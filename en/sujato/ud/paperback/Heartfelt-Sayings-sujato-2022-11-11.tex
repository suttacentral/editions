\documentclass[12pt,openany]{book}%
\usepackage{lastpage}%
%
\usepackage[inner=1in, outer=1in, top=.7in, bottom=1in, papersize={6in,9in}, headheight=13pt]{geometry}
\usepackage{polyglossia}
\usepackage[12pt]{moresize}
\usepackage{soul}%
\usepackage{microtype}
\usepackage{tocbasic}
\usepackage{realscripts}
\usepackage{epigraph}%
\usepackage{setspace}%
\usepackage{sectsty}
\usepackage{fontspec}
\usepackage{marginnote}
\usepackage[bottom]{footmisc}
\usepackage{enumitem}
\usepackage{fancyhdr}
\usepackage{extramarks}
\usepackage{graphicx}
\usepackage{verse}
\usepackage{relsize}
\usepackage{etoolbox}
\usepackage[a-3u]{pdfx}

\hypersetup{
colorlinks=true,
urlcolor=black,
linkcolor=black,
citecolor=black
}

% use a small amount of tracking on small caps
\SetTracking[ spacing = {25*,166, } ]{ encoding = *, shape = sc }{ 25 }

% add a blank page
\newcommand{\blankpage}{
\newpage
\thispagestyle{empty}
\mbox{}
\newpage
}

% define languages
\setdefaultlanguage[]{english}
\setotherlanguage[script=Latin]{sanskrit}

%\usepackage{pagegrid}
%\pagegridsetup{top-left, step=.25in}

% define fonts
% use if arno sanskrit is unavailable
%\setmainfont{Gentium Plus}
%\newfontfamily\Semiboldsubheadfont[]{Gentium Plus}
%\newfontfamily\Semiboldnormalfont[]{Gentium Plus}
%\newfontfamily\Lightfont[]{Gentium Plus}
%\newfontfamily\Marginalfont[]{Gentium Plus}
%\newfontfamily\Allsmallcapsfont[RawFeature=+c2sc]{Gentium Plus}
%\newfontfamily\Noligaturefont[Renderer=Basic]{Gentium Plus}
%\newfontfamily\Noligaturecaptionfont[Renderer=Basic]{Gentium Plus}
%\newfontfamily\Fleuronfont[Ornament=1]{Gentium Plus}

% use if arno sanskrit is available. display is applied to \chapter and \part, subhead to \section and \subsection. When specifying semibold, the italic must be defined.
\setmainfont[Numbers=OldStyle]{Arno Pro}
\newfontfamily\Semibolddisplayfont[BoldItalicFont = Arno Pro Semibold Italic Display]{Arno Pro Semibold Display} %
\newfontfamily\Semiboldsubheadfont[BoldItalicFont = Arno Pro Semibold Italic Subhead]{Arno Pro Semibold Subhead}
\newfontfamily\Semiboldnormalfont[BoldItalicFont = Arno Pro Semibold Italic]{Arno Pro Semibold}
\newfontfamily\Marginalfont[RawFeature=+subs]{Arno Pro Regular}
\newfontfamily\Allsmallcapsfont[RawFeature=+c2sc]{Arno Pro}
\newfontfamily\Noligaturefont[Renderer=Basic]{Arno Pro}
\newfontfamily\Noligaturecaptionfont[Renderer=Basic]{Arno Pro Caption}

% chinese fonts
\newfontfamily\cjk{Noto Serif TC}
\newcommand*{\langlzh}[1]{\cjk{#1}\normalfont}%

% logo
\newfontfamily\Logofont{sclogo.ttf}
\newcommand*{\sclogo}[1]{\large\Logofont{#1}}

% use subscript numerals for margin notes
\renewcommand*{\marginfont}{\Marginalfont}

% ensure margin notes have consistent vertical alignment
\renewcommand*{\marginnotevadjust}{-.17em}

% use compact lists
\setitemize{noitemsep,leftmargin=1em}
\setenumerate{noitemsep,leftmargin=1em}
\setdescription{noitemsep, style=unboxed, leftmargin=0em}

% style ToC
\DeclareTOCStyleEntries[
  raggedentrytext,
  linefill=\hfill,
  pagenumberwidth=.5in,
  pagenumberformat=\normalfont,
  entryformat=\normalfont
]{tocline}{chapter,section}


  \setlength\topsep{0pt}%
  \setlength\parskip{0pt}%

% define new \centerpars command for use in ToC. This ensures centering, proper wrapping, and no page break after
\def\startcenter{%
  \par
  \begingroup
  \leftskip=0pt plus 1fil
  \rightskip=\leftskip
  \parindent=0pt
  \parfillskip=0pt
}
\def\stopcenter{%
  \par
  \endgroup
}
\long\def\centerpars#1{\startcenter#1\stopcenter}

% redefine part, so that it adds a toc entry without page number
\let\oldcontentsline\contentsline
\newcommand{\nopagecontentsline}[3]{\oldcontentsline{#1}{#2}{}}

    \makeatletter
\renewcommand*\l@part[2]{%
  \ifnum \c@tocdepth >-2\relax
    \addpenalty{-\@highpenalty}%
    \addvspace{0em \@plus\p@}%
    \setlength\@tempdima{3em}%
    \begingroup
      \parindent \z@ \rightskip \@pnumwidth
      \parfillskip -\@pnumwidth
      {\leavevmode
       \setstretch{.85}\large\scshape\centerpars{#1}\vspace*{-1em}\llap{#2}}\par
       \nobreak
         \global\@nobreaktrue
         \everypar{\global\@nobreakfalse\everypar{}}%
    \endgroup
  \fi}
\makeatother

\makeatletter
\def\@pnumwidth{2em}
\makeatother

% define new sectioning command, which is only used in volumes where the pannasa is found in some parts but not others, especially in an and sn

\newcommand*{\pannasa}[1]{\clearpage\thispagestyle{empty}\begin{center}\vspace*{14em}\setstretch{.85}\huge\itshape\scshape\MakeLowercase{#1}\end{center}}

    \makeatletter
\newcommand*\l@pannasa[2]{%
  \ifnum \c@tocdepth >-2\relax
    \addpenalty{-\@highpenalty}%
    \addvspace{.5em \@plus\p@}%
    \setlength\@tempdima{3em}%
    \begingroup
      \parindent \z@ \rightskip \@pnumwidth
      \parfillskip -\@pnumwidth
      {\leavevmode
       \setstretch{.85}\large\itshape\scshape\lowercase{\centerpars{#1}}\vspace*{-1em}\llap{#2}}\par
       \nobreak
         \global\@nobreaktrue
         \everypar{\global\@nobreakfalse\everypar{}}%
    \endgroup
  \fi}
\makeatother

% don't put page number on first page of toc (relies on etoolbox)
\patchcmd{\chapter}{plain}{empty}{}{}

% global line height
\setstretch{1.05}

% allow linebreak after em-dash
\catcode`\—=13
\protected\def—{\unskip\textemdash\allowbreak}

% style headings with secsty. chapter and section are defined per-edition
\partfont{\setstretch{.85}\normalfont\centering\textsc}
\subsectionfont{\setstretch{.85}\Semiboldsubheadfont}%
\subsubsectionfont{\setstretch{.85}\Semiboldnormalfont}

% style elements of suttatitle
\newcommand*{\suttatitleacronym}[1]{\smaller[2]{#1}\vspace*{.3em}}
\newcommand*{\suttatitletranslation}[1]{\linebreak{#1}}
\newcommand*{\suttatitleroot}[1]{\linebreak\smaller[2]\itshape{#1}}

\DeclareTOCStyleEntries[
  indent=3.3em,
  dynindent,
  beforeskip=.2em plus -2pt minus -1pt,
]{tocline}{section}

\DeclareTOCStyleEntries[
  indent=0em,
  dynindent,
  beforeskip=.4em plus -2pt minus -1pt,
]{tocline}{chapter}

\newcommand*{\tocacronym}[1]{\hspace*{-3.3em}{#1}\quad}
\newcommand*{\toctranslation}[1]{#1}
\newcommand*{\tocroot}[1]{(\textit{#1})}
\newcommand*{\tocchapterline}[1]{\bfseries\itshape{#1}}


% redefine paragraph and subparagraph headings to not be inline
\makeatletter
% Change the style of paragraph headings %
\renewcommand\paragraph{\@startsection{paragraph}{4}{\z@}%
            {-2.5ex\@plus -1ex \@minus -.25ex}%
            {1.25ex \@plus .25ex}%
            {\noindent\Semiboldnormalfont\normalsize}}

% Change the style of subparagraph headings %
\renewcommand\subparagraph{\@startsection{subparagraph}{5}{\z@}%
            {-2.5ex\@plus -1ex \@minus -.25ex}%
            {1.25ex \@plus .25ex}%
            {\noindent\Semiboldnormalfont\small}}
\makeatother

% use etoolbox to suppress page numbers on \part
\patchcmd{\part}{\thispagestyle{plain}}{\thispagestyle{empty}}
  {}{\errmessage{Cannot patch \string\part}}

% and to reduce margins on quotation
\patchcmd{\quotation}{\rightmargin}{\leftmargin 1.2em \rightmargin}{}{}
\AtBeginEnvironment{quotation}{\small}

% titlepage
\newcommand*{\titlepageTranslationTitle}[1]{{\begin{center}\begin{large}{#1}\end{large}\end{center}}}
\newcommand*{\titlepageCreatorName}[1]{{\begin{center}\begin{normalsize}{#1}\end{normalsize}\end{center}}}

% halftitlepage
\newcommand*{\halftitlepageTranslationTitle}[1]{\setstretch{2.5}{\begin{Huge}\uppercase{\so{#1}}\end{Huge}}}
\newcommand*{\halftitlepageTranslationSubtitle}[1]{\setstretch{1.2}{\begin{large}{#1}\end{large}}}
\newcommand*{\halftitlepageFleuron}[1]{{\begin{large}\Fleuronfont{{#1}}\end{large}}}
\newcommand*{\halftitlepageByline}[1]{{\begin{normalsize}\textit{{#1}}\end{normalsize}}}
\newcommand*{\halftitlepageCreatorName}[1]{{\begin{LARGE}{\textsc{#1}}\end{LARGE}}}
\newcommand*{\halftitlepageVolumeNumber}[1]{{\begin{normalsize}{\Allsmallcapsfont{\textsc{#1}}}\end{normalsize}}}
\newcommand*{\halftitlepageVolumeAcronym}[1]{{\begin{normalsize}{#1}\end{normalsize}}}
\newcommand*{\halftitlepageVolumeTranslationTitle}[1]{{\begin{Large}{\textsc{#1}}\end{Large}}}
\newcommand*{\halftitlepageVolumeRootTitle}[1]{{\begin{normalsize}{\Allsmallcapsfont{\textsc{\itshape #1}}}\end{normalsize}}}
\newcommand*{\halftitlepagePublisher}[1]{{\begin{large}{\Noligaturecaptionfont\textsc{#1}}\end{large}}}

% epigraph
\renewcommand{\epigraphflush}{center}
\renewcommand*{\epigraphwidth}{.85\textwidth}
\newcommand*{\epigraphTranslatedTitle}[1]{\vspace*{.5em}\footnotesize\textsc{#1}\\}%
\newcommand*{\epigraphRootTitle}[1]{\footnotesize\textit{#1}\\}%
\newcommand*{\epigraphReference}[1]{\footnotesize{#1}}%

% custom commands for html styling classes
\newcommand*{\scnamo}[1]{\begin{center}\textit{#1}\end{center}}
\newcommand*{\scendsection}[1]{\begin{center}\textit{#1}\end{center}}
\newcommand*{\scendsutta}[1]{\begin{center}\textit{#1}\end{center}}
\newcommand*{\scendbook}[1]{\begin{center}\uppercase{#1}\end{center}}
\newcommand*{\scendkanda}[1]{\begin{center}\textbf{#1}\end{center}}
\newcommand*{\scend}[1]{\begin{center}\textit{#1}\end{center}}
\newcommand*{\scuddanaintro}[1]{\textit{#1}}
\newcommand*{\scendvagga}[1]{\begin{center}\textbf{#1}\end{center}}
\newcommand*{\scrule}[1]{\textbf{#1}}
\newcommand*{\scadd}[1]{\textit{#1}}
\newcommand*{\scevam}[1]{\textsc{#1}}
\newcommand*{\scspeaker}[1]{\hspace{2em}\textit{#1}}
\newcommand*{\scbyline}[1]{\begin{flushright}\textit{#1}\end{flushright}\bigskip}

% custom command for thematic break = hr
\newcommand*{\thematicbreak}{\begin{center}\rule[.5ex]{6em}{.4pt}\begin{normalsize}\quad\Fleuronfont{•}\quad\end{normalsize}\rule[.5ex]{6em}{.4pt}\end{center}}

% manage and style page header and footer. "fancy" has header and footer, "plain" has footer only

\pagestyle{fancy}
\fancyhf{}
\fancyfoot[RE,LO]{\thepage}
\fancyfoot[LE,RO]{\footnotesize\lastleftxmark}
\fancyhead[CE]{\setstretch{.85}\Noligaturefont\MakeLowercase{\textsc{\firstrightmark}}}
\fancyhead[CO]{\setstretch{.85}\Noligaturefont\MakeLowercase{\textsc{\firstleftmark}}}
\renewcommand{\headrulewidth}{0pt}
\fancypagestyle{plain}{ %
\fancyhf{} % remove everything
\fancyfoot[RE,LO]{\thepage}
\fancyfoot[LE,RO]{\footnotesize\lastleftxmark}
\renewcommand{\headrulewidth}{0pt}
\renewcommand{\footrulewidth}{0pt}}

% style footnotes
\setlength{\skip\footins}{1em}

\makeatletter
\newcommand{\@makefntextcustom}[1]{%
    \parindent 0em%
    \thefootnote.\enskip #1%
}
\renewcommand{\@makefntext}[1]{\@makefntextcustom{#1}}
\makeatother

% hang quotes (requires microtype)
\microtypesetup{
  protrusion = true,
  expansion  = true,
  tracking   = true,
  factor     = 1000,
  patch      = all,
  final
}

% Custom protrusion rules to allow hanging punctuation
\SetProtrusion
{ encoding = *}
{
% char   right left
  {-} = {    , 500 },
  % Double Quotes
  \textquotedblleft
      = {1000,     },
  \textquotedblright
      = {    , 1000},
  \quotedblbase
      = {1000,     },
  % Single Quotes
  \textquoteleft
      = {1000,     },
  \textquoteright
      = {    , 1000},
  \quotesinglbase
      = {1000,     }
}

% make latex use actual font em for parindent, not Computer Modern Roman
\AtBeginDocument{\setlength{\parindent}{1em}}%
%

% Default values; a bit sloppier than normal
\tolerance 1414
\hbadness 1414
\emergencystretch 1.5em
\hfuzz 0.3pt
\clubpenalty = 10000
\widowpenalty = 10000
\displaywidowpenalty = 10000
\hfuzz \vfuzz
 \raggedbottom%

\title{Heartfelt Sayings}
\author{Bhikkhu Sujato}
\date{}%
% define a different fleuron for each edition
\newfontfamily\Fleuronfont[Ornament=1]{Arno Pro}

% Define heading styles per edition for chapter and section. Suttatitle can be either of these, depending on the volume. 

\let\oldfrontmatter\frontmatter
\renewcommand{\frontmatter}{%
\chapterfont{\setstretch{.85}\normalfont\centering}%
\sectionfont{\setstretch{.85}\Semiboldsubheadfont}%
\oldfrontmatter}

\let\oldmainmatter\mainmatter
\renewcommand{\mainmatter}{%
\chapterfont{\setstretch{.85}\normalfont\centering}%
\sectionfont{\setstretch{.85}\normalfont\centering}%
\oldmainmatter}

\let\oldbackmatter\backmatter
\renewcommand{\backmatter}{%
\chapterfont{\setstretch{.85}\normalfont\centering}%
\sectionfont{\setstretch{.85}\Semiboldsubheadfont}%
\oldbackmatter}
%
%
\begin{document}%
\normalsize%
\frontmatter%
\setlength{\parindent}{0cm}

\pagestyle{empty}

\maketitle

\blankpage%
\begin{center}

\vspace*{2.2em}

\halftitlepageTranslationTitle{Heartfelt Sayings}

\vspace*{1em}

\halftitlepageTranslationSubtitle{An uplifting translation of the Udāna.}

\vspace*{2em}

\halftitlepageFleuron{•}

\vspace*{2em}

\halftitlepageByline{translated and introduced by}

\vspace*{.5em}

\halftitlepageCreatorName{Bhikkhu Sujato}

\vspace*{4em}

\halftitlepageVolumeNumber{}

\smallskip

\halftitlepageVolumeAcronym{Ud}

\smallskip

\halftitlepageVolumeTranslationTitle{}

\smallskip

\halftitlepageVolumeRootTitle{}

\vspace*{\fill}

\sclogo{0}
 \halftitlepagePublisher{SuttaCentral}

\end{center}

\newpage
%
\setstretch{1.05}

\begin{footnotesize}

\textit{Heartfelt Sayings} is a translation of the Udāna by Bhikkhu Sujato.

\medskip

Creative Commons Zero (CC0)

To the extent possible under law, Bhikkhu Sujato has waived all copyright and related or neighboring rights to \textit{Heartfelt Sayings}.

\medskip

This work is published from Australia.

\begin{center}
\textit{This translation is an expression of an ancient spiritual text that has been passed down by the Buddhist tradition for the benefit of all sentient beings. It is dedicated to the public domain via Creative Commons Zero (CC0). You are encouraged to copy, reproduce, adapt, alter, or otherwise make use of this translation. The translator respectfully requests that any use be in accordance with the values and principles of the Buddhist community.}
\end{center}

\medskip

\begin{description}
    \item[Web publication date] 2021
    \item[This edition] 2022-11-11 13:32:54
    \item[Publication type] paperback
    \item[Edition] ed5
    \item[Number of volumes] 1
    \item[Publication ISBN] 978-1-76132-055-2
    \item[Publication URL] https://suttacentral.net/editions/ud/en/sujato
    \item[Source URL] https://github.com/suttacentral/bilara-data/tree/published/translation/en/sujato/sutta/kn/ud
    \item[Publication number] scpub18
\end{description}

\medskip

Published by SuttaCentral

\medskip

\textit{SuttaCentral,\\
c/o Alwis \& Alwis Pty Ltd\\
Kaurna Country,\\
Suite 12,\\
198 Greenhill Road,\\
Eastwood,\\
SA 5063,\\
Australia}

\end{footnotesize}

\newpage

\setlength{\parindent}{1.5em}%%
\newpage

\vspace*{\fill}

\begin{center}
\epigraph{Seclusion is happiness for the contented\\
who see the teaching they have learned.\\
Kindness for the world is happiness\\
for one who’d not harm a living creature.}
{
\epigraphTranslatedTitle{With Mucalinda}
\epigraphRootTitle{Mucalindasutta}
\epigraphReference{\textsanskrit{Udāna} 2.1}
}
\end{center}

\vspace*{2in}

\vspace*{\fill}

\blankpage%

\setlength{\parindent}{1em}
%
\tableofcontents
\newpage
\pagestyle{fancy}
%
\chapter*{The SuttaCentral Editions Series}
\addcontentsline{toc}{chapter}{The SuttaCentral Editions Series}
\markboth{The SuttaCentral Editions Series}{The SuttaCentral Editions Series}

Since 2005 SuttaCentral has provided access to the texts, translations, and parallels of early Buddhist texts. In 2018 we started creating and publishing our own translations of these seminal spiritual classics. The “Editions” series now makes selected translations available as books in various forms, including print, PDF, and EPUB.

Editions are selected from our most complete, well-crafted, and reliable translations. They aim to bring these texts to a wider audience in forms that reward mindful reading. Care is taken with every detail of the production, and we aim to meet or exceed professional best standards in every way. These are the core scriptures underlying the entire Buddhist tradition, and we believe that they deserve to be preserved and made available in highest quality without compromise.

SuttaCentral is a charitable organization. Our work is accomplished by volunteers and through the generosity of our donors. Everything we create is offered to all of humanity free of any copyright or licensing restrictions. 

%
\chapter*{Preface}
\addcontentsline{toc}{chapter}{Preface}
\markboth{Preface}{Preface}

Joy is the most underrated emotion. Joy or rejoicing or celebration is \textit{\textsanskrit{muditā}} in Pali, one of the four “meditations of \textsanskrit{Brahmā}” (\textit{\textsanskrit{brahmavihāra}}). At one point, I wanted to translate it as “pride”, not in the sense of “vanity” but in “taking pleasure in the accomplishments of oneself or others”. I didn’t go ahead with that, but I still think about it. It is the dual aspect of “pride” that interests me.

For many years, philosophers in the Christian traditions taught that pride was the greatest of sins, tantamount to imagining oneself God. But for Buddhists, that is precisely the point. \textit{\textsanskrit{Muditā}} does, in fact, lead to becoming God, or \textsanskrit{Brahmā} as the ancient Indians called them. For that is ultimately what the four “meditations of \textsanskrit{Brahmā}” are: what God did that made them become God. Any one of us can do the same thing and become God in our own right—if that is what we want.

I’m proud of the work that I’ve achieved in the suttas. It makes me happy to think about. It gives me confidence and assurance when I hear that others have found joy in my work too. And in my meditation, that joy lifts me up and sustains me. It is purely wholesome and leads only to freedom.

I want others to feel that same joy. This glorious tradition belongs to all of us. We all have something to offer, so let us honor our tradition with our time and love and intelligence.

%
\chapter*{Heartfelt Sayings: inspirational passages and stories}
\addcontentsline{toc}{chapter}{Heartfelt Sayings: inspirational passages and stories}
\markboth{Heartfelt Sayings: inspirational passages and stories}{Heartfelt Sayings: inspirational passages and stories}

\scbyline{Bhikkhu Sujato, 2022}

The \textsanskrit{Udāna} is a collection of 80 discourses that are are inspiring, accessible, and epigrammatic. It forms an ideal introduction to the Buddha’s teachings; a combination of simple, catchy, and profound that remains as popular today as it has ever been. The collection speaks of meditation, wisdom, and freedom in the context of dramatic, sometimes quirky, stories, loosely arranged to follow the life of the Buddha. It finds space for ethical examples, ecstatic celebrations of liberation, and solemn meditations on the nature of \textsanskrit{Nibbāna}.

Each discourse has a narrative background culminating in a short passage of heightened significance that conveys the spiritual essence of the text. These passages give the collection as a whole its name. The word \textit{\textsanskrit{udāna}} literally means “up-breath” and it is translated as “heartfelt saying” or “inspired utterance”. These passages are usually, but not always, in verse. The commentary explains \textit{\textsanskrit{udāna}} as an overflowing of joy in the heart.

This sense of \textit{\textsanskrit{udāna}} appears to be specifically Buddhist. The word \textit{\textsanskrit{udāna}} appears in the Brahmanical \textsanskrit{Upaniṣads} as one of the five “breaths” which form one aspect of the self (eg. \textsanskrit{Bṛhadāraṇyaka} \textsanskrit{Upaniṣad} 3,9.26, \textsanskrit{Chāndogya} \textsanskrit{Upaniṣad} 3,13.5). There is, however, nothing in these passages that suggests any connection with the Buddhist usage. While the Buddhist texts do have a similar concept of different breaths moving around the body, they describe them with quite different terminology MN 140:17.5, while reserving the word \textit{\textsanskrit{udāna}} for the sense “heartfelt saying”.

The prose setting and the \textit{\textsanskrit{udāna}} proper are joined by a stock phrase recording that the Buddha, having understood the events of the narrative, spoke the \textit{\textsanskrit{udāna}}. A key term in this stock phrase, \textit{attha}, has been interpreted by some translators as “meaning” (Ireland, \textsanskrit{Ṭhānissaro}, Mahendra) or “significance” (Ānandajoti). Others, however, take \textit{attha} in the sense of “connection” (Strong) or else “matter” (Masefield), a sense also accepted by Cone in her \textit{Dictionary of Pali}. 

How are we to resolve this? Well, grammatically, \textit{\textsanskrit{etamatthaṁ} \textsanskrit{viditvā}} reads more naturally as “this matter” than “the meaning of this”, for which the genitive is used for example at SN 22.1:7.3 (\textit{etassa \textsanskrit{bhāsitassa} \textsanskrit{atthamaññātuṁ}}). We find a further clue in the rather curious detail that in one and only one discourse of the \textsanskrit{Udāna}, there is a tag line that echoes the tag found in all discourses of the Itivuttaka. There, it says that “this \textit{attha} was spoken by the Buddha”. At the end of Ud 1.10  the tag says “this \textit{\textsanskrit{udāna}} was spoken by the Buddha”. Since the \textit{\textsanskrit{udāna}} can only mean the text or substance of what was spoken, it seems the same must apply to \textit{attha}. In these cases, then, \textit{attha} does not have its more common textual sense of “meaning”, but rather refers to the passage or matter or substance of what was spoken.

\section*{The place of the \textsanskrit{Udāna} and its relation to the Dhammapada}

The \textsanskrit{Udāna} is the third book of the Khuddaka \textsanskrit{Nikāya} in the Pali \textsanskrit{Tipiṭaka}. There’s also an \textit{\textsanskrit{udāna}} in the list of the Buddha’s teachings known as the the nine sections or genres of the teaching (\textit{\textsanskrit{navaṅgadhamma}}) (eg. AN 4.6, MN  22:10.2). This list appears in a standard form within the \textsanskrit{Nikāyas}, and is found in similar form in the Chinese Āgamas and elsewhere, although there we also find an extended list of twelve sections. The \textsanskrit{Nikāyas} are not mentioned at such an early stage, which suggests that the nine sections were an earlier way of organizing and categorizing the Dhamma, one which may have originated during the Buddha’s lifetime. 

This raises the question as to whether the \textsanskrit{Udāna} as we have it today is identical with the \textit{\textsanskrit{udāna}} referred to in the nine sections. To answer this we shall have to consider the close relation of the \textsanskrit{Udāna} with another class of Buddhist scripture, the Dhammapada (Sayings of the Dhamma). 

The Dhammapada is notably absent from the nine sections. And if we look to collections outside of Pali, we find a number of Dhammapada-style texts that are called \textsanskrit{Udānavarga} (Handbook of Heartfelt Sayings). These are attributed to the \textsanskrit{Sarvāstivāda}, which was an early school of Buddhism based in north-east India, especially Kashmir. Today we find several versions in Sanskrit, Chinese, and Tibetan, all similar but with their own idiosyncrasies. Like the Pali Dhammapada, the \textsanskrit{Udānavarga} is purely verse, without narrative. On the other hand, some texts called Dhammapada supply a narrative background for their verses (T 211 and T 212 in the Chinese Taishō \textsanskrit{Tipiṭaka}). Meanwhile the Pali Dhammapada, while being purely verse in the canon, is accompanied by a set of narrative stories in its commentary, which contain many of the most beloved stories in the Theravada tradition.

It seems, then, that the original idea of the \textit{\textsanskrit{udāna}} was a collection of short sayings, often in verse. These sayings would have been taught together with a narrative that framed and gave weight and significance to the verses. However, the \textit{\textsanskrit{udānas}} themselves would have been relatively fixed at an early stage, while the stories would remain relatively fluid. In practice, the stories would have been conveyed by a teacher in a flexible way; this style of Dhamma teaching is still popular today. The narratives vary more than the verses, and we often find that the same verse is accompanied by a quite different background story. This literary form is a common one in Buddhism and we find variations of it in such diverse places as the \textsanskrit{Jātakas} and the Vinaya \textsanskrit{Vibhaṅgas}. At some point, these collections became known as Dhammapada. In the Pali tradition two separate texts emerged, the \textsanskrit{Udāna} with narrative background in the canonical text, and the Dhammapada with narrative background in the commentary. 

What is common to all these collections, however, is the character and flavor of the collected verses. They address universal themes of the Dhamma in a pithy and appealing way. Typically the verses stand individually, sometimes in pairs, and and collected in chapters by theme. While the detailed list of chapters and verses differs from one collection to the next, they share many hundreds of verses in common. They differ in the selection and arrangement of verses, not in doctrine.

To answer our question, then, it seems that the text of the \textsanskrit{Udāna} as we have it today is neither identical with the \textit{\textsanskrit{udāna}} referred to in the nine sections, nor is it completely different. Rather, at first \textit{\textsanskrit{udāna}} referred to a somewhat fluid genre of short passages of a popular and uplifting nature. Over time, these crystallized as the collections we know today as Dhammapada and \textsanskrit{Udāna}. Each of these collections share a common structure, flavor, and much of the content. The verses would have been taught with background stories, which over time became collected, sometimes in the canonical texts, and sometimes in commentaries.

As a point of clarification, it is worth noting that there is a separate class of verses in Buddhist texts that are called \textit{\textsanskrit{uddāna}} (with double d). Despite the similar spelling, this is a completely different kind of verse. An \textit{\textsanskrit{uddāna}} is a summary that appears at the end of a chapter or larger section, and lists the titles of the suttas or chapters. It performs much the same function as a table of contents in a modern edition, and appears very widely throughout all the early Buddhist literature. Traditionally it would act as a mnemonic, so that reciters could check that that had remembered all the suttas of that chapter. Most modern translators, including myself, do not translate these. While \textit{\textsanskrit{udāna}} traces its root to \textit{\textsanskrit{āna}} “breath”, \textit{\textsanskrit{uddāna}} is from the root \textit{dayati} “to bind”, and means a “set” or “batch”. 

\section*{\textit{\textsanskrit{Udānas}} outside the \textsanskrit{Udāna}}

The \textsanskrit{Udāna} marks its heartfelt verses, the \textit{\textsanskrit{udāna}} proper, by saying that the Buddha “expresses this heartfelt sentiment” (\textit{\textsanskrit{udānaṁ} \textsanskrit{udānesi}}). If we consider the \textit{\textsanskrit{udāna}} as a literary trope, we find it is not restricted to the book called \textsanskrit{Udāna}, but is used quite widely in the \textsanskrit{Nikāyas}, and even the Vinaya. 

Some of these \textit{\textsanskrit{udānas}} are collected in the \textsanskrit{Udāna}. For example, the first three suttas of the \textsanskrit{Udāna} also appear as the opening sections of the Vinaya Khandhaka (Pli Tv Kd 1). When, at the end of his life, the Buddha decided to relinquish his life-force, he spoke an \textit{\textsanskrit{udāna}} to mark the dramatic occasion. The verse appears, with narrative context, in several places both in the \textsanskrit{Udāna} and elsewhere (Ud 6.1, AN 8.70, SN 51.10, DN 16:3.10.3). That last sutta, the \textsanskrit{Mahāparinibbāna}, features two other \textit{\textsanskrit{udānas}}, on crossing the river (DN 16:1.34.3) and on giving (DN 16:4.43.2), which are found in consecutive suttas of the \textsanskrit{Udāna} (Ud 8.5, Ud 8.6).

In other cases, an \textit{\textsanskrit{udāna}} is not collected in the \textsanskrit{Udāna}. The \textsanskrit{Aṅgulimālasutta}, for example, concludes its tale of redemption with a series of verses described as \textit{\textsanskrit{udānas}}. Several of these verses appear in the Dhammapada, not in the \textsanskrit{Udāna}, reinforcing the close connection between these collections.

An \textit{\textsanskrit{udāna}} is often an emotional reaction to a particular context. When inspired by the the Buddha, for example, devoted layfolk such as the brahmin \textsanskrit{Kāraṇapālī}  (AN 5.194:9.1), King Pasenadi (MN 87:29.5), the brahmin \textsanskrit{Brahmāyu} (MN 91:23.1), the brahmin \textsanskrit{Jāṇussoṇi} (MN 27:8.2), or the brahmin lady \textsanskrit{Dhanañjānī} (SN 7.1:2.1) uttered the triple \textit{\textsanskrit{udāna}} of homage that we still use today: \textit{namo tassa bhagavato arahato \textsanskrit{sammāsambuddhassa}}. 

On the other hand, an \textit{\textsanskrit{udāna}} might have no particular Dhammic significance outside of context. The Licchavi \textsanskrit{Mahānāma}, when he saw the dissolute youths of his clan respecting the Buddha, was so pleased that he uttered an \textit{\textsanskrit{udāna}} to the Buddha: “The Vajjis will grow up! The Vajjis will grow up!” (\textit{bhavissanti \textsanskrit{vajjī}, bhavissanti \textsanskrit{vajjī}}) (AN 5.58:3.4).

An \textit{\textsanskrit{udāna}} need not be a Buddhist saying. At MN 80:2.4 the non-Buddhist wanderer Vekhanasa expressed to the Buddha his affirmation, “This is the ultimate splendor, this is the ultimate splendor.” The Buddha examined him and found his saying to be hollow, as Vekhanasa could not even describe what he was talking about.

At SN 22.55  we find an “\textsanskrit{Udānasutta}”, which opens with the Buddha making the following \textit{\textsanskrit{udāna}}: “It might not be, and it might not be mine. It will not be, and it will not be mine.” The \textsanskrit{Udāna} includes the same saying in a different context at Ud 7.8. Now, this saying is found several times elsewhere in the Suttas, in subtle variants, without being called an \textit{\textsanskrit{udāna}} (eg. MN 106:10.1). This shows that the genre \textit{\textsanskrit{udāna}} need not be limited to sayings that are explicitly called \textit{\textsanskrit{udāna}}, but rather, is a generic term for pithy and inspired phrases. 

While I translate \textit{\textsanskrit{udāna}} to emphasize their “heartfelt” nature, which was evidently the original intent of the term, it is, sadly, the case that repetition can dull even the most inspiring of teachings. When the monk \textsanskrit{Cūḷapanthaka} was scheduled to give a teaching for the nuns, they complained that he would probably just give them the same boring old \textit{\textsanskrit{udāna}} that they‘d heard so many times before. And that did indeed prove to be the case. However, when he heard their complaints, \textsanskrit{Cūḷapanthaka} changed his teaching approach, to much more effective results (Monks’ \textsanskrit{Pācittiya} 22). 

This incomplete survey shows that an \textit{\textsanskrit{udāna}} need not be a Buddhist teaching, or even a teaching at all; that it need not be verse; that it need not be labelled as an \textit{\textsanskrit{udāna}}; and that the sense of “inspiration” might not always apply. Not all \textit{\textsanskrit{udānas}} were collected in the book called \textsanskrit{Udāna}. In some cases \textit{\textsanskrit{udānas}} were obviously not suitable candidates for inclusion, but in other cases, such as the verses of \textsanskrit{Aṅgulimāla}, the choice to put them in the Dhammapada rather than the \textsanskrit{Udāna} seems to have rested with the discretion of the redactors. 

\section*{The narrative purpose of the \textsanskrit{Udāna}}

If the mere fact of being an \textit{\textsanskrit{udāna}} is not sufficient for inclusion in the \textsanskrit{Udāna}, might there be another purpose to their selection? John Ireland, in the introduction to his translation of the \textsanskrit{Udāna}, did not think so, saying there was “often no discernible theme linking the utterances” and that many pieces were “taken from elsewhere in the \textsanskrit{Pāli} Canon, with no obvious systematization.”

Venerable \textsanskrit{Anālayo}, in his article on the \textsanskrit{Udāna} for the Encyclopedia of Buddhism, discusses several kinds of thematic and literary links between the different suttas, and identifies a main theme for each chapter. But he doesn’t posit any overarching purpose in the selection and arrangement.

Venerable Ānandajoti, however, in the introduction to his translation of the \textsanskrit{Udāna}, notes:

\begin{quotation}%
Some of the most memorable stories in the Canon have found their way into this collection, which seems to have an overall structural plan, in that it begins with events that happened just after the Sambodhi (also recorded in the \textsanskrit{Mahāvagga} of the Vinaya); and the last chapter includes many events from the last days of the Buddha as recorded in the \textsanskrit{Mahāparinibbānasutta} (\textsanskrit{Dīghanikāya} 16). Note that the \textsanskrit{Udāna} ends, not with the Buddha’s \textsanskrit{parinibbāna}, following which no \textsanskrit{udāna} was spoken, of course; but with the \textsanskrit{parinibbāna} of one of the Buddha’s leading disciples Ven. Dabba Mallaputta.

%
\end{quotation}

In the introduction to his translation, Venerable \textsanskrit{Ṭhānissaro} also points out this narrative cohesion, although his essay looks more at the thematic rather than narrative unity. He argues that the \textsanskrit{Udāna} has the unifying aesthetic flavor of the “astounding”. This recurs in many of the extraordinary events depicted in the text. While other flavors complement the main one, he notes that the redactors of the \textsanskrit{Udāna} carefully avoid the incompatible flavor of “digust”, even in contexts where it would be appropriate. This is a sign of the care and attention to detail of the redaction process.

Following up on the idea of the narrative arc of the text, I believe that the redactors of the \textsanskrit{Udāna} did indeed have an overall purpose in mind; and moreover, that they announced this fact clearly with their choice of opening suttas. The \textsanskrit{Udāna} begins in exactly the same way as the section of the Vinaya known as the Khandhakas (or \textsanskrit{Mahāvagga} per Ānandajoti): with the Buddha’s ecstatic proclamation of dependent origination immediately after his awakening. The Khandhakas go on to tell many of the key events of the Buddha’s life until his passing away, and even after, doing so as a narrative context for the monastic Vinaya. The \textsanskrit{Udāna} covers much of the same ground, shifting the emphasis to inspiration and meditative freedom.

It’s hard to over-estimate the significance of the Buddha’s life story to all Buddhists, and especially to the early generations who selected and arranged texts such as the \textsanskrit{Udāna}. Key events were told and retold, contextualized in particular ways to draw out certain patterns of meaning.

While the narrative frame of the \textsanskrit{Udāna} as a whole is not explicit, careful consideration reveals that it is not just in the opening discourses, but the text as a whole is arranged to follow the life of the Buddha. Like the Khandhakas, however, it does not cover his whole life. From the accounts of the very first days of Awakening they move to the establishment of the dispensation, touching on some crucial events of the Buddha’s life, such as the betrayal by Devadatta. And they both conclude with the Buddha’s passing and the establishment of his legacy. For this, the \textsanskrit{Mahāparinibbānasutta} is the key text. In the Pali, this is not found in the Khandhakas, but it dovetails on to the narrative of the Councils at the Khandhaka’s end. In several other Vinayas, however, it is found in the Khandhakas themselves, confirming that it is meant to be seen as part of the same great story.

In both texts, the narrative arc is most pronounced in the beginning and ending chapters, and even then, usually in the first portions of each chapter. Middle chapters are less clearly defined. This is, however, not an indication that the texts have lost the plot, but rather that they follows the standard form of Indian narrative. If you make sure you have a strong and clearly defined opening and closing, then in the middle you have the freedom to take the audience on a wandering journey that may appear random, but which always bears the destination in mind.

In the \textsanskrit{Udāna}, the arc is obscured because, once the theme of the chapter is established (usually in the first discourse of each chapter), other discourses are added. Some of these relate to their particular chapter thematically rather than narratively. 

Others are added in a way that might seem random, although I don’t believe that that is the case. I think what is happening is that, since many of the middle chapters deal with struggles facing the dispensation, each chapter counterbalances this by including inspiring stories of individual practitioners who overcome challenges in their meditations. This keeps alive the themes of awakening and happiness announced in the first two chapters, which are woven like a thread binding each chapter into the whole even as new themes are introduced.

Let’s take a bird’s eye view of how the \textsanskrit{Udāna} builds this narrative as we consider the text chapter by chapter. 

\section*{Ud 1: Awakening—the true brahmin}

The first chapter establishes the most astonishing thesis of Buddhism: that it is possible, in this very life, and without assistance from divine entities of magical powers, to let go of all suffering and realize human perfection simply through the power of understanding.

The chapter opens in \textsanskrit{Uruvelā} with three discourses celebrating the Buddha’s awakening. It thus signals that the narrative of the \textsanskrit{Udāna} begins with the Buddha’s awakening. Now, the life story of the Buddha as a whole follows the three-fold pattern of the classical hero myth. The first part deals with the Buddha’s “origin story”: his life at home in Kapilavatthu, and the reason he left on his quest. The second part is the six years in the wilderness culminating in his realization of awakening. The third part begins with the awakening and tells of how the Buddha brought the fruits of his experience back into the world for the benefit of others. And this third part, the Return, is what the \textsanskrit{Udāna} is about.

This narrative arc finds its fullest expression in the portion of the Vinaya known as the Khandhakas. Indeed, the Khandhakas begin with a series of teachings that are found in the \textsanskrit{Udāna} as discourses 1–4 of the first chapter, and the first discourse of the second chapter. The Khandhakas wrap up their main narrative with the Buddha’s passing away, and along the way deal with some of the same events we find in the \textsanskrit{Udāna}, such as Devadatta’s betrayal. The overall theme of the two texts is the same: how is the radical experience of the Buddha’s awakening brought back into a world that has very different priorities? While the themes and some narrative beats are the same, the Khandhakas focus on legal procedures and good conduct, while the \textsanskrit{Udāna} is more concerned with inspiring people to take up the challenge of freedom.

Each of the texts in this chapter deals in one way or another with the idea of the “brahmin”, and thus can be seen in relation to the “Chapter on Brahmins” that concludes the Dhammapada. The brahmins were the most influential religious group in the Buddha’s time, posing a major challenge to the Buddha and his followers in establishing their new religion. The Buddha co-opted the term “brahmin” and redefined it, not as a hereditary caste, but as one who was spiritually awakened. 

In the first three discourses, this awakening is attributed to the Buddha’s realization of dependent origination. In brief, this set of twelve conditions shows how, due to ignorance, we make moral choices that propel our consciousness into future rebirth. Once born in a new life, we grow to become attached to our experiences of pleasure and pain, not understanding that this is how we got here in the first place. Craving kicks in again and the round continues. The discourses of the \textsanskrit{Nidāna} \textsanskrit{Saṁyutta} (SN 12, “Linked Discourses on Causation”) analyze this process from many angles, but here the focus is on the inspirational outcome of understanding (Ud 1.3):

\begin{quotation}%
When things become clear \\
to the keen, meditating brahmin, \\
he remains, scattering \textsanskrit{Māra}’s army, \\
as the sun lights up the sky.

%
\end{quotation}

The Buddha spent several weeks after his awakening simply sitting in meditation, enjoying the bliss of freedom. During this time he was approached by a “whiny” brahmin, occasioning an explanation of the real spiritual meaning of brahmin (Ud 1.4). From there, the chapter follows the theme rather than narrative.

Many mendicants are identified as brahmins, showing that it was not the Buddha alone who could find perfection (Ud 1.5). One of these is Venerable \textsanskrit{Mahākassapa}, who showed his noble nature by refusing to accept the alms from deities—the most elevated food imaginable—and instead visited the streets of the poor and destitute, giving them a chance to make merit (Ud 1.6). The rest of the chapter continues to illustrate the qualities of real and fake brahmins. 

There is a challenging episode when the mendicant Venerable \textsanskrit{Saṅgāmaji} is approached by his ex-wife with a child she says is his. She asks him to help look after her and the child. This is a call-back to the discourse with \textsanskrit{Mahākassapa}, which praises him as one who does not support a family. \textsanskrit{Saṅgāmaji} ignores his ex-wife’s pleading, and she leaves with the child. On the face of it, it’s a shocking story, where \textsanskrit{Saṅgāmaji}’s heartlessness is praised as equanimity. Yet the story is more subtle than it might seem. 

Near the end of the text, the ex-wife makes to depart, leaving the child behind; but when she sees that even this provokes no response, she returns for the child. Now, the text describes her behavior here as \textit{\textsanskrit{vippakāra}}, a somewhat obscure term that is usually translated following the commentary as “bad manners” or “misbehavior”. However, this word normally means “change, alteration” (Cone’s Dictionary of Pali), and in its only other occurrence in the canon it means simply a change in posture (\textsanskrit{Parivāra} 15:37.6). Here, therefore, it merely refers to the observation that she turned around and went back for the child. What that means is that neither here nor anywhere else in the text is she criticized for her actions.

We don’t know the circumstances of \textsanskrit{Saṅgāmaji}’s departure from his wife. What kind of marriage was it? Why did he leave? We do not know. The commentary fills in some details, but since these are mostly lifted from the famous story of \textsanskrit{Raṭṭhapāla}, it is difficult to feel too confident in their veracity. And even then, it does not discuss his former marriage or how he left home. We are left with a story of an ex-wife’s all-too-human need for her husband’s support, and a confirmed renunciate’s determination to continue on his path. The narrative highlights the tension between their responses, each determined by a different scale of value. 

The chapter concludes with a memorable combination of narrative and teaching, the story of \textsanskrit{Bāhiya}. The core of the Buddha’s teaching to \textsanskrit{Bāhiya} is enigmatic, yet has become famous in modern meditation circles:

\begin{quotation}%
In the seen will be merely the seen; in the heard will be merely the heard; in the thought will be merely the thought; in the known will be merely the known.

%
\end{quotation}

Here, as always in early texts, the third term \textit{muta} means “thought”, not “sensed” per the commentary. While it’s tempting to take this pithy teaching as a guide to meditation, it’s worth remembering that the narrative emphasizes that \textsanskrit{Bāhiya} had already lived as an ascetic for many years. Indeed, he regarded himself as already awakened. The passage appears in another sutta as well, where it is also given to an experienced mendicant (SN 35.95). \textsanskrit{Bāhiya}’s belief in his own awakening was a case of over-estimation, but his spiritual sincerity is testified by the fact that once he realized this he rightaway accepted the fact and rushed to find the real thing. And his innate spiritual potential is shown when he immediately realizes full awakening upon hearing the teaching.

Oddly, perhaps, the \textit{\textsanskrit{udāna}} here is not the central teaching, but rather the concluding verses given after \textsanskrit{Bāhiya}’s untimely death. Anticipating the final chapter of the \textsanskrit{Udāna}, they offer a solemn meditation on \textsanskrit{Nibbāna}.

\section*{Ud 2: With Mucalinda—spiritual and worldly happiness}

The second chapter picks up the narrative thread, and we find ourselves back in \textsanskrit{Uruvelā} (Ud 2.1). We have established the goal of Buddhism—freedom from worldly attachments—but may be left with the feeling that this is a cold and distant state; admirable, perhaps, but not all that enticing. The Buddha was well aware that his teaching could be intimidating, even terrifying. So, just as an ad for a meditation retreat would show a smiling meditator, he made sure to emphasize that his is a path of happiness. Happiness is already introduced in the first chapter, as the Buddha’s meditation is said to be the “bliss of freedom”.  That one idea becomes the seed for the second chapter.

It opens with an awe-inspiring event. As the Buddha meditates on the bliss of freedom, a great unseasonal rainstorm blows for seven days. As it happens, while I write this on the east coast of Australia, we are also experiencing a great and unseasonal period of rain, with floods threatening peoples lives and homes. We shelter in our comfortable homes, and protect ourselves with umbrellas and coats if we venture out even for a few minutes in the rain. Yet the Buddha sat for a whole week of storm and thunder, undisturbed in his meditation. He may not have had an umbrella, but something even better appeared: the spread hood of the giant dragon-king Mucalinda. Encircling him seven times, the dragon kept him safe.

This episode evokes a deep range of mythical and symbolic connotations: a sacred serpent of astonishing power rises from the earth and coils seven times around the person as it grows towards transcendence. One of the aims is to show the proper harmonious relation of the Buddha and the pre-Buddhist religious practices of worshiping the spirits of nature. The \textit{\textsanskrit{udāna}} with which the episode ends is at once distantly related to the events of the narrative, yet curiously appropriate. Inspired by the experience of the “bliss of freedom”, the Buddha gives a memorable teaching on the nature of happiness. What is significant about it is that it brings happiness back to ordinary things—simplicity and kindness—showing that these more relatable forms of happiness are what grow into the Buddha’s unfathomable bliss.

The narrative frame is once more left behind as the next sutta picks up the idea of spiritual bliss as greater than that of the world. The bliss of awakening outweighs that even of kings (Ud 2.2, Ud 2.10), while the cruelty of boys highlights the desire for happiness among all creatures (Ud 2.3). The search for lower happiness corrupts ascetics (Ud 2.4), prevents people from fully committing to the Dhamma (Ud 2.5), leads inevitably to grief (Ud 2.7), and subjects one to unwanted authority (Ud 2.9).

The chapter further explores the relationship of male ascetics with women and children. In one discourse, a non-Buddhist wanderer, presumably of a non-celibate order, has to find oil to ease his pregnant wife’s difficult delivery (Ud 2.6). He can only find it at a distribution center of the king; but there, oil is only given to be consumed on the spot, not to be taken away. An early commentary, perhaps, on the dire effect of bureaucratic rigidity on social welfare programs. Desperate, he consumes too much oil and is stricken with illness, unable to help himself, much less his wife. 

This tragic story gets its counterpart in the next discourse, where the proper way for mendicants to support women and children is shown (Ud 2.8). \textsanskrit{Suppavāsā} the Koliyan was suffering an extended and painful pregnancy and labor. The thought of the Buddha helped her endure, and when he learned of her travails, he gave her his blessing. That was enough for her to finally give birth to a healthy child. When she had recovered, she invited the Buddha with the Sangha for a meal, and they made special arrangements so that this could happen. \textsanskrit{Suppavāsā} was overjoyed to serve the Sangha and see her son with the monks, especially when Venerable \textsanskrit{Sāriputta} spoke with him and gave his blessings. This discourse heightens the drama with the miraculous detail of \textsanskrit{Suppavāsā}’s seven-year pregnancy, but the message is a practical one. A mendicant is not a breadwinner or a doctor; they have chosen a different path. But that doesn’t mean they don’t care. Their role is to offer spiritual and emotional support, and sometimes this is exactly what is needed.

\section*{Ud 3: With Nanda—a mendicant’s equanimity in the face of pleasure and pain}

The third chapter complicates the theme of happiness. While it is true that the Buddha’s path is one of happiness, it is also true that so long as we are here in the world our lives are affected also by suffering. The discourses in this chapter speak of “trembling not at pleasure and pain”, and extol the peaceful state of the mendicants who remain equanimous and unflustered in the face of adversity. The narrative does not directly continue from the scene of the awakening, but it does continue in the same general direction. In the Khandhakas, we see that the idealized and pure days of the early dispensation become increasingly complicated with the influx of new recruits, not all of them exemplars of purity. This chapter follows a similar course.

It begins, however, with a reminder of the power of the awakened mind. As any meditator will know, you don’t have to sit long to realize that your body is a constant source of pain. The Buddha’s teachings don’t focus on pain in meditation, and never extol it as a particularly useful or meaningful part of the path. But they also don’t deny it. It turns out, even an enlightened mendicant endures pain in meditation (Ud 3.1). This is attributed to their deeds in past lives. It’s important to note, however, that this doesn’t necessarily mean that they did something specific in the past that is causing this pain now. Rather, it is because we created kamma in the past that we have been reborn with these bodies in this life, and so long as these bodies persist, we shall experience pain.

The chapter is named after the second discourse, the famous episode of Nanda and the dove-footed nymphs (Ud 3.2). A close relative of the Buddha, Nanda, like so many of the Sakyan clan, struggled in the Sangha. Unable to stop thinking of his former sweetheart, he planned to disrobe. The Buddha dissuaded him, and Nanda was able to achieve freedom from pleasure and pain. This episode occurred after the Buddha returned to his family in the Sakyan realm, an event told in the first chapter of the Khandhakas. There too there is a tension and ambiguity in the going forth of some of the Sakyans, with the Buddha’s father lamenting the renunciation of his grandson, \textsanskrit{Rāhula}. The complicated tensions created by the presence of the Buddha’s family in the Sangha comes to a head later, in the betrayal of Devadatta.

As the Sangha grew there came to be lax monks making an unseemly racket. But when the Buddha dismissed them they were genuinely ashamed. They reformed and latter rejoined the Buddha in “imperturbable” meditation (Ud 3.3). Accomplished meditators sit like mountains (Ud 3.4, Ud 3.5). 

But even great monks sometimes have personality quirks; Venerable Pilindavaccha would sometimes slip into referring to his fellows as “lowlifes”. But the Buddha, seeing that this was due to conditioning from past lives,  urges the monks to not get upset by that (Ud 3.6). 

The austere monk \textsanskrit{Mahākassapa} is revealed to wander for alms indiscriminately, refusing offerings from the gods so the poor can make merit (Ud 3.7). By contrast, monks who wander for alms in search of pleasure are admonished to remain unstirred (Ud 3.8). 

The chapter ends by returning to \textsanskrit{Uruvelā} for a profound meditation on the nature of the world and escape from it. Elevating the theme of poise and balance, it posits that both holding on to and getting rid of continued existence keep one trapped. The chapter ends with the word \textit{\textsanskrit{tādi}}, a “poised one”, one who is “such”, unaffected by likes and dislikes. 

\section*{Ud 4: With Meghiya—controlling the mind}

The fourth chapter deals with the harm caused by an undisciplined mind, which affects even those who follow his path. It starts with the story of the truculent Meghiya. He was the Buddha’s attendant before Ānanda, so this must still be set fairly early in the Buddha’s career. Meghiya’s impulsive personality means he is benefitted by association and friendship rather than solitary retreat (Ud 4.1). The Buddha recommends a balanced set of four meditations: the perception of ugliness in the body to give up attachment to the body, meditation on love to give up hate, mindfulness of breathing to cut off thinking, and perception of impermanence to uproot the conceit “I am”.

Restless mendicants are also taught calm and restraint in the next sutta (Ud 4.2). The danger of an unrestrained mind is further emphasized in the story of a devoted lay follower suddenly killed (Ud 4.3) while wandering in the wilds between the villages. Restlessness is also to the fore in the tragically amusing story of the wanton behavior of certain native spirits (Ud 4.4). 

The misbehaving Sangha at Kosambi causes the Buddha to abandon them (Ud 4.5). This story is told in more detail in the Vinaya at the tenth Khandhaka (“At \textsanskrit{Kosambī}”), and so here at the middle of the \textsanskrit{Udāna} we are roughly keeping pace with the middle of the Khandhakas. 

Lack of self-control also leads some non-Buddhist ascetics to launch a vile campaign against the Buddha. They manipulate a non-Buddhist nun to act so as to arouse suspicion regarding the Buddha’s intentions, only to brutally murder her and attempt to pin it on the Buddha (Ud 4.8). 

Great mendicants celebrate in peace, having tamed these excesses (Ud 4.6, Ud 4.7, Ud 4.9, Ud 4.10).

\section*{Ud 5: \textsanskrit{Soṇa}—all}

The fifth chapter widens the scope, emphasizing the universal quality of the Dhamma with the key word \textit{sabba} “all”. As the dispensation grows and matures it emphasizes the inclusivenss of the Dhamma, how its struggles affect all without limit. Thus even the greatest of kings realizes his beloved queen, like all beings, loves themselves more than him (Ud 5.1). 

Likewise, even the Buddha’s own mother passed away shortly after he was born, a fate that all beings, even the Buddha, must face (Ud 5.2). 

The story of Suppabuddha shows that even one shunned by society as he was for his leprosy main attain the Dhamma; he also appears several times in the Dhammapada.

 Cruel boys tormenting fish are led to understand that they share a fear of pain with all creatures (Ud 5.4). 

 While the Sangha is universal and inclusive like the ocean, it does not tolerate bad behavior. The presence of an ill-conducted monk prompts the Buddha to establish of a formal legal proceeding, the Sabbath (\textit{uposatha}) to unify the Sangha (Ud 5.5). The same development is told rather less dramatically in the second chapter of the Khandhakas, where the Sabbath is first established on the pattern of pre-existing practices, and then the Buddha decides to ask the mendicants to recite the monastic code. The primary purpose of the Sabbath is to bring together all (\textit{sabbeheva}) of the mendicants living within a monastic boundary (\textit{\textsanskrit{sīmā}}). 

 The efficacy of this is tested by Devadatta (Ud 5.8), who tries to split the Sangha by performing his own Sabbath. Meanwhile, the geographical spread of the teaching highlights the question of how to organize ordination in distant lands. \textsanskrit{Soṇa} from \textsanskrit{Avantī} can only ordain after great efforts, a story that, once again, finds its echo in the Khandhakas (Pli Tv Kd 5, Ud 5.6). 

 The problem with all-inclusiveness is what to do when some members of the community do not behave well. Thus a sub-theme of the chapter is “evil”. A wise person would shun evil (Ud 5.3), which will create suffering though all you want is happiness (Ud 5.4), which is why the Buddha rejects evil (Ud 5.6), while for one used to bad deeds it is easy to continue on the paths of evil (Ud 5.8).

\section*{Ud 6: Blind From Birth—ascetics caught in views fail to see}

The sixth chapter does not open by continuing the linear narrative, but skips forward to with an episode near the end of the Buddha’s life, as, knowing that the end is near, he relinquishes his life force. This passage is shared with the \textsanskrit{Mahāparinibbāna} Sutta (DN 16). 

The remainder of the chapter is unified by the theme of losing the way, like moths in the flame of a lamp (Ud 6.9). Those who have fixed “views” (\textit{\textsanskrit{diṭṭhi}}) are trapped in their preconceptions and cannot see the path.

There was a pair of gangs who were fighting over a courtesan, lost in their lust for a woman they will never attain (Ud 6.8). This discourse is capped by a powerful and enigmatic \textit{\textsanskrit{udāna}} in prose, where the Buddha compared them to those who fall into the two extremes, of regarding the observances as the sessence, or of indulging in sensual pleasures. While there are many different expressions of the two “extremes” found in the Suttas, this one is especially reminiscent of the famous first discourse, which was given for non-Buddhist ascetics.

The failings of such “ascetics” are spelled out in many of the remaining discourses of this chapter. Sometimes they are outright frauds; spies, in fact, doing the king’s bidding (Ud 6.2). The idea that ascetics might act as spies is one with a long history. Kautilya’s \textsanskrit{Arthaśāstra} recommends that monks and nuns be recruited as spies, and rumors of this practice continue to the present day in Buddhist lands. This discourse ends with the pointed reminder that one should not trade in the Dhamma. The Dhamma is not something to be bought and sold for profit in the marketplace.

Other ascetics are holders of wrong views (Ud 6.4, Ud 6.5, Ud 6.6). They  miss the point of the spiritual life (Ud 6.9), so that when the Buddha appears their glory fades away (Ud 6.10), like glow-worms when the sun rises.

\section*{Ud 7: The Lesser Chapter—the floods of the world and the clear water of the Dhamma}

As if preparing for the end, the seventh chapter returns to the opening theme of Awakening, celebrating the enlightenment of the Buddha and others. The chapter as a whole serves to recollect and reinforce the efficacy of the Dhamma, from the beginning to the end, showing us that despite the overwhelming strength of the world’s passions, the Dhamma can steer us to the other side. The dominant image is that of water, whether the floods crossed (Ud 7.1, Ud 7.3), the streams dried or cut (Ud 7.2, Ud 7.9), the fish trapped (Ud 7.4), or the well cleared.

It begins with the monk Bhaddiya the Dwarf, who becomes awakened (Ud 7.1), but is not recognized as such (Ud 7.2) due to his unsightly appearance (Ud 7.5). 

Despite the Buddha’s long and mostly pleasant residence near \textsanskrit{Sāvatthī}, a couple of discourses lament the power that attachment still holds over the minds of the people (Ud 7.3, Ud 7.4). It’s not clear what prompted this; perhaps a festival of some sort. But it reminds us of how the Buddha, at the start of his ministry, was reluctant to teach due to the power of defilements.

The Buddha, of course, changed his mind, and subsequent discourses drive home the efficacy of the path that is able to overcome even such deep-rooted defilements. We catch a rare glimpse of Venerable \textsanskrit{Añña} \textsanskrit{Koṇḍañña}, the first disciple to realize the four noble truths, all those years ago in Benares. Here he is sitting in meditation at \textsanskrit{Sāvatthī}, having uprooted all grounds for defilements, living proof of the Dhamma’s power. Similar discourses celebrate the Buddha himself, as well as Venerable \textsanskrit{Mahākaccāna}, who is praised for his mindfulness of the body.

The last couple of discourses echo, or perhaps anticipate, themes found in the \textsanskrit{Mahāparinibbānasutta}. (Ud 7.9) depicts the increasingly petty attempts of non-Buddhist ascetics to obstruct him. They soil a well that the Buddha wished to drink from, only for it to become miraculously clear, a passage anticipating the incident with ox carts soiling the \textsanskrit{Kakutthā} river (DN 16:4.22.1), which in fact appears in the next chapter of the \textsanskrit{Udāna} (Ud 8.5). 

Then the tragic deaths by fire of five hundred harem women led by Queen \textsanskrit{Sāmāvatī} of Kosambi (Ud 7.10) prompts an inquiry as to their rebirth, echoing the similar questions of the lay folk at \textsanskrit{Nādika} (DN 16:2.5.1). 

\section*{Ud 8: \textsanskrit{Nibbāna}—the final extinguishment of existence}

The theme of death, and the connection with the \textsanskrit{Mahāparinibbānasutta}, dominates the final chapter. 

The first four discourses consist of a series of solemn declarations regarding the nature of \textsanskrit{Nibbāna}. They are not directly part of the \textsanskrit{Mahāparinibbāna} narrative, but they expand on the theme of what happens at the end end of life for an enlightened one. 

These discourses have long exercised the Buddhist imagination, as they consist of some of the Buddha’s most explicit and evocative statements on this famously enigmatic subject. The word \textit{\textsanskrit{nibbāna}} refers to the extinguishment of a flame, and most of the Buddha’s staments on the matter reflect this quality of ending, of the passing away of suffering. It is naturally exciting, then, to see the Buddha so whole-heartedly affirming the reality of “that dimension” (Ud 8.1) that is “uninclined” (Ud 8.2), yet which most assuredly “is” (Ud 8.3) without being caught in “coming and going” (Ud 8.4). 

It is tempting to see here a hint of \textsanskrit{Nibbāna} as an eternal state of transcendence; subtle beyond subtlety, of course, but nonetheless, \emph{something}. A closer look reveals, however, that the Buddha was not so swift to overturn everything else he said on the subject. The stronger the positive affirmation of \textsanskrit{Nibbāna}, the clearer it becomes that what is affirmed is a series of negations: “\emph{there is} an \emph{un}born, \emph{un}produced, \emph{un}made, \emph{un}conditioned”. There is no affirmatiopn of a eternal transcendent state; rather, an affirmation of the end of all the cycles of worldly suffering.

In the discourses on Cunda (Ud 8.5) and the villagers of \textsanskrit{Pāṭali} (Ud 8.6) we return to passages that are shared with the \textsanskrit{Mahāparinibbānasutta} (DN 16:4.13.1, DN 16:1.19.1). Though his own demise was imminent, the Buddha exhibits his care for others, dispelling any remorse that Cunda may feel for serving the Buddha’s last meal, and reaffirming the power of good deeds. And in the episode at \textsanskrit{Pāṭali}, the Buddha also reaffirms the value of the fundamental virtue of giving, keeping his connection with the simple good things of the world, even as he is about to “cross over”, a journey that is symbolically foreshadowed in the crossing of the Ganges. 

A couple of discourses remind us of how easy it is, even for those close to the Buddha, to forget the depth of his wisdom. The monk \textsanskrit{Nāgasamāla} ignores the Buddha’s directions and heads down the wrong path, to dire results. Death is also the theme of the lamentation of \textsanskrit{Visākhā} (Ud 8.8), whose desire for children traps her in grief. 

While these discourses again do not directly relate to the \textsanskrit{Mahāparinibbāna} narrative, they echo an important theme: the crisis of faith that arises at the passing of the Buddha. Depictions of the Buddha’s passing typically contrast the serene equanimity of the fully-awakened with the grievous lament of the rest of us. When learning of the Buddha’s demise, the great disciple \textsanskrit{Mahākappasa} perceived right away that some mendicants would take advantage of the absence of their Teacher to go their own way, choosing the wronmg fork in the road. Hence he set about establishing the Councils that would reaffirm the \textsanskrit{Saṅgha}’s committtment to maintaining the Buddha’s teaching for the future.

The \textsanskrit{Udāna} culminates with the spectacular passing of Dabba, who, knowing his life was at an end, immolated himself in the middle of the Sangha using only the power of the meditation on fire (Ud 8.9, Ud 8.10). Here the utterly remainderless nature of \textsanskrit{Nibbāna} is stressed, as if to remove any ambiguity the opening discourses of the chapter may have invited. 

\section*{Narrative Throughlines}

We have seen how the life of the Buddha informs the structure of the \textsanskrit{Udāna}, providing a firm if flexible template. We’ve also seen how the text creates unity by constantly referring back to the “inspirational” experiences of awakening by the Buddha and others. But there is yet another literary technique that is used by the redactors to create unity and a forward motion in the text. 

In multiple cases we find the text refers back to the same ideas or themes. And when it does so, it rather consistently \emph{advances} the theme each time. Thus by picking out two or three discourses through the \textsanskrit{Udāna} on a specific theme we get a progressive learning on that topic. I’ve already pointed out how in the final chapter, the last discourses on \textsanskrit{Nibbāna} seem to act as a rebalancing for the opening discourses; they are meant to be read as a whole. We have met some similar cases along the way. Here are a few examples, and there surely will be more.

\begin{itemize}%
\item The cruelty of boys: At first the Buddha merely observes and comments on their behavior (Ud 2.3). Later, he engages and persuades them (Ud 5.4).%
\item Kings: Monks discuss kings in an envious and worldly way (Ud 2.2); then the drawbacks of a king’s power are shown (Ud 2.9); then a king renounces the throne (Ud 2.10).\begin{itemize}%
\item Native spirits (\textit{yakkhas}): A gross goblin gets his kicks by annoying the Buddha (Ud 1.7). Later, two spirits debate the virtue of ascetics, showing their intelligence and the diverse nature of the \textit{yakkha} community (Ud 4.4).%
\end{itemize}

%
\item Ascetics and children: An unwanted child is coldly rejected (Ud 1.8); then an ascetic unwisely attempts to act as a medical doctor to his demise (Ud 2.6); finally the Buddha and his disciples compassionately offer spiritual support for the mother and child (Ud 2.8).%
\item Kassapa’s almsround: In Ud 1.6  Venerable \textsanskrit{Mahākassapa}, recovering from illness, goes for alms to the poor, refusing the offerings of the deities. In Ud 3.7  we have a similar story, but raised up and exalted: here Kassapa is emerging from deep meditation; and his encounter with the deities is told in more detail, with a personal offering from Sakka himself in disguise.%
\end{itemize}

In some cases, the way the themes are developed show signs of similarity to one another. Consider, for example, the discourses on boys and kings. In the second chapter, the Buddha is merely a passive observer, and we hear of these people second-hand. He doesn’t engage with them directly. Perhaps this is an early stage in the dispensation, where he has little recognition and influence. In chapter 5 we see him speaking with the boys. Similarly, in chapter 4 we seem him becoming acquainted with kings, to the extent that he had to leave society to find solitude (Ud 4.5), and got caught up in a criminal investigation (Ud 4.8). In these cases, however, we still don’t see him speaking with kings. But by chapter 5, King Pasenadi is comfortable enough with the Buddha to discuss his domestic conversation (Ud 5.1), and eventually he is trusted enough that the king would reveal state secrets (Ud 6.2). 

If we take these discourses individually, seeing them as a semi-random assemblage of teachings on diverse themes, we overlook the careful and often interesting work of the redactors. They did not put everything on the surface. Just as, in life, we learn lessons gradually, from one event and then perhaps another event much later, in the \textsanskrit{Udāna} we find hidden connections that create a greater whole out of the parts.

\section*{A Brief Textual History}

A printed edition of the \textsanskrit{Udāna} was published by the Pali Text Society in 1885. It was edited by Paul Steinthal from one Burmese and two Sinhalese manuscripts. The provenance of the manuscripts is not given in detail, beyond that one of the Sinhala-script manuscripts was gifted to the PTS by the Thera \textsanskrit{Sūriyagoda} Sonuttara of Kandy, while the other was made for T.W. Rhys Davids at Kaluttara, presumably while he was staying in Sri Lanka. It’s worth remembering that all our modern Pali editions stem from manuscripts that were kindly and freely offered by Buddhists for international scholarship, despite the fact that the Buddhists were at the time under the colonial yoke. 

Steinthal regards the Burmese manuscript as the most accurate of the three, and further argues that the third one—prepared at Kaluttara in Sri Lanka—appears to share a common heritage with the Burmese manuscript. The Kandy manuscript appears relatively independent.

In the forewords to their translations, both Ireland and Masefield note the poor quality of the PTS edition. I have mostly ignored it, and as usual rely on the well-edited edition of the \textsanskrit{Mahāsaṅgīti} text.

As one might expect from such a short and engaging text, there have been several translations into English. 

Major-General D.M. Strong holds the honor of making the first English translation, published by Luzac \& Co. in 1902 under the title \textit{The Solemn Utterances of the Buddha}. Described in his obituary as a “worthy old soldier” whose entire family was “interested in music, art, religion, and science”, he died only a year after publishing his translation, leaving further translations unpublished. His work, while obviously superseded by later translations, is still quite readable and remains credible.

The Pali Text Society’s first translation was that of F.L. Woodward in 1935. It was published as \textit{\textsanskrit{Udāna}: Verses of Uplift} together with \textit{Itivuttaka: As It Was Said} under the collective title \textit{The Minor Anthologies of the Pali Canon Part II}. It was made in Tasmania following his retirement as principal of Mahinda College in Galle, Sri Lanka. At Rowella on the Tamar River near Launceston he lived a solitary life of contemplation, scholarship, and vegetarianism. His draft translations were sent by slow boat all the way to Oxford for review by C.A.F Rhys Davids. His devotion to the decidedly unprofitable field of Pali translation cost him his career and his wealth, and this decorated Cambridge graduate would sometimes become so cold he had to line his trousers with newspaper. 

The first of what one might call the modern translations is that of John Ireland, originally published by the Buddhist Publication Society in 1990, and later, following the precedent of the PTS edition, reprinted together with the Itivuttaka as “Two Classics From the Pali canon”. Ireland’s work draws strongly on the advances in Pali translation made by Bhikkhu \textsanskrit{Ñāṇamoḷi} and others, and employs a more rigorous and accurate treatment of terminology, while remaining readable. 

Peter Masefield’s 1997 translation with the PTS, simply titled \textit{The \textsanskrit{Udāna}} serves primarily as a companion to his translation of the commentary, and he has by and large stuck closely to commentarial readings. His work was one of the first translations of a complete Pali commentary.

Venerables Ānandajoti and \textsanskrit{Ṭhānissaro} have also both translated the text. Ānandajoti’s version, published in 2008 as \textit{Exalted Utterances}, was based on the Buddha Jayanthi edition. As always it is highly literal and accurate, informed by his unsurpassed knowledge of verse forms. It was my first port of call when I needed help. \textsanskrit{Ṭhānissaro}’s version \textit{\textsanskrit{Udāna}: Exclamations} was published soon after in 2012, relying primarily on the Thai edition.

Finally, Anagarika Mahendra has published an annotated bilingual edition \textit{\textsanskrit{Udāna}: Book of Inspired Utterances} in 2022 with Dhamma Publishers. This is part of his much longer series of translations, and is based on the digital text of the \textsanskrit{Chaṭṭha} \textsanskrit{Saṅgāyana} \textsanskrit{Tipiṭaka} (VRI). It is a literal translation, intended to help the student reading the Pali.

I respect and honor all of these editors and translators who have done so much to bring this ancient scripture to modern light, and without whose work my own would not be possible.

%
\chapter*{Acknowledgements}
\addcontentsline{toc}{chapter}{Acknowledgements}
\markboth{Acknowledgements}{Acknowledgements}

I remember with gratitude all those from whom I have learned the Dhamma, especially Ajahn Brahm and Bhikkhu Bodhi, the two monks who more than anyone else showed me the depth, meaning, and practical value of the Suttas.

Special thanks to Dustin and Keiko Cheah and family, who sponsored my stay in Qi Mei while I made this translation.

Thanks also for Blake Walshe, who provided essential software support for my translation work.

Throughout the process of translation, I have frequently sought feedback and suggestions from the community on the SuttaCentral community on our forum, “Discuss and Discover”. I want to thank all those who have made suggestions and contributed to my understanding, as well as to the moderators who have made the forum possible. A special thanks is due to \textsanskrit{Sabbamittā}, a true friend of all, who has tirelessly and precisely checked my work.

Finally my everlasting thanks to all those people, far too many to mention, who have supported SuttaCentral, and those who have supported my life as a monastic. None of this would be possible without you.

%
\mainmatter%
\pagestyle{fancy}%
\addtocontents{toc}{\let\protect\contentsline\protect\nopagecontentsline}
\part*{Heartfelt Sayings}
\addcontentsline{toc}{part}{Heartfelt Sayings}
\markboth{}{}
\addtocontents{toc}{\let\protect\contentsline\protect\oldcontentsline}

%
\addtocontents{toc}{\let\protect\contentsline\protect\nopagecontentsline}
\chapter*{The Chapter on Awakening }
\addcontentsline{toc}{chapter}{\tocchapterline{The Chapter on Awakening }}
\addtocontents{toc}{\let\protect\contentsline\protect\oldcontentsline}

%
\section*{{\suttatitleacronym Ud 1.1}{\suttatitletranslation Upon Awakening (1st) }{\suttatitleroot Paṭhamabodhisutta}}
\addcontentsline{toc}{section}{\tocacronym{Ud 1.1} \toctranslation{Upon Awakening (1st) } \tocroot{Paṭhamabodhisutta}}
\markboth{Upon Awakening (1st) }{Paṭhamabodhisutta}
\extramarks{Ud 1.1}{Ud 1.1}

\scevam{So\marginnote{1.1} I have heard. }At one time, when he was first awakened, the Buddha was staying near \textsanskrit{Uruvelā} at the root of the tree of awakening on the bank of the \textsanskrit{Nerañjarā} River. There the Buddha sat cross-legged for seven days without moving, experiencing the bliss of freedom. When seven days had passed, the Buddha emerged from that state of immersion. In the first part of the night, he reflected on dependent origination in forward order: 

“When\marginnote{2.1} this exists, that is; due to the arising of this, that arises. That is: Ignorance is a condition for choices. Choices are a condition for consciousness. Consciousness is a condition for name and form. Name and form are conditions for the six sense fields. The six sense fields are conditions for contact. Contact is a condition for feeling. Feeling is a condition for craving. Craving is a condition for grasping. Grasping is a condition for continued existence. Continued existence is a condition for rebirth. Rebirth is a condition for old age and death, sorrow, lamentation, pain, sadness, and distress to come to be. That is how this entire mass of suffering originates.” 

Then,\marginnote{3.1} understanding this matter, on that occasion the Buddha expressed this heartfelt sentiment: 

\begin{verse}%
“When\marginnote{4.1} things become clear \\
to the keen, meditating brahmin, \\
his doubts are dispelled, \\
since he understands each thing and its cause.” 

%
\end{verse}

%
\section*{{\suttatitleacronym Ud 1.2}{\suttatitletranslation Upon Awakening (2nd) }{\suttatitleroot Dutiyabodhisutta}}
\addcontentsline{toc}{section}{\tocacronym{Ud 1.2} \toctranslation{Upon Awakening (2nd) } \tocroot{Dutiyabodhisutta}}
\markboth{Upon Awakening (2nd) }{Dutiyabodhisutta}
\extramarks{Ud 1.2}{Ud 1.2}

\scevam{So\marginnote{1.1} I have heard. }At one time, when he was first awakened, the Buddha was staying near \textsanskrit{Uruvelā} at the root of the tree of awakening on the bank of the \textsanskrit{Nerañjarā} River. There the Buddha sat cross-legged for seven days without moving, experiencing the bliss of freedom. When seven days had passed, the Buddha emerged from that state of immersion. In the second part of the night, he reflected on dependent origination in reverse order: 

“When\marginnote{2.1} this doesn’t exist, that is not; due to the cessation of this, that ceases. That is: When ignorance ceases, choices cease. When choices cease, consciousness ceases. When consciousness ceases, name and form cease. When name and form cease, the six sense fields cease. When the six sense fields cease, contact ceases. When contact ceases, feeling ceases. When feeling ceases, craving ceases. When craving ceases, grasping ceases. When grasping ceases, continued existence ceases. When continued existence ceases, rebirth ceases. When rebirth ceases, old age and death, sorrow, lamentation, pain, sadness, and distress cease. That is how this entire mass of suffering ceases.” 

Then,\marginnote{3.1} understanding this matter, on that occasion the Buddha expressed this heartfelt sentiment: 

\begin{verse}%
“When\marginnote{4.1} things become clear \\
to the keen, meditating brahmin, \\
his doubts are dispelled, \\
since he’s known the end of conditions.” 

%
\end{verse}

%
\section*{{\suttatitleacronym Ud 1.3}{\suttatitletranslation Upon Awakening (3rd) }{\suttatitleroot Tatiyabodhisutta}}
\addcontentsline{toc}{section}{\tocacronym{Ud 1.3} \toctranslation{Upon Awakening (3rd) } \tocroot{Tatiyabodhisutta}}
\markboth{Upon Awakening (3rd) }{Tatiyabodhisutta}
\extramarks{Ud 1.3}{Ud 1.3}

\scevam{So\marginnote{1.1} I have heard. }At one time, when he was first awakened, the Buddha was staying near \textsanskrit{Uruvelā} at the root of the tree of awakening on the bank of the \textsanskrit{Nerañjarā} River. There the Buddha sat cross-legged for seven days without moving, experiencing the bliss of freedom. When seven days had passed, the Buddha emerged from that state of immersion. In the last part of the night, he reflected on dependent origination in forward and reverse order: 

“When\marginnote{2.1} this exists, that is; due to the arising of this, that arises. When this doesn’t exist, that is not; due to the cessation of this, that ceases. That is: Ignorance is a condition for choices. Choices are a condition for consciousness. Consciousness is a condition for name and form. Name and form are conditions for the six sense fields. The six sense fields are conditions for contact. Contact is a condition for feeling. Feeling is a condition for craving. Craving is a condition for grasping. Grasping is a condition for continued existence. Continued existence is a condition for rebirth. Rebirth is a condition for old age and death, sorrow, lamentation, pain, sadness, and distress to come to be. That is how this entire mass of suffering originates. 

When\marginnote{3.1} ignorance fades away and ceases with nothing left over, choices cease. When choices cease, consciousness ceases. When consciousness ceases, name and form cease. When name and form cease, the six sense fields cease. When the six sense fields cease, contact ceases. When contact ceases, feeling ceases. When feeling ceases, craving ceases. When craving ceases, grasping ceases. When grasping ceases, continued existence ceases. When continued existence ceases, rebirth ceases. When rebirth ceases, old age and death, sorrow, lamentation, pain, sadness, and distress cease. That is how this entire mass of suffering ceases.” 

Then,\marginnote{4.1} understanding this matter, on that occasion the Buddha expressed this heartfelt sentiment: 

\begin{verse}%
“When\marginnote{5.1} things become clear \\
to the keen, meditating brahmin, \\
he remains, scattering \textsanskrit{Māra}’s army, \\
as the sun lights up the sky.” 

%
\end{verse}

%
\section*{{\suttatitleacronym Ud 1.4}{\suttatitletranslation Whiny }{\suttatitleroot Huṁhuṅkasutta}}
\addcontentsline{toc}{section}{\tocacronym{Ud 1.4} \toctranslation{Whiny } \tocroot{Huṁhuṅkasutta}}
\markboth{Whiny }{Huṁhuṅkasutta}
\extramarks{Ud 1.4}{Ud 1.4}

\scevam{So\marginnote{1.1} I have heard. }At one time, when he was first awakened, the Buddha was staying near \textsanskrit{Uruvelā} at the goatherd’s banyan tree on the bank of the \textsanskrit{Nerañjarā} River. There the Buddha sat cross-legged for seven days without moving, experiencing the bliss of freedom. When seven days had passed, the Buddha emerged from that state of immersion. 

Then\marginnote{2.1} a certain brahmin of the whiny sort went up to the Buddha and exchanged greetings with him. When the greetings and polite conversation were over, he stood to one side, and said, “Master Gotama, how do you define a brahmin? And what are the things that make one a brahmin?” 

Then,\marginnote{3.1} understanding this matter, on that occasion the Buddha expressed this heartfelt sentiment: 

\begin{verse}%
“A\marginnote{4.1} brahmin who has banished bad qualities, \\
—not whiny, not stained, but self-controlled, \\
a complete knowledge master who has completed the spiritual journey—\\
may rightly proclaim the brahmin doctrine, \\
not proud of anything in the world.” 

%
\end{verse}

%
\section*{{\suttatitleacronym Ud 1.5}{\suttatitletranslation The Brahmin }{\suttatitleroot Brāhmaṇasutta}}
\addcontentsline{toc}{section}{\tocacronym{Ud 1.5} \toctranslation{The Brahmin } \tocroot{Brāhmaṇasutta}}
\markboth{The Brahmin }{Brāhmaṇasutta}
\extramarks{Ud 1.5}{Ud 1.5}

\scevam{So\marginnote{1.1} I have heard. }At one time the Buddha was staying near \textsanskrit{Sāvatthī} in Jeta’s Grove, \textsanskrit{Anāthapiṇḍika}’s monastery. Now at that time a number of senior monks approached the Buddha—Venerables \textsanskrit{Sāriputta}, \textsanskrit{Mahāmoggallāna}, \textsanskrit{Mahākassapa}, \textsanskrit{Mahākaccāna}, \textsanskrit{Mahākoṭṭhita}, \textsanskrit{Mahākappina}, \textsanskrit{Mahācunda}, Anuruddha, Revata, and Nanda. 

The\marginnote{2.1} Buddha saw them coming off in the distance, and addressed the mendicants: “These, mendicants, are brahmins coming! These are brahmins coming!” When he said this, a certain mendicant of the brahmin caste asked the Buddha, “Sir, how do you define a brahmin? And what are the things that make one a brahmin?” 

Then,\marginnote{3.1} understanding this matter, on that occasion the Buddha expressed this heartfelt sentiment: 

\begin{verse}%
“Having\marginnote{4.1} banished bad qualities, \\
those who live always mindful, \\
with fetters ended, awakened, \\
they are the world’s true brahmins.” 

%
\end{verse}

%
\section*{{\suttatitleacronym Ud 1.6}{\suttatitletranslation With Mahākassapa }{\suttatitleroot Mahākassapasutta}}
\addcontentsline{toc}{section}{\tocacronym{Ud 1.6} \toctranslation{With Mahākassapa } \tocroot{Mahākassapasutta}}
\markboth{With Mahākassapa }{Mahākassapasutta}
\extramarks{Ud 1.6}{Ud 1.6}

\scevam{So\marginnote{1.1} I have heard. }At one time the Buddha was staying near \textsanskrit{Rājagaha}, in the Bamboo Grove, the squirrels’ feeding ground. Now at that time Venerable \textsanskrit{Mahākassapa} was staying in the Pipphali cave, and he was sick, suffering, gravely ill. Then after some time he recovered from that illness. It occurred to him, “Why not enter \textsanskrit{Rājagaha} for almsfood?” 

Now\marginnote{2.1} at that time five hundred deities were ready and eager for the chance to offer alms to \textsanskrit{Mahākassapa}. But \textsanskrit{Mahākasspa} refused those deities. In the morning, he robed up, took his bowl and robe, and entered \textsanskrit{Rājagaha} for alms. He went to the streets of the poor, the destitute, and the weavers. The Buddha saw him wandering for alms in the streets of the poor, the destitute, and the weavers. 

Then,\marginnote{3.1} understanding this matter, on that occasion the Buddha expressed this heartfelt sentiment: 

\begin{verse}%
“The\marginnote{4.1} stranger, providing for no other,\footnote{Compare ud1.8. } \\
tamed, grounded in the essence, \\
with defilements ended and flaws purged: \\
that’s who I call a brahmin.” 

%
\end{verse}

%
\section*{{\suttatitleacronym Ud 1.7}{\suttatitletranslation At Ajakalāpaka }{\suttatitleroot Ajakalāpakasutta}}
\addcontentsline{toc}{section}{\tocacronym{Ud 1.7} \toctranslation{At Ajakalāpaka } \tocroot{Ajakalāpakasutta}}
\markboth{At Ajakalāpaka }{Ajakalāpakasutta}
\extramarks{Ud 1.7}{Ud 1.7}

\scevam{So\marginnote{1.1} I have heard. }At one time the Buddha was staying in \textsanskrit{Pāvā} at the \textsanskrit{Ajakalāpaka} Tree-shrine, the haunt of the native spirit \textsanskrit{Ajakalāpaka}. Now at that time the Buddha was meditating in the open during the dark of night, while a gentle rain drizzled down. Then \textsanskrit{Ajakalāpaka}, wanting to make the Buddha feel fear, terror, and goosebumps, approached him, and while not far away yelled three times, “Grr! Arrgh!” Then he added, “Now that’s a goblin for you, ascetic!” 

Then,\marginnote{2.1} understanding this matter, on that occasion the Buddha expressed this heartfelt sentiment: 

\begin{verse}%
“When\marginnote{3.1} a brahmin has got over \\
their own issues, \\
they transcend this goblin \\
and his grunts.” 

%
\end{verse}

%
\section*{{\suttatitleacronym Ud 1.8}{\suttatitletranslation With Saṅgāmaji }{\suttatitleroot Saṅgāmajisutta}}
\addcontentsline{toc}{section}{\tocacronym{Ud 1.8} \toctranslation{With Saṅgāmaji } \tocroot{Saṅgāmajisutta}}
\markboth{With Saṅgāmaji }{Saṅgāmajisutta}
\extramarks{Ud 1.8}{Ud 1.8}

\scevam{So\marginnote{1.1} I have heard. }At one time the Buddha was staying near \textsanskrit{Sāvatthī} in Jeta’s Grove, \textsanskrit{Anāthapiṇḍika}’s monastery. Now at that time around Venerable \textsanskrit{Saṅgāmaji} had arrived at \textsanskrit{Sāvatthī} to see the Buddha. His former wife heard that he had arrived, and went to the Jetavana, taking their boy. 

Now\marginnote{2.1} at that time Venerable \textsanskrit{Saṅgāmaji} was sitting at the root of a tree for the day’s meditation. Then his former wife went up to him and said, “I have a little child, ascetic, so please provide for me.” When she said this, \textsanskrit{Saṅgāmaji} kept silent. 

For\marginnote{3.1} a second time she said, “I have a little child, ascetic, so please provide for me.” For a second time, \textsanskrit{Saṅgāmaji} kept silent. 

For\marginnote{4.1} a third time she said, “I have a little child, ascetic, so please provide for me.” For a third time, \textsanskrit{Saṅgāmaji} kept silent. 

Then\marginnote{5.1} she put down the boy in front of \textsanskrit{Saṅgāmaji}, saying, “This is your child, ascetic. Provide for him.” 

But\marginnote{6.1} \textsanskrit{Saṅgāmaji} neither looked at the boy nor spoke to him. Then his former wife went a little distance away. Looking back, she saw \textsanskrit{Saṅgāmaji} ignoring the boy, and thought, “This ascetic doesn’t even want his child.”\footnote{\textit{Atthiko} is tricky to translate due to the vagueness of the base term. Ānandajoti has “need”, while Thanissaro has “care about”. But it is usually used in the sense of “want”, as a synonym of \textit{\textsanskrit{gavesī}}, etc., or one who desires a kingdom. } She returned to pick up the boy, then left. With clairvoyance that is purified and superhuman, the Buddha saw how \textsanskrit{Saṅgāmaji}’s former wife went back for the child.\footnote{For \textit{\textsanskrit{vippakāraṁ}}, Ānandajoti has “bad manners”, while Thanissaro has “misbehaving”. Neither of these seems justified. PTS dictionary gives only commentarial citations for this word, but it also occurs at pli-tv-pvr15:37.7, where it is defined as changes in posture. It seems the Buddha merely observed that the former wife had changed her mind and gone back. Otherwise, the sutta passes no judgment on her behavior. } 

Then,\marginnote{7.1} understanding this matter, on that occasion the Buddha expressed this heartfelt sentiment: 

\begin{verse}%
“When\marginnote{8.1} she came he was not glad, \\
when she left he did not grieve. \\
Victorious in battle, freed from chains, \\
that’s who I call a brahmin.” 

%
\end{verse}

%
\section*{{\suttatitleacronym Ud 1.9}{\suttatitletranslation Dreadlocked Ascetics }{\suttatitleroot Jaṭilasutta}}
\addcontentsline{toc}{section}{\tocacronym{Ud 1.9} \toctranslation{Dreadlocked Ascetics } \tocroot{Jaṭilasutta}}
\markboth{Dreadlocked Ascetics }{Jaṭilasutta}
\extramarks{Ud 1.9}{Ud 1.9}

\scevam{So\marginnote{1.1} I have heard. }At one time the Buddha was staying near \textsanskrit{Gayā} on \textsanskrit{Gayā} Head. Now at that time—during the cold spell when the snow falls in the dead of winter—many dreadlocked ascetics plunged in and out of the \textsanskrit{Gayā} river. Plunging and showering, they served the sacred flame, thinking: “Through this there is purity.” 

The\marginnote{2.1} Buddha saw them plunging in and out. 

Then,\marginnote{3.1} understanding this matter, on that occasion the Buddha expressed this heartfelt sentiment: 

\begin{verse}%
“Purity\marginnote{4.1} doesn’t come from water, \\
no matter how many people bathe there. \\
One who has truth and principle, \\
they are pure, they are brahmins.” 

%
\end{verse}

%
\section*{{\suttatitleacronym Ud 1.10}{\suttatitletranslation With Bāhiya }{\suttatitleroot Bāhiyasutta}}
\addcontentsline{toc}{section}{\tocacronym{Ud 1.10} \toctranslation{With Bāhiya } \tocroot{Bāhiyasutta}}
\markboth{With Bāhiya }{Bāhiyasutta}
\extramarks{Ud 1.10}{Ud 1.10}

\scevam{So\marginnote{1.1} I have heard. }At one time the Buddha was staying near \textsanskrit{Sāvatthī} in Jeta’s Grove, \textsanskrit{Anāthapiṇḍika}’s monastery. Now at that time \textsanskrit{Bāhiya} of the Bark Cloth was residing by \textsanskrit{Suppāraka} on the ocean shore, where he was honored, respected, revered, venerated, and esteemed. And he received robes, almsfood, lodgings, and medicines and supplies for the sick. Then as he was in private retreat this thought came to his mind, “I am one of those in the world who are perfected or on the path to perfection.” 

Then\marginnote{2.1} a deity who was a former relative of \textsanskrit{Bāhiya}, having compassion and wanting what’s best for him, approached him and said: “\textsanskrit{Bāhiya}, you’re not a perfected one, nor on the path to perfection. You don’t have the practice by which you might become a perfected one or one on the path to perfection.” 

“Then\marginnote{3.1} who exactly are those in the world who are perfected or on the path to perfection?” “In the northern lands there is a city called \textsanskrit{Sāvatthī}. There that Blessed One is now staying, the perfected one, the fully awakened Buddha. He is a perfected one and teaches the Dhamma for the sake of perfection.” 

Impelled\marginnote{4.1} by that deity, \textsanskrit{Bāhiya} left \textsanskrit{Suppāraka} right away. Sojourning no more than a single night in any place, he made his way to \textsanskrit{Anāthapiṇḍika}’s Monastery in the Jeta Grove at \textsanskrit{Sāvatthī}. At that time several mendicants were walking mindfully in the open air. \textsanskrit{Bāhiya} approached them and said, “Sirs, where is the Blessed One at present, the perfected one, the fully awakened Buddha? For I want to see him.” “He has entered an inhabited area for almsfood, \textsanskrit{Bāhiya}.” 

Then\marginnote{5.1} \textsanskrit{Bāhiya} rushed out of the Jeta Grove and entered \textsanskrit{Sāvatthī}, where he saw the Buddha walking for alms. He was impressive and inspiring, with peaceful faculties and mind, attained to the highest self-control and serenity, like an elephant with tamed, guarded, and controlled faculties. \textsanskrit{Bāhiya} went up to the Buddha, bowed down with his head at the Buddha’s feet, and said, “Sir, let the Blessed One teach me the Dhamma! Let the Holy One teach me the Dhamma! That would be for my lasting welfare and happiness.” The Buddha said this: “It’s not the time, \textsanskrit{Bāhiya}, so long as I have entered an inhabited area for almsfood.” 

For\marginnote{6.1} a second time, \textsanskrit{Bāhiya} said, “But you never know, sir, when life is at risk, either the Buddha’s or my own. Let the Blessed One teach me the Dhamma! Let the Holy One teach me the Dhamma! That would be for my lasting welfare and happiness.” For a second time, the Buddha said, “It’s not the time, \textsanskrit{Bāhiya}, so long as I have entered an inhabited area for almsfood.” 

For\marginnote{7.1} a third time, \textsanskrit{Bāhiya} said, “But you never know, sir, when life is at risk, either the Buddha’s or my own. Let the Blessed One teach me the Dhamma! Let the Holy One teach me the Dhamma! That would be for my lasting welfare and happiness.” 

“In\marginnote{8.1} that case, \textsanskrit{Bāhiya}, you should train like this: ‘In the seen will be merely the seen; in the heard will be merely the heard; in the thought will be merely the thought; in the known will be merely the known.’ That’s how you should train. When you have trained in this way, you won’t be ‘by that’. When you’re not ‘by that’, you won’t be ‘in that’. When you’re not ‘in that’, you won’t be in this world or the world beyond or between the two. Just this is the end of suffering.” 

Then,\marginnote{9.1} due to this brief Dhamma teaching of the Buddha, \textsanskrit{Bāhiya}’s mind was right away freed from defilements by not grasping. 

And\marginnote{10.1} when the Buddha had given \textsanskrit{Bāhiya} this brief advice he left. But soon after the Buddha had left, a cow with a baby calf charged at \textsanskrit{Bāhiya} and took his life. 

Then\marginnote{11.1} the Buddha wandered for alms in \textsanskrit{Sāvatthī}. After the meal, on his return from almsround, he departed the city together with several mendicants and saw that \textsanskrit{Bāhiya} had passed away. He said to the monks, “Mendicants, pick up \textsanskrit{Bāhiya}’s corpse. Having lifted it onto a cot and carried it, cremate it and build a monument. Mendicants, one of your spiritual companions has passed away.” 

“Yes,\marginnote{12.1} sir,” replied those mendicants. They did as the Buddha asked, then returned to the Buddha and said, “Sir, \textsanskrit{Bāhiya}’s corpse has been cremated and a monument built for him. Where has he been reborn in his next life?” “Mendicants, \textsanskrit{Bāhiya} was astute. He practiced in line with the teachings, and did not trouble me about the teachings. \textsanskrit{Bāhiya} of the Bark Cloth has become fully extinguished.” 

Then,\marginnote{13.1} understanding this matter, on that occasion the Buddha expressed this heartfelt sentiment: 

\begin{verse}%
“Where\marginnote{14.1} water and earth, \\
fire and air find no footing: \\
there no star does shine, \\
nor does the sun shed its light; \\
there the moon glows not, \\
yet no darkness is found. 

And\marginnote{15.1} when a sage, a brahmin, finds understanding \\
through their own sagacity, \\
then from forms and formless, \\
from pleasure and pain they are released.” 

%
\end{verse}

This\marginnote{16.1} too is a heartfelt saying that was spoken by the Blessed One: that is what I heard.\footnote{This tag is unique to the \textsanskrit{Udāna}, and seems to be a relic of an earlier oral organization. It is virtually identical to the tag concluding each discourse of the Itivuttaka, except there we find \textit{attha} instead of \textit{\textsanskrit{udāna}}. } 

%
\addtocontents{toc}{\let\protect\contentsline\protect\nopagecontentsline}
\chapter*{The Chapter with Mucalinda }
\addcontentsline{toc}{chapter}{\tocchapterline{The Chapter with Mucalinda }}
\addtocontents{toc}{\let\protect\contentsline\protect\oldcontentsline}

%
\section*{{\suttatitleacronym Ud 2.1}{\suttatitletranslation With Mucalinda }{\suttatitleroot Mucalindasutta}}
\addcontentsline{toc}{section}{\tocacronym{Ud 2.1} \toctranslation{With Mucalinda } \tocroot{Mucalindasutta}}
\markboth{With Mucalinda }{Mucalindasutta}
\extramarks{Ud 2.1}{Ud 2.1}

\scevam{So\marginnote{1.1} I have heard. }At one time, when he was first awakened, the Buddha was staying near \textsanskrit{Uruvelā} at the root of the Mucalinda tree on the bank of the \textsanskrit{Nerañjarā} River. There the Buddha sat cross-legged for seven days without moving, experiencing the bliss of freedom. 

Just\marginnote{2.1} then a great storm blew up out of season, bringing seven days of rain, cold winds, and clouds. Mucalinda, the dragon king, came out from his abode, encircled the Buddha’s body with seven coils and spread his large hood over his head, thinking, “May the Buddha not be hot or cold, nor be bothered by flies, mosquitoes, wind, sun, or reptiles.” 

When\marginnote{3.1} seven days had passed, the Buddha emerged from that state of immersion. When he knew the sky was clear, Mucalinda unwrapped his coils from the Buddha’s body. Hiding his own form, he manifested in the form of a brahmin youth. He stood in front of the Buddha, venerating him with joined palms. 

Then,\marginnote{4.1} understanding this matter, on that occasion the Buddha expressed this heartfelt sentiment: 

\begin{verse}%
“Seclusion\marginnote{5.1} is happiness for the contented \\
who see the teaching they have learned. \\
Kindness for the world is happiness \\
for one who’d not harm a living creature. 

Dispassion\marginnote{6.1} for the world is happiness \\
for one who has gone beyond sensual pleasures. \\
But dispelling the conceit ‘I am’ \\
is truly the ultimate happiness.” 

%
\end{verse}

%
\section*{{\suttatitleacronym Ud 2.2}{\suttatitletranslation Kings }{\suttatitleroot Rājasutta}}
\addcontentsline{toc}{section}{\tocacronym{Ud 2.2} \toctranslation{Kings } \tocroot{Rājasutta}}
\markboth{Kings }{Rājasutta}
\extramarks{Ud 2.2}{Ud 2.2}

\scevam{So\marginnote{1.1} I have heard. }At one time the Buddha was staying near \textsanskrit{Sāvatthī} in Jeta’s Grove, \textsanskrit{Anāthapiṇḍika}’s monastery. Now at that time, after the meal, on return from almsround, several mendicants sat together in the assembly hall and this discussion came up among them: “Which of these two kings has greater wealth, riches, treasury, dominion, vehicles, forces, might, and power: King Seniya \textsanskrit{Bimbisāra} of \textsanskrit{Māgadha} or King Pasenadi of Kosala?” At that point the conversation among those mendicants was left unfinished. 

Then\marginnote{2.1} in the late afternoon, the Buddha came out of retreat, went to the assembly hall, sat down on the seat spread out, and addressed the mendicants: “Mendicants, what were you sitting talking about just now? What conversation was left unfinished?” 

So\marginnote{3.1} the mendicants told him what they had been talking about when the Buddha arrived. The Buddha said, 

“Mendicants,\marginnote{4.1} it is not appropriate for you gentlemen who have gone forth in faith from the lay life to homelessness to talk about such things. When you’re sitting together you should do one of two things: discuss the teachings or keep noble silence.” 

Then,\marginnote{5.1} understanding this matter, on that occasion the Buddha expressed this heartfelt sentiment: 

\begin{verse}%
“Neither\marginnote{6.1} the pleasures of the senses, \\
nor even divine happiness, \\
is worth even a sixteenth part \\
of the happiness of craving’s end.” 

%
\end{verse}

%
\section*{{\suttatitleacronym Ud 2.3}{\suttatitletranslation A Stick }{\suttatitleroot Daṇḍasutta}}
\addcontentsline{toc}{section}{\tocacronym{Ud 2.3} \toctranslation{A Stick } \tocroot{Daṇḍasutta}}
\markboth{A Stick }{Daṇḍasutta}
\extramarks{Ud 2.3}{Ud 2.3}

\scevam{So\marginnote{1.1} I have heard. }At one time the Buddha was staying near \textsanskrit{Sāvatthī} in Jeta’s Grove, \textsanskrit{Anāthapiṇḍika}’s monastery. Now at that time, between \textsanskrit{Sāvatthī} and the Jeta Grove, several boys were hitting a snake with a stick. Then the Buddha robed up in the morning and, taking his bowl and robe, entered \textsanskrit{Sāvatthī} for alms. He saw the boys hitting the snake. 

Then,\marginnote{2.1} understanding this matter, on that occasion the Buddha expressed this heartfelt sentiment: 

\begin{verse}%
“Creatures\marginnote{3.1} love happiness, \\
so if you harm them with a stick \\
in search of your own happiness, \\
after death you’ll find no happiness. 

Creatures\marginnote{4.1} love happiness, \\
so if you don’t harm them with a stick \\
in search of your own happiness, \\
after death you will find happiness.” 

%
\end{verse}

%
\section*{{\suttatitleacronym Ud 2.4}{\suttatitletranslation Esteem }{\suttatitleroot Sakkārasutta}}
\addcontentsline{toc}{section}{\tocacronym{Ud 2.4} \toctranslation{Esteem } \tocroot{Sakkārasutta}}
\markboth{Esteem }{Sakkārasutta}
\extramarks{Ud 2.4}{Ud 2.4}

\scevam{So\marginnote{1.1} I have heard. }At one time the Buddha was staying near \textsanskrit{Sāvatthī} in Jeta’s Grove, \textsanskrit{Anāthapiṇḍika}’s monastery. Now at that time the Buddha was honored, respected, revered, venerated, and esteemed. And he received robes, almsfood, lodgings, and medicines and supplies for the sick. And the mendicant \textsanskrit{Saṅgha} was also honored, respected, revered, venerated, and esteemed. And they received robes, almsfood, lodgings, and medicines and supplies for the sick. But the wanderers who followed other paths were not honored, respected, revered, venerated, and esteemed. And they didn’t receive robes, almsfood, lodgings, and medicines and supplies for the sick. Then those wanderers who followed other paths, unable to bear the esteem of the mendicant Sangha, abused, attacked, harassed, and troubled the mendicants in the village and the wilderness. 

Then\marginnote{2.1} several mendicants went up to the Buddha, bowed, sat down to one side, and told him what had happened. 

Then,\marginnote{3.1} understanding this matter, on that occasion the Buddha expressed this heartfelt sentiment: 

\begin{verse}%
“When\marginnote{4.1} struck by pleasure and pain in the village or wilderness, \\
regard it not as self or other. \\
Contacts strike because of attachment; \\
how would contacts strike one free of attachment?” 

%
\end{verse}

%
\section*{{\suttatitleacronym Ud 2.5}{\suttatitletranslation A Lay Follower }{\suttatitleroot Upāsakasutta}}
\addcontentsline{toc}{section}{\tocacronym{Ud 2.5} \toctranslation{A Lay Follower } \tocroot{Upāsakasutta}}
\markboth{A Lay Follower }{Upāsakasutta}
\extramarks{Ud 2.5}{Ud 2.5}

\scevam{So\marginnote{1.1} I have heard. }At one time the Buddha was staying near \textsanskrit{Sāvatthī} in Jeta’s Grove, \textsanskrit{Anāthapiṇḍika}’s monastery. Now at that time a certain lay follower from \textsanskrit{Icchānaṅgalaka} arrived at \textsanskrit{Sāvatthī} on some business. Having concluded his business in \textsanskrit{Sāvatthī} he went to see the Buddha, bowed, and sat down to one side. The Buddha said this to him: “It’s been a long time, lay follower, since you took the opportunity to come here.” 

“For\marginnote{2.1} a long time I’ve wanted to come and see the Buddha, but I wasn’t able, being prevented by my many duties and responsibilities.” 

Then,\marginnote{3.1} understanding this matter, on that occasion the Buddha expressed this heartfelt sentiment: 

\begin{verse}%
“One\marginnote{4.1} who has nothing is happy indeed, \\
a learned person who has assessed the teaching. \\
See how troubled are those with attachments, \\
a person bound tight to people.” 

%
\end{verse}

%
\section*{{\suttatitleacronym Ud 2.6}{\suttatitletranslation The Pregnant Woman }{\suttatitleroot Gabbhinīsutta}}
\addcontentsline{toc}{section}{\tocacronym{Ud 2.6} \toctranslation{The Pregnant Woman } \tocroot{Gabbhinīsutta}}
\markboth{The Pregnant Woman }{Gabbhinīsutta}
\extramarks{Ud 2.6}{Ud 2.6}

\scevam{So\marginnote{1.1} I have heard. }At one time the Buddha was staying near \textsanskrit{Sāvatthī} in Jeta’s Grove, \textsanskrit{Anāthapiṇḍika}’s monastery. Now at that time a certain wanderer had a young brahmin wife who was pregnant and about to give birth. She said to him, “Go, brahmin, bring oil for my delivery.” 

The\marginnote{2.1} wanderer said, “But where, my dear, can I get oil?” For a second time, she said, “Go, brahmin, bring oil for my delivery.” For a second time, the wanderer said, “But where, my dear, can I get oil?” For a third time, she said, “Go, brahmin, bring oil for my delivery.” 

Now\marginnote{3.1} at that time ghee and oil were being given away to any ascetic or brahmin at the storehouse of King Pasenadi of Kosala. But it was only to drink there, not to take away. 

Knowing\marginnote{4.1} this, the wanderer thought, “Why don’t I go to the king’s storehouse, drink as much oil as I can, then come home and throw it up so it can be used for the delivery?” 

Then\marginnote{5.1} he did just that. But when he got home he was unable to either bring it up or pass it out. He rolled to and fro, suffering painful, sharp, severe, acute feelings. 

Then\marginnote{6.1} the Buddha robed up in the morning and, taking his bowl and robe, entered \textsanskrit{Sāvatthī} for alms. He saw the wanderer in agony. 

Then,\marginnote{7.1} understanding this matter, on that occasion the Buddha expressed this heartfelt sentiment: 

\begin{verse}%
“Oh!\marginnote{8.1} How happy are those with nothing! \\
Hence knowledge masters are people with nothing. \\
See how troubled are those with attachments, \\
a person bound tight to people.” 

%
\end{verse}

%
\section*{{\suttatitleacronym Ud 2.7}{\suttatitletranslation An Only Son }{\suttatitleroot Ekaputtakasutta}}
\addcontentsline{toc}{section}{\tocacronym{Ud 2.7} \toctranslation{An Only Son } \tocroot{Ekaputtakasutta}}
\markboth{An Only Son }{Ekaputtakasutta}
\extramarks{Ud 2.7}{Ud 2.7}

\scevam{So\marginnote{1.1} I have heard. }At one time the Buddha was staying near \textsanskrit{Sāvatthī} in Jeta’s Grove, \textsanskrit{Anāthapiṇḍika}’s monastery. Now at that time a certain lay follower’s dear and beloved only child passed away. 

Then\marginnote{2.1} in the middle of the day several lay followers with wet clothes and hair went up to the Buddha, bowed, and sat down to one side. The Buddha said to them: “Why, lay followers, have you come here in the middle of the day with wet clothes and hair?” 

The\marginnote{3.1} lay follower replied, “Sir, my dear and beloved only child has passed away. That’s why we came here in the middle of the day with wet clothes and hair.” 

Then,\marginnote{4.1} understanding this matter, on that occasion the Buddha expressed this heartfelt sentiment: 

\begin{verse}%
“Hosts\marginnote{5.1} of gods and most human beings are bound\footnote{The PTS reading \textit{\textsanskrit{piyarūpāsāta}} is tempting, and supported by the \textsanskrit{Udānavarga}’s \textit{\textsanskrit{priyarūpasātagrathitā}}. The commentary reads \textit{\textsanskrit{sukhavedanassādena}}, which appears to support \textit{\textsanskrit{assāda}}. But perhaps originally \textit{\textsanskrit{sāta}} was glossed with \textit{\textsanskrit{sukhavedanā}}, and \textit{\textsanskrit{assāda}} was read back into the text by mistake. } \\
to what seems dear and pleasant. \\
Miserable and exhausted, \\
they fall under the sway of the King of Death. 

The\marginnote{6.1} diligent, who day and night \\
leave behind what seems pleasant, \\
dig out the root of misery—\\
Death’s bait so hard to escape.” 

%
\end{verse}

%
\section*{{\suttatitleacronym Ud 2.8}{\suttatitletranslation Suppavāsā }{\suttatitleroot Suppavāsāsutta}}
\addcontentsline{toc}{section}{\tocacronym{Ud 2.8} \toctranslation{Suppavāsā } \tocroot{Suppavāsāsutta}}
\markboth{Suppavāsā }{Suppavāsāsutta}
\extramarks{Ud 2.8}{Ud 2.8}

\scevam{So\marginnote{1.1} I have heard. }At one time the Buddha was staying near \textsanskrit{Kuṇḍiyā} in the \textsanskrit{Kuṇḍadhāna} Grove. Now at that time \textsanskrit{Suppavāsā} the Koliyan had been pregnant for seven years, and in difficult labor for seven days. While suffering painful, sharp, severe, acute feelings, three thoughts helped her endure: “Oh! The Blessed One is indeed a fully awakened Buddha, who teaches the Dhamma for giving up suffering such as this. Oh! The \textsanskrit{Saṅgha} of the Buddha’s disciples is indeed practicing well, who practice for giving up suffering such as this. Oh! Extinguishment is so very blissful, where such suffering as this is not found.” 

Then\marginnote{2.1} \textsanskrit{Suppavāsā} addressed her husband, “Please, master, go to the Buddha, and in my name bow with your head to his feet. Ask him if he is healthy and well, nimble, strong, and living comfortably. And then say: ‘\textsanskrit{Suppavāsā} the Koliyan has been pregnant for seven years, and in difficult labor for seven days. While suffering painful, sharp, severe, acute feelings, three thoughts help her endure: “Oh! The Blessed One is indeed a fully awakened Buddha, who teaches the Dhamma for giving up suffering such as this. Oh! The \textsanskrit{Saṅgha} of the Buddha’s disciples is indeed practicing well, who practice for giving up suffering such as this. Oh! Extinguishment is so very blissful, where such suffering as this is not found.”’” 

“Excellent\marginnote{3.1} idea,” he replied. He went to the Buddha and told him of his wife’s struggles. The Buddha said: 

“May\marginnote{4.1} \textsanskrit{Suppavāsā} the Koliyan be happy and healthy! May she give birth to a healthy child!” As soon as he spoke, \textsanskrit{Suppavāsā}, happy and healthy, gave birth to a healthy child. 

Saying\marginnote{5.1} “Yes, sir,” the Koliyan gentleman approved and agreed with what the Buddha said. He got up from his seat, bowed, and respectfully circled the Buddha, keeping him on his right. Then he returned to his own house. He saw that his wife, happy and healthy, had given birth to a healthy child, and thought, “It’s incredible, it’s amazing! The Realized One has such psychic power and might! For as soon as he spoke, \textsanskrit{Suppavāsā}, happy and healthy, gave birth to a healthy child.” He became uplifted and overjoyed, full of rapture and happiness. 

Then\marginnote{6.1} \textsanskrit{Suppavāsā} addressed her husband, “Please, master, go to the Buddha, and in my name bow with your head to his feet. Ask him if he is healthy and well, nimble, strong, and living comfortably. And then say, ‘\textsanskrit{Suppavāsā} the Koliyan, who was pregnant for seven years, and in difficult labor for seven days, is now happy and healthy and has given birth to a healthy child. She invites the mendicant \textsanskrit{Saṅgha} headed by the Buddha to a meal for seven days. Sir, may the Buddha please accept seven meals from \textsanskrit{Suppavāsā}.’” 

“Excellent\marginnote{7.1} idea,” he replied. He went to the Buddha, told him the good news, and conveyed his wife’s invitation. 

Now\marginnote{9.1} at that time a certain lay follower had already invited the Sangha of monks headed by the Buddha for the meal on the following day. That lay follower was Venerable \textsanskrit{Mahāmoggallāna}’s supporter. Then the Buddha addressed Venerable \textsanskrit{Mahāmoggallāna}, “Please, \textsanskrit{Moggallāna}, go to the that lay follower and say to him, ‘\textsanskrit{Suppavāsā} the Koliyan, who was pregnant for seven years, and in difficult labor for seven days, is now happy and healthy and has given birth to a healthy child. She invites the mendicant \textsanskrit{Saṅgha} headed by the Buddha to a meal for seven days. Let \textsanskrit{Suppavāsā} make seven meals, afterwards you can make yours.’ He is your supporter.” 

“Yes,\marginnote{10.1} sir,” replied \textsanskrit{Mahāmoggallāna}. He went to that lay follower and conveyed the Buddha’s request. 

“If,\marginnote{11.1} sir, Venerable \textsanskrit{Mahāmoggallāna} can guarantee me three things—wealth, life, and faith—then let \textsanskrit{Suppavāsā} make seven meals, afterwards I shall make mine.” “I can guarantee you two things—wealth and life. But as for faith, you alone are the guarantor.” 

“If,\marginnote{12.1} sir Venerable \textsanskrit{Mahāmoggallāna} can guarantee me two things—wealth and life—then let \textsanskrit{Suppavāsā} make seven meals, afterwards I shall make mine.” 

Having\marginnote{13.1} persuaded that lay follower, \textsanskrit{Mahāmoggallāna} went to the Buddha and said, “I’ve persuaded the lay follower. Let \textsanskrit{Suppavāsā} make seven meals, afterwards he shall make his.” 

For\marginnote{14.1} seven days \textsanskrit{Suppavāsā} served and satisfied the Buddha with her own hands with a variety of delicious foods. And she made her little boy bow to the Buddha and the mendicant Sangha. 

Then\marginnote{15.1} \textsanskrit{Sāriputta} said to the boy, “I hope you’re keeping well, little boy; I hope you’re alright. I hope that you are not in pain.” “How could I be keeping well? How could I be alright? For seven years I lived in a pot of blood.”\footnote{\textit{Lohitakumbhiya} is probably a play on the common phrase \textit{lohakumbhiya}, “copper pot”, which is commonly used in descriptions of hell. The boy, it seems, had an unusually rapid linguistic development. } 

Then\marginnote{16.1} \textsanskrit{Suppavāsā}, thinking, “My child is conversing with the General of the Dhamma!” was uplifted and overjoyed, full of rapture and happiness. Knowing this, the Buddha said to her, “Would you like to have another child like this?” “Sir, I would like to have seven more children like this!” 

Then,\marginnote{17.1} understanding this matter, on that occasion the Buddha expressed this heartfelt sentiment: 

\begin{verse}%
“Pain\marginnote{18.1} in the guise of pleasure, \\
the disliked in the guise of the liked, \\
suffering in the guise of happiness, \\
overpower the negligent.” 

%
\end{verse}

%
\section*{{\suttatitleacronym Ud 2.9}{\suttatitletranslation With Visākhā }{\suttatitleroot Visākhāsutta}}
\addcontentsline{toc}{section}{\tocacronym{Ud 2.9} \toctranslation{With Visākhā } \tocroot{Visākhāsutta}}
\markboth{With Visākhā }{Visākhāsutta}
\extramarks{Ud 2.9}{Ud 2.9}

\scevam{So\marginnote{1.1} I have heard. }At one time the Buddha was staying near \textsanskrit{Sāvatthī} in the Eastern Monastery, the stilt longhouse of \textsanskrit{Migāra}’s mother. Now at that time \textsanskrit{Visākhā} was caught up in some business with King Pasenadi. But the king’s settlement did not meet her expectations. 

Then,\marginnote{2.1} in the middle of the day, she went to the Buddha, bowed, and sat down. The Buddha said to her: “So, \textsanskrit{Visākhā}, where are you coming from in the middle of the day?” “Sir, I am caught up in some business with King Pasenadi. But the king’s settlement did not meet my expectations.” 

Then,\marginnote{3.1} understanding this matter, on that occasion the Buddha expressed this heartfelt sentiment: 

\begin{verse}%
“All\marginnote{4.1} under another’s control is suffering, \\
all under one’s own authority is pleasing; \\
what’s shared is stressful for both parties, \\
for bonds are hard to escape.” 

%
\end{verse}

%
\section*{{\suttatitleacronym Ud 2.10}{\suttatitletranslation With Bhaddiya }{\suttatitleroot Bhaddiyasutta}}
\addcontentsline{toc}{section}{\tocacronym{Ud 2.10} \toctranslation{With Bhaddiya } \tocroot{Bhaddiyasutta}}
\markboth{With Bhaddiya }{Bhaddiyasutta}
\extramarks{Ud 2.10}{Ud 2.10}

\scevam{So\marginnote{1.1} I have heard. }At one time the Buddha was staying near Anupiya in a mango grove. Now at that time, Venerable Bhaddiya son of \textsanskrit{Kāḷīgodhā}, even in the wilderness, at the foot of a tree, or in an empty dwelling, frequently expressed this heartfelt sentiment: “Oh, what bliss! Oh, what bliss!” 

Several\marginnote{2.1} mendicants heard him and thought, “Without a doubt, Venerable Bhaddiya leads the spiritual life dissatisfied. It’s when recalling the pleasures of royalty he formerly enjoyed as a lay person that, even in the wilderness, at the foot of a tree, or in an empty dwelling, he frequently expresses this heartfelt sentiment: ‘Oh, what bliss! Oh, what bliss!’” 

Then\marginnote{3.1} those mendicants went up to the Buddha, bowed, sat down to one side, and told him what was happening. 

So\marginnote{4.1} the Buddha addressed a certain monk, “Please, monk, in my name tell the mendicant Bhaddiya that the teacher summons him.” 

“Yes,\marginnote{5.1} sir,” that monk replied. He went to Bhaddiya and said to him, “Reverend Bhaddiya, the teacher summons you.” “Yes, reverend,” Bhaddiya replied. He went to the Buddha, bowed, and sat down to one side. The Buddha said to him: 

“Is\marginnote{6.1} it really true, Bhaddiya, that even in the wilderness, at the foot of a tree, or in an empty dwelling, you frequently express this heartfelt sentiment: ‘Oh, what bliss! Oh, what bliss!’?” “Yes, sir.” 

“But\marginnote{7.1} why do you say this?” “Formerly, as a lay person ruling the land, my guard was well organized within and without the royal compound, within and without the city, and within and without the country. But although I was guarded and defended in this way, I remained fearful, scared, suspicious, and nervous. But these days, even when alone in the wilderness, at the foot of a tree, or in an empty dwelling, I’m not fearful, scared, suspicious, or nervous. I live relaxed, unruffled, surviving on charity, my heart free as a wild deer. It is for this reason that, even in the wilderness, at the foot of a tree, or in an empty dwelling, I frequently expressed this heartfelt sentiment: ‘Oh, what bliss! Oh, what bliss!’” 

Then,\marginnote{8.1} understanding this matter, on that occasion the Buddha expressed this heartfelt sentiment: 

\begin{verse}%
“They\marginnote{9.1} who hide no anger within, \\
gone beyond any kind of existence; \\
happy, free from fear and sorrow—\\
even the gods can’t see them.” 

%
\end{verse}

%
\addtocontents{toc}{\let\protect\contentsline\protect\nopagecontentsline}
\chapter*{The Chapter with Nanda }
\addcontentsline{toc}{chapter}{\tocchapterline{The Chapter with Nanda }}
\addtocontents{toc}{\let\protect\contentsline\protect\oldcontentsline}

%
\section*{{\suttatitleacronym Ud 3.1}{\suttatitletranslation Born of the Fruits of deeds }{\suttatitleroot Kammavipākajasutta}}
\addcontentsline{toc}{section}{\tocacronym{Ud 3.1} \toctranslation{Born of the Fruits of deeds } \tocroot{Kammavipākajasutta}}
\markboth{Born of the Fruits of deeds }{Kammavipākajasutta}
\extramarks{Ud 3.1}{Ud 3.1}

\scevam{So\marginnote{1.1} I have heard. }At one time the Buddha was staying near \textsanskrit{Sāvatthī} in Jeta’s Grove, \textsanskrit{Anāthapiṇḍika}’s monastery. Now, at that time a certain mendicant was sitting not far from the Buddha, cross-legged, with his body straight. As a result of past deeds, he suffered painful, sharp, severe, and acute feelings, which he endured unbothered, with mindfulness and awareness. 

The\marginnote{2.1} Buddha saw him meditating and enduring that pain. 

Then,\marginnote{3.1} understanding this matter, on that occasion the Buddha expressed this heartfelt sentiment: 

\begin{verse}%
“A\marginnote{4.1} mendicant who has left all deeds behind, \\
shaking off the dust of past deeds, \\
unselfish, steady, poised, \\
has no need to complain.” 

%
\end{verse}

%
\section*{{\suttatitleacronym Ud 3.2}{\suttatitletranslation With Nanda }{\suttatitleroot Nandasutta}}
\addcontentsline{toc}{section}{\tocacronym{Ud 3.2} \toctranslation{With Nanda } \tocroot{Nandasutta}}
\markboth{With Nanda }{Nandasutta}
\extramarks{Ud 3.2}{Ud 3.2}

\scevam{So\marginnote{1.1} I have heard. }At one time the Buddha was staying near \textsanskrit{Sāvatthī} in Jeta’s Grove, \textsanskrit{Anāthapiṇḍika}’s monastery. Now at that time Venerable Nanda, the Buddha’s brother and maternal cousin, informed several mendicants: “I lead the spiritual life dissatisfied. I am unable to keep up the spiritual life. I shall resign the training and return to a lesser life.” 

Then\marginnote{2.1} a mendicant went up to the Buddha, bowed, sat down to one side, and told him what was happening. 

So\marginnote{3.1} the Buddha addressed a certain monk, “Please, monk, in my name tell the mendicant Nanda that the teacher summons him.” “Yes, sir,” that monk replied. He went to Nanda and said to him, “Reverend Nanda, the teacher summons you.” 

“Yes,\marginnote{4.1} reverend,” Nanda replied. He went to the Buddha, bowed, and sat down to one side. The Buddha said to him: 

“Is\marginnote{5.1} it really true, Nanda, that you informed several mendicants that you are unable to keep up the spiritual life; that you shall resign the training and return to a lesser life?” “Yes, sir,” he replied. 

“But\marginnote{6.1} why are you so dissatisfied with the spiritual life?” “As I left my house, sir, the finest lady of the Sakyan land, her hair half-combed, glanced at me and said, ‘Hurry back, master.’ Recalling that, I am dissatisfied and shall resign the training.” 

Then\marginnote{7.1} the Buddha took Nanda by the arm and, as easily as a strong person would extend or contract their arm, vanished from Jeta’s Grove and reappeared among the gods of the Thirty-Three. 

Now\marginnote{8.1} at that time five hundred dove-footed nymphs had come to attend to Sakka, the lord of gods. Then the Buddha said to Nanda, “Nanda, do you see these five hundred dove-footed nymphs?” “Yes, sir,” he replied. 

“What\marginnote{9.1} do you think, Nanda? Who is more attractive, good-looking, and lovely—the finest lady of the Sakyan land, or these five hundred dove-footed nymphs?” “Compared to these five hundred dove-footed nymphs, the finest lady of the Sakyan land is like a deformed monkey with its ears and nose cut off. She doesn’t count, there’s no comparison, she’s not worth a fraction. These five hundred dove-footed nymphs are far more attractive, good-looking, and lovely.” 

“Rejoice,\marginnote{10.1} Nanda, rejoice! I guarantee you five hundred dove-footed nymphs.” “If, sir, you guarantee me five hundred dove-footed nymphs, I shall happily lead the spiritual life under the Buddha.” 

Then\marginnote{11.1} the Buddha took Nanda by the arm and, as easily as a strong person would extend or contract their arm, vanished from the gods of the Thirty-Three and reappeared at Jeta’s Grove. 

The\marginnote{12.1} mendicants heard, “It seems Venerable Nanda—who is both the Buddha’s half-brother and maternal cousin—leads the spiritual life for the sake of nymphs. And it seems that the Buddha guaranteed him five hundred dove-footed nymphs.” 

Monks\marginnote{13.1} who were his friends accused him of being a hireling and a lackey, “It seems Nanda is a hireling, it seems he is a lackey: he leads the spiritual life for the sake of nymphs. And it seems that the Buddha guaranteed him five hundred dove-footed nymphs.” 

Then\marginnote{14.1} Nanda—embarrassed, ashamed, and disgusted at being called a hireling and a lackey—living alone, withdrawn, diligent, keen, and resolute, soon realized the supreme end of the spiritual path in this very life. He lived having achieved with his own insight the goal for which gentlemen rightly go forth from the lay life to homelessness. He understood: “Rebirth is ended; the spiritual journey has been completed; what had to be done has been done; there is no return to any state of existence.” Venerable Nanda became one of the perfected. 

Then,\marginnote{15.1} late at night, a glorious deity, lighting up the entire Jeta’s Grove, went up to the Buddha, bowed, stood to one side, and said to him: “Sir, Venerable Nanda—who is both the Buddha’s half-brother and maternal cousin—has realized the undefiled freedom of heart and freedom by wisdom in this very life. He lives having realized it with his own insight due to the ending of defilements.” And the knowledge also came to the Buddha: “Nanda has realized the undefiled freedom of heart and freedom by wisdom in this very life. He lives having realized it with his own insight due to the ending of defilements.” 

Then,\marginnote{16.1} when the night had passed, Nanda went to the Buddha, bowed, sat down to one side, and said to him, “Sir, you guaranteed me five hundred dove-footed nymphs. I release you from that promise.” “Nanda, I comprehended your mind and knew that you had realized the undefiled freedom of heart and freedom by wisdom. And deities also told me about this. As soon as your mind was freed from defilements by not grasping, I was released from that promise.” 

Then,\marginnote{17.1} understanding this matter, on that occasion the Buddha expressed this heartfelt sentiment: 

\begin{verse}%
“The\marginnote{18.1} mendicant who has crossed over the bog, \\
who has crushed the thorns of sensuality, \\
who has reached the end of delusion, \\
trembles not at pleasure and pain.” 

%
\end{verse}

%
\section*{{\suttatitleacronym Ud 3.3}{\suttatitletranslation With Yasoja }{\suttatitleroot Yasojasutta}}
\addcontentsline{toc}{section}{\tocacronym{Ud 3.3} \toctranslation{With Yasoja } \tocroot{Yasojasutta}}
\markboth{With Yasoja }{Yasojasutta}
\extramarks{Ud 3.3}{Ud 3.3}

\scevam{So\marginnote{1.1} I have heard. }At one time the Buddha was staying near \textsanskrit{Sāvatthī} in Jeta’s Grove, \textsanskrit{Anāthapiṇḍika}’s monastery. Now at that time around five hundred mendicants headed by Yasoja arrived at \textsanskrit{Sāvatthī} to see the Buddha. At that, those visiting mendicants, while exchanging pleasantries with the resident mendicants, preparing their lodgings, and putting away their bowls and robes, made a dreadful racket.\footnote{The narrative employs a unique scheme in introducing the repetitions: \textit{tedha}, \textit{tete}, teme\_. } 

Then\marginnote{2.1} the Buddha said to Venerable Ānanda, “Ānanda, who’s making that dreadful racket? You’d think it was fishermen hauling in a catch!” “Sir, those five hundred mendicants headed by Yasoja have arrived at \textsanskrit{Sāvatthī} to see the Buddha. It’s those visiting mendicants who, while exchanging pleasantries with the resident mendicants, preparing their lodgings, and putting away their bowls and robes, made a dreadful racket.” “Well then, Ānanda, in my name tell those mendicants that the teacher summons them.” 

“Yes,\marginnote{3.1} sir,” Ānanda replied. He went to those mendicants and said, “Venerables, the teacher summons you.” “Yes, reverend,” replied those mendicants. Then they rose from their seats and went to the Buddha, bowed, and sat down to one side. The Buddha said to them, 

“Mendicants,\marginnote{4.1} what’s with that dreadful racket? You’d think it was fishermen hauling in a catch!” When he said this, Venerable Yasoja said to the Buddha, “Sir, these five hundred mendicants have arrived at \textsanskrit{Sāvatthī} to see the Buddha. It’s these visiting mendicants who, while exchanging pleasantries with the resident mendicants, preparing their lodgings, and putting away their bowls and robes, made a dreadful racket.” “Go away, mendicants, I dismiss you. You are not to stay in my presence.” 

“Yes,\marginnote{5.1} sir,” replied those mendicants. They got up from their seats, bowed, and respectfully circled the Buddha, keeping him on their right. They set their lodgings in order and left, taking their bowls and robes. Traveling stage by stage in the land of the \textsanskrit{Vajjīs}, they arrived at the \textsanskrit{Vaggumudā} River. They built leaf huts near the riverbank and there they entered the rainy season. 

Then\marginnote{6.1} Venerable Yasoja, having entered the rainy season, addressed the mendicants: “Out of compassion, reverends, the Buddha dismissed us, wanting what’s best for us. Come, let us live in such a way that the Buddha would be pleased with us.” “Yes, reverend,” they replied. Then those mendicants, living alone, withdrawn, diligent, keen, and resolute, all realized the three knowledges in that same rainy season. 

When\marginnote{7.1} the Buddha had stayed in \textsanskrit{Sāvatthī} as long as he wished, he set out for \textsanskrit{Vesālī}. Traveling stage by stage, he arrived at \textsanskrit{Vesālī}, where he stayed in the hall with the peaked roof. 

Then,\marginnote{8.1} having applied his mind to comprehending the minds of the mendicants staying on the bank of the river \textsanskrit{Vaggumudā}, the Buddha said to Venerable Ānanda, “A light, it appears to me, has arisen in this quarter, Ānanda; a brightness has arisen. I’m not put off at the thought of going to where the mendicants are staying on the bank of the river \textsanskrit{Vaggumudā}.\footnote{Here the Buddha invents the concept of ”faint praise”. } Send a message to those mendicants: ‘Venerables, the teacher summons you. He wants to see you.’” 

“Yes,\marginnote{9.1} sir,” Ānanda replied. He went to one of the mendicants and said, “Please, Reverend, go to the mendicants staying on the bank of the river \textsanskrit{Vaggumudā} and say to them, ‘Venerables, the teacher summons you. He wants to see you.’” 

“Yes,\marginnote{10.1} reverend,” replied that mendicant. Then, as easily as a strong person would extend or contract their arm, he vanished from the Great Wood, in the hall with the peaked roof, and reappeared in front of those mendicants on the bank of the river \textsanskrit{Vaggumudā}. Then he said to those mendicants, “Venerables, the teacher summons you. He wants to see you.” 

“Yes,\marginnote{11.1} reverend,” replied those mendicants. They set their lodgings in order and took their bowls and robes. Then, as easily as a strong person would extend or contract their arm, they vanished from the bank of the river \textsanskrit{Vaggumudā}, and reappeared in the presence of the Buddha in the Great Wood, in the hall with the peaked roof. But at that time the Buddha was sitting immersed in imperturbable meditation. Then those mendicants thought, “What kind of meditation is the Buddha practicing right now?” They thought, “He is practicing the imperturbable meditation.” They all sat in imperturbable meditation. 

And\marginnote{12.1} then, as the night was getting late, in the first watch of the night, Venerable Ānanda got up from his seat, arranged his robe over one shoulder, raised his joined palms toward the Buddha and said, “Sir, the night is getting late. It is the first watch of the night, and the visiting mendicants have been sitting long. Sir, please greet the visiting mendicants.” But the Buddha kept silent. 

For\marginnote{13.1} a second time, as the night was getting late, in the middle watch of the night, Ānanda got up from his seat, arranged his robe over one shoulder, raised his joined palms toward the Buddha and said, “Sir, the night is getting late. It is the second watch of the night, and the visiting mendicants have been sitting long. Sir, please greet the visiting mendicants.” But for a second time the Buddha kept silent. 

For\marginnote{14.1} a third time, as the night was getting late, in the last watch of the night, as dawn stirred, bringing joy to the night, Ānanda got up from his seat, arranged his robe over one shoulder, raised his joined palms toward the Buddha and said, “Sir, the night is getting late. It is the last watch of the night; dawn stirs, bringing joy to the night, and the visiting mendicants have been sitting long. Sir, please greet the visiting mendicants.” 

Then\marginnote{15.1} the Buddha emerged from that immersion and addressed Ānanda, “If you’d known, Ānanda, you wouldn’t have said so much.\footnote{Ānandajoti has ‘about them’ for \textit{te} here, but in parallel phrases we find \textit{no}, so this must be the second person singular pronoun. } Both I and these five hundred mendicants have been sitting in imperturbable meditation.” 

Then,\marginnote{16.1} understanding this matter, on that occasion the Buddha expressed this heartfelt sentiment: 

\begin{verse}%
“A\marginnote{17.1} mendicant who has beaten the thorns of sensuality—\\
and abuse, killing, and caging—\\
steady as a mountain, imperturbable, \\
trembles not at pleasure and pain.” 

%
\end{verse}

%
\section*{{\suttatitleacronym Ud 3.4}{\suttatitletranslation With Sāriputta }{\suttatitleroot Sāriputtasutta}}
\addcontentsline{toc}{section}{\tocacronym{Ud 3.4} \toctranslation{With Sāriputta } \tocroot{Sāriputtasutta}}
\markboth{With Sāriputta }{Sāriputtasutta}
\extramarks{Ud 3.4}{Ud 3.4}

\scevam{So\marginnote{1.1} I have heard. }At one time the Buddha was staying near \textsanskrit{Sāvatthī} in Jeta’s Grove, \textsanskrit{Anāthapiṇḍika}’s monastery. Now at that time Venerable \textsanskrit{Sāriputta} was sitting not far from the Buddha, cross-legged, with his body straight, and mindfulness established right there. The Buddha saw him meditating there. 

Then,\marginnote{2.1} understanding this matter, on that occasion the Buddha expressed this heartfelt sentiment: 

\begin{verse}%
“As\marginnote{3.1} a rocky mountain \\
is unwavering and well grounded, \\
so when delusion ends, \\
a monk, like a mountain, doesn’t tremble.” 

%
\end{verse}

%
\section*{{\suttatitleacronym Ud 3.5}{\suttatitletranslation With Mahāmoggallāna }{\suttatitleroot Mahāmoggallānasutta}}
\addcontentsline{toc}{section}{\tocacronym{Ud 3.5} \toctranslation{With Mahāmoggallāna } \tocroot{Mahāmoggallānasutta}}
\markboth{With Mahāmoggallāna }{Mahāmoggallānasutta}
\extramarks{Ud 3.5}{Ud 3.5}

\scevam{So\marginnote{1.1} I have heard. }At one time the Buddha was staying near \textsanskrit{Sāvatthī} in Jeta’s Grove, \textsanskrit{Anāthapiṇḍika}’s monastery. Now at that time Venerable \textsanskrit{Mahāmoggallāna} was sitting not far from the Buddha, cross-legged, with his body straight and mindfulness of the body well-established in himself. The Buddha saw him meditating there. 

Then,\marginnote{2.1} understanding this matter, on that occasion the Buddha expressed this heartfelt sentiment: 

\begin{verse}%
“With\marginnote{3.1} mindfulness of the body established, \\
restrained in the six fields of contact, \\
a mendicant always immersed in \textsanskrit{samādhi} \\
would know quenching in themselves.” 

%
\end{verse}

%
\section*{{\suttatitleacronym Ud 3.6}{\suttatitletranslation With Pilindavaccha }{\suttatitleroot Pilindavacchasutta}}
\addcontentsline{toc}{section}{\tocacronym{Ud 3.6} \toctranslation{With Pilindavaccha } \tocroot{Pilindavacchasutta}}
\markboth{With Pilindavaccha }{Pilindavacchasutta}
\extramarks{Ud 3.6}{Ud 3.6}

\scevam{So\marginnote{1.1} I have heard. }At one time the Buddha was staying near \textsanskrit{Rājagaha}, in the Bamboo Grove, the squirrels’ feeding ground. Now at that time Venerable Pilindavaccha addressed the mendicants as “lowlifes”. Then several mendicants went up to the Buddha, bowed, sat down to one side, and said to him, “Sir, Venerable Pilindavaccha addresses the mendicants as ‘lowlifes’.” 

So\marginnote{2.1} the Buddha addressed a certain monk, “Please, monk, in my name tell the mendicant Pilindavaccha that the Teacher summons him.” “Yes, sir,” that monk replied. He went to Pilindavaccha and said to him, “Reverend Pilindavaccha, the teacher summons you.” 

“Yes,\marginnote{3.1} reverend,” Pilindavaccha replied. He went to the Buddha, bowed, and sat down to one side. The Buddha said to him: “Is it really true, Vaccha, that you addressed the mendicants as ‘lowlifes’?” “Yes, sir,” he replied. 

Then,\marginnote{4.1} having applied his mind to Pilindavaccha’s past lives, the Buddha said to the mendicants, “Mendicants, don’t complain about the mendicant Vaccha. He doesn’t addresses the mendicants as ‘lowlifes’ out of hate. For five hundred lives without interruption he was reborn in a brahmin family. For a long time, he has addressed people as ‘lowlife’. That’s why he addresses the mendicants as ‘lowlifes’.” 

Then,\marginnote{5.1} understanding this matter, on that occasion the Buddha expressed this heartfelt sentiment: 

\begin{verse}%
“In\marginnote{6.1} whom dwells no deceit or conceit, \\
rid of greed, unselfish, with no need for hope, \\
with anger eliminated, quenched: \\
they are a brahmin, an ascetic, a mendicant.” 

%
\end{verse}

%
\section*{{\suttatitleacronym Ud 3.7}{\suttatitletranslation Sakka’s Heartfelt Saying }{\suttatitleroot Sakkudānasutta}}
\addcontentsline{toc}{section}{\tocacronym{Ud 3.7} \toctranslation{Sakka’s Heartfelt Saying } \tocroot{Sakkudānasutta}}
\markboth{Sakka’s Heartfelt Saying }{Sakkudānasutta}
\extramarks{Ud 3.7}{Ud 3.7}

\scevam{So\marginnote{1.1} I have heard. }At one time the Buddha was staying near \textsanskrit{Rājagaha}, in the Bamboo Grove, the squirrels’ feeding ground. Now at that time Venerable \textsanskrit{Mahākassapa} was staying in the Pipphali cave. Having entered a certain state of immersion, he sat cross-legged for seven days without moving. When seven days had passed, \textsanskrit{Mahākassapa} emerged from that state of immersion. It occurred to him, “Why not enter \textsanskrit{Rājagaha} for almsfood?” 

Now\marginnote{2.1} at that time five hundred deities were ready and eager for the chance to offer alms to \textsanskrit{Mahākassapa}. But \textsanskrit{Mahākasspa} refused those deities. In the morning, he robed up, took his bowl and robe, and entered \textsanskrit{Rājagaha} for alms. 

Now\marginnote{2.3} at that time Sakka, lord of Gods, wished to give alms to \textsanskrit{Mahākassapa}. Having manifested in the appearance of a weaver, he worked the loom while the demon maiden \textsanskrit{Sujā} fed the shuttle. Then, as \textsanskrit{Mahākassapa} wandered indiscriminately for almsfood in \textsanskrit{Rājagaha}, he approached Sakka’s house. Seeing \textsanskrit{Mahākassapa} coming off in the distance, Sakka came out of his house, greeted him, and took the bowl from his hand. He re-entered the house and filled the bowl with rice from the pot. That almsfood had many tasty soups and sauces.\footnote{I’m not sure if \textit{\textsanskrit{anekarasabyañjano}} is correct, although the word does occur at pv13:15.4 and pv27:12.4. The commentary, while not definitive, suggests \textit{\textsanskrit{anekasūpabyañjano}} (\textit{anekehi \textsanskrit{sūpehi} ceva \textsanskrit{byañjanehi} ca}), which would be more idiomatic. However it is not attested. } Then it occurred to \textsanskrit{Mahākassapa}, “Now, what being is this who has such psychic power?” It occurred to him, “This is Sakka, lord of Gods.” Knowing this, he said to Sakka, “This is your doing, Kosiya; don’t do such a thing again.” “But sir, Kassapa, we too need merit! We too ought make merit.” 

Then\marginnote{4.1} Sakka bowed and respectfully circled \textsanskrit{Mahākassapa}, keeping him on his right. Then he rose into the air and, sitting cross-legged in the sky, expressed this heartfelt sentiment three times: “Oh the gift, the best gift is well established in Kassapa! Oh the gift, the best gift is well established in Kassapa! Oh the gift, the best gift is well established in Kassapa!” With clairaudience that is purified and superhuman, the Buddha heard Sakka express this heartfelt sentiment while sitting in the sky. 

Then,\marginnote{5.1} understanding this matter, on that occasion the Buddha expressed this heartfelt sentiment: 

\begin{verse}%
“A\marginnote{6.1} mendicant who relies on alms, \\
self-supported, providing for no other; \\
the poised one is envied by even the gods, \\
calm and ever mindful.” 

%
\end{verse}

%
\section*{{\suttatitleacronym Ud 3.8}{\suttatitletranslation One Who Eats Only Almsfood }{\suttatitleroot Piṇḍapātikasutta}}
\addcontentsline{toc}{section}{\tocacronym{Ud 3.8} \toctranslation{One Who Eats Only Almsfood } \tocroot{Piṇḍapātikasutta}}
\markboth{One Who Eats Only Almsfood }{Piṇḍapātikasutta}
\extramarks{Ud 3.8}{Ud 3.8}

\scevam{So\marginnote{1.1} I have heard. }At one time the Buddha was staying near \textsanskrit{Sāvatthī} in Jeta’s Grove, \textsanskrit{Anāthapiṇḍika}’s monastery. Now at that time, after the meal, on return from almsround, several mendicants sat together in the pavilion by the kari tree and this discussion came up among them: 

“Reverends,\marginnote{2.1} when a mendicant who eats only almsfood is wandering for alms, from time to time they get to see pleasing sights, hear pleasing sounds, smell pleasing smells, taste pleasing tastes, and encounter pleasing touches. They wander for alms being honored, respected, revered, venerated, and esteemed. Come, we too should eat only almsfood. From time to time we too will get to see pleasing sights, hear pleasing sounds, smell pleasing smells, taste pleasing tastes, and encounter pleasing touches. We too shall wander for alms being honored, respected, revered, venerated, and esteemed.” At that point the conversation among those mendicants was left unfinished. 

Then\marginnote{3.1} in the late afternoon, the Buddha came out of retreat and went to the pavilion by the kari tree, where he sat on the seat spread out and addressed the mendicants: “Mendicants, what were you sitting talking about just now? What conversation was left unfinished?” 

So\marginnote{4.1} the mendicants told him what they had been talking about. The Buddha said, 

“Mendicants,\marginnote{6.1} it is not appropriate for you gentlemen who have gone forth in faith from the lay life to homelessness to talk about such things. When you’re sitting together you should do one of two things: discuss the teachings or keep noble silence.” 

Then,\marginnote{7.1} understanding this matter, on that occasion the Buddha expressed this heartfelt sentiment: 

\begin{verse}%
“A\marginnote{8.1} mendicant who relies on alms, \\
self-supported, providing for no other; \\
the poised one is envied by even the gods, \\
but not if they’re after popularity and reputation.” 

%
\end{verse}

%
\section*{{\suttatitleacronym Ud 3.9}{\suttatitletranslation Professions }{\suttatitleroot Sippasutta}}
\addcontentsline{toc}{section}{\tocacronym{Ud 3.9} \toctranslation{Professions } \tocroot{Sippasutta}}
\markboth{Professions }{Sippasutta}
\extramarks{Ud 3.9}{Ud 3.9}

\scevam{So\marginnote{1.1} I have heard. }At one time the Buddha was staying near \textsanskrit{Sāvatthī} in Jeta’s Grove, \textsanskrit{Anāthapiṇḍika}’s monastery. Now at that time, after the meal, on return from almsround, several mendicants sat together in the pavilion and this discussion came up among them: “Who knows a craft? Who is studying which craft? Which is the best craft?” 

In\marginnote{2.1} answer, some said that elephant-craft is the best of crafts. Others said that the best craft is horse-craft, or chariot-craft, or archery, or swordsmanship, or computing, or accounting, or calculating, or writing, or poetry, or cosmology, or geomancy. At that point the conversation among those mendicants was left unfinished. 

Then\marginnote{3.1} in the late afternoon, the Buddha came out of retreat and went to the assembly hall. He sat down on the seat spread out, and addressed the mendicants: “Mendicants, what were you sitting talking about just now? What conversation was left unfinished?” 

So\marginnote{4.1} the mendicants told him what they had been talking about when the Buddha arrived. The Buddha said, 

“Mendicants,\marginnote{6.1} it is not appropriate for you gentlemen who have gone forth in faith from the lay life to homelessness to talk about such things. When you’re sitting together you should do one of two things: discuss the teachings or keep noble silence.” 

Then,\marginnote{7.1} understanding this matter, on that occasion the Buddha expressed this heartfelt sentiment: 

\begin{verse}%
“Living\marginnote{8.1} without a craft, light, desiring the good, \\
with senses controlled, everywhere free; \\
a migrant with no shelter, unselfish, with no need for hope, \\
having given up conceit, wandering alone: that is a mendicant.” 

%
\end{verse}

%
\section*{{\suttatitleacronym Ud 3.10}{\suttatitletranslation The World }{\suttatitleroot Lokasutta}}
\addcontentsline{toc}{section}{\tocacronym{Ud 3.10} \toctranslation{The World } \tocroot{Lokasutta}}
\markboth{The World }{Lokasutta}
\extramarks{Ud 3.10}{Ud 3.10}

\scevam{So\marginnote{1.1} I have heard. }At one time, when he was first awakened, the Buddha was staying near \textsanskrit{Uruvelā} at the root of the tree of awakening on the bank of the \textsanskrit{Nerañjarā} River. There the Buddha sat cross-legged for seven days without moving, experiencing the bliss of freedom. 

When\marginnote{2.1} seven days had passed, the Buddha emerged from that state of immersion and surveyed the world with the eye of a Buddha. He saw sentient beings tormented with many torments, and burning with many fevers born of greed, hate, and delusion. 

Then,\marginnote{3.1} understanding this matter, on that occasion the Buddha expressed this heartfelt sentiment: 

\begin{verse}%
“This\marginnote{4.1} world, born in torment, \\
overcome by contact, speaks of disease as the self. \\
For whatever it thinks it is, \\
it turns out to be something else. 

The\marginnote{5.1} world is attached to continued existence, overcome by continued existence, \\
taking pleasure only in continued existence, yet it becomes something else. \\
What it enjoys, that is the fear; \\
what it fears, that is the suffering. \\
But this spiritual life is led \\
in order to give up continued existence. 

%
\end{verse}

Of\marginnote{6.1} the ascetics and brahmins who say that through continued existence one is freed from continued existence, none are themselves freed from continued existence, I say. Of the ascetics and brahmins who say that through annihilation of existence one escapes from continued existence, none have themselves escaped from continued existence, I say. 

For\marginnote{7.1} this suffering originates dependent on all attachment. With the ending of all grasping there is no origination of suffering.\footnote{Thai editions read \textit{\textsanskrit{sabbupadhiṁ} hi}, which is supported by the \textsanskrit{Udānavarga}. } Just look at this world! Mired in all sorts of ignorance, beings in love with being are not released from continued existence.\footnote{Accepting \textit{\textsanskrit{bhavā} \textsanskrit{aparimuttā}} from the Buddha Jayanthi edition. For \textit{puthu}, the commentary says \textit{\textsanskrit{bahū}, \textsanskrit{visuṁ} \textsanskrit{visuṁ} \textsanskrit{vā}}. } Whatever states of continued existence there are—everywhere, all over—all are impermanent, suffering, and perishable. 

\begin{verse}%
One\marginnote{8.1} who sees truly like this, \\
with right wisdom, \\
gives up craving for continued existence, \\
while not look forward to ending existence. \\
Extinguishment comes from the ending of all cravings; \\
fading away and cessation with nothing left over. 

There\marginnote{9.1} is no further existence \\
for that mendicant extinguished without grasping. \\
Victorious in battle, such a one has defeated \textsanskrit{Māra}; \\
they’ve gone beyond all states of existence.” 

%
\end{verse}

%
\addtocontents{toc}{\let\protect\contentsline\protect\nopagecontentsline}
\chapter*{The Chapter with Meghiya }
\addcontentsline{toc}{chapter}{\tocchapterline{The Chapter with Meghiya }}
\addtocontents{toc}{\let\protect\contentsline\protect\oldcontentsline}

%
\section*{{\suttatitleacronym Ud 4.1}{\suttatitletranslation With Meghiya }{\suttatitleroot Meghiyasutta}}
\addcontentsline{toc}{section}{\tocacronym{Ud 4.1} \toctranslation{With Meghiya } \tocroot{Meghiyasutta}}
\markboth{With Meghiya }{Meghiyasutta}
\extramarks{Ud 4.1}{Ud 4.1}

\scevam{So\marginnote{1.1} I have heard. }At one time the Buddha was staying near \textsanskrit{Cālikā}, on the \textsanskrit{Cālikā} mountain. Now, at that time Venerable Meghiya was the Buddha’s attendant. Then Venerable Meghiya went up to the Buddha, bowed, stood to one side, and said to him, “Sir, I’d like to enter Jantu village for alms.” “Please, Meghiya, go at your convenience.” 

Then\marginnote{2.1} Meghiya robed up in the morning and, taking his bowl and robe, entered Jantu village for alms. After the meal, on his return from almsround in Jantu village, he went to the shore of \textsanskrit{Kimikālā} river. As he was going for a walk along the shore of the river he saw a lovely and delightful mango grove. When he saw this he thought, “Oh, this mango grove is lovely and delightful! It’s truly good enough for meditation for a gentleman who wants to meditate. If the Buddha allows me, I’ll come back to this mango grove to meditate.” 

Then\marginnote{3.1} Venerable Meghiya went up to the Buddha, bowed, sat down to one side, and told him what had happened, adding, 

“If\marginnote{4.1} the Buddha allows me, I’ll go back to that mango grove to meditate.” 

When\marginnote{5.1} he had spoken, the Buddha said to him, “We’re alone, Meghiya. Wait until another mendicant comes.” 

For\marginnote{6.1} a second time Meghiya said to the Buddha, “Sir, the Buddha has nothing more to do, and nothing that needs improvement. But I have. If you allow me, I’ll go back to that mango grove to meditate.” For a second time the Buddha said, “We’re alone, Meghiya. Wait until another mendicant comes.” 

For\marginnote{7.1} a third time Meghiya said to the Buddha, “Sir, the Buddha has nothing more to do, and nothing that needs improvement. But I have. If you allow me, I’ll go back to that mango grove to meditate.” “Meghiya, since you speak of meditation, what can I say? Please, Meghiya, go at your convenience.” 

Then\marginnote{8.1} Meghiya got up from his seat, bowed, and respectfully circled the Buddha, keeping him on his right. Then he went to that mango grove, and, having plunged deep into it, sat at the root of a certain tree for the day’s meditation. But while Meghiya was meditating in that mango grove he was beset mostly by three kinds of bad, unskillful thoughts, namely, sensual, malicious, and cruel thoughts. 

Then\marginnote{9.1} he thought, “It’s incredible, it’s amazing! I’ve gone forth from the lay life to homelessness out of faith, but I’m still harassed by these three kinds of bad, unskillful thoughts: sensual, malicious, and cruel thoughts.” 

Then\marginnote{10.1} in the late afternoon, Venerable Meghiya came out of retreat and went to the Buddha. He bowed, sat down to one side, and told the Buddha what had happened. 

“Meghiya,\marginnote{11.1} when the heart’s release is not ripe, five things help it ripen. What five? 

Firstly,\marginnote{12.1} a mendicant has good friends, companions, and associates. This is the first thing … 

Furthermore,\marginnote{13.1} a mendicant is ethical, restrained in the monastic code, conducting themselves well and seeking alms in suitable places. Seeing danger in the slightest fault, they keep the rules they’ve undertaken. This is the second thing … 

Furthermore,\marginnote{14.1} a mendicant gets to take part in talk about self-effacement that helps open the heart and leads solely to disillusionment, dispassion, cessation, peace, insight, awakening, and extinguishment when they want, without trouble or difficulty. That is, talk about fewness of wishes, contentment, seclusion, aloofness, arousing energy, ethics, immersion, wisdom, freedom, and the knowledge and vision of freedom.’ This is the third thing … 

Furthermore,\marginnote{15.1} a mendicant lives with energy roused up for giving up unskillful qualities and embracing skillful qualities. They are strong, staunchly vigorous, not slacking off when it comes to developing skillful qualities. This is the fourth thing … 

Furthermore,\marginnote{16.1} a mendicant is wise. They have the wisdom of arising and passing away which is noble, penetrative, and leads to the complete ending of suffering. This is the fifth thing that, when the heart’s release is not ripe, helps it ripen. These are the five things that, when the heart’s release is not ripe, help it ripen. 

A\marginnote{17.1} mendicant with good friends, companions, and associates can expect to be ethical … 

A\marginnote{18.1} mendicant with good friends, companions, and associates can expect to take part in talk about self-effacement that helps open the heart … 

A\marginnote{19.1} mendicant with good friends, companions, and associates can expect to be energetic … 

A\marginnote{20.1} mendicant with good friends, companions, and associates can expect to be wise … 

But\marginnote{21.1} then, a mendicant grounded on these five things should develop four further things. They should develop the perception of ugliness to give up greed, love to give up hate, mindfulness of breathing to cut off thinking, and perception of impermanence to uproot the conceit ‘I am’. When you perceive impermanence, the perception of not-self becomes stabilized. Perceiving not-self, you uproot the conceit ‘I am’ and attain extinguishment in this very life.” 

Then,\marginnote{22.1} understanding this matter, on that occasion the Buddha expressed this heartfelt sentiment: 

\begin{verse}%
“With\marginnote{23.1} thoughts whether low or fine, \\
excitement in the mind arises.\footnote{Following variant \textit{\textsanskrit{anuggatā}}. \textsanskrit{Udānavarga} here has \textit{\textsanskrit{samudgatāṁ}}. Commentary allows both \textit{anugata} and \textit{anuggata}. Here we read \textit{anu-(g)\textsanskrit{gatā}}, below \textit{an-uggate}. } \\
Not understanding these thoughts in the mind, \\
one with mind astray runs all over the place. 

Having\marginnote{24.1} understood these thoughts in the mind, \\
an awakened one—keen, restrained, and mindful—\\
has given up them all; \\
excitement in the mind no longer arises.” 

%
\end{verse}

%
\section*{{\suttatitleacronym Ud 4.2}{\suttatitletranslation Restless }{\suttatitleroot Uddhatasutta}}
\addcontentsline{toc}{section}{\tocacronym{Ud 4.2} \toctranslation{Restless } \tocroot{Uddhatasutta}}
\markboth{Restless }{Uddhatasutta}
\extramarks{Ud 4.2}{Ud 4.2}

\scevam{So\marginnote{1.1} I have heard. }At one time the Buddha was staying in the sal forest of the Mallas at Upavattana near \textsanskrit{Kusinārā}. Now at that time several mendicants were staying not far from the Buddha in a wilderness hut. They were restless, insolent, fickle, scurrilous, loose-tongued, unmindful, lacking situational awareness and immersion, with straying minds and undisciplined faculties. 

The\marginnote{2.1} Buddha saw those mendicants staying nearby. 

Then,\marginnote{3.1} understanding this matter, on that occasion the Buddha expressed this heartfelt sentiment: 

\begin{verse}%
“Unguarded\marginnote{4.1} in body, \\
ruined by wrong view,\footnote{The variant \textit{\textsanskrit{micchādiṭṭhigatena}} is the more obvious reading. However, \textit{\textsanskrit{micchādiṭṭhihatā}} appears at an4.49 without variants. Commentary accepts \textit{hata} in both places. Here it glosses \textit{\textsanskrit{sassatādimicchābhinivesadūsitena}}. } \\
overcome with dullness and drowsiness, \\
you fall under \textsanskrit{Māra}’s sway. 

That’s\marginnote{5.1} why you should guard the mind,\footnote{Commentary: \textit{rakkhitacitto assa}, literally “be one of guarded mind”. } \\
with right thoughts your pasture, \\
and right view at the fore. \\
Having understood rise and fall, \\
a mendicant who has overcome dullness and drowsiness \\
would cast off all bad destinies.” 

%
\end{verse}

%
\section*{{\suttatitleacronym Ud 4.3}{\suttatitletranslation The Cowherd }{\suttatitleroot Gopālakasutta}}
\addcontentsline{toc}{section}{\tocacronym{Ud 4.3} \toctranslation{The Cowherd } \tocroot{Gopālakasutta}}
\markboth{The Cowherd }{Gopālakasutta}
\extramarks{Ud 4.3}{Ud 4.3}

\scevam{So\marginnote{1.1} I have heard. }At one time the Buddha was wandering in the land of the Kosalans together with a large \textsanskrit{Saṅgha} of mendicants. And then the Buddha left the road, went to the root of a certain tree, and sat down on the seat spread out. 

Then\marginnote{2.1} a certain cowherd went up to the Buddha, bowed, and sat down to one side. The Buddha educated, encouraged, fired up, and inspired him with a Dhamma talk. 

Then\marginnote{3.1} the cowherd said to the Buddha, “Sir, may the Buddha together with the mendicant \textsanskrit{Saṅgha} please accept tomorrow’s meal from me.” The Buddha consented in silence. The cowherd got up from his seat, circumambulated the Buddha with his right side toward him, and left. 

And\marginnote{4.1} when the night had passed the cowherd had plenty of thick milk-rice prepared in his own home, with fresh ghee. Then he had the Buddha informed of the time, saying, “Sir, it’s time. The meal is ready.” 

The\marginnote{5.1} Buddha robed up, took his bowl and robe and, together with the Sangha of monks, went to the house of that cowherd, where he sat down on the prepared seat in the dining hall. Then the cowherd served and satisfied the mendicant \textsanskrit{Saṅgha} headed by the Buddha with his own hands with a thick milk-rice and fresh ghee. When the Buddha had eaten and washed his hand and bowl, the cowherd took a low seat and sat to one side. The Buddha then instructed, inspired, and gladdened him with a teaching, after which he got up and left. But soon after the Buddha had left, the cowherd was killed by a certain man in the gap between village borders. 

Then\marginnote{6.1} several mendicants went up to the Buddha, bowed, sat down to one side, and told him what had happened. 

Then,\marginnote{7.1} understanding this matter, on that occasion the Buddha expressed this heartfelt sentiment: 

\begin{verse}%
“A\marginnote{8.1} wrongly directed mind \\
would do you more harm \\
than a hater to the hated, \\
or an enemy to their foe.” 

%
\end{verse}

%
\section*{{\suttatitleacronym Ud 4.4}{\suttatitletranslation The Spirit’s Blow }{\suttatitleroot Yakkhapahārasutta}}
\addcontentsline{toc}{section}{\tocacronym{Ud 4.4} \toctranslation{The Spirit’s Blow } \tocroot{Yakkhapahārasutta}}
\markboth{The Spirit’s Blow }{Yakkhapahārasutta}
\extramarks{Ud 4.4}{Ud 4.4}

\scevam{So\marginnote{1.1} I have heard. }At one time the Buddha was staying near \textsanskrit{Rājagaha}, in the Bamboo Grove, the squirrels’ feeding ground. At that time Venerables \textsanskrit{Sāriputta} and \textsanskrit{Moggallāna} were staying near the pigeons’ alcove. Now at that time Venerable \textsanskrit{Sāriputta} was sitting outdoors in the moonlight, his head freshly shaven, having entered a certain state of immersion. 

Now\marginnote{2.1} at that time two native spirits who were friends were on their way from the north to the south on some business. They saw \textsanskrit{Sāriputta} meditating there. One of the spirits said to the other, “I feel inspired, friend, to give this ascetic a blow on the head!” The other spirit replied, “Enough, friend, don’t hit the ascetic! That is an eminent ascetic, powerful and mighty!” 

For\marginnote{3.1} a second time the first spirit said to the other, “I feel inspired, friend, to give this ascetic a blow on the head!” For a second time, the other spirit replied, “Enough, friend, don’t hit the ascetic! That is an eminent ascetic, powerful and mighty!” For a third time the first spirit said to the other, “I feel inspired, friend, to give this ascetic a blow on the head!” For a third time, the other spirit replied, “Enough, friend, don’t hit the ascetic! That is an eminent ascetic, powerful and mighty!” 

Ignoring\marginnote{4.1} his friend, the first spirit struck \textsanskrit{Sāriputta}. The blow was so strong it would have felled a bull elephant seven or seven and a half cubits tall, or split apart a great mountain peak. But then the spirit, crying out, “I burn, I burn!” fell into the Great Hell right there. 

With\marginnote{5.1} clairvoyance that is purified and superhuman, Venerable \textsanskrit{Moggallāna} saw that spirit striking Venerable \textsanskrit{Sāriputta}. He approached him and said, “I hope you’re keeping well, reverend; I hope you’re alright. I hope that you are not in pain.” “I am alright, Reverend \textsanskrit{Moggallāna}; but my head does hurt a little.” 

“It’s\marginnote{6.1} incredible, Reverend \textsanskrit{Sāriputta}, it’s amazing! How mighty and powerful is Venerable \textsanskrit{Sāriputta}! Just now, a native spirit struck you on the head. The blow was so strong it would have felled a bull elephant seven or seven and a half cubits tall, or split apart a great mountain peak. Yet you say, ‘I am alright, Reverend \textsanskrit{Moggallāna}; but my head does hurt a little.’” 

“It’s\marginnote{7.1} incredible, Reverend \textsanskrit{Moggallāna}, it’s amazing! How mighty and powerful is Venerable \textsanskrit{Moggallāna}, in that he can even see a native spirit! Whereas I can’t even see a mud-goblin right now.” 

With\marginnote{8.1} clairaudience that is purified and superhuman, the Buddha heard that discussion between those two spiritual giants. 

Then,\marginnote{9.1} understanding this matter, on that occasion the Buddha expressed this heartfelt sentiment: 

\begin{verse}%
“One\marginnote{10.1} whose mind is like a rock, \\
steady, never trembling, \\
free of desire for desirable things, \\
not getting annoyed when things are annoying: \\
from where will suffering strike one \\
whose mind is developed like this?” 

%
\end{verse}

%
\section*{{\suttatitleacronym Ud 4.5}{\suttatitletranslation A Bull Elephant }{\suttatitleroot Nāgasutta}}
\addcontentsline{toc}{section}{\tocacronym{Ud 4.5} \toctranslation{A Bull Elephant } \tocroot{Nāgasutta}}
\markboth{A Bull Elephant }{Nāgasutta}
\extramarks{Ud 4.5}{Ud 4.5}

\scevam{So\marginnote{1.1} I have heard. }At one time the Buddha was staying near Kosambi, in Ghosita’s Monastery. Now at that time Buddha lived crowded by monks, nuns, laymen, and laywomen; by rulers and their ministers, and teachers of other paths and their disciples. Crowded, he lived in suffering and discomfort. Then he thought, “These days I live crowded by monks, nuns, laymen, and laywomen; by rulers and their ministers, and teachers of other paths and their disciples. Crowded, I live in suffering and discomfort. Why don’t I live alone, withdrawn from the group?” 

Then\marginnote{2.1} the Buddha robed up in the morning and, taking his bowl and robe, entered Kosambi for alms. After the meal, on his return from almsround, he set his lodgings in order himself. Taking his bowl and robe, without informing his attendants or taking leave of the mendicant \textsanskrit{Saṅgha}, he set out to go wandering alone towards \textsanskrit{Pārileyya}, with no companion. When he eventually arrived, he stayed in a protected forest grove, at the foot of a sacred sal tree. 

A\marginnote{3.1} certain bull elephant was also living crowded by other males, females, younglings, and cubs. He ate the grass they’d trampled, and they ate the broken branches he dragged down. He drank muddy water, and after his bath the female elephants bumped into him. Crowded, he lived in suffering and discomfort. Then he thought, “These days I live crowded by other males, females, younglings, and cubs. I eat the grass they’ve trampled, and they eat the broken branches I’ve dragged down. I drink muddy water, and after my bath the female elephants bump into me. Crowded, I live in suffering and discomfort. Why don’t I live alone, withdrawn from the herd?” 

So\marginnote{4.1} he left the herd and went to \textsanskrit{Pārileyya}, where he approached the Buddha in the protected forest grove at the foot of a sacred sal tree. There he attended on the Buddha, clearing the vegetation from the place where the Buddha stayed, and using his trunk to set out water for drinking and washing. 

Then\marginnote{5.1} as the Buddha was in private retreat this thought came to his mind, “Formerly I lived crowded by monks, nuns, laymen, and laywomen; by rulers and their ministers, and teachers of other paths and their disciples. Crowded, I live in suffering and discomfort. But now I live uncrowded by monks, nuns, laymen, and laywomen; by rulers and their ministers, and teachers of other paths and their disciples. Being uncrowded, I live in happiness and comfort.” 

And\marginnote{6.1} to the bull elephant also this thought came to mind, “Formerly I lived crowded by other males, females, younglings, and cubs. I ate the grass they’d trampled, and they ate the broken branches I’d dragged down. I drank muddy water, and after my bath the female elephants bumped into me. Crowded, I lived in suffering and discomfort. But now I live uncrowded by other males, females, younglings, and cubs. I eat untrampled grass, and other elephants don’t eat the broken branches I have dragged down. I don’t drink muddy water, and the female elephants don’t bump into me after my bath. Being uncrowded, I live in happiness and comfort.” 

Then,\marginnote{7.1} understanding his own seclusion and knowing what that elephant was thinking, on that occasion the Buddha expressed this heartfelt sentiment: 

\begin{verse}%
“The\marginnote{8.1} giant elephant, \\
with tusks like chariot-poles, \\
agrees heart to heart with the spiritual giant, \\
since each finds joy in the woods alone.”\footnote{Commentary reads \textit{vane \textsanskrit{araññe}}, so text’s \textit{mano} must be wrong. } 

%
\end{verse}

%
\section*{{\suttatitleacronym Ud 4.6}{\suttatitletranslation The Alms-gatherer }{\suttatitleroot Piṇḍolasutta}}
\addcontentsline{toc}{section}{\tocacronym{Ud 4.6} \toctranslation{The Alms-gatherer } \tocroot{Piṇḍolasutta}}
\markboth{The Alms-gatherer }{Piṇḍolasutta}
\extramarks{Ud 4.6}{Ud 4.6}

\scevam{So\marginnote{1.1} I have heard. }At one time the Buddha was staying near \textsanskrit{Sāvatthī} in Jeta’s Grove, \textsanskrit{Anāthapiṇḍika}’s monastery. Now at that time Venerable \textsanskrit{Bhāradvāja} the Alms-gatherer was sitting not far from the Buddha, cross-legged, with his body straight. He was one who lived in the wilderness, ate only almsfood, wore rag robes, and owned just three robes. He was of few wishes, content, secluded, aloof, and energetic, an advocate of austerities, dedicated to the higher mind. 

The\marginnote{2.1} Buddha saw him meditating there. 

Then,\marginnote{3.1} understanding this matter, on that occasion the Buddha expressed this heartfelt sentiment: 

\begin{verse}%
“Not\marginnote{4.1} speaking ill nor doing harm; \\
restraint in the monastic code; \\
moderation in eating; \\
staying in remote lodgings; \\
commitment to the higher mind—\\
this is the instruction of the Buddhas.” 

%
\end{verse}

%
\section*{{\suttatitleacronym Ud 4.7}{\suttatitletranslation With Sāriputta }{\suttatitleroot Sāriputtasutta}}
\addcontentsline{toc}{section}{\tocacronym{Ud 4.7} \toctranslation{With Sāriputta } \tocroot{Sāriputtasutta}}
\markboth{With Sāriputta }{Sāriputtasutta}
\extramarks{Ud 4.7}{Ud 4.7}

\scevam{So\marginnote{1.1} I have heard. }At one time the Buddha was staying near \textsanskrit{Sāvatthī} in Jeta’s Grove, \textsanskrit{Anāthapiṇḍika}’s monastery. Now at that time Venerable \textsanskrit{Sāriputta} was sitting not far from the Buddha, cross-legged, with his body straight. He was of few wishes, content, secluded, aloof, energetic, dedicated to the higher mind. 

The\marginnote{2.1} Buddha saw him meditating there. 

Then,\marginnote{3.1} understanding this matter, on that occasion the Buddha expressed this heartfelt sentiment: 

\begin{verse}%
“A\marginnote{4.1} sage of higher consciousness, diligent, \\
training in the ways of sagacity: \\
there are no sorrows for one thus poised, \\
calm and ever mindful.” 

%
\end{verse}

%
\section*{{\suttatitleacronym Ud 4.8}{\suttatitletranslation With Sundarī }{\suttatitleroot Sundarīsutta}}
\addcontentsline{toc}{section}{\tocacronym{Ud 4.8} \toctranslation{With Sundarī } \tocroot{Sundarīsutta}}
\markboth{With Sundarī }{Sundarīsutta}
\extramarks{Ud 4.8}{Ud 4.8}

\scevam{So\marginnote{1.1} I have heard. }At one time the Buddha was staying near \textsanskrit{Sāvatthī} in Jeta’s Grove, \textsanskrit{Anāthapiṇḍika}’s monastery. Now at that time the Buddha was honored, respected, revered, venerated, and esteemed. And he received robes, almsfood, lodgings, and medicines and supplies for the sick. And the mendicant \textsanskrit{Saṅgha} was also honored, respected, revered, venerated, and esteemed. And they received robes, almsfood, lodgings, and medicines and supplies for the sick. But the wanderers who followed other paths were not honored, respected, revered, venerated, and esteemed. And they didn’t receive robes, almsfood, lodgings, and medicines and supplies for the sick. 

Then\marginnote{2.1} those wanderers who followed other paths, unable to bear the esteem of the mendicant Sangha, approached the female wanderer \textsanskrit{Sundarī} and said, “Sister, are you able to do something for the welfare of your kin?”\footnote{Anandajoti has “venture”, Thanissaro has “dare”, but a similar phrase at pli-tv-kd8:12.1.5 clearly has the weaker sense of “able”, per Ireland and commentary (\textit{\textsanskrit{ussahasīti} sakkosi}). } “What can I do, venerables? How can I help?\footnote{Accepting the variant \textit{\textsanskrit{kiṁ} \textsanskrit{mayā} \textsanskrit{sakkā}}, which is found in all similar passages. Reading, as Thanissaro and Anandajoti do, “what can I \emph{not} do” is pleasingly dramatic, but not very idiomatic for Pali. Far more likely a simple reinforcement is meant. Commentary ignores this phrase, implying it thought there was nothing unusual. } I’d even give my life for the welfare of my kin.” 

“Well\marginnote{3.1} then, sister, frequently visit Jeta’s Grove.” “Yes, venerables,” she replied, and did as they asked. 

When\marginnote{4.1} those wanderers knew that \textsanskrit{Sundarī} had been clearly seen by many people frequently visiting Jeta’s Grove, they killed her and dumped her in the ditch around Jeta’s Grove. Then they went to see King Pasenadi of Kosala and said to him, “Great king, we cannot find the female wanderer \textsanskrit{Sundarī}.” “But where do you suspect she is?” “At Jeta’s Grove, great king.” “Well then, search Jeta’s Grove.” 

So\marginnote{5.1} the wanderers searched Jeta’s Grove. They pulled her body up from the ditch where they had dumped it, and lifted it on a bier. Having entered \textsanskrit{Sāvatthī}, they went from street to street and from square to square, complaining to people: 

“See\marginnote{6.1} the deed of the Sakyan ascetics! Shameless are these Sakyan ascetics, immoral and wicked. They are liars and fake celibates. Sure, they claim to be of principled and moral conduct, to be celibate, truthful, ethical, and of good character.\footnote{The idiom \textit{ime hi \textsanskrit{nāma}} appears mildly emphatic in a disparaging way. Compare \textit{\textsanskrit{kathaṁ} hi \textsanskrit{nāma}} just below. } But they have no asceticism, no spirituality. Asceticism and spirituality are lost to them! Where is their asceticism, where their spirituality? They have abandoned asceticism and spirituality! How on earth can a man, having done a man’s business, kill a woman!” 

Then\marginnote{7.1} at that time when the people of \textsanskrit{Sāvatthī} saw the mendicants they abused and insulted them with rude, harsh words: 

“Shameless\marginnote{8.1} are these Sakyan ascetics, immoral, wicked, liars, and fake celibates. Sure, they claim to be of principled and moral conduct, to be celibate, truthful, ethical, and of good character. But they have no asceticism, no spirituality. Asceticism and spirituality are lost to them! Where is their asceticism, where their spirituality? They have abandoned asceticism and spirituality! How on earth can a man, having done a man’s business, kill a woman!” 

Then\marginnote{9.1} several mendicants robed up in the morning and, taking their bowls and robes, entered \textsanskrit{Sāvatthī} for alms. Then, after the meal, when they returned from almsround, they went up to the Buddha, bowed, sat down to one side, and told him what was happening. 

“That\marginnote{11.1} rumor, mendicants, won’t last long. It will only be seven days, then it will vanish. So you may respond to those critics with this verse: 

\begin{verse}%
A\marginnote{12.1} liar goes to hell, \\
as does one who denies what they did. \\
Both are equal in the hereafter, \\
those men of base deeds.” 

%
\end{verse}

The\marginnote{13.1} mendicants memorized that verse in the Buddha’s presence, then used it to respond to those critics. 

People\marginnote{15.1} thought, “These Sakyan ascetics didn’t do it, it was not done by them, they swear it.” That rumor didn’t last long. It was seven days, then it vanished. 

Then\marginnote{16.1} several mendicants went up to the Buddha, bowed, sat down to one side, and said to him, 

“It’s\marginnote{17.1} incredible, sir, it’s amazing! How well said this was by the Buddha: ‘That rumor, mendicants, won’t last long. It will only be seven days, then it will vanish.’ That rumor has vanished, sir.” 

Then,\marginnote{18.1} understanding this matter, on that occasion the Buddha expressed this heartfelt sentiment: 

\begin{verse}%
“People\marginnote{19.1} out of control stab with words, \\
like they stab a tusker in battle with darts. \\
When they hear a harsh word spoken, \\
a mendicant should endure with no anger in heart.” 

%
\end{verse}

%
\section*{{\suttatitleacronym Ud 4.9}{\suttatitletranslation With Upasena }{\suttatitleroot Upasenasutta}}
\addcontentsline{toc}{section}{\tocacronym{Ud 4.9} \toctranslation{With Upasena } \tocroot{Upasenasutta}}
\markboth{With Upasena }{Upasenasutta}
\extramarks{Ud 4.9}{Ud 4.9}

\scevam{So\marginnote{1.1} I have heard. }At one time the Buddha was staying near \textsanskrit{Rājagaha}, in the Bamboo Grove, the squirrels’ feeding ground. Then as Venerable Upasena son of \textsanskrit{Vaṅgantā} was in private retreat this thought came to his mind: “I’m so fortunate, so very fortunate! My Teacher is the Blessed One, perfected, a fully awakened Buddha. Well explained is the teaching and training in which I have gone forth from the lay life to homelessness. My spiritual companions are ethical and of good character. I have fulfilled the precepts. My mind is unified and serene. I am a perfected one who has ended the defilements. I am of great might and power. My life has been good, and my death will be good.” 

Then,\marginnote{2.1} knowing what that Venerable Upasena was thinking, on that occasion the Buddha expressed this heartfelt sentiment: 

\begin{verse}%
“One\marginnote{3.1} who has no guilt in life, \\
nor grief at facing death: \\
that wise one who has seen the truth, \\
grieves not among the grieving. 

For\marginnote{4.1} the mendicant with peaceful mind, \\
who has cut off craving for continued existence, \\
Transmigration through births is finished, \\
there are no future lives for them.” 

%
\end{verse}

%
\section*{{\suttatitleacronym Ud 4.10}{\suttatitletranslation The Peacefulness of Sāriputta }{\suttatitleroot Sāriputtaupasamasutta}}
\addcontentsline{toc}{section}{\tocacronym{Ud 4.10} \toctranslation{The Peacefulness of Sāriputta } \tocroot{Sāriputtaupasamasutta}}
\markboth{The Peacefulness of Sāriputta }{Sāriputtaupasamasutta}
\extramarks{Ud 4.10}{Ud 4.10}

\scevam{So\marginnote{1.1} I have heard. }At one time the Buddha was staying near \textsanskrit{Sāvatthī} in Jeta’s Grove, \textsanskrit{Anāthapiṇḍika}’s monastery. Now at that time Venerable \textsanskrit{Sāriputta} was sitting not far from the Buddha, cross-legged, with his body straight, reviewing his own peacefulness. 

The\marginnote{2.1} Buddha saw him meditating there. 

Then,\marginnote{3.1} understanding this matter, on that occasion the Buddha expressed this heartfelt sentiment: 

\begin{verse}%
“With\marginnote{4.1} mind at peace, so full of peace, \\
for a mendicant who has cut the cord, \\
transmigration through births is finished: \\
they’re freed from \textsanskrit{Māra}’s bonds.” 

%
\end{verse}

%
\addtocontents{toc}{\let\protect\contentsline\protect\nopagecontentsline}
\chapter*{The Chapter with Soṇa }
\addcontentsline{toc}{chapter}{\tocchapterline{The Chapter with Soṇa }}
\addtocontents{toc}{\let\protect\contentsline\protect\oldcontentsline}

%
\section*{{\suttatitleacronym Ud 5.1}{\suttatitletranslation Who Is More Dear? }{\suttatitleroot Piyatarasutta}}
\addcontentsline{toc}{section}{\tocacronym{Ud 5.1} \toctranslation{Who Is More Dear? } \tocroot{Piyatarasutta}}
\markboth{Who Is More Dear? }{Piyatarasutta}
\extramarks{Ud 5.1}{Ud 5.1}

\scevam{So\marginnote{1.1} I have heard. }At one time the Buddha was staying near \textsanskrit{Sāvatthī} in Jeta’s Grove, \textsanskrit{Anāthapiṇḍika}’s monastery. Now at that time King Pasenadi of Kosala was upstairs in the royal longhouse together with Queen \textsanskrit{Mallikā}. Then King Pasenadi addressed Queen \textsanskrit{Mallikā}, “\textsanskrit{Mallikā}, is there anyone more dear to you than yourself?” 

“No,\marginnote{2.1} great king, there isn’t. But is there anyone more dear to you than yourself?” “For me also, \textsanskrit{Mallikā}, there’s no-one.” 

Then\marginnote{3.1} King Pasenadi of Kosala came downstairs from the stilt longhouse, went to the Buddha, bowed, sat down to one side, and told him what had happened. 

Then,\marginnote{5.1} understanding this matter, on that occasion the Buddha expressed this heartfelt sentiment: 

\begin{verse}%
“Having\marginnote{6.1} explored every quarter with the mind, \\
Likewise for others, each holds themselves dear; \\
so one who loves themselves would harm no other.” 

%
\end{verse}

%
\section*{{\suttatitleacronym Ud 5.2}{\suttatitletranslation Short-lived }{\suttatitleroot Appāyukasutta}}
\addcontentsline{toc}{section}{\tocacronym{Ud 5.2} \toctranslation{Short-lived } \tocroot{Appāyukasutta}}
\markboth{Short-lived }{Appāyukasutta}
\extramarks{Ud 5.2}{Ud 5.2}

\scevam{So\marginnote{1.1} I have heard. }At one time the Buddha was staying near \textsanskrit{Sāvatthī} in Jeta’s Grove, \textsanskrit{Anāthapiṇḍika}’s monastery. Then in the late afternoon, Venerable Ānanda came out of retreat and went to the Buddha. He bowed, sat down to one side, and said to him: “It’s incredible, sir, it’s amazing! How short-lived was the Blessed One’s mother! For seven days after the Blessed One was born, his mother passed away and was reborn in the host of Joyful Gods.”\footnote{Here I use “Blessed One” rather than my usual “Buddha” for \textit{\textsanskrit{bhagavā}}. Normally the difference is immaterial, as it is simply the most common way to address the Buddha. In this case, however, the passage Ānanda is referring to uses \textit{bodhisatta}, and it is quite striking that he changes it here. } 

“That’s\marginnote{2.1} so true, Ānanda! For the mothers of beings intent of awakening are short-lived.\footnote{The Buddha switches back to \textit{bodhisatta}, and depersonalizes it by using the plural. } Seven days after the beings intent on awakening are born, their mothers pass away and are reborn in the host of Joyful Gods.” 

Then,\marginnote{3.1} understanding this matter, on that occasion the Buddha expressed this heartfelt sentiment: 

\begin{verse}%
“Whether\marginnote{4.1} born or to be born, \\
all depart, leaving the body behind. \\
The skillful, understanding that all is lost, \\
would keenly practice the spiritual life.” 

%
\end{verse}

%
\section*{{\suttatitleacronym Ud 5.3}{\suttatitletranslation With Suppabuddha the Leper }{\suttatitleroot Suppabuddhakuṭṭhisutta}}
\addcontentsline{toc}{section}{\tocacronym{Ud 5.3} \toctranslation{With Suppabuddha the Leper } \tocroot{Suppabuddhakuṭṭhisutta}}
\markboth{With Suppabuddha the Leper }{Suppabuddhakuṭṭhisutta}
\extramarks{Ud 5.3}{Ud 5.3}

\scevam{So\marginnote{1.1} I have heard. }At one time the Buddha was staying near \textsanskrit{Rājagaha}, in the Bamboo Grove, the squirrels’ feeding ground. At that time in \textsanskrit{Rājagaha} there was a leper called Suppabuddha. He was poor, destitute, and pitiful. Now, at that time the Buddha was teaching Dhamma, surrounded by a large assembly. 

Suppabuddha\marginnote{2.1} saw the crowd seated off in the distance and thought, “Surely some variety of foods will be distributed there. Why don’t I approach the crowd? Hopefully I’ll get something to eat.” 

So\marginnote{3.1} he approached the crowd where he saw the Buddha teaching Dhamma, surrounded by a large assembly. When he saw this he thought, “There’s no food being distributed here. The ascetic Gotama is teaching Dhamma in an assembly. Why don’t I also listen to the teaching?” Right there he sat down to one side, thinking, “I also will listen to the teaching.” 

Then\marginnote{4.1} the Buddha focused on comprehending the minds of everyone in the assembly, thinking, “Who here is capable of understanding the teaching?” He saw Suppabuddha sitting in the assembly,\footnote{It’s interesting to note that, despite the perhaps derogatory sense of \textit{\textsanskrit{kuṭṭhi}} (“leper”), there is no question that Suppabuddha sits in the assembly and is not physically exlcuded. } and thought, “He is capable of understanding the teaching.” He gave a step by step talk especially for Suppabuddha on giving, ethical conduct, and heaven. He explained the drawbacks of sensual pleasures, so sordid and corrupt, and the benefit of renunciation. When the Buddha knew that Suppabuddha’s mind was ready, supple, without hindrances, elated, and confident, he revealed the teaching unique to the Buddhas: suffering, its origin, its cessation, and the path. Just as a clean cloth rid of stains would properly absorb dye, in that very seat the stainless, immaculate vision of the Dhamma in Suppabuddha: “Everything that has a beginning has an end.” 

Then\marginnote{5.1} Suppabuddha saw, attained, understood, and fathomed the Dhamma. He went beyond doubt, got rid of indecision, and became self-assured and independent of others regarding the Teacher’s instructions. He rose from his seat and went to the Buddha, bowed, sat down to one side, and said: 

“Excellent,\marginnote{6.1} sir! Excellent! As if he were righting the overturned, or revealing the hidden, or pointing out the path to the lost, or lighting a lamp in the dark so people with good eyes can see what’s there, the Buddha has made the teaching clear in many ways. I go for refuge to the Buddha, to the teaching, and to the mendicant \textsanskrit{Saṅgha}. From this day forth, may the Buddha remember me as a lay follower who has gone for refuge for life.” 

After\marginnote{7.1} Suppabuddha had been educated, encouraged, fired up, and inspired with a Dhamma talk by the Buddha, he got up from his seat, bowed, and respectfully circled the Buddha before leaving. But soon after Suppabuddha had left, a cow with a baby calf charged at him and took his life. 

Then\marginnote{8.1} several mendicants went up to the Buddha, bowed, sat down to one side, and said, “The leper called Suppabuddha, after being educated, encouraged, fired up, and inspired with a Dhamma talk by the Buddha, has passed away. Where has he been reborn in his next life?” 

“Mendicants,\marginnote{9.1} Suppabuddha was astute. He practiced in line with the teachings, and did not trouble me about the teachings. With the ending of three fetters, Suppabuddha is a stream-enterer, not liable to be reborn in the underworld, bound for awakening.” 

When\marginnote{10.1} he said this, one of the mendicants said to the Buddha, “What is the cause, sir, what is the reason why Suppabuddha became a leper, poor, destitute, and pitiful?” 

“Once\marginnote{11.1} upon a time, mendicants, Suppabuddha used to be a financier’s son right here in \textsanskrit{Rājagaha}. On his way to visit a park, he saw \textsanskrit{Tagarasikhī}, a Buddha awakened for themselves, entering the city for alms. When he saw this he thought, “Who is this leper wandering about with his leper’s robe?” Before leaving, he spat on the ground and turned his left side to \textsanskrit{Tagarasikhī}. As a result of that deed he burned in hell for many years, for many hundreds, many thousands, many hundreds of thousands of years. And as a residual result of that same deed, he became a leper right here in \textsanskrit{Rājagaha}, poor, destitute and pitiful. But owing to faith in the teaching and training proclaimed by the Realized One, he undertook ethics, learning, generosity, and wisdom. After undertaking these things, when his body broke up, after death, he was reborn in a good place, a heavenly realm, in the company of the gods of the Thirty-Three. There he outshines the other gods in beauty and glory.” 

Then,\marginnote{12.1} understanding this matter, on that occasion the Buddha expressed this heartfelt sentiment: 

\begin{verse}%
“As\marginnote{13.1} a well-sighted man would avoid rough paths, \\
so long as strength is found; \\
an astute person in the living world, \\
would shun bad deeds.” 

%
\end{verse}

%
\section*{{\suttatitleacronym Ud 5.4}{\suttatitletranslation The Boys }{\suttatitleroot Kumārakasutta}}
\addcontentsline{toc}{section}{\tocacronym{Ud 5.4} \toctranslation{The Boys } \tocroot{Kumārakasutta}}
\markboth{The Boys }{Kumārakasutta}
\extramarks{Ud 5.4}{Ud 5.4}

\scevam{So\marginnote{1.1} I have heard. }At one time the Buddha was staying near \textsanskrit{Sāvatthī} in Jeta’s Grove, \textsanskrit{Anāthapiṇḍika}’s monastery. Now at that time, between \textsanskrit{Sāvatthī} and the Jeta Grove, several boys were tormenting some fish. 

Then\marginnote{2.1} the Buddha robed up in the morning and, taking his bowl and robe, entered \textsanskrit{Sāvatthī} for alms. He saw the boys tormenting the fish. He went to them and said, “Boys, do you fear pain? Do you dislike pain?” “Yes, sir,” they replied. “We dislike pain.” 

Then,\marginnote{3.1} understanding this matter, on that occasion the Buddha expressed this heartfelt sentiment: 

\begin{verse}%
“If\marginnote{4.1} you fear pain, \\
if you dislike pain, \\
don’t do bad deeds \\
either openly or in secret. 

If\marginnote{5.1} you should do a bad deed, \\
now or in the future, \\
you won’t be freed from suffering, \\
though you fly away and flee.” 

%
\end{verse}

%
\section*{{\suttatitleacronym Ud 5.5}{\suttatitletranslation Sabbath }{\suttatitleroot Uposathasutta}}
\addcontentsline{toc}{section}{\tocacronym{Ud 5.5} \toctranslation{Sabbath } \tocroot{Uposathasutta}}
\markboth{Sabbath }{Uposathasutta}
\extramarks{Ud 5.5}{Ud 5.5}

\scevam{So\marginnote{1.1} I have heard. }At one time the Buddha was staying near \textsanskrit{Sāvatthī} in the Eastern Monastery, the stilt longhouse of \textsanskrit{Migāra}’s mother. Now, at that time it was the sabbath, and the Buddha was sitting surrounded by the \textsanskrit{Saṅgha} of monks. 

And\marginnote{2.1} then, as the night was getting late, in the first watch of the night, Venerable Ānanda got up from his seat, arranged his robe over one shoulder, raised his joined palms toward the Buddha and said, “Sir, the night is getting late. It is the first watch of the night, and the \textsanskrit{Saṅgha} has been sitting long. Please, sir, may the Buddha recite the monastic code to the mendicants.” But when he said this, the Buddha kept silent. 

For\marginnote{3.1} a second time, as the night was getting late, in the middle watch of the night, Ānanda got up from his seat, arranged his robe over one shoulder, raised his joined palms toward the Buddha and said, “Sir, the night is getting late. It is the first watch of the night, and the \textsanskrit{Saṅgha} has been sitting long. Please, sir, may the Buddha recite the monastic code to the mendicants.” But for a second time the Buddha kept silent. 

For\marginnote{4.1} a third time, as the night was getting late, in the last watch of the night, as dawn stirred, bringing joy to the night, Ānanda got up from his seat, arranged his robe over one shoulder, raised his joined palms toward the Buddha and said, “Sir, the night is getting late. It is the last watch of the night and dawn stirs, bringing joy to the night. Please, sir, may the Buddha recite the monastic code to the mendicants.” “Ānanda, the assembly is not pure.” 

Then\marginnote{5.1} Venerable \textsanskrit{Mahāmoggallāna} thought, “Who is the Buddha talking about?” Then he focused on comprehending the minds of everyone in the \textsanskrit{Saṅgha}. He saw that unethical person, of bad qualities, filthy, with suspicious behavior, underhand, no true ascetic or spiritual practitioner—though claiming to be one—rotten inside, corrupt, and depraved, sitting in the middle of the \textsanskrit{Saṅgha}. When he saw him he got up from his seat, went up to him and said, “Get up, reverend. The Buddha has seen you. You can’t live in communion with the mendicants.” But when he said this, that person kept silent. 

For\marginnote{6.1} a second time and a third time, he asked that monk to leave. But for a third time that person kept silent. 

Then\marginnote{7.1} Venerable \textsanskrit{Mahāmoggallāna} took that person by the arm, ejected him out the gate, and bolted the door. Then he went up to the Buddha, and said to him, “I have ejected that person. The assembly is pure. Please, sir, may the Buddha recite the monastic code to the mendicants.” “It’s incredible, \textsanskrit{Moggallāna}, it’s amazing, how that silly man waited to be taken by the arm!” 

Then\marginnote{8.1} the Buddha said to the mendicants, “From this day forth, mendicants, I will not perform the sabbath or recite the monastic code. Now you should perform the sabbath and recite the monastic code. It’s impossible, mendicants, it can’t happen that a Realized One could recite the monastic code in an impure assembly. 

Seeing\marginnote{9.1} these eight incredible and amazing things the demons love the ocean. What eight? 

The\marginnote{10.1} ocean gradually slants, slopes, and inclines, with no abrupt precipice. This is the first thing the demons love about the ocean. 

Furthermore,\marginnote{11.1} the ocean is consistent and doesn’t overflow its boundaries. This is the second thing the demons love about the ocean. 

Furthermore,\marginnote{12.1} the ocean doesn’t accommodate a corpse, but quickly carries it to the shore and strands it on the beach. This is the third thing the demons love about the ocean. 

Furthermore,\marginnote{13.1} when they reach the ocean, all the great rivers—that is, the Ganges, \textsanskrit{Yamunā}, \textsanskrit{Aciravatī}, \textsanskrit{Sarabhū}, and \textsanskrit{Mahī}—lose their names and clans and are simply considered ‘the ocean’. This is the fourth thing the demons love about the ocean. 

Furthermore,\marginnote{14.1} for all the world’s streams that reach it, and the rain that falls from the sky, the ocean never empties or fills up. This is the fifth thing the demons love about the ocean. 

Furthermore,\marginnote{15.1} the ocean has just one taste, the taste of salt. This is the sixth thing the demons love about the ocean. 

Furthermore,\marginnote{16.1} the ocean is full of many kinds of treasures, such as pearls, gems, beryl, conch, quartz, coral, silver, gold, rubies, and emeralds. This is the seventh thing the demons love about the ocean. 

Furthermore,\marginnote{17.1} many great beings live in the ocean, such as leviathans, leviathan-gulpers, leviathan-gulper-gulpers, demons, dragons, and fairies. In the ocean there are life-forms a hundred leagues long, or even two hundred, three hundred, four hundred, or five hundred leagues long. This is the eighth thing the demons love about the ocean. Seeing these eight incredible and amazing things the demons love the ocean. 

In\marginnote{18.1} the same way, seeing eight incredible and amazing things, mendicants, the mendicants love this teaching and training. What eight? 

The\marginnote{19.1} ocean gradually slants, slopes, and inclines, with no abrupt precipice. In the same way in this teaching and training the penetration to enlightenment comes from gradual training, progress, and practice, not abruptly. This is the first thing the mendicants love about this teaching and training. 

The\marginnote{20.1} ocean is consistent and doesn’t overflow its boundaries. In the same way, when a training rule is laid down for my disciples they wouldn’t break it even for the sake of their own life. This is the second thing the mendicants love about this teaching and training. 

The\marginnote{21.1} ocean doesn’t accommodate a corpse, but quickly carries it to the shore and strands it on the beach. In the same way, the \textsanskrit{Saṅgha} doesn’t accommodate a person who is unethical, of bad qualities, filthy, with suspicious behavior, underhand, no true ascetic or spiritual practitioner—though claiming to be one—rotten inside, corrupt, and depraved. But they quickly gather and expel them. Even if such a person is sitting in the middle of the \textsanskrit{Saṅgha}, they’re far from the \textsanskrit{Saṅgha}, and the \textsanskrit{Saṅgha} is far from them. This is the third thing the mendicants love about this teaching and training. 

Furthermore,\marginnote{22.1} when they reach the ocean, all the great rivers—that is, the Ganges, \textsanskrit{Yamunā}, \textsanskrit{Aciravatī}, \textsanskrit{Sarabhū}, and \textsanskrit{Mahī}—lose their names and clans and are simply considered ‘the ocean’. In the same way, when they go forth from the lay life to homelessness, all four castes—aristocrats, brahmins, merchants, and workers—lose their former names and clans and are simply considered ‘Sakyan ascetics’. This is the fourth thing the mendicants love about this teaching and training. 

For\marginnote{23.1} all the world’s streams that reach it, and the rain that falls from the sky, the ocean never empties or fills up. In the same way, though several mendicants become fully extinguished through the element of extinguishment with nothing left over, the element of extinguishment never empties or fills up. This is the fifth thing the mendicants love about this teaching and training. 

The\marginnote{24.1} ocean has just one taste, the taste of salt. In the same way, this teaching and training has one taste, the taste of freedom. This is the sixth thing the mendicants love about this teaching and training. 

The\marginnote{25.1} ocean is full of many kinds of treasures, such as pearls, gems, beryl, conch, quartz, coral, silver, gold, rubies, and emeralds. In the same way, this teaching and training is full of many kinds of treasures, such as the four kinds of mindfulness meditation, the four right efforts, the four bases of psychic power, the five faculties, the five powers, the seven awakening factors, and the noble eightfold path. This is the seventh thing the mendicants love about this teaching and training. 

Many\marginnote{26.1} great beings live in the ocean, such as leviathans, leviathan-gulpers, leviathan-gulper-gulpers, demons, dragons, and fairies. In the ocean there are life-forms a hundred leagues long, or even two hundred, three hundred, four hundred, or five hundred leagues long. In the same way, great beings live in this teaching and training, and these are those beings. The stream-enterer and the one practicing to realize the fruit of stream-entry. The once-returner and the one practicing to realize the fruit of once-return. The non-returner and the one practicing to realize the fruit of non-return. The perfected one, and the one practicing for perfection. This is the eighth thing the mendicants love about this teaching and training. Seeing these eight incredible and amazing things, the mendicants love this teaching and training.” 

Then,\marginnote{27.1} understanding this matter, on that occasion the Buddha expressed this heartfelt sentiment: 

\begin{verse}%
“The\marginnote{28.1} rain saturates things that are covered up; \\
it doesn’t saturate things that are open. \\
Therefore you should open up a covered thing, \\
so the rain will not saturate it.” 

%
\end{verse}

%
\section*{{\suttatitleacronym Ud 5.6}{\suttatitletranslation With Soṇa }{\suttatitleroot Soṇasutta}}
\addcontentsline{toc}{section}{\tocacronym{Ud 5.6} \toctranslation{With Soṇa } \tocroot{Soṇasutta}}
\markboth{With Soṇa }{Soṇasutta}
\extramarks{Ud 5.6}{Ud 5.6}

\scevam{So\marginnote{1.1} I have heard. }At one time the Buddha was staying near \textsanskrit{Sāvatthī} in Jeta’s Grove, \textsanskrit{Anāthapiṇḍika}’s monastery. Now at that time Venerable \textsanskrit{Mahākaccāna} was staying in the land of the Avantis near Kuraraghara on Steep Mountain. And the layman \textsanskrit{Soṇa} of the Sharp Ears was \textsanskrit{Mahākaccāna}’s attendant. 

Then\marginnote{2.1} as \textsanskrit{Soṇa} was in private retreat this thought came to his mind, “As I understand Venerable \textsanskrit{Mahākaccāna}’s teachings, it’s not easy for someone living at home to lead the spiritual life utterly full and pure, like a polished shell. Why don’t I shave off my hair and beard, dress in ocher robes, and go forth from lay life to homelessness?” 

Then\marginnote{3.1} \textsanskrit{Soṇa} went up to \textsanskrit{Mahākaccāna}, bowed, sat down to one side, and told him what he was thinking. Then he said, 

“May\marginnote{4.1} Venerable \textsanskrit{Mahākaccāna} please give me the going forth!” 

When\marginnote{5.1} this was said, \textsanskrit{Mahākaccāna} said to him, “It’s hard to lead the spiritual life as long as you live, eating in one part of the day and sleeping alone. Come now, \textsanskrit{Soṇa}, while remaining a layperson just as you are, devote yourself to the instructions of the Buddhas, leading the spiritual life at suitable times, eating in one part of the day and sleeping alone.” Then \textsanskrit{Soṇa}’s aspiration to go forth died down. 

For\marginnote{6.1} a second time, while in private retreat the thought came to \textsanskrit{Soṇa} that he should go forth, but the outcome was the same. 

For\marginnote{7.1} a third time, as \textsanskrit{Soṇa} was in private retreat this thought came to his mind, “As I understand Venerable \textsanskrit{Mahākaccāna}’s teachings, it’s not easy for someone living at home to lead the spiritual life utterly full and pure, like a polished shell. Why don’t I shave off my hair and beard, dress in ocher robes, and go forth from lay life to homelessness?” For a third time, \textsanskrit{Soṇa} went up to \textsanskrit{Mahākaccāna}, bowed, sat down to one side, and told him what he was thinking. Then he said, 

“May\marginnote{8.1} Venerable \textsanskrit{Mahākaccāna} please give me the going forth!” 

Then\marginnote{9.1} \textsanskrit{Mahākaccāna} gave \textsanskrit{Soṇa} the going forth. Now at that time the southern region, including Avanti, was short of monks. It took three years and much struggle and difficulty before Venerable \textsanskrit{Mahākaccāyana} was able to assemble from here and there a Sangha consisting of ten monks and give Venerable \textsanskrit{Soṇa} the full ordination. 

Then\marginnote{10.1} as Venerable \textsanskrit{Soṇa} was in private retreat this thought came to his mind, “I have not personally seen the Buddha. I have only heard reports that that Blessed One is like this or like that. If my mentor allows, I should go to see that Blessed One, the perfected one, the fully awakened Buddha.” 

Then\marginnote{11.1} in the late afternoon, \textsanskrit{Soṇa} came out of retreat, went up to \textsanskrit{Mahākaccāna}, bowed, sat down to one side, and told him what he was thinking. Then \textsanskrit{Mahākaccāna} said, 

“Good,\marginnote{13.1} good, \textsanskrit{Soṇa}! Go to see the Blessed One, the perfected one, the fully awakened Buddha. You will see that Blessed One who is impressive and inspiring, with peaceful faculties and mind, attained to the highest self-control and serenity, like an elephant with tamed, guarded, and controlled faculties. On seeing him, in my name bow with your head to his feet. Ask him if he is healthy and well, nimble, strong, and living comfortably, saying, ‘Sir, my mentor Venerable \textsanskrit{Mahākaccāna} bows with his head to your feet. He asks if you are healthy and well, nimble, strong, and living comfortably.’” 

Saying,\marginnote{14.1} “Yes, sir,” \textsanskrit{Soṇa} welcomed and agreed with \textsanskrit{Mahākaccāna}’s words. He got up from his seat, bowed, and respectfully circled \textsanskrit{Mahākaccāna}, keeping him on his right. Then he set his lodgings in order and, taking his bowl and robe, set out for \textsanskrit{Sāvatthī}. Eventually he came to \textsanskrit{Sāvatthī} and Jeta’s Grove. He went up to the Buddha, bowed, and sat down to one side. \textsanskrit{Soṇa} said to the Buddha, “Sir, my mentor Venerable \textsanskrit{Mahākaccāna} bows with his head to your feet. He asks if you are healthy and well, nimble, strong, and living comfortably.” 

“I\marginnote{15.1} hope you’re keeping well, mendicant; I hope you’re all right. And I hope you have arrived from your journey unwearied, having had no trouble getting almsfood.” “I’m keeping well, sir; I’m all right. And I have arrived from my journey unwearied, having had no trouble getting almsfood.” 

Then\marginnote{16.1} the Buddha said to Venerable Ānanda, “Prepare lodgings for this visiting mendicant.” Then Venerable Ānanda thought, “When the Buddha orders me to prepare lodgings for a specific mendicant, he wishes to stay in the same dwelling with that mendicant. The Buddha wishes to stay together with Venerable \textsanskrit{Soṇa}.” He prepared lodgings for \textsanskrit{Soṇa} in the same dwelling where the Buddha was staying. 

The\marginnote{17.1} Buddha spent most of the night siting meditation in the open. Then he got up from his seat, washed his feet and entered the dwelling. Venerable \textsanskrit{Soṇa} did the same. Then the Buddha rose at the crack of dawn and addressed \textsanskrit{Soṇa}, “Speak some Dhamma, mendicant, as you feel inspired.” 

“Yes,\marginnote{18.1} sir,” replied \textsanskrit{Soṇa}. He chanted all sixteen discourses in the Chapter of the Eights. When \textsanskrit{Soṇa} finished his chanting, the Buddha applauded, saying, “Good, good, mendicant! You have learned the sixteen discourses of the Chapter of the Eights well, you have attended and remembered it well. You are a good speaker, with a polished, clear, and articulate voice that expresses the meaning. How many rains have you been ordained, mendicant?” “I have one rains, Blessed One.” “But why did it take you so long to make it?” “Sir, I have long seen the drawbacks of sensual pleasures, yet living in a house is cramped, with many duties and much to do.” 

Then,\marginnote{19.1} understanding this matter, on that occasion the Buddha expressed this heartfelt sentiment: 

\begin{verse}%
“Seeing\marginnote{20.1} the danger of the world, \\
I understood the reality without attachments. \\
The Noble One does not delight in evil, \\
the Pure One does not delight in evil.” 

%
\end{verse}

%
\section*{{\suttatitleacronym Ud 5.7}{\suttatitletranslation With Revata the Doubter }{\suttatitleroot Kaṅkhārevatasutta}}
\addcontentsline{toc}{section}{\tocacronym{Ud 5.7} \toctranslation{With Revata the Doubter } \tocroot{Kaṅkhārevatasutta}}
\markboth{With Revata the Doubter }{Kaṅkhārevatasutta}
\extramarks{Ud 5.7}{Ud 5.7}

\scevam{So\marginnote{1.1} I have heard. }At one time the Buddha was staying near \textsanskrit{Sāvatthī} in Jeta’s Grove, \textsanskrit{Anāthapiṇḍika}’s monastery. Now at that time Venerable Revata the Doubter was sitting not far from the Buddha, cross-legged, with his body straight, reviewing his own purification through overcoming doubt. 

The\marginnote{2.1} Buddha saw him meditating there. 

Then,\marginnote{3.1} understanding this matter, on that occasion the Buddha expressed this heartfelt sentiment: 

\begin{verse}%
“Any\marginnote{4.1} doubts about this world or the world beyond, \\
about one’s own experiences or those of another: \\
those who meditate give them all up, \\
keenly practicing the spiritual life.” 

%
\end{verse}

%
\section*{{\suttatitleacronym Ud 5.8}{\suttatitletranslation Schism in the Saṅgha }{\suttatitleroot Saṁghabhedasutta}}
\addcontentsline{toc}{section}{\tocacronym{Ud 5.8} \toctranslation{Schism in the Saṅgha } \tocroot{Saṁghabhedasutta}}
\markboth{Schism in the Saṅgha }{Saṁghabhedasutta}
\extramarks{Ud 5.8}{Ud 5.8}

\scevam{So\marginnote{1.1} I have heard. }At one time the Buddha was staying near \textsanskrit{Rājagaha}, in the Bamboo Grove, the squirrels’ feeding ground. Now at that time Venerable Ānanda robed up in the morning and, taking his bowl and robe, entered \textsanskrit{Rājagaha} for alms. 

Devadatta\marginnote{2.1} saw him wandering for alms, so he went up to him and said, “From this day forth, Reverend Ānanda, I shall perform the sabbath and legal proceedings of the \textsanskrit{Saṅgha} apart from the Buddha and the \textsanskrit{Saṅgha} of mendicants.” 

Then\marginnote{3.1} Ānanda wandered for alms in \textsanskrit{Rājagaha}. After the meal, on his return from almsround, he went to the Buddha, bowed, sat down to one side, and told him what had just happened, adding: 

“Today,\marginnote{4.1} sir, Devadatta will split the \textsanskrit{Saṅgha}. He will perform the sabbath and legal proceedings of the \textsanskrit{Saṅgha}.” 

Then,\marginnote{5.1} understanding this matter, on that occasion the Buddha expressed this heartfelt sentiment: 

\begin{verse}%
“It’s\marginnote{6.1} easy for the good to do good, \\
and hard for the good to do bad. \\
It’s easy for the bad to do bad, \\
but for the noble ones, bad is hard to do.” 

%
\end{verse}

%
\section*{{\suttatitleacronym Ud 5.9}{\suttatitletranslation Teasing }{\suttatitleroot Sadhāyamānasutta}}
\addcontentsline{toc}{section}{\tocacronym{Ud 5.9} \toctranslation{Teasing } \tocroot{Sadhāyamānasutta}}
\markboth{Teasing }{Sadhāyamānasutta}
\extramarks{Ud 5.9}{Ud 5.9}

\scevam{So\marginnote{1.1} I have heard. }At one time the Buddha was wandering in the land of the Kosalans together with a large \textsanskrit{Saṅgha} of mendicants. Now at that time several students were passing by not far from the Buddha in a teasing manner. The Buddha saw them. 

Then,\marginnote{2.1} understanding this matter, on that occasion the Buddha expressed this heartfelt sentiment: 

\begin{verse}%
“Dolts\marginnote{3.1} pretending to be astute, \\
they talk, their words right out of bounds. \\
They blab at will, their mouths agape, \\
and no-one knows what leads them on.” 

%
\end{verse}

%
\section*{{\suttatitleacronym Ud 5.10}{\suttatitletranslation With Cūḷapanthaka }{\suttatitleroot Cūḷapanthakasutta}}
\addcontentsline{toc}{section}{\tocacronym{Ud 5.10} \toctranslation{With Cūḷapanthaka } \tocroot{Cūḷapanthakasutta}}
\markboth{With Cūḷapanthaka }{Cūḷapanthakasutta}
\extramarks{Ud 5.10}{Ud 5.10}

\scevam{So\marginnote{1.1} I have heard. }At one time the Buddha was staying near \textsanskrit{Sāvatthī} in Jeta’s Grove, \textsanskrit{Anāthapiṇḍika}’s monastery. Now at that time Venerable \textsanskrit{Cūḷapanthaka} was sitting not far from the Buddha, cross-legged, with his body straight, and mindfulness established right there. 

The\marginnote{2.1} Buddha saw him meditating there. 

Then,\marginnote{3.1} understanding this matter, on that occasion the Buddha expressed this heartfelt sentiment: 

\begin{verse}%
“Steady\marginnote{4.1} in body, steady in mind, \\
standing, sitting or lying down: \\
a mendicant focusing on this mindfulness \\
gains an ever higher distinction. \\
And when they have done so, \\
they vanish from the King of Death.” 

%
\end{verse}

%
\addtocontents{toc}{\let\protect\contentsline\protect\nopagecontentsline}
\chapter*{The Chapter on Blind From Birth }
\addcontentsline{toc}{chapter}{\tocchapterline{The Chapter on Blind From Birth }}
\addtocontents{toc}{\let\protect\contentsline\protect\oldcontentsline}

%
\section*{{\suttatitleacronym Ud 6.1}{\suttatitletranslation Surrendering the Life Force }{\suttatitleroot Āyusaṅkhārossajjanasutta}}
\addcontentsline{toc}{section}{\tocacronym{Ud 6.1} \toctranslation{Surrendering the Life Force } \tocroot{Āyusaṅkhārossajjanasutta}}
\markboth{Surrendering the Life Force }{Āyusaṅkhārossajjanasutta}
\extramarks{Ud 6.1}{Ud 6.1}

\scevam{So\marginnote{1.1} I have heard. }At one time the Buddha was staying near \textsanskrit{Vesālī}, at the Great Wood, in the hall with the peaked roof. Then the Buddha robed up in the morning and, taking his bowl and robe, entered \textsanskrit{Vesālī} for alms. Then, after the meal, on his return from almsround, he addressed Venerable Ānanda: “Ānanda, get your sitting cloth. Let’s go to the \textsanskrit{Cāpāla} shrine for the day’s meditation.” 

“Yes,\marginnote{2.1} sir,” replied Ānanda. Taking his sitting cloth he followed behind the Buddha. Then the Buddha went up to the \textsanskrit{Cāpāla} shrine, where he sat on the seat spread out. When he was seated he said to Venerable Ānanda: 

“Ānanda,\marginnote{3.1} \textsanskrit{Vesālī} is lovely. And the Udena, Gotamaka, Sattamba, Bahuputta, \textsanskrit{Sārandada}, and \textsanskrit{Cāpāla} Tree-shrines are all lovely. Whoever has developed and cultivated the four bases of psychic power—made them a vehicle and a basis, kept them up, consolidated them, and properly implemented them—may, if they wish, live on for the eon or what’s left of the eon. The Realized One has developed and cultivated the four bases of psychic power, made them a vehicle and a basis, kept them up, consolidated them, and properly implemented them. If he wished, the Realized One could live on for the eon or what’s left of the eon.” 

But\marginnote{4.1} Ānanda didn’t get it, even though the Buddha dropped such an obvious hint, such a clear sign. He didn’t beg the Buddha, “Sir, may the Blessed One please remain for the eon! May the Holy One please remain for the eon! That would be for the welfare and happiness of the people, out of compassion for the world, for the benefit, welfare, and happiness of gods and humans.” For his mind was as if possessed by \textsanskrit{Māra}. For a second time … and for a third time, the Buddha said to Ānanda: 

“Ānanda,\marginnote{5.1} \textsanskrit{Vesālī} is lovely. And the Udena, Gotamaka, Sattamba, Bahuputta, \textsanskrit{Sārandada}, and \textsanskrit{Cāpāla} Tree-shrines are all lovely. Whoever has developed and cultivated the four bases of psychic power—made them a vehicle and a basis, kept them up, consolidated them, and properly implemented them—may, if they wish, live on for the eon or what’s left of the eon. The Realized One has developed and cultivated the four bases of psychic power, made them a vehicle and a basis, kept them up, consolidated them, and properly implemented them. If he wished, the Realized One could live on for the eon or what’s left of the eon.” 

But\marginnote{6.1} Ānanda didn’t get it, even though the Buddha dropped such an obvious hint, such a clear sign. He didn’t beg the Buddha, “Sir, may the Blessed One please remain for the eon! May the Holy One please remain for the eon! That would be for the welfare and happiness of the people, out of compassion for the world, for the benefit, welfare, and happiness of gods and humans.” For his mind was as if possessed by \textsanskrit{Māra}. 

Then\marginnote{7.1} the Buddha said to Venerable Ānanda, “Go now, Ānanda, at your convenience.” “Yes, sir,” replied Ānanda. He rose from his seat, bowed, and respectfully circled the Buddha, keeping him on his right, before sitting at the root of a tree close by. 

And\marginnote{8.1} then, not long after Ānanda had left, \textsanskrit{Māra} the Wicked went up to the Buddha, stood to one side, and said to him: 

“May\marginnote{9.1} the Blessed One now become fully extinguished! May the Holy One now become fully extinguished! Now is the time for the Buddha to become fully extinguished. Sir, you once made this statement: ‘Wicked One, I will not become fully extinguished until I have monk disciples who are competent, educated, assured, learned, have memorized the teachings, and practice in line with the teachings. Not until they practice properly, living in line with the teaching. Not until they’ve learned their own tradition, and explain, teach, assert, establish, disclose, analyze, and make it clear. Not until they can legitimately and completely refute the doctrines of others that come up, and teach with a demonstrable basis.’ Today you do have such monk disciples. May the Blessed One now become fully extinguished! May the Holy One now become fully extinguished! Now is the time for the Buddha to become fully extinguished. 

Sir,\marginnote{10.1} you once made this statement: ‘Wicked One, I will not become fully extinguished until I have nun disciples who are competent, educated, assured, learned …’ … Today you do have such nun disciples. 

‘Wicked\marginnote{11.1} One, I will not become fully extinguished until I have layman disciples who are competent, educated, assured, learned …’ Today you do have such layman disciples. 

‘Wicked\marginnote{12.1} One, I will not become fully extinguished until I have laywoman disciples who are competent, educated, assured, learned …’ Today you do have such laywoman disciples. May the Blessed One now become fully extinguished! May the Holy One now become fully extinguished! Now is the time for the Buddha to become fully extinguished. 

Sir,\marginnote{13.1} you once made this statement: ‘Not until my spiritual path is successful and prosperous, extensive, popular, widespread, and well proclaimed wherever there are gods and humans.’ Today your spiritual path is successful and prosperous, extensive, popular, widespread, and well proclaimed wherever there are gods and humans. May the Blessed One now become fully extinguished! May the Holy One now become fully extinguished! Now is the time for the Buddha to become fully extinguished. 

When\marginnote{14.1} this was said, the Buddha said to \textsanskrit{Māra}, “Relax, Wicked One. The final extinguishment of the Realized One will be soon. Three months from now the Realized One will finally be extinguished.” 

So\marginnote{15.1} at the \textsanskrit{Cāpāla} Tree-shrine the Buddha, mindful and aware, surrendered the life force. When he did so there was a great earthquake, awe-inspiring and hair-raising, and thunder cracked the sky. 

Then,\marginnote{16.1} understanding this matter, on that occasion the Buddha expressed this heartfelt sentiment: 

\begin{verse}%
“Weighing\marginnote{17.1} up the incomparable against an extension of life, \\
the sage surrendered the life force. \\
Happy inside, serene, \\
he burst out of this self-made chain like a suit of armor.” 

%
\end{verse}

%
\section*{{\suttatitleacronym Ud 6.2}{\suttatitletranslation Seven Matted-Hair Ascetics }{\suttatitleroot Sattajaṭilasutta}}
\addcontentsline{toc}{section}{\tocacronym{Ud 6.2} \toctranslation{Seven Matted-Hair Ascetics } \tocroot{Sattajaṭilasutta}}
\markboth{Seven Matted-Hair Ascetics }{Sattajaṭilasutta}
\extramarks{Ud 6.2}{Ud 6.2}

\scevam{So\marginnote{1.1} I have heard. }At one time the Buddha was staying near \textsanskrit{Sāvatthī} in the Eastern Monastery, the stilt longhouse of \textsanskrit{Migāra}’s mother. Then in the late afternoon, the Buddha came out of retreat and sat outside the gate. Then King Pasenadi of Kosala went up to the Buddha, bowed, and sat down to one side. 

Now\marginnote{2.1} at that time seven matted-hair ascetics, seven Jain ascetics, seven naked ascetics, seven one-cloth ascetics, and seven wanderers passed by not far from the Buddha. Their armpits and bodies were hairy, and their nails were long; and they carried their stuff with shoulder-poles. 

King\marginnote{3.1} Pasenadi saw them passing by. He got up from his seat, arranged his robe over one shoulder, knelt with his right knee on the ground, raised his joined palms toward those various ascetics, and pronounced his name three times: “Sirs, I am Pasenadi, king of Kosala! I am Pasenadi, king of Kosala! I am Pasenadi, king of Kosala!” 

Then,\marginnote{4.1} soon after those ascetics had left, King Pasenadi went up to the Buddha, bowed, sat down to one side, and said to him, “Sir, are they among those in the world who are perfected ones or who are on the path to perfection?” 

“Great\marginnote{5.1} king, as a layman enjoying sensual pleasures, living at home with your children, using sandalwood imported from \textsanskrit{Kāsi}, wearing garlands, perfumes, and makeup, and accepting gold and money, it’s hard for you to know who is perfected or on the path to perfection. 

You\marginnote{6.1} can get to know a person’s ethics by living with them. But only after a long time, not casually; only when paying attention, not when inattentive; and only by the wise, not the witless. You can get to know a person’s purity by dealing with them. … You can get to know a person’s resilience in times of trouble. … You can get to know a person’s wisdom by discussion. But only after a long time, not casually; only when paying attention, not when inattentive; and only by the wise, not the witless.” 

“It’s\marginnote{7.1} incredible, sir, it’s amazing, how well said this was by the Buddha. … 

Sir,\marginnote{8.1} these are my spies, my undercover agents returning after spying on the country. First they go undercover, then I have them report to me. And now—when they have washed off the dust and dirt, and are nicely bathed and anointed, with hair and beard dressed, and dressed in white—they will amuse themselves, supplied and provided with the five kinds of sensual stimulation.” 

Then,\marginnote{9.1} understanding this matter, on that occasion the Buddha expressed this heartfelt sentiment: 

\begin{verse}%
“Don’t\marginnote{10.1} strive in every situation, \\
don’t become another’s man. \\
Don’t live depending on another, \\
and don’t use the teaching to make money.”\footnote{Commentary: \textit{\textsanskrit{dhanādiatthāya} dhammaṃ na katheyya}, “Don’t teach the Dhamma for the sake of money, etc.” } 

%
\end{verse}

%
\section*{{\suttatitleacronym Ud 6.3}{\suttatitletranslation The Buddha’s Reviewing }{\suttatitleroot Paccavekkhaṇasutta}}
\addcontentsline{toc}{section}{\tocacronym{Ud 6.3} \toctranslation{The Buddha’s Reviewing } \tocroot{Paccavekkhaṇasutta}}
\markboth{The Buddha’s Reviewing }{Paccavekkhaṇasutta}
\extramarks{Ud 6.3}{Ud 6.3}

\scevam{So\marginnote{1.1} I have heard. }At one time the Buddha was staying near \textsanskrit{Sāvatthī} in Jeta’s Grove, \textsanskrit{Anāthapiṇḍika}’s monastery. Now at that time the Buddha was sitting reviewing his own giving up of many bad, unskillful qualities, and the many skillful qualities he had fully developed. 

Then,\marginnote{2.1} knowing the many bad, unskillful qualities that he had given up and the many skillful qualities he had fully developed, on that occasion the Buddha expressed this heartfelt sentiment: 

\begin{verse}%
“What\marginnote{3.1} was before then was not; \\
what before was not then was. \\
It never was, nor will it be, \\
nor is it found today.” 

%
\end{verse}

%
\section*{{\suttatitleacronym Ud 6.4}{\suttatitletranslation Followers of Other Paths (1st) }{\suttatitleroot Paṭhamanānātitthiyasutta}}
\addcontentsline{toc}{section}{\tocacronym{Ud 6.4} \toctranslation{Followers of Other Paths (1st) } \tocroot{Paṭhamanānātitthiyasutta}}
\markboth{Followers of Other Paths (1st) }{Paṭhamanānātitthiyasutta}
\extramarks{Ud 6.4}{Ud 6.4}

\scevam{So\marginnote{1.1} I have heard. }At one time the Buddha was staying near \textsanskrit{Sāvatthī} in Jeta’s Grove, \textsanskrit{Anāthapiṇḍika}’s monastery. Now at that time several ascetics, brahmins, and wanderers who followed various other paths were residing in \textsanskrit{Sāvatthī}, holding different views and opinions, relying on different views. 

There\marginnote{2.1} were some ascetics and brahmins who had this doctrine and view: “The cosmos is eternal. This is the only truth, other ideas are silly.” 

Others\marginnote{3.1} held views such as the following, each regarding their own view as true and others as silly. “The cosmos is not eternal.” 

“The\marginnote{4.1} world is finite.” 

“The\marginnote{5.1} world is infinite.” 

“The\marginnote{6.1} soul and the body are the same thing.” 

“The\marginnote{7.1} soul and the body are different things.” 

“A\marginnote{8.1} Realized One exists after death.” 

“A\marginnote{9.1} Realized One doesn’t exist after death.” 

“A\marginnote{10.1} Realized One both exists and doesn’t exist after death.” 

“A\marginnote{11.1} Realized One neither exists nor doesn’t exist after death.” 

They\marginnote{12.1} were arguing, quarreling, and disputing, continually wounding each other with barbed words: “Such is Truth, such is not Truth! Such is not Truth, such is Truth!” 

Then\marginnote{13.1} several mendicants robed up in the morning and, taking their bowls and robes, entered \textsanskrit{Sāvatthī} for alms. Then, after the meal, when they returned from almsround, they went up to the Buddha, bowed, sat down to one side, and told him what was happening. The Buddha said: 

“The\marginnote{16.1} wanderers who follow other paths are blind and sightless. They don't understand what is beneficial or what is not beneficial, nor what is the truth and what is not the truth. That’s why they are arguing, quarreling, and disputing, continually wounding each other with barbed words. 

Once\marginnote{17.1} upon a time, mendicants, right here in \textsanskrit{Sāvatthī} there was a certain king. Then the king addressed a man, ‘Please, mister, gather all those blind from birth throughout \textsanskrit{Sāvatthī} and bring them together in one place.’ ‘Yes, Your Majesty,’ that man replied. He did as the king asked, then said to him, ‘Your Majesty, the blind people throughout \textsanskrit{Sāvatthī} have been gathered.’ ‘Well then, my man, show them an elephant.’ ‘Yes, Your Majesty,’ that man replied. He did as the king asked. 

To\marginnote{18.1} some of the blind people he showed the elephant’s head, saying, ‘Here is the elephant.’ To some of them he showed the elephant’s ear, saying, ‘Here is the elephant.’ To some of them he showed the elephant’s tusk, saying, ‘Here is the elephant.’ To some of them he showed the elephant’s trunk, saying, ‘Here is the elephant.’ To some of them he showed the elephant’s flank, saying, ‘Here is the elephant.’ To some of them he showed the elephant’s leg, saying, ‘Here is the elephant.’ To some of them he showed the elephant’s thigh, saying, ‘Here is the elephant.’ To some of them he showed the elephant’s tail, saying, ‘Here is the elephant.’ To some of them he showed the tip of the elephant’s tail, saying, ‘Here is the elephant.’ 

Then\marginnote{19.1} he approached the king and said, ‘Your Majesty, the blind people have been shown the elephant. Please go at your convenience.’ 

Then\marginnote{20.1} the king went up to the blind people and said, ‘Have you seen the elephant?’ ‘Yes, Your Majesty, we have been shown the elephant.’ ‘Then tell us, what kind of thing is an elephant?’ 

The\marginnote{21.1} blind people who had been shown the elephant’s head said, ‘Your Majesty, an elephant is like a pot.’ 

Those\marginnote{22.1} who had been shown the ear said, ‘An elephant is like a winnowing fan.’ 

Those\marginnote{23.1} who had been shown the tusk said, ‘An elephant is like a ploughshare.’ 

Those\marginnote{24.1} who had been shown the trunk said, ‘An elephant is like a plough-pole.’ 

Those\marginnote{25.1} who had been shown the flank said, ‘An elephant is like a storehouse.’ 

Those\marginnote{26.1} who had been shown the leg said, ‘An elephant is like a pillar.’ 

Those\marginnote{27.1} who had been shown the thigh said, ‘An elephant is like a mortar.’ 

Those\marginnote{28.1} who had been shown the tail said, ‘An elephant is like a pestle.’ 

Those\marginnote{29.1} who had been shown the tip of the tail said, ‘An elephant is like a broom.’ 

Saying,\marginnote{30.1} ‘Such is an elephant, not such! Such is not an elephant, such is!’ they punched each other with their fists. At that, the king was pleased. 

In\marginnote{31.1} the same way, mendicants, the wanderers who follow other paths are blind and sightless. They don't understand what is beneficial or what is not beneficial, nor what is the truth and what is not the truth. That’s why they are arguing, quarreling, and disputing, continually wounding each other with barbed words. ‘Such is Truth, such is not! Such is not Truth, such is!’” 

Then,\marginnote{32.1} understanding this matter, on that occasion the Buddha expressed this heartfelt sentiment: 

\begin{verse}%
“Some\marginnote{33.1} ascetics and brahmins, it seems, \\
cling to these things. \\
Arguing, they quarrel, \\
the folk who see just one part.” 

%
\end{verse}

%
\section*{{\suttatitleacronym Ud 6.5}{\suttatitletranslation Followers of Other Paths (2nd) }{\suttatitleroot Dutiyanānātitthiyasutta}}
\addcontentsline{toc}{section}{\tocacronym{Ud 6.5} \toctranslation{Followers of Other Paths (2nd) } \tocroot{Dutiyanānātitthiyasutta}}
\markboth{Followers of Other Paths (2nd) }{Dutiyanānātitthiyasutta}
\extramarks{Ud 6.5}{Ud 6.5}

\scevam{So\marginnote{1.1} I have heard. }At one time the Buddha was staying near \textsanskrit{Sāvatthī} in Jeta’s Grove, \textsanskrit{Anāthapiṇḍika}’s monastery. Now at that time several ascetics, brahmins, and wanderers who followed various other paths were residing in \textsanskrit{Sāvatthī}, holding different views and opinions, relying on different views. 

There\marginnote{2.1} were some ascetics and brahmins who had this doctrine and view: “The self and the cosmos are eternal. This is the only truth, other ideas are silly.” 

Others\marginnote{3.1} held views such as the following, each regarding their own view as true and others as silly. “The self and the cosmos are not eternal.” 

“The\marginnote{4.1} self and the cosmos are both eternal and not eternal.” 

“The\marginnote{5.1} self and the cosmos are neither eternal nor not eternal.” 

“The\marginnote{6.1} self and the cosmos are made by oneself.” 

“The\marginnote{7.1} self and the cosmos are made by another.” 

“The\marginnote{8.1} self and the cosmos are made by both oneself and another.” 

“The\marginnote{9.1} self and the cosmos have arisen by chance, not made by oneself or another.” 

“Pleasure\marginnote{10.1} and pain are eternal, and the self and the cosmos.”\footnote{\textit{\textsanskrit{Attā} ca loko ca} here and following is missing from the parallel passage in DN 29:36.9. I assume it has been inserted in error and translate accordingly. } 

“Pleasure\marginnote{11.1} and pain are not eternal, and the self and the cosmos.” 

“Pleasure\marginnote{12.1} and pain are both eternal and not eternal, and the self and the cosmos.” 

“Pleasure\marginnote{13.1} and pain are neither eternal nor not eternal, and the self and the cosmos.” 

“Pleasure\marginnote{14.1} and pain are made by oneself, and the self and the cosmos.” 

“Pleasure\marginnote{15.1} and pain are made by another, and the self and the cosmos.” 

“Pleasure\marginnote{16.1} and pain are made by both oneself and another, and the self and the cosmos.” 

“Pleasure\marginnote{17.1} and pain have arisen by chance, not made by oneself or another, and the self and the cosmos.” 

They\marginnote{18.1} were arguing, quarreling, and disputing, continually wounding each other with barbed words: “Such is Truth, such is not Truth! Such is not Truth, such is Truth!” 

Then\marginnote{19.1} several mendicants robed up in the morning and, taking their bowls and robes, entered \textsanskrit{Sāvatthī} for alms. Then, after the meal, when they returned from almsround, they went up to the Buddha, bowed, sat down to one side, and told him what was happening. 

Then,\marginnote{23.1} understanding this matter, on that occasion the Buddha expressed this heartfelt sentiment: 

\begin{verse}%
“Some\marginnote{24.1} ascetics and brahmins, it seems, \\
cling to these things. \\
They flounder in mid-stream, \\
without reaching a firm footing.” 

%
\end{verse}

%
\section*{{\suttatitleacronym Ud 6.6}{\suttatitletranslation Followers of Other Paths (3rd) }{\suttatitleroot Tatiyanānātitthiyasutta}}
\addcontentsline{toc}{section}{\tocacronym{Ud 6.6} \toctranslation{Followers of Other Paths (3rd) } \tocroot{Tatiyanānātitthiyasutta}}
\markboth{Followers of Other Paths (3rd) }{Tatiyanānātitthiyasutta}
\extramarks{Ud 6.6}{Ud 6.6}

\scevam{So\marginnote{1.1} I have heard. }At one time the Buddha was staying near \textsanskrit{Sāvatthī} in Jeta’s Grove, \textsanskrit{Anāthapiṇḍika}’s monastery. Now at that time several ascetics, brahmins, and wanderers who followed various other paths were residing in \textsanskrit{Sāvatthī}, holding different views and opinions, relying on different views. 

There\marginnote{2.1} were some ascetics and brahmins who had this doctrine and view: “The self and the cosmos are eternal. This is the only truth, other ideas are silly.” 

Others\marginnote{3.1} held views such as the following, each regarding their own view as true and others as silly. “The self and the cosmos are not eternal.” 

“The\marginnote{4.1} self and the cosmos are both eternal and not eternal.” 

“The\marginnote{5.1} self and the cosmos are neither eternal nor not eternal.” 

“The\marginnote{6.1} self and the cosmos are made by oneself.” 

“The\marginnote{7.1} self and the cosmos are made by another.” 

“The\marginnote{8.1} self and the cosmos are made by both oneself and another.” 

“The\marginnote{9.1} self and the cosmos have arisen by chance, not made by oneself or another.” 

“Pleasure\marginnote{10.1} and pain are eternal, and the self and the cosmos.” 

“Pleasure\marginnote{11.1} and pain are not eternal, and the self and the cosmos.” 

“Pleasure\marginnote{12.1} and pain are both eternal and not eternal, and the self and the cosmos.” 

“Pleasure\marginnote{13.1} and pain are neither eternal nor not eternal, and the self and the cosmos.” 

“Pleasure\marginnote{14.1} and pain are made by oneself, and the self and the cosmos.” 

“Pleasure\marginnote{15.1} and pain are made by another, and the self and the cosmos.” 

“Pleasure\marginnote{16.1} and pain are made by both oneself and another, and the self and the cosmos.” 

“Pleasure\marginnote{17.1} and pain have arisen by chance, not made by oneself or another, and the self and the cosmos.” 

They\marginnote{18.1} were arguing, quarreling, and disputing, continually wounding each other with barbed words: ‘Such is Truth, such is not! Such is not Truth, such is Truth!” 

Then\marginnote{19.1} several mendicants robed up in the morning and, taking their bowls and robes, entered \textsanskrit{Sāvatthī} for alms. Then, after the meal, when they returned from almsround, they went up to the Buddha, bowed, sat down to one side, and told him what was happening. 

Then,\marginnote{23.1} understanding this matter, on that occasion the Buddha expressed this heartfelt sentiment: 

\begin{verse}%
“Folk\marginnote{24.1} are fixated on the I-maker, \\
which is tied up with the other-maker. \\
There are some who do not realize this, \\
they do not see the dart. 

But\marginnote{25.1} when they see this dart, \\
they do not think, ‘I make it’, \\
nor ‘another makes it’. 

These\marginnote{26.1} folk are caught up in conceit, \\
tied by conceit, schackled by conceit. \\
Vehemently defending their views, \\
they don’t escape transmigration.” 

%
\end{verse}

%
\section*{{\suttatitleacronym Ud 6.7}{\suttatitletranslation With Subhūti }{\suttatitleroot Subhūtisutta}}
\addcontentsline{toc}{section}{\tocacronym{Ud 6.7} \toctranslation{With Subhūti } \tocroot{Subhūtisutta}}
\markboth{With Subhūti }{Subhūtisutta}
\extramarks{Ud 6.7}{Ud 6.7}

\scevam{So\marginnote{1.1} I have heard. }At one time the Buddha was staying near \textsanskrit{Sāvatthī} in Jeta’s Grove, \textsanskrit{Anāthapiṇḍika}’s monastery. Now at that time Venerable \textsanskrit{Subhūti} was sitting not far from the Buddha, cross-legged, with his body straight, having attained the immersion free of placing the mind. 

The\marginnote{2.1} Buddha saw him meditating there. 

Then,\marginnote{3.1} understanding this matter, on that occasion the Buddha expressed this heartfelt sentiment: 

\begin{verse}%
“In\marginnote{4.1} whom mental vibrations are cleared away, \\
internally clipped off entirely, \\
perceiving the formless, beyond attachments, \\
having overcome the four yokes, they are not born again.”\footnote{Reading \textit{\textsanskrit{jātim} eti} following \textsanskrit{Śarīrārthagāthā} 32, which  here is very close to the Pali. The commentary also accepts this reading: \textit{‘Na \textsanskrit{jāti} \textsanskrit{metī}’tipi \textsanskrit{paṭhanti}, so evattho}. } 

%
\end{verse}

%
\section*{{\suttatitleacronym Ud 6.8}{\suttatitletranslation The Courtesan }{\suttatitleroot Gaṇikāsutta}}
\addcontentsline{toc}{section}{\tocacronym{Ud 6.8} \toctranslation{The Courtesan } \tocroot{Gaṇikāsutta}}
\markboth{The Courtesan }{Gaṇikāsutta}
\extramarks{Ud 6.8}{Ud 6.8}

\scevam{So\marginnote{1.1} I have heard. }At one time the Buddha was staying near \textsanskrit{Rājagaha}, in the Bamboo Grove, the squirrels’ feeding ground. Now at that time two gangs were both hopelessly in love with a certain courtesan. Quarreling, arguing, and disputing, they attacked each other with fists, stones, rods, and swords, resulting in death and deadly pain. 

Then\marginnote{2.1} several mendicants robed up in the morning and, taking their bowls and robes, entered \textsanskrit{Rājagaha} for alms. Then, after the meal, when they returned from almsround, they went up to the Buddha, bowed, sat down to one side, and told him what was happening. 

Then,\marginnote{4.1} understanding this matter, on that occasion the Buddha expressed this heartfelt sentiment: 

“What\marginnote{5.1} has been attained and what is to be attained are both strewn over with dust for that one training while still sick. Those for whom the training is the essence, or precepts and observances, celibacy, and service as the essence: this is one extreme. Those who say, ‘There’s nothing wrong with sensual pleasures’: this is the second extreme. Thus these two extremes swell the charnel grounds, while the charnel grounds swell wrong view. Not realizing these two extremes, some get stuck and some overreach. Those who realize these things, who were not found there, who did not conceive by that, there is no cycle of rebirths to be found.” 

%
\section*{{\suttatitleacronym Ud 6.9}{\suttatitletranslation Hastening By }{\suttatitleroot Upātidhāvantisutta}}
\addcontentsline{toc}{section}{\tocacronym{Ud 6.9} \toctranslation{Hastening By } \tocroot{Upātidhāvantisutta}}
\markboth{Hastening By }{Upātidhāvantisutta}
\extramarks{Ud 6.9}{Ud 6.9}

\scevam{So\marginnote{1.1} I have heard. }At one time the Buddha was staying near \textsanskrit{Sāvatthī} in Jeta’s Grove, \textsanskrit{Anāthapiṇḍika}’s monastery. Now at that time the Buddha was meditating in the open during the dark of night, while oil lamps were burning. 

And\marginnote{2.1} many moths were falling down and crashing down into the lamps, coming to grief and ruin. The Buddha saw the moths coming to grief. 

Then,\marginnote{3.1} understanding this matter, on that occasion the Buddha expressed this heartfelt sentiment: 

\begin{verse}%
“Hastening\marginnote{4.1} by, they miss the essence, \\
sprouting ever more new bonds. \\
Like moths falling in the flame, \\
some have become fixed in what is seen or heard.” 

%
\end{verse}

%
\section*{{\suttatitleacronym Ud 6.10}{\suttatitletranslation Arising }{\suttatitleroot Uppajjantisutta}}
\addcontentsline{toc}{section}{\tocacronym{Ud 6.10} \toctranslation{Arising } \tocroot{Uppajjantisutta}}
\markboth{Arising }{Uppajjantisutta}
\extramarks{Ud 6.10}{Ud 6.10}

\scevam{So\marginnote{1.1} I have heard. }At one time the Buddha was staying near \textsanskrit{Sāvatthī} in Jeta’s Grove, \textsanskrit{Anāthapiṇḍika}’s monastery. Then Venerable Ānanda went up to the Buddha, bowed, sat down to one side, and said to him: 

“Sir,\marginnote{2.1} so long as the Realized Ones, the perfected ones, the fully awakened Buddhas do not arise in the world, the wanderers who follow other paths are honored, respected, revered, venerated, and esteemed. And they receive robes, almsfood, lodgings, and medicines and supplies for the sick. But when the Realized Ones do arise in the world, the wanderers who follow other paths are no longer honored, respected, revered, venerated, and esteemed. And they do not receive robes, almsfood, lodgings, and medicines and supplies for the sick. Now only the Buddha is honored, respected, revered, venerated, and esteemed. And he receives robes, almsfood, lodgings, and medicines and supplies for the sick.” 

“That’s\marginnote{3.1} so true, Ānanda. So long as the Realized Ones do not arise in the world, the wanderers who follow other paths are honored, respected, revered, venerated, and esteemed. And they receive robes, almsfood, lodgings, and medicines and supplies for the sick. But when the Realized Ones do arise in the world, the wanderers who follow other paths are no longer honored in this way. Now only the Realized One is honored, respected, revered, venerated, and esteemed. And he receives robes, almsfood, lodgings, and medicines and supplies for the sick. And so does the mendicant \textsanskrit{Saṅgha}.” 

Then,\marginnote{4.1} understanding this matter, on that occasion the Buddha expressed this heartfelt sentiment: 

\begin{verse}%
“The\marginnote{5.1} glow-worm shines so long \\
as the beacon of the sun does not rise. \\
But when the sun has come up, \\
that light is erased and shines no more. 

So\marginnote{6.1} too the reasoners shine bright \\
so long as the Buddhas don’t arise in the world. \\
The reasoners are not purified, nor are their disciples. \\
Having bad views, they are not freed from suffering.” 

%
\end{verse}

%
\addtocontents{toc}{\let\protect\contentsline\protect\nopagecontentsline}
\chapter*{The Lesser Chapter }
\addcontentsline{toc}{chapter}{\tocchapterline{The Lesser Chapter }}
\addtocontents{toc}{\let\protect\contentsline\protect\oldcontentsline}

%
\section*{{\suttatitleacronym Ud 7.1}{\suttatitletranslation Bhaddiya the Dwarf (1st) }{\suttatitleroot Paṭhamalakuṇḍakabhaddiyasutta}}
\addcontentsline{toc}{section}{\tocacronym{Ud 7.1} \toctranslation{Bhaddiya the Dwarf (1st) } \tocroot{Paṭhamalakuṇḍakabhaddiyasutta}}
\markboth{Bhaddiya the Dwarf (1st) }{Paṭhamalakuṇḍakabhaddiyasutta}
\extramarks{Ud 7.1}{Ud 7.1}

\scevam{So\marginnote{1.1} I have heard. }At one time the Buddha was staying near \textsanskrit{Sāvatthī} in Jeta’s Grove, \textsanskrit{Anāthapiṇḍika}’s monastery. Now at that time Venerable \textsanskrit{Sāriputta} was educating, encouraging, firing up, and inspiring Venerable Bhaddiya the Dwarf with a Dhamma talk. 

Then\marginnote{2.1} after being taught like this Bhaddiya’s mind was freed from defilements by not grasping. 

The\marginnote{3.1} Buddha saw what had happened. 

Then,\marginnote{4.1} understanding this matter, on that occasion the Buddha expressed this heartfelt sentiment: 

\begin{verse}%
“Above,\marginnote{5.1} below, everywhere free, \\
not contemplating ‘I am this’. \\
Freed like this, he has crossed the flood \\
not crossed before, so as to not be reborn.” 

%
\end{verse}

%
\section*{{\suttatitleacronym Ud 7.2}{\suttatitletranslation Bhaddiya the Dwarf (2nd) }{\suttatitleroot Dutiyalakuṇḍakabhaddiyasutta}}
\addcontentsline{toc}{section}{\tocacronym{Ud 7.2} \toctranslation{Bhaddiya the Dwarf (2nd) } \tocroot{Dutiyalakuṇḍakabhaddiyasutta}}
\markboth{Bhaddiya the Dwarf (2nd) }{Dutiyalakuṇḍakabhaddiyasutta}
\extramarks{Ud 7.2}{Ud 7.2}

\scevam{So\marginnote{1.1} I have heard. }At one time the Buddha was staying near \textsanskrit{Sāvatthī} in Jeta’s Grove, \textsanskrit{Anāthapiṇḍika}’s monastery. Now at that time Venerable \textsanskrit{Sāriputta} was educating, encouraging, firing up, and inspiring Venerable Bhaddiya the Dwarf with even more Dhamma talk, thinking that he was still a trainee. 

The\marginnote{2.1} Buddha saw what was happening. 

Then,\marginnote{3.1} understanding this matter, on that occasion the Buddha expressed this heartfelt sentiment: 

\begin{verse}%
“They’ve\marginnote{4.1} cut the cycle, gone to the wishless; \\
the streams are dried, they flow no more. \\
Cut, the cycle no longer turns. \\
Just this is the end of suffering.” 

%
\end{verse}

%
\section*{{\suttatitleacronym Ud 7.3}{\suttatitletranslation Clinging (1st) }{\suttatitleroot Paṭhamasattasutta}}
\addcontentsline{toc}{section}{\tocacronym{Ud 7.3} \toctranslation{Clinging (1st) } \tocroot{Paṭhamasattasutta}}
\markboth{Clinging (1st) }{Paṭhamasattasutta}
\extramarks{Ud 7.3}{Ud 7.3}

\scevam{So\marginnote{1.1} I have heard. }At one time the Buddha was staying near \textsanskrit{Sāvatthī} in Jeta’s Grove, \textsanskrit{Anāthapiṇḍika}’s monastery. Now at that time most of the people in \textsanskrit{Sāvatthī} overly clung to sensual pleasures. Lustful, greedy, tied, infatuated, they lived completely addicted to sensual pleasures. 

Then\marginnote{2.1} several mendicants robed up in the morning and, taking their bowls and robes, entered \textsanskrit{Sāvatthī} for alms. Then, after the meal, when they returned from almsround, they went up to the Buddha, bowed, sat down to one side, and told him what was happening. 

Then,\marginnote{3.1} understanding this matter, on that occasion the Buddha expressed this heartfelt sentiment: 

\begin{verse}%
“Clinging\marginnote{4.1} to sensual pleasures, to the chains of the senses, \\
blind to the faults of the fetters, \\
clinging to the chain of the fetters, \\
there’s no way they can cross the flood so vast.” 

%
\end{verse}

%
\section*{{\suttatitleacronym Ud 7.4}{\suttatitletranslation Clinging (2nd) }{\suttatitleroot Dutiyasattasutta}}
\addcontentsline{toc}{section}{\tocacronym{Ud 7.4} \toctranslation{Clinging (2nd) } \tocroot{Dutiyasattasutta}}
\markboth{Clinging (2nd) }{Dutiyasattasutta}
\extramarks{Ud 7.4}{Ud 7.4}

\scevam{So\marginnote{1.1} I have heard. }At one time the Buddha was staying near \textsanskrit{Sāvatthī} in Jeta’s Grove, \textsanskrit{Anāthapiṇḍika}’s monastery. Now at that time most of the people in \textsanskrit{Sāvatthī} clung to sensual pleasures. Lustful, greedy, tied, infatuated, attached, and blinded, they lived completely addicted to sensual pleasures. 

Then\marginnote{2.1} the Buddha robed up in the morning and, taking his bowl and robe, entered \textsanskrit{Sāvatthī} for alms. He saw how attached the people were. 

Then,\marginnote{3.1} understanding this matter, on that occasion the Buddha expressed this heartfelt sentiment: 

\begin{verse}%
“Blinded\marginnote{4.1} by sensual pleasures, wrapped in a net, \\
they are smothered over by craving; \\
bound by the kinsman of the negligent, \\
like a fish caught in a funnel-net trap. \\
They chase old age and death, \\
like a suckling calf its mother.” 

%
\end{verse}

%
\section*{{\suttatitleacronym Ud 7.5}{\suttatitletranslation Another Discourse with Bhaddiya the Dwarf }{\suttatitleroot Aparalakuṇḍakabhaddiyasutta}}
\addcontentsline{toc}{section}{\tocacronym{Ud 7.5} \toctranslation{Another Discourse with Bhaddiya the Dwarf } \tocroot{Aparalakuṇḍakabhaddiyasutta}}
\markboth{Another Discourse with Bhaddiya the Dwarf }{Aparalakuṇḍakabhaddiyasutta}
\extramarks{Ud 7.5}{Ud 7.5}

\scevam{So\marginnote{1.1} I have heard. }At one time the Buddha was staying near \textsanskrit{Sāvatthī} in Jeta’s Grove, \textsanskrit{Anāthapiṇḍika}’s monastery. Then Venerable Bhaddiya the Dwarf, closely following several mendicants, approached the Buddha. 

The\marginnote{2.1} Buddha saw Venerable Bhaddiya coming off in the distance—ugly, unsightly, deformed, and despised by most of the mendicants. The Buddha addressed the mendicants: 

“Mendicants,\marginnote{3.1} do you see this monk coming—ugly, unsightly, deformed, and despised by most of the mendicants?” “Yes, sir.” 

“That\marginnote{4.1} mendicant is very mighty and powerful. It’s not easy to find an attainment that he has not already attained. And he has realized the supreme end of the spiritual path in this very life. He lives having achieved with his own insight the goal for which gentlemen rightly go forth from the lay life to homelessness.” 

Then,\marginnote{5.1} understanding this matter, on that occasion the Buddha expressed this heartfelt sentiment: 

\begin{verse}%
“With\marginnote{6.1} flawless wheel and white canopy, \\
the one-spoke chariot rolls on. \\
See it come, untroubled, \\
with stream cut, unbound.” 

%
\end{verse}

%
\section*{{\suttatitleacronym Ud 7.6}{\suttatitletranslation The Ending of Craving }{\suttatitleroot Taṇhāsaṅkhayasutta}}
\addcontentsline{toc}{section}{\tocacronym{Ud 7.6} \toctranslation{The Ending of Craving } \tocroot{Taṇhāsaṅkhayasutta}}
\markboth{The Ending of Craving }{Taṇhāsaṅkhayasutta}
\extramarks{Ud 7.6}{Ud 7.6}

\scevam{So\marginnote{1.1} I have heard. }At one time the Buddha was staying near \textsanskrit{Sāvatthī} in Jeta’s Grove, \textsanskrit{Anāthapiṇḍika}’s monastery. Now at that time Venerable \textsanskrit{Koṇḍañña} Who Understood was sitting not far from the Buddha, cross-legged, with his body straight, reviewing the freedom through the ending of craving. 

The\marginnote{2.1} Buddha saw him meditating there. 

Then,\marginnote{3.1} understanding this matter, on that occasion the Buddha expressed this heartfelt sentiment: 

\begin{verse}%
“There\marginnote{4.1} is no root or ground or leaves for them, \\
so where would creepers sprout from? \\
That wise one is released from bonds: \\
who is worthy to criticize them? \\
Even the gods praise them, \\
and by \textsanskrit{Brahmā}, too, they’re praised.” 

%
\end{verse}

%
\section*{{\suttatitleacronym Ud 7.7}{\suttatitletranslation The Ending of Proliferation }{\suttatitleroot Papañcakhayasutta}}
\addcontentsline{toc}{section}{\tocacronym{Ud 7.7} \toctranslation{The Ending of Proliferation } \tocroot{Papañcakhayasutta}}
\markboth{The Ending of Proliferation }{Papañcakhayasutta}
\extramarks{Ud 7.7}{Ud 7.7}

\scevam{So\marginnote{1.1} I have heard. }At one time the Buddha was staying near \textsanskrit{Sāvatthī} in Jeta’s Grove, \textsanskrit{Anāthapiṇḍika}’s monastery. Now at that time the Buddha was sitting reviewing his own giving up of the concepts of identity that emerge from the proliferation of perceptions. 

Then,\marginnote{2.1} understanding his own giving up of the concepts of identity that emerge from the proliferation of perceptions, on that occasion the Buddha expressed this heartfelt sentiment: 

\begin{verse}%
“There\marginnote{3.1} is no proliferation remaining in them, \\
the reins and bar are escaped; \\
the sage who lives without craving \\
is never scorned by the world with its gods.” 

%
\end{verse}

%
\section*{{\suttatitleacronym Ud 7.8}{\suttatitletranslation Kaccāna }{\suttatitleroot Kaccānasutta}}
\addcontentsline{toc}{section}{\tocacronym{Ud 7.8} \toctranslation{Kaccāna } \tocroot{Kaccānasutta}}
\markboth{Kaccāna }{Kaccānasutta}
\extramarks{Ud 7.8}{Ud 7.8}

\scevam{So\marginnote{1.1} I have heard. }At one time the Buddha was staying near \textsanskrit{Sāvatthī} in Jeta’s Grove, \textsanskrit{Anāthapiṇḍika}’s monastery. Now at that time Venerable \textsanskrit{Mahākaccāna} was sitting not far from the Buddha, cross-legged, with his body straight and mindfulness of the body well-established in himself. 

The\marginnote{2.1} Buddha saw him meditating there. 

Then,\marginnote{3.1} understanding this matter, on that occasion the Buddha expressed this heartfelt sentiment: 

\begin{verse}%
“Their\marginnote{4.1} mindfulness would always \\
be established in the body, constant: \\
‘It might not be, and it might not be mine, \\
It will not be, and it will not be mine.’ \\
Meditating stage by stage on that, \\
in time they’d cross over clinging.” 

%
\end{verse}

%
\section*{{\suttatitleacronym Ud 7.9}{\suttatitletranslation The Well }{\suttatitleroot Udapānasutta}}
\addcontentsline{toc}{section}{\tocacronym{Ud 7.9} \toctranslation{The Well } \tocroot{Udapānasutta}}
\markboth{The Well }{Udapānasutta}
\extramarks{Ud 7.9}{Ud 7.9}

\scevam{So\marginnote{1.1} I have heard. }At one time the Buddha was wandering in the land of the Mallas together with a large \textsanskrit{Saṅgha} of mendicants when he arrived at a brahmin town of the Mallas named \textsanskrit{Thūṇa}. The brahmins and householders of \textsanskrit{Thūṇa} heard: “It seems the ascetic Gotama—a Sakyan, gone forth from a Sakyan family—while wandering in the land of the Mallas has arrived at \textsanskrit{Thūṇa}, together with a large \textsanskrit{Saṅgha} of mendicants.” They filled the well with grass and chaff right to the top, thinking, “Don’t let these shavelings, these fake ascetics drink the water.” 

And\marginnote{2.1} then the Buddha left the road, went to the root of a tree, and sat down on the seat spread out. When he was seated he said to Venerable Ānanda: “Please, Ānanda, fetch me some water from that well.” 

When\marginnote{3.1} he said this, Venerable Ānanda said to the Buddha, “Just now, sir, the brahimns and householders of \textsanskrit{Thūṇa} filled the well with grass and chaff right to the top, thinking, ‘Don’t let these shavelings, these fake ascetics drink the water.’” 

For\marginnote{4.1} a second time, and for a third time, the Buddha said to Ānanda: “Please, Ānanda, fetch me some water from that well.” “Yes, sir,” replied Ānanda. Taking his bowl he went to the well. As he approached the well, all the grass and chaff erupted out of the well-mouth. The water stood transparent, unclouded, and clear right up to the top, seeming to overflow. 

Then\marginnote{5.1} Venerable Ānanda thought, “It’s incredible, it’s amazing! The Realized One has such psychic power and might! For when I approached this well, all the grass and chaff erupted out of the well-mouth. The water stood transparent, unclouded, and clear right up to the top, seeming to overflow.” Gathering a bowl of drinking water he went back to the Buddha, and said to him, “It’s incredible, sir, it’s amazing! The Realized One has such psychic power and might! For when I approached that well, all the grass and chaff erupted out of the well-mouth. The water stood transparent, unclouded, and clear right up to the top, seeming to overflow. Drink the water, Blessed One! Drink the water, Holy One!” 

Then,\marginnote{6.1} understanding this matter, on that occasion the Buddha expressed this heartfelt sentiment: 

\begin{verse}%
“What\marginnote{7.1} difference would a well make \\
if water is there all the time? \\
Having cut off craving at the root, \\
who would go out on a quest?” 

%
\end{verse}

%
\section*{{\suttatitleacronym Ud 7.10}{\suttatitletranslation King Udena }{\suttatitleroot Utenasutta}}
\addcontentsline{toc}{section}{\tocacronym{Ud 7.10} \toctranslation{King Udena } \tocroot{Utenasutta}}
\markboth{King Udena }{Utenasutta}
\extramarks{Ud 7.10}{Ud 7.10}

\scevam{So\marginnote{1.1} I have heard. }At one time the Buddha was staying near Kosambi, in Ghosita’s Monastery. Now at that time, while King Udena was visiting a park, his royal compound burned down. Five hundred women died, with Queen \textsanskrit{Sāmāvatī} at their head. 

Then\marginnote{2.1} several mendicants robed up in the morning and, taking their bowls and robes, entered Kosambi for alms. Then, after the meal, when they returned from almsround, they went up to the Buddha, bowed, sat down to one side, and told him had happened. They asked the Buddha, “Sir, where have those laywomen been reborn in the next life?” 

“Among\marginnote{3.1} those laywomen there were stream-enterers, once-returners, and non-returners. None of those laywomen died without some fruit of the practice.” 

Then,\marginnote{4.1} understanding this matter, on that occasion the Buddha expressed this heartfelt sentiment: 

\begin{verse}%
“The\marginnote{5.1} world is caught up in delusion, \\
but is looked on as making sense. \\
The fool caught up in attachment \\
is surrounded by darkness. \\
It seems as if eternal, \\
but for one who sees, there is nothing.” 

%
\end{verse}

%
\addtocontents{toc}{\let\protect\contentsline\protect\nopagecontentsline}
\chapter*{The Chapter with the Pāṭali Villagers}
\addcontentsline{toc}{chapter}{\tocchapterline{The Chapter with the Pāṭali Villagers}}
\addtocontents{toc}{\let\protect\contentsline\protect\oldcontentsline}

%
\section*{{\suttatitleacronym Ud 8.1}{\suttatitletranslation About Extinguishment (1st) }{\suttatitleroot Paṭhamanibbānapaṭisaṁyuttasutta}}
\addcontentsline{toc}{section}{\tocacronym{Ud 8.1} \toctranslation{About Extinguishment (1st) } \tocroot{Paṭhamanibbānapaṭisaṁyuttasutta}}
\markboth{About Extinguishment (1st) }{Paṭhamanibbānapaṭisaṁyuttasutta}
\extramarks{Ud 8.1}{Ud 8.1}

\scevam{So\marginnote{1.1} I have heard. }At one time the Buddha was staying near \textsanskrit{Sāvatthī} in Jeta’s Grove, \textsanskrit{Anāthapiṇḍika}’s monastery. Now at that time the Buddha was educating, encouraging, firing up, and inspiring the mendicants with a Dhamma talk about extinguishment. And those mendicants were paying heed, paying attention, engaging wholeheartedly, and lending an ear. 

Then,\marginnote{2.1} understanding this matter, on that occasion the Buddha expressed this heartfelt sentiment: 

“There\marginnote{3.1} is, mendicants, that dimension where there is no earth, no water, no fire, no wind; no dimension of infinite space, no dimension of infinite consciousness, no dimension of nothingness, no dimension of neither perception nor non-perception; no this world, no other world, no moon or sun. There, mendicants, I say there is no coming or going or remaining or passing away or reappearing. It is not established, does not proceed, and has no support. Just this is the end of suffering.” 

%
\section*{{\suttatitleacronym Ud 8.2}{\suttatitletranslation About Extinguishment (2nd) }{\suttatitleroot Dutiyanibbānapaṭisaṁyuttasutta}}
\addcontentsline{toc}{section}{\tocacronym{Ud 8.2} \toctranslation{About Extinguishment (2nd) } \tocroot{Dutiyanibbānapaṭisaṁyuttasutta}}
\markboth{About Extinguishment (2nd) }{Dutiyanibbānapaṭisaṁyuttasutta}
\extramarks{Ud 8.2}{Ud 8.2}

\scevam{So\marginnote{1.1} I have heard. }At one time the Buddha was staying near \textsanskrit{Sāvatthī} in Jeta’s Grove, \textsanskrit{Anāthapiṇḍika}’s monastery. Now at that time the Buddha was educating, encouraging, firing up, and inspiring the mendicants with a Dhamma talk about extinguishment. And those mendicants were paying heed, paying attention, engaging wholeheartedly, and lending an ear. 

Then,\marginnote{2.1} understanding this matter, on that occasion the Buddha expressed this heartfelt sentiment: 

\begin{verse}%
“It’s\marginnote{3.1} hard to see what they call the ‘uninclined’, \\
for the truth is not easy to see. \\
For one who has penetrated craving, \\
who knows and sees, there is nothing.” 

%
\end{verse}

%
\section*{{\suttatitleacronym Ud 8.3}{\suttatitletranslation About Extinguishment (3rd) }{\suttatitleroot Tatiyanibbānapaṭisaṁyuttasutta}}
\addcontentsline{toc}{section}{\tocacronym{Ud 8.3} \toctranslation{About Extinguishment (3rd) } \tocroot{Tatiyanibbānapaṭisaṁyuttasutta}}
\markboth{About Extinguishment (3rd) }{Tatiyanibbānapaṭisaṁyuttasutta}
\extramarks{Ud 8.3}{Ud 8.3}

\scevam{So\marginnote{1.1} I have heard. }At one time the Buddha was staying near \textsanskrit{Sāvatthī} in Jeta’s Grove, \textsanskrit{Anāthapiṇḍika}’s monastery. Now at that time the Buddha was educating, encouraging, firing up, and inspiring the mendicants with a Dhamma talk about extinguishment. And those mendicants were paying heed, paying attention, engaging wholeheartedly, and lending an ear. 

Then,\marginnote{2.1} understanding this matter, on that occasion the Buddha expressed this heartfelt sentiment: 

“There\marginnote{3.1} is, mendicants, an unborn, unproduced, unmade, and unconditioned. If there were no unborn, unproduced, unmade, and unconditioned, then you would find no escape here from the born, produced, made, and conditioned. But since there is an unborn, unproduced, unmade, and unconditioned, an escape is found from the born, produced, made, and conditioned.” 

%
\section*{{\suttatitleacronym Ud 8.4}{\suttatitletranslation About Extinguishment (4th) }{\suttatitleroot Catutthanibbānapaṭisaṁyuttasutta}}
\addcontentsline{toc}{section}{\tocacronym{Ud 8.4} \toctranslation{About Extinguishment (4th) } \tocroot{Catutthanibbānapaṭisaṁyuttasutta}}
\markboth{About Extinguishment (4th) }{Catutthanibbānapaṭisaṁyuttasutta}
\extramarks{Ud 8.4}{Ud 8.4}

\scevam{So\marginnote{1.1} I have heard. }At one time the Buddha was staying near \textsanskrit{Sāvatthī} in Jeta’s Grove, \textsanskrit{Anāthapiṇḍika}’s monastery. Now at that time the Buddha was educating, encouraging, firing up, and inspiring the mendicants with a Dhamma talk about extinguishment. And those mendicants were paying heed, paying attention, engaging wholeheartedly, and lending an ear. 

Then,\marginnote{2.1} understanding this matter, on that occasion the Buddha expressed this heartfelt sentiment: 

“For\marginnote{3.1} the dependent there is agitation. For the independent there’s no agitation. When there’s no agitation there is tranquility. When there is tranquility there’s no inclination. When there’s no inclination, there’s no coming and going. When there’s no coming and going, there’s no passing away and reappearing. When there’s no passing away and reappearing there’s no this world or world beyond or between the two. Just this is the end of suffering.” 

%
\section*{{\suttatitleacronym Ud 8.5}{\suttatitletranslation With Cunda }{\suttatitleroot Cundasutta}}
\addcontentsline{toc}{section}{\tocacronym{Ud 8.5} \toctranslation{With Cunda } \tocroot{Cundasutta}}
\markboth{With Cunda }{Cundasutta}
\extramarks{Ud 8.5}{Ud 8.5}

\scevam{So\marginnote{1.1} I have heard. }At one time the Buddha was wandering in the land of the Mallas together with a large \textsanskrit{Saṅgha} when he arrived at \textsanskrit{Pāvā}. There he stayed in Cunda the smith’s mango grove. 

Cunda\marginnote{2.1} heard that the Buddha had arrived and was staying in his mango grove. Then he went to the Buddha, bowed, and sat down to one side. The Buddha educated, encouraged, fired up, and inspired him with a Dhamma talk. Then Cunda said to the Buddha, “Sir, may the Buddha together with the mendicant \textsanskrit{Saṅgha} please accept tomorrow’s meal from me.” The Buddha consented in silence. 

Then,\marginnote{3.1} knowing that the Buddha had consented, Cunda got up from his seat, bowed, and respectfully circled the Buddha, keeping him on his right, before leaving. And when the night had passed Cunda had a variety of delicious foods prepared in his own home, and plenty of pork on the turn. Then he had the Buddha informed of the time, saying, “Sir, it’s time. The meal is ready.” 

Then\marginnote{4.1} the Buddha robed up in the morning and, taking his bowl and robe, went to the home of Cunda together with the mendicant \textsanskrit{Saṅgha}, where he sat on the seat spread out and addressed Cunda, “Cunda, please serve me with the pork on the turn that you’ve prepared. And serve the mendicant \textsanskrit{Saṅgha} with the other foods.” “Yes, sir,” replied Cunda, and did as he was asked. 

Then\marginnote{5.1} the Buddha addressed Cunda, “Cunda, any pork on the turn that’s left over, you should bury it in a pond. I don’t see anyone in this world—with its gods, \textsanskrit{Māras}, and \textsanskrit{Brahmās}, this population with its ascetics and brahmins, its gods and humans—who could properly digest it except for the Realized One.” “Yes, sir,” replied Cunda. He did as he was asked, then came back to the Buddha, bowed, and sat down to one side. Then the Buddha educated, encouraged, fired up, and inspired him with a Dhamma talk, after which he got up from his seat and left. 

After\marginnote{6.1} the Buddha had eaten Cunda’s meal, he fell severely ill with bloody dysentery, struck by dreadful pains, close to death. But he endured unbothered, with mindfulness and situational awareness. 

Then\marginnote{7.1} the Buddha said to Venerable Ānanda, “Come, Ānanda, let’s go to \textsanskrit{Kusinārā}.” “Yes, sir,” Ānanda replied. 

\begin{verse}%
I’ve\marginnote{8.1} heard that after eating \\
the meal of Cunda the smith, \\
the wise one fell severely ill, \\
with pains, close to death. 

A\marginnote{9.1} severe sickness struck the Teacher \\
who had eaten the pork on the turn. \\
While still purging the Buddha said: \\
“I’ll go to the citadel of \textsanskrit{Kusinārā}.” 

%
\end{verse}

Then\marginnote{10.1} the Buddha left the road and went to the root of a certain tree, where he addressed Ānanda, “Please, Ānanda, fold my outer robe in four and spread it out for me. I am tired and will sit down.” “Yes, sir,” replied Ānanda, and did as he was asked. The Buddha sat on the seat spread out, and said to Venerable Ānanda, “Please, Ānanda, fetch me some water. I am thirsty and will drink.” 

When\marginnote{11.1} he said this, Venerable Ānanda said to the Buddha, “Sir, just now around five hundred carts have passed by. The shallow water has been churned up by their wheels, and it flows cloudy and murky. The \textsanskrit{Kakutthā} river is not far away, with clear, sweet, cool water, clean, with smooth banks, delightful. There the Buddha can drink and cool his limbs.” 

For\marginnote{12.1} a second time, and a third time, the Buddha said to Ānanda, “Please, Ānanda, fetch me some water. I am thirsty and will drink.” “Yes, sir,” replied Ānanda. Taking his bowl he went to the river. Now, though the shallow water in that river had been churned up by wheels, and flowed cloudy and murky, when Ānanda approached it flowed transparent, clear, and unclouded. 

Then\marginnote{13.1} Venerable Ānanda thought, “It’s incredible, it’s amazing! The Realized One has such psychic power and might! For though the shallow water in that river had been churned up by wheels, and flowed cloudy and murky, when I approached it flowed transparent, clear, and unclouded.” Gathering a bowl of drinking water he went back to the Buddha, and said to him, “It’s incredible, sir, it’s amazing! The Realized One has such psychic power and might! For though the shallow water in that river had been churned up by wheels, and flowed cloudy and murky, when I approached it flowed transparent, clear, and unclouded. Drink the water, Blessed One! Drink the water, Holy One!” 

So\marginnote{14.1} the Buddha drank the water. Then the Buddha together with a large \textsanskrit{Saṅgha} of mendicants went to the \textsanskrit{Kakutthā} River. He plunged into the river and bathed and drank. And when he had emerged, he went to the mango grove, where he addressed Venerable Cundaka, “Please, Cundaka, fold my outer robe in four and spread it out for me. I am tired and will lie down.” 

“Yes,\marginnote{15.1} sir,” replied Cundaka, and did as he was asked. And then the Buddha laid down in the lion’s posture—on the right side, placing one foot on top of the other—mindful and aware, and focused on the time of getting up. But Cundaka sat down right there in front of the Buddha. 

\begin{verse}%
Having\marginnote{16.1} gone to \textsanskrit{Kakutthā} Creek, \\
whose water was transparent, sweet, and clear, \\
the Teacher, being tired, plunged in, \\
the Realized One, without compare in the world. 

And\marginnote{17.1} after bathing and drinking the Teacher emerged. \\
Before the group of mendicants, in the middle, the Buddha, \\
the Teacher who rolled forth the present dispensation, \\
the great hermit went to the mango grove. \\
He addressed the mendicant named Cundaka: \\
“Spread out my folded robe so I can lie down.” 

The\marginnote{18.1} evolved one urged Cunda, \\
who quickly spread the folded robe. \\
The Teacher lay down so tired, \\
while Cunda sat there before him. 

%
\end{verse}

Then\marginnote{19.1} the Buddha said to Venerable Ānanda, “Now it may happen, Ānanda, that others may give rise to some regret for Cunda the smith: ‘It’s your loss, friend Cunda, it’s your misfortune, in that the Realized One became fully extinguished after eating his last almsmeal from you.’ You should get rid of remorse in Cunda the smith like this: 

‘You’re\marginnote{20.1} fortunate, friend Cunda, you’re so very fortunate, in that the Realized One became fully extinguished after eating his last almsmeal from you. I have heard and learned this in the presence of the Buddha. There are two almsmeal offerings that have identical fruit and result, and are more fruitful and beneficial than other almsmeal offerings. What two? The almsmeal after eating which a Realized One understands the supreme perfect awakening; and the almsmeal after eating which he becomes fully extinguished through the element of extinguishment with nothing left over. These two almsmeal offerings have identical fruit and result, and are more fruitful and beneficial than other almsmeal offerings. 

You’ve\marginnote{21.1} accumulated a deed that leads to long life, beauty, happiness, fame, heaven, and sovereignty.’ You should dispel remorse in Cunda the smith like this.” 

Then,\marginnote{22.1} understanding this matter, on that occasion the Buddha expressed this heartfelt sentiment: 

\begin{verse}%
“A\marginnote{23.1} giver’s merit grows; \\
enmity doesn’t build up when you have self-control. \\
A skillful person gives up bad things—\\
with the end of greed, hate, and delusion, they’re extinguished.” 

%
\end{verse}

%
\section*{{\suttatitleacronym Ud 8.6}{\suttatitletranslation The Layfolk of Pāṭali Village }{\suttatitleroot Pāṭaligāmiyasutta}}
\addcontentsline{toc}{section}{\tocacronym{Ud 8.6} \toctranslation{The Layfolk of Pāṭali Village } \tocroot{Pāṭaligāmiyasutta}}
\markboth{The Layfolk of Pāṭali Village }{Pāṭaligāmiyasutta}
\extramarks{Ud 8.6}{Ud 8.6}

\scevam{So\marginnote{1.1} I have heard. }At one time the Buddha was wandering in the land of the Magadhans together with a large \textsanskrit{Saṅgha} of mendicants when he arrived at the village of \textsanskrit{Pāṭali}. The lay followers of \textsanskrit{Pāṭali} Village heard that he had arrived. So they went to see him, bowed, sat down to one side, and said to him, “Sir, please consent to come to our guest house.” The Buddha consented in silence. 

Then,\marginnote{2.1} knowing that the Buddha had consented, the lay followers of \textsanskrit{Pāṭali} Village got up from their seat, bowed, and respectfully circled the Buddha, keeping him on their right. Then they went to the guest house, where they spread carpets all over, prepared seats, set up a water jar, and placed a lamp. Then they went back to the Buddha, bowed, stood to one side, and told him of their preparations, saying: “Please, sir, come at your convenience.” 

In\marginnote{3.1} the morning, the Buddha robed up and, taking his bowl and robe, went to the guest house together with the \textsanskrit{Saṅgha} of mendicants. Having washed his feet he entered the guest house and sat against the central column facing east. The \textsanskrit{Saṅgha} of mendicants also washed their feet, entered the guest house, and sat against the west wall facing east, with the Buddha right in front of them. The lay followers of \textsanskrit{Pāṭali} Village also washed their feet, entered the guest house, and sat against the east wall facing west, with the Buddha right in front of them. Then the Buddha addressed them: 

“Householders,\marginnote{4.1} there are these five drawbacks for an unethical person because of their failure in ethics. What five? Firstly, an unethical person loses substantial wealth on account of negligence. This is the first drawback. 

Furthermore,\marginnote{5.1} an unethical person gets a bad reputation. This is the second drawback. 

Furthermore,\marginnote{6.1} an unethical person enters any kind of assembly timid and embarrassed, whether it’s an assembly of aristocrats, brahmins, householders, or ascetics. This is the third drawback. 

Furthermore,\marginnote{7.1} an unethical person feels lost when they die. This is the fourth drawback. 

Furthermore,\marginnote{8.1} an unethical person, when their body breaks up, after death, is reborn in a place of loss, a bad place, the underworld, hell. This is the fifth drawback. These are the five drawbacks for an unethical person because of their failure in ethics. 

There\marginnote{9.1} are these five benefits for an ethical person because of their accomplishment in ethics. What five? Firstly, an ethical person gains substantial wealth on account of diligence. This is the first benefit. 

Furthermore,\marginnote{10.1} an ethical person gets a good reputation. This is the second benefit. 

Furthermore,\marginnote{11.1} an ethical person enters any kind of assembly bold and self-assured, whether it’s an assembly of aristocrats, brahmins, householders, or ascetics. This is the third benefit. 

Furthermore,\marginnote{12.1} an ethical person dies not feeling lost. This is the fourth benefit. 

Furthermore,\marginnote{13.1} when an ethical person’s body breaks up, after death, they’re reborn in a good place, a heavenly realm. This is the fifth benefit. These are the five benefits for an ethical person because of their accomplishment in ethics.” 

The\marginnote{14.1} Buddha spent most of the night educating, encouraging, firing up, and inspiring the lay followers of \textsanskrit{Pāṭali} Village with a Dhamma talk. Then he dismissed them, “The night is getting late, householders. Please go at your convenience.” And then the lay followers of \textsanskrit{Pāṭali} Village approved and agreed with what the Buddha said. They got up from their seat, bowed, and respectfully circled the Buddha, keeping him on their right, before leaving. Soon after they left the Buddha entered a private cubicle. 

Now\marginnote{15.1} at that time the Magadhan ministers Sunidha and \textsanskrit{Vassakāra} were building a citadel at \textsanskrit{Pāṭali} Village to keep the Vajjis out. At that time thousands of deities were taking possession of building sites in \textsanskrit{Pāṭali} Village. Illustrious rulers or royal ministers inclined to build houses at sites possessed by illustrious deities. Middling rulers or royal ministers inclined to build houses at sites possessed by middling deities. Lesser rulers or royal ministers inclined to build houses at sites possessed by lesser deities. 

With\marginnote{16.1} clairvoyance that is purified and superhuman, the Buddha saw those deities taking possession of building sites in \textsanskrit{Pāṭali} Village, and the people building houses in accord with the station of the deities. The Buddha rose at the crack of dawn and addressed Ānanda, 

“Ānanda,\marginnote{17.1} who is building a citadel at \textsanskrit{Pāṭali} Village?” “Sir, the Magadhan ministers Sunidha and \textsanskrit{Vassakāra} are building a citadel to keep the Vajjis out.” “It’s as if they were building the citadel in consultation with the gods of the Thirty-Three. With clairvoyance that is purified and superhuman, I saw those deities taking possession of building sites. Illustrious rulers or royal ministers inclined to build houses at sites possessed by illustrious deities. Middling rulers or royal ministers inclined to build houses at sites possessed by middling deities. Lesser rulers or royal ministers inclined to build houses at sites possessed by lesser deities. As far as the civilized region extends, as far as the trading zone extends, this will be the chief city: the \textsanskrit{Pāṭaliputta} trade center. But \textsanskrit{Pāṭaliputta} will face three threats: from fire, flood, and dissension.” 

Then\marginnote{18.1} the Magadhan ministers Sunidha and \textsanskrit{Vassakāra} approached the Buddha, and exchanged greetings with him. When the greetings and polite conversation were over, they stood to one side and said, “Would Master Gotama together with the mendicant \textsanskrit{Saṅgha} please accept today’s meal from me?” 

Then,\marginnote{19.1} knowing that the Buddha had consented, they went to their own guest house, where they had a variety of delicious foods prepared. Then they had the Buddha informed of the time, saying, “It’s time, Master Gotama, the meal is ready.” 

Then\marginnote{20.1} the Buddha robed up in the morning and, taking his bowl and robe, went to their guest house together with the mendicant \textsanskrit{Saṅgha}, where he sat on the seat spread out. Then Sunidha and \textsanskrit{Vassakāra} served and satisfied the mendicant \textsanskrit{Saṅgha} headed by the Buddha with their own hands with a variety of delicious foods. 

When\marginnote{21.1} the Buddha had eaten and washed his hand and bowl, Sunidha and \textsanskrit{Vassakāra} took a low seat and sat to one side. The Buddha expressed his appreciation with these verses: 

\begin{verse}%
“In\marginnote{22.1} the place he makes his dwelling, \\
having fed the astute \\
and the virtuous here, \\
the restrained spiritual practitioners, 

he\marginnote{23.1} should dedicate an offering \\
to the deities there. \\
Venerated, they venerate him; \\
honored, they honor him. 

After\marginnote{24.1} that they have compassion for him, \\
like a mother for the child at her breast. \\
A man beloved of the deities \\
always sees nice things.” 

%
\end{verse}

When\marginnote{25.1} the Buddha had expressed his appreciation to Sunidha and \textsanskrit{Vassakāra} with these verses, he got up from his seat and left. 

Sunidha\marginnote{26.1} and \textsanskrit{Vassakāra} followed behind the Buddha, thinking, “The gate through which the ascetic Gotama departs today shall be named the Gotama Gate. The ford at which he crosses the Ganges River shall be named the Gotama Ford.” 

Then\marginnote{27.1} the gate through which the Buddha departed was named the Gotama Gate. Then the Buddha came to the Ganges River. Now at that time the Ganges was full to the brim so a crow could drink from it. Wanting to cross from the near to the far shore, some people were seeking a boat, some a dinghy, while some were tying up a raft. But, as easily as a strong person would extend or contract their arm, the Buddha, together with the mendicant \textsanskrit{Saṅgha}, vanished from the near shore and landed on the far shore. 

He\marginnote{28.1} saw all those people wanting to cross over. 

Then,\marginnote{29.1} understanding this matter, on that occasion the Buddha expressed this heartfelt sentiment: 

\begin{verse}%
“Those\marginnote{30.1} who cross a deluge or stream \\
have built a bridge and left the marshes behind. \\
While some people are still tying a raft, \\
intelligent people have crossed over.” 

%
\end{verse}

%
\section*{{\suttatitleacronym Ud 8.7}{\suttatitletranslation A Fork in the Road }{\suttatitleroot Dvidhāpathasutta}}
\addcontentsline{toc}{section}{\tocacronym{Ud 8.7} \toctranslation{A Fork in the Road } \tocroot{Dvidhāpathasutta}}
\markboth{A Fork in the Road }{Dvidhāpathasutta}
\extramarks{Ud 8.7}{Ud 8.7}

\scevam{So\marginnote{1.1} I have heard. }At one time the Buddha was traveling along a road in the Kosalan lands with Venerable \textsanskrit{Nāgasamāla} as his second monk. \textsanskrit{Nāgasamāla} saw a fork in the road and said to the Buddha, “Sir, this is the road, let us go this way.” But when he said this the Buddha responded, “\textsanskrit{Nāgasamāla}, this is the road, let us go this way.” 

For\marginnote{2.1} a second time, and a  third time \textsanskrit{Nāgasamāla} said to the Buddha, “Sir, this is the road, let us go this way.” And for a third time the Buddha responded, “\textsanskrit{Nāgasamāla}, this is the road, let us go this way.” Then \textsanskrit{Nāgasamāla} put the Buddha’s bowl and robes down on the ground right there and left, saying, “Sir, here are your bowl and robes.” 

Then\marginnote{3.1} as \textsanskrit{Nāgasamāla} was going down that road, he was set upon by bandits who struck him with fists and feet, broke his bowl, and tore up his outer robe. Then \textsanskrit{Nāgasamāla}—with his bowl broken and his outer robe torn—went to the Buddha and told him what had happened. 

Then,\marginnote{4.1} understanding this matter, on that occasion the Buddha expressed this heartfelt sentiment: 

\begin{verse}%
“Walking\marginnote{5.1} together, dwelling as one, \\
the knowledge master mixes with foolish folk. \\
Knowing this, they give up wickedness, \\
like a milk-drinking heron the water.” 

%
\end{verse}

%
\section*{{\suttatitleacronym Ud 8.8}{\suttatitletranslation With Visākhā }{\suttatitleroot Visākhāsutta}}
\addcontentsline{toc}{section}{\tocacronym{Ud 8.8} \toctranslation{With Visākhā } \tocroot{Visākhāsutta}}
\markboth{With Visākhā }{Visākhāsutta}
\extramarks{Ud 8.8}{Ud 8.8}

\scevam{So\marginnote{1.1} I have heard. }At one time the Buddha was staying near \textsanskrit{Sāvatthī} in the Eastern Monastery, the stilt longhouse of \textsanskrit{Migāra}’s mother. Now at that time the dear and beloved granddaughter of \textsanskrit{Visākhā} \textsanskrit{Migāra}’s Mother had just passed away. Then, in the middle of the day, \textsanskrit{Visākhā} with wet clothes and hair went to the Buddha, bowed, and sat down. The Buddha said to her, 

“So,\marginnote{2.1} \textsanskrit{Visākhā}, where are you coming from in the middle of the day with wet clothes and hair?” “Sir, my beloved granddaughter has just passed away. That’s why I came here in the middle of the day with wet clothes and hair.” “\textsanskrit{Visākhā}, would you like as many children and grandchildren as there are people in the whole of \textsanskrit{Sāvatthī}?” “I would, sir.” 

“But\marginnote{3.1} \textsanskrit{Visākhā}, how many people pass away each day in \textsanskrit{Sāvatthī}?” “Every day, sir, there are ten people passing away in \textsanskrit{Sāvatthī}. Or else there are nine, eight, seven, six, five, four, three, two, or at least one person who passes away every day in \textsanskrit{Sāvatthī}. \textsanskrit{Sāvatthī} is never without someone passing away.” 

“What\marginnote{4.1} do you think, \textsanskrit{Visākhā}? Would there ever be a time when your clothes and hair were not wet?” “No, sir. Enough, sir, with so many children and grandchildren.” 

“Those\marginnote{5.1} who have a hundred loved ones, \textsanskrit{Visākhā}, have a hundred sufferings. Those who have ninety loved ones, or eighty, seventy, sixty, fifty, forty, thirty, twenty, ten, nine, eight, seven, six, five, four, three, two, or one loved one have one suffering. Those who have no loved ones have no suffering. They are free of sorrow, stains, and anguish I say.” 

Then,\marginnote{6.1} understanding this matter, on that occasion the Buddha expressed this heartfelt sentiment: 

\begin{verse}%
“All\marginnote{7.1} the sorrows and lamentations \\
and the countless forms of suffering in the world \\
occur because of those that we love; \\
without loved ones they do not occur. 

That’s\marginnote{8.1} why those who have no loved ones at all in the world \\
are happy and free of grief. \\
So aspiring to the sorrowless and stainless, \\
have no loved ones in the world at all.” 

%
\end{verse}

%
\section*{{\suttatitleacronym Ud 8.9}{\suttatitletranslation With Dabba (1st) }{\suttatitleroot Paṭhamadabbasutta}}
\addcontentsline{toc}{section}{\tocacronym{Ud 8.9} \toctranslation{With Dabba (1st) } \tocroot{Paṭhamadabbasutta}}
\markboth{With Dabba (1st) }{Paṭhamadabbasutta}
\extramarks{Ud 8.9}{Ud 8.9}

\scevam{So\marginnote{1.1} I have heard. }At one time the Buddha was staying near \textsanskrit{Rājagaha}, in the Bamboo Grove, the squirrels’ feeding ground. Then Venerable Dabba the Mallian went up to the Buddha, bowed, sat down to one side, and said to him: “Holy One, it is the time for my full extinguishment.” “Please, Dabba, do as you see fit.” 

Then\marginnote{2.1} Dabba rose from his seat, bowed and respectfully circled the Buddha, keeping him on his right. Then he rose into the air and, sitting cross-legged in the sky, entered and withdrew from the fire element before becoming fully extinguished. 

Then\marginnote{3.1} when he became fully extinguished while sitting in the sky, his body burning and combusting left neither ashes nor soot to be found. It’s like when ghee or oil blaze and burn, and neither ashes nor soot are found. 

Then,\marginnote{4.1} understanding this matter, on that occasion the Buddha expressed this heartfelt sentiment: 

\begin{verse}%
“The\marginnote{5.1} body is broken up, perception has ceased, \\
all feelings have become cool; \\
choices are stilled, \\
and consciousness come to an end.” 

%
\end{verse}

%
\section*{{\suttatitleacronym Ud 8.10}{\suttatitletranslation Dabba (2nd) }{\suttatitleroot Dutiyadabbasutta}}
\addcontentsline{toc}{section}{\tocacronym{Ud 8.10} \toctranslation{Dabba (2nd) } \tocroot{Dutiyadabbasutta}}
\markboth{Dabba (2nd) }{Dutiyadabbasutta}
\extramarks{Ud 8.10}{Ud 8.10}

\scevam{So\marginnote{1.1} I have heard. }At one time the Buddha was staying near \textsanskrit{Sāvatthī} in Jeta’s Grove, \textsanskrit{Anāthapiṇḍika}’s monastery. There the Buddha addressed the mendicants, “Mendicants!” “Venerable sir,” they replied. The Buddha said this: 

“Mendicants,\marginnote{2.1} when Dabba the Mallian rose into the air and, sitting cross-legged in the sky, entered and withdrew from the fire element before becoming fully extinguished, his body burning and combusting left neither ashes nor soot to be found. It’s like when ghee or oil blaze and burn, and neither ashes nor soot are found. In the same way, when Dabba the Mallian rose into the air and, sitting cross-legged in the sky, entered and withdrew from the fire element before becoming fully extinguished, his body burning and combusting left neither ashes nor soot to be found.” 

Then,\marginnote{3.1} understanding this matter, on that occasion the Buddha expressed this heartfelt sentiment: 

\begin{verse}%
“When\marginnote{4.1} an iron bar is struck \\
by heat and flame \\
the heat gradually dissipates, \\
and where it has gone no-one knows. 

In\marginnote{5.1} the same way for the rightly released, \\
who have crossed the flood of sensual bonds, \\
and attained unshakable happiness, \\
where they have gone cannot be found.” 

%
\end{verse}

\scendbook{The Heartfelt Sayings are finished. }

%
\backmatter%
%
\chapter*{Colophon}
\addcontentsline{toc}{chapter}{Colophon}
\markboth{Colophon}{Colophon}

\section*{The Translator}

Bhikkhu Sujato was born as Anthony Aidan Best on 4/11/1966 in Perth, Western Australia. He grew up in the pleasant suburbs of Mt Lawley and Attadale alongside his sister Nicola, who was the good child. His mother, Margaret Lorraine Huntsman née Pinder, said “he’ll either be a priest or a poet”, while his father, Anthony Thomas Best, advised him to “never do anything for money”. He attended Aquinas College, a Catholic school, where he decided to become an atheist. At the University of WA he studied philosophy, aiming to learn what he wanted to do with his life. Finding that what he wanted to do was play guitar, he dropped out. His main band was named Martha’s Vineyard, which achieved modest success in the indie circuit. Then it broke up, because everyone thought they personally were reason for the success, which, oddly enough, turns out not to have been the case. 

A seemingly random encounter with a roadside joey took him to Thailand, where he entered his first meditation retreat at Wat Ram Poeng, Chieng Mai in 1992. He decided to devote himself to the Buddha’s path, and took full ordination in Wat Pa Nanachat in 1994, where his teachers were Ajahn Pasanno and Ajahn Jayasaro. In 1997 he returned to Perth to study with Ajahn Brahm at Bodhinyana Monastery. 

He spent several years practicing in seclusion in Malaysia and Thailand before establishing Santi Forest Monastery in Bundanoon, NSW, in 2003. There he was instrumental in supporting the establishment of the Theravada bhikkhuni order in Australia and advocating for women’s rights. He continues to teach in Australia and globally, with a special concern for the moral implications of climate change and other forms of environmental destruction. He has published a series of books of original and groundbreaking research on early Buddhism. 

In 2005 he founded SuttaCentral together with Rod Bucknell and John Kelly. In 2015, seeing the need for a complete, accurate, plain English translation of the Pali texts, he undertook the task, spending nearly three years in isolation on the isle of Qi Mei off the coast of the nation of Taiwan. He completed the four main \textsanskrit{Nikāyas} in 2018, and the early books of the Khuddaka \textsanskrit{Nikāya} were complete by 2021. All this work is dedicated to the public domain and is entirely free of copyright encumbrance. 

In 2019 he returned to Sydney where, together with Bhikkhu Akaliko, he established Lokanta Vihara (The Monastery at the End of the World). 

\section*{Creation Process}

Translated from the Pali. Primary source was the \textsanskrit{Mahāsaṅgīti} edition, with reference to several English translations, especially those of John Ireland and Bhikkhu Ānandajoti.

\section*{The Translation}

The \textsanskrit{Udāna} has a distinctive form, being comprised of Dhammapada-style verses together with contextual narratives in prose. It thus straddles the styles of the prose and verse Suttas. This translation aims to make a clear, readable, and accurate rendering.

\section*{About SuttaCentral}

SuttaCentral publishes early Buddhist texts. Since 2005 we have provided root texts in Pali, Chinese, Sanskrit, Tibetan, and other languages, parallels between these texts, and translations in many modern languages. We build on the work of generations of scholars, and offer our contribution freely.

SuttaCentral is driven by volunteer contributions, and in addition we employ professional developers. We offer a sponsorship program for high quality translations from the original languages. Financial support for SuttaCentral is handled by the SuttaCentral Development Trust, a charitable trust registered in Australia.

\section*{About Bilara}

“Bilara” means “cat” in Pali, and it is the name of our Computer Assisted Translation (CAT) software. Bilara is a web app that enables translators to translate early Buddhist texts into their own language. These translations are published on SuttaCentral with the root text and translation side by side.

\section*{About SuttaCentral Editions}

The SuttaCentral Editions project makes high quality books from selected Bilara translations. These are published in formats including HTML, EPUB, PDF, and print.

If you want to print any of our Editions, please let us know and we will help prepare a file to your specifications.

%
\end{document}