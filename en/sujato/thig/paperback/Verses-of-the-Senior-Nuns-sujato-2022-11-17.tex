\documentclass[12pt,openany]{book}%
\usepackage{lastpage}%
%
\usepackage[inner=1in, outer=1in, top=.7in, bottom=1in, papersize={6in,9in}, headheight=13pt]{geometry}
\usepackage{polyglossia}
\usepackage[12pt]{moresize}
\usepackage{soul}%
\usepackage{microtype}
\usepackage{tocbasic}
\usepackage{realscripts}
\usepackage{epigraph}%
\usepackage{setspace}%
\usepackage{sectsty}
\usepackage{fontspec}
\usepackage{marginnote}
\usepackage[bottom]{footmisc}
\usepackage{enumitem}
\usepackage{fancyhdr}
\usepackage{extramarks}
\usepackage{graphicx}
\usepackage{verse}
\usepackage{relsize}
\usepackage{etoolbox}
\usepackage[a-3u]{pdfx}

\hypersetup{
colorlinks=true,
urlcolor=black,
linkcolor=black,
citecolor=black
}

% use a small amount of tracking on small caps
\SetTracking[ spacing = {25*,166, } ]{ encoding = *, shape = sc }{ 25 }

% add a blank page
\newcommand{\blankpage}{
\newpage
\thispagestyle{empty}
\mbox{}
\newpage
}

% define languages
\setdefaultlanguage[]{english}
\setotherlanguage[script=Latin]{sanskrit}

%\usepackage{pagegrid}
%\pagegridsetup{top-left, step=.25in}

% define fonts
% use if arno sanskrit is unavailable
%\setmainfont{Gentium Plus}
%\newfontfamily\Semiboldsubheadfont[]{Gentium Plus}
%\newfontfamily\Semiboldnormalfont[]{Gentium Plus}
%\newfontfamily\Lightfont[]{Gentium Plus}
%\newfontfamily\Marginalfont[]{Gentium Plus}
%\newfontfamily\Allsmallcapsfont[RawFeature=+c2sc]{Gentium Plus}
%\newfontfamily\Noligaturefont[Renderer=Basic]{Gentium Plus}
%\newfontfamily\Noligaturecaptionfont[Renderer=Basic]{Gentium Plus}
%\newfontfamily\Fleuronfont[Ornament=1]{Gentium Plus}

% use if arno sanskrit is available. display is applied to \chapter and \part, subhead to \section and \subsection. When specifying semibold, the italic must be defined.
\setmainfont[Numbers=OldStyle]{Arno Pro}
\newfontfamily\Semibolddisplayfont[BoldItalicFont = Arno Pro Semibold Italic Display]{Arno Pro Semibold Display} %
\newfontfamily\Semiboldsubheadfont[BoldItalicFont = Arno Pro Semibold Italic Subhead]{Arno Pro Semibold Subhead}
\newfontfamily\Semiboldnormalfont[BoldItalicFont = Arno Pro Semibold Italic]{Arno Pro Semibold}
\newfontfamily\Marginalfont[RawFeature=+subs]{Arno Pro Regular}
\newfontfamily\Allsmallcapsfont[RawFeature=+c2sc]{Arno Pro}
\newfontfamily\Noligaturefont[Renderer=Basic]{Arno Pro}
\newfontfamily\Noligaturecaptionfont[Renderer=Basic]{Arno Pro Caption}

% chinese fonts
\newfontfamily\cjk{Noto Serif TC}
\newcommand*{\langlzh}[1]{\cjk{#1}\normalfont}%

% logo
\newfontfamily\Logofont{sclogo.ttf}
\newcommand*{\sclogo}[1]{\large\Logofont{#1}}

% use subscript numerals for margin notes
\renewcommand*{\marginfont}{\Marginalfont}

% ensure margin notes have consistent vertical alignment
\renewcommand*{\marginnotevadjust}{-.17em}

% use compact lists
\setitemize{noitemsep,leftmargin=1em}
\setenumerate{noitemsep,leftmargin=1em}
\setdescription{noitemsep, style=unboxed, leftmargin=0em}

% style ToC
\DeclareTOCStyleEntries[
  raggedentrytext,
  linefill=\hfill,
  pagenumberwidth=.5in,
  pagenumberformat=\normalfont,
  entryformat=\normalfont
]{tocline}{chapter,section}


  \setlength\topsep{0pt}%
  \setlength\parskip{0pt}%

% define new \centerpars command for use in ToC. This ensures centering, proper wrapping, and no page break after
\def\startcenter{%
  \par
  \begingroup
  \leftskip=0pt plus 1fil
  \rightskip=\leftskip
  \parindent=0pt
  \parfillskip=0pt
}
\def\stopcenter{%
  \par
  \endgroup
}
\long\def\centerpars#1{\startcenter#1\stopcenter}

% redefine part, so that it adds a toc entry without page number
\let\oldcontentsline\contentsline
\newcommand{\nopagecontentsline}[3]{\oldcontentsline{#1}{#2}{}}

    \makeatletter
\renewcommand*\l@part[2]{%
  \ifnum \c@tocdepth >-2\relax
    \addpenalty{-\@highpenalty}%
    \addvspace{0em \@plus\p@}%
    \setlength\@tempdima{3em}%
    \begingroup
      \parindent \z@ \rightskip \@pnumwidth
      \parfillskip -\@pnumwidth
      {\leavevmode
       \setstretch{.85}\large\scshape\centerpars{#1}\vspace*{-1em}\llap{#2}}\par
       \nobreak
         \global\@nobreaktrue
         \everypar{\global\@nobreakfalse\everypar{}}%
    \endgroup
  \fi}
\makeatother

\makeatletter
\def\@pnumwidth{2em}
\makeatother

% define new sectioning command, which is only used in volumes where the pannasa is found in some parts but not others, especially in an and sn

\newcommand*{\pannasa}[1]{\clearpage\thispagestyle{empty}\begin{center}\vspace*{14em}\setstretch{.85}\huge\itshape\scshape\MakeLowercase{#1}\end{center}}

    \makeatletter
\newcommand*\l@pannasa[2]{%
  \ifnum \c@tocdepth >-2\relax
    \addpenalty{-\@highpenalty}%
    \addvspace{.5em \@plus\p@}%
    \setlength\@tempdima{3em}%
    \begingroup
      \parindent \z@ \rightskip \@pnumwidth
      \parfillskip -\@pnumwidth
      {\leavevmode
       \setstretch{.85}\large\itshape\scshape\lowercase{\centerpars{#1}}\vspace*{-1em}\llap{#2}}\par
       \nobreak
         \global\@nobreaktrue
         \everypar{\global\@nobreakfalse\everypar{}}%
    \endgroup
  \fi}
\makeatother

% don't put page number on first page of toc (relies on etoolbox)
\patchcmd{\chapter}{plain}{empty}{}{}

% global line height
\setstretch{1.05}

% allow linebreak after em-dash
\catcode`\—=13
\protected\def—{\unskip\textemdash\allowbreak}

% style headings with secsty. chapter and section are defined per-edition
\partfont{\setstretch{.85}\normalfont\centering\textsc}
\subsectionfont{\setstretch{.85}\Semiboldsubheadfont}%
\subsubsectionfont{\setstretch{.85}\Semiboldnormalfont}

% style elements of suttatitle
\newcommand*{\suttatitleacronym}[1]{\smaller[2]{#1}\vspace*{.3em}}
\newcommand*{\suttatitletranslation}[1]{\linebreak{#1}}
\newcommand*{\suttatitleroot}[1]{\linebreak\smaller[2]\itshape{#1}}

\DeclareTOCStyleEntries[
  indent=3.3em,
  dynindent,
  beforeskip=.2em plus -2pt minus -1pt,
]{tocline}{section}

\DeclareTOCStyleEntries[
  indent=0em,
  dynindent,
  beforeskip=.4em plus -2pt minus -1pt,
]{tocline}{chapter}

\newcommand*{\tocacronym}[1]{\hspace*{-3.3em}{#1}\quad}
\newcommand*{\toctranslation}[1]{#1}
\newcommand*{\tocroot}[1]{(\textit{#1})}
\newcommand*{\tocchapterline}[1]{\bfseries\itshape{#1}}


% redefine paragraph and subparagraph headings to not be inline
\makeatletter
% Change the style of paragraph headings %
\renewcommand\paragraph{\@startsection{paragraph}{4}{\z@}%
            {-2.5ex\@plus -1ex \@minus -.25ex}%
            {1.25ex \@plus .25ex}%
            {\noindent\Semiboldnormalfont\normalsize}}

% Change the style of subparagraph headings %
\renewcommand\subparagraph{\@startsection{subparagraph}{5}{\z@}%
            {-2.5ex\@plus -1ex \@minus -.25ex}%
            {1.25ex \@plus .25ex}%
            {\noindent\Semiboldnormalfont\small}}
\makeatother

% use etoolbox to suppress page numbers on \part
\patchcmd{\part}{\thispagestyle{plain}}{\thispagestyle{empty}}
  {}{\errmessage{Cannot patch \string\part}}

% and to reduce margins on quotation
\patchcmd{\quotation}{\rightmargin}{\leftmargin 1.2em \rightmargin}{}{}
\AtBeginEnvironment{quotation}{\small}

% titlepage
\newcommand*{\titlepageTranslationTitle}[1]{{\begin{center}\begin{large}{#1}\end{large}\end{center}}}
\newcommand*{\titlepageCreatorName}[1]{{\begin{center}\begin{normalsize}{#1}\end{normalsize}\end{center}}}

% halftitlepage
\newcommand*{\halftitlepageTranslationTitle}[1]{\setstretch{2.5}{\begin{Huge}\uppercase{\so{#1}}\end{Huge}}}
\newcommand*{\halftitlepageTranslationSubtitle}[1]{\setstretch{1.2}{\begin{large}{#1}\end{large}}}
\newcommand*{\halftitlepageFleuron}[1]{{\begin{large}\Fleuronfont{{#1}}\end{large}}}
\newcommand*{\halftitlepageByline}[1]{{\begin{normalsize}\textit{{#1}}\end{normalsize}}}
\newcommand*{\halftitlepageCreatorName}[1]{{\begin{LARGE}{\textsc{#1}}\end{LARGE}}}
\newcommand*{\halftitlepageVolumeNumber}[1]{{\begin{normalsize}{\Allsmallcapsfont{\textsc{#1}}}\end{normalsize}}}
\newcommand*{\halftitlepageVolumeAcronym}[1]{{\begin{normalsize}{#1}\end{normalsize}}}
\newcommand*{\halftitlepageVolumeTranslationTitle}[1]{{\begin{Large}{\textsc{#1}}\end{Large}}}
\newcommand*{\halftitlepageVolumeRootTitle}[1]{{\begin{normalsize}{\Allsmallcapsfont{\textsc{\itshape #1}}}\end{normalsize}}}
\newcommand*{\halftitlepagePublisher}[1]{{\begin{large}{\Noligaturecaptionfont\textsc{#1}}\end{large}}}

% epigraph
\renewcommand{\epigraphflush}{center}
\renewcommand*{\epigraphwidth}{.85\textwidth}
\newcommand*{\epigraphTranslatedTitle}[1]{\vspace*{.5em}\footnotesize\textsc{#1}\\}%
\newcommand*{\epigraphRootTitle}[1]{\footnotesize\textit{#1}\\}%
\newcommand*{\epigraphReference}[1]{\footnotesize{#1}}%

% custom commands for html styling classes
\newcommand*{\scnamo}[1]{\begin{center}\textit{#1}\end{center}}
\newcommand*{\scendsection}[1]{\begin{center}\textit{#1}\end{center}}
\newcommand*{\scendsutta}[1]{\begin{center}\textit{#1}\end{center}}
\newcommand*{\scendbook}[1]{\begin{center}\uppercase{#1}\end{center}}
\newcommand*{\scendkanda}[1]{\begin{center}\textbf{#1}\end{center}}
\newcommand*{\scend}[1]{\begin{center}\textit{#1}\end{center}}
\newcommand*{\scuddanaintro}[1]{\textit{#1}}
\newcommand*{\scendvagga}[1]{\begin{center}\textbf{#1}\end{center}}
\newcommand*{\scrule}[1]{\textbf{#1}}
\newcommand*{\scadd}[1]{\textit{#1}}
\newcommand*{\scevam}[1]{\textsc{#1}}
\newcommand*{\scspeaker}[1]{\hspace{2em}\textit{#1}}
\newcommand*{\scbyline}[1]{\begin{flushright}\textit{#1}\end{flushright}\bigskip}

% custom command for thematic break = hr
\newcommand*{\thematicbreak}{\begin{center}\rule[.5ex]{6em}{.4pt}\begin{normalsize}\quad\Fleuronfont{•}\quad\end{normalsize}\rule[.5ex]{6em}{.4pt}\end{center}}

% manage and style page header and footer. "fancy" has header and footer, "plain" has footer only

\pagestyle{fancy}
\fancyhf{}
\fancyfoot[RE,LO]{\thepage}
\fancyfoot[LE,RO]{\footnotesize\lastleftxmark}
\fancyhead[CE]{\setstretch{.85}\Noligaturefont\MakeLowercase{\textsc{\firstrightmark}}}
\fancyhead[CO]{\setstretch{.85}\Noligaturefont\MakeLowercase{\textsc{\firstleftmark}}}
\renewcommand{\headrulewidth}{0pt}
\fancypagestyle{plain}{ %
\fancyhf{} % remove everything
\fancyfoot[RE,LO]{\thepage}
\fancyfoot[LE,RO]{\footnotesize\lastleftxmark}
\renewcommand{\headrulewidth}{0pt}
\renewcommand{\footrulewidth}{0pt}}

% style footnotes
\setlength{\skip\footins}{1em}

\makeatletter
\newcommand{\@makefntextcustom}[1]{%
    \parindent 0em%
    \thefootnote.\enskip #1%
}
\renewcommand{\@makefntext}[1]{\@makefntextcustom{#1}}
\makeatother

% hang quotes (requires microtype)
\microtypesetup{
  protrusion = true,
  expansion  = true,
  tracking   = true,
  factor     = 1000,
  patch      = all,
  final
}

% Custom protrusion rules to allow hanging punctuation
\SetProtrusion
{ encoding = *}
{
% char   right left
  {-} = {    , 500 },
  % Double Quotes
  \textquotedblleft
      = {1000,     },
  \textquotedblright
      = {    , 1000},
  \quotedblbase
      = {1000,     },
  % Single Quotes
  \textquoteleft
      = {1000,     },
  \textquoteright
      = {    , 1000},
  \quotesinglbase
      = {1000,     }
}

% make latex use actual font em for parindent, not Computer Modern Roman
\AtBeginDocument{\setlength{\parindent}{1em}}%
%

% Default values; a bit sloppier than normal
\tolerance 1414
\hbadness 1414
\emergencystretch 1.5em
\hfuzz 0.3pt
\clubpenalty = 10000
\widowpenalty = 10000
\displaywidowpenalty = 10000
\hfuzz \vfuzz
 \raggedbottom%

\title{Verses of the Senior Nuns}
\author{Bhikkhu Sujato}
\date{}%
% define a different fleuron for each edition
\newfontfamily\Fleuronfont[Ornament=36]{Arno Pro}

% Define heading styles per edition for chapter and section. Suttatitle can be either of these, depending on the volume. 

\let\oldfrontmatter\frontmatter
\renewcommand{\frontmatter}{%
\chapterfont{\setstretch{.85}\normalfont\centering}%
\sectionfont{\setstretch{.85}\Semiboldsubheadfont}%
\oldfrontmatter}

\let\oldmainmatter\mainmatter
\renewcommand{\mainmatter}{%
\chapterfont{\setstretch{.85}\normalfont\centering}%
\sectionfont{\setstretch{.85}\normalfont\centering}%
\oldmainmatter}

\let\oldbackmatter\backmatter
\renewcommand{\backmatter}{%
\chapterfont{\setstretch{.85}\normalfont\centering}%
\sectionfont{\setstretch{.85}\Semiboldsubheadfont}%
\oldbackmatter}

%
%
\begin{document}%
\normalsize%
\frontmatter%
\setlength{\parindent}{0cm}

\pagestyle{empty}

\maketitle

\blankpage%
\begin{center}

\vspace*{2.2em}

\halftitlepageTranslationTitle{Verses of the Senior Nuns}

\vspace*{1em}

\halftitlepageTranslationSubtitle{A friendly translation of the Therīgāthā}

\vspace*{2em}

\halftitlepageFleuron{•}

\vspace*{2em}

\halftitlepageByline{translated and introduced by}

\vspace*{.5em}

\halftitlepageCreatorName{Bhikkhu Sujato}

\vspace*{4em}

\halftitlepageVolumeNumber{}

\smallskip

\halftitlepageVolumeAcronym{Thig}

\smallskip

\halftitlepageVolumeTranslationTitle{}

\smallskip

\halftitlepageVolumeRootTitle{}

\vspace*{\fill}

\sclogo{0}
 \halftitlepagePublisher{SuttaCentral}

\end{center}

\newpage
%
\setstretch{1.05}

\begin{footnotesize}

\textit{Verses of the Senior Nuns} is a translation of the Therīgāthā by Bhikkhu Sujato.

\medskip

Creative Commons Zero (CC0)

To the extent possible under law, Bhikkhu Sujato has waived all copyright and related or neighboring rights to \textit{Verses of the Senior Nuns}.

\medskip

This work is published from Australia.

\begin{center}
\textit{This translation is an expression of an ancient spiritual text that has been passed down by the Buddhist tradition for the benefit of all sentient beings. It is dedicated to the public domain via Creative Commons Zero (CC0). You are encouraged to copy, reproduce, adapt, alter, or otherwise make use of this translation. The translator respectfully requests that any use be in accordance with the values and principles of the Buddhist community.}
\end{center}

\medskip

\begin{description}
    \item[Web publication date] 2019
    \item[This edition] 2022-11-17 09:04:40
    \item[Publication type] paperback
    \item[Edition] ed5
    \item[Number of volumes] 1
    \item[Publication ISBN] 978-1-76132-046-0
    \item[Publication URL] https://suttacentral.net/editions/thig/en/sujato
    \item[Source URL] https://github.com/suttacentral/bilara-data/tree/published/translation/en/sujato/sutta/kn/thig
    \item[Publication number] scpub6
\end{description}

\medskip

Published by SuttaCentral

\medskip

\textit{SuttaCentral,\\
c/o Alwis \& Alwis Pty Ltd\\
Kaurna Country,\\
Suite 12,\\
198 Greenhill Road,\\
Eastwood,\\
SA 5063,\\
Australia}

\end{footnotesize}

\newpage

\setlength{\parindent}{1.5em}%%
\newpage

\vspace*{\fill}

\begin{center}
\epigraph{Sleep softly, little nun,\\
wrapped in the cloth you sewed yourself;\\
for your desire has been quelled,\\
like vegetables boiled dry in a pot.}
{
\epigraphTranslatedTitle{Verse for an Unnamed Nun}
\epigraphRootTitle{}
\epigraphReference{\textsanskrit{Therīgāthā} 1.1}
}
\end{center}

\vspace*{2in}

\vspace*{\fill}

\blankpage%

\setlength{\parindent}{1em}
%
\tableofcontents
\newpage
\pagestyle{fancy}
%
\chapter*{The SuttaCentral Editions Series}
\addcontentsline{toc}{chapter}{The SuttaCentral Editions Series}
\markboth{The SuttaCentral Editions Series}{The SuttaCentral Editions Series}

Since 2005 SuttaCentral has provided access to the texts, translations, and parallels of early Buddhist texts. In 2018 we started creating and publishing our own translations of these seminal spiritual classics. The “Editions” series now makes selected translations available as books in various forms, including print, PDF, and EPUB.

Editions are selected from our most complete, well-crafted, and reliable translations. They aim to bring these texts to a wider audience in forms that reward mindful reading. Care is taken with every detail of the production, and we aim to meet or exceed professional best standards in every way. These are the core scriptures underlying the entire Buddhist tradition, and we believe that they deserve to be preserved and made available in highest quality without compromise.

SuttaCentral is a charitable organization. Our work is accomplished by volunteers and through the generosity of our donors. Everything we create is offered to all of humanity free of any copyright or licensing restrictions. 

%
\chapter*{Preface to the \textsanskrit{Therīgāthā}}
\addcontentsline{toc}{chapter}{Preface to the \textsanskrit{Therīgāthā}}
\markboth{Preface to the \textsanskrit{Therīgāthā}}{Preface to the \textsanskrit{Therīgāthā}}

While writing the introduction for this book, I found myself reflecting on my own approach to analysis of texts. I find myself increasingly put off by heavily theoretical approaches, or by analyses that invent a scheme of categorization into which the texts must fall. When teaching the Suttas, my approach is pretty much the same as it is when reading them: open the book and start reading. Listen to what it is actually saying.

The key is empathy and critical discernment. I never assume that I am in a position of moral authority from which to judge people of another time and place. I’m here to learn, not to condemn. And the further people are away from me, the more I have to learn from them.

To me, a genuine inquiry begins with the willingness to question my own values and assumptions. But over the years, I have picked up one or two things that might be useful for others. So if, as a writer and a teacher, I can clarify some facts, smooth the path, and set an example of an honest inquiry, I’ll be happy.

When teaching Suttas, I’ve noticed two biasses that obscure vision. For some folks, the process of reading is an entirely passive venture, in which their only concern is to find the single, correct, and authoritative meaning. For others, their subjective feelings or theories about the text are paramount, and they feel good when they succeed in squeezing an ancient sacred text into their preconceptions. 

Both approaches are lazy and far from wisdom. Understanding arises when you see the hidden connections between distant things. It’s not about passing a test or proving your ideological purity. It’s about that moment when you \emph{see}. You can’t control it or predict it. 

Getting your facts straight is important. It takes discipline and years of hard work to learn how to sift out one’s own views and to listen with clarity and empathy. It’s crucial to do the work to ground opinions on the facts, for uninformed opinions are worth less than nothing. 

But this is just the beginning. The meaning of those facts is something else entirely. And in a sacred text, meaning is never exhausted. May it deepen and grow with you on your journey. 

%
\chapter*{Verses of the Senior Nuns: a reflective life}
\addcontentsline{toc}{chapter}{Verses of the Senior Nuns: a reflective life}
\markboth{Verses of the Senior Nuns: a reflective life}{Verses of the Senior Nuns: a reflective life}

\scbyline{Bhikkhu Sujato, 2022}

The \textsanskrit{Therīgāthā} or “Verses of the Senior Nuns” is the ninth book in the Khuddhaka \textsanskrit{Nikāya} of the Pali Canon or \textsanskrit{Tipiṭaka}. It is a collection of 522 verses associated with seventy-three senior nuns, most of whom were alive in the Buddha’s time.

These verses celebrate the bliss of freedom and the life of meditation, full of proud and joyous proclamations of their spiritual attainments and their gratitude to other nuns as guides and teachers. The verses express the Dhamma through images that are immediate and personal. They speak of the fading of the hair’s lustre rather than of impermanence; of the trembling of failing limbs rather than of old age; of “searing and sizzling” greed and hate rather than abandoning them.

The \textsanskrit{Therīgāthā} is one of the oldest spiritual texts that record primarily women’s voices. It stems from the same general period as the Hebraic Books of Ruth and Esther, and like those books, it is a natural touchstone for those who wish to reflect on women’s roles in ancient religion.

It is a pair with the \textsanskrit{Theragāthā}, the “Verses of the Senior Monks”. Together these collections constitute one of the oldest and largest records of the voices of contemplatives.

The verses mostly stem from the time of the Buddha or a little later. Some have tried to argue that these collections were generally somewhat late. But it should be noted that in the Introduction to his 1971 translation, K.R. Norman, with his unparalleled historical and linguistic expertise, dismissed most of the arguments for lateness. He accepted that most of the nuns were alive during or soon after the Buddha’s time, and identified archaic \textsanskrit{Magadhī} features in some verses. He concluded that the text was probably composed over a three hundred year period from the late 6th century to the late 3rd century.

However, this appears to rely on the so-called “long chronology” of the Buddha’s life, which puts his death around 480 BCE. Under the “median chronology” which is accepted by many scholars currently, the Buddha’s death was closer to 400 BCE. Adjusting for this, and noting that Norman rejects the argument that any of the texts must be post-Ashokan, we should probably round the period of composition closer to two centuries, from the mid-5th century to the mid-3rd century. Most of the verses stem from the early period, with only a few, readily identifiable, texts being added in the later stages.

In my introduction to the \textsanskrit{Theragāthā}, I gave a general background. Most of those remarks apply equally here, so in this essay, I will focus on those things that are specific to the \textsanskrit{Therīgāthā} and refer you to the \textsanskrit{Theragāthā} for the basics.

\section*{The Complex Question of Authorship}

It’s unfortunate that, even within the limited scope of the \textsanskrit{Therīgāthā}, one of the shorter verse collections in the canon, many of the verses are not, in fact, by the nuns themselves. It’s difficult to count the number exactly, as attribution is not always clear, but roughly 100 of the 524 verses are not actually by the bhikkhunis. Rather, they were spoken to them by the Buddha or another interlocutor, or about them by a third party or narrator. In a few cases (noted below) the commentary says verses were added by the redactors at the Council.

In other cases, even where the bhikkhunis are speaking, the verses echo or paraphrase teachings from elsewhere in the canon. There is also some confusion about to whom certain verses belong. Some of the nuns share the same name; in other cases the name is unknown. Certain verses are sometimes said to be spoken by certain bhikkhunis—in the \textsanskrit{Therīgāthā}, the \textsanskrit{Bhikkhunī} \textsanskrit{Saṁyutta}, the \textsanskrit{Apadāna}, or the commentary—yet they do not appear under those names in the \textsanskrit{Therīgāthā} itself.

Some of the bhikkhunis appear both in the \textsanskrit{Therīgāthā} and the \textsanskrit{Bhikkhunī} \textsanskrit{Saṁyutta} (SN 5). The selected verses there are framed as a series of encounters between ten of the nuns and \textsanskrit{Māra} in the Dark Forest near \textsanskrit{Sāvatthī}. The verses are mostly similar to the corresponding portions of the \textsanskrit{Therīgāthā}. But some of them appear in a slightly different form, while in other cases, especially the “\textsanskrit{Cālā}” sisters (\textsanskrit{Cālā}, \textsanskrit{Upacālā}, and \textsanskrit{Sīsūpacālā}), the verses are assigned to different nuns.

There was no copyright in the Buddha’s day, and everyone, including the Buddha, freely repeated the sayings of others. The nuns were no different. What are we to draw from this? On the one hand, we would love to hear more about the lives and personal experiences of the nuns, making the few cases where they do speak of these things even more precious. On the other hand, it shows that for the nuns, what mattered was the Dhamma, not their own lives. If we over-personalize and over-dramatize their lives, making that the centrepiece, we are not listening to what they are trying to tell us.

In most cases, we know little about the nuns apart from the verses themselves. In some cases, the nuns are known from elsewhere in the Suttas or Vinaya, and in addition, some information, albeit legendary, is added by the commentary. All this must be handled with care.

\section*{The \textsanskrit{Therīgāthā} as a Women’s Text}

The \textsanskrit{Therīgāthā} is feminist in the sense of foregrounding women’s voices and experiences, and on occasion pointing to the specific ways that the suffering of women is due to gendered discrimination. The overall tenor of the \textsanskrit{Therīgāthā} is vibrant, proud, and celebratory. At the same time, though, these are the voices of women in a very different time and place whose words do not exist to serve our agendas. There’s nothing feminist about eliding, paraphrasing, or interpreting away the voices of women because they’re not saying what we would want them to say.

The \textsanskrit{Therīgāthā} remains our primary source of information about ancient nuns, along with the Vinaya, the code of monastic discipline. The \textsanskrit{Therīgāthā} presents women generally in a positive light and in their own voices, whereas the Vinaya is by its nature concerned with bad behaviour. In addition, the Vinaya has been passed down through the monks’ community and bears the signs of their editorial hand, but this is not the case in the \textsanskrit{Therīgāthā}.

Let’s look at an example of how the monks’ editorial hand reveals itself in the Vinaya. The Vinaya retains a terminology around women’s ordination that is quite distinct from that of the monks. Where the monks call their preceptor an \textit{\textsanskrit{upajjhāya}}, the nuns have \textit{\textsanskrit{pavattinī}}. Where the monks’ student is a \textit{\textsanskrit{saddhivihārī}}, for the nuns it is \textit{\textsanskrit{sahajīvinī}}. And while the monks call ordination \textit{\textsanskrit{upasampadā}}, the nuns call it \textit{\textsanskrit{vuṭṭhāpana}}.

How are these terms related? To understand this we need to know that the Vinaya texts are historically layered. The most clear-cut example of this is the distinction between the monastic rules (\textit{\textsanskrit{pātimokkha}}) and the analysis of those rules (\textit{\textsanskrit{vibhaṅga}}), which evolved after the rules were laid down. This was established by scholars in the 19th century, and the evidence that has come to light since then—such as comparative studies of different Vinayas—has confirmed the validity of the original insight. It is based on multiple independent grounds and is one of the firmest and most widely accepted consensus opinions in Buddhist studies.

Now, what we find when we look at these different layers is that in the portions that are earlier, such as the list of rules (\textit{\textsanskrit{pātimokkha}}), the nuns’ special terms are used. In the portions we know are later, such as the analysis of the rules (\textit{\textsanskrit{vibhaṅga}}), the nuns’ special terms are explained as being equivalent to the monks’ terms. In other words, the monk editors explained the unfamiliar \textsanskrit{bhikkhunī} vocabulary in terms that they understood. What they didn’t do, however, was go back and change the \textit{\textsanskrit{pātimokkha}} itself, even though it would have been a simple and perhaps justifiable standardization. In some schools, this may have happened, but in the Pali, it didn’t: the \textsanskrit{Theravāda} school was particularly scrupulous about such things.

It is the term for ordination itself that is the most significant here. The meaning of the word itself doesn’t matter. What matters is contextual usage. When the nuns’ word \textit{\textsanskrit{vuṭṭhāpana}} is used, only nuns are mentioned as performing ordination. When the monks’ word \textit{\textsanskrit{upasampadā}} is used, nuns’ ordination must be performed by both nuns and monks. In other words, a procedure which was originally done by nuns for themselves was usurped by the monks, who made themselves the gatekeepers for the ordination of nuns, and hence controlled who can be a nun and who cannot. Since ordination is the only way that a celibate community can “reproduce”, this is a vital issue of reproductive rights for the nuns’ community.

So when the monks made changes to the \textsanskrit{bhikkhunī} Vinaya texts, they left traces. We have reasonably firm grounds for identifying such changes, and when we do, they pertain to the later layers of the text. And we have no similar grounds for saying that the editorial hand of the monks is visible in the \textsanskrit{Therīgāthā}; for example, there is no mention of monks ordaining nuns. For this reason, if we want to understand the life of the bhikkhunis, the \textsanskrit{Therīgāthā} must be our primary source, not the Vinaya.

The \textsanskrit{Therīgāthā} offers a clear and inspiring call to the spiritual life, one that belongs firmly to the women. In modern times, it has become a key text in developing feminist perspectives on early Buddhism as history, and on modern Buddhism as potential. These voices have foregrounded the \textsanskrit{Therīgāthā} in new ways, opening a new chapter in Buddhism, one that better represents the full spectrum of Buddhist practitioners both in ancient times and the present. Yet academic feminist studies are undermined by a lack of familiarity with the source material, which leaves them riddled with factual errors and mistaken assumptions. On such shaky ground, they read theory into the text, all too often eliding the voices of the women instead of hearing them. These theory-laden readings become rapidly outdated as the preoccupations of gender studies shifts, with the only constant factor being that the lives, voices, and beliefs of the ancient \textsanskrit{bhikkhunīs} are subject to judgment and scrutiny by modern theorists, while modern theory is never subject to scrutiny in light of the words of the ancient \textsanskrit{bhikkhunīs}.

One systematic problem that dogs studies of gender in the \textsanskrit{Therīgāthā} is credulous reliance on the commentary by \textsanskrit{Dhammapāla}. The commentary stems from a millennium later, in a different country thousands of kilometres away. Yet it is too often treated as a reliable record of information about the nuns’ lives. It isn’t. Unless a story has independent corroboration in other sources—and few of them do—the stories depicted in the commentary should be regarded only as the stories told about the bhikkhunis in the \textsanskrit{Theravāda} community. They tell us not about the bhikkhunis, but about how the commentators, who of course were male, responded to the lives and teachings of the ancient nuns of legend.

The very first verse of the \textsanskrit{Therīgāthā} illustrates this well (Thig 1.1. A similar verse at Thig 1.16 is spoken to an elder nun.) The text attributes it only to a certain unnamed nun, identifying neither the speaker nor the nun spoken to.

\begin{quotation}%
Sleep softly, little nun, \\
wrapped in the cloth you sewed yourself; \\
for your desire has been quelled, \\
like vegetables boiled dry in a pot.

%
\end{quotation}

The words record a tender and personal moment between two women with a rare warmth and intimacy. The kitchen metaphor speaks to a shared experience, an assumed closeness. The speaker is a woman who is drawing from her life and who does not need a man’s authority to express words of gentle comfort. Her friend is lying down to sleep; perhaps the sleepy nun is unwell, or perhaps she is weary after a long journey or hard work. The verse has a tenderness that belies the confidence of what she is saying. She addresses the sleepy nun with the unique diminutive \textit{therike} (“little nun”), but she employs this familiar form to affirm her friend’s enlightenment. It is at once bold and quiet, understated and momentous.

We know so little about these women that even to know that her name was unknown is a significant detail. But the commentary, relentlessly backfilling the spaces in the text, says she was actually \emph{named} \textsanskrit{Therikā} even before ordaining, due to her sturdy body (the root can carry the sense of either seniority or solidity). The commentary is not consistent on this point, as it sometimes also refers to her as an unnamed nun. This indicates that there were multiple commentarial sources whose viewpoints are not fully resolved in \textsanskrit{Dhammapāla}’s edition.

The commentary goes on to identify her with the so-called “\textsanskrit{Maṇḍapadāyikā}” of the \textsanskrit{Therīpadāna} (Thi Ap 3). But the name \textsanskrit{Maṇḍapadāyikā} is obviously artificial: it just means “giver of a pavilion”. Late texts like the \textsanskrit{Apadānas} often invent names to frame a pious story of making merit. Since there never was anyone called \textsanskrit{Maṇḍapadāyikā}, the name is conveniently available for identification with our unknown nun of the \textsanskrit{Therīgāthā}. Once that is done, the commentary can trace her spiritual path to an act of merit in a far distant age of a past Buddha: a woman’s journey must begin with an act of service to a man.

The commentary then tells us that in this, her final life, she was married to a husband who would not agree to her desire to go forth, until a conflagration in the kitchen caused her to deepen her insight into Dhamma and reject sensual desires. After this, seeing that normal home life was now impossible, her husband allowed her to go forth. She cannot decide for herself but must rely on a man’s choice. Now, of course, there is a long history of women being subject to the choices of their husbands. But there is an equally long history of men compiling texts that frame women’s compliance as a sacred duty. The verse itself says nothing of a husband, so the commentary must reframe her story to fit the moralizing expectations of the male commentators.

Remember, this is the first text in the \textsanskrit{Therīgāthā}. The commentary is not just explaining this verse: it is setting expectations for the whole collection and by implication, the whole bhikkhuni order. The permission of the husband is one of the criteria for women’s ordination that was added to the Vinaya at some point, just as the requirement that ordination is certified by monks was added. The commentator is deliberately importing this despite its irrelevance to the text, making us read the \textsanskrit{Therīgāthā} through the lens of the Vinaya, reminding students that compliance with male authority is required before a woman may take ordination and seek freedom. It has to do this because nowhere in the \textsanskrit{Therīgāthā} is there anything about getting permission from a husband.

Indeed, husbands make an appearance in only a few poems: as a loved one tragically lost (Thig 10.1), as a lazy ingrate (Thig 15.1), or as an object of disgust (Thig 1.11, Thig 2.3). Sometimes a husband is not mentioned even when we might expect it, as in the verses of the nuns \textsanskrit{Saṅghā} (Thig 1.18), \textsanskrit{Sakulā} (Thig 5.7), and \textsanskrit{Guttā} (Thig 6.7), which speak of leaving behind all that they find dear—home, children, and wealth. Or else take the poem of \textsanskrit{Bhaddā} \textsanskrit{Kāpilānī}, where she begins by praising the spectacular attainments of her former husband, Kassapa, only to boldly claim to have realized the exact same attainments (Thig 4.1). She’s not speaking of her need to get his permission, but of the fact of her spiritual equality. In other poems, it is the husband who is set on his path by the wife (Thig 13.4).

Returning to the commentarial account of our sleepy nun, it says that after her ordination, she was brought to the Buddha, who spoke the verse. This is highly incongruous: why is the Buddha talking to her about sleeping? There’s nothing in the backstory to justify it. The verse sounds like the voice of a friend to a friend, not like the address of a teacher to a student. But for the commentary, the verse belongs to a man.

To sum up: the verse records the fond words of one woman to another. The commentary, ignoring this, claims that she started her path with an offering to a man, invents a husband whose permission she needed to go forth, and attributes her verse to a man.

This doesn’t mean that we have nothing to learn from the commentary. But it does mean that the voices of the bhikkhunis in the \textsanskrit{Therīgāthā} and the voices of the commentators are two quite different things. The commentary should be critically assessed as a male response to the \textsanskrit{Therīgāthā}, not as an essential framing for it.

When the \textsanskrit{bhikkhunī} \textsanskrit{Vimalā} recalled her former days as a sex worker, she positioned herself, not as the victim of a man, but as the agent of her life (Thig 5.2:3). She had a toxic relationship with other women, despising those less beautiful and famous. And she used her beauty to entice men, laughing at them as she manipulated them to get what she wanted.

\begin{quotation}%
\textit{\textsanskrit{akāsiṁ} vividhaṃ \textsanskrit{māyaṁ}} \\
I created an intricate illusion

%
\end{quotation}

It was through her work, her agency, that she did her job of enticing men. This is no mere sophistic detail, as it speaks to the heart of Buddhism, that we are agents who form our own world, and do not merely passively occupy it. She was the one who choose to create a world of illusion that ensnares, and she was the one who decided to use her wisdom to find the truth that frees.

In the case of the \textsanskrit{bhikkhunī} \textsanskrit{Khemā}, the sensual temptation by a “man” came after she was ordained. The young man—who turned out to be none other than \textsanskrit{Māra}—harassed her, as he did so many of the nuns, playing the nice guy who wants to take her to see a band (Thig 6.3). \textsanskrit{Khemā} objects, pointing out that her body is “rotting, ailing, and frail” and saying that she is “repelled” by it and has given up sensual desires. \textsanskrit{Māra} the “terminator” (\textit{\textsanskrit{antakāra}}) is summarily vanquished by \textsanskrit{Khemā}’s power. What \textsanskrit{Khemā} sees and \textsanskrit{Māra} does not is that, even while she is still young and beautiful, the body already has the nature of impermanence and decay. She’s not seeing it with the physical eye, but with the eye of insight, while \textsanskrit{Māra} is still trapped in the realm of the senses.

\textsanskrit{Māra} features as the fall guy in several other poems that serve to illustrate the fearlessness of the nuns. They always see through his disguise but rarely does he get taken down as hard as when he tried to gaslight \textsanskrit{Somā} with his sexist putdowns. He tells her that women are too weak to attain the state realized by the sages. Many men have tried this one since, but it doesn’t really work when you’re speaking to a woman who has already attained that goal herself.

\begin{quotation}%
What difference does womanhood make \\
when the mind is serene, \\
and knowledge is present \\
as you rightly discern the Dhamma.

%
\end{quotation}

The theme of leaving behind womanhood also features in the verses of \textsanskrit{Mahāpajāpati} \textsanskrit{Gotamī}, the Buddha’s aunt and stepmother (Thig 6.6). Later generations have seen her as either an icon of womanhood, the founding leader of the female \textsanskrit{Saṅgha}, or else as a morality fable for why women should not be ordained. In the Vinaya, she features at the very start of the bhikkhuni community. And she re-appears throughout the Vinaya as an active force of leadership, a crucial mediator between the nuns and the Buddha.

It is a curious thing, then, that not a single one of the bhikkhunis mentions her at all. They frequently speak of the women who have taught them the Dhamma with gratitude and love, yet \textsanskrit{Mahāpajāpati} somehow never comes up. I believe that this is because she was not, in fact, the founder of the bhikkhuni \textsanskrit{Saṅgha}. I think that she joined the \textsanskrit{Saṅgha} when she was already elderly; that she was conceited about her status as the Buddha’s mother; and that, as was the case for several of the Buddha’s relatives, special rules were laid down to ensure she fitted in properly.

And I think she was raised up as an icon following the Buddha’s death—specifically, around the time of the Second Council—as interest in the Buddha’s teachings waned and interest in his life grew. Her story became the lens through which the story of all the bhikkhunis was seen, as it still is today. Some monks at the time, seeking greater control over the bhikkhuni community, took the rules imposed on her for good reasons, extended them, and applied them to all bhikkhunis for no good reason. These rules dominate patriarchal discourse about bhikkhunis to this day, yet once again, no bhikkhuni in the \textsanskrit{Therīgāthā} sees fit to mention them. The bitter pill was wrapped in a human interest story of drama and pathos. And a spectacular story of \textsanskrit{Mahāpajāpati}’s death was invented for the \textsanskrit{Apadāna} in the hope that people would be distracted by shiny things.

The entire \textsanskrit{Therīgāthā}, including the verses of \textsanskrit{Mahāpajāpati} herself, stands completely outside this discourse. \textsanskrit{Mahāpajāpati} says nothing of her role in founding the bhikkhuni \textsanskrit{Saṅgha}, nor does she acknowledge any of her supposed bhikkhuni students. She doesn’t position herself as a female leader or role model. Instead, her own words send a rather different message.

\begin{quotation}%
Previously I was a mother, a son, \\
a father, a brother, and a grandmother. \\
Failing to grasp the true nature of things, \\
I transmigrated without reward.

Since I have seen the Blessed One, \\
this bag of bones is my last. \\
Transmigration through births is finished, \\
now there’ll be no more future lives.

%
\end{quotation}

She echoes the famous lines of her son immediately after his awakening when he recalled his long “journey without reward”. It is in not the state of womanhood or any other that freedom is to be found, but only when all such limitations have been left behind.

\section*{A Celebration of Freedom}

The \textsanskrit{Therīgāthā} is a proud celebration of free women, unembarrassed and unashamed. We have already discussed at some length the first verse of the collection. Here I’d like to highlight some further verses.

The second verse (Thig 1.2) shifts register but keeps the focus on freedom. Here the nun is being addressed and exhorted to find freedom. It’s a simple verse, which doesn’t aim to convey doctrine but to encourage. The rubric (a special tag in prose that follows the verses) identifies the speaker as the Buddha and the nun as a “trainee” (\textit{\textsanskrit{sikkhamānā}}). This was special ordination status established primarily for girls of eighteen, rather than the usual twenty years for \textsanskrit{bhikkhunī} ordination. Older women are sometimes said to have also undertaken this stage (Thig 5.8). This is to be expected. As Buddhist ordination procedures evolved, requirements introduced for a limited purpose rapidly became applied universally. And I think that is the case here. Certainly not all did, for \textsanskrit{Bhaddā} \textsanskrit{Kuṇḍalakesā} was called to full ordination directly by the Buddha (Thig 5.9).

Only this verse and Thig 2.1 explicitly say that the nun was not the speaker, and in both cases they were trainees. It suggests that the Buddha himself took the time to give heart to these women who were new on the path, to assure them without hesitation or qualification that they could attain the same freedom that he had found.

The nature of the speaker also affects the reading of the third verse (Thig 1.3). Here, a nun called \textsanskrit{Puṇṇā} is addressed with a similarly bold and encouraging call to destroy ignorance. The tag line that identifies the speaker, however, says that “\textsanskrit{Puṇṇā}” spoke these verses. The commentary, contradicting the rubric, says it was the Buddha speaking. The next series of verses, up to Thig 1.10, are also addressed to the nuns, and according to the commentary, the speaker in all cases was the Buddha. These verses all lack the intimate touch of the opening verse; they rely on standard imagery, and where they are personalized, they merely pun on the women’s names. This is just the kind of thing a teacher would do to personalize teaching if they knew little about them but their name. I do it all the time.

Thig 1.11 brings us the first poem in first person, and a return to the personal voice; a poem, it seems, by a nun for nuns. Rather than the tender comfort of the opening verse, however, here we have what seems to be a winking adaptation of a verse by the monk \textsanskrit{Sumaṅgala} at Thag 1.43. \textsanskrit{Sumaṅgala} celebrates his release from three crooked things—sickles, ploughs, and hoes. It’s pretty straightforward, which is why I think \textsanskrit{Muttā} adapted her verse from there, rather than the other way around. She similarly celebrates her release from three crooked things, one of them being her husband. But that’s only the start of the innuendo. The other “crooked” things are the mortar and pestle. On the surface, it’s an allusion to kitchen drudgery; but inescapably, it’s also about sex. It’s a mortar and pestle.

The line is constructed knowingly, with sly humour; the reader is led to expect a threefold listing of kitchen appliances, then along comes the husband, suddenly recontextualizing what came before. It’s the classic rule of three employed so often when telling jokes.

An odd problem with the line opens up a further layer of innuendo. When the monk describes three crooked things, the tools he mentions are, in fact, crooked. But a mortar and pestle are not crooked; the PTS edition of Rhys Davids’ translation even includes a photo of a distinctly uncrooked mortar and pestle (plate facing page 14). The commentary seems to be aware of this, and it allows that \textit{khujja} can mean something that \emph{is} crooked or something that \emph{makes you} crooked. (Commentary to Thag 1.43: \textit{\textsanskrit{khujjasabhāvehi} khujjakērehi \textsanskrit{vā}}; commentary to Thig 1.11: \textit{\textsanskrit{khujjakaraṇahetutāya} \textsanskrit{tadubhayaṁ} “khujja”nti \textsanskrit{vuttaṁ}}.) The commentary explains that the husband was a hunchback and hence \emph{is} crooked, whereas the kitchen tools \emph{make you} crooked due to long hours bent over them. If the dual sense proposed by the commentary is to be accepted—and I believe the context demands it—then it’s problematic to translate it as “three crooked things” per Norman and most other translators, for it leaves us with a line that doesn’t quite make sense.

Now, given that the three items in the line work as a whole, and that they aim to set up a punchline about the husband, it makes more sense to me if all three items are things that \emph{make you} crooked, rather than assuming that the third item, the husband, \emph{is} crooked. It rather sours the verse if she ends up just making a dig at a disabled husband. I think the point of the verse is more sly: the drudgery of the kitchen bends you over no less than the drudgery in the bedroom.

Most of the poems are much more straightforward. \textsanskrit{Jentā} announces that she has developed all the factors of awakening and will not be reborn (Thig 2.2). An \textsanskrit{Uttamā} makes the same claim, and in addition, claims to be the rightful daughter of the Buddha. \textsanskrit{Dhammā} (Thig 1.17), \textsanskrit{Cittā} (Thig 2.5), and \textsanskrit{Mettikā} (Thig 2.6) speak of the triumph of their insight despite the failing of their bodies. \textsanskrit{Selā} (Thig 3.7) is just one of many nuns who proclaims her triumph over delusion.

Some nuns found peace only after travelling through the depths of despair (Thig 2.10, Thig 3.1, Thig 3.1, Thig 5.1, Thig 5.3, Thig 6.8). \textsanskrit{Ubbirī} is distraught in lamenting her daughter (Thig 3.5), \textsanskrit{Paṭācārā} and \textsanskrit{Vāseṭṭhī} in lamenting a son (Thig 6.1, Thig 6.2), while \textsanskrit{Sundarī} has overcome the grief even at losing many children (Thig 13.4). \textsanskrit{Candā} was a homeless widow who endured seven years of hardship on the streets before meeting a nun to inspire her (Thig 5.12).

This should not be overinterpreted. The women speak of these things to show how they triumphed over them, not to romanticize them or encourage others to follow such a path. Most of the nuns do not have such dramatic stories to tell. For \textsanskrit{Dantikā}, insight came when seeing an elephant at a ford (Thig 3.4); for \textsanskrit{Paṭācārā}, when she saw a lamp going out (Thig 5.10). The experience of suffering as a woman is something that the early Buddhism texts acknowledge with compassion since it is the reality of those women’s lives. It acted as a spur to practice, or simply to contrast the freedom that they now experience. It doesn’t mean that anyone, especially a woman, \emph{has} to undergo such extremes of suffering.

The long poem of \textsanskrit{Subhā} does not speak of any existential despair. Rather, as a young woman she immediately understood the Dhamma as soon as she heard it (Thig 13.5). Sometimes all it takes is a single teaching. Coming from a wealthy family, she speaks eloquently of the trap of money and how it breeds conflict and corruption. She declares that she will cross over on the same path travelled by the great sages. The poem finishes with verses in her praise. The final verse, which according to the commentary was added by redactors at the Council, attributes the verses of praise to no less a figure than Sakka, the lord of gods.

Not all the verses focus on the personal journey of the women. The verses of \textsanskrit{Sukkā}, for example, record an unnamed third party complaining that too many of the folk of \textsanskrit{Rājagaha} are as if drunk on mead, not paying attention properly to her teaching, which is like nectar of cool water (Thig 3.6). The verses of \textsanskrit{Bhaddā} \textsanskrit{Kuṇḍalakesā} include praise for a layperson who made much merit by offering a robe to one as holy as her (Thig 5.9).

\textsanskrit{Puṇṇikā}’s poem is not about her path, but about how she gave a lesson to a deluded brahmin (Thig 12.1). The framing of the verses is not quite clear. According to the commentary, the opening verses—where \textsanskrit{Puṇṇikā} tells the brahmin how as a water carrier she feared her mistresses, and proceeds to ask him what he’s afraid of—were spoken by \textsanskrit{Puṇṇikā} before she was ordained as a nun. But the only clear indication of tense is the use of the aorist in the past tense, which suggests, rather, that she is already a nun and is telling an anecdote of her past. This would explain why she speaks with such boldness, and also why the brahmin addresses her with the respectful \textit{bhoti}. Even though the very basis of his religious beliefs is being challenged, the brahmin listens and responds well. \textsanskrit{Puṇṇikā}’s verses include the classic rebuttal to the efficacy of bathing for purity:

\begin{quotation}%
Would not they all go to heaven, then: \\
all the frogs and the turtles, \\
gharials, crocodiles, \\
and other water-dwellers too?

%
\end{quotation}

Closely related to this is the poem of \textsanskrit{Rohinī} (Thig 13.2). In this case, the recalcitrant brahmin is her father, who objects to his daughter’s devotion to the “ascetics”, arguing that they are lazy good-for-nothings. In this case, \textsanskrit{Rohinī} must not yet be ordained, because her father observes that she says “ascetics” even when falling asleep and waking up. The assumption is that she would later go on to ordain; and while the text itself does not confirm this, her father hints at it when he says she “will become an ascetic”. She responds, not as an obsequious and obedient daughter, but by extolling the virtues of ascetics at length.

The verses of \textsanskrit{Cāpā} give a unique twist to this scenario (Thig 13.3). Here we are thrown in the middle of an argument between a man and his wife, who are torn between their desire for each other and their aspirations for a spiritual life. The fight gets so vicious they even threaten their child. Yet ultimately \textsanskrit{Kāḷa}, the husband is set on his path and \textsanskrit{Cāpā} conveys her blessings. The poem doesn’t say that she ordained.

In yet another case of a man being redeemed through his encounter with a woman, the brahmin \textsanskrit{Sujāta} marvels at how \textsanskrit{Sundarī} can respond with such equanimity even when she has lost seven children. She attributes her calm to the teaching of the Buddha, upon which the brahmin went forth, and she later followed suit (Thig 13.4).

Sometimes the path of the nuns has not been from domestic or emotional travail, but from one religious practice to another. Such was the case of \textsanskrit{Nanduttarā}, who recalls both her devotion to meaningless worship and mortification, as well as her infatuation with her appearance: the two extremes (Thig 5.5; see also \textsanskrit{Mittā} at Thig 2.7). \textsanskrit{Khemā} also reports a fruitless former practice of worshipping stars and serving the sacred flame (Thig 6.3). For \textsanskrit{Mittākāḷī}, genuine insight only came long after ordaining for the wrong reasons (Thig 5.6).

In addition to the notable scarcity of husbands, there are few references to monks. When acknowledging teachers, the nuns mention either the Buddha or another nun: \textsanskrit{Paṭācārā} (Thig 5.11, Thig 5.12, Thig 7.1), \textsanskrit{Uppalavaṇṇā} (Thig 13.5), \textsanskrit{Jinadattā} (Thig 15.1), or else an unnamed nun (Thig 3.2, Thig 5.1, Thig 5.8, Thig 6.8, Thig 13.4). Typically these nuns are said to have conveyed the central teachings of Buddhism such as the four noble truths, the aggregates, elements, and so on. These are the central topics of the \textsanskrit{Saṁyutta} \textsanskrit{Nikāya}, and we can therefore conclude that this, or its ancestor, was carefully studied by the nuns.

Only \textsanskrit{Sakulā} reports learning the Dhamma from a monk, and that was when she was still a laywoman (Thig 5.7). This is especially noteworthy given that, according to the Vinaya, the monks were supposed to be teaching the nuns every fortnight. Yet somehow these regular sessions are never mentioned by the nuns, just as the procedure of ordination by monks is never mentioned.

\section*{The Dramatic Verses}

Since the poems of the \textsanskrit{Therīgāthā} are arranged from short to long, and since there is a general tendency for texts in Buddhism to grow over time, it’s fair to assume that the final poems of the collection are somewhat later than most. This applies especially to the final three poems, each of which develops a complex dramatic scenario. These dramatic elements appear in earlier poems also, but not to the same extent.

These literary compositions distinguish the \textsanskrit{Therīgāthā} from the \textsanskrit{Theragāthā}. There, most of the long poems are produced by simply compiling several shorter passages of verse, which often have no relation to each other. The \textsanskrit{Therīgāthā} has only one, rather modest, poem in this style, that of \textsanskrit{Uppalavaṇṇā} (Thig 11.1).

Of course, in describing these verses as “dramatic” I am not suggesting that they were used for actual stage performances. There’s no evidence for theatrical presentations in the \textsanskrit{Saṅgha} at such an early date. Nor is it historically meaningful to analyze how such texts were shaped by “Indian aesthetic theories” since there is no evidence of any such thing until at least half a millennium later, and even then, no evidence that the theories ever influenced Buddhist literature. Nonetheless, anyone who has ever given a Dhamma talk has, in some sense, made a dramatic presentation of the Dhamma, and knows how important it is to hold an audience’s attention through the narrative fundamentals of emotion, engagement, conflict, and humour.

The most dramatic confrontation of all is that between a rogue and the young nun \textsanskrit{Subhā} (Thig 14.1; this is not the same \textsanskrit{Subhā} we have met before at Thig 13.5). The lateness of the poem is suggested by the setting verse, an unusual feature for poetry, which was ascribed by the commentary to the redactors. The rogue blocks her path, and despite her strong objections tries to seduce her. But she is having none of it, even with his lengthy and admittedly eloquent evocation of the sensual joys they will find together. He lavishes special praise on the beauty of her eyes, probably thinking he is being romantic. But he is really not prepared for what she does next.

The long poem by \textsanskrit{Isidāsi} sets the scene with a private conversation between two nuns. The conversation is rather unusually given a location, which is in \textsanskrit{Pāṭaliputta}. This immediately sets the dialogue at a date considerably after the death of the Buddha, as in his time \textsanskrit{Pāṭaliputta} was just a small village. In confirmation, the commentary states that the narrative verses were added by the redactors at a Council. It doesn’t say which one, but it must be the Second or Third. As Norman points out, this only tells us when the narrative frame was added to the poem, which must have been composed earlier.

The poem builds an extended narrative, which is unusual in the early texts, discussing the specific results of past kamma over many lifetimes, which again is something we don’t see often. It mentions three nuns—\textsanskrit{Isidāsī}, \textsanskrit{Bodhī}, and \textsanskrit{Jinadattā}—none of whom are met elsewhere in the canon. All this means that while we can safely say that this poem is later than most in the collection, we cannot fix the date with any confidence.

The poem has \textsanskrit{Isidāsī} pleading her case to go forth with her father. She uses a rather specific phrasing (Thig 15.1:32.4):

\begin{quotation}%
\textit{\textsanskrit{Kammaṁ} \textsanskrit{taṁ} \textsanskrit{nijjaressāmi}} \\
I shall wear that bad deed away.

%
\end{quotation}

Rhys Davids says her aspirations are “Jainistic” (\textit{Psalms of the Sisters}, xxii) and Norman concurs (\textit{Elders’ Verses II}, 176). And it is indeed true that the wearing away of past kamma is regarded as a core teaching of the Jains. But what makes a practice truly Jain is that this wearing away is done by self-mortification, of which there is no hint here. In fact, in a dialogue with a Jain, the Buddha, always apt to adapt and respond to the language used by other religions, reframed the idea of “wearing away kamma”. Instead of wearing away by self-mortification, it can be done by letting go the defilements that underlie the creation of kamma (AN 4.195:6.3). Such passages don’t show any hidden influence of Jainism. Rather, they show how the Buddhists were very conscious and deliberate in how they responded to the language and ideas of others.

Nonetheless, it remains the case that such language is more characteristic of Jainism. It’s an unusual choice of words. Since it was put in the mouth of the young \textsanskrit{Isidāsī} before she went forth, maybe she was just using words she had picked up about kamma without a clear understanding of the differences between the schools. After all, modern Buddhists do this all the time. Perhaps; but it seems like an unnecessarily complex linguistic conceit.

\textsanskrit{Isidāsī}’s long take of woe recounts how before ordaining as a nun, she had been married, but despite being the perfect wife, her husband just couldn’t stand her. She was kicked out and handed from husband to husband, each one a less appealing catch than the last. Finally, they were reduced to tempting a homeless ascetic into discarding his vows for her, but even after that he still couldn’t stand to be with her. She swore to her father that she had done nothing to deserve such treatment. It’s as if there is something wrong with her inside, something that she cannot see, and that no amount of effort on her part can overcome. But then the nun \textsanskrit{Jinadattā} came to her house for alms. She was so inspired she took ordination herself.

She became enlightened and could recollect the seven lives that had led her to this point, including the one that started it all. Long ago, she had been born as a man and had sex with another man’s wife. This was the root bad kamma that drove her to a series of distressing rebirths. Repeatedly she was born as a male animal who suffered castration, then as a slave who had neither male nor female genitals. Eventually, she was born as a girl subject to violence and abandonment.

The depiction of kamma and its effects here is subtly different from the normal presentation in the early texts. Normally the idea is that if you do a bad deed, you will experience bad results because of that. For example, you might be born in a lower realm, or if you are born in the human realm—which is \emph{always} the result of good kamma—you might still have the bad kamma to be born suffering a chronic illness.

There are two fallacies to be wary of here.

First, the fact that kamma creates some results does not mean that all results were created by kamma. In other words, if A then B does not imply if B then A. The Buddha listed multiple causes for illness, for example, only one of which was kamma. Unless we are like \textsanskrit{Isidāsī} and have the psychic ability to recollect past lives, we do not know. What we do know, however, is that transmigration is long. And in that long journey, all of us have done many good things and many bad things.

This is important in the current case because it is one of the few examples in the early texts that might be used to argue that being born as a woman is a result of bad kamma, a belief that is commonly held in Buddhist cultures today. This might also be held to apply to intersex people, since in one birth she is biologically neither male nor female. But in such delicate cases, it is crucial to not overinterpret the text. Looking at the lives described in the text, in each case it is not the \emph{mere fact} of biological sex that is painful. She was reborn in a life of suffering, and in her case, sex characteristics were part of that.

The very next poem, discussed below, appears as a counterpoint, perhaps deliberately, to this fallacy. \textsanskrit{Sumedhā} is repeatedly reborn in a happy life as a woman because of her good kamma as a woman. The entire framing of \textsanskrit{Isidāsī}’s text shows how as a woman she triumphed over her circumstances and found freedom from all this. She must have performed good kamma in the past to be born as a human with the capacity to understand and practice the Dhamma. And so have we. The real question facing us is, what are we choosing to do about it?

The second fallacy is to think that kamma determines the choices of others. No: kamma determines what you experience, not what others do. Yet in \textsanskrit{Isidāsī}’s telling, her bad deed in the past determines how others treat her in her many past lives. When she was born as a monkey, she did not have any unusual sex characteristics. It was the monkey chief who castrated her at seven days of age. How is that \emph{her} kamma forced the monkey chief to do that? Is he not responsible for his own deeds? The same pattern plays out in life after life. She is badly mistreated, mostly at the hands of males. Yet in each case, their kamma is \emph{their} kamma and is not forced upon them by \emph{her} misdeeds.

This misunderstanding of kamma is very prevalent in the Buddhist community today. We hear, for example, that \textsanskrit{Moggallāna} died being set upon by bandits due to his past kamma which he could not escape. But this story from the commentaries never addresses the problem: how did \textsanskrit{Moggallāna}’s misdeeds cause others to commit murder?

So we can add doctrinal evolution to the list of reasons for concluding that this text is late. Of course, this does not mean it is worthless. It means it is a record of a teaching by women from the period \emph{after} the Buddha, which is even rarer than teachings from the Buddha’s life. The question of authorship is a complex one: is the teaching by \textsanskrit{Isidāsī}? Or is it related by her friend \textsanskrit{Bodhī} with whom she shared her story? Who was it that cast the story in verse? My intuition is that the text was composed within the women’s community to reconcile women’s circumstances and struggles with the more deterministic understanding of kamma that was already evolving a century or two after the Buddha.

In this light, the casual reference to the “wearing away” of kamma, while not formally contradicting Buddhist doctrine, takes on a new light. A deterministic reading of kamma is not present in the early texts, yet it became common in schools such as \textsanskrit{Theravāda}. Why? Was it purely a result of internal doctrinal developments? Or was it influenced by encounters with followers of other religions, such as the Jains? The distinctions made by the Buddha in the early texts are often subtle and debated even among scholars, not to speak of regular Buddhists. By itself, this one passage cannot be decisive, but it does belong in a broader discussion of such issues.

The final poem, attributed to \textsanskrit{Sumedhā}, is also late, but as with the poem of \textsanskrit{Isidāsī}, it is not possible to date with any precision. The poem quotes liberally from the prose Suttas, including not just general doctrines such as we find commonly in the \textsanskrit{Therīgāthā}, but multiple detailed specific references to particular passages. She urges her folks to “remember” these, implying that they were commonly known teachings. But this doesn’t tell us much, as such teachings may have been well known even in the Buddha’s life; the Suttas are full of such cross-references.

The nations and the kings don’t help with dating, as they seem to be otherwise unattested. Certainly, they are not part of the normal roster of places and people familiar from the early texts. \textsanskrit{Sumedhā} is the daughter of King \textsanskrit{Koñca} of the city of \textsanskrit{Mantāvatī}, about which I can find no information. Her betrothed is King \textsanskrit{Anīkaratta} of \textsanskrit{Vāraṇavatī}. A \textsanskrit{Varaṇāvatī} is mentioned in the Atharvaveda, but it is unclear what it is; perhaps a river. The \textsanskrit{Mahābhārata} mentions a \textsanskrit{Vāraṇāvata}, but this does not help us much. If it is the same city, we only know that, according to the Monier-Williams dictionary, it is about an eight-day journey from \textsanskrit{Hastināpura}. \textsanskrit{Hastināpura} is located on the Ganges in modern Uttar Pradesh, about 100km northeast of Delhi. Eight days journey is around 300km, so even if this identification were correct, it would only tell us that we are in northwest India, not far from the scope of the early Buddhist region. So the most we can say is that this story may have been set in a region into which Buddhism had expanded a century or two after the Buddha’s death. Such tales often serve as the “conversion story” for a country.

The story is a variation on the universal folk tale of the young woman betrothed against her will. It is constructed with melodramatic flair: the hapless parents, the brilliant and wilful daughter, the handsome king, and the mysterious kingdom. She is probably a teenager at this point, and while the sympathies of the story lie with her, I can’t help feeling sorry for the poor parents, subject to a relentless haranguing by a girl convinced that she knows it all.

The climax of the story builds tension by splitting into two narrative frames: the relentless approach of the royal suitor, and \textsanskrit{Sumedhā}’s equally relentless ascent to \textsanskrit{jhāna} and insight. It’s a brilliant narrative device. The king begs for her hand through the door, but she just delivers another scathing takedown of the futility of the world’s delights. She opens the door, only to see the lot of them sitting on the floor and weeping in despair. But she’s still not done. She tells them that this is nothing; they’ve cried much more than that in the long journey of rebirth. She gathers up the hardest of the hard core teachings from the Suttas and launches them in salvo after salvo at her dear and beloved as they huddle on the floor in tears.

It worked: the handsome king got up and begged the parents on \textsanskrit{Sumedhā}’s behalf to let her go forth. She did so and rapidly attained \textsanskrit{Nibbāna}.

Years later, on her deathbed, she revealed her past lives. She and \textsanskrit{Isidāsī} are the only two to speak of details of their past lives in the \textsanskrit{Therīgāthā}, although many nuns say they can recollect them. Unlike \textsanskrit{Isidāsī}, here she is not speaking of any bad kamma. On the contrary, she tells of how in the far distant age of the Buddha \textsanskrit{Koṇāgamana} she made merit together with two female friends. They offered no less than a new monastery, regarded as the greatest of material offerings. As women, they had access to considerable wealth, which they used to benefit others. As a result, they all experienced many good rebirths before realizing enlightenment in this final life. This kind of narrative appears very rarely in early texts but became the standard template for the \textsanskrit{Apadānas}, so it is yet another sign that this is a late poem.

In the story of \textsanskrit{Isidāsī}, she committed bad kamma as a man, and consequently experienced suffering in many lives as a woman subject to the brutality of men. Here, \textsanskrit{Sumedhā} does good kamma as a woman, together with her female friends, and consequently experiences happiness in many good lives as a woman. There is no single, simple narrative around kamma, sex, and gender, and early Buddhist texts do not try to construct one. The only narrative they are concerned with is that doing good leads to good results. In this way, the \textsanskrit{Therīgāthā} finishes on a high note, a lavish and exultant celebration of the determination, intelligence, and spiritual capacity of an extraordinary human being who happens to have been a woman.

\section*{A Brief Textual History}

The \textsanskrit{Therīgāthā} was published in 1883 by the Pali Text Society, as edited by Richard Pischel, one of the greats of 19th century German Indology. He made use of four manuscripts, two in Burmese script and two in Sinhala, as well as a manuscript of the commentary, which embeds the full text within it.

He notes that all these sources share serious blunders and that they must all stem from a single source. This is not unexpected, as we know that the Sri Lankan texts were re-introduced back from Burma. He found that one of the manuscripts, part of the Phayre collection in London, is in all respects superior to the rest. The textual corruption goes back a long way, as he notes there are several places where even the commentator of 500 CE had before him a corrupt text. As a result, his footnotes contain a profusion of variant readings, even though he Germanically avers that he only included those that seemed “really important”. So difficult was the task that he said, “without the commentary, I should hardly have ventured to publish this text at all.”

The second edition of this text was published in 1966 (reprinted 1990) with two new Appendices: additional variant readings supplied by K.R. Norman, and metrical analysis by L. Alsdorf.

The \textsanskrit{Therīgāthā} has since attracted a substantial body of translation and study. The first English translation was by C.A.F. Rhys Davids as \textit{Psalms of the Early Buddhists, I: Psalms of the Sisters} in 1909 with the Pali Text Society. Her translation was enthusiastic and informed by a serious study of Pali. Much of her commentary, however, is inevitably dated, and her flowery translation style was deliberately archaic even when it was made. And her decision to embed the translations within what she described as the “legends of legends” of the commentarial stories was, I think, unfortunate. Better to let the verses stand for themselves, and keep the commentary for a separate work.

In 1971 K.R. Norman published his translation \textit{Elders’ Verses II: \textsanskrit{Therīgāthā}} with the PTS. This was part of his extraordinary series of editions of Pali verses. It included a detailed analysis of metrical, textual, and linguistic issues to accompany his admittedly dry translation style. His analysis brings together a huge amount of relevant data which is an invaluable point of reference. His analysis of composition, dating, attribution, and the role of the commentary are essential backgrounds for any serious study.

In his preface he acknowledges the advances in understanding the \textsanskrit{Therīgāthā} since Pischel’s text, especially noting the improvements from “oriental editions” which appeared in the last “sixty years” (i.e. between 1910 and 1970). Most of these are the various editions that emerged from the Sixth Council, in Burmese (3rd edition, 1961), Cambodian (1958), and \textsanskrit{Devanāgarī} (1959), as well as the Thai edition of 1926–8 and the Hemaviratne edition of the commentary in Sinhalese characters (1918). While acknowledging the “many excellences” of the Burmese Sixth Council edition, he cautions that it “gives the impression of having been subjected to a considerable amount of normalization, which naturally greatly reduces its value”. My translation is based on the  \textsanskrit{Mahāsaṅgīti} edition, which is a digital version of this text.

The balance between readability and accuracy that is sorely lacking in the PTS editions was finally realized in 2015, with Charles Hallisey’s \textit{\textsanskrit{Therīgāthā}: Poems of the First Buddhist Women}, published by the Harvard University Press as part of the Murty Classical Library of India series. This is an elegant and mature translation, which includes a useful introduction as well as the Pali text.

A further full translation by \textsanskrit{Anāgārika} Mahendra under the title \textit{\textsanskrit{Therīgāthāpāḷi}: Book of Verses of Elder Bhikkhunis} was published in 2017 through Dhamma Publishers. This edition includes Pali text with translation and notes.

Readers should beware of a literary fraud masquerading as a translation. \textit{The First Free Women: Poems of the Early Buddhist Nuns} by Matty Weingast was published by Shambhala Publications in 2020. It is a work of original poetry, apparently made in the belief that channelling inspiration from the ancient bhikkhunis was a valid translation approach. The resulting work was ecstatically received by many Buddhist teachers who praised the uplifting of an ancient scripture of distinctively female voices, unaware or uncaring that it was composed in 2019 by a man in California. The publisher revised their fraudulent marketing somewhat after public pressure, but as of May 2022, their webpage for the book is still full of claims that it is a translation.

There have also been some partial translations. The best known is Susan Murcott’s \textit{First Buddhist Women: Poems and Stories of Awakening}, published through Parallax Press in 1991. She translated most of the poems and presented them together with her own reflections as well as paraphrases of the Pali commentary. As a feminist reading, it highlights the problematic status of women in ancient India as reflected in the \textsanskrit{Therīgāthā}. But its analysis is inevitably blunted by its closeness to the commentary, which frames the lens through which the texts are seen. For example, the first verse draws from the commentary for its title “An Unknown Wife”, whereas the verse itself says nothing about her being married.

Thirty-two of the poems were translated by \textsanskrit{Ṭhānissaro} Bhikkhu in his \textit{Poems of the Elders, An Anthology from the Theragatha and Therigatha} of 2005.

A 2009 translation of fourteen poems by Francis Booth \textit{Songs of the Elder Sisters} offers some of the most delightful passages in a refreshingly simple and unburdened form.

%
\chapter*{Acknowledgements}
\addcontentsline{toc}{chapter}{Acknowledgements}
\markboth{Acknowledgements}{Acknowledgements}

I remember with gratitude all those from whom I have learned the Dhamma, especially Ajahn Brahm and Bhikkhu Bodhi, the two monks who more than anyone else showed me the depth, meaning, and practical value of the Suttas.

Special thanks to Dustin and Keiko Cheah and family, who sponsored my stay in Qi Mei while I made this translation.

Thanks also for Blake Walshe, who provided essential software support for my translation work.

Throughout the process of translation, I have frequently sought feedback and suggestions from the community on the SuttaCentral community on our forum, “Discuss and Discover”. I want to thank all those who have made suggestions and contributed to my understanding, as well as to the moderators who have made the forum possible. A special thanks is due to \textsanskrit{Sabbamittā}, a true friend of all, who has tirelessly and precisely checked my work.

Finally my everlasting thanks to all those people, far too many to mention, who have supported SuttaCentral, and those who have supported my life as a monastic. None of this would be possible without you.

%
\mainmatter%
\pagestyle{fancy}%
\addtocontents{toc}{\let\protect\contentsline\protect\nopagecontentsline}
\part*{Verses of the Senior Nuns}
\addcontentsline{toc}{part}{Verses of the Senior Nuns}
\markboth{}{}
\addtocontents{toc}{\let\protect\contentsline\protect\oldcontentsline}

%
\addtocontents{toc}{\let\protect\contentsline\protect\nopagecontentsline}
\chapter*{The Book of the Ones }
\addcontentsline{toc}{chapter}{\tocchapterline{The Book of the Ones }}
\addtocontents{toc}{\let\protect\contentsline\protect\oldcontentsline}

%
\section*{{\suttatitleacronym Thig 1.1}{\suttatitletranslation An Unnamed Nun (1st) }{\suttatitleroot Aññatarātherīgāthā}}
\addcontentsline{toc}{section}{\tocacronym{Thig 1.1} \toctranslation{An Unnamed Nun (1st) } \tocroot{Aññatarātherīgāthā}}
\markboth{An Unnamed Nun (1st) }{Aññatarātherīgāthā}
\extramarks{Thig 1.1}{Thig 1.1}

\scnamo{Homage to that Blessed One, the perfected one, the fully awakened Buddha! }

\begin{verse}%
Sleep\marginnote{1.1} softly, little nun, \\
wrapped in the cloth you sewed yourself; \\
for your desire has been quelled, \\
like vegetables boiled dry in a pot. 

%
\end{verse}

\scendsutta{That is how this verse was recited by a certain unnamed nun. }

%
\section*{{\suttatitleacronym Thig 1.2}{\suttatitletranslation Muttā (1st) }{\suttatitleroot Muttātherīgāthā}}
\addcontentsline{toc}{section}{\tocacronym{Thig 1.2} \toctranslation{Muttā (1st) } \tocroot{Muttātherīgāthā}}
\markboth{Muttā (1st) }{Muttātherīgāthā}
\extramarks{Thig 1.2}{Thig 1.2}

\begin{verse}%
\textsanskrit{Muttā},\marginnote{1.1} be released from your bonds, \\
like the moon released from the eclipse. \\
When your mind is released, \\
enjoy your alms free of debt. 

%
\end{verse}

\scendsutta{That is how the Buddha regularly advised the trainee nun \textsanskrit{Muttā} with these verses. }

%
\section*{{\suttatitleacronym Thig 1.3}{\suttatitletranslation Puṇṇā }{\suttatitleroot Puṇṇātherīgāthā}}
\addcontentsline{toc}{section}{\tocacronym{Thig 1.3} \toctranslation{Puṇṇā } \tocroot{Puṇṇātherīgāthā}}
\markboth{Puṇṇā }{Puṇṇātherīgāthā}
\extramarks{Thig 1.3}{Thig 1.3}

\begin{verse}%
\textsanskrit{Puṇṇā},\marginnote{1.1} be filled with good qualities, \\
like the moon on the fifteenth day. \\
When your wisdom is full, \\
shatter the mass of darkness. 

%
\end{verse}

\scendsutta{That is how this verse was recited by the senior nun \textsanskrit{Puṇṇā}. }

%
\section*{{\suttatitleacronym Thig 1.4}{\suttatitletranslation Tissā }{\suttatitleroot Tissātherīgāthā}}
\addcontentsline{toc}{section}{\tocacronym{Thig 1.4} \toctranslation{Tissā } \tocroot{Tissātherīgāthā}}
\markboth{Tissā }{Tissātherīgāthā}
\extramarks{Thig 1.4}{Thig 1.4}

\begin{verse}%
\textsanskrit{Tissā},\marginnote{1.1} train in the trainings—\\
don’t let the practice pass you by. \\
Detached from all attachments, \\
live in the world free of defilements. 

%
\end{verse}

%
\section*{{\suttatitleacronym Thig 1.5}{\suttatitletranslation Another Tissā }{\suttatitleroot Aññatarātissātherīgāthā}}
\addcontentsline{toc}{section}{\tocacronym{Thig 1.5} \toctranslation{Another Tissā } \tocroot{Aññatarātissātherīgāthā}}
\markboth{Another Tissā }{Aññatarātissātherīgāthā}
\extramarks{Thig 1.5}{Thig 1.5}

\begin{verse}%
\textsanskrit{Tissā},\marginnote{1.1} apply yourself to good qualities—\\
don’t let the moment pass you by. \\
For if you miss your moment, \\
you’ll grieve when sent to hell. 

%
\end{verse}

%
\section*{{\suttatitleacronym Thig 1.6}{\suttatitletranslation Dhīrā }{\suttatitleroot Dhīrātherīgāthā}}
\addcontentsline{toc}{section}{\tocacronym{Thig 1.6} \toctranslation{Dhīrā } \tocroot{Dhīrātherīgāthā}}
\markboth{Dhīrā }{Dhīrātherīgāthā}
\extramarks{Thig 1.6}{Thig 1.6}

\begin{verse}%
\textsanskrit{Dhīrā},\marginnote{1.1} touch cessation, \\
the blissful stilling of perception. \\
Win extinguishment, \\
the supreme sanctuary. 

%
\end{verse}

%
\section*{{\suttatitleacronym Thig 1.7}{\suttatitletranslation Vīrā }{\suttatitleroot Vīrātherīgāthā}}
\addcontentsline{toc}{section}{\tocacronym{Thig 1.7} \toctranslation{Vīrā } \tocroot{Vīrātherīgāthā}}
\markboth{Vīrā }{Vīrātherīgāthā}
\extramarks{Thig 1.7}{Thig 1.7}

\begin{verse}%
She’s\marginnote{1.1} known as \textsanskrit{Vīrā} because of her heroic qualities, \\
a nun with faculties developed. \\
She bears her final body, \\
having vanquished \textsanskrit{Māra} and his mount. 

%
\end{verse}

%
\section*{{\suttatitleacronym Thig 1.8}{\suttatitletranslation Mittā (1st) }{\suttatitleroot Mittātherīgāthā}}
\addcontentsline{toc}{section}{\tocacronym{Thig 1.8} \toctranslation{Mittā (1st) } \tocroot{Mittātherīgāthā}}
\markboth{Mittā (1st) }{Mittātherīgāthā}
\extramarks{Thig 1.8}{Thig 1.8}

\begin{verse}%
Having\marginnote{1.1} gone forth out of faith, \\
appreciate your spiritual friends, \textsanskrit{Mittā}. \\
Develop skillful qualities \\
for the sake of finding sanctuary. 

%
\end{verse}

%
\section*{{\suttatitleacronym Thig 1.9}{\suttatitletranslation Bhadrā }{\suttatitleroot Bhadrātherīgāthā}}
\addcontentsline{toc}{section}{\tocacronym{Thig 1.9} \toctranslation{Bhadrā } \tocroot{Bhadrātherīgāthā}}
\markboth{Bhadrā }{Bhadrātherīgāthā}
\extramarks{Thig 1.9}{Thig 1.9}

\begin{verse}%
Having\marginnote{1.1} gone forth out of faith, \\
appreciate your blessings, \textsanskrit{Bhadrā}. \\
Develop skillful qualities \\
for the sake of the supreme sanctuary. 

%
\end{verse}

%
\section*{{\suttatitleacronym Thig 1.10}{\suttatitletranslation Upasamā }{\suttatitleroot Upasamātherīgāthā}}
\addcontentsline{toc}{section}{\tocacronym{Thig 1.10} \toctranslation{Upasamā } \tocroot{Upasamātherīgāthā}}
\markboth{Upasamā }{Upasamātherīgāthā}
\extramarks{Thig 1.10}{Thig 1.10}

\begin{verse}%
\textsanskrit{Upasamā},\marginnote{1.1} cross the flood, \\
Death’s domain so hard to pass. \\
When you have vanquished \textsanskrit{Māra} and his mount, \\
bear your final body. 

%
\end{verse}

%
\section*{{\suttatitleacronym Thig 1.11}{\suttatitletranslation Muttā (2nd) }{\suttatitleroot Muttātherīgāthā}}
\addcontentsline{toc}{section}{\tocacronym{Thig 1.11} \toctranslation{Muttā (2nd) } \tocroot{Muttātherīgāthā}}
\markboth{Muttā (2nd) }{Muttātherīgāthā}
\extramarks{Thig 1.11}{Thig 1.11}

\begin{verse}%
I’m\marginnote{1.1} well freed, so very well freed, \\
freed from the three things that bent me over: \\
the mortar, the pestle, \\
and my humpbacked husband. \\
I’m freed from birth and death; \\
the conduit to rebirth is eradicated. 

%
\end{verse}

%
\section*{{\suttatitleacronym Thig 1.12}{\suttatitletranslation Dhammadinnā }{\suttatitleroot Dhammadinnātherīgāthā}}
\addcontentsline{toc}{section}{\tocacronym{Thig 1.12} \toctranslation{Dhammadinnā } \tocroot{Dhammadinnātherīgāthā}}
\markboth{Dhammadinnā }{Dhammadinnātherīgāthā}
\extramarks{Thig 1.12}{Thig 1.12}

\begin{verse}%
One\marginnote{1.1} who is eager and determined \\
would be filled with awareness. \\
Their mind not bound to pleasures of sense, \\
they’re said to be heading upstream. 

%
\end{verse}

%
\section*{{\suttatitleacronym Thig 1.13}{\suttatitletranslation Visākhā }{\suttatitleroot Visākhātherīgāthā}}
\addcontentsline{toc}{section}{\tocacronym{Thig 1.13} \toctranslation{Visākhā } \tocroot{Visākhātherīgāthā}}
\markboth{Visākhā }{Visākhātherīgāthā}
\extramarks{Thig 1.13}{Thig 1.13}

\begin{verse}%
Do\marginnote{1.1} the Buddha’s bidding, \\
you won’t regret it. \\
Having quickly washed your feet, \\
sit in a discreet place to meditate. 

%
\end{verse}

%
\section*{{\suttatitleacronym Thig 1.14}{\suttatitletranslation Sumanā }{\suttatitleroot Sumanātherīgāthā}}
\addcontentsline{toc}{section}{\tocacronym{Thig 1.14} \toctranslation{Sumanā } \tocroot{Sumanātherīgāthā}}
\markboth{Sumanā }{Sumanātherīgāthā}
\extramarks{Thig 1.14}{Thig 1.14}

\begin{verse}%
Having\marginnote{1.1} seen the elements as suffering, \\
don’t get reborn again. \\
When you’ve discarded desire for rebirth, \\
you will live at peace. 

%
\end{verse}

%
\section*{{\suttatitleacronym Thig 1.15}{\suttatitletranslation Uttarā (1st) }{\suttatitleroot Uttarātherīgāthā}}
\addcontentsline{toc}{section}{\tocacronym{Thig 1.15} \toctranslation{Uttarā (1st) } \tocroot{Uttarātherīgāthā}}
\markboth{Uttarā (1st) }{Uttarātherīgāthā}
\extramarks{Thig 1.15}{Thig 1.15}

\begin{verse}%
I\marginnote{1.1} was restrained \\
in body, speech, and mind. \\
Having plucked out craving, root and all, \\
I’m cooled and quenched. 

%
\end{verse}

%
\section*{{\suttatitleacronym Thig 1.16}{\suttatitletranslation Sumanā, Who Went Forth Late in Life }{\suttatitleroot Vuḍḍhapabbajitasumanātherīgāthā}}
\addcontentsline{toc}{section}{\tocacronym{Thig 1.16} \toctranslation{Sumanā, Who Went Forth Late in Life } \tocroot{Vuḍḍhapabbajitasumanātherīgāthā}}
\markboth{Sumanā, Who Went Forth Late in Life }{Vuḍḍhapabbajitasumanātherīgāthā}
\extramarks{Thig 1.16}{Thig 1.16}

\begin{verse}%
Sleep\marginnote{1.1} softly, old lady, \\
wrapped in the cloth you sewed yourself; \\
for your desire has been quelled, \\
you’re cooled and quenched. 

%
\end{verse}

%
\section*{{\suttatitleacronym Thig 1.17}{\suttatitletranslation Dhammā }{\suttatitleroot Dhammātherīgāthā}}
\addcontentsline{toc}{section}{\tocacronym{Thig 1.17} \toctranslation{Dhammā } \tocroot{Dhammātherīgāthā}}
\markboth{Dhammā }{Dhammātherīgāthā}
\extramarks{Thig 1.17}{Thig 1.17}

\begin{verse}%
I\marginnote{1.1} wandered for alms \\
though feeble, leaning on a staff. \\
My limbs wobbled \\
and I fell to the ground right there. \\
Seeing the danger of the body, \\
my mind was freed. 

%
\end{verse}

%
\section*{{\suttatitleacronym Thig 1.18}{\suttatitletranslation Saṅghā }{\suttatitleroot Saṁghātherīgāthā}}
\addcontentsline{toc}{section}{\tocacronym{Thig 1.18} \toctranslation{Saṅghā } \tocroot{Saṁghātherīgāthā}}
\markboth{Saṅghā }{Saṁghātherīgāthā}
\extramarks{Thig 1.18}{Thig 1.18}

\begin{verse}%
I\marginnote{1.1} gave up my home, my child, my cattle, \\
and all that I love, and went forth. \\
And now that I’ve given up desire and hate, \\
dispelled ignorance, \\
and plucked out craving, root and all, \\
I’m at peace, I’m quenched. 

%
\end{verse}

\scendsection{The Book of the Ones is finished. }

%
\addtocontents{toc}{\let\protect\contentsline\protect\nopagecontentsline}
\chapter*{The Book of the Twos }
\addcontentsline{toc}{chapter}{\tocchapterline{The Book of the Twos }}
\addtocontents{toc}{\let\protect\contentsline\protect\oldcontentsline}

%
\section*{{\suttatitleacronym Thig 2.1}{\suttatitletranslation Abhirūpanandā }{\suttatitleroot Abhirūpanandātherīgāthā}}
\addcontentsline{toc}{section}{\tocacronym{Thig 2.1} \toctranslation{Abhirūpanandā } \tocroot{Abhirūpanandātherīgāthā}}
\markboth{Abhirūpanandā }{Abhirūpanandātherīgāthā}
\extramarks{Thig 2.1}{Thig 2.1}

\begin{verse}%
\textsanskrit{Nandā},\marginnote{1.1} see this bag of bones as \\
diseased, filthy, and rotten. \\
With mind unified and serene, \\
meditate on the ugly aspects of the body. 

Meditate\marginnote{2.1} on the signless, \\
give up the underlying tendency to conceit; \\
and when you comprehend conceit, \\
you will live at peace. 

%
\end{verse}

\scendsutta{That is how the Buddha regularly advised the trainee nun \textsanskrit{Nandā} with these verses. }

%
\section*{{\suttatitleacronym Thig 2.2}{\suttatitletranslation Jentā }{\suttatitleroot Jentātherīgāthā}}
\addcontentsline{toc}{section}{\tocacronym{Thig 2.2} \toctranslation{Jentā } \tocroot{Jentātherīgāthā}}
\markboth{Jentā }{Jentātherīgāthā}
\extramarks{Thig 2.2}{Thig 2.2}

\begin{verse}%
Of\marginnote{1.1} the seven awakening factors, \\
the path for attaining extinguishment, \\
I have developed them all, \\
just as the Buddha taught. 

For\marginnote{2.1} I have seen the Blessed One, \\
and this bag of bones is my last. \\
Transmigration through births is finished, \\
now there’ll be no more future lives. 

%
\end{verse}

\scendsutta{That is how these verses were recited by the senior nun \textsanskrit{Jentā}. }

%
\section*{{\suttatitleacronym Thig 2.3}{\suttatitletranslation Sumaṅgala’s Mother }{\suttatitleroot Sumaṅgalamātātherīgāthā}}
\addcontentsline{toc}{section}{\tocacronym{Thig 2.3} \toctranslation{Sumaṅgala’s Mother } \tocroot{Sumaṅgalamātātherīgāthā}}
\markboth{Sumaṅgala’s Mother }{Sumaṅgalamātātherīgāthā}
\extramarks{Thig 2.3}{Thig 2.3}

\begin{verse}%
I’m\marginnote{1.1} well freed, well freed, \\
so very well freed from the pestle! \\
My shameless husband popped up like a mushroom, \\
my mortar wafted like eels. 

Greed\marginnote{2.1} and hate sizzle and hiss \\
as I plunge them (in cool water). \\
Having gone to the root of a tree, \\
I meditate happily, thinking, “Oh, what bliss!” 

%
\end{verse}

%
\section*{{\suttatitleacronym Thig 2.4}{\suttatitletranslation Aḍḍhakāsi }{\suttatitleroot Aḍḍhakāsitherīgāthā}}
\addcontentsline{toc}{section}{\tocacronym{Thig 2.4} \toctranslation{Aḍḍhakāsi } \tocroot{Aḍḍhakāsitherīgāthā}}
\markboth{Aḍḍhakāsi }{Aḍḍhakāsitherīgāthā}
\extramarks{Thig 2.4}{Thig 2.4}

\begin{verse}%
The\marginnote{1.1} price for my services \\
amounted to the nation of \textsanskrit{Kāsi}. \\
By setting that price, \\
the townsfolk made me priceless. 

Then,\marginnote{2.1} growing disillusioned with my form, \\
I became dispassionate. \\
Don’t journey on and on, \\
transmigrating through rebirths! \\
I've realized the three knowledges, \\
and fulfilled the Buddha’s instructions. 

%
\end{verse}

%
\section*{{\suttatitleacronym Thig 2.5}{\suttatitletranslation Cittā }{\suttatitleroot Cittātherīgāthā}}
\addcontentsline{toc}{section}{\tocacronym{Thig 2.5} \toctranslation{Cittā } \tocroot{Cittātherīgāthā}}
\markboth{Cittā }{Cittātherīgāthā}
\extramarks{Thig 2.5}{Thig 2.5}

\begin{verse}%
Though\marginnote{1.1} I’m skinny, \\
sick, and very feeble, \\
I climb the mountain, \\
leaning on a staff. 

Having\marginnote{2.1} laid down my outer robe, \\
and overturned my bowl, \\
propping myself against a rock, \\
I shattered the mass of darkness. 

%
\end{verse}

%
\section*{{\suttatitleacronym Thig 2.6}{\suttatitletranslation Mettikā }{\suttatitleroot Mettikātherīgāthā}}
\addcontentsline{toc}{section}{\tocacronym{Thig 2.6} \toctranslation{Mettikā } \tocroot{Mettikātherīgāthā}}
\markboth{Mettikā }{Mettikātherīgāthā}
\extramarks{Thig 2.6}{Thig 2.6}

\begin{verse}%
Though\marginnote{1.1} in pain, \\
feeble, my youth long gone, \\
I climb the mountain, \\
leaning on a staff. 

Having\marginnote{2.1} laid down my outer robe \\
and overturned my bowl, \\
sitting on a rock, \\
my mind was freed. \\
I’ve attained the three knowledges, \\
and fulfilled the Buddha’s instructions. 

%
\end{verse}

%
\section*{{\suttatitleacronym Thig 2.7}{\suttatitletranslation Mittā (2nd) }{\suttatitleroot Mittātherīgāthā}}
\addcontentsline{toc}{section}{\tocacronym{Thig 2.7} \toctranslation{Mittā (2nd) } \tocroot{Mittātherīgāthā}}
\markboth{Mittā (2nd) }{Mittātherīgāthā}
\extramarks{Thig 2.7}{Thig 2.7}

\begin{verse}%
I\marginnote{1.1} rejoiced in the host of gods, \\
having observed the sabbath \\
complete in all eight factors, \\
on the fourteenth and the fifteenth days, 

and\marginnote{2.1} the eighth day of the fortnight, \\
as well as on the fortnightly special displays. \\
Today I eat just once a day, \\
my head is shaven, I wear the outer robe. \\
I don’t long for the host of gods, \\
for stress has been removed from my heart. 

%
\end{verse}

%
\section*{{\suttatitleacronym Thig 2.8}{\suttatitletranslation To Abhayā’s Mother From Her Daughter }{\suttatitleroot Abhayamātutherīgāthā}}
\addcontentsline{toc}{section}{\tocacronym{Thig 2.8} \toctranslation{To Abhayā’s Mother From Her Daughter } \tocroot{Abhayamātutherīgāthā}}
\markboth{To Abhayā’s Mother From Her Daughter }{Abhayamātutherīgāthā}
\extramarks{Thig 2.8}{Thig 2.8}

\begin{verse}%
My\marginnote{1.1} dear mother, I examined this body, \\
up from the soles of the feet, \\
and down from the tips of the hairs, \\
so impure and foul-smelling. 

Meditating\marginnote{2.1} like this, \\
all my lust is eradicated. \\
The fever of passion is cut off, \\
I’m cooled and quenched. 

%
\end{verse}

%
\section*{{\suttatitleacronym Thig 2.9}{\suttatitletranslation Abhayā }{\suttatitleroot Abhayātherīgāthā}}
\addcontentsline{toc}{section}{\tocacronym{Thig 2.9} \toctranslation{Abhayā } \tocroot{Abhayātherīgāthā}}
\markboth{Abhayā }{Abhayātherīgāthā}
\extramarks{Thig 2.9}{Thig 2.9}

\begin{verse}%
\textsanskrit{Abhayā},\marginnote{1.1} the body is fragile, \\
yet ordinary people are attached to it. \\
I'll lay down the body, \\
aware and mindful. 

Though\marginnote{2.1} subject to so many painful things, \\
I have, through my love of diligence, \\
reached the ending of craving, \\
and fulfilled the Buddha’s instructions. 

%
\end{verse}

%
\section*{{\suttatitleacronym Thig 2.10}{\suttatitletranslation Sāmā }{\suttatitleroot Sāmātherīgāthā}}
\addcontentsline{toc}{section}{\tocacronym{Thig 2.10} \toctranslation{Sāmā } \tocroot{Sāmātherīgāthā}}
\markboth{Sāmā }{Sāmātherīgāthā}
\extramarks{Thig 2.10}{Thig 2.10}

\begin{verse}%
Four\marginnote{1.1} or five times \\
I left my dwelling. \\
I had failed to find peace of heart, \\
or any control over my mind. \\
Now it is the eighth night \\
since craving was eradicated. 

Though\marginnote{2.1} subject to so many painful things, \\
I have, through my love of diligence, \\
reached the ending of craving, \\
and fulfilled the Buddha’s instructions. 

%
\end{verse}

\scendsection{The Book of the Twos is finished. }

%
\addtocontents{toc}{\let\protect\contentsline\protect\nopagecontentsline}
\chapter*{The Book of the Threes }
\addcontentsline{toc}{chapter}{\tocchapterline{The Book of the Threes }}
\addtocontents{toc}{\let\protect\contentsline\protect\oldcontentsline}

%
\section*{{\suttatitleacronym Thig 3.1}{\suttatitletranslation Another Sāmā }{\suttatitleroot Aparāsāmātherīgāthā}}
\addcontentsline{toc}{section}{\tocacronym{Thig 3.1} \toctranslation{Another Sāmā } \tocroot{Aparāsāmātherīgāthā}}
\markboth{Another Sāmā }{Aparāsāmātherīgāthā}
\extramarks{Thig 3.1}{Thig 3.1}

\begin{verse}%
In\marginnote{1.1} the twenty-five years \\
since I went forth, \\
I don’t know that I had ever found \\
serenity in my mind. 

I\marginnote{2.1} had failed to find peace of heart, \\
or any control over my mind. \\
When I remembered the victor’s instructions, \\
I was struck with a sense of urgency. 

Though\marginnote{3.1} subject to so many painful things, \\
I have, through my love of diligence, \\
reached the ending of craving, \\
and fulfilled the Buddha’s instructions. \\
This is the seventh day \\
since my craving dried up. 

%
\end{verse}

%
\section*{{\suttatitleacronym Thig 3.2}{\suttatitletranslation Uttamā }{\suttatitleroot Uttamātherīgāthā}}
\addcontentsline{toc}{section}{\tocacronym{Thig 3.2} \toctranslation{Uttamā } \tocroot{Uttamātherīgāthā}}
\markboth{Uttamā }{Uttamātherīgāthā}
\extramarks{Thig 3.2}{Thig 3.2}

\begin{verse}%
Four\marginnote{1.1} or five times \\
I left my dwelling. \\
I had failed to find peace of heart, \\
or any control over my mind. 

I\marginnote{2.1} approached a nun \\
in whom I had faith. \\
She taught me the Dhamma: \\
the aggregates, sense fields, and elements. 

When\marginnote{3.1} I had heard her teaching, \\
in accordance with her instructions, \\
I sat cross-legged for seven days without moving, \\
given over to rapture and bliss. \\
On the eighth day I stretched out my feet, \\
having shattered the mass of darkness. 

%
\end{verse}

%
\section*{{\suttatitleacronym Thig 3.3}{\suttatitletranslation Another Uttamā }{\suttatitleroot Aparāuttamātherīgāthā}}
\addcontentsline{toc}{section}{\tocacronym{Thig 3.3} \toctranslation{Another Uttamā } \tocroot{Aparāuttamātherīgāthā}}
\markboth{Another Uttamā }{Aparāuttamātherīgāthā}
\extramarks{Thig 3.3}{Thig 3.3}

\begin{verse}%
Of\marginnote{1.1} the seven awakening factors, \\
the path for attaining extinguishment, \\
I have developed them all, \\
just as the Buddha taught. 

I\marginnote{2.1} attain the meditations on emptiness \\
and signlessness whenever I want. \\
I am the Buddha’s rightful daughter, \\
always delighting in quenching. 

All\marginnote{3.1} sensual pleasures are cut off, \\
whether human or divine. \\
Transmigration through births is finished, \\
now there’ll be no more future lives. 

%
\end{verse}

%
\section*{{\suttatitleacronym Thig 3.4}{\suttatitletranslation Dantikā }{\suttatitleroot Dantikātherīgāthā}}
\addcontentsline{toc}{section}{\tocacronym{Thig 3.4} \toctranslation{Dantikā } \tocroot{Dantikātherīgāthā}}
\markboth{Dantikā }{Dantikātherīgāthā}
\extramarks{Thig 3.4}{Thig 3.4}

\begin{verse}%
Leaving\marginnote{1.1} my day’s meditation \\
on Vulture’s Peak Mountain, \\
I saw an elephant on the riverbank \\
having just come up from his bath. 

A\marginnote{2.1} man, taking a pole with a hook, \\
asked the elephant, “Give me your foot.” \\
The elephant presented his foot, \\
and the man mounted him. 

Seeing\marginnote{3.1} a wild beast so tamed, \\
submitting to human control, \\
my mind became serene: \\
\emph{that} is why I’ve gone to the forest! 

%
\end{verse}

%
\section*{{\suttatitleacronym Thig 3.5}{\suttatitletranslation Ubbirī }{\suttatitleroot Ubbiritherīgāthā}}
\addcontentsline{toc}{section}{\tocacronym{Thig 3.5} \toctranslation{Ubbirī } \tocroot{Ubbiritherīgāthā}}
\markboth{Ubbirī }{Ubbiritherīgāthā}
\extramarks{Thig 3.5}{Thig 3.5}

\begin{verse}%
“You\marginnote{1.1} cry ‘Please be living!’ in the forest. \\
\textsanskrit{Ubbirī}, get a hold of yourself! \\
Eighty-four thousand people, \\
all named ‘living being’, \\
have been burnt in this funeral ground: \\
which one do you grieve for?” 

“Oh!\marginnote{2.1} For you have plucked the arrow from me, \\
so hard to see, stuck in the heart. \\
You’ve swept away the grief for my daughter \\
in which I once was mired. 

Today\marginnote{3.1} I’ve plucked the arrow, \\
I’m hungerless, extinguished. \\
I go for refuge to that sage, the Buddha, \\
to his teaching, and to the Sangha.” 

%
\end{verse}

%
\section*{{\suttatitleacronym Thig 3.6}{\suttatitletranslation Sukkā }{\suttatitleroot Sukkātherīgāthā}}
\addcontentsline{toc}{section}{\tocacronym{Thig 3.6} \toctranslation{Sukkā } \tocroot{Sukkātherīgāthā}}
\markboth{Sukkā }{Sukkātherīgāthā}
\extramarks{Thig 3.6}{Thig 3.6}

\begin{verse}%
“What’s\marginnote{1.1} up with these people in \textsanskrit{Rājagaha}? \\
They sprawl like they’ve been drinking mead! \\
They don’t attend on \textsanskrit{Sukkā} \\
as she teaches the Buddha’s instructions. 

But\marginnote{2.1} the wise—\\
it’s as if they drink it up, \\
so irresistible, delicious and nutritious, \\
like travelers enjoying a cool cloud.” 

“She’s\marginnote{3.1} known as \textsanskrit{Sukkā} because of her bright qualities, \\
free of greed, serene. \\
She bears her final body, \\
having vanquished \textsanskrit{Māra} and his mount.” 

%
\end{verse}

%
\section*{{\suttatitleacronym Thig 3.7}{\suttatitletranslation Selā }{\suttatitleroot Selātherīgāthā}}
\addcontentsline{toc}{section}{\tocacronym{Thig 3.7} \toctranslation{Selā } \tocroot{Selātherīgāthā}}
\markboth{Selā }{Selātherīgāthā}
\extramarks{Thig 3.7}{Thig 3.7}

\begin{verse}%
“There’s\marginnote{1.1} no escape in the world, \\
so what will seclusion do for you? \\
Enjoy the delights of sensual pleasure; \\
don’t regret it later.” 

“Sensual\marginnote{2.1} pleasures are like swords and stakes \\
the aggregates are their chopping block. \\
What you call sensual delight \\
is now no delight for me. 

Relishing\marginnote{3.1} is destroyed in every respect, \\
and the mass of darkness is shattered. \\
So know this, Wicked One: \\
you’re beaten, terminator!” 

%
\end{verse}

%
\section*{{\suttatitleacronym Thig 3.8}{\suttatitletranslation Somā }{\suttatitleroot Somātherīgāthā}}
\addcontentsline{toc}{section}{\tocacronym{Thig 3.8} \toctranslation{Somā } \tocroot{Somātherīgāthā}}
\markboth{Somā }{Somātherīgāthā}
\extramarks{Thig 3.8}{Thig 3.8}

\begin{verse}%
“That\marginnote{1.1} state’s very challenging; \\
it’s for the sages to attain. \\
It’s not possible for a woman, \\
with her two-fingered wisdom.” 

“What\marginnote{2.1} difference does womanhood make \\
when the mind is serene, \\
and knowledge is present \\
as you rightly discern the Dhamma. 

Relishing\marginnote{3.1} is destroyed in every respect, \\
and the mass of darkness is shattered. \\
So know this, Wicked One: \\
you’re beaten, terminator!” 

%
\end{verse}

\scendsection{The Book of the Threes is finished. }

%
\addtocontents{toc}{\let\protect\contentsline\protect\nopagecontentsline}
\chapter*{The Book of the Fours }
\addcontentsline{toc}{chapter}{\tocchapterline{The Book of the Fours }}
\addtocontents{toc}{\let\protect\contentsline\protect\oldcontentsline}

%
\section*{{\suttatitleacronym Thig 4.1}{\suttatitletranslation Bhaddā Daughter of Kapila }{\suttatitleroot Bhaddākāpilānītherīgāthā}}
\addcontentsline{toc}{section}{\tocacronym{Thig 4.1} \toctranslation{Bhaddā Daughter of Kapila } \tocroot{Bhaddākāpilānītherīgāthā}}
\markboth{Bhaddā Daughter of Kapila }{Bhaddākāpilānītherīgāthā}
\extramarks{Thig 4.1}{Thig 4.1}

\begin{verse}%
Kassapa\marginnote{1.1} is the son and heir of the Buddha, \\
whose mind is immersed in \textsanskrit{samādhi}. \\
He knows his past lives, \\
he sees heaven and places of loss, 

and\marginnote{2.1} has attained the end of rebirth: \\
that sage has perfect insight. \\
It’s because of these three knowledges \\
that the brahmin is a master of the three knowledges. 

In\marginnote{3.1} exactly the same way, \textsanskrit{Bhaddā} daughter of Kapila \\
is master of the three knowledges, conqueror of death. \\
She bears her final body, \\
having vanquished \textsanskrit{Māra} and his mount. 

Seeing\marginnote{4.1} the danger of the world, \\
both of us went forth. \\
Now we are tamed, our defilements have ended; \\
we’ve become cooled and quenched. 

%
\end{verse}

\scendsection{The Book of the Fours is finished. }

%
\addtocontents{toc}{\let\protect\contentsline\protect\nopagecontentsline}
\chapter*{The Book of the Fives }
\addcontentsline{toc}{chapter}{\tocchapterline{The Book of the Fives }}
\addtocontents{toc}{\let\protect\contentsline\protect\oldcontentsline}

%
\section*{{\suttatitleacronym Thig 5.1}{\suttatitletranslation An Unnamed Nun (2nd) }{\suttatitleroot Aññataratherīgāthā}}
\addcontentsline{toc}{section}{\tocacronym{Thig 5.1} \toctranslation{An Unnamed Nun (2nd) } \tocroot{Aññataratherīgāthā}}
\markboth{An Unnamed Nun (2nd) }{Aññataratherīgāthā}
\extramarks{Thig 5.1}{Thig 5.1}

\begin{verse}%
In\marginnote{1.1} the twenty-five years \\
since I went forth \\
I have not found peace of mind, \\
even for as long as a finger-snap. 

Failing\marginnote{2.1} to find peace of heart, \\
corrupted by sensual desire, \\
I cried with flailing arms \\
as I entered a dwelling. 

I\marginnote{3.1} approached a nun \\
in whom I had faith. \\
She taught me the Dhamma: \\
the aggregates, sense fields, and elements. 

When\marginnote{4.1} I heard her teaching, \\
I retired to a discreet place. \\
I know my past lives; \\
my clairvoyance is purified; 

I\marginnote{5.1} comprehend the minds of others; \\
my clairaudience is purified; \\
I've realized the psychic powers, \\
and attained the ending of defilements. \\
I have realized the six kinds of direct knowledge, \\
and fulfilled the Buddha’s instructions. 

%
\end{verse}

%
\section*{{\suttatitleacronym Thig 5.2}{\suttatitletranslation Vimalā, the Former Courtesan }{\suttatitleroot Vimalātherīgāthā}}
\addcontentsline{toc}{section}{\tocacronym{Thig 5.2} \toctranslation{Vimalā, the Former Courtesan } \tocroot{Vimalātherīgāthā}}
\markboth{Vimalā, the Former Courtesan }{Vimalātherīgāthā}
\extramarks{Thig 5.2}{Thig 5.2}

\begin{verse}%
Intoxicated\marginnote{1.1} by my appearance, \\
my figure, my beauty, my fame, \\
and owing to my youth, \\
I despised other women. 

I\marginnote{2.1} adorned this body, \\
so fancy, cooed over by fools, \\
and stood at the brothel door, \\
like a hunter laying a snare. 

I\marginnote{3.1} stripped for them, \\
revealing my many hidden treasures. \\
Creating an intricate illusion, \\
I laughed, teasing those men. 

Today,\marginnote{4.1} having wandered for alms, \\
my head shaven, wearing the outer robe, \\
I sat at the root of a tree to meditate; \\
I've gained freedom from thought. 

All\marginnote{5.1} bonds are cut off, \\
both human and divine. \\
Having wiped out all defilements, \\
I have become cooled and quenched. 

%
\end{verse}

%
\section*{{\suttatitleacronym Thig 5.3}{\suttatitletranslation Sīhā }{\suttatitleroot Sīhātherīgāthā}}
\addcontentsline{toc}{section}{\tocacronym{Thig 5.3} \toctranslation{Sīhā } \tocroot{Sīhātherīgāthā}}
\markboth{Sīhā }{Sīhātherīgāthā}
\extramarks{Thig 5.3}{Thig 5.3}

\begin{verse}%
Due\marginnote{1.1} to improper attention, \\
I was racked by desire for pleasures of the senses. \\
I was restless in the past, \\
lacking control over my mind. 

Overcome\marginnote{2.1} by corruptions, \\
pursuing perceptions of the beautiful, \\
I gained no peace of mind. \\
Under the sway of lustful thoughts, 

thin,\marginnote{3.1} pale, and wan, \\
for seven years I wandered, \\
full of pain, \\
finding no happiness by day or night. 

Taking\marginnote{4.1} a rope \\
I entered deep into the forest, thinking: \\
“It’s better that I hang myself \\
than I return to a lesser life.” 

I\marginnote{5.1} made a strong noose \\
and tied it to the branch of a tree. \\
Casting it round my neck, \\
my mind was freed. 

%
\end{verse}

%
\section*{{\suttatitleacronym Thig 5.4}{\suttatitletranslation Sundarīnandā }{\suttatitleroot Sundarīnandātherīgāthā}}
\addcontentsline{toc}{section}{\tocacronym{Thig 5.4} \toctranslation{Sundarīnandā } \tocroot{Sundarīnandātherīgāthā}}
\markboth{Sundarīnandā }{Sundarīnandātherīgāthā}
\extramarks{Thig 5.4}{Thig 5.4}

\begin{verse}%
“\textsanskrit{Nandā},\marginnote{1.1} see this bag of bones as \\
diseased, filthy, and rotten. \\
With mind unified and serene, \\
meditate on the ugly aspects of the body: 

as\marginnote{2.1} this is, so is that, \\
as that is, so is this. \\
A foul stink wafts from it, \\
it is the fools’ delight.” 

Reflecting\marginnote{3.1} in such a way, \\
tireless all day and night, \\
having broken through \\
with my own wisdom, I saw. 

Being\marginnote{4.1} diligent, \\
properly investigating, \\
I truly saw this body \\
both inside and out. 

Then,\marginnote{5.1} growing disillusioned with the body, \\
I became dispassionate within. \\
Diligent, detached, \\
I’m quenched and at peace. 

%
\end{verse}

%
\section*{{\suttatitleacronym Thig 5.5}{\suttatitletranslation Nanduttarā }{\suttatitleroot Nanduttarātherīgāthā}}
\addcontentsline{toc}{section}{\tocacronym{Thig 5.5} \toctranslation{Nanduttarā } \tocroot{Nanduttarātherīgāthā}}
\markboth{Nanduttarā }{Nanduttarātherīgāthā}
\extramarks{Thig 5.5}{Thig 5.5}

\begin{verse}%
In\marginnote{1.1} the past I worshiped the sacred flame, \\
the moon, the sun, and the gods. \\
Having gone to a river ford, \\
I plunged into the water. 

Undertaking\marginnote{2.1} many vows, \\
I shaved half my head. \\
Preparing a bed on the ground, \\
I ate no food at night. 

I\marginnote{3.1} loved my ornaments and decorations; \\
and with baths and oil-massages, \\
I pandered to this body, \\
racked by desire for pleasures of the senses. 

But\marginnote{4.1} then I gained faith, \\
and went forth to homelessness. \\
Truly seeing the body, \\
desire for sensual pleasure is eradicated. 

All\marginnote{5.1} rebirths are cut off, \\
wishes and aspirations too. \\
Detached from all attachments, \\
I've found peace of mind. 

%
\end{verse}

%
\section*{{\suttatitleacronym Thig 5.6}{\suttatitletranslation Mittākāḷī }{\suttatitleroot Mittākāḷītherīgāthā}}
\addcontentsline{toc}{section}{\tocacronym{Thig 5.6} \toctranslation{Mittākāḷī } \tocroot{Mittākāḷītherīgāthā}}
\markboth{Mittākāḷī }{Mittākāḷītherīgāthā}
\extramarks{Thig 5.6}{Thig 5.6}

\begin{verse}%
Having\marginnote{1.1} gone forth out of faith \\
from the lay life to homelessness, \\
I wandered here and there, \\
jealous of possessions and honors. 

Neglecting\marginnote{2.1} the highest goal, \\
I pursued the lowest. \\
Under the sway of corruptions, \\
I never knew the goal of the ascetic life. 

I\marginnote{3.1} was struck with a sense of urgency \\
as I was sitting in my hut: \\
“I’m walking the wrong path, \\
under the sway of craving. 

My\marginnote{4.1} life is short, \\
trampled by old age and sickness. \\
Before this body breaks apart, \\
there is no time for me to be careless.” 

I\marginnote{5.1} examined in line with reality \\
the rise and fall of the aggregates. \\
I stood up with mind liberated, \\
having fulfilled the Buddha’s instructions. 

%
\end{verse}

%
\section*{{\suttatitleacronym Thig 5.7}{\suttatitletranslation Sakulā }{\suttatitleroot Sakulātherīgāthā}}
\addcontentsline{toc}{section}{\tocacronym{Thig 5.7} \toctranslation{Sakulā } \tocroot{Sakulātherīgāthā}}
\markboth{Sakulā }{Sakulātherīgāthā}
\extramarks{Thig 5.7}{Thig 5.7}

\begin{verse}%
While\marginnote{1.1} staying at home \\
I heard the teaching from a monk. \\
I saw the stainless Dhamma, \\
extinguishment, the imperishable state. 

Leaving\marginnote{2.1} behind my son and my daughter, \\
my riches and my grain, \\
I had my hair cut off, \\
and went forth to homelessness. 

As\marginnote{3.1} a trainee nun, \\
I developed the direct path. \\
I gave up greed and hate, \\
along with associated defilements. 

When\marginnote{4.1} I was fully ordained as a nun, \\
I recollected my past lives, \\
and purified my clairvoyance, \\
immaculate and fully developed. 

Conditions\marginnote{5.1} are born of causes, crumbling; \\
having seen them as other, \\
I gave up all defilements, \\
I’m cooled and quenched. 

%
\end{verse}

%
\section*{{\suttatitleacronym Thig 5.8}{\suttatitletranslation Soṇā }{\suttatitleroot Soṇātherīgāthā}}
\addcontentsline{toc}{section}{\tocacronym{Thig 5.8} \toctranslation{Soṇā } \tocroot{Soṇātherīgāthā}}
\markboth{Soṇā }{Soṇātherīgāthā}
\extramarks{Thig 5.8}{Thig 5.8}

\begin{verse}%
I\marginnote{1.1} gave birth to ten sons \\
in this form, this bag of bones. \\
Then, when feeble and old, \\
I approached a nun. 

She\marginnote{2.1} taught me the Dhamma: \\
the aggregates, sense fields, and elements. \\
When I heard her teaching, \\
I shaved off my hair and went forth. 

When\marginnote{3.1} I was a trainee nun, \\
my clairvoyance was clarified, \\
and I knew my past lives, \\
the places I used to live. 

I\marginnote{4.1} meditate on the signless, \\
my mind unified and serene. \\
I achieved the immediate liberation, \\
extinguished by not grasping. 

The\marginnote{5.1} five aggregates are fully understood; \\
they remain, but their root is cut. \\
Curse you, wretched old age! \\
now there’ll be no more future lives. 

%
\end{verse}

%
\section*{{\suttatitleacronym Thig 5.9}{\suttatitletranslation Bhaddā of the Curly Hair }{\suttatitleroot Bhaddākuṇḍalakesātherīgāthā}}
\addcontentsline{toc}{section}{\tocacronym{Thig 5.9} \toctranslation{Bhaddā of the Curly Hair } \tocroot{Bhaddākuṇḍalakesātherīgāthā}}
\markboth{Bhaddā of the Curly Hair }{Bhaddākuṇḍalakesātherīgāthā}
\extramarks{Thig 5.9}{Thig 5.9}

\begin{verse}%
My\marginnote{1.1} hair mown off, covered in mud, \\
I used to wander wearing just one robe. \\
I saw fault where there was none, \\
and was blind to the actual fault. 

Leaving\marginnote{2.1} my day’s meditation \\
on Vulture’s Peak Mountain, \\
I saw the stainless Buddha \\
at the fore of the mendicant \textsanskrit{Saṅgha}. 

I\marginnote{3.1} bent my knee and bowed, \\
and in his presence raised my joined palms. \\
“Come \textsanskrit{Bhaddā},” he said; \\
that was my ordination. 

“I’ve\marginnote{4.1} wandered among the \textsanskrit{Aṅgans} and Magadhans, \\
the \textsanskrit{Vajjīs}, \textsanskrit{Kāsīs}, and Kosalans. \\
I have eaten the almsfood of the nations \\
free of debt for fifty years.” 

“O!\marginnote{5.1} He has made so much merit! \\
That lay follower is so very wise. \\
He gave a robe to \textsanskrit{Bhaddā}, \\
who is released from all ties.” 

%
\end{verse}

%
\section*{{\suttatitleacronym Thig 5.10}{\suttatitletranslation Paṭācārā }{\suttatitleroot Paṭācārātherīgāthā}}
\addcontentsline{toc}{section}{\tocacronym{Thig 5.10} \toctranslation{Paṭācārā } \tocroot{Paṭācārātherīgāthā}}
\markboth{Paṭācārā }{Paṭācārātherīgāthā}
\extramarks{Thig 5.10}{Thig 5.10}

\begin{verse}%
Plowing\marginnote{1.1} the fields, \\
sowing seeds in the ground, \\
supporting partners and children, \\
young men acquire wealth. 

I\marginnote{2.1} am accomplished in ethics, \\
and I do the Teacher’s bidding, \\
being neither lazy nor restless—\\
why then do I not achieve quenching? 

Having\marginnote{3.1} washed my feet, \\
I took note of the water, \\
seeing the foot-washing water \\
flowing from high ground to low. 

My\marginnote{4.1} mind became serene, \\
like a fine thoroughbred steed. \\
Then, taking a lamp, \\
I entered my dwelling, \\
inspected the bed, \\
and sat on my cot. 

Then,\marginnote{5.1} grabbing the pin, \\
I drew out the wick. \\
The liberation of my heart \\
was like the quenching of the lamp. 

%
\end{verse}

%
\section*{{\suttatitleacronym Thig 5.11}{\suttatitletranslation Thirty Nuns }{\suttatitleroot Tiṁsamattātherīgāthā}}
\addcontentsline{toc}{section}{\tocacronym{Thig 5.11} \toctranslation{Thirty Nuns } \tocroot{Tiṁsamattātherīgāthā}}
\markboth{Thirty Nuns }{Tiṁsamattātherīgāthā}
\extramarks{Thig 5.11}{Thig 5.11}

\begin{verse}%
“Taking\marginnote{1.1} a pestle, \\
young men pound grain. \\
Supporting partners and children, \\
young men acquire wealth. 

Do\marginnote{2.1} the Buddha’s bidding, \\
you won’t regret it. \\
Having quickly washed your feet, \\
sit in a discreet place to meditate. \\
Devoted to serenity of heart, \\
do the Buddha’s bidding.” 

After\marginnote{3.1} hearing her words, \\
the instructions of \textsanskrit{Paṭācārā}, \\
they washed their feet \\
and retired to a discreet place. \\
Devoted to serenity of heart, \\
they did the Buddha’s bidding. 

In\marginnote{4.1} the first watch of the night, \\
they recollected their past lives. \\
In the middle watch of the night, \\
they purified their clairvoyance. \\
In the last watch of the night, \\
they shattered the mass of darkness. 

They\marginnote{5.1} rose and paid homage at her feet: \\
“We have done your bidding; \\
we shall abide honoring you, \\
as the thirty gods honor Indra, \\
undefeated in battle. \\
Masters of the three knowledges, we are free of defilements.” 

%
\end{verse}

\scendsutta{That is how thirty senior nuns declared their enlightenment in the presence of \textsanskrit{Paṭācārā}. }

%
\section*{{\suttatitleacronym Thig 5.12}{\suttatitletranslation Candā }{\suttatitleroot Candātherīgāthā}}
\addcontentsline{toc}{section}{\tocacronym{Thig 5.12} \toctranslation{Candā } \tocroot{Candātherīgāthā}}
\markboth{Candā }{Candātherīgāthā}
\extramarks{Thig 5.12}{Thig 5.12}

\begin{verse}%
I\marginnote{1.1} used to be in a sorry state. \\
As a childless widow, \\
bereft of friends or relatives, \\
I got neither food nor clothes. 

I\marginnote{2.1} took a bowl and a staff \\
and went begging from family to family. \\
For seven years I wandered, \\
burned by heat and cold. 

Then\marginnote{3.1} I saw a nun \\
receiving food and drink. \\
Approaching her, I said: \\
“Send me forth to homelessness.” 

Out\marginnote{4.1} of compassion for me, \\
\textsanskrit{Paṭācārā} gave me the going forth. \\
Then, having advised me, \\
she urged me on to the ultimate goal. 

After\marginnote{5.1} hearing her words, \\
I did her bidding. \\
The lady’s advice was not in vain: \\
master of the three knowledges, I am free of defilements. 

%
\end{verse}

\scendsection{The Book of the Fives is finished. }

%
\addtocontents{toc}{\let\protect\contentsline\protect\nopagecontentsline}
\chapter*{The Book of the Sixes }
\addcontentsline{toc}{chapter}{\tocchapterline{The Book of the Sixes }}
\addtocontents{toc}{\let\protect\contentsline\protect\oldcontentsline}

%
\section*{{\suttatitleacronym Thig 6.1}{\suttatitletranslation Paṭācārā, Who Had a Following of Five Hundred }{\suttatitleroot Pañcasatamattātherīgāthā}}
\addcontentsline{toc}{section}{\tocacronym{Thig 6.1} \toctranslation{Paṭācārā, Who Had a Following of Five Hundred } \tocroot{Pañcasatamattātherīgāthā}}
\markboth{Paṭācārā, Who Had a Following of Five Hundred }{Pañcasatamattātherīgāthā}
\extramarks{Thig 6.1}{Thig 6.1}

\begin{verse}%
“One\marginnote{1.1} whose path you do not know, \\
not whence they came nor where they went; \\
though they came from who knows where, \\
you mourn that being, crying, ‘Oh my son!’ 

But\marginnote{2.1} one whose path you do know, \\
whence they came or where they went; \\
that one you do not lament—\\
such is the nature of living creatures. 

Unasked\marginnote{3.1} he came, \\
he left without leave. \\
He must have come from somewhere, \\
and stayed who knows how many days. \\
He left from here by one road, \\
he will go from there by another. 

Departing\marginnote{4.1} with the form of a human, \\
he will go on transmigrating. \\
As he came, so he went: \\
why cry over that?” 

“Oh!\marginnote{5.1} For you have plucked the arrow from me, \\
so hard to see, stuck in the heart. \\
You’ve swept away the grief for my son, \\
in which I once was mired. 

Today\marginnote{6.1} I’ve plucked the arrow, \\
I’m hungerless, extinguished. \\
I go for refuge to that sage, the Buddha, \\
to his teaching, and to the Sangha.” 

%
\end{verse}

\scendsutta{That is how \textsanskrit{Paṭācārā}, who had a following of five hundred, declared her enlightenment. }

%
\section*{{\suttatitleacronym Thig 6.2}{\suttatitletranslation Vāseṭṭhī }{\suttatitleroot Vāseṭṭhītherīgāthā}}
\addcontentsline{toc}{section}{\tocacronym{Thig 6.2} \toctranslation{Vāseṭṭhī } \tocroot{Vāseṭṭhītherīgāthā}}
\markboth{Vāseṭṭhī }{Vāseṭṭhītherīgāthā}
\extramarks{Thig 6.2}{Thig 6.2}

\begin{verse}%
Struck\marginnote{1.1} down with grief for my son, \\
deranged, out of my mind, \\
naked, my hair flying, \\
I wandered here and there. 

I\marginnote{2.1} lived on rubbish heaps, \\
in cemeteries and highways. \\
For three years I wandered, \\
stricken by hunger and thirst. 

Then\marginnote{3.1} I saw the Holy One, \\
who had gone to the city of \textsanskrit{Mithilā}. \\
Tamer of the untamed, \\
the Awakened One fears nothing from any quarter. 

Regaining\marginnote{4.1} my mind, \\
I paid homage and sat down. \\
Out of compassion \\
Gotama taught me the Dhamma. 

After\marginnote{5.1} hearing his teaching, \\
I went forth to homelessness. \\
Applying myself to the Teacher’s words, \\
I realized the state of grace. 

All\marginnote{6.1} sorrows are cut off, \\
given up, they end here. \\
I've fully understood the basis \\
from which grief comes to be. 

%
\end{verse}

%
\section*{{\suttatitleacronym Thig 6.3}{\suttatitletranslation Khemā }{\suttatitleroot Khemātherīgāthā}}
\addcontentsline{toc}{section}{\tocacronym{Thig 6.3} \toctranslation{Khemā } \tocroot{Khemātherīgāthā}}
\markboth{Khemā }{Khemātherīgāthā}
\extramarks{Thig 6.3}{Thig 6.3}

\begin{verse}%
“You’re\marginnote{1.1} so young and beautiful! \\
I too am young, just a youth. \\
Come, \textsanskrit{Khemā}, let us enjoy \\
the music of a five-piece band.” 

“This\marginnote{2.1} body is rotting, \\
ailing and frail, \\
I’m horrified and repelled by it, \\
and I’ve eradicated sensual craving. 

Sensual\marginnote{3.1} pleasures are like swords and stakes; \\
the aggregates are their chopping block. \\
What you call sensual delight \\
is now no delight for me. 

Relishing\marginnote{4.1} is destroyed in every respect, \\
and the mass of darkness is shattered. \\
So know this, Wicked One: \\
you’re beaten, terminator!” 

“Worshiping\marginnote{5.1} the stars, \\
serving the sacred flame in a grove; \\
failing to grasp the true nature of things, \\
foolish me, I thought this was purity. 

But\marginnote{6.1} now I worship the Awakened One, \\
supreme among men. \\
Doing the teacher’s bidding, \\
I am released from all suffering.” 

%
\end{verse}

%
\section*{{\suttatitleacronym Thig 6.4}{\suttatitletranslation Sujātā }{\suttatitleroot Sujātātherīgāthā}}
\addcontentsline{toc}{section}{\tocacronym{Thig 6.4} \toctranslation{Sujātā } \tocroot{Sujātātherīgāthā}}
\markboth{Sujātā }{Sujātātherīgāthā}
\extramarks{Thig 6.4}{Thig 6.4}

\begin{verse}%
I\marginnote{1.1} was adorned with jewelry and all dressed up, \\
with garlands, and sandalwood makeup piled on, \\
all covered over with decorations, \\
and surrounded by my maids. 

Taking\marginnote{2.1} food and drink, \\
staples and dainties in no small amount, \\
I left my house \\
and took myself to the park. 

I\marginnote{3.1} enjoyed myself there and played about, \\
and then, returning to my own house, \\
I saw a monastic dwelling, \\
and so I entered the \textsanskrit{Añjana} grove at \textsanskrit{Sāketa}. 

Seeing\marginnote{4.1} the light of the world, \\
I paid homage and sat down. \\
Out of compassion \\
the seer taught me the Dhamma. 

When\marginnote{5.1} I heard the great hermit, \\
I penetrated the truth. \\
Right there I encountered the Dhamma, \\
the stainless, deathless state. 

Then,\marginnote{6.1} having understood the true teaching, \\
I went forth to homelessness. \\
I’ve attained the three knowledges; \\
the Buddha’s bidding was not in vain. 

%
\end{verse}

%
\section*{{\suttatitleacronym Thig 6.5}{\suttatitletranslation Anopamā }{\suttatitleroot Anopamātherīgāthā}}
\addcontentsline{toc}{section}{\tocacronym{Thig 6.5} \toctranslation{Anopamā } \tocroot{Anopamātherīgāthā}}
\markboth{Anopamā }{Anopamātherīgāthā}
\extramarks{Thig 6.5}{Thig 6.5}

\begin{verse}%
I\marginnote{1.1} was born into an eminent family, \\
affluent and wealthy, \\
endowed with a beautiful complexion and figure; \\
Majjha’s true-born daughter. 

I\marginnote{2.1} was sought by princes, \\
coveted by sons of the wealthy. \\
One sent a messenger to my father: \\
“Give me \textsanskrit{Anopamā}! 

However\marginnote{3.1} much your daughter \\
\textsanskrit{Anopamā} weighs, \\
I'll give you eight times that \\
in gold and gems.” 

When\marginnote{4.1} I saw the Awakened One, \\
the world’s Elder, unsurpassed, \\
I paid homage at his feet, \\
then sat down to one side. 

Out\marginnote{5.1} of compassion, \\
Gotama taught me the Dhamma. \\
While sitting in that seat, \\
I realized the third fruit. 

Then,\marginnote{6.1} having shaved off my hair, \\
I went forth to homelessness. \\
This is the seventh day \\
since my craving dried up. 

%
\end{verse}

%
\section*{{\suttatitleacronym Thig 6.6}{\suttatitletranslation Mahāpajāpati Gotamī }{\suttatitleroot Mahāpajāpatigotamītherīgāthā}}
\addcontentsline{toc}{section}{\tocacronym{Thig 6.6} \toctranslation{Mahāpajāpati Gotamī } \tocroot{Mahāpajāpatigotamītherīgāthā}}
\markboth{Mahāpajāpati Gotamī }{Mahāpajāpatigotamītherīgāthā}
\extramarks{Thig 6.6}{Thig 6.6}

\begin{verse}%
Oh\marginnote{1.1} Buddha, my hero: homage to you! \\
Supreme among all beings, \\
who released me from suffering, \\
and many other beings as well. 

All\marginnote{2.1} suffering is fully understood; \\
craving—its cause—is dried up; \\
the eightfold path has been developed; \\
and cessation has been realized by me. 

Previously\marginnote{3.1} I was a mother, a son, \\
a father, a brother, and a grandmother. \\
Failing to grasp the true nature of things, \\
I transmigrated without reward. 

Since\marginnote{4.1} I have seen the Blessed One, \\
this bag of bones is my last. \\
Transmigration through births is finished, \\
now there’ll be no more future lives. 

I\marginnote{5.1} see the disciples in harmony, \\
energetic and resolute, \\
always staunchly vigorous—\\
this is homage to the Buddhas! 

It\marginnote{6.1} was truly for the benefit of many \\
that \textsanskrit{Māyā} gave birth to Gotama. \\
He swept away the mass of suffering \\
for those stricken by sickness and death. 

%
\end{verse}

%
\section*{{\suttatitleacronym Thig 6.7}{\suttatitletranslation Guttā }{\suttatitleroot Guttātherīgāthā}}
\addcontentsline{toc}{section}{\tocacronym{Thig 6.7} \toctranslation{Guttā } \tocroot{Guttātherīgāthā}}
\markboth{Guttā }{Guttātherīgāthā}
\extramarks{Thig 6.7}{Thig 6.7}

\begin{verse}%
\textsanskrit{Guttā},\marginnote{1.1} you have given up your child, \\
your wealth, and all that you love. \\
Foster the goal for which you went forth; \\
do not fall under the mind’s control. 

Beings\marginnote{2.1} deceived by the mind, \\
playing in \textsanskrit{Māra}’s domain, \\
ignorant, they journey on, \\
transmigrating through countless rebirths. 

Sensual\marginnote{3.1} desire and ill will, \\
and identity view; \\
misapprehension of precepts and observances, \\
and doubt as the fifth. 

O\marginnote{4.1} nun, when you have given up \\
these lower fetters, \\
you won’t come back \\
to this world again. 

And\marginnote{5.1} when you’re rid of desire, \\
conceit, ignorance, and restlessness, \\
having cut the fetters, \\
you’ll make an end to suffering. 

Having\marginnote{6.1} wiped out transmigration, \\
and fully understood rebirth, \\
hungerless in this very life, \\
you will live at peace. 

%
\end{verse}

%
\section*{{\suttatitleacronym Thig 6.8}{\suttatitletranslation Vijayā }{\suttatitleroot Vijayātherīgāthā}}
\addcontentsline{toc}{section}{\tocacronym{Thig 6.8} \toctranslation{Vijayā } \tocroot{Vijayātherīgāthā}}
\markboth{Vijayā }{Vijayātherīgāthā}
\extramarks{Thig 6.8}{Thig 6.8}

\begin{verse}%
Four\marginnote{1.1} or five times \\
I left my dwelling; \\
I had failed to find peace of heart, \\
or any control over my mind. 

I\marginnote{2.1} approached a nun \\
and politely questioned her. \\
She taught me the Dhamma: \\
the elements and sense fields, 

the\marginnote{3.1} four noble truths, \\
the faculties and the powers, \\
the awakening factors, and the eightfold path \\
for the attainment of the highest goal. 

After\marginnote{4.1} hearing her words, \\
I did her bidding. \\
In the first watch of the night, \\
I recollected my past lives. 

In\marginnote{5.1} the middle watch of the night, \\
I purified my clairvoyance. \\
In the last watch of the night, \\
I shattered the mass of darkness. 

I\marginnote{6.1} then meditated pervading my body \\
with rapture and bliss. \\
On the seventh day I stretched out my feet, \\
having shattered the mass of darkness. 

%
\end{verse}

\scendsection{The Book of the Sixes is finished. }

%
\addtocontents{toc}{\let\protect\contentsline\protect\nopagecontentsline}
\chapter*{The Book of the Sevens }
\addcontentsline{toc}{chapter}{\tocchapterline{The Book of the Sevens }}
\addtocontents{toc}{\let\protect\contentsline\protect\oldcontentsline}

%
\section*{{\suttatitleacronym Thig 7.1}{\suttatitletranslation Uttarā (2nd) }{\suttatitleroot Uttarātherīgāthā}}
\addcontentsline{toc}{section}{\tocacronym{Thig 7.1} \toctranslation{Uttarā (2nd) } \tocroot{Uttarātherīgāthā}}
\markboth{Uttarā (2nd) }{Uttarātherīgāthā}
\extramarks{Thig 7.1}{Thig 7.1}

\begin{verse}%
“Taking\marginnote{1.1} a pestle, \\
young men pound grain. \\
Supporting partners and children, \\
young men acquire wealth. 

Work\marginnote{2.1} at the Buddha’s bidding, \\
you won’t regret it. \\
Having quickly washed your feet, \\
sit in a discreet place to meditate. 

Establish\marginnote{3.1} the mind, \\
unified and serene. \\
Examine conditions \\
as other, not as self.” 

“After\marginnote{4.1} hearing her words, \\
the instructions of \textsanskrit{Paṭācārā}, \\
I washed my feet \\
and retired to a discreet place. 

In\marginnote{5.1} the first watch of the night, \\
I recollected my past lives. \\
In the middle watch of the night, \\
I purified my clairvoyance. 

In\marginnote{6.1} the last watch of the night, \\
I shattered the mass of darkness. \\
I rose up master of the three knowledges: \\
your bidding has been done. 

I\marginnote{7.1} shall abide honoring you \\
as the thirty gods honor Sakka, \\
undefeated in battle. \\
Master of the three knowledges, I am free of defilements.” 

%
\end{verse}

%
\section*{{\suttatitleacronym Thig 7.2}{\suttatitletranslation Cālā }{\suttatitleroot Cālātherīgāthā}}
\addcontentsline{toc}{section}{\tocacronym{Thig 7.2} \toctranslation{Cālā } \tocroot{Cālātherīgāthā}}
\markboth{Cālā }{Cālātherīgāthā}
\extramarks{Thig 7.2}{Thig 7.2}

\begin{verse}%
“As\marginnote{1.1} a nun with developed faculties, \\
having established mindfulness, \\
I penetrated that peaceful state, \\
the blissful stilling of conditions.” 

“In\marginnote{2.1} whose name did you shave your head? \\
You look like an ascetic, \\
but you don’t believe in any creed. \\
Why do you live as if lost?” 

“Followers\marginnote{3.1} of other creeds \\
rely on their views. \\
They don’t understand the Dhamma, \\
for they’re no experts in the Dhamma. 

But\marginnote{4.1} there is one born in the Sakyan clan, \\
the unrivaled Buddha; \\
he taught me the Dhamma \\
for going beyond views. 

Suffering,\marginnote{5.1} suffering’s origin, \\
suffering’s transcendence, \\
and the noble eightfold path \\
that leads to the stilling of suffering. 

After\marginnote{6.1} hearing his words, \\
I happily did his bidding. \\
I’ve attained the three knowledges \\
and fulfilled the Buddha’s instructions. 

Relishing\marginnote{7.1} is destroyed in every respect, \\
and the mass of darkness is shattered. \\
So know this, Wicked One: \\
you’re beaten, terminator!” 

%
\end{verse}

%
\section*{{\suttatitleacronym Thig 7.3}{\suttatitletranslation Upacālā }{\suttatitleroot Upacālātherīgāthā}}
\addcontentsline{toc}{section}{\tocacronym{Thig 7.3} \toctranslation{Upacālā } \tocroot{Upacālātherīgāthā}}
\markboth{Upacālā }{Upacālātherīgāthā}
\extramarks{Thig 7.3}{Thig 7.3}

\begin{verse}%
“A\marginnote{1.1} nun with faculties developed, \\
mindful, seeing clearly, \\
I penetrated that peaceful state, \\
which sinners do not cultivate.” 

“Why\marginnote{2.1} don’t you approve of rebirth? \\
When you’re born, you get to enjoy sensual pleasures. \\
Enjoy the delights of sensual pleasure; \\
don’t regret it later.” 

“Death\marginnote{3.1} comes to those who are born; \\
and when born they fall into suffering: \\
the chopping off of hands and feet, \\
killing, caging, misery. 

But\marginnote{4.1} there is one born in the Sakyan clan, \\
an awakened champion. \\
He taught me the Dhamma \\
for passing beyond rebirth: 

suffering,\marginnote{5.1} suffering’s origin, \\
suffering’s transcendence, \\
and the noble eightfold path \\
that leads to the stilling of suffering. 

After\marginnote{6.1} hearing his words, \\
I happily did his bidding. \\
I’ve attained the three knowledges \\
and fulfilled the Buddha’s instructions. 

Relishing\marginnote{7.1} is destroyed in every respect, \\
and the mass of darkness is shattered. \\
So know this, Wicked One: \\
you’re beaten, terminator!” 

%
\end{verse}

\scendsection{The Book of the Sevens is finished. }

%
\addtocontents{toc}{\let\protect\contentsline\protect\nopagecontentsline}
\chapter*{The Book of the Eights }
\addcontentsline{toc}{chapter}{\tocchapterline{The Book of the Eights }}
\addtocontents{toc}{\let\protect\contentsline\protect\oldcontentsline}

%
\section*{{\suttatitleacronym Thig 8.1}{\suttatitletranslation Sīsūpacālā }{\suttatitleroot Sīsūpacālātherīgāthā}}
\addcontentsline{toc}{section}{\tocacronym{Thig 8.1} \toctranslation{Sīsūpacālā } \tocroot{Sīsūpacālātherīgāthā}}
\markboth{Sīsūpacālā }{Sīsūpacālātherīgāthā}
\extramarks{Thig 8.1}{Thig 8.1}

\begin{verse}%
“A\marginnote{1.1} nun accomplished in ethics, \\
her sense faculties well-restrained, \\
would realize the peaceful state, \\
so irresistible, delicious and nutritious.” 

“There\marginnote{2.1} are the Gods of the Thirty-Three, and those of Yama; \\
also the Joyful Deities, \\
the Gods Who Love to Create, \\
and the Gods Who Control the Creations of Others. \\
Set your heart on such places, \\
where you used to live.” 

“The\marginnote{3.1} Gods of the Thirty-Three, and those of Yama; \\
also the Joyful Deities, \\
the Gods Who Love to Create, \\
and the Gods Who Control the Creations of Others—

time\marginnote{4.1} after time, life after life, \\
are governed by identity. \\
They haven’t transcended identity, \\
those who transmigrate through birth and death. 

All\marginnote{5.1} the world is on fire, \\
all the world is alight, \\
all the world is ablaze, \\
all the world is rocking. 

The\marginnote{6.1} Buddha taught me the Dhamma, \\
unshakable, incomparable, \\
not frequented by ordinary people; \\
my mind adores that place. 

After\marginnote{7.1} hearing his words, \\
I happily did his bidding. \\
I’ve attained the three knowledges, \\
and fulfilled the Buddha’s instructions. 

Relishing\marginnote{8.1} is destroyed in every respect, \\
and the mass of darkness is shattered. \\
So know this, Wicked One: \\
you’re beaten, terminator!” 

%
\end{verse}

\scendsection{The Book of the Eights is finished. }

%
\addtocontents{toc}{\let\protect\contentsline\protect\nopagecontentsline}
\chapter*{The Book of the Nines }
\addcontentsline{toc}{chapter}{\tocchapterline{The Book of the Nines }}
\addtocontents{toc}{\let\protect\contentsline\protect\oldcontentsline}

%
\section*{{\suttatitleacronym Thig 9.1}{\suttatitletranslation Vaḍḍha’s Mother }{\suttatitleroot Vaḍḍhamātutherīgāthā}}
\addcontentsline{toc}{section}{\tocacronym{Thig 9.1} \toctranslation{Vaḍḍha’s Mother } \tocroot{Vaḍḍhamātutherīgāthā}}
\markboth{Vaḍḍha’s Mother }{Vaḍḍhamātutherīgāthā}
\extramarks{Thig 9.1}{Thig 9.1}

\begin{verse}%
“\textsanskrit{Vaḍḍha},\marginnote{1.1} please never ever \\
get entangled in the world. \\
My child, do not partake \\
in suffering again and again. 

For\marginnote{2.1} happy dwell the sages, \textsanskrit{Vaḍḍha}, \\
unstirred, their doubts cut off, \\
cooled and tamed, \\
and free of defilements. 

\textsanskrit{Vaḍḍha},\marginnote{3.1} foster the path \\
that the hermits have walked, \\
for the attainment of vision, \\
and for making an end of suffering.” 

“Mother,\marginnote{4.1} you speak with such assurance \\
to me on this matter. \\
My dear mom, I can’t help thinking \\
that no entanglements are found in you.” 

“\textsanskrit{Vaḍḍha},\marginnote{5.1} not a jot or a skerrick \\
of entanglement is found in me \\
for any conditions at all, \\
whether low, high, or middling. 

All\marginnote{6.1} defilements are ended for me, \\
meditating and diligent. \\
I’ve attained the three knowledges \\
and fulfilled the Buddha’s instructions.” 

“Oh\marginnote{7.1} so excellent was the goad \\
my mother spurred me with! \\
Owing to her compassion, she spoke \\
verses on the ultimate goal. 

On\marginnote{8.1} hearing her words, \\
advised by my mother, \\
I was struck with righteous urgency \\
for the sake of finding sanctuary. 

Striving,\marginnote{9.1} resolute, \\
tireless all day and night, \\
urged on by my mother, \\
I realized supreme peace. 

%
\end{verse}

\scendsection{The Book of the Nines is finished. }

%
\addtocontents{toc}{\let\protect\contentsline\protect\nopagecontentsline}
\chapter*{The Book of the Elevens }
\addcontentsline{toc}{chapter}{\tocchapterline{The Book of the Elevens }}
\addtocontents{toc}{\let\protect\contentsline\protect\oldcontentsline}

%
\section*{{\suttatitleacronym Thig 10.1}{\suttatitletranslation Kisāgotamī }{\suttatitleroot Kisāgotamītherīgāthā}}
\addcontentsline{toc}{section}{\tocacronym{Thig 10.1} \toctranslation{Kisāgotamī } \tocroot{Kisāgotamītherīgāthā}}
\markboth{Kisāgotamī }{Kisāgotamītherīgāthā}
\extramarks{Thig 10.1}{Thig 10.1}

\begin{verse}%
“Pointing\marginnote{1.1} out how the world works, \\
the sages have praised good friendship. \\
Associating with good friends, \\
even a fool becomes astute. 

Associate\marginnote{2.1} with good people, \\
for that is how wisdom grows. \\
Should you associate with good people, \\
you would be freed from all suffering. 

And\marginnote{3.1} you would understand suffering, \\
its origin and cessation, \\
the eightfold path, \\
and so the four noble truths.” 

“‘A\marginnote{4.1} woman’s life is painful,’ \\
explained the Buddha, guide for those who wish to train, \\
‘and for a co-wife it’s especially so. \\
After giving birth just once, 

some\marginnote{5.1} women even cut their own throat, \\
while refined ladies take poison. \\
Being guilty of killing a person, \\
they undergo ruin both here and beyond.’” 

“I\marginnote{6.1} was on the road and about to give birth., \\
when I saw my husband dead. \\
I gave birth there on the road \\
before I’d reached my own house. 

My\marginnote{7.1} two children have died, \\
and on the road my husband lies dead—oh woe is me! \\
Mother, father, and brother \\
all burning up on the same pyre.” 

“Oh\marginnote{8.1} woe is you whose family is lost, \\
your suffering has no measure; \\
you have been shedding tears \\
for many thousands of lives.” 

“While\marginnote{9.1} staying in the charnel ground, \\
I saw my son’s flesh being eaten. \\
With my family destroyed, condemned by all, \\
and my husband dead, I realized the deathless. 

I’ve\marginnote{10.1} developed the noble eightfold path \\
leading to the deathless. \\
I’ve realized quenching, \\
as seen in the mirror of the Dhamma. 

I’ve\marginnote{11.1} plucked out the dart, \\
laid down the burden, and done what needed to be done.” \\
The senior nun \textsanskrit{Kisāgotamī}, \\
her mind released, said this. 

%
\end{verse}

\scendsection{The Book of the Elevens is finished. }

%
\addtocontents{toc}{\let\protect\contentsline\protect\nopagecontentsline}
\chapter*{The Book of the Twelves }
\addcontentsline{toc}{chapter}{\tocchapterline{The Book of the Twelves }}
\addtocontents{toc}{\let\protect\contentsline\protect\oldcontentsline}

%
\section*{{\suttatitleacronym Thig 11.1}{\suttatitletranslation Uppalavaṇṇā }{\suttatitleroot Uppalavaṇṇātherīgāthā}}
\addcontentsline{toc}{section}{\tocacronym{Thig 11.1} \toctranslation{Uppalavaṇṇā } \tocroot{Uppalavaṇṇātherīgāthā}}
\markboth{Uppalavaṇṇā }{Uppalavaṇṇātherīgāthā}
\extramarks{Thig 11.1}{Thig 11.1}

\begin{verse}%
“The\marginnote{1.1} two of us were co-wives, \\
though we were mother and daughter. \\
I was struck with a sense of urgency, \\
so astonishing and hair-raising! 

Curse\marginnote{2.1} those filthy sensual pleasures, \\
so nasty and thorny, \\
where we, both mother and daughter, \\
had to be co-wives together. 

Seeing\marginnote{3.1} the danger in sensual pleasures, \\
seeing renunciation as sanctuary, \\
I went forth in \textsanskrit{Rājagaha} \\
from the lay life to homelessness. 

I\marginnote{4.1} know my past lives; \\
my clairvoyance is clarified; \\
I comprehend the minds of others; \\
my clairaudience is purified; 

I've\marginnote{5.1} realized the psychic powers, \\
and attained the ending of defilements. \\
I’ve realized the six kinds of direct knowledge, \\
and fulfilled the Buddha’s instructions. 

I\marginnote{6.1} created a four-horsed chariot \\
using my psychic powers. \\
Then I bowed at the feet of the Buddha, \\
the glorious protector of the world.” 

“You’ve\marginnote{7.1} come to this sal tree all crowned with flowers, \\
and stand at its root all alone. \\
But you have no companion with you, \\
silly girl, aren’t you afraid of rascals?” 

“Even\marginnote{8.1} if 100,000 rascals like this \\
were to gang up, \\
I’d stir not a hair nor tremble. \\
What could you do to me all alone, \textsanskrit{Māra}? 

I’ll\marginnote{9.1} vanish, \\
or I’ll enter your belly; \\
I could stand between your eyebrows \\
and you still wouldn’t see me. 

I’m\marginnote{10.1} the master of my own mind, \\
I’ve developed the bases of psychic power well. \\
I’ve realized the six kinds of direct knowledge, \\
and fulfilled the Buddha’s instructions. 

Sensual\marginnote{11.1} pleasures are like swords and stakes; \\
the aggregates are their chopping block. \\
What you call sensual delight \\
is now no delight for me. 

Relishing\marginnote{12.1} is destroyed in every respect, \\
and the mass of darkness is shattered. \\
So know this, Wicked One: \\
you’re beaten, terminator!” 

%
\end{verse}

\scendsection{The Book of the Twelves is finished. }

%
\addtocontents{toc}{\let\protect\contentsline\protect\nopagecontentsline}
\chapter*{The Book of the Sixteens }
\addcontentsline{toc}{chapter}{\tocchapterline{The Book of the Sixteens }}
\addtocontents{toc}{\let\protect\contentsline\protect\oldcontentsline}

%
\section*{{\suttatitleacronym Thig 12.1}{\suttatitletranslation Puṇṇikā }{\suttatitleroot Puṇṇātherīgāthā}}
\addcontentsline{toc}{section}{\tocacronym{Thig 12.1} \toctranslation{Puṇṇikā } \tocroot{Puṇṇātherīgāthā}}
\markboth{Puṇṇikā }{Puṇṇātherīgāthā}
\extramarks{Thig 12.1}{Thig 12.1}

\begin{verse}%
“I\marginnote{1.1} used to be a water-carrier. Even when it was cold, \\
I would always plunge into the water, \\
afraid of my mistresses’ beatings, \\
harassed by fear of abuse and anger. 

Brahmin,\marginnote{2.1} what are you afraid of, \\
that you always plunge into the water, \\
your limbs trembling \\
in the freezing cold?” 

“Oh,\marginnote{3.1} but you already know, \\
Madam \textsanskrit{Puṇṇikā}, when you ask me: \\
I am doing good deeds, \\
to ward off the wickedness I have done. 

Whosoever\marginnote{4.1} young or old \\
performs a wicked deed, \\
by ablution in water they are \\
released from their wicked deed.” 

“Who\marginnote{5.1} on earth told you this, \\
one ignoramus to another: \\
‘Actually, by ablution in water one is \\
released from a wicked deed.’ 

Would\marginnote{6.1} not they all go to heaven, then: \\
all the frogs and the turtles, \\
gharials, crocodiles, \\
and other water-dwellers too? 

Butchers\marginnote{7.1} of sheep and pigs, \\
fishermen, animal trappers, \\
bandits, executioners, \\
and others of evil deeds: \\
by ablution in water they too would be \\
released from their wicked deeds. 

If\marginnote{8.1} these rivers washed away \\
the bad deeds of the past, \\
then they’d also wash off goodness, \\
and thereby you would be excluded. 

Brahmin,\marginnote{9.1} the thing that you are afraid of, \\
when you always plunge into the water, \\
do not do that very thing, \\
don’t let the cold harm your skin.” 

“I\marginnote{10.1} have been on the wrong path, \\
and you’ve guided me to the noble path. \\
Madam, I give to you \\
this ablution cloth.” 

“Keep\marginnote{11.1} the cloth for yourself, \\
I do not want it. \\
If you fear suffering, \\
if you don’t like suffering, 

then\marginnote{12.1} don’t do bad deeds \\
either openly or in secret. \\
If you should do a bad deed, \\
or you’re doing one now, 

you\marginnote{13.1} won’t be freed from suffering, \\
though you fly away and flee. \\
If you fear suffering, \\
if you don’t like suffering, 

go\marginnote{14.1} for refuge to the Buddha, the poised, \\
to his teaching and to the Sangha. \\
Undertake the precepts, \\
that will be good for you.” 

“I\marginnote{15.1} go for refuge to the Buddha, the poised, \\
to his teaching and to the Sangha. \\
I undertake the precepts, \\
that will be good for me. 

In\marginnote{16.1} the past I was related to \textsanskrit{Brahmā}, \\
today I truly am a brahmin! \\
I am master of the three knowledges, accomplished in wisdom, \\
I’m a scholar and a bathed initiate.” 

%
\end{verse}

\scendsection{The Book of the Sixteens is finished. }

%
\addtocontents{toc}{\let\protect\contentsline\protect\nopagecontentsline}
\chapter*{The Book of the Twenties }
\addcontentsline{toc}{chapter}{\tocchapterline{The Book of the Twenties }}
\addtocontents{toc}{\let\protect\contentsline\protect\oldcontentsline}

%
\section*{{\suttatitleacronym Thig 13.1}{\suttatitletranslation Ambapālī }{\suttatitleroot Ambapālītherīgāthā}}
\addcontentsline{toc}{section}{\tocacronym{Thig 13.1} \toctranslation{Ambapālī } \tocroot{Ambapālītherīgāthā}}
\markboth{Ambapālī }{Ambapālītherīgāthā}
\extramarks{Thig 13.1}{Thig 13.1}

\begin{verse}%
My\marginnote{1.1} hair was as black as bees, \\
graced with curly tips; \\
now old, it has become like hemp bark—\\
the word of the truthful one is confirmed. 

Crowned\marginnote{2.1} with flowers, \\
my head was as fragrant as a perfume box; \\
now old, it smells like dog fur—\\
the word of the truthful one is confirmed. 

My\marginnote{3.1} hair was as thick as a well-planted forest, \\
it shone, parted with brush and pins; \\
now old, it’s patchy and sparse—\\
the word of the truthful one is confirmed. 

With\marginnote{4.1} plaits of black and ribbons of gold, \\
it was so pretty, adorned with braids; \\
now old, my head’s gone bald—\\
the word of the truthful one is confirmed. 

My\marginnote{5.1} eyebrows used to look so nice, \\
like crescents painted by an artist; \\
now old, they droop with wrinkles—\\
the word of the truthful one is confirmed. 

My\marginnote{6.1} eyes shone brilliant as gems, \\
wide and deepest blue; \\
ruined by age, they shine no more—\\
the word of the truthful one is confirmed. 

My\marginnote{7.1} nose was like a perfect peak, \\
lovely in my bloom of youth; \\
now old, it’s shriveled like a pepper; \\
the word of the truthful one is confirmed. 

My\marginnote{8.1} ear-lobes were so pretty, \\
like lovingly crafted bracelets; \\
now old, they droop with wrinkles—\\
the word of the truthful one is confirmed. 

My\marginnote{9.1} teeth used to be so pretty, \\
bright as a jasmine flower; \\
now old, they’re broken and yellow—\\
the word of the truthful one is confirmed. 

My\marginnote{10.1} singing was sweet as a cuckoo \\
wandering in the forest groves; \\
now old, it’s patchy and croaking—\\
the word of the truthful one is confirmed. 

My\marginnote{11.1} neck used to be so pretty, \\
like a polished shell of conch; \\
now old, it’s bowed and bent—\\
the word of the truthful one is confirmed. 

My\marginnote{12.1} arms used to be so pretty, \\
like rounded cross-bars; \\
now old, they droop like a trumpet-flower tree—\\
the word of the truthful one is confirmed. 

My\marginnote{13.1} hands used to be so pretty, \\
adorned with lovely golden rings; \\
now old, they’re like red radishes—\\
the word of the truthful one is confirmed. 

My\marginnote{14.1} breasts used to be so pretty, \\
swelling, round, close, and high; \\
now they droop like water bags—\\
the word of the truthful one is confirmed. 

My\marginnote{15.1} body used to be so pretty, \\
like a polished slab of gold; \\
now it’s covered with fine wrinkles—\\
the word of the truthful one is confirmed. 

Both\marginnote{16.1} my thighs used to be so pretty, \\
like an elephant’s trunk; \\
now old, they’re like bamboo—\\
the word of the truthful one is confirmed. 

My\marginnote{17.1} calves used to be so pretty, \\
adorned with cute golden anklets; \\
now old, they’re like sesame sticks—\\
the word of the truthful one is confirmed. 

Both\marginnote{18.1} my feet used to be so pretty, \\
plump as if with cotton-wool; \\
now old, they’re cracked and wrinkly—\\
the word of the truthful one is confirmed. 

This\marginnote{19.1} bag of bones once was such, \\
but now it’s withered, home to so much pain; \\
like a house in decay with plaster crumbling—\\
the word of the truthful one is confirmed. 

%
\end{verse}

%
\section*{{\suttatitleacronym Thig 13.2}{\suttatitletranslation Rohinī }{\suttatitleroot Rohinītherīgāthā}}
\addcontentsline{toc}{section}{\tocacronym{Thig 13.2} \toctranslation{Rohinī } \tocroot{Rohinītherīgāthā}}
\markboth{Rohinī }{Rohinītherīgāthā}
\extramarks{Thig 13.2}{Thig 13.2}

\begin{verse}%
“You\marginnote{1.1} fell asleep saying ‘ascetics’; \\
you woke up saying ‘ascetics’; \\
you only praise ascetics, madam—\\
surely you’ll become an ascetic. 

You\marginnote{2.1} provide ascetics \\
with abundant food and drink. \\
I ask you now, \textsanskrit{Rohiṇī}: \\
why do you like ascetics? 

They\marginnote{3.1} don’t like to work, they’re lazy, \\
they live on charity; \\
always on the lookout, greedy for sweets—\\
so why do you like ascetics?” 

“Dad,\marginnote{4.1} for a long time now \\
you’ve questioned me about ascetics. \\
I shall extol for you \\
their wisdom, ethics, and vigor. 

They\marginnote{5.1} like to work, they’re not lazy; \\
by giving up greed and hate, \\
they do the best kind of work—\\
that’s why I like ascetics. 

As\marginnote{6.1} for the three roots of evil, \\
by pure deeds they shake them off. \\
They have given up all wickedness—\\
that’s why I like ascetics. 

Their\marginnote{7.1} bodily actions are pure; \\
their actions of speech likewise; \\
their actions of mind are pure—\\
that’s why I like ascetics. 

Immaculate\marginnote{8.1} as a conch-shell, \\
they’re pure inside and out, \\
full of bright qualities—\\
that’s why I like ascetics. 

They’re\marginnote{9.1} learned and memorize the teaching, \\
noble, living righteously, \\
teaching the text and its meaning: \\
that’s why I like ascetics. 

They’re\marginnote{10.1} learned and memorize the teaching, \\
noble, living righteously, \\
unified in mind, and mindful—\\
that’s why I like ascetics. 

Traveling\marginnote{11.1} afar, and mindful, \\
thoughtful in counsel, and stable, \\
they understand the end of suffering—\\
that’s why I like ascetics. 

When\marginnote{12.1} they leave a village, \\
they don’t look back with longing, \\
but proceed without concern—\\
that’s why I like ascetics. 

They\marginnote{13.1} hoard no goods in storerooms, \\
nor in pots or baskets. \\
They seek food prepared by others—\\
that’s why I like ascetics. 

They\marginnote{14.1} don’t receive silver, \\
or gold whether coined or uncoined; \\
feeding on whatever comes that day, \\
that’s why I like ascetics. 

They\marginnote{15.1} have gone forth from different families, \\
even different countries, \\
and yet they all love one another—\\
that’s why I like ascetics.” 

“Dear\marginnote{16.1} \textsanskrit{Rohinī}, it was truly for our benefit \\
that you were born in our family! \\
You have faith and such keen respect \\
for the Buddha, his teaching, and the Sangha. 

For\marginnote{17.1} you understand this \\
supreme field of merit. \\
These ascetics will henceforth \\
receive our religious donation, too. 

For\marginnote{18.1} there we will place our sacrifice, \\
and it shall be abundant.” \\
“If you fear suffering, \\
if you don’t like suffering, 

go\marginnote{19.1} for refuge to the Buddha, the poised, \\
to his teaching and to the Sangha. \\
Undertake the precepts, \\
that will be good for you.” 

“I\marginnote{20.1} go for refuge to the Buddha, the poised, \\
to his teaching and to the Sangha. \\
I undertake the precepts, \\
that will be good for me. 

In\marginnote{21.1} the past I was related to \textsanskrit{Brahmā}, \\
now I genuinely am a brahmin. \\
Possessing the three knowledges, I’m a genuine scholar, \\
I’m a knowledge master, a bathed initiate.” 

%
\end{verse}

%
\section*{{\suttatitleacronym Thig 13.3}{\suttatitletranslation Cāpā }{\suttatitleroot Cāpātherīgāthā}}
\addcontentsline{toc}{section}{\tocacronym{Thig 13.3} \toctranslation{Cāpā } \tocroot{Cāpātherīgāthā}}
\markboth{Cāpā }{Cāpātherīgāthā}
\extramarks{Thig 13.3}{Thig 13.3}

\begin{verse}%
“Once\marginnote{1.1} I carried a hermit’s staff, \\
but these days I hunt deer. \\
My desires have made me unable to cross \\
from the awful marsh to the far shore. 

Thinking\marginnote{2.1} me so in love with her, \\
\textsanskrit{Cāpā} kept our son happy. \\
Having cut \textsanskrit{Cāpā}’s bond, \\
I’ll go forth once again.” 

“Don’t\marginnote{3.1} be mad at me, great hero! \\
Don’t be mad at me, great sage! \\
If you’re mired in anger you can’t stay pure, \\
let alone practice austerities.” 

“I’m\marginnote{4.1} going to leave \textsanskrit{Nālā}! \\
For who’d stay here at \textsanskrit{Nālā}! \\
With their figures, the women trap \\
ascetics who live righteously.” 

“Please,\marginnote{5.1} \textsanskrit{Kāḷa}, come back to me. \\
Enjoy pleasures like you did before. \\
I’ll be under your control, \\
along with any relatives I have.” 

“\textsanskrit{Cāpā},\marginnote{6.1} if even a quarter \\
of what you say were true, \\
it would be a splendid thing \\
for a man in love with you!” 

“\textsanskrit{Kāḷa},\marginnote{7.1} I am like a sprouting iris \\
flowering on a mountain top, \\
like a blossoming pomegranate, \\
like a trumpet-flower tree on an isle; 

my\marginnote{8.1} limbs are anointed with yellow sandalwood, \\
and I wear the finest \textsanskrit{Kāsi} cloth: \\
when I am so very beautiful, \\
how can you abandon me and leave?” 

“You’re\marginnote{9.1} like a fowler \\
who wants to catch a bird; \\
but you won’t trap me \\
with your captivating form.” 

“But\marginnote{10.1} this child, my fruit, \\
was begotten by you, \textsanskrit{Kāḷa}. \\
When I have this child, \\
how can you abandon me and leave?” 

“The\marginnote{11.1} wise give up \\
children, family, and wealth. \\
Great heroes go forth \\
like elephants breaking their bonds.” 

“Now,\marginnote{12.1} this son of yours: \\
I’ll strike him to the ground right here, \\
with a stick or with a knife! \\
Grieving your son, you will not leave.” 

“Even\marginnote{13.1} if you feed our son \\
to jackals and dogs, \\
I’d never return again, you bitch, \\
not even for the child’s sake.” 

“Well\marginnote{14.1} then, sir, tell me, \\
where will you go, \textsanskrit{Kāḷa}? \\
To what village or town, \\
city or capital?” 

“Last\marginnote{15.1} time we had followers, \\
we weren’t ascetics, we just thought we were. \\
We wandered from village to village, \\
to cities and capitals. 

But\marginnote{16.1} now the Blessed One, the Buddha, \\
on the bank of the \textsanskrit{Nerañjara} River, \\
teaches the Dhamma so that living creatures \\
may abandon all suffering. \\
I shall go to his presence, \\
he shall be my Teacher.” 

“Now\marginnote{17.1} please convey my respects \\
to the supreme protector of the world. \\
Circling him to your right, \\
dedicate my religious donation.” 

“This\marginnote{18.1} is the proper thing to do, \\
just as you have said to me. \\
I’ll convey your respects \\
to the supreme protector of the world. \\
Circling him to my right, \\
I’ll dedicate your religious donation.” 

Then\marginnote{19.1} \textsanskrit{Kāḷa} set out \\
for the bank of the \textsanskrit{Nerañjara} River. \\
He saw the Awakened One \\
teaching the deathless state: 

suffering,\marginnote{20.1} suffering’s origin, \\
suffering’s transcendence, \\
and the noble eightfold path \\
that leads to the stilling of suffering. 

He\marginnote{21.1} paid homage at his feet, \\
circling him to his right, \\
and conveyed \textsanskrit{Cāpā}’s dedication; \\
then he went forth to homelessness. \\
He attained the three knowledges, \\
and fulfilled the Buddha’s instructions. 

%
\end{verse}

%
\section*{{\suttatitleacronym Thig 13.4}{\suttatitletranslation Sundarī }{\suttatitleroot Sundarītherīgāthā}}
\addcontentsline{toc}{section}{\tocacronym{Thig 13.4} \toctranslation{Sundarī } \tocroot{Sundarītherīgāthā}}
\markboth{Sundarī }{Sundarītherīgāthā}
\extramarks{Thig 13.4}{Thig 13.4}

\begin{verse}%
“Before,\marginnote{1.1} when your children passed away, \\
you would expose them to be eaten. \\
All day and all night \\
you’d be racked with despair. 

Today,\marginnote{2.1} brahmin lady, you have exposed \\
seven children in all to be eaten; \\
\textsanskrit{Vāseṭṭhī}, what is the reason why \\
you’re not so filled with despair?” 

“Many\marginnote{3.1} hundreds of children, \\
hundreds of family circles, \\
both mine and yours, brahmin, \\
have been eaten in the past. 

Having\marginnote{4.1} known the escape \\
from rebirth and death \\
I neither grieve nor lament, \\
nor do I despair.” 

“Wow,\marginnote{5.1} \textsanskrit{Vaseṭṭhī}, the words you speak \\
really are amazing! \\
Whose teaching did you understand \\
that you say these things?” 

“Brahmin,\marginnote{6.1} the Awakened One \\
at the city of \textsanskrit{Mithilā}, \\
teaches the Dhamma so that living creatures \\
may abandon all suffering. 

After\marginnote{7.1} hearing the perfected one’s teaching, \\
brahmin, which is free of all attachments, \\
having understood the true teaching there, \\
I’ve swept away grief for children.” 

“I\marginnote{8.1} too shall go \\
to the city of \textsanskrit{Mithilā}. \\
Hopefully the Buddha may release me \\
from all suffering.” 

The\marginnote{9.1} brahmin saw the Buddha, \\
liberated, free of attachments. \\
He taught him the Dhamma, \\
the sage gone beyond suffering: 

suffering,\marginnote{10.1} suffering’s origin, \\
suffering’s transcendence, \\
and the noble eightfold path \\
that leads to the stilling of suffering. 

Having\marginnote{11.1} understood the true teaching there, \\
he chose to go forth. \\
Three days later \\
\textsanskrit{Sujāta} realized the three knowledges. 

“Please,\marginnote{12.1} charioteer, go; \\
take back this carriage. \\
Bidding my brahmin lady good health, say: \\
‘The brahmin has now gone forth. \\
After three days, \\
\textsanskrit{Sujāta} realized the three knowledges.’” 

Then\marginnote{13.1} taking the carriage, \\
along with a thousand coins, the charioteer \\
bade the brahmin lady good health, and said: \\
“The brahmin has now gone forth. \\
After three days, \\
\textsanskrit{Sujāta} realized the three knowledges.” 

Hearing\marginnote{14.1} that the brahmin had the three knowledges, the lady replied: \\
“I present to you this horse and carriage, \\
O charioteer, along with 1000 coins, \\
and a full bowl as a gift.” 

“Keep\marginnote{15.1} the horse and carriage, lady, \\
along with the thousand coins. \\
I too shall go forth in his presence, \\
the one of such splendid wisdom.” 

“Elephants,\marginnote{16.1} cattle, jeweled earrings, \\
such opulent domestic wealth: \\
having given it up, your father went forth, \\
enjoy these riches \textsanskrit{Sundarī}, \\
you are the family heir.” 

“Elephants,\marginnote{17.1} cattle, jeweled earrings, \\
such delightful domestic wealth: \\
having given it up, my father went forth, \\
racked by grief for his son. \\
I too shall go forth, \\
racked by grief for my brother.” 

“\textsanskrit{Sundarī},\marginnote{18.1} may the wish you desire \\
come true. \\
Leftovers as gleanings, \\
and cast-off rags as robes—\\
make do with these, \\
free of defilements regarding the next life.” 

“Ma’am,\marginnote{19.1} while I am still a trainee nun, \\
my clairvoyance is clarified; \\
I know my past lives, \\
the places I used to live. 

Relying\marginnote{20.1} on a fine lady like you, \\
a senior nun who beautifies the Sangha, \\
I’ve attained the three knowledges, \\
and fulfilled the Buddha’s instructions. 

Give\marginnote{21.1} me permission ma’am, \\
I wish to go to \textsanskrit{Sāvatthī}, \\
where I shall roar my lion’s roar \\
before the best of Buddhas.” 

“\textsanskrit{Sundarī},\marginnote{22.1} see the Teacher! \\
Golden colored, golden skinned, \\
tamer of the untamed, \\
the Awakened One who fears nothing from any quarter.” 

“See\marginnote{23.1} \textsanskrit{Sundarī} coming, \\
liberated, free of attachments. \\
desireless, detached, \\
her task completed, without defilements.” 

“Having\marginnote{24.1} set forth from \textsanskrit{Bārāṇasī} \\
and come to your presence, great hero, \\
your disciple \textsanskrit{Sundarī} \\
bows at your feet. 

You\marginnote{25.1} are the Buddha, you are the Teacher, \\
I am your rightful daughter, brahmin, \\
born of your mouth. \\
I’ve completed the task and am free of defilements.” 

“Then\marginnote{26.1} welcome, good lady, \\
you’re by no means unwelcome. \\
For this is how the tamed come \\
bowing at the Teacher’s feet; \\
desireless, detached, \\
the task completed, without defilements.” 

%
\end{verse}

%
\section*{{\suttatitleacronym Thig 13.5}{\suttatitletranslation Subhā, the Smith’s Daughter }{\suttatitleroot Subhākammāradhītutherīgāthā}}
\addcontentsline{toc}{section}{\tocacronym{Thig 13.5} \toctranslation{Subhā, the Smith’s Daughter } \tocroot{Subhākammāradhītutherīgāthā}}
\markboth{Subhā, the Smith’s Daughter }{Subhākammāradhītutherīgāthā}
\extramarks{Thig 13.5}{Thig 13.5}

\begin{verse}%
“I\marginnote{1.1} was so young, my clothes so fresh, \\
at that time I heard the teaching. \\
Being diligent, \\
I comprehended the truth; 

and\marginnote{2.1} then I became profoundly dispassionate \\
towards all sensual pleasures. \\
Seeing fear in identity, \\
I longed for renunciation. 

Giving\marginnote{3.1} up my family circle, \\
bonded servants and workers, \\
and my flourishing villages and lands, \\
so delightful and pleasant, 

I\marginnote{4.1} went forth; \\
all that is no small wealth. \\
Now that I’ve gone forth in faith like this, \\
in the true teaching so well proclaimed, 

since\marginnote{5.1} I desire to have nothing, \\
it would not be appropriate \\
to take back gold and money, \\
having already got rid of them. 

Money\marginnote{6.1} or gold \\
doesn’t lead to peace and awakening. \\
It doesn’t befit an ascetic, \\
it’s not the wealth of the noble ones; 

it’s\marginnote{7.1} just greed and vanity, \\
confusion and growing decadence, \\
dubious, troublesome—\\
there is nothing lasting there. 

Depraved\marginnote{8.1} and heedless, \\
unenlightened folk, their hearts corrupt, \\
fight each other, \\
creating conflict. 

Killing,\marginnote{9.1} caging, misery, \\
loss, grief, and lamentation; \\
those sunk in sensual pleasures \\
see many disastrous things. 

My\marginnote{10.1} family, why do you urge me on \\
to pleasures, as if you were my enemies? \\
You know I’ve gone forth, \\
seeing fear in sensual pleasures. 

It’s\marginnote{11.1} not due to gold, coined or uncoined, \\
that defilements come to an end. \\
Sensual pleasures are enemies and murderers, \\
hostile forces that bind you to thorns. 

My\marginnote{12.1} family, why do you urge me on \\
to pleasures, as if you were my enemies? \\
You know I’ve gone forth, \\
shaven, wrapped in my outer robe. 

Leftovers\marginnote{13.1} as gleanings, \\
and cast-off rags as robes—\\
that’s what’s fitting for me, \\
the essentials of the homeless life. 

Great\marginnote{14.1} hermits expel sensual pleasures, \\
both human and divine. \\
Safe in their sanctuary, they are freed, \\
having found unshakable happiness. 

May\marginnote{15.1} I not encounter sensual pleasures, \\
for no shelter is found in them. \\
Sensual pleasures are enemies and murderers, \\
as painful as a bonfire. 

Greed\marginnote{16.1} is an obstacle, a threat, \\
full of anguish and thorns; \\
it is out of balance, \\
a great gateway to confusion. 

Hazardous\marginnote{17.1} and terrifying, \\
sensual pleasures are like a snake’s head, \\
where fools delight, \\
the blind ordinary folk. 

Stuck\marginnote{18.1} in the swamp of sensuality, \\
there are so many ignorant in the world. \\
They know nothing of the end \\
of rebirth and death. 

Because\marginnote{19.1} of sensual pleasures, \\
people jump right on to the path that goes to a bad place. \\
So many walk the path \\
that brings disease onto themselves. 

That’s\marginnote{20.1} how sensual pleasures create enemies; \\
they are so tormenting, so corrupting, \\
trapping beings with the world’s material delights, \\
they are nothing less than the bonds of death. 

Maddening,\marginnote{21.1} enticing, \\
sensual pleasures derange the mind. \\
They’re a snare laid by \textsanskrit{Māra} \\
for the corruption of beings. 

Sensual\marginnote{22.1} pleasures are infinitely dangerous, \\
they’re full of suffering, a terrible poison; \\
offering little gratification, they’re makers of strife, \\
withering bright qualities away. 

Since\marginnote{23.1} I’ve created so much ruination \\
because of sensual pleasures, \\
I will not relapse to them again, \\
but will always delight in quenching. 

Fighting\marginnote{24.1} against sensual pleasures, \\
longing for that cool state, \\
I shall meditate diligently \\
for the ending of all fetters. 

Sorrowless,\marginnote{25.1} stainless, secure: \\
I’ll follow that path, \\
the straight noble eightfold way \\
by which the hermits have crossed over.” 

“Look\marginnote{26.1} at this: \textsanskrit{Subhā} the smith’s daughter, \\
standing firm in the teaching. \\
She has entered the imperturbable state, \\
meditating at the root of a tree. 

It’s\marginnote{27.1} just eight days since she went forth, \\
full of faith in the beautiful teaching. \\
Guided by \textsanskrit{Uppalavaṇṇā}, \\
she is master of the three knowledges, conqueror of death. 

This\marginnote{28.1} one is freed from slavery and debt, \\
a nun with faculties developed. \\
Detached from all attachments, \\
she has completed the task and is free of defilements.” 

Thus\marginnote{29.1} did Sakka, lord of all creatures, \\
along with a host of gods, \\
having come by their psychic powers, \\
honor \textsanskrit{Subhā}, the smith’s daughter. 

%
\end{verse}

\scendsection{The Book of the Twenties is finished. }

%
\addtocontents{toc}{\let\protect\contentsline\protect\nopagecontentsline}
\chapter*{The Book of the Thirties }
\addcontentsline{toc}{chapter}{\tocchapterline{The Book of the Thirties }}
\addtocontents{toc}{\let\protect\contentsline\protect\oldcontentsline}

%
\section*{{\suttatitleacronym Thig 14.1}{\suttatitletranslation Subhā of Jīvaka’s Mango Grove }{\suttatitleroot Subhājīvakambavanikātherīgāthā}}
\addcontentsline{toc}{section}{\tocacronym{Thig 14.1} \toctranslation{Subhā of Jīvaka’s Mango Grove } \tocroot{Subhājīvakambavanikātherīgāthā}}
\markboth{Subhā of Jīvaka’s Mango Grove }{Subhājīvakambavanikātherīgāthā}
\extramarks{Thig 14.1}{Thig 14.1}

\begin{verse}%
Going\marginnote{1.1} to the lovely mango grove \\
of \textsanskrit{Jīvaka}, the nun \textsanskrit{Subhā} \\
was held up by a rascal. \\
\textsanskrit{Subhā} said this to him: 

“What\marginnote{2.1} harm have I done to you, \\
that you stand in my way? \\
Sir, it’s not proper that a man \\
should touch a woman gone forth. 

This\marginnote{3.1} training was taught by the Holy One, \\
it is a serious matter in my teacher’s instructions. \\
I am pure and rid of blemishes, \\
so why do you stand in my way? 

One\marginnote{4.1} whose mind is sullied against one unsullied; \\
one who is lustful against one free of lust; \\
unblemished, my heart is freed in every respect, \\
so why do you stand in my way?” 

“You’re\marginnote{5.1} young and flawless—\\
what will going-forth do for you? \\
Throw away the ocher robe, \\
come and play in the blossom grove. 

Everywhere,\marginnote{6.1} the scent of pollen wafts sweet, \\
born of the flowering woods. \\
The start of spring is a happy time—\\
come and play in the blossom grove. 

And\marginnote{7.1} trees crested with flowers \\
cry out, as it were, in the breeze. \\
But what kind of fun will you have \\
if you plunge into the woods all alone? 

Frequented\marginnote{8.1} by packs of predators, \\
and she-elephants aroused by rutting bulls; \\
you wish to go without a friend \\
to the deserted, awe-inspiring forest. 

Like\marginnote{9.1} a shining doll of gold, \\
like a nymph wandering in a park of colorful vines, \\
your matchless beauty will shine \\
in graceful clothes of exquisite muslin. 

I’ll\marginnote{10.1} be under your sway, \\
if we are to stay in the forest. \\
I love no creature more than you, \\
O pixie with such bashful eyes. 

Were\marginnote{11.1} you to take up my invitation—\\
‘Come, be happy, and live in a house’—\\
you’ll stay in a longhouse sheltered from wind; \\
let the ladies look to your needs. 

Dressed\marginnote{12.1} in exquisite muslin, \\
put on your garlands and your cosmetics. \\
I’ll make all sorts of adornments for you, \\
of gold and gems and pearls. 

Climb\marginnote{13.1} onto a costly bed, \\
its coverlet so clean and nice, \\
with a new woolen mattress, \\
so fragrant, sprinkled with sandalwood. 

As\marginnote{14.1} a blue lily risen from the water \\
remains untouched by men, \\
so too, O chaste and holy lady, \\
your limbs grow old unshared.” 

“This\marginnote{15.1} carcass is full of putrefaction, it swells \\
the charnel ground, for its nature is to fall apart. \\
What do you think is so essential in it \\
that you stare at me so crazily?” 

“Your\marginnote{16.1} eyes are like those of a doe, \\
or a pixie in the mountains; \\
seeing them, \\
my sensual desire grows all the more. 

Set\marginnote{17.1} in your flawless face of golden sheen, \\
your eyes compare to a blue lily’s bud; \\
seeing them, \\
my sensual excitement grows all the more. 

Though\marginnote{18.1} you may wander far, I’ll still think of you, \\
with your lashes so long, and your vision so clear. \\
I love no eyes more than yours, \\
O pixie with such bashful eyes.” 

“You’re\marginnote{19.1} setting out on the wrong road! \\
You’re looking to take the moon for your toy! \\
You’re trying to leap over Mount Meru! \\
You, who are hunting a child of the Buddha! 

For\marginnote{20.1} in this world with all its gods, \\
there will be no more lust anywhere in me. \\
I don’t even know what kind it could be, \\
it’s been smashed root and all by the path. 

Cast\marginnote{21.1} out like sparks from fiery coals, \\
it’s worth no more than a bowl of poison. \\
I don’t even see what kind it could be, \\
it’s been smashed root and all by the path. 

Well\marginnote{22.1} may you try to seduce the type of lady \\
who has not reflected on these things, \\
or who has never attended the Teacher: \\
but \emph{this} is a lady who knows—now you’re in trouble! 

No\marginnote{23.1} matter if I am reviled or praised, \\
or feel pleasure or pain: I stay mindful. \\
Knowing that conditions are ugly, \\
my mind clings to nothing. 

I\marginnote{24.1} am a disciple of the Holy One, \\
riding in the carriage of the eightfold path. \\
The dart pulled out, free of defilements, \\
I’m happy to have reached an empty place. 

I’ve\marginnote{25.1} seen brightly painted \\
dolls and wooden puppets, \\
tied to sticks and strings, \\
and made to dance in many ways. 

But\marginnote{26.1} when the sticks and strings are taken off—\\
loosed, disassembled, dismantled, \\
irrecoverable, stripped to parts—\\
on what could the mind be fixed? 

That’s\marginnote{27.1} what my body is really like, \\
without those things it can’t go on. \\
This being so, \\
on what could the mind be fixed? 

It’s\marginnote{28.1} like when you see a mural on a wall, \\
painted with orpiment, \\
and your vision gets confused, \\
falsely perceiving that it is a person. 

Though\marginnote{29.1} it’s as worthless as a magic trick, \\
or a golden tree seen in a dream, \\
you blindly chase what is hollow, \\
like a puppet show among the people. 

An\marginnote{30.1} eye is just a ball in a socket, \\
with a pupil in the middle, and tears, \\
and mucus comes from there as well, \\
and so different eye-parts are lumped all together.” 

The\marginnote{31.1} pretty lady ripped out her eye. \\
With no attachment in her mind at all, she said: \\
“Come now, take this eye,” \\
and gave it to the man right then. 

And\marginnote{32.1} at that moment he lost his lust, \\
and asked for her forgiveness: \\
“May you be well, O chaste and holy lady; \\
such a thing will not happen again. 

Attacking\marginnote{33.1} a person such as this \\
is like holding on to a blazing fire, \\
or grabbing a deadly viper! \\
May you be well, please forgive me.” 

When\marginnote{34.1} that nun was released \\
she went to the presence of the excellent Buddha. \\
Seeing the one with excellent marks of merit, \\
her eye became just as it was before. 

%
\end{verse}

\scendsection{The Book of the Thirties is finished. }

%
\addtocontents{toc}{\let\protect\contentsline\protect\nopagecontentsline}
\chapter*{The Book of the Forties }
\addcontentsline{toc}{chapter}{\tocchapterline{The Book of the Forties }}
\addtocontents{toc}{\let\protect\contentsline\protect\oldcontentsline}

%
\section*{{\suttatitleacronym Thig 15.1}{\suttatitletranslation Isidāsī }{\suttatitleroot Isidāsītherīgāthā}}
\addcontentsline{toc}{section}{\tocacronym{Thig 15.1} \toctranslation{Isidāsī } \tocroot{Isidāsītherīgāthā}}
\markboth{Isidāsī }{Isidāsītherīgāthā}
\extramarks{Thig 15.1}{Thig 15.1}

\begin{verse}%
In\marginnote{1.1} \textsanskrit{Pāṭaliputta}, the cream of the world, \\
the city named for a flower, \\
there were two nuns from the Sakyan clan, \\
both of them ladies of quality. 

One\marginnote{2.1} was named \textsanskrit{Isidāsī}, the second \textsanskrit{Bodhī}. \\
They both were accomplished in ethics, \\
lovers of meditation and chanting, \\
learned, crushing corruptions. 

They\marginnote{3.1} wandered for alms and had their meal. \\
When they had washed their bowls, \\
they sat happily in a private place \\
and started a conversation. 

“You’re\marginnote{4.1} so lovely, Venerable \textsanskrit{Isidāsī}, \\
your youth has not yet faded. \\
What problem did you see that made you \\
dedicate your life to renunciation?” 

Being\marginnote{5.1} pressed like this in private, \\
\textsanskrit{Isidāsī}, skilled in teaching Dhamma, \\
voiced the following words. \\
“\textsanskrit{Bodhī}, hear how I went forth. 

In\marginnote{6.1} the fine town of \textsanskrit{Ujjenī}, \\
my father was a financier, a good and moral man. \\
I was his only daughter, \\
dear, beloved, and cherished. 

Then\marginnote{7.1} some suitors came for me \\
from the top family of \textsanskrit{Sāketa}. \\
They were sent by a financier abounding in wealth, \\
to whom my father then gave me as daughter-in-law. 

Come\marginnote{8.1} morning and come night, \\
I bowed with my head to the feet \\
of my father and mother-in-law, \\
just as I had been told. 

Whenever\marginnote{9.1} I saw my husband’s sisters, \\
his brothers, his servants, \\
or even he, my one and only, \\
I nervously gave them a seat. 

Whatever\marginnote{10.1} they wanted—food and drink, \\
treats, or whatever was in the cupboard—\\
I brought out and offered to them, \\
ensuring each got what was fitting. 

Having\marginnote{11.1} risen bright and early, \\
I approached the main house, \\
washed my hands and feet, \\
and went to my husband with joined palms. 

Taking\marginnote{12.1} a comb, adornments, \\
eyeshadow, and a mirror, \\
I myself did the makeup for my husband, \\
as if I were his beautician. 

I\marginnote{13.1} myself cooked the rice; \\
I myself washed the pots. \\
I looked after my husband \\
like a mother her only child. 

Thus\marginnote{14.1} I showed my devotion to him, \\
a loving, virtuous, and humble servant, \\
getting up early, and working tirelessly: \\
yet still my husband did me wrong. 

He\marginnote{15.1} said to his mother and father: \\
‘I’ll take my leave and go, \\
I can’t stand to live together with \textsanskrit{Isidāsī} \\
staying in the same house.’ 

‘Son,\marginnote{16.1} don’t speak like this! \\
\textsanskrit{Isidāsī} is astute and competent, \\
she gets up early and works tirelessly, \\
son, why doesn’t she please you?’ 

‘She\marginnote{17.1} hasn’t done anything to hurt me, \\
but I just can’t stand to live with her. \\
As far as I’m concerned, she’s just horrible. \\
I’ve had enough, I’ll take my leave and go.’ 

When\marginnote{18.1} they heard his words, \\
my father-in-law and mother-in-law asked me: \\
‘What did you do wrong? \\
Tell us honestly, have no fear.’ 

‘I’ve\marginnote{19.1} done nothing wrong, \\
I haven’t hurt him, or said anything bad. \\
What can I possibly do, \\
when my husband finds me so hateful?’ 

They\marginnote{20.1} led me back to my father’s home, \\
distraught, overcome with suffering, and said: \\
‘By caring for our son, \\
we’ve lost her, so lovely and lucky!’ 

Next\marginnote{21.1} my dad gave me to the household \\
of a second wealthy family-man. \\
For this he got half the bride-price \\
of that which the financier paid. 

In\marginnote{22.1} his house I also lived a month, \\
before he too wanted me gone; \\
though I served him like a slave, \\
virtuous and doing no wrong. 

My\marginnote{23.1} father then spoke to a beggar for alms, \\
a tamer of others and of himself: \\
‘Be my son-in-law; \\
set aside your rags and bowl.’ 

He\marginnote{24.1} stayed a fortnight before he said to my dad: \\
‘Give me back my rag robes, \\
my bowl, and my cup—\\
I’ll wander begging for alms again.’ 

So\marginnote{25.1} then my mum and my dad \\
and my whole group of relatives said: \\
‘What has not been done for you here? \\
Quickly, tell us what we can do for you!’ 

When\marginnote{26.1} they spoke to him like this he said, \\
‘If I can make do for myself, that is enough. \\
I can’t stand to live together with \textsanskrit{Isidāsī} \\
staying in the same house.’ 

Released,\marginnote{27.1} he left. \\
But I sat all alone contemplating: \\
‘Having taken my leave, I’ll go, \\
either to die or to go forth.’ 

But\marginnote{28.1} then the venerable lady \textsanskrit{Jinadattā}, \\
learned and virtuous, \\
who had memorized the monastic law, \\
came to my dad’s house in search of alms. 

When\marginnote{29.1} I saw her, \\
I got up from my seat and prepared it for her. \\
When she had taken her seat, \\
I honored her feet and offered her a meal, 

satisfying\marginnote{30.1} her with food and drink, \\
treats, or whatever was in the cupboard. \\
Then I said: \\
‘Ma’am, I wish to go forth!’ 

But\marginnote{31.1} my dad said to me: \\
‘Child, practice Dhamma right here! \\
Satisfy ascetics and twice-born brahmins \\
with food and drink.’ 

Then\marginnote{32.1} I said to my dad, \\
crying, my joined palms raised to him: \\
‘I’ve done bad things in the past; \\
I shall wear that bad deed away.’ 

And\marginnote{33.1} my dad said to me: \\
‘May you attain awakening, the highest state, \\
and may you find the extinguishment \\
that was realized by the best of men!’ 

I\marginnote{34.1} bowed down to my mother and father, \\
and my whole group of relatives; \\
and then, seven days after going forth, \\
I realized the three knowledges. 

I\marginnote{35.1} know my last seven lives; \\
I shall relate to you the deeds \\
of which this life is the fruit and result: \\
focus your whole mind on that. 

In\marginnote{36.1} the city of Erakacca \\
I was a goldsmith with lots of money. \\
Drunk on the pride of youth, \\
I had sex with someone else’s wife. 

Having\marginnote{37.1} passed away from there, \\
I burned in hell for a long time. \\
Rising up from there \\
I was conceived in a monkey’s womb. 

When\marginnote{38.1} I was only seven days old, \\
I was castrated by the monkey chief. \\
This was the fruit of that deed, \\
because of adultery with another’s wife. 

Having\marginnote{39.1} passed away from there, \\
passing away in Sindhava grove, \\
I was conceived in the womb \\
of a lame, one-eyed she-goat. 

I\marginnote{40.1} carried children on my back for twelve years, \\
and all the while I was castrated, \\
worm-eaten, and tail-less, \\
because of adultery with another’s wife. 

Having\marginnote{41.1} passed away from there, \\
I was reborn in a cow \\
owned by a cattle merchant. \\
A red calf, castrated, for twelve months 

I\marginnote{42.1} drew a big plow. \\
I shouldered a cart, \\
blind, tail-less, feeble, \\
because of adultery with another’s wife. 

Having\marginnote{43.1} passed away from there, \\
I was born of a slave in the street, \\
with neither male nor female parts, \\
because of adultery with another’s wife. 

I\marginnote{44.1} died at thirty years of age, \\
and was reborn as a girl in a carter’s family. \\
We were poor, of little wealth, \\
greatly oppressed by creditors. 

Because\marginnote{45.1} of the huge interest we owed, \\
I was dragged away screaming, \\
taken by force from the family home \\
by a caravan leader. 

When\marginnote{46.1} I was sixteen years old, \\
his son named \textsanskrit{Giridāsa}, \\
seeing that I was a girl of marriageable age, \\
took me as his wife. 

He\marginnote{47.1} also had another wife, \\
a virtuous and well-known lady of quality, \\
faithful to her husband; \\
yet I stirred up resentment in her. 

As\marginnote{48.1} the fruit of that deed, \\
they abandoned me and left, \\
though I served them like a slave. \\
Now I’ve made an end to this as well.” 

%
\end{verse}

\scendsection{The Book of the Forties is finished. }

%
\addtocontents{toc}{\let\protect\contentsline\protect\nopagecontentsline}
\chapter*{The Great Book}
\addcontentsline{toc}{chapter}{\tocchapterline{The Great Book}}
\addtocontents{toc}{\let\protect\contentsline\protect\oldcontentsline}

%
\section*{{\suttatitleacronym Thig 16.1}{\suttatitletranslation Sumedhā }{\suttatitleroot Sumedhātherīgāthā}}
\addcontentsline{toc}{section}{\tocacronym{Thig 16.1} \toctranslation{Sumedhā } \tocroot{Sumedhātherīgāthā}}
\markboth{Sumedhā }{Sumedhātherīgāthā}
\extramarks{Thig 16.1}{Thig 16.1}

\begin{verse}%
In\marginnote{1.1} \textsanskrit{Mantāvatī} city, \textsanskrit{Sumedhā}, \\
the daughter of King \textsanskrit{Koñca}’s chief queen, \\
was converted by those \\
who practice the Buddha’s teaching. 

She\marginnote{2.1} was virtuous, a brilliant speaker, \\
learned, and trained in the Buddha’s instructions. \\
She went up to her mother and father and said: \\
“Pay heed, both of you! 

I\marginnote{3.1} delight in extinguishment! \\
No life is eternal, not even that of the gods; \\
what then of sensual pleasures, so hollow, \\
offering little gratification and much anguish. 

Sensual\marginnote{4.1} pleasures are bitter as the venom of a snake, \\
yet fools are infatuated by them. \\
Sent to hell for a very long time, \\
they are beaten and tortured. 

Those\marginnote{5.1} who grow in wickedness \\
always sorrow in the underworld due to their own bad deeds. \\
They’re fools, unrestrained in body, \\
mind, and speech. 

Those\marginnote{6.1} witless, senseless fools, \\
obstructed by the origin of suffering, \\
are ignorant, not understanding the noble truths \\
when they are being taught. 

Most\marginnote{7.1} people, mum, ignorant of the truths \\
taught by the excellent Buddha, \\
look forward to the next life, \\
longing for rebirth among the gods. 

Yet\marginnote{8.1} even rebirth among the gods \\
in an impermanent state is not eternal. \\
But fools are not scared \\
of being reborn time and again. 

Four\marginnote{9.1} lower realms and two other realms \\
may be gained somehow or other. \\
But for those who end up in a lower realm, \\
there is no way to go forth in the hells. 

May\marginnote{10.1} you both grant me permission to go forth \\
in the dispensation of him of the ten powers. \\
Living at ease, I shall apply myself \\
to giving up rebirth and death. 

What’s\marginnote{11.1} the point in hope, in a new life, \\
in this useless, hollow body? \\
Grant me permission, I shall go forth \\
to make an end of craving for a new life. 

A\marginnote{12.1} Buddha has arisen, the time has come, \\
the unlucky moment has passed. \\
As long as I live I’ll never betray \\
my ethical precepts or my celibate path.” 

Then\marginnote{13.1} \textsanskrit{Sumedhā} said to her parents: \\
“So long as I remain a lay person, \\
I’ll refuse to eat any food, \\
until I’ve fallen under the sway of death.” 

Upset,\marginnote{14.1} her mother burst into tears, \\
while her father, though grieved, \\
tried his best to persuade her \\
as she lay collapsed on the longhouse roof. 

“Get\marginnote{15.1} up child, why do you grieve so? \\
You’re already betrothed to be married! \\
King \textsanskrit{Anīkaratta} the handsome \\
is in \textsanskrit{Vāraṇavatī}: he is your betrothed. 

You\marginnote{16.1} shall be the chief queen, \\
wife of King \textsanskrit{Anīkaratta}. \\
Ethical precepts, the celibate path—\\
going forth is hard to do, my child. 

As\marginnote{17.1} a royal there is command, wealth, authority, \\
and the happiness of possessions. \\
Enjoy sensual pleasures while you’re still young! \\
Let your wedding take place, my child!” 

Then\marginnote{18.1} \textsanskrit{Sumedhā} said to him: \\
“Let this not come to pass! Existence is hollow! \\
I shall either go forth or die, \\
but I shall never marry. 

Why\marginnote{19.1} cling to this rotting body so foul, \\
stinking of fluids, \\
a horrifying water-bag of corpses, \\
always oozing, full of filth? 

Knowing\marginnote{20.1} it like I do, what’s the point? \\
A carcass is vile, smeared with flesh and blood, \\
food for birds and swarms of worms—\\
why have we been given it? 

Before\marginnote{21.1} long the body, bereft of consciousness, \\
is carried out to the charnel ground, \\
to be tossed aside like an old log \\
by relatives in disgust. 

When\marginnote{22.1} they’ve tossed it away in the charnel ground, \\
to be eaten by others, your own parents \\
bathe themselves, disgusted; \\
what then of people at large? 

They’re\marginnote{23.1} attached to this hollow carcass, \\
this mass of sinews and bone; \\
this rotting body \\
full of saliva, tears, feces, and pus. 

If\marginnote{24.1} anyone were to dissect it, \\
turning it inside out, \\
the unbearable stench \\
would disgust even their own mother. 

Properly\marginnote{25.1} examining \\
the aggregates, elements, and sense fields \\
as conditioned, rooted in birth, suffering—\\
why would I wish for marriage? 

Let\marginnote{26.1} three hundred sharp swords \\
fall on my body everyday! \\
Even if the slaughter lasted 100 years \\
it’d be worth it if it led to the end of suffering. 

One\marginnote{27.1} who understands the Teacher’s words \\
would put up with this slaughter: \\
‘Long for you is transmigration \\
being killed time and time again.’ 

Among\marginnote{28.1} gods and humans, \\
in the realm of animals or that of demons, \\
among the ghosts or in the hells, \\
endless killings are seen. 

The\marginnote{29.1} hells are full of killing, \\
for the corrupt who have fallen to the underworld. \\
Even among the gods there is no shelter, \\
for no happiness excels extinguishment. 

Those\marginnote{30.1} who are committed to the dispensation \\
of him of the ten powers attain extinguishment. \\
Living at ease, they apply themselves \\
to giving up rebirth and death. 

On\marginnote{31.1} this very day, dad, I shall renounce: \\
what’s to enjoy in hollow riches? \\
I’m disillusioned with sensual pleasures, \\
they’re like vomit, made like a palm stump.” 

As\marginnote{32.1} she spoke thus to her father, \\
\textsanskrit{Anīkaratta}, to whom she was betrothed, \\
approached from \textsanskrit{Vāraṇavatī} \\
at the time appointed for the marriage. 

Then\marginnote{33.1} \textsanskrit{Sumedhā} took up a knife, \\
and cut off her hair, so black, thick, and soft. \\
Shutting herself in the longhouse, \\
she entered the first absorption. 

And\marginnote{34.1} as she entered it there, \\
\textsanskrit{Anīkaratta} arrived at the city. \\
Then in the longhouse, \textsanskrit{Sumedhā} \\
well developed the perception of impermanence. 

As\marginnote{35.1} she investigated in meditation, \\
\textsanskrit{Anīkaratta} quickly climbed the stairs. \\
His limbs adorned with gems and gold, \\
he begged \textsanskrit{Sumedhā} with joined palms: 

“As\marginnote{36.1} a royal there is command, wealth, authority, \\
and the happiness of possessions. \\
Enjoy sensual pleasures while you’re still young! \\
Sensual pleasures are hard to find in the world! 

I’ve\marginnote{37.1} handed royalty to you—\\
enjoy riches, give gifts! \\
Don’t be sad; \\
your parents are upset.” 

\textsanskrit{Sumedhā},\marginnote{38.1} having no use for sensual pleasures, \\
and having done away with delusion, spoke right back: \\
“Do not take pleasure in sensuality! \\
See the danger in sensual pleasures! 

\textsanskrit{Mandhātā},\marginnote{39.1} king of four continents, \\
foremost in enjoying sensual pleasures, \\
died unsated, \\
his desires unfulfilled. 

Were\marginnote{40.1} the seven jewels to rain from the sky \\
all over the ten directions, \\
there would be no sating of sensual pleasures: \\
people die insatiable. 

Like\marginnote{41.1} a butcher’s knife and chopping block, \\
sensual pleasures are like a snake’s head. \\
They burn like a fire-brand, \\
they resemble a skeleton. 

Sensual\marginnote{42.1} pleasures are impermanent and unstable, \\
they’re full of suffering, a terrible poison; \\
like a hot iron ball, \\
the root of misery, their fruit is pain. 

Sensual\marginnote{43.1} pleasures are like fruits of a tree, \\
like lumps of meat, painful, \\
they trick you like a dream; \\
sensual pleasures are like borrowed goods. 

Sensual\marginnote{44.1} pleasures are like swords and stakes; \\
a disease, a boil, misery and trouble. \\
Like a pit of glowing coals, \\
the root of misery, fear and slaughter. 

Thus\marginnote{45.1} sensual pleasures have been explained \\
to be obstructions, so full of suffering. \\
Please leave! As for me, \\
I have no trust in a new life. 

What\marginnote{46.1} can someone else do for me \\
when their own head is burning? \\
When stalked by old age and death, \\
you should strive to destroy them.” 

She\marginnote{47.1} opened the door \\
and saw her parents with \textsanskrit{Anīkaratta}, \\
sitting crying on the floor. \\
And so she said this: 

“Transmigration\marginnote{48.1} is long for fools, \\
crying again and again at that with no known beginning—\\
the death of a father, \\
the killing of a brother or of themselves. 

Remember\marginnote{49.1} the ocean of tears, of milk, of blood—\\
transmigration with no known beginning. \\
Remember the bones piled up \\
by beings transmigrating. 

Remember\marginnote{50.1} the four oceans \\
compared with tears, milk, and blood. \\
Remember bones piled up high as Mount Vipula \\
in the course of a single eon. 

Transmigration\marginnote{51.1} with no known beginning \\
is compared to this broad land of India; \\
if divided into lumps the size of jujube seeds, \\
they’d still be fewer than his mother’s mothers. 

Remember\marginnote{52.1} the grass, sticks, and leaves, \\
compare that with no known beginning: \\
if split into pieces four inches in size, \\
they’d still be fewer than his father’s fathers. 

Remember\marginnote{53.1} the one-eyed turtle and the yoke with a hole \\
blown in the ocean from east to west—\\
sticking the head in the hole \\
is a metaphor for gaining a human birth. 

Remember\marginnote{54.1} the form of this unlucky body, \\
insubstantial as a lump of foam. \\
See the aggregates as impermanent, \\
remember the hells so full of anguish. 

Remember\marginnote{55.1} those swelling the charnel grounds \\
again and again in life after life. \\
Remember the threat of the marsh crocodile! \\
Remember the four truths! 

When\marginnote{56.1} the deathless is there to be found, \\
why would you drink the five bitter poisons? \\
For every enjoyment of sensual pleasures \\
is so much more bitter than them. 

When\marginnote{57.1} the deathless is there to be found, \\
why would you burn for sensual pleasures? \\
For every enjoyment of sensual pleasures \\
is burning, boiling, bubbling, seething. 

When\marginnote{58.1} there is freedom from enmity, \\
why would you want your enemy, sensual pleasures? \\
Like kings, fire, robbers, flood, and people you dislike, \\
sensual pleasures are very much your enemy. 

When\marginnote{59.1} liberation is there to be found, \\
what good are sensual pleasures that kill and bind? \\
For though unwilling, when sensual pleasures are there, \\
they are subject to the pain of killing and binding. 

As\marginnote{60.1} a blazing grass torch \\
burns one who grasps it without letting go, \\
sensual pleasures are like a grass torch, \\
burning those who do not let go. 

Don’t\marginnote{61.1} give up abundant happiness \\
for the trivial joys of sensual pleasure. \\
Don’t fret later, \\
like a catfish on a hook. 

Deliberately\marginnote{62.1} control yourself among sensual pleasures! \\
You’re like a dog fixed to a chain: \\
sensual pleasures will surely devour you \\
as hungry outcasts would a dog. 

Harnessed\marginnote{63.1} to sensual pleasure, \\
you undergo endless pain, \\
along with much mental anguish: \\
relinquish sensual pleasures, they don’t last! 

When\marginnote{64.1} the unaging is there to be found, \\
what good are sensual pleasures in which is old age? \\
All rebirths everywhere \\
are bonded to death and sickness. 

This\marginnote{65.1} is the ageless, this is the deathless! \\
This is the ageless and deathless, the sorrowless state! \\
Free of enmity, unconstricted, \\
faultless, fearless, without tribulations. 

This\marginnote{66.1} deathless has been realized by many; \\
even today it can be obtained \\
by those who properly apply themselves; \\
but it’s impossible if you don’t try.” 

So\marginnote{67.1} said \textsanskrit{Sumedhā}, \\
lacking delight in conditioned things. \\
Soothing \textsanskrit{Anīkaratta}, \\
\textsanskrit{Sumedhā} cast her hair on the ground. 

Standing\marginnote{68.1} up, \textsanskrit{Anīkaratta} \\
raised his joined palms to her father and begged: \\
“Let go of \textsanskrit{Sumedhā}, so that she may go forth! \\
She will see the truth of liberation.” 

Released\marginnote{69.1} by her mother and father, \\
she went forth, afraid of grief and fear. \\
While still a trainee nun she realized the six direct knowledges, \\
along with the highest fruit. 

The\marginnote{70.1} extinguishment of the princess \\
was incredible and amazing; \\
on her deathbed, she declared \\
her several past lives. 

“In\marginnote{71.1} the time of the Buddha \textsanskrit{Koṇāgamana}, \\
we three friends gave the gift \\
of a newly-built dwelling \\
in the \textsanskrit{Saṅgha}’s monastery. 

Ten\marginnote{72.1} times, a hundred times, \\
a thousand times, ten thousand times, \\
we were reborn among the gods, \\
let alone among humans. 

We\marginnote{73.1} were mighty among the gods, \\
let alone among humans! \\
I was queen to a king with the seven treasures—\\
I was the treasure of a wife. 

That\marginnote{74.1} was the cause, that the origin, that the root, \\
that was the acceptance of the dispensation; \\
that first meeting culminated in extinguishment \\
for one delighting in the teaching. 

So\marginnote{75.1} say those who have faith in the words \\
of the one unrivaled in wisdom. \\
They’re disillusioned with being reborn, \\
and being disillusioned they become dispassionate.” 

%
\end{verse}

That\marginnote{76.1} is how these verses were recited by the senior nun \textsanskrit{Sumedhā}. 

\scendsection{The Great Book is finished. }

\scendbook{The Verses of the Senior Nuns are finished. }

%
\backmatter%
\chapter*{Colophon}
\addcontentsline{toc}{chapter}{Colophon}
\markboth{Colophon}{Colophon}

\section*{The Translator}

Bhikkhu Sujato was born as Anthony Aidan Best on 4/11/1966 in Perth, Western Australia. He grew up in the pleasant suburbs of Mt Lawley and Attadale alongside his sister Nicola, who was the good child. His mother, Margaret Lorraine Huntsman née Pinder, said “he’ll either be a priest or a poet”, while his father, Anthony Thomas Best, advised him to “never do anything for money”. He attended Aquinas College, a Catholic school, where he decided to become an atheist. At the University of WA he studied philosophy, aiming to learn what he wanted to do with his life. Finding that what he wanted to do was play guitar, he dropped out. His main band was named Martha’s Vineyard, which achieved modest success in the indie circuit. 

A seemingly random encounter with a roadside joey took him to Thailand, where he entered his first meditation retreat at Wat Ram Poeng, Chieng Mai in 1992. Feeling the call to the Buddha’s path, he took full ordination in Wat Pa Nanachat in 1994, where his teachers were Ajahn Pasanno and Ajahn Jayasaro. In 1997 he returned to Perth to study with Ajahn Brahm at Bodhinyana Monastery. 

He spent several years practicing in seclusion in Malaysia and Thailand before establishing Santi Forest Monastery in Bundanoon, NSW, in 2003. There he was instrumental in supporting the establishment of the Theravada bhikkhuni order in Australia and advocating for women’s rights. He continues to teach in Australia and globally, with a special concern for the moral implications of climate change and other forms of environmental destruction. He has published a series of books of original and groundbreaking research on early Buddhism. 

In 2005 he founded SuttaCentral together with Rod Bucknell and John Kelly. In 2015, seeing the need for a complete, accurate, plain English translation of the Pali texts, he undertook the task, spending nearly three years in isolation on the isle of Qi Mei off the coast of the nation of Taiwan. He completed the four main \textsanskrit{Nikāyas} in 2018, and the early books of the Khuddaka \textsanskrit{Nikāya} were complete by 2021. All this work is dedicated to the public domain and is entirely free of copyright encumbrance. 

In 2019 he returned to Sydney where he established Lokanta Vihara (The Monastery at the End of the World). 

\section*{Creation Process}

Primary source was the digital \textsanskrit{Mahāsaṅgīti} edition of the Pali \textsanskrit{Tipiṭaka}. Translated from the Pali, with reference to several English translations, especially those of K.R. Norman.

\section*{The Translation}

This translation aims to make a clear, readable, and accurate rendering of the \textsanskrit{Therīgāthā}. The initial draft was by Jessica Walton, and it was revised and finished by Bhikkhu Sujato in 2019. The terminology has been brought in line with Bhikkhu Sujato’s translation of the four \textsanskrit{Nikāyas}.

\section*{About SuttaCentral}

SuttaCentral publishes early Buddhist texts. Since 2005 we have provided root texts in Pali, Chinese, Sanskrit, Tibetan, and other languages, parallels between these texts, and translations in many modern languages. We build on the work of generations of scholars, and offer our contribution freely.

SuttaCentral is driven by volunteer contributions, and in addition we employ professional developers. We offer a sponsorship program for high quality translations from the original languages. Financial support for SuttaCentral is handled by the SuttaCentral Development Trust, a charitable trust registered in Australia.

\section*{About Bilara}

“Bilara” means “cat” in Pali, and it is the name of our Computer Assisted Translation (CAT) software. Bilara is a web app that enables translators to translate early Buddhist texts into their own language. These translations are published on SuttaCentral with the root text and translation side by side.

\section*{About SuttaCentral Editions}

The SuttaCentral Editions project makes high quality books from selected Bilara translations. These are published in formats including HTML, EPUB, PDF, and print.

If you want to print any of our Editions, please let us know and we will help prepare a file to your specifications.

%
\end{document}