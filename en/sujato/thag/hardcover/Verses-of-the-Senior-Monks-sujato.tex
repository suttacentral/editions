\documentclass[12pt,openany]{book}%
\usepackage{lastpage}%
%
\usepackage{ragged2e}
\usepackage{verse}
\usepackage[a-3u]{pdfx}
\usepackage[inner=1in, outer=1in, top=.7in, bottom=1in, papersize={6in,9in}, headheight=13pt]{geometry}
\usepackage{polyglossia}
\usepackage[12pt]{moresize}
\usepackage{soul}%
\usepackage{microtype}
\usepackage{tocbasic}
\usepackage{realscripts}
\usepackage{epigraph}%
\usepackage{setspace}%
\usepackage{sectsty}
\usepackage{fontspec}
\usepackage{marginnote}
\usepackage[bottom]{footmisc}
\usepackage{enumitem}
\usepackage{fancyhdr}
\usepackage{emptypage}
\usepackage{extramarks}
\usepackage{graphicx}
\usepackage{relsize}
\usepackage{etoolbox}

% improve ragged right headings by suppressing hyphenation and orphans. spaceskip plus and minus adjust interword spacing; increase rightskip stretch to make it want to push a word on the first line(s) to the next line; reduce parfillskip stretch to make line length more equal . spacefillskip and xspacefillskip can be deleted to use defaults.
\protected\def\BalancedRagged{
\leftskip     0pt
\rightskip    0pt plus 10em
\spaceskip=1\fontdimen2\font plus .5\fontdimen3\font minus 1.5\fontdimen4\font
\xspaceskip=1\fontdimen2\font plus 1\fontdimen3\font minus 1\fontdimen4\font
\parfillskip  0pt plus 15em
\relax
}

\hypersetup{
colorlinks=true,
urlcolor=black,
linkcolor=black,
citecolor=black,
allcolors=black
}

% use a small amount of tracking on small caps
\SetTracking[ spacing = {25*,166, } ]{ encoding = *, shape = sc }{ 25 }

% add a blank page
\newcommand{\blankpage}{
\newpage
\thispagestyle{empty}
\mbox{}
\newpage
}

% define languages
\setdefaultlanguage[]{english}
\setotherlanguage[script=Latin]{sanskrit}

%\usepackage{pagegrid}
%\pagegridsetup{top-left, step=.25in}

% define fonts
% use if arno sanskrit is unavailable
%\setmainfont{Gentium Plus}
%\newfontfamily\Marginalfont[]{Gentium Plus}
%\newfontfamily\Allsmallcapsfont[RawFeature=+c2sc]{Gentium Plus}
%\newfontfamily\Noligaturefont[Renderer=Basic]{Gentium Plus}
%\newfontfamily\Noligaturecaptionfont[Renderer=Basic]{Gentium Plus}
%\newfontfamily\Fleuronfont[Ornament=1]{Gentium Plus}

% use if arno sanskrit is available. display is applied to \chapter and \part, subhead to \section and \subsection.
\setmainfont[
  FontFace={sb}{n}{Font = {Arno Pro Semibold}},
  FontFace={sb}{it}{Font = {Arno  Pro Semibold Italic}}
]{Arno Pro}

% create commands for using semibold
\DeclareRobustCommand{\sbseries}{\fontseries{sb}\selectfont}
\DeclareTextFontCommand{\textsb}{\sbseries}

\newfontfamily\Marginalfont[RawFeature=+subs]{Arno Pro Regular}
\newfontfamily\Allsmallcapsfont[RawFeature=+c2sc]{Arno Pro}
\newfontfamily\Noligaturefont[Renderer=Basic]{Arno Pro}
\newfontfamily\Noligaturecaptionfont[Renderer=Basic]{Arno Pro Caption}

% chinese fonts
\newfontfamily\cjk{Noto Serif TC}
\newcommand*{\langlzh}[1]{\cjk{#1}\normalfont}%

% logo
\newfontfamily\Logofont{sclogo.ttf}
\newcommand*{\sclogo}[1]{\large\Logofont{#1}}

% use subscript numerals for margin notes
\renewcommand*{\marginfont}{\Marginalfont}

% ensure margin notes have consistent vertical alignment
\renewcommand*{\marginnotevadjust}{-.17em}

% use compact lists
\setitemize{noitemsep,leftmargin=1em}
\setenumerate{noitemsep,leftmargin=1em}
\setdescription{noitemsep, style=unboxed, leftmargin=1em}

% style ToC
\DeclareTOCStyleEntries[
  raggedentrytext,
  linefill=\hfill,
  pagenumberwidth=.5in,
  pagenumberformat=\normalfont,
  entryformat=\normalfont
]{tocline}{chapter,section}


  \setlength\topsep{0pt}%
  \setlength\parskip{0pt}%

% define new \centerpars command for use in ToC. This ensures centering, proper wrapping, and no page break after
\def\startcenter{%
  \par
  \begingroup
  \leftskip=0pt plus 1fil
  \rightskip=\leftskip
  \parindent=0pt
  \parfillskip=0pt
}
\def\stopcenter{%
  \par
  \endgroup
}
\long\def\centerpars#1{\startcenter#1\stopcenter}

% redefine part, so that it adds a toc entry without page number
\let\oldcontentsline\contentsline
\newcommand{\nopagecontentsline}[3]{\oldcontentsline{#1}{#2}{}}

    \makeatletter
\renewcommand*\l@part[2]{%
  \ifnum \c@tocdepth >-2\relax
    \addpenalty{-\@highpenalty}%
    \addvspace{0em \@plus\p@}%
    \setlength\@tempdima{3em}%
    \begingroup
      \parindent \z@ \rightskip \@pnumwidth
      \parfillskip -\@pnumwidth
      {\leavevmode
       \setstretch{.85}\large\scshape\centerpars{#1}\vspace*{-1em}\llap{#2}}\par
       \nobreak
         \global\@nobreaktrue
         \everypar{\global\@nobreakfalse\everypar{}}%
    \endgroup
  \fi}
\makeatother

\makeatletter
\def\@pnumwidth{2em}
\makeatother

% define new sectioning command, which is only used in volumes where the pannasa is found in some parts but not others, especially in an and sn

\newcommand*{\pannasa}[1]{\clearpage\thispagestyle{empty}\begin{center}\vspace*{14em}\setstretch{.85}\huge\itshape\scshape\MakeLowercase{#1}\end{center}}

    \makeatletter
\newcommand*\l@pannasa[2]{%
  \ifnum \c@tocdepth >-2\relax
    \addpenalty{-\@highpenalty}%
    \addvspace{.5em \@plus\p@}%
    \setlength\@tempdima{3em}%
    \begingroup
      \parindent \z@ \rightskip \@pnumwidth
      \parfillskip -\@pnumwidth
      {\leavevmode
       \setstretch{.85}\large\itshape\scshape\lowercase{\centerpars{#1}}\vspace*{-1em}\llap{#2}}\par
       \nobreak
         \global\@nobreaktrue
         \everypar{\global\@nobreakfalse\everypar{}}%
    \endgroup
  \fi}
\makeatother

% don't put page number on first page of toc (relies on etoolbox)
\patchcmd{\chapter}{plain}{empty}{}{}

% global line height
\setstretch{1.05}

% allow linebreak after em-dash
\catcode`\—=13
\protected\def—{\unskip\textemdash\allowbreak}

% style headings with secsty. chapter and section are defined per-edition
\partfont{\setstretch{.85}\normalfont\centering\textsc}
\subsectionfont{\setstretch{.95}\normalfont\BalancedRagged}%
\subsubsectionfont{\setstretch{1}\normalfont\itshape\BalancedRagged}

% style elements of suttatitle
\newcommand*{\suttatitleacronym}[1]{\smaller[2]{#1}\vspace*{.3em}}
\newcommand*{\suttatitletranslation}[1]{\linebreak{#1}}
\newcommand*{\suttatitleroot}[1]{\linebreak\smaller[2]\itshape{#1}}

\DeclareTOCStyleEntries[
  indent=3.3em,
  dynindent,
  beforeskip=.2em plus -2pt minus -1pt,
]{tocline}{section}

\DeclareTOCStyleEntries[
  indent=0em,
  dynindent,
  beforeskip=.4em plus -2pt minus -1pt,
]{tocline}{chapter}

\newcommand*{\tocacronym}[1]{\hspace*{-3.3em}{#1}\quad}
\newcommand*{\toctranslation}[1]{#1}
\newcommand*{\tocroot}[1]{(\textit{#1})}
\newcommand*{\tocchapterline}[1]{\bfseries\itshape{#1}}


% redefine paragraph and subparagraph headings to not be inline
\makeatletter
% Change the style of paragraph headings %
\renewcommand\paragraph{\@startsection{paragraph}{4}{\z@}%
            {-2.5ex\@plus -1ex \@minus -.25ex}%
            {1.25ex \@plus .25ex}%
            {\noindent\normalfont\itshape\small}}

% Change the style of subparagraph headings %
\renewcommand\subparagraph{\@startsection{subparagraph}{5}{\z@}%
            {-2.5ex\@plus -1ex \@minus -.25ex}%
            {1.25ex \@plus .25ex}%
            {\noindent\normalfont\itshape\footnotesize}}
\makeatother

% use etoolbox to suppress page numbers on \part
\patchcmd{\part}{\thispagestyle{plain}}{\thispagestyle{empty}}
  {}{\errmessage{Cannot patch \string\part}}

% and to reduce margins on quotation
\patchcmd{\quotation}{\rightmargin}{\leftmargin 1.2em \rightmargin}{}{}
\AtBeginEnvironment{quotation}{\small}

% titlepage
\newcommand*{\titlepageTranslationTitle}[1]{{\begin{center}\begin{large}{#1}\end{large}\end{center}}}
\newcommand*{\titlepageCreatorName}[1]{{\begin{center}\begin{normalsize}{#1}\end{normalsize}\end{center}}}

% halftitlepage
\newcommand*{\halftitlepageTranslationTitle}[1]{\setstretch{2.5}{\begin{Huge}\uppercase{\so{#1}}\end{Huge}}}
\newcommand*{\halftitlepageTranslationSubtitle}[1]{\setstretch{1.2}{\begin{large}{#1}\end{large}}}
\newcommand*{\halftitlepageFleuron}[1]{{\begin{large}\Fleuronfont{{#1}}\end{large}}}
\newcommand*{\halftitlepageByline}[1]{{\begin{normalsize}\textit{{#1}}\end{normalsize}}}
\newcommand*{\halftitlepageCreatorName}[1]{{\begin{LARGE}{\textsc{#1}}\end{LARGE}}}
\newcommand*{\halftitlepageVolumeNumber}[1]{{\begin{normalsize}{\Allsmallcapsfont{\textsc{#1}}}\end{normalsize}}}
\newcommand*{\halftitlepageVolumeAcronym}[1]{{\begin{normalsize}{#1}\end{normalsize}}}
\newcommand*{\halftitlepageVolumeTranslationTitle}[1]{{\begin{Large}{\textsc{#1}}\end{Large}}}
\newcommand*{\halftitlepageVolumeRootTitle}[1]{{\begin{normalsize}{\Allsmallcapsfont{\textsc{\itshape #1}}}\end{normalsize}}}
\newcommand*{\halftitlepagePublisher}[1]{{\begin{large}{\Noligaturecaptionfont\textsc{#1}}\end{large}}}

% epigraph
\renewcommand{\epigraphflush}{center}
\renewcommand*{\epigraphwidth}{.85\textwidth}
\newcommand*{\epigraphTranslatedTitle}[1]{\vspace*{.5em}\footnotesize\textsc{#1}\\}%
\newcommand*{\epigraphRootTitle}[1]{\footnotesize\textit{#1}\\}%
\newcommand*{\epigraphReference}[1]{\footnotesize{#1}}%

% map
\newsavebox\IBox

% custom commands for html styling classes
\newcommand*{\scnamo}[1]{\begin{Center}\textit{#1}\end{Center}\bigskip}
\newcommand*{\scendsection}[1]{\begin{Center}\begin{small}\textit{#1}\end{small}\end{Center}\addvspace{1em}}
\newcommand*{\scendsutta}[1]{\begin{Center}\textit{#1}\end{Center}\addvspace{1em}}
\newcommand*{\scendbook}[1]{\bigskip\begin{Center}\uppercase{#1}\end{Center}\addvspace{1em}}
\newcommand*{\scendkanda}[1]{\begin{Center}\textbf{#1}\end{Center}\addvspace{1em}} % use for ending vinaya rule sections and also samyuttas %
\newcommand*{\scend}[1]{\begin{Center}\begin{small}\textit{#1}\end{small}\end{Center}\addvspace{1em}}
\newcommand*{\scendvagga}[1]{\begin{Center}\textbf{#1}\end{Center}\addvspace{1em}}
\newcommand*{\scrule}[1]{\textsb{#1}}
\newcommand*{\scadd}[1]{\textit{#1}}
\newcommand*{\scevam}[1]{\textsc{#1}}
\newcommand*{\scspeaker}[1]{\hspace{2em}\textit{#1}}
\newcommand*{\scbyline}[1]{\begin{flushright}\textit{#1}\end{flushright}\bigskip}
\newcommand*{\scexpansioninstructions}[1]{\begin{small}\textit{#1}\end{small}}
\newcommand*{\scuddanaintro}[1]{\medskip\noindent\begin{footnotesize}\textit{#1}\end{footnotesize}\smallskip}

\newenvironment{scuddana}{%
\setlength{\stanzaskip}{.5\baselineskip}%
  \vspace{-1em}\begin{verse}\begin{footnotesize}%
}{%
\end{footnotesize}\end{verse}
}%

% custom command for thematic break = hr
\newcommand*{\thematicbreak}{\begin{center}\rule[.5ex]{6em}{.4pt}\begin{normalsize}\quad\Fleuronfont{•}\quad\end{normalsize}\rule[.5ex]{6em}{.4pt}\end{center}}

% manage and style page header and footer. "fancy" has header and footer, "plain" has footer only

\pagestyle{fancy}
\fancyhf{}
\fancyfoot[RE,LO]{\thepage}
\fancyfoot[LE,RO]{\footnotesize\lastleftxmark}
\fancyhead[CE]{\setstretch{.85}\Noligaturefont\MakeLowercase{\textsc{\firstrightmark}}}
\fancyhead[CO]{\setstretch{.85}\Noligaturefont\MakeLowercase{\textsc{\firstleftmark}}}
\renewcommand{\headrulewidth}{0pt}
\fancypagestyle{plain}{ %
\fancyhf{} % remove everything
\fancyfoot[RE,LO]{\thepage}
\fancyfoot[LE,RO]{\footnotesize\lastleftxmark}
\renewcommand{\headrulewidth}{0pt}
\renewcommand{\footrulewidth}{0pt}}
\fancypagestyle{plainer}{ %
\fancyhf{} % remove everything
\fancyfoot[RE,LO]{\thepage}
\renewcommand{\headrulewidth}{0pt}
\renewcommand{\footrulewidth}{0pt}}

% style footnotes
\setlength{\skip\footins}{1em}

\makeatletter
\newcommand{\@makefntextcustom}[1]{%
    \parindent 0em%
    \thefootnote.\enskip #1%
}
\renewcommand{\@makefntext}[1]{\@makefntextcustom{#1}}
\makeatother

% hang quotes (requires microtype)
\microtypesetup{
  protrusion = true,
  expansion  = true,
  tracking   = true,
  factor     = 1000,
  patch      = all,
  final
}

% Custom protrusion rules to allow hanging punctuation
\SetProtrusion
{ encoding = *}
{
% char   right left
  {-} = {    , 500 },
  % Double Quotes
  \textquotedblleft
      = {1000,     },
  \textquotedblright
      = {    , 1000},
  \quotedblbase
      = {1000,     },
  % Single Quotes
  \textquoteleft
      = {1000,     },
  \textquoteright
      = {    , 1000},
  \quotesinglbase
      = {1000,     }
}

% make latex use actual font em for parindent, not Computer Modern Roman
\AtBeginDocument{\setlength{\parindent}{1em}}%
%

% Default values; a bit sloppier than normal
\tolerance 1414
\hbadness 1414
\emergencystretch 1.5em
\hfuzz 0.3pt
\clubpenalty = 10000
\widowpenalty = 10000
\displaywidowpenalty = 10000
\hfuzz \vfuzz
 \raggedbottom%

\title{Verses of the Senior Monks}
\author{Bhikkhu Sujato}
\date{}%
% define a different fleuron for each edition
\newfontfamily\Fleuronfont[Ornament=35]{Arno Pro}

% Define heading styles per edition for chapter and section. Suttatitle can be either of these, depending on the volume. 

\let\oldfrontmatter\frontmatter
\renewcommand{\frontmatter}{%
\chapterfont{\setstretch{.85}\normalfont\centering}%
\sectionfont{\setstretch{.85}\normalfont\BalancedRagged}%
\oldfrontmatter}

\let\oldmainmatter\mainmatter
\renewcommand{\mainmatter}{%
\chapterfont{\setstretch{.85}\normalfont\centering}%
\sectionfont{\setstretch{.85}\normalfont\centering}%
\oldmainmatter}

\let\oldbackmatter\backmatter
\renewcommand{\backmatter}{%
\chapterfont{\setstretch{.85}\normalfont\centering}%
\sectionfont{\setstretch{.85}\normalfont\BalancedRagged}%
\pagestyle{plainer}%
\oldbackmatter}
%
%
\begin{document}%
\normalsize%
\frontmatter%
\setlength{\parindent}{0cm}

\pagestyle{empty}

\maketitle

\blankpage%
\begin{center}

\vspace*{2.2em}

\halftitlepageTranslationTitle{Verses of the Senior Monks}

\vspace*{1em}

\halftitlepageTranslationSubtitle{An approachable translation of the Theragāthā}

\vspace*{2em}

\halftitlepageFleuron{•}

\vspace*{2em}

\halftitlepageByline{translated and introduced by}

\vspace*{.5em}

\halftitlepageCreatorName{Bhikkhu Sujato}

\vspace*{4em}

\halftitlepageVolumeNumber{}

\smallskip

\halftitlepageVolumeAcronym{Thag}

\smallskip

\halftitlepageVolumeTranslationTitle{}

\smallskip

\halftitlepageVolumeRootTitle{}

\vspace*{\fill}

\sclogo{0}
 \halftitlepagePublisher{SuttaCentral}

\end{center}

\newpage
%
\setstretch{1.05}

\begin{footnotesize}

\textit{Verses of the Senior Monks} is a translation of the Theragāthā by Bhikkhu Sujato.

\medskip

Creative Commons Zero (CC0)

To the extent possible under law, Bhikkhu Sujato has waived all copyright and related or neighboring rights to \textit{Verses of the Senior Monks}.

\medskip

This work is published from Australia.

\begin{center}
\textit{This translation is an expression of an ancient spiritual text that has been passed down by the Buddhist tradition for the benefit of all sentient beings. It is dedicated to the public domain via Creative Commons Zero (CC0). You are encouraged to copy, reproduce, adapt, alter, or otherwise make use of this translation. The translator respectfully requests that any use be in accordance with the values and principles of the Buddhist community.}
\end{center}

\medskip

\begin{description}
    \item[Web publication date] 2014
    \item[This edition] 2025-01-11 11:58:19
    \item[Publication type] hardcover
    \item[Edition] ed3
    \item[Number of volumes] 1
    \item[Publication ISBN] 978-1-76132-016-3
    \item[Volume ISBN] 
    \item[Publication URL] \href{https://suttacentral.net/editions/thag/en/sujato}{https://suttacentral.net/editions/thag/en/sujato}
    \item[Source URL] \href{https://github.com/suttacentral/bilara-data/tree/published/translation/en/sujato/sutta/kn/thag}{https://github.com/suttacentral/bilara-data/tree/published/translation/en/sujato/sutta/kn/thag}
    \item[Publication number] scpub1
\end{description}

\medskip

Map of Jambudīpa is by Jonas David Mitja Lang, and is released by him under Creative Commons Zero (CC0).

\medskip

Published by SuttaCentral

\medskip

\textit{SuttaCentral,\\
c/o Alwis \& Alwis Pty Ltd\\
Kaurna Country,\\
Suite 12,\\
198 Greenhill Road,\\
Eastwood,\\
SA 5063,\\
Australia}

\end{footnotesize}

\newpage

\setlength{\parindent}{1em}%%
\newpage

\vspace*{\fill}

\begin{center}
\epigraph{See this wisdom of the Realized Ones!\\
Like a fire blazing in the night,\\
giving light, giving vision,\\
they dispel the doubt of those who’ve come.}
{
\epigraphTranslatedTitle{\textsanskrit{Kaṅkhārevata}}
\epigraphRootTitle{}
\epigraphReference{\textsanskrit{Theragāthā} 1.3}
}
\end{center}

\vspace*{2in}

\vspace*{\fill}

\newgeometry{inner=0mm, outer=.5in, top=.6in, bottom=0mm}
\setlength{\parindent}{0em}
\sbox\IBox{\includegraphics{/app/sutta_publisher/images/jambudipa_map.png}}%
\includegraphics[trim=0 0 \dimexpr\wd\IBox-\textwidth{} 0,clip]{/app/sutta_publisher/images/jambudipa_map.png}
\newpage
\includegraphics[trim=\textwidth{} 0 0 0,clip]{/app/sutta_publisher/images/jambudipa_map.png}
\newpage
\restoregeometry

\blankpage%

\setlength{\parindent}{1em}
%
\tableofcontents
\newpage
\pagestyle{fancy}
%
\chapter*{The SuttaCentral Editions Series}
\addcontentsline{toc}{chapter}{The SuttaCentral Editions Series}
\markboth{The SuttaCentral Editions Series}{The SuttaCentral Editions Series}

Since 2005 SuttaCentral has provided access to the texts, translations, and parallels of early Buddhist texts. In 2018 we started creating and publishing our translations of these seminal spiritual classics. The “Editions” series now makes selected translations available as books in various forms, including print, PDF, and EPUB.

Editions are selected from our most complete, well-crafted, and reliable translations. They aim to bring these texts to a wider audience in forms that reward mindful reading. Care is taken with every detail of the production, and we aim to meet or exceed professional best standards in every way. These are the core scriptures underlying the entire Buddhist tradition, and we believe that they deserve to be preserved and made available in the highest quality without compromise.

SuttaCentral is a charitable organization. Our work is accomplished by volunteers and through the generosity of our donors. Everything we create is offered to all of humanity free of any copyright or licensing restrictions. 

%
\chapter*{Preface}
\addcontentsline{toc}{chapter}{Preface}
\markboth{Preface}{Preface}

The monks of the \textsanskrit{Theragāthā} are my brothers. I can feel them through the years, their struggles and triumphs, their deep joys and nagging sorrows. They are also my idols, those humans who I would most like to be. 

The Buddha’s teachings are amazing. But it is not about the philosophy, it is about the people, about what mattered to them and what matters to us. In their fierce forests, their leaf huts, their tattered robes, the monks of old could not be more different than us.

Except in another way they are just like us. For the comforts of our civilization are but a veneer, a veil between us and the world as it has been lived by humans for thousands of years. How is it that when we look into our heart of hearts we yearn only to be free, and when we hear the words of people in such distant times and places, we find the same yearning?

To renounce is to dedicate one’s life to having less. For we who have taken the mendicant path, having things is a hassle, and having no things is freedom. It is strange to look at the world of consumerism and see how people have fought so hard for such a long time for the freedom to live their lives, to make their own choices, to be who they want to be. And what do they choose? To trap themselves in the plastic prison of their desires? To insist on getting what they want while giving not a single thought to wanting wisely? 

We shall leave all of this behind. We might as well start now.

%
\chapter*{Verses of the Senior Monks: a meditative life}
\addcontentsline{toc}{chapter}{Verses of the Senior Monks: a meditative life}
\markboth{Verses of the Senior Monks: a meditative life}{Verses of the Senior Monks: a meditative life}

\scbyline{Bhikkhu Sujato, 2022}

The \textsanskrit{Theragāthā} is a classic Pali collection of verses by early Buddhist monks. There is a parallel collection of nuns’ verses, the \textsanskrit{Therīgāthā}. The \textsanskrit{Theragāthā} consists of 1289 verses, collected according to the monk with whom they were traditionally associated. These poems speak from the personal experience of monks living in or near the time of the Buddha. More than any other text we find here a range of voices expressing the fears, inspirations, struggles, and triumphs of the spiritual search.

The \textsanskrit{Theragāthā} was the first major Pali work that I translated, and it was here that I developed my ideas of how to translate. I aimed to make a translation that is both precise and readable so that this astonishing work of ancient spiritual insight might enjoy the wider audience it so richly deserves. This essay is a revised version of the Introduction to the first paperback edition of this translation in 2014.

\section*{About the \textsanskrit{Theragāthā}}

I’d like to give a very brief and non-technical introduction to the text. If you are interested in a more detailed technical analysis, you can read Norman’s long introduction. This especially focuses on the metrical styles of the text, but in addition, it contains analyses of dating, composition, and authorship that are essential for any serious study.

Each of the verses of the \textsanskrit{Theragāthā} is collected under the name of a certain monk. Verses appear under the names of 264 monks, although occasionally a monk may have more than one set of verses. In many cases, the verses were composed by, or at least were supposed to be composed by, these monks. Generally speaking, I see no reason why the bulk of the verses should not be authentic. The \textsanskrit{Gāthās} are not wholly independent texts: many of the verses are found elsewhere in Buddhist texts. But while the concept of “authorship” was indeed a flexible one, most of the more striking verses do have a personal flavor of their author. The fancy stylings of the monk \textsanskrit{Vaṅgīsa} show off his poetic virtuosity (\href{https://suttacentral.net/thag21.1}{Thag 21.1}), while the emotional struggle of \textsanskrit{Tālapuṭa} expresses a deeply personal internal conflict and yearning (\href{https://suttacentral.net/thag19.1}{Thag 19.1}).

However, not all the verses can be ascribed to the monks in question. Sometimes the verses are in a dialogue form; or they may be teaching verses addressed to a monk; or they may be verses about a monk; in some cases they have been added by later redactors. Often the verses are in a vague third person, which leaves it ambiguous whether it was meant to be by the monk or about him. And sometimes verses are repeated, both within the \textsanskrit{Theragāthā} and in other Buddhist texts, so a speaker of a verse is not always its composer. Technically, then, the collection is “verses associated with the senior monks”.

I have used the term “senior monk” rather than “elder” to render \textit{thera} for a couple of reasons. First, it will make it easier to distinguish the collection from the \textsanskrit{Therīgāthā}. More importantly, not all the monks here are really “elders” in the sense of being wizened old men. Usually in Sangha usage a \textit{thera} is simply one who has completed ten years as a monk, so a monk of thirty years of age, while hardly an “elder”, may be a \textit{thera}.

As well as being collected according to the name of the associated monk, the texts are organized by number (the \textit{\textsanskrit{aṅguttara}} principle). That is, the first sets of verses are those where a monk is associated with only one verse; then two, three, and so on. There is, in addition, an occasional connection of subject matter or literary style from one verse to the other; and, rarely, a thin narrative context (eg. \href{https://suttacentral.net/thag16.1}{Thag 16.1}).

The numbering of the collections needs a little attention. The texts may be referenced by three means, all of which are available on SuttaCentral; either by simple verse count, or by chapter and verse, or by the page number of the PTS Pali edition.

The primary system used in SuttaCentral is the chapter and verse, as this collects all the verses associated with a given monk in one place. This chapter and verse system is not used in the PTS editions, but it is used in the \textsanskrit{Mahāsaṅgīti} text on which the translation is based. However this system can be a little confusing—or at least, I was confused by it! From the ones to the fourteens there is no problem. There is no set of fifteen verses, so we skip from the fourteens to the sixteens. Here the numbering of the sections goes out of alignment with the number of verses: the fifteenth section (\href{https://suttacentral.net/thag15.1}{Thag 15.1}) consists of a set of sixteen verses. The sixteenth section (\href{https://suttacentral.net/thag16.1}{Thag 16.1} etc.) then consists of sets of twenty or more verses, and so on.

In terms of dating, the \textsanskrit{Theragāthā} belongs firmly to the corpus of early Buddhist texts. Most of the monks are said to have lived in the time of the Buddha, and there seems no good reason to doubt this. In a few cases, due to the content of the text, the vocabulary or meter, or the statements in the commentary, the verses appear to date from as late as the time of King Ashoka. Norman suggests a period of composition of almost 300 years; however, if we adopt, as it seems we should, the “median chronology” that places the death of the Buddha not long before 400 BCE, then the period of composition would be closer to 200 years.

The \textsanskrit{Theragāthā} is placed as the eighth book of the Khuddaka \textsanskrit{Nikāya}, where it follows such late texts as the \textsanskrit{Vimānavatthu} and the Petavatthu. No particular conclusion can be drawn from this, however. There is, it’s true, a general tendency to group early texts at the start, while late additions are added at the end, but this general pattern admits many individual exceptions. The first book in the collection, after all, is the \textsanskrit{Khuddakapāṭha}, which is one of the latest books in the canon.

It seems the tradition was not tired of hearing the stories of monastics from the days of old, for a pair of texts, the Thera- and \textsanskrit{Therī}-\textsanskrit{Apadāna}, was developed to tell their past life stories. While ostensibly relating tales of most of the same monks and nuns as in the Thera- and \textsanskrit{Therīgātha}, these texts, which probably date 300–400 years after the Buddha, have no claim to historical authenticity. In place of the varied, vivid, and challenging verses of the earlier works, which focus on the life and practice to be done in this life, these works attribute the Awakening of the monastics to pious acts of merit-making in far-gone ages.

As with all Pali texts, the \textsanskrit{Theragāthā} is passed down in the tradition alongside a commentary, in this case, written by \textsanskrit{Dhammapāla} approximately 1,000 years after the text itself. I have consulted the commentary in cases where the meaning of the verse was unclear to me, and for myself, as for all serious scholars, the commentaries have proven invaluable.

As well as providing the normal kinds of linguistic and doctrinal analysis, the \textsanskrit{Theragāthā} commentary gives background stories for the lives of the monks, many of whom we know little about apart from the \textsanskrit{Theragāthā} itself. These stories draw largely from the \textsanskrit{Apadānas}. In some cases, the stories provide context to make sense of the verses, and there seems little doubt that these verses, as is the normal way in Pali, were passed down from the earliest times with some form of narrative context and explanation. Like the \textsanskrit{Jātakas}, the Dhammapada, or the \textsanskrit{Udāna}, the verses formed the emotional and doctrinal kernel of the story. However, in the form that we have today, the commentary speaks to a set of concerns and ideas that date long after the \textsanskrit{Theragāthā} itself. While the commentary is invaluable in understanding what the meaning of these texts was for the Theravadin tradition, only in rare cases does it provide genuine historical information about the monks.

What is striking to me is just how clear-cut the demarcation of Pali texts really is. The Thera- and \textsanskrit{Therīgāthā} lie on the far side of a dividing line in Pali literature. They are concerned with seclusion, meditation, mindfulness, and above all, liberation. From the time of Ashoka or thereabouts, texts such as the \textsanskrit{Apadānas} became concerned with glorifying the Buddha, and especially with encouraging acts of merit for attaining heaven or enlightenment in future lives. Such concerns are notable for their absence from the \textsanskrit{Theragāthā}; when they are present, such as Sela’s verses extolling the Buddha, they remain grounded in human experience, rather than the elaborate fantasies of later days.

There are very few exceptions, such as \href{https://suttacentral.net/thag1.96}{Thag 1.96} \textsanskrit{Khaṇḍasumana}, which says how after offering a flower he rejoiced in heaven for 800 million years, and then attained \textit{\textsanskrit{Nibbāna}} with what was left over. But this is just so out of place. Among the countless verses that speak of retreating to solitude, of devotion to \textit{\textsanskrit{jhāna}}, of renouncing everything in the world, such sentiments seem as if from a different world of thought—a different religion even. Yet the \textsanskrit{Apadānas} consist of little more than lengthy elaborations of this kind of story.

The classical \textsanskrit{Theragāthā} verse is a song of liberation, rejoicing in a simple life lived with nature. Here’s a typical example, from \href{https://suttacentral.net/thag1.22}{Thag 1.22}, the verse of Cittaka:

\begin{verse}%
Crested peacocks with beautiful blue necks \\
Cry out in \textsanskrit{Karaṃvī}. \\
Aroused by a cool breeze, \\
They awaken the sleeper to practice \textit{\textsanskrit{jhāna}}.

%
\end{verse}

But the verses embrace a wide range of subjects; straightforward doctrinal statements, lamentations of the decline of the Sangha, eulogy of great monks, or simple narrative.

While the texts are mostly direct and clear-hearted, some of the most interesting verses are those that speak from the mind’s contradictions, the longings that bedevil the spiritual life. Nowhere has this very human ambiguity been expressed better than in the extended set of verses by \textsanskrit{Tālapuṭa} (\href{https://suttacentral.net/thag19.1}{Thag 19.1}). Employing an unusually sophisticated poetic style—only exceeded in this regard by \textsanskrit{Vaṅgīsa}, in whose verses we can discern the beginnings of the decadent poetics of later generations—and addressing his stubborn mind in the second person (A rare appearance of the neuter vocative) he berates it for its inconstancy:

\begin{verse}%
Oh, when will the winter clouds rain freshly \\
As I wear my robe in the forest, \\
Walking the path trodden by the sages? \\
When will it be? …

For many years you begged me, \\
“Enough of living in a house for you!” \\
Why do you not urge me on, mind, \\
Now I’ve gone forth as an ascetic?

%
\end{verse}

There is one text that deserves a special mention. The verses of \textsanskrit{Vaḍḍha} at \href{https://suttacentral.net/thag5.5}{Thag 5.5} are not an independent composition but continue the poem begun under the name of \textsanskrit{Vaḍḍha}’s Mother in the \textsanskrit{Therīgāthā} (\href{https://suttacentral.net/thig9.1/en/sujato}{Thig 9.1}). That these are one composition split in two can be seen by several linguistic features, such as the use of a demonstrative pronoun (\textit{\textsanskrit{tassāhaṁ} \textsanskrit{vacanaṁ} \textsanskrit{sutvā}}) in \href{https://suttacentral.net/thig9.1/en/sujato}{Thig 9.1} echoed with a relative pronoun (\textit{\textsanskrit{yassāhaṁ} \textsanskrit{vacanaṁ} \textsanskrit{sutvā}}) in \href{https://suttacentral.net/thag5.5}{Thag 5.5}. The story tells of how \textsanskrit{Vaḍḍha} was urged to practice by his mother, who he realizes has seen the Dhamma herself. The plot twist comes when we realize that his sister, too, has seen the Dhamma.

Of all the texts in the Pali canon, it is in the verses of these senior monks, and the nuns of the \textsanskrit{Therīgāthā}, that we come closest to the personal experience of living in the time of the Buddha, struggling with, and eventually overcoming, the causes of suffering that are so captivating. I hope that this new translation can help bring these experiences to life for a new audience.

\section*{Gender and Empathy: Comparing \textsanskrit{Theragāthā} with \textsanskrit{Therīgāthā}}

The division of texts into those authored by men and by women invites consideration of gender. The basic division is along the lines of the traditional gender binary, and it is tempting to analyze them in this light. Yet the movement of the Dhamma, and many specific details in the texts, suggests a blurring of the lines, a nuancing of the black and white that invites participation by those whose gender is not so easily categorized. We have all been men and women in our countless past lives, and trans and non-binary as well. Gendered language and culture are our past and our present, but it does not have to be our future.

Questions of gender are rarely raised by the nuns and monks in the \textsanskrit{Theragāthā} and \textsanskrit{Therīgāthā}, and overwhelmingly, the teachings, practices, and insights are the same regardless of sex and gender. The word for “womanhood” (\textit{\textsanskrit{itthibhāva}}), for example, is mentioned twice in the \textsanskrit{Therīgāthā}, once in response to a sexist attack by \textsanskrit{Māra} the wicked deity (\href{https://suttacentral.net/thig3.8/en/sujato\#2.1}{Thig 3.8:2.1}), and once in reference to a teaching of the Buddha (\href{https://suttacentral.net/thig10.1/en/sujato\#4.1}{Thig 10.1:4.1}). There, gender is raised in a spirit of empathy, recognizing the special kinds of suffering that a woman must endure, sometimes at the hands of men. This is sometimes overinterpreted to imply that such suffering helps women to get enlightened. But the texts never say this. We all have plenty of suffering, and more is not better. It’s dangerous to imply that women \emph{should} suffer because it will lead to their spiritual betterment: that is the message of every exploitative guru. The Buddha was sensitive to the sufferings women had to undergo, and the purpose of his path was so that they could be free.

If we wish to find the differences between the genders, then, it is tempting to use the extra information included in the commentaries, which provide fairly extensive backstories for the monks and nuns. Unfortunately, these narratives are unlikely to contain much in the way of accurate history, and by and large, are tales told within the tradition as collected by male commentators a thousand years later in a different country.

Dialogue around the body and its beauty is found more often among the nuns, but perspectives on the female body are not different in kind from those towards the male body. The Buddha was often described as beautiful, a point which the texts elucidate in great detail. But when an adoring disciple longed to behold the Buddha’s body, he rebuked him, describing his own body as “putrid” (\href{https://suttacentral.net/sn22.87/en/sujato\#3.1}{SN 22.87:3.1}). In the \textsanskrit{Theragāthā}, the monk Nanda reflects on his freedom from his former addiction to beauty and ornamentation (\href{https://suttacentral.net/thag2.19}{Thag 2.19}). Worrying about appearance is not just for women.

The famous verses of \textsanskrit{Ambapālī} (\href{https://suttacentral.net/thig13.1/en/sujato}{Thig 13.1}), where she eloquently details the faded beauty of her aging body, are perhaps the finest expression of the impermanence of the body in all Buddhist literature. But her poetry does not appear in a vacuum. She was a wealthy courtesan who, outfoxing the powerful youths of the Vajji clan, presented the Buddha with the gift of a monastery. As a sex worker, she is not depicted as a victim in need of saving, but as a powerful and independent woman who exercises her choice to help the Buddha and the Sangha. 

She met the Buddha near the end of his life when he described his own body as like a “decrepit cart held together with straps” (\href{https://suttacentral.net/dn16/en/sujato\#2.25.11}{DN 16:2.25.11}), and said he could only escape the weariness and pain of the body in deep meditation. The imposing physical form of the young warrior prince was long gone: it was the greying Buddha of his final days that \textsanskrit{Ambapālī} met. And the sight seems to have made a deep impression on her, for her verses are a meditation on aging.

\textsanskrit{Ambapālī}’s unique contribution lies not in her contemplation of her body as impermanent, but in how she expresses this in verses of unequaled poetic grace. Starting with the crowning glory of her hair, she works gradually through the parts of the body, loosely echoing the standard meditation on parts of the body, which also starts with the hair. Each verse evokes with vivid and specific metaphors the before and after, how she was in her youth and how she is now, in her old age. 

\begin{verse}%
My nose was like a perfect peak, \\
lovely in my bloom of youth; \\
now old, it’s shriveled like a pepper; \\
the word of the truthful one is confirmed.

%
\end{verse}

The verses must have been composed some decades after the Buddha’s death, for she was not old when she met him in \textsanskrit{Vesālī}. She employs to the full the poetic arts in which she had trained as a courtesan, showing off the old skills that she learned with the intent to entice, now employed for dispassion. Yet nowhere is there a skerrick of shame or reproach for her past career as a sex worker. It is simply how the sensual world is. These are verses of dignity and acceptance, not of shame and repentance. She is still the same woman she was then, but her beauty is now in the wisdom that takes form in her words. Where the Buddha was blunt and unsparing in his assessment of his own body, \textsanskrit{Ambapālī} elevates each detail, devoting the same loving care to both the glories of her past and the fading of the present. Her body in all its changes is her witness to the truth. And the verses, through describing and accepting both the beautiful and the ugly, create a new kind of beauty. Their effect is not repulsive or depressing, but uplifting.

While \textsanskrit{Ambapālī} creates a new kind of effect through her poetic skills, we also find more simple expressions of the contemplation of the body. In later forms of Buddhist literature, it is often the case that such contemplation is gendered: it is male observers seeing the decay of the female body. But this is not the case in early Buddhism. To illustrate this, let’s look at a short passage that is uttered by three mendicants: the nuns \textsanskrit{Abhirūpānandā} (\href{https://suttacentral.net/thig2.1/en/sujato}{Thig 2.1}) and \textsanskrit{Sundarīnandā} (\href{https://suttacentral.net/thig5.4/en/sujato}{Thig 5.4}), and the monk Kulla (\href{https://suttacentral.net/thag6.4}{Thag 6.4}). The verse is in the second person as it is addressed to the mendicant concerned, and the only difference between the versions is the name. The nuns are addressed as “\textsanskrit{Nandā}”.

\begin{verse}%
\textsanskrit{Nandā}, see this bag of bones as \\
diseased, filthy, and rotten.

%
\end{verse}

From this we learn that the monks and nuns received the same meditation instructions. They undertook the same practices, and did so within a shared body of teaching and understanding, aiming for the same goal, of freedom from attachment.

There is an interesting difference, however. The nuns reflect in the way that is normal in the early texts, starting with “this body”, i.e. their own depersonalized body, rather than one that is objectified and externalized. For Kulla, however, his reflection came about because he went to a cemetery for meditation, where he saw the worm-eaten corpse of a woman’s body, prompting the reflection on the body’s repulsiveness. At first glance, this is reminiscent of how in later texts an objectified woman’s body becomes the object of repulsion for men.

All is not, however, as it seems. The first verse of Kulla is in the first person, while the second verse, which contains the passage shared with the nuns, is in the second person: an unnamed speaker is addressing Kulla. But if he entered the cemetery for solitary meditation, who is talking to him? Of course, it’s not anyone: Kulla is reflecting on a teaching that had previously been given to him, presumably by the Buddha. So the Buddha’s teaching to Kulla turns out to have been identical to that which he gave the nuns: to reflect on “this body” as repulsive. It is Kulla who, spurred by the sight of the rotting female corpse, takes up the “teaching as a mirror”, reflecting that his body is no different. The whole point is to undermine the distinction between the body he has and the one he observes, between internal and external, between the “male” body and the “female” body, to see that they share the same nature. Such meditations don’t enforce gender, they erase it.

These are strong practices to be sure. And the Buddhist texts warn of the distressing consequences of undertaking them when not balanced and emotionally ready. For those who do undertake them in the right way, they lead not to distress, but to solace; not to anxiety but to relief; and ultimately to the stillness and peace of immersion in meditation. \textsanskrit{Sundarīnandā} says she “saw” the truth, and now is “quenched and at peace”. Similarly, Kulla says:

\begin{verse}%
Even the music of a five-piece band \\
can never give such pleasure \\
as when, with unified mind, \\
you rightly discern the Dhamma.

%
\end{verse}

So the monks and nuns were both given the same teaching; they both used that teaching to develop a meditative reflection; doing so they found peace of mind; and the peace of mind led to freedom and happiness. When the opportunity came for the monk Kulla to objectify and differentiate his own body from a female body, he took the opposite approach, erasing differences by seeing himself in her.

The Buddhist approach is not “body positivity” but “body neutrality”. Those who see the body entranced by desire need the strong pill of repulsion to clear their minds. But this would be exactly the wrong approach for someone affected by anxiety or insecurity regarding their body, for whom a solid foundation in breath meditation or loving-kindness meditation is recommended. An uncritical body positivity too often leads, in a fairly straight line, to exploitation and abuse of women at the hands of men on the one hand, and body-image distress among women on the other.  

The early Buddhist texts have both positive and negative depictions of bodies, and this is completely realistic: bodies are both beautiful and disgusting. Buddhism is the original anti-essentialism. We don’t focus on one aspect of the body or another because that is the essence of what the body is, but to counter imbalances and lead to a healthy and reasonable equanimity. To take this seriously is to challenge the implicit binding of both men and women to their bodies and to move beyond such binary divisions. 

\section*{How This Translation Came About}

The process of creating the translation was this. For many years, we at SuttaCentral had gathered translations in many modern languages, aiming for as broad a coverage as possible. I was keen to create a complete set of translations for early Buddhist texts. Yet at that time, not all early Buddhist texts were freely available on the internet in English, and I wanted to change that. In 2013 I was approached by Jessica Walton (then \textsanskrit{Ayyā} \textsanskrit{Nibbidā}), a student of mine, who wanted a project to help learn Pali. I suggested that she work on the Thera/Theri-\textsanskrit{gāthā}, in the hope that we could create a freely available translation.

Of course, this is a terrible job for a student—these are among the more difficult texts in the Pali Canon. But I hoped that it would prove useful, and so it has. I suggested that Jessica use Norman’s translation together with the Pali, and work on creating a more readable rendering. She did this, mostly working on her own.

When she was happy with that, she passed the project over to me, and when I got the chance I took it up. I then went over the text in detail, modifying virtually every one of Jessica’s lines, while still keeping many of her turns of phrase. Without her work, this translation would not have been completed. Since this translation was first released in 2014, I have continued to revise, correct, and update in line with my subsequent work.

I also referred heavily to Norman’s translation, which enabled me to make sense of the many obscurities of vocabulary and syntax found in the text. Only rarely have I departed from Norman’s linguistic interpretations, and I have adopted his renderings on occasions when I felt I couldn’t do better.

There are, however, many occasions when Norman’s work is limited by his purely linguistic approach. There is no better example of this than \href{https://suttacentral.net/thag6.7/en/sujato\#1.1}{Thag 6.7:1.1}. The Pali begins \textit{\textsanskrit{uṭṭhehi} \textsanskrit{nisīda}}, on which Norman notes:

\begin{quotation}%
The collocation of “stand up” and “sit down” is strange and one or other of the words is used metaphorically.

%
\end{quotation}

He then renders the verse thus:

\begin{quotation}%
Stand up, \textsanskrit{Kātiyāna}, pay attention; do not be full of sleep, be awake. May the kinsman of the indolent, king death, not conquer lazy you, as though with a snare.

%
\end{quotation}

But to any meditator there is nothing strange about this at all; it just means to get up and meditate. I render the verse:

\begin{verse}%
Get up, \textsanskrit{Kātiyāna}, and sit! \\
Don’t sleep too much, be wakeful. \\
Don’t be lazy, and let the kinsman of the heedless, \\
The king of death, catch you in his trap.

%
\end{verse}

In addition to Norman’s translation, I consulted translations by Bhikkhu \textsanskrit{Ṭhānissaro} and Bhikkhu Bodhi for a few verses. However, I did not consult the Rhys Davids translation at all.

In the original translation, I followed the common practice of leaving a few terms in Pali, as they refer to refined spiritual concepts for which we simply have no parallels in the West. I have subsequently decided to translate these, but the notes I made are, I think, still relevant, and I have adapted and expanded them here.

\subsection*{\textit{\textsanskrit{Samādhi}} (“immersion”)}

This is the poster child for those who believe that it’s best to just leave problematic terms in the original. But if we leave it in the Indic, whose \textit{\textsanskrit{samādhi}} do we mean? In the Brahmanical tradition, \textit{\textsanskrit{samādhi}} means the transcendent union of the individual self with the cosmic divinity, or else the death of a sage. In Thailand, \textit{\textsanskrit{samādhi}} commonly means “meditate”, so to “sit \textit{\textsanskrit{samādhi}}” doesn’t necessarily imply abiding in deep absorption. In the vipassana schools, \textit{\textsanskrit{samādhi}} refers to a moment-to-moment mindful awareness of changing phenomena. In the Suttas, \textit{\textsanskrit{samādhi}} means a deep state of meditative absorption, or \textit{\textsanskrit{jhāna}}. What a reader takes away depends not on any intrinsic meaning of the word—for words have no intrinsic meaning—but on their prior conditioning.

In Buddhism, or at least in early Buddhism, it is an exalted term, as it was for the brahmins, although of course without the metaphysical implications. It means the transcendence of the realm of the senses, the union of the mind in a deep, serene, stillness; a state of mind so powerful it literally makes you God. Through the practice of \textit{\textsanskrit{samādhi}} one would be reborn in the realms of the highest divinities, the Brahma gods.

It could perhaps be rendered as “coalescence” or “stillness”. My teacher Ajahn Brahm prefers “stillness”, but when I tried using this rendering I found that in context it was often misleading, as it sounded like the meditators were just being still. Also, even as a psychological term, it is very subjective when used without context—what Ajahn Brahm means by stillness is one thing, but what most people think is stillness is quite another. When speaking, one can add context to make the meaning clear, but a translator enjoys no such luxury.

Certainly it does \emph{not} mean “concentration”, which is, I believe, the single most damaging translation in Buddhism. This rendering has misled an entire generation of meditators, who think they have to force themselves to focus on a single point to gain \textit{\textsanskrit{samādhi}}. This is very different from the “vast”, “immeasurable” state of \textit{\textsanskrit{samādhi}} taught by the Buddha, as “broad as the great earth”.

In central doctrinal contexts, \textit{\textsanskrit{samādhi}} is defined as the four \textit{\textsanskrit{jhānas}}. It is true that  \textit{\textsanskrit{samādhi}} is used in a somewhat flexible way in the Suttas, and the meaning is on occasion broader than this. But such contexts tend to be personal, poetic, or specific discussions with a limited scope. When used in the central teachings of Buddhism—the threefold training, the eightfold path, the five spiritual faculties, and so on—\textit{\textsanskrit{samādhi}} means \textit{\textsanskrit{jhāna}}.

After a long time, I settled on “immersion” for \textit{\textsanskrit{samādhi}}. I came upon it by seeing how the word “immerse” was used in common English. It means a state of mind that is absorbed or engrossed: “I was so immersed in the book, I didn’t hear you come home.” Tech folks use it to describe an application that draws a user in without being distracted by things outside. I think this captures an important part of the meaning of \textit{\textsanskrit{samādhi}}. Importantly, it retains something of a sense of being a special or technical word, so it’s less likely to be misunderstood.

\subsection*{\textit{\textsanskrit{Jhāna}} (“absorption”)}

This is also an exalted state, and cannot be translated as “meditation”, which is, rather, the practice that leads to \textit{\textsanskrit{jhāna}}. In the Suttas this is called \textit{\textsanskrit{satipaṭṭhāna}} (“mindfulness meditation”).

Terms derived from the same root as root \textit{\textsanskrit{jhāna}} often emphasize mental steadiness or focus: a donkey gazing at their food, or the ruminations of a depressed and obsessive person obsessing. But in Pali, the term has a dual sense, for it also refers to lamps burning, and this is applied quite explicitly in the context of meditation (\href{https://suttacentral.net/mn127/en/sujato\#16.4}{MN 127:16.4}).

I think the Pali usage draws from the Brahmanical concept of \textit{\textsanskrit{dhī}}, which is the divine inspiration of the rising sun, filling the world with light, and raising the mind to awakening. \textit{\textsanskrit{Dhī}} is used twice in the most famous of the Vedic verses, the \textsanskrit{Gāyatrī} Mantra,  which is recited at dawn by Brahmans: “We lift our minds to the divine radiance of the glorious sun: may he waken our minds!” This verse is referred to several times in the Pali texts, where it is called the \textsanskrit{Sāvittī}, which is a name for the Sun in its role as empowerer or vivifier (\href{https://suttacentral.net/snp3.4/en/sujato\#7.3}{Snp 3.4:7.3}, \href{https://suttacentral.net/mn92/en/sujato\#26.2}{MN 92:26.2}, \href{https://suttacentral.net/pli-tv-kd6/en/brahmali\#35.8.3}{Vinaya Khandhaka 6:35.8.3}).

This would suggest a rendering like “illumination”, but this is not a natural idiom for meditation in English. “Absorption” has served translators fairly well, and I follow suit.

While the dual roots of \textit{\textsanskrit{jhāna}} are well known, it is a curious detail that \textit{\textsanskrit{samādhi}}, too, has dual roots, and they are parallel to those for \textit{\textsanskrit{jhāna}}. Like \textit{\textsanskrit{jhāna}}, \textit{\textsanskrit{samādhi}} has a primary sense of composure, centeredness, gathering, like a tortoise drawing its limbs in the shell. Less commonly, a homonym refers to the “kindling” of a fire. So both \textit{jhana} and \textit{\textsanskrit{samādhi}} convey the dual senses of “coalescence” and “illumination”, which reflect the extremely common association of these states with the light and radiance experienced by the meditator.

\section*{A Brief Textual History}

The \textsanskrit{Theragāthā} was published by the Pali Text Society in 1883, with Hermann Oldenberg and Richard Pischel serving as editors. They made use of two manuscripts in Burmese scripts, and one in Sinhala, as well as a commentary in Sinhalese characters. Oldenburg pointed out the widespread occurrence of mistakes common to all three editions as an unmistakable sign that they all ultimately hailed from the same source. As is well known, almost all Sinhalese manuscripts available today were copied from originals brought back from Burma, and do not constitute a separate Sinhalese lineage.

A second edition of this text was published in 1966 (reprinted 1990) with two new Appendices: additional variant readings supplied by K.R. Norman and metrical analysis by L. Alsdorf.

My translation is based on the \textsanskrit{Mahāsaṅgīti} edition of the Pali canon as published on SuttaCentral. It numbers 1289 verses as opposed to the 1279 of the PTS editions. The extra verses arise, not from a difference in substance, but from the inclusion of repetitions that were absent from the PTS editions. The first set of extra verses is at verse 1020 and the second at verse 1161. Up to verse 1020, therefore, the numbering is the same in the SuttaCentral and PTS editions.

The \textsanskrit{Theragāthā} has been fully translated into English twice before, both times published by the Pali Text Society. The efforts of the former translators are indispensable, and their work makes each succeeding attempt that much easier. Nevertheless, the limitations of these earlier translations are well known.

The first translation was by Caroline A.F. Rhys Davids in 1913 as \textit{Psalms of the Brethren}. It employed the then-fashionable approach of using archaic language as a signifier of religious significance. Modern readers unfamiliar with early 20th-century writing might be forgiven for thinking that this was simply how people spoke back then. But even at the time it was old-fashioned, and quite deliberately so. The translation was the first attempt by a serious scholar to capture both the meaning and the excitement of the text, as well as quotes from the commentaries.

A much-needed update was supplied by K.R. Norman in 1969 as \textit{Elders’ Verses, volume I}, later reissued without notes in paperback as \textit{Poems of Early Buddhist Monks}. As if reacting to the emotionality of Rhys David’s version, Norman’s translation employed what Norman himself described as “a starkness and austerity of words which borders on the ungrammatical”. It is an exemplary linguistic study, and it serves well as a resource for scholars and as a reference point for future translators.

\textsanskrit{Sāmaṇera} Mahinda published a bilingual Pali-English edition in 2022 under the title \textit{\textsanskrit{Theragāthā}: Verses of the Elder Bhikkhus} through Dhamma Publishers. 97 of the \textsanskrit{Theragāthā} verses have also been translated by Ven. \textsanskrit{Ṭhānissaro}.

%
\chapter*{Acknowledgements}
\addcontentsline{toc}{chapter}{Acknowledgements}
\markboth{Acknowledgements}{Acknowledgements}

I remember with gratitude all those from whom I have learned the Dhamma, especially Ajahn Brahm and Bhikkhu Bodhi, the two monks who more than anyone else showed me the depth, meaning, and practical value of the Suttas.

Special thanks to Dustin and Keiko Cheah and family, who sponsored my stay in Qi Mei while I made this translation.

Thanks also for Blake Walshe, who provided essential software support for my translation work.

Throughout the process of translation, I have frequently sought feedback and suggestions from the SuttaCentral community on our forum, “Discuss and Discover”. I want to thank all those who have made suggestions and contributed to my understanding, as well as to the moderators who have made the forum possible. These translations were significantly improved due to the careful work of my proofreaders: \textsanskrit{Ayyā} \textsanskrit{Pāsādā}, John and Lynn Kelly, and Derek Sola. Special thanks are due to \textsanskrit{Sabbamittā}, a true friend of all, who has tirelessly and precisely checked my work.

Finally my everlasting thanks to all those people, far too many to mention, who have supported SuttaCentral, and those who have supported my life as a monastic. None of this would be possible without you.

%
\mainmatter%
\pagestyle{fancy}%
\addtocontents{toc}{\let\protect\contentsline\protect\nopagecontentsline}
\part*{The Book of the Ones }
\addcontentsline{toc}{part}{The Book of the Ones }
\markboth{}{}
\addtocontents{toc}{\let\protect\contentsline\protect\oldcontentsline}

%
\addtocontents{toc}{\let\protect\contentsline\protect\nopagecontentsline}
\chapter*{Chapter One }
\addcontentsline{toc}{chapter}{\tocchapterline{Chapter One }}
\addtocontents{toc}{\let\protect\contentsline\protect\oldcontentsline}

%
\section*{{\suttatitleacronym Thag 1.1}{\suttatitletranslation Subhūti }{\suttatitleroot Subhūtittheragāthā}}
\addcontentsline{toc}{section}{\tocacronym{Thag 1.1} \toctranslation{Subhūti } \tocroot{Subhūtittheragāthā}}
\markboth{Subhūti }{Subhūtittheragāthā}
\extramarks{Thag 1.1}{Thag 1.1}

\subsection*{Background }

\scnamo{Homage to that Blessed One, the perfected one, the fully awakened Buddha! }

\begin{verse}%
Like\marginnote{1.1} the lions of mighty fang \\
who roar in mountain caves—\\
hear now from those who’ve practiced well \\
their own verses about themselves. 

What\marginnote{2.1} their name, and what their clan, \\
and how they lived by the teaching; \\
how dedicated were those wise ones, \\
as they meditated tirelessly. 

Clearly\marginnote{3.1} discerning in every case, \\
they reached the state that does not pass. \\
Reviewing their completed task, \\
they spoke about it in these words. 

%
\end{verse}

\subsection*{\textsanskrit{Subhūti} }

\begin{verse}%
My\marginnote{4.1} little hut is roofed and pleasant, \\>sheltered from the wind: \\
so rain, heavens, as you please!\footnote{The word \textit{deva} (“god”) is closely connected with the idea of the sky. The “god” of thunder and rain (perhaps to be identified with the god named Pajjuna) is a terrifying threat to those who are exposed outdoors. Here the monk snug in his little hut invokes the god in contrast to the safety of peace. This idea recurs many times in the \textsanskrit{Theragāthā}. At \href{https://suttacentral.net/snp1.2/en/sujato}{Snp 1.2} the same notion occurs in the context of making household offerings to the gods. } \\
My mind is serene and freed, \\
I practice wholeheartedly: so rain, heavens! 

%
\end{verse}

\scendsutta{That is how this verse was recited by the senior venerable \textsanskrit{Subhūti}. }

%
\section*{{\suttatitleacronym Thag 1.2}{\suttatitletranslation Mahākoṭṭhita }{\suttatitleroot Mahākoṭṭhikattheragāthā}}
\addcontentsline{toc}{section}{\tocacronym{Thag 1.2} \toctranslation{Mahākoṭṭhita } \tocroot{Mahākoṭṭhikattheragāthā}}
\markboth{Mahākoṭṭhita }{Mahākoṭṭhikattheragāthā}
\extramarks{Thag 1.2}{Thag 1.2}

\begin{verse}%
Calm\marginnote{1.1} and still, \\
thoughtful in counsel, not restless—\\
he shakes off bad qualities \\
as the gale shakes leaves off a tree. 

%
\end{verse}

\scendsutta{That is how this verse was recited by the senior venerable \textsanskrit{Mahākoṭṭhita}. }

%
\section*{{\suttatitleacronym Thag 1.3}{\suttatitletranslation Revata the Doubter }{\suttatitleroot Kaṅkhārevatattheragāthā}}
\addcontentsline{toc}{section}{\tocacronym{Thag 1.3} \toctranslation{Revata the Doubter } \tocroot{Kaṅkhārevatattheragāthā}}
\markboth{Revata the Doubter }{Kaṅkhārevatattheragāthā}
\extramarks{Thag 1.3}{Thag 1.3}

\begin{verse}%
See\marginnote{1.1} this wisdom of the Realized Ones! \\
Like a fire blazing in the night, \\
giving light, giving vision, \\
they dispel the doubt of those who’ve come. 

%
\end{verse}

\scendsutta{That is how this verse was recited by the senior venerable Revata the Doubter. }

%
\section*{{\suttatitleacronym Thag 1.4}{\suttatitletranslation Puṇṇa (1st) }{\suttatitleroot Puṇṇattheragāthā}}
\addcontentsline{toc}{section}{\tocacronym{Thag 1.4} \toctranslation{Puṇṇa (1st) } \tocroot{Puṇṇattheragāthā}}
\markboth{Puṇṇa (1st) }{Puṇṇattheragāthā}
\extramarks{Thag 1.4}{Thag 1.4}

\begin{verse}%
Associate\marginnote{1.1} only with the virtuous, \\
the astute ones who see the goal. \\
The wise ones, diligent and clear-seeing, \\
realize the goal \\
so great and profound, \\
hard to see, subtle, and fine. 

%
\end{verse}

\scendsutta{That is how this verse was recited by the senior venerable \textsanskrit{Puṇṇa} son of \textsanskrit{Mantāṇī}.\footnote{This \textsanskrit{Puṇṇa} is famous for his conversation with \textsanskrit{Sāriputta} at \href{https://suttacentral.net/mn24/en/sujato}{MN 24}. } }

%
\section*{{\suttatitleacronym Thag 1.5}{\suttatitletranslation Dabba }{\suttatitleroot Dabbattheragāthā}}
\addcontentsline{toc}{section}{\tocacronym{Thag 1.5} \toctranslation{Dabba } \tocroot{Dabbattheragāthā}}
\markboth{Dabba }{Dabbattheragāthā}
\extramarks{Thag 1.5}{Thag 1.5}

\begin{verse}%
Once\marginnote{1.1} hard to tame, now tamed himself; \\
clever, content, with doubt overcome; \\
victorious since his fears have vanished: \\
Dabba is steadfast, and has become fully quenched. 

%
\end{verse}

\scendsutta{That is how this verse was recited by the senior venerable Dabba. }

%
\section*{{\suttatitleacronym Thag 1.6}{\suttatitletranslation Sītavaniya }{\suttatitleroot Sītavaniyattheragāthā}}
\addcontentsline{toc}{section}{\tocacronym{Thag 1.6} \toctranslation{Sītavaniya } \tocroot{Sītavaniyattheragāthā}}
\markboth{Sītavaniya }{Sītavaniyattheragāthā}
\extramarks{Thag 1.6}{Thag 1.6}

\begin{verse}%
The\marginnote{1.1} monk who went to the Cool Grove is solitary, \\
content and serene, \\
victorious, with goosebumps vanished, \\
guarding mindfulness of the body, steadfast. 

%
\end{verse}

\scendsutta{That is how this verse was recited by the senior venerable \textsanskrit{Sītavaniya}. }

%
\section*{{\suttatitleacronym Thag 1.7}{\suttatitletranslation Bhalliya }{\suttatitleroot Bhalliyattheragāthā}}
\addcontentsline{toc}{section}{\tocacronym{Thag 1.7} \toctranslation{Bhalliya } \tocroot{Bhalliyattheragāthā}}
\markboth{Bhalliya }{Bhalliyattheragāthā}
\extramarks{Thag 1.7}{Thag 1.7}

\begin{verse}%
He\marginnote{1.1} has swept away the army of the King of Death, \\
as a fragile bridge of reeds by a great flood. \\
Victorious since his fears have vanished: \\
tame and steadfast, he has become quenched. 

%
\end{verse}

\scendsutta{That is how this verse was recited by the senior venerable Bhalliya. }

%
\section*{{\suttatitleacronym Thag 1.8}{\suttatitletranslation Vīra }{\suttatitleroot Vīrattheragāthā}}
\addcontentsline{toc}{section}{\tocacronym{Thag 1.8} \toctranslation{Vīra } \tocroot{Vīrattheragāthā}}
\markboth{Vīra }{Vīrattheragāthā}
\extramarks{Thag 1.8}{Thag 1.8}

\begin{verse}%
Once\marginnote{1.1} hard to tame, now tamed himself; \\
a hero, content, with doubt overcome; \\
victorious, with goosebumps vanished, \\
\textsanskrit{Vīra} is steadfast, and has become fully quenched. 

%
\end{verse}

\scendsutta{That is how this verse was recited by the senior venerable \textsanskrit{Vīra}. }

%
\section*{{\suttatitleacronym Thag 1.9}{\suttatitletranslation Pilindavaccha }{\suttatitleroot Pilindavacchattheragāthā}}
\addcontentsline{toc}{section}{\tocacronym{Thag 1.9} \toctranslation{Pilindavaccha } \tocroot{Pilindavacchattheragāthā}}
\markboth{Pilindavaccha }{Pilindavacchattheragāthā}
\extramarks{Thag 1.9}{Thag 1.9}

\begin{verse}%
It\marginnote{1.1} was welcome, not unwelcome, \\
the advice I got was good. \\
Of the well-explained teachings, \\
I arrived at the best. 

%
\end{verse}

\scendsutta{That is how this verse was recited by the senior venerable Pilindavaccha. }

%
\section*{{\suttatitleacronym Thag 1.10}{\suttatitletranslation Puṇṇamāsa (1st) }{\suttatitleroot Puṇṇamāsattheragāthā}}
\addcontentsline{toc}{section}{\tocacronym{Thag 1.10} \toctranslation{Puṇṇamāsa (1st) } \tocroot{Puṇṇamāsattheragāthā}}
\markboth{Puṇṇamāsa (1st) }{Puṇṇamāsattheragāthā}
\extramarks{Thag 1.10}{Thag 1.10}

\begin{verse}%
A\marginnote{1.1} knowledge master, peaceful and self-controlled, \\
is rid of concern for this world and the world beyond. \\
Unsullied in the midst of all things, \\
they’d know the arising and passing of the world. 

%
\end{verse}

\scendsutta{That is how this verse was recited by the senior venerable \textsanskrit{Puṇṇamāsa}. }

%
\addtocontents{toc}{\let\protect\contentsline\protect\nopagecontentsline}
\chapter*{Chapter Two }
\addcontentsline{toc}{chapter}{\tocchapterline{Chapter Two }}
\addtocontents{toc}{\let\protect\contentsline\protect\oldcontentsline}

%
\section*{{\suttatitleacronym Thag 1.11}{\suttatitletranslation Cūḷavaccha }{\suttatitleroot Cūḷavacchattheragāthā}}
\addcontentsline{toc}{section}{\tocacronym{Thag 1.11} \toctranslation{Cūḷavaccha } \tocroot{Cūḷavacchattheragāthā}}
\markboth{Cūḷavaccha }{Cūḷavacchattheragāthā}
\extramarks{Thag 1.11}{Thag 1.11}

\begin{verse}%
A\marginnote{1.1} monk full of joy \\
in the teaching proclaimed by the Buddha \\
would realize the peaceful state, \\
the blissful stilling of conditions. 

%
\end{verse}

%
\section*{{\suttatitleacronym Thag 1.12}{\suttatitletranslation Mahāvaccha }{\suttatitleroot Mahāvacchattheragāthā}}
\addcontentsline{toc}{section}{\tocacronym{Thag 1.12} \toctranslation{Mahāvaccha } \tocroot{Mahāvacchattheragāthā}}
\markboth{Mahāvaccha }{Mahāvacchattheragāthā}
\extramarks{Thag 1.12}{Thag 1.12}

\begin{verse}%
Empowered\marginnote{1.1} by wisdom, \\>with precepts and observances intact, \\
serene, loving absorption, mindful, \\
eating just the needed food, \\
one should await one’s time here, free of desire. 

%
\end{verse}

%
\section*{{\suttatitleacronym Thag 1.13}{\suttatitletranslation Vanavaccha (1st) }{\suttatitleroot Vanavacchattheragāthā}}
\addcontentsline{toc}{section}{\tocacronym{Thag 1.13} \toctranslation{Vanavaccha (1st) } \tocroot{Vanavacchattheragāthā}}
\markboth{Vanavaccha (1st) }{Vanavacchattheragāthā}
\extramarks{Thag 1.13}{Thag 1.13}

\begin{verse}%
Glistening,\marginnote{1.1} they look like blue storm clouds, \\
with waters cool and streams so clear, \\
and covered all in ladybugs: \\
these rocky crags delight me! 

%
\end{verse}

%
\section*{{\suttatitleacronym Thag 1.14}{\suttatitletranslation The Novice Sivaka }{\suttatitleroot Sivakasāmaṇeragāthā}}
\addcontentsline{toc}{section}{\tocacronym{Thag 1.14} \toctranslation{The Novice Sivaka } \tocroot{Sivakasāmaṇeragāthā}}
\markboth{The Novice Sivaka }{Sivakasāmaṇeragāthā}
\extramarks{Thag 1.14}{Thag 1.14}

\begin{verse}%
My\marginnote{1.1} mentor said to me: \\
“Let’s leave here, \textsanskrit{Sīvaka}.” \\
My body lives in the village, \\
but my mind has gone to the wilderness. \\
I go there even when lying down—\\
you can't snare those who know. 

%
\end{verse}

%
\section*{{\suttatitleacronym Thag 1.15}{\suttatitletranslation Kuṇḍadhāna }{\suttatitleroot Kuṇḍadhānattheragāthā}}
\addcontentsline{toc}{section}{\tocacronym{Thag 1.15} \toctranslation{Kuṇḍadhāna } \tocroot{Kuṇḍadhānattheragāthā}}
\markboth{Kuṇḍadhāna }{Kuṇḍadhānattheragāthā}
\extramarks{Thag 1.15}{Thag 1.15}

\begin{verse}%
Five\marginnote{1.1} to cut, five to drop, \\
and five more to develop. \\
When a mendicant slips five chains \\
they’re said to have crossed the flood. 

%
\end{verse}

%
\section*{{\suttatitleacronym Thag 1.16}{\suttatitletranslation Belaṭṭhasīsa }{\suttatitleroot Belaṭṭhasīsattheragāthā}}
\addcontentsline{toc}{section}{\tocacronym{Thag 1.16} \toctranslation{Belaṭṭhasīsa } \tocroot{Belaṭṭhasīsattheragāthā}}
\markboth{Belaṭṭhasīsa }{Belaṭṭhasīsattheragāthā}
\extramarks{Thag 1.16}{Thag 1.16}

\begin{verse}%
Just\marginnote{1.1} as a fine thoroughbred \\
proceeds with ease, \\
tail and mane flying in the wind; \\
so my days and nights \\
proceed with ease, \\
full of joy not of the flesh. 

%
\end{verse}

%
\section*{{\suttatitleacronym Thag 1.17}{\suttatitletranslation Dāsaka }{\suttatitleroot Dāsakattheragāthā}}
\addcontentsline{toc}{section}{\tocacronym{Thag 1.17} \toctranslation{Dāsaka } \tocroot{Dāsakattheragāthā}}
\markboth{Dāsaka }{Dāsakattheragāthā}
\extramarks{Thag 1.17}{Thag 1.17}

\begin{verse}%
One\marginnote{1.1} who gets drowsy from overeating, \\
fond of sleep, rolling round the bed \\
like a great hog stuffed with grain: \\
that dullard returns to the womb again and again. 

%
\end{verse}

%
\section*{{\suttatitleacronym Thag 1.18}{\suttatitletranslation Siṅgāla’s Father }{\suttatitleroot Siṅgālapituttheragāthā}}
\addcontentsline{toc}{section}{\tocacronym{Thag 1.18} \toctranslation{Siṅgāla’s Father } \tocroot{Siṅgālapituttheragāthā}}
\markboth{Siṅgāla’s Father }{Siṅgālapituttheragāthā}
\extramarks{Thag 1.18}{Thag 1.18}

\begin{verse}%
There\marginnote{1.1} was an heir of the Buddha, \\
a monk in \textsanskrit{Bhesakaḷā} forest, \\
who suffused the entire earth \\
with the perception of bones. \\
I think he will quickly \\
get rid of sensual desire. 

%
\end{verse}

%
\section*{{\suttatitleacronym Thag 1.19}{\suttatitletranslation Kula }{\suttatitleroot Kulattheragāthā}}
\addcontentsline{toc}{section}{\tocacronym{Thag 1.19} \toctranslation{Kula } \tocroot{Kulattheragāthā}}
\markboth{Kula }{Kulattheragāthā}
\extramarks{Thag 1.19}{Thag 1.19}

\begin{verse}%
Irrigators\marginnote{1.1} guide water, \\
fletchers shape arrows, \\
carpenters carve timber; \\
those true to their vows tame themselves. 

%
\end{verse}

%
\section*{{\suttatitleacronym Thag 1.20}{\suttatitletranslation Ajita }{\suttatitleroot Ajitattheragāthā}}
\addcontentsline{toc}{section}{\tocacronym{Thag 1.20} \toctranslation{Ajita } \tocroot{Ajitattheragāthā}}
\markboth{Ajita }{Ajitattheragāthā}
\extramarks{Thag 1.20}{Thag 1.20}

\begin{verse}%
I\marginnote{1.1} do not fear death; \\
nor do I long for life. \\
I’ll lay down this body, \\
aware and mindful. 

%
\end{verse}

%
\addtocontents{toc}{\let\protect\contentsline\protect\nopagecontentsline}
\chapter*{Chapter Three }
\addcontentsline{toc}{chapter}{\tocchapterline{Chapter Three }}
\addtocontents{toc}{\let\protect\contentsline\protect\oldcontentsline}

%
\section*{{\suttatitleacronym Thag 1.21}{\suttatitletranslation Nigrodha }{\suttatitleroot Nigrodhattheragāthā}}
\addcontentsline{toc}{section}{\tocacronym{Thag 1.21} \toctranslation{Nigrodha } \tocroot{Nigrodhattheragāthā}}
\markboth{Nigrodha }{Nigrodhattheragāthā}
\extramarks{Thag 1.21}{Thag 1.21}

\begin{verse}%
I’m\marginnote{1.1} not afraid of fear, \\
for our teacher is expert in freedom from death. \\
Mendicants advance by the path \\
where no fear remains. 

%
\end{verse}

%
\section*{{\suttatitleacronym Thag 1.22}{\suttatitletranslation Cittaka }{\suttatitleroot Cittakattheragāthā}}
\addcontentsline{toc}{section}{\tocacronym{Thag 1.22} \toctranslation{Cittaka } \tocroot{Cittakattheragāthā}}
\markboth{Cittaka }{Cittakattheragāthā}
\extramarks{Thag 1.22}{Thag 1.22}

\begin{verse}%
Crested\marginnote{1.1} peacocks with beautiful blue necks \\
cry out in \textsanskrit{Karaṁvī}. \\
Stirred by a cool breeze, \\
they wake the sleeper to practice absorption. 

%
\end{verse}

%
\section*{{\suttatitleacronym Thag 1.23}{\suttatitletranslation Gosāla }{\suttatitleroot Gosālattheragāthā}}
\addcontentsline{toc}{section}{\tocacronym{Thag 1.23} \toctranslation{Gosāla } \tocroot{Gosālattheragāthā}}
\markboth{Gosāla }{Gosālattheragāthā}
\extramarks{Thag 1.23}{Thag 1.23}

\begin{verse}%
I’ll\marginnote{1.1} eat honey and milk-rice \\
in \textsanskrit{Veḷugumba}. \\
And then, skillfully scrutinizing \\
the rise and fall of the aggregates, \\
I’ll return to my forest hill \\
and foster seclusion. 

%
\end{verse}

%
\section*{{\suttatitleacronym Thag 1.24}{\suttatitletranslation Sugandha }{\suttatitleroot Sugandhattheragāthā}}
\addcontentsline{toc}{section}{\tocacronym{Thag 1.24} \toctranslation{Sugandha } \tocroot{Sugandhattheragāthā}}
\markboth{Sugandha }{Sugandhattheragāthā}
\extramarks{Thag 1.24}{Thag 1.24}

\begin{verse}%
See\marginnote{1.1} the excellence of the teaching! \\
Just one rainy season after I went forth, \\
I attained the three knowledges \\
and fulfilled the Buddha’s instructions. 

%
\end{verse}

%
\section*{{\suttatitleacronym Thag 1.25}{\suttatitletranslation Nandiya }{\suttatitleroot Nandiyattheragāthā}}
\addcontentsline{toc}{section}{\tocacronym{Thag 1.25} \toctranslation{Nandiya } \tocroot{Nandiyattheragāthā}}
\markboth{Nandiya }{Nandiyattheragāthā}
\extramarks{Thag 1.25}{Thag 1.25}

\begin{verse}%
Dark\marginnote{1.1} One, after attacking such a monk—\\
one who has arrived at the fruit, \\
and whose mind is always full of light—\\
you’ll fall into suffering. 

%
\end{verse}

%
\section*{{\suttatitleacronym Thag 1.26}{\suttatitletranslation Abhaya }{\suttatitleroot Abhayattheragāthā}}
\addcontentsline{toc}{section}{\tocacronym{Thag 1.26} \toctranslation{Abhaya } \tocroot{Abhayattheragāthā}}
\markboth{Abhaya }{Abhayattheragāthā}
\extramarks{Thag 1.26}{Thag 1.26}

\begin{verse}%
Having\marginnote{1.1} heard the fine words \\
of the Buddha, the kinsman of the Sun, \\
I penetrated the subtle truth, \\
like a hair-tip with an arrow. 

%
\end{verse}

%
\section*{{\suttatitleacronym Thag 1.27}{\suttatitletranslation Lomasakaṅgiya }{\suttatitleroot Lomasakaṅgiyattheragāthā}}
\addcontentsline{toc}{section}{\tocacronym{Thag 1.27} \toctranslation{Lomasakaṅgiya } \tocroot{Lomasakaṅgiyattheragāthā}}
\markboth{Lomasakaṅgiya }{Lomasakaṅgiyattheragāthā}
\extramarks{Thag 1.27}{Thag 1.27}

\begin{verse}%
With\marginnote{1.1} my chest I’ll thrust aside \\
the grasses, vines, \\
reeds and creepers, \\
and foster seclusion. 

%
\end{verse}

%
\section*{{\suttatitleacronym Thag 1.28}{\suttatitletranslation Jambugāmikaputta }{\suttatitleroot Jambugāmikaputtattheragāthā}}
\addcontentsline{toc}{section}{\tocacronym{Thag 1.28} \toctranslation{Jambugāmikaputta } \tocroot{Jambugāmikaputtattheragāthā}}
\markboth{Jambugāmikaputta }{Jambugāmikaputtattheragāthā}
\extramarks{Thag 1.28}{Thag 1.28}

\begin{verse}%
Aren’t\marginnote{1.1} you obsessed with clothes? \\
Don’t you just love jewelry? \\
Is it not you—and no-one else—\\
who spreads the scent of virtue? 

%
\end{verse}

%
\section*{{\suttatitleacronym Thag 1.29}{\suttatitletranslation Hārita (1st) }{\suttatitleroot Hāritattheragāthā}}
\addcontentsline{toc}{section}{\tocacronym{Thag 1.29} \toctranslation{Hārita (1st) } \tocroot{Hāritattheragāthā}}
\markboth{Hārita (1st) }{Hāritattheragāthā}
\extramarks{Thag 1.29}{Thag 1.29}

\begin{verse}%
Straighten\marginnote{1.1} yourself, \\
like a fletcher straightens an arrow. \\
When your mind is sincere, \textsanskrit{Hārita}, \\
break ignorance to bits! 

%
\end{verse}

%
\section*{{\suttatitleacronym Thag 1.30}{\suttatitletranslation Uttiya (1st) }{\suttatitleroot Uttiyattheragāthā}}
\addcontentsline{toc}{section}{\tocacronym{Thag 1.30} \toctranslation{Uttiya (1st) } \tocroot{Uttiyattheragāthā}}
\markboth{Uttiya (1st) }{Uttiyattheragāthā}
\extramarks{Thag 1.30}{Thag 1.30}

\begin{verse}%
When\marginnote{1.1} I was ill in the past, \\
mindfulness arose in me. \\
Now I am ill once more—\\
it’s time for me to be heedful. 

%
\end{verse}

%
\addtocontents{toc}{\let\protect\contentsline\protect\nopagecontentsline}
\chapter*{Chapter Four }
\addcontentsline{toc}{chapter}{\tocchapterline{Chapter Four }}
\addtocontents{toc}{\let\protect\contentsline\protect\oldcontentsline}

%
\section*{{\suttatitleacronym Thag 1.31}{\suttatitletranslation Gahvaratīriya }{\suttatitleroot Gahvaratīriyattheragāthā}}
\addcontentsline{toc}{section}{\tocacronym{Thag 1.31} \toctranslation{Gahvaratīriya } \tocroot{Gahvaratīriyattheragāthā}}
\markboth{Gahvaratīriya }{Gahvaratīriyattheragāthā}
\extramarks{Thag 1.31}{Thag 1.31}

\begin{verse}%
Pestered\marginnote{1.1} by flies and mosquitoes \\
in the wilds, the formidable forest, \\
one should mindfully endure, \\
like an elephant at the head of the battle. 

%
\end{verse}

%
\section*{{\suttatitleacronym Thag 1.32}{\suttatitletranslation Suppiya }{\suttatitleroot Suppiyattheragāthā}}
\addcontentsline{toc}{section}{\tocacronym{Thag 1.32} \toctranslation{Suppiya } \tocroot{Suppiyattheragāthā}}
\markboth{Suppiya }{Suppiyattheragāthā}
\extramarks{Thag 1.32}{Thag 1.32}

\begin{verse}%
I’ll\marginnote{1.1} swap old age for the unaging, \\
burning for quenching—\\
the ultimate peace, \\
the supreme sanctuary from the yoke. 

%
\end{verse}

%
\section*{{\suttatitleacronym Thag 1.33}{\suttatitletranslation Sopāka (1st) }{\suttatitleroot Sopākattheragāthā}}
\addcontentsline{toc}{section}{\tocacronym{Thag 1.33} \toctranslation{Sopāka (1st) } \tocroot{Sopākattheragāthā}}
\markboth{Sopāka (1st) }{Sopākattheragāthā}
\extramarks{Thag 1.33}{Thag 1.33}

\begin{verse}%
Just\marginnote{1.1} as a mother would be good \\
to her beloved only child; \\
so, to creatures all and everywhere, \\
let one be good. 

%
\end{verse}

%
\section*{{\suttatitleacronym Thag 1.34}{\suttatitletranslation Posiya }{\suttatitleroot Posiyattheragāthā}}
\addcontentsline{toc}{section}{\tocacronym{Thag 1.34} \toctranslation{Posiya } \tocroot{Posiyattheragāthā}}
\markboth{Posiya }{Posiyattheragāthā}
\extramarks{Thag 1.34}{Thag 1.34}

\begin{verse}%
It's\marginnote{1.1} always better for a smart person \\
to avoid sharing a seat with such women. \\
I went from the village to the wilderness; \\
from there I entered a house. \\
Though I was there to be fed, \\
I got up and left without taking leave. 

%
\end{verse}

%
\section*{{\suttatitleacronym Thag 1.35}{\suttatitletranslation Sāmaññakāni }{\suttatitleroot Sāmaññakānittheragāthā}}
\addcontentsline{toc}{section}{\tocacronym{Thag 1.35} \toctranslation{Sāmaññakāni } \tocroot{Sāmaññakānittheragāthā}}
\markboth{Sāmaññakāni }{Sāmaññakānittheragāthā}
\extramarks{Thag 1.35}{Thag 1.35}

\begin{verse}%
Seeking\marginnote{1.1} happiness, they find it through this practice. \\
They get a good reputation and grow in fame, \\
those who develop the direct route: \\
the noble eightfold path to realize freedom from death. 

%
\end{verse}

%
\section*{{\suttatitleacronym Thag 1.36}{\suttatitletranslation Kumāputta }{\suttatitleroot Kumāputtattheragāthā}}
\addcontentsline{toc}{section}{\tocacronym{Thag 1.36} \toctranslation{Kumāputta } \tocroot{Kumāputtattheragāthā}}
\markboth{Kumāputta }{Kumāputtattheragāthā}
\extramarks{Thag 1.36}{Thag 1.36}

\begin{verse}%
Learning\marginnote{1.1} is good, living well is good, \\
the life without abode is always good. \\
Questions on the meaning, actions that are skillful: \\
this is the ascetic life for one who has nothing. 

%
\end{verse}

%
\section*{{\suttatitleacronym Thag 1.37}{\suttatitletranslation Kumāputtasahāyaka }{\suttatitleroot Kumāputtasahāyakattheragāthā}}
\addcontentsline{toc}{section}{\tocacronym{Thag 1.37} \toctranslation{Kumāputtasahāyaka } \tocroot{Kumāputtasahāyakattheragāthā}}
\markboth{Kumāputtasahāyaka }{Kumāputtasahāyakattheragāthā}
\extramarks{Thag 1.37}{Thag 1.37}

\begin{verse}%
Some\marginnote{1.1} travel to different countries, \\
wandering undisciplined. \\
If they lose their meditation, \\
what will such rotten conduct achieve? \\
So you should dispel aggression, \\
practicing absorption undistracted. 

%
\end{verse}

%
\section*{{\suttatitleacronym Thag 1.38}{\suttatitletranslation Gavampati }{\suttatitleroot Gavampatittheragāthā}}
\addcontentsline{toc}{section}{\tocacronym{Thag 1.38} \toctranslation{Gavampati } \tocroot{Gavampatittheragāthā}}
\markboth{Gavampati }{Gavampatittheragāthā}
\extramarks{Thag 1.38}{Thag 1.38}

\begin{verse}%
His\marginnote{1.1} psychic power made the river Sarabhu stand still; \\
Gavampati is unbound and unperturbed. \\
The gods bow to that great sage, \\>who has slipped all chains, \\
and gone beyond rebirth. 

%
\end{verse}

%
\section*{{\suttatitleacronym Thag 1.39}{\suttatitletranslation Tissa (1st) }{\suttatitleroot Tissattheragāthā}}
\addcontentsline{toc}{section}{\tocacronym{Thag 1.39} \toctranslation{Tissa (1st) } \tocroot{Tissattheragāthā}}
\markboth{Tissa (1st) }{Tissattheragāthā}
\extramarks{Thag 1.39}{Thag 1.39}

\begin{verse}%
Like\marginnote{1.1} they’re struck by a sword, \\
like their head was on fire, \\
a mendicant should wander mindful \\
to give up sensual desire. 

%
\end{verse}

%
\section*{{\suttatitleacronym Thag 1.40}{\suttatitletranslation Vaḍḍhamāna }{\suttatitleroot Vaḍḍhamānattheragāthā}}
\addcontentsline{toc}{section}{\tocacronym{Thag 1.40} \toctranslation{Vaḍḍhamāna } \tocroot{Vaḍḍhamānattheragāthā}}
\markboth{Vaḍḍhamāna }{Vaḍḍhamānattheragāthā}
\extramarks{Thag 1.40}{Thag 1.40}

\begin{verse}%
Like\marginnote{1.1} they’re struck by a sword, \\
like their head was on fire, \\
a mendicant should wander mindful, \\
to give up desire for rebirth. 

%
\end{verse}

%
\addtocontents{toc}{\let\protect\contentsline\protect\nopagecontentsline}
\chapter*{Chapter Five }
\addcontentsline{toc}{chapter}{\tocchapterline{Chapter Five }}
\addtocontents{toc}{\let\protect\contentsline\protect\oldcontentsline}

%
\section*{{\suttatitleacronym Thag 1.41}{\suttatitletranslation Sirivaḍḍha }{\suttatitleroot Sirivaḍḍhattheragāthā}}
\addcontentsline{toc}{section}{\tocacronym{Thag 1.41} \toctranslation{Sirivaḍḍha } \tocroot{Sirivaḍḍhattheragāthā}}
\markboth{Sirivaḍḍha }{Sirivaḍḍhattheragāthā}
\extramarks{Thag 1.41}{Thag 1.41}

\begin{verse}%
Lightning\marginnote{1.1} flashes down \\
on the cleft of \textsanskrit{Vebhāra} and \textsanskrit{Paṇḍava}. \\
But in the mountain cleft he is absorbed in \textsanskrit{jhāna}—\\
the son of the Buddha, inimitable and unaffected. 

%
\end{verse}

%
\section*{{\suttatitleacronym Thag 1.42}{\suttatitletranslation Revata of the Acacia Wood }{\suttatitleroot Khadiravaniyattheragāthā}}
\addcontentsline{toc}{section}{\tocacronym{Thag 1.42} \toctranslation{Revata of the Acacia Wood } \tocroot{Khadiravaniyattheragāthā}}
\markboth{Revata of the Acacia Wood }{Khadiravaniyattheragāthā}
\extramarks{Thag 1.42}{Thag 1.42}

\begin{verse}%
\textsanskrit{Cāla},\marginnote{1.1} \textsanskrit{Upacāla}, and \textsanskrit{Sīsupacāla} \\
meditate mindfully! \\
I’ve come to you like a hair-splitter. 

%
\end{verse}

%
\section*{{\suttatitleacronym Thag 1.43}{\suttatitletranslation Sumaṅgala }{\suttatitleroot Sumaṅgalattheragāthā}}
\addcontentsline{toc}{section}{\tocacronym{Thag 1.43} \toctranslation{Sumaṅgala } \tocroot{Sumaṅgalattheragāthā}}
\markboth{Sumaṅgala }{Sumaṅgalattheragāthā}
\extramarks{Thag 1.43}{Thag 1.43}

\begin{verse}%
Well\marginnote{1.1} freed! Well freed! \\
I’m very well freed from three crooked things: \\
my sickles, my ploughs, \\
and my little hoes. \\
Even if they were here, right here—\\
I’d be done with them, done! \\
Practice absorption \textsanskrit{Sumaṅgala}! \\>Practice absorption \textsanskrit{Sumaṅgala}! \\
Stay heedful, \textsanskrit{Sumaṅgala}! 

%
\end{verse}

%
\section*{{\suttatitleacronym Thag 1.44}{\suttatitletranslation Sānu }{\suttatitleroot Sānuttheragāthā}}
\addcontentsline{toc}{section}{\tocacronym{Thag 1.44} \toctranslation{Sānu } \tocroot{Sānuttheragāthā}}
\markboth{Sānu }{Sānuttheragāthā}
\extramarks{Thag 1.44}{Thag 1.44}

\begin{verse}%
Mum,\marginnote{1.1} they weep for the dead, \\
or for one who’s alive but has disappeared. \\
I’m alive and you can see me, \\
so mum, why do you weep for me? 

%
\end{verse}

%
\section*{{\suttatitleacronym Thag 1.45}{\suttatitletranslation Ramaṇīyavihārin }{\suttatitleroot Ramaṇīyavihārittheragāthā}}
\addcontentsline{toc}{section}{\tocacronym{Thag 1.45} \toctranslation{Ramaṇīyavihārin } \tocroot{Ramaṇīyavihārittheragāthā}}
\markboth{Ramaṇīyavihārin }{Ramaṇīyavihārittheragāthā}
\extramarks{Thag 1.45}{Thag 1.45}

\begin{verse}%
Though\marginnote{1.1} a fine thoroughbred may stumble, \\
it soon stands firm again. \\
Even so is one accomplished in vision, \\
a disciple of the Buddha. 

%
\end{verse}

%
\section*{{\suttatitleacronym Thag 1.46}{\suttatitletranslation Samiddhi }{\suttatitleroot Samiddhittheragāthā}}
\addcontentsline{toc}{section}{\tocacronym{Thag 1.46} \toctranslation{Samiddhi } \tocroot{Samiddhittheragāthā}}
\markboth{Samiddhi }{Samiddhittheragāthā}
\extramarks{Thag 1.46}{Thag 1.46}

\begin{verse}%
I\marginnote{1.1} went forth out of faith \\
from the lay life to homelessness. \\
My mindfulness and wisdom have grown, \\
my mind is serene. \\
Make whatever illusions you want, \\
it doesn’t bother me. 

%
\end{verse}

%
\section*{{\suttatitleacronym Thag 1.47}{\suttatitletranslation Ujjaya }{\suttatitleroot Ujjayattheragāthā}}
\addcontentsline{toc}{section}{\tocacronym{Thag 1.47} \toctranslation{Ujjaya } \tocroot{Ujjayattheragāthā}}
\markboth{Ujjaya }{Ujjayattheragāthā}
\extramarks{Thag 1.47}{Thag 1.47}

\begin{verse}%
Homage\marginnote{1.1} to you, O Buddha, O hero, \\
freed in every way! \\
Meditating in the fruits of your practice, \\
I live without defilements. 

%
\end{verse}

%
\section*{{\suttatitleacronym Thag 1.48}{\suttatitletranslation Sañjaya }{\suttatitleroot Sañjayattheragāthā}}
\addcontentsline{toc}{section}{\tocacronym{Thag 1.48} \toctranslation{Sañjaya } \tocroot{Sañjayattheragāthā}}
\markboth{Sañjaya }{Sañjayattheragāthā}
\extramarks{Thag 1.48}{Thag 1.48}

\begin{verse}%
Since\marginnote{1.1} I went forth \\
from the lay life to homelessness, \\
I’ve not been aware of any thought \\
that is ignoble and hateful. 

%
\end{verse}

%
\section*{{\suttatitleacronym Thag 1.49}{\suttatitletranslation Rāmaṇeyyaka }{\suttatitleroot Rāmaṇeyyakattheragāthā}}
\addcontentsline{toc}{section}{\tocacronym{Thag 1.49} \toctranslation{Rāmaṇeyyaka } \tocroot{Rāmaṇeyyakattheragāthā}}
\markboth{Rāmaṇeyyaka }{Rāmaṇeyyakattheragāthā}
\extramarks{Thag 1.49}{Thag 1.49}

\begin{verse}%
Even\marginnote{1.1} with all the sounds, \\
the chirping and cheeping of the birds, \\
my mind doesn’t waver, \\
for I’m devoted to oneness. 

%
\end{verse}

%
\section*{{\suttatitleacronym Thag 1.50}{\suttatitletranslation Vimala (1st) }{\suttatitleroot Vimalattheragāthā}}
\addcontentsline{toc}{section}{\tocacronym{Thag 1.50} \toctranslation{Vimala (1st) } \tocroot{Vimalattheragāthā}}
\markboth{Vimala (1st) }{Vimalattheragāthā}
\extramarks{Thag 1.50}{Thag 1.50}

\begin{verse}%
The\marginnote{1.1} celestial lake \textsanskrit{Dharaṇī} pours, \\>the heavenly gale blows,\footnote{\textit{\textsanskrit{Dharaṇī}} is an allusion to the celestial lake from which the rains come (\href{https://suttacentral.net/dn32/en/sujato\#7.48}{DN 32:7.48}). | \textit{\textsanskrit{Māluto}} is a poetic word for the “breeze”, derived from the Vedic deities the Maruts, windy gods of the thunderstorm. } \\
and lightning flashes across the sky! \\
But my thoughts are stilled, \\
my mind is serene. 

%
\end{verse}

%
\addtocontents{toc}{\let\protect\contentsline\protect\nopagecontentsline}
\chapter*{Chapter Six }
\addcontentsline{toc}{chapter}{\tocchapterline{Chapter Six }}
\addtocontents{toc}{\let\protect\contentsline\protect\oldcontentsline}

%
\section*{{\suttatitleacronym Thag 1.51}{\suttatitletranslation Godhika }{\suttatitleroot Godhikattheragāthā}}
\addcontentsline{toc}{section}{\tocacronym{Thag 1.51} \toctranslation{Godhika } \tocroot{Godhikattheragāthā}}
\markboth{Godhika }{Godhikattheragāthā}
\extramarks{Thag 1.51}{Thag 1.51}

\begin{verse}%
The\marginnote{1.1} heavens rain like a sweet song; \\
my little hut is roofed and pleasant, \\>sheltered from the wind. \\
My mind is serene: \\
so rain forth, heavens, if you wish. 

%
\end{verse}

%
\section*{{\suttatitleacronym Thag 1.52}{\suttatitletranslation Subāhu }{\suttatitleroot Subāhuttheragāthā}}
\addcontentsline{toc}{section}{\tocacronym{Thag 1.52} \toctranslation{Subāhu } \tocroot{Subāhuttheragāthā}}
\markboth{Subāhu }{Subāhuttheragāthā}
\extramarks{Thag 1.52}{Thag 1.52}

\begin{verse}%
The\marginnote{1.1} heavens rain like a sweet song. \\
My little hut is roofed and pleasant, \\>sheltered from the wind. \\
My mind is immersed in my body: \\
so rain forth, heavens, if you wish. 

%
\end{verse}

%
\section*{{\suttatitleacronym Thag 1.53}{\suttatitletranslation Valliya (1st) }{\suttatitleroot Valliyattheragāthā}}
\addcontentsline{toc}{section}{\tocacronym{Thag 1.53} \toctranslation{Valliya (1st) } \tocroot{Valliyattheragāthā}}
\markboth{Valliya (1st) }{Valliyattheragāthā}
\extramarks{Thag 1.53}{Thag 1.53}

\begin{verse}%
The\marginnote{1.1} heavens rain like a sweet song. \\
My little hut is roofed and pleasant, \\>sheltered from the wind. \\
I meditate there, diligent: \\
so rain forth, heavens, if you wish. 

%
\end{verse}

%
\section*{{\suttatitleacronym Thag 1.54}{\suttatitletranslation Uttiya (2nd) }{\suttatitleroot Uttiyattheragāthā}}
\addcontentsline{toc}{section}{\tocacronym{Thag 1.54} \toctranslation{Uttiya (2nd) } \tocroot{Uttiyattheragāthā}}
\markboth{Uttiya (2nd) }{Uttiyattheragāthā}
\extramarks{Thag 1.54}{Thag 1.54}

\begin{verse}%
The\marginnote{1.1} heavens rain like a sweet song; \\
my little hut is roofed and pleasant, \\>sheltered from the wind. \\
I dwell there without a partner: \\
so rain forth, heavens, if you wish. 

%
\end{verse}

%
\section*{{\suttatitleacronym Thag 1.55}{\suttatitletranslation Añjanavaniya }{\suttatitleroot Añjanavaniyattheragāthā}}
\addcontentsline{toc}{section}{\tocacronym{Thag 1.55} \toctranslation{Añjanavaniya } \tocroot{Añjanavaniyattheragāthā}}
\markboth{Añjanavaniya }{Añjanavaniyattheragāthā}
\extramarks{Thag 1.55}{Thag 1.55}

\begin{verse}%
I\marginnote{1.1} plunged into the \textsanskrit{Añjana} forest \\
and made a little hut to live in. \\
I’ve attained the three knowledges \\
and fulfilled the Buddha’s instructions. 

%
\end{verse}

%
\section*{{\suttatitleacronym Thag 1.56}{\suttatitletranslation Kuṭivihārin (1st) }{\suttatitleroot Kuṭivihārittheragāthā}}
\addcontentsline{toc}{section}{\tocacronym{Thag 1.56} \toctranslation{Kuṭivihārin (1st) } \tocroot{Kuṭivihārittheragāthā}}
\markboth{Kuṭivihārin (1st) }{Kuṭivihārittheragāthā}
\extramarks{Thag 1.56}{Thag 1.56}

\begin{verse}%
“Who\marginnote{1.1} is in this little hut?” \\>“A monk is in this little hut, \\
free of lust, his mind serene. \\
My friend, you should know this: \\
your little hut wasn’t built in vain.” 

%
\end{verse}

%
\section*{{\suttatitleacronym Thag 1.57}{\suttatitletranslation Kuṭivihārin (2nd) }{\suttatitleroot Dutiyakuṭivihārittheragāthā}}
\addcontentsline{toc}{section}{\tocacronym{Thag 1.57} \toctranslation{Kuṭivihārin (2nd) } \tocroot{Dutiyakuṭivihārittheragāthā}}
\markboth{Kuṭivihārin (2nd) }{Dutiyakuṭivihārittheragāthā}
\extramarks{Thag 1.57}{Thag 1.57}

\begin{verse}%
This\marginnote{1.1} was your old hut, \\
but you still want a new hut. \\
Let go of hope for a hut, monk! \\
A new hut will only bring more suffering. 

%
\end{verse}

%
\section*{{\suttatitleacronym Thag 1.58}{\suttatitletranslation Ramaṇīyakuṭika }{\suttatitleroot Ramaṇīyakuṭikattheragāthā}}
\addcontentsline{toc}{section}{\tocacronym{Thag 1.58} \toctranslation{Ramaṇīyakuṭika } \tocroot{Ramaṇīyakuṭikattheragāthā}}
\markboth{Ramaṇīyakuṭika }{Ramaṇīyakuṭikattheragāthā}
\extramarks{Thag 1.58}{Thag 1.58}

\begin{verse}%
My\marginnote{1.1} little hut is pleasing, delightful, \\
a gift given in faith. \\
I’ve no need of girls: \\
go, ladies, to those in need! 

%
\end{verse}

%
\section*{{\suttatitleacronym Thag 1.59}{\suttatitletranslation Kosalavihārin }{\suttatitleroot Kosalavihārittheragāthā}}
\addcontentsline{toc}{section}{\tocacronym{Thag 1.59} \toctranslation{Kosalavihārin } \tocroot{Kosalavihārittheragāthā}}
\markboth{Kosalavihārin }{Kosalavihārittheragāthā}
\extramarks{Thag 1.59}{Thag 1.59}

\begin{verse}%
I\marginnote{1.1} went forth out of faith \\
and built a little hut in the wilderness. \\
I’m heedful, ardent, \\
aware, and mindful. 

%
\end{verse}

%
\section*{{\suttatitleacronym Thag 1.60}{\suttatitletranslation Sīvali }{\suttatitleroot Sīvalittheragāthā}}
\addcontentsline{toc}{section}{\tocacronym{Thag 1.60} \toctranslation{Sīvali } \tocroot{Sīvalittheragāthā}}
\markboth{Sīvali }{Sīvalittheragāthā}
\extramarks{Thag 1.60}{Thag 1.60}

\begin{verse}%
My\marginnote{1.1} wishes—the purpose I had \\
for entering this hut—came true. \\
Abandoning the tendency to conceit, \\
I’ll realize knowledge and liberation. 

%
\end{verse}

%
\addtocontents{toc}{\let\protect\contentsline\protect\nopagecontentsline}
\chapter*{Chapter Seven }
\addcontentsline{toc}{chapter}{\tocchapterline{Chapter Seven }}
\addtocontents{toc}{\let\protect\contentsline\protect\oldcontentsline}

%
\section*{{\suttatitleacronym Thag 1.61}{\suttatitletranslation Vappa }{\suttatitleroot Vappattheragāthā}}
\addcontentsline{toc}{section}{\tocacronym{Thag 1.61} \toctranslation{Vappa } \tocroot{Vappattheragāthā}}
\markboth{Vappa }{Vappattheragāthā}
\extramarks{Thag 1.61}{Thag 1.61}

\begin{verse}%
One\marginnote{1.1} who sees \\
sees those who see and those who don’t. \\
One who doesn’t see \\
sees neither. 

%
\end{verse}

%
\section*{{\suttatitleacronym Thag 1.62}{\suttatitletranslation Vajjiputta (1st) }{\suttatitleroot Vajjiputtattheragāthā}}
\addcontentsline{toc}{section}{\tocacronym{Thag 1.62} \toctranslation{Vajjiputta (1st) } \tocroot{Vajjiputtattheragāthā}}
\markboth{Vajjiputta (1st) }{Vajjiputtattheragāthā}
\extramarks{Thag 1.62}{Thag 1.62}

\begin{verse}%
We\marginnote{1.1} dwell alone in the wilderness, \\
like a log rejected in a forest. \\
Lots of people are jealous of me, \\
as beings in hell are of one going to heaven. 

%
\end{verse}

%
\section*{{\suttatitleacronym Thag 1.63}{\suttatitletranslation Pakkha }{\suttatitleroot Pakkhattheragāthā}}
\addcontentsline{toc}{section}{\tocacronym{Thag 1.63} \toctranslation{Pakkha } \tocroot{Pakkhattheragāthā}}
\markboth{Pakkha }{Pakkhattheragāthā}
\extramarks{Thag 1.63}{Thag 1.63}

\begin{verse}%
They\marginnote{1.1} fall, collapsed and fallen; \\
greedy, they return. \\
The work is done, the joyful is enjoyed, \\
happiness is found through happiness. 

%
\end{verse}

%
\section*{{\suttatitleacronym Thag 1.64}{\suttatitletranslation Vimalakoṇḍañña }{\suttatitleroot Vimalakoṇḍaññattheragāthā}}
\addcontentsline{toc}{section}{\tocacronym{Thag 1.64} \toctranslation{Vimalakoṇḍañña } \tocroot{Vimalakoṇḍaññattheragāthā}}
\markboth{Vimalakoṇḍañña }{Vimalakoṇḍaññattheragāthā}
\extramarks{Thag 1.64}{Thag 1.64}

\begin{verse}%
I\marginnote{1.1} arose from the one named after a tree; \\
I was born of the one whose banner shines. \\
The banner killer has destroyed the great banner, \\
by means of the banner itself. 

%
\end{verse}

%
\section*{{\suttatitleacronym Thag 1.65}{\suttatitletranslation Ukkhepakatavaccha }{\suttatitleroot Ukkhepakatavacchattheragāthā}}
\addcontentsline{toc}{section}{\tocacronym{Thag 1.65} \toctranslation{Ukkhepakatavaccha } \tocroot{Ukkhepakatavacchattheragāthā}}
\markboth{Ukkhepakatavaccha }{Ukkhepakatavacchattheragāthā}
\extramarks{Thag 1.65}{Thag 1.65}

\begin{verse}%
Vaccha\marginnote{1.1} has tossed away \\
what he built over many years. \\
Sitting comfortably, uplifted with joy, \\
he teaches this to householders. 

%
\end{verse}

%
\section*{{\suttatitleacronym Thag 1.66}{\suttatitletranslation Meghiya }{\suttatitleroot Meghiyattheragāthā}}
\addcontentsline{toc}{section}{\tocacronym{Thag 1.66} \toctranslation{Meghiya } \tocroot{Meghiyattheragāthā}}
\markboth{Meghiya }{Meghiyattheragāthā}
\extramarks{Thag 1.66}{Thag 1.66}

\begin{verse}%
He\marginnote{1.1} counseled me, the great hero, \\
the one who has gone beyond all things. \\
When I heard his teaching \\
I stayed close by him, mindful. \\
I’ve attained the three knowledges \\
and fulfilled the Buddha’s instructions. 

%
\end{verse}

%
\section*{{\suttatitleacronym Thag 1.67}{\suttatitletranslation Ekadhammasavanīya }{\suttatitleroot Ekadhammasavanīyattheragāthā}}
\addcontentsline{toc}{section}{\tocacronym{Thag 1.67} \toctranslation{Ekadhammasavanīya } \tocroot{Ekadhammasavanīyattheragāthā}}
\markboth{Ekadhammasavanīya }{Ekadhammasavanīyattheragāthā}
\extramarks{Thag 1.67}{Thag 1.67}

\begin{verse}%
My\marginnote{1.1} defilements have been burnt away \\>by practicing absorption—\\
rebirth into all states of existence is eradicated, \\
transmigration through births is finished, \\
now there’ll be no more future lives. 

%
\end{verse}

%
\section*{{\suttatitleacronym Thag 1.68}{\suttatitletranslation Ekudāniya }{\suttatitleroot Ekudāniyattheragāthā}}
\addcontentsline{toc}{section}{\tocacronym{Thag 1.68} \toctranslation{Ekudāniya } \tocroot{Ekudāniyattheragāthā}}
\markboth{Ekudāniya }{Ekudāniyattheragāthā}
\extramarks{Thag 1.68}{Thag 1.68}

\begin{verse}%
A\marginnote{1.1} sage of higher consciousness, diligent, \\
training in the ways of sagacity: \\
there are no sorrows for the unaffected, \\
calm and ever mindful. 

%
\end{verse}

%
\section*{{\suttatitleacronym Thag 1.69}{\suttatitletranslation Channa }{\suttatitleroot Channattheragāthā}}
\addcontentsline{toc}{section}{\tocacronym{Thag 1.69} \toctranslation{Channa } \tocroot{Channattheragāthā}}
\markboth{Channa }{Channattheragāthā}
\extramarks{Thag 1.69}{Thag 1.69}

\begin{verse}%
Hearing\marginnote{1.1} the sweet Dhamma taught by the master, \\
all-knowing, of superb knowledge, \\
I’ve entered the path to realize freedom from death—\\
he is the expert on the road to sanctuary from the yoke. 

%
\end{verse}

%
\section*{{\suttatitleacronym Thag 1.70}{\suttatitletranslation Puṇṇa (2nd) }{\suttatitleroot Puṇṇattheragāthā}}
\addcontentsline{toc}{section}{\tocacronym{Thag 1.70} \toctranslation{Puṇṇa (2nd) } \tocroot{Puṇṇattheragāthā}}
\markboth{Puṇṇa (2nd) }{Puṇṇattheragāthā}
\extramarks{Thag 1.70}{Thag 1.70}

\begin{verse}%
Ethical\marginnote{1.1} conduct is best in this life,\footnote{This \textsanskrit{Puṇṇa} departs for the distant land of \textsanskrit{Sunāparanta} at \href{https://suttacentral.net/mn145/en/sujato}{MN 145}. } \\
but one with wisdom is supreme. \\
Someone with both virtue and wisdom \\
is victorious among men and gods. 

%
\end{verse}

%
\addtocontents{toc}{\let\protect\contentsline\protect\nopagecontentsline}
\chapter*{Chapter Eight }
\addcontentsline{toc}{chapter}{\tocchapterline{Chapter Eight }}
\addtocontents{toc}{\let\protect\contentsline\protect\oldcontentsline}

%
\section*{{\suttatitleacronym Thag 1.71}{\suttatitletranslation Vacchapāla }{\suttatitleroot Vacchapālattheragāthā}}
\addcontentsline{toc}{section}{\tocacronym{Thag 1.71} \toctranslation{Vacchapāla } \tocroot{Vacchapālattheragāthā}}
\markboth{Vacchapāla }{Vacchapālattheragāthā}
\extramarks{Thag 1.71}{Thag 1.71}

\begin{verse}%
For\marginnote{1.1} one who sees the meaning \\>so very subtle and fine; \\
who is skilled in thought and humble in manner; \\
who has cultivated mature ethics, \\
it’s not hard to gain extinguishment. 

%
\end{verse}

%
\section*{{\suttatitleacronym Thag 1.72}{\suttatitletranslation Ātuma }{\suttatitleroot Ātumattheragāthā}}
\addcontentsline{toc}{section}{\tocacronym{Thag 1.72} \toctranslation{Ātuma } \tocroot{Ātumattheragāthā}}
\markboth{Ātuma }{Ātumattheragāthā}
\extramarks{Thag 1.72}{Thag 1.72}

\begin{verse}%
A\marginnote{1.1} young bamboo is hard to extract \\
when the point is grown and become all woody. \\
That’s how I feel with the wife \\>who was arranged for me. \\
Give me permission—now I’ve gone forth. 

%
\end{verse}

%
\section*{{\suttatitleacronym Thag 1.73}{\suttatitletranslation Māṇava }{\suttatitleroot Māṇavattheragāthā}}
\addcontentsline{toc}{section}{\tocacronym{Thag 1.73} \toctranslation{Māṇava } \tocroot{Māṇavattheragāthā}}
\markboth{Māṇava }{Māṇavattheragāthā}
\extramarks{Thag 1.73}{Thag 1.73}

\begin{verse}%
Seeing\marginnote{1.1} an old person, and one suffering from disease, \\
and a corpse come to the end of life, \\
I went forth, becoming a wanderer, \\
and giving up the pleasures of the senses. 

%
\end{verse}

%
\section*{{\suttatitleacronym Thag 1.74}{\suttatitletranslation Suyāmana }{\suttatitleroot Suyāmanattheragāthā}}
\addcontentsline{toc}{section}{\tocacronym{Thag 1.74} \toctranslation{Suyāmana } \tocroot{Suyāmanattheragāthā}}
\markboth{Suyāmana }{Suyāmanattheragāthā}
\extramarks{Thag 1.74}{Thag 1.74}

\begin{verse}%
Sensual\marginnote{1.1} desire, ill will, \\
dullness and drowsiness, \\
restlessness, and doubt \\
are not found in a monk at all. 

%
\end{verse}

%
\section*{{\suttatitleacronym Thag 1.75}{\suttatitletranslation Susārada }{\suttatitleroot Susāradattheragāthā}}
\addcontentsline{toc}{section}{\tocacronym{Thag 1.75} \toctranslation{Susārada } \tocroot{Susāradattheragāthā}}
\markboth{Susārada }{Susāradattheragāthā}
\extramarks{Thag 1.75}{Thag 1.75}

\begin{verse}%
Good\marginnote{1.1} is the sight of those who’ve practiced well: \\
doubt is cut off and intelligence grows—\\
even a fool grows wise! \\
That’s why it’s good to meet good people. 

%
\end{verse}

%
\section*{{\suttatitleacronym Thag 1.76}{\suttatitletranslation Piyañjaha }{\suttatitleroot Piyañjahattheragāthā}}
\addcontentsline{toc}{section}{\tocacronym{Thag 1.76} \toctranslation{Piyañjaha } \tocroot{Piyañjahattheragāthā}}
\markboth{Piyañjaha }{Piyañjahattheragāthā}
\extramarks{Thag 1.76}{Thag 1.76}

\begin{verse}%
Settle\marginnote{1.1} down when others spring up; \\
spring up when others settle down. \\
Remain when others have departed; \\
don’t delight when others delight. 

%
\end{verse}

%
\section*{{\suttatitleacronym Thag 1.77}{\suttatitletranslation Hatthārohaputta }{\suttatitleroot Hatthārohaputtattheragāthā}}
\addcontentsline{toc}{section}{\tocacronym{Thag 1.77} \toctranslation{Hatthārohaputta } \tocroot{Hatthārohaputtattheragāthā}}
\markboth{Hatthārohaputta }{Hatthārohaputtattheragāthā}
\extramarks{Thag 1.77}{Thag 1.77}

\begin{verse}%
In\marginnote{1.1} the past my mind wandered \\
how it wished, where it liked, as it pleased. \\
Now I’ll carefully guide it, \\
as a trainer with a hook guides a rutting elephant. 

%
\end{verse}

%
\section*{{\suttatitleacronym Thag 1.78}{\suttatitletranslation Meṇḍasira }{\suttatitleroot Meṇḍasirattheragāthā}}
\addcontentsline{toc}{section}{\tocacronym{Thag 1.78} \toctranslation{Meṇḍasira } \tocroot{Meṇḍasirattheragāthā}}
\markboth{Meṇḍasira }{Meṇḍasirattheragāthā}
\extramarks{Thag 1.78}{Thag 1.78}

\begin{verse}%
Transmigrating\marginnote{1.1} through countless rebirths, \\
I’ve journeyed without reward. \\
I’ve suffered, but now \\
the mass of suffering has fallen away. 

%
\end{verse}

%
\section*{{\suttatitleacronym Thag 1.79}{\suttatitletranslation Rakkhita }{\suttatitleroot Rakkhitattheragāthā}}
\addcontentsline{toc}{section}{\tocacronym{Thag 1.79} \toctranslation{Rakkhita } \tocroot{Rakkhitattheragāthā}}
\markboth{Rakkhita }{Rakkhitattheragāthā}
\extramarks{Thag 1.79}{Thag 1.79}

\begin{verse}%
All\marginnote{1.1} my lust is given up, \\
all my hate is eradicated, \\
all my delusion is gone: \\
I’m cooled, quenched. 

%
\end{verse}

%
\section*{{\suttatitleacronym Thag 1.80}{\suttatitletranslation Ugga }{\suttatitleroot Uggattheragāthā}}
\addcontentsline{toc}{section}{\tocacronym{Thag 1.80} \toctranslation{Ugga } \tocroot{Uggattheragāthā}}
\markboth{Ugga }{Uggattheragāthā}
\extramarks{Thag 1.80}{Thag 1.80}

\begin{verse}%
Any\marginnote{1.1} deeds I have done, \\
a little or a lot, \\
are all completely exhausted: \\
now there’ll be no more future lives. 

%
\end{verse}

%
\addtocontents{toc}{\let\protect\contentsline\protect\nopagecontentsline}
\chapter*{Chapter Nine }
\addcontentsline{toc}{chapter}{\tocchapterline{Chapter Nine }}
\addtocontents{toc}{\let\protect\contentsline\protect\oldcontentsline}

%
\section*{{\suttatitleacronym Thag 1.81}{\suttatitletranslation Samitigutta }{\suttatitleroot Samitiguttattheragāthā}}
\addcontentsline{toc}{section}{\tocacronym{Thag 1.81} \toctranslation{Samitigutta } \tocroot{Samitiguttattheragāthā}}
\markboth{Samitigutta }{Samitiguttattheragāthā}
\extramarks{Thag 1.81}{Thag 1.81}

\begin{verse}%
Any\marginnote{1.1} bad things I’ve done \\
in previous lives, \\
are to be experienced right here, \\
not in any other place. 

%
\end{verse}

%
\section*{{\suttatitleacronym Thag 1.82}{\suttatitletranslation Kassapa }{\suttatitleroot Kassapattheragāthā}}
\addcontentsline{toc}{section}{\tocacronym{Thag 1.82} \toctranslation{Kassapa } \tocroot{Kassapattheragāthā}}
\markboth{Kassapa }{Kassapattheragāthā}
\extramarks{Thag 1.82}{Thag 1.82}

\begin{verse}%
Go,\marginnote{1.1} child, to any place \\
where there’s plenty of food, \\
where it’s safe and free of peril—\\
may you not be overcome by sorrow! 

%
\end{verse}

%
\section*{{\suttatitleacronym Thag 1.83}{\suttatitletranslation Sīha }{\suttatitleroot Sīhattheragāthā}}
\addcontentsline{toc}{section}{\tocacronym{Thag 1.83} \toctranslation{Sīha } \tocroot{Sīhattheragāthā}}
\markboth{Sīha }{Sīhattheragāthā}
\extramarks{Thag 1.83}{Thag 1.83}

\begin{verse}%
Meditate\marginnote{1.1} diligently, \textsanskrit{Sīha}, \\
tireless all day and night. \\
Develop skillful qualities, \\
and quickly discard this bag of bones. 

%
\end{verse}

%
\section*{{\suttatitleacronym Thag 1.84}{\suttatitletranslation Nīta }{\suttatitleroot Nītattheragāthā}}
\addcontentsline{toc}{section}{\tocacronym{Thag 1.84} \toctranslation{Nīta } \tocroot{Nītattheragāthā}}
\markboth{Nīta }{Nītattheragāthā}
\extramarks{Thag 1.84}{Thag 1.84}

\begin{verse}%
Sleeping\marginnote{1.1} all night, \\
happily socializing by day, \\
when will the simpleton \\
make an end of suffering? 

%
\end{verse}

%
\section*{{\suttatitleacronym Thag 1.85}{\suttatitletranslation Sunāga }{\suttatitleroot Sunāgattheragāthā}}
\addcontentsline{toc}{section}{\tocacronym{Thag 1.85} \toctranslation{Sunāga } \tocroot{Sunāgattheragāthā}}
\markboth{Sunāga }{Sunāgattheragāthā}
\extramarks{Thag 1.85}{Thag 1.85}

\begin{verse}%
Skilled\marginnote{1.1} in the patterns of the mind, \\
understanding the nectar of seclusion, \\
practicing absorption, alert, mindful: \\
such a person would realize pleasure not of the flesh. 

%
\end{verse}

%
\section*{{\suttatitleacronym Thag 1.86}{\suttatitletranslation Nāgita }{\suttatitleroot Nāgitattheragāthā}}
\addcontentsline{toc}{section}{\tocacronym{Thag 1.86} \toctranslation{Nāgita } \tocroot{Nāgitattheragāthā}}
\markboth{Nāgita }{Nāgitattheragāthā}
\extramarks{Thag 1.86}{Thag 1.86}

\begin{verse}%
Elsewhere\marginnote{1.1} there are many other doctrines; \\
paths that, unlike this, don’t lead to extinguishment. \\
For the Buddha himself instructs the \textsanskrit{Saṅgha}; \\
the Teacher shows the palms of his hands. 

%
\end{verse}

%
\section*{{\suttatitleacronym Thag 1.87}{\suttatitletranslation Paviṭṭha }{\suttatitleroot Paviṭṭhattheragāthā}}
\addcontentsline{toc}{section}{\tocacronym{Thag 1.87} \toctranslation{Paviṭṭha } \tocroot{Paviṭṭhattheragāthā}}
\markboth{Paviṭṭha }{Paviṭṭhattheragāthā}
\extramarks{Thag 1.87}{Thag 1.87}

\begin{verse}%
The\marginnote{1.1} aggregates are seen as they truly are; \\
all rebirths are shattered; \\
transmigration through births is finished; \\
now there’ll be no more future lives. 

%
\end{verse}

%
\section*{{\suttatitleacronym Thag 1.88}{\suttatitletranslation Ajjuna }{\suttatitleroot Ajjunattheragāthā}}
\addcontentsline{toc}{section}{\tocacronym{Thag 1.88} \toctranslation{Ajjuna } \tocroot{Ajjunattheragāthā}}
\markboth{Ajjuna }{Ajjunattheragāthā}
\extramarks{Thag 1.88}{Thag 1.88}

\begin{verse}%
I\marginnote{1.1} was able to lift myself up \\
from the water to the shore. \\
While being swept away by the great flood, \\
I penetrated the truths. 

%
\end{verse}

%
\section*{{\suttatitleacronym Thag 1.89}{\suttatitletranslation Devasabha (1st) }{\suttatitleroot (Paṭhama) Devasabhattheragāthā}}
\addcontentsline{toc}{section}{\tocacronym{Thag 1.89} \toctranslation{Devasabha (1st) } \tocroot{(Paṭhama) Devasabhattheragāthā}}
\markboth{Devasabha (1st) }{(Paṭhama) Devasabhattheragāthā}
\extramarks{Thag 1.89}{Thag 1.89}

\begin{verse}%
I’ve\marginnote{1.1} crossed the bogs, \\
I’ve avoided the cliffs, \\
I’m freed from floods and ties, \\
and I've wiped out all conceit. 

%
\end{verse}

%
\section*{{\suttatitleacronym Thag 1.90}{\suttatitletranslation Sāmidatta }{\suttatitleroot Sāmidattattheragāthā}}
\addcontentsline{toc}{section}{\tocacronym{Thag 1.90} \toctranslation{Sāmidatta } \tocroot{Sāmidattattheragāthā}}
\markboth{Sāmidatta }{Sāmidattattheragāthā}
\extramarks{Thag 1.90}{Thag 1.90}

\begin{verse}%
The\marginnote{1.1} five aggregates are fully understood; \\
they remain, but their root is cut. \\
Transmigration through births is finished, \\
now there’ll be no more future lives. 

%
\end{verse}

%
\addtocontents{toc}{\let\protect\contentsline\protect\nopagecontentsline}
\chapter*{Chapter Ten }
\addcontentsline{toc}{chapter}{\tocchapterline{Chapter Ten }}
\addtocontents{toc}{\let\protect\contentsline\protect\oldcontentsline}

%
\section*{{\suttatitleacronym Thag 1.91}{\suttatitletranslation Paripuṇṇaka }{\suttatitleroot Paripuṇṇakattheragāthā}}
\addcontentsline{toc}{section}{\tocacronym{Thag 1.91} \toctranslation{Paripuṇṇaka } \tocroot{Paripuṇṇakattheragāthā}}
\markboth{Paripuṇṇaka }{Paripuṇṇakattheragāthā}
\extramarks{Thag 1.91}{Thag 1.91}

\begin{verse}%
What\marginnote{1.1} I consumed today is considered better \\
than delicious grain of a hundred flavors—\\
the Dhamma taught by the Buddha, \\
Gotama of infinite vision. 

%
\end{verse}

%
\section*{{\suttatitleacronym Thag 1.92}{\suttatitletranslation Vijaya }{\suttatitleroot Vijayattheragāthā}}
\addcontentsline{toc}{section}{\tocacronym{Thag 1.92} \toctranslation{Vijaya } \tocroot{Vijayattheragāthā}}
\markboth{Vijaya }{Vijayattheragāthā}
\extramarks{Thag 1.92}{Thag 1.92}

\begin{verse}%
One\marginnote{1.1} whose defilements have ended; \\
who’s not attached to food; \\
whose range is the liberation \\
of the signless and the empty: \\
their track is hard to trace, \\
like birds in the sky. 

%
\end{verse}

%
\section*{{\suttatitleacronym Thag 1.93}{\suttatitletranslation Eraka }{\suttatitleroot Erakattheragāthā}}
\addcontentsline{toc}{section}{\tocacronym{Thag 1.93} \toctranslation{Eraka } \tocroot{Erakattheragāthā}}
\markboth{Eraka }{Erakattheragāthā}
\extramarks{Thag 1.93}{Thag 1.93}

\begin{verse}%
Sensual\marginnote{1.1} pleasures are suffering, Eraka! \\
Sensual pleasures aren’t happiness, Eraka! \\
One who enjoys sensual pleasures \\
enjoys suffering, Eraka! \\
One who doesn’t enjoy sensual pleasures \\
doesn’t enjoy suffering, Eraka! 

%
\end{verse}

%
\section*{{\suttatitleacronym Thag 1.94}{\suttatitletranslation Mettaji }{\suttatitleroot Mettajittheragāthā}}
\addcontentsline{toc}{section}{\tocacronym{Thag 1.94} \toctranslation{Mettaji } \tocroot{Mettajittheragāthā}}
\markboth{Mettaji }{Mettajittheragāthā}
\extramarks{Thag 1.94}{Thag 1.94}

\begin{verse}%
Homage\marginnote{1.1} to that Blessed One, \\
the glorious Sakyan! \\
Having reached the best, \\
he beautifully taught the best teaching. 

%
\end{verse}

%
\section*{{\suttatitleacronym Thag 1.95}{\suttatitletranslation Cakkhupāla }{\suttatitleroot Cakkhupālattheragāthā}}
\addcontentsline{toc}{section}{\tocacronym{Thag 1.95} \toctranslation{Cakkhupāla } \tocroot{Cakkhupālattheragāthā}}
\markboth{Cakkhupāla }{Cakkhupālattheragāthā}
\extramarks{Thag 1.95}{Thag 1.95}

\begin{verse}%
I’m\marginnote{1.1} blind, my eyes are ruined, \\
I’m traveling a desolate road. \\
Even if I have to crawl I’ll keep going—\\
though not with wicked companions. 

%
\end{verse}

%
\section*{{\suttatitleacronym Thag 1.96}{\suttatitletranslation Khaṇḍasumana }{\suttatitleroot Khaṇḍasumanattheragāthā}}
\addcontentsline{toc}{section}{\tocacronym{Thag 1.96} \toctranslation{Khaṇḍasumana } \tocroot{Khaṇḍasumanattheragāthā}}
\markboth{Khaṇḍasumana }{Khaṇḍasumanattheragāthā}
\extramarks{Thag 1.96}{Thag 1.96}

\begin{verse}%
I\marginnote{1.1} offered a single flower \\
and then amused myself in heavens \\
for 800 million years; \\
with what’s left over I’ve become quenched. 

%
\end{verse}

%
\section*{{\suttatitleacronym Thag 1.97}{\suttatitletranslation Tissa (2nd) }{\suttatitleroot Tissattheragāthā}}
\addcontentsline{toc}{section}{\tocacronym{Thag 1.97} \toctranslation{Tissa (2nd) } \tocroot{Tissattheragāthā}}
\markboth{Tissa (2nd) }{Tissattheragāthā}
\extramarks{Thag 1.97}{Thag 1.97}

\begin{verse}%
Giving\marginnote{1.1} up a valuable bronze bowl, \\
and a precious golden one, too, \\
I took a bowl made of clay: \\
this is my second initiation. 

%
\end{verse}

%
\section*{{\suttatitleacronym Thag 1.98}{\suttatitletranslation Abhaya }{\suttatitleroot Abhayattheragāthā}}
\addcontentsline{toc}{section}{\tocacronym{Thag 1.98} \toctranslation{Abhaya } \tocroot{Abhayattheragāthā}}
\markboth{Abhaya }{Abhayattheragāthā}
\extramarks{Thag 1.98}{Thag 1.98}

\begin{verse}%
When\marginnote{1.1} you see a sight, mindfulness is lost \\
as you focus on a pleasant feature. \\
Experiencing it with a mind full of desire, \\
you keep clinging to it. \\
Your defilements grow, \\
leading to the root of rebirth. 

%
\end{verse}

%
\section*{{\suttatitleacronym Thag 1.99}{\suttatitletranslation Uttiya (3rd) }{\suttatitleroot Uttiyattheragāthā}}
\addcontentsline{toc}{section}{\tocacronym{Thag 1.99} \toctranslation{Uttiya (3rd) } \tocroot{Uttiyattheragāthā}}
\markboth{Uttiya (3rd) }{Uttiyattheragāthā}
\extramarks{Thag 1.99}{Thag 1.99}

\begin{verse}%
When\marginnote{1.1} you hear a sound, mindfulness is lost \\
as you focus on a pleasant feature. \\
Experiencing it with a mind full of desire, \\
you keep clinging to it. \\
Your defilements grow, \\
leading to transmigration. 

%
\end{verse}

%
\section*{{\suttatitleacronym Thag 1.100}{\suttatitletranslation Devasabha (2nd) }{\suttatitleroot (Dutiya) Devasabhattheragāthā}}
\addcontentsline{toc}{section}{\tocacronym{Thag 1.100} \toctranslation{Devasabha (2nd) } \tocroot{(Dutiya) Devasabhattheragāthā}}
\markboth{Devasabha (2nd) }{(Dutiya) Devasabhattheragāthā}
\extramarks{Thag 1.100}{Thag 1.100}

\begin{verse}%
Accomplished\marginnote{1.1} in the four right efforts, \\
mindfulness meditation is your range; \\
festooned with the flowers of liberation, \\
undefiled, you’ll be fully quenched. 

%
\end{verse}

%
\addtocontents{toc}{\let\protect\contentsline\protect\nopagecontentsline}
\chapter*{Chapter Eleven }
\addcontentsline{toc}{chapter}{\tocchapterline{Chapter Eleven }}
\addtocontents{toc}{\let\protect\contentsline\protect\oldcontentsline}

%
\section*{{\suttatitleacronym Thag 1.101}{\suttatitletranslation Belaṭṭhānika }{\suttatitleroot Belaṭṭhānikattheragāthā}}
\addcontentsline{toc}{section}{\tocacronym{Thag 1.101} \toctranslation{Belaṭṭhānika } \tocroot{Belaṭṭhānikattheragāthā}}
\markboth{Belaṭṭhānika }{Belaṭṭhānikattheragāthā}
\extramarks{Thag 1.101}{Thag 1.101}

\begin{verse}%
He’s\marginnote{1.1} given up the household life, \\>but he has no purpose. \\
Living for his belly, lazy, \\>he uses his snout as a plow, \\
like a great hog stuffed with grain. \\
That dullard is reborn again and again. 

%
\end{verse}

%
\section*{{\suttatitleacronym Thag 1.102}{\suttatitletranslation Setuccha }{\suttatitleroot Setucchattheragāthā}}
\addcontentsline{toc}{section}{\tocacronym{Thag 1.102} \toctranslation{Setuccha } \tocroot{Setucchattheragāthā}}
\markboth{Setuccha }{Setucchattheragāthā}
\extramarks{Thag 1.102}{Thag 1.102}

\begin{verse}%
Deceived\marginnote{1.1} by conceit, \\
defiled among conditions, \\
oppressed by gain and loss, \\
they don’t reach immersion. 

%
\end{verse}

%
\section*{{\suttatitleacronym Thag 1.103}{\suttatitletranslation Bandhura }{\suttatitleroot Bandhurattheragāthā}}
\addcontentsline{toc}{section}{\tocacronym{Thag 1.103} \toctranslation{Bandhura } \tocroot{Bandhurattheragāthā}}
\markboth{Bandhura }{Bandhurattheragāthā}
\extramarks{Thag 1.103}{Thag 1.103}

\begin{verse}%
I\marginnote{1.1} have no need of this—\\
I’m happy and satisfied with the sweet teaching. \\
I’ve drunk the best, the supreme nectar: \\
I won’t go near poison. 

%
\end{verse}

%
\section*{{\suttatitleacronym Thag 1.104}{\suttatitletranslation Khitaka }{\suttatitleroot Khitakattheragāthā}}
\addcontentsline{toc}{section}{\tocacronym{Thag 1.104} \toctranslation{Khitaka } \tocroot{Khitakattheragāthā}}
\markboth{Khitaka }{Khitakattheragāthā}
\extramarks{Thag 1.104}{Thag 1.104}

\begin{verse}%
Hey!\marginnote{1.1} My body is light, \\
full of so much rapture and happiness. \\
My body feels like it’s floating, \\
like cotton in a gale. 

%
\end{verse}

%
\section*{{\suttatitleacronym Thag 1.105}{\suttatitletranslation Malitavambha }{\suttatitleroot Malitavambhattheragāthā}}
\addcontentsline{toc}{section}{\tocacronym{Thag 1.105} \toctranslation{Malitavambha } \tocroot{Malitavambhattheragāthā}}
\markboth{Malitavambha }{Malitavambhattheragāthā}
\extramarks{Thag 1.105}{Thag 1.105}

\begin{verse}%
Dissatisfied,\marginnote{1.1} one should not stay; \\
and even if happy, one should depart. \\
One who sees clearly wouldn’t stay \\
in a place that was not conducive to the goal. 

%
\end{verse}

%
\section*{{\suttatitleacronym Thag 1.106}{\suttatitletranslation Suhemanta }{\suttatitleroot Suhemantattheragāthā}}
\addcontentsline{toc}{section}{\tocacronym{Thag 1.106} \toctranslation{Suhemanta } \tocroot{Suhemantattheragāthā}}
\markboth{Suhemanta }{Suhemantattheragāthā}
\extramarks{Thag 1.106}{Thag 1.106}

\begin{verse}%
Though\marginnote{1.1} the meaning has a hundred facets, \\
and bears a hundred characteristics, \\
the simpleton sees only one factor, \\
while the sage sees a hundred. 

%
\end{verse}

%
\section*{{\suttatitleacronym Thag 1.107}{\suttatitletranslation Dhammasava }{\suttatitleroot Dhammasavattheragāthā}}
\addcontentsline{toc}{section}{\tocacronym{Thag 1.107} \toctranslation{Dhammasava } \tocroot{Dhammasavattheragāthā}}
\markboth{Dhammasava }{Dhammasavattheragāthā}
\extramarks{Thag 1.107}{Thag 1.107}

\begin{verse}%
After\marginnote{1.1} investigating, I went forth \\
from the lay life to homelessness. \\
I’ve attained the three knowledges \\
and fulfilled the Buddha’s instructions. 

%
\end{verse}

%
\section*{{\suttatitleacronym Thag 1.108}{\suttatitletranslation Dhammasavapitu }{\suttatitleroot Dhammasavapituttheragāthā}}
\addcontentsline{toc}{section}{\tocacronym{Thag 1.108} \toctranslation{Dhammasavapitu } \tocroot{Dhammasavapituttheragāthā}}
\markboth{Dhammasavapitu }{Dhammasavapituttheragāthā}
\extramarks{Thag 1.108}{Thag 1.108}

\begin{verse}%
At\marginnote{1.1} 120 years old \\
I went forth to homelessness. \\
I’ve attained the three knowledges \\
and fulfilled the Buddha’s instructions. 

%
\end{verse}

%
\section*{{\suttatitleacronym Thag 1.109}{\suttatitletranslation Saṅgharakkhita }{\suttatitleroot Saṁgharakkhitattheragāthā}}
\addcontentsline{toc}{section}{\tocacronym{Thag 1.109} \toctranslation{Saṅgharakkhita } \tocroot{Saṁgharakkhitattheragāthā}}
\markboth{Saṅgharakkhita }{Saṁgharakkhitattheragāthā}
\extramarks{Thag 1.109}{Thag 1.109}

\begin{verse}%
Even\marginnote{1.1} on retreat he doesn’t heed the counsel \\
of the one with supreme compassion for his welfare. \\
He lives with unrestrained faculties, \\
like a young deer in the woods. 

%
\end{verse}

%
\section*{{\suttatitleacronym Thag 1.110}{\suttatitletranslation Usabha (1st) }{\suttatitleroot Usabhattheragāthā}}
\addcontentsline{toc}{section}{\tocacronym{Thag 1.110} \toctranslation{Usabha (1st) } \tocroot{Usabhattheragāthā}}
\markboth{Usabha (1st) }{Usabhattheragāthā}
\extramarks{Thag 1.110}{Thag 1.110}

\begin{verse}%
The\marginnote{1.1} trees on the mountain tops have grown tall, \\
freshly sprinkled by towering clouds. \\
For Usabha, who loves seclusion, \\>and who thinks only of wilderness, \\
goodness flourishes more and more. 

%
\end{verse}

%
\addtocontents{toc}{\let\protect\contentsline\protect\nopagecontentsline}
\chapter*{Chapter Twelve }
\addcontentsline{toc}{chapter}{\tocchapterline{Chapter Twelve }}
\addtocontents{toc}{\let\protect\contentsline\protect\oldcontentsline}

%
\section*{{\suttatitleacronym Thag 1.111}{\suttatitletranslation Jenta }{\suttatitleroot Jentattheragāthā}}
\addcontentsline{toc}{section}{\tocacronym{Thag 1.111} \toctranslation{Jenta } \tocroot{Jentattheragāthā}}
\markboth{Jenta }{Jentattheragāthā}
\extramarks{Thag 1.111}{Thag 1.111}

\begin{verse}%
Going\marginnote{1.1} forth is hard; living at home is hard; \\
Dhamma is profound; money is hard to come by. \\
Getting by is difficult for we \\>who accept whatever comes, \\
so we should always think about impermanence. 

%
\end{verse}

%
\section*{{\suttatitleacronym Thag 1.112}{\suttatitletranslation Vacchagotta }{\suttatitleroot Vacchagottattheragāthā}}
\addcontentsline{toc}{section}{\tocacronym{Thag 1.112} \toctranslation{Vacchagotta } \tocroot{Vacchagottattheragāthā}}
\markboth{Vacchagotta }{Vacchagottattheragāthā}
\extramarks{Thag 1.112}{Thag 1.112}

\begin{verse}%
I\marginnote{1.1} am a master of the three knowledges, \\>I’m a great meditator, \\
an expert in serenity of heart. \\
I’ve realized my own true goal \\
and fulfilled the Buddha’s instructions. 

%
\end{verse}

%
\section*{{\suttatitleacronym Thag 1.113}{\suttatitletranslation Vanavaccha (2nd) }{\suttatitleroot Vanavacchattheragāthā}}
\addcontentsline{toc}{section}{\tocacronym{Thag 1.113} \toctranslation{Vanavaccha (2nd) } \tocroot{Vanavacchattheragāthā}}
\markboth{Vanavaccha (2nd) }{Vanavacchattheragāthā}
\extramarks{Thag 1.113}{Thag 1.113}

\begin{verse}%
The\marginnote{1.1} water’s clear and the rocks are broad, \\
monkeys and deer are all around; \\
festooned with dewy moss, \\
these rocky crags delight me! 

%
\end{verse}

%
\section*{{\suttatitleacronym Thag 1.114}{\suttatitletranslation Adhimutta (1st) }{\suttatitleroot Adhimuttattheragāthā}}
\addcontentsline{toc}{section}{\tocacronym{Thag 1.114} \toctranslation{Adhimutta (1st) } \tocroot{Adhimuttattheragāthā}}
\markboth{Adhimutta (1st) }{Adhimuttattheragāthā}
\extramarks{Thag 1.114}{Thag 1.114}

\begin{verse}%
Your\marginnote{1.1} body is uncomfortably heavy, \\
and life is running out; \\
greedy for physical pleasure, \\
how can you find happiness as an ascetic? 

%
\end{verse}

%
\section*{{\suttatitleacronym Thag 1.115}{\suttatitletranslation Mahānāma }{\suttatitleroot Mahānāmattheragāthā}}
\addcontentsline{toc}{section}{\tocacronym{Thag 1.115} \toctranslation{Mahānāma } \tocroot{Mahānāmattheragāthā}}
\markboth{Mahānāma }{Mahānāmattheragāthā}
\extramarks{Thag 1.115}{Thag 1.115}

\begin{verse}%
By\marginnote{1.1} Mount \textsanskrit{Nesādaka}, \\
with its famous covering \\
of abundant shrubs and trees, \\
you’re found deficient. 

%
\end{verse}

%
\section*{{\suttatitleacronym Thag 1.116}{\suttatitletranslation Pārāsariya (1st) }{\suttatitleroot Pārāpariyattheragāthā}}
\addcontentsline{toc}{section}{\tocacronym{Thag 1.116} \toctranslation{Pārāsariya (1st) } \tocroot{Pārāpariyattheragāthā}}
\markboth{Pārāsariya (1st) }{Pārāpariyattheragāthā}
\extramarks{Thag 1.116}{Thag 1.116}

\begin{verse}%
I’ve\marginnote{1.1} given up the six spheres of sense-contact,\footnote{A brahmin teacher named \textsanskrit{Pārāsariya} (also spelled \textsanskrit{Pārāpariya} or \textsanskrit{Pārāsiviya}) features in \href{https://suttacentral.net/mn152/en/sujato}{MN 152}. In \href{https://suttacentral.net/ja222/en/sujato}{Ja 222} and \href{https://suttacentral.net/ja353/en/sujato}{Ja 353} another teacher warns that doing evil will lead to regret. The three similarly-named monks in the \textsanskrit{Theragāthā} (\href{https://suttacentral.net/thag1.116/en/sujato}{Thag 1.116}, \href{https://suttacentral.net/thag16.2/en/sujato}{Thag 16.2}, \href{https://suttacentral.net/thag16.10/en/sujato}{Thag 16.10}) share the same clan name (Sanskrit \textsanskrit{Pārāśarya}) as descendants of the great sage \textsanskrit{Parāśara}. } \\
my sense doors are guarded and well restrained; \\
I’ve ejected the root of misery \\
and attained the ending of defilements. 

%
\end{verse}

%
\section*{{\suttatitleacronym Thag 1.117}{\suttatitletranslation Yasa }{\suttatitleroot Yasattheragāthā}}
\addcontentsline{toc}{section}{\tocacronym{Thag 1.117} \toctranslation{Yasa } \tocroot{Yasattheragāthā}}
\markboth{Yasa }{Yasattheragāthā}
\extramarks{Thag 1.117}{Thag 1.117}

\begin{verse}%
I’m\marginnote{1.1} well-anointed and well-dressed, \\
adorned with all my jewellery. \\
I’ve attained the three knowledges \\
and fulfilled the Buddha’s instructions. 

%
\end{verse}

%
\section*{{\suttatitleacronym Thag 1.118}{\suttatitletranslation Kimbila (1st) }{\suttatitleroot Kimilattheragāthā}}
\addcontentsline{toc}{section}{\tocacronym{Thag 1.118} \toctranslation{Kimbila (1st) } \tocroot{Kimilattheragāthā}}
\markboth{Kimbila (1st) }{Kimilattheragāthā}
\extramarks{Thag 1.118}{Thag 1.118}

\begin{verse}%
Old\marginnote{1.1} age falls like a curse; \\
it’s the same body, but it seems like someone else’s. \\
I remember myself as if I were someone else, \\
but I’m still the same, I haven’t been away. 

%
\end{verse}

%
\section*{{\suttatitleacronym Thag 1.119}{\suttatitletranslation Vajjiputta (2nd) }{\suttatitleroot Vajjiputtattheragāthā}}
\addcontentsline{toc}{section}{\tocacronym{Thag 1.119} \toctranslation{Vajjiputta (2nd) } \tocroot{Vajjiputtattheragāthā}}
\markboth{Vajjiputta (2nd) }{Vajjiputtattheragāthā}
\extramarks{Thag 1.119}{Thag 1.119}

\begin{verse}%
You’ve\marginnote{1.1} left for the jungle, the root of a tree, \\
with extinguishment in your heart. \\
Practice absorption, Gotama, don’t be negligent! \\
What is this hullabaloo to you? 

%
\end{verse}

%
\section*{{\suttatitleacronym Thag 1.120}{\suttatitletranslation Isidatta }{\suttatitleroot Isidattattheragāthā}}
\addcontentsline{toc}{section}{\tocacronym{Thag 1.120} \toctranslation{Isidatta } \tocroot{Isidattattheragāthā}}
\markboth{Isidatta }{Isidattattheragāthā}
\extramarks{Thag 1.120}{Thag 1.120}

\begin{verse}%
The\marginnote{1.1} five aggregates are fully understood, \\
they remain, but their root is cut. \\
I have reached the end of suffering \\
and attained the ending of defilements. 

%
\end{verse}

%
\addtocontents{toc}{\let\protect\contentsline\protect\nopagecontentsline}
\part*{The Book of the Twos }
\addcontentsline{toc}{part}{The Book of the Twos }
\markboth{}{}
\addtocontents{toc}{\let\protect\contentsline\protect\oldcontentsline}

%
\addtocontents{toc}{\let\protect\contentsline\protect\nopagecontentsline}
\chapter*{Chapter One }
\addcontentsline{toc}{chapter}{\tocchapterline{Chapter One }}
\addtocontents{toc}{\let\protect\contentsline\protect\oldcontentsline}

%
\section*{{\suttatitleacronym Thag 2.1}{\suttatitletranslation Uttara (1st) }{\suttatitleroot Uttarattheragāthā}}
\addcontentsline{toc}{section}{\tocacronym{Thag 2.1} \toctranslation{Uttara (1st) } \tocroot{Uttarattheragāthā}}
\markboth{Uttara (1st) }{Uttarattheragāthā}
\extramarks{Thag 2.1}{Thag 2.1}

\begin{verse}%
No\marginnote{1.1} life is permanent, \\
and no conditions last forever. \\
The aggregates are reborn \\
and pass away, again and again. 

Knowing\marginnote{2.1} this danger, \\
I have no need for another life. \\
I’ve escaped all sensual pleasures, \\
and attained the ending of defilements. 

%
\end{verse}

\scendsutta{That is how these verses were recited by the senior venerable Uttara. }

%
\section*{{\suttatitleacronym Thag 2.2}{\suttatitletranslation Bhāradvāja the Alms-Gatherer }{\suttatitleroot Piṇḍolabhāradvājattheragāthā}}
\addcontentsline{toc}{section}{\tocacronym{Thag 2.2} \toctranslation{Bhāradvāja the Alms-Gatherer } \tocroot{Piṇḍolabhāradvājattheragāthā}}
\markboth{Bhāradvāja the Alms-Gatherer }{Piṇḍolabhāradvājattheragāthā}
\extramarks{Thag 2.2}{Thag 2.2}

\begin{verse}%
You\marginnote{1.1} can’t live by fasting, \\
but food doesn’t lead to peace of heart. \\
Seeing how this bag of bones is sustained by food, \\
I wander, seeking. 

They\marginnote{2.1} know it’s just a swamp, \\
this homage and veneration in respectable families. \\
Honor is a subtle dart, hard to extract, \\
and hard for a sinner to give up. 

%
\end{verse}

\scendsutta{That is how these verses were recited by the senior venerable \textsanskrit{Bhāradvāja} the Alms-Gatherer. }

%
\section*{{\suttatitleacronym Thag 2.3}{\suttatitletranslation Valliya (2nd) }{\suttatitleroot Valliyattheragāthā}}
\addcontentsline{toc}{section}{\tocacronym{Thag 2.3} \toctranslation{Valliya (2nd) } \tocroot{Valliyattheragāthā}}
\markboth{Valliya (2nd) }{Valliyattheragāthā}
\extramarks{Thag 2.3}{Thag 2.3}

\begin{verse}%
A\marginnote{1.1} monkey went up to the little hut \\
with five doors. \\
He circles around, knocking \\
on each door, again and again. 

Stand\marginnote{2.1} still monkey, don’t run! \\
Things are different now; \\
you’ve been caught by wisdom—\\
you won’t go far. 

%
\end{verse}

%
\section*{{\suttatitleacronym Thag 2.4}{\suttatitletranslation Gaṅgātīriya }{\suttatitleroot Gaṅgātīriyattheragāthā}}
\addcontentsline{toc}{section}{\tocacronym{Thag 2.4} \toctranslation{Gaṅgātīriya } \tocroot{Gaṅgātīriyattheragāthā}}
\markboth{Gaṅgātīriya }{Gaṅgātīriyattheragāthā}
\extramarks{Thag 2.4}{Thag 2.4}

\begin{verse}%
My\marginnote{1.1} hut on the bank of the Ganges \\
is made from three palm leaves. \\
My alms-bowl is a funeral pot, \\
my robe is cast-off rags. 

In\marginnote{2.1} my first two rainy seasons \\
I spoke only one word. \\
In my third rainy season \\
the mass of darkness was shattered. 

%
\end{verse}

%
\section*{{\suttatitleacronym Thag 2.5}{\suttatitletranslation Ajina }{\suttatitleroot Ajinattheragāthā}}
\addcontentsline{toc}{section}{\tocacronym{Thag 2.5} \toctranslation{Ajina } \tocroot{Ajinattheragāthā}}
\markboth{Ajina }{Ajinattheragāthā}
\extramarks{Thag 2.5}{Thag 2.5}

\begin{verse}%
Even\marginnote{1.1} a master of the three knowledges, \\
undefiled, conqueror of death, \\
is looked down upon for being unknown \\
by ignorant fools. 

But\marginnote{2.1} any person here \\
who gets food and drink \\
is honored by them, \\
even if they are of bad character. 

%
\end{verse}

%
\section*{{\suttatitleacronym Thag 2.6}{\suttatitletranslation Meḷajina }{\suttatitleroot Meḷajinattheragāthā}}
\addcontentsline{toc}{section}{\tocacronym{Thag 2.6} \toctranslation{Meḷajina } \tocroot{Meḷajinattheragāthā}}
\markboth{Meḷajina }{Meḷajinattheragāthā}
\extramarks{Thag 2.6}{Thag 2.6}

\begin{verse}%
When\marginnote{1.1} I heard the Teacher \\
speaking Dhamma, \\
I wasn’t aware of any doubt \\
in the all-knowing, unconquered one, 

the\marginnote{2.1} caravan leader, the great hero, \\
the most excellent of charioteers. \\
I have no doubt \\
in the path or practice. 

%
\end{verse}

%
\section*{{\suttatitleacronym Thag 2.7}{\suttatitletranslation Rādha }{\suttatitleroot Rādhattheragāthā}}
\addcontentsline{toc}{section}{\tocacronym{Thag 2.7} \toctranslation{Rādha } \tocroot{Rādhattheragāthā}}
\markboth{Rādha }{Rādhattheragāthā}
\extramarks{Thag 2.7}{Thag 2.7}

\begin{verse}%
Just\marginnote{1.1} as rain seeps into \\
a poorly roofed house, \\
lust seeps into \\
an undeveloped mind. 

Just\marginnote{2.1} as rain doesn’t seep into \\
a well roofed house, \\
lust doesn’t seep into \\
a well-developed mind. 

%
\end{verse}

%
\section*{{\suttatitleacronym Thag 2.8}{\suttatitletranslation Surādha }{\suttatitleroot Surādhattheragāthā}}
\addcontentsline{toc}{section}{\tocacronym{Thag 2.8} \toctranslation{Surādha } \tocroot{Surādhattheragāthā}}
\markboth{Surādha }{Surādhattheragāthā}
\extramarks{Thag 2.8}{Thag 2.8}

\begin{verse}%
Rebirth\marginnote{1.1} is ended for me; \\
the victor’s instruction is fulfilled; \\
what they call a “net” is given up; \\
the conduit to rebirth is eradicated. 

I’ve\marginnote{2.1} reached the goal \\
for the sake of which I went forth \\
from the lay life to homelessness: \\
the ending of all fetters. 

%
\end{verse}

%
\section*{{\suttatitleacronym Thag 2.9}{\suttatitletranslation Gotama (1st) }{\suttatitleroot Gotamattheragāthā}}
\addcontentsline{toc}{section}{\tocacronym{Thag 2.9} \toctranslation{Gotama (1st) } \tocroot{Gotamattheragāthā}}
\markboth{Gotama (1st) }{Gotamattheragāthā}
\extramarks{Thag 2.9}{Thag 2.9}

\begin{verse}%
Sages\marginnote{1.1} sleep at ease \\
when they’re not bound to women. \\
For the truth is hard to find among them \\
and one must always be guarded. 

Sensual\marginnote{2.1} pleasure, you’ve been slain! \\
We’re not in your debt any more. \\
Now we go to extinguishment, \\
where there is no sorrow. 

%
\end{verse}

%
\section*{{\suttatitleacronym Thag 2.10}{\suttatitletranslation Vasabha }{\suttatitleroot Vasabhattheragāthā}}
\addcontentsline{toc}{section}{\tocacronym{Thag 2.10} \toctranslation{Vasabha } \tocroot{Vasabhattheragāthā}}
\markboth{Vasabha }{Vasabhattheragāthā}
\extramarks{Thag 2.10}{Thag 2.10}

\begin{verse}%
First\marginnote{1.1} one kills oneself, \\
then one kills others. \\
One kills oneself, really dead, \\
like one who kills birds using a dead bird as a decoy. 

A\marginnote{2.1} brahmin’s color is not on the outside; \\
a brahmin is colored on the inside. \\
Whoever harbors bad deeds \\
is truly a dark one, Sujampati. 

%
\end{verse}

%
\addtocontents{toc}{\let\protect\contentsline\protect\nopagecontentsline}
\chapter*{Chapter Two }
\addcontentsline{toc}{chapter}{\tocchapterline{Chapter Two }}
\addtocontents{toc}{\let\protect\contentsline\protect\oldcontentsline}

%
\section*{{\suttatitleacronym Thag 2.11}{\suttatitletranslation Mahācunda }{\suttatitleroot Mahācundattheragāthā}}
\addcontentsline{toc}{section}{\tocacronym{Thag 2.11} \toctranslation{Mahācunda } \tocroot{Mahācundattheragāthā}}
\markboth{Mahācunda }{Mahācundattheragāthā}
\extramarks{Thag 2.11}{Thag 2.11}

\begin{verse}%
It’s\marginnote{1.1} from eagerness to learn that learning grows; \\
when you’re learned, wisdom grows; \\
by wisdom, you know the goal; \\
knowing the goal brings happiness. 

You\marginnote{2.1} should frequent remote lodgings \\
and practice to be released from fetters. \\
If you don’t find enjoyment there, \\
live in the \textsanskrit{Saṅgha}, self-guarded and mindful. 

%
\end{verse}

%
\section*{{\suttatitleacronym Thag 2.12}{\suttatitletranslation Jotidāsa }{\suttatitleroot Jotidāsattheragāthā}}
\addcontentsline{toc}{section}{\tocacronym{Thag 2.12} \toctranslation{Jotidāsa } \tocroot{Jotidāsattheragāthā}}
\markboth{Jotidāsa }{Jotidāsattheragāthā}
\extramarks{Thag 2.12}{Thag 2.12}

\begin{verse}%
People\marginnote{1.1} who act harshly—\footnote{The reading and derivation of \textit{\textsanskrit{veṭhamissena}} are unclear. \textit{\textsanskrit{Veṭha}} means “twist, strap, turban”. \textit{Missa} means “mixed” or “plaited” (\href{https://suttacentral.net/pli-tv-bu-vb-ss2/en/sujato\#2.1.21}{Bu Ss 2:2.1.21}. Thus it probably refers to a kind of strong twisted material used to tie or bind, i.e. rope. } \\
attacking people, \\
binding them with rope, \\
hurting them in all kinds of ways—\\
they’re treated in the same way; \\
their deeds don’t vanish. 

Whatever\marginnote{2.1} deeds a person does, \\
whether good or bad, \\
they are the heir to each \\
and every deed they do. 

%
\end{verse}

%
\section*{{\suttatitleacronym Thag 2.13}{\suttatitletranslation Heraññakāni }{\suttatitleroot Heraññakānittheragāthā}}
\addcontentsline{toc}{section}{\tocacronym{Thag 2.13} \toctranslation{Heraññakāni } \tocroot{Heraññakānittheragāthā}}
\markboth{Heraññakāni }{Heraññakānittheragāthā}
\extramarks{Thag 2.13}{Thag 2.13}

\begin{verse}%
The\marginnote{1.1} days and nights rush by, \\
and then life is cut short. \\
The life of mortals wastes away, \\
like the water in tiny streams. 

But\marginnote{2.1} while doing bad deeds \\
the fool doesn’t realize—\\
it’ll be bitter later on; \\
for the result will be bad for them. 

%
\end{verse}

%
\section*{{\suttatitleacronym Thag 2.14}{\suttatitletranslation Somamitta }{\suttatitleroot Somamittattheragāthā}}
\addcontentsline{toc}{section}{\tocacronym{Thag 2.14} \toctranslation{Somamitta } \tocroot{Somamittattheragāthā}}
\markboth{Somamitta }{Somamittattheragāthā}
\extramarks{Thag 2.14}{Thag 2.14}

\begin{verse}%
If\marginnote{1.1} you’re lost in the middle of a great sea, \\
and you clamber up on a little log, you’ll sink. \\
So too, a person who lives well \\
sinks by relying on a lazy person. \\
Hence you should avoid such \\
a lazy person who lacks energy. 

Dwell\marginnote{2.1} with the noble ones \\
who are secluded and determined \\
and always energetic; \\
the astute who practice absorption. 

%
\end{verse}

%
\section*{{\suttatitleacronym Thag 2.15}{\suttatitletranslation Sabbamitta }{\suttatitleroot Sabbamittattheragāthā}}
\addcontentsline{toc}{section}{\tocacronym{Thag 2.15} \toctranslation{Sabbamitta } \tocroot{Sabbamittattheragāthā}}
\markboth{Sabbamitta }{Sabbamittattheragāthā}
\extramarks{Thag 2.15}{Thag 2.15}

\begin{verse}%
People\marginnote{1.1} are attached to people; \\
people depend on people; \\
people are hurt by people; \\
and people hurt people. 

So\marginnote{2.1} what’s the point of people, \\
or those born of people? \\
Go, abandon these people, \\
who’ve hurt so many people. 

%
\end{verse}

%
\section*{{\suttatitleacronym Thag 2.16}{\suttatitletranslation Mahākāḷa }{\suttatitleroot Mahākāḷattheragāthā}}
\addcontentsline{toc}{section}{\tocacronym{Thag 2.16} \toctranslation{Mahākāḷa } \tocroot{Mahākāḷattheragāthā}}
\markboth{Mahākāḷa }{Mahākāḷattheragāthā}
\extramarks{Thag 2.16}{Thag 2.16}

\begin{verse}%
There’s\marginnote{1.1} a big black woman who looks like a crow. \\
She broke off thigh-bones, first one then another; \\
she broke off arm-bones, first one then another; \\
she broke off a skull like a curd-bowl, and then \\
arranged them and sat nearby. 

When\marginnote{2.1} an ignorant person builds up attachments, \\
that dullard returns to suffering again and again. \\
So let one who understands \\>not build up attachments: \\
may I never again lie with a broken skull! 

%
\end{verse}

%
\section*{{\suttatitleacronym Thag 2.17}{\suttatitletranslation Tissa (3rd) }{\suttatitleroot Tissattheragāthā}}
\addcontentsline{toc}{section}{\tocacronym{Thag 2.17} \toctranslation{Tissa (3rd) } \tocroot{Tissattheragāthā}}
\markboth{Tissa (3rd) }{Tissattheragāthā}
\extramarks{Thag 2.17}{Thag 2.17}

\begin{verse}%
A\marginnote{1.1} shaven one wrapped in the outer robe \\
gets many enemies \\
when they receive food and drink, \\
clothes and bedding. 

Knowing\marginnote{2.1} this danger, \\
this great fear in honors, \\
a mendicant should wander mindful, \\
with few possessions, not festering. 

%
\end{verse}

%
\section*{{\suttatitleacronym Thag 2.18}{\suttatitletranslation Kimbila (2nd) }{\suttatitleroot Kimilattheragāthā}}
\addcontentsline{toc}{section}{\tocacronym{Thag 2.18} \toctranslation{Kimbila (2nd) } \tocroot{Kimilattheragāthā}}
\markboth{Kimbila (2nd) }{Kimilattheragāthā}
\extramarks{Thag 2.18}{Thag 2.18}

\begin{verse}%
In\marginnote{1.1} the Eastern Bamboo Park \\
the Sakyan friends,\footnote{This is a reference to \href{https://suttacentral.net/mn31/en/sujato\#2.1}{MN 31:2.1} = \href{https://suttacentral.net/mn128/en/sujato\#8.1}{MN 128:8.1}, where Anuruddha, Kimbila, and Nandiya were practicing together in this grove. } \\
having given up great wealth, \\
are happy with the scraps in their bowls. 

Energetic,\marginnote{2.1} resolute, \\
always staunchly vigorous; \\
having given up mundane delights, \\
they enjoy the delights of the Dhamma. 

%
\end{verse}

%
\section*{{\suttatitleacronym Thag 2.19}{\suttatitletranslation Nanda }{\suttatitleroot Nandattheragāthā}}
\addcontentsline{toc}{section}{\tocacronym{Thag 2.19} \toctranslation{Nanda } \tocroot{Nandattheragāthā}}
\markboth{Nanda }{Nandattheragāthā}
\extramarks{Thag 2.19}{Thag 2.19}

\begin{verse}%
Because\marginnote{1.1} of focusing on the wrong things, \\
I was addicted to ornamentation. \\
I was vain, fickle, \\
racked by desire for pleasures of the senses. 

But\marginnote{2.1} with the help of the Buddha, \\
the kinsman of the Sun, so skilled in means, \\
I practiced rationally and extracted \\
attachment to continued existence from my mind. 

%
\end{verse}

%
\section*{{\suttatitleacronym Thag 2.20}{\suttatitletranslation Sirima }{\suttatitleroot Sirimattheragāthā}}
\addcontentsline{toc}{section}{\tocacronym{Thag 2.20} \toctranslation{Sirima } \tocroot{Sirimattheragāthā}}
\markboth{Sirima }{Sirimattheragāthā}
\extramarks{Thag 2.20}{Thag 2.20}

\begin{verse}%
If\marginnote{1.1} others praise one \\
who has no immersion, \\
they praise in vain, \\
as one has no immersion. 

If\marginnote{2.1} others criticize one \\
who does have immersion, \\
they criticize in vain, \\
as one does have immersion. 

%
\end{verse}

%
\addtocontents{toc}{\let\protect\contentsline\protect\nopagecontentsline}
\chapter*{Chapter Three }
\addcontentsline{toc}{chapter}{\tocchapterline{Chapter Three }}
\addtocontents{toc}{\let\protect\contentsline\protect\oldcontentsline}

%
\section*{{\suttatitleacronym Thag 2.21}{\suttatitletranslation Uttara (2nd) }{\suttatitleroot Uttarattheragāthā}}
\addcontentsline{toc}{section}{\tocacronym{Thag 2.21} \toctranslation{Uttara (2nd) } \tocroot{Uttarattheragāthā}}
\markboth{Uttara (2nd) }{Uttarattheragāthā}
\extramarks{Thag 2.21}{Thag 2.21}

\begin{verse}%
I’ve\marginnote{1.1} fully understood the aggregates; \\
I’ve eradicated craving; \\
I’ve developed the factors of awakening, \\
I’ve attained the ending of defilements. 

Having\marginnote{2.1} fully understood the aggregates, \\
having plucked out the weaver of the web, \\
having developed the factors of awakening, \\
being undefiled, I will be fully extinguished. 

%
\end{verse}

%
\section*{{\suttatitleacronym Thag 2.22}{\suttatitletranslation Bhaddaji }{\suttatitleroot Bhaddajittheragāthā}}
\addcontentsline{toc}{section}{\tocacronym{Thag 2.22} \toctranslation{Bhaddaji } \tocroot{Bhaddajittheragāthā}}
\markboth{Bhaddaji }{Bhaddajittheragāthā}
\extramarks{Thag 2.22}{Thag 2.22}

\begin{verse}%
There\marginnote{1.1} was a king named \textsanskrit{Panāda} \\
who had a sacrificial post all golden. \\
Its height was sixteen times its width, \\
and the top was a thousandfold. 

It\marginnote{2.1} had a thousand panels and a hundred ball-caps, \\
all adorned with banners, and made of sun gold.\footnote{Gold is also called \textit{harita} at Atharvaveda 11.2.12a. } \\
There danced the centaurs, \\
numbering seven times six thousand. 

%
\end{verse}

%
\section*{{\suttatitleacronym Thag 2.23}{\suttatitletranslation Sobhita }{\suttatitleroot Sobhitattheragāthā}}
\addcontentsline{toc}{section}{\tocacronym{Thag 2.23} \toctranslation{Sobhita } \tocroot{Sobhitattheragāthā}}
\markboth{Sobhita }{Sobhitattheragāthā}
\extramarks{Thag 2.23}{Thag 2.23}

\begin{verse}%
As\marginnote{1.1} a monk, mindful and wise, \\
empowered and full of energy, \\
I recollected five hundred eons \\
in a single night. 

Developing\marginnote{2.1} the four mindfulness meditations, \\
the seven factors of awakening, and the eightfold path,\footnote{The Pali text says only “developing the seven and eight”. } \\
I recollected five hundred eons \\
in a single night. 

%
\end{verse}

%
\section*{{\suttatitleacronym Thag 2.24}{\suttatitletranslation Valliya (3rd) }{\suttatitleroot Valliyattheragāthā}}
\addcontentsline{toc}{section}{\tocacronym{Thag 2.24} \toctranslation{Valliya (3rd) } \tocroot{Valliyattheragāthā}}
\markboth{Valliya (3rd) }{Valliyattheragāthā}
\extramarks{Thag 2.24}{Thag 2.24}

\begin{verse}%
The\marginnote{1.1} duty of one whose energy is strong; \\
the duty of one who longs to wake up: \\
that I’ll do, I won’t fail—\\
see my energy and vigor! 

Teach\marginnote{2.1} me the path, \\
the direct route that culminates in freedom from death. \\
I’ll know it with wisdom, \\
as the Ganges knows the ocean. 

%
\end{verse}

%
\section*{{\suttatitleacronym Thag 2.25}{\suttatitletranslation Vītasoka }{\suttatitleroot Vītasokattheragāthā}}
\addcontentsline{toc}{section}{\tocacronym{Thag 2.25} \toctranslation{Vītasoka } \tocroot{Vītasokattheragāthā}}
\markboth{Vītasoka }{Vītasokattheragāthā}
\extramarks{Thag 2.25}{Thag 2.25}

\begin{verse}%
The\marginnote{1.1} barber approached \\
to shave my head. \\
I picked up a mirror \\
and examined my body. 

My\marginnote{2.1} body appeared hollow; \\
I once was blind, but the darkness disappeared. \\
My fancy hairdo has been cut off: \\
now there’ll be no more future lives. 

%
\end{verse}

%
\section*{{\suttatitleacronym Thag 2.26}{\suttatitletranslation Puṇṇamāsa (2nd) }{\suttatitleroot Puṇṇamāsattheragāthā}}
\addcontentsline{toc}{section}{\tocacronym{Thag 2.26} \toctranslation{Puṇṇamāsa (2nd) } \tocroot{Puṇṇamāsattheragāthā}}
\markboth{Puṇṇamāsa (2nd) }{Puṇṇamāsattheragāthā}
\extramarks{Thag 2.26}{Thag 2.26}

\begin{verse}%
I\marginnote{1.1} gave up the five hindrances \\
for the sake of sanctuary from the yoke. \\
I took Dhamma as a mirror \\
for knowing and seeing myself. 

I\marginnote{2.1} examined this body, \\
all of it, inside and out. \\
Internally and externally \\
my body appeared hollow. 

%
\end{verse}

%
\section*{{\suttatitleacronym Thag 2.27}{\suttatitletranslation Nandaka (1st) }{\suttatitleroot Nandakattheragāthā}}
\addcontentsline{toc}{section}{\tocacronym{Thag 2.27} \toctranslation{Nandaka (1st) } \tocroot{Nandakattheragāthā}}
\markboth{Nandaka (1st) }{Nandakattheragāthā}
\extramarks{Thag 2.27}{Thag 2.27}

\begin{verse}%
Though\marginnote{1.1} a fine thoroughbred may stumble, \\
it soon stands firm again. \\
It gains even more urgency, \\
and draws its load undeterred. 

Even\marginnote{2.1} so is one accomplished in vision, \\
a disciple of the Buddha. \\
Remember me as a thoroughbred, \\
the Buddha’s rightful son. 

%
\end{verse}

%
\section*{{\suttatitleacronym Thag 2.28}{\suttatitletranslation Bharata }{\suttatitleroot Bharatattheragāthā}}
\addcontentsline{toc}{section}{\tocacronym{Thag 2.28} \toctranslation{Bharata } \tocroot{Bharatattheragāthā}}
\markboth{Bharata }{Bharatattheragāthā}
\extramarks{Thag 2.28}{Thag 2.28}

\begin{verse}%
Come,\marginnote{1.1} Nandaka, let’s go \\
to visit our preceptor. \\
We’ll roar our lion’s roar \\
before the best of Buddhas. 

The\marginnote{2.1} sage gave us the going forth \\
out of compassion, so we could realize \\
the ending of all fetters—\\
now we have reached that goal. 

%
\end{verse}

%
\section*{{\suttatitleacronym Thag 2.29}{\suttatitletranslation Bhāradvāja }{\suttatitleroot Bhāradvājattheragāthā}}
\addcontentsline{toc}{section}{\tocacronym{Thag 2.29} \toctranslation{Bhāradvāja } \tocroot{Bhāradvājattheragāthā}}
\markboth{Bhāradvāja }{Bhāradvājattheragāthā}
\extramarks{Thag 2.29}{Thag 2.29}

\begin{verse}%
This\marginnote{1.1} is how the wise roar: \\
like lions in mountain caves, \\
heroes, triumphant in battle, \\
having vanquished \textsanskrit{Māra} and his mount. 

I’ve\marginnote{2.1} served the teacher; \\
I’ve honored the Dhamma and the \textsanskrit{Saṅgha}; \\
I’m happy and joyful, \\
because I’ve seen my son free of defilements. 

%
\end{verse}

%
\section*{{\suttatitleacronym Thag 2.30}{\suttatitletranslation Kaṇhadinna }{\suttatitleroot Kaṇhadinnattheragāthā}}
\addcontentsline{toc}{section}{\tocacronym{Thag 2.30} \toctranslation{Kaṇhadinna } \tocroot{Kaṇhadinnattheragāthā}}
\markboth{Kaṇhadinna }{Kaṇhadinnattheragāthā}
\extramarks{Thag 2.30}{Thag 2.30}

\begin{verse}%
I\marginnote{1.1} regularly sat close by true persons \\
and learnt the teaching. \\
What I learned, I practiced, \\
the direct route that culminates in freedom from death. 

I’ve\marginnote{2.1} slain the desire to be reborn, \\
it won’t be found in me again. \\
It was not, and it won’t be in me, \\
and it isn’t found in me now. 

%
\end{verse}

%
\addtocontents{toc}{\let\protect\contentsline\protect\nopagecontentsline}
\chapter*{Chapter Four }
\addcontentsline{toc}{chapter}{\tocchapterline{Chapter Four }}
\addtocontents{toc}{\let\protect\contentsline\protect\oldcontentsline}

%
\section*{{\suttatitleacronym Thag 2.31}{\suttatitletranslation Migasira }{\suttatitleroot Migasirattheragāthā}}
\addcontentsline{toc}{section}{\tocacronym{Thag 2.31} \toctranslation{Migasira } \tocroot{Migasirattheragāthā}}
\markboth{Migasira }{Migasirattheragāthā}
\extramarks{Thag 2.31}{Thag 2.31}

\begin{verse}%
When\marginnote{1.1} I had gone forth \\
in the teaching of the Buddha, \\
while letting go, I rose up; \\
escaping the sensual realm. 

Then,\marginnote{2.1} as the supreme one looked on, \\
my mind was freed. \\
My freedom is unshakable \\
with the ending of all fetters. 

%
\end{verse}

%
\section*{{\suttatitleacronym Thag 2.32}{\suttatitletranslation Sivaka }{\suttatitleroot Sivakattheragāthā}}
\addcontentsline{toc}{section}{\tocacronym{Thag 2.32} \toctranslation{Sivaka } \tocroot{Sivakattheragāthā}}
\markboth{Sivaka }{Sivakattheragāthā}
\extramarks{Thag 2.32}{Thag 2.32}

\begin{verse}%
Houses\marginnote{1.1} are impermanent—\\
on and on, life after life. \\
I’ve been searching for the house-builder—\\
painful is birth again and again. 

I’ve\marginnote{2.1} seen you, house-builder! \\
You won’t build a house again. \\
Your rafters are all broken, \\
your ridgepole is shattered. \\
My mind is released from limits: \\
in this very life it will dissipate. 

%
\end{verse}

%
\section*{{\suttatitleacronym Thag 2.33}{\suttatitletranslation Upavāna }{\suttatitleroot Upavāṇattheragāthā}}
\addcontentsline{toc}{section}{\tocacronym{Thag 2.33} \toctranslation{Upavāna } \tocroot{Upavāṇattheragāthā}}
\markboth{Upavāna }{Upavāṇattheragāthā}
\extramarks{Thag 2.33}{Thag 2.33}

\begin{verse}%
The\marginnote{1.1} perfected one, the Holy One in the world, \\
the sage is afflicted by winds. \\
If there’s hot water, \\
give it to the sage, brahmin. 

I\marginnote{2.1} wish to bring it to the one \\
who is esteemed by the estimable, \\
honored by the honorable, \\
and venerated by the venerable. 

%
\end{verse}

%
\section*{{\suttatitleacronym Thag 2.34}{\suttatitletranslation Isidinna }{\suttatitleroot Isidinnattheragāthā}}
\addcontentsline{toc}{section}{\tocacronym{Thag 2.34} \toctranslation{Isidinna } \tocroot{Isidinnattheragāthā}}
\markboth{Isidinna }{Isidinnattheragāthā}
\extramarks{Thag 2.34}{Thag 2.34}

\begin{verse}%
I’ve\marginnote{1.1} seen lay disciples \\>who have memorized discourses, \\
saying, “Sensual pleasures are impermanent”. \\
But they’re obsessed with jeweled earrings, \\
concerned for their partners and children. 

To\marginnote{2.1} be honest, they don’t know Dhamma, \\
even though they say \\>“Sensual pleasures are impermanent”. \\
They don’t have the power to cut their lust, \\
so they cling to children, wives, and wealth. 

%
\end{verse}

%
\section*{{\suttatitleacronym Thag 2.35}{\suttatitletranslation Sambulakaccāna }{\suttatitleroot Sambulakaccānattheragāthā}}
\addcontentsline{toc}{section}{\tocacronym{Thag 2.35} \toctranslation{Sambulakaccāna } \tocroot{Sambulakaccānattheragāthā}}
\markboth{Sambulakaccāna }{Sambulakaccānattheragāthā}
\extramarks{Thag 2.35}{Thag 2.35}

\begin{verse}%
The\marginnote{1.1} heavens rain, the heavens pour, \\
I’m staying alone in a frightful hole. \\
But while I’m staying alone in that frightful hole, \\
I’ve no fear, no dread, no goosebumps. 

This\marginnote{2.1} is my normal state \\
while staying alone in a frightful hole: \\
I’ve no fear, \\
no dread, no goosebumps. 

%
\end{verse}

%
\section*{{\suttatitleacronym Thag 2.36}{\suttatitletranslation Nitaka }{\suttatitleroot Nitakattheragāthā}}
\addcontentsline{toc}{section}{\tocacronym{Thag 2.36} \toctranslation{Nitaka } \tocroot{Nitakattheragāthā}}
\markboth{Nitaka }{Nitakattheragāthā}
\extramarks{Thag 2.36}{Thag 2.36}

\begin{verse}%
Whose\marginnote{1.1} mind is like a rock, \\
steady, never trembling—\\
free of desire for desirable things, \\
not getting annoyed when things are annoying? \\
From where will suffering strike one \\
whose mind is developed like this? 

My\marginnote{2.1} mind is like a rock, \\
steady, never trembling—\\
free of desire for desirable things, \\
not getting annoyed when things are annoying. \\
From where will suffering strike me \\
whose mind is developed like this? 

%
\end{verse}

%
\section*{{\suttatitleacronym Thag 2.37}{\suttatitletranslation Soṇapoṭiriya }{\suttatitleroot Soṇapoṭiriyattheragāthā}}
\addcontentsline{toc}{section}{\tocacronym{Thag 2.37} \toctranslation{Soṇapoṭiriya } \tocroot{Soṇapoṭiriyattheragāthā}}
\markboth{Soṇapoṭiriya }{Soṇapoṭiriyattheragāthā}
\extramarks{Thag 2.37}{Thag 2.37}

\begin{verse}%
Night,\marginnote{1.1} with her garland of stars, \\
is not only for sleeping. \\
For those who know, \\
this night is really for waking. 

Were\marginnote{2.1} I to fall from the back of an elephant, \\
trampled by the tuskers that follow, \\
better for me to die in battle, \\
than to live on in defeat. 

%
\end{verse}

%
\section*{{\suttatitleacronym Thag 2.38}{\suttatitletranslation Nisabha }{\suttatitleroot Nisabhattheragāthā}}
\addcontentsline{toc}{section}{\tocacronym{Thag 2.38} \toctranslation{Nisabha } \tocroot{Nisabhattheragāthā}}
\markboth{Nisabha }{Nisabhattheragāthā}
\extramarks{Thag 2.38}{Thag 2.38}

\begin{verse}%
One\marginnote{1.1} who’s given up the five sensual titillations, \\
so pleasing and delightful, \\
and who’s left the home life out of faith—\\
let them make an end to suffering! 

I\marginnote{2.1} don’t long for death; \\
I don’t long for life; \\
I await my time, \\
aware and mindful. 

%
\end{verse}

%
\section*{{\suttatitleacronym Thag 2.39}{\suttatitletranslation Usabha (2nd) }{\suttatitleroot Usabhattheragāthā}}
\addcontentsline{toc}{section}{\tocacronym{Thag 2.39} \toctranslation{Usabha (2nd) } \tocroot{Usabhattheragāthā}}
\markboth{Usabha (2nd) }{Usabhattheragāthā}
\extramarks{Thag 2.39}{Thag 2.39}

\begin{verse}%
Arranging\marginnote{1.1} a robe over my shoulder, \\
the color of young mango sprouts, \\
I entered the village for alms \\
sitting on an elephant’s neck! 

But\marginnote{2.1} when I dismounted from the elephant, \\
I was struck with a sense of urgency. \\
I burned with shame, but then I found peace, \\
and attained the ending of defilements. 

%
\end{verse}

%
\section*{{\suttatitleacronym Thag 2.40}{\suttatitletranslation Kappaṭakura }{\suttatitleroot Kappaṭakurattheragāthā}}
\addcontentsline{toc}{section}{\tocacronym{Thag 2.40} \toctranslation{Kappaṭakura } \tocroot{Kappaṭakurattheragāthā}}
\markboth{Kappaṭakura }{Kappaṭakurattheragāthā}
\extramarks{Thag 2.40}{Thag 2.40}

\begin{verse}%
This\marginnote{1.1} fellow, “Rag-rice”, he sure is a rag! \\
Into the vase of freedom from death, \\>polished and overflowing, \\
sufficient teaching has been poured; \\
the path to build up absorptions has been laid out. 

Don’t\marginnote{2.1} nod off, Rag—\\
I’ll smack your ear! \\
Nodding off in the middle of the \textsanskrit{Saṅgha}? \\
You know no bounds. 

%
\end{verse}

%
\addtocontents{toc}{\let\protect\contentsline\protect\nopagecontentsline}
\chapter*{Chapter Five }
\addcontentsline{toc}{chapter}{\tocchapterline{Chapter Five }}
\addtocontents{toc}{\let\protect\contentsline\protect\oldcontentsline}

%
\section*{{\suttatitleacronym Thag 2.41}{\suttatitletranslation Kassapa the Prince }{\suttatitleroot Kumārakassapattheragāthā}}
\addcontentsline{toc}{section}{\tocacronym{Thag 2.41} \toctranslation{Kassapa the Prince } \tocroot{Kumārakassapattheragāthā}}
\markboth{Kassapa the Prince }{Kumārakassapattheragāthā}
\extramarks{Thag 2.41}{Thag 2.41}

\begin{verse}%
Oh,\marginnote{1.1} the Buddhas! Oh, the Dhammas!\footnote{“Kassapa the Prince” (\textsanskrit{Kumārakassapa}) was ordained at twenty (\href{https://suttacentral.net/pli-tv-kd1/en/sujato\#75.1.1}{pli{-}tv{-}kd1:75.1.1}). He features in the \textsanskrit{Pāyāsisutta} (\href{https://suttacentral.net/dn23/en/sujato}{DN 23}), the Vammikasutta (\href{https://suttacentral.net/mn23/en/sujato}{MN 23}), and he was declared the foremost of those with brilliant speech (\href{https://suttacentral.net/an1.217/en/sujato}{AN 1.217}). } \\
Oh, the accomplishments of the Teacher! \\
Here a disciple may realize \\
such a teaching for themselves. 

Through\marginnote{2.1} countless eons \\
they obtained substantial forms. \\
This is their last, \\
their very final body \\
in the transmigration through births and deaths; \\
now there’ll be no more future lives. 

%
\end{verse}

%
\section*{{\suttatitleacronym Thag 2.42}{\suttatitletranslation Dhammapāla }{\suttatitleroot Dhammapālattheragāthā}}
\addcontentsline{toc}{section}{\tocacronym{Thag 2.42} \toctranslation{Dhammapāla } \tocroot{Dhammapālattheragāthā}}
\markboth{Dhammapāla }{Dhammapālattheragāthā}
\extramarks{Thag 2.42}{Thag 2.42}

\begin{verse}%
The\marginnote{1.1} young monk \\
who is devoted to the teaching of the Buddha, \\
wakeful while others sleep—\\
his life is not in vain. 

So\marginnote{2.1} let the wise devote themselves \\
to faith, ethical behavior, \\
confidence, and insight into the teaching, \\
remembering the instructions of the Buddhas. 

%
\end{verse}

%
\section*{{\suttatitleacronym Thag 2.43}{\suttatitletranslation Brahmāli }{\suttatitleroot Brahmālittheragāthā}}
\addcontentsline{toc}{section}{\tocacronym{Thag 2.43} \toctranslation{Brahmāli } \tocroot{Brahmālittheragāthā}}
\markboth{Brahmāli }{Brahmālittheragāthā}
\extramarks{Thag 2.43}{Thag 2.43}

\begin{verse}%
Whose\marginnote{1.1} faculties have become serene, \\
like horses tamed by a charioteer? \\
With conceit and defilements given up, \\
who is such as envied by even the gods? 

My\marginnote{2.1} faculties have become serene, \\
like horses tamed by a charioteer. \\
With conceit and defilements given up, \\
I am such as envied by even the gods. 

%
\end{verse}

%
\section*{{\suttatitleacronym Thag 2.44}{\suttatitletranslation Mogharājā }{\suttatitleroot Mogharājattheragāthā}}
\addcontentsline{toc}{section}{\tocacronym{Thag 2.44} \toctranslation{Mogharājā } \tocroot{Mogharājattheragāthā}}
\markboth{Mogharājā }{Mogharājattheragāthā}
\extramarks{Thag 2.44}{Thag 2.44}

\begin{verse}%
“Your\marginnote{1.1} skin is nasty but your heart is good; \\
\textsanskrit{Mogharājā}, you’re always immersed in \textsanskrit{samādhi}. \\
But in the nights of winter, so dark and cold, \\
how will you get by, monk?” 

“I’ve\marginnote{2.1} heard that all the Magadhans \\
have an abundance of grain. \\
I’ll make my bed under a thatched roof, \\
just like those who live in comfort.” 

%
\end{verse}

%
\section*{{\suttatitleacronym Thag 2.45}{\suttatitletranslation Visākhapañcālaputta }{\suttatitleroot Visākhapañcālaputtattheragāthā}}
\addcontentsline{toc}{section}{\tocacronym{Thag 2.45} \toctranslation{Visākhapañcālaputta } \tocroot{Visākhapañcālaputtattheragāthā}}
\markboth{Visākhapañcālaputta }{Visākhapañcālaputtattheragāthā}
\extramarks{Thag 2.45}{Thag 2.45}

\begin{verse}%
One\marginnote{1.1} should not suspend others from the \textsanskrit{Saṅgha}, \\>nor raise objections against them;\footnote{This refers to specific procedures of Vinaya. } \\
and neither disparage nor raise one’s voice against \\>one who has crossed to the further shore. \\
One should not praise oneself among the assemblies, \\
not restless, measured in speech, \\>and true to your vows. 

For\marginnote{2.1} one who sees the meaning \\>so very subtle and fine, \\
who is skilled in thought and humble in manner, \\
who has cultivated mature ethics—\\
it’s not hard to gain extinguishment. 

%
\end{verse}

%
\section*{{\suttatitleacronym Thag 2.46}{\suttatitletranslation Cūḷaka }{\suttatitleroot Cūḷakattheragāthā}}
\addcontentsline{toc}{section}{\tocacronym{Thag 2.46} \toctranslation{Cūḷaka } \tocroot{Cūḷakattheragāthā}}
\markboth{Cūḷaka }{Cūḷakattheragāthā}
\extramarks{Thag 2.46}{Thag 2.46}

\begin{verse}%
The\marginnote{1.1} peacocks cry out with their fair crests and tails, \\
their lovely blue necks and fair faces, \\>their beautiful song and their call. \\
This broad earth is lush with grass and dew, \\
and the firmament is full of beautiful clouds. 

One\marginnote{2.1} practicing absorption is happy in mind, \\>and their appearance is uplifting; \\
going forth in the teaching of the Buddha \\>is easy for a good person. \\
You should realize that supreme state that does not pass, \\
so very pure, subtle, and hard to see. 

%
\end{verse}

%
\section*{{\suttatitleacronym Thag 2.47}{\suttatitletranslation Anūpama }{\suttatitleroot Anūpamattheragāthā}}
\addcontentsline{toc}{section}{\tocacronym{Thag 2.47} \toctranslation{Anūpama } \tocroot{Anūpamattheragāthā}}
\markboth{Anūpama }{Anūpamattheragāthā}
\extramarks{Thag 2.47}{Thag 2.47}

\begin{verse}%
The\marginnote{1.1} conceited mind, addicted to pleasure, \\
impales itself on its own stake. \\
It always goes where \\
there’s a stake, a chopping board. 

I\marginnote{2.1} declare you the loser mind! \\
I declare you the insidious mind! \\
You’ve found the teacher so hard to find—\\
don’t lead me away from the goal. 

%
\end{verse}

%
\section*{{\suttatitleacronym Thag 2.48}{\suttatitletranslation Vajjita }{\suttatitleroot Vajjitattheragāthā}}
\addcontentsline{toc}{section}{\tocacronym{Thag 2.48} \toctranslation{Vajjita } \tocroot{Vajjitattheragāthā}}
\markboth{Vajjita }{Vajjitattheragāthā}
\extramarks{Thag 2.48}{Thag 2.48}

\begin{verse}%
Transmigrating\marginnote{1.1} for such a long time, \\
I’ve proceeded through various states of rebirth, \\
not seeing the noble truths, \\
a blind, unenlightened person. 

But\marginnote{2.1} when I became heedful, \\
transmigrations were mown down.\footnote{For \textit{\textsanskrit{saṁsāra}} in plural see \href{https://suttacentral.net/dn18/en/sujato\#10.7}{DN 18:10.7}. } \\
All states of rebirth are cut off; \\
now there’ll be no more future lives. 

%
\end{verse}

%
\section*{{\suttatitleacronym Thag 2.49}{\suttatitletranslation Sandhita }{\suttatitleroot Sandhitattheragāthā}}
\addcontentsline{toc}{section}{\tocacronym{Thag 2.49} \toctranslation{Sandhita } \tocroot{Sandhitattheragāthā}}
\markboth{Sandhita }{Sandhitattheragāthā}
\extramarks{Thag 2.49}{Thag 2.49}

\begin{verse}%
Beneath\marginnote{1.1} the Bodhi Tree, \\
bright green and growing, \\
being mindful, my perception \\
became one with the Buddha. 

It’s\marginnote{2.1} been thirty one eons \\
since I gained that perception; \\
and it’s due to that perception \\
that I’ve attained the ending of defilements. 

%
\end{verse}

%
\addtocontents{toc}{\let\protect\contentsline\protect\nopagecontentsline}
\part*{The Book of the Threes }
\addcontentsline{toc}{part}{The Book of the Threes }
\markboth{}{}
\addtocontents{toc}{\let\protect\contentsline\protect\oldcontentsline}

%
\addtocontents{toc}{\let\protect\contentsline\protect\nopagecontentsline}
\chapter*{Chapter One }
\addcontentsline{toc}{chapter}{\tocchapterline{Chapter One }}
\addtocontents{toc}{\let\protect\contentsline\protect\oldcontentsline}

%
\section*{{\suttatitleacronym Thag 3.1}{\suttatitletranslation Aṅgaṇikabhāradvāja }{\suttatitleroot Aṅgaṇikabhāradvājattheragāthā}}
\addcontentsline{toc}{section}{\tocacronym{Thag 3.1} \toctranslation{Aṅgaṇikabhāradvāja } \tocroot{Aṅgaṇikabhāradvājattheragāthā}}
\markboth{Aṅgaṇikabhāradvāja }{Aṅgaṇikabhāradvājattheragāthā}
\extramarks{Thag 3.1}{Thag 3.1}

\begin{verse}%
Seeking\marginnote{1.1} purity the wrong way, \\
I served the sacred fire in a grove. \\
Not knowing the path to purity, \\
I mortified my flesh in search of immortality. 

I’ve\marginnote{2.1} gained this happiness by means of happiness: \\
see the excellence of the teaching! \\
I’ve attained the three knowledges \\
and fulfilled the Buddha’s instructions. 

I\marginnote{3.1} used to be brahmin by kin, \\
but now I really am a brahmin! \\
I am master of the three knowledges, \\>I’m a bathed initiate, \\
I’m a scholar and a knowledge master. 

%
\end{verse}

%
\section*{{\suttatitleacronym Thag 3.2}{\suttatitletranslation Paccaya }{\suttatitleroot Paccayattheragāthā}}
\addcontentsline{toc}{section}{\tocacronym{Thag 3.2} \toctranslation{Paccaya } \tocroot{Paccayattheragāthā}}
\markboth{Paccaya }{Paccayattheragāthā}
\extramarks{Thag 3.2}{Thag 3.2}

\begin{verse}%
I\marginnote{1.1} went forth five days ago, \\
a trainee, my heart’s desire unfulfilled. \\
I entered my dwelling \\
and resolved in my heart: 

I\marginnote{2.1} won’t eat; I won’t drink; \\
I won’t leave my dwelling; \\
nor will I lie down on my side—\\
not until the dart of craving is plucked. 

See\marginnote{3.1} my energy and vigor \\
as I meditate like this! \\
I’ve attained the three knowledges \\
and fulfilled the Buddha’s instructions. 

%
\end{verse}

%
\section*{{\suttatitleacronym Thag 3.3}{\suttatitletranslation Bākula }{\suttatitleroot Bākulattheragāthā}}
\addcontentsline{toc}{section}{\tocacronym{Thag 3.3} \toctranslation{Bākula } \tocroot{Bākulattheragāthā}}
\markboth{Bākula }{Bākulattheragāthā}
\extramarks{Thag 3.3}{Thag 3.3}

\begin{verse}%
Whoever\marginnote{1.1} wishes to do afterwards \\
what they should have done before \\
has lost the causes for happiness, \\
and afterwards they’re tormented by regrets. 

You\marginnote{2.1} should only say what you would do; \\
you shouldn’t say what you wouldn’t do. \\
The wise will recognize \\
one who talks without doing. 

Oh!\marginnote{3.1} Extinguishment is so very blissful, \\
as taught by the fully awakened Buddha: \\
sorrowless, stainless, secure, \\
where suffering all ceases. 

%
\end{verse}

%
\section*{{\suttatitleacronym Thag 3.4}{\suttatitletranslation Dhaniya }{\suttatitleroot Dhaniyattheragāthā}}
\addcontentsline{toc}{section}{\tocacronym{Thag 3.4} \toctranslation{Dhaniya } \tocroot{Dhaniyattheragāthā}}
\markboth{Dhaniya }{Dhaniyattheragāthā}
\extramarks{Thag 3.4}{Thag 3.4}

\begin{verse}%
If\marginnote{1.1} you wish to live in happiness, \\
longing for the ascetic life, \\
don’t look down on the \textsanskrit{Saṅgha}’s robes, \\
or its food and drinks. 

If\marginnote{2.1} you wish to live in happiness, \\
longing for the ascetic life, \\
stay in the \textsanskrit{Saṅgha}’s lodgings \\
like a snake making use of a mouse’s hole. 

If\marginnote{3.1} you wish to live in happiness, \\
longing for the ascetic life, \\
develop this one quality: \\
be content with whatever is offered. 

%
\end{verse}

%
\section*{{\suttatitleacronym Thag 3.5}{\suttatitletranslation Mātaṅgaputta }{\suttatitleroot Mātaṅgaputtattheragāthā}}
\addcontentsline{toc}{section}{\tocacronym{Thag 3.5} \toctranslation{Mātaṅgaputta } \tocroot{Mātaṅgaputtattheragāthā}}
\markboth{Mātaṅgaputta }{Mātaṅgaputtattheragāthā}
\extramarks{Thag 3.5}{Thag 3.5}

\begin{verse}%
“It’s\marginnote{1.1} too cold, too hot, \\
too late,” they say. \\
When the young neglect their work like this, \\
opportunities pass them by. 

But\marginnote{2.1} one who considers heat and cold \\
as no more than blades of grass—\\
he does his duties as a man, \\
and his happiness never fails. 

With\marginnote{3.1} my chest I’ll thrust aside \\
the grasses, vines, and creepers, \\
and foster seclusion. 

%
\end{verse}

%
\section*{{\suttatitleacronym Thag 3.6}{\suttatitletranslation Khujjasobhita }{\suttatitleroot Khujjasobhitattheragāthā}}
\addcontentsline{toc}{section}{\tocacronym{Thag 3.6} \toctranslation{Khujjasobhita } \tocroot{Khujjasobhitattheragāthā}}
\markboth{Khujjasobhita }{Khujjasobhitattheragāthā}
\extramarks{Thag 3.6}{Thag 3.6}

\begin{verse}%
“One\marginnote{1.1} of those monks who live in \textsanskrit{Pāṭaliputta}—\\
such brilliant speakers, and very learned—\\
stands at the door: \\
the old man, Khujjasobhita. 

One\marginnote{2.1} of those monks who live in \textsanskrit{Pāṭaliputta}—\\
such brilliant speakers, and very learned—\\
stands at the door: \\
an old man, trembling in the gale.” 

“By\marginnote{3.1} war well fought, by sacrifice well made, \\
by victory in battle; \\
by leading the spiritual life: \\
that’s how one prospers in happiness.” 

%
\end{verse}

%
\section*{{\suttatitleacronym Thag 3.7}{\suttatitletranslation Vāraṇa }{\suttatitleroot Vāraṇattheragāthā}}
\addcontentsline{toc}{section}{\tocacronym{Thag 3.7} \toctranslation{Vāraṇa } \tocroot{Vāraṇattheragāthā}}
\markboth{Vāraṇa }{Vāraṇattheragāthā}
\extramarks{Thag 3.7}{Thag 3.7}

\begin{verse}%
Anyone\marginnote{1.1} among men \\
who harms other creatures: \\
that person will fall \\
both from this world and the next. 

But\marginnote{2.1} someone with a mind of love, \\
sympathetic for all creatures: \\
a person like that \\
creates much merit. 

One\marginnote{3.1} should train in following good advice, \\
in attending closely to ascetics, \\
in sitting alone in hidden places, \\
and in calming the mind. 

%
\end{verse}

%
\section*{{\suttatitleacronym Thag 3.8}{\suttatitletranslation Vassika }{\suttatitleroot Vassikattheragāthā}}
\addcontentsline{toc}{section}{\tocacronym{Thag 3.8} \toctranslation{Vassika } \tocroot{Vassikattheragāthā}}
\markboth{Vassika }{Vassikattheragāthā}
\extramarks{Thag 3.8}{Thag 3.8}

\begin{verse}%
I\marginnote{1.1} was the only one in my family \\
who had faith and wisdom. \\
It’s good for my relatives that I’m \\
firm in principle, and ethical. 

I\marginnote{2.1} corrected my family out of sympathy, \\
telling them off out of love \\
for my family and relatives. \\
They performed a service for the monks 

and\marginnote{3.1} then they passed away, \\
finding happiness in the heaven of the Thirty-three. \\
There, my brothers and mother \\
enjoy all the pleasures they desire. 

%
\end{verse}

%
\section*{{\suttatitleacronym Thag 3.9}{\suttatitletranslation Yasoja }{\suttatitleroot Yasojattheragāthā}}
\addcontentsline{toc}{section}{\tocacronym{Thag 3.9} \toctranslation{Yasoja } \tocroot{Yasojattheragāthā}}
\markboth{Yasoja }{Yasojattheragāthā}
\extramarks{Thag 3.9}{Thag 3.9}

\begin{verse}%
“With\marginnote{1.1} knobbly knees, \\
thin and veiny, \\
eating and drinking but little—\\
this person’s spirit is undaunted.” 

“Pestered\marginnote{2.1} by flies and mosquitoes \\
in the wilds, the formidable forest, \\
one should mindfully endure, \\
like an elephant at the head of the battle. 

A\marginnote{3.1} monk alone is like the supreme Divinity; \\
a pair of monks are like gods; \\
three are like a village; \\
and more than that is a rabble.” 

%
\end{verse}

%
\section*{{\suttatitleacronym Thag 3.10}{\suttatitletranslation Sāṭimattiya }{\suttatitleroot Sāṭimattiyattheragāthā}}
\addcontentsline{toc}{section}{\tocacronym{Thag 3.10} \toctranslation{Sāṭimattiya } \tocroot{Sāṭimattiyattheragāthā}}
\markboth{Sāṭimattiya }{Sāṭimattiyattheragāthā}
\extramarks{Thag 3.10}{Thag 3.10}

\begin{verse}%
In\marginnote{1.1} the past you had faith, \\
today you have none. \\
What’s yours is yours alone—\\
I’ve done nothing wrong. 

Faith\marginnote{2.1} is impermanent, fickle: \\
or so I have seen. \\
Passions wax and wane: \\
why would a sage waste away on that account? 

The\marginnote{3.1} meal of a sage is cooked \\
bit by bit in this family or that. \\
I’ll walk for alms, \\
for my legs are strong. 

%
\end{verse}

%
\section*{{\suttatitleacronym Thag 3.11}{\suttatitletranslation Upāli }{\suttatitleroot Upālittheragāthā}}
\addcontentsline{toc}{section}{\tocacronym{Thag 3.11} \toctranslation{Upāli } \tocroot{Upālittheragāthā}}
\markboth{Upāli }{Upālittheragāthā}
\extramarks{Thag 3.11}{Thag 3.11}

\begin{verse}%
One\marginnote{1.1} newly gone forth, \\
who has left their home out of faith, \\
should mix with spiritual friends, \\
who are tireless and pure of livelihood. 

One\marginnote{2.1} newly gone forth, \\
who has left their home out of faith, \\
a mendicant staying in the \textsanskrit{Saṅgha}, \\
being wise, would train in monastic discipline. 

One\marginnote{3.1} newly gone forth, \\
who has left their home out of faith, \\
skilled in what is appropriate and what is not, \\
would wander undistracted. 

%
\end{verse}

%
\section*{{\suttatitleacronym Thag 3.12}{\suttatitletranslation Uttarapāla }{\suttatitleroot Uttarapālattheragāthā}}
\addcontentsline{toc}{section}{\tocacronym{Thag 3.12} \toctranslation{Uttarapāla } \tocroot{Uttarapālattheragāthā}}
\markboth{Uttarapāla }{Uttarapālattheragāthā}
\extramarks{Thag 3.12}{Thag 3.12}

\begin{verse}%
I\marginnote{1.1} was, indeed, an astute scholar, \\
competent to think on the meaning. \\
The five kinds of sensual stimulation in the world, \\
so delusory, were my downfall. 

Leaping\marginnote{2.1} into \textsanskrit{Māra}’s domain, \\
I was struck by a powerful dart. \\
But I was able to free myself \\
from the trap laid by the King of Death. 

I\marginnote{3.1} have given up all sensual pleasures; \\
all rebirths are shattered; \\
transmigration through births is finished; \\
now there’ll be no more future lives. 

%
\end{verse}

%
\section*{{\suttatitleacronym Thag 3.13}{\suttatitletranslation Abhibhūta }{\suttatitleroot Abhibhūtattheragāthā}}
\addcontentsline{toc}{section}{\tocacronym{Thag 3.13} \toctranslation{Abhibhūta } \tocroot{Abhibhūtattheragāthā}}
\markboth{Abhibhūta }{Abhibhūtattheragāthā}
\extramarks{Thag 3.13}{Thag 3.13}

\begin{verse}%
Listen\marginnote{1.1} up, all my relatives, \\
those who have gathered here: \\
I’ll teach you Dhamma! \\
Painful is birth again and again. 

Rouse\marginnote{2.1} yourselves, try harder! \\
Devote yourselves to the instructions of the Buddha! \\
Crush the army of death, \\
as an elephant a hut of reeds. 

Whoever\marginnote{3.1} shall meditate diligently \\
in this teaching and training, \\
giving up transmigration, \\
will make an end to suffering. 

%
\end{verse}

%
\section*{{\suttatitleacronym Thag 3.14}{\suttatitletranslation Gotama (2nd) }{\suttatitleroot Gotamattheragāthā}}
\addcontentsline{toc}{section}{\tocacronym{Thag 3.14} \toctranslation{Gotama (2nd) } \tocroot{Gotamattheragāthā}}
\markboth{Gotama (2nd) }{Gotamattheragāthā}
\extramarks{Thag 3.14}{Thag 3.14}

\begin{verse}%
Transmigrating,\marginnote{1.1} I went to hell, \\
and to the ghost realm time and again. \\
Many times I dwelt long \\
in the animal realm, so full of pain. 

I\marginnote{2.1} was also reborn as a human, \\
and from time to time I went to heaven. \\
I’ve stayed in realms of form and formlessness, \\
among the neither-percipient-nor-non-percipient, \\>and the non-percipient. 

I\marginnote{3.1} know well these created states are worthless—\footnote{\textit{Sambhava} means “creation, production”. The text is bridging the gap between \textit{bhava} as the “states of existence” referred to above, and the “creation’ of those states, i.e. craving, referred to below. } \\
conditioned, unstable, always in motion. \\
When I understood that this is created in myself, \\
mindful, I found peace. 

%
\end{verse}

%
\section*{{\suttatitleacronym Thag 3.15}{\suttatitletranslation Hārita (2nd) }{\suttatitleroot Hāritattheragāthā}}
\addcontentsline{toc}{section}{\tocacronym{Thag 3.15} \toctranslation{Hārita (2nd) } \tocroot{Hāritattheragāthā}}
\markboth{Hārita (2nd) }{Hāritattheragāthā}
\extramarks{Thag 3.15}{Thag 3.15}

\begin{verse}%
Whoever\marginnote{1.1} wishes to do afterwards \\
tasks they should have done before \\
has lost the causes for happiness, \\
and afterwards they’re tormented by regrets. 

You\marginnote{2.1} should only say what you would do; \\
you shouldn’t say what you wouldn’t do. \\
The wise will recognize \\
one who talks without doing. 

Oh!\marginnote{3.1} Extinguishment is so very blissful, \\
as taught by the fully awakened Buddha: \\
sorrowless, stainless, secure, \\
where suffering all ceases. 

%
\end{verse}

%
\section*{{\suttatitleacronym Thag 3.16}{\suttatitletranslation Vimala (2nd) }{\suttatitleroot Vimalattheragāthā}}
\addcontentsline{toc}{section}{\tocacronym{Thag 3.16} \toctranslation{Vimala (2nd) } \tocroot{Vimalattheragāthā}}
\markboth{Vimala (2nd) }{Vimalattheragāthā}
\extramarks{Thag 3.16}{Thag 3.16}

\begin{verse}%
Shunning\marginnote{1.1} bad friends, \\
associate with the best of people. \\
Stick to the advice he gave you, \\
aspiring to unshakable happiness. 

If\marginnote{2.1} you’re lost in the middle of a great sea, \\
and you clamber up on a little log, you’ll sink. \\
So too, a person who lives well \\
sinks by relying on a lazy person. \\
Hence you should avoid such \\
a lazy person who lacks energy. 

Dwell\marginnote{3.1} with the noble ones \\
who are secluded and determined \\
and always energetic; \\
the astute who practice absorption. 

%
\end{verse}

%
\addtocontents{toc}{\let\protect\contentsline\protect\nopagecontentsline}
\part*{The Book of the Fours }
\addcontentsline{toc}{part}{The Book of the Fours }
\markboth{}{}
\addtocontents{toc}{\let\protect\contentsline\protect\oldcontentsline}

%
\addtocontents{toc}{\let\protect\contentsline\protect\nopagecontentsline}
\chapter*{Chapter One }
\addcontentsline{toc}{chapter}{\tocchapterline{Chapter One }}
\addtocontents{toc}{\let\protect\contentsline\protect\oldcontentsline}

%
\section*{{\suttatitleacronym Thag 4.1}{\suttatitletranslation Nāgasamāla }{\suttatitleroot Nāgasamālattheragāthā}}
\addcontentsline{toc}{section}{\tocacronym{Thag 4.1} \toctranslation{Nāgasamāla } \tocroot{Nāgasamālattheragāthā}}
\markboth{Nāgasamāla }{Nāgasamālattheragāthā}
\extramarks{Thag 4.1}{Thag 4.1}

\begin{verse}%
Adorned\marginnote{1.1} with jewelry and all dressed up, \\
with garlands, and sandalwood makeup piled on, \\
along the main street is a lady—\\
a dancer dancing as the music plays. 

I\marginnote{2.1} entered for alms, \\
and while walking along I glanced at her, \\
adorned with jewelry and all dressed up, \\
like a snare of death laid down. 

Then\marginnote{3.1} the realization \\
came upon me—\\
the danger became clear, \\
and I grew firmly disillusioned. 

Then\marginnote{4.1} my mind was freed—\\
see the excellence of the teaching! \\
I’ve attained the three knowledges \\
and fulfilled the Buddha’s instructions. 

%
\end{verse}

%
\section*{{\suttatitleacronym Thag 4.2}{\suttatitletranslation Bhagu }{\suttatitleroot Bhaguttheragāthā}}
\addcontentsline{toc}{section}{\tocacronym{Thag 4.2} \toctranslation{Bhagu } \tocroot{Bhaguttheragāthā}}
\markboth{Bhagu }{Bhaguttheragāthā}
\extramarks{Thag 4.2}{Thag 4.2}

\begin{verse}%
Overwhelmed\marginnote{1.1} by drowsiness, \\
I left my dwelling. \\
Stepping up to the path for walking meditation, \\
I fell to the ground right there. 

I\marginnote{2.1} rubbed my limbs, and again \\
I stepped up on the path for walking meditation. \\
I walked meditation up and down the path, \\
serene inside myself. 

Then\marginnote{3.1} the realization \\
came upon me—\\
the danger became clear, \\
and I grew firmly disillusioned. 

Then\marginnote{4.1} my mind was freed—\\
see the excellence of the teaching! \\
I’ve attained the three knowledges, \\
and fulfilled the Buddha’s instructions. 

%
\end{verse}

%
\section*{{\suttatitleacronym Thag 4.3}{\suttatitletranslation Sabhiya }{\suttatitleroot Sabhiyattheragāthā}}
\addcontentsline{toc}{section}{\tocacronym{Thag 4.3} \toctranslation{Sabhiya } \tocroot{Sabhiyattheragāthā}}
\markboth{Sabhiya }{Sabhiyattheragāthā}
\extramarks{Thag 4.3}{Thag 4.3}

\begin{verse}%
When\marginnote{1.1} others do not understand, \\
let us, who do understand this, \\
restrain ourselves in this regard; \\
for that is how conflicts are laid to rest. 

And\marginnote{2.1} when those who don’t understand \\
behave as though they were immortal, \\
those who understand the Dhamma \\
are like the healthy among the sick. 

Any\marginnote{3.1} lax act, \\
any corrupt observance, \\
or suspicious spiritual life, \\
is not very fruitful. 

Whoever\marginnote{4.1} has no respect \\
for their spiritual companions \\
is as far from true Dhamma \\
as the firmament from the earth. 

%
\end{verse}

%
\section*{{\suttatitleacronym Thag 4.4}{\suttatitletranslation Nandaka (2nd) }{\suttatitleroot Nandakattheragāthā}}
\addcontentsline{toc}{section}{\tocacronym{Thag 4.4} \toctranslation{Nandaka (2nd) } \tocroot{Nandakattheragāthā}}
\markboth{Nandaka (2nd) }{Nandakattheragāthā}
\extramarks{Thag 4.4}{Thag 4.4}

\begin{verse}%
Curse\marginnote{1.1} you mortal frame, you stink! \\
You’re on \textsanskrit{Māra}’s side, you fester! \\
O body, you have nine streams \\
that are flowing all the time. 

Don’t\marginnote{2.1} think much of mortal frames; \\
don’t disparage the Realized Ones. \\
They’re not even aroused by heaven, \\
let alone by humans. 

But\marginnote{3.1} those who are fools and simpletons, \\
with bad advisors, shrouded in delusion, \\
that kind of person is aroused by bodies, \\
when \textsanskrit{Māra} has laid down the snare. 

Those\marginnote{4.1} in whom greed, hate, and ignorance \\
have faded away; \\
such people are not aroused by bodies, \\
they’ve cut the strings, they’re no longer bound. 

%
\end{verse}

%
\section*{{\suttatitleacronym Thag 4.5}{\suttatitletranslation Jambuka }{\suttatitleroot Jambukattheragāthā}}
\addcontentsline{toc}{section}{\tocacronym{Thag 4.5} \toctranslation{Jambuka } \tocroot{Jambukattheragāthā}}
\markboth{Jambuka }{Jambukattheragāthā}
\extramarks{Thag 4.5}{Thag 4.5}

\begin{verse}%
For\marginnote{1.1} fifty-five years \\
I wore dust and dirt. \\
Eating one meal a month, \\
I tore out my hair and beard. 

I\marginnote{2.1} stood on one foot; \\
I rejected seats; \\
I ate dried-out dung; \\
I didn’t accept food set aside for me. 

I\marginnote{3.1} did many deeds of this kind, \\
which lead to a bad destination. \\
Swept away by a great flood, \\
I went to the Buddha for refuge. 

See\marginnote{4.1} the going for refuge! \\
See the excellence of the teaching! \\
I’ve attained the three knowledges \\
and fulfilled the Buddha’s instructions. 

%
\end{verse}

%
\section*{{\suttatitleacronym Thag 4.6}{\suttatitletranslation Senaka }{\suttatitleroot Senakattheragāthā}}
\addcontentsline{toc}{section}{\tocacronym{Thag 4.6} \toctranslation{Senaka } \tocroot{Senakattheragāthā}}
\markboth{Senaka }{Senakattheragāthā}
\extramarks{Thag 4.6}{Thag 4.6}

\begin{verse}%
It\marginnote{1.1} was so welcome for me \\
during the \textsanskrit{Gayā} spring festival \\
to see the Awakened One \\
teaching the supreme Dhamma. 

He\marginnote{2.1} was glorious, the tutor of a community, \\
a leader who had realized the highest. \\
In all the world with its gods, \\
he was the victor of unequaled vision. 

A\marginnote{3.1} great giant, a great hero, \\
a great light free of defilement. \\
With the utter ending of all defilements, \\
the teacher fears nothing from any quarter. 

For\marginnote{4.1} a long time, sadly, I was corrupted, \\
fettered by the bond of wrong view. \\
That Blessed One, Senaka, \\
released me from all ties. 

%
\end{verse}

%
\section*{{\suttatitleacronym Thag 4.7}{\suttatitletranslation Sambhūta }{\suttatitleroot Sambhūtattheragāthā}}
\addcontentsline{toc}{section}{\tocacronym{Thag 4.7} \toctranslation{Sambhūta } \tocroot{Sambhūtattheragāthā}}
\markboth{Sambhūta }{Sambhūtattheragāthā}
\extramarks{Thag 4.7}{Thag 4.7}

\begin{verse}%
Hurrying\marginnote{1.1} when it’s time to dawdle; \\
dawdling when it’s time to hurry; \\
being so disorganized \\
a fool falls into suffering. 

Their\marginnote{2.1} good fortune wastes away \\
like the moon in the waning fortnight. \\
They become disgraced \\
and alienated from their friends. 

Dawdling\marginnote{3.1} when it’s time to dawdle; \\
hurrying when it’s time to hurry; \\
being so well organized, \\
an astute person comes into happiness. 

Their\marginnote{4.1} good fortune flourishes \\
like the moon in the waxing fortnight. \\
They become famous and respected, \\
not alienated from their friends. 

%
\end{verse}

%
\section*{{\suttatitleacronym Thag 4.8}{\suttatitletranslation Rāhula }{\suttatitleroot Rāhulattheragāthā}}
\addcontentsline{toc}{section}{\tocacronym{Thag 4.8} \toctranslation{Rāhula } \tocroot{Rāhulattheragāthā}}
\markboth{Rāhula }{Rāhulattheragāthā}
\extramarks{Thag 4.8}{Thag 4.8}

\begin{verse}%
I\marginnote{1.1} am known as “Fortunate \textsanskrit{Rāhula}”, \\
because I’m accomplished in both ways: \\
I am the son of the Buddha, \\
whose eye sees clearly in all things. 

Since\marginnote{2.1} my defilements have ended, \\
since there are no more future lives—\\
I’m perfected, worthy of offerings, \\
master of the three knowledges, seer of freedom from death. 

Blinded\marginnote{3.1} by sensual pleasures, trapped in a net, \\
they are smothered over by craving; \\
bound by the kinsman of the negligent, \\
like a fish caught in a funnel-net trap. 

Having\marginnote{4.1} thrown off those sensual pleasures, \\
having cut \textsanskrit{Māra}’s bond, \\
and having plucked out craving, root and all: \\
I’m cooled, quenched. 

%
\end{verse}

%
\section*{{\suttatitleacronym Thag 4.9}{\suttatitletranslation Candana }{\suttatitleroot Candanattheragāthā}}
\addcontentsline{toc}{section}{\tocacronym{Thag 4.9} \toctranslation{Candana } \tocroot{Candanattheragāthā}}
\markboth{Candana }{Candanattheragāthā}
\extramarks{Thag 4.9}{Thag 4.9}

\begin{verse}%
Covered\marginnote{1.1} over with gold, \\
surrounded by all her maids, \\
with my son upon her hip, \\
my wife came to me. 

I\marginnote{2.1} saw her coming, \\
the mother of my son, \\
adorned with jewelry and all dressed up, \\
like a snare of death laid down. 

Then\marginnote{3.1} the realization \\
came upon me—\\
the danger became clear, \\
and I was firmly disillusioned. 

Then\marginnote{4.1} my mind was freed—\\
see the excellence of the teaching! \\
I’ve attained the three knowledges \\
and fulfilled the Buddha’s instructions. 

%
\end{verse}

%
\section*{{\suttatitleacronym Thag 4.10}{\suttatitletranslation Dhammika }{\suttatitleroot Dhammikattheragāthā}}
\addcontentsline{toc}{section}{\tocacronym{Thag 4.10} \toctranslation{Dhammika } \tocroot{Dhammikattheragāthā}}
\markboth{Dhammika }{Dhammikattheragāthā}
\extramarks{Thag 4.10}{Thag 4.10}

\begin{verse}%
“Dhamma\marginnote{1.1} surely protects \\>one who practices Dhamma; \\
the teaching brings happiness when practiced well. \\
This is the benefit of practicing Dhamma: \\
one doesn’t go to a bad destination. 

It’s\marginnote{2.1} not the case that Dhamma \\>and what is not Dhamma \\
lead to the same results. \\
What is not Dhamma leads to hell, \\
while Dhamma takes you to a good place. 

So\marginnote{3.1} you should rouse up enthusiasm for the teachings; \\
such rejoicing is owing to the Holy One, the unaffected. \\
The disciples of the best of Holy Ones \\>are firm in the teaching; \\
those attentive ones are led on, \\>headed to the very best of refuges.” 

“The\marginnote{4.1} boil has been burst from its root, \\
the net of craving is eradicated. \\
He has ended transmigration, he has nothing, \\
he’s like the full moon on a bright night.” 

%
\end{verse}

%
\section*{{\suttatitleacronym Thag 4.11}{\suttatitletranslation Sappaka }{\suttatitleroot Sappakattheragāthā}}
\addcontentsline{toc}{section}{\tocacronym{Thag 4.11} \toctranslation{Sappaka } \tocroot{Sappakattheragāthā}}
\markboth{Sappaka }{Sappakattheragāthā}
\extramarks{Thag 4.11}{Thag 4.11}

\begin{verse}%
When\marginnote{1.1} the crane with its beautiful white wings, \\
startled by fear of the dark thundercloud, \\
flees, seeking shelter—\\
then the River \textsanskrit{Ajakaraṇī} delights me. 

When\marginnote{2.1} the crane, so pure and white, \\
startled by fear of the dark thundercloud, \\
seeks a cave to shelter in, but can’t see one—\\
then the River \textsanskrit{Ajakaraṇī} delights me. 

Who\marginnote{3.1} wouldn’t be delighted \\
by the black plum trees \\
that adorn both banks of the river, \\
there, behind my cave? 

Rid\marginnote{4.1} of snakes, that death-mad swarm, \\
the lazy frogs croak: \\
“Today isn’t the time to stray from mountain streams; \\
\textsanskrit{Ajakaraṇī} is safe, pleasant, and delightful.” 

%
\end{verse}

%
\section*{{\suttatitleacronym Thag 4.12}{\suttatitletranslation Mudita }{\suttatitleroot Muditattheragāthā}}
\addcontentsline{toc}{section}{\tocacronym{Thag 4.12} \toctranslation{Mudita } \tocroot{Muditattheragāthā}}
\markboth{Mudita }{Muditattheragāthā}
\extramarks{Thag 4.12}{Thag 4.12}

\begin{verse}%
I\marginnote{1.1} went forth to save my life. \\
But I embraced faith \\
after receiving full ordination. \\
I strove, strong in effort: 

gladly,\marginnote{2.1} let this body be broken! \\
Let this scrap of meat disintegrate! \\
Let both my legs fall off \\
at the knees! 

I\marginnote{3.1} won’t eat, I won’t drink, \\
I won’t leave my dwelling, \\
nor will I lie down on my side, \\
until the dart of craving is plucked. 

As\marginnote{4.1} I meditate like this, \\
see my energy and vigor! \\
I’ve attained the three knowledges \\
and fulfilled the Buddha’s instructions. 

%
\end{verse}

%
\addtocontents{toc}{\let\protect\contentsline\protect\nopagecontentsline}
\part*{The Book of the Fives }
\addcontentsline{toc}{part}{The Book of the Fives }
\markboth{}{}
\addtocontents{toc}{\let\protect\contentsline\protect\oldcontentsline}

%
\addtocontents{toc}{\let\protect\contentsline\protect\nopagecontentsline}
\chapter*{Chapter One }
\addcontentsline{toc}{chapter}{\tocchapterline{Chapter One }}
\addtocontents{toc}{\let\protect\contentsline\protect\oldcontentsline}

%
\section*{{\suttatitleacronym Thag 5.1}{\suttatitletranslation Rājadatta }{\suttatitleroot Rājadattattheragāthā}}
\addcontentsline{toc}{section}{\tocacronym{Thag 5.1} \toctranslation{Rājadatta } \tocroot{Rājadattattheragāthā}}
\markboth{Rājadatta }{Rājadattattheragāthā}
\extramarks{Thag 5.1}{Thag 5.1}

\begin{verse}%
I,\marginnote{1.1} a monk, went to a charnel ground \\
and saw a woman’s body abandoned there, \\
discarded in a cemetery, \\
full of worms that devoured. 

Some\marginnote{2.1} men were disgusted, \\
seeing her dead and rotten; \\
but sexual desire arose in me, \\
I was as if blind to her oozing body. 

Quicker\marginnote{3.1} than the cooking of rice \\
I left that place! \\
Mindful and aware, \\
I retired to a discreet place. 

Then\marginnote{4.1} the realization \\
came upon me—\\
the danger became clear, \\
and I was firmly disillusioned. 

Then\marginnote{5.1} my mind was freed—\\
see the excellence of the teaching! \\
I’ve attained the three knowledges \\
and fulfilled the Buddha’s instructions. 

%
\end{verse}

%
\section*{{\suttatitleacronym Thag 5.2}{\suttatitletranslation Subhūta }{\suttatitleroot Subhūtattheragāthā}}
\addcontentsline{toc}{section}{\tocacronym{Thag 5.2} \toctranslation{Subhūta } \tocroot{Subhūtattheragāthā}}
\markboth{Subhūta }{Subhūtattheragāthā}
\extramarks{Thag 5.2}{Thag 5.2}

\begin{verse}%
When\marginnote{1.1} a person, wishing for a certain outcome, \\
applies themselves where they ought not; \\
not achieving what they worked for, \\
they say: “That’s a sign of my bad luck.” 

When\marginnote{2.1} a misfortune is extracted and beaten, \\
to surrender it in part would be like losing at dice; \\
but to surrender it all you’d have to be blind, \\
not seeing the even and the uneven. 

You\marginnote{3.1} should only say what you would do; \\
you shouldn’t say what you wouldn’t do. \\
The wise will recognize \\
one who talks without doing. 

Just\marginnote{4.1} like a glorious flower \\
that’s colorful but lacks fragrance; \\
well-spoken speech is fruitless \\
for one who does not act on it. 

Just\marginnote{5.1} like a glorious flower \\
that's both colorful and fragrant, \\
well-spoken speech is fruitful \\
for one who acts on it. 

%
\end{verse}

%
\section*{{\suttatitleacronym Thag 5.3}{\suttatitletranslation Girimānanda }{\suttatitleroot Girimānandattheragāthā}}
\addcontentsline{toc}{section}{\tocacronym{Thag 5.3} \toctranslation{Girimānanda } \tocroot{Girimānandattheragāthā}}
\markboth{Girimānanda }{Girimānandattheragāthā}
\extramarks{Thag 5.3}{Thag 5.3}

\begin{verse}%
The\marginnote{1.1} heavens rain like a sweet song; \\
my little hut is roofed and pleasant, \\>sheltered from the wind; \\
I meditate there, peaceful: \\
so rain forth, heavens, if you wish. 

The\marginnote{2.1} heavens rain like a sweet song; \\
my little hut is roofed and pleasant, \\>sheltered from the wind; \\
I meditate there, my mind at peace: \\
so rain forth, heavens, if you wish. 

The\marginnote{3.1} heavens rain like a sweet song; \\
my little hut is roofed and pleasant, \\>sheltered from the wind; \\
I meditate there, free of greed: \\
so rain forth, heavens, if you wish. 

The\marginnote{4.1} heavens rain like a sweet song; \\
my little hut is roofed and pleasant, \\>sheltered from the wind; \\
I meditate there, free of hate: \\
so rain forth, heavens, if you wish. 

The\marginnote{5.1} heavens rain like a sweet song; \\
my little hut is roofed and pleasant, \\>sheltered from the wind; \\
I meditate there, free of delusion: \\
so rain forth, heavens, if you wish. 

%
\end{verse}

%
\section*{{\suttatitleacronym Thag 5.4}{\suttatitletranslation Sumana (1st) }{\suttatitleroot Sumanattheragāthā}}
\addcontentsline{toc}{section}{\tocacronym{Thag 5.4} \toctranslation{Sumana (1st) } \tocroot{Sumanattheragāthā}}
\markboth{Sumana (1st) }{Sumanattheragāthā}
\extramarks{Thag 5.4}{Thag 5.4}

\begin{verse}%
My\marginnote{1.1} mentor helped me to learn, \\
hoping I would practice those teachings. \\
Aspiring to freedom from death, \\
I’ve done what had to be done. 

I’ve\marginnote{2.1} realized the Dhamma, \\
witnessing it for myself, not based on hearsay. \\
With purified knowledge, free of doubt, \\
I declare it in your presence. 

I\marginnote{3.1} know my past lives, \\
my clairvoyance is purified, \\
I’ve realized my own true goal, \\
and fulfilled the Buddha’s instructions. 

Being\marginnote{4.1} diligent in the training, \\
I learned your teachings well. \\
All my defilements are ended; \\
now there’ll be no more future lives. 

You\marginnote{5.1} advised me in noble observances; \\
sympathetic, you helped teach me. \\
Your instruction was not in vain—\\
I, your pupil, am fully trained. 

%
\end{verse}

%
\section*{{\suttatitleacronym Thag 5.5}{\suttatitletranslation Vaḍḍha }{\suttatitleroot Vaḍḍhattheragāthā}}
\addcontentsline{toc}{section}{\tocacronym{Thag 5.5} \toctranslation{Vaḍḍha } \tocroot{Vaḍḍhattheragāthā}}
\markboth{Vaḍḍha }{Vaḍḍhattheragāthā}
\extramarks{Thag 5.5}{Thag 5.5}

\begin{verse}%
Oh\marginnote{1.1} so well was the goad\footnote{These verses continue from \href{https://suttacentral.net/thig9.1/en/sujato}{Thig 9.1} so I use only close quote. } \\
shown to me by my mother, \\
on hearing whose words, \\
advised by my mother, \\
energetic and resolute, \\
I realized supreme awakening. 

I’m\marginnote{2.1} perfected, worthy of offerings, \\
master of the three knowledges, seer of freedom from death. \\
I’ve conquered the army of Namuci, \\
and live without defilements. 

Those\marginnote{3.1} defilements that were found in me, \\
internally and externally, \\
are all cut off without remainder, \\
and will not arise again. 

My\marginnote{4.1} self-assured sister \\
said this to me: \\
‘Now neither you nor I \\
have any entanglements.’ 

Suffering\marginnote{5.1} is at an end; \\
this bag of bones is my last \\
in the transmigration through births and deaths; \\
now there’ll be no more future lives.” 

%
\end{verse}

%
\section*{{\suttatitleacronym Thag 5.6}{\suttatitletranslation Nadīkassapa }{\suttatitleroot Nadīkassapattheragāthā}}
\addcontentsline{toc}{section}{\tocacronym{Thag 5.6} \toctranslation{Nadīkassapa } \tocroot{Nadīkassapattheragāthā}}
\markboth{Nadīkassapa }{Nadīkassapattheragāthā}
\extramarks{Thag 5.6}{Thag 5.6}

\begin{verse}%
It\marginnote{1.1} was truly for my benefit \\
that the Buddha went to the river \textsanskrit{Nerañjara}. \\
When I heard his teaching, \\
I shunned wrong view. 

I\marginnote{2.1} used to perform a diverse spectrum of sacrifices; \\
I served the sacred flame, \\
imagining, “This is purity.” \\
I was a blind, ordinary person. 

Caught\marginnote{3.1} in the thicket of wrong view, \\
deluded by misapprehension. \\
Thinking impurity was purity, \\
I was blind and ignorant. 

I’ve\marginnote{4.1} abandoned wrong view; \\
all rebirths are shattered. \\
I serve the fire for those worthy of a religious donation: \\
I bow to the Realized One. 

I’ve\marginnote{5.1} given up all delusion; \\
craving for continued existence is shattered; \\
transmigration through births is finished; \\
now there’ll be no more future lives. 

%
\end{verse}

%
\section*{{\suttatitleacronym Thag 5.7}{\suttatitletranslation Gayākassapa }{\suttatitleroot Gayākassapattheragāthā}}
\addcontentsline{toc}{section}{\tocacronym{Thag 5.7} \toctranslation{Gayākassapa } \tocroot{Gayākassapattheragāthā}}
\markboth{Gayākassapa }{Gayākassapattheragāthā}
\extramarks{Thag 5.7}{Thag 5.7}

\begin{verse}%
Three\marginnote{1.1} times a day—\\
morning, midday, and evening—\\
I plunged into the water at \textsanskrit{Gayā} \\
for the \textsanskrit{Gayā} spring festival. 

“Any\marginnote{2.1} bad things I’ve done \\
in previous lives, \\
I’ll now wash away right here”—\\
such was the view I used to hold. 

Having\marginnote{3.1} heard the fine words, \\
a passage meaningful and principled, \\
I rationally reflected \\
on the true, essential goal. 

I’ve\marginnote{4.1} washed away all bad things; \\
I’m stainless, clean, pristine; \\
the pure heir of the pure one, \\
a true-born son of the Buddha. 

When\marginnote{5.1} I plunged into the eightfold stream, \\
all bad things were washed away. \\
I’ve attained the three knowledges \\
and fulfilled the Buddha’s instructions. 

%
\end{verse}

%
\section*{{\suttatitleacronym Thag 5.8}{\suttatitletranslation Vakkali }{\suttatitleroot Vakkalittheragāthā}}
\addcontentsline{toc}{section}{\tocacronym{Thag 5.8} \toctranslation{Vakkali } \tocroot{Vakkalittheragāthā}}
\markboth{Vakkali }{Vakkalittheragāthā}
\extramarks{Thag 5.8}{Thag 5.8}

\begin{verse}%
“Struck\marginnote{1.1} by a wind ailment \\
while dwelling in a forest grove, \\
you’ve entered a tough place for gathering alms—\\
how will you get by, monk?” 

“Pervading\marginnote{2.1} this bag of bones \\
with abundant rapture and happiness, \\
putting up with what’s tough, \\
I’ll dwell in the forest. 

Developing\marginnote{3.1} mindfulness meditation, \\
the faculties and the powers, \\
developing the factors of awakening, \\
I’ll dwell in the forest. 

Having\marginnote{4.1} seen those who are energetic, resolute, \\
always staunchly vigorous, \\
harmonious and united, \\
I’ll dwell in the forest. 

Recollecting\marginnote{5.1} the Buddha—\\
the best, the tamed, the serene—\\
tireless all day and night \\
I’ll dwell in the forest.” 

%
\end{verse}

%
\section*{{\suttatitleacronym Thag 5.9}{\suttatitletranslation Vijitasena }{\suttatitleroot Vijitasenattheragāthā}}
\addcontentsline{toc}{section}{\tocacronym{Thag 5.9} \toctranslation{Vijitasena } \tocroot{Vijitasenattheragāthā}}
\markboth{Vijitasena }{Vijitasenattheragāthā}
\extramarks{Thag 5.9}{Thag 5.9}

\begin{verse}%
I’ll\marginnote{1.1} cage you, mind, \\
like an elephant in a stockade. \\
Born of the flesh, that net of the senses, \\
I won’t urge you to do bad. 

Caged,\marginnote{2.1} you won’t go anywhere, \\
like an elephant who can’t find an open gate. \\
Loser mind, you won’t wander again and again, \\
bullying, in love with wickedness. 

Just\marginnote{3.1} as a strong trainer with a hook \\
takes a wild, newly captured elephant \\
and wins it over against its will, \\
so I’ll win you over. 

Just\marginnote{4.1} as a fine charioteer, skilled in the taming \\
of fine horses, tames a thoroughbred, \\
so I’ll tame you, \\
firmly established in the five powers. 

I’ll\marginnote{5.1} bind you with mindfulness; \\
devout, I shall tame you; \\
kept in check by harnessed energy, \\
mind, you won’t go far from here. 

%
\end{verse}

%
\section*{{\suttatitleacronym Thag 5.10}{\suttatitletranslation Yasadatta }{\suttatitleroot Yasadattattheragāthā}}
\addcontentsline{toc}{section}{\tocacronym{Thag 5.10} \toctranslation{Yasadatta } \tocroot{Yasadattattheragāthā}}
\markboth{Yasadatta }{Yasadattattheragāthā}
\extramarks{Thag 5.10}{Thag 5.10}

\begin{verse}%
With\marginnote{1.1} fault-finding mind, the simpleton \\
listens to the victor’s instruction. \\
They’re as far from the true teaching \\
as the earth is from the sky. 

With\marginnote{2.1} fault-finding mind, the simpleton \\
listens to the victor’s instruction. \\
They fall away from the true teaching, \\
like the moon in the waning fortnight. 

With\marginnote{3.1} fault-finding mind, the simpleton \\
listens to the victor’s instruction. \\
They wither away in the true teaching, \\
like a fish in a little puddle. 

With\marginnote{4.1} fault-finding mind, the simpleton \\
listens to the victor’s instruction. \\
They don’t thrive in the true teaching, \\
like a rotten seed in a field. 

But\marginnote{5.1} one with contented mind \\
who listens to the victor’s instruction—\\
having wiped out all defilements; \\
having witnessed the unshakable; \\
having arrived at ultimate peace—\\
the undefiled are fully extinguished.” 

%
\end{verse}

%
\section*{{\suttatitleacronym Thag 5.11}{\suttatitletranslation Soṇa of the Sharp Ears }{\suttatitleroot Soṇakuṭikaṇṇattheragāthā}}
\addcontentsline{toc}{section}{\tocacronym{Thag 5.11} \toctranslation{Soṇa of the Sharp Ears } \tocroot{Soṇakuṭikaṇṇattheragāthā}}
\markboth{Soṇa of the Sharp Ears }{Soṇakuṭikaṇṇattheragāthā}
\extramarks{Thag 5.11}{Thag 5.11}

\begin{verse}%
I’ve\marginnote{1.1} received ordination; \\
I am liberated, without defilements; \\
I’ve seen the Blessed One myself, \\
and even stayed together with him. 

The\marginnote{2.1} Blessed One, the teacher, \\
spent much of the night in the open; \\
then he, who is so skilled in meditation, \\
entered his dwelling. 

Spreading\marginnote{3.1} his outer robe, \\
Gotama made his bed, \\
like a lion in a rocky cave, \\
with fear and dread given up. 

Then,\marginnote{4.1} with lovely enunciation, \\
\textsanskrit{Soṇa}, a disciple of the Buddha, \\
recited the true teaching \\
before the best of Buddhas. 

When\marginnote{5.1} he has completely understood \\>the five aggregates, \\
developed the direct route, \\
and arrived at ultimate peace, \\
undefiled, he’ll be fully quenched. 

%
\end{verse}

%
\section*{{\suttatitleacronym Thag 5.12}{\suttatitletranslation Kosiya }{\suttatitleroot Kosiyattheragāthā}}
\addcontentsline{toc}{section}{\tocacronym{Thag 5.12} \toctranslation{Kosiya } \tocroot{Kosiyattheragāthā}}
\markboth{Kosiya }{Kosiyattheragāthā}
\extramarks{Thag 5.12}{Thag 5.12}

\begin{verse}%
Whatever\marginnote{1.1} attentive one, understanding their teacher’s words, \\
stays with them, their fondness growing; \\
that astute person is indeed devoted—\\
knowing the teachings, they’re distinguished. 

When\marginnote{2.1} extreme stresses arise, \\
one who does not tremble, but reflects instead, \\
that astute person is indeed strong—\\
knowing the teachings, they’re distinguished. 

Steady\marginnote{3.1} as the ocean, imperturbable, \\
their wisdom is deep, they see the subtle meaning; \\
that astute person is indeed unfaltering—\\
knowing the teachings, they’re distinguished. 

They’re\marginnote{4.1} very learned, \\>and have memorized the teaching, \\
living in line with the teachings—\\
that astute person is indeed such—\\
knowing the teachings, they’re distinguished. 

They\marginnote{5.1} know the meaning of what is said, \\
and act accordingly; \\
that astute person is indeed a master of meaning—\\
knowing the teachings, they’re distinguished. 

%
\end{verse}

%
\addtocontents{toc}{\let\protect\contentsline\protect\nopagecontentsline}
\part*{The Book of the Sixes }
\addcontentsline{toc}{part}{The Book of the Sixes }
\markboth{}{}
\addtocontents{toc}{\let\protect\contentsline\protect\oldcontentsline}

%
\addtocontents{toc}{\let\protect\contentsline\protect\nopagecontentsline}
\chapter*{Chapter One }
\addcontentsline{toc}{chapter}{\tocchapterline{Chapter One }}
\addtocontents{toc}{\let\protect\contentsline\protect\oldcontentsline}

%
\section*{{\suttatitleacronym Thag 6.1}{\suttatitletranslation Uruveḷakassapa }{\suttatitleroot Uruveḷakassapattheragāthā}}
\addcontentsline{toc}{section}{\tocacronym{Thag 6.1} \toctranslation{Uruveḷakassapa } \tocroot{Uruveḷakassapattheragāthā}}
\markboth{Uruveḷakassapa }{Uruveḷakassapattheragāthā}
\extramarks{Thag 6.1}{Thag 6.1}

\begin{verse}%
Seeing\marginnote{1.1} the demonstrations \\
of the renowned Gotama \\
was not enough for me to bow to him—\\
I was blinded by jealousy and conceit. 

Knowing\marginnote{2.1} my thoughts, \\
the trainer of men scolded me. \\
I was struck with a sense of urgency, \\
so astonishing and hair-raising! 

Rejecting\marginnote{3.1} the petty powers \\
I had before as a matted-hair ascetic, \\
I then went forth \\
in the victor’s instruction. 

I\marginnote{4.1} used to be content with sacrifice, \\
the realm of sensual pleasures was my priority. \\
But later I eradicated desire, \\
and hatred and also delusion. 

I\marginnote{5.1} know my past lives; \\
my clairvoyance is clarified; \\
I have psychic powers, and know the minds of others; \\
I have attained clairaudience. 

I’ve\marginnote{6.1} attained the goal \\
for the sake of which I went forth \\
from the lay life to homelessness—\\
the ending of all fetters. 

%
\end{verse}

%
\section*{{\suttatitleacronym Thag 6.2}{\suttatitletranslation Tekicchakāri }{\suttatitleroot Tekicchakārittheragāthā}}
\addcontentsline{toc}{section}{\tocacronym{Thag 6.2} \toctranslation{Tekicchakāri } \tocroot{Tekicchakārittheragāthā}}
\markboth{Tekicchakāri }{Tekicchakārittheragāthā}
\extramarks{Thag 6.2}{Thag 6.2}

\begin{verse}%
“The\marginnote{1.1} rice has been harvested \\
and gathered on the threshing-floor—\\
but I don’t get any almsfood! \\
How will I get by?” 

“In\marginnote{2.1} faith, recollect the immeasurable Buddha! \\
Your body soaked with rapture, \\>you’ll always be full of joy. 

In\marginnote{3.1} faith, recollect the immeasurable teaching! \\
Your body soaked with rapture, \\>you’ll always be full of joy. 

In\marginnote{4.1} faith, recollect the immeasurable \textsanskrit{Saṅgha}! \\
Your body soaked with rapture, \\>you’ll always be full of joy.” 

“You\marginnote{5.1} stay in the open, \\
though these winter nights are cold. \\
Don’t perish, overcome with cold; \\
enter your dwelling, with door shut fast.” 

“I’ll\marginnote{6.1} realize the four immeasurable states, \\
and meditate happily in them. \\
I won’t perish, overcome with cold; \\
I’ll dwell unperturbed.” 

%
\end{verse}

%
\section*{{\suttatitleacronym Thag 6.3}{\suttatitletranslation Mahānāga }{\suttatitleroot Mahānāgattheragāthā}}
\addcontentsline{toc}{section}{\tocacronym{Thag 6.3} \toctranslation{Mahānāga } \tocroot{Mahānāgattheragāthā}}
\markboth{Mahānāga }{Mahānāgattheragāthā}
\extramarks{Thag 6.3}{Thag 6.3}

\begin{verse}%
Whoever\marginnote{1.1} has no respect \\
for their spiritual companions \\
falls away from the true teaching, \\
like a fish in a little puddle. 

Whoever\marginnote{2.1} has no respect \\
for their spiritual companions \\
doesn’t thrive in the true teaching, \\
like a rotten seed in a field. 

Whoever\marginnote{3.1} has no respect \\
for their spiritual companions \\
is far from extinguishment, \\
in the teaching of the Dhamma king. 

Whoever\marginnote{4.1} does have respect \\
for their spiritual companions \\
doesn’t fall away from the true teaching, \\
like a fish in plenty of water. 

Whoever\marginnote{5.1} does have respect \\
for their spiritual companions \\
thrives in the true teaching, \\
like a fine seed in a field. 

Whoever\marginnote{6.1} does have respect \\
for their spiritual companions \\
is close to extinguishment \\
in the teaching of the Dhamma king. 

%
\end{verse}

%
\section*{{\suttatitleacronym Thag 6.4}{\suttatitletranslation Kulla }{\suttatitleroot Kullattheragāthā}}
\addcontentsline{toc}{section}{\tocacronym{Thag 6.4} \toctranslation{Kulla } \tocroot{Kullattheragāthā}}
\markboth{Kulla }{Kullattheragāthā}
\extramarks{Thag 6.4}{Thag 6.4}

\begin{verse}%
I,\marginnote{1.1} Kulla, went to a charnel ground \\
and saw a woman’s body abandoned there, \\
discarded in a cemetery, \\
full of worms that devoured. 

“See\marginnote{2.1} this bag of bones, Kulla—\\
diseased, filthy, rotten, \\
oozing and trickling, \\
a fool’s delight.” 

Taking\marginnote{3.1} the teaching as a mirror \\
for realizing knowledge and vision, \\
I examined this body, \\
hollow, inside and out. 

As\marginnote{4.1} this is, so is that; \\
as that is, so is this. \\
As below, so above; \\
as above, so below. 

As\marginnote{5.1} by day, so by night; \\
as by night, so by day. \\
As before, so behind; \\
as behind, so before. 

Even\marginnote{6.1} the music of a five-piece band \\
can never give such pleasure \\
as when, with unified mind, \\
you rightly discern the Dhamma. 

%
\end{verse}

%
\section*{{\suttatitleacronym Thag 6.5}{\suttatitletranslation Māluṅkyaputta (1st) }{\suttatitleroot Mālukyaputtattheragāthā}}
\addcontentsline{toc}{section}{\tocacronym{Thag 6.5} \toctranslation{Māluṅkyaputta (1st) } \tocroot{Mālukyaputtattheragāthā}}
\markboth{Māluṅkyaputta (1st) }{Mālukyaputtattheragāthā}
\extramarks{Thag 6.5}{Thag 6.5}

\begin{verse}%
When\marginnote{1.1} a person lives heedlessly, \\
craving grows in them like a camel’s foot creeper. \\
They jump from life to life, like a langur \\
greedy for fruit in a forest grove.\footnote{The \textit{\textsanskrit{vānara}} is a type of human-like monkey often identified with the langur. See \href{https://suttacentral.net/ja57/en/sujato}{Ja 57}. } 

Whoever\marginnote{2.1} is beaten by this wretched craving, \\
this attachment to the world, \\
their sorrow grows, \\
like grass in the rain. 

But\marginnote{3.1} whoever prevails over this wretched craving, \\
so hard to get over in the world, \\
their sorrows fall from them, \\
like a drop from a lotus-leaf. 

I\marginnote{4.1} say this to you, good people, \\
all those who have gathered here: \\
dig up the root of craving, \\
as you’d dig up the grass in search of roots. \\
Don’t let \textsanskrit{Māra} break you again and again, \\
like a stream breaking a reed. 

Act\marginnote{5.1} on the Buddha’s words, \\
don’t let the moment pass you by. \\
For if you miss your moment \\
you’ll grieve when sent to hell. 

Negligence\marginnote{6.1} is always dust; \\
dust follows right behind negligence. \\
Through diligence and knowledge, \\
pluck out the dart from yourself. 

%
\end{verse}

%
\section*{{\suttatitleacronym Thag 6.6}{\suttatitletranslation Sappadāsa }{\suttatitleroot Sappadāsattheragāthā}}
\addcontentsline{toc}{section}{\tocacronym{Thag 6.6} \toctranslation{Sappadāsa } \tocroot{Sappadāsattheragāthā}}
\markboth{Sappadāsa }{Sappadāsattheragāthā}
\extramarks{Thag 6.6}{Thag 6.6}

\begin{verse}%
In\marginnote{1.1} the twenty-five years \\
since I went forth, \\
I have not found peace of mind, \\
even as long as a finger-snap. 

Since\marginnote{2.1} I couldn’t get my mind unified, \\
I was racked by desire for pleasures of the senses. \\
Wailing, with outstretched arms, \\
I left my dwelling. 

Shall\marginnote{3.1} I … or shall I take my life? \\
What’s the point of living? \\
For how on earth can one such as me die \\
after resigning the training? 

Then\marginnote{4.1} I picked up a razor, \\
I sat on a cot: \\
the razor was ready \\
to slice my vein. 

Then\marginnote{5.1} the realization \\
came upon me—\\
the danger became clear, \\
and I was firmly disillusioned. 

Then\marginnote{6.1} my mind was freed—\\
see the excellence of the teaching! \\
I’ve attained the three knowledges, \\
and fulfilled the Buddha’s instructions. 

%
\end{verse}

%
\section*{{\suttatitleacronym Thag 6.7}{\suttatitletranslation Kātiyāna }{\suttatitleroot Kātiyānattheragāthā}}
\addcontentsline{toc}{section}{\tocacronym{Thag 6.7} \toctranslation{Kātiyāna } \tocroot{Kātiyānattheragāthā}}
\markboth{Kātiyāna }{Kātiyānattheragāthā}
\extramarks{Thag 6.7}{Thag 6.7}

\begin{verse}%
Get\marginnote{1.1} up, \textsanskrit{Kātiyāna}, and sit! \\
Don’t sleep too much, be wakeful. \\
Don’t be lazy and let the kinsman of the negligent, \\
the King of Death, catch you in his trap. 

Like\marginnote{2.1} a wave in the mighty ocean, \\
rebirth and old age sweep you under. \\
Make a safe island of yourself, \\
for you have no other shelter. 

The\marginnote{3.1} teacher has mastered this path, \\
which transcends chains, \\>and the fear of birth and old age. \\
Be diligent in the first and final watches of the night, \\
and dedicate yourself to practice. 

Free\marginnote{4.1} yourself from your former bonds! \\
Wearing your outer robe, with shaven head, \\>eating almsfood, \\
don’t delight in play or sleep, \\
dedicate yourself to absorption, \textsanskrit{Kātiyāna}. 

Meditate\marginnote{5.1} and conquer, \textsanskrit{Kātiyāna}, \\
you’re an expert in the path to sanctuary  from the yoke. \\
Attaining unexcelled purity, \\
you’ll be extinguished as a flame by water. 

A\marginnote{6.1} lamp of feeble flame \\
is bent down by the wind, like a creeper; \\
just so, kinsman of Indra, \\
shake off \textsanskrit{Māra}, without grasping. \\
Free of lust for feelings, \\
await your time here, cooled. 

%
\end{verse}

%
\section*{{\suttatitleacronym Thag 6.8}{\suttatitletranslation Migajāla }{\suttatitleroot Migajālattheragāthā}}
\addcontentsline{toc}{section}{\tocacronym{Thag 6.8} \toctranslation{Migajāla } \tocroot{Migajālattheragāthā}}
\markboth{Migajāla }{Migajālattheragāthā}
\extramarks{Thag 6.8}{Thag 6.8}

\begin{verse}%
It\marginnote{1.1} was well-taught by the Clear-eyed One, \\
the Buddha, kinsman of the Sun, \\
who has transcended all fetters, \\
and destroyed all rolling-on. 

Emancipating,\marginnote{2.1} it leads across, \\
drying up the root of craving, \\
and, having cut off the poisonous root, \\>the slaughterhouse, \\
it leads to quenching. 

By\marginnote{3.1} breaking the root of unknowing, \\
it smashes the mechanism of deeds, \\
and drops the thunderbolt of knowledge \\
on the taking up of consciousnesses. 

It\marginnote{4.1} informs us of our feelings, \\
releasing us from grasping, \\
contemplating with understanding \\
all states of existence as a pit of burning coals. 

It’s\marginnote{5.1} very sweet and very deep, \\
holding birth and death at bay; \\
it is the noble eightfold path—\\
the stilling of suffering, bliss. 

Knowing\marginnote{6.1} deed as deed \\
and result as result; \\
seeing dependently originated phenomena \\
as if they were in a clear light; \\
leading to the great sanctuary and peace, \\
it’s excellent at the end. 

%
\end{verse}

%
\section*{{\suttatitleacronym Thag 6.9}{\suttatitletranslation Jenta, the High Priest’s Son }{\suttatitleroot Purohitaputtajentattheragāthā}}
\addcontentsline{toc}{section}{\tocacronym{Thag 6.9} \toctranslation{Jenta, the High Priest’s Son } \tocroot{Purohitaputtajentattheragāthā}}
\markboth{Jenta, the High Priest’s Son }{Purohitaputtajentattheragāthā}
\extramarks{Thag 6.9}{Thag 6.9}

\begin{verse}%
I\marginnote{1.1} was drunk with the pride of birth \\
and wealth and authority. \\
I wandered about intoxicated \\
with my own gorgeous body. 

No-one\marginnote{2.1} was my equal or my better—\\
or so I thought. \\
I was such an arrogant fool, \\
stuck up, waving my own flag. 

I\marginnote{3.1} never paid homage to anyone: \\
not even my mother or father, \\
nor others esteemed as respectable. \\
I was stiff with pride, lacking regard for others. 

When\marginnote{4.1} I saw the foremost leader, \\
the most excellent of charioteers, \\
shining like the sun, \\
at the fore of the mendicant \textsanskrit{Saṅgha}, 

I\marginnote{5.1} discarded conceit and vanity, \\
and, with a clear and confident heart, \\
I bowed down with my head \\
to the most excellent of all beings. 

The\marginnote{6.1} conceit of superiority \\>and the conceit of inferiority \\
have been given up and eradicated. \\
The conceit “I am” is cut off, \\
and every kind of conceit is destroyed. 

%
\end{verse}

%
\section*{{\suttatitleacronym Thag 6.10}{\suttatitletranslation Sumana (2nd) }{\suttatitleroot Sumanattheragāthā}}
\addcontentsline{toc}{section}{\tocacronym{Thag 6.10} \toctranslation{Sumana (2nd) } \tocroot{Sumanattheragāthā}}
\markboth{Sumana (2nd) }{Sumanattheragāthā}
\extramarks{Thag 6.10}{Thag 6.10}

\begin{verse}%
I\marginnote{1.1} was only seven years old \\
and had just gone forth \\
when I overcame the mighty serpent king \\
with my psychic powers. 

I\marginnote{2.1} brought water for my mentor \\
from the great lake Anotatta. \\
When he saw me, \\
my teacher declared: 

“\textsanskrit{Sāriputta},\marginnote{3.1} see this \\
young boy coming, \\
carrying a water pot, \\
serene inside himself. 

His\marginnote{4.1} conduct inspires confidence, \\
he is of lovely deportment: \\
he is Anuruddha’s novice, \\
assured in psychic powers. 

Made\marginnote{5.1} a thoroughbred by a thoroughbred, \\
made good by the good, \\
educated and trained by Anuruddha, \\
who has completed his task. 

Having\marginnote{6.1} attained ultimate peace \\
and witnessed the unshakable, \\
that novice Sumana has the wish: \\
‘May no-one find me out!’” 

%
\end{verse}

%
\section*{{\suttatitleacronym Thag 6.11}{\suttatitletranslation Nhātakamuni }{\suttatitleroot Nhātakamunittheragāthā}}
\addcontentsline{toc}{section}{\tocacronym{Thag 6.11} \toctranslation{Nhātakamuni } \tocroot{Nhātakamunittheragāthā}}
\markboth{Nhātakamuni }{Nhātakamunittheragāthā}
\extramarks{Thag 6.11}{Thag 6.11}

\begin{verse}%
“Struck\marginnote{1.1} by a wind ailment \\
while dwelling in a forest grove, \\
you’ve entered a tough resort for gathering alms—\\
how will you get by, monk?” 

“Pervading\marginnote{2.1} this bag of bones \\
with abundant rapture and happiness, \\
putting up with what’s tough, \\
I’ll dwell in the forest. 

Developing\marginnote{3.1} the seven awakening factors, \\
the faculties and the powers, \\
endowed with subtle absorptions, \\
I’ll dwell without defilements. 

Freed\marginnote{4.1} from corruptions, \\
my pure mind is unclouded. \\
Frequently reviewing this, \\
I’ll meditate without defilements. 

Those\marginnote{5.1} defilements that were found in me, \\
internally and externally, \\
are all cut off without remainder, \\
and will not arise again. 

The\marginnote{6.1} five aggregates are fully understood, \\
they remain, but their root is cut. \\
I have reached the ending of suffering, \\
now there’ll be no more future lives.” 

%
\end{verse}

%
\section*{{\suttatitleacronym Thag 6.12}{\suttatitletranslation Brahmadatta }{\suttatitleroot Brahmadattattheragāthā}}
\addcontentsline{toc}{section}{\tocacronym{Thag 6.12} \toctranslation{Brahmadatta } \tocroot{Brahmadattattheragāthā}}
\markboth{Brahmadatta }{Brahmadattattheragāthā}
\extramarks{Thag 6.12}{Thag 6.12}

\begin{verse}%
From\marginnote{1.1} where would come anger for one free of anger, \\
tamed, living justly, \\
freed by right knowledge, \\
peaceful and unaffected? 

When\marginnote{2.1} you get angry at an angry person \\
you just make things worse for yourself. \\
When you don’t get angry at an angry person \\
you win a battle hard to win. 

When\marginnote{3.1} you know that the other is angry, \\
you act for the good of both \\
yourself and the other \\
if you’re mindful and stay calm. 

People\marginnote{4.1} unfamiliar with the teaching \\
consider one who heals both \\
oneself and the other \\
to be a fool. 

If\marginnote{5.1} anger arises in you, \\
reflect on the simile of the saw; \\
if craving for flavors arises in you, \\
remember the simile of the child’s flesh. 

If\marginnote{6.1} your mind runs off \\
to sensual pleasures and future lives, \\
quickly curb it with mindfulness, \\
as one would curb a greedy cow eating grain. 

%
\end{verse}

%
\section*{{\suttatitleacronym Thag 6.13}{\suttatitletranslation Sirimaṇḍa }{\suttatitleroot Sirimaṇḍattheragāthā}}
\addcontentsline{toc}{section}{\tocacronym{Thag 6.13} \toctranslation{Sirimaṇḍa } \tocroot{Sirimaṇḍattheragāthā}}
\markboth{Sirimaṇḍa }{Sirimaṇḍattheragāthā}
\extramarks{Thag 6.13}{Thag 6.13}

\begin{verse}%
The\marginnote{1.1} rain saturates things that are covered up; \\
it doesn’t saturate things that are open. \\
Therefore you should open up a covered thing, \\
so the rain will not saturate it. 

The\marginnote{2.1} world is beaten down by death \\
and surrounded by old age. \\
The dart of craving has struck it down, \\
and it’s always fuming with desire. 

The\marginnote{3.1} world is beaten down by death, \\
caged by old age, \\
beaten constantly without respite, \\
like a thief being flogged. 

Three\marginnote{4.1} things are coming, like a wall of flame: \\
death, disease, and old age. \\
No power can stand before them, \\
and no speed’s enough to flee. 

Don’t\marginnote{5.1} waste your day, \\
a little or a lot. \\
Every night that passes \\
shortens your life by that much. 

Walking\marginnote{6.1} or standing, \\
sitting or lying down: \\
your final night draws near; \\
you have no time to be careless. 

%
\end{verse}

%
\section*{{\suttatitleacronym Thag 6.14}{\suttatitletranslation Sabbakāmi }{\suttatitleroot Sabbakāmittheragāthā}}
\addcontentsline{toc}{section}{\tocacronym{Thag 6.14} \toctranslation{Sabbakāmi } \tocroot{Sabbakāmittheragāthā}}
\markboth{Sabbakāmi }{Sabbakāmittheragāthā}
\extramarks{Thag 6.14}{Thag 6.14}

\begin{verse}%
This\marginnote{1.1} two-legged body is dirty and stinking, \\
full of different carcasses, \\
and oozing all over the place—\\
but still it is cherished! 

Like\marginnote{2.1} a lurking deer by a trick, \\
like a fish by a hook, \\
like a langur by tar—\footnote{The \textit{\textsanskrit{vānara}} is a type of human-like monkey often identified with the langur. See \href{https://suttacentral.net/ja57/en/sujato}{Ja 57}. } \\
they trap an ordinary person. 

Sights,\marginnote{3.1} sounds, tastes, smells, \\
and touches so delightful: \\
these five kinds of sensual stimulation \\
are seen in a woman’s body. 

Those\marginnote{4.1} ordinary people, their minds full of lust, \\
who pursue those women: \\
they swell the horrors of the charnel ground, \\
piling up future lives. 

One\marginnote{5.1} who, being mindful, \\
avoids them \\
like a snake’s head with their foot, \\
gets over clinging to the world. 

Seeing\marginnote{6.1} the danger in sensual pleasures, \\
seeing renunciation as sanctuary, \\
I’ve escaped all sensual pleasures, \\
and attained the ending of defilements. 

%
\end{verse}

%
\addtocontents{toc}{\let\protect\contentsline\protect\nopagecontentsline}
\part*{The Book of the Sevens }
\addcontentsline{toc}{part}{The Book of the Sevens }
\markboth{}{}
\addtocontents{toc}{\let\protect\contentsline\protect\oldcontentsline}

%
\addtocontents{toc}{\let\protect\contentsline\protect\nopagecontentsline}
\chapter*{Chapter One }
\addcontentsline{toc}{chapter}{\tocchapterline{Chapter One }}
\addtocontents{toc}{\let\protect\contentsline\protect\oldcontentsline}

%
\section*{{\suttatitleacronym Thag 7.1}{\suttatitletranslation Sundarasamudda }{\suttatitleroot Sundarasamuddattheragāthā}}
\addcontentsline{toc}{section}{\tocacronym{Thag 7.1} \toctranslation{Sundarasamudda } \tocroot{Sundarasamuddattheragāthā}}
\markboth{Sundarasamudda }{Sundarasamuddattheragāthā}
\extramarks{Thag 7.1}{Thag 7.1}

\begin{verse}%
Adorned\marginnote{1.1} with jewelry and all dressed up, \\
with her garland and her makeup on, \\
and her feet so brightly rouged: \\
the courtesan was wearing sandals. 

Stepping\marginnote{2.1} off her sandals in front of me, \\
her palms joined in greeting, \\
smiling, she spoke to me \\
so softly and so sweet: 

“You’re\marginnote{3.1} too young to go forth—\\
come, stay in my teaching! \\
Enjoy human sensual pleasures, \\
I’ll give you riches. \\
I promise this is the truth—\\
I swear it by the Sacred Flame. 

And\marginnote{4.1} when we’ve grown old together, \\
both of us leaning on staffs, \\
we shall both go forth, \\
and hold a perfect hand on both counts.” 

I\marginnote{5.1} saw the courtesan seducing me, \\
her palms joined in greeting, \\
adorned with jewelry and all dressed up, \\
like a snare of death laid down. 

Then\marginnote{6.1} the realization \\
came upon me—\\
the danger became clear \\
and I grew firmly disillusioned. 

Then\marginnote{7.1} my mind was freed—\\
see the excellence of the Dhamma! \\
I’ve attained the three knowledges, \\
and fulfilled the Buddha’s instructions. 

%
\end{verse}

%
\section*{{\suttatitleacronym Thag 7.2}{\suttatitletranslation Bhaddiya the Dwarf }{\suttatitleroot Lakuṇḍakabhaddiyattheragāthā}}
\addcontentsline{toc}{section}{\tocacronym{Thag 7.2} \toctranslation{Bhaddiya the Dwarf } \tocroot{Lakuṇḍakabhaddiyattheragāthā}}
\markboth{Bhaddiya the Dwarf }{Lakuṇḍakabhaddiyattheragāthā}
\extramarks{Thag 7.2}{Thag 7.2}

\begin{verse}%
Bhaddiya\marginnote{1.1} has plucked out craving, root and all, \\
and in a jungle thicket \\
on the far side of the Wild Mango Monastery, \\
he practices absorption; he is truly well-favoured. 

Some\marginnote{2.1} delight in clay drums, \\
in arched harps, and in cymbals. \\
But here, at the foot of a tree, \\
I delight in the Buddha’s teaching. 

If\marginnote{3.1} the Buddha were to grant me one wish, \\
and I were to get what I wished for, \\
I’d choose for the whole world \\
constant mindfulness of the body. 

Those\marginnote{4.1} who’ve judged me on appearance, \\
and those swayed by my voice, \\
are full of desire and greed; \\
they don’t know me. 

Not\marginnote{5.1} knowing what’s inside, \\
nor seeing what’s outside, \\
the fool shut in on every side, \\
gets carried away by a voice. 

Not\marginnote{6.1} knowing what’s inside, \\
but discerning what’s outside, \\
seeing the fruit outside, \\
they’re also carried away by a voice. 

Understanding\marginnote{7.1} what’s inside, \\
and discerning what’s outside, \\
of unhindered vision, \\
they don’t get carried away by a voice. 

%
\end{verse}

%
\section*{{\suttatitleacronym Thag 7.3}{\suttatitletranslation Bhadda }{\suttatitleroot Bhaddattheragāthā}}
\addcontentsline{toc}{section}{\tocacronym{Thag 7.3} \toctranslation{Bhadda } \tocroot{Bhaddattheragāthā}}
\markboth{Bhadda }{Bhaddattheragāthā}
\extramarks{Thag 7.3}{Thag 7.3}

\begin{verse}%
I\marginnote{1.1} was an only child, \\
loved by my mother and father. \\
They had me by practicing \\
many prayers and observances. 

Out\marginnote{2.1} of sympathy for me \\
wishing me well and wanting the best for me, \\
my mother and father \\
took me to see the Buddha. 

“We\marginnote{3.1} had this son with difficulty; \\
he is delicate and dainty. \\
We offer him to you, Lord, \\
to attend upon the victor.” 

The\marginnote{4.1} teacher, having accepted me, \\
declared to Ānanda: \\
“Quickly give him the going-forth—\\
this one will be a thoroughbred!” 

After\marginnote{5.1} he, the teacher, had sent me forth, \\
the victor entered his dwelling. \\
Before the sun set \\
my mind was freed. 

The\marginnote{6.1} teacher didn’t neglect me; \\
when he came out from seclusion, \\
he said: “Come Bhadda!” \\
That was my ordination. 

At\marginnote{7.1} seven years old \\
I received ordination. \\
I’ve attained the three knowledges; \\
oh, the excellence of the teaching! 

%
\end{verse}

%
\section*{{\suttatitleacronym Thag 7.4}{\suttatitletranslation Sopāka (2nd) }{\suttatitleroot Sopākattheragāthā}}
\addcontentsline{toc}{section}{\tocacronym{Thag 7.4} \toctranslation{Sopāka (2nd) } \tocroot{Sopākattheragāthā}}
\markboth{Sopāka (2nd) }{Sopākattheragāthā}
\extramarks{Thag 7.4}{Thag 7.4}

\begin{verse}%
I\marginnote{1.1} saw the supreme person \\
walking mindfully in the shade of the terrace, \\
so I approached, \\
and bowed to the supreme among men. 

Arranging\marginnote{2.1} my robe over one shoulder \\
and clasping my hands together, \\
I walked alongside that stainless one, \\
supreme among all beings. 

The\marginnote{3.1} wise one, expert in questions, \\
questioned me. \\
Brave and fearless, \\
I answered the teacher. 

When\marginnote{4.1} all his questions were answered, \\
the Realized One congratulated me. \\
Looking around the mendicant \textsanskrit{Saṅgha}, \\
he said the following: 

“It\marginnote{5.1} is a blessing for the people of \textsanskrit{Aṅga} and Magadha \\
that this person enjoys their \\
robe and almsfood, \\
requisites and lodgings, \\
their respect and service—\\
it’s a blessing for them,” he declared. 

“\textsanskrit{Sopāka},\marginnote{6.1} from this day on \\
you are invited to come and see me. \\
And \textsanskrit{Sopāka}, let this \\
be your ordination.” 

At\marginnote{7.1} seven years old \\
I received ordination. \\
I bear my final body—\\
oh, the excellence of the teaching! 

%
\end{verse}

%
\section*{{\suttatitleacronym Thag 7.5}{\suttatitletranslation Sarabhaṅga }{\suttatitleroot Sarabhaṅgattheragāthā}}
\addcontentsline{toc}{section}{\tocacronym{Thag 7.5} \toctranslation{Sarabhaṅga } \tocroot{Sarabhaṅgattheragāthā}}
\markboth{Sarabhaṅga }{Sarabhaṅgattheragāthā}
\extramarks{Thag 7.5}{Thag 7.5}

\begin{verse}%
I\marginnote{1.1} broke the reeds off with my hands, \\
made a hut, and stayed there. \\
That’s how I became known \\
as “Reed-breaker”. 

But\marginnote{2.1} now it’s not appropriate \\
for me to break reeds with my hands. \\
The training rules have been laid down for us \\
by Gotama the renowned. 

Previously,\marginnote{3.1} I, \textsanskrit{Sarabhaṅga}, \\
didn’t see the disease in its entirety. \\
But now I have seen the disease, \\
as I’ve practiced what was taught \\>by he who is beyond the gods. 

Gotama\marginnote{4.1} traveled by that direct route; \\
the same path traveled by \textsanskrit{Vipassī}, \\
by \textsanskrit{Sikhī}, \textsanskrit{Vessabhū}, \\
Kakusandha, \textsanskrit{Koṇāgamana}, and Kassapa. 

These\marginnote{5.1} seven Buddhas have plunged into the ending, \\
free of craving, without grasping, \\
having become Dhamma, unaffected. \\
They have taught this Dhamma 

out\marginnote{6.1} of compassion for living creatures—\\
suffering, origin, path, \\
and cessation, the ending of suffering. \\
In these four noble truths, 

the\marginnote{7.1} endless suffering of transmigration \\
finally comes to an end. \\
When the body breaks up, \\
and life comes to an end, \\
there are no future lives; \\
I’m everywhere well freed. 

%
\end{verse}

%
\addtocontents{toc}{\let\protect\contentsline\protect\nopagecontentsline}
\part*{The Book of the Eights }
\addcontentsline{toc}{part}{The Book of the Eights }
\markboth{}{}
\addtocontents{toc}{\let\protect\contentsline\protect\oldcontentsline}

%
\addtocontents{toc}{\let\protect\contentsline\protect\nopagecontentsline}
\chapter*{Chapter One }
\addcontentsline{toc}{chapter}{\tocchapterline{Chapter One }}
\addtocontents{toc}{\let\protect\contentsline\protect\oldcontentsline}

%
\section*{{\suttatitleacronym Thag 8.1}{\suttatitletranslation Mahākaccāyana }{\suttatitleroot Mahākaccāyanattheragāthā}}
\addcontentsline{toc}{section}{\tocacronym{Thag 8.1} \toctranslation{Mahākaccāyana } \tocroot{Mahākaccāyanattheragāthā}}
\markboth{Mahākaccāyana }{Mahākaccāyanattheragāthā}
\extramarks{Thag 8.1}{Thag 8.1}

\begin{verse}%
Don’t\marginnote{1.1} get involved in lots of work, \\
avoid people, and don’t try to acquire things. \\
If you’re eager and greedy for flavors, \\
you’ll miss the goal that brings such happiness. 

They\marginnote{2.1} know it really is a bog, \\
this homage and veneration in respectable families. \\
Honor is a subtle dart, hard to extract, \\
and hard for a sinner to give up. 

The\marginnote{3.1} deeds of a mortal aren’t bad \\
because of what others do. \\
You yourself should not do bad, \\
for mortals have deeds as their kin. 

You’re\marginnote{4.1} not a bandit because of what someone says, \\
you’re not a sage because of what someone says; \\
but as you know yourself, \\
so the gods will know you. 

When\marginnote{5.1} others do not understand, \\
let us, who do understand this, \\
restrain ourselves in this regard; \\
for that is how conflicts are laid to rest. 

A\marginnote{6.1} wise person lives on \\
even after loss of wealth; \\
but without gaining wisdom, \\
even a rich person doesn’t really live. 

All\marginnote{7.1} is heard with the ear, \\
all is seen with the eye; \\
the attentive ought not discard \\
all that is seen and heard. 

Though\marginnote{8.1} you have eyes, be as if blind; \\
though you have ears, be as if deaf; \\
though you have wisdom, be as if dumb; \\
though you have strength, be as if feeble. \\
And when issues come up \\
lie as still as a corpse. 

%
\end{verse}

%
\section*{{\suttatitleacronym Thag 8.2}{\suttatitletranslation Sirimitta }{\suttatitleroot Sirimittattheragāthā}}
\addcontentsline{toc}{section}{\tocacronym{Thag 8.2} \toctranslation{Sirimitta } \tocroot{Sirimittattheragāthā}}
\markboth{Sirimitta }{Sirimittattheragāthā}
\extramarks{Thag 8.2}{Thag 8.2}

\begin{verse}%
Free\marginnote{1.1} of anger and acrimony, \\
free of deceit, and rid of slander; \\
that’s how such a mendicant \\
doesn’t grieve after passing away. 

Free\marginnote{2.1} of anger and acrimony, \\
free of deceit, and rid of slander; \\
that’s how a mendicant \\>with sense doors always guarded \\
doesn’t grieve after passing away. 

Free\marginnote{3.1} of anger and acrimony, \\
free of deceit, and rid of slander; \\
that’s how a mendicant of good morals \\
doesn’t grieve after passing away. 

Free\marginnote{4.1} of anger and acrimony, \\
free of deceit, and rid of slander; \\
that’s how a mendicant with good friends \\
doesn’t grieve after passing away. 

Free\marginnote{5.1} of anger and acrimony, \\
free of deceit, and rid of slander; \\
that’s how a mendicant of good wisdom, \\
doesn’t grieve after passing away. 

Whoever\marginnote{6.1} has faith in the Realized One, \\
unwavering and well grounded; \\
whose ethical conduct is good, \\
praised and loved by the noble ones; 

who\marginnote{7.1} has confidence in the \textsanskrit{Saṅgha}, \\
and correct view: \\
they’re said to be prosperous; \\
their life is not in vain. 

So\marginnote{8.1} let the wise devote themselves \\
to faith, ethical behavior, \\
confidence, and insight into the teaching, \\
remembering the instructions of the Buddhas. 

%
\end{verse}

%
\section*{{\suttatitleacronym Thag 8.3}{\suttatitletranslation Mahāpanthaka }{\suttatitleroot Mahāpanthakattheragāthā}}
\addcontentsline{toc}{section}{\tocacronym{Thag 8.3} \toctranslation{Mahāpanthaka } \tocroot{Mahāpanthakattheragāthā}}
\markboth{Mahāpanthaka }{Mahāpanthakattheragāthā}
\extramarks{Thag 8.3}{Thag 8.3}

\begin{verse}%
When\marginnote{1.1} I first saw the Teacher \\
who fears nothing from any quarter, \\
I was struck with a sense of urgency, \\
seeing the supreme among men. 

Anyone\marginnote{2.1} who, having found such a Teacher, \\
would lose them again, \\
is like one who, when approached by Lady Luck, \\
would ward her off with their hands and feet.\footnote{The same idiom recurs at \href{https://suttacentral.net/mn49/en/sujato\#5.9}{MN 49:5.9}. } 

Then\marginnote{3.1} I left behind my children and wives, \\
my riches and my grain; \\
I had my hair and beard cut off, \\
and went forth to homelessness. 

Endowed\marginnote{4.1} with the monastic training and livelihood, \\
my sense faculties well-restrained, \\
paying homage to the Buddha, \\
I meditated undefeated. 

Then\marginnote{5.1} a wish occurred to me, \\
my heart’s truest wish: \\
I wouldn’t sit down, not even for a short while, \\
until the dart of craving was plucked. 

As\marginnote{6.1} I meditate like this, \\
see my energy and vigor! \\
I’ve attained the three knowledges, \\
and fulfilled the Buddha’s instructions. 

I\marginnote{7.1} know my past lives, \\
my clairvoyance is clarified; \\
I’m perfected, worthy of offerings, \\
liberated, free of attachments. 

Then,\marginnote{8.1} at the end of the night, \\
as the rising of the sun drew near, \\
all craving was dried up, \\
so I sat down cross-legged. 

%
\end{verse}

%
\addtocontents{toc}{\let\protect\contentsline\protect\nopagecontentsline}
\part*{The Book of the Nines }
\addcontentsline{toc}{part}{The Book of the Nines }
\markboth{}{}
\addtocontents{toc}{\let\protect\contentsline\protect\oldcontentsline}

%
\addtocontents{toc}{\let\protect\contentsline\protect\nopagecontentsline}
\chapter*{Chapter One }
\addcontentsline{toc}{chapter}{\tocchapterline{Chapter One }}
\addtocontents{toc}{\let\protect\contentsline\protect\oldcontentsline}

%
\section*{{\suttatitleacronym Thag 9.1}{\suttatitletranslation Bhūta }{\suttatitleroot Bhūtattheragāthā}}
\addcontentsline{toc}{section}{\tocacronym{Thag 9.1} \toctranslation{Bhūta } \tocroot{Bhūtattheragāthā}}
\markboth{Bhūta }{Bhūtattheragāthā}
\extramarks{Thag 9.1}{Thag 9.1}

\begin{verse}%
When\marginnote{1.1} an astute person knows, \\>“Old age and death are suffering; \\
yet an ignorant ordinary person is bound to them”, \\
completely understanding suffering, being mindful, \\>practicing absorption: \\
there is no greater pleasure than this. 

When\marginnote{2.1} clinging, the carrier \\>of suffering, \\
and craving, the carrier  \\>of this painful mass of proliferation, \\
are destroyed, and one is mindful, \\>practicing absorption: \\
there is no greater pleasure than this. 

When\marginnote{3.1} the eightfold way, so full of grace, \\
the supreme path, cleanser of all corruptions, \\
is seen with wisdom; and one is mindful, \\>practicing absorption: \\
there is no greater pleasure than this. 

When\marginnote{4.1} one develops that peaceful state, \\
sorrowless, stainless, unconditioned, \\
cleanser of all corruptions, \\>cutter of fetters and bonds: \\
there is no greater pleasure than this. 

When\marginnote{5.1} the thunder-cloud rumbles in the sky, \\
while the rain pours on the path of birds all around, \\
and a monk has gone to a mountain cave, \\>practicing absorption: \\
there is no greater pleasure than this. 

When\marginnote{6.1} sitting on a riverbank covered in flowers, \\
garlanded with brightly colored forest plants, \\
one is truly happy, \\>practicing absorption: \\
there is no greater pleasure than this. 

When\marginnote{7.1} it is midnight in a lonely forest, \\
and the lions roar as the heavens pour, \\
and a monk has gone to a mountain cave, \\>practicing absorption: \\
there is no greater pleasure than this. 

When\marginnote{8.1} one’s own thoughts have been cut off, \\
between the mountains, sheltered inside a cleft, \\
without stress or heartlessness, \\>practicing absorption: \\
there is no greater pleasure than this. 

When\marginnote{9.1} one is happy, destroyer of stains, heartlessness, and sorrow, \\
free of obstructions, entanglements, and thorns, \\
and with all defilements annihilated, \\>practicing absorption: \\
there is no greater pleasure than this. 

%
\end{verse}

%
\addtocontents{toc}{\let\protect\contentsline\protect\nopagecontentsline}
\part*{The Book of the Tens }
\addcontentsline{toc}{part}{The Book of the Tens }
\markboth{}{}
\addtocontents{toc}{\let\protect\contentsline\protect\oldcontentsline}

%
\addtocontents{toc}{\let\protect\contentsline\protect\nopagecontentsline}
\chapter*{Chapter One }
\addcontentsline{toc}{chapter}{\tocchapterline{Chapter One }}
\addtocontents{toc}{\let\protect\contentsline\protect\oldcontentsline}

%
\section*{{\suttatitleacronym Thag 10.1}{\suttatitletranslation Kāḷudāyī }{\suttatitleroot Kāḷudāyittheragāthā}}
\addcontentsline{toc}{section}{\tocacronym{Thag 10.1} \toctranslation{Kāḷudāyī } \tocroot{Kāḷudāyittheragāthā}}
\markboth{Kāḷudāyī }{Kāḷudāyittheragāthā}
\extramarks{Thag 10.1}{Thag 10.1}

\begin{verse}%
“The\marginnote{1.1} trees are now crimson, venerable sir, \\
they’ve shed their foliage, and are ready to fruit. \\
They’re splendid, as if aflame; \\
great hero, this season is full of flavor. 

The\marginnote{2.1} blossoming trees are delightful, \\
wafting their scent all around, in all directions. \\
They’ve shed their leaves and wish to fruit, \\
hero, it is time to depart from here. 

It\marginnote{3.1} is neither too hot nor too cold, \\
venerable sir, it’s a pleasant season for traveling. \\
Let the \textsanskrit{Sākiyans} and Koliyans see you, \\
heading west across the \textsanskrit{Rohiṇī} river. 

In\marginnote{4.1} hope, the field is plowed; \\
the seed is sown in hope; \\
in hope, merchants travel the seas, \\
carrying rich cargoes. \\
The hope that I stand for: \\
may it succeed! 

Again\marginnote{5.1} and again, they sow the seed; \\
again and again, the king of the heavens sends rain; \\
again and again, farmers plow the field; \\
again and again, grain is produced for the nation. 

Again\marginnote{6.1} and again, the beggars wander, \\
again and again, the donors give. \\
Again and again, when the donors have given, \\
again and again, they take their place in heaven. 

A\marginnote{7.1} hero of vast wisdom purifies seven generations \\
of the family in which they’re born. \\
Sakya, I believe you’re the king of kings, \\
since you fathered the one who is truly called a sage. 

The\marginnote{8.1} father of the great seer is named Suddhodana; \\
and the Buddha’s mother is named \textsanskrit{Māyā}. \\
Having borne the Bodhisatta in her belly, \\
she rejoices in the heaven of the thirty-three. 

When\marginnote{9.1} she died and passed away from here, \\
she was blessed with heavenly sensual pleasures; \\
enjoying the five kinds of sensual stimulation. \\
\textsanskrit{Gotamī} is surrounded by those hosts of gods.” 

“I’m\marginnote{10.1} the son of the Buddha, the incomparable \textsanskrit{Aṅgīrasa}, the unaffected, \\
the bearer of the unbearable. \\
You, Sakya, are my father’s father; \\
Gotama, you are my grandfather in the Dhamma.” 

%
\end{verse}

%
\section*{{\suttatitleacronym Thag 10.2}{\suttatitletranslation Ekavihāriya }{\suttatitleroot Ekavihāriyattheragāthā}}
\addcontentsline{toc}{section}{\tocacronym{Thag 10.2} \toctranslation{Ekavihāriya } \tocroot{Ekavihāriyattheragāthā}}
\markboth{Ekavihāriya }{Ekavihāriyattheragāthā}
\extramarks{Thag 10.2}{Thag 10.2}

\begin{verse}%
If\marginnote{1.1} no-one else is found \\
before or behind, \\
it’s extremely pleasant \\
to be dwelling alone in a forest grove. 

Come\marginnote{2.1} now, I’ll go alone \\
to the wilderness praised by the Buddha. \\
It’s pleasant for a mendicant \\
to be dwelling alone and resolute. 

Alone\marginnote{3.1} and self-disciplined, \\
I’ll quickly enter the delightful forest, \\
which gives joy to meditators, \\
and is frequented by rutting elephants. 

In\marginnote{4.1} the Cool Grove, so full of flowers, \\
in a cool mountain cave, \\
I’ll bathe my limbs \\
and walk mindfully alone. 

When\marginnote{5.1} will I dwell alone, \\
without a companion, \\
in the great wood, so delightful, \\
my task complete, free of defilements? 

This\marginnote{6.1} is what I want to do: \\
may my wish succeed! \\
I’ll make it happen myself, \\
for no-one can do another’s duty. 

Fastening\marginnote{7.1} my armor, \\
I’ll enter the forest. \\
I won’t leave \\
without attaining the end of defilements. 

As\marginnote{8.1} the cool gale blows \\
with fragrant scent, \\
I’ll split ignorance apart, \\
sitting on the mountain-peak. 

In\marginnote{9.1} a forest grove covered with blossoms, \\
in a cave so very cool, \\
I take pleasure in the Mountainfold, \\
happy with the happiness of freedom. 

I’ve\marginnote{10.1} got all I wished for \\
like the moon on the fifteenth day. \\
With the utter ending of all defilements, \\
now there’ll be no more future lives. 

%
\end{verse}

%
\section*{{\suttatitleacronym Thag 10.3}{\suttatitletranslation Mahākappina }{\suttatitleroot Mahākappinattheragāthā}}
\addcontentsline{toc}{section}{\tocacronym{Thag 10.3} \toctranslation{Mahākappina } \tocroot{Mahākappinattheragāthā}}
\markboth{Mahākappina }{Mahākappinattheragāthā}
\extramarks{Thag 10.3}{Thag 10.3}

\begin{verse}%
If\marginnote{1.1} you’re prepared for the future, \\
both the good and the bad, \\
then those, whether enemies or well-wishers, \\
who examine you for weakness will see none. 

One\marginnote{2.1} who has fulfilled, developed, \\
and gradually consolidated \\
mindfulness of breathing \\
as it was taught by the Buddha: \\
they light up the world, \\
like the moon freed from clouds. 

Yes,\marginnote{3.1} my mind is clean, \\
limitless and well-developed; \\
it has broken through and been uplifted—\\
it radiates in every direction. 

A\marginnote{4.1} wise person lives on \\
even after loss of wealth; \\
but without gaining wisdom, \\
even a rich person doesn’t really live. 

Wisdom\marginnote{5.1} questions what is learned; \\
wisdom grows fame and reputation; \\
a person who has wisdom \\
finds happiness even among sufferings. 

It’s\marginnote{6.1} not a teaching just for today; \\
it isn’t incredible or amazing. \\
When you’re born, you die—\\
what’s amazing about that? 

For\marginnote{7.1} anyone who is born, \\
death always follows after life. \\
Everyone who is born here dies here; \\
such is the nature of living creatures. 

The\marginnote{8.1} things that are useful for the living \\
are of no use for the dead—not fame, not celebrity, \\
not praise by ascetics and brahmins. \\
For the dead, there is only weeping. 

And\marginnote{9.1} weeping impairs the eye and the body; \\
complexion, strength, and intelligence decline. \\
Your enemies rejoice; \\
but your well-wishers are not happy. 

So\marginnote{10.1} you should wish that those who stay in your family \\
have intelligence and learning, \\
and do their duty through the power of wisdom, \\
just as you’d cross a full river by boat. 

%
\end{verse}

%
\section*{{\suttatitleacronym Thag 10.4}{\suttatitletranslation Cūḷapanthaka }{\suttatitleroot Cūḷapanthakattheragāthā}}
\addcontentsline{toc}{section}{\tocacronym{Thag 10.4} \toctranslation{Cūḷapanthaka } \tocroot{Cūḷapanthakattheragāthā}}
\markboth{Cūḷapanthaka }{Cūḷapanthakattheragāthā}
\extramarks{Thag 10.4}{Thag 10.4}

\begin{verse}%
My\marginnote{1.1} progress was slow, \\
I was despised in the past. \\
Even my brother turned me away, \\
saying, “Go home now.” 

Turned\marginnote{2.1} away at the gate \\
of the \textsanskrit{Saṅgha}’s monastery, \\
I stood there sadly, \\
longing for the dispensation. 

Then\marginnote{3.1} the Buddha came \\
and touched my head. \\
Taking me by the arm, \\
he brought me into the \textsanskrit{Saṅgha}’s monastery. 

The\marginnote{4.1} Teacher, out of sympathy, \\
gave me a foot-wiping cloth, saying: \\
“Focus your awareness \\
exclusively on this clean cloth.” 

After\marginnote{5.1} hearing his words, \\
I happily did his bidding. \\
I practiced meditative immersion \\
for the attainment of the highest goal. 

I\marginnote{6.1} know my past lives, \\
my clairvoyance is clarified; \\
I’ve attained the three knowledges, \\
and fulfilled the Buddha’s instructions. 

I,\marginnote{7.1} Panthaka, created a thousand \\
images of myself, \\
and sat in the delightful mango grove \\
until the time for the meal offering was announced. 

Then\marginnote{8.1} the teacher sent to me \\
a messenger to announce the time. \\
When the time was announced, \\
I flew to him through the air. 

I\marginnote{9.1} paid homage at the teacher’s feet, \\
and sat to one side. \\
When he knew I was seated, \\
the teacher received the offering. 

Recipient\marginnote{10.1} of gifts from the whole world, \\
receiver of sacrifices, \\
field of merit for humanity, \\
he received the religious donation. 

%
\end{verse}

%
\section*{{\suttatitleacronym Thag 10.5}{\suttatitletranslation Kappa }{\suttatitleroot Kappattheragāthā}}
\addcontentsline{toc}{section}{\tocacronym{Thag 10.5} \toctranslation{Kappa } \tocroot{Kappattheragāthā}}
\markboth{Kappa }{Kappattheragāthā}
\extramarks{Thag 10.5}{Thag 10.5}

\begin{verse}%
Filled\marginnote{1.1} with different kinds of dirt, \\
a great producer of dung, \\
like a stagnant cesspool, \\
a huge boil, a bad wound, 

full\marginnote{2.1} of pus and blood, \\
sunk in a toilet-pit, \\
trickling with fluids, \\
this rotting body always oozes. 

Bound\marginnote{3.1} by sixty tendons, \\
coated with a fleshy coating, \\
clothed in a jacket of skin, \\
this rotting body is worthless. 

Held\marginnote{4.1} together by a skeleton of bones, \\
and bound by sinews; \\
it assumes postures \\
due to a complex of many things. 

We\marginnote{5.1} set out in the certainty of death, \\
in the presence of the King of Death; \\
and having discarded the body right here, \\
a person goes where he likes. 

Shrouded\marginnote{6.1} by ignorance, \\
tied by the four ties, \\
this body is sinking in the flood, \\
caught in the net of the underlying tendencies. 

Yoked\marginnote{7.1} to the five hindrances, \\
afflicted by thought, \\
stuck to the root of craving, \\
hidden by delusion: 

that\marginnote{8.1} is how the body goes on, \\
propelled by the mechanism of deeds. \\
But existence ends in perishing; \\
separated, the body perishes. 

Those\marginnote{9.1} blind, ordinary people \\
who think of this body as theirs, \\
swell the horrors of the charnel ground, \\
taking hold of future lives. 

The\marginnote{10.1} undefiled who shun this body \\
like a snake smeared with dung, \\
having expelled the root of rebirth, \\
will be fully quenched. 

%
\end{verse}

%
\section*{{\suttatitleacronym Thag 10.6}{\suttatitletranslation Upasena son of Vaṅgantā }{\suttatitleroot Vaṅgantaputtaupasenattheragāthā}}
\addcontentsline{toc}{section}{\tocacronym{Thag 10.6} \toctranslation{Upasena son of Vaṅgantā } \tocroot{Vaṅgantaputtaupasenattheragāthā}}
\markboth{Upasena son of Vaṅgantā }{Vaṅgantaputtaupasenattheragāthā}
\extramarks{Thag 10.6}{Thag 10.6}

\begin{verse}%
In\marginnote{1.1} order to go on retreat, \\
a monk should stay in lodgings \\
that are secluded and quiet, \\
frequented by beasts of prey. 

Having\marginnote{2.1} gathered scraps from rubbish heaps, \\
cemeteries and streets, \\
and making an outer robe from them, \\
one should wear that coarse robe. 

Humbling\marginnote{3.1} their heart, \\
a mendicant should walk for alms \\
from family to family indiscriminately, \\
with sense doors guarded, well-restrained. 

They\marginnote{4.1} should be content even with coarse food, \\
not hoping for lots of flavors. \\
The mind that’s greedy for flavors \\
doesn’t enjoy absorption. 

With\marginnote{5.1} few wishes, content, \\
a sage should live secluded, \\
mixing with neither \\
householders nor the homeless. 

They\marginnote{6.1} should present themselves \\
as if stupid or dumb; \\
an astute person would not speak overly long \\
in the midst of the \textsanskrit{Saṅgha}. 

They\marginnote{7.1} would not insult anyone, \\
and would avoid causing damage. \\
Restrained in the monastic code, \\
they would eat in moderation. 

Expert\marginnote{8.1} in the arising of thought, \\
they would grasp well the pattern of the mind. \\
They would be devoted to practicing \\
serenity and discernment at the right time. 

Though\marginnote{9.1} endowed with energy and perseverance, \\
and always devoted to meditation, \\
a wise person would not be too sure of themselves, \\
until they have attained the end of suffering. 

For\marginnote{10.1} a mendicant who meditates in this way, \\
longing for purification, \\
all their defilements wither away, \\
and they realize quenching. 

%
\end{verse}

%
\section*{{\suttatitleacronym Thag 10.7}{\suttatitletranslation Another Gotama }{\suttatitleroot (Apara) Gotamattheragāthā}}
\addcontentsline{toc}{section}{\tocacronym{Thag 10.7} \toctranslation{Another Gotama } \tocroot{(Apara) Gotamattheragāthā}}
\markboth{Another Gotama }{(Apara) Gotamattheragāthā}
\extramarks{Thag 10.7}{Thag 10.7}

\begin{verse}%
You\marginnote{1.1} should understand your own purpose, \\
and consider the dispensation carefully, \\
as well as what’s appropriate \\
for one who has entered the ascetic life. 

Good\marginnote{2.1} friendship in the community, \\
undertaking plenty of training, \\
eagerness to learn from the teachers—\\
this is appropriate for an ascetic. 

Respect\marginnote{3.1} for the Buddha, \\
reverence for the Dhamma as it really is, \\
esteem for the \textsanskrit{Saṅgha}—\\
this is appropriate for an ascetic. 

Commitment\marginnote{4.1} to good conduct and alms-resort, \\
a livelihood that is pure and blameless, \\
and stilling the mind—\\
this is appropriate for an ascetic. 

An\marginnote{5.1} impressive deportment \\>in things that should be done, \\
and in those better avoided; \\
commitment to the higher mind—\\
this is appropriate for an ascetic. 

Wilderness\marginnote{6.1} lodgings, \\
remote and quiet, \\
fit for use by a sage—\\
this is appropriate for an ascetic. 

Ethics,\marginnote{7.1} learning, \\
investigation of teachings in line with the truth, \\
and penetration of the truths—\\
this is appropriate for an ascetic. 

Developing\marginnote{8.1} the perceptions \\
of impermanence, non-self, and unattractiveness, \\
and dissatisfaction with the whole world—\\
this is appropriate for an ascetic. 

Developing\marginnote{9.1} the awakening factors, \\
the bases for psychic power, the faculties and powers, \\
and the noble eightfold path—\\
this is appropriate for an ascetic. 

A\marginnote{10.1} sage should abandon craving, \\
defilements shattered, root and all, \\
they should live liberated—\\
this is appropriate for an ascetic. 

%
\end{verse}

%
\addtocontents{toc}{\let\protect\contentsline\protect\nopagecontentsline}
\part*{The Book of the Elevens }
\addcontentsline{toc}{part}{The Book of the Elevens }
\markboth{}{}
\addtocontents{toc}{\let\protect\contentsline\protect\oldcontentsline}

%
\addtocontents{toc}{\let\protect\contentsline\protect\nopagecontentsline}
\chapter*{Chapter One }
\addcontentsline{toc}{chapter}{\tocchapterline{Chapter One }}
\addtocontents{toc}{\let\protect\contentsline\protect\oldcontentsline}

%
\section*{{\suttatitleacronym Thag 11.1}{\suttatitletranslation Saṅkicca }{\suttatitleroot Saṅkiccattheragāthā}}
\addcontentsline{toc}{section}{\tocacronym{Thag 11.1} \toctranslation{Saṅkicca } \tocroot{Saṅkiccattheragāthā}}
\markboth{Saṅkicca }{Saṅkiccattheragāthā}
\extramarks{Thag 11.1}{Thag 11.1}

\begin{verse}%
“What\marginnote{1.1} good does it do you to be in the grove, my dear? \\
You’re like a little bird in the monsoon! \\
The city of \textsanskrit{Verambhā} is nice for you—\\
seclusion is for meditators.” 

“Just\marginnote{2.1} as the wind in \textsanskrit{Verambhā} \\
scatters the monsoon clouds as they pour down, \\
so the city scatters \\
my perception of seclusion.” 

“It’s\marginnote{3.1} all black and born of an egg—\\
the crow whose abode is the charnel ground \\
rouses my mindfulness, \\
based on dispassion for the body.” 

“He\marginnote{4.1} who is not guarded by others, \\
and who does not guard others, \\
truly sleeps at ease, mendicant,\footnote{The parallel to this verse at \href{https://suttacentral.net/ja10/en/sujato}{Ja 10} is identical except it is addressed to a king. The form there is unambiguously vocative (\textit{\textsanskrit{rāja}}), so I take \textit{bhikkhu} here as also vocative, although the commentary reads it as nominative. } \\
unconcerned for sensual pleasures.” 

“The\marginnote{5.1} water’s clear and the rocks are broad, \\
monkeys and deer are all around; \\
festooned with dewy moss, \\
these rocky crags delight me! 

I’ve\marginnote{6.1} stayed in the wilderness, \\
in caves and caverns \\
and remote lodgings \\
frequented by beasts of prey. 

‘May\marginnote{7.1} these beings be killed! \\
May they be slaughtered! May they suffer!’—\\
I’m not aware of having any such \\
ignoble, hateful thoughts. 

I’ve\marginnote{8.1} served the teacher \\
and fulfilled the Buddha’s instructions. \\
The heavy burden is laid down, \\
the conduit to rebirth is eradicated. 

I’ve\marginnote{9.1} attained the goal \\
for the sake of which I went forth \\
from the lay life to homelessness—\\
the ending of all fetters. 

I\marginnote{10.1} don’t long for death; \\
I don’t long for life; \\
I await my time, \\
like a worker waiting for their wages. 

I\marginnote{11.1} don’t long for death; \\
I don’t long for life; \\
I await my time, \\
aware and mindful.” 

%
\end{verse}

%
\addtocontents{toc}{\let\protect\contentsline\protect\nopagecontentsline}
\part*{The Book of the Twelves }
\addcontentsline{toc}{part}{The Book of the Twelves }
\markboth{}{}
\addtocontents{toc}{\let\protect\contentsline\protect\oldcontentsline}

%
\addtocontents{toc}{\let\protect\contentsline\protect\nopagecontentsline}
\chapter*{Chapter One }
\addcontentsline{toc}{chapter}{\tocchapterline{Chapter One }}
\addtocontents{toc}{\let\protect\contentsline\protect\oldcontentsline}

%
\section*{{\suttatitleacronym Thag 12.1}{\suttatitletranslation Sīlava }{\suttatitleroot Sīlavattheragāthā}}
\addcontentsline{toc}{section}{\tocacronym{Thag 12.1} \toctranslation{Sīlava } \tocroot{Sīlavattheragāthā}}
\markboth{Sīlava }{Sīlavattheragāthā}
\extramarks{Thag 12.1}{Thag 12.1}

\begin{verse}%
One\marginnote{1.1} should train just in ethical conduct, \\
for in this world, when ethical conduct is \\
cultivated and well-trained, \\
it provides all success. 

Wishing\marginnote{2.1} for three kinds of happiness—\\
praise, prosperity, \\
and to delight in heaven after passing away—\\
the wise would take care of their ethics. 

The\marginnote{3.1} well-behaved have many friends, \\
because of their self-restraint. \\
But one lacking ethics, of bad conduct, \\
drives away their friends. 

A\marginnote{4.1} person whose ethics are bad has \\
ill-repute and infamy. \\
A person whose conduct is ethical always has \\
a good reputation, fame, and praise. 

Ethical\marginnote{5.1} conduct is the starting point and foundation; \\
the mother at the head \\
of all good things: \\
that’s why you should purify your ethics. 

Ethics\marginnote{6.1} provide a boundary and a restraint, \\
an enjoyment for the mind; \\
the ford where all the Buddhas cross over: \\
that’s why you should purify your ethics. 

Ethics\marginnote{7.1} are the matchless power; \\
ethics are the ultimate weapon; \\
ethics are the best ornament; \\
ethics are a marvelous coat of armor. 

Ethics\marginnote{8.1} are a mighty bridge; \\
ethics are the unsurpassed scent; \\
ethics are the best perfume, \\
that float from place to place. 

Ethics\marginnote{9.1} are the best provision; \\
ethics are the unsurpassed supply for a journey; \\
ethics are the best vehicle \\
that take you from place to place. 

In\marginnote{10.1} this life they’re criticized; \\
after departing they grieve in a lower realm; \\
a fool is unhappy everywhere, \\
because they are unsteady in ethics. 

In\marginnote{11.1} this life they’re renowned; \\
after departing they’re happy in heaven; \\
the attentive are happy everywhere, \\
because they are steady in ethics. 

Ethical\marginnote{12.1} conduct is best in this life, \\
but one with wisdom is supreme. \\
Someone with both virtue and wisdom \\
is victorious among men and gods. 

%
\end{verse}

%
\section*{{\suttatitleacronym Thag 12.2}{\suttatitletranslation Sunīta }{\suttatitleroot Sunītattheragāthā}}
\addcontentsline{toc}{section}{\tocacronym{Thag 12.2} \toctranslation{Sunīta } \tocroot{Sunītattheragāthā}}
\markboth{Sunīta }{Sunītattheragāthā}
\extramarks{Thag 12.2}{Thag 12.2}

\begin{verse}%
I\marginnote{1.1} was born in a low-class family. \\
We were poor, with little to eat. \\
My job was lowly—\\
I threw out the old flowers. 

Shunned\marginnote{2.1} by people, \\
I was disregarded and held in contempt. \\
I humbled my heart \\
and paid respects to many people. 

Then\marginnote{3.1} I saw the Buddha \\
at the fore of the mendicant \textsanskrit{Saṅgha}; \\
the great hero \\
was entering the capital city of the Magadhans. 

I\marginnote{4.1} dropped my flail \\
and approached to pay homage. \\
Out of sympathy for me, \\
the supreme man stood still. 

When\marginnote{5.1} I had paid homage at the Teacher’s feet, \\
I stood to one side \\
and asked the supreme being \\
for the going-forth. 

Then\marginnote{6.1} the Teacher, being sympathetic, \\
and having sympathy for the whole world, \\
said to me, “Come, monk!” \\
That was my ordination. 

Staying\marginnote{7.1} alone in the wilderness, \\
meditating tirelessly, \\
I have completed what the Teacher taught, \\
just as the victor advised me. 

In\marginnote{8.1} the first watch of the night, \\
I recollected my past lives. \\
In the middle watch, \\
I purified my clairvoyance. \\
In the last watch, \\
I shattered the mass of darkness. 

At\marginnote{9.1} the end of the night, \\
as the sunrise drew near, \\
Indra and the Divinity came \\
and revered me with joined hands. 

“Homage\marginnote{10.1} to you, O thoroughbred! \\
Homage to you, supreme among men! \\
Since your defilements are ended, \\
you, sir, are worthy of a religious donation.” 

When\marginnote{11.1} he saw me honored \\
by the assembly of gods, \\
the teacher smiled \\
and said the following: 

“By\marginnote{12.1} fervor and spiritual practice, \\
by restraint and by self-control: \\
that’s how to become a brahmin, \\
this is the supreme brahmin.” 

%
\end{verse}

%
\addtocontents{toc}{\let\protect\contentsline\protect\nopagecontentsline}
\part*{The Book of the Thirteens }
\addcontentsline{toc}{part}{The Book of the Thirteens }
\markboth{}{}
\addtocontents{toc}{\let\protect\contentsline\protect\oldcontentsline}

%
\addtocontents{toc}{\let\protect\contentsline\protect\nopagecontentsline}
\chapter*{Chapter One }
\addcontentsline{toc}{chapter}{\tocchapterline{Chapter One }}
\addtocontents{toc}{\let\protect\contentsline\protect\oldcontentsline}

%
\section*{{\suttatitleacronym Thag 13.1}{\suttatitletranslation Soṇakoḷivisa }{\suttatitleroot Soṇakoḷivisattheragāthā}}
\addcontentsline{toc}{section}{\tocacronym{Thag 13.1} \toctranslation{Soṇakoḷivisa } \tocroot{Soṇakoḷivisattheragāthā}}
\markboth{Soṇakoḷivisa }{Soṇakoḷivisattheragāthā}
\extramarks{Thag 13.1}{Thag 13.1}

\begin{verse}%
He\marginnote{1.1} who was special in the kingdom, \\
the footman to the king of \textsanskrit{Aṅga}, \\
today is special in the Dhamma—\\
\textsanskrit{Soṇa} has gone beyond suffering. 

Five\marginnote{2.1} to cut, five to drop, \\
and five more to develop. \\
When a mendicant slips five chains \\
they’re said to have crossed the flood. 

If\marginnote{3.1} a monk is insolent and negligent, \\
concerned only with externals, \\
their ethics, immersion, and wisdom \\
do not become fulfilled. 

They\marginnote{4.1} disregard what should be done, \\
and do what should not be done. \\
For the insolent and the negligent, \\
their defilements only grow. 

Those\marginnote{5.1} that have properly undertaken \\
constant mindfulness of the body, \\
don’t cultivate what should not be done, \\
but always do what should be done. \\
Mindful and aware, \\
their defilements come to an end. 

The\marginnote{6.1} straight path has been explained—\\
go on it and don’t turn back. \\
Urge yourself on \\
and take yourself to extinguishment. 

When\marginnote{7.1} my energy was over-exerted, \\
the supreme Teacher in the world \\
created the simile of the arched harp for me. \\
The Clear-eyed One taught the Dhamma, \\
and when I heard what he said, \\
I happily did his bidding. 

Practicing\marginnote{8.1} serenity of mind \\
for the attainment of the highest goal. \\
I’ve attained the three knowledges \\
and fulfilled the Buddha’s instructions. 

When\marginnote{9.1} you’re dedicated to renunciation \\
and seclusion of the heart; \\
when you’re dedicated to kindness \\
and the end of grasping; 

when\marginnote{10.1} you’re dedicated to the ending of craving \\
and clarity of heart; \\
and you’ve seen the arising of the senses, \\
your mind is rightly freed. 

For\marginnote{11.1} that one, rightly freed, \\
a mendicant with peaceful mind, \\
there’s nothing to be improved, \\
and nothing more to do. 

As\marginnote{12.1} the wind cannot stir \\
a solid mass of rock, \\
so too sights, tastes, sounds, \\
smells, and touches—the lot—

and\marginnote{13.1} ideas, whether liked or disliked, \\
don’t disturb the unaffected one. \\
Their mind is steady and unfettered \\
as they observe disappearance. 

%
\end{verse}

%
\addtocontents{toc}{\let\protect\contentsline\protect\nopagecontentsline}
\part*{The Book of the Fourteens }
\addcontentsline{toc}{part}{The Book of the Fourteens }
\markboth{}{}
\addtocontents{toc}{\let\protect\contentsline\protect\oldcontentsline}

%
\addtocontents{toc}{\let\protect\contentsline\protect\nopagecontentsline}
\chapter*{Chapter One }
\addcontentsline{toc}{chapter}{\tocchapterline{Chapter One }}
\addtocontents{toc}{\let\protect\contentsline\protect\oldcontentsline}

%
\section*{{\suttatitleacronym Thag 14.1}{\suttatitletranslation Revata of the Acacia Wood }{\suttatitleroot Khadiravaniyarevatattheragāthā}}
\addcontentsline{toc}{section}{\tocacronym{Thag 14.1} \toctranslation{Revata of the Acacia Wood } \tocroot{Khadiravaniyarevatattheragāthā}}
\markboth{Revata of the Acacia Wood }{Khadiravaniyarevatattheragāthā}
\extramarks{Thag 14.1}{Thag 14.1}

\begin{verse}%
Since\marginnote{1.1} I’ve gone forth \\
from the lay life to homelessness, \\
I’ve not been aware of any thought \\
that is ignoble and hateful. 

“May\marginnote{2.1} these beings be killed! \\
May they be slaughtered! May they suffer!”—\\
I’ve not been aware of any such thought \\
in all this long while. 

I\marginnote{3.1} have been aware of loving-kindness, \\
limitless and well-developed; \\
gradually consolidated \\
as it was taught by the Buddha. 

I’m\marginnote{4.1} friend and comrade to all, \\
sympathetic for all beings! \\
I develop a mind of love, \\
always delighting in harmlessness. 

The\marginnote{5.1} unfaltering, the unshakable: \\
I gladden that mind. \\
I develop a divine meditation, \\
which sinners do not cultivate. 

When\marginnote{6.1} in a meditation  free of placing the mind, \\
a disciple of the Buddha \\
is at that moment blessed \\
with noble silence. 

As\marginnote{7.1} a rocky mountain \\
is unwavering and well grounded, \\
so when delusion ends, \\
a monk, like a mountain, doesn’t tremble. 

To\marginnote{8.1} the man who has not a blemish \\
who is always seeking purity, \\
even a hair-tip of evil \\
seems as big as a cloud. 

As\marginnote{9.1} a frontier city \\
is guarded inside and out, \\
so you should ward yourselves—\\
don’t let the moment pass you by. 

I\marginnote{10.1} don’t long for death; \\
I don’t long for life; \\
I await my time, \\
like a worker waiting for their wages. 

I\marginnote{11.1} don’t long for death; \\
I don’t long for life; \\
I await my time, \\
aware and mindful. 

I’ve\marginnote{12.1} served the teacher \\
and fulfilled the Buddha’s instructions. \\
The heavy burden is laid down, \\
the conduit to rebirth is eradicated. 

I’ve\marginnote{13.1} attained the goal \\
for the sake of which I went forth \\
from the lay life to homelessness—\\
the ending of all fetters. 

Persist\marginnote{14.1} with diligence: \\
this is my instruction. \\
Come, I’ll be fully quenched—\\
I’m liberated in every way. 

%
\end{verse}

%
\section*{{\suttatitleacronym Thag 14.2}{\suttatitletranslation Godatta }{\suttatitleroot Godattattheragāthā}}
\addcontentsline{toc}{section}{\tocacronym{Thag 14.2} \toctranslation{Godatta } \tocroot{Godattattheragāthā}}
\markboth{Godatta }{Godattattheragāthā}
\extramarks{Thag 14.2}{Thag 14.2}

\begin{verse}%
Just\marginnote{1.1} as a fine thoroughbred, \\
yoked to a carriage, endures the load. \\
Though oppressed by the heavy burden, \\
it doesn’t shake off the yoke. 

So\marginnote{2.1} too, those who are as full of wisdom \\
as the ocean is with water, \\
don’t look down on others; \\
this is the teaching of the noble ones \\>for living creatures. 

People\marginnote{3.1} who fall under the sway of time, \\
the sway of rebirth in this or that state, \\
undergo suffering, \\
and those young men grieve in this life. 

Elated\marginnote{4.1} by things that bring happiness, \\
downcast by things that bring suffering: \\
this pair destroys the fool \\
who doesn’t see things as they are. 

But\marginnote{5.1} those who in suffering, and in happiness, \\
and in the middle have overcome the weaver—\\
they stand like a boundary pillar, \\
neither elated nor downcast. 

Not\marginnote{6.1} to gain nor loss, \\
not to fame nor reputation, \\
not to criticism nor praise, \\
not to suffering nor happiness—

the\marginnote{7.1} wise cling to nothing, \\
like a droplet on a lotus-leaf. \\
They are happy everywhere, \\
and victorious everywhere. 

There’s\marginnote{8.1} legitimate loss, \\
and there’s illegitimate gain. \\
Legitimate loss is better \\
than illegitimate gain. 

There’s\marginnote{9.1} the fame of the unintelligent, \\
and there’s the disrepute of the discerning. \\
The disrepute of the discerning is better \\
than the fame of the unintelligent. 

There’s\marginnote{10.1} praise by simpletons, \\
and there’s criticism by the discerning. \\
Criticism by the discerning is better \\
than praise by fools. 

There’s\marginnote{11.1} the happiness of sensual pleasures, \\
and there’s the suffering of seclusion. \\
The suffering of seclusion is better \\
than the happiness of sensual pleasures. 

There’s\marginnote{12.1} life without principles, \\
and there’s death with principles. \\
Death with principles is better \\
than life without principles. 

Those\marginnote{13.1} who’ve given up desire and anger, \\
their minds at peace \\>regarding rebirth in this or that state, \\
wander in the world unattached, \\
for them nothing is beloved or unloved. 

Having\marginnote{14.1} developed the awakening factors, \\
the faculties and the powers, \\
having arrived at ultimate peace, \\
the undefiled become fully quenched. 

%
\end{verse}

%
\addtocontents{toc}{\let\protect\contentsline\protect\nopagecontentsline}
\part*{The Book of the Sixteens }
\addcontentsline{toc}{part}{The Book of the Sixteens }
\markboth{}{}
\addtocontents{toc}{\let\protect\contentsline\protect\oldcontentsline}

%
\addtocontents{toc}{\let\protect\contentsline\protect\nopagecontentsline}
\chapter*{Chapter One }
\addcontentsline{toc}{chapter}{\tocchapterline{Chapter One }}
\addtocontents{toc}{\let\protect\contentsline\protect\oldcontentsline}

%
\section*{{\suttatitleacronym Thag 15.1}{\suttatitletranslation Koṇḍañña Who Understood }{\suttatitleroot Aññāsikoṇḍaññattheragāthā}}
\addcontentsline{toc}{section}{\tocacronym{Thag 15.1} \toctranslation{Koṇḍañña Who Understood } \tocroot{Aññāsikoṇḍaññattheragāthā}}
\markboth{Koṇḍañña Who Understood }{Aññāsikoṇḍaññattheragāthā}
\extramarks{Thag 15.1}{Thag 15.1}

\begin{verse}%
“My\marginnote{1.1} confidence grew \\
as I heard the teaching, so full of flavor. \\
Dispassion is what was taught, \\
without any grasping at all.” 

“There\marginnote{2.1} are so many pretty things \\
in this vast territory. \\
They disturb one’s thoughts, it seems to me, \\
attractive, provoking lust. 

Just\marginnote{3.1} as a rain cloud would settle \\
the dust blown up by the wind, \\
so thoughts settle down \\
when seen with wisdom. 

All\marginnote{4.1} conditions are impermanent—\\
when this is seen with wisdom \\
one grows disillusioned with suffering: \\
this is the path to purity. 

All\marginnote{5.1} conditions are suffering—\\
when this is seen with wisdom \\
one grows disillusioned with suffering: \\
this is the path to purity. 

All\marginnote{6.1} things are not-self—\\
when this is seen with wisdom \\
one grows disillusioned with suffering: \\
this is the path to purity.” 

“The\marginnote{7.1} senior monk who was awakened \\>right after the Buddha,\footnote{The first two lines of this verse extolling \textsanskrit{Koṇḍañña} were spoken by \textsanskrit{Vaṅgīsa} at \href{https://suttacentral.net/sn8.9/en/sujato\#3.1}{SN 8.9:3.1} and \href{https://suttacentral.net/thag21.1/en/sujato\#38.1}{Thag 21.1:38.1}. } \\
\textsanskrit{Koṇḍañña}, is keenly energetic. \\
He has given up birth and death, \\
and has completed the spiritual journey.” 

“There\marginnote{8.1} are floods, snares, and strong posts, \\
and a mountain hard to crack; \\
snapping the posts and snares, \\
breaking the mountain so hard to break, \\
crossing over to the far shore, \\
a meditator is freed from \textsanskrit{Māra}’s bonds. 

When\marginnote{9.1} a mendicant is haughty and fickle, \\
relying on bad friends, \\
they sink down in the great flood, \\
overcome by a wave. 

But\marginnote{10.1} one not restless or fickle, \\
alert, with senses restrained, \\
intelligent, with good friends, \\
makes an end of suffering. 

With\marginnote{11.1} knobbly knees, \\
thin and veiny, \\
eating and drinking in moderation—\\
this person’s spirit is undaunted. 

Pestered\marginnote{12.1} by flies and mosquitoes \\
in the wilds, the formidable forest, \\
one should mindfully endure, \\
like an elephant at the head of the battle. 

I\marginnote{13.1} don’t long for death; \\
I don’t long for life; \\
I await my time, \\
like a worker waiting for their wages. 

I\marginnote{14.1} don’t long for death; \\
I don’t long for life; \\
I await my time, \\
aware and mindful. 

I’ve\marginnote{15.1} served the teacher \\
and fulfilled the Buddha’s instructions. \\
The heavy burden is laid down, \\
the conduit to rebirth is eradicated. 

I’ve\marginnote{16.1} attained the goal \\
for the sake of which I went forth \\
from the lay life to homelessness—\\
what use do I have for protégés?” 

%
\end{verse}

%
\section*{{\suttatitleacronym Thag 15.2}{\suttatitletranslation Udāyī }{\suttatitleroot Udāyittheragāthā}}
\addcontentsline{toc}{section}{\tocacronym{Thag 15.2} \toctranslation{Udāyī } \tocroot{Udāyittheragāthā}}
\markboth{Udāyī }{Udāyittheragāthā}
\extramarks{Thag 15.2}{Thag 15.2}

\begin{verse}%
Awakened\marginnote{1.1} as a human being, \\
self-tamed and immersed in \textsanskrit{samādhi}, \\
following the spiritual path, \\
he loves peace of mind. 

Revered\marginnote{2.1} by people, \\
gone beyond all things, \\
even the gods revere him; \\
so I’ve heard from the perfected one. 

He\marginnote{3.1} has transcended all fetters, \\
and escaped from entanglements. \\
Delighting to renounce sensual pleasures, \\
he’s freed like lustrous gold from stone. 

That\marginnote{4.1} giant outshines all, \\
like the Himalaya beside other mountains. \\
Of all those named “giant”, \\
he is truly named, supreme. 

I\marginnote{5.1} shall extol the giant for you, \\
for he does nothing monstrous. \\
Sweetness and harmlessness \\
are two feet of the giant. 

Mindfulness\marginnote{6.1} and awareness \\
are his two other feet. \\
Faith is the giant’s trunk, \\
and equanimity his white tusks. 

Mindfulness\marginnote{7.1} is his neck, his head is wisdom—\\
investigation and thinking about principles. \\
His belly is the sacred hearth of the Dhamma, \\
and his tail is seclusion. 

Practicing\marginnote{8.1} absorption, enjoying the breath, \\
he is serene within. \\
The giant is serene when walking, \\
the giant is serene when standing, 

the\marginnote{9.1} giant is serene when lying down, \\
and when sitting, the giant is serene. \\
The giant is restrained everywhere: \\
this is the accomplishment of the giant. 

He\marginnote{10.1} eats blameless things, \\
he doesn’t eat blameworthy things. \\
When he gets food and clothes, \\
he avoids storing them up. 

Having\marginnote{11.1} severed all bonds, \\
fetters large and small, \\
wherever he goes, \\
he goes without concern. 

A\marginnote{12.1} white lotus, \\
fragrant and delightful, \\
sprouts in water and grows there, \\
but the water doesn’t cling to it. 

Just\marginnote{13.1} so the Buddha is born in the world, \\
and lives in the world, \\
but the world doesn’t stick to him, \\
as water does not stick to the lotus. 

A\marginnote{14.1} great blazing fire \\
dies down when the fuel runs out. \\
And when the coals have gone out \\
it’s said to be “quenched”. 

This\marginnote{15.1} simile is taught by the discerning \\
to express the meaning clearly. \\
Great giants will understand \\
what the giant taught the giant. 

Free\marginnote{16.1} of greed, free of hate, \\
free of delusion, undefiled; \\
the giant, giving up his body, \\
undefiled, will be fully quenched. 

%
\end{verse}

%
\addtocontents{toc}{\let\protect\contentsline\protect\nopagecontentsline}
\part*{The Book of the Twenties }
\addcontentsline{toc}{part}{The Book of the Twenties }
\markboth{}{}
\addtocontents{toc}{\let\protect\contentsline\protect\oldcontentsline}

%
\addtocontents{toc}{\let\protect\contentsline\protect\nopagecontentsline}
\chapter*{Chapter One }
\addcontentsline{toc}{chapter}{\tocchapterline{Chapter One }}
\addtocontents{toc}{\let\protect\contentsline\protect\oldcontentsline}

%
\section*{{\suttatitleacronym Thag 16.1}{\suttatitletranslation Adhimutta (2nd) }{\suttatitleroot Adhimuttattheragāthā}}
\addcontentsline{toc}{section}{\tocacronym{Thag 16.1} \toctranslation{Adhimutta (2nd) } \tocroot{Adhimuttattheragāthā}}
\markboth{Adhimutta (2nd) }{Adhimuttattheragāthā}
\extramarks{Thag 16.1}{Thag 16.1}

\begin{verse}%
“Those\marginnote{1.1} who we killed in the past, \\
whether for sacrifice or for wealth, \\
without exception were afraid; \\
they trembled and they squealed. 

But\marginnote{2.1} you’re not scared; \\
you look even calmer than before. \\
Why don’t you cry out \\
in such a terrifying situation?” 

“There\marginnote{3.1} isn’t any mental suffering \\
for one without hope, village chief. \\
All fears are left behind \\
by one whose fetters have ended. 

When\marginnote{4.1} the conduit to rebirth is ended, \\
and the truth is seen as it is, \\
there is no fear of death; \\
it’s like laying down a burden. 

I’ve\marginnote{5.1} lived the spiritual life well, \\
and developed the path well, too. \\
I do not fear death; \\
it’s like the passing of a disease. 

I’ve\marginnote{6.1} lived the spiritual life well, \\
and developed the path well, too. \\
I’ve seen that there’s nothing gratifying in existences, \\
like someone who has tasted poison, \\>then thrown it out. 

One\marginnote{7.1} who has gone beyond, without grasping, \\
they’ve completed the task \\>and are free of defilements. \\
They are content at the end of life, \\
like someone released from execution. 

Having\marginnote{8.1} realized the supreme Dhamma, \\
without needing anything in the whole world, \\
one doesn’t grieve at death; \\
for it’s like escaping from a burning house. 

Whatever\marginnote{9.1} has come to pass, \\
wherever life is obtained, \\
there is no Lord of all that: \\
so said the great seer. 

Whoever\marginnote{10.1} understands this \\
as it was taught by the Buddha \\
doesn’t grab on to any new life, \\
like you wouldn’t grab a hot iron ball. 

It\marginnote{11.1} doesn’t occur to me, ‘I existed in the past’; \\
nor, ‘I will exist in the future’. \\
All conditions will disappear—\\
why weep over that? 

Seeing\marginnote{12.1} truly \\
the bare arising of phenomena, \\
and the bare process of conditions, \\
there is no fear, village chief. 

The\marginnote{13.1} world is like grass and sticks: \\
when this is seen with wisdom, \\
not finding anything to be mine, \\
you won’t grieve, thinking ‘I don’t have it’. 

I’m\marginnote{14.1} fed up with the body; \\
I have no need for another life. \\
This body will be broken up, \\
and there won’t be another. 

Do\marginnote{15.1} what you want \\
with my corpse. \\
I won’t be angry or attached \\
on account of that.” 

When\marginnote{16.1} they heard these words, \\
so astonishing and hair-raising, \\
the young men laid down their swords \\
and spoke these words: 

“What\marginnote{17.1} have you practiced, Venerable? \\
And who is your tutor? \\
Whose instructions do we follow \\
to gain the sorrowless state?” 

“The\marginnote{18.1} knower of all, the seer of all: \\
the victor is my tutor. \\
He is a Teacher of great compassion, \\
healer of the whole world. 

He\marginnote{19.1} taught this Dhamma, \\
leading to ending, unsurpassed. \\
Following his instructions, \\
you can gain the sorrowless state.” 

When\marginnote{20.1} the bandits heard the good words of the seer, \\
they laid down their swords and weapons. \\
Some refrained from their former deeds, \\
while others chose the going-forth. 

Having\marginnote{21.1} gone forth in the teaching of the Holy One, \\
those astute ones developed \\>the awakening factors and the powers. \\
Elated, happy, their faculties complete, \\
they realized the state of extinguishment, the unconditioned. 

%
\end{verse}

%
\section*{{\suttatitleacronym Thag 16.2}{\suttatitletranslation Pārāsariya (2nd) }{\suttatitleroot Pārāpariyattheragāthā}}
\addcontentsline{toc}{section}{\tocacronym{Thag 16.2} \toctranslation{Pārāsariya (2nd) } \tocroot{Pārāpariyattheragāthā}}
\markboth{Pārāsariya (2nd) }{Pārāpariyattheragāthā}
\extramarks{Thag 16.2}{Thag 16.2}

\begin{verse}%
This\marginnote{1.1} thought came to the ascetic, \\
the monk \textsanskrit{Pārāsariya}, \\
as he was seated alone \\
meditating in seclusion: 

“Following\marginnote{2.1} what procedure, \\
what observance, what conduct, \\
may a person do what they need for themselves, \\
without harming anyone else? 

The\marginnote{3.1} faculties of human beings \\
can lead to both welfare and harm. \\
Unguarded they lead to harm; \\
guarded they lead to welfare. 

By\marginnote{4.1} protecting the faculties, \\
taking care of the faculties, \\
I can do what I need for myself \\
without harming anyone else. 

If\marginnote{5.1} your eye wanders \\
among sights without check, \\
not seeing the danger, \\
you’re not freed from suffering. 

If\marginnote{6.1} your ear wanders \\
among sounds without check, \\
not seeing the danger, \\
you’re not freed from suffering. 

If,\marginnote{7.1} not seeing the escape, \\
you indulge in a smell, \\
you’re not freed from suffering, \\
being besotted by smells. 

Recollecting\marginnote{8.1} the sour, \\
the sweet and the bitter, \\
captivated by craving for taste, \\
you don’t understand the heart. 

Recollecting\marginnote{9.1} lovely \\
and pleasurable touches, \\
full of desire, you experience \\
many kinds of suffering because of lust. 

Unable\marginnote{10.1} to protect \\
the mind from such thoughts, \\
suffering follows them \\
because of all five. 

This\marginnote{11.1} body is full of pus and blood, \\
it’s home to many carcasses; \\
but cunning people decorate it \\
like a lovely painted casket. 

You\marginnote{12.1} don’t understand that \\
the sweetness of honey turns bitter, \\
and the bonds to those we love cause pain, \\
like a razor’s edge smeared with honey. 

Full\marginnote{13.1} of lust for the sight of a woman, \\
for the voice and the smells of a woman, \\
for a woman’s touch, \\
you experience many kinds of suffering. 

All\marginnote{14.1} of a woman’s streams \\
flow from five to five. \\
Whoever, being energetic, \\
is able to curb these, 

purposeful\marginnote{15.1} and firm in principle, \\
is clever and clear-seeing. \\
Though he might enjoy himself, \\
his duty is connected with the teaching and its goal. 

One\marginnote{16.1} who’s diligent and discerning, \\
thinking, “This ought not be done”, \\
would avoid a useless task \\
that’s doomed to failure. 

Whatever\marginnote{17.1} is meaningful, \\
and whatever happiness is principled, \\
let one undertake and follow that: \\
this is the best happiness. 

They\marginnote{18.1} want to get hold of what belongs to others \\
by any means, fair or foul. \\
They kill, injure, and torment, \\
violently plundering what belongs to others. 

Just\marginnote{19.1} as a strong person when building \\
knocks out a peg with a peg, \\
so the skillful person \\
knocks out the faculties with the faculties. 

Developing\marginnote{20.1} faith, energy, immersion, \\
mindfulness, and wisdom; \\
destroying the five with the five, \\
the brahmin walks on untroubled. 

Purposeful\marginnote{21.1} and firm in principle, \\
having fulfilled in every respect \\
the instructions spoken by the Buddha, \\
that person prospers in happiness.” 

%
\end{verse}

%
\section*{{\suttatitleacronym Thag 16.3}{\suttatitletranslation Telakāni }{\suttatitleroot Telakānittheragāthā}}
\addcontentsline{toc}{section}{\tocacronym{Thag 16.3} \toctranslation{Telakāni } \tocroot{Telakānittheragāthā}}
\markboth{Telakāni }{Telakānittheragāthā}
\extramarks{Thag 16.3}{Thag 16.3}

\begin{verse}%
For\marginnote{1.1} a long time, sadly, \\
though I keenly contemplated the teaching, \\
I gained no peace of mind. \\
So I asked this of ascetics and brahmins: 

“Who\marginnote{2.1} in the world have crossed over? \\
Whose attainment culminates in freedom from death? \\
Whose teaching do I accept \\
to understand the ultimate goal? 

I\marginnote{3.1} was hooked inside, \\
like a fish gulping bait; \\
bound like the titan Vepaciti \\
in Mahinda’s trap.\footnote{Mahinda (“Great Indra”) is a name of Sakka, the nemesis of Vepaciti. } 

Dragging\marginnote{4.1} it along, I’m not free \\
from grief and lamentation. \\
Who will free me from bonds in the world, \\
so that I may know awakening? 

What\marginnote{5.1} ascetic or brahmin \\
points out what is frail? \\
Whose teaching do I accept \\
to sweep away old age and death? 

Tied\marginnote{6.1} up with uncertainty and doubt, \\
secured by the power of aggression, \\
stiff as a mind beset by anger; \\
the arrow of covetousness, 

propelled\marginnote{7.1} by the bow of craving, \\
is stuck in my twice-fifteen ribcage—\footnote{The commentary explains “twice-fifteen” as the twenty kinds of substantialist view and ten kinds of wrong view. This is implausible, as it clearly has an organic sense. I count it as 24 ribs, plus two bones each for the collar-bones, the shoulder-blades, and the sternum, making thirty in total. The sternum is technically one bone, but it is divided into upper (manubrium) and lower (gladiolus) with a notch between. } \\
see how it stands in my breast, \\
breaking my strong heart. 

Speculative\marginnote{8.1} views are not abandoned, \\
they are sharpened by memories and intentions;\footnote{Read \textit{\textsanskrit{saṅkappasara}} per PTS for MS’s \textit{\textsanskrit{saṅkappapara}}. } \\
and pierced by this I tremble, \\
like a leaf blowing in the gale. 

Having\marginnote{9.1} arisen within, \\
what belongs to me burns quickly, \\
in that place where the body always heads \\
with its six sense-fields of contact. 

I\marginnote{10.1} don’t see a healer \\
who can pull out my dart of doubt \\
without a lance \\
or some other blade. 

Without\marginnote{11.1} knife or wound, \\
who will pull out this dart \\
that’s stuck inside me, \\
without harming any part of my body? 

He\marginnote{12.1} really would be the Lord of the Dhamma, \\
the best one to cure the damage of poison; \\
when I have fallen into deep waters, \\
he would show me his hand and the shore. 

I’ve\marginnote{13.1} plunged into a lake, \\
and I can’t wash off the mud and dirt. \\
It’s full of fraud, jealousy, aggression, \\
and dullness and drowsiness. 

Like\marginnote{14.1} a thunder-cloud of restlessness, \\
like a rain cloud of fetters; \\
lustful thoughts are winds \\
that sweep off a person with bad views. 

The\marginnote{15.1} streams flow everywhere; \\
a weed springs up and remains. \\
Who will block the streams? \\
Who will cut the weed?” 

“Venerable\marginnote{16.1} sir, build a dam \\
to block the streams. \\
Don’t let your mind-made streams \\
cut you down suddenly like a tree.” 

That\marginnote{17.1} is how, when I was full of fear, \\
seeking the far shore from the near, \\
my shelter was the teacher \\>whose weapon is wisdom, \\
surrounded by the \textsanskrit{Saṅgha} of seers. 

As\marginnote{18.1} I was being swept away, \\
he gave me a strong, simple ladder, \\
made of the heartwood of Dhamma, \\
and he said to me: “Do not fear.” 

I\marginnote{19.1} climbed the tower of mindfulness meditation, \\
and checked back down \\
at people delighting in substantial reality, \\
as I had obsessed over in the past. 

When\marginnote{20.1} I saw the path, \\
as I was embarking on the ship, \\
without fixating on the self, \\
I saw the supreme landing-place. 

The\marginnote{21.1} dart that arises in oneself, \\
and that which stems from the conduit to rebirth: \\
he taught the supreme path \\
for the canceling of these. 

For\marginnote{22.1} a long time it had lain within me; \\
for a long time it was fixed in me: \\
the Buddha cast off the knot, \\
curing the damage of poison. 

%
\end{verse}

%
\section*{{\suttatitleacronym Thag 16.4}{\suttatitletranslation Raṭṭhapāla }{\suttatitleroot Raṭṭhapālattheragāthā}}
\addcontentsline{toc}{section}{\tocacronym{Thag 16.4} \toctranslation{Raṭṭhapāla } \tocroot{Raṭṭhapālattheragāthā}}
\markboth{Raṭṭhapāla }{Raṭṭhapālattheragāthā}
\extramarks{Thag 16.4}{Thag 16.4}

\begin{verse}%
“See\marginnote{1.1} this fancy puppet, \\
a body built of sores, \\
diseased, obsessed over, \\
in which nothing lasts at all. 

See\marginnote{2.1} this fancy figure, \\
with its gems and earrings; \\
it is bones encased in skin, \\
made pretty by its clothes. 

Rouged\marginnote{3.1} feet \\
and powdered face \\
may be enough to beguile a fool, \\
but not a seeker of the far shore. 

Hair\marginnote{4.1} in eight braids \\
and eyeshadow \\
may be enough to beguile a fool, \\
but not a seeker of the far shore. 

A\marginnote{5.1} rotting body all adorned \\
like a freshly painted makeup box \\
may be enough to beguile a fool, \\
but not a seeker of the far shore. 

The\marginnote{6.1} hunter laid his snare, \\
but the deer didn’t spring the trap. \\
I’ve eaten the bait and now I go, \\
leaving the trapper to lament. 

The\marginnote{7.1} hunter’s snare is broken, \\
but the deer didn’t spring the trap. \\
I’ve eaten the bait and now I go, \\
leaving the deer-hunter to grieve.” 

“I\marginnote{8.1} see rich people in the world who, \\
because of delusion, give not the money they’ve earned. \\
Greedily, they hoard their riches, \\
yearning for ever more sensual pleasures. 

A\marginnote{9.1} king who conquered the earth by force, \\
ruling the land from sea to sea, \\
unsatisfied with the near shore of the ocean, \\
would still yearn for the further shore. 

Not\marginnote{10.1} just the king, but others too, \\
reach death not rid of craving. \\
They leave the body still wanting, \\
for in this world sensual pleasures never satisfy. 

Relatives\marginnote{11.1} lament, their hair disheveled, \\
saying ‘Ah! Alas! They’re not immortal!’ \\
They take out the body wrapped in a shroud, \\
heap up a pyre, and burn it there. 

It’s\marginnote{12.1} poked with stakes while being burnt, \\
in just a single cloth, all wealth gone. \\
Relatives, friends, and companions \\
can’t help you when you’re dying. 

Heirs\marginnote{13.1} take your riches, \\
while beings fare on according to their deeds. \\
Riches don’t follow you when you die; \\
nor do children, wife, wealth, nor kingdom. 

Longevity\marginnote{14.1} isn’t gained by riches, \\
nor does wealth banish old age; \\
for the attentive say this life is short, \\
it’s perishable and not eternal. 

The\marginnote{15.1} rich and the poor feel its touch; \\
the fool and the attentive one feel it too. \\
But the fool lies stricken by their own folly, \\
while the attentive don’t tremble at the touch. 

Therefore\marginnote{16.1} wisdom’s much better than wealth, \\
since by wisdom \\>you reach consummation in this life. \\
But if because of delusion \\>you don’t reach consummation, \\
you’ll do evil deeds in life after life. 

One\marginnote{17.1} who enters a womb and the world beyond, \\
will transmigrate from one life to the next. \\
While someone of little wisdom, \\>placing faith in them, \\
also enters a womb and the world beyond. 

As\marginnote{18.1} a bandit caught in a window \\
is punished for his own bad deeds; \\
so after departing, in the world beyond, \\
people are punished for their own bad deeds. 

Sensual\marginnote{19.1} pleasures are diverse, sweet, delightful, \\
appearing in disguise they disturb the mind. \\
Seeing danger in sensual stimulations, \\
I went forth, O King. 

As\marginnote{20.1} fruit falls from a tree, so people fall, \\
young and old, when the body breaks up. \\
Seeing this, too, I went forth, O King; \\
the ascetic life is unfailingly better.” 

“I\marginnote{21.1} went forth out of faith \\
joining the victor’s dispensation. \\
My going forth wasn’t wasted; \\
I enjoy my food free of debt. 

I\marginnote{22.1} saw sensual pleasures as burning, \\
gold as a cutting blade, \\
conception in a womb as suffering, \\
and the hells as very fearful. 

Knowing\marginnote{23.1} this danger, \\
I was struck with a sense of urgency. \\
I was stabbed, but then I found peace, \\
attaining the end of defilements. 

I’ve\marginnote{24.1} served the teacher \\
and fulfilled the Buddha’s instructions. \\
The heavy burden is laid down, \\
the conduit to rebirth is eradicated. 

I’ve\marginnote{25.1} reached the goal \\
for the sake of which I went forth \\
from the lay life to homelessness—\\
the ending of all fetters.” 

%
\end{verse}

%
\section*{{\suttatitleacronym Thag 16.5}{\suttatitletranslation Māluṅkyaputta (2nd) }{\suttatitleroot Mālukyaputtattheragāthā}}
\addcontentsline{toc}{section}{\tocacronym{Thag 16.5} \toctranslation{Māluṅkyaputta (2nd) } \tocroot{Mālukyaputtattheragāthā}}
\markboth{Māluṅkyaputta (2nd) }{Mālukyaputtattheragāthā}
\extramarks{Thag 16.5}{Thag 16.5}

\begin{verse}%
When\marginnote{1.1} you see a sight, mindfulness is lost \\
as you focus on a pleasant feature. \\
Experiencing it with a mind full of desire, \\
you keep clinging to it. 

Many\marginnote{2.1} feelings grow \\
arising from sights. \\
The mind is damaged \\
by covetousness and cruelty. \\
Heaping up suffering like this, \\
you’re said to be far from extinguishment. 

When\marginnote{3.1} you hear a sound, mindfulness is lost \\
as you focus on a pleasant feature. \\
Experiencing it with a mind full of desire, \\
you keep clinging to it. 

Many\marginnote{4.1} feelings grow \\
arising from sounds. \\
The mind is damaged \\
by covetousness and cruelty. \\
Heaping up suffering like this, \\
you’re said to be far from extinguishment. 

When\marginnote{5.1} you smell an odor, mindfulness is lost \\
as you focus on a pleasant feature. \\
Experiencing it with a mind full of desire, \\
you keep clinging to it. 

Many\marginnote{6.1} feelings grow \\
arising from smells. \\
The mind is damaged \\
by covetousness and cruelty. \\
Heaping up suffering like this, \\
you’re said to be far from extinguishment. 

When\marginnote{7.1} you enjoy a taste, mindfulness is lost \\
as you focus on a pleasant feature. \\
Experiencing it with a mind full of desire, \\
you keep clinging to it. 

Many\marginnote{8.1} feelings grow \\
arising from tastes. \\
The mind is damaged \\
by covetousness and cruelty. \\
Heaping up suffering like this, \\
you’re said to be far from extinguishment. 

When\marginnote{9.1} you sense a touch, mindfulness is lost \\
as you focus on a pleasant feature. \\
Experiencing it with a mind full of desire, \\
you keep clinging to it. 

Many\marginnote{10.1} feelings grow \\
arising from touches. \\
The mind is damaged \\
by covetousness and cruelty. \\
Heaping up suffering like this, \\
you’re said to be far from extinguishment. 

When\marginnote{11.1} you know an idea, mindfulness is lost \\
as you focus on a pleasant feature. \\
Experiencing it with a mind full of desire, \\
you keep clinging to it. 

Many\marginnote{12.1} feelings grow \\
arising from ideas. \\
The mind is damaged \\
by covetousness and cruelty. \\
Heaping up suffering like this, \\
you’re said to be far from extinguishment. 

There’s\marginnote{13.1} no desire for sights \\
when you see a sight with mindfulness. \\
Experiencing it with a mind free of desire, \\
you don’t keep clinging to it. 

Even\marginnote{14.1} as you see a sight \\
and undergo a feeling, \\
you wear away, you don’t heap up: \\
that’s how to live mindfully. \\
Eroding suffering like this, \\
you’re said to be in the presence of extinguishment. 

There’s\marginnote{15.1} no desire for sounds \\
when you hear a sound with mindfulness. \\
Experiencing it with a mind free of desire, \\
you don’t keep clinging to it. 

Even\marginnote{16.1} as you hear a sound \\
and undergo a feeling, \\
you wear away, you don’t heap up: \\
that’s how to live mindfully. \\
Eroding suffering like this, \\
you’re said to be in the presence of extinguishment. 

There’s\marginnote{17.1} no desire for odors \\
when you smell an odor with mindfulness. \\
Experiencing it with a mind free of desire, \\
you don’t keep clinging to it. 

Even\marginnote{18.1} as you smell an odor \\
and undergo a feeling, \\
you wear away, you don’t heap up: \\
that’s how to live mindfully. \\
Eroding suffering like this, \\
you’re said to be in the presence of extinguishment. 

There’s\marginnote{19.1} no desire for tastes \\
when you enjoy a taste with mindfulness. \\
Experiencing it with a mind free of desire, \\
you don’t keep clinging to it. 

Even\marginnote{20.1} as you savor a taste \\
and undergo a feeling, \\
you wear away, you don’t heap up: \\
that’s how to live mindfully. \\
Eroding suffering like this, \\
you’re said to be in the presence of extinguishment. 

There’s\marginnote{21.1} no desire for touches \\
when you sense a touch with mindfulness. \\
Experiencing it with a mind free of desire, \\
you don’t keep clinging to it. 

Even\marginnote{22.1} as you sense a touch \\
and get familiar with how it feels, \\
you wear away, you don’t heap up: \\
that’s how to live mindfully. \\
Eroding suffering like this, \\
you’re said to be in the presence of extinguishment. 

There’s\marginnote{23.1} no desire for ideas \\
when you know an idea with mindfulness. \\
Experiencing it with a mind free of desire, \\
you don’t keep clinging to it. 

Even\marginnote{24.1} as you know an idea \\
and get familiar with how it feels, \\
you wear away, you don’t heap up: \\
that’s how to live mindfully. \\
Eroding suffering like this, \\
you’re said to be in the presence of extinguishment. 

%
\end{verse}

%
\section*{{\suttatitleacronym Thag 16.6}{\suttatitletranslation Sela }{\suttatitleroot Selattheragāthā}}
\addcontentsline{toc}{section}{\tocacronym{Thag 16.6} \toctranslation{Sela } \tocroot{Selattheragāthā}}
\markboth{Sela }{Selattheragāthā}
\extramarks{Thag 16.6}{Thag 16.6}

\begin{verse}%
“O\marginnote{1.1} Blessed One, your body’s perfect, \\
you’re radiant, handsome, lovely to behold; \\
golden colored, \\
with teeth so white; you’re strong. 

The\marginnote{2.1} characteristics \\
of a handsome man, \\
the marks of a great man, \\
are all found on your body. 

Your\marginnote{3.1} eyes are clear, your face is fair, \\
you’re formidable, sincere, majestic. \\
In the midst of the \textsanskrit{Saṅgha} of ascetics, \\
you shine like the sun. 

You’re\marginnote{4.1} a mendicant fine to see, \\
with skin that shines like lustrous gold. \\
But with such excellent appearance, \\
what do you want with the ascetic life? 

You’re\marginnote{5.1} fit to be a king, \\
a wheel-turning monarch, chief of charioteers, \\
victorious in the four quarters, \\
lord of the Black Plum Tree Land. 

Aristocrats,\marginnote{6.1} nobles, and kings \\
ought follow your rule. \\
Gotama, you should reign \\
as king of kings, lord of men!” 

“I\marginnote{7.1} am a king, Sela”, \\
\scspeaker{said the Buddha to Sela, }\\
“the supreme king of the teaching. \\
By the teaching I roll forth the wheel \\
which cannot be rolled back.”\footnote{This is of course a reference to the first sermon (\href{https://suttacentral.net/sn56.11/en/sujato}{SN 56.11}). } 

“You\marginnote{8.1} claim to be awakened,” \\
\scspeaker{said Sela the brahmin, }\\
“the supreme king of the teaching. \\
‘I roll forth the teaching’: \\
so you say, Gotama. 

Then\marginnote{9.1} who is your general,\footnote{The title “General of the Dhamma” belongs to \textsanskrit{Sāriputta} (\href{https://suttacentral.net/ud2.8/en/sujato\#16.2}{Ud 2.8:16.2}, \href{https://suttacentral.net/thag18.1/en/sujato\#33.1}{Thag 18.1:33.1}). } \\
the disciple who follows the Teacher’s way? \\
Who keeps rolling the wheel \\
of the teaching you rolled forth?” 

“By\marginnote{10.1} me the wheel was rolled forth,” \\
\scspeaker{said the Buddha, }\\
“the supreme wheel of the teaching. \\
\textsanskrit{Sāriputta}, taking after the Realized One, \\
keeps it rolling on.\footnote{\textit{\textsanskrit{Anujāta}} is said at \href{https://suttacentral.net/iti74/en/sujato\#4.1}{Iti 74:4.1} to be a child who “takes after” the good qualities of their parents. } 

I\marginnote{11.1} have known what should be known,\footnote{As at \href{https://suttacentral.net/mn91/en/sujato\#31.5}{MN 91:31.5}. } \\
and developed what should be developed, \\
and given up what should be given up: \\
and so, brahmin, I am a Buddha. 

Dispel\marginnote{12.1} your doubt in me—\\
make up your mind, brahmin! \\
The sight of a Buddha \\
is hard to find again.\footnote{\textit{\textsanskrit{Abhiṇhaso}} means “repeatedly”. Here the force of the saying is, I think, “It is hard enough to encounter a Buddha even once, let alone repeatedly.” } 

I\marginnote{13.1} am a Buddha, brahmin, \\
the supreme surgeon, \\
one of those whose appearance in the world \\
is hard to find again. 

A\marginnote{14.1} manifestation of divinity, unequaled, \\
crusher of \textsanskrit{Māra}’s army; \\
having subdued all my opponents, \\
I rejoice, fearing nothing from any quarter.” 

“Pay\marginnote{15.1} heed, sirs, to what \\
is spoken by the Clear-eyed One. \\
The surgeon, the great hero, \\
roars like a lion in the jungle. 

A\marginnote{16.1} manifestation of divinity, unequaled, \\
crusher of \textsanskrit{Māra}’s army; \\
who would not be inspired by him, \\
even one born in a dark class?\footnote{The “dark class” refers to those born in an unfortunate state (\href{https://suttacentral.net/an6.57/en/sujato\#11.1}{AN 6.57:11.1}). Sela is saying that the Buddha’s path is for everyone, not just the fortunate. } 

Those\marginnote{17.1} who wish may follow me; \\
those who don’t may go. \\
Right here, I’ll go forth in his presence, \\
the one of such splendid wisdom.” 

“Sir,\marginnote{18.1} if you endorse \\
the teaching of the Buddha, \\
we’ll also go forth in his presence, \\
the one of such splendid wisdom.” 

“These\marginnote{19.1} three hundred brahmins \\
with joined palms held up, ask: \\
‘May we lead the spiritual life \\
in your presence, Blessed One?’” 

“The\marginnote{20.1} spiritual life is well explained,” \\
\scspeaker{said the Buddha, }\\
“apparent in the present life, immediately effective. \\
Here the going forth isn’t in vain \\
for one who trains with diligence.” 

“This\marginnote{21.1} is the eighth day since \\
we went for refuge, O Clear-eyed One. \\
In these seven days, Blessed One, \\
we’ve become tamed in your teaching. 

You\marginnote{22.1} are the Buddha, you are the Teacher,\footnote{This verse and the next are also at \href{https://suttacentral.net/snp3.6/en/sujato\#49.1}{Snp 3.6:49.1}. } \\
you are the sage who has overcome \textsanskrit{Māra}; \\
you have cut off the underlying tendencies, \\
you’ve crossed over, and you bring humanity across. 

You\marginnote{23.1} have transcended attachments, \\
your defilements are shattered; \\
by not grasping, like a lion, \\
you’ve given up fear and dread. 

These\marginnote{24.1} three hundred mendicants \\
stand with joined palms raised. \\
Stretch out your feet, great hero: \\
let these giants bow to the Teacher.” 

%
\end{verse}

%
\section*{{\suttatitleacronym Thag 16.7}{\suttatitletranslation Kāḷigodhāputtabhaddiya }{\suttatitleroot Kāḷigodhāputtabhaddiyattheragāthā}}
\addcontentsline{toc}{section}{\tocacronym{Thag 16.7} \toctranslation{Kāḷigodhāputtabhaddiya } \tocroot{Kāḷigodhāputtabhaddiyattheragāthā}}
\markboth{Kāḷigodhāputtabhaddiya }{Kāḷigodhāputtabhaddiyattheragāthā}
\extramarks{Thag 16.7}{Thag 16.7}

\begin{verse}%
I\marginnote{1.1} rode on an elephant’s neck, \\
wearing luxurious clothes. \\
I ate boiled fine rice \\
with pure meat sauce. 

Today\marginnote{2.1} I am fortunate, persistent, \\
happy with the scraps in my bowl: \\
Bhaddiya son of \textsanskrit{Godhā} \\
practices absorption without grasping. 

Wearing\marginnote{3.1} rags, persistent, \\
happy with the scraps in my bowl: \\
Bhaddiya son of \textsanskrit{Godhā} \\
practices absorption without grasping. 

Living\marginnote{4.1} on almsfood, persistent, \\
happy with the scraps in my bowl: \\
Bhaddiya son of \textsanskrit{Godhā} \\
practices absorption without grasping. 

Possessing\marginnote{5.1} only three robes, persistent, \\
happy with the scraps in my bowl: \\
Bhaddiya son of \textsanskrit{Godhā} \\
practices absorption without grasping. 

Wandering\marginnote{6.1} for alms indiscriminately, persistent, \\
happy with the scraps in my bowl: \\
Bhaddiya son of \textsanskrit{Godhā} \\
practices absorption without grasping. 

Sitting\marginnote{7.1} alone, persistent, \\
happy with the scraps in my bowl: \\
Bhaddiya son of \textsanskrit{Godhā} \\
practices absorption without grasping. 

Eating\marginnote{8.1} only what’s put in the alms-bowl, persistent, \\
happy with the scraps in my bowl: \\
Bhaddiya son of \textsanskrit{Godhā} \\
practices absorption without grasping. 

Never\marginnote{9.1} eating too late, persistent, \\
happy with the scraps in my bowl: \\
Bhaddiya son of \textsanskrit{Godhā} \\
practices absorption without grasping. 

Living\marginnote{10.1} in the wilderness, persistent, \\
happy with the scraps in my bowl: \\
Bhaddiya son of \textsanskrit{Godhā} \\
practices absorption without grasping. 

Living\marginnote{11.1} at the foot of a tree, persistent, \\
happy with the scraps in my bowl: \\
Bhaddiya son of \textsanskrit{Godhā} \\
practices absorption without grasping. 

Living\marginnote{12.1} in the open, persistent, \\
happy with the scraps in my bowl: \\
Bhaddiya son of \textsanskrit{Godhā} \\
practices absorption without grasping. 

Living\marginnote{13.1} in a charnel ground, persistent, \\
happy with the scraps in my bowl: \\
Bhaddiya son of \textsanskrit{Godhā} \\
practices absorption without grasping. 

Accepting\marginnote{14.1} whatever seat is offered, persistent, \\
happy with the scraps in my bowl: \\
Bhaddiya son of \textsanskrit{Godhā} \\
practices absorption without grasping. 

Not\marginnote{15.1} lying down to sleep, persistent, \\
happy with the scraps in my bowl: \\
Bhaddiya son of \textsanskrit{Godhā} \\
practices absorption without grasping. 

Few\marginnote{16.1} in wishes, persistent, \\
happy with the scraps in my bowl: \\
Bhaddiya son of \textsanskrit{Godhā} \\
practices absorption without grasping. 

Content,\marginnote{17.1} persistent, \\
happy with the scraps in my bowl: \\
Bhaddiya son of \textsanskrit{Godhā} \\
practices absorption without grasping. 

Secluded,\marginnote{18.1} persistent, \\
happy with the scraps in my bowl: \\
Bhaddiya son of \textsanskrit{Godhā} \\
practices absorption without grasping. 

Aloof,\marginnote{19.1} persistent, \\
happy with the scraps in my bowl: \\
Bhaddiya son of \textsanskrit{Godhā} \\
practices absorption without grasping. 

Energetic,\marginnote{20.1} persistent, \\
happy with the scraps in my bowl: \\
Bhaddiya son of \textsanskrit{Godhā} \\
practices absorption without grasping. 

Giving\marginnote{21.1} up a valuable bronze bowl, \\
and a precious golden one, too, \\
I took a bowl made of clay: \\
this is my second initiation. 

I\marginnote{22.1} used to live in a citadel with walls so high, \\
with battlements strong and gates, \\
all guarded by swordsmen—\\
and yet I trembled with fear. 

Today\marginnote{23.1} I am fortunate, free of cowardice, \\
with fear and dread given up. \\
Bhaddiya son of \textsanskrit{Godhā} \\
has plunged into the forest and practices absorption. 

Established\marginnote{24.1} in the entire spectrum of ethics, \\
developing mindfulness and wisdom, \\
gradually I attained \\
the ending of all fetters. 

%
\end{verse}

%
\section*{{\suttatitleacronym Thag 16.8}{\suttatitletranslation Aṅgulimāla }{\suttatitleroot Aṅgulimālattheragāthā}}
\addcontentsline{toc}{section}{\tocacronym{Thag 16.8} \toctranslation{Aṅgulimāla } \tocroot{Aṅgulimālattheragāthā}}
\markboth{Aṅgulimāla }{Aṅgulimālattheragāthā}
\extramarks{Thag 16.8}{Thag 16.8}

\begin{verse}%
“While\marginnote{1.1} walking, ascetic, you say ‘I’ve stopped.’\footnote{The story of \textsanskrit{Aṅgulimāla} is told in more detail in \href{https://suttacentral.net/mn86/en/sujato}{MN 86}, which includes most of the verses found here. } \\
And I have stopped, but you tell me I’ve not. \\
I’m asking you this, ascetic: \\
how is it you’ve stopped and I have not?” 

“\textsanskrit{Aṅgulimāla},\marginnote{2.1} I have forever stopped—\\
I’ve laid aside violence towards all creatures. \\
But you can’t stop yourself \\>from harming living creatures; \\
that’s why I’ve stopped, but you have not.” 

“Oh,\marginnote{3.1} at long last a renowned great seer, \\
an ascetic has followed me into this deep wood. \\
Now that I’ve heard your verse on Dhamma, \\
I shall discard a thousand evils.” 

With\marginnote{4.1} these words, \\>the bandit hurled his sword and weapons \\
down a cliff into an abyss. \\
He venerated the Holy One’s feet, \\
and asked the Buddha for the going forth right away. 

Then\marginnote{5.1} the Buddha, the compassionate great seer, \\
the teacher of the world with its gods, \\
said to him, “Come, monk!” \\
And with that he became a monk. 

“He\marginnote{6.1} who once was heedless, \\
but turned to heedfulness, \\
lights up the world, \\
like the moon freed from clouds. 

Someone\marginnote{7.1} whose bad deed \\
is supplanted by the good, \\
lights up the world, \\
like the moon freed from clouds. 

A\marginnote{8.1} young mendicant \\
devoted to the Buddha’s teaching, \\
lights up the world, \\
like the moon freed from clouds. 

May\marginnote{9.1} even my enemies \\>hear a Dhamma talk! \\
May even my enemies \\>devote themselves to the Buddha’s teaching! \\
May even my enemies \\>associate with those good people \\
who establish others in the Dhamma! 

May\marginnote{10.1} even my enemies \\>hear Dhamma at the right time, \\
from those who teach acceptance, \\
praising acquiescence; \\
and may they follow that path! 

For\marginnote{11.1} then they’d never wish harm \\
upon myself or others. \\
Having arrived at ultimate peace, \\
they’d look after creatures firm and frail. 

For\marginnote{12.1} irrigators guide the water, \\
and fletchers straighten arrows; \\
carpenters carve timber—\\
but the astute tame themselves. 

Some\marginnote{13.1} tame by using the rod, \\
some with goads, and some with whips. \\
But the unaffected one tamed me \\
without rod or sword. 

My\marginnote{14.1} name is ‘Harmless’, \\
though I used to be harmful. \\
The name I bear today is true, \\
for I do no harm to anyone. 

I\marginnote{15.1} used to be a bandit, \\
the notorious \textsanskrit{Aṅgulimāla}. \\
Swept away in a great flood, \\
I went to the Buddha for refuge. 

I\marginnote{16.1} used to have blood on my hands, \\
the notorious \textsanskrit{Aṅgulimāla}. \\
See the refuge I’ve found—\\
the conduit to rebirth is eradicated. 

I’ve\marginnote{17.1} done many of the sort of deeds \\
that lead to a bad destination. \\
The result of my deeds has already struck me, \\
so I enjoy my food free of debt. 

Fools\marginnote{18.1} and simpletons \\
devote themselves to negligence. \\
But the wise protect diligence \\
as their best treasure. 

Don’t\marginnote{19.1} devote yourself to negligence, \\
or delight in erotic intimacy. \\
For if you’re diligent and practice absorption, \\
you’ll attain ultimate happiness. 

It\marginnote{20.1} was welcome, not unwelcome, \\
the advice I got was good. \\
Of the well-explained teachings, \\
I arrived at the best. 

It\marginnote{21.1} was welcome, not unwelcome, \\
the advice I got was good. \\
I’ve attained the three knowledges, \\
and fulfilled the Buddha’s instructions.” 

“In\marginnote{22.1} the wilderness, at a tree’s root,\footnote{The following verses are not found in \href{https://suttacentral.net/mn86/en/sujato}{MN 86}. } \\
on mountains, or in caves—\\
it used to be that wherever I stood, \\
my mind was anxious. 

But\marginnote{23.1} now I lie down happily and stand up happily, \\
I live my life happily, \\
out of \textsanskrit{Māra}’s reach; \\
the teacher had sympathy for me. 

I\marginnote{24.1} used to be of brahmin birth, \\
highborn on both sides, \\
now I’m a son of the Holy One, \\
the Teacher, King of Dhamma. 

I\marginnote{25.1} am rid of craving, free of grasping, \\
my sense doors are guarded and well-restrained. \\
I’ve destroyed the root of misery, \\
and attained the ending of defilements. 

I’ve\marginnote{26.1} served the teacher \\
and fulfilled the Buddha’s instructions. \\
The heavy burden is laid down, \\
the conduit to rebirth is eradicated.” 

%
\end{verse}

%
\section*{{\suttatitleacronym Thag 16.9}{\suttatitletranslation Anuruddha }{\suttatitleroot Anuruddhattheragāthā}}
\addcontentsline{toc}{section}{\tocacronym{Thag 16.9} \toctranslation{Anuruddha } \tocroot{Anuruddhattheragāthā}}
\markboth{Anuruddha }{Anuruddhattheragāthā}
\extramarks{Thag 16.9}{Thag 16.9}

\begin{verse}%
Leaving\marginnote{1.1} my mother and father behind, \\
as well as sisters, kinsmen, and brothers; \\
having given up the five sensual titillations, \\
Anuruddha practices absorption. 

Surrounded\marginnote{2.1} by song and dance, \\
awakened by cymbals and gongs, \\
he did not find purification \\
while delighting in \textsanskrit{Māra}’s domain. 

But\marginnote{3.1} he has gone beyond all that, \\
and delights in the Buddha’s teaching. \\
Having crossed over the entire flood, \\
Anuruddha practices absorption. 

Sights,\marginnote{4.1} sounds, tastes, smells, \\
and touches so delightful: \\
having crossed over these as well, \\
Anuruddha practices absorption. 

Returning\marginnote{5.1} from almsround, \\
alone, without companion, \\
seeking rags from the dust heap, \\
Anuruddha is without defilements. 

The\marginnote{6.1} thoughtful sage \\
selected rags from the dust heap; \\
he picked them up, washed, dyed, and wore them; \\
Anuruddha is without defilements. 

The\marginnote{7.1} principles of someone \\
who has many wishes and is not content, \\
who socializes and is conceited, \\
are wicked and corrupting. 

But\marginnote{8.1} someone who is mindful, few of wishes, \\
content and untroubled, \\
delighting in seclusion, joyful, \\
always resolute and energetic; 

their\marginnote{9.1} principles are skillful, \\
leading to awakening; \\
they are without defilements—\\
so said the great seer. 

“Knowing\marginnote{10.1} my thoughts, \\
the supreme Teacher in the world \\
came to me in a mind-made body, \\
using his psychic power. 

He\marginnote{11.1} taught me more \\
than I had thought of. \\
The Buddha who loves non-proliferation \\
taught me non-proliferation. 

Understanding\marginnote{12.1} that teaching, \\
I happily did his bidding. \\
I’ve attained the three knowledges, \\
and have fulfilled the Buddha’s instructions. 

For\marginnote{13.1} the last fifty-five years \\
I have not lain down to sleep. \\
Twenty-five years have passed \\
since I eradicated drowsiness.” 

“There\marginnote{14.1} was no more breathing \\
for the unaffected one of steady heart. \\
Imperturbable, committed to peace, \\
the Clear-eyed One was fully quenched. 

He\marginnote{15.1} put up with painful feelings \\
without flinching. \\
The liberation of his heart \\
was like the extinguishing of a lamp.” 

“Now\marginnote{16.1} these touches and the other four \\
are the last to be experienced by the sage; \\
nor will there be other phenomena \\
since the Buddha was fully quenched. 

O\marginnote{17.1} Penelope, weaver of the web—\\
there’s no more abodes in the host of gods. \\
Transmigration through births is finished, \\
now there’ll be no more future lives.” 

“The\marginnote{18.1} mendicant by whom the galaxy \\
with the age of the Divinity are known in an hour—\\
that master of psychic ability sees the gods \\
at the time they pass away and are reborn.” 

“In\marginnote{19.1} the past I was \textsanskrit{Annabhāra}, \\
a poor carrier of fodder. \\
I practiced as an ascetic, \\
the renowned \textsanskrit{Upariṭṭha}. 

Then\marginnote{20.1} I was reborn in the Sakyan clan, \\
where I was known as ‘Anuruddha’. \\
Surrounded by song and dance, \\
I was awakened by cymbals and gongs. 

Then\marginnote{21.1} I saw the Buddha, \\
the Teacher, fearing nothing from any quarter; \\
filling my mind with confidence in him, \\
I went forth to homelessness. 

I\marginnote{22.1} know my past lives, \\
the places I used to live. \\
I was born as Sakka, \\
and stayed among the thirty-three gods. 

Seven\marginnote{23.1} times I was a king of men \\
ruling a kingdom, \\
victorious in the four quarters, \\
lord of the Black Plum Tree Land. \\
Without rod or sword, \\
I governed by principle. 

Seven\marginnote{24.1} from here, seven from there—\\
fourteen transmigrations in all. \\
I shall remember my past lives: \\
at that time I stayed in the realm of the gods. 

I\marginnote{25.1} have gained complete tranquility \\
in immersion with five factors. \\
Peaceful, serene, \\
my clairvoyance is purified. 

Steady\marginnote{26.1} in five-factored absorption, \\
I know the passing away and rebirth of beings, \\
their coming and going, \\
their lives in this state and that. 

I’ve\marginnote{27.1} served the teacher \\
and fulfilled the Buddha’s instructions. \\
The heavy burden is laid down, \\
the conduit to rebirth is eradicated. 

In\marginnote{28.1} the Vajjian village of \textsanskrit{Veḷuva}, \\
my life will come to an end. \\
Beneath a thicket of bamboos, \\
being undefiled, I will be fully extinguished.” 

%
\end{verse}

%
\section*{{\suttatitleacronym Thag 16.10}{\suttatitletranslation Pārāsariya (3rd) }{\suttatitleroot Pārāpariyattheragāthā}}
\addcontentsline{toc}{section}{\tocacronym{Thag 16.10} \toctranslation{Pārāsariya (3rd) } \tocroot{Pārāpariyattheragāthā}}
\markboth{Pārāsariya (3rd) }{Pārāpariyattheragāthā}
\extramarks{Thag 16.10}{Thag 16.10}

\begin{verse}%
This\marginnote{1.1} thought came to the ascetic \\
in the forest full of flowers, \\
as he was seated alone \\
meditating in seclusion: 

“The\marginnote{2.1} behavior of the mendicants \\
these days seems different \\
from when the protector of the world, \\
the best of men, was still here. 

Their\marginnote{3.1} robes were only for covering the private parts, \\
and protection from the cold and wind. \\
They ate in moderation, \\
content with whatever they were offered. 

Whether\marginnote{4.1} food was fine or coarse, \\
a little or a lot, \\
they ate only for sustenance, \\
without greed or gluttony. 

They\marginnote{5.1} weren’t so very eager \\
for the requisites of life, \\
such as tonics and other necessities, \\
as they were for the ending of defilements. 

In\marginnote{6.1} the wilderness, at a tree’s root, \\
in caves and caverns, \\
fostering seclusion, \\
they lived with that as their final goal. 

They\marginnote{7.1} were used to simple things, unburdensome, \\
gentle, not pompous at heart, \\
unsullied, not scurrilous, \\
their thoughts were intent on the goal. 

That’s\marginnote{8.1} why they inspired confidence, \\
in their movements, eating, and practice; \\
their deportment was as smooth \\
as a stream of oil. 

With\marginnote{9.1} the utter ending of all defilements, \\
those senior monks have now been quenched. \\
They were great meditators and great benefactors—\\
there are few like them today. 

With\marginnote{10.1} the ending \\
of good principles and understanding, \\
the victor’s teaching, \\
full of all excellent qualities, has fallen apart. 

This\marginnote{11.1} is the season \\
for bad principles and defilements. \\
Those who are ready for seclusion \\
are all that’s left of the true Dhamma. 

As\marginnote{12.1} they grow, the defilements \\
possess most people; \\
they play with fools, it seems to me, \\
like monsters with the mad. 

Overcome\marginnote{13.1} by defilements, \\
they run here and there \\
among the bases for defilement, \\
as if they had declared war on themselves. 

Having\marginnote{14.1} forsaken the true teaching, \\
they argue with each other.\footnote{\textit{\textsanskrit{Bhaṇḍare}} is reflexive third plural present. } \\
Following wrong views \\
they think, ‘This is better.’ 

They\marginnote{15.1} cut off their wealth, \\
children, and wife to go forth. \\
But then they do what they shouldn’t, \\
for the sake of a measly spoon of almsfood. 

They\marginnote{16.1} eat until their bellies are full, \\
and then they lie to sleep on their backs. \\
When they wake up, they keep on chatting, \\
the kind of talk that the teacher criticized. 

Valuing\marginnote{17.1} all the arts and crafts, \\
they train themselves in them. \\
Not being settled inside, they think, \\
‘This is the goal of the ascetic life.’ 

They\marginnote{18.1} provide clay, oil, and talcum powder, \\
water, lodgings, and food \\
for householders, \\
expecting more in return. 

And\marginnote{19.1} in addition, tooth-picks, portia flowers, \\
flowers, food to eat, \\
well-cooked almsfood, \\
mangoes and myrobalans. 

In\marginnote{20.1} medicine they are like doctors, \\
in business like householders, \\
in makeup like prostitutes, \\
in sovereignty like lords. 

Cheats,\marginnote{21.1} frauds, \\
false witnesses, sly: \\
using multiple plans, \\
they enjoy things of the flesh. 

Pursuing\marginnote{22.1} shams, contrivances, and plans, \\
by such means \\
they accumulate a lot of wealth \\
for the sake of their own livelihood. 

They\marginnote{23.1} assemble the community \\
for business rather than Dhamma. \\
They teach the Dhamma to others \\
for gain, not for the goal. 

Those\marginnote{24.1} barred from the \textsanskrit{Saṅgha} \\
quarrel over the \textsanskrit{Saṅgha}’s property. \\
Lacking conscience, they do not care \\
that they live on the earnings of others. 

Some\marginnote{25.1} with shaven head and robe \\
are not devoted to practice, \\
but wish only to be honored, \\
besotted with property and reverence. 

When\marginnote{26.1} things have come to this, \\
it’s not easy these days \\
to realize what has not yet been realized, \\
or to preserve what has been realized. 

When\marginnote{27.1} shoeless on a thorny path, \\
one would walk \\
very mindfully; \\
that’s how a sage should walk in the village. 

Remembering\marginnote{28.1} the meditators of old, \\
and recollecting their conduct, \\
even in the latter days, \\
it’s still possible to realize freedom from death.” 

That\marginnote{29.1} is what the ascetic, whose faculties \\
were fully developed, said in the \textsanskrit{sāl} tree grove. \\
The brahmin, the seer, became fully extinguished, \\
putting an end to all future lives. 

%
\end{verse}

%
\addtocontents{toc}{\let\protect\contentsline\protect\nopagecontentsline}
\part*{The Book of the Thirties }
\addcontentsline{toc}{part}{The Book of the Thirties }
\markboth{}{}
\addtocontents{toc}{\let\protect\contentsline\protect\oldcontentsline}

%
\addtocontents{toc}{\let\protect\contentsline\protect\nopagecontentsline}
\chapter*{Chapter One }
\addcontentsline{toc}{chapter}{\tocchapterline{Chapter One }}
\addtocontents{toc}{\let\protect\contentsline\protect\oldcontentsline}

%
\section*{{\suttatitleacronym Thag 17.1}{\suttatitletranslation Phussa }{\suttatitleroot Phussattheragāthā}}
\addcontentsline{toc}{section}{\tocacronym{Thag 17.1} \toctranslation{Phussa } \tocroot{Phussattheragāthā}}
\markboth{Phussa }{Phussattheragāthā}
\extramarks{Thag 17.1}{Thag 17.1}

\begin{verse}%
Seeing\marginnote{1.1} many who inspire confidence, \\
evolved and well-restrained, \\
the seer of the \textsanskrit{Paṇḍara} clan,\footnote{No \textit{\textsanskrit{paṇḍara}} clan is known elsewhere. Given that the text goes on to discuss robe colors of mendicants, the name, which means “white”, probably refers to ascetics wearing white robes, the \textit{\textsanskrit{pāṇḍarabhikṣu}} of Yaśomitra’s \textsanskrit{Sphuṭārthā} \textsanskrit{Abhidharmakośavyākhyā} 300 and \textsanskrit{Mahāvyutpatti} 179. Soma is exalted as “white-robed” (\textit{\textsanskrit{pāṇḍaravāsa}}) at \textsanskrit{Bṛhadāraṇyaka} \textsanskrit{Upaniṣad} 2.1.3. } \\
asked the one known as Phussa: 

“In\marginnote{2.1} future times, \\
what desire and motivation \\
and behavior will people have? \\
Please answer my question.” 

“Listen\marginnote{3.1} to my words, \\
O seer known as a \textsanskrit{Paṇḍara}, \\
and remember them carefully, \\
I will describe the future. 

In\marginnote{4.1} the future many will be \\
angry and hostile, \\
offensive, stubborn, and devious, \\
jealous, holding divergent views. 

Imagining\marginnote{5.1} they understand \\>the depths of the teaching, \\
they resort to the near shore. \\
Superficial and disrespectful towards the teaching, \\
they lack respect for one another. 

In\marginnote{6.1} the future \\
many dangers will arise in the world. \\
Idiots will defile \\
the Dhamma that was taught so well. 

Though\marginnote{7.1} bereft of good qualities, \\
unlearned prattlers, too sure of themselves, \\
will become powerful \\
in running \textsanskrit{Saṅgha} proceedings.\footnote{This is probably an indirect reference to the Second Council, where one of the difficulties was ensuring the proceedings were carried out by monks on integrity. } 

Though\marginnote{8.1} possessing good qualities, \\
the conscientious and unbiased, \\>acting in the proper spirit, \\
will become weak \\
in running \textsanskrit{Saṅgha} proceedings. 

In\marginnote{9.1} the future, simpletons will accept \\
currency and gold,\footnote{Another reference to the Second Council, whose primary issue was the use of money by monastics. } \\
fields and land, goats and sheep, \\
and bonded servants, male and female. 

Fools\marginnote{10.1} looking for fault in others, \\
but unsteady in their own ethics, \\
will wander about, insolent, \\
like cantankerous beasts. 

They’ll\marginnote{11.1} be haughty, \\
wrapped in robes of blue;\footnote{The \textsanskrit{Śāriputraparipṛcchā} (T24, no. 1465, 900.c18) says that the \textsanskrit{Mahīśāsaka} school wore blue robes. They were probably based in \textsanskrit{Mahissatī}, which is close by the kingdom of the \textsanskrit{Pāṇḍyas}. Perhaps this sutta records tensions in the early community of the southern region, leading to the formation of distinct groups of monastics differentiated by their robe colors. } \\
deceivers and flatterers, pompous and fake, \\
they’ll wander as if they were noble ones. 

With\marginnote{12.1} hair sleeked back with oil, \\
fickle, their eyes painted with eye-liner, \\
they’ll travel on the high-road, \\
wrapped in robes of ivory color. 

The\marginnote{13.1} deep-dyed ocher robe, \\
worn without disgust by the free, \\
they will come to loathe, \\
besotted by white clothes.\footnote{Text here uses \textit{\textsanskrit{odāta}} rather than \textit{\textsanskrit{paṇḍara}} for “white”. } 

They’ll\marginnote{14.1} want lots of possessions, \\
and be lazy, lacking energy. \\
Weary of the forest, \\
they’ll stay within villages. 

Being\marginnote{15.1} unrestrained, they’ll keep company with \\
those who acquire profit, \\
and who always enjoy wrong livelihood, \\
following their example. 

They\marginnote{16.1} won’t respect those \\
who don’t get lots of stuff, \\
and they won’t associate with the attentive, \\
even though they’re very amiable. 

Disparaging\marginnote{17.1} their own banner,\footnote{\textit{Milakkhu} (Sanskrit \textit{mleccha}) has the sense “copper-colored”, “vermillion” as well as the more common “foreigner”. } \\
dyed a vermilion color, \\
some will wear the white banner \\
of those of other religions. 

Then\marginnote{18.1} they’ll have no respect \\
for the ocher robe. \\
The mendicants will not reflect \\
on the nature of the ocher robe. 

This\marginnote{19.1} awful lack of reflection \\
was unthinkable to the elephant, \\
who was overcome by suffering, \\
injured by an arrow strike. 

Then\marginnote{20.1} the six-tusked elephant,\footnote{The “six-tusked” (\textit{chaddanta}) elephant was a mystical beast. The story of how he was struck by an arrow and uttered these verses is told in \href{https://suttacentral.net/ja514/en/sujato}{Ja 514}. } \\
seeing the deep-dyed banner of the perfected ones, \\
straight away spoke these verses \\
connected with the goal. 

‘One\marginnote{21.1} who, not free of stains themselves,\footnote{This verse and the next are found at \href{https://suttacentral.net/dhp9/en/sujato}{Dhp 9} f. and \href{https://suttacentral.net/ja514/en/sujato\#26.1}{Ja 514:26.1} f. The text and commentary here do not specify which verses were spoken by the elephant, so I assume they are the same two mentioned in the \textsanskrit{Jātaka}. } \\
would wear the robe stained in ocher, \\
bereft of self-control and truth: \\
they are not worthy of the ocher robe. 

One\marginnote{22.1} who’s purged all their stains, \\
steady in ethics, \\
possessing truth and self-control: \\
they are truly worthy of the ocher robe.’ 

Devoid\marginnote{23.1} of virtue, a simpleton, \\
wild, doing what they like, \\
their minds astray, indolent: \\
they are not worthy of the ocher robe. 

One\marginnote{24.1} accomplished in ethics, \\
free of greed, serene, \\
their heart’s intention pure: \\
they are truly worthy of the ocher robe. 

The\marginnote{25.1} restless, insolent fool, \\
who has no ethics at all, \\
is worthy of a white robe—\\
what use is an ocher robe for them? 

In\marginnote{26.1} the future, monks and nuns \\
with corrupt hearts, lacking regard for others, \\
will disparage those \\
with hearts of loving-kindness. 

Though\marginnote{27.1} trained in wearing the robe \\
by senior monks, \\
simpletons will not listen, \\
wild, doing what they like. 

With\marginnote{28.1} that kind of attitude to training, \\
those fools won’t respect each other, \\
or take any notice of their mentors, \\
like a wild colt with its charioteer. 

Even\marginnote{29.1} so, in the future, \\
this will be the practice \\
of monks and nuns \\
when the latter days have come. 

Before\marginnote{30.1} this frightening future arrives, \\
be easy to admonish, \\
courteous in speech, \\
and respect one another. 

Have\marginnote{31.1} hearts of love and compassion, \\
and please do keep your precepts. \\
Be energetic, resolute, \\
and always staunchly vigorous. 

Seeing\marginnote{32.1} negligence as fearful, \\
and diligence as a sanctuary, \\
develop the eightfold path, \\
realizing the state free of death.” 

%
\end{verse}

\scendsutta{… }

%
\section*{{\suttatitleacronym Thag 17.2}{\suttatitletranslation Sāriputta }{\suttatitleroot Sāriputtattheragāthā}}
\addcontentsline{toc}{section}{\tocacronym{Thag 17.2} \toctranslation{Sāriputta } \tocroot{Sāriputtattheragāthā}}
\markboth{Sāriputta }{Sāriputtattheragāthā}
\extramarks{Thag 17.2}{Thag 17.2}

\begin{verse}%
“One\marginnote{1.1} who’s mindful \\>as per their conduct and mindfulness, \\
diligent \\>as per their intentions and meditation, \\
happy inside, serene, solitary, contented: \\
that is what they call a mendicant. 

When\marginnote{2.1} eating fresh or dried food, \\
one shouldn’t be overly replete. \\
A mendicant should wander mindfully, \\
with empty stomach, taking limited food. 

Four\marginnote{3.1} or five mouthfuls before you’re full, \\
drink some water; \\
this is enough for a resolute mendicant \\
to live in comfort. 

If\marginnote{4.1} they cover themselves with a robe \\
that’s allowable and fit for purpose; \\
this is enough for a resolute mendicant \\
to live in comfort. 

When\marginnote{5.1} sitting cross-legged, \\
the rain doesn’t fall on the knees; \\
this is enough for a resolute mendicant \\
to live in comfort.” 

“When\marginnote{6.1} you’ve seen happiness as suffering, \\
and suffering as a dart, \\
and that there’s nothing between the two—\\
what keeps you in the world? \\>What would you become? 

Thinking,\marginnote{7.1} ‘May I have nothing to do \\>with those of bad wishes, \\
lazy, lacking energy, \\
unlearned, lacking regard for others’—\\
what keeps you in the world? \\>What would you become?” 

“An\marginnote{8.1} intelligent, learned person, \\
steady in ethics, \\
devoted to serenity of heart—\\
let them stand at the head.” 

“A\marginnote{9.1} beast who likes to proliferate, \\
enjoying proliferation, \\
fails to win extinguishment, \\
the supreme sanctuary from the yoke. 

But\marginnote{10.1} one who gives up proliferation, \\
enjoying the state of non-proliferation, \\
wins extinguishment, \\
the supreme sanctuary from the yoke.” 

“Whether\marginnote{11.1} in the village or the wilderness, \\
in a valley or the uplands, \\
wherever the perfected ones live \\
is a delightful place.” 

“Delightful\marginnote{12.1} are the wildernesses \\
where no people delight. \\
Those free of greed will delight there, \\
not those who seek sensual pleasures.” 

“Regard\marginnote{13.1} one who sees your faults \\
as a guide to a hidden treasure. \\
Stay close to one so wise and astute \\
who corrects you when you need it. \\
Sticking close to such an impartial person, \\
things get better, not worse.” 

“Advise\marginnote{14.1} and instruct; \\
curb wickedness: \\
for you shall be loved by the good, \\
and disliked by the bad.” 

“The\marginnote{15.1} Blessed One, the Buddha, the seer \\
was teaching Dhamma to another. \\
As he taught the Dhamma, \\
I lent an ear to get the meaning. 

My\marginnote{16.1} listening wasn’t wasted: \\
I’m freed, without defilements.” 

“Not\marginnote{17.1} for knowledge of past lives, \\
nor even for clairvoyance; \\
not for psychic powers, \\>or reading the minds of others, \\
nor for knowing people’s \\>passing away and being reborn; \\
not for purifying the power of clairaudience, \\
did I have any wish.” 

“His\marginnote{18.1} only shelter is the foot of a tree; \\
shaven, wrapped in his outer robe, \\
the senior monk foremost in wisdom, \\
Upatissa himself practices absorption. 

When\marginnote{19.1} in a meditation  free of placing the mind, \\
a disciple of the Buddha \\
is at that moment blessed \\
with noble silence. 

As\marginnote{20.1} a rocky mountain \\
is unwavering and well grounded, \\
so when delusion ends, \\
a monk, like a mountain, doesn’t tremble.” 

“To\marginnote{21.1} the man who has not a blemish, \\
who is always seeking purity, \\
even a hair-tip of evil \\
seems as big as a cloud.” 

“I\marginnote{22.1} don’t long for death; \\
I don’t long for life; \\
I will lay down this body, \\
aware and mindful. 

I\marginnote{23.1} don’t long for death; \\
I don’t long for life; \\
I await my time, \\
like a worker waiting for their wages.” 

“Both\marginnote{24.1} what came before and what follows after \\
are nothing but death, not freedom from death. \\
Practice, don’t perish—\\
don’t let the moment pass you by. 

Just\marginnote{25.1} like a frontier city, \\
is guarded inside and out, \\
so you should ward yourselves—\\
don’t let the moment pass you by. \\
For if you miss your moment \\
you’ll grieve when sent to hell.” 

“Calm\marginnote{26.1} and still, \\
thoughtful in counsel, not restless—\\
he shakes off bad qualities \\
as the gale shakes leaves off a tree. 

Calm\marginnote{27.1} and still, \\
thoughtful in counsel, not restless—\\
he plucks off bad qualities \\
as the gale shakes leaves off a tree. 

Calm\marginnote{28.1} and free of despair, \\
clear and unclouded, \\
of good morals, intelligent: \\
one would make an end of suffering.” 

“Some\marginnote{29.1} householders, and even some renunciants, \\
are not to be trusted. \\
Some who were good later become bad; \\
while some who were bad become good.” 

“Sensual\marginnote{30.1} desire, ill will, \\
dullness and drowsiness, \\
restlessness, and doubt: \\
these are the five mental stains for a monk.” 

“Whether\marginnote{31.1} they’re honored \\
or not honored, or both, \\
their immersion doesn’t waver \\
as they live diligently. 

They\marginnote{32.1} persistently practice absorption \\
with subtle view and discernment. \\
Rejoicing in the ending of grasping, \\
they’re said to be a true person.” 

“The\marginnote{33.1} oceans and the earth, \\
the mountains and the wind—\\
no simile can do justice \\
to the Teacher’s magnificent liberation.” 

“The\marginnote{34.1} senior monk who keeps the wheel rolling, \\
he is very wise and serene. \\
Like earth, like water, like fire, \\
he is neither attracted nor repelled. 

He\marginnote{35.1} has attained the perfection of wisdom, \\
so intelligent and thoughtful. \\
He is bright, but seems to be dull; \\
he always wanders, quenched.” 

“I’ve\marginnote{36.1} served the teacher \\
and fulfilled the Buddha’s instructions. \\
The heavy burden is laid down, \\
the conduit to rebirth is eradicated.” 

“Persist\marginnote{37.1} with diligence: \\
this is my instruction. \\
Come, I’ll be fully quenched—\\
I am everywhere free.” 

%
\end{verse}

%
\section*{{\suttatitleacronym Thag 17.3}{\suttatitletranslation Ānanda }{\suttatitleroot Ānandattheragāthā}}
\addcontentsline{toc}{section}{\tocacronym{Thag 17.3} \toctranslation{Ānanda } \tocroot{Ānandattheragāthā}}
\markboth{Ānanda }{Ānandattheragāthā}
\extramarks{Thag 17.3}{Thag 17.3}

\begin{verse}%
“The\marginnote{1.1} astute would not make friends \\
with the slanderous or the hateful, \\
with a miser or a gloater, \\
for it’s bad to consort with sinners. 

The\marginnote{2.1} astute would make friends \\
with the faithful and the pleasant, \\
the wise and the learned, \\
for it’s a blessing to consort with true persons.” 

“See\marginnote{3.1} this fancy puppet, \\
a body built of sores, \\
diseased, obsessed over, \\
in which nothing lasts at all. 

See\marginnote{4.1} this fancy puppet, \\
with its gems and earrings; \\
it is bones encased in skin, \\
made pretty by its clothes. 

Rouged\marginnote{5.1} feet \\
and powdered face \\
may be enough to beguile a fool, \\
but not a seeker of the far shore. 

Hair\marginnote{6.1} in eight braids \\
and eyeliner \\
may be enough to beguile a fool, \\
but not a seeker of the far shore. 

A\marginnote{7.1} rotting body all adorned \\
like a freshly painted makeup box \\
may be enough to beguile a fool, \\
but not a seeker of the far shore.” 

“Gotama\marginnote{8.1} is learned, a brilliant speaker, \\
the attendant to the Buddha. \\
With burden put down, detached, \\
Gotama made his bed. 

Defilements\marginnote{9.1} ended, detached, \\
he has got over clinging and become quenched. \\
He bears his final body, \\
having gone beyond birth and death.” 

“Gotama\marginnote{10.1} stands firm \\
on the path that leads to quenching, \\
where the teachings of the Buddha, \\
the kinsman of the Sun, are grounded.” 

“82,000\marginnote{11.1} from the Buddha, \\
and 2,000 more from the monks: \\
84,000 teachings I’ve learned, \\
and these are what I promulgate.” 

“A\marginnote{12.1} person of little learning \\
ages like an ox—\\
their flesh grows, \\
but not their wisdom. 

A\marginnote{13.1} learned person who, on account of their learning, \\
looks down on someone of little learning, \\
strikes me as like \\
a blind man holding a lamp. 

You\marginnote{14.1} should stay close to a learned person—\\
don’t lose what you’ve learned. \\
It is the root of the spiritual life, \\
which is why you should memorize the teaching. 

Knowing\marginnote{15.1} the sequence and meaning of the teaching, \\
expert in the interpretation of terms, \\
they make sure it is well memorized, \\
and then examine the meaning. 

Accepting\marginnote{16.1} the teachings, they become enthusiastic; \\
making an effort, they weigh up the teaching. \\
When it’s time, they strive \\
serene inside themselves. 

If\marginnote{17.1} you want to understand the teaching, \\
you should befriend the sort of person \\
who is learned and has memorized the teachings, \\
a wise disciple of the Buddha. 

One\marginnote{18.1} who is learned \\>and has memorized the teaching, \\
a keeper of the great seer’s treasury, \\
is a visionary for the whole world, \\
learned and honorable. 

Delighting\marginnote{19.1} in the teaching, enjoying the teaching, \\
contemplating the teaching, \\
a mendicant who recollects the teaching \\
doesn’t decline in the true teaching.” 

“When\marginnote{20.1} your body is pampered and heavy, \\
while your remaining time is running out, \\
greedy for physical pleasure, \\
how can you be comfortable as an ascetic?” 

“I’m\marginnote{21.1} completely disorientated! \\
The teachings don’t spring to mind! \\
With the passing of our good friend, \\
everything seems dark. 

When\marginnote{22.1} your friend has passed away, \\
and your Teacher is past and gone, \\
there’s no friend like \\
mindfulness of the body. 

The\marginnote{23.1} old have passed away, \\
and I don’t agree with the new. \\
Today I meditate alone \\
like a bird snug in its nest.” 

“Many\marginnote{24.1} international visitors \\
have come to visit. \\
Don’t block the audience, \\
let the congregation see me.” 

“Lots\marginnote{25.1} of international visitors \\
have come to visit. \\
The teacher grants them the opportunity, \\
the Clear-eyed One doesn’t turn them away.” 

“In\marginnote{26.1} the twenty five years that have passed \\
since I became a trainee, \\
no sensual perception has arisen in me: \\
see the excellence of the teaching! 

In\marginnote{27.1} the twenty-five years \\
since I became a trainee, \\
no malicious perception has arisen in me: \\
see the excellence of the teaching!” 

“For\marginnote{28.1} 25 years \\
I attended on the Buddha \\
with loving deeds, \\
like a shadow that never left. 

For\marginnote{29.1} 25 years \\
I attended on the Buddha \\
with loving words, \\
like a shadow that never left. 

For\marginnote{30.1} 25 years \\
I attended on the Buddha \\
with loving thoughts, \\
like a shadow that never left. 

While\marginnote{31.1} the Buddha was walking mindfully, \\
I walked behind him. \\
As he taught the Dhamma, \\
knowledge arose in me.” 

“I’m\marginnote{32.1} a trainee, who has more to do; \\
my heart’s desire is still unfulfilled. \\
Yet the Teacher, who was so compassionate to me, \\
has become completely quenched. 

Then\marginnote{33.1} there was terror! \\
Then they had goosebumps! \\
When the Buddha, endowed with all fine qualities, \\
was fully quenched.” 

“Ānanda,\marginnote{34.1} who was learned \\>and had memorized the teaching, \\
a keeper of the great seer’s treasury, \\
a visionary for the entire world, \\
has become fully quenched. 

He\marginnote{35.1} was learned \\>and had memorized the teaching, \\
a keeper of the great seer’s treasury, \\
a visionary for the entire world, \\
in thick of night he dispelled the dark. 

He\marginnote{36.1} is the seer who remembered the teachings, \\
and mastered their sequence, holding them firm. \\
The senior monk who memorized the teaching, \\
Ānanda was a mine of gems.” 

“I’ve\marginnote{37.1} served the teacher \\
and fulfilled the Buddha’s instructions. \\
The heavy burden is laid down, \\
now there’ll be no more future lives.” 

%
\end{verse}

%
\addtocontents{toc}{\let\protect\contentsline\protect\nopagecontentsline}
\part*{The Book of the Forties }
\addcontentsline{toc}{part}{The Book of the Forties }
\markboth{}{}
\addtocontents{toc}{\let\protect\contentsline\protect\oldcontentsline}

%
\addtocontents{toc}{\let\protect\contentsline\protect\nopagecontentsline}
\chapter*{Chapter One }
\addcontentsline{toc}{chapter}{\tocchapterline{Chapter One }}
\addtocontents{toc}{\let\protect\contentsline\protect\oldcontentsline}

%
\section*{{\suttatitleacronym Thag 18.1}{\suttatitletranslation Mahākassapa }{\suttatitleroot Mahākassapattheragāthā}}
\addcontentsline{toc}{section}{\tocacronym{Thag 18.1} \toctranslation{Mahākassapa } \tocroot{Mahākassapattheragāthā}}
\markboth{Mahākassapa }{Mahākassapattheragāthā}
\extramarks{Thag 18.1}{Thag 18.1}

\begin{verse}%
“You\marginnote{1.1} shouldn’t live for the adulation of a following; \\
it turns your mind, making it hard to get immersion. \\
Seeing that popularity is suffering, \\
you shouldn’t consent to a following. 

A\marginnote{2.1} sage should not visit respectable families; \\
it turns your mind, making it hard to get immersion. \\
If you’re eager and greedy for flavors, \\
you’ll miss the goal that brings such happiness. 

They\marginnote{3.1} know it really is a bog, \\
this homage and veneration in respectable families. \\
Honor is a subtle dart, hard to extract, \\
and hard for a sinner to give up.” 

“I\marginnote{4.1} came down from my lodging \\
and entered the city for alms. \\
I courteously stood by \\
while a leper ate. 

With\marginnote{5.1} his putrid hand \\
he offered me a morsel. \\
Putting the morsel in my bowl, \\
his finger dropped off right there. 

Sitting\marginnote{6.1} by a wall, \\
I ate that lump of rice. \\
I did not feel any disgust \\
while eating or afterwards. 

Anyone\marginnote{7.1} who makes use of \\
leftovers for food, \\
rancid urine as medicine, \\
the root of a tree as lodging, \\
and cast-off rags as robes, \\
is at ease in any quarter.” 

“Where\marginnote{8.1} some have fallen to ruin \\
while climbing the mountain, \\
there Kassapa ascends; \\
an heir of the Buddha, \\
aware and mindful, \\
owing to his psychic powers. 

Returning\marginnote{9.1} from almsround, \\
Kassapa ascends the mountain, \\
and practices absorption without grasping, \\
with fear and dread given up. 

Returning\marginnote{10.1} from almsround, \\
Kassapa ascends the mountain, \\
and practices absorption without grasping, \\
quenched amongst those who burn. 

Returning\marginnote{11.1} from almsround, \\
Kassapa ascends the mountain, \\
and practices absorption without grasping, \\
his task completed, free of defilements.” 

“Strewn\marginnote{12.1} with garlands of the musk-rose tree, \\
these regions are so delightful, so lovely, \\
echoing with the trumpeting of elephants: \\
these rocky crags delight me! 

Glistening,\marginnote{13.1} they look like blue storm clouds, \\
with waters cool and streams so clear, \\
and covered all in ladybugs: \\
these rocky crags delight me! 

Like\marginnote{14.1} the peak of a blue storm cloud, \\
or like a fine bungalow, lovely, \\
echoing with the trumpeting of elephants: \\
these rocky crags delight me! 

The\marginnote{15.1} rain comes down on the lovely flats, \\
in the mountains frequented by seers. \\
Echoing with the cries of peacocks, \\
these rocky crags delight me! 

It’s\marginnote{16.1} enough for me, who loves absorption, \\
to remain resolute. \\
It’s enough for me, \\
a resolute monk who loves the goal. 

It’s\marginnote{17.1} enough for me, \\
a resolute monk who loves comfort. \\
It’s enough for me, \\
resolute and unaffected, loving meditation. 

Covered\marginnote{18.1} with flowers of flax, \\
like the welkin covered with clouds, \\
full of flocks of many different birds, \\
these rocky crags delight me! 

Empty\marginnote{19.1} of householders, \\
frequented by herds of deer, \\
full of flocks of many different birds, \\
these rocky crags delight me! 

The\marginnote{20.1} water’s clear and the rocks are broad, \\
monkeys and deer are all around; \\
festooned with dewy moss, \\
these rocky crags delight me!” 

“Even\marginnote{21.1} the music of a five-piece band \\
can never give such pleasure \\
as when, with unified mind, \\
you rightly discern the Dhamma.” 

“Don’t\marginnote{22.1} get involved in lots of work, \\
avoid people, and don’t try to acquire things. \\
If you’re eager and greedy for flavors, \\
you’ll miss the goal that brings such happiness. 

Don’t\marginnote{23.1} get involved in lots of work, \\
avoid what doesn’t lead to the goal. \\
The body gets worn out and fatigued, \\
and when you ache, you won’t find serenity.” 

“You\marginnote{24.1} won’t see yourself \\
by merely reciting words, \\
wandering stiff-necked \\
and thinking, ‘I’m better than them.’ 

The\marginnote{25.1} fool is no better, \\
but they think they are. \\
The wise don’t praise \\
pompous people. 

Whoever\marginnote{26.1} is not affected \\
by the modes of conceit—\\
‘I am better’, ‘I’m not better’, \\
‘I am worse’, or ‘I am the same’—

with\marginnote{27.1} such understanding, unaffected, \\
steady in ethics, \\
and devoted to serenity of mind: \\
that is who the wise praise.” 

“Whoever\marginnote{28.1} has no respect \\
for their spiritual companions \\
is as far from the true teaching \\
as the earth is from the sky. 

Those\marginnote{29.1} whose conscience and prudence \\
are always rightly established, \\
thrive in the spiritual life; \\
for them, there are no future lives. 

When\marginnote{30.1} a mendicant who is haughty and fickle \\
wears rags from the rubbish-heap, \\
that doesn’t make them shine: \\
they’re like a monkey in a lion skin. 

But\marginnote{31.1} if they not restless or fickle, \\
alert, with senses restrained, \\
then, wearing rags from the rubbish-heap, they shine \\
like a lion in a mountain cave.” 

“These\marginnote{32.1} many gods \\
powerful and glorious, \\
all 10,000 of them, \\
belong to the host of the Divinity. 

They\marginnote{33.1} stand with joined palms \\
honoring \textsanskrit{Sāriputta}, \\
the general of the Dhamma, the hero, \\
the serene great meditator: 

‘Homage\marginnote{34.1} to you, O thoroughbred! \\
Homage to you, supreme among men! \\
We don’t understand \\
the basis of your absorption. 

The\marginnote{35.1} profound domain of the Buddhas \\
is truly amazing. \\
We don’t understand, \\
though we’ve gathered here to split hairs.’ 

When\marginnote{36.1} he saw the host of gods \\
paying homage to \textsanskrit{Sāriputta}—\\
who is truly worthy of homage—\\
Kappina smiled.” 

“As\marginnote{37.1} far as the range of the Buddha extends, \\
I am outstanding in austerities. \\
I have no equal, \\
apart from the great sage himself. 

I’ve\marginnote{38.1} served the teacher \\
and fulfilled the Buddha’s instructions. \\
The heavy burden is laid down, \\
now there’ll be no more future lives.” 

“Like\marginnote{39.1} a lotus flower \\
to which water will not stick, \\
Gotama the immeasurable is unstained \\
by robes, lodgings, or food. \\
He inclines to renunciation, \\
and has escaped the three states of existence. 

The\marginnote{40.1} great sage’s neck is mindfulness meditation; \\
faith is his hands, and wisdom his head. \\
Having great knowledge, \\
he always wanders, quenched.” 

%
\end{verse}

%
\addtocontents{toc}{\let\protect\contentsline\protect\nopagecontentsline}
\part*{The Book of the Fifties }
\addcontentsline{toc}{part}{The Book of the Fifties }
\markboth{}{}
\addtocontents{toc}{\let\protect\contentsline\protect\oldcontentsline}

%
\addtocontents{toc}{\let\protect\contentsline\protect\nopagecontentsline}
\chapter*{Chapter One }
\addcontentsline{toc}{chapter}{\tocchapterline{Chapter One }}
\addtocontents{toc}{\let\protect\contentsline\protect\oldcontentsline}

%
\section*{{\suttatitleacronym Thag 19.1}{\suttatitletranslation Tālapuṭa }{\suttatitleroot Tālapuṭattheragāthā}}
\addcontentsline{toc}{section}{\tocacronym{Thag 19.1} \toctranslation{Tālapuṭa } \tocroot{Tālapuṭattheragāthā}}
\markboth{Tālapuṭa }{Tālapuṭattheragāthā}
\extramarks{Thag 19.1}{Thag 19.1}

\begin{verse}%
Oh,\marginnote{1.1} when will I stay in a mountain cave, \\
alone, with no companion, \\
discerning all states of existence as impermanent? \\
This hope of mine, \\>when will it be? 

Oh,\marginnote{2.1} when will I stay happily in the forest, \\
a sage wearing a torn robe, dressed in ocher, \\
unselfish, with no need for hope, \\
with greed, hate, and delusion destroyed? 

Oh,\marginnote{3.1} when will I stay alone in the wood, \\
fearless, discerning this body as impermanent, \\
a nest of death and disease, \\
oppressed by death and old age; \\>when will it be? 

Oh,\marginnote{4.1} when will I live, \\>having grasped the sharp sword of wisdom \\
and cut the creeper of craving \\>that tangles around everything, \\
the mother of fear, the bringer of suffering? \\
When will it be? 

Oh,\marginnote{5.1} when will I, seated on the lion’s throne, \\
swiftly grasp the sword of the sages, \\
forged by wisdom, of fiery might, \\
and swiftly break \textsanskrit{Māra} and his army? \\>When will it be? 

Oh,\marginnote{6.1} when will I be seen striving in the assemblies \\
with those who are virtuous, unaffected, respecting the Dhamma, \\
seeing things as they are, with faculties subdued? \\
When will it be? 

Oh,\marginnote{7.1} when will I focus on my own goal \\>at the Mountainfold, \\
free of oppression by laziness, hunger, thirst, \\
wind, heat, insects, and reptiles? \\
When will it be? 

Oh,\marginnote{8.1} when will I, serene and mindful, \\
understand the four truths, \\
that were realized by the great seer, \\
and are so very hard to see? \\>When will it be? 

Oh,\marginnote{9.1} when will I, devoted to serenity, \\
see with understanding the infinite sights, \\
sounds, smells, tastes, touches, and ideas \\
as burning? \\>When will it be? 

Oh,\marginnote{10.1} when will I not be distraught \\
because of criticism, \\
nor elated because of praise? \\
When will it be? 

Oh,\marginnote{11.1} when will I discern the aggregates \\
and the infinite varieties of phenomena, \\
both internal and external, as no more than \\
wood, grass, and creepers? \\>When will it be? 

Oh,\marginnote{12.1} when will the monsoon clouds in season \\
freshly wet me in my robe in the forest, \\
walking the path trodden by the sages? \\
When will it be? 

Oh,\marginnote{13.1} when will I rise up, \\>intent on attaining freedom from death, \\
hearing, in the mountain cave, \\
the cry of the crested peacock in the forest? \\
When will it be? 

Oh,\marginnote{14.1} when will I cross the Ganges, \textsanskrit{Yamunā}, \\
and \textsanskrit{Sarasvatī} rivers, the \textsanskrit{Pātāla} country, \\
and the dangerous \textsanskrit{Baḷavāmukha} sea, \\
by psychic power unobstructed? When will it be? 

Oh,\marginnote{15.1} when will I be devoted to absorption, \\
rejecting entirely the signs of beauty, \\
splitting apart desire for sensual stimulation, \\
like an elephant that wanders free of ties? \\>When will it be? 

Oh,\marginnote{16.1} when will I realize the teaching of the great seer \\
and be content, like a pauper in debt \\
harassed by creditors, who finds a hidden treasure? \\
When will it be? 

For\marginnote{17.1} many years you begged me, \\
“Enough of living in a house for you!” \\
Why do you not urge me on, mind, \\
now that I’ve gone forth as an ascetic? 

Didn’t\marginnote{18.1} you entice me, mind: \\
“On the Mountainfold, the birds with colorful wings, \\
greeting the thunder, Mahinda’s voice,\footnote{Mahinda is “Great Indra”, the god of thunder. } \\
will delight you as you meditate in the forest?” 

In\marginnote{19.1} my family circle, friends, loved ones, and relatives; \\
and in the world, sports and play, \\>and sensual pleasures; \\
all these I gave up when I entered this life: \\
and even then you’re not content with me, mind! 

This\marginnote{20.1} is mine alone, it doesn’t belong to others; \\
when it is time to don your armor, why lament? \\
Observing that all this is unstable, \\
I went forth, seeking the state free of death. 

The\marginnote{21.1} methodical teacher, supreme among people, \\
great physician, \\>guide for those who wish to train, said: \\
“The mind fidgets like a monkey, \\
so it’s very hard to control if you are not free of lust.” 

Sensual\marginnote{22.1} pleasures are diverse, sweet, delightful; \\
an ignorant ordinary person is bound to them. \\
Seeking to be reborn again, they wish for suffering; \\
led on by their mind, they’re relegated to hell. 

“Staying\marginnote{23.1} in the grove resounding with cries \\
of peacocks and herons, \\>and adorned by leopards and tigers, \\
abandon concern for the body, without fail!” \\
So you used to urge me, mind. 

“Develop\marginnote{24.1} the absorptions and spiritual faculties, \\
the powers, awakening factors, and immersion; \\
realize the three knowledges \\>in the teaching of the Buddha!” \\
So you used to urge me, mind. 

“Develop\marginnote{25.1} the eightfold path to realize freedom from death \\
emancipating, plunging into the end of all suffering, \\
and cleansing all defilements!” \\
So you used to urge me, mind. 

“Reflect\marginnote{26.1} rationally on the aggregates as suffering, \\
and abandon that from which suffering arises; \\
make an end of suffering in this very life!” \\
So you used to urge me, mind. 

“Rationally\marginnote{27.1} discern that impermanence is suffering, \\
that emptiness is non-self, and that misery is death. \\
Uproot the wandering mind!” \\
So you used to urge me, mind. 

“Bald,\marginnote{28.1} unsightly, accursed, \\
seek alms amongst families, bowl in hand. \\
Devote yourself to the word of the teacher, the great seer!” \\
So you used to urge me, mind. 

“Wander\marginnote{29.1} the streets well-restrained, \\
mentally unsnared to families and sensual pleasures, \\
like the full moon on a bright night!” \\
So you used to urge me, mind. 

“Be\marginnote{30.1} a wilderness-dweller and an alms-eater, \\
one who lives in charnel grounds, a rag-robe wearer, \\
one who never lies down, \\>always delighting in ascetic practices.” \\
So you used to urge me, mind. 

Mind,\marginnote{31.1} when you urge me to the impermanent and unstable, \\
you’re acting like someone who plants trees, \\
then, when they’re about to fruit, \\
wishes to cut down the very same trees. 

Incorporeal\marginnote{32.1} mind, far-traveler, lone-wanderer: \\
I won’t do your bidding any more. \\
Sensual pleasures are suffering, painful, \\>and very dangerous; \\
I’ll wander with my mind focused only on extinguishment. 

I\marginnote{33.1} didn’t go forth due to bad luck or shamelessness, \\
or due to a whim or banishment, \\
nor for the sake of a livelihood; \\
it was because I agreed \\>to the promise you made, mind. 

“Having\marginnote{34.1} few wishes, abandoning disparagement, \\
the stilling of suffering: \\>these are praised by true persons.” \\
So you used to urge me, mind, \\
but now you keep on with your old habits! 

Craving,\marginnote{35.1} ignorance, the loved and unloved, \\
pretty sights, pleasant feelings, \\
and the delightful kinds of sensual stimulation: \\
I’ve vomited them up, I can’t swallow them back. 

I’ve\marginnote{36.1} done your bidding everywhere, mind! \\
For many births, I’ve done nothing to upset you. \\
Yet the creation in myself is because of your ingratitude—\footnote{Read \textit{\textsanskrit{akataññutāya}}. The “creation in myself” (\textit{ajjhattasambhavo}) is craving. } \\
for a long time I’ve transmigrated \\>in the suffering you’ve made. 

Only\marginnote{37.1} you, mind, make a brahmin; \\
you make an aristocrat or a royal seer. \\
Sometimes we become peasants or menials; \\
and life as a god is also on account of you. 

You\marginnote{38.1} alone make us titans; \\
because of you we’re born in hell. \\
Then sometimes we become animals, \\
and life as a ghost is also on account of you. 

Come\marginnote{39.1} what may, you won’t betray me again, \\
dazzling me with your ever-changing display! \\
You play with me like I’m mad—\\
but how have I ever failed you, mind? 

In\marginnote{40.1} the past my mind wandered \\
how it wished, where it liked, as it pleased. \\
Now I’ll carefully guide it, \\
as a trainer with a hook guides a rutting elephant. 

The\marginnote{41.1} teacher willed that this world appear to me \\
as impermanent, unstable, insubstantial. \\
Mind, let me leap into the victor’s teaching, \\
carry me over the great flood, so hard to pass. 

Things\marginnote{42.1} have changed, mind! \\
Nothing could make me return to your control! \\
I’ve gone forth in the teaching of the great seer, \\
those like me don’t come to ruin. 

Mountains,\marginnote{43.1} oceans, rivers, the earth; \\
the four quarters, the intermediate directions, \\>below and in the sky; \\
the three realms of existence \\>are all impermanent and troubled—\\
where can you go to find happiness, mind? 

Mind,\marginnote{44.1} what will you do to someone \\>who has made the ultimate commitment? \\
Nothing could make me a follower \\>under your control, mind; \\
I’d never touch a bellows \\>with a mouth open at each end; \\
curse this mortal frame flowing with nine streams! 

You’ve\marginnote{45.1} ascended the mountain peak, \\>full of nature’s beauty, \\
frequented by boars and antelopes, \\
a grove sprinkled with fresh water in the monsoon; \\
and there you’ll be happy in your cave-home. 

Peacocks\marginnote{46.1} with beautiful necks and crests, \\
colorful tail-feathers and wings, \\
crying out at the resounding thunder: \\
they’ll delight you as you meditate in the forest. 

When\marginnote{47.1} the heavens have rained, and the grass is four inches high, \\
and the grove is full of flowers like a cloud, \\
between the mountains, like the fork of a tree, I’ll lie; \\
it will be as soft as cotton-buds. 

I’ll\marginnote{48.1} act as a master does: \\
let whatever I get be enough for me. \\
And that’s why I’ll make you as supple \\
as a tireless worker makes a cat-skin bag. 

I’ll\marginnote{49.1} act as a master does: \\
let whatever I get be enough for me. \\
I’ll control you with my energy, \\
as a skilled trainer controls an elephant with a hook. 

Now\marginnote{50.1} that you’re well-tamed and reliable, \\
I can use you, \\>like a trainer uses a straight-running horse, \\
to practice the path so full of grace, \\
cultivated by those who take care of their minds. 

I\marginnote{51.1} shall strongly fasten you to a meditation subject, \\
as an elephant is tied to a post with firm rope. \\
You’ll be well-guarded by me, \\>well-developed by mindfulness, \\
and unattached to rebirth in all states of existence. 

With\marginnote{52.1} wisdom you’ll cut short the one following the wrong path, \\
curb them by practice, and settle them on the right path. \\
Having seen arising and passing away \\>with respect to the cause of suffering,\footnote{Take \textit{\textsanskrit{samudayaṁ}} as accusative of relation, equivalent to locative (although treated as genitive in commentary). } \\
you’ll be an heir to the greatest teacher. 

Under\marginnote{53.1} the sway of the four distortions, mind, \\
you dragged me around like a bull in a pit; \\
but now you won’t associate \\>with the great sage of compassion, \\
the cutter of fetters and bonds? 

Like\marginnote{54.1} a deer roaming free in the colorful forest, \\
I’ll ascend the lovely mountain wreathed in monsoon clouds, \\
and rejoice to be on that hill, free of folk—\\
there is no doubt you’ll perish, mind. 

The\marginnote{55.1} men and women who live \\>under your will and command, \\
whatever pleasure they experience, \\
they are ignorant and fall under \textsanskrit{Māra}’s control; \\
loving life, they’re your disciples, mind. 

%
\end{verse}

%
\addtocontents{toc}{\let\protect\contentsline\protect\nopagecontentsline}
\part*{The Book of the Sixties }
\addcontentsline{toc}{part}{The Book of the Sixties }
\markboth{}{}
\addtocontents{toc}{\let\protect\contentsline\protect\oldcontentsline}

%
\addtocontents{toc}{\let\protect\contentsline\protect\nopagecontentsline}
\chapter*{Chapter One }
\addcontentsline{toc}{chapter}{\tocchapterline{Chapter One }}
\addtocontents{toc}{\let\protect\contentsline\protect\oldcontentsline}

%
\section*{{\suttatitleacronym Thag 20.1}{\suttatitletranslation Mahāmoggallāna }{\suttatitleroot Mahāmoggallānattheragāthā}}
\addcontentsline{toc}{section}{\tocacronym{Thag 20.1} \toctranslation{Mahāmoggallāna } \tocroot{Mahāmoggallānattheragāthā}}
\markboth{Mahāmoggallāna }{Mahāmoggallānattheragāthā}
\extramarks{Thag 20.1}{Thag 20.1}

\begin{verse}%
“Living\marginnote{1.1} in the wilderness, eating only almsfood, \\
happy with the scraps in our bowls, \\
let us tear apart the army of death, \\
while remaining serene within. 

Living\marginnote{2.1} in the wilderness, eating only almsfood, \\
happy with the scraps in our bowls, \\
let us crush the army of death, \\
as an elephant a hut of reeds. 

Living\marginnote{3.1} at the foot of a tree, persistent, \\
happy with the scraps in our bowls, \\
let us tear apart the army of death, \\
while remaining serene within. 

Living\marginnote{4.1} at the foot of a tree, persistent, \\
happy with the scraps in our bowls, \\
let us crush the army of death, \\
as an elephant a hut of reeds.” 

“You\marginnote{5.1} little hut, made of a chain of bones, \\
sewn together with flesh and sinew; \\
curse you mortal frame, you stink, \\
you cherish the parts of others! 

You\marginnote{6.1} sack of dung encased in skin! \\
You demoness with horns on your chest! \\
O body, you have nine streams \\
that are flowing all the time. 

With\marginnote{7.1} its nine streams, \\
your body stinks, full of dung. \\
A monk seeking purity \\
would avoid it like excrement. 

If\marginnote{8.1} they knew you \\
like I do, \\
they’d keep far away, \\
like a cesspit in the monsoon.” 

“So\marginnote{9.1} it is, great hero! \\
As you say, ascetic! \\
But some flounder here \\
like an old bull stuck in a bog.” 

“Whoever\marginnote{10.1} might think \\
of making the sky yellow, \\
or some other color, \\
would only trouble themselves. 

This\marginnote{11.1} mind is like the sky: \\
serene inside itself. \\
Evil-minded one, don’t attack me, \\
you’ll end up like a moth in a mass of fire.” 

“See\marginnote{12.1} this fancy puppet, \\
a body built of sores, \\
diseased, obsessed over, \\
in which nothing lasts at all. 

See\marginnote{13.1} this fancy figure, \\
with its gems and earrings; \\
it is bones encased in skin, \\
made pretty by its clothes. 

Rouged\marginnote{14.1} feet \\
and powdered face \\
may be enough to beguile a fool, \\
but not a seeker of the far shore. 

Hair\marginnote{15.1} in eight braids \\
and eyeshadow \\
may be enough to beguile a fool, \\
but not a seeker of the far shore. 

A\marginnote{16.1} rotting body all adorned \\
like a freshly painted makeup box \\
may be enough to beguile a fool, \\
but not a seeker of the far shore. 

The\marginnote{17.1} hunter laid his snare, \\
but the deer didn’t spring the trap. \\
I’ve eaten the bait and now I go, \\
leaving the trapper to lament. 

The\marginnote{18.1} hunter’s trap is broken, \\
but the deer didn’t spring the trap. \\
I’ve eaten the bait and now I go, \\
leaving the deer-hunter to grieve.” 

“Then\marginnote{19.1} there was terror! \\
Then they had goosebumps! \\
When \textsanskrit{Sāriputta}, \\>endowed with many fine qualities,\footnote{When this verse is spoken at the Buddha’s death, it says “all fine qualities”. } \\
became quenched. 

Oh!\marginnote{20.1} Conditions are impermanent, \\
their nature is to rise and fall; \\
having arisen, they cease; \\
their stilling is blissful.” 

“Those\marginnote{21.1} who see the five aggregates \\
as other, not as self, \\
penetrate a subtle thing, \\
like a hair-tip with an arrow. 

Those\marginnote{22.1} who see conditions \\
as other, not as self, \\
pierce a fine thing, \\
like a hair-tip with an arrow.” 

“Like\marginnote{23.1} they’re struck by a sword, \\
like their head was on fire, \\
a mendicant should wander mindful, \\
to give up sensual desire. 

Like\marginnote{24.1} they’re struck by a sword, \\
like their head was on fire, \\
a mendicant should wander mindful, \\
to give up desire for rebirth.” 

“Urged\marginnote{25.1} by the developed one, \\
who bore his final body, \\
I shook the stilt longhouse of \textsanskrit{Migāra}’s mother \\
with my big toe.” 

“Not\marginnote{26.1} by being slack, \\
or with little strength \\
is extinguishment realized, \\
the release from all ties.” 

“This\marginnote{27.1} young monk, \\
this best of men, \\
bears his final body, \\
having vanquished \textsanskrit{Māra} and his mount.” 

“Lightning\marginnote{28.1} flashes down \\
on the cleft of \textsanskrit{Vebhāra} and \textsanskrit{Paṇḍava}. \\
But in the mountain cleft he is absorbed in \textsanskrit{jhāna}—\\
the son of the Buddha, inimitable and unaffected.” 

“Calm\marginnote{29.1} and still, \\
the sage in his remote lodging, \\
the heir to the best of Buddhas, \\
is honored even by the Divinity. 

Calm\marginnote{30.1} and still, \\
the sage in his remote lodging, \\
is heir to the best of Buddhas: \\
Brahmin, you should honor Kassapa! 

Even\marginnote{31.1} if someone were to be born again and again \\
a hundred times in the human realm, \\
and always as a brahmin, \\
a student accomplished in the Vedas; 

and\marginnote{32.1} if he were to become a reciter, \\
a master of the three Vedas: \\
honoring such a person \\
isn’t worth a sixteenth of that. 

One\marginnote{33.1} who attains the eight liberations \\
forwards and backwards \\
before breakfast, \\
and then goes on almsround—

don’t\marginnote{34.1} attack such a mendicant! \\
Don’t ruin yourself, brahmin! \\
Let your heart have trust \\
in the perfected one, the unaffected; \\
quickly venerate him with joined palms: \\
don’t let your head explode!” 

“If\marginnote{35.1} you prioritize transmigration, \\
you don’t see the true teaching. \\
You’re following a twisted path, \\
a bad path that will lead you down. 

Like\marginnote{36.1} a worm smeared with dung, \\
he is besotted with conditions. \\
Consumed by gain and honor, \\
\textsanskrit{Poṭṭhila} goes on, hollow.” 

“See\marginnote{37.1} \textsanskrit{Sāriputta} coming! \\
It is good to see him; \\
he is freed in both ways, \\
serene inside himself; 

free\marginnote{38.1} of thorns, with yoking ended, \\
master of the three knowledges, \\>conqueror of death; \\
worthy of offerings, \\
a supreme field of merit for the people.” 

“These\marginnote{39.1} many gods, \\
powerful and glorious, \\
all 10,000 of them, \\
are priests of Divinity. \\
They stand with joined palms \\
honoring \textsanskrit{Moggallāna}: 

‘Homage\marginnote{40.1} to you, O thoroughbred! \\
Homage to you, supreme among men! \\
Since your defilements are ended, \\
you, sir, are worthy of teacher’s offerings.’ 

Venerated\marginnote{41.1} by men and gods, \\
he has arisen, the master of death. \\
He is unsmeared by conditions, \\
as a lotus-flower by water.” 

“The\marginnote{42.1} mendicant by whom the galaxy \\
with the age of the Divinity are known in an hour—\\
that master of psychic ability sees the gods \\
at the time they pass away and are reborn.” 

“\textsanskrit{Sāriputta}\marginnote{43.1} is full of wisdom, \\
ethics, and peace. \\
Even a mendicant who has crossed over \\
might at best equal him. 

But\marginnote{44.1} in a moment I can create the likenesses \\
of ten million times 100,000 people! \\
I’m skilled in transformations; \\
I’m a master of psyshic powers. 

A\marginnote{45.1} member of the \textsanskrit{Moggallāna} clan, \\>attained to perfection and mastery \\
in immersion and knowledge, \\>wise in the teachings of the unattached, \\
with serene faculties, has burst his bonds \\
like an elephant bursts a vine. 

I’ve\marginnote{46.1} served the teacher \\
and fulfilled the Buddha’s instructions. \\
The heavy burden is laid down, \\
the conduit to rebirth is eradicated. 

I’ve\marginnote{47.1} attained the goal \\
for the sake of which I went forth \\
from the lay life to homelessness—\\
the end of all fetters.” 

“What\marginnote{48.1} kind of hell was that, \\
where \textsanskrit{Dūsī} was roasted \\
after attacking the disciple Vidhura \\
along with the brahmin Kakusandha? 

There\marginnote{49.1} were 100 iron spikes, \\
each one individually painful. \\
That’s the kind of hell \\
where \textsanskrit{Dūsī} was roasted \\
after attacking the disciple Vidhura \\
along with the brahmin Kakusandha. 

Dark\marginnote{50.1} One, if you attack \\
a mendicant who directly knows this, \\
a disciple of the Buddha, \\
you’ll fall into suffering. 

There\marginnote{51.1} are mansions that last an eon \\
standing in the middle of a lake. \\
Sapphire-colored, brilliant, \\
they sparkle and shine. \\
Dancing there are nymphs \\
shining in all different colors. 

Dark\marginnote{52.1} One, if you attack \\
a mendicant who directly knows this, \\
a disciple of the Buddha, \\
you’ll fall into suffering. 

I’m\marginnote{53.1} the one who, urged by the Buddha, \\
shook the stilt longhouse of \textsanskrit{Migāra}’s mother \\
with his big toe \\
as the \textsanskrit{Saṅgha} of mendicants watched. 

Dark\marginnote{54.1} One, if you attack \\
a mendicant who directly knows this, \\
a disciple of the Buddha, \\
you’ll fall into suffering. 

I’m\marginnote{55.1} the one who shook the Palace of Victory \\
with his big toe \\
owing to psychic power, \\
inspiring deities to awe. 

Dark\marginnote{56.1} One, if you attack \\
a mendicant who directly knows this, \\
a disciple of the Buddha, \\
you’ll fall into suffering. 

I’m\marginnote{57.1} the one who asked Sakka \\
in the Palace of Victory: \\
‘Sir, I hope you recall \\
the one who is freed through the ending of craving?’\footnote{\textsanskrit{Moggallāna} asked Sakka if he remembered the teaching on this topic that he had received from the Buddha (\href{https://suttacentral.net/mn37/en/sujato\#8.1}{MN 37:8.1}). | The compound \textit{\textsanskrit{taṇhākkhayavimuttiyo}} is translated as a feminine plural by Norman in \emph{Elders’ Verses}, but \href{https://suttacentral.net/mn37/en/sujato\#2.2}{MN 37:2.2} refers to “the mendicant who is freed” in singular. Resolve to \textit{\textsanskrit{taṇhākkhayavimutti} yo}; \textit{vimutti} agrees with \textit{yo} as the nominative singular of the masculine agent noun in \textit{-in}, which occurs in the same phrase at \href{https://suttacentral.net/an4.38/en/sujato\#6.2}{AN 4.38:6.2} and \href{https://suttacentral.net/iti55/en/sujato\#4.2}{Iti 55:4.2}. } \\
And I’m the one to whom Sakka \\
admitted the truth when asked. 

Dark\marginnote{58.1} One, if you attack \\
a mendicant who directly knows this, \\
a disciple of the Buddha, \\
you’ll fall into suffering. 

I’m\marginnote{59.1} the one who asked the Divinity \\
in the Hall of Justice before the assembly: \\
‘Reverend, do you still have the same view \\
that you had in the past? \\
Or do you see the radiance \\
transcending the realm of divinity?’ 

And\marginnote{60.1} I’m the one to whom the Divinity \\
admitted the truth when asked. \\
‘Good sir, I don’t have that view \\
that I had in the past. 

I\marginnote{61.1} see the radiance \\
transcending the realm of divinity. \\
So how could I say today \\
that I am permanent and eternal?’ 

Dark\marginnote{62.1} One, if you attack \\
a mendicant who directly knows this, \\
a disciple of the Buddha, \\
you’ll fall into suffering. 

I’m\marginnote{63.1} the one who touched the peak of Mount Neru \\
using the power of meditative liberation. \\
I’ve visited the forests of the people \\
who dwell in the Eastern Continent. 

Dark\marginnote{64.1} One, if you attack \\
a mendicant who directly knows this, \\
a disciple of the Buddha, \\
you’ll fall into suffering. 

Though\marginnote{65.1} a fire doesn’t think: \\
‘I’ll burn the fool!’ \\
Still the fool who attacks \\
the fire gets burnt. 

In\marginnote{66.1} the same way, \textsanskrit{Māra}, \\
in attacking the Realized One, \\
you’ll only burn yourself, \\
like a fool touching the flames. 

\textsanskrit{Māra}’s\marginnote{67.1} done a bad thing \\
in attacking the Realized One. \\
Wicked One, do you imagine that \\
your wickedness won’t bear fruit? 

Your\marginnote{68.1} deeds heap up wickedness \\
that will last a long time, Terminator! \\
Give up on the Buddha, \textsanskrit{Māra}! \\
And hold no hope for the mendicants!” 

That\marginnote{69.1} is how, in the \textsanskrit{Bhesekaḷā} grove,\footnote{The commentary says that this verse was added at the Council. } \\
the mendicant condemned \textsanskrit{Māra}. \\
That spirit, downcast, \\
disappeared right there. 

%
\end{verse}

\scendsutta{That is how these verses were recited by the senior venerable \textsanskrit{Mahāmoggallāna}. }

%
\addtocontents{toc}{\let\protect\contentsline\protect\nopagecontentsline}
\part*{The Great Book }
\addcontentsline{toc}{part}{The Great Book }
\markboth{}{}
\addtocontents{toc}{\let\protect\contentsline\protect\oldcontentsline}

%
\addtocontents{toc}{\let\protect\contentsline\protect\nopagecontentsline}
\chapter*{Chapter One}
\addcontentsline{toc}{chapter}{\tocchapterline{Chapter One}}
\addtocontents{toc}{\let\protect\contentsline\protect\oldcontentsline}

%
\section*{{\suttatitleacronym Thag 21.1}{\suttatitletranslation Vaṅgīsa }{\suttatitleroot Vaṅgīsattheragāthā}}
\addcontentsline{toc}{section}{\tocacronym{Thag 21.1} \toctranslation{Vaṅgīsa } \tocroot{Vaṅgīsattheragāthā}}
\markboth{Vaṅgīsa }{Vaṅgīsattheragāthā}
\extramarks{Thag 21.1}{Thag 21.1}

\begin{verse}%
“Now\marginnote{1.1} that I’ve gone forth \\
from the lay life to homelessness, \\
I’m overrun \\
by the rude thoughts of the Dark One. 

Even\marginnote{2.1} if a thousand mighty princes and great archers, \\
well trained, with strong bows, \\
were to completely surround me; \\
I would never flee. 

And\marginnote{3.1} even if women come, \\
many more than that, \\
they won’t scare me, \\
for I stand firm in the teaching. 

I\marginnote{4.1} heard this with my own ears \\
from the Buddha, kinsman of the Sun, \\
about the path going to extinguishment; \\
that’s what delights my mind. 

Wicked\marginnote{5.1} One, if you come near me \\
as I meditate like this, \\
I’ll make sure that you, Death, \\
won’t even see the path I take.” 

“Giving\marginnote{6.1} up discontent and desire, \\
along with all thoughts of domestic life, \\
they wouldn’t get entangled in anything; \\
unentangled, disentangled: that’s a real mendicant. 

Whether\marginnote{7.1} on this earth or in the sky, \\
whatever in the world is included in form \\
wears out, it is all impermanent; \\
the thoughtful live having comprehended this truth. 

People\marginnote{8.1} are bound to their attachments, \\
to what is seen, heard, felt, and thought. \\
Unstirred, dispel desire for these things; \\
for one called ‘a sage’ does not cling to them. 

Attached\marginnote{9.1} to the sixty wrong views, \\>and full of their own opinions, \\
ordinary people are fixed in wrong principles. \\
But that mendicant wouldn’t join a sectarian group, \\
still less would they utter lewd speech. 

Clever,\marginnote{10.1} long serene, \\
free of deceit, alert, without envy, \\
the sage has reached the state of peace; \\
quenched, he awaits his time.” 

“Give\marginnote{11.1} up conceit, Gotama! \\
Completely abandon the different kinds of conceit! \\
Besotted with the different kinds of conceit, \\
you’ve had regrets for a long time. 

Smeared\marginnote{12.1} by smears and slain by conceit, \\
people fall into hell. \\
When people slain by conceit are reborn in hell, \\
they grieve for a long time. 

But\marginnote{13.1} a mendicant who practices rightly, \\
winner of the path, never grieves. \\
They enjoy happiness and a good reputation, \\
and they rightly call him a ‘Seer of Truth’. 

So\marginnote{14.1} don’t be hard-hearted, be energetic, \\
with hindrances given up, be pure. \\
Then with conceit given up completely, \\
use knowledge to make an end, and be calmed.” 

“I’ve\marginnote{15.1} got a burning desire for pleasure; \\
my mind is on fire! \\
Please, out of compassion, Gotama, \\
tell me how to quench the flames.” 

“Your\marginnote{16.1} mind is on fire \\
because of a perversion of perception. \\
Turn away from the sign \\
that’s attractive, provoking lust. 

With\marginnote{17.1} your mind unified and serene, \\
meditate on the ugly aspects of the body. \\
With mindfulness immersed in the body, \\
be full of disillusionment. 

Meditate\marginnote{18.1} on the signless, \\
give up the tendency to conceit; \\
and when you comprehend conceit, \\
you will live at peace.” 

“Speak\marginnote{19.1} only such words \\
that do not hurt yourself \\
nor harm others; \\
such speech is truly well spoken. 

Speak\marginnote{20.1} only pleasing words, \\
words gladly welcomed. \\
Pleasing words are those \\
that bring nothing bad to others. 

Truth\marginnote{21.1} itself is the undying word: \\
this is an ancient teaching. \\
Good people say that the teaching and its meaning \\
are grounded in the truth. 

The\marginnote{22.1} words spoken by the Buddha \\
for finding the sanctuary, extinguishment, \\
for making an end of suffering: \\
this really is the best kind of speech.” 

“Deep\marginnote{23.1} in wisdom, intelligent, \\
expert in what is the path \\>and what is not the path; \\
\textsanskrit{Sāriputta}, so greatly wise, \\
teaches Dhamma to the mendicants. 

He\marginnote{24.1} teaches in brief, \\
or he speaks at length. \\
His call, like a myna bird, \\
overflows with inspiration. 

While\marginnote{25.1} he teaches \\
the mendicants listen to his sweet voice, \\
sounding attractive, \\
clear and graceful. \\
They listen joyfully, \\
their hearts elated.” 

“Today,\marginnote{26.1} on the fifteenth day sabbath, \\
five hundred monks have gathered together \\>to purify their precepts. \\
These untroubled sages \\>have cut off their fetters and bonds, \\
they will not be reborn again. 

Just\marginnote{27.1} as a wheel-rolling monarch \\
surrounded by ministers \\
travels all around this \\
land that’s girt by sea. 

So\marginnote{28.1} disciples with the three knowledges, \\
conquerors of death, \\
revere the winner of the battle, \\
the unsurpassed caravan leader. 

All\marginnote{29.1} are sons of the Blessed One—\\
there is no rubbish here. \\
I bow to the kinsman of the Sun, \\
destroyer of the dart of craving.” 

“Over\marginnote{30.1} a thousand mendicants \\
revere the Holy One \\
as he teaches the immaculate Dhamma, \\
extinguishment, fearing nothing from any quarter. 

They\marginnote{31.1} listen to the immaculate Dhamma \\
taught by the fully awakened Buddha; \\
the Buddha is so brilliant, \\
at the fore of the mendicant \textsanskrit{Saṅgha}. 

Blessed\marginnote{32.1} One, your name is ‘Giant’, \\
seventh of the sages. \\
You are like a great cloud \\
that rains on your disciples. 

I’ve\marginnote{33.1} left my day’s meditation, \\
out of desire to see the teacher. \\
Great hero, your disciple \textsanskrit{Vaṅgīsa} \\
bows at your feet.” 

“Having\marginnote{34.1} overcome \textsanskrit{Māra}’s devious path, \\
you wander with hard-heartedness dissolved.\footnote{\href{https://suttacentral.net/sn8.8/en/sujato\#8.2}{SN 8.8:8.2} has the second person \textit{carasi}, and it seems reasonable to assume that \textsanskrit{Vaṅgīsa} is continuing to address the Buddha directly. | From the same place accept the reading \textit{khila} rather than \textit{\textsanskrit{khīla}}. } \\
See him, the liberator from bonds, unattached,\footnote{Here \textsanskrit{Vaṅgīsa} turns to address the monks directly. } \\
analyzing the teaching. 

You\marginnote{35.1} have explained in many ways \\
the path to cross the flood. \\
The Seers of Truth stand unfaltering\footnote{These lines follow Bodhi’s reading at \href{https://suttacentral.net/sn8.8/en/sujato\#9.3}{SN 8.8:9.3}. } \\
in the freedom from death you’ve explained. 

As\marginnote{36.1} the bringer of light who has pierced the truth, \\
you’ve seen what lies beyond all realms.\footnote{\textit{Ṭhiti} is \textit{\textsanskrit{viññāṇaṭṭhiti}} (“stations of consciousness”), i.e. realms in which consciousness can be reborn. } \\
When you saw and realized this for yourself, \\
you taught it first to the group of five. 

When\marginnote{37.1} the Dhamma has been so well taught, \\
how could those who know it be negligent? \\
That’s why, being diligent, we should always train \\
respectfully in the Buddha’s teaching.” 

“The\marginnote{38.1} senior monk who was awakened \\>right after the Buddha,\footnote{Spoken at \href{https://suttacentral.net/sn8.9/en/sujato\#3.1}{SN 8.9:3.1}, while the first two lines are included in \textsanskrit{Koṇḍañña}’s verses at \href{https://suttacentral.net/thag21.1/en/sujato\#38.1}{Thag 21.1:38.1}. } \\
\textsanskrit{Koṇḍañña}, is keenly energetic. \\
He regularly gains blissful meditative states, \\
and the three kinds of seclusion. 

Whatever\marginnote{39.1} can be attained by a disciple \\
who does the Teacher’s bidding, \\
he has attained it all, \\
through diligently training himself. 

With\marginnote{40.1} great power and the three knowledges, \\
expert in comprehending the minds of others, \\
\textsanskrit{Koṇḍañña}, the heir to the Buddha, \\
bows at the Teacher’s feet.” 

“As\marginnote{41.1} the sage, who has gone beyond suffering, \\
sits upon the mountain slope, \\
he is revered by disciples with the three knowledges, \\
conquerors of death. 

\textsanskrit{Moggallāna},\marginnote{42.1} of great psychic power, \\
comprehends with his mind, \\
scrutinizing their minds, \\
liberated, free of attachments. 

So\marginnote{43.1} they revere Gotama, \\
the sage gone beyond suffering, \\
who is endowed with all path factors, \\
and with a multitude of attributes.” 

“Like\marginnote{44.1} the moon on a cloudless night, \\
like the shining immaculate sun, \\
so too \textsanskrit{Aṅgīrasa}, O great sage, \\
your glory outshines the entire world.” 

“We\marginnote{45.1} used to wander, drunk on poetry, \\
from village to village, town to town. \\
Then we saw the Buddha, \\
who has gone beyond all things. 

He,\marginnote{46.1} the sage gone beyond suffering, \\
taught me the Dhamma. \\
When we heard the Dhamma, \\>we became confident—\\
faith arose in us. 

Hearing\marginnote{47.1} him speak of \\
the aggregates, the sense-fields, \\
and the elements, I understood; \\
and then I went forth to homelessness. 

It\marginnote{48.1} is for the benefit of many \\
that the Realized Ones arise—\\
the men and women \\
who follow their instructions. 

It\marginnote{49.1} is truly for their benefit \\
that the sage realized awakening—\\
for the monks and for the nuns \\
who see that they’ve reached certainty. 

The\marginnote{50.1} Clear-eyed One, the Buddha, \\
the kinsman of the Sun, \\
has well taught the four noble truths \\
out of sympathy for living creatures. 

Suffering,\marginnote{51.1} suffering’s origin, \\
suffering’s transcendence, \\
and the noble eightfold path \\
that leads to the stilling of suffering. 

As\marginnote{52.1} these things were taught, \\
so I have seen them. \\
I’ve realized my own true goal, \\
and fulfilled the Buddha’s instructions. 

It\marginnote{53.1} was so welcome for me \\
to be in the presence of the Buddha. \\
Of the well-explained teachings, \\
I arrived at the best. 

I’ve\marginnote{54.1} realized the perfection of direct knowledge; \\
my clairaudience is purified; \\
I am master of three knowledges, \\>attained in psychic power, \\
I’m expert at reading the minds of others.” 

“I\marginnote{55.1} ask the teacher unrivaled in wisdom, \\
who has cut off all doubts in this very life: \\
a monk has died at \textsanskrit{Aggāḷava}, who was \\
well-known, famous, and quenched. 

Nigrodhakappa\marginnote{56.1} was his name; \\
it was given to that brahmin by you, Blessed One. \\
He wandered in your honor, yearning for freedom, \\
energetic, a resolute Seer of Truth. 

O\marginnote{57.1} Sakyan, All-seer, \\
we all wish to know about that disciple. \\
Our ears are eager to hear, \\
for you are the most excellent teacher. 

Cut\marginnote{58.1} off our doubt, declare this to us; \\
your wisdom is vast, tell us of his quenching! \\
All-seer, speak among us, \\
like the thousand-eyed Sakka \\>in the midst of the gods! 

Whatever\marginnote{59.1} ties there are, or paths to delusion, \\
or things on the side of unknowing, \\>or that are bases of doubt \\
vanish on reaching a Realized One, \\
for his eye is the best of all people’s. 

If\marginnote{60.1} no man were ever to disperse corruptions, \\
like the wind dispersing the clouds, \\
darkness would veil the whole world; \\
not even brilliant men would shine. 

The\marginnote{61.1} attentive are bringers of light; \\
my hero, that is what I think of you. \\
We’ve come for your discernment and knowledge: \\
here in this assembly, declare to us about \textsanskrit{Kappāyana}. 

Swiftly\marginnote{62.1} send forth your sweet, sweet voice, \\
like a goose stretching its neck, gently honking, \\
lucid-flowing, with lovely tone: \\
alert, we all listen to you. 

You\marginnote{63.1} have entirely abandoned birth and death; \\
restrained and pure, speak the Dhamma! \\
For ordinary people \\>have no wish-granter,\footnote{\textit{\textsanskrit{Kāmakāro}}: “wish-granter”. Ordinary people cannot simply get what they want. } \\
but Realized Ones \\>have a comprehensibility-granter.\footnote{\textit{\textsanskrit{Saṅkheyyakāro}}: “comprehensibility-granter”. \textit{\textsanskrit{Saṅkheyya}} is “calculable, comprehensible”, not “after comprehension”. The Buddha has a gift for explaining things in a way that makes them comprehensible. } 

Your\marginnote{64.1} answer is definitive, and we will adopt it, \\
for you have perfect understanding. \\
We raise our joined palms one last time, \\
one of unrivaled wisdom, \\>don’t deliberately confuse us. 

Knowing\marginnote{65.1} the teaching of the noble ones \\>from top to bottom, \\
one of unrivaled energy, \\>don’t deliberately confuse us. \\
Like a man in the baking summer sun \\>would long for water, \\
I long for your voice, so let the sound rain forth. 

Surely\marginnote{66.1} \textsanskrit{Kappāyana} did not \\>lead the spiritual life in vain? \\
Did he realize quenching, \\
or did he still have a residue? \\
Let us hear what kind of liberation he had!” 

“He\marginnote{67.1} cut off craving for name and form right here,” \\
\scspeaker{said the Buddha, }\\
“the river of darkness that had long lain within him. \\
He has entirely crossed over birth and death.” \\
So declared the Blessed One, the leader of the five. 

“Now\marginnote{68.1} that I have heard your words, \\
seventh of sages, I am confident. \\
My question, it seems, was not in vain, \\
the brahmin did not deceive me. 

As\marginnote{69.1} he spoke, so he acted; \\
he was a disciple of the Buddha. \\
He cut the net of death the deceiver, \\
so extended and strong. 

Blessed\marginnote{70.1} One, \textsanskrit{Kappāyana} saw \\
the starting point of grasping. \\
He has indeed gone far beyond \\
Death’s dominion so hard to pass. 

God\marginnote{71.1} of gods, best of men, I bow to you, \\
and to your son, \\
who followed your example, a great hero; \\
a giant, true-born son of a giant.” 

%
\end{verse}

\scendsutta{That is how these verses were recited by the senior venerable \textsanskrit{Vaṅgīsa}. }

\scendbook{The Verses of the Senior Monks are finished. }

%
\backmatter%
\chapter*{Colophon}
\addcontentsline{toc}{chapter}{Colophon}
\markboth{Colophon}{Colophon}

\section*{The Translator}

Bhikkhu Sujato was born as Anthony Aidan Best on 4/11/1966 in Perth, Western Australia. He grew up in the pleasant suburbs of Mt Lawley and Attadale alongside his sister Nicola, who was the good child. His mother, Margaret Lorraine Huntsman née Pinder, said “he’ll either be a priest or a poet”, while his father, Anthony Thomas Best, advised him to “never do anything for money”. He attended Aquinas College, a Catholic school, where he decided to become an atheist. At the University of WA he studied philosophy, aiming to learn what he wanted to do with his life. Finding that what he wanted to do was play guitar, he dropped out. His main band was named Martha’s Vineyard, which achieved modest success in the indie circuit. 

A seemingly random encounter with a roadside joey took him to Thailand, where he entered his first meditation retreat at Wat Ram Poeng, Chieng Mai in 1992. Feeling the call to the Buddha’s path, he took full ordination in Wat Pa Nanachat in 1994, where his teachers were Ajahn Pasanno and Ajahn Jayasaro. In 1997 he returned to Perth to study with Ajahn Brahm at Bodhinyana Monastery. 

He spent several years practicing in seclusion in Malaysia and Thailand before establishing Santi Forest Monastery in Bundanoon, NSW, in 2003. There he was instrumental in supporting the establishment of the Theravada bhikkhuni order in Australia and advocating for women’s rights. He continues to teach in Australia and globally, with a special concern for the moral implications of climate change and other forms of environmental destruction. He has published a series of books of original and groundbreaking research on early Buddhism. 

In 2005 he founded SuttaCentral together with Rod Bucknell and John Kelly. In 2015, seeing the need for a complete, accurate, plain English translation of the Pali texts, he undertook the task, spending nearly three years in isolation on the isle of Qi Mei off the coast of the nation of Taiwan. He completed the four main \textsanskrit{Nikāyas} in 2018, and the early books of the Khuddaka \textsanskrit{Nikāya} were complete by 2021. All this work is dedicated to the public domain and is entirely free of copyright encumbrance. 

In 2019 he returned to Sydney where he established Lokanta Vihara (The Monastery at the End of the World). 

\section*{Creation Process}

Translated from the Pali. The primary source was the \textsanskrit{Mahāsaṅgīti} edition, with reference to several English translations, especially those of K.R. Norman.

\section*{The Translation}

This translation aims to make a clear, readable, and accurate rendering of the \textsanskrit{Theragāthā}. The initial edition was by Jessica Walton and Bhikkhu Sujato was published in 2014 through SuttaCentral. A revised edition, bringing the terminology in line with the subsequently-translated four \textsanskrit{Nikāyas}, was published in 2019.

\section*{About SuttaCentral}

SuttaCentral publishes early Buddhist texts. Since 2005 we have provided root texts in Pali, Chinese, Sanskrit, Tibetan, and other languages, parallels between these texts, and translations in many modern languages. Building on the work of generations of scholars, we offer our contribution freely.

SuttaCentral is driven by volunteer contributions, and in addition we employ professional developers. We offer a sponsorship program for high quality translations from the original languages. Financial support for SuttaCentral is handled by the SuttaCentral Development Trust, a charitable trust registered in Australia.

\section*{About Bilara}

“Bilara” means “cat” in Pali, and it is the name of our Computer Assisted Translation (CAT) software. Bilara is a web app that enables translators to translate early Buddhist texts into their own language. These translations are published on SuttaCentral with the root text and translation side by side.

\section*{About SuttaCentral Editions}

The SuttaCentral Editions project makes high quality books from selected Bilara translations. These are published in formats including HTML, EPUB, PDF, and print.

You are welcome to print any of our Editions.

%
\end{document}