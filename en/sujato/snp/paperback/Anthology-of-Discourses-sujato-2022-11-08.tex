\documentclass[12pt,openany]{book}%
\usepackage{lastpage}%
%
\usepackage[inner=1in, outer=1in, top=.7in, bottom=1in, papersize={6in,9in}, headheight=13pt]{geometry}
\usepackage{polyglossia}
\usepackage[12pt]{moresize}
\usepackage{soul}%
\usepackage{microtype}
\usepackage{tocbasic}
\usepackage{realscripts}
\usepackage{epigraph}%
\usepackage{setspace}%
\usepackage{sectsty}
\usepackage{fontspec}
\usepackage{marginnote}
\usepackage[bottom]{footmisc}
\usepackage{enumitem}
\usepackage{fancyhdr}
\usepackage{extramarks}
\usepackage{graphicx}
\usepackage{verse}
\usepackage{relsize}
\usepackage{etoolbox}
\usepackage[a-3u]{pdfx}

\hypersetup{
colorlinks=true,
urlcolor=black,
linkcolor=black,
citecolor=black
}

% use a small amount of tracking on small caps
\SetTracking[ spacing = {25*,166, } ]{ encoding = *, shape = sc }{ 25 }

% add a blank page
\newcommand{\blankpage}{
\newpage
\thispagestyle{empty}
\mbox{}
\newpage
}

% define languages
\setdefaultlanguage[]{english}
\setotherlanguage[script=Latin]{sanskrit}

%\usepackage{pagegrid}
%\pagegridsetup{top-left, step=.25in}

% define fonts
% use if arno sanskrit is unavailable
%\setmainfont{Gentium Plus}
%\newfontfamily\Semiboldsubheadfont[]{Gentium Plus}
%\newfontfamily\Semiboldnormalfont[]{Gentium Plus}
%\newfontfamily\Lightfont[]{Gentium Plus}
%\newfontfamily\Marginalfont[]{Gentium Plus}
%\newfontfamily\Allsmallcapsfont[RawFeature=+c2sc]{Gentium Plus}
%\newfontfamily\Noligaturefont[Renderer=Basic]{Gentium Plus}
%\newfontfamily\Noligaturecaptionfont[Renderer=Basic]{Gentium Plus}
%\newfontfamily\Fleuronfont[Ornament=1]{Gentium Plus}

% use if arno sanskrit is available. display is applied to \chapter and \part, subhead to \section and \subsection. When specifying semibold, the italic must be defined.
\setmainfont[Numbers=OldStyle]{Arno Pro}
\newfontfamily\Semibolddisplayfont[BoldItalicFont = Arno Pro Semibold Italic Display]{Arno Pro Semibold Display} %
\newfontfamily\Semiboldsubheadfont[BoldItalicFont = Arno Pro Semibold Italic Subhead]{Arno Pro Semibold Subhead}
\newfontfamily\Semiboldnormalfont[BoldItalicFont = Arno Pro Semibold Italic]{Arno Pro Semibold}
\newfontfamily\Marginalfont[RawFeature=+subs]{Arno Pro Regular}
\newfontfamily\Allsmallcapsfont[RawFeature=+c2sc]{Arno Pro}
\newfontfamily\Noligaturefont[Renderer=Basic]{Arno Pro}
\newfontfamily\Noligaturecaptionfont[Renderer=Basic]{Arno Pro Caption}

% chinese fonts
\newfontfamily\cjk{Noto Serif TC}
\newcommand*{\langlzh}[1]{\cjk{#1}\normalfont}%

% logo
\newfontfamily\Logofont{sclogo.ttf}
\newcommand*{\sclogo}[1]{\large\Logofont{#1}}

% use subscript numerals for margin notes
\renewcommand*{\marginfont}{\Marginalfont}

% ensure margin notes have consistent vertical alignment
\renewcommand*{\marginnotevadjust}{-.17em}

% use compact lists
\setitemize{noitemsep,leftmargin=1em}
\setenumerate{noitemsep,leftmargin=1em}
\setdescription{noitemsep, style=unboxed, leftmargin=0em}

% style ToC
\DeclareTOCStyleEntries[
  raggedentrytext,
  linefill=\hfill,
  pagenumberwidth=.5in,
  pagenumberformat=\normalfont,
  entryformat=\normalfont
]{tocline}{chapter,section}


  \setlength\topsep{0pt}%
  \setlength\parskip{0pt}%

% define new \centerpars command for use in ToC. This ensures centering, proper wrapping, and no page break after
\def\startcenter{%
  \par
  \begingroup
  \leftskip=0pt plus 1fil
  \rightskip=\leftskip
  \parindent=0pt
  \parfillskip=0pt
}
\def\stopcenter{%
  \par
  \endgroup
}
\long\def\centerpars#1{\startcenter#1\stopcenter}

% redefine part, so that it adds a toc entry without page number
\let\oldcontentsline\contentsline
\newcommand{\nopagecontentsline}[3]{\oldcontentsline{#1}{#2}{}}

    \makeatletter
\renewcommand*\l@part[2]{%
  \ifnum \c@tocdepth >-2\relax
    \addpenalty{-\@highpenalty}%
    \addvspace{0em \@plus\p@}%
    \setlength\@tempdima{3em}%
    \begingroup
      \parindent \z@ \rightskip \@pnumwidth
      \parfillskip -\@pnumwidth
      {\leavevmode
       \setstretch{.85}\large\scshape\centerpars{#1}\vspace*{-1em}\llap{#2}}\par
       \nobreak
         \global\@nobreaktrue
         \everypar{\global\@nobreakfalse\everypar{}}%
    \endgroup
  \fi}
\makeatother

\makeatletter
\def\@pnumwidth{2em}
\makeatother

% define new sectioning command, which is only used in volumes where the pannasa is found in some parts but not others, especially in an and sn

\newcommand*{\pannasa}[1]{\clearpage\thispagestyle{empty}\begin{center}\vspace*{14em}\setstretch{.85}\huge\itshape\scshape\MakeLowercase{#1}\end{center}}

    \makeatletter
\newcommand*\l@pannasa[2]{%
  \ifnum \c@tocdepth >-2\relax
    \addpenalty{-\@highpenalty}%
    \addvspace{.5em \@plus\p@}%
    \setlength\@tempdima{3em}%
    \begingroup
      \parindent \z@ \rightskip \@pnumwidth
      \parfillskip -\@pnumwidth
      {\leavevmode
       \setstretch{.85}\large\itshape\scshape\lowercase{\centerpars{#1}}\vspace*{-1em}\llap{#2}}\par
       \nobreak
         \global\@nobreaktrue
         \everypar{\global\@nobreakfalse\everypar{}}%
    \endgroup
  \fi}
\makeatother

% don't put page number on first page of toc (relies on etoolbox)
\patchcmd{\chapter}{plain}{empty}{}{}

% global line height
\setstretch{1.05}

% allow linebreak after em-dash
\catcode`\—=13
\protected\def—{\unskip\textemdash\allowbreak}

% style headings with secsty. chapter and section are defined per-edition
\partfont{\setstretch{.85}\normalfont\centering\textsc}
\subsectionfont{\setstretch{.85}\Semiboldsubheadfont}%
\subsubsectionfont{\setstretch{.85}\Semiboldnormalfont}

% style elements of suttatitle
\newcommand*{\suttatitleacronym}[1]{\smaller[2]{#1}\vspace*{.3em}}
\newcommand*{\suttatitletranslation}[1]{\linebreak{#1}}
\newcommand*{\suttatitleroot}[1]{\linebreak\smaller[2]\itshape{#1}}

\DeclareTOCStyleEntries[
  indent=3.3em,
  dynindent,
  beforeskip=.2em plus -2pt minus -1pt,
]{tocline}{section}

\DeclareTOCStyleEntries[
  indent=0em,
  dynindent,
  beforeskip=.4em plus -2pt minus -1pt,
]{tocline}{chapter}

\newcommand*{\tocacronym}[1]{\hspace*{-3.3em}{#1}\quad}
\newcommand*{\toctranslation}[1]{#1}
\newcommand*{\tocroot}[1]{(\textit{#1})}
\newcommand*{\tocchapterline}[1]{\bfseries\itshape{#1}}


% redefine paragraph and subparagraph headings to not be inline
\makeatletter
% Change the style of paragraph headings %
\renewcommand\paragraph{\@startsection{paragraph}{4}{\z@}%
            {-2.5ex\@plus -1ex \@minus -.25ex}%
            {1.25ex \@plus .25ex}%
            {\noindent\Semiboldnormalfont\normalsize}}

% Change the style of subparagraph headings %
\renewcommand\subparagraph{\@startsection{subparagraph}{5}{\z@}%
            {-2.5ex\@plus -1ex \@minus -.25ex}%
            {1.25ex \@plus .25ex}%
            {\noindent\Semiboldnormalfont\small}}
\makeatother

% use etoolbox to suppress page numbers on \part
\patchcmd{\part}{\thispagestyle{plain}}{\thispagestyle{empty}}
  {}{\errmessage{Cannot patch \string\part}}

% and to reduce margins on quotation
\patchcmd{\quotation}{\rightmargin}{\leftmargin 1.2em \rightmargin}{}{}
\AtBeginEnvironment{quotation}{\small}

% titlepage
\newcommand*{\titlepageTranslationTitle}[1]{{\begin{center}\begin{large}{#1}\end{large}\end{center}}}
\newcommand*{\titlepageCreatorName}[1]{{\begin{center}\begin{normalsize}{#1}\end{normalsize}\end{center}}}

% halftitlepage
\newcommand*{\halftitlepageTranslationTitle}[1]{\setstretch{2.5}{\begin{Huge}\uppercase{\so{#1}}\end{Huge}}}
\newcommand*{\halftitlepageTranslationSubtitle}[1]{\setstretch{1.2}{\begin{large}{#1}\end{large}}}
\newcommand*{\halftitlepageFleuron}[1]{{\begin{large}\Fleuronfont{{#1}}\end{large}}}
\newcommand*{\halftitlepageByline}[1]{{\begin{normalsize}\textit{{#1}}\end{normalsize}}}
\newcommand*{\halftitlepageCreatorName}[1]{{\begin{LARGE}{\textsc{#1}}\end{LARGE}}}
\newcommand*{\halftitlepageVolumeNumber}[1]{{\begin{normalsize}{\Allsmallcapsfont{\textsc{#1}}}\end{normalsize}}}
\newcommand*{\halftitlepageVolumeAcronym}[1]{{\begin{normalsize}{#1}\end{normalsize}}}
\newcommand*{\halftitlepageVolumeTranslationTitle}[1]{{\begin{Large}{\textsc{#1}}\end{Large}}}
\newcommand*{\halftitlepageVolumeRootTitle}[1]{{\begin{normalsize}{\Allsmallcapsfont{\textsc{\itshape #1}}}\end{normalsize}}}
\newcommand*{\halftitlepagePublisher}[1]{{\begin{large}{\Noligaturecaptionfont\textsc{#1}}\end{large}}}

% epigraph
\renewcommand{\epigraphflush}{center}
\renewcommand*{\epigraphwidth}{.85\textwidth}
\newcommand*{\epigraphTranslatedTitle}[1]{\vspace*{.5em}\footnotesize\textsc{#1}\\}%
\newcommand*{\epigraphRootTitle}[1]{\footnotesize\textit{#1}\\}%
\newcommand*{\epigraphReference}[1]{\footnotesize{#1}}%

% custom commands for html styling classes
\newcommand*{\scnamo}[1]{\begin{center}\textit{#1}\end{center}}
\newcommand*{\scendsection}[1]{\begin{center}\textit{#1}\end{center}}
\newcommand*{\scendsutta}[1]{\begin{center}\textit{#1}\end{center}}
\newcommand*{\scendbook}[1]{\begin{center}\uppercase{#1}\end{center}}
\newcommand*{\scendkanda}[1]{\begin{center}\textbf{#1}\end{center}}
\newcommand*{\scend}[1]{\begin{center}\textit{#1}\end{center}}
\newcommand*{\scuddanaintro}[1]{\textit{#1}}
\newcommand*{\scendvagga}[1]{\begin{center}\textbf{#1}\end{center}}
\newcommand*{\scrule}[1]{\textbf{#1}}
\newcommand*{\scadd}[1]{\textit{#1}}
\newcommand*{\scevam}[1]{\textsc{#1}}
\newcommand*{\scspeaker}[1]{\hspace{2em}\textit{#1}}
\newcommand*{\scbyline}[1]{\begin{flushright}\textit{#1}\end{flushright}\bigskip}

% custom command for thematic break = hr
\newcommand*{\thematicbreak}{\begin{center}\rule[.5ex]{6em}{.4pt}\begin{normalsize}\quad\Fleuronfont{•}\quad\end{normalsize}\rule[.5ex]{6em}{.4pt}\end{center}}

% manage and style page header and footer. "fancy" has header and footer, "plain" has footer only

\pagestyle{fancy}
\fancyhf{}
\fancyfoot[RE,LO]{\thepage}
\fancyfoot[LE,RO]{\footnotesize\lastleftxmark}
\fancyhead[CE]{\setstretch{.85}\Noligaturefont\MakeLowercase{\textsc{\firstrightmark}}}
\fancyhead[CO]{\setstretch{.85}\Noligaturefont\MakeLowercase{\textsc{\firstleftmark}}}
\renewcommand{\headrulewidth}{0pt}
\fancypagestyle{plain}{ %
\fancyhf{} % remove everything
\fancyfoot[RE,LO]{\thepage}
\fancyfoot[LE,RO]{\footnotesize\lastleftxmark}
\renewcommand{\headrulewidth}{0pt}
\renewcommand{\footrulewidth}{0pt}}

% style footnotes
\setlength{\skip\footins}{1em}

\makeatletter
\newcommand{\@makefntextcustom}[1]{%
    \parindent 0em%
    \thefootnote.\enskip #1%
}
\renewcommand{\@makefntext}[1]{\@makefntextcustom{#1}}
\makeatother

% hang quotes (requires microtype)
\microtypesetup{
  protrusion = true,
  expansion  = true,
  tracking   = true,
  factor     = 1000,
  patch      = all,
  final
}

% Custom protrusion rules to allow hanging punctuation
\SetProtrusion
{ encoding = *}
{
% char   right left
  {-} = {    , 500 },
  % Double Quotes
  \textquotedblleft
      = {1000,     },
  \textquotedblright
      = {    , 1000},
  \quotedblbase
      = {1000,     },
  % Single Quotes
  \textquoteleft
      = {1000,     },
  \textquoteright
      = {    , 1000},
  \quotesinglbase
      = {1000,     }
}

% make latex use actual font em for parindent, not Computer Modern Roman
\AtBeginDocument{\setlength{\parindent}{1em}}%
%

% Default values; a bit sloppier than normal
\tolerance 1414
\hbadness 1414
\emergencystretch 1.5em
\hfuzz 0.3pt
\clubpenalty = 10000
\widowpenalty = 10000
\displaywidowpenalty = 10000
\hfuzz \vfuzz
 \raggedbottom%

\title{Anthology of Discourses}
\author{Bhikkhu Sujato}
\date{}%
% define a different fleuron for each edition
\newfontfamily\Fleuronfont[Ornament=11]{Arno Pro}

% Define heading styles per edition for chapter and section. Suttatitle can be either of these, depending on the volume. 

\let\oldfrontmatter\frontmatter
\renewcommand{\frontmatter}{%
\chapterfont{\setstretch{.85}\normalfont\centering}%
\sectionfont{\setstretch{.85}\Semiboldsubheadfont}%
\oldfrontmatter}

\let\oldmainmatter\mainmatter
\renewcommand{\mainmatter}{%
\chapterfont{\setstretch{.85}\normalfont\centering}%
\sectionfont{\setstretch{.85}\normalfont\centering}%
\oldmainmatter}

\let\oldbackmatter\backmatter
\renewcommand{\backmatter}{%
\chapterfont{\setstretch{.85}\normalfont\centering}%
\sectionfont{\setstretch{.85}\Semiboldsubheadfont}%
\oldbackmatter}
%
%
\begin{document}%
\normalsize%
\frontmatter%
\setlength{\parindent}{0cm}

\pagestyle{empty}

\maketitle

\blankpage%
\begin{center}

\vspace*{2.2em}

\halftitlepageTranslationTitle{Anthology of Discourses}

\vspace*{1em}

\halftitlepageTranslationSubtitle{A refreshing translation of the Suttanipāta}

\vspace*{2em}

\halftitlepageFleuron{•}

\vspace*{2em}

\halftitlepageByline{translated and introduced by}

\vspace*{.5em}

\halftitlepageCreatorName{Bhikkhu Sujato}

\vspace*{4em}

\halftitlepageVolumeNumber{}

\smallskip

\halftitlepageVolumeAcronym{Snp}

\smallskip

\halftitlepageVolumeTranslationTitle{}

\smallskip

\halftitlepageVolumeRootTitle{}

\vspace*{\fill}

\sclogo{0}
 \halftitlepagePublisher{SuttaCentral}

\end{center}

\newpage
%
\setstretch{1.05}

\begin{footnotesize}

\textit{Anthology of Discourses} is a translation of the Suttanipāta by Bhikkhu Sujato.

\medskip

Creative Commons Zero (CC0)

To the extent possible under law, Bhikkhu Sujato has waived all copyright and related or neighboring rights to \textit{Anthology of Discourses}.

\medskip

This work is published from Australia.

\begin{center}
\textit{This translation is an expression of an ancient spiritual text that has been passed down by the Buddhist tradition for the benefit of all sentient beings. It is dedicated to the public domain via Creative Commons Zero (CC0). You are encouraged to copy, reproduce, adapt, alter, or otherwise make use of this translation. The translator respectfully requests that any use be in accordance with the values and principles of the Buddhist community.}
\end{center}

\medskip

\begin{description}
    \item[Web publication date] 2021
    \item[This edition] 2022-11-08 08:10:33
    \item[Publication type] paperback
    \item[Edition] ed5
    \item[Number of volumes] 1
    \item[Publication ISBN] 978-1-76132-005-7
    \item[Publication URL] https://suttacentral.net/editions/snp/en/sujato
    \item[Source URL] https://github.com/suttacentral/bilara-data/tree/published/translation/en/sujato/sutta/kn/snp
    \item[Publication number] scpub17
\end{description}

\medskip

Published by SuttaCentral

\medskip

\textit{SuttaCentral,\\
c/o Alwis \& Alwis Pty Ltd\\
Kaurna Country,\\
Suite 12,\\
198 Greenhill Road,\\
Eastwood,\\
SA 5063,\\
Australia}

\end{footnotesize}

\newpage

\setlength{\parindent}{1.5em}%%
\newpage

\vspace*{\fill}

\begin{center}
\epigraph{The little creeks flow on babbling,\\
while silent flow the great rivers.}
{
\epigraphTranslatedTitle{About \textsanskrit{Nālaka}}
\epigraphRootTitle{\textsanskrit{Nālakasutta}}
\epigraphReference{Sutta \textsanskrit{Nipāta} 3.11:42}
}
\end{center}

\vspace*{2in}

\vspace*{\fill}

\blankpage%

\setlength{\parindent}{1em}
%
\tableofcontents
\newpage
\pagestyle{fancy}
%
\chapter*{The SuttaCentral Editions Series}
\addcontentsline{toc}{chapter}{The SuttaCentral Editions Series}
\markboth{The SuttaCentral Editions Series}{The SuttaCentral Editions Series}

Since 2005 SuttaCentral has provided access to the texts, translations, and parallels of early Buddhist texts. In 2018 we started creating and publishing our own translations of these seminal spiritual classics. The “Editions” series now makes selected translations available as books in various forms, including print, PDF, and EPUB.

Editions are selected from our most complete, well-crafted, and reliable translations. They aim to bring these texts to a wider audience in forms that reward mindful reading. Care is taken with every detail of the production, and we aim to meet or exceed professional best standards in every way. These are the core scriptures underlying the entire Buddhist tradition, and we believe that they deserve to be preserved and made available in highest quality without compromise.

SuttaCentral is a charitable organization. Our work is accomplished by volunteers and through the generosity of our donors. Everything we create is offered to all of humanity free of any copyright or licensing restrictions. 

%
\chapter*{Preface to the \textsanskrit{Suttanipāta}}
\addcontentsline{toc}{chapter}{Preface to the \textsanskrit{Suttanipāta}}
\markboth{Preface to the \textsanskrit{Suttanipāta}}{Preface to the \textsanskrit{Suttanipāta}}

In case it’s not obvious from the Introduction, I really love the \textsanskrit{Suttanipāta}. When I was a young monk I tried to memorize it. I didn’t succeed, but I got more than halfway through before life intervened. It’s been my friend and teacher for half my life.

At times, I have to admit, it has been a difficult relationship. The \textsanskrit{Suttanipāta} can be stubborn, refusing to give up its secrets. Even after a quarter century of study, and with the assistance of so much great scholarship, it wasn’t until I translated it that I felt I got certain passages. And then, when I wrote the Introduction, it was only \emph{then} that I got it. I’ll probably be “getting it” for quite some time to come.

Scripture is nor like the news, nor like a novel or a tweet. It stays with you. It gets into your bones. It sinks into them and stays there. Words and phrases turn up when you least expect them. Perspectives and resonances shape how you see the world. When you learn scripture, you might forget some things, but you never unlearn them.

When I was finishing up my Introduction, I asked myself what I probably should have asked at the start: who is this for? It’s too long and complicated for a general reader, too Buddhist for academics, too wide-ranging for meditators. There’s too much historical stuff; it’s not relevant enough.

I was thinking about the texts mentioned by Ashoka, and it occurs to me that scholars have spilled considerable ink to identify the texts without noticing perhaps the most relevant detail. He mentions the discourses on the “fears of the future”. For him, living over a century later, the Buddha’s future was his present. He looked to those texts because they were relevant. And this is the beginning of a long history of reading Suttas based on “what is relevant for me”.

So who is my Introduction for? I guess it must be for me. It’s what I would have wanted to read when starting out. If it’s not for you, maybe one day you’ll write your own. Until then, enjoy your journey.

%
\chapter*{Anthology of Discourses: an island in the stream}
\addcontentsline{toc}{chapter}{Anthology of Discourses: an island in the stream}
\markboth{Anthology of Discourses: an island in the stream}{Anthology of Discourses: an island in the stream}

\scbyline{Bhikkhu Sujato, 2022}

\section*{General Remarks}

The \textsanskrit{Suttanipāta} is full of poems that are vivid, personal, and direct. They are highly concentrated. A little has a lot of flavor, and they are best when savored slowly and reflectively. The collection as a whole is the most vibrant and rewarding of all the minor works of the Pali canon.

Within the Pali Canon, the \textsanskrit{Suttanipāta} is the fifth book of the Khuddaka \textsanskrit{Nikāya}. It is mainly verse, with a few prose sections filling out the narrative background.

The \textsanskrit{Suttanipāta} is unique to the Pali tradition. It is the only collection of early discourses that has no parallels, not even a mention, in the northern traditions. The Thera and \textsanskrit{Therī} \textsanskrit{gāthās} have no parallel collections, but they are mentioned, so it seems as if they once existed but have been lost. The \textsanskrit{Suttanipāta} seems to be completely unknown, and so far as we know, no school other than the Theravada ever had a similar collection.

While there are no parallels to the whole, there are many parallels to specific poems, sections, and verses. The Ratanasutta, \textsanskrit{Khaggavisāṇasutta}, and \textsanskrit{Aṭṭhakavagga} have important parallels outside the Pali canon. In addition, several texts, such as the Selasutta, are found elsewhere in the Pali canon. Three of the \textsanskrit{Suttanipāta} discourses are included in the little handbook known as the \textsanskrit{Khuddakapāṭha}, which was evidently compiled as an introductory curriculum for novices. These are the \textsanskrit{Maṅgalasutta}, Ratanasutta, and Mettasutta, whose popularity as blessing chants (\textit{paritta}) continues today. And as always, individual verses are scattered across the many verse collections in all schools of Buddhism.

\subsection*{Form and Structure}

The \textsanskrit{Suttanipāta} has a unique structure. It is comprised of \emph{poems}, that is, coherent sets of verses, some of which have a slim narrative background in prose. Other early Buddhist verse collections consist of groups of individual verses connected by theme (Dhammapada), or verses associated with a prose background story (\textsanskrit{Udāna}, \textsanskrit{Jātaka}). In the Thera- and \textsanskrit{Therīgāthā}, verses are collected because of their association with a given monk or nun. Some of these are poems, but we also find individual verses or clusters of verses with no particular relation to others gathered under the same person.

Perhaps the closest literary cousin of the \textsanskrit{Suttanipāta} is the \textsanskrit{Sagāthāvagga} of the \textsanskrit{Saṁyutta} \textsanskrit{Nikāya}, which likewise consists of various sets of verses, often in dialogue form, and with a minimal narrative background. Several of the \textsanskrit{Suttanipāta} texts are shared with the \textsanskrit{Sagāthāvagga}, such as the \textsanskrit{Kasibhāradvājasutta} (Snp 1.4).

The poems of the \textsanskrit{Suttanipāta} are, like all early Buddhist texts, organized in \textit{vaggas} (“chapters”). Unusually, all the \textit{vaggas} of the \textsanskrit{Suttanipāta} consist of more than ten texts.

The poems of the first three vaggas are loosely associated through slender thematic and stylistic connections. However the final two chapters—\textsanskrit{Aṭṭhakavagga} and \textsanskrit{Pārāyanavagga}—are tightly knit, and existed as self-contained works before being added to the \textsanskrit{Suttanipāta}. They are referred to by name in several places in the prose Suttas, and the \textsanskrit{Aṭṭhakavagga} has a parallel in the Chinese canon.

The verses display a wide range of metrical styles. They stem from a transitional period, with ancient metres from Vedic times side by side with innovative rhythms suitable for musical accompaniment.

\subsection*{Dating: against the \textsanskrit{gāthā} theory}

The final two chapters of the \textsanskrit{Suttanipāta}, together with certain other texts such as the \textsanskrit{Khaggavisāṇasutta}, show signs of being old texts. This has led to the oft-repeated claim that the \textsanskrit{Suttanipāta} is a uniquely early book. It is not. Many of the poems in the \textsanskrit{Suttanipāta} are not especially early and some of them are quite late, notably the opening verses of the \textsanskrit{Nālakasutta} (Snp 3.11) and the introduction to the \textsanskrit{Pārāyanavagga} (Snp 5.1). No serious scholar argues that the \textsanskrit{Suttanipāta} as a whole is an early collection, and I believe this view is based on nothing more than carelessness.

A more prudent argument, which is endorsed by some scholars, restricts itself to the early portions of the \textsanskrit{Suttanipāta}. It is widely agreed that certain portions of the \textsanskrit{Suttanipāta} are among the earliest passages of Pali. But some go on to argue that these portions are especially early compared to the bulk of the prose texts and even attempt to derive an “earliest Buddhism” from them. In my \textit{A History of Mindfulness} I called this the “\textsanskrit{gāthā} theory”. Here I will briefly review the arguments given for this, and show why they do not support the conclusions that have been drawn from them.

\begin{description}%
\item[The language is archaic.] Pali verses affect an archaic style, as is a common feature of verse. \footnote{Both Bodhi and Norman implicitly accept the commentarial reading here, i.e. that this refers to “wrong thoughts”. However the identical line at Ud 6.7:4.1 clearly refers to \textsanskrit{jhāna}. Both also follow the commentary in reading \textit{\textsanskrit{vidhūpitā}} as “burning up”, but \textit{\textsanskrit{vidhūpa}} refers to clearing the mist, smoke, or incense. }%
\item[The metres are early.] The leading expert in the field, K.R. Norman, says that “dating by metre is not particularly helpful”. \footnote{Bodhi, Norman, and Thanissaro all render \textit{vitatha} as “unreal”. Normally it’s used in the context of “true” speech or the “reality” or “unerringness” of say dependent origination. To say the whole world is “unreal” seems overly idealistic for early Buddhism, so I translate in accord with the apparent sense. }%
\item[Some verses are quoted in the canon.] This confirms only that the quoted verses are older than the prose texts that quote them. Both the prose and the verse collections were compiled over time, so it is hardly surprising that some portions of verse are older than some portions of prose. The contrary is also true, although less obvious. The verses frequently quote from or allude to doctrines from the prose texts, and often would make little sense without them. Moreover, there are plenty of cases where prose suttas are quoted by other prose suttas.%
\item[The \textsanskrit{Aṭṭhakavagga}, \textsanskrit{Pārāyanavagga}, and \textsanskrit{Khaggavisāṇasutta} have their own canonical commentary, the Niddesa.] There are plenty of prose suttas that serve as commentaries on other prose suttas. What we consider to be a “commentary” is somewhat arbitrary. But if we take it simply as “a text that explains another text”, the \textsanskrit{Anattalakkhaṇasutta}—the second sermon—is a commentary on the first sermon, the Dhammacakkappavattanasutta, drawing out the implications of the five aggregates that were mentioned in passing there. The \textsanskrit{Saccavibaṅgasutta} is an explicit commentary on the first sermon. The Niddesa is a much later style of commentary, comparable to other analytical “commentaries” in the canon. In the Abhidhamma, we have the \textsanskrit{Vibhaṅga}, which serves as a commentary on key passages from the \textsanskrit{Saṁyutta}, and the \textsanskrit{Puggalapaññatti}, which does the same for the \textsanskrit{Aṅguttara}. Moreover, the \textsanskrit{Vibhaṅga} of the Vinaya serves as a commentary on the \textsanskrit{Pātimokkha}. The \textsanskrit{Parivāra} is a commentary of sorts on the \textsanskrit{Vibhaṅga}, so the \textsanskrit{Pātimokkha} has two layers of canonical commentary.%
\item[Technical terms and formulaic doctrines appear less often.] This is the normal character of verse. Poetry is for inspiration, not information. Nonetheless, we do find references to technical categories in the verse—such as the five hindrances or the five faculties—but they are sometimes in altered form.%
\item[The mendicants wandered as hermits in the wilderness rather than settling in monasteries.] The verses are encouraging an idealized lifestyle, not describing how mendicants actually lived. The prose Suttas and the Vinaya depict a mix of wandering and settled life for mendicants, which seems reasonable. The \textsanskrit{Sāmaññaphalasutta} depicts the Buddha with a large community being visited by the king, yet the body of the teaching speaks of the meditating monk “gone to the forest, to the root of a tree, or to an empty hut”. The case of the \textsanskrit{Pātimokkha} is particularly troubling for this theory, as it is universally regarded as an early text, and, like the early portions of the \textsanskrit{Suttanipāta}, it has its own canonical commentary. Yet it contains several references to building regulations and must stem from a time when buildings for mendicants were not only normal but could be large and luxurious.%
\end{description}

It is also worth noting that, where we have parallel versions of these texts, they display no less variation than other early parallels. The Chinese Arthavargiya, leaving aside the extensive addition of stories, has numerous differences in content as well as a different sequence. The \textsanskrit{Khaggavisāṇasutta} has signs of added repetitions, misplaced lines, and missing text even in the Pali, while the \textsanskrit{Mahāvastu} has both added and missing verses, as well as a range of other variations.

The obsession with seeing portions of the \textsanskrit{Suttanipāta} as early is a holdover of the mid-20th century enthusiasm for discovering a “Buddha before Buddhism”, seeking a “truly authentic” teaching before it was institutionalized as rigid doctrine. Somehow, this search always ends up conflated with the racially-charged effort to divest Buddhism of its “cultural” (read “Asian”) elements. The \textsanskrit{Suttanipāta} is not in need of rescue, and a romanticizing and reductive lens adds nothing and distorts much.

\subsection*{Themes}

Since the collection is primarily organized by literary style, it contains texts on a range of themes. These are familiar elsewhere in Buddhist texts, and I will briefly summarize them here.

\begin{itemize}%
\item The virtues of renunciation and the proper life of the ascetic. \footnote{Norman points out that this verse is not a proper response to the previous and suggests that a verse has been lost. I agree, but I disagree there is a contradiction in the sense. Rather, the verse tells a story over time. I translate to bring out the meaning. }%
\item Ethical virtues and dangers. \footnote{I’m not convinced by the commentary’s “not separated”. It doesn’t seem to occur in this meaning elsewhere, and lacks a connection with the opening line. Surely the whole verse must be about the children, which requires that there be a comparison made with the first line. The point is, I think, that the children follow the father and have become self-sufficient. This also echoes better with the Buddha’s response. }%
\item Revaluation of contemporary, often Brahmanical, views. \footnote{Norman and Bodhi read the middle form \textit{\textsanskrit{carāmase}} as imperative, apparently, but that doesn’t make sense to me; in this context optative would make more sense, but I am not aware of this form. Rather, I take it as the present indicative in the sense of definite future. }%
\item Conversion of local deities. \footnote{The exact nuance of this line is hard to pin down. Later Pali texts equate \textit{\textsanskrit{abhilāpa}} with “naming”, but it doesn’t seem to occur in any early texts with this sense. However \textit{\textsanskrit{lapanā}} is used in the sense of “flattery” in eg. MN 117:29. }%
\item The virtues of the Buddha and the Triple Gem. \footnote{\textit{\textsanskrit{Padumī}} is uncertain, see Bodhi’s note 475. But other Sanskrit sources attribute the epithet to the elephant’s predeliction to bathing and eating in lotus ponds, which would seem much more likely. }%
\item Biography of the Buddha. \footnote{The reference is to the smelting of gold, AN 3.101:1.9. }%
\item Meditation and mental development. \footnote{\textit{\textsanskrit{Sakkhipuṭṭho}} is only elsewhere used in the context of being asked to bear witness in court. I think the context is relevant here, as the Buddha is offering personal testimony on a matter that otherwise he may not speak. }%
\item Philosophy. \footnote{Norman and Bodhi both read \textit{pada} as “way”, but it surely means “passage” as eg. AN 4.191:1.7. }%
\item Letting go of disputatious views. \footnote{Several of the examples are gendered so I use “man”. }%
\end{itemize}

It is important to not over-interpret. Perhaps the most striking themes, sometimes taken as emblematic of the collection as a whole, are the first and last in the above list: the virtues of renunciation, and the dangers of disputatious views. If one assumes that the \textsanskrit{Suttanipāta} is a uniquely archaic text, it is tempting to see these positions as more authentic to the Buddha’s original teachings than the bulk of the prose texts. This is not the case. The virtue of renunciation and a simple wandering life is mentioned in many places in the prose Suttas, as is the danger of getting involved in disputations based on theoretical views. Perhaps they are emphasized more in the \textsanskrit{Suttanipāta}, but this is hardly a sign of any substantially different doctrine. It is a mistake to develop a theory of Early Buddhism based on a few verses.

\subsection*{The \textsanskrit{Suttanipāta} and Ashoka}

The earliest epigraphic evidence of Buddhist scriptures is King Ashoka’s Calcutta-\textsanskrit{Bairāṭ} edict, the third Minor Rock Edict, in which Ashoka lists several texts whose study would be for the long-lasting of the Dhamma. It is a rare case where Ashoka specifically addresses the Buddhist community. The texts are named, and there seems no reason to seek them outside of what we today consider the early Buddhist texts. Yet the titles of Suttas are often fluid, and it is difficult to identify them with precision, as they could refer to a number of texts in the canon.

\begin{quotation}%
Vinaya-samukase, Aliya-\textsanskrit{vasāṇi}, \textsanskrit{Anāgata}-\textsanskrit{bhayāni}, Muni-\textsanskrit{gāthā}, Moneya-\textsanskrit{sūte}, \textsanskrit{Upaṭisa}-pasine, \textsanskrit{cā} Laghulo-\textsanskrit{vāde} \textsanskrit{musā}-\textsanskrit{vādaṁ}

%
\end{quotation}

Some of these seven texts have been identified with poems of the \textsanskrit{Suttanipāta}. Most confidently, the Muni-\textsanskrit{gāthā} with the Munisutta (Snp 1.12). Less confidently, \textsanskrit{Upaṭisa}-pasine with the \textsanskrit{Sāriputtasutta} (Snp 4.16), bearing in mind that Upatissa is \textsanskrit{Sāriputta}’s given name. However, there are many questions by \textsanskrit{Sāriputta} in the canon. Finally, the second portion of the \textsanskrit{Nālakasutta} deals with “sagacity”, so perhaps this is the Moneya-\textsanskrit{sūte}.

\subsection*{The \textsanskrit{Suttanipāta} and the south}

Many of the suttas are presented without the usual narrative frame that locates the sutta. However, when locations are mentioned or implied, there seems to be an emphasis on the south. Most obviously, of course, the \textsanskrit{Pārāyanavagga}, which is the southern-most location in all the suttas, and which details the stops along the journey of the “southern road”.

These stops include \textsanskrit{Ujjenī}, the capital of \textsanskrit{Avantī}. As I will discuss further in the relevant chapter, \textsanskrit{Avantī} has a close association with the \textsanskrit{Aṭṭhakavagga}, as it was the primary text taught there by \textsanskrit{Mahākaccāna}, the founder of Buddhism in \textsanskrit{Avantī}. The Dhaniyasutta (Snp 1.2) is set on the banks of the \textsanskrit{Mahī} river, which is presumably near the Mahissati mentioned in the \textsanskrit{Pārāyanavagga}. The \textsanskrit{Kasibhāradvājasutta} is located in the “southern hills”, which, while still within Magadha, lay towards the southern extreme of the Buddha’s normal domain. The region was visited only a few times in the Suttas, but one of those times we find the laywoman \textsanskrit{Veḷukaṇṭakī} rising early to recite the verses of the \textsanskrit{Pārāyana} (AN 7.53:1.3).

Other discourses are located in the Buddha’s heartland. Hemavata and \textsanskrit{Sātāgira} are spirits of the mountain lands, and so probably belong in the north (Snp 1.9). The Selasutta is also somewhat north in \textsanskrit{Aṅguttarāpāna}. Thus the \textsanskrit{Suttanipāta} as a whole covers much of the normal area of early Buddhism.

Given the unusual emphasis on the south, however, it is worth asking if the collection as a whole stems from there. It could easily have been that the popularity of these southern texts led to an entire collection coalescing around them. If this is so, then it would be natural to look to \textsanskrit{Avantī}, among the students of \textsanskrit{Mahākaccāna}’s lineage, for the creation of the book. It was probably used as a conversion text for brahmins in the area.

Support, albeit indirect, for this comes from the unusual presence of the \textit{\textsanskrit{gīti}} (or “old \textit{\textsanskrit{āryā}}”) style of metre in the \textsanskrit{Suttanipāta}. This is found in only three Jain suttas in Ardhamagadhi and three suttas in the Pali, namely the Mettasutta (Snp 1.8), the \textsanskrit{Tuvaṭakasutta} (Snp 4.14), and the verses of homage in the \textsanskrit{Upālisutta} (MN 56). The \textsanskrit{Upālisutta} is a dialogue with the Jains, and \textsanskrit{Upāli} is using a metre that is most closely associated with the Jains, a detail that attests to the verses’ authenticity.

According to K.R. Norman, when the Vedic peoples arrived in northern India with their chanted verses some centuries before the Buddha, these melded with the more musical poetic singing of the local Dravidian peoples to give rise to new families of poetic styles. This innovation was ongoing during the period of composition of the Pali canon, and is responsible for the diverse metrical styles in the \textsanskrit{Suttanipāta}. \footnote{Reading \textit{\textsanskrit{bhavantaṁ}} which is the regular form used with \textit{Gotama}. } The \textit{\textsanskrit{gīti}} is the oldest example of this transition. Given that extant examples stem from either Magadha or \textsanskrit{Mahārāṣṭra}, Norman postulates that the style arose somewhere between Magadha and \textsanskrit{Mahārāṣṭra}. And that would locate it in \textsanskrit{Avantī}.

There is a further southern connection hidden in the origins of the “dark hermit” Asita (Snp 3.11). The Pali does not say where he is from, but the parallels in both the \textsanskrit{Mahāvastu} and the \textsanskrit{Nidānakathā} place him in the south. Other “dark hermits” of a similar character are elsewhere said to hail from the south as well.

If it is the case that the \textsanskrit{Suttanipāta} was composed in the south by \textsanskrit{Mahākaccāna}’s students, this would explain why the text is unique to the Theravada tradition. The founder of Buddhism in Sri Lanka, and hence of the Theravada lineage, was Ashoka’s son Mahinda. He was born in \textsanskrit{Ujjenī}, and on his journey to Sri Lanka visited his mother in nearby Vidisa. It is not hard to imagine a connection between Mahinda, with his affinity for analytical riddles, and \textsanskrit{Mahākaccāna}, the famously analytical teacher, founder of Buddhism in \textsanskrit{Avantī} and proponent of the \textsanskrit{Aṭṭhakavagga}.

Drawing this out even further, perhaps we could consider the formation of the Niddesa in this light. Like the \textsanskrit{Suttanipāta} itself, this is unique to the Theravada tradition. The two main texts it comments on, the \textsanskrit{Aṭṭhakavagga} and \textsanskrit{Pārāyanavagga}, are both associated with the south. Given that we know that the \textsanskrit{Aṭṭhakavagga} was a core part of \textsanskrit{Mahakaccāna}’s teaching curriculum and that he was a teacher renowned for analytical and linguistic interpretation of texts, it would make sense that the Niddesa was produced by his school. Perhaps it formalizes the teaching methods employed by \textsanskrit{Mahakaccāna} himself.

All this is quite speculative, of course. But if there is anything to it, perhaps the unique character of the \textsanskrit{Suttanipāta} would lend support to the idea that the Pali canon was learned by Mahinda in \textsanskrit{Avantī}, and represents the recension of that area. This is supported by the observation that Pali is similar to Western dialects such as those of the Girnar inscriptions. \footnote{“Cause” or “reason” are not wrong, but they miss the force of the metaphor, which is about going from one state to another. }

\section*{The Serpent Chapter}

The chapter offers advice for lay folk, counters the pretensions of the brahmins, and co-opts the animist worship of native spirits. But it begins and ends with a direct statement on the life that the Buddha believed to be most worth living: a mendicant who “sheds this world and the next”, “the sage meditating secluded in the woods”. While the structure of the chapter is certainly loose, this thematic return suggests that it was compiled with an overall plan in mind.

\subsection*{The Serpent}

The \textsanskrit{Suttanipāta} opens with the Uragasutta (Snp 1.1), which draws out the metaphor of a snake shedding its skin. Analyzing the language, the simple and flexible style and metre, and the doctrinal focus, Jayawickrama concluded that, “The tone of the sutta is generally archaic and the language preserves an early stratum of Pali”. \footnote{\textit{\textsanskrit{Dhammakāmo}} is apparently always used in this sense. } The reference to the “five hindrances” nonetheless shows that the text depended on knowledge of the prose suttas (Snp 1.1:17.1).

It was a popular poem, judging by the existence of parallels found in the \textsanskrit{Gāndhārī} Dharmapada, Patna Dhammapada, and \textsanskrit{Udānavarga}. A comparative study by Ven Ānandajoti concluded that, while agreeing that the text is old, verses 6 and 10–13 in the Pali were likely added later, for they are not found in the parallels.

Jaywickrama points out that, while there are many prosaic words for “snake” in Pali, \textit{uraga} is the only one used in an elevated sense. In AN 6.43, the \textit{uraga} is compared with noble animals such as the elephant, the bull, and the horse. Hence I have rendered it as “serpent”.

Most of us will encounter the snake metaphor from the safety of our comfortable and snake-free homes. Even in Australia—and I’m sorry to disappoint—it isn’t really the case that deadly snakes lurk around every corner. Having said which, I’m writing this from my father’s home in the hills up from Coffs Harbour. There once was a huge carpet python sleeping beneath the bed I’m using. Bloated with the ducklings in its belly, it was even bigger than the one in the workshop. But carpet pythons are mostly harmless, unlike the black snake that slithered across the TV while people were watching it. But it’s the brown snakes that you really have to look out for. Not to mention the tiger snakes, and the cunningly-named death adder. But I promise, not everywhere in Australia is like this! Anyway, we have anti-venom and few people die of snakebites in Australia. In India, on the other hand, snakes are still a deadly menace, killing around 58,000 people each year.

So while we might relate to a snake sentimentally, appreciating its cool elegance, for the mendicants living in the forests 2,500 years ago, a snakebite meant death. Indeed, the Pali texts record several monks dying this way (AN 4.67). When the Uragasutta compares a mendicant with a snake, it is associating them with a dangerous and mysterious power over life and death. Elsewhere, the Suttas warn us to beware of four things when they are young: a prince, a snake, a fire, and a mendicant (SN 3.1). Each of these may look harmless when small, but can be deadly when grown.

A snake often appears in a negative aspect: a taker of life, a loathsome spitter of poison, a shape-shifting master of deception. But the impression that the snake makes on the human mind is nothing if not complicated. We are afraid and fascinated in equal measure. One of the snake’s many astonishing properties is its ability to shed its skin. And in this we can see a manifestation of its deathly power not for destruction but for transformation.

The first verse offers a succinct summary of the entire Buddhist path, employing the snake metaphor in both dark and bright aspects.

\begin{verse}%
When anger surges, they drive it out,\\

as with medicine a snake’s spreading venom.\\

Such a mendicant sheds this world and the next,\\

as a snake its old worn-out skin.

%
\end{verse}

The first couplet reflects where we are: riven with anger. If anger is a poison, then there is on the one hand a threat to life; but on the other, there is a possibility of a cure. Now, the treatments available at that time—which involved using disgusting substances like ash or dung as an emetic (Khandhaka 6:14.6.1)—were risky at best. But survival was not impossible, because the venom was still “spreading”: it is an active, ongoing process, not something that is fated or destined. If anger is likewise “spreading”, it too can be arrested.

Thus far the teaching aligns with what we might expect from a psychologist, or indeed, from common sense. But now the verse inverts the snake metaphor. The death threatened in the opening manifests as something rather more mysterious: the shedding of skin, abandoning “this world and the next”. The Pali here is \textit{\textsanskrit{orapāraṁ}}, more literally, “the near shore and the far”. Often these terms refer to the mundane world of suffering and the transcendence realization of \textsanskrit{Nibbāna}, but here they must refer to “this world” in which one is born, and “another world” to which one goes after death, so I have translated accordingly. Verses commonly employ non-standard terms, whether for poetic expression or simply to fit the metre.

Whereas most psychology aims to alleviate extremes of distress and promote a healthy and balanced happiness, and most religion offers a promise of bliss in a future life, the Buddha promised something quite different: freedom.

Taking the metaphor at face value and without regard for context, one might infer that the Sutta is teaching that the true essence of a person lives inside their “skin”—i.e. the ephemeral physical body—and that when they shed their “skin” they will reveal this true essence. Such a reading would support the theory that the Buddha’s teaching was essentially the same as that of the \textsanskrit{Upaniṣads}.

However, if the Buddha wanted to teach the survival of an abiding essence, he could surely have done a better job of it. Long before he sat under a fig tree by the \textsanskrit{Uruvelā} river, the snake metaphor had already been employed to express this idea. Consider the following verse from \textsanskrit{Bṛhadāraṇyaka} \textsanskrit{Upaniṣad} 4.4.7:

\begin{quotation}%
This corpse lies as lifeless as the slough of a snake shed in an anthill. But this incorporeal, immortal life-breath is sheer divinity, sheer incandescence.

%
\end{quotation}

Like the Uragasutta, this employs a snake shedding its skin as a metaphor for spiritual transcendence. But the passage goes on to speak of an essential force that is divine, radiant, and immortal. This is the concept of the transcendent Self that is so prominent in the \textsanskrit{Upaniṣads} and so conspicuously absent from the Uragasutta, and which the Buddha rejected on so many occasions.

That said, the shared metaphor does raise a question. Is it simply a result of various sages independently observing a common feature of nature and applying it as a spiritual metaphor? Or is the Buddhist text consciously responding to the earlier usage of the metaphor, and possibly even to this specific \textsanskrit{Upaniṣadic} passage? This question invites us to take a closer look at the context.

The \textsanskrit{Upaniṣadic} quote is a comment on a verse that speaks of releasing all the heart’s desires. The entire chapter is concerned with the passage of the “Self” from this world to the next world, and how this is propelled by a person’s deeds (\textit{karma}). Indeed, it is in this chapter that we find perhaps the earliest clear statement of the doctrine of \textit{karma} as moral deeds (\textsanskrit{Bṛhadāraṇyaka} \textsanskrit{Upaniṣad} 4.4.5). This idea is so central to Buddhism that it is tempting to believe it was a commonly accepted notion at the time. But it is not found in the earlier Vedic texts, and here it is introduced in an esoteric dialogue. And even here it is not presented as a doctrine universally accepted, for others are said to hold a different belief: the self is purely desire and attains that for which it wishes. In this alternate theory, it is the force of will that matters, not the moral quality of actions.

This means there are at least two details in this passage that echo ideas found in the Suttas: the snake shedding its skin, and the doctrine of \textit{karma}. Now, it often happens that when you notice one detail shared between the Suttas and a passage of the \textsanskrit{Upaniṣads}, you will soon find another. And it turns out that in this case there are not just two, but many ideas, phrases, and metaphors running parallel. Here are a few of them:

\begin{itemize}%
\item The ancient way that is rediscovered. \footnote{Both Norman and Bodhi read \textit{dissati}, I read \textit{dussati}. }%
\item The aged body like a cart. \footnote{Both Norman and Bodhi add a term suggesting transgression here, in line with the commentary and the apparent sense. I don’t, as I suspect the text is corrupt. Could \textit{diss} be \textit{duss}? }%
\item The ability to see the effects of merit and demerit. \footnote{Note that \textit{\textsanskrit{jātu}} here is not connected to \textit{\textsanskrit{jāti}} “birth”. Rather, \textit{na hi \textsanskrit{jātu}} is an idiom meaning “absolutely not”. Commentary: \textit{ekaṃseneva}. }%
\item comparing heavenly realms through multiplication. \footnote{Norman has “untruth”, Bodhi “destructive”. Neither notices \textit{\textsanskrit{vebhūtiyaṁ}} at  DN 30:2.21.2 and DN 28:11.2, where “divisive” fits all cases. }%
\item The functions of seeing, hearing, thinking, etc.. \footnote{Lit. “the counselor speaks meaningfully.” }%
\item The image of freedom as the detachment of a mango from its stalk. \footnote{Read \textit{\textsanskrit{pasaṁsasi}} per PTS. }%
\item The identification with the Self as “I am this”. \footnote{The commentary, followed by Norman, Bodhi, and \textsanskrit{Ñāṇadīpa}, all take \textit{\textsanskrit{pamāya}} here in the sense of “crushed” (Sanskrit: \textit{\textsanskrit{pramṛṇati}}). However, \textit{\textsanskrit{pamāya}} occurs at Snp 4.12:17.1, where, being beside \textit{vinicchaye} (“judge, assess”), it is clearly an absolutive of \textit{\textsanskrit{pamiṇāti}} “having measured”. Here too it sits beside a word (\textit{\textsanskrit{saṅkhāya}}) having the sense to reckon or calculate. In the simile of rebirth and the field, the “seed” is consciousness, which belonging to the first noble truth is not “crushed” but “fully known”. If the seeds are already “crushed” then choosing not to water them (in the next line) is a tad redundant. }%
\item The idea that the ignorant suffer. \footnote{Compare the stock idiom \textit{\textsanskrit{Saṅkhampi} na upeti upanidhimpi na upeti \textsanskrit{kalabhāgampi} na upeti} at eg. SN 20.2:1.7; also Snp 5.7:6.3. }%
\item The observation that thinking is fatiguing compared to awareness. \footnote{\textit{\textsanskrit{Dinnaṁ}} and \textit{\textsanskrit{payataṁ}} are synonyms. }%
\item Giving up the desire for children, renouncing the home and the worldly “quest” (\textit{\textsanskrit{eṣana}}), and wandering as a mendicant in search of truth. \footnote{Read \textit{kujja}. }%
\item The identification of the state transcending rebirth
\begin{itemize}%
\item as that which is unborn, unaging, undying, and fearless. \footnote{\textit{\textsanskrit{Komāra}} is hypermetrical and has probably been inserted by analogy with AN 5.192:5.4. }%
\item as that which neither increases nor decreases. \footnote{According to Baudh 2.1.2.13, \textit{\textsanskrit{agamyā} \textsanskrit{gamanaṁ}} means not transgressing with women considered inappropriate, such as the female friend of a male or female teacher. It doesn’t mean “outside caste”. }%
\item as that which is the negation of everything. \footnote{\textit{\textsanskrit{Kiccākicca}} means “all kinds of duties, various business”, not “what is to be done and not done” (per both Norman and Bodhi). See eg. Thag 16.10:20.2. }%
\item as the divinity that is the cosmos. \footnote{A line elsewhere only used to describe hell. Perhaps the Buddha was no fan of the suburbs. }%
\end{itemize}

%
\end{itemize}

The next chapter has even more, but I’ll stop there. Independently, each point may be simply a coincidence, but taken as a whole they paint a compelling picture. There’s hardly a verse that is not reminiscent of the Suttas in some way. And the similarity is not just in specifics, for the Uragasutta’s emphasis on paradoxical phrasing and mysterious transcendence sounds a lot like an \textsanskrit{Upaniṣad}.

The closeness is not just in content but in historical context as well. The \textsanskrit{Upaniṣadic} chapters were spoken by the sage \textsanskrit{Yājñavalkya} to King Janaka of Videha, perhaps a century or two before the Buddha. The Buddha is recorded as having visited Videha only once. And it is a curious thing that on that visit he spoke with and ultimately converted \textsanskrit{Brahmāyu}, perhaps the oldest living Brahmanical sage (MN 91). Surely this text has a missionary purpose, to convert the most respected brahmin in this ancient home of some of the most famous Brahmanical dialogues. It’s just possible that \textsanskrit{Brahmāyu} was a direct student of \textsanskrit{Yājñavalkya}, or perhaps was part of his lineage. Regardless, there are many instances where the Buddha’s teaching clearly echoes and responds to that of \textsanskrit{Yājñavalkya}, and we will encounter this again in the \textsanskrit{Suttanipāta}, especially the final chapter.

Similarity, however, does not mean identity. If the Buddha was drawing on and echoing ideas from these dialogues, he did not do so uncritically; the differences are just as important as the similarities.

\textsanskrit{Yājñavalkya} expresses his view with clarity and elegance: the Self is the person made of consciousness, the immortal light in the heart among the vital energies (BrhUp 4.3.7). There is no mistaking what he meant. The folk who have tried to extract a similar view from the Suttas are reduced to seeking out in dusty allusions and twisted metaphors the same thing that \textsanskrit{Yājñavalkya} had already stated directly.

Given this context, might it be that the Serpent Sutta was composed as a deliberate rejoinder to \textsanskrit{Yājñavalkya}? It’s possible, but it’s hard to say definitively. This is often the case in the Suttas. The \textsanskrit{Upaniṣadic} view is there, but not explicitly, so it is easy to overlook. Only when seeing the overall pattern does it become clear.

But I would make another proposal. The \textsanskrit{Suttanipāta} as a whole ends with the \textsanskrit{Pārāyanavagga}, a series of conversations with brahmins that closely evoke the \textsanskrit{Upaniṣads}. I think the Serpent Sutta was added at the beginning of the \textsanskrit{Suttanipāta} to frame the entire collection as an implicit rejection of the \textsanskrit{Upaniṣadic} doctrine. Right from the beginning, we hear an unequivocal rejection of any positive assertion of a Self, the core of \textsanskrit{Yājñavalkya}’s teaching.

The Buddha adopts that which is of value in \textsanskrit{Yājñavalkya}’s analysis: namely, that desire, hate, and ignorance create suffering and bind us to a cycle of rebirth. But when speaking of “shedding” the skin, he says nothing of a new or revealed state of the Self. It is the shedding itself that is the point. They find “no substance” in any existence, going beyond “any state of existence” because they know that it is all “not what it seems”. His response to the \textsanskrit{Upaniṣads} is subtle and nuanced; what he does \emph{not} say is as significant as what he does say.

\subsection*{With Dhaniya the Cowherd}

The second discourse, Dhaniyasutta (Snp 1.2) is of a very different nature. Here we leave aside subtle paradoxes and calls to transcendence for a dialogue between the Buddha and a farmer. The farmer boasts of his worldly security and contentment amidst the uncertainties of the world, but the Buddha betters him with his spiritual contentment. This manner of exchanging verses in spiritual one-upmanship is reminiscent of the \textsanskrit{Sagāthavagga}.

This discourse is set by the \textsanskrit{Mahī} river, said to be one of the five great rivers of ancient India. The commentary to the Selasutta locates the country of \textsanskrit{Aṅguttarāpana} north of the \textsanskrit{Mahī}, which would mean it must be the modern Kosi River. The Kosi is known as the “sorrow of Bihar” due to its capricious flooding and frequent changing of courses. In 2008 it broke its banks and flooded millions of people. So it certainly fits with the dangerous and threatening tone of the Dhaniyasutta.

The exchange is interrupted by a thunderstorm that rather conveniently illustrates the Buddha’s point. This, and the unexpected appearance of \textsanskrit{Māra} near the end, lends the dialogue a dramatic flair.

Within the \textsanskrit{Suttanipāta} as a whole, it functions as a more accessible entry point after the somewhat forbidding opening poem. In this, the \textsanskrit{Suttanipāta} echoes a number of other collections, which similarly begin with an intimidating, difficult text, followed by a more practical and accessible one. \footnote{I’m not entirely convinced that \textit{\textsanskrit{ganthetvā}} means “composed” here. It’s the only early usage in this sense, and the reading is derived from the highly polemical commentary. It may mean just that they “put together” i.e. “selected” favorable passages. This would be less nasty to the brahmins and more historically plausible (as we know that the Vedas are, in fact, old.) }

The key term, introduced in the verse of \textsanskrit{Māra} near the end, is \textit{upadhi}. \textsanskrit{Māra} uses it in a positive sense, as the children and cattle that bring delight to one who has them. Here we can see the objective sense of the word as a “close support”, the things in one’s life that make living possible and worthwhile. The Buddha, in saying that the same things bring you sorrow, shifts the meaning to the subjective sense, as it is one’s inner craving and desire that draws one to sorrow again and again, as said in Snp 5.5:2.4.

These final verses, where \textsanskrit{Māra} rather abruptly intervenes, were likely not part of the original composition. Jayawickrama points out that they are added after the poem already reaches a satisfying conclusion with the conversion of Dhaniya and his wife. \footnote{No doubt an exaggeration, but sacrifices on this scale have been performed in modern times. } The two extra verses are found independently spoken by \textsanskrit{Māra} at SN 4.8, and spoken by a deity at SN 1.12. They also have a more regular metrical structure than the bulk of the Dhaniyasutta. Taken together these points strongly indicate that the verses were added later. Nonetheless, on the basis of metre, language, style, and doctrine, Jayawickrama concludes that the discourse is fairly old.

\subsection*{The Horned Rhinoceros}

With the third discourse, the famous \textsanskrit{Khaggavisāṇasutta} (Snp 1.3), the mood changes once more. Here we find neither philosophical paradoxes nor dramatic dialogue, but a heartfelt and pragmatic plea for the solitary life of a wandering ascetic.

The poem strikes an uncompromising tone, a principled rejection of worldly attachments regardless of the cost, and an equally unsentimental acknowledgement of the tribulations of a solitary life in the forest. Perhaps we could read the three opening Suttas as respective responses to Brahmanical philosophy, sensual enjoyment of the lay life, and asceticism.

The \textsanskrit{Khaggavisāṇasutta} is universally accepted as an early text, due to its subject matter, style, and prevalence of archaic linguistic forms. While not wishing to dispute the fact that the \textsanskrit{Khaggavisāṇasutta} is early, there is a range of considerations that suggest that in its current form it has been subject to considerable development. I will discuss these points further before drawing my conclusions.

It is commented on in full in the Niddesa and is also quoted in the \textsanskrit{Apadāna}. Twelve of the verses are found in the \textsanskrit{Mahāvastu} (Mvu 33). This is a text of the \textsanskrit{Lokuttaravāda}-\textsanskrit{Mahāsaṅghikas}, who stem from the other party in the first schism, dating sometime after Ashoka. It contains twelve verses as compared to forty-one in the Pali, of which seven closely correspond. In addition, a verse is found in the \textsanskrit{Divyāvadāna} of the \textsanskrit{Mūlasarvastivādins} (Divy 182.013), corresponding to the second verse in the Pali. Finally, a version in forty verses has been found in a \textsanskrit{Gandhārī} manuscript. \footnote{\textit{\textsanskrit{Kāmakāro}}: “wish-maker”. At Kv 23.3 \textit{\textsanskrit{issariyakāmakārikā}} means a “sovereign act of will”. Commentary here says ordinary people may not simply know or say what they want. } While the number of verses is the same as the Pali, several of the verses vary, as does their sequence.

The \textsanskrit{Khaggavisāṇasutta} and its central image of “wandering alone” is a Buddhist expression of a broader Indian literature extolling the virtues of the solitary life. This has been explored in a recent article by Kristoffer af Edholm. \footnote{\textit{\textsanskrit{Saṅkheyyakāro}}: “comprehensibility-maker”. \textit{\textsanskrit{Saṅkheyya}} is always used in the sense of “calculable, comprehensible”, not “after comprehension”. The point, in line with the previous and succeeding verses, is that the Buddha explains things in a way that makes them clear. It’s a clever pair of lines, best served by sticking close to the literal form. } He presents the \textsanskrit{Aitareyabrāhmaṇa}’s story of Rohita, an exiled prince in the forest who is exhorted by Indra (in disguise as a brahmin) in the virtues of the solitary forest life, wandering pre-eminent and untiring as the Sun. He shows that the idea has ancient roots and that Indra himself is the archetypal solitary hero, always wandering alone. Many texts exhort the brahmin students and renunciants to wander alone, thus following the example of Indra and the Sun. Jain literature likewise extols the solitary wanderer, who crushes his defilements like a war elephant crushing his foe, like a lion, like the sun, like a bird free in the sky. And the Jains even use the same term \textit{\textsanskrit{khaggivisāṇa}}.

In all these cases, what wanders alone is a being, not a thing. (The sun was regarded, at least poetically, as a divinity.) Nonetheless, there has been a long discussion in the Indological community about whether the term \textit{\textsanskrit{khaggivisāṇa}} means “that which has a sword for its horn” (i.e. a “rhinoceros”), or “the horn of a rhinoceros”. The most recent discussion on this is by Bhikkhu Bodhi, who supports Norman in reading \textit{\textsanskrit{khaggivisāṇa}} as “the horn of the rhinoceros” against both those who would read it as “a rhinoceros” and those who treat it as deliberately ambiguous. He shows that \textit{khagga} in other passages denotes the animal, and argues that “if \textit{khagga} is the rhinoceros, then \textit{\textsanskrit{khaggavisāṇa}} is the horn of that animal.” By analogy, however, if “rhino” is the animal, then “rhinoceros” must be the horn (“ceros”) of that animal. Similar abbreviations are found in Sanskrit, as pointed out by Dhivan Thomas Jones. \footnote{Neither Norman nor Bodhi note \textit{sandhiyatimeva} in identical context at AN 3.132:3.2. There Bodhi has “remains on friendly terms”, I have “stay in touch”. I assume there is a confusion of the negative which is required to give the correct sense. If I’m right, the corruption must predate the commentary. }

It remains the case that throughout the literature, the subject is said to “wander alone” like something else that wanders alone, not to “wander alone” like something else that is alone. This includes the \textsanskrit{Khaggavisāṇasutta} itself, where the mendicant is urged to wander alone like a tusker, or like a king who flees his realm, or like a wild deer. Nowhere else do we split “wander” from “alone”, as is required if we are to read “wander alone like the rhinoceros’ horn (is alone)”.

I wonder, however, whether \textit{khagga} and \textit{\textsanskrit{khaggavisāṇa}} are fully identical. When an elephant is used in this context, it is not just any elephant, but a \textit{\textsanskrit{nāga}}, a fully grown tusker or bull elephant. It seems that the male Indian rhinoceros has similar habits, in that he will associate with a breeding herd for some time, but is often found alone. I suspect that \textit{\textsanskrit{khaggavisāṇa}} might likewise be a male “horned rhinoceros”. Against this is the fact that female Indian rhinos also have horns. It is also the case, however, that in ancient India there were two species of rhinoceros. The Indian Javan Rhinoceros, which was hunted to extinction by the 1920s, is unusual in that the cow has no horn.

This may not be a decisive problem, though, as female elephants have horns, yet we still use “tusker” for a bull. Obviously, the phallic imagery plays a part here. It’s also true that females, both elephants and rhinos, may sometimes “wander alone”. The point is not that these things are categorical truths, but that they are evocative ideas employed by a poet to paint a mood. This is all very tenuous, but I am inclined to think that \textit{\textsanskrit{khaggavisāṇa}} means “bull rhino”.

The imagery is tough and masculine: rejecting affection, living independently like a lion, and suspicious of others’ motives. An overly harsh attitude to seclusion can lead to a certain hard-heartedness, so the poem does not neglect to mention the development of love and compassion. Indeed, when the Sutta is quoted in the \textsanskrit{Apadāna}, there is an extra verse right at the start, which begins the whole discourse speaking of love and compassion (Tha-ap 2:9.3). It gives rather a different framing to the poem.

A closer look at this verse shows that three of the lines are the same as the verse that follows, meaning it adds only one line to the \textsanskrit{Suttanipāta} version: \textit{mettena cittena \textsanskrit{hitānukampī}}. In fact, this verse appears four times with only the third line changed—once in the \textsanskrit{Suttanipāta}, the extra verse in the \textsanskrit{Apadāna}, in the \textsanskrit{Mahāvastu}, and in the \textsanskrit{Gandhārī}. The verse opens with the same two lines:

\begin{verse}%
When you’ve laid down arms toward all creatures,\\

not harming even a single one,

%
\end{verse}

Then the \textsanskrit{Suttanipāta} rather abruptly changes course: “don’t wish for a child, let alone a companion.” But the other three versions all continue on the theme of love. The first verse of the \textsanskrit{Apadāna} has here, “live full of compassion with a loving heart”. It turns out that the \textsanskrit{Gandhārī} version starts with the very same verse including the same third line (\textit{\textsanskrit{metreṇa} \textsanskrit{citiṇa} \textsanskrit{hitaṇ}‍\textsanskrit{ukaṁpi}}). The \textsanskrit{Mahāvastu} has a different line on a similar theme, “having laid down arms against creatures firm and frail”. \footnote{\textit{Atthadaso}: Bodhi follows comm in “seeing the good”, Norman has “seeing the goal”. I usually have “seeing the meaning”, and I think that is supported here. It is at the start of the practice, after several lines talking about the teaching. The point is, I think, that a reflective person doesn't just hear the teaching, but understands the point of it and puts it into practice. }

It is also noteworthy that the “lost line” of the \textsanskrit{Apadāna} and the \textsanskrit{Gandhārī} is also found in the \textsanskrit{Mahāvastu}: \textit{\textsanskrit{maitreṇa} cittena \textsanskrit{hitānukaṃpī}}. But there it is in a different verse along with compassion, joy, and equanimity. Thus we find verses added or subtracted, and lines swapped in different places more or less aptly. The “lost line” of the \textsanskrit{Apadāna} would seem to be a case where a later text preserves a verse lost to the generally earlier texts.

The fact that this is found in the \textsanskrit{Mahāvastu}, however, does not mean that the \textsanskrit{Mahāvastu} version as a whole is earlier. For five of its twelve verses expand on the topic of “affection” (\textit{sneha}), adding minor variations to a topic dealt with in only one verse of the Pali and the \textsanskrit{Gandhārī}.

All of our current versions have clearly been subject to a degree of revision and flexibility. This is easily done in such a format, where there is no logical sequence of ideas, merely verses on a related theme.

The tradition—including the Niddesa, the \textsanskrit{Apadāna}, the commentary, the \textsanskrit{Mahāvastu}, and the \textsanskrit{Divyāvadāna}—is unanimous in attributing these verses to the \textit{paccekabuddhas}. The association is so close that the \textsanskrit{Divyāvadāna} uses “one who is like a rhinoceros” as an epithet for a Paccekabuddha, even outside the context of the poem (Divy 37, 490.003). Paccekabuddhas were a mysterious class of enlightened sages who lived a solitary life of meditation without establishing a dispensation. They are not prominent in the suttas, featuring in one late sutta of the Majjhima (MN 116), and in four passing mentions in the \textsanskrit{Aṅguttara}. But the later texts of the Khuddaka show the growing popularity of the idea, mentioning them fifty-seven times. Clearly, if they were part of early Buddhism at all, it was in a very minor role. Given that the \textsanskrit{Khaggavisāṇasutta} itself says nothing of the Paccekabuddhas, it seems unlikely that this was the original intent. It would seem that the idea of the Paccekabuddha was part of the pan-sectarian development of what one might call “Buddhology” in the post-Ashokan period, along with the \textsanskrit{Jātakas}, the \textit{\textsanskrit{pāramī}} (“perfections”), the ideal of the Bodhisattva, the Buddha image, and the literary legends detailing the life of the Buddha.

Rejecting the traditional Paccekabuddha connection, scholars such as Jayawickrama and \textsanskrit{Saddhātissa} have argued that the sutta represents an early period of ascetic Buddhism before settled monastic life. Bhikkhu Bodhi, on the other hand, points out that the Vinaya \textsanskrit{Mahāvagga} and its parallels depict settled monastic life developing soon after the Buddha began teaching. As an alternative theory, he suggests the \textsanskrit{Khaggavisāṇasutta} may have been meant for a class of forest-dwelling, reclusive mendicants.

Bodhi’s point about the early introduction of building is well-taken. Let alone the Vinaya \textsanskrit{Mahāvagga}, this is found even in the oldest portion of the Vinaya, namely the \textsanskrit{Pātimokkha}, which is common to all the schools and is universally accepted as an early document. There we find not just mention of buildings, but requirements as to building methods and materials, safety regulations, restrictions on the size of huts, environmental regulations governing prospective sites, procedures for communal agreement on new construction, and acknowledgement of the role of sponsors. This speaks to a developed and sophisticated approach to communal buildings in the lifetime of the Buddha.

It thus seems unlikely that these verses predate buildings for monastics, but does this mean they were addressed to a distinct class of reclusive mendicants? There are a couple of verses in the \textsanskrit{Khaggavisāṇasutta} that, to my mind, suggest a quite different origin. If we trace them back to their source, it turns out that they originate not from an early period or from a special class of renunciants, but as a call for those who have allowed settled monastic life to dull their inspiration.

These verses are numbers 11 and 12. They nuance the virtue of solitariness: if you can find a good companion, that is best, but if not, live alone. While this is more conciliatory than the sterner stuff typical of the \textsanskrit{Khaggavisāṇasutta}, we have already seen that the messages of love and compassion are also central to the text.

These verses also appear in the \textsanskrit{Gandhārī}, although in a different position. Nonetheless, they are clearly an interpolation. They have a distinct form and act a pair, as the first verse lacks the usual ending tagline and instead leads on to the second. What is more, these verses occur independently in a quite different context, which is found in the Vinaya (Khandhaka 10:3.1.32–39), as well as the Majjhima (MN 128:6.30–37) and the \textsanskrit{Jātakas} (Ja 428:8.1–9.4. They are also found at Dhp 328–9, where there is no context). So it is clear that they have an independent origin, but it seems that the redactors of old considered their message and style similar enough to justify their inclusion in the \textsanskrit{Khaggavisāṇasutta}. This probably predated the split between the Pali and \textsanskrit{Gandhārī} versions.

Now, the background to these verses concerns the so-called incident at Kosambi, which is related in detail in the Vinaya. There, monks had a rather dispiriting falling out, allowing a trivial difference of opinion to spiral into an acrimonious dispute. The Buddha, unable to reconcile them, gave up and retreated alone into the forest where he spoke these verses. This famous story is overlooked entirely in the commentary on the \textsanskrit{Khaggavisāṇasutta}. But surely it gives us a clue to the role such verses play in the community.

Each verse of the \textsanskrit{Khaggavisāṇasutta} is an \emph{ought}. When a spiritual teacher sees that a community is out of balance, they teach what people should do, not what they are already doing. The narrative background shows that these two verses were a reaction to increasing settledness and institutionalization. And if that is true of these verses, it may well be the case for the \textsanskrit{Khaggavisāṇasutta} as a whole.

This highlights a further fact, one that we know with far more certainty than anything about the origins of the text. The \textsanskrit{Khaggavisāṇasutta} has been maintained and passed down for thousands of years in monastic communities. It was widely popular across the traditions, and the overwhelming majority of those who learned and recited it, and later, those who read and studied it, have been settled monastics. Why would this not be the original audience? Spiritual teachings do not act in a unilateral way, where one thing is addressed to one group of people who are already doing that thing. No: the whole point is to instill a yearning for something else.

The monastic vocation, to this day, has a tension between a life of institutionalized security and one of wandering freedom. This is part of the core dynamic of our communities. It cannot be resolved by a decree; all mendicants feel the call of the solitary life. To live alone, aloof from the dramas and desires of other people, the open road and the creatures and the trees your only companions; who has not felt the yearning for such a life?

While it does not affect the above argument, I should note that there is a confusion in these verses that needs clearing up. The reading of the two verses differs in the editions. The Pali Text Society edition abbreviates the tagline with just \textit{eko care}, implying that it should be expanded as the usual “wander alone like a rhinoceros” (\textit{eko care \textsanskrit{khaggavisāṇakappo}}). The Buddha Jayanthi Tipitaka edition spells out the “rhinoceros” tag here, supporting the PTS reading. This has been followed by Norman and Bodhi, who both note that this line differs from its parallels. If this reading is correct, then these verses have been adapted to this context by giving them the “rhinoceros” tag. In the \textsanskrit{Mahāsaṅgīti} edition, however, these verses are essentially identical with their parallels, ending with the tag “wander alone like a tusker in the wilds” (\textit{eko care \textsanskrit{mātaṅgaraññeva} \textsanskrit{nāgo}}). That this is no oversight is clear from the fact that MS acknowledges the PTS and BJT variant. The commentary is no help, as it is silent on this line. In the \textsanskrit{Apadāna} version, the “elephant” tag is found in all three editions. The \textsanskrit{Mahāsaṅgīti} edition, however, has the “rhinoceros” version in the Niddesa (Cnd 23:140.4), as does the BJT and apparently also the PTS, though once again it is abbreviated. The \textsanskrit{Mahāsaṅgīti} readings throughout are the same as the VRI edition on which it is based, representing the Burmese edition of the Sixth Council. The \textsanskrit{Gandhārī} has “rhinceros”.

Clearly, the last line of the verse originally referred to an “elephant” and it was changed to “rhinoceros” to fit the context of the \textsanskrit{Khaggavisāṇasutta}. Editors have evidently been swayed in both ways, either to keep consistency with the original context or creating a new consistency with the new context. Given the support of the \textsanskrit{Gandhārī}, I am inclined to think the change was a deliberate choice of the composer or compiler at an early date.

Returning to the question of dating, we have seen that the \textsanskrit{Khaggavisāṇasutta} is subject to many of the same issues found throughout the verse collections. Verses are added or lost, lines are swapped around, and exact terms and phrases are varied. The interpretation of the Niddesa, the earliest commentary, is not infallible, and it employs a doctrinal framework, the Paccekabuddha, which is later than the original context. Note that it is the last text of the Niddesa, following the \textsanskrit{Pārāyanavagga}, which also includes late passages. Verses have been multiplied in a formulaic way, there are repetitions, imports from elsewhere, and awkward transitions. The whole is, like the chapters of the Dhammapada, a compilation, a form that readily lends itself to expansion and reshuffling. Comparison with \textsanskrit{Gandhārī} and Sanskrit versions reveals no less variation than we might find in any other verse collections. Attempts to identify the text with an early period of renunciant life are not persuasive. Where we can establish context, it suggests that the text began as it continued and as it is today: a call for renewal praising the solitary wandering life primarily for those who are not living that life.

\subsection*{\textsanskrit{Bhāradvāja} the Farmer}

With the fourth Sutta, we see a return to a Brahmanical context, but this time in a more conventional way (Snp 1.4). The \textsanskrit{Kasibhāradvājasutta} tells of a brahmin who works as a farmer. To celebrate the sowing season, he distributes food for anyone who comes.

The sutta is found also in the \textsanskrit{Sagāthāvagga} of the \textsanskrit{Saṁyutta} (SN 7.1) with a shorter narrative portion. Jayawickrama describes this text as a “pastoral ballad”. \footnote{\textit{Suci} here refers to his appearance, as agreed by the commentary, so “pure” is not very useful. } He concludes that the \textsanskrit{Suttanipāta} version is adapted and expanded from the \textsanskrit{Saṁyutta} version, while also showing the influence of the Sundarikasutta (SN 7.9) with a shorter narrative portion.

The events unfold in the Southern Hills, an outlying district rarely visited by the Buddha and his \textsanskrit{Saṅgha} (Khandhaka 1:53.1.3; see the discussion on the quotations of the \textsanskrit{Parāyanavagga}). This gives some context for the farmer’s unsettled reaction to the sight of the Buddha, as if it were the first time he had seen such an ascetic. And it suggests that the text served as a conversion narrative for this region.

It is worth noting that the name \textsanskrit{Bhāradvāja} is used in the Suttas as a conventional appellation for a (usually snooty) brahmin, and it is regularly prepended with a nickname which is simply the thing discussed in the Sutta. Here \textit{kasi} means simply “farming”. Below we shall meet the \textsanskrit{Bhāradvāja} \textit{aggika}, which means “fire-worshipper”. These are more like descriptive epithets than personal names.

Hymns for farming are found as far back as the old Vedic tradition, many centuries before the \textsanskrit{Upaniṣads}. The Vedic hymns focus on the celebration of abundance and prosperity in life, and that includes agriculture. Early Vedic hymns include invocations to the “Lord of the Field” to bless the ploughing and the crops (Rig Veda 4.57.4). These invocations did not remain merely as dead letters, for they are cited in much later ritual texts, which say that a brahmin should recite them when touching the plough (\textsanskrit{Saṅkhāyana} Grihya \textsanskrit{Sūtra} 4.13). Such verses, and others (eg. Rig Veda 10.101.3), invoke the different parts of the ploughing—goad, plough, seed, furrow—in a way not dissimilar to the \textsanskrit{Kasibhāradvājasutta}. Atharvaveda 6.142.3 says that the “givers” of the grain shall be inexhaustible, perhaps suggesting a food distribution practice.

To pour seed on the ground is an act of faith: the belief that, gods willing, the seed will grow and supply food the next year. By distributing food—the fruits of last year’s sowing—the brahmin is mirroring the act of sowing itself and thus amplifying its fruitfulness. That this is not an entirely selfless act becomes apparent when he sees the Buddha and tells him he can only eat if he farms. His distrust of ascetics hails from the more worldly traditions of the Vedas, rather than the mystical path of renunciate sages like \textsanskrit{Yājñavalkya}.

Drawing on similar rhetorical tactics that he used in the Dhaniyasutta, the Buddha ascribes a spiritual meaning to each of the items of the farmer. This kind of point-by-point spiritual metaphor is a characteristic of the \textsanskrit{Upaniṣads}, especially in those portions that ascribe a hidden meaning to each aspect of the ritual and the rites. That this is central to the \textsanskrit{Upaniṣadic} project is shown by the fact that the \textsanskrit{Bṛhadāraṇyaka} \textsanskrit{Upaniṣad} opens with such a set of correspondences (\textit{sandhi}).

\begin{quotation}%
The head of the sacrificial horse is the dawn; its eye is the sun; its vital breath is the wind …

%
\end{quotation}

And so it goes. It was in this way that the \textsanskrit{Upaniṣadic} philosophers were able to break new ground, extending and deepening their philosophy, while still maintaining that they were continuing the same tradition. Hence the leitmotif of the \textsanskrit{Upaniṣads}: \textit{ya evam veda} “one who knows this …”. Each act, each detail, has a higher meaning, which a true brahmin understands. The Buddha is using exactly the same technique to convince the brahmin that there is a higher meaning behind his traditional duties.

In a curious twist, the brahmin offers his milk-rice to the Buddha “whose farming has the deathless as its fruit”. The Buddha rejects his offering—a very rare circumstance, as mendicants are normally required to accept any offering with gratitude and humility. The poem says this food has been “enchanted by a spell”, and indeed, when it is discarded it sizzles and fumes in the water.

The implications of this episode are complex. Dietary laws were, and are, fundamental to the notion of caste in India. The brahmins—in theory—took only the purest food and did not share a plate with others. The Āpastamba \textsanskrit{Dharmasūtra} says that the student must “not leave any food uneaten. If he is unable to do so, he should bury the leftovers in the ground, (or) throw them in the water” (1.4.12). This was in direct contrast with the practice of ascetics such as the Buddha, who accepted food from anyone without discrimination.

The brahmin has performed a ritual by chanting. That this is literally an “enchantment” rather than simply “chanting over” is shown by the fact that the food has acquired a mysterious magical property. The brahmin caste represents the divinity \textsanskrit{Brahmā} on earth, from whom all power and creative force derive. In their belief system, they are a manifestation of divine energy, and through their chanting, they channel that energy into the material realm. Thus the milk rice is infused with a potency, which, like all power, is dangerous in the wrong hands.

The Buddha can consume it since his power supersedes that of \textsanskrit{Brahmā}. Yet he chooses not to. The text is silent as to why. But we can speculate that the magical act of consuming such potent food would, in the eyes of the brahmins, mark the Buddha as an acolyte of \textsanskrit{Brahmā} himself, else how could he survive it? By rejecting the ritually potent food, the Buddha is affirming his independence from the Brahmanical system. He has no need for their enchantments. For him, food is simply a material means of sustenance.

\subsection*{With Cunda the Smith}

Next there follows the Cundasutta (Snp 1.5), which the Buddha apparently taught near the end of his life to the smith Cunda at his home in \textsanskrit{Pāvā}, when he famously gave the Buddha his last meal. It is omitted from the account of the meeting between Cunda and the Buddha in the Pali \textsanskrit{Mahāparinibbānasutta}, but it is found in the Sanskrit versions of that text. Both the Sanskrit text and the Pali commentary give the same origin: a monk steals the golden dish on which Cunda had served the meal.

Seeing such a badly-behaved monk, Cunda asks the Buddha to define the different kinds of ascetic (\textit{\textsanskrit{samaṇa}}) that are found in the world. The Buddha gives three examples of good ascetics and one of bad. He encourages lay disciples to use discernment and not equate one with the other.

The topic of good and bad mendicants was a pressing one as the young Buddhist community faced the imminent demise of its beloved teacher. The senior remaining teacher, Venerable \textsanskrit{Mahākassapa}, saw the signs of decay right after the Parinibbana and convened the First Council to establish the \textsanskrit{Saṅgha} going forward. This discourse with Cunda shows that it was not only the monastics who were concerned about the maintenance of discipline in the \textsanskrit{Saṅgha}.

\subsection*{Downfalls}

Continuing the theme of practical advice for lay folk, we next find the first of the great popular Suttas of the \textsanskrit{Suttanipāta}, the \textsanskrit{Parābhavasutta} (Snp 1.6). This gives an extensive list of moral failings into which lay folk might fall, framed within the question of a deity in the manner of the \textsanskrit{Sagāthāvagga}. It is a perennially popular discourse for recitation and sermons.

The ethics are not specifically Buddhist but would have been broadly acceptable to anyone in the Buddha‘s culture, and for the most part, remain so today. It is notable, however, that the actor whose morality is discussed is assumed to be a straight male, so I have translated using the masculine gender.

Jayawickrama notes that the language and style, which are characteristically simple and early, are so similar to Ashoka’s edicts it seems he may have been familiar with such suttas as this, the \textsanskrit{Maṅgalasutta}, and the Vasalasutta.

\subsection*{The Lowlife}

In the Vasalasutta (Snp 1.7) we meet another \textsanskrit{Bhāradvāja}, this time an orthodox fire-worshipping brahmin. The worship of fire is one of the oldest, if not the oldest, rites of humanity, and it was a part of the ancestral Indo-Aryan religion. There is an inherent tension between the performance of religious rituals and the principles of morality: if doing the ritual is truly effective, then why keep moral rules?

Like the other \textsanskrit{Bhāradvāja} we have already met, this gentleman is no great fan of ascetics. When he sees the Buddha, he doesn’t just not offer him anything, but yells abuse, calling him a “shaveling, fake ascetic, lowlife”. While being met with abuse and ridicule in the village is a part of the mendicant’s life, Buddhist mendicants both past and present mostly encounter a warm and respectful welcome when on alms round, and abuse is very much the exception.

The Sutta details a series of moral failings in a way not dissimilar to the previous. The Buddha, responding to the personal attack on him, turns it around, claiming that one who abuses ascetics or reviles the Buddha is the real lowlife. He argues that deeds, not birth, make you a lowlife or a brahmin, implying that the brahmin himself is no brahmin but an outcaste.

He tells a story of \textsanskrit{Mātaṅga}, an outcaste who lived a life of such virtue that he became a \textsanskrit{Brahmā} divinity himself, implying that the brahmin had all this time been worshipping an outcaste. This story has apparently been drawn from the vast ocean of Indian story, for it also appears in the \textsanskrit{Mahābharata} \textsanskrit{Anuśāsika}-parvan (Mbh XIII, 3, 198 ff), as well as in the \textsanskrit{Mātaṅga} \textsanskrit{Jātaka} (Ja 497). As is not infrequently the case, it would seem that the Buddhist tradition maintains the oldest form of a story better known from its Brahmanical retelling. But as Jayawickrama observes, it is likely that all these sources draw from a story circulating in the long oral tradition. For this as well as linguistic reasons, he suggests that the story, as well as the prose introduction, are likely to have been an expansion of the early ethical core of the verses (PBR).

The Buddha points out that it is precisely those brahmins who are most often seen reciting hymns and performing pious deeds who turn out to be deeply involved in corrupt deeds. This remains as relevant to Buddhism now as it was to Brahmanism then.

\subsection*{The Discourse on Love}

Shifting to a positive mood, we next find the famous Mettasutta (Snp 1.8). Rather than a series of verses on a theme, this is a coherent poem divided roughly into three sections: verses 1–3ab lay the foundation for virtuous conduct; 3cd–6 describe the meditation on love; while verses 7–9 depict the exultant state of liberation that results. Verse 10 is in a different metre and appears to be a later addition, describing in a highly compressed form the development of insight for attaining first stream-entry, then non-return.

I now translate \textit{\textsanskrit{mettā}} as “love” rather than the Buddhist neologism “loving-kindness”. The latter has become widely accepted and is justified by arguing that “love” has too much of a sensual connotation. And it is true that Pali distinguishes sensual love (\textit{\textsanskrit{kāma}}, \textit{pema}, etc.) from spiritual love (\textit{\textsanskrit{mettā}}), much like the Greek \emph{eros} and \emph{agape}. I once asked a Catholic contemplative monk about this. His native language was Italian. He said there is no equivalent distinction in modern Italian; they just use \emph{amore} in both cases and let the context make the meaning clear. I adopted the same approach, and it seems to work fine.

I suspect that the real reason for the rendering of “loving-kindness” is that we can be uncomfortable expressing emotions. “Loving-kindness” is a more distancing word; it’s emotionally cooler than “love”. I prefer the more direct, ordinary language expression.

The opening lines pose something of an interpretive problem, as evidenced by the complex discussion in the commentary. The second line refers to someone who has realized the “peaceful state” of \textsanskrit{Nibbāna} (\textit{\textsanskrit{santaṁ} \textsanskrit{padaṁ}}). The term for “realized” (\textit{abhisamecca}) is usually reserved for the breakthrough to the four noble truths, eg. SN 56.4:1.1). Such a person must then be a stream-enterer at least (SN 56.49:1.7, see also \textit{\textsanskrit{āgataphalā} \textsanskrit{abhisametāvinī} \textsanskrit{viññātasāsanā}} at Aniyata 1:2.1.26). Is the poem then restricted to the practice of a stream-enterer? Given the universal nature of its themes, this seems unlikely.

Jayawickrama suggests that “the peaceful state” need not mean \textsanskrit{Nibbāna}, but this does not seem to be supported in early Pali. Bodhi addresses this issue in his note 685, where he confirms that a stream-enterer is meant, rejecting the commentary’s alternative explanations. This is clearly the most straightforward reading of the lines.

There remains, however, the problem of how this fits with the remainder of the text. The stream-enterer is “perfected in ethics”. Yet the subject of the Mettasutta is exhorted to have moral integrity, to be admonishable, to not be overly demanding of donor families, and to not act in a blameworthy way, all of which are things that would surely come naturally to a stream-enterer. Further, as we shall see below, the original poem merely taught as far as rebirth in the \textsanskrit{Brahmā} realm, and it is unlikely that the Buddha should teach this to a stream-enterer. Overall, the poem has a high degree of unity and purpose, and this all feels ill-fitting.

Revisiting the commentary, it offers several different approaches. Among them, it suggests the absolutive may be read in an infinitive sense. That is, instead of “what has been realized”, what it means is “in order to realize” (\textit{abhisamecca \textsanskrit{viharitukāmo}}). While the primary gloss on \textit{abhisamecca} confirms that it is an absolutive (\textit{\textsanskrit{abhisameccāti} \textsanskrit{abhisamāgantvā}}), it also explains it in terms of an infinitive (\textit{\textsanskrit{adhigantukāmena}}). Further, it speaks of a mendicant who “is practising to attain that state” (\textit{\textsanskrit{tadadhigamāya} \textsanskrit{paṭipajjamāno}}).

If we are to adopt this reading, we are left with a further puzzle in the term \textit{atthakusala}. The word \textit{attha} means many things, but here it is interpreted by the commentary as what is good or beneficial, and this is followed by most or all modern interpreters. It does, however, create a similar problem to that discussed in the previous paragraph, because it is generally considered that only a stream-enterer is accomplished in ethics. Those who have not seen the four noble truths can, of course, live a good life, but they are not yet “experts”. The commentary appears to be aware of this problem, as it offers an infinitive reading for \textit{atthakusala}, that is, “one who wishes to dwell in the fourfold ethical purity” (\textit{\textsanskrit{catupārisuddhisīle} \textsanskrit{patiṭṭhātukāmo}}).

However, such a reading is not necessary here, for \textit{atthakusala} occurs in the Suttas, where it refers to expertise in the scriptures (AN 5.169:3.1):

\begin{quotation}%
a mendicant is skilled in the meaning (\textit{atthakusala}), skilled in the teaching, skilled in terminology, skilled in phrasing, and skilled in sequence.

%
\end{quotation}

Now we have a much more satisfying reading for the opening lines. They are addressed to someone who has already mastered the textual teachings and wishes to realize them in practice. This follows exactly the same pattern as practised by the \textit{bodhisatta} under his former teachers, where he first learned the scriptures and then undertook meditation. It is also the fundamental framework for monastic practice, the Gradual Training, where the mendicant learns the Dhamma and then goes off to meditate.

This also allows us to situate the opening of the Mettasutta more precisely with the added final verse, which begins with the words “avoiding views”. This is a major theme of the \textsanskrit{Suttanipāta}, especially the \textsanskrit{Aṭṭhakavagga}, which formed one of the kernels around which the \textsanskrit{Suttanipāta} formed. The redactors who added this verse must have known the \textsanskrit{Aṭṭhakavagga}, with its oft-repeated admonishment to avoid the trap of heated and angry debating on views. It is quite possible, likely even, that this verse was added when the Mettasutta was included in the \textsanskrit{Suttanipāta}, creating a connection with the themes of the \textsanskrit{Aṭṭhakavagga}, offering a meditation path especially suited to those with a tendency to become heated in their arguments.

The Mettasutta urges us to spread love to all beings, including those who are “born or to be born” (\textit{\textsanskrit{bhūtā} \textsanskrit{vā} \textsanskrit{sambhavesī} \textsanskrit{vā}}). The latter phrase, which occurs only here and in a stock phrase at eg. SN 12.11:1.3, evidently refers to beings who are in the process of taking a new life, an idea that is evidently at odds with the early Theravadin insistence that one life follows immediately after another, with no “in-between state” (Kv 8.2). A majority of early Buddhist schools accepted this state, and this phrase in the Mettasutta joins a list of other contexts in the early texts that show with reasonable certainty that the early Buddhists did too.

This doctrinal point speaks to the earliness of the poem, as does the metre (old Ārya), the coherent flow of ideas, and the focus on the universal subject of \textit{\textsanskrit{mettā}} without a forced “Buddhistic” perspective, the lack of which evidently prompted the later addition of the final verse.

It is assumed in the Buddhist tradition that \textit{\textsanskrit{mettā}} and the other \textit{\textsanskrit{brahmavihāras}} (\textit{\textsanskrit{karuṇā}} or compassion, \textit{\textsanskrit{muditā}} or rejoicing, and \textit{\textsanskrit{upekkhā}} or equanimity) are pre-Buddhistic, and belong with the very many ideas that the Buddha happily adopted from his religious surroundings. While they are not, to my knowledge, attested in any surviving pre-Buddhist texts, they are found in the \textsanskrit{Yogasūtra} (1.33), several later \textsanskrit{Upaniṣads}, the Jain \textsanskrit{Tattvarthasūtra} (7.11), and even a Tibetan Bon text of the eleventh century, “A Cavern of Treasures” (\textit{mDzod-phug}).

If we look at the way the word \textit{\textsanskrit{brahmavihāra}} is used in the suttas, we find that apart from the Mettasutta, there is one other verse where the term particularly refers to the practice of \textit{\textsanskrit{mettā}} (Thag 14.1:5.3). The term is sometimes used for mindfulness of breathing, along with \textit{\textsanskrit{ariyavihāra}} and \textit{\textsanskrit{tathāgatavihāra}}, in which case the term \textit{\textsanskrit{brahmā}} must refer to the being of that name, rather than an abstract sense such as “divine” or “holy” (SN 54.11:3.2, SN 54.12:8.4).

While the group of four qualities are commonly taught, they are referred to as \textit{\textsanskrit{brahmavihāras}} specifically to indicate that such a practice leads to rebirth in the \textsanskrit{Brahmā} realm (MN 83:6.2, AN 5.192:6.8, DN 17:2.13.8). Elsewhere the Buddha clarifies that such practices, which he did in his past life, do not lead to \textsanskrit{Nibbāna}, unlike his own eightfold path (DN 19:61.4). When someone who is a Buddhist disciple practices them, then if they do not realize full enlightenment in this life, they will do so after being reborn in the \textsanskrit{Brahmā} realm (AN 4.125, AN 4.126)

This lends a greater specificity to the line “this is a meditation of \textsanskrit{Brahmā} in this life”, which would have been the final line of the original poem. The sense is that through this meditation, one can live like the god \textsanskrit{Brahmā} in this life; and such a life leads to being reborn as a \textsanskrit{Brahmā} in the next. While it is unusual to find a Buddhist text that ends with such a rebirth, it is not unique. In the Tevijjasutta the Buddha teaches this path to some brahmin questioners (DN 13). \textsanskrit{Sāriputta} did the same in MN 97, although the Buddha evidently felt he should have gone further. What is unusual, however, is that such suttas are taught to those who were not committed Buddhists, while the introductory passages of the Mettasutta evidently address Buddhist mendicants.

As with other didactic texts of the early period, the poem is noteworthy for its plain style and lack of metaphors. But this restraint shows no lack on behalf of the poet, for when the metaphor of a mother’s love for her child is introduced, it comes at the climax of the poem. The restraint creates a heightened emotion here, which is one of the secrets behind this poem’s enduring popularity.

\subsection*{The Buddha Teaches \textsanskrit{Sātāgira} and Hemavata the Native Spirits}

The Hemavatasutta finds two \textit{yakkhas} (“native spirits”) having a perhaps unlikely conversation about the Buddha’s spiritual qualities as they debate whether to visit him on the \textit{uposatha} (“Sabbath”) and proceed to ask a series of questions (Snp 1.9). The questions are of a riddling form—“In what has the world arisen?”—and seem to presuppose a serious interest in Buddhism, or at least, in the kinds of problems discussed among ancient Indian ascetics.

There is nothing in the content itself, however, that suggests the speakers were \textit{yakkhas}. Most of the verses are found elsewhere in the \textsanskrit{Saṁyuttanikāya}, and it is likely that the current Sutta is a compilation, although there is nothing to say its contents are late.

Generally speaking, the \textit{yakkhas} of the Suttas are morally neutral, though they sometimes succumb to aggressive impulses. Rarely do they display such an elevated sense of morality. \textsanskrit{Sātāgira} and Hemavata appear together in the \textsanskrit{Āṭānāṭiyasutta} (DN 32:10.10), while thousands of spirits from these two places, of which our \textit{yakkhas} are evidently the chiefs, attend the Great Congregation in the \textsanskrit{Mahāsamayasutta} (DN 20:7.7). These late passages probably tell us little beyond the fact that their authors were familiar with the Hemavatasutta.

\textit{Yakkhas} are local deities, the spirits of particular places. Their names reveal them to be titulary deities of the mountainous regions: the Himalayas and “Mount \textsanskrit{Sāta}”, which the commentary says was in the middle region. They end by promising to travel far and wide spreading the good news of the Dhamma. This suggests that the verses acted as a missionary text for the Himalayan region, into which Buddhism was spreading in the century after the Buddha’s death.

\subsection*{\textsanskrit{Āḷavaka} the Native Spirit}

The ferocious side of the \textit{yakkhas} is expressed by the monstrous \textsanskrit{Āḷavaka} (Snp 1.10). He was given an extensive backstory in Pali commentaries, and a quite different one in the Dhammapada that was translated into Chinese as the Fa-chu pi-yu-ching. \footnote{Reading \textit{tato}. }

He comes out full of bully and bluster in the prose narrative, but the verses settle down to a conventional exchange of Dhamma questions.

\textsanskrit{Āḷavaka}, like \textsanskrit{Sātāgira} and Hemavata, is a titular deity who announced his zeal for spreading the Dhamma, but his town \textsanskrit{Āḷavī} was less remote, lying between \textsanskrit{Sāvatthī} and \textsanskrit{Rājagaha}.

\subsection*{Victory Over Desire for the Body}

Though separated by two Suttas, the Vijayasutta (Snp 1.11) might be considered as a pair with the Mettasutta, as the former teaches the meditation on love that overcomes hate, while this teaches the meditation on the parts of the body that overcomes desire (verse 11, \textit{\textsanskrit{kāye} \textsanskrit{chandaṁ} \textsanskrit{virājaye}}). I have observed that the Mettasutta appears to be meditation instructions for one who has already studied the texts, and the Vijayasutta makes this explicit, saying that the practice is for a wise mendicant who has “learned the Buddha’s words” (\textit{\textsanskrit{sutvāna} \textsanskrit{buddhavacanaṁ}}).

Rather curiously, there is no explanation of the title \textit{vijaya} (“victory”) in the Sutta or its commentary. The commentary gives an alternate title, \textit{\textsanskrit{kāyavicchandanikasutta}}, and I have incorporated this in my translation as it makes the sense plain.

The meditation on the body is found commonly in the Suttas, and here is expressed in poetic form. Indeed, it seems as if the poem was constructed to provide a poetic source for body contemplation that summarizes the relevant teachings of the \textsanskrit{Satipaṭṭhānasutta} (MN 10).

The \textsanskrit{Satipaṭṭhānasutta} begins its list of body contemplations with mindfulness of breathing, then continues with a series of meditations. The poem falls naturally into four portions, each of which corresponds with a section on body contemplation, leaving aside the sections on mindfulness of breathing and the elements.

\begin{enumerate}%
\item Verse 1 is the contemplation of postures, which corresponds with the second and third sections on body contemplation.%
\item Verses 2–7 lay out the various parts of the body for contemplation, corresponding with the fourth section on body contemplation. There is the notable addition of the “hollow head filled with brains” which is lacking from the \textsanskrit{Satipaṭṭhānasutta} but found in later canonical texts (Kp 3, Mil 3.1.1:3.40, Ps 1.1:34.30, Ne 17:11.2).%
\item Verses 8–9 describe the rotting body in the charnel ground, as per the sixth section on body contemplation.%
\item Verses 10–14 describe the nature of the meditation itself and its benefits, corresponding to the refrains in the \textsanskrit{Satipaṭṭhānasutta} that speak of comparing one’s own body with the corpse, contemplating internally and externally, and seeing it clearly for the sake of wisdom.%
\end{enumerate}

These parallels are too consistent to be a coincidence. The sections on body contemplation in the \textsanskrit{Satipaṭṭhānasutta} are, however, widely variable in different versions of that text. The exact details seem to have been open for interpretation until quite a late date, which is why I consider it one of the latest prose texts in the canon. It is possible that the Vijayasutta was based on the \textsanskrit{Satipaṭṭhānasutta} more-or-less as we have it today, with the addition of the “brain”. If this were the case, the text must be one of the latest additions to the \textsanskrit{Suttanipāta}. Jayawickrama, in fact, regards it as one of the latest texts in the \textsanskrit{Suttanipāta}, although he does not go into the reasons why.

It would seem likely that this poem was composed to ensure that students of the \textsanskrit{Suttanipāta} had appropriate instructions on this important form of meditation. This thesis is supported, albeit obliquely, by the fact that in the final verse we find an emphasis on overcoming conceit, which is not mentioned in the \textsanskrit{Satipaṭṭhānasutta}, but which is a major theme of the old portions of the \textsanskrit{Suttanipāta}.

The text refers to the “nine streams” of liquid that flow from the body. This idea is found a few times in the \textsanskrit{Theragāthā} (Thag 4.4:1.3, Thag 19.1:44.4, Thag 20.1:6.3). There, the meditation on the body is presented in a reflective way, as a personal response to the meditation. They do not list the parts of the body and hence, unlike the Vijayasutta, they do not sound like a meditation manual.

Verses 2–7, the second of the sections mentioned above, is found in the \textsanskrit{Nigrodhamigajātaka} (Ja 12), not in the canonical verse portion, but in the commentary. There it is quoted as part of a series of verses spoken by the deeply spiritual wife of a merchant, who scorns her husband’s expectation that she get dressed up for a festival. This shows that this set of verses must have circulated with a degree of independence. Perhaps it was the kernel around which the whole sutta was formed.

Notably in this story, the woman herself is the agent of the telling, unlike the commentary to the Vijayasutta, where women’s bodies are the locus of desire for both the women themselves and the men who see them. The Sutta, as is normal in the early texts, specifically avoids any gendered description of the body. The contemplation does not objectify the bodies of women, or of anyone at all, but rather focuses inwards, on one’s own body, erasing the distinction between inner and outer, between “mine” and “theirs”.

\subsection*{The Sage}

The final Sutta of the chapter, the Munisutta exalts the solitary life of the renunciate (Snp 1.12). The imagery of the steadfast sage aloof from worldly impulses brings us full circle with the Uragasutta and the \textsanskrit{Khaggavisāṇasutta}.

In contrast with the \textsanskrit{Kasibhāradvājasutta}, rather than co-opting the imagery of agricultural flourishing, here the Buddha defiantly opposes it, speaking of ending new growth. The metaphors draw upon a specific set of ideas found more explicitly in prose suttas such as SN 22.54.

The commentary takes \textit{\textsanskrit{pamāya}} at Snp 1.12:3.1 in the sense of “crushed” (Sanskrit: \textit{\textsanskrit{pramṛṇati}}). However, \textit{\textsanskrit{pamāya}} occurs at Snp 4.12:17.1, where, as it is beside \textit{vinicchaye} (“judge, assess”), it is clearly an absolutive of \textit{\textsanskrit{pamiṇāti}} in the sense “having measured”. Here too it sits beside a word having the sense to reckon or calculate (\textit{\textsanskrit{saṅkhāya}}). In the simile of rebirth and the field, the “seed” is consciousness (AN 3.76:2.3), which, as it belongs to the first noble truth, is not “crushed” but “fully known”. If the seeds are already “crushed” then choosing not to water them (in the next line) is a tad redundant. But if they are “fully known” (the first noble truth) then to not water them (with craving, the second noble truth) makes for a better sense. The idea is that when we stop performing actions motivated by craving, we are no longer setting up the conditions for the stream of consciousness to continue flowing in the future. Consciousness, not being “established” or “planted” in a new life, ceases.

The phrase \textit{na upeti \textsanskrit{saṅkhaṁ}} in Snp 1.12:3.4 is taken from a common idiom used in making comparisons, “it doesn’t count, there no comparison, it’s not worth a fraction” (eg. SN 56.52:1.6.: \textit{\textsanskrit{saṅkhampi} na upeti, upanidhampi na upeti, \textsanskrit{kalabhāgampi} na upeti}). We find it also describing the \textit{muni} at Snp 5.7:6.3. In both cases, it refers to the fact that, since the sage has left behind any reliance on judgments and discriminations, they may no longer be defined by the limiting capabilities of language.

\section*{The Lesser Chapter}

The second chapter is similar in themes and poetic forms to the first. While any thematic development is thin at best, it is possible to discern a thematic arc around the idea of the “teaching” and the “teacher”.

Titles of several poems include the word \textit{dhamma} in the sense of both “teachings” and the life of virtue following those teachings. These start with the Dhammikasutta (Snp 2.6). Despite the title, this speaks mainly of those who do not follow a life of righteousness; of a mendicant who does not even know what has been taught in the Dhamma by the Buddha. The first line uses two terms side by side, \textit{\textsanskrit{dhammacariyaṁ} \textsanskrit{brahmacariyaṁ}}, and the next sutta, the \textsanskrit{Brahmaṇadhammasutta} takes up the theme of the \textit{brahmacariya}, again mainly speaking of the fall from ideal conduct. Salvation is offered in the next sutta, “The Boat”, whose alternate title is “The Teaching”, which describes the ideal teacher as one who can help a student cross the flood. The \textsanskrit{Kiṁsīlasutta} next describes the attitude and behavior of a good student, while the \textsanskrit{Uṭṭhānasutta} urges the application of the teaching in meditation, and the \textsanskrit{Rāhulasutta} continues to develop the same theme. The Nigrodhakappasutta is a eulogy by \textsanskrit{Vaṅgīsa} for his departed teacher. The \textsanskrit{Sammāparibbājanīyasutta} describes the right way to wander for someone who has understood the teachings having reached independence. Finally, the Dhammikasutta finds the layman Dhammika longing to hear the “teaching”, and the Buddha explaining the paths of both the mendicant and the layperson who practice after hearing the teaching.

\subsection*{Gems}

In between two sets of benedictive verses, the Ratanasutta offers a subtle and insightful survey of the famed Triple Gem—the Buddha, the Dhamma, and the \textsanskrit{Saṅgha} (Snp 2.1). Here, as elsewhere in the Suttas, the Buddha, the Dhamma, and the \textsanskrit{Saṅgha} are not called “Gems”. Rather, the “gems” are good qualities that are found in them, much like the gemstones found in the great ocean (AN 8.19:8.1). But it seems likely that the idea of the “Triple Gem” evolved from this Sutta.

Each verse ends with a tagline. Usually, such taglines make a statement of fact: “this is a cause of downfall” or “this is the highest blessing”. Here, however, the tagline is a “blessing” in imperative form—“may there be well-being” or more colloquially, “may you be well”. This wish (\textit{sotthi hotu}) is elsewhere found in a more everyday sense, as a simple plea for safety (DN 3:1.23.11). The sense here, with the implication that the power of truthful words will have a powerful effect in themselves, finds its only early parallel in the \textsanskrit{Aṅgulimālasutta}. There, powerful words of truth wishing well for a mother in travail resulted in a successful birth and healthy mother and child (MN 86:15.8).

Such texts skirt a fine line. They don’t exactly say that the words of Dhamma exert a magical force, but they don’t exactly not say that either. It’s possible to read them as words of encouragement and uplift, whose effects are purely motivational. But that is surely not how generations of Buddhists have taken them.

The doctrinal content of the Ratanasutta leans heavily on concepts and ideas developed in the prose Suttas, and much of it would be incomprehensible on its own. A stream-enterer, for example, is said to “not take up an eighth life”, which draws on the notion that a stream-enterer, having uprooted certain fundamental fetters, is reborn no more than seven times.

A more subtle example is the “immersion with immediate fruit” spoken of in verse five, said to be the unequaled state of meditation. This is not a common technical term in the Suttas. The key term is “immediate” (\textit{\textsanskrit{ānantarika}}). This only occurs in one other Sutta, which is one of the latest in the canon, and there it is only said that such a state is hard to comprehend (DN 34:1.2.21).

\textit{Ānantarika} is closely aligned with a spectrum of more familiar terms—\textit{\textsanskrit{akālika}}, \textit{\textsanskrit{sandiṭṭhika}}, \textit{\textsanskrit{diṭṭheva} dhamme}—all of which convey the sense that the Dhamma is to be realized “in this very life”, as opposed to a religious practice that only bears fruit in a future life (\textit{\textsanskrit{samparāyika}}). Thus the “immersion of immediate result” is the practice of \textit{\textsanskrit{jhāna}} within the context of the eightfold path, which results in the realization of the noble fruits. Compare the case of the Buddha’s former teachers, whose path resulted in a state of rebirth in a refined and long-lasting heavenly realm, which was nonetheless still impermanent and not free from suffering. Both traditions practice deep \textit{\textsanskrit{samādhi}}, but only when informed by the right view of the four noble truths will that \textit{\textsanskrit{samādhi}} lead to awakening.

The benedictive verses at the beginning and end, the framing of each verse as a blessing, and the prominence of advanced doctrinal content suggest that the Ratanasutta must stem from a slightly later phase within the early texts. However, it can’t be very late, for the Ratanasutta has a close parallel in the \textsanskrit{Mahāvastu} of the \textsanskrit{Mahāsaṅghika} school. Given that the \textsanskrit{Mahāsaṅghika} and the \textsanskrit{Theravāda} split in the first schism, perhaps two centuries after the Buddha, it is likely that the composition predates this.

\subsection*{Putrescence}

The Āmagandhasutta (Snp 2.2) is one of the few early discourses to directly discuss vegetarianism. An unnamed interlocuter claims that pure folk only eat vegetables, not lying to get what they want. The list of foods appears to be purely wild vegetables and grains, so he is advocating a diet of vegan food gathered from nature. This was a traditional ascetic practice found among certain circles. We hear many \textsanskrit{Jātaka} stories of brahmin renunciates who retreated to the Himalayas to live on such a diet, which was regarded as a special mark of ascetic prowess (Ja 536). The speaker, it would seem, is familiar with such ascetics. He criticizes “Kassapa” for eating delicious food cooked by others. And he calls out Kassapa’s hypocrisy in claiming that “putrescence is not proper for me”, yet all the while he is eating rice with the flesh of fowl. This seems to be the basis on which he hints Kassapa has lied to get good food.

The word translated as “putrescence” is \textit{\textsanskrit{āmagandha}}, literally “raw smell”. It’s the smell of a rotting corpse and doesn’t have any intrinsic connection to meat eating \emph{per se}. Rather, it appears throughout the Suttas in the sense of “moral decay or corruption” (AN 3.128:2.2). According to the \textsanskrit{Mahāgovindasutta}, it was \textsanskrit{Brahmā} \textsanskrit{Sanaṅkumāra} himself who taught that the real meaning of “putrescence” was the defilements such as anger (DN 19:46.15). Thus Kassapa in the Āmagandhasutta, who is identified as a “kinsman of \textsanskrit{Brahmā}”, is advocating the same position as \textsanskrit{Brahmā} himself. But only the Āmagandhasutta connects the notions of “putrescence” and eating meat.

Before proceeding, let’s briefly review the position of the early texts on meat. It is accepted as normal throughout the Suttas and Vinaya that the Buddha and his community ate meat if they wished. Nonetheless, when the topic comes up it is usually hedged around with cautions and restrictions.

In the \textsanskrit{Jīvakasutta}, the layman \textsanskrit{Jīvaka} reports the rumor that the Buddha eats meat especially killed for him (MN 55). The Buddha denies doing so and repeats the well-known Vinaya allowance that meat or fish may be eaten by a mendicant unless it is seen, heard, or suspected to have been killed on purpose for the mendicant. Normally a mendicant will simply accept whatever is placed in the bowl, and since they wander at random through a village, no one is preparing food for them specially. The bulk of the Sutta shifts focus to the elevated practice of the four divine abidings and restraint by the mendicants and analyzes in detail the ethical boundaries that are crossed by someone who kills an animal to offer the mendicants. From the moment they order the beast to be fetched for killing, they make bad kamma. Thus the emphasis of the Sutta as a whole is not to justify meat eating but to establish the strict criteria under which meat eating is permitted, and to graphically illustrate the evil of killing for food.

In the Vinaya itself, the threefold allowance is not the only word on meat eating. A mendicant is forbidden from eating ten kinds of meat—human, elephant, horse, dog, snake, lion, tiger, leopard, bear, and hyena (Khandhaka 6:23.1.1 ff.). And given that it is not always obvious whether such meat is being served, it is an offense to eat any meat without first having checked that it is allowable (Khandhaka 6:23.9.9). It’s also an offense to accept raw meat (DN 1:1.10.9). Meat is regarded as one of the luxurious foods (DN 26:19.6), asking for which is forbidden (Bhikkhu \textsanskrit{Pācittiya} 39:2.10.1). Certain ascetics refused all meat (DN 25:8.5); these included Nanda Vaccha, Kisa \textsanskrit{Saṅkicca}, and Makkhali \textsanskrit{Gosāla} (MN 36:5.3). But when the Jains falsely accused the Buddha of eating meat slaughtered especially for him, the Buddha dismissed them (AN 8.12:31.4). It seems the early Jains, like the Buddhists, would accept meat that was not from an animal slaughtered on their behalf. He likewise rejected Devadatta’s proposal that all mendicants must be vegetarian, along with a range of other ascetic practices, as it too was merely a pretense to attack the Buddha. Instead, he left the decision to individual mendicants (Khandhaka 17:3.14.13). Other ascetics went to the opposite extreme, consuming only meat and alcohol, apparently blazing the trail for certain delusional teachers today (DN 24:1.11.4). While the position of brahmins on meat is ambiguous, there is at least one passage where a brahmin laments his meat-eating as being unrighteous (Sekhiya 69:1.32).

Thus the general position of the Suttas on meat-eating for mendicants is to restrict its usage and leave the final decision up to the individual. This is not an endorsement or encouragement, but a way of managing a delicate situation. As for the lay folk, there are no special pronouncements as such. But it is considered wrong livelihood to trade in animals or meat (AN 5.177). And of course, it is wrong to kill or harm an animal under any circumstances, or to have one killed. Again, there is restriction without prohibition.

In this discussion, I have left aside the identity of “Kassapa”. It is a common name for brahmins. The Sutta is concluded with two verses which, according to the commentary, were added by the redactors at the Council. It’s an unusual ending, which suggests that a pre-existing set of verses was adapted for the context. Now, in these closing verses “Kassapa” is identified as a Buddha, who must be the legendary Buddha of the past of that name. Such a legendary attribution cannot be taken literally, but rather is another sign of an imported text. Further, the main series of verses, which end with the tagline “this is putrescence, not eating meat”, deal only with general ethical matters and don’t have any distinctively Buddhist teachings. The last two verses of the teaching (before the closing verses) lack this ending tag. They say nothing of “putrescence”, but contain a Buddhist criticism of Brahmanical ideas of ritual purification.

We saw above that the Suttas attribute the teaching on “putrescence” as moral decay to \textsanskrit{Brahmā}. And the relevant portions of the Āmagandhasutta are purely a discussion between two brahmins, with the Buddhist content added later. It seems to me, then, that the main teaching here is a Brahmanical dialogue that was adopted as a Buddhist text by adding some framing verses and a background story. Perhaps this explains why, while the position on meat-eating here does not technically contradict that found in the rest of the canon, it is more pushy, lacking the tendency towards constraints that we find elsewhere.

The puzzling thing in all this is the question of supply and demand. Economic theory tells us that demand for a certain good drives its production. If people buy more meat, suppliers will kill more animals. But the Suttas don’t think in terms of a generalized concept of economic demand for meat, only a personal and individual one. To us, this seems like such a natural and obvious argument, yet even the Buddha’s critics don’t make it. Why is that?

In pre-industrial societies, the supply of meat is relatively fixed. Chickens run around the yard, cows graze in a field, fish swim in a stream. A certain number of animals are killed for their flesh, but well-functioning societies do not over-cull and deplete their supply. But too many animals are also a problem, as they eat the crops. Typically, larger animals are killed for celebrations or special days, or when the supply becomes excessive and must be burned off. This is ritualized in the form of the sacrifice. In addition, there are no simple means to store large quantities of meat long-term, so animals are usually slaughtered and eaten right away.

Thus the early texts employ the concept of “available meat” (\textit{\textsanskrit{pavattamaṁsa}}): it was either there or it wasn’t. The animals would be killed regardless, which is why it only becomes an ethical issue for the mendicants when animals are killed on purpose for them. There’s a good illustration of this in the Vinaya (Khandhaka 6:23.2.8). The laywoman \textsanskrit{Suppiyā} tried to order some meat, but there was none for sale in the city of Benares since no slaughtering had been done that day. No meat in the entire city: unthinkable to us, but normal to them.

There was no concept of increasing the supply of meat to cater to demand because the material means of increasing supply simply did not exist. It is only in recent centuries, with the development of scientific and industrial techniques in animal husbandry, that we have learned to expand the supply of meat at will. Producing more meat takes energy, and that energy is supplied by the industrial application of fossil fuels.

Arguably, then, the concept of “available meat” no longer applies, except in limited cases such as roadkill. All meat is produced on demand, even if that demand is indirect. When monastics accept meat, this acts as an implicit endorsement for the lay community. It is not uncommon in Theravada communities that even those who were formerly vegetarian start eating meat, which contributes to expanding demand for meat in society as a whole.

This demand drives the machinery of death in the slaughterhouses, the grotesque horrors of the factory farms, the grim scouring of the oceans, and the ecocidal madness of climate collapse. Those who eat meat belong to the most privileged generation that has ever lived on this planet, blessed with an astonishing quantity and variety of delicious foods, yet they choose to demand the flesh of living creatures. It is certainly possible to argue that this is “allowed” in the ancient texts. But the economic, material, and social context has completely changed. Is this the best we can do—to take what we want because it is “allowed”?

\subsection*{Conscience}

The Hirisutta is named after its first word, but the theme is, rather, good friendship (Snp 2.3). Even though the poem has only five verses, it appears at first to be somewhat disjointed, as the final two verses seem disconnected from the first three. Perhaps it is simply a somewhat clumsy compilation. But I suspect the whole is more coherent than it appears.

The poem begins with a description of a bad friend, who betrays you or looks for flaws. It especially emphasizes that a good friend acts for one’s benefit, not just talking. The third verse ends with a moving and intimate image: a true friend rests with you like a child lying on the breast.

It is the fourth verse that seems to break the pattern. The syntax is tricky, but I think the key is the word \textit{porisa}. One is said to carry the “\textit{porisa}” burden. This is explained in the commentary here as being “fit for a person (\textit{purisa})”. \textit{Porisa} is also used in the sense of a “servant”. I think the sense here is that one serves others by acting as a good friend to them. In doing such service—the virtues spoken of in the first three verses—one can expect worldly happiness and praise.

The final verse then brings in the twist: the benefits of even such virtuous and loving companionship pale in comparison to the blissful nectar of seclusion. It raises the virtue from the mundane to the transcendent.

\subsection*{Blessings}

The \textsanskrit{Maṅgalasutta} redefines the superstitious notion of an “auspicious sign” (\textit{\textsanskrit{maṅgala}}) in practical and ethical terms (\textsanskrit{Maṅgalasutta}, Snp 2.4). The prose background serves to authorize the new definition in the eyes of the gods, who you would think are the experts in such matters. \textit{\textsanskrit{Maṅgala}} shades in meaning from a prophetic augury (“auspice”) to a more general sense of “felicity” or “good fortune”.

People still seek to know the future by reading tea leaves or checking their horoscope. Such things are popular all over the world, not limited to one particular culture or religion. The Suttas mention countless such practices and dismiss them all. One who has seen the truth will not believe in such things (AN 6.93), but the Buddha was not harsh about this, as he did allow mendicants to step on a cloth as a \textit{\textsanskrit{maṅgala}} in concession to the superstitions of layfolk (Khandhaka 15:21.4.11), even though he would not do so himself (MN 85:7.16).

The “blessings” in the \textsanskrit{Maṅgalasutta} are things that are both good in and of themselves and which lead to a good outcome in the future. The sense of creating an auspicious future through auspicious deeds comes out more clearly in AN 3.155, where a lucky star (\textit{sunakkhatta}) and a good auspice (\textit{\textsanskrit{sumaṅgala}}) lead to happy results. In the \textsanskrit{Maṅgalasutta} this is implicit in the structure of the text, which moves from the simple building blocks of a good life to the sublime bliss of freedom.

It is in its recognition of the simple things that the \textsanskrit{Maṅgalasutta} distinguishes itself. Many Suttas eulogize the virtues of renunciation, but few speak of getting an education in a craft or skill, or which recognize the crucial importance of living in a region hospitable for practice.

The very first line of the Buddha’s response has a special significance for me. When I first read it, devouring all the Suttas as a newly-enthused meditator, it struck a discordant note. “Not to associate with fools”—isn’t that mean? And who gets to say who is a fool? Today, however, I think this is a crucial statement, and its placement at the start of a text that lays the foundation for a blessed life is no accident. If we become caught up in the words of fools—those who preach anger and division, who spin delusional conspiracies in the name of truth—then our minds become twisted, and any possibility of moral or spiritual progress is lost. All the more valuable are the compassionate and reasonable words of the Buddha.

\subsection*{With Spiky, the Native Spirit}

Unlike the \textit{yakkhas} we met previously, here Kharo (“Shaggy”) and \textsanskrit{Sūciloma} (“Spiky”) are named not after places, but after their rather unkempt and intimidating coverings (Snp 2.5). But as with \textsanskrit{Āḷavaka}, once the unpleasantries of the introduction were over, the questions are not merely civil, but positively sophisticated: “where do the mind’s thoughts originate, like a crow let loose by boys?”

The Buddha’s response is insightful and challenging. But at the end, there is no word as to the \textit{yakkhas}’ response, unlike the spirits of place who committed themselves to spread the Dhamma in their region. The prose introduction, however, allows for a different kind of symbolic reading.

In the Hemavatasutta, the \textit{yakkhas} are discussing the observance of a sabbath day, and propose to see the Buddha. So this discourse can be read as a conversion narrative, where the Buddhist observance displaces an earlier animist observance in the mountainous regions. In the \textsanskrit{Āḷavakasutta}, it is the Buddha who is intruding into \textsanskrit{Āḷavaka}’s home at \textsanskrit{Āḷavī}, just as Buddhism came into places that already had animist forms of worship. While there was initial discomfort, soon enough \textsanskrit{Āḷavaka} was happy with the intrusion.

The \textsanskrit{Sūcilomasutta}, on the other hand, is set in the very heartland of Buddhism, at \textsanskrit{Gayā}, close by the seat of the Buddha’s enlightenment. The Buddha is still sitting in the haunt of the \textit{yakkha}—because even there, animism predates the Buddha—but it is the \textit{yakkhas} who pass close by the Buddha. They press up against him, just as the ancient animist practices pressed up against Buddhism even in its heartland; it was a prickly situation. The Buddhists saw the \textit{yakkhas}—and by extension, their worshippers—as coarse, verging on violent, and saw themselves as a civilizing and uplifting influence. They didn’t dismiss others because of their different practices but saw in them a potential for growth.

\subsection*{A Righteous Life}

Like the Hirisutta, the Dhammacariyasutta is named after the first word. But it’s a bit misleading, because the bulk of the Sutta is about unrighteous conduct, and only the beginning and the end return to good conduct. The commentarial title Kapilasutta is spurious, as there is no mention of a Kapila in the text.

The mission statement of this text is in the first two verses: if someone goes forth as a mendicant but is of bad character and conduct, things only get worse for them. Monastic life is no magical cure for depravity. And indeed, for one who presents themselves to the world as a paragon of virtue, the consequences of bad behavior are even worse.

The poem denounces those of bad conduct and urges the \textsanskrit{Saṅgha} to expel wicked mendicants, as the Buddha did also at AN 8.10 and AN 8.20. There are several legal mechanisms in the Vinaya for such matters.

One interesting detail worth noting: the Buddha does not do the expulsion himself. He built the \textsanskrit{Saṅgha} to take responsibility for its actions. And just as it is up to the \textsanskrit{Saṅgha} to decide who to accept in their ranks, it is up to them to decide who to expel.

\subsection*{Brahmanical Traditions}

Asked by some senior brahmins about the ancient Brahmanical ways, the Buddha laments that brahmins have fallen far from their pure ancient traditions (Snp 2.7). The nadir was the ritual slaughter of cows, a vile practice that was specifically called for by the brahmins themselves against the outraged cries of the gods.

The notion that contemporary brahmins had fallen into corruption is found elsewhere in the Suttas. The \textsanskrit{Soṇasutta} speaks of five ancient Brahmanical practices that today are only followed by dogs, not by brahmins (AN 5.191); all five are also in the \textsanskrit{Brāhmaṇadhammikasutta}. If this feels too harsh, the \textsanskrit{Chāndogya} \textsanskrit{Upaniṣad} includes the famous “prayer of the dogs”, where brahmins are compared to dogs begging for food (ChU 1.12).

The Buddha depicts ancient brahmins as having been renunciant seers devoted to austerity and meditation. This generally accords with the description in the \textsanskrit{Aggaññasutta} (DN 27:22.6), although not every detail is the same. There, they are said to have gone to the village for alms, whereas here food was left outside their doors. There, taking up recitation was felt to be a decline from meditation, whereas here their devotion to chanting is praised. Nonetheless, the overall tenor is that the brahmins lived then much like the bhikkhus do today.

The text is full of fascinating details, and while the overarching message is clear, it is not easy to know how to read it. What exactly is the source of the Buddha’s information? Did he use his psychic powers? The text says nothing of this, and, as a rule of thumb, it’s better to prefer the lesser miracle.

Might the depiction be simply a legend or parable, a just-so story invented by the Buddhists to show their superiority to the brahmins? The lives and values of the ancient brahmins are markedly similar to the mendicants who followed the Buddha, and the text certainly serves to exalt that way of life. The oldest Brahmanical text, the Rig Veda, is not a renunciate text; it glorifies the gods for the sake of worldly prosperity. We know that an ascetic tradition emerged within Brahmanism, but it’s a stretch to argue that this was the normal and accepted practice among ancient brahmins. There is a world of difference, however, between selecting details to tell a story, and simply making things up.

Does the text perhaps reflect things that the Buddha had seen and heard around him, things said by brahmins, or about them? Possibly, although if it were common knowledge, you’d expect that the brahmins would not have to ask the Buddha about it.

There is at least one detail that suggests the text is drawing on specific Brahmanical sources. The text refers to a period of 48 years of the “spiritual life” (\textit{brahmacariya}). Some manuscripts add \textit{\textsanskrit{komāra}} (“boy”) here; but this is hypermetrical, and must have been copied over from AN 5.192:5.4, which says this was a period for learning the “hymns”, i.e. the Vedas. This refers to an apprenticeship under a teacher from the time of childhood to master the Vedic texts. The Suttas speak of boys as young as sixteen as having achieved such mastery.

The practice of \textit{brahmacariya} is not common in pre-Buddhist texts; it’s referred to only once in a late Rig Vedic verse (Rig Veda 10,109.05), and occasionally in the Chandogya \textsanskrit{Upaniṣad} (ChUp 2.23.1), where it appears as one of the ways to attain a heavenly rebirth.

A more promising connection is found in the old Brahmanical law books. Like all ancient Indian texts, the date of these is uncertain, but they are not too remote from the Pali canon. In the \textsanskrit{Baudhāyanadharmasūtra} we find the same concept of \textit{brahmacariya} that is described in the \textsanskrit{Brāhmaṇadhammikasutta}.

\begin{quotation}%
\textit{\textsanskrit{aṣṭācatvāriṃśad} \textsanskrit{varṣāṇi} \textsanskrit{paurāṇaṃ} vedabrahmacaryam}\\

For the ancients, the studentship for learning the Vedas lasted forty-eight years.

%
\end{quotation}

Likewise, the \textsanskrit{Āpastambadharmasūtra} attributes to “some” (\textit{eke}) the same period (Ap1.11.30.2: \textit{\textsanskrit{tathā} vratena-\textsanskrit{aṣṭācatvāriṃśat} \textsanskrit{parīmāṇena}}). The similarity is too pronounced to be a coincidence: both Buddhist and Brahmanical sources speak of a period of forty-eight years for studying Vedic mantras, and this is ascribed in both cases to the ancient brahmins. At least in some details, then, the \textsanskrit{Brāhmaṇadhammikasutta} must be drawing upon genuine Brahmanical traditions.

I wondered whether these texts would reflect other aspects of the \textsanskrit{Brāhmaṇadhammikasutta}. As I’m no student of the Brahmanical texts, I’m not in a position to assess all the different traditions of Brahmanism, but perhaps a simple survey restricted to the \textsanskrit{Baudhāyanadharmasūtra} might help answer our question. Let us go through the first portion of the text and see how the ideals of brahmin as described in the \textsanskrit{Brāhmaṇadhammikasutta} and \textsanskrit{Baudhāyanadharmasūtra} compare.

A brahmin should be restrained \footnote{Following the suggestion in PTS dict under \textit{niketa} that it is related to \textit{ketu}. } and austere. \footnote{While this whole sutta flirts with the idea of self-mortification, the commentary, followed by Bodhi, takes it to the next level here, identifying \textit{\textsanskrit{vedanā}} as extreme suffering. No doubt in the previous verse such suffering might apply, but we have just been told that the faculties, including \textit{\textsanskrit{samādhi}} have grown strong. The next line indicates freedom from sensuality. All this agrees with the normal Sutta description of the feeling in \textsanskrit{jhāna}. } They owned no cattle, \footnote{The meaning of these lines is quite obscure. The \textit{\textsanskrit{muñja}} was used for the brahmin’s girdle, and as such became an epithet of Vishnu and Shiva. The sense of “girdle” in English conveys the idea of preparing oneself for a righteous challenge, “gird thy loins”. I take \textit{parihare} as optative. } for their true wealth was in recitation, which was a gift from \textsanskrit{Brahmā}. \footnote{\textit{Dhiratthu} is a curse, lit. “damn my life”. But it's hard to render as a literal curse without sounding like it’s weird, swearing, or casting a spell. } They lived a mendicant lifestyle, \footnote{And unfired pot also used as simile for weakness at MN 122:27.1. } accepting food freely donated by others. \footnote{Norman and Bodhi both follow comm. here without comment: \textit{manussehi}. But \textit{\textsanskrit{māṇava}} is not used elsewhere in this sense (PTS Dict’s ref to Thig 7.1 notwithstanding), nor does such a sense seem to be attested in the Sanskrit dictionaries. The context is the Buddha’s unattachment to “clan” and it surely has the sense here of “youngling”, either children or students. } Despite their asceticism, and in a quite un-bhikkhu-like touch, they wore multi-colored robes. \footnote{It’s a translation: we should use the normal word used these days. }

When married, a brahmin only approached his wife for sex during the fertile fortnight of her period. \footnote{Norman and Bodhi both have “approach and ask”, but \textit{\textsanskrit{upasaṅkamma}} is absolutive: he is already there. } For the Brahmanical text, it is an equal offense to have sex outside the fertile period, and also to \emph{not} have sex in the fertile period. The Buddha adopts the former detail, but not the latter: Buddhism is not a fertility religion. It’s also noteworthy that for the \textsanskrit{Brāhmaṇadhammikasutta}, marriage is purely a love match entered with mutual consent; I have not been able to identify this concept in the Brahmanical text (Snp 2.7:9).

The central value of the \textsanskrit{Brāhmaṇadhammikasutta}, the one whose loss precipitated the fall of the brahmins, was harmlessness. \footnote{Bodhi translates \textit{\textsanskrit{ārādhaye}} as simple present tense here (“succeeds”) and in the identical line below as imperative (“should accomplish”). Norman has the optative mood in both cases, which is surely correct. } And harmlessness is constantly listed as a central virtue in the \textsanskrit{Baudhāyanadharmasūtra}. The Self is purified by harmlessness to living creatures as the mind is purified by truth. \footnote{The “speaker’s mark” is in MS treated as part of the line, which is unusual if not unique. According to Norman’s note here however, we should consider this line a seven-syllable sloka (perhaps restoring the variant \textit{ca} to make it eight). I have adjusted the Pali punctuation accordingly. } The brahmin has vows of harmlessness, truth, not stealing, sexual restraint, and generosity. \footnote{\textit{Bajjhati} (“bound”) seems odd to me in this line. I think it’s more likely these are three semi-synonyms. As it stands, the Buddha does not actually answer the question, since he only discusses the positive side. PTS notes a reading \textit{bujjhati} in the commentary, which I follow. However the commentary reads \textit{bajjhati} in the VRI edition, and the comment itself supports this, since it treats it as a bad thing. If it is a mistake, then, it is an old one. } “Austerity” (\textit{tapas}) is defined as beginning with harmlessness. \footnote{Comm allows \textit{kappa} in the sense of both “mental activity” and “aeon” here. For the sake of consistency with later uses of the term, it might seem prudent to use “mental activity” here, and most translators have done so. Yet the context in this verse suggests it is about the aeons of transmigration. } An ascetic forest dweller is urged to not harm even gadflies or gnats. \footnote{I think Norman’s reconstruction of this to root \textit{\textsanskrit{ñā}} (followed by Bodhi) is too speculative. The word makes fine sense as is, and is more metaphorically connected with the sense of \textit{khetta}. }

It seems clear then that the \textsanskrit{Brāhmaṇadhammikasutta} draws upon genuine Brahmanical traditions, and by and large presents them accurately. Occasionally, it will adjust details in a more Buddhist light.

If one reads the \textsanskrit{Brāhmaṇadhammikasutta} with its commentary, however, there are a couple of details that are a little puzzling, as they appear to endorse the caste doctrine. If this were the case, it would suggest that the text had been rather carelessly put together from Brahmanical sources, in contrast to the careful selection of details we have seen above.

According to the commentary, the ancient brahmins are praised for only marrying within their caste (Snp 2.7:8.1), and a cause of their decline was the rejection of the doctrine of caste (Snp 2.7:33.4). In his summary introducing his translation of this sutta, Bhikkhu Bodhi remarks that “There seems to be here an incongruity between this line of the text and the view expressed elsewhere that caste distinctions are purely conventional.” However, a closer reading shows that neither of these phrases requires reading in this sense.

Snp 2.7:8.1 says \textit{Na \textsanskrit{brāhmaṇā} \textsanskrit{aññamagamuṁ}}, literally “brahmins did not go (for sex) to another”. This is clarified in Baudh 2.1.2.13, which says certain women are “not to be approached” (\textit{\textsanskrit{agamyā}-\textsanskrit{gamanaṁ}}). This includes women who are to be shunned (\textit{\textsanskrit{apapātrāṁ}}, “those who eat from separate bowls”). This is a stronger term than merely “of a different caste”, and refers to those who have committed a transgression serious enough for them to be set apart. Indeed, the \textsanskrit{Baudhāyanadharmasūtra} praises a brahmin who honors all four castes with offerings of food in his home (Baudh 2.3.5.11). Similar is the \textit{\textsanskrit{patitā}}, a woman who has “fallen” from caste. Notions of caste are involved here, but it is not as simple as saying that a brahmin can never have sex outside of caste; the text is more specific and restricted. Compare what it says about having children, where the requirement to remain within one’s own caste is clear and emphatic. \footnote{Reading \textit{vijeyya}; both \textit{viceyya} and \textit{vijeyya} are accepted in the commentary. }

But caste is not the only criterion. The passage also forbids a brahmin from relations with a female friend of a teacher. Interestingly, the text refers without comment to both male and female teachers (\textit{\textsanskrit{gurvī}}). Thus the basic idea is about setting boundaries around appropriate relations. The Buddhist text speaks of the same idea, but pointedly does not frame it in terms of caste.

And as to the second problematic passage, \textit{\textsanskrit{jātivāda}} in the Suttas doesn’t quite mean “doctrine of caste”, which a literal rendering would suggest. Rather, it refers to ancestral knowledge. Consider a stock passage such as AN 5.19:2.9:

\begin{quotation}%
For I am well born on both my mother’s and father’s side, of pure descent, irrefutable and impeccable in \emph{questions of ancestry} back to the seventh paternal generation.

%
\end{quotation}

It can’t mean that “everyone in my ancestral lineage held the correct doctrine of caste”. It means the ancestral knowledge that has been passed down through the generations. This would of course include information about one’s caste and parentage. In Snp 3.1:19.2, the Buddha identifies his \textit{\textsanskrit{jāti}} as “Sakyan”, which is a people rather than a caste. The burden of the the \textsanskrit{Brāhmaṇadhammikasutta}, in fact, is to show that \textit{\textsanskrit{jāti}} is not just an assertion of caste and its privileges, but carries with it a set of rules and principles mandating a standard of moral conduct.

The \textsanskrit{Brāhmaṇadhammikasutta} is not saying that the brahmins became corrupt because they rejected the doctrine of caste; rather, it is because they neglected their ancestral heritage. This reinforces the central message of the Sutta, and it doesn’t require an odd apparent endorsement of caste.

In sum, there seems no doubt that the \textsanskrit{Brāhmaṇadhammikasutta} is drawing upon genuine Brahmanical traditions, which have much in common with the \textsanskrit{Baudhāyanadharmasūtra}. The \textsanskrit{Baudhāyanadharmasūtra} itself claims to draw on the ancient traditions of, in order, the Vedas, the ancillary literature, and the traditions of disciples (Baudh 1.1.1.1–4). Thus both texts position themselves as descriptions of the noble practices of the brahmins of old.

Law books of any kind don’t depict what people actually do, but what the lawmakers think they should do. The \textsanskrit{Brāhmaṇadhammikasutta} carefully selects details that accord with the story it wants to tell, rather than presenting the traditions as a whole. But it does not merely create an idealized brahmin by projecting the ideal of a mendicant into the past, for it includes many details that do not apply to bhikkhus, such as colored robes and family life.

Now, at some point, these ancient paragons of Brahmanical virtue arranged for harmless sacrifices. The Brahmanical tradition includes many sacrifices that do not involve killing, such as the \textit{agnihotra}, where ghee is offered to the fire. This tradition harks back to ancient Indo-Aryan rites, and related rituals are maintained among inheritors of the ancient Indo-Aryan religions from Iran to Lithuania. It may well stem from much earlier, and be related to the taming of fire itself. So there is support for the idea that ancient Brahmins practiced harmless rites.

This doesn’t mean, however, that they did not also practice the harmful rituals of animal sacrifice. Such practices also hark from human antiquity and are deeply embedded in Vedic beliefs and texts. The \textsanskrit{Brāhmaṇadhammikasutta}, however, insists that there was no sacrifice of cows, speaking tenderly of the loving debt we owe them.

The ancient Indo-Aryans were pastoralists, so the cow was central to their lifestyle. The Vedas treat cows with respect, calling them \textit{aghnya} “not to be killed”. In the Rig Veda, we find verses of gratitude and sympathy with the cow that are not at all dissimilar to the sentiments of the \textsanskrit{Brāhmaṇadhammikasutta}. \footnote{The commentary has led translators astray here. \textit{\textsanskrit{Paṇḍarāni}} means not “senses” per commentary (followed without remark by Norman); nor “translucencies” per Bodhi; nor “white flowers” per Thanissaro. In the EBTs, \textit{\textsanskrit{paṇḍara}} simply means “white, pure” and is a synonym for \textit{suddhi} in the next line. A true \textit{\textsanskrit{paṇḍita}} is pure both inside and out. }

\begin{quotation}%
Cow, may you be rich in milk through abundant fodder; that we also may be rich (in abundance); eat grass at all seasons, and, roaming (at will), drink pure water.

%
\end{quotation}

But this does not tell the whole picture. In the Rig Veda, Indra devours the offering of oxen (\textit{\textsanskrit{ukṣan}}), even fifteen or twenty, filling himself on their fat. \footnote{Using ‘wise scholar’ rather than my normal ‘astute’ to capture, albeit lamely, a pun with \textit{\textsanskrit{paṇḍara}}. } He eats bulls (\textit{\textsanskrit{vṛṣa}̱bha}) cooked for him, \footnote{Commentarial “abode of energy” for \textit{\textsanskrit{viriyavāso}} doesn’t sound right to me, though followed by Bodhi and Norman. I read \textit{\textsanskrit{viriyavā} so}. } and buffaloes (\textit{\textsanskrit{mahiṣa}}) in their hundreds. \footnote{Reading \textit{\textsanskrit{vīro}}. } Agni is called “one whose food is the ox and the cow (\textit{\textsanskrit{vaśā}})”. \footnote{Earlier translations misconstrued this verse. The gerunds \textit{\textsanskrit{sutvā}} and \textit{\textsanskrit{abhiññāya}} apply to their respective following accusatives \textit{\textsanskrit{sabbadhammaṁ}} and \textit{\textsanskrit{sāvajjānavajjaṁ}}, not both to \textit{\textsanskrit{sabbadhammaṁ}} per Norman and Bodhi. \textit{Dhamma} means “teachings” here, not “phenomena” per Bodhi. The sense is that to be a “scholar” it is not enough to have “learned” the teachings, one must know for oneself. The commentary well explains this sense, but it doesn’t come through in the translations. Thanissaro attempts this sense, but only by adding an unsupported reading of \textit{\textsanskrit{abhiññāya}} as dative. } A cow was to be slaughtered for the wedding. \footnote{The three perceptions are sensuality, malice, and cruelty. Since \textit{\textsanskrit{paṅka}} is regularly said to be a term for \textit{\textsanskrit{kāma}}, it seems we should read “the bog of the three perceptions” rather than “the three perceptions and the bog” per Bodhi and Thanissaro. } That the sacrifice and eating of cows did not fall out of favor after the Vedic period is shown by such ritual texts as the \textsanskrit{Taittirīya} \textsanskrit{Brāhmaṇa}, which emphasizes that the cow (\textit{go}) was a sacrifice and was food, \footnote{It seems to me that the syntax reads more cleanly if the nominatives in this line and the next are read together; and the clauses following \textit{sabb-} are likewise read in parallel. } the Aitareya \textsanskrit{Brāhmaṇa}, which prescribes the killing of a cow to feed a guest, \footnote{Following Norman and Bodhi in reading \textit{\textsanskrit{parivajjayitā}} here. } and the \textsanskrit{Bṛhadāraṇyaka} \textsanskrit{Upaniṣad}, which recommends eating beef (\textit{\textsanskrit{rṣabha}}) to help procure a son. \footnote{I think Bodhi, Norman, and Thanissaro have missed two related points. First, \textit{pariyanta} means “limit, boundary” not “end”; and second, \textit{\textsanskrit{nāmarūpa}} is not ended in this life but rather “limited”. An arahant still has name \& form but only until \textit{\textsanskrit{parinibbāna}}; it is “limited”. This further highlights the rather awkward inclusion of \textit{\textsanskrit{nāmarūpa}} in the list of defilements, which are ended. I think we should, rather, include the defilements with the “deeds causing suffering”. This has the further advantage of dividing this long verse into 4 + 2 lines rather than 3 + 3. } All this agrees with the Pali canon, where a “cow-butcher” (\textit{\textsanskrit{goghātaka}}) is a common enough livelihood to have regular apprentices (\textit{\textsanskrit{goghātakantevāsī}}).

That the historical picture is complex should come as no surprise. The Brahmanical tradition contains multitudes of texts produced over vast extents of space and time. Regardless, the \textsanskrit{Brāhmaṇadhammikasutta} is emphatic that ancient Brahmanical customs required purely harmless rites. Gradually, however, the prosperity of the Brahmins caused corruption to seep into them, jealous of the finery of the kings and their ladies.

Among the glories of the kings is their planned urban environment: their homes were neatly laid out in rows (Snp 2.7:18.4). This is a stock phrase that in the Suttas describes Hell (MN 129:16.4, MN 130:16.4, AN 3.36:16.2), and elsewhere an ideal vision of a city revealed in a magic gem (Ja 541:101.4)—an ambiguity that might seem strange unless you’ve lived in the suburbs. But there is one passage where this describes an actual city: the \textsanskrit{Mahājanaka} \textsanskrit{Jātaka} (Ja 539:25.2).

This tells of the Great Janaka, the wise and wealthy king of Mithila, the capital of Videha. He returned from exile to win the hand of the princess, his cousin \textsanskrit{Sīvali}, and the crown of the glorious city. Eventually, however, he saw through the danger of such pleasures and became determined to go forth. He saw even his own home as being like one of the hells—thus showing that the description of hell as being like an ordered city is no accident. Donning the ochre robe and bowl and taking up a mendicant’s duty, he left, aspiring for \textit{\textsanskrit{jhāna}}, and in the Himalayas met a Brahmanical ascetic named \textsanskrit{Nārada} of the Kassapa clan.

The point here is that Mithila is an actual city, and “Janaka” an actual king, or more accurately, a kingly lineage. A historical Janaka of Mithila was the royal sponsor of \textsanskrit{Yājñavalkya}. For his skill in debate, he presented \textsanskrit{Yājñavalkya} with vast wealth—herds of cattle with gold-tipped horns—even though \textsanskrit{Yājñavalkya} claimed to be a renunciant. \textsanskrit{Yājñavalkya} is the putative author of many Brahmanical scriptures, and the corrupt brahmins are said to have compiled scriptures.

As to what scriptures were compiled or composed, they cannot be the Vedas, as the Suttas, like the brahmins themselves, regarded the Vedas as descending from an immemorial past. So they must be the post-Vedic scriptures, most likely the \textsanskrit{Brahmaṇas}, which are a class of texts that, among other things, detail the sacrifices and the fees payable to brahmins therefrom. The oldest of these is the Śatapatha \textsanskrit{Brāhmaṇa}, of which the final portion is the \textsanskrit{Bṛhadāraṇyaka} \textsanskrit{Upaniṣad} and the whole of which is attributed to \textsanskrit{Yājñavalkya}. It prescribes fees for rites such as human sacrifice, \footnote{I think \textit{pattipatta} is a play on words here. } horse sacrifice, \footnote{Oddly, at AN 5.38:5.4 we find \textit{sorata} with \textit{sakhila} in a similar sense to \textit{akhila} here. } and universal sacrifice. \footnote{The commentary says these were two groups of wise devas. But it seems it refers to the legendary sages \textsanskrit{Nārada} and his nephew Parvata, whose friendship, estrangement, and reconciliation is told in the \textsanskrit{Viṣṇu} \textsanskrit{Purāṇa}. \textsanskrit{Nārada} is known for his learning and wisdom, which supports the commentary to an extent. Why these two are invoked here is a mystery; presumably they were important to Sabhiya. They occur in a similar context, but with many more gods, in Ja 547. } Fees would include women, \footnote{I can’t find any translation or note that captures this line properly. A \textit{nimitta} is a sign or portent, and is commonly used in the context of omens, prophecy, or fortune-telling. Here it means that there is no portent or omen by which one can predict the extent of life. } gold, \footnote{Accepting \textit{\textsanskrit{niccaṁ}} per Norman and Bodhi. } silver, \footnote{Commentary says this means, “I cannot bring back the dead” followed by Norman, Bodhi, and Thanissaro. But I think that reading is suspect. I think the text elides a second negative and should read: \textit{na eso \textsanskrit{alabbhā}}, literally “This is not not-to-be-gotten by me”, i.e. “I cannot escape this”. } cattle, \footnote{I can’t find any evidence to support the commentary’s gloss as “liar”. I assume it is shorthand for \textit{\textsanskrit{vebhūtiya}} which is a synonym of \textit{\textsanskrit{pesuṇiya}} (DN 30:2.21.2 and DN 28:11.2). } oxen and furniture. \footnote{I can’t find any explanation for the shift to direct address in second person here. Perhaps these verses were inserted from a separate source. } The brahmins are depicted as sharing the wealth of the sacrifice that they gain. \footnote{Norman and Bodhi both accept commentary’s \textit{kilesarajaṃ attani pakkhipasi}, but this seems too narrow; their deeds harmed others too. }

It is no great stretch to imagine that the \textsanskrit{Brāhmaṇadhammikasutta} is thinking of the kingdom of Mithila, and the corrupting influence of the lavish sponsorship of brahmins under the Janakas. Of course, it is not explicitly set in that period, but rather in the legendary days of King \textsanskrit{Okkāka} (Sanskrit: \textit{\textsanskrit{Ikṣvāku}}). He was the founder of the Solar lineage of kings and is claimed as a distant ancestor by the Buddha. He does not appear to be a historical figure, however, but rather a culture hero who introduced the cultivation of sugar cane (\textit{\textsanskrit{ikṣuḥ}}).

Eventually, the corruption reached such an extent that the brahmins demanded the ultimate sin: the slaughter of cows for sacrifice. At that, all the gods cried out with one voice: “This is a crime against nature!” (\textit{adhammo}). The intervention of the gods against the sacrifice is a common theme in religious evolution. Sacrifice lurks in the past, and a tale is told of how god intervenes to reduce or eliminate harm. Think of the gods led by Zeus revolted at being served human flesh by Tantalus, or how Jehovah stopped Abraham from killing his son Isaac, accepting a ram instead. In the Śatapatha \textsanskrit{Brāhmaṇa}, an anonymous voice intervenes to advise against human sacrifice. \footnote{Commentary says the hell-keepers are the subject here, hell-dwellers the subject of the next line. But this seems forced and unnecessary. Those going to hell are screaming, not uttering sweet words. }

Once the endorsement of the gods was severed due to this unnatural violence, humanity became sickly and divided, the social order fell into decay, and we ended up with the troubled world of today.

\subsection*{The Boat}

The \textsanskrit{Nāvasutta}, or as it is also known, the Dhammasutta (“Discourse on the Teaching”) is a short eulogy of the blessings of a good teacher (Snp 2.8). It shows the way out of the quandary posed by the two previous suttas on the corruptions of the spiritual life, through reliance on a good teacher. It is the first of a series of texts on the virtues of a good teacher as the foundation for spiritual progress.

The \textsanskrit{Nāvasutta} employs the well-known simile of the raft from MN 22:24. There, a man cobbles together a raft from grass and twigs and makes his way across the river safely, abandoning the raft on the far side. Here, one in a strong boat can bring across not only themselves but many others.

The image of a boat occurs elsewhere, such as the \textsanskrit{Kāmasutta} (Snp 4.1:6), but so far as I know this is the only place where the boat is envisaged as having room for more than one. In the “Questions of Dhotaka”, the Buddha denies that he can release anyone else, and instead says that the practitioner can themselves cross the flood by living mindfully (Snp 5.6:4). Probably it is no more than a metaphor slightly stretched, but the idea that one’s own efforts bring others across seems more at home in later forms of Buddhism.

The thrust of the text, however, is simply that it is good to use discernment when selecting a spiritual teacher. For as inspiring and helpful as a good teacher is, a bad teacher only passes on confusion and doubt.

\subsection*{What Morality?}

Named after the question in the opening line, the \textsanskrit{Kiṁsīlasutta} builds on the previous sutta; once a teacher is chosen, a student will honor them by applying their teaching in practice (Snp 2.9).

The Sutta begins with the proper attitude that a student should display to a teacher, respecting the teacher and treasuring the rare chance to learn Dhamma. The emphasis is on the subjective side of learning; without joy, eagerness, and humility, the path of learning is rocky indeed.

But learning is not for its own sake, for its essence is understanding; and the essence of understanding is the meditative immersion of the mind in \textit{\textsanskrit{samādhi}}.

\subsection*{Get Up!}

Again continuing the theme, the \textsanskrit{Uṭṭhānasutta} urges the meditator to remember that \textit{\textsanskrit{samādhi}} doesn’t just happen, it arises through diligent meditation (Snp 2.10). It sounds like the words of a slightly frustrated teacher as they shake a lazy student out of bed.

Many a well-meaning meditator has started with all good intentions, only to flag along the way. This is especially a trap in monastic life, where it is all too easy to slip into complacency. Hence teachings very much like this are a staple in meditation communities.

\subsection*{With \textsanskrit{Rāhula}}

The Buddha’s son \textsanskrit{Rāhula} appears several times in the Suttas, and in the \textsanskrit{Rāhulasutta} we find a brief but complete survey of the monastic training, beginning once more with the honor owed to a teacher (Snp 2.11).

The voice of the first two verses is not entirely clear from the text. The commentary explains that the first verse was asked by the Buddha to \textsanskrit{Rāhula} concerning \textsanskrit{Rāhula}’s regular teacher, Venerable \textsanskrit{Sāriputta}. The Buddha wants to know if his son has become over-familiar to the point of losing respect for his expert teacher (\textit{\textsanskrit{paṇḍita}}), but \textsanskrit{Rāhula} says he bears him nothing but respect. The teacher is said to “hold aloft the torch for all humanity” (\textit{\textsanskrit{ukkādhāro} \textsanskrit{manussānaṁ}}), which fits \textsanskrit{Sāriputta}’s role as the one who keeps the Wheel of Dhamma rolling (Snp 3.7:22.1).

The remainder of the Sutta is a concise summary of the monastic path otherwise known as the “gradual training”: the aspiration to end suffering; restraint and good friendship; moderation in material requisites; mindfulness and restraint; and meditation on the ugliness of the body; all culminating with abandoning conceit.

\subsection*{\textsanskrit{Vaṅgīsa} and his Mentor Nigrodhakappa}

The Nigrodhakappasutta is a series of elaborate verses by the monk \textsanskrit{Vaṅgīsa}, whose mentor, Nigrodhakappa, has recently passed (Snp 2.12). \textsanskrit{Vaṅgīsa} asks the Buddha whether Nigrodhakappa was fully enlightened. \textsanskrit{Vaṅgīsa} was an accomplished poet, whose skill in extemporaneous verse was praised by the Buddha (AN 1.212). The quality of \textit{\textsanskrit{paṭibhāna}} is regarded as an aspect of wisdom; it refers to the spontaneous arising of a wise and eloquent expression of the Dhamma.

Ancient Indian literary culture is confounding to modern sensibilities, accustomed as we are to think of “literature” and “writing” as going hand in hand. In ancient India, a supremely sophisticated literary culture—with advanced linguistics, philosophy, and poetics—was created and maintained entirely through the oral tradition. Throughout the Suttas, we see poetry spontaneously improvised. \textsanskrit{Vaṅgīsa}’s style shows that such improvisations could reach a high degree of embellishment and sophistication.

The verses are full of wordplay, with unusual forms and vocabulary that are a delightful challenge to convey in translation. Consider just a single line:

\begin{quotation}%
\textit{\textsanskrit{khippaṁ} \textsanskrit{giraṁ} eraya vaggu \textsanskrit{vagguṁ}}\\

Swiftly send forth your graceful voice

%
\end{quotation}

The line opens with the short bright sound \textit{khip}, chiming like a high-hat. \textit{\textsanskrit{Girā}} is a poetic term for a song or a saying. The guttural \textit{gi} takes the same place of enunciation as \textit{khip}, but the added voicing grounds it, like a chime then a drum. It is followed by \textit{eraya}, which is a unique form in the Pali canon: a stem imperative from the rare \textit{erayati}, to raise the voice. It leads on to the punctuated ending of \textit{vaggu \textsanskrit{vagguṁ}}. This is a reduplication of a word whose meaning is exactly how it sounds: sweet-sounding (cp. English “melody”). It can be read either as an emphatic reduplication “send forth your sweet sweet voice”, or with the first element as a vocative “sweet one, send forth your sweet voice”; I read it in the first sense.

Notice that the first pair of words and the last pair of words sonically echo each other. In each case, we have a pair of two-syllable words, which are consonant-heavy, with the emphasis on gutturals. The vowels are short, and are repeated in each pair: \textit{i} + \textit{a}, then \textit{a} + \textit{u}. So we have a pair of chiming words that bookend the line with solid rhythmic punctuation. They are joined together with the fluid \textit{eraya}, which entirely lacks guttural, or indeed any plosive consonants. It’s a musical line, its percussive chimes united with fluid melody.

\textsanskrit{Vaṅgīsa}’s sophistication is not just in sounds; he loves to play with meanings as well. Consider these enigmatic lines:

\begin{quotation}%
\textit{Na \textsanskrit{kāmakāro} hi \textsanskrit{puthujjanānaṁ},}\\
\textit{\textsanskrit{Saṅkheyyakāro} ca \textsanskrit{tathāgatānaṁ}.}

%
\end{quotation}

The suffix \textit{-\textsanskrit{kāra}} usually means “maker”, and \textit{\textsanskrit{kāmakāro}} appears in Cp 16:11.4 and Ja 524:26.3 as “wish-granter”, “fulfiller of desire”, while \textit{\textsanskrit{issariyakāmakārikā}} at Kv 23.3:1.1 means “the will of God”, and at Pe 6:73.2, \textit{\textsanskrit{akāmakāri}} is given as a meaning of “not-self”. The point here is that in the sphere of worldly desires, it is not possible to simply get what you want.

\textsanskrit{Vaṅgīsa} invents the term \textit{\textsanskrit{saṅkheyyakāra}} in contrast. \textit{\textsanskrit{Saṅkhā}} means “calculation, reckoning” and \textit{\textsanskrit{saṅkheyya}} means “calculable”. The Buddhas have a gift for explaining things in a way that makes their meaning clear. Unlike the illusory wish granter imagined by ordinary folk, this is real.

For all the poetic ingenuity on display, \textsanskrit{Vaṅgīsa}’s question is a simple one: was his teacher Nigrodhakappa enlightened? The Buddha’s response is almost comical in its brevity, underscoring the restraint and directness of his rhetoric compared to \textsanskrit{Vaṅgīsa}’s. Rarely in the Suttas do we see such a stark contrast in personal styles. He affirms \textsanskrit{Vaṅgīsa}’s hopes in just three lines, within which he makes room for a powerful simile of his own: Nigrodhakappa cut off craving like a “river of darkness”.

\subsection*{The Right Way to Wander}

In the \textsanskrit{Sammāparibbājanīyasutta}, an unnamed party asks the Buddha about the proper way for a mendicant to wander the world (Snp 2.13). This expands on the \textsanskrit{Rāhulasutta}, detailing the life of an ascetic. In contrast with the theme of the teacher, we see a return to the solitary wandering life extolled in the Rhinoceros Sutta and elsewhere.

Nonetheless, even though this sutta does not explicitly continue the theme of teachers, I believe it is meant to extend that theme. In the Vinaya, it is normally the case that a student must complete several years of training under a qualified teacher before they are ready to wander alone. The \textsanskrit{Sammāparibbājanīyasutta} cannot be expected to outline all this in detail, but it does specify that the wanderer has learned the teaching and understood it. The virtues and practices of the sutta will be familiar to anyone who has studied the Vinaya or the less formal guide to monastic conduct in the Gradual Training. It would seem, then, that the \textsanskrit{Sammāparibbājanīyasutta} extends the theme of the teacher, speaking of the time when both teacher and student have done their job and one has graduated from studentship.

In the first line, we see the Buddha rejecting \textit{\textsanskrit{maṅgalā}}, which here we might render as “superstitions”. Here, rather than being co-opted to a Buddhist meaning as in the \textsanskrit{Maṅgalasutta}, they are outright dismissed. This sets the tone of the Sutta, which is not at all conciliatory.

Each verse outlines a specific quality with which the mendicant wanders the world. There is no specific mention of meditation, or indeed, of any specific practices that a mendicant should pursue. Rather, the focus is on the inner qualities of freedom. Again, I think this characterizes teachings to qualified students. There is no need to go over the basics. What is needed is to set the overall intention and aims. The sutta sounds to me like a direct address by the Buddha to qualified students who wish to seek seclusion; a graduation speech if you will.

\subsection*{With Dhammika}

The Dhammikasutta presents us with something of a puzzle, for it is initiated by the layman Dhammika who, despite his prominent position as leader of a large retinue, is mentioned nowhere else (Snp 2.14). He asks a straightforward question: what is the practice both for mendicants as well as for lay folk? But he proceeds to dress his question up; not, like \textsanskrit{Vaṅgīsa}, with literary embellishment, but with praise for the Buddha, and reference to several previous occasions when he had persuaded difficult audiences.

He specifically references the dragon king \textsanskrit{Eravaṇa}, who is recorded as encountering the Buddha on only one occasion, told in the \textsanskrit{Mahāsamayasutta} (DN 20:11.7). The next verse mentions the Great King \textsanskrit{Vessavaṇa}; and while he is a staple of Buddhist cosmology, he is here identified with Kuvera, which must be a reference to the \textsanskrit{Āṭānāṭiyasutta} (DN 32:7.40). Thus these verses show familiarity with these Suttas, and by referencing them Dhammika assumes his audience is also familiar. But these are two of the latest Suttas in the \textsanskrit{Dīghanikāya}, showing an interest in magical protection that is rare in the canon.

Compounding the puzzle is the fact that when the Buddha begins, he addresses the mendicants rather than Dhammika. The layman Dhammika only exists as a single mention in the opening. Given that his name simply means “follower of Dhamma”, that he occurs nowhere else, that he is familiar with late Suttas, and that both question and answer frequently reference \textit{dhamma}, one might be forgiven for thinking that Dhammika is no more than a literary invention. Indeed, if we skip forward to the final verse, it urges a layperson to undertake a legitimate (\textit{dhammika}) business. Perhaps the Sutta was originally named after this, and the introduction was supplied later.

Nonetheless, the text is a handy source for comparing the ethical injunctions for lay and renunciate communities.

For renunciates, the basic precepts such as non-violence are skipped over—apparently they are assumed—and we begin with monastic deportment while on alms round in the village. Retreat, restraint in speech, and contentment with requisites round out the basic elements of monastic conduct, but there is no mention of meditation.

For layfolk, the emphasis is on the fundamental precepts beginning with not killing, with a special emphasis on not drinking.

There’s an odd disjunct, as after teaching the five precepts, the text once again introduces the basic precepts (Snp 2.14:26–7), this time as the eight precepts. If this seems like an interpolated passage, it is, as it is also found several times in the \textsanskrit{Aṅguttaranikāya}.

Following the Sabbath (\textit{uposatha}), the lay follower should make an offering to the Sangha. The specific practice of offering to the Sangha the day after the Sabbath is only mentioned here. The impression that this is a late verse is reinforced by the fact that some of the terms, such as \textit{\textsanskrit{pasannamānasa}}, are found only in later texts.

The last verse swerves again to mention the virtues of respect for parents and lawful business practices. As a result, the diligent layperson can expect to become one of the gods called “Self-luminous”.

All this is suggestive of a later text, perhaps compiled in the Ashokan era, with its focus on the basics of ethics and the goal of heaven. Nonetheless, the ideals and practices taught here, as well as the direct and pragmatic style, remain well within the scope of early Buddhism.

\section*{The Great Chapter}

The middle chapter of the \textsanskrit{Suttanipāta} opens with a distinct change of mood, from the fundamentals of ethics to the life of the Buddha. In contrast with the central position of the Buddha’s life in later teachings, the early Suttas contain relatively few accounts of the Buddha’s early life, making the few that exist of even greater interest. The \textsanskrit{Pabbajjāsutta} (Snp 3.1) and the \textsanskrit{Padhānasutta} that follows (Snp 3.2) are the root sources for several details that featured in later legends of the Buddha’s practice and awakening. Other Suttas in this chapter also touch on major events in the Buddha’s life—Devadatta’s follower \textsanskrit{Kokālika} (Snp 3.10), and the blessing of the newborn Siddhattha (Snp 3.12). Still others have a devotional flavor, while the \textsanskrit{Dvayatānupassanāsutta} offers a unique perspective on core doctrines such as dependent origination (Snp 3.12).

\subsection*{Going Forth}

The \textsanskrit{Pabbajjāsutta} recounts an episode evidently from shortly after Siddhattha had gone forth, when he met \textsanskrit{Bimbisāra}, the king of Magadha, for the first time. The king was deeply impressed by Siddhattha’s bearing and made an extravagant offer to him. Unusually, we have two other complete Indic versions of the \textsanskrit{Pabbajjāsutta}, one in the \textsanskrit{Mahāsanghika} \textsanskrit{Lokuttaravādin} \textsanskrit{Mahāvastu} in Hybrid Sanskrit, one in the \textsanskrit{Saṅghabhedavastu}, a chapter of the \textsanskrit{Mūlasarvāstāda} Vinaya, in Sanskrit.

I feel that these Suttas represent a lost potential, a way that was not chosen. Their style is direct, charming, and effective. They have a dramatic flair that accentuates the personal, which makes them affecting in a way that the more elaborate legends have lost sight of. In piling up miracles and smoothing out the rough, later legends buried the Buddha’s humanity, like a movie smothered with too many effects. They aim to impress and overwhelm but end up oppressive and tacky.

The bulk of the text is taken up with a narrative introduction in verse, the speaker of which is not specified in the text. Unusually, it is voiced in the first person (“I shall extol …”). The commentary identifies the speaker as Venerable Ānanda, and it is certainly the case that, in addition to his role in reciting all the Discourses, Ānanda was instrumental in creating the biography of the Buddha. It’s a striking and unique opening for a Sutta.

The first three verses, including this opening, are missing from the parallel in the \textsanskrit{Mahāvastu}, which otherwise appears quite early. However, they are found in the \textsanskrit{Saṅghabhedavastu}, which stems from a school (\textsanskrit{Mūlasarvāstivāda}) that spilt from the \textsanskrit{Theravāda} later than the split with the \textsanskrit{Mahāsaṅghika} \textsanskrit{Lokuttaravādins} who compiled the \textsanskrit{Mahāvastu}. Perhaps these verses were added after the first schism, which would place them after the time of Ashoka. This is very tentative, though, as the \textsanskrit{Mahāvastu} presents the verses as part of a narrative flow, and might have left the opening verses out to avoid introducing a new narrative voice.

The Pali commentary sets this encounter shortly after the Buddha’s renunciation, before his practice under the Brahmanical teachers \textsanskrit{Āḷāra} \textsanskrit{Kālāma} and Uddaka \textsanskrit{Rāmaputta}. In this, the \textsanskrit{Saṅghabhedavastu} agrees, but the \textsanskrit{Mahāvastu} places the event between these two apprenticeships.

Several details speak to the relative earliness of this text compared to the bulk of biographical legends.

\begin{itemize}%
\item In the Pali, the Buddha is referred to as \textit{buddha}, not \textit{bodhisatta}, even though he is still not enlightened, suggesting that the doctrinal distinction between \textit{buddha} and \textit{bodhisatta} was not yet established. However, note that this is found only in the Pali, and the two Sanskrit versions do not say \textit{buddha}.%
\item The Buddha is called \textit{bhikkhu}.%
\item The Buddha feels the need to explain to \textsanskrit{Bimbisāra} where the land of the Sakyans is.%
\item The Buddha acknowledged that his land was subject to Kosala and was not an independent kingdom.%
\item \textsanskrit{Bimbisāra} goes to see him in a single chariot, apparently alone; compare the elaborate visit by his son \textsanskrit{Ajātasattu} in DN 2. The expected retinue is added in the \textsanskrit{Mahāvastu}.%
\item The style is plain and draws on idioms found in the early texts.%
\item The text ends abruptly and is not concerned with \textsanskrit{Bimbisāra}’s response.%
\end{itemize}

A possible exception to the naturalism of the narrative is the mention that \textsanskrit{Bimbisāra} saw the Buddha’s “marks” (\textit{\textsanskrit{lakkhaṇa}}), which may refer to the putative Brahmanical legend of the 32 marks of the great man. But this is not specified, and most of the marks, if they were genuine physical characteristics, would not be visible on the street from a rooftop. More likely \textsanskrit{Bimbisāra} simply saw that he bore himself well, with a mindful and dignified carriage. This impression is reinforced in the Sanskrit versions, both of which depict \textsanskrit{Bimbisāra} himself as speaking of the marks, rather than saying that he saw them. The 32 marks of the great man are an obscure element of Brahmanical lore, and there is no reason why \textsanskrit{Bimbisāra} should know or care about them.

On the other hand, several indications suggest a degree of artifice in the account. \textsanskrit{Bimbisāra} himself, while firmly ensconced in Buddhist lore, is an enigmatic figure in the early texts. He is mentioned fairly often, but usually in his absence. He is more like an idealized Buddhist king than a human personality like his counterpart Pasenadi of Kosala, or even his son, \textsanskrit{Ajātasattu}. Given that \textsanskrit{Bimbisāra}’s kingdom of Magadha ultimately triumphed and, within a century, dominated the entire region, it would make sense for the Buddhists to treat him as their own. We know that the Buddhist King Ashoka, the most magnificent heir to the Magadhan throne, was ecumenical in his own words and deeds, yet came to be depicted in Buddhist texts as highly partisan. It’s possible that with \textsanskrit{Bimbisāra}, too, his Buddhist sympathies were exaggerated in later legends. It should come as no surprise that the Jains, also, claim \textsanskrit{Bimbisāra} as one of their own. Most likely he supported all the religions in his realm, as was the custom.

Regardless of his personal faith, however, there is nothing unusual in the idea of a king seeking out a dignified ascetic or wandering sage. The final verses of \textsanskrit{Bimbisāra}, however, seem quite extraordinary, depending on how they are read (Snp 3.1:16–17).

If we keep the sense of each verse contained within the verse, it seems \textsanskrit{Bimbisāra} first praised the Bodhisatta’s dignified appearance, then offered him wealth and glory at the head of an army. This is how the verses are interpreted by the commentary, followed by Bhikkhu Bodhi. But no king in his right mind would promise an army to a stranger, let alone to a dedicated renunciant.

If we ignore the verse boundaries, however, the description of the aristocrat naturally agrees with him riding in an army. Then, having praised the Bodhisatta’s appearance, \textsanskrit{Bimbisāra} offers him “pleasures” (\textit{bhoga}). This reading is adopted by K.R. Norman. It is still an inappropriate gift for a renunciant, but not nearly as outrageous. It is exactly the kind of thing a then-young king might do; on the one hand, out of a genuine desire to serve, and on the other, as a test. It is, in fact, the same test that Devadatta failed when he became corrupted by the riches offered by \textsanskrit{Bimbisāra}’s troubled son \textsanskrit{Ajātasattu}.

The Sanskrit versions are quite different at this point, which is usually a sign that the lines were difficult even for the ancients. The \textsanskrit{Saṅghabhedavastu} says that the mendicant life is no good for one such as the Bodhisatta, and he offers pleasures such as dwellings and women. The \textsanskrit{Mahāvastu} compresses the passage, and has a reading that the translator Jones describes as “untranslatable”.

This is a good example of what I call “the principle of least meaning”. Texts, and especially sacred scriptures, tend to become overburdened with meaning and interpretation. When translating, always choose the \emph{thinnest} reading possible, the one that adds the least meaning to the passage. In this case, to offer food and enjoyments to a renunciant is an everyday thing, but to offer an army is astonishing and unprecedented. Unless the text requires it, prefer the lesser meaning.

The outcome was never in doubt. If the Bodhisatta had succumbed to temptation, he would never have become the Buddha and we would not hear his story. But this was not the final temptation he was to face. He politely rejects \textsanskrit{Bimbisāra}’s offer and commits to “striving”.

In his response to the king, he identifies his country, going into a level of detail that suggests that he did not assume \textsanskrit{Bimbisāra} was already familiar with it. His home was said to be on the slopes of the Himalayas, which is perhaps a slight exaggeration, as it is north but not exactly mountainous. But the most interesting reference is that it is ruled by one who is “native” (\textit{niketino}). This seems to imply that the Sakyans are indigenous to the region, an implication that is confirmed in the commentary. He says they are “in the land of the Kosalans” (\textit{kosalesu}), acknowledging that his own land has been subsumed within the growing Kosalan realm.

\subsection*{Striving}

The \textsanskrit{Padhānasutta} picks up the story sometime later, when the Bodhisatta was meditating on the banks of the \textsanskrit{Nerañjara} river (Snp 3.2). He is confronted by \textsanskrit{Māra} and his ten armies and responds with a bold and confident assertion of his future enlightenment.

The title word \textit{\textsanskrit{padhāna}} means both “striving, putting forth energy”, and also the active application of energy in practice, and thus comes close in meaning to “meditation”; a \textit{\textsanskrit{padhānasāla}} is a “meditation hall”. Like the \textsanskrit{Pabbajjāsutta}, we have two almost complete parallels in Indic languages, in the \textsanskrit{Mahāvastu} again, and in the Lalitavisatra, a popular legend of the Buddha in Sanskrit; while it is a later text, this passage is mostly verbatim.

The chronology of these episodes is not entirely clear. According to the commentary, this Sutta depicts the time the Bodhisatta was engaged in the practice of severe, Jain-like austerities. \textsanskrit{Māra} says he is thin, discolored, and near death, which is how he is described in the prose accounts of this period of “striving” (MN 36:26.1). It was immediately after this encounter that he realized the futility of such practice and undertook the gentler path of \textit{\textsanskrit{jhāna}}.

It’s an odd narrative twist, and one that I feel is unpersuasive. The whole point of the \textsanskrit{Padhānasutta} is the Bodhisatta’s absolute confidence and commitment, and it feels strange that he would then simply realize that he was wrong after all and completely change course, dismissing all his work as “pointless” (\textit{\textsanskrit{anatthasañhita}}). Clearly whatever he was doing was not pointless, for it allowed him to overcome \textsanskrit{Māra}’s armies.

Moreover, the qualities and practices spoken of in the Sutta sound Buddhist, and quite unlike the Jains or other practitioners of self-torment. He says nothing about burning up past kamma, or that pleasure is to be gained by pain, which is the belief that drove him to extremes of austerity (MN 85:10.2). On the contrary, he speaks of having faith, energy, and wisdom, of mindfulness, and especially of the calm mind of immersion in \textit{\textsanskrit{samādhi}}. The prose accounts of austerities say nothing like this. Indeed, it was the fact that he could not find tranquillity that drove him to abandon austerities. When listing the “ten armies of \textsanskrit{Māra}” he names only psychological obstacles to meditation as understood in Buddhism, which are not mentioned in the prose accounts.

The Bodhisatta says to \textsanskrit{Māra} that he had attained the “supreme” (\textit{uttama}) “feeling” (\textit{\textsanskrit{vedanā}}), an unusual phrase. This is reminiscent of the prose claim that he had experienced the “ultimate” (\textit{parama}) “painful feeling” (\textit{dukkha \textsanskrit{vedanā}}, MN 85:30.2), hence Bhikkhu Bodhi translates it as “I have experienced extreme pain”. But it would be wise to not insist on this point, for the \textsanskrit{Mahāvastu} has the supreme “state” (\textit{\textsanskrit{padaṁ}}), a term for \textsanskrit{Nibbāna}, while the \textsanskrit{Saṅghabhedavastu} has supreme “intention” (\textit{\textsanskrit{cetanā}}), which would refer to his avowed dedication to Awakening. These are not merely variant readings, but three quite different interpretations of the narrative, each of which finds support in the text.

\begin{itemize}%
\item \textit{\textsanskrit{Vedanā}}: Verses 9 and 10 speak of the drying up of blood and flesh, reaching the extremes of painful “feeling” through self-mortification.%
\item \textit{\textsanskrit{Cetanā}}: Verses 16 through 19 express a vivid and undaunted “intention” to triumph against all odds and vanquish his demons.%
\item \textit{Pada}: Verses 7 and 8 say he has no need for merit and possesses wisdom, while in verse 11 he claims to be pure, and in verse 5 he is referred to as the “Buddha”, all of which suggest this episode happened after he was enlightened. Note too that the verb “attained” (\textit{patta})—which is found in all three versions—is normally used of reaching a spiritual achievement, not of experiencing a feeling.%
\end{itemize}

If we see this episode through the lens of the prose accounts, which emphasize the distinct break between the periods of self-torment and the undertaking of \textit{\textsanskrit{jhāna}}, this Sutta uneasily straddles the divide. Perhaps, then, the mistake is ours, and we should not try to impose that framework on the text, ignoring the prose accounts and simply taking the \textsanskrit{Padhānasutta} as-is.

It depicts the Bodhisatta undertaking a strenuous program of meditation. He is practising rigorous self-denial in respect of food and rest, but not the full range of austerities. It was while practising in this way that he purified his mind of obstacles, not because of pain, but because of energy and determination. There was no abrupt shift from the path of austerities to that of \textit{\textsanskrit{jhāna}}.

In this light, we can propose a fourth explanation for the difficult phrase, “attained the supreme feeling” (\textit{\textsanskrit{pattassuttamavedanaṁ}}). This occurs immediately after the Bodhisatta has said that his mind has become clear, with strong mindfulness, wisdom, and immersion. The next line speaks of disinterest in sensuality and the purity of a being. Throughout the Suttas, \textit{\textsanskrit{jhāna}} is an “attainment” regarded as the highest feeling, one that is realized through immersion, which rejects sensual pleasure, which is a state of purity, and which was a pivotal realization on the Bodhisatta’s journey.

There is no doubt that the division between self-torment and meditation, which was a fundamental feature of the Dhamma starting with the very first sermon, was never observed quite so clearly in Buddhist culture, as is attested by the popularity of images of the “fasting Bodhisattva”. The practice of austerities was always a troubling episode for later hagiographers. If such austerities were such a waste of time, why did the Bodhisatta do them in the first place, given that he had already been developing the \textit{\textsanskrit{pāramīs}} for many lifetimes? Of course, the \textit{\textsanskrit{pāramīs}} are not part of early Buddhism, and in the Suttas, he undertook austerities because he did not know what he was doing.

This Sutta suggests that this ambiguity about austerity was present from an early time. It would be wrong to revise the whole history of Buddhism based on a single, rather ambiguous, text, but it does suggest that the clear division between austerities and \textit{\textsanskrit{jhāna}} became emphasized in the prose accounts, probably to distinguish the Buddha from the Jains. Reality is rarely so clear-cut.

There is a curious shift in narrative voice. The opening verse has the Buddha speaking in the first person (\textit{\textsanskrit{maṁ}}), but after \textsanskrit{Māra}’s verses, the text refers to the Buddha in the third person. Jayawickrama suggests we correct to \textit{\textsanskrit{naṁ}} or \textit{\textsanskrit{taṁ} (i)\textsanskrit{maṁ}}, but Norman rejects these emendations based on the commentary and regards the phrase as an accusative version of the common idiom \textit{so’\textsanskrit{haṁ}}; the same idiom occurs in dative in verse 11: \textit{tassa me}. Here the commentary must be right, for the reading in the \textsanskrit{Mahāvastu} does not admit such ambiguity (\textit{\textsanskrit{mayā}}). The Lalitavistara rephrases this line, perhaps because its more refined literary sensibilities did not admit such a lapse. Given that the opening verses are quite different in the three versions, it’s likely they were added later to introduce the story, but the shift in voice remains unexplained. Perhaps it is simply a sign of a slightly raw and unedited original.

Less inexplicable is the text’s occasional flirtation with a literal depiction of \textsanskrit{Māra}’s armies. When Siddhattha says that he sees \textsanskrit{Māra} on his mount, surrounded by his forces in full battle-array (Snp 3.2:18.1), it sounds decidedly un-psychological and more like the flamboyant representations of \textsanskrit{Māra} and his armies that became so popular in later legend and art. But this is one of the several verses of the Pali that have no parallel in the \textsanskrit{Mahāvastu} or the Lalitavistara, and which seem to have been added to give extra dramatic flair.

Both of the Sanskritic versions have a much briefer final section, so this was probably expanded in the Pali. One point of agreement, however, is that even in its few verses the \textsanskrit{Mahāvastu}, like the Pali, mentions that he will lead many disciples to freedom from \textsanskrit{Māra}. This is one of the aspects of \textsanskrit{Māra}’s mythology that is often overlooked. Yes, \textsanskrit{Māra} is a psychological metaphor for defilements. But he is also something larger: a personalized representation of the darkness that afflicts the world. If \textsanskrit{Māra} is defeated once, he can be defeated again, which is why he was so determined to ensure the Bodhisatta could not succeed in his quest. Myth exalts the personal to the universal. This is a story of a man sitting under a tree on the bank of the \textsanskrit{Nerañjara} river 2,500 years ago, battling his personal demons. But it is also all of us, now, wherever we are, battling our demons. And no matter how many times we are defeated, we are inspired to get up once more, for we know that victory is possible: it has happened once, so it can happen again.

\subsection*{Well-Spoken Words}

In the \textsanskrit{Aṅguttara}-style \textsanskrit{Subhāsitasutta}, the Buddha lists four factors of “well-spoken speech” and reiterates his teaching with a simple verse. \textsanskrit{Vaṅgīsa} feels inspired to speak and lends his poetic flair to expanding and elaborating the teaching (SN 3.3). The discourse is repeated at SN 8.5, and \textsanskrit{Vaṅgīsa}’s verses are included in the \textsanskrit{Theragāthā} (Thag 21.1:19–22). As a short discourse on practical ethics, with a special acknowledgement of the role of the teacher, this feels like it would be better suited for the previous chapter.

There is no real parallel in the \textsanskrit{Aṅguttara}, the closest being AN 5.198, which lists five factors of well-spoken speech with a prose introduction similar to the \textsanskrit{Subhāsitasutta}. The five factors, however, differ.

In the Buddha’s portion of the Sutta, the first factor of “well-spoken speech” is explained as “well-spoken speech”, an oddity that is not cleared up in his verse. The four factors, however, map against the regular four kinds of right speech, albeit in a different order.

\begin{itemize}%
\item Truthful speech: 1 = 4%
\item Harmonious speech: 2 = (1)%
\item Pleasant: 3 = 3%
\item Meaningful speech: 4 = 2%
\end{itemize}

It seems that here, “well-spoken speech” implies speech leading to harmony, which more-or-less agrees with \textsanskrit{Vaṅgīsa}’s expansion. But I can’t help feeling that we’re missing something in understanding why this Sutta is formed the way it is.

It’s noteworthy that for the Buddha, the verse is plain and redundant in meaning; it serves only as a mnemonic to support the prose. \textsanskrit{Vaṅgīsa} doesn’t merely decorate the verses, but expands the meaning, claiming, for example, that “truth is the undying word”. His final verse corresponds with speech on the Dhamma, but he does not mention that word, instead speaking about the Buddha’s words. While the word \textit{dhamma} can have many meanings in the Suttas, the sense of “the Buddha’s teachings” is perhaps the most common. He also re-orders the teaching so that it climaxes with the Buddha’s words, a significant shift since the final factor is felt to be the most important. For the Buddha, the most important thing is that speech is true; for \textsanskrit{Vaṅgīsa}, the most important thing is that it is spoken by the Buddha. These verses have the same order in the \textsanskrit{Udānavarga}, so it is no accident (Uv 8.11–15).

\subsection*{With \textsanskrit{Bhāradvāja} of \textsanskrit{Sundarikā} on the Sacrificial Cake}

The Buddha encounters another \textsanskrit{Bhāradvāja} in the \textsanskrit{Pūraḷāsasutta}, also known as the \textsanskrit{Sundarikabhāradvājasutta} (Snp 3.4). This \textsanskrit{Bhāradvāja} is engaged in the ancient Vedic rite of fire worship on the banks of the \textsanskrit{Sundarikā} river. The same encounter is told at SN 7.9, but while the prose setting is the same, only four verses are shared. Our current Sutta is much longer, and in some aspects appears corrupted, so it is likely to have been expanded from SN 7.9. A \textsanskrit{Bhāradvāja} of \textsanskrit{Sundarikā} appears as a champion of the virtues of ritual bathing in MN 7:19.1 and is there converted and ordained just as here, so they cannot be the same person.

The contours of all these encounters follow the same template. The brahmin believes in the efficacy of Vedic ritual, but the Buddha counters this with his practical ethics, leading to the Brahmin’s conversion, ordination, and ultimately arahantship.

In the prose introduction, \textsanskrit{Sundarikabhāradvāja} wishes to share the leftovers from his sacrifice with someone, and spies the Buddha meditating. The Buddha has pulled his robe over his head—to ward off insects or inclement weather—and not until he uncovers it does \textsanskrit{Sundarikabhāradvāja} realize that he is shaven. When he asks the Buddha’s caste, the Buddha begins the verse exchange by asserting that he has no caste, and accuses the brahmin of asking an inappropriate question. The brahmin seems to think this is a little harsh, saying it’s merely a formality among brahmins. Indeed, the Buddha did not object when King \textsanskrit{Bimbisāra} asked the same question (Snp 3.1:17.4), and in SN 7.9 he merely says to ask about conduct rather than caste. This line is unmetrical and is the first of several lines in this passage that appear to be prose.

The rather confrontational tone continues when the Buddha, out of nowhere, challenges the brahmin regarding knowledge of the \textsanskrit{Gāyatrī} Mantra. This most famous of Vedic verses is called the \textsanskrit{Sāvitti} in Pali, but the description of the number of lines and syllables leaves no doubt as to what it refers to. Quite apart from the oddness of the Buddha issuing such a challenge, it’s a lame flex. Everyone knows this verse.

The brahmin, sensibly, ignores this and asks why so many people perform the sacrifice, to which the Buddha says the donor profits if they give to a \textit{\textsanskrit{vedagū}}. The Buddha doesn’t explain himself, but to him, this means someone who is freed by mastering the wisdom of the threefold knowledges (\textit{\textsanskrit{tevijjā}}). To the brahmin, on the other hand, it means someone who has memorized the Vedas. Coming after the \textsanskrit{Gāyatrī} challenge, he has every reason to think the Buddha shares his understanding. Nonetheless, even this brief and somewhat incoherent exchange is sufficient to convince him of the Buddha’s wisdom. Conversion narratives in the Suttas often have this kind of abruptness, almost as if something is missing. Perhaps the missing piece is the power of the Buddha’s presence, his overwhelming charisma.

He asks the Buddha about a profitable way to perform a sacrifice, prompting the long series of verses that make up the main teaching passage. Here the incoherence of the opening exchange is left behind, and we return to the Buddha’s familiar practice of redefining Brahmanical ideas in his terms. The discourse ends with the Buddha refusing to eat “enchanted” food, as in Snp 1.4. Just as any kind of wood can produce fire, any person who has conducted themselves well and left defilements behind can be a true brahmin.

\subsection*{With \textsanskrit{Māgha}}

The \textsanskrit{Māghasutta} tells us yet another story of how a brahmin approached the Buddha to ask about making an offering or a sacrifice (Snp 3.5). This time he is a \textit{\textsanskrit{māṇava}}, an apprenticed student, yet he appears to already have confidence in the Buddha. He asks how an offering is purified, and the Buddha says, as in the previous Sutta, that it is when the recipient is pure and unattached. When asked about who in the world is unattached, the Buddha gives a long series of verses, similar to the previous Sutta, extolling the arahants as the pure ones worthy of offerings.

Satisfied, \textsanskrit{Māgha} asks about how to accomplish an offering, to which the Buddha replies that the giver should also be pure, with a heart full of love. Finally, \textsanskrit{Māgha} asks how to be reborn in the \textsanskrit{Brahmā} realm. The Buddha, rather than questioning the point of this wish, says that to be reborn in the \textsanskrit{Brahmā} realm, one must offer to the worthy with the three-fold sacrifice, which at DN 5:15.1 is defined as having no regrets before, during, or after giving. This is a controversial answer, as it is normally understood that to be reborn in the Brahma realm it takes more than generosity: one must develop at least first \textit{\textsanskrit{jhāna}}.

For example, AN 4.15 talks of developing absorption through the four \textit{\textsanskrit{brahmavihāras}} (love, compassion, rejoicing, and equanimity), and being reborn in various \textsanskrit{Brahmā} realms as a result. The Sutta distinguishes between an ordinary person—who remains in that realm so long as it lasts, then is reborn in a lower realm—and a disciple of the Buddha, who becomes extinguished in that realm.

However, the idea is not entirely without parallel in the suttas. For example, the \textsanskrit{Dānamahapphalasutta} says that one who gives with the thought that the gift is an adornment and prerequisite of the mind will be reborn in the \textsanskrit{Brahmā} realm, and even become a non-returner (AN 7.52:12.12).

This gives a clue as to the interpretation. Giving as adornment and a prerequisite for the mind means that the act of giving is part of the development of the mind in the eightfold path. It is not meant to discount the importance of the remainder of the path. But giving is one of those factors that make it possible for all the rest of the factors to come to fulfilment. And that includes the development of \textit{\textsanskrit{jhāna}}. While this doesn’t completely resolve the difference between the two perspectives, it does show that they are linked.

\subsection*{With Sabhiya}

The Sabhiyasutta tells the story of the wanderer Sabhiya and his questions for the Buddha (Snp 3.6). Ultimately he ordains and becomes an arahant. This is probably the same Sabhiya whose verses at Thag 4.3 are an admonishment to lax monks. The Sabhiya \textsanskrit{Kaccāna} of SN 44.11 and MN 127 may or may not be the same person.

Sabhiya has, quite literally, a mission from god. A former relative has appeared to him in divine form after death and charged him with finding the answers to a series of questions. Like King \textsanskrit{Ajātasattu} in the \textsanskrit{Sāmaññaphalasutta} (DN 2), he had questioned the six prominent \textit{\textsanskrit{samaṇa}} teachers and succeeded only in annoying them. Sabhiya was disillusioned to the point where he was about to give up on his spiritual practice altogether. But he went at last to the Buddha, who welcomed his questions with warmth and openness.

Sabhiya asks about the proper way to understand various terms, all of which describe a spiritually ideal person. There are twenty terms in all, divided into five groups of four. The division appears to be merely a literary artifact without meaningful significance.

The Buddha answers each of the questions, speaking to the spiritual ideal embodied in the words, rather than simply listing off the dictionary meanings. These often rely on wordplay that is not easy to represent elegantly in English. Indeed, some of the puns are obscured in the Indic dialects, and linguists use them to trace the dialectical history of the text. The Hybrid Sanskrit parallel in the \textsanskrit{Mahāvastu} offers additional source material for this inquiry, as does the Chinese translation, which is included in the \textsanskrit{Abhiniṣkramaṇasūtra}, a late biography.

His depression banished, Sabhiya expresses his joy in a long series of devotional verses and asks the Buddha for ordination.

\subsection*{With Sela}

The Selasutta, which is also found at MN 92, offers another story of conversion and ordination resulting in arahantship (Snp 3.7). It shares a few passages in common with the Sabhiyasutta.

This time, however, the narrative is more complex. The Buddha is sojourning in \textsanskrit{Āpaṇa}, a city in what is today eastern Bihar, which is near the easternmost extent of his travels. Since he rarely ventured so far east, the matted-hair Brahmanical ascetic \textsanskrit{Keṇiya} was excited to receive him and offer a meal for the whole \textsanskrit{Saṅgha} of over a thousand mendicants. The Buddha cautioned him about the extent of the offering, both due to the size of the \textsanskrit{Saṅgha} and the fact that he was a follower of the brahmins. If it seems unlikely that a matted-hair ascetic should have the considerable resources to make such an offering, consider that today it is common for bhikkhus to organize events on a similar scale.

While \textsanskrit{Keṇiya} was busying himself with preparations, the prominent brahmin Sela came by with three hundred students. On inquiry, \textsanskrit{Keṇiya} says he is preparing for a large sacrifice (\textit{\textsanskrit{mahāyañña}}), which is an interestingly direct example of how offerings to the \textsanskrit{Saṅgha} took over the role of the sacrifice in Brahmanism.

Sela is thrilled to hear the word “Buddha”, and wants to know whether the Buddha possesses the famed “32 marks of a great man”. The bulk of the discourse consists of Sela’s verses extolling the Buddha, and the Buddha’s responses. Inspired, Sela asks for ordination together with his students and went forth right there.

The meal with \textsanskrit{Keṇiya} was the next morning, and following it the Buddha gave the verses of \textit{\textsanskrit{anumodanā}}. Today, such recitations are typically believed to be either a quasi-magical “blessing” given by the mendicants or an act of sharing merit with departed relatives. But in the Suttas, the \textit{\textsanskrit{anumodanā}} is a reflection on cause and effect, emphasizing the merit created by the giver due to their actions.

Sela and his students went on to practice meditation in retreat and attain arahantship. They returned to the Buddha where Sela announced their achievement and offered further homage to the Buddha.

\subsection*{The Dart}

The Sallasutta offers a reflection on the inevitability of death and the “dart” of grief that a meditator should pluck out (Snp 3.8). Its blunt declarations are a departure from the narrative and devotional texts that characterize this chapter, and perhaps would feel more at home in the \textsanskrit{Aṭṭhakavagga}.

Some might find such teachings to be too strong, but for others the bluntness is refreshing, and I would count myself as one of those. The Buddha is not being confrontational for the sake of it. He is simply asking us to accept the facts of life from which we normally want to hide. It is natural to respond to death with grief, but the grieving creates nothing but more distress. Sometimes it is as if the relatives who remain feel obliged to show excessive grief as a demonstration of love and loyalty, but this is not the Buddha’s way. Our grieving does nothing to help the dead, so why do we hurt ourselves?

The opening lines say that there is no \textit{nimitta} for mortal life. This doesn’t mean that there is no such thing as a “sign” of life. \textit{Nimitta} is a term used in horoscopes and prophecy, and here it means there is no way the end of life can be predicted. The Buddha is countering those who claim to be able to predict the course of life, a practice that is still common in India today.

For all the hold the grief has on the mind, the Buddha says it can be dispelled as lightly as a tuft of cotton blown by the wind. The Sutta is not morbidly dwelling on death; it is promising us that we can be free from the spell of death and grief.

\subsection*{With \textsanskrit{Vāseṭṭha}}

In the \textsanskrit{Vāseṭṭhasutta} (Snp 3.9), the brahmins \textsanskrit{Vāseṭṭha} and \textsanskrit{Bhāradvāja}, each the accomplished student of a recognized master, hold opposing views on the fundamental question of what makes a person a brahmin. Both of these are ancestral clan names of the brahmins, and it is not easy to know the actual individuals behind them. \textsanskrit{Bhāradvāja} takes the traditional view that it is one’s ancestry that matters, while \textsanskrit{Vāseṭṭha} points to the importance of how you live your life.

Such is the fundamental dilemma of any religious or spiritual path once it has become institutionalized. The defining marks of sanctity become expressed in an external form, be it caste, lineage, garb, or ordination. It is this external form that, of necessity, governs the external privileges of religious communities, such as who gets to perform rituals and, even more to the point, who gets to own the real estate. Being a good person is not enough, and sometimes it seems that it is an impediment.

Despite the presence of many expert brahmins in the area, \textsanskrit{Vāseṭṭha} suggests they ask the Buddha. Given his respectful approach to the Buddha and the fact that it was his opinion that agreed with the Buddha’s perspective, it seems likely that he was already at least somewhat familiar with the Buddha’s teachings.

The Buddha points out that in the natural world, species are defined by their birth, each to their own kind. But in the human realm, this is not so. There is no biological characteristic by which humans can be classed. This is a radical position, which rejects any form of “essentialist” classification of humanity. In modern terms this would include pseudo-scientific notions such as “race”, which have no definitive biological meaning, but rather are mediated through culture and history. Strikingly, the Buddha explicitly mentions the genitalia here, thus rejecting gender essentialism based on biology. As is the case with race, modern biology rejects any single biological sex characteristic, acknowledging that biological sex is a complex phenomenon with diverse manifestations.

Rather than innate biological differences, the Buddha says that humans are differentiated by “convention”, that is, through what is socially and culturally agreed upon. And such distinctions are based on a person’s deeds. Whether a farmer, a trader, or a ruler, it is how a person lives their life that makes them who they are. Among these livelihoods, the Buddha includes a ritualist who performs priestly rites and sacrifices, showing what he thought of the conventional duties of a brahmin.

The Buddha goes on to speak of a true brahmin, who becomes so when they are free of attachments, living an ideal life of non-harming, wisdom, and moral purity. It is deeds that make the world go round, and deeds that define a person in the eyes of others; but a true brahmin has made an end to all deeds.

The entire Sutta is also found in MN 98, while most of the verses are found in the Dhammapadas.

\subsection*{With \textsanskrit{Kokālika}}

Any hero needs a nemesis, and the Buddha’s nemesis was his cousin Devadatta. Devadatta’s story is told primarily in the “Chapter on Schism in the \textsanskrit{Saṅgha}” of the Vinaya (\textit{\textsanskrit{Saṁghabhedakakkhandhaka}}, Khandhaka 17), but there are many discourses that record various incidents or statements relating to him. Such is the \textsanskrit{Kokālikasutta}, which records the attack by an acolyte of Devadatta on the Buddha’s two chief disciples, \textsanskrit{Sāriputta} and \textsanskrit{Moggallāna} (Snp 3.10). The context is found in the above-mentioned chapter on schism, where \textsanskrit{Kokālika} is one of the first monks approached by Devadatta to enlist in his scheme to split the \textsanskrit{Saṅgha} and replace the Buddha.

Both there and here, \textsanskrit{Kokālika} is said to have issued a harsh and groundless accusation against \textsanskrit{Sāriputta} and \textsanskrit{Moggallāna}, accusing them of having wicked desires. In the Vinaya, however, he warns Devadatta, when \textsanskrit{Sāriputta} and \textsanskrit{Moggallāna} come to redeem the errant monks who have followed Devadatta. Here he has the temerity to accuse the Buddha. This approach seems clumsy and doomed to failure, and on that basis might be seen as unlikely, were it not for the fact that history furnishes us with plentiful examples of failed coup attempts by goonish dolts.

Failing to turn the Buddha against his chief students, \textsanskrit{Kokālika} leaves, but horrifying boils erupt on his body; he dies and goes to hell. The remainder of the Sutta is given over to describing the length of duration of the hells and the gruesome torment that awaits a person there who has maligned the noble ones.

The idea that maligning a person of purity is an especially heinous act is not original to Buddhism, for the brahmins have a similar doctrine regarding anyone who maligns a brahmin. It’s important to recognize that there are plenty of cases in the Suttas and Vinaya where senior monks, and even the Buddha, were criticized, and it did not result in such a drastic kammic result. If the criticism is made in good faith, then it is considered whether it is true or not, and in some cases, the Buddha was happy to adopt the recommendations made by critics. In the case of \textsanskrit{Kokālika}, however, he and the other followers of Devadatta had repeatedly demonstrated their lack of integrity, so by this point, the Buddha simply rejected \textsanskrit{Kokālika}’s bad faith accusations out of hand.

The prose section and the first four verses find parallels in the Pali at SN 6.10 and AN 10.89, while the fifth and sixth verses have several parallels in Dhammapadas and elsewhere. The remaining verses, which detail the torments of hell, are unique to this Sutta. They have presumably been added later, and the commentary ascribed them to \textsanskrit{Moggallāna}. They have much in common with the similarly-late \textsanskrit{Devadūta} Sutta (MN 130). According to the commentary, the final two verses beginning with the counting of the years in the Pink Lotus Hell were not mentioned in the (now lost) commentarial text upon which the commentary was based. Most likely, these verses were added by redactors, presumably at some point during the transmission in Sri Lanka.

\subsection*{About \textsanskrit{Nālaka}}

The \textsanskrit{Nālakasutta} is divided into two portions (Snp 3.11). The introductory narrative tells the famous story of how the hermit Asita visited the newborn Siddhattha and prophesied his future Buddhahood. He was old and would not live to see the Buddha himself, so he urged his nephew \textsanskrit{Nālaka} to seek out the Buddha when the time came. Many years later, \textsanskrit{Nālaka} heard that the Buddha had indeed appeared in fulfillment of Asita’s prediction. The second portion of the Sutta tells of how \textsanskrit{Nālaka} went to the Buddha and asked about the way of the sage (\textit{muni}).

Those who know the story of the Buddha’s life will recall the familiar tale of how Asita was the former chaplain of Siddhattha’s grandfather and teacher to his father Suddhodana. He visited the court of King Suddhodana in his grand palace, where a multitude of brahmin soothsayers had foretold that the prince would become either a wheel-turning monarch or an all-seeing Buddha. They will be well aware of how Asita, examining the boy for the 32 marks of a Great Man, was the first to realize that the other soothsayers were wrong: the boy was surely destined to become a Buddha. And they will know that the reason Asita would never see the Buddha was that he was destined to be reborn for many aeons in the formless realms, where beings are removed from involvement in the material dimensions.

What they may not know is that in this, the earliest version of events, none of these details are found. Asita was not a beloved former teacher and appears nowhere else in the Suttas. There is no mention of a king, a palace, or a court therein, only Suddhodana with his family in his home. No soothsayers are spoken of, nor the 32 marks. There was no “examination” of the boy and his marks; Asita merely “saw” him lying on his cot. Nor does the text speak of conflicting prophecies, for the boy’s destiny was known even to the mere worldly gods before Asita appeared. Asita is not said to have been destined for the formless realms; rather, he is simply old and will not live long. The rebirth in the formless realms was probably intended to show that Asita belonged to the most exalted circles of Brahmanical meditators, while also filling an awkward plot hole: why could he not simply return to see the Buddha from wherever he had been reborn?

The \textsanskrit{Nālakasutta} is, in fact, a rare witness of a transitional phase in the rapidly-evolving legend of the Buddha. Along with MN 123 and DN 14, it is one of the earliest sources for certain crucial aspects of the Buddha’s mythology, in particular the idea that Siddhattha was, from the time of his birth, a \textit{bodhisatta} destined for enlightenment. Nonetheless, the dramatic and mythological significance of events is not yet fully drawn out, and the relatively simple account of the \textsanskrit{Nālakasutta} was, for all practical purposes, soon overshadowed by the more developed legends. Yet a Sutta like this speaks quietly with a very specific voice. It deserves a hearing on its own terms.

According to the commentaries, the introduction was composed by Ānanda at the request of \textsanskrit{Mahākassapa} at the First Council. This confirms that it was not the Buddha’s words. But it must be considerably later than this, and probably originated a few centuries after the Buddha’s passing. Jayawickrama identifies a range of late features, including a multiplicity of late and Sanskritic words, decorative poetics, and a variety of metres, all of which set the introduction quite apart from the passages that follow.

The introductory verses share these late features with those of the \textsanskrit{Pārāyanavagga}. Both passages are called \textit{\textsanskrit{vatthugāthā}}, a term used nowhere else. In both cases, an elder sage sets the action in motion by urging their younger counterparts to seek out the Buddha. This legendary narrative serves to frame a set of teachings that stem from an earlier age. Such stories fulfill a need for the Buddhist community in the time that they were composed. The \textsanskrit{Pārāyanavagga} is a conversion narrative, which supports the geographical expansion of the Buddha’s teachings in the south of India. The \textsanskrit{Nālakasutta} fulfills a more universal need: to exalt the Buddha’s teachings within the cosmic and eternal significance of the Buddha as a person.

The Buddha gave only sparse details of his early life, so after his death, the Buddhist community swiftly moved to complete the narrative. It is conventional for the birth of a great hero to be heralded with prophecy. The Buddha’s silence on this topic leaves a gap that yearns to be filled. Never mind that the Buddha consistently spoke against the very idea of using marks and signs to tell the future. And leave aside the inconvenient detail that the very essence of his teaching, from his first words until his last, was the practice of the eightfold path, not the fulfillment of a destiny. The mythic impulse is not so easily dismissed. In the early Buddhist community, as the historical Buddha faded from living memory, there grew an insatiable need for stories to keep him alive.

Myth serves this purpose, for it tells of things that “never were, but are always”. This is not a paradox, but a simple psychological reality. For those who are born into a religious tradition, there is no first moment when the story of the Buddha is heard. It has always been there, told and retold in story, in painting, and in song, surrounding you from before you had the language to hear it. There never was a time when Siddhattha did not become a Buddha. The prophecy of Asita is not a historical record of a soothsayer and his prediction, but rather, confirms the eternal and inevitable reality of the Buddha. It is not a historical account of what \emph{actually} happened, but a mythological account of what \emph{must have} happened.

The sutta opens with Asita’s meditative vision of the gods. They are dancing and singing with exuberance, celebrating the birth of the Bodhisatta. The gods are, in this mythological context, able to see the significance of this event, and even to predict specific details like the teaching of the First Sermon in the “Grove of the Hermits”, i.e. the Isipatana near Benares. Asita then hurries to see the newborn babe, where he confirms the prophecy. But he bursts into tears when he realizes that he is too old and will die before witnessing the magnificence of Awakening.

Asita’s tears at the Buddha’s birth echo the tears of Ānanda, who, when he learned of the Buddha’s imminent demise, could not bear to face life without his beloved Teacher. Did Ānanda—the putative author of this story—wish to emphasize Asita’s emotional vulnerability here, drawing on his own experience? We cannot say, but we can say that this kind of narrative echoing is an outstanding feature of early Buddhist mythic narratives. Ostensibly unrelated texts in different collections are formed with an eye to mirroring related events. While the \textit{\textsanskrit{vatthugāthā}} as we have it appears to be later than Ānanda, there is no doubt that Ānanda himself was the literary founder of the Buddha’s legend. The texts are profoundly influenced by his methods and sensibilities, and it is, I believe, likely that such details originated from him.

What can we say about the sage Asita? He appears without explanation or context, yet unhesitatingly takes a position of prominence in the home of a powerful man. It is as if the text assumes that the audience will be familiar with him, and will take such confidence for granted. A survey of related early texts suggests that Asita fulfills an archetype that we shall call the “dark hermit”. It is, I believe, the loss of awareness of this archetype that prompted the later tradition to say that he was the family chaplain since they needed a new explanation for the familiarity and respect with which Asita was received in Suddhodana’s home.

The word \textit{asita} can have a variety of meanings in Pali, including “black” and “unattached”. The latter seems like a good name for a hermit, and it is indeed the explanation given in some Sanskrit and Tibetan sources, where it is taken as \textit{\textsanskrit{akleṣa}}. But the text also refers to him as \textit{\textsanskrit{kaṇhasiri}} “dark splendor”, and later texts also call him \textit{\textsanskrit{kāladevala}}, “Dark Devala”. It would seem, then, that the relevant sense of \textit{asita} is “black”. The Pali commentary confirms that this is a reference to his skin color.

His name and status as a “dark hermit” (\textit{isi}) link him with other passages in a way that is surprisingly revealing. The similarly-named hermit Asita Devala (“Devala the Dark”) appears in the \textsanskrit{Assalāyanasutta} (MN 93), where the Buddha relates to the prideful brahmin student \textsanskrit{Assalāyana} how Asita Devala challenged seven hermits on the doctrine of caste. In the \textsanskrit{Ambaṭṭhasutta} (DN 3) we find yet another “black” hermit, this one named \textsanskrit{Kaṇha}, whose story is also told by the Buddha in a conversation with a prideful brahmin student.

All of these passages, while quite different, have a range of features in common. A powerful hermit (\textit{isi}) named “black” appears in a quasi-legendary narrative. They are an outsider, who though respected and associated with the Brahmanical tradition, is not exactly a part of it. They challenge the accepted order of things. \textsanskrit{Kaṇha} is the son of a slave girl, and his descendants became regarded as brahmins. Asita Devala, likewise challenging the Brahmanical notions of lineage, is notably not identified as a brahmin, in contrast with the “seven Brahmin hermits” (\textit{satta \textsanskrit{brāhmaṇisayo}}).

The “dark hermits” appear in marginal spaces. Asita Devala is in a wilderness hermitage, while \textsanskrit{Kaṇha} learns magic in the south. The latter detail is especially interesting, as, during the Buddha’s time, the south of India was largely unknown, and was regarded as outside the sphere of civilized Aryan lands. While the \textsanskrit{Nālakasutta} itself says nothing of Asita’s origins, both the \textsanskrit{Mahāvastu} and the \textsanskrit{Nidānakathā} place him in the south. In the \textsanskrit{Pārāyanavagga} we shall meet another sage who, having made his way to the remote southern lands, encountered an exponent of dark magics there.

All of this evokes a tradition of outsider ascetics of a dark skin color, known for their magical abilities, and associated with the south. It would seem likely that these are a cultural memory, and perhaps cultural reality, of non-brahmin or pre-brahmin ascetics or shamans who predated the arrival of the brahmins in India. They interacted with the brahmins in complex ways, sometimes being adopted by them and changing Brahmanical culture from the inside.

All this begs the question: are these various hermits, in fact, one and the same? Several modern commentators have remarked on the apparent “confusion” between the Asita of the \textsanskrit{Nālakasutta} and the Asita Devala of the \textsanskrit{Assalāyanasutta}. But this way of thinking assumes that there were originally different individuals, whose separate identities later became conflated. As a monk, however, I am keenly aware of how, when an ordained person appears to others, they are wrapped in two robes: the ochre robes that protect the body, and the robes of preconceptions and projections. I would suggest that the “dark hermit”, while a genuine reality of life in ancient India, was perceived in terms of an archetype or stereotype. When we encounter a dark hermit in our texts, we are seeing a partially-differentiated figure drawn from the archetype. That is not to say that there is not an actual person at the root of the story, only that their depiction is shaped by cultural stereotypes.

Since \textit{\textsanskrit{kaṇha}} is simply the Pali spelling of Sanskrit \textit{\textsanskrit{kṛṣṇa}}, it is further tempting to associate these hermits with the famed god Krishna of later Hinduism. Indeed, it may be that the deity known as Krishna is first attested in Buddhist texts, for a tale in the \textsanskrit{Ummaggajātaka} (Ja 542) tells of how the outcaste maiden \textsanskrit{Jampāvatī} was made queen by King \textsanskrit{Kaṇha} \textsanskrit{Vāsudeva}. Like the dark hermits, he transgresses the expected conventions of lineage. \textsanskrit{Vāsudeva} is, of course, a common name of Krishna. It would be, however, overly literal to claim that these were the same historical figures. These contexts are vague and legendary, framed as reports of days of yore, and associated with weird magics. They tell us something about how such figures were thought of, and how they challenged the narrow assumptions of Brahmanical theories of lineage.

All this tells us that Asita the “Black” sage or “Dark Splendour” appears in the \textsanskrit{Nālakasutta} as a figure of legend, a marvelous and revered font of wisdom and magic, come to disrupt the notions of legacy and succession. Unlike the other dark hermits, he is implied to be a brahmin, or at least, so much is suggested when he is described as a “master of marks and hymns” (\textit{\textsanskrit{lakkhaṇamantapāragū}}). “Hymns” are the Vedas, while “marks” are signs discerned as a basis for prophecy, most famously the 32 marks of the Great Man. Like other Brahmanical figures of note—including Indra and \textsanskrit{Brahmā} himself—he uplifts and validates the Buddha in the eyes of the brahmins.

The descriptions of the gods and the home of Suddhodana are unusually colorful and vivid, although still fairly restrained compared to later Indian literature. Notably, the verses mostly avoid using royal language in the depiction of Siddhattha’s home and family. Suddhodana is not referred to as king, and his residence is described simply as a “home” (\textit{bhavana}). The infant is referred to as \textit{\textsanskrit{kumāra}}, which can mean “prince”, but can equally well mean simply “boy”. Given that there is no mention of a king, the humbler translation seems preferable. The lavish descriptors and honorifics are used only for the boy, not for his family or home. The only “royal” language is the \textit{\textsanskrit{antepūra}} from which the sage departs at the end. This was a kind of walled compound inside which the rulers resided. But the description overall is quite compatible with what we know of the Sakyans from the early sources, namely that they were an aristocratic republic that elected “rulers” (\textit{\textsanskrit{rājā}}) from the leading clans to serve as rulers for limited periods. In later legend, of course, Suddhodana was elevated to the status of a king, but it seems the current Sutta predates this.

The gods on Meru’s peak refer to the newborn baby as \textit{bodhisatta}, “one intent on Awakening”, which is the only time this term occurs in the \textsanskrit{Suttanipāta}. The earliest usage of \textit{bodhisatta} was, it seems, to refer to Siddhattha after he had left home and was actively seeking Awakening. There is no suggestion in the major Sutta passages that such Awakening was predestined, or indeed, that predestination or prophecy was possible. The burden of the \textit{\textsanskrit{vatthugāthā}} is to establish the reality and reliability of prophecy. From here, it was but a small step to infer that the \textit{bodhisatta} was already destined to become Awakening in a past life long ago.

As to the main teaching passage, it is one of the classic “sage” (\textit{muni}) texts in the \textsanskrit{Suttanipāta}. In his opening question, \textsanskrit{Nālaka} tells the Buddha that the Asita’s words had come true. This establishes a connection between the two parts and suggests that, while the details of the \textit{\textsanskrit{vatthugāthā}} are late, they may have drawn from a genuine story of a sage named Asita who spoke of a Buddha. \textsanskrit{Nālaka}’s brief mention, however, says nothing of prophecy, and might just as well have been spoken in a situation similar to the \textsanskrit{Pārāyanavagga}, where an elder sage encouraged a student to seek out the Buddha. It is rarely the case that such legendary narratives are either entirely factual or entirely invented.

The Buddha urges \textsanskrit{Nālaka} to practice restraint in all circumstances, leaving behind the temptations of the worldly life. He is not sugar-coating it but warns that the path will be hard. Seeing one’s oneness with all creatures, one would never harm any of them. The life of a renunciant is solitary, devoted to meditation under a tree, and walking for alms content to receive little or nothing. The text occasionally evokes teachings from elsewhere in the \textsanskrit{Nikāyas}, as when the Buddha says to practice as if “licking a razor’s edge”, per SN 35.235:4.1 or Thag 16.2:12.4 with “tongue pressed on the roof of the mouth”, per MN 20:7.2, etc.

The sutta concludes in praise of the virtues of silence.

\subsection*{Contemplating Pairs}

The \textsanskrit{Dvayatānupassanāsutta} finishes off the “Great Chapter” with a long discourse in mixed prose and verse (Snp 3.12). It is an unusual discourse for the \textsanskrit{Suttanipāta}. Length-wise it would fit with the Suttas of the Majjhima. Doctrinally, too, it focuses on the four noble truths and other core doctrines in a way that is reminiscent of the longer prose discourses. The fact, however, that it is based around the number two, and that each section consists of a prose statement and verse paraphrase, suggests an extended text in the \textsanskrit{Aṅguttara} style.

It seems that the compilers of the \textsanskrit{Piṭakas} tried to ensure that major doctrines were found in each of the collections. This was, presumably, because each collection would have been memorized by different students. There would have been interaction between them, but it makes sense for students of every collection to have at least some teachings on the fundamental doctrines. The tradition held that the four noble truths can be found in every teaching, even if indirectly, but it seems there was a need for a direction exposition as well.

The discourse begins with the Buddha gathering with the assembled \textsanskrit{Saṅgha} at the Eastern Monastery near \textsanskrit{Sāvatthī} for the fortnightly Sabbath (\textit{uposatha}). The Buddha engages the \textsanskrit{Saṅgha} by prompting them to consider how they would respond if asked a series of questions. Notice that the Buddha rarely taught by simply expounding a long text to an audience; he usually found a way to draw them in. Here, the question is, why do we listen to the teachings on liberation? And the answer is, to get to know the pairs of teachings, which are then explained.

There are sixteen such pairs. The first pair is the noble truths of suffering and its origin, contrasted with suffering’s cessation and the path. Each subsequent pair follows the same pattern: the first item shows the arising of suffering, and the second, its cessation.

Apart from this general thematic pattern, the Sutta does not try to relate each item sequentially. Each item is simply said to be another way of contemplating the pairs. But it is noteworthy that seven items, namely three through nine, correspond exactly with the normal factors of dependent origination starting with ignorance and going up to grasping. The verses on grasping, moreover, speak of three further factors of dependent origination: continued existence (\textit{bhava}), rebirth (\textit{\textsanskrit{jāti}}), and suffering (\textit{dukkha}). That makes ten of the factors of dependent origination. Left out are “name and form” (\textit{\textsanskrit{nāmarūpa}}), which is mentioned later, in the fifteen items on deceptiveness; and the six senses, which appear in the final set of verses. Thus all the factors of dependent origination appear, albeit in different ways.

Moreover, several further items that do not appear in the standard list of twelve factors are nonetheless associated with dependent origination. The word “condition” (\textit{paccaya}) occurs no less than 22 times throughout the text. If we start at the beginning, the four noble truths themselves are integrated with dependent origination in several ways in the sutta. The second item, “attachment” (\textit{upadhi}) is similar in meaning to “grasping”, and is found in this sense in the context of dependent origination (SN 12.66:3.6). The tenth factor, “instigation” (\textit{\textsanskrit{ārambha}}), is not a standard technical term in this sense, and normally it is associated with right effort. The commentary says that it is “energy connected with action” (\textit{\textsanskrit{kammasampayuttavīriya}}), but this seems implausible, as when \textit{\textsanskrit{ārambha}} is used with \textit{viriya} it is invariably in a positive sense. Here, rather, it appears to have the sense of “originating kamma” as at AN 4.195:6.2, which would make it more or less a synonym of “choices” (\textit{\textsanskrit{saṅkhārā}}). The next factor is “sustenance” or “fuel” (\textit{\textsanskrit{āhāra}}), which occurs frequently in the context of dependent origination as another aspect of “grasping”. Following this, the topic is simply “dependence” (\textit{nissita}) itself. We could go on, but it seems clear that virtually all of the items are connected with dependent origination either directly or indirectly.

There is an important detail that, I believe, is key to understanding the construction of this sutta: numbers. In my essay introducing the \textsanskrit{Aṅguttaranikāya}, I pointed out that numbers are used in the Suttas consciously and with a freighted meaning. Here, the clue is in the title: it’s about contemplating twos. This is brought out in an obvious way in the contemplation of pairs of dhammas. But notice that the sutta is built on another set of pairs: prose and verse. So that means that each item consists of two sets of two, which is expressed as a fourfold matrix:

\begin{itemize}%
\item origin in prose%
\item cessation in prose%
\item origin in verse%
\item cessation in verse%
\end{itemize}

Reinforcing this, the very first item is the “four” noble truths, which is also divided into two sets of two. This shows that the text is thinking in terms of fitting multiples of two into each other. But if two twos make four, four fours make sixteen. And there are sixteen items in the whole sutta. This suggests that the overall text has been consciously planned out to reflect this set of doubly-expanded twos.

This use of multiplied sets of two and four is a characteristic feature of teachings on causality, signifying that it is an expansion of the four noble truths. The standard sequence of dependent origination has twelve items, which is doubled when treated in reverse. The treatment of causality in the Abhidhamma \textsanskrit{Paṭṭhāna} follows suit, with twenty-four distinct kinds of condition, signifying a further developed and expanded treatment of causality. The number sixteen serves a similar purpose here. These nested numerical sets are like a mandala, dividing and subdividing the same fundamental pattern to reveal a more granular detail.

As indicated by the existence of the \textsanskrit{Aṅguttaranikāya}, it was a standard practice to use numbers for organizing texts. This can be traced back to the very first discourse, the Dhammacakkappavattanasutta, which uses a similar scheme of two, four, eight, and twelve. There, however, the numerical structure is additive: a series of distinct doctrines with related numerical expressions, are taught in sequence: the two extremes, the four noble truths, the eightfold path, and the twelve modes. Here, the numerical structure is recursive: pairs multiply to fours, and fours multiply to sixteens. For comparison, I would point to the Dasuttarasutta (DN 34), which is built on a scheme of ten groups of ten. This is the final discourse in the \textsanskrit{Dīghanikāya} and is one of the latest Suttas in the canon. And compared with that, the numerical structure of the \textsanskrit{Dvayatānupassanāsutta} is both more complex and less explicit.

The underlying significance of this numerological system points to universality and order. Since the Buddha did away with any notion of a creator deity or controlling spirit or consciousness, he taught dependent origination as a way of showing that the word operates according to comprehensible principles and patterns. Yes, we experience terrible suffering, but suffering is not inexplicable or random: it follows rules. And those rules, like the laws of science, apply everywhere and at every time. And this means that we can implement an organized and effective response to suffering; namely, the path. The teaching (\textit{dhamma}) unfolds as a reflection of reality (\textit{dhamma}), which provides a guide for a living (\textit{dhamma}).

All this suggests that the \textsanskrit{Dvayatānupassanāsutta} is neither early nor an assemblage that has evolved over time. It was quite deliberately composed with a high degree of artfulness and care according to a single vision. It is unlikely that this was the Buddha, as he avoided such methods. His use of numerology is much more direct. Like the Dasuttarasutta, the \textsanskrit{Dvayatānupassanāsutta} is on the verge of being overly clever, more concerned with formalism than meaning. Such formalisms would soon come to dominate Buddhist philosophy in the Abhidhamma period, where the entire movement relied on the method of recursively multiplying sets of dhammas; although to be sure, the \textsanskrit{Dvayatānupassanāsutta} is far more restrained.

As early as 1907, La Vallée Poussin singled out the \textsanskrit{Dvayatānupassanāsutta} as the “most remarkable” of the suttas that appear to presage the Abhidhamma. \footnote{Read \textit{aggini-\textsanskrit{samaṁ} \textsanskrit{jalitaṁ}}, as at Snp 3.10:22.2 below. } He suggested that dependent origination itself was probably “only a recast of this primitive fragment of Abhidhamma”. His lead has been taken up by some scholars since, who have argued that the \textsanskrit{Dvayatānupassanāsutta} includes an incomplete list of the usual items of dependent origination, and further, that it is unsystematic. But the Sutta is, on the contrary, more elaborate than most of those on dependent origination. We can’t just reduce a text to a list of items. DN 14 \textsanskrit{Mahānidānasutta} is another example of a long and elaborate sutta on dependent origination that has fewer than the traditional twelve items. In any case, as I showed above, the \textsanskrit{Dvayatānupassanāsutta} does include virtually all of dependent origination, even if not all are explicitly mentioned as major items. And I think the numerological structure should finally put paid to this argument. It is not unsystematic but has been carefully and artfully constructed.

Thus I agree with Jayawickrama that the \textsanskrit{Dvayatānupassanāsutta} is late rather than early.

One final point. It is often said that insights in physics are driven by aesthetics as much as evidence. The truth, we feel, should be beautiful. And while the beauty of physics may be inscrutable to those like me who struggle to grasp the basic mathematics, it is both real and inspiring to those who understand. I believe the same can be said of the Abhidhamma. It may be cold and austere, but there is a driving sense of balance and symmetry that underlies the whole project. From this perspective, the \textsanskrit{Dvayatānupassanāsutta}, perhaps more than any other text, strives to harmonize poetic beauty with doctrinal formalism.

\section*{The Chapter of the Eights}

The “Chapter of the Eights” or \textsanskrit{Aṭṭhakavagga}, the fourth chapter of the \textsanskrit{Suttanipāta}, is a uniquely significant collection. Unlike the first three chapters, the poems here are quite unified in theme and language; but unlike the last chapter, they are distinct poems, not drawn together in a narrative.

There is a close parallel in the Chinese canon, \langlzh{義足經} (Taishō Vol. 4, No. 198), making this is the only chapter of the \textsanskrit{Suttanipāta} with a parallel outside of Pali. It has been translated to English by P.V. Bapat. \footnote{The commentary says that these two verses are absent from the old “Great Commentary”. Bodhi takes this as a sign they were a later addition. Norman thinks otherwise, arguing that these verses answer the question asked in the prose introduction. But that question has already been answered in the prose, so I don’t think this is a compelling reason. Norman also argues on the basis of metre, which I’m not competent to assess. On the whole, however, I tend to agree with Bodhi here. } In addition, fragments of a Gandhari version have been published, which confirm that the title was \textsanskrit{Arthapadasūtra}, the “Meaningful Verses” or “Beneficial Sayings”. Elsewhere, the form \textsanskrit{Arthavargīya} is also found. The existence of the chapter as a distinct work elsewhere confirms that it existed independently before being collected in the \textsanskrit{Suttanipāta}.

The poems of this chapter have a distinctive flavor and feel. They are blunt and direct, conveying the sense of an exchange between peers, rather than a work of conversion or persuasion. The main topics, especially the renunciation of sexuality and disputatious views, are familiar from elsewhere in the canon, but here they have a special emphasis. In addition, the language used is very specific, and there are several words and idioms used repeatedly in these poems that have a special meaning in context, a meaning that has not always been captured in previous translations. Much as I love this text, it has always felt somewhat opaque to me, as if there were a meaning just out of focus. It is only after making my translation, near the end of translating all the early Pali discourses, that I understood the reason for this. The language and terminology are used in a very specific way, which differs from the prose \textsanskrit{Nikāyas}, yet is highly consistent within itself. For this reason, I discuss some of the terms that, in my view, have led to this opacity of meaning.

Nonetheless, the uniqueness of the \textsanskrit{Aṭṭhakavagga} should not be over-stressed. There are no significant doctrinal differences from the major prose Nikayas, nor any reason to think the collection is any earlier than the prose texts. All of the themes of the \textsanskrit{Aṭṭhakavagga} can be found elsewhere. Differences are a matter of context and emphasis.

\subsection*{Intertextual references and parallels}

The \textsanskrit{Aṭṭhakavagga} is often mentioned alongside the \textsanskrit{Pārāyanavagga}, as both of them receive a canonical commentary in the Niddesa, and both are quoted by name elsewhere in the canon. But while the \textsanskrit{Pārāyanavagga} is quoted and referred to six times in the four Pali \textsanskrit{Nikāyas}, the \textsanskrit{Aṭṭhakavagga} is mentioned only once, in the \textsanskrit{Hāliddikānisutta} (SN 22.3). This sutta is taught by \textsanskrit{Mahākaccāna} in \textsanskrit{Avantī}, where he was staying “near Kuraraghara on Steep Mountain”. Now, \textsanskrit{Avantī} is geographically significant. It lies on the “southern road” at the edge of the “sixteen nations” of the heartland of early Buddhism. It was one of the earlier places to which Buddhism spread, probably originally by \textsanskrit{Mahākaccāna} himself. It’s nearly a thousand kilometres southwest of \textsanskrit{Sāvatthī} or about a month as the monk walks.

\textsanskrit{Mahākaccāna} responds to a question by the householder \textsanskrit{Hāliddikāni} regarding a verse from “The Questions of \textsanskrit{Māgandiya}” (Snp 4.11). \textsanskrit{Mahākaccāna}’s response is quite interpretive. While the verse itself appears to deal straightforwardly with one who leaves home, he treats it in terms of the five aggregates as a metaphor for being unattached to consciousness.

The only other early intertextual reference to the \textsanskrit{Aṭṭhakavagga} also involves \textsanskrit{Mahākaccāna} at the very same location in \textsanskrit{Avantī}. \textsanskrit{Mahākaccāna}’s attendant \textsanskrit{Soṇa} fulfilled his desire to ordain, but only after \textsanskrit{Mahākaccāna}, with great difficulty, managed to scrape together the required ten bhikkhus from here and there. \textsanskrit{Soṇa} wanted to see the Buddha and received \textsanskrit{Mahākaccāna}’s blessing, together with a request for the Buddha to relax five Vinaya rules in outlying districts. These included reducing the quorum of monks to perform a valid ordination from ten to five, and allowing shoes for the rough ground. On his arrival at \textsanskrit{Sāvatthī}, he was granted the honor of staying together with the Buddha in his hut. In the morning, \textsanskrit{Soṇa} was invited by the Buddha to recite in his presence, and he chose to recite the entire \textsanskrit{Aṭṭhakavagga}. The Buddha praised his chanting, spoke a verse in response, and went on to grant \textsanskrit{Mahākaccāna}’s reasonable requests.

It can hardly be a coincidence that the two intertextual references to the \textsanskrit{Aṭṭhakavagga} are both in conjunction with \textsanskrit{Mahākaccāna}. It seems this text was a favorite of his and was regularly taught to his students, including his lay students. Given the paucity of bhikkhus in the region, and the fact that both the stories revolve around renunciation of the home life, it seems reasonable to infer that he taught this text specifically to inspire renunciate zeal among the new Buddhist community of \textsanskrit{Avantī}. The topics of the \textsanskrit{Aṭṭhakavagga} also fit what we know of \textsanskrit{Mahākaccāna}. He is perhaps best known for his analysis of “proliferation” (\textit{\textsanskrit{papañca}}, MN 18), a topic prominent in the \textsanskrit{Aṭṭhakavagga} (Snp 4.11:13.4, Snp 4.14:2.1). More generally, his style, emphasizing a rigorous analysis of the six senses, complements the deconstruction of baseless speculation.

The Chinese text has been translated by Bapat. The suttas are substantially the same, although there is a slight difference in the sequence of the final seven suttas. More significantly, the poems are embedded in a narrative background. In the Pali, narrative backgrounds are found in the commentary. Generally speaking, it is normal for Buddhist verses to have a narrative background in prose, and these are sometimes included in the canon, and sometimes in the commentary. Typically, the Pali tradition is more conservative in this regard and tends to keep the prose narratives for the commentary, as in the Dhammapada or the \textsanskrit{Jātakas}. This is not always the case, of course, for in the \textsanskrit{Suttanipāta} itself there are several prose backgrounds, and a collection like the \textsanskrit{Udāna} has them throughout. Again, however, it is generally the case that those backgrounds that made it to the canon in Pali still retain the spare, elemental flavor of early Pali prose, while those in the commentary become more elaborate.

The narrative is typically more free and adaptable, and it is common to find different stories attached to the same verse. In this case, according to Bapat, seven of the sixteen background stories in the Arthapada Sutra are completely different from the Pali.

\subsection*{The meaning of “eight”}

The Pali word \textit{\textsanskrit{aṭṭhaka}} in the title of the \textsanskrit{Aṭṭhakavagga} is equivalent to the Sanskrit \textit{\textsanskrit{aṣṭaka}}, “group of eight”. Four of the Suttas of the \textsanskrit{Aṭṭhakavagga} include the phrase \textit{\textsanskrit{aṭṭhaka}} in their titles, and they do indeed consist of a group of eight verses each. This suggests that the chapter was named after these texts and that we should translate it as “The Chapter of the Eights”. It is, however, worth noting that no Pali commentary confirms this.

This conclusion is brought into question by the fact that when the title appears in the northern traditions, the reading \textit{artha} predominates in the sense of either “meaning” or “weal”. In addition to the Chinese translation of the chapter, it is referred to in many Vinaya texts as well as later treatises, the details of which are ably discussed by Jayawickrama. The reading “eight” is found only in a single place in the \textsanskrit{Mahāsaṅghika} Vinaya, although that text also has \textit{artha}.

Normally in Pali, the Sanskrit \textit{artha} is found as \textit{attha}, while \textit{\textsanskrit{aṣṭa}} is found as \textit{\textsanskrit{aṭṭha}}. It is, however, not uncommon to find the spelling \textit{\textsanskrit{aṭṭha}} for \textit{artha} as well, hence the confusion. Jayawickrama argues that \textit{artha} was the earlier reading, but in this, he is not followed by Norman or Bodhi. Neither author explains why they reject Jayawickrama’s argument, other than Bodhi noting that Jayawickrama himself elsewhere translated it as “eights”. I will step into the breach.

Jayawickrama presents a linguistic case for \textit{\textsanskrit{aṭṭhaka}} representing the Sanskrit \textit{arthaka}. But this merely allows for the possibility, and it remains the fact that this form is not elsewhere attested in Pali. He also argues the general importance of the concept of \textit{attha} as “weal” or “meaning”. But it’s not clear how such a general argument is relevant in naming the chapter. Indeed the argument could be inverted—since all Buddhist texts speak of the “weal”, why should this text be named for it? After all, \textit{attha}, though an extremely common word, occurs only four times in the \textsanskrit{Aṭṭhakavagga} and never in the sense of “weal” or “meaning”. It’s found in Snp 4.11:9.3, Snp 4.11:8.3, and Snp 4.11:9.3 in the sense of “topic”, and Snp 4.16:17.2 in the sense of “sake” or “goal”. None of these is significant enough to justify naming the whole chapter.

Jayawickrama regards it as “mere coincidence” that four of the suttas have eight verses, and speaks of the “weakness” of the Pali redactors for numerical organization. But as I have shown in my discussion of the \textsanskrit{Dvāyatānupassanasutta}, and at more length in my introduction to the \textsanskrit{Aṅguttara}, numbers were consciously employed to express the harmony and symmetry of the teachings. Jayawickrama’s dismissal is a good example of how scholars have too often overlooked this feature of the scriptures.

Delving deeper, we notice that the chapter of the “eights” consists of sixteen suttas. This is an example of the method of nested doubling, which is commonly found throughout the suttas. Take, for example, the four noble truths and the eightfold path. Each of these is nested in the other: the fourth truth is the eightfold path, and the first path factor is the four truths. This nested doubling shows that each aspect of the Dhamma is both a part of the greater whole, while also being a full expression of the wisdom of that whole. The fact that it is found in such fundamental teachings shows that it was used by the Buddha himself. The redactors, in organizing the scriptures, employed the same techniques that the Buddha had used when teaching the discourses.

That the number sixteen is significant is supported by the fact that the suttas of the \textsanskrit{Aṭṭhakavagga} are so enumerated in the \textsanskrit{Udāna} (Ud 5.6:18.1: \textit{\textsanskrit{soḷasa} \textsanskrit{aṭṭhakavaggikāni}}), which, incidentally, also confirms that the number of suttas was settled at an early date. The number sixteen is found also immediately preceding, as the number of pairs in the \textsanskrit{Dvāyatānupassanasutta}, and immediately following, in the “sixteen questions” that make up the main content of the \textsanskrit{Pārāyanavagga}. The fact that there are \emph{four} suttas of \emph{eight} verses, which lend their names to the Chapter of the Eights, consisting of \emph{sixteen} poems, should not be seen as a coincidence, but as a revealing insight into the minds of the redactors.

I would even go beyond this and ask whether all the chapters of the \textsanskrit{Suttanipāta} might have had the numerical pattern of expanding fours as their organizing principle. None have the standard ten suttas per vagga. The first and third chapters have twelve suttas. The \textsanskrit{Aṭṭhakavagga} has sixteen suttas, and so does the \textsanskrit{Pārāyanavagga} (leaving aside the late opening and closing passages). This leaves only the second chapter outside the pattern, with fourteen suttas. It doesn’t seem unlikely that this rather motley chapter has been expanded a little. I don’t want to push this argument too far, as it is only a general idea admitting many exceptions. Still, I do believe that an awareness of numerical patterns is a useful addition to the toolset of text analysis.

\subsection*{Characteristic terms and idioms}

The \textsanskrit{Aṭṭhakavagga}, and to a lesser extent the \textsanskrit{Pārāyanavagga}, employs several specialized Pali terms dealing with theories or views. These are not completely different from their normal senses throughout the canon, yet they are quite specific in this context and easily trip up the reader. Here I list these characteristic terms.

\begin{itemize}%
\item \textit{(no) upeti}, \textit{(an)upaya}: (does not) get involved \footnote{Bodhi and Norman follow the PTS reading \textit{sakkacca} here, but it seems uncharacteristic to me. Why mention they have just honored Indra? On the other hand, it’s common to express “a group with its leader”. }%
\item \textit{chanda}: preference, favored opinion \footnote{Notice the use of the humble \textit{bhavana} rather than the royal \textit{\textsanskrit{pasāda}}. There is an almost complete lack of royal language in this text, the sole exception being the \textit{antepura} “royal compound” in Snp 3.11:17.2. }%
\item \textit{kusala}: (one who claims to be) an expert \footnote{Here, as so often, “autumn” (per Bodhi and Norman) conveys quite the wrong impression. Autumn is the time of gathering clouds, \textit{sarada} is the time after the rains. }%
\item \textit{\textsanskrit{ñāṇa}}, \textit{\textsanskrit{jānati}}: notion; unjustified knowledge \footnote{Here the commentary glosses \textit{\textsanskrit{paṇḍu}} as \textit{ratta} (“red”), followed by both Norman (“pale red”) and Bodhi. But \textit{\textsanskrit{paṇḍu}} means “pale, white, cream, yellowish” and I can’t see anywhere in Pali or Sanskrit to suggest a meaning “red”. Given that it’s the standard color of a luxurious rug, perhaps it is due to a change in fashions; in the Buddha’s day luxury was cream-colored, but later it became red. }%
\item \textit{(pa)kappa}, \textit{(pa)kappeti}: formulation, a view created by imagination and speculation \footnote{Neither Bodhi’s “fortitude” nor Norman’s “peerless one” quite capture the force of \textit{asamadhura}. The implication is that, as the forger of the path, the Buddha carries a burden greater than any other. }%
\item \textit{vikappeti}: to judge, in the sense of think critically of someone or to justify oneself \footnote{Note the idiom \textit{parato \textsanskrit{ghosaṁ}} here. }%
\item \textit{purakkhata}, \textit{purekkharoti}: promote (one’s views); often paired with \textit{pakappeti}, “to make things up and promote them”. Those whose views are insecurely grounded feel the need to push them on others. \footnote{This is a succinct definition of \textit{buddha}, not an alternate thing that you might hear people say, per Bodhi and Norman. }%
\item \textit{pacceti}: to believe \footnote{Both Norman and Bodhi have “doctrine” here, while commentary is silent. I’m not aware of \textit{samaya} in the sense of “doctrine” in the early texts, nor is it listed in the senses of \textit{samaya} in the commentarial analysis in the \textsanskrit{Dhammasaṅgaṇī} commentary, \textsanskrit{Kāmāvacarakusala}. Given that the concern in previous lines is the notion of an awakened one, surely we have here an abbreviated \textit{abhisamaya} i.e. “breakthrough, enlightenment experience”. }%
\item \textit{(sam)(ug)\textsanskrit{gahīta}}: adopted; specifically, a view or belief that has been adopted \footnote{I find it curious that this epithet is used of both the Buddha and Asita. }%
\item \textit{(vi)nicchaya}: judgment; weighing one view against another \footnote{Possibly the only usage of “hope” in a spiritual sense in the EBTs. }%
\item \textit{nivesa}, \textit{\textsanskrit{niviṭṭha}}: dogma; a view that is held over-strongly \footnote{Bodhi accepts the commentarial gloss of \textit{pada} as \textit{\textsanskrit{paṭipadā}} here, but I find his reasoning unconvincing. Yes, the text speaks of practice, but this correlates to the first part of the answer. Here, at the end of the question, it relates to the end of the text, which speaks of the state of sagacity. }%
\item \textit{samatta}, \textit{\textsanskrit{samādāya}}: embraced, taken up (a belief or practice) \footnote{Note that, despite the etymology, \textit{\textsanskrit{uccāvaca}} seems to be used more in the sense of “diversity” rather than “high and low”; eg. the colors of a rainbow. Here all things are like tongues of flame, i.e. there are many different ones, but they are not treated as better or worse. }%
\item \textit{(\textsanskrit{diṭṭhi})\textsanskrit{paribbasāna}}: maintaining (a view); continuing to dwell on one’s own view as the best \footnote{As other instances of this line make clear, the referent of \textit{yattha} is \textit{\textsanskrit{kāmā}}, i.e. the pleasures of the senses, not the desire and greed (of the previous line). Bodhi gets this nuance right, Norman does not. }%
\end{itemize}

In addition, there are a couple of characteristic idioms that deserve discussion.

Perhaps the most striking phrases in the \textsanskrit{Aṭṭhakavagga} are those where the enlightened sage is described as neither taking up nor setting down. This appears, and is meant to appear, as a paradox. Normally, of course, we would expect that such a person would be beyond grasping, not that they would be beyond non-grasping. What it points to, though, is that the sage is no longer engaged in the process of grasping and letting go. It is because the enlightened sage has already let go of everything that they no longer need to let go of anything.

This must be handled carefully in translation because the Pali phrases sometimes use a past participle, which if rendered with grammatical literalness, implies a process that happened in the past: a sage has never grasped anything or let anything go. Obviously, this cannot be the case: what could it even mean?

The word used most commonly in this way is \textit{atta}. This is a past participle from the verb \textit{\textsanskrit{ādiyati}} (“to take up”) and is not the more common word \textit{\textsanskrit{attā}} (“self”). Rather confusingly, the Niddesa and commentary accept both explanations, which, while not linguistically justified, serves to illustrate the sense.

If we look at a typical case, the meaning becomes clear. In the \textsanskrit{Duṭṭhaṭṭhakasutta} we are first told that a confused person is continually “rejecting” (\textit{\textsanskrit{nirassatī}}) and “taking up” (\textit{\textsanskrit{ādiyatī}}) different teachings (Snp 4.3:6.4). Here the present active tense is used. Then, at the end of the sutta, the enlightened sage is said to have shaken off all views, and hence is no longer engaged in picking up and putting down (Snp 4.3:8.3: \textit{\textsanskrit{attā} \textsanskrit{nirattā} na hi tassa atthi}). The use of the past participle emphasizes that the whole process is behind them. The commentary glosses the past participles here with present participles (\textit{\textsanskrit{gahaṇaṁ} \textsanskrit{muñcanaṁ}}), confirming that this sense is intended. The Niddesa offers a slightly different interpretation, glossing \textit{atta} with another past participle (\textit{gahita}) and \textit{niratta} with a future passive participle (\textit{\textsanskrit{muñcitabba}}), thus yielding the sense, “there is nothing they have grasped or that they need to let go”. Thus the traditional commentaries agree, and the contextual usage supports, that while the linguistic phrasing appears paradoxical, the sense is not. The Buddha uses paradoxes to stimulate interest and inquiry, not to confuse a plain matter.

The same idea is elsewhere expressed using various terms and grammatical tenses. We have already seen the contrast between the present active tense and the past participle in the \textsanskrit{Duṭṭhaṭṭhakasutta}. In the \textsanskrit{Suddhaṭṭhakasutta} we find further examples. The unsullied one is said to be “letting go without creating anything new here” (\textit{\textsanskrit{attañjaho} nayidha \textsanskrit{pakubbamāno}}, Snp 4.4:3.4), expressed with present participles. On the other hand, the unenlightened “grab on and let go (like a monkey)” (\textit{\textsanskrit{uggahāyanti} nirassajanti}, Snp 4.4:4.3), using present active tense. The same grammatical form is employed in the \textsanskrit{Attadaṇḍasutta}, “they neither take nor reject” (\textit{\textsanskrit{nādeti} na \textsanskrit{nirassatī}}, Snp 4.15:20.4). Again, the enlightened one is “neither in love with passion nor besotted by dispassion” (\textit{na \textsanskrit{rāgarāgī} na \textsanskrit{virāgaratto}}, Snp 4.4:8.3), here using personal nouns to express the same idea. The same terms are also used in the present active sense in the \textsanskrit{Jarāsutta}, (\textit{na hi so rajjati no virajjati}, Snp 4.6:10.4). As another example, consider the \textsanskrit{Paramaṭṭhakasutta}, where we find “having let go what was picked up, one does not grasp” (\textit{\textsanskrit{attaṁ} \textsanskrit{pahāya} \textsanskrit{anupādiyāno}}, Snp 4.5:5.1). The \textsanskrit{Purābhedasutta} returns to the past participle forms when it says a sage has nothing “picked up or put down” (\textit{\textsanskrit{attā} \textsanskrit{vāpi} \textsanskrit{nirattā} \textsanskrit{vā}}, Snp 4.10:11.3). The \textsanskrit{Tuvaṭakasutta} has a similar phrasing (\textit{natthi \textsanskrit{attā} kuto \textsanskrit{nirattā} \textsanskrit{vā}}, Snp 4.14:5.4). The \textsanskrit{Purābhedasutta} has a further variation, leaning on an unusual sense of \textit{\textsanskrit{saddhā}}, “neither hungering nor growing dispassionate” (\textit{na saddho na virajjati}, Snp 4.10:6.4). Of a similar sense is the final line of the \textsanskrit{Mahābyūhasutta}, “not formulating, not abstaining, not longing” (\textit{na kappiyo \textsanskrit{nūparato} na patthiyo}, Snp 4.13:20.4).

Clearly, all of these variations express a similar idea, which is not defined by a specific terminology or grammar.

\subsection*{Meditation and the ways of knowing}

The second characteristic idiom is not so distinctive of the \textsanskrit{Aṭṭhakavagga}, yet is used in a somewhat specific way. In the suttas, we commonly find mention of what is “seen, heard, thought, and cognized” (\textit{\textsanskrit{diṭṭhaṁ} \textsanskrit{sutaṁ} \textsanskrit{mutaṁ} \textsanskrit{viññātaṁ}}). These are derived from the \textsanskrit{Upaniṣads}. Following Jayatillecke’s incisive analysis, it is clear that these are a shorthand for the kinds of ways that philosophies and beliefs are known \footnote{Bodhi has “inferno”, Norman “hell” but the normal sense of “abyss” better fits the metaphor of “crossing over”. } They may come from what has been “seen” as per the \textsanskrit{Suddhaṭṭhakasutta}, i.e. the sight of a holy person; “heard”, or less literally “learned”, through a teaching or the passing down of a scriptural tradition; “thought” in the case of rationalist philosophers; and “cognized” by meditators in states of profound immersion. Note that \textit{\textsanskrit{diṭṭha}} (“what has been seen”) is not the same as \textit{\textsanskrit{diṭṭhi}} (“a view or theory”). A “view” may arise from what is “seen”, or from any of the other ways of knowing.

The tradition, however, starting with the Abhidhamma, typically treats these as a mere summary of the six senses. This is linguistically implausible, as \textit{muta} indicates what has been “thought” rather than what has been “smelt, tasted, and touched”. Nevertheless, the older meaning is not unknown to the traditions, for the commentary to the \textsanskrit{Paramaṭṭhakasutta}, for example, explains “with regard to what has been seen, heard, or thought” as “views that have arisen constructed on these bases” (\textit{\textsanskrit{diṭṭhe} sute \textsanskrit{sīlavate} muteti etesu \textsanskrit{vatthūsu} \textsanskrit{uppannadiṭṭhisaṅkhāte}}).

These terms are used throughout the suttas in this sense. In the \textsanskrit{Aṭṭhakavagga}, however, they occur with two distinctive features.

The first rather remarkable feature is that the last item, the “cognized” is missing. This indicates that these portions of the \textsanskrit{Aṭṭhakavagga} are not concerned with meditative practitioners, but with those who have arrived at their beliefs through faith or reason.

In this respect, the \textsanskrit{Pārāyanavagga} differs. As it is a conversation with meditators, we usually find \textit{\textsanskrit{viññāta}} there as normal (Snp 5.17:3.4, Snp 5.9:3.1). There is one exception to this, but it proves the rule. The \textsanskrit{Nandamāṇavapucchā} finds the Buddha explaining how a sage is described, and repeating the same line in the same context as in the \textsanskrit{Māgaṇḍiyasutta} of the \textsanskrit{Aṭṭhakavagga}: \textit{Na \textsanskrit{diṭṭhiyā} na \textsanskrit{sutiyā} na \textsanskrit{ñāṇena}} (Snp 4.9:5.1 = Snp 5.8:2.1). This line is unique in that we find the form \textit{suti} to parallel \textit{\textsanskrit{diṭṭhi}}. \textit{Suti} is the noun form (“learning”) of the past participle \textit{suta} (what has been “learned”). It is a standard term for the brahmins (Sanskrit \textit{\textsanskrit{śruti}}), where it refers to the revealed Vedic texts that were “heard” from the gods, or more prosaically, passed down orally from time immemorial. Curiously this appears to be its only occurrence in early Pali. In context, it does not refer to the ways of knowing as such, but rather to the qualities of an individual who has come to possess “views” and “learning”. Thus I read the expression as an instrumental of relation and render “in terms of”.

Meditation is addressed only rarely in the \textsanskrit{Aṭṭhakavagga}. The word \textit{\textsanskrit{samādhi}} occurs once only, in the \textsanskrit{Tuvaṭakasutta} (Snp 4.14:7.4), while \textit{\textsanskrit{jhāna}} occurs only twice, once in the same sutta, and once in the \textsanskrit{Sāriputtasutta} (Snp 4.14:11.1, Snp 4.16:18.2. \textit{\textsanskrit{Jhāyati}} occurs at Snp 4.7:5.2, but there it means “brood”.) The \textsanskrit{Kalahavivādasutta} also deals with deep meditations, without using the standard terminology (Snp 4.11). Mindfulness is only mentioned in four suttas, again the \textsanskrit{Tuvaṭakasutta} and \textsanskrit{Sāriputtasutta}, and in addition the \textsanskrit{Kāmasutta} and \textsanskrit{Purābhedasutta} (Snp 4.1, Snp 4.10). These are the discourses that deal with the mendicant lifestyle, rather than those that deal with disputations and views.

The striking thing is that, while the “seen, heard, and thought” appear in various combinations about a dozen times in the \textsanskrit{Aṭṭhakavagga}, they do not occur in any of these suttas. There is no overlap between them and meditation.

The \textsanskrit{Māgaṇḍiyasutta}, however, might prove an exception to this, as there we find \textit{\textsanskrit{ñāṇa}} instead of \textit{muta} (Snp 4.9:5.1). The Niddesa explains \textit{\textsanskrit{ñāṇa}} here as knowledge of the ownership of deeds, knowledge in conformity with the (four noble) truths, knowledge of the (six) direct knowledges, and knowledge arising from (meditative) attainment (Mnd 9:32.8). Thus it assumes that we are speaking of meditative knowledges, and that \textit{\textsanskrit{ñāṇa}} here has the same scope as the normal \textit{\textsanskrit{viññāta}} (“the cognized”). This is possible, but it seems to me more likely that here \textit{\textsanskrit{ñāṇa}} is simply a synonym for \textit{muta}. Generally, the \textsanskrit{Aṭṭhakavagga} speaks of \textit{\textsanskrit{ñāṇa}} that has arisen from thought (Snp 4.3:2.4, Snp 4.5:4.2), or from seeing a pure person (Snp 4.4:1.3).

This is one of the reasons why the “views” spoken of are inadequate: they are based on inadequate means of knowing. These suttas are addressing people who are deeply convinced that their own powers of rationality forge a clear and incontrovertible path to the truth. Meditation helps unbind that attachment to rationality Of course, meditation itself has its own pitfalls, which are discussed at length in DN 1 and elsewhere.

The second distinctive feature is that these terms are often combined with “precepts” (\textit{\textsanskrit{sīla}}) and/or vows (\textit{vata}). For example, a perfected one would not speak in terms of “what has been seen, heard, thought, or precepts and vows” (\textit{\textsanskrit{diṭṭhe} sute \textsanskrit{sīlavate} mute \textsanskrit{vā}}, Snp 4.4:3.2). Such combinations would make little sense if the seen and heard were referring simply to sense experience. Rather, these are all examples of ways of coming to know philosophical or spiritual truths. Precepts and vows are adopted as behaviors that support the realization of the truth.

One final characteristic of the \textsanskrit{Aṭṭhakavagga} deserves a mention. In keeping with its direct, unsentimental tenor, the \textsanskrit{Aṭṭhakavagga}, and to a lesser extent the \textsanskrit{Pārāyanavagga}, has an unrelenting insistence on the necessity for self-reliance. In no less than six places we are told that it is not possible for one person to purify another (Snp 4.2:2.2, Snp 4.4:2.3, Snp 4.4:3.1 , Snp 4.6:10.3 , Snp 4.13:14.4 , Snp 4.14:5.2). As always, this is repeating an idea that is featured in the prose Suttas as well, but its prominence here is striking.

\subsection*{Sensual Pleasures}

The Buddhist \textsanskrit{Kāmasutta} could hardly be further from the notorious Sanskrit sutra of the same name (Snp 4.1). It is a blunt and unapologetic dismissal of sensual desires of all kinds, including, but not limited to, sexual pleasures.

The Pali (or Sanskrit) word \textit{\textsanskrit{kāma}}, usually rendered “sensual pleasures”, has a range of nuances, which this discourse nicely illustrates. The opening line has \textit{\textsanskrit{kāma}} twice; literally, “one has \textit{\textsanskrit{kāma}} for \textit{\textsanskrit{kāma}}”. This plays on the subjective and objective senses of the word: “one has sensual desire for sensual pleasure”. The objective sense is illustrated in the discourse by such things as “fields, lands and gold” and so on. Notice that in this objective sense, \textit{\textsanskrit{kāma}} may be singular (as in the first line) or plural, as here, whereas the subjective “sensual desire” is singular throughout.

Often, rather than illustrating specifics, the many kinds sensual titillation are described as the \textit{\textsanskrit{kāmaguṇa}}, which emphasizes the stimulating power of the pleasures of the senses. In this way, \textit{\textsanskrit{kāma}} also extends to the sense of “pleasure”, the enjoyment of the senses.

\textit{\textsanskrit{Kāma}} in the sense of “desire” is, in itself, neutral. While it most commonly described the unskillful infatuation with sensuality, it can also be \textit{\textsanskrit{dhammakāma}}, the desire to achieve the good, to practice the path.

Despite the Buddha’s disdain for sensual pleasures in general, he accepted a moderate enjoyment of sensual pleasures as a normal part of lay life and saw the gaining and sharing of sensual pleasures as one of the chief benefits of that life. For all their unsatisfactoriness, pleasures are still pleasures. It is tempting to infer from this that, in a strongly-worded discourse such as this, the Buddha was addressing renunciates. However, we do not know the audience, and it would be unwise to leap to this conclusion. The Buddha was not afraid to teach people what they need, rather than what they want. Later in the \textsanskrit{Aṭṭhakavagga}, we find the \textsanskrit{Māgaṇḍiyasutta} (Snp 4.9), where the Buddha addresses a layman in even stronger terms.

The sutta points out the transience of sensual pleasures, and how one in their thrall is weakened and vulnerable. But notice that the opening lines are phrased with “if”; they offer an argument. \emph{If} a person follows this road, \emph{then} these are the consequences. The Buddha always offered an alternative, hard as it may be to imagine. We have a choice, and that choice is non-attachment. This is nicely illustrated here with the simile of the leaky boat, which foreshadows the imagery of the flood and the far shore that recur throughout this chapter and the next.

\subsection*{Eight on the Cave}

We now begin the four poems of the “eights”. Each of these is comprised of eight verses, and given a title composed from \textit{\textsanskrit{aṭṭhaka}} “group of eight” and a word in the first line. In this case, that word is \textit{guha} “cave”. The \textsanskrit{Guhaṭṭhakasutta} continues in a similar vein to the \textsanskrit{Kāmasutta}, deploring the dangers of sensual attachments (Snp 4.2).

The metaphor of the cave invites comparison with the famous allegory of the cave in Plato’s Republic. Both Plato and the Buddha used the cave to illustrate the state of worldly ignorance. We are trapped in the darkness of delusion like a person in a cave. The Buddha did not develop an extended allegory like Plato, but it is hard not to see a connection between the ways of thought of these two great philosophers. For Plato, the cave represents the world of delusions in which we are trapped, a shadow world that is only a poor echo of the reality perceived by the awake. The Buddhist tradition, starting with the Niddesa, saw, rather, the cave as a metaphor for the body, which traps a person in bondage to sensual pleasures. It is, however, perhaps possible to read the \textsanskrit{Guhaṭṭhakasutta} more like Plato’s allegory, where the cave is the world of delusion of the senses rather than the body \emph{per se}.

Following this thread, another interesting difference between the two lies in the connotations of the idea of a cave. Plato was a worldly philosopher, engaged in the project of examining the ideal nation, who found his inspiration in vibrant discussion with the intelligentsia. For him, a cave was a return to a primitive way of being. The Buddha, on the other hand, found wisdom in seclusion, and it is rather striking that throughout the suttas, a cave is almost always presented in a positive light, as a quiet and secluded place for meditation. The Buddha himself is noted as meditating in caves, and they are one of the standard places he recommended. In India, a cave provides cool relief from the fierce heat of the day, and later generations of Buddhists up until the present have adopted the cave as a resort for meditation.

This irony is brought out in the opening verse, which says that one trapped in a cave is “far from seclusion”. It is speaking of those who have left the material traps of the world behind to meditate in a cave, but who remain trapped there by their own desires that tie them to the world they left behind. Thus we can see the metaphor of the cave as evoking a rich range of connotations.

In the second verse, the Buddha brings in an argument about time, one which is a crucial component of his perspective. Whereas it might be assumed that a spiritual path forsakes the pleasures of the present for the sake of future lives, the Buddha insisted that it is sensual pleasures that are bound to time, and the path of the Dhamma that yields its benefits in the present life. While it is true that we enjoy sensual pleasures in the present, they never provide true satisfaction. So we are always longing for more or fearing that we will lose what we have. This desire sets up the attachments that bind us to rebirth. Thus impelled by desire, the unenlightened are anxious in the present and fearful when contemplating their mortality. One who fully realizes the Dhamma, on the other hand, is freed from bondage to time. Such freedom, our sutta is careful to note, must be attained by oneself, for “one cannot free another” (Snp 4.2:2.2), an idea that becomes the central theme of Snp 4.4.

As with the \textsanskrit{Kāmasutta}, the \textsanskrit{Guhaṭṭhakasutta} employs a range of vivid and uncompromising imagery. The Buddha depicts the common lot of humanity as lost, floundering, ever yearning for that which can never satisfy. It is a disturbing vision, one that is only redeemed by the possibility of freedom.

\subsection*{Eight on Malice}

The \textsanskrit{Duṭṭhaṭṭhakasutta} shifts focus from sensuality to views, a particular temptation of the renunciant (Snp 4.3). The word \textit{\textsanskrit{duṭṭha}} is formally ambiguous in Pali, for it is represented by two distinct words in Sanskrit: \textit{\textsanskrit{dviṣṭa}} = “malicious” and \textit{\textsanskrit{duṣṭa}} = “corrupt”. Different translators have opted for one or other of these, but the usage in the Suttas, confirmed by the Niddesa, establishes “malicious” as the correct sense.

This sutta is the first of several that highlight the problematic nature of views and our tendency to become attached to them (Snp 4.4, Snp 4.5, Snp 4.8, Snp 4.9, Snp 4.12, Snp 4.13). The fundamental thesis of these suttas is laid out right at the start of the \textsanskrit{Duṭṭhaṭṭhakasutta}. It depicts two kinds of speakers: some malicious, some truthful. But when they debate, a sage remains aloof, which is why they have no “barrenness” (\textit{khila}). Now, “barrenness” is a technical term with a specific meaning. It refers to the hard-heartedness that can come up when there is a lack of love and forgiveness, especially for one’s spiritual companions. So this verse is contrasting the intellectual pursuit of winning an argument with a state of emotional balance and love. Underlying this contrast is another fundamental principle of the Dhamma: that rationality, while useful, does not bring us to the truth. The liberating vision of the truth arises only from \textit{\textsanskrit{samādhi}}, which is a deep state of emotional wholeness.

Right view is, of course, the very foundation of the path. Yet even attachment to right view can be a problem, as illustrated by the famous simile of the raft. In the spiritual life, in Buddhism no less than elsewhere, it is sadly common to find people who are deeply attached to the letter, or to their specific interpretation. For a certain kind of person, this becomes a dangerous obsession, which if left unchecked, can lead a person to rigidity and fundamentalism, and even to a departure from reason and sanity. It is no small thing to see a person lose their grip on reality as they pursue their obsession with views to their own detriment and that of the people around them.

The Buddha, as so often, preempts the findings of modern psychology by pointing out that what we take to be an objective rational argument is merely “led by preference, dogmatic in belief”. This is why it is not until profound emotional wholeness and balance has been reached through \textit{\textsanskrit{samādhi}} that deep insights will transform one’s beliefs and ideas into right view.

\subsection*{Eight on the Pure}

The \textsanskrit{Suddhaṭṭhakasutta} focuses on views and beliefs that have been adopted based on the sight of a holy person (Snp 4.4). It was, and remains, a common belief that the mere sight of a purified sage was sufficient to grant blessings, perhaps even enlightenment. The Buddha rejects this, arguing that it is not possible for one person to purify another.

The Buddha expresses a similar sentiment in, for example, SN 22.87:3.1. The monk Vakkali—whose devotion is mentioned at the end of the \textsanskrit{Pārāyanavagga}—longs for the sight of the Buddha, but the Buddha says, “what can this putrid body do for you?” He then explains that “one who sees the teaching sees me, one who sees me sees the teaching.”

The \textsanskrit{Suddhaṭṭhakasutta} employs the famous simile of the monkey, although not quite in the sense of “thinking of one thing and then the next”, but rather, of adopting one view or belief and then the next. It would seem that the spiritual supermarket is no new thing. While the sight of a pure being might bring inspiration, that alone is not enough to sustain a spiritual path. And underlying this is the egoistic notion that there is someone who observes that pure being; and the observer must be able to reliably discern who is pure and who is not.

Elsewhere the Buddha was rather scathing in his dismissal of those who, while themselves impure, presumed to judge who was pure (AN 6.44 = AN 10.75). Being unable to truly judge the purity of another, such a person will drift from one teacher to another, perpetually unsatisfied. One who sees no longer goes from one teacher to another seeking purity.

\subsection*{Eight on the Ultimate}

The \textsanskrit{Paramaṭṭhakasutta} (Snp 4.5) offers yet another perspective on the trap of dogmatic views, analyzing the problems that arise when one argues that one’s own belief is the “ultimate” (\textit{parama}). When someone falls into this trap, they are committed to judging all others as lesser, which embroils them in arguments. A view is established on limited grounds—“on what is seen, heard, thought, or on precepts and vows”—and it cannot transcend those limitations. Such a view can never be “ultimate”, for it remains a personal perspective. While the Buddhist view is, of course, regarded by Buddhists as “true”, this is not something to be argued over, but rather, to be relied on as a guide for liberation.

The adoption of a view is not disinterested but depends on one seeing some kind of advantage for oneself. This is one of the fundamental psychological mechanisms of a cult. Adherents see themselves as special, as chosen because they hold a view that is “better” than that of others. The deeper the underlying insecurities, the more fragile the view and the more strident its insistence. This is one of the reasons cults tend to gravitate towards outlandish and extreme views: the further adherents must depart from reality, the stronger their faith.

Any religious view, including Buddhism, is susceptible to these failings. The world is not short of those who would claim to judge their teaching as the best, or falling into the conceit of equality, asserting that all teachings are the same. Each of these is based on the same underlying conceit: that we sit above all these spiritual wisdoms and can survey and judge them.

One who has truly let go has no need to be led by precepts and vows, for the external forms of religious practice have already served their purpose. This is sometimes misconstrued to argue that one who is free has no need to adhere to precepts, but this is not what the Buddha is saying. The underlying impulses that would motivate someone to break precepts are no longer present, so they keep precepts because it is natural to them, not because they need to progress on the path.

\subsection*{Old Age}

In the blunt manner of the \textsanskrit{Aṭṭhakavagga}, the \textsanskrit{Jarāsutta} begins by stating that life is short, and few live longer than a century (Snp 4.6). Despite the title, the sutta mentions old age only in passing, and its main theme is, rather, death and the separation from all one holds dear. The Chinese title, translated by Bapat as “The Death of Both Old and Young” is more apt. Your loved ones are said to be like the visions in a dream when you wake up. This is, of course, a central theme of the Dhamma, which these verses present in a simple, pithy form.

The poem shifts perspective in the middle, apparently alluding to sages of the past who had left home to wander forth. The Buddhist mendicant should follow their example, being independent and unattached to families and loved ones.

The final verse presents perhaps the most distilled essence of the \textsanskrit{Aṭṭhakavagga} since each line contains one of the major themes of the text. Describing the realized sage, the first line says they do not “conceive”, which means they do not construct a view based on self; the second line is the seen, heard, and thought, which is the basis upon which they do not conceive; the third line says they do not want to be purified by another; the fourth, that they are neither desiring nor growing desireless.

\subsection*{With Tissametteyya}

The Tissametteyyasutta is just as blunt as the \textsanskrit{Jarāsutta}, on a different theme. The Buddha is questioned by the monk Tissametteyya on the dangers of indulgence in sex for a renunciate. There is a Tissametteyya in the \textsanskrit{Pārāyanavagga} (Snp 5.3) and another in the \textsanskrit{Apadāna} (Tha-ap 403), but there is nothing to say whether any or all of these are the same person. We can say, however, that none of these shares anything but the name with the prophesied Buddha Metteyya of the coming age.

Specifically, the text addresses the state that befalls a renunciate who gives way to sexual temptation. Abandoning their commitment to a higher purpose, they become scorned and depressed. They have given away everything for a transient, hollow pleasure, and fall into bitterness.

These reflections are designed to spur a renunciate to cherish their chosen path and find meaning in seclusion. The Buddha, however, makes a specific point of warning the ascetic against thinking themselves “better” because of their choices.

The sutta closes with the astute observation that, for all their pleasures, those who have chosen the way of sensuality envy the renunciate, free and unworried.

\subsection*{With \textsanskrit{Pasūra}}

The \textsanskrit{Pasūrasutta} returns to the theme of views and their inadequacy, centering the discussion in the context of the debate. The energetic contest of philosophical views was a distinctive feature of ancient India, and it would seem that it often devolved into point-scoring and competitiveness rather than a genuine search for truth.

The Buddha is addressing a certain \textsanskrit{Pasūra}, who does not seem to appear elsewhere. The Buddha rather sarcastically puns on his name, the base of which (\textit{\textsanskrit{sūra}}) means “hero, warrior”, comparing the debater with, not a genuine hero, but a drunken soldier looking for a fight.

Once again, the Buddha appears unconcerned with likeability. He’s dressing \textsanskrit{Pasūra} down, not trying to mollify him. Indeed, the sternness of his critique seems of a piece with his equally stern denunciation of the dangers of sensuality.

The sutta begins by speaking of those who dogmatically assert the purity of their own teachings. The third line says they expound their view depending on the “beautiful” (\textit{subha}); the same line also occurs at Snp 4.12:15.3. Elsewhere, \textit{subha} is used to describe a meditation attainment, raising the possibility that we are dealing with meditators who have arrived at their views based on the “beautiful illumination” of \textit{\textsanskrit{jhāna}}. On the other hand, the \textsanskrit{Pasūrasutta} gives no impression that we are dealing with a deep meditator, and it is perhaps more likely that this is just a term for the “beauty” of their own doctrine. In MN 79:9.3 the wanderer \textsanskrit{Sakuludāyī} propounds the doctrine of the “ultimate splendor” (\textit{paramo \textsanskrit{vaṇṇo}}), but when questioned as to what it is cannot explain. It would seem the “ultimate splendor” is merely a baseless theory, and perhaps the same is true here.

The psychology of the debater is dissected mercilessly. They rush to a debate where they call each other fools and boast of their expertise. But they are needy and nervous, so they get upset when they lose, and all excited when they win. The more they win, the greater grows their conceit, until they become addicted to the contest.

What the debater cannot deal with, however, is the one who refuses to join the contest. Who do they become when there is no one to dispute? All they understand is winning and losing, and when they meet someone for whom winning and losing is less than nothing, they finally have no rejoinder but silence.

\subsection*{With \textsanskrit{Māgaṇḍiya}}

If suttas were ranked by bluntness, the \textsanskrit{Māgaṇḍiyasutta} would surely win (Snp 4.9). Returning to the theme of the dangers of sensuality, the Buddha begins by reminding \textsanskrit{Māgaṇḍiya} of how he even rejected \textsanskrit{Māra}’s daughters. Still less is he tempted by, and I quote, “this thing full of piss and shit”, saying he wouldn’t even touch it with his foot.

The next verse, spoken by \textsanskrit{Māgaṇḍiya}, clarifies that the Buddha was referring to a woman, but the text says nothing further about her. The commentary, however, says she is \textsanskrit{Māgaṇḍiya}’s daughter, who he has offered to the Buddha in marriage; and in this, the parallels in the Arthapada and the \textsanskrit{Divyāvadāna} are in agreement.

The text and its commentary are clear that the “thing” that would not be touched is the “body”, rather than the woman. This important distinction is lost when translators, including Bodhi and Norman, render the verse so that the Buddha would not touch “her”.

A close look at how the pronouns are used shows that this reading is unlikely. Line three refers to the body with the neuter pronoun \textit{\textsanskrit{idaṁ}}. In the next line, we have the ambiguous pronoun \textit{\textsanskrit{naṁ}}, which could be either feminine or neuter, but it is surely likely to simply agree with the neuter in the previous line. This is supported by the commentarial gloss of \textit{nena}, which cannot be feminine. The Buddha is speaking of the body as disgusting, just as he spoke of his own body in the same way. It was the later traditions that normalized the objectification of the feminine body as an object of disgust. The suttas are very precise in this matter and do not treat the contemplated body as feminine.

The question of \textsanskrit{Māgaṇḍiya}’s identity is not a simple one. The \textsanskrit{Suttanipāta} itself reveals little. The \textsanskrit{Divyāvadāna}, as a collection of stories, gives an elaborate background which locates \textsanskrit{Māgaṇḍiya} in the Kuru country, a detail with which both the Arthapada and the Pali commentaries agree. It also depicts him as a wanderer. A wanderer called \textsanskrit{Māgaṇḍiya}, also in the Kuru country, features in MN 75. The Pali commentaries say he was the nephew of the \textsanskrit{Māgaṇḍiya} of the \textsanskrit{Suttanipāta}. He begins with a strong antipathy towards the Buddha, accusing him of being a “life-destroyer”, only to receive a rather severe dressing down on the topic of sensual pleasures. Ultimately, the sutta tells us, he was converted and became enlightened. To further complicate the situation, there is an isolated reference to an order of wanderers called \textsanskrit{Māgaṇḍikas} who were evidently followers of \textsanskrit{Māgaṇḍiya}, but about whom nothing else is known AN 5.298.

Given all this, it seems not unlikely that the two \textsanskrit{Māgaṇḍiyas}—both of them wanderers in the Kuru country with an abrasive attitude towards the Buddha, who were sternly lectured about sensuality—are the same person. If \textsanskrit{Māgaṇḍiya} were a wanderer, this would explain the Buddha’s uncharacteristically stern rebuke; a lay person is expected to enjoy sense pleasures, but the Buddha believed this was unbecoming of religious renunciates. His initial encounter in the \textsanskrit{Suttanipāta} was unsatisfying, and we note that neither the Pali nor the Arthapada records that he was converted. If he was nursing a grudge against the Buddha, this would explain his immediate antipathy when he heard the Buddha mentioned in MN 75. Ultimately, however, his defensiveness broke down and the truth of the Buddha’s words was undeniable.

After this dramatic opening, \textsanskrit{Māgaṇḍiya} asks, if the Buddha rejects even such a prize as this beautiful woman, what his views and practices are. The shift in tone is abrupt, and Jayawickrama suggests that the later verses may have been a separate poem, noting that the \textsanskrit{Divyāvadāna} version, after opening similarly, leaves out all the subsequent verses. However, the Arthapada does include them and moreover provides a more gradual transition by including two verses that discuss sensual desire before rejoining the Pali in shifting to a discussion on views. Perhaps this was the original form of the poem. On the other hand, the fact that the three extant witnesses of the text all differ considerably at this point suggests that there was an early confusion.

The Buddha makes an essential point when he explains that purity is not spoken of in terms of view, learning, or knowledge, or precepts and vows; but neither is it spoken of without these things. These are all normal dimensions of spiritual life, but in and of themselves they are not a sign of purity, nor does their practice guarantee anything. To cling to them as the ultimate is a mistake; to reject them because they are not the ultimate is equally a mistake. \textsanskrit{Māgaṇḍiya} objects, saying this is all too confusing. The Buddha argues that he is only confused because he is attached to the dogmas that he has adopted, getting caught up in proving the rightness of his own view.

From here the sutta pivots to the theme of the secluded sage, wandering aloof and unsullied by both sensuality and dogma.

\subsection*{Before the Breakup}

The \textsanskrit{Purābhedasutta} opens with a question about how one can be at peace (Snp 4.10). The Buddha replies that one is rid of craving “before the breakup” (of the body at death), giving the sutta its name. Here he returns to emphasize that the benefits of the Dhamma are to be experienced in the present life.

Avoiding the dense argumentation of the more contentious suttas, the sutta proceeds to eulogize the freed sage. In the process, it touches on most of the characteristic teachings of the \textsanskrit{Aṭṭhakavagga}, and might even serve as a summary of the whole chapter.

\subsection*{Quarrels and Disputes}

Presented as a dialogue with an unnamed interlocuter, the \textsanskrit{Kalahavivādasutta} offers some of the most complex and demanding philosophy of the \textsanskrit{Aṭṭhakavagga} (Snp 4.11). The questions and answers unfold deeper and deeper levels of the Dhamma, ultimately leading to a mysterious expression of profound transformation.

The opening of the sutta is conventional enough, especially in the context of the \textsanskrit{Aṭṭhakavagga}: why do we argue? But the manner of questioning is unusual. Typically in such cases, the questioner will either ask a single question and receive a single answer; or they will ask a series of distinct questions and receive distinct answers. Here, however, they present a cluster of related questions that dig into a particular point; and the answers, appropriately, tend to focus on the cluster as a whole.

The first verse, for example, asks about the origins of arguments, sorrows, stinginess, conceit, and slander. The Buddha says these all stem from attachment to what we hold dear; and moreover, that they reinforce each other.

Since the underlying problem turns out to be the things we hold dear, the next question probes into where they come from. The Buddha attributes it to desire, which is the ultimate source of all a person’s hopes and wishes.

The next verse asks about the origin of desire, which turns out to be pleasure and pain. By now, the sequence bears an unmistakable similarity to Dependent Origination. Here, however, the causal sequence pursues the origins of conflict, rather than the process of transmigration. This approach is not unique to the \textsanskrit{Kalahavivādasutta}, for we find a similar analysis in AN 9.23, which is integrated with the full dependent origination in DN 15.

This verse also introduces a concept that will grow more significant as we go on. The Buddha says we form judgments based on seeing the “appearance and disappearance of forms” (\textit{\textsanskrit{rūpesu} \textsanskrit{disvā} \textsanskrit{vibhavaṁ} \textsanskrit{bhavañca}}). The exact sense of these words is tricky. Normally the pair \textit{bhava} and \textit{vibhava} occurs in the four noble truths in the sense of the craving to “continue life in a state of existence”, or the craving to “annihilate oneself and end existence”. Here, however, they are said to arise not from craving but from contact and seem to refer to the manifestation or vanishing of perceptions, which, as we learn later on, especially refer to the forms perceived in deep meditation.

The questioner digs further, asking where contact comes from; and also, a bit randomly, about possessiveness. The latter is once more dependent on desire and provides a further point of similarity with the analysis of conflict in DN 15, the \textsanskrit{Mahānidānasutta} (excerpted at AN 9.23). As for contacts, the Buddha says that they spring from “name and form”. Now, normally in dependent origination, the six senses appear between contact and name and form; here they are missing. This once more connects us with the \textsanskrit{Mahānidānasutta}, which likewise omits the six senses.

This is no coincidence, for the explanation of this point in the \textsanskrit{Mahānidānasutta} is very relevant to the \textsanskrit{Kalahavivādasutta}. The \textsanskrit{Mahānidānasutta} gives a unique analysis of the relation between contact and name and form. It identifies two kinds of contact: “impingement contact” (\textit{\textsanskrit{paṭighasamphassa}}), which characterizes the physical realm of “form”, and “linguistic contact” (\textit{adhivacanasamphassa}), which characterizes the mental realm of “name”. While the term “impingement contact” does not occur elsewhere in the suttas, the cessation of “impingement perceptions” (\textit{\textsanskrit{paṭighasaññā}}) is part of the normal description of the process of attaining the formless states. When entering such a state, the “impingement contact” ceases.

The ideas are by now becoming quite abstruse, and the terse words of the verses by themselves border on incomprehensible. However, seen in the light of the broader treatment of deep meditation in the suttas, they start to make sense. The idea of “forms” and specifically, “vision of forms” is often associated with the experience of \textit{\textsanskrit{samādhi}}, specifically the first four jhanas, which for this reason came to be known as the “form” jhanas. Here the “forms” are the reflections or echoes of material properties that are seen by the meditator, such as lights or other visions. These “disappear” when the meditator enters the first of the “formless” attainments.

This passage should be read in conjunction with Snp 5.15, which similarly speaks of someone who “perceives the disappearance of form” (\textit{\textsanskrit{vibhūtarūpasaññissa}}) and who has “entirely given up the body” (\textit{\textsanskrit{sabbakāyappahāyino}}). There, the meaning is made clear in the next lines, which say they see “nothing at all” (\textit{natthi \textsanskrit{kiñcīti} passato}), which can only refer to the formless attainment known as the “dimension of nothingness”.

The interlocutor, understandably, wants to know how to practice to experience the ending of form, and also pleasure and pain. The Buddha does not directly answer the question about the disappearance of happiness and suffering, but in this context, it must be the fourth jhana, the immediate basis from which “forms vanish” and the formless attainments are reached.

As to the practice of one for whom forms disappear, the Buddha describes someone who is neither of normal perception, nor distorted perception (of deranged consciousness), nor without perception (ruling out both the obscure “non-percipient realm” as well as the meditative state of the cessation of perception and feeling). Finally, they are also not “percipient of the disappeared”, which must refer to the form that has disappeared in the formless attainments. One who is in such a state perceives what remains when form has vanished. But they are not yet in such a state. They have attained the fourth jhana and are in the process of shedding perceptions of form, which will lead to the formless attainments.

The final line of this verse emphasizes the psychological purpose of this path of practice. The diffusion of thought and desire into the world that creates our sense of self springs from perception; specifically, the diverse perceptions of enticing things. By proceeding into higher and higher states of meditation, such perceptions erode and fade away, undermining the tendency of the mind towards proliferation.

Our questioner is still not done. He wants to know whether this attainment (of the formless dimensions) is the ultimate, or whether there is something else. The states of meditation of which the Buddha is speaking are indeed deep and refined, and many spiritual practitioners who experience them assume that they are the final goal of the spiritual path. Such, it would seem, includes the Buddha’s former teachers, \textsanskrit{Ālāra} \textsanskrit{Kālāma} and Udaka \textsanskrit{Rāmaputta}, who were evidently Brahmanical sages of the \textsanskrit{Upaniṣadic} traditions. It would also include the brahmins of the \textsanskrit{Pārāyanavagga}, who frequently question the Buddha on related matters. Such sages were the elite, the highest peak of practitioners before the Buddha, and hence they are referred to as “astute” by the questioner.

The Buddha agrees that some astute folk do claim that this is the ultimate. By this, he is of course not endorsing their views, merely saying that it is a view that some hold. He contrasts this with a competing view, according to which there is nothing remaining (\textit{\textsanskrit{anupādisesa}}). Now, on the face of it, this sounds like \textsanskrit{Nibbāna}, which is described in the same way. Yet just below, the Buddha dismisses these states as “dependent”, which rules \textsanskrit{Nibbāna} out.

He is speaking of some other sectarian view according to which nothing is left. The key is, I believe, in the word “occasion” (\textit{samaya}). This appears to refer to the annihilation teaching that the individual self will disintegrate at the time of death. This is different from the Buddhist view, which is that the notion of a “self” is merely a convention to which we become attached through craving, distorted perceptions, and mental proliferation.

Thus the two contrasting positions are those of certain eternalists and annihilationists. Notice the subtle shade in the Buddha’s answer; he echoes the questioner’s respectful reference to eternalists as “astute” (\textit{\textsanskrit{paṇḍita}}), while the annihilationists merely “claim to be experts” (\textit{\textsanskrit{kusalā} \textsanskrit{vadānā}}).

In the final verse, the Buddha, at last, speaks of the state of true freedom. All these meditations and doctrines are dependent and hence remain within the cycle of transmigration. Bringing the text full circle, one who has undone the most fundamental roots from which disputes arise would never engage in disputes again.

\subsection*{The Shorter Discourse on Arrayed For Battle}

The \textsanskrit{Cūḷabyūhasutta} continues the theme of views and disputations (Snp 4.12). The title of this sutta and the next echo the military theme of the \textsanskrit{Pasūrasutta}, by implication comparing the disputing parties as armies arrayed for battle. This military imagery, however, is only in the titles and does not appear in the suttas themselves.

The sutta opens with an unnamed questioner, who, seeing that all the so-called experts disagree, asks whether any of them speak truly. The Buddha responds in his characteristic way, by showing that the disputants fall into a duality. If adherence to your own view makes you wise, then all are wise; and if the rejection of anothers’ view makes you a fool, then all are fools. This gets to the heart of much disputation to this day: some seek validation by adherence to an orthodoxy, others by being a controversialist. Neither of these positions has any relevance to a seeker of the truth. If adherence to one’s own view is the only standard of wisdom, then everyone else is a fool. The Buddha rejects this extreme position; he is, throughout the suttas, an analytical and critical inquirer, who accepts or rejects aspects of different philosophical systems based on reason and truth, not on whether they agree with his own view.

Each of the disputants has a different take on what is true and what is false, and the questioner wonders why they cannot agree. The Buddha responds that there is only one truth, not a second, and that if people understood this they would not argue. Here he rejects any subjectivist notion of truth, arguing that different interpretations show only a lack of understanding, driven by ego and attachment.

The questioner, unsatisfied, digs down further. I am not entirely happy with the logic of the next couple of verses, which depend on the reading of the third line. If we take \textit{\textsanskrit{sutāni}} as “(truths) learned (through oral tradition)”, the questioner is asking whether the different truths they talk about arise through adherence to tradition or through speculative reasoning. This makes sense, as these are two of the most prominent kinds of spiritual practitioner in the Buddha’s day. If, however, we read \textit{su \textsanskrit{tāni}} then \textit{su} is a question particle and \textit{\textsanskrit{tāni}} merely a pronoun referring to the truths, in which case the questioner is asking whether there really are two truths or whether people just think so.

The former reading, to my mind, connects back to the previous discussion more satisfactorily, whereas on the second reading they are just asking again a question that the Buddha just answered rather definitively. On the other hand, the second reading connects forward better to the subsequent discussion, as in the next verse the Buddha speaks to whether there really are two truths, not whether some people have “learned” from the oral tradition. Curiously, none of the ancient sources has a definitive reading on this; but the lack of a gloss in Niddesa and commentary more likely indicates that the term was considered rather inconsequential, which agrees with \textit{su \textsanskrit{tāni}}.

The Buddha’s response is also tricky, and here I depart from the tradition, which regards “permanence” as something that is grasped through mistaken perception. It seems to me the grammar of the lines is, rather, that there are not many permanent truths, apart from what is perceived as such. Or to flip it around, that there is one lasting truth, and people only perceive that there are many. Here, I believe \textit{nicca} does not refer to the distorted perception of permanence (\textit{\textsanskrit{niccasaññā}}), but rather to the “fixed regularity of the Dhamma” (\textit{\textsanskrit{dhammaniyāmatā}}). The Dhamma is lasting, not in the sense of a permanently existing thing, but in the sense that the four noble truths are always an accurate description of how the world is.

The remainder of the text develops the theme of how a view, once adopted as the highest, leads to contempt for those of a different view, and hence to conflict.

\subsection*{The Longer Discourse on Arrayed For Battle}

Sharing more than just themes with the preceding discourse, the \textsanskrit{Mahābyūhasutta}’s opening line echoes the term \textit{\textsanskrit{diṭṭhiparibbasāna}}, “(those who) maintain their own view” (Snp 4.13; cf. Snp 4.5:1.1). This linguistic sharing extends to the remainder of the chapter as well, as the \textsanskrit{Mahābyūhasutta} makes use of almost the special terms and usages characteristic of the \textsanskrit{Aṭṭhakavagga}. Nonetheless, while the discussion covers familiar ground, here the Buddha develops a number of distinct and interesting arguments.

Where the \textsanskrit{Cūḷabyūhasuttam} discusses the question of the diversity of views and the unity of the truth, here the focus shifts to the consequences of attachments to views. The questioner asks whether the proponent of a view might be praised or criticized for it, but for the Buddha both of these are inadequate reasons to argue. Views are far more consequential than that. Adherence to wrong view is a commitment to a delusional and harmful way of seeing the world that leads to suffering in future lives.

The Buddha’s opening verse presents the alternative to disputation with a striking image. He speaks of the “sanctuary” (\textit{khema}), a metaphor whose roots go back to the origins of the Indo-European cultures in the pastoral nomads of the Pontic–Caspian steppes. Their hard days wandering the grasslands with cattle would be rewarded with a glimpse of a “sanctuary”, an oasis promising water, rest, and safety. Such is the promise of letting go when surrounded by those embroiled in disputes. The imagery of wandering is revisited in the fifth verse, where the Buddha compares someone who has fallen from the vows to a traveler who loses their caravan.

The Buddha emphasizes that the wise do not get involved because they do not rest their ideas in the “seen and the heard”. Once again, this emphasizes the transmission of religious or philosophical understanding via the vision of a holy one or the learning of scripture, not the understanding of sense experience, which is, of course, something all have in common and is not divisive.

The reference to those who “champion ethics” (\textit{\textsanskrit{sīluttma}}) is reminiscent of the Brahmajalasutta, where the Buddha criticizes those who take ethics to be the highest. While the Buddha always spoke of ethics as the foundation of the spiritual life, he was consistently critical of those who regarded adherence to mere norms of behavior as the essence of purity.

In addition to a general critique of attachment to views based on hearsay, the Buddha alludes to several more specific ideologies and practices that are found elsewhere in the canon. Some follow the notion of \textit{\textsanskrit{jigucchā}}, which I translate clumsily as “(mortification) in disgust of sin”. This refers to the self-mortifiers, who believed that by torturing their bodies they could free themselves from the accumulated suffering due to their past deeds. I use the word “sin”, which otherwise is best avoided in Buddhist contexts because it implies an inherent burden of suffering due to actions in the far past. The term “heading upstream” is clarified in MN 102:11.3, where it refers to those who look for salvation “upstream” in a future life. Here they moan their way through their painful self-mortification in the belief that it will all be worth it when they find bliss in the next life. Others rely on “prayer” (\textit{jappa}), muttering holy mantras in the hope of redemption.

The twelfth verse is a difficult one, and it is parsed in various ways by translators. But I think it is making an interesting argument. Spiritual teachers often establish the validity of their doctrine by reference to the path that they have traveled (\textit{\textsanskrit{sakāyanāni}}). They have learned hard-won wisdom through irrefutable experience, and the doctrine they teach should be revered as an expression of that truth (\textit{\textsanskrit{saddhammapūjā}}). Yet the fact remains that they each teach different and quite incompatible things. So if the truth was based on the validity (\textit{\textsanskrit{tathiyā}}) of the teacher’s journey, there must be many different truths. Since we have already established that there are not, in fact, many different truths, the testimony of a teacher’s experience cannot be a guide to the truth. For the Buddha, the testimony of his experience is not, in and of itself, a proof of the reality of his teaching; it is merely an example to set students on the path to finding the truth for themselves.

Some claim to see the truth for themselves, perhaps even based on a “beautiful” vision in some state of \textit{\textsanskrit{samādhi}} (Snp 4.13:16.3), yet they still fall back on views (Snp 4:13.14.1). The Buddha argues that if they had really seen, views would have no purpose for them. Even an accurate direct insight into reality is still limited; we see what we have seen, and no more (Snp 4.13:15.1). Again, this echoes certain meditators of the Brahmajalasutta (DN 1) or, say, the \textsanskrit{Mahākammavibhaṅgasutta} (MN 136), who attain jhana and accurate insight into certain aspects of the truth, nonetheless see only partially, and still fall into wrong view by taking their partial vision to be complete. It is “not easy to educate” (\textit{na hi \textsanskrit{subbināyo}}) such a person who already “knows” the truth.

The peaceful sage has no need for views, so they take no side among the factions when disputes arise.

\subsection*{Speedy}

The \textsanskrit{Tuvaṭakasutta} opens with a questioner addressing the Buddha with the exalted title, “Kinsman of the Sun”, an allusion to his supposed descent from the Solar lineage of Indic royalty, as well of course for his spiritual brilliance. (Snp 4.14). This lets us know that the question is that of a seeker who wishes to follow the Buddha’s path, not that of a skeptic or disputant. He first asks the Buddha to speak of how a mendicant becomes free of all attachments. The Buddha’s response focuses on the now-familiar theme of non-attachment to views and conceit. The second question concerns the mendicant’s way of life that leads to such letting go. Thus this sutta combines two of the great themes of the \textsanskrit{Aṭṭhakavagga}: non-attachment to views, and the renunciant life.

The opening verse of the Buddha calls back to the \textsanskrit{Kalahavivādasutta}, where we learned that “concepts of identity due to proliferation spring from perception” (Snp 4.11:13.4). Here the Buddha says that the fundamental conceit that “I am the thinker” lies at the root of such “concepts of identity due to proliferation”. As with virtually all the occurrences of this subtle idea, the exact parsing of the phrase is difficult and must be done in light of the more explicit, although still far from obvious, prose passages, especially the \textsanskrit{Madhupiṇḍikasutta} (MN 18).

Here the word \textit{\textsanskrit{mantā}} has been variously interpreted in both grammar and meaning. The root sense is to think, and it carries senses of secret wisdom or words of guidance. Niddesa followed by the commentary takes \textit{\textsanskrit{mantā}} to be a noun meaning wisdom and explains the form as a truncated instrumental. This gives rise to a translation like that of Bodhi, “\textbf{By reflection}, he should stop … ”. Norman, rather, takes \textit{\textsanskrit{mantā}} as the nominative of an agent noun \textit{mantar} (“thinker”), and renders, “\textbf{Being a thinker}, he would put a stop …”. \textsanskrit{Ñāṇadīpa} reads it similarly, translating: “the \textbf{deep thinker} should put an end to.” I agree that it is in all probability an agent noun, but I construe it as “They would cut off the idea ‘I am the thinker’…” (\textit{\textsanskrit{mantā} \textsanskrit{asmīti}}). This aligns it with other phrases that deal with conceit, such as \textit{\textsanskrit{seyyohamasmīti}}, “I am better”.

The idea is that one who seeks true wisdom must avoid the trap of identifying with their mere conceptual understanding, which is, of course, an overriding theme of the \textsanskrit{Aṭṭhakavagga}. The next verse expands this idea, warning them not to become proud of what they know, avoiding the trap of thinking in terms of “better”, “worse”, or “equal”. One who thinks in this way is always judging and comparing against others and is never free.

The fourth verse contains another difficult translation, due to a confluence of ambiguous words, variant readings, and unusual usages. I find the renderings by Bodhi, Norman, and \textsanskrit{Ñāṇadīpa} to be near-incomprehensible. Up until now the Buddha has been speaking of avoiding conceited comparisons and judgements that arise because of limited knowledge, and I think this verse develops that theme. Thus I take the opening word as \textit{\textsanskrit{puṭṭho}} “questioned”, rather than the variant reading \textit{\textsanskrit{phuṭṭho}} “contacted”, as I do not see any reason why the topic should shift to sense stimulation. We may then take \textit{\textsanskrit{anekarūpehi}} in its normal sense of “in many ways”, rather than taking \textit{\textsanskrit{rūpa}} to mean “form”. In the next line we have \textit{\textsanskrit{vikappayaṁ}}, which is a special \textsanskrit{Aṭṭhakavagga} term having the sense of “judging” or “justifying” someone in relation to others. This yields a satisfying sense:

\begin{verse}%
When questioned in many ways,\\

they wouldn’t keep justifying themselves.

%
\end{verse}

I think the overall sense of the verse is the same as AN 9.14:11.11: “It’s good that you answered each question. But don’t get conceited because of that.” (\textit{\textsanskrit{Sādhu} kho \textsanskrit{tvaṁ}, samiddhi, \textsanskrit{puṭṭho} \textsanskrit{puṭṭho} vissajjesi, tena ca \textsanskrit{mā} \textsanskrit{maññī}}).

The Buddha concludes this portion of the poem with the striking image of a mendicant who remains unstirred like the “middle of the ocean”. It is an ambiguous metaphor since the “middle” could refer to either the horizontal or vertical dimension. The commentary explains \textit{majjhe} “in the middle” in both ways: firstly as between top and bottom layers, or alternatively as in-between the mountains (i.e. land masses). In the Buddha’s day, deep ocean voyages were rare and somewhat legendary, and there is no evidence that the Buddha himself ever saw the ocean, unless his use of oceanic images is that evidence. This image may refer to a folk belief that the center of the ocean was void of waves; after all, we still call the largest ocean the “Pacific” due to a similar misapprehension. But I think it refers to the vertical dimension. The turbulence of the ocean surface conceals a surprising stillness beneath.

The last line of this verse echoes the oceanic metaphor. The mendicant is said to have no \textit{ussada}, which literally refers to a “prominence” or “swelling up” like a wave. As a psychological word, \textit{ussada} means “pride”, and I would normally translate it as such. The English word “pride” also harks back to the sense of “prominence”, but for us the metaphor has become almost completely worn away, remaining only in the occasional usage of “proud” to mean “a raised-up area”. The usage here suggests that in Pali the metaphorical connection between the ideas of “pride” and “swelling” in the word \textit{ussada} has not been lost. In modern English, the metaphorical sense of “pride” as “prominence” or “swelling” has so receded that we must recreate it with tautological idioms like “swell with pride” or “puffed up with pride”. I translate it here as “swell with pride”.

The questioner now turns to the path: how does one practice to reach such a state? In particular, he asks about the “monastic code” (\textit{\textsanskrit{pātimokkha}}) and “immersion” (\textit{\textsanskrit{samādhi}}). Thus we can take the sutta as a whole to encompass the three great divisions of the Buddha’s path: the first section dealt with wisdom, and now we discuss ethics (\textit{\textsanskrit{sīla}}) and meditation (\textit{\textsanskrit{samādhi}}). The Buddha barely touches on \textit{\textsanskrit{samādhi}} in his response, focusing instead on a description of the restrained and meditative lifestyle of the ideal mendicant.

As one would expect in poetry, this does not outline the literal rules of the monastic code. Nonetheless, it alludes in many ways to the same ideas that the monastic code formalizes as rules. Take, for example, the admonition in Snp 4.14:10.1 to not store edibles or clothing. This is codified in multiple Vinaya rules, such as Nissaggiya \textsanskrit{Pācittiyas} 1 and 21, which prohibit keeping an extra robe or bowl for more than ten days; Nissaggiya \textsanskrit{Pācittiya} 23, which prohibits storing certain medicinal foods longer than seven days; and \textsanskrit{Pācittiya} 38, which prohibits eating most food if it has been stored overnight. Or take Snp 4.14:10.1, which says they would not let their eyes wander in the village, a behavior which is covered by Sekhiyas 7 and 8. The injunction to avoid buying and selling is found in similar terms in Nissaggiya \textsanskrit{Pācittiya} 20.

While the Vinaya, however, is a legal system that aims to keep the worst behavior within bounds, here the Buddha is presenting a lifestyle worth aspiring to. What is front and center is not the details of what is and is not allowed, but the mental purity and freedom that such a life offers.

The Buddha says that his followers would not get involved in various of what elsewhere are called the “low arts” or “useless knowledges” (\textit{\textsanskrit{tiracchānavijja}}). In the Vinaya, these are prohibited for monks in Kd 15, and also addressed in Nuns’ \textsanskrit{Pācittiyas} 49 and 50. But the most famous discussion of them occurs in the first chapter of the \textsanskrit{Dīghanikāya}, where several suttas beginning with the Brahmajalasutta contain an extensive list. These include divining dreams, astrology, reading omens, and the like. Despite the very clear position of the Buddha, these are all practices that are commonly seen in the Buddhist \textsanskrit{Saṅgha} today.

A couple of the details here are worth noting. The passage includes the only canonical reference to the collection of texts known as the Arthava (Pali: \textit{\textsanskrit{āthabbaṇa}}), which later became known as the fourth Veda. It is explained in the Niddesa as essentially black magic, the casting of harmful spells on enemies. The Atharvaveda as it exists today deals with folk rites pertaining to home life, only a portion of which aim at doing harm, while others are for healing, invoking spirits, long life, and so on. This does not mean that the Niddesa is incorrect, as it is likely the term evolved towards a more positive meaning over time.

One of the practices that a mendicant should avoid is \textit{\textsanskrit{gabbhakaraṇaṁ}}, “treating a fetus”. \textit{\textsanskrit{Karaṇa}} is rendered literally as “make” by most translators: Norman (“the art of impregnation”); \textsanskrit{Ñāṇadīpa} (“causing of conception”); and Bodhi (“making women fertile”). Presumably, a ritual invocation or medicinal remedy is imagined, rather than the more normal means of causing pregnancy. Niddesa’s interesting and quite different explanation is omitted by Bodhi, so I translate it here.

\begin{quotation}%
“Treating a fetus” means “they sustain a fetus”. There are two reasons a fetus does not survive: because of germs (\textit{\textsanskrit{pāṇaka}}), or because of a disturbance of winds (\textit{\textsanskrit{vātakuppa}}). They give medicine to ward off germs and the disturbance of winds.

%
\end{quotation}

This gives a glimpse into the medical understanding of pregnancy at the time. We have come a long way, but it is worth bearing in mind that this offers a purely medical description of the problems and solutions, rather than a magical invocation. The presence of countless tiny “creatures” (\textit{\textsanskrit{pāṇaka}}) inhabiting the body and causing illness is a precursor to the modern theories of bacteria. The concept of a “disturbance of winds” is likewise not as removed from modern understanding as it might seem, since it probably refers to various forms of seizure and other afflictions that were ascribed to physical imbalances of energy.

The phrase is evidently related to the term \textit{\textsanskrit{viruddhagabbhakaraṇa}} (DN 1:1.26.2), which is translated by Rhys Davids as “using charms to procure abortion”, and in exactly the opposite sense by Bodhi as “rejuvenating the fetuses of abortive women”. While the commentary on the \textsanskrit{Tuvaṭakasutta} says nothing on \textit{\textsanskrit{gabbhakaraṇa}}, the commentary on DN 1 gives quite an extensive explanation.

\begin{quotation}%
“Treatment of impacted fetuses” means treatment of impacted, stuck, non-surviving, dead fetuses. Further, it means giving medicine, etc., so that it is not lost. For a fetus may be lost by three means: wind, germs, or karma. When being lost by winds, a cooling and calming drink is given as medicine. When being lost by germs, a remedy against germs is given. But when being lost by karma, even the Buddha himself is not able to prevent it.

%
\end{quotation}

This translation is similar to that of the Niddesa, and it rules out the interpretations of both Rhys Davids and Bodhi. One detail in which it differs is that it adds karma to the list of causes, which is conspicuously absent from the Niddesa.

While it is true that the commentary refers to “dead” fetuses, the present participles used in describing treatment make it clear that it was a medical intervention in an ongoing crisis while giving birth, not a magic spell of rejuvenation. Since \textit{viruddha} means “obstructed” and the gloss \textit{\textsanskrit{vilīna}} means “stuck”, the treatment is for a difficult birth due to an “impacted” fetus. It would seem that in the not unusual case when delivery was not going smoothly, they would call on the services of a monk or other holy personage to somehow ease the passage. Despite the tragedy and pain of the travail, however, a mendicant should not be serving as a doctor.

\subsection*{Taking Up Arms}

The \textsanskrit{Attadaṇḍasutta} is another of the formally divided texts of the \textsanskrit{Suttanipāta} (Snp 4.15). Like the \textsanskrit{Nālakasutta}, the first portion deals with the early life of the Buddha, and the latter portion with the training of a mendicant. Unlike the \textsanskrit{Nālakasutta}, however, the \textsanskrit{Attadaṇḍasutta} throughout has all the hallmarks of an early text. It offers a strikingly different perspective on the Buddha’s going-forth. Rather than being a purely personal existential crisis, here the decision to go forth is prompted by war.

The texts of the \textsanskrit{Suttanipāta} offer a coherent account of the early stages of the Buddha’s life. The \textsanskrit{Nālakasutta} tells of his birth. The \textsanskrit{Attadaṇḍasutta} tells of his motivations for going forth. The \textsanskrit{Pabbajjāsutta} tells us what he did after going forth (Snp 3.1). And the \textsanskrit{Padhānasutta} tells of his struggles as he practiced for enlightenment (Snp 3.2). The time after the enlightenment is not told, but that story is picked up in the \textsanskrit{Udāna}. These two collections taken together offer the earliest version of the Buddha’s life that is told primarily in verse. And as such, they are the immediate precursors to the versified Buddha legend as found in the later books of the \textsanskrit{Khuddhakanikāya} such as the \textsanskrit{Buddhavaṁsa}.

Non-violence has always been central to the Buddhist code of personal ethics. But the relation to war is more complex, and not discussed in detail by the Buddha. Modern ethics, drawing from a long Christian tradition, teaches us that there is such a thing as a “just war”, a cause whose moral clarity outweighs even the endless horrors of war. Certainly, the Buddha never articulated such a notion.

Rather, the moralities of war should be seen through the fourfold lens of the kinds of karma: bright deeds with bright results, dark deeds with dark results, bright and dark deeds with bright and dark results, and neither dark nor bright deeds with neither dark nor bright results, which lead to the ending of deeds (AN 4.232). The decision to go to war, to inflict pain and suffering on innocent people in the name of pride and rapacious greed, is an expression of the darkest evil that humanity is capable of. But what of defending one’s country? To protect one’s nation and loved ones is surely a bright deed; but in doing so, it means killing enemy soldiers, who are human beings. So this is mixed karma, both bright and dark. And this is the choice that the young Siddhattha would have been facing had he continued as one of the nobility of the Sakyan people. This sutta tells of how he made a different choice, undertaking the path that leads to the ending of karma.

The first line issues a challenge to the way we have been conditioned to think about war: “Peril stems from those who take up arms”. We are told that arming oneself is for protection, but in reality, it is those with arms who create wars. This is not to make a “both sides” argument; there is a genuine difference between those who idolize weapons and violence and those who take up arms solely to defend themselves. You can see the difference in times of peace: the wise are always seeking a way to disarm, while fools seize on any justification for more.

The Buddha calls his audience to look at people in conflict. This is a necessary message for Buddhist practitioners today: don’t look away. It is important that we see and understand the true nature and cost of war, lest we be complicit in the next one.

The Buddha was pointing to actual war, not making a merely theoretical observation. There must be at least two: one he saw long ago when still living in Kapilavatthu as the young Siddhattha, and one that prompted the teaching of this sutta. As to what they are, it is hard to say.

There were several conflicts in the Buddha’s lifetime. Prominent among them was the series of wars between Pasenadi of Kosala and \textsanskrit{Ajātasattu} of Magadha over Benares (SN 3.14, SN 3.15). Kosala had ruled Benares for some time, but \textsanskrit{Ajātasattu}, seeking control of the trade routes on the Ganges, tried to invade it. He was initially successful, but Pasenadi rallied, routing \textsanskrit{Ajātasattu}’s forces and capturing the king. In this case, Pasenadi behaved honorably and released his nephew \textsanskrit{Ajātasattu} after dismantling his army. But it was ultimately for nought, as there was no stopping the expansion of the Magadhan empire, which soon after retook Benares.

To my mind, this war is a good candidate for the one referenced in the \textsanskrit{Attadaṇḍasutta}. It was in the heartland of Buddhism, and it involved the two most prominent kings, both of whom the Buddha knew personally, and we know the Buddha commented on it.

The traditions, however, say that it was a more personal story, involving the Sakyan realm. The Pali commentary identifies the war with the famous conflict between the Sakyans and Koliyans over water supplies. However, this seems like an unlikely candidate, as the conflict is not attested in canonical sources, and according to the commentarial accounts, it was stopped before there was any violence.

The Arthapada gives a different background. This is the conflict between the Sakyans and Pasenadi’s son, \textsanskrit{Viḍūḍabha}, sparked by \textsanskrit{Viḍūḍabha}’s resentment over being called the “son of a slave girl” by the Sakyans. His wounded pride ultimately lead to him invading and utterly wiping out the Sakyan nation. But if this story has any truth to it, it cannot have happened during the Buddha’s lifetime, since according to the canonical source in DN 16, the Sakyans are said to have built a monument to honor the Buddha’s relics after he died.

Both of these accounts follow the tendency of the traditions to turn everything in the Buddha’s life into a family drama. But life is not a soap opera. Wars happen all the time, and they are tragic and devastating even if they do not involve your family. The reality is that the Buddha spent relatively little time as an adult in the Sakyan realm, and if he is calling his community to observe a war, it is more likely that it involved the great powers in the regions where he lived.

The earlier conflict, however, the one that he relates seeing as a young man, would presumably have involved his own people. Neither the Pali commentary nor the Arthapada, however, identify this. We know from the \textsanskrit{Suttanipāta} itself that the Sakyan people were subject to the Kosalans (Snp 3.1:18.4, confirmed at DN 27:8.3). I am not aware of any accounts of how this situation came to be, but it is safe to assume that the prideful Sakyans would not have submitted to Kosala without a fight. Indeed, the war with \textsanskrit{Viḍūḍabha} can be seen as a resurgence of that ancient conflict; such resentments echo down through generations. This is all, of course, purely speculative, but it does seem reasonable to suppose that the young Siddhattha witnessed conflict between the Sakyans and the Kosalans. If this was indeed one of the reasons for him to go forth, perhaps it explains why he mentioned the Sakyan’s state of submission to Kosala in the \textsanskrit{Pabbajjāsutta}, whose events happened immediately after this.

Regardless of the details of the war, it left the young Siddhattha deeply moved. It triggered not just a revulsion to the horror of war, but an existential crisis at the nature of human life. Humanity was floundering, “like a fish in a little puddle”, echoing imagery from the \textsanskrit{Guhaṭṭhakasutta} (Snp 4.2:6.1). The imagery is subtle foreshadowing, for in the \textsanskrit{Guhaṭṭhakasutta} it describes people who are lost in their attachments to sensual pleasures, and in the \textsanskrit{Mahādukkhakkhandhasutta} (MN 13) the Buddha attributes the origins of conflict to humanity’s inability to curb its insatiable desires. But we have not reached that point in the \textsanskrit{Attadaṇḍasutta} just yet.

When the Buddha speaks of how “fear came upon him” when he saw populations locked in conflict, he was not speaking merely of the immediate threat of war itself, terrible as that is. The deeper problem is that it seemed as if all were trapped in a cycle and no one could envisage any escape. Each line unfolds in striking and disturbing imagery, “the world around was hollow, all directions were in turmoil”.

The ending of the third verse and the beginning of the fourth pose a challenge for the translator. They contain the words \textit{anosita} and \textit{\textsanskrit{osāna}}, the former of which is simply the negative of the past participle of the latter. Nonetheless, both Norman and Bodhi translate these words as “unoccupied” and “at the end” respectively, obscuring the fact that they are the same word in a different form. Meanwhile, \textsanskrit{Ñāṇadīpa} has “obstruct” in both cases, achieving consistency at the expense of comprehensibility.

The root sense is “lay to rest”. Originally it probably meant the place where one laid down one’s burdens at the end of the day (cf. \textit{khema}). From there we see two main applications, to “end” (cf. \textit{\textsanskrit{pariyosāna}}), or to “reside”. The fact that it follows right after the mention of finding “home” in the “world” in an unafflicted “quarter” shows that the sense “reside” applies.

As it happens, English also has a word that means both “residence” and “ending”, and which likewise originates in the idea of establishing a safe place to rest at the end of a journey: settlement.

The Niddesa, followed by the commentary, explains these lines in a metaphorical sense. For \textit{anosita} it has, “Youth is settled by old age, health is settled by illness” and so on; and for \textit{\textsanskrit{osāna}}, “Old age settles all youth, illness settles all health …”. This is in the tradition of the \textsanskrit{Hāliddikānisutta} (SN 22.3), which as we have seen, explains a verse about “leaving shelter” in terms of the five aggregates. While this is a valid pedagogical approach, it doesn’t help us understand the basic meaning. What we can say, though, is that Niddesa connects \textit{anosita} with \textit{\textsanskrit{osāna}}.

A final subtlety in the translation is the particle \textit{tveva}, which here has a contrastive sense. Putting these ideas together, we arrive at a translation that is not only consistent and coherent, but which reveals a new layer of meaning.

\begin{verse}%
Wanting a home for myself,\\

I saw nowhere unsettled.\\

Yet even in their settlement they fight

%
\end{verse}

Here the existential plight comes into sharper focus. All that people want, it would seem, is a safe place to make their life. Land, water, food, family. Yet somehow, it is never enough: even when they have such a place, they still end up fighting. And this is why the situation appears so hopeless: even victory and achieving all you want merely sow the seeds for future conflict.

Thus it is here that the young Siddhattha’s quest turned inwards. Surely there must be something in humanity that causes us to always seek out violence? And he saw it, that “dart”, an arrow nestling deep in the heart. The pain and anguish of this is the ultimate reason we can never be truly satisfied, and never find our place of rest.

The text does not identify the dart, but it is commonly used throughout the suttas as a metaphor for suffering and its various causes. The Niddesa explains it as the “seven darts”, and while this is not a standard category, most or all of these are called darts elsewhere in the suttas: greed, hate, delusion, conceit, views, sorrow, and doubt.

The fact that it is unexplained, however, suggests that here, the exact metaphorical target of “dart” is less important than the metaphor itself. It is a military image. It tells us that those who inflict violence are already wounded. They carry their injury with them before they even go into battle; it is the reason they fight. And it is why, long after they have returned home and the scars of war have healed, they still carry that pain within them. And it lies there, festering, passed on from generation to generation, waiting for the next war.

For many, the realization that the spur to violence lies within us all is a depressing realization of the nature of humanity. But not for the young Siddhattha. If it lies within us, it can be extracted.

One struck by the dart is said to “run around”, and once again the metaphor is left implicit. But it evokes our never-ending search for solutions in the world, be they through violence, or through war's handmaid, indulgence. Plucking it out stops the running around and the opposite, “sinking down”. Throughout the suttas, “sinking down” is used as a metaphor for laziness and apathy. Here it evokes the depression and hopelessness that can easily come upon one who contemplates humanity’s propensity for endless cycles of violence.

Having depicted his youthful existential crisis in lines of uncommon brevity and precision, the Buddha now goes on to speak of the training that leads to plucking out the dart. Here we return to the themes of non-attachment, contentment, moral integrity, and effort to which the trainee aspires, and which the perfected sage embodies.

\subsection*{With \textsanskrit{Sāriputta}}

The chapter is wrapped up with the \textsanskrit{Sāriputtasutta}, where the Buddha’s chief disciple questions the Buddha on the path of the renunciant (Snp 4.16). Here \textsanskrit{Sāriputta} shows his characteristic grace and devotion, as well as his magnanimity of spirit, as he questions on behalf of the people assembled there.

He opens by saying he has not previously “seen” or “heard” of a Teacher like the Buddha. We have learned from the rest of the \textsanskrit{Aṭṭhakavagga} that these refer to the ways of knowing: to behold the sight of a holy one or to encounter a teaching or oral tradition. \textsanskrit{Sāriputta}, let us remember, was formerly a follower of \textsanskrit{Sañjaya} \textsanskrit{Belaṭṭhiputta}, so he would have been familiar with the various “groups” or communities and their leaders before meeting the Buddha. It almost sounds as if this is \textsanskrit{Sāriputta}’s first meeting with the Buddha, but perhaps this is just a rhetorical flourish.

The final line of the verse mentions that the Buddha has arrived from Tusita heaven, dropping the reference casually as if assuming everyone would get it. This is presumably because the canonical texts say on multiple occasions that the bodhisatta had arrived in this life from Tusita heaven (AN 4.127:1.3, AN 8.70:16.1, MN 123:6.3, DN 14:1.17.1). This is a good example of how the \textsanskrit{Aṭṭhakavagga} relies on the prose suttas.

The Niddesa and the Pali commentary agree that this refers to the Buddha’s birth. Yet at the same time, the commentary, as well as the Arthapada, set the discourse following the Buddha’s descent from the heaven of the Thirty-Three (\textsanskrit{Tāvatiṁsa}) after teaching the Dhamma to his mother there. Since this event is not found in early texts, and since it is the wrong heaven, it cannot be what \textsanskrit{Sāriputta} was referring to. To add to the confusion, the Pali texts say the Buddha’s mother was reborn in Tusita, not the \textsanskrit{Tāvatiṁsa}; on this point, various northern texts differ.

This story originated in the northern regions, probably among the \textsanskrit{Mūlasarvāstivādins} of Mathura, as a morality tale about filial piety and the power of maternal love. Its roots appear to lie in a coded mythic narrative about how Buddhism brought an end to human sacrifice; the voracious \textit{\textsanskrit{yakkhinī}} \textsanskrit{Hārītī}’s disturbing habit of eating children was ended by the Buddha when he showed her how much she loved her children. The commentary to the Abhidhamma adapted this famous tale to serve as an origin story for the Abhidhamma, probably to compete with the similarly divine origin legend of the contemporary \textsanskrit{Mahāyāna} sutras. Evidently, the \textsanskrit{Theravāda} tradition has become confused as they tried to graft the incident of teaching the Abhidhamma onto the early suttas. It is worth noting that the \textsanskrit{Suttanipāta} commentary does not mention the Abhidhamma in this context, either because it is earlier than the telling in the Abhidhamma commentary, or because it is just an abbreviated retelling.

\textsanskrit{Sāriputta} asks a series of questions about a mendicant in training. These are not answered by the Buddha one by one, but the general outline of the Buddha‘s response does follow the questions.

\begin{itemize}%
\item How many dangers will they face? (verses 10, 11)%
\item How many adversities must they overcome? (12–16)%
\item What kind of speech should they employ? (various)%
\item What kind of alms resort? (17)%
\item What precepts should they undertake? (18, 19)%
\item What training should they pursue? (20, 21)%
\end{itemize}

The answers focus on the practical application of Dhamma in a mendicant’s life, emphasizing integrity, endurance, and dedication to meditation.

On the last point, this sutta emphasizes that a mendicant should be devoted to absorption (\textit{\textsanskrit{jhāna}}). This is placed within the framework of the mendicant’s restrained life, just as it is in the prose suttas. After mastering sense restraint (Snp 4.16:20.3), they have dispelled their desire for those objects of the senses (Snp 4.16:21.1). The language here (\textit{vineyya \textsanskrit{chandaṁ}}) is similar to the standard formula for \textit{\textsanskrit{satipaṭṭhāna}} (\textit{vineyya loke \textsanskrit{abhijjhādomanassaṁ}}), which likewise refers to the curbing of defilements through the practice of sense restraint. This “mindful” mendicant’s heart is well freed (\textit{suvimuttacitto}), a phrase which can apply to the mind in \textit{\textsanskrit{jhāna}}, and all the way to the abandoning of greed, hate, and delusion, (AN 10.20:11.2). Then they “investigate the Dhamma” (\textit{\textsanskrit{dhammaṁ} \textsanskrit{parivīmaṁsamāno}}), referring to the discernment (or “insight”, \textit{\textsanskrit{vipassanā}}) practices otherwise known as \textit{\textsanskrit{dhammānupassanā}} or \textit{\textsanskrit{dhammavicayasambojjhaṅga}}. Finally, with their mind unified in \textit{\textsanskrit{samādhi}}, they “shatter the darkness”, realizing enlightenment, and bringing the \textsanskrit{Aṭṭhakavagga} to a fittingly powerful conclusion.

\section*{The Way to the Far Shore}

The final chapter of the \textsanskrit{Suttanipāta}, the \textsanskrit{Pārāyanavagga}, is an inspirational work of devotion, philosophical subtlety, and profound meditation. From the company of the dogmatic, argumentative ascetics of the \textsanskrit{Aṭṭhakavagga}, desperate to prove their rightness above all else, we step into an altogether more elevated circle of elite contemplatives.

From a literary perspective, it is a single cohesive work, which in terms of its unity and scope is more comparable to a single sutta of the \textsanskrit{Dīghanikāya} than it is to anything else in the \textsanskrit{Suttanipāta}. It has been carefully composed to create a narrative with emotional stakes, building to a moving climax. In the opening section, we are introduced to the main characters. We learn of \textsanskrit{Bāvari}, with his earnest and sincere character. We are meant to like him: he is our primary emotional connection. Then he is faced with an imminent and unjust threat, which he cannot face alone. But he has friends, both human and divine, so our sphere of interest expands to include them, reminding us of the crucial role of spiritual friendship. A resolution to the threat is proposed, which instigates a journey of discovery that, painfully, leaves \textsanskrit{Bāvari} behind. The journey is no trivial one; it covers about 2,500 kilometers, taking perhaps three months. In the end, the threat is rendered irrelevant, for the journey has led to an outcome that surpasses hope. The story returns to \textsanskrit{Bāvari}, but he is not the man he once was.

This narrative arc forms the setting for the sixteen sets of questions that form the heart of the chapter. Each set of questions is personal and specific. We have seen that in some suttas of the \textsanskrit{Suttanipāta}, the questions can be generic and the actors interchangeable. Not so here. We come to know each of the students and can feel the connection between the topic of the question asked and the spiritual need and aspiration that underlies it. The questions serve not only to address a variety of interesting doctrinal points but to reinforce our emotional connection with the characters of the story, heightening the resolution in the final passages.

\subsection*{The Composition of the \textsanskrit{Pārāyanavagga}}

This text is no simple compilation, but an extended composition guided by a hand of considerable literary sophistication. The commentary attributes the composition to Venerable Ānanda. He may well have had a hand in the older portions, the text as a whole is much later. The sixteen questions are the oldest portion, and, as usual, they would have been passed down together with a story of their provenance. But someone must have been responsible for creating the narrative arc.

Several of the questions are cited in the Pali Suttas, as well as in other Buddhist literature, where they are said to be “in the \textsanskrit{Pārāyanavagga}”. The Niddesa comments on the questions but merely repeats the narrative portions without comment. This suggests that the framing narrative was formed considerably later, an impression borne out by a variety of evidence.

The story begins in the distant land of \textsanskrit{Dakkhiṇapatha}, the “Southway”, a region that falls outside the normal domain of early Buddhism, and which was sometimes said by the Brahmins to be unholy ground. By the time of King Ashoka, about a century and a half after the Buddha, this became the trade route to Andhra, and from there to Sri Lanka. Several of the places named in the voyage are also named in the account of the Buddhist missions in the time of Ashoka, showing that more than a century later, they still lay outside the “center”.

The narrative form of the story, featuring a curse, is found commonly in later literature. And several of the terms are late. \textit{\textsanskrit{Vāsana}}, for example, is a common term in later Buddhism for the “traces” of karmic influence, but not found in this sense at all in the early texts. In the Pali canon it is only found here, and in the Netti and the Milinda, both of which are post-Ashokan texts of dubious canonicity. Jayawickrama cites several other examples of Sanskritic terms and idioms.

This agreement of multiple independent indications makes it certain that the introduction is late. But what was the motivation? Why set a story in such a far-off land, and fill it with Sanskritic idioms and allusions? In the Ashokan and later years, Buddhism spread to many new lands, from Kashmir in the northwest, to Sri Lanka in the southeast. It is common, if not universal, to find the advance of Buddhism recorded in conversion narratives. A new land must have been visited by the Buddha, or at least by his disciples. Buddhism cannot have been a new introduction, but part of the culture of the land since far-off days. As time went on, such conversion narratives became more elaborate and fanciful. So it seems likely that the introductory narrative served as a conversion tale for the once-remote land of \textsanskrit{Dakkhiṇapatha}.

\subsection*{The Quotations}

In our discussion of the \textsanskrit{Aṭṭhakavagga}, we noted that the two quotations found in the early canon are very specific: they are both in the context of Venerable \textsanskrit{Mahākaccāna} in \textsanskrit{Avantī}; nowhere is the \textsanskrit{Aṭṭhakavagga} quoted by the Buddha himself. The six canonical quotations of the \textsanskrit{Pārāyanavagga} follow a quite different pattern. They are not geographically significant, occurring in the early heartland of \textsanskrit{Sāvatthī} or Benares, with one exception.

That is the account of the lay devotee \textsanskrit{Veḷukaṇṭakī}, an accomplished meditator, who rose before dawn and recited the verses of the \textsanskrit{Pārāyana} (AN 7.53:2.2). A passing deity, the great king \textsanskrit{Vessavaṇa}, overheard and applauded her, just like the Buddha applauded \textsanskrit{Soṇa} for his recitation of the \textsanskrit{Aṭṭhakavagga}. \textsanskrit{Veḷukaṇṭakī} lives in the Southern Hills (\textit{\textsanskrit{dakkhiṇāgiri}}), which lie within the Magadhan realm about 500 kilometers to the south-west of \textsanskrit{Rājagaha}, along the road towards \textsanskrit{Avantī}, and, beyond, to the distant residence of \textsanskrit{Bāvari}. While this is still within the early region of Buddhism, it was still regarded as an outlying district, so much so that the lay folk complained about how rarely such districts were visited by the Buddha and his mendicants (Khandhaka 1:53.1.3). Although this is only one data point, it does locate the \textsanskrit{Pārāyanavagga} to the south. If nothing else, the fact that the Southern Hills was considered an outlying district during the Buddha’s lifetime puts into perspective just how distant lay the hermitage of \textsanskrit{Bāvari} on the banks of the \textsanskrit{Godhavarī}.

This case is unusual, not only for the setting but because no actual verses are quoted, just referred to. In the remaining five quotations, actual verses are cited and discussed. And the manner of these discussions reveals much about how the early Buddhist community approached the problematics of text interpretation.

In three of these cases, the Buddha presents a short teaching, then caps it off by saying that this was what he was indicating with a certain verse in the \textsanskrit{Pārāyana}. Two of these (AN 3.32, AN 4.41) quote the closing verse from the Questions of \textsanskrit{Puṇṇaka} (Snp 5.4:6.1), while AN 3.33 quotes from the Questions of Udaya (Snp 5.14:2.1). All three suttas speak of profound and subtle states of meditation, and in each case, the relation between the prose and the verse is far from obvious. These are not like the ordinary verses of the Anguttara, where the verses merely serve as a mnemonic, summing up the prose with little flair or imagination. Nor are they like the explanation of the \textsanskrit{Aṭṭhakavagga} verse offered by \textsanskrit{Mahākaccāna}, where he was extending the original meaning by allegory. Rather, they express the same profound truth in quite different forms.

This very point is further emphasized in the quotation found in AN 6.61, which in the \textsanskrit{Pārāyana} is attributed to Tissametteyya, who here is simply called Metteyya. He has no relation to the future Buddha of the same name, although certain later legends did make that connection.

This quote is unique in that the quotation is different than the quoted passage. In some other cases there are variant readings, but here we find an entirely different word in the first line, although the sense “having understood” is the same (AN 6.61:2.1, \textit{yo ubhonte \textsanskrit{viditvāna}}; Snp 5.3:3.1, \textit{so \textsanskrit{ubhantamabhiññāya}}).

A more interesting detail of difference is the opening pronoun, which in the \textsanskrit{Pārāyana} is a demonstrative pronoun (“that” person). Subsequent lines also use a demonstrative pronoun, showing that each is an independent statement that calls back to the opening lines.

\begin{verse}%
That thoughtful one, having known both ends,\\

is not stuck in the middle.\\

He is a great man

%
\end{verse}

But when it is quoted in the \textsanskrit{Aṅguttara}, the opening lines are omitted, so the pronoun is left dangling. Most editions here use the relative pronoun \textit{yo} (the person “who”). Rather than linking back to the previous lines, this links forward to the subsequent lines.

\begin{verse}%
The thoughtful one who, having known both ends,\\

is not stuck in the middle:\\

he is a great man

%
\end{verse}

Thus the \textsanskrit{Aṅguttara} version fits its context better since it does not depend on the omitted lines. But at the same time, it subtly alters the sense. I translate each context differently to bring out the variation.

Tissametteyya asks of the one who has known both ends and is not stuck in the middle. This imagery is familiar from Buddhism, but we are not told what it meant to him. The Rig Veda sometimes speaks of the middle as noon, while the ends are dawn and dusk (RV 5.47.3, RV 6.43.2), an image that might be broadened to cover life versus death or light versus darkness. The implication is that he is looking for a path that does not merely avoid the dark in favor of the light but transcends even the light. The Buddha seemingly affirms this when he explains that the intended meaning was that contact is one extreme, the origin of contact is the second extreme, and the cessation of contact is the middle. The seamstress is craving, for it binds creatures into one rebirth after another (AN 6.61:12.4).

Why does the Buddha connect Tissametteyya’s image of the seamstress with craving? Tissametteyya is speaking on a metaphorical level, but from where is he drawing his metaphor? A seamstress would have been a common sight in those days, and perhaps he was simply speaking from everyday life. But when he refers to the “great man” he is drawing on scripture, as the Introduction makes clear.

The seamstress did feature in then-current Brahmanical literature. Rig Veda 2.32.4 speaks of the goddess \textsanskrit{Rākā} as the seamstress, invoking her blessings for prosperity and progeny. She appears with her “unbreakable needle” in the Atharvaveda, where she is beseeched to grant a hundred heroic sons (RV 20.11.8). In the Aitareya Brahmana—a ritual text of around 600 BCE based on the Rig Veda—she is said to sew the semen in a man to form a male child (3.37). The “seamstress” was a goddess of fertility, which explains why the Buddha explained her as “craving”.

The literary form of the discussion tells us more about how early Buddhists interpreted scripture. The verse is brought up in a discussion among senior mendicants while at the Deer Park in Benares, and each is invited to offer an interpretation. Each reading is different, although all follow the same form. They take the main idea of the verse—that having understood the two extremes, one would not be stuck in the middle—and fill out the metaphor with teachings found commonly in the prose suttas.

An ambiguous poetic text like this might have \emph{many} meanings, but it cannot have just \emph{any} meaning. No one is just pulling an interpretation out of nowhere because it feels right to them. On the contrary, they rely on their deep understanding of the prose suttas, which offer a range of interpretive possibilities.

The Buddha ultimately praises all of the readings as “well-spoken”, while endorsing one interpretation as his actual intended meaning. In doing so he shows how to avoid, not just the “extremes” of the text, but also extremes in interpretation. Texts are not rigidly determined with a single reductive meaning, nor are they a blank canvas for the imagination. And even the “correct” interpretation is not something to get stuck on.

The final of the quoted verses, in SN 12.31, reinforces this notion of interpretive humility. The Buddha quotes a verse from the Questions of Ajita (Snp 5.2:7.1) and asks Venerable \textsanskrit{Sāriputta} to explain it. Yet even he, the General of the Dhamma, an enlightened master, senior teacher, and wisest monk in the \textsanskrit{Saṅgha}, does not answer until the Buddha drops the slightest of hints: “Do you see that this has come to be?” Just this is enough to indicate the approach the Buddha was looking for, and \textsanskrit{Sāriputta} unhesitatingly answers in terms of conditionality. He unpacks the verse with precision, detailing the exact doctrinal implications of the terms used in the verse.

\subsection*{The Introduction}

The narrative setting offers a popular dramatic context for the more refined philosophies discussed within (Snp 5.1). As I have already discussed, it is a late addition. Yet such material should not be dismissed lightly. It remains true today, as it was then, that elevated philosophy is often stimulated by mundane worldly crises or needs. While the details of curses, deities, and sacrifice might seem bizarre, at its heart the story speaks of real people, with desires, flaws, and fears, and thus situates the spiritual quest in a relatable humanity.

We are introduced to the brahmin \textsanskrit{Bāvari}, who had traveled from “the fair city of the Kosalans”, namely \textsanskrit{Sāvatthī}, to the distant south. He settled by the banks of the \textsanskrit{Godhāvarī} river, in a region that seems to have been shared by the nations of Assaka and \textsanskrit{Aḷaka}. It was a little south of the city of \textsanskrit{Patiṭṭhāna}, which is modern Paithan. To this day, the nearby city of Paithan is renowned as a center for Brahmanical sages.

That the sixteen brahmins came from here is not directly confirmed in the verses themselves. Nonetheless, the imagery of rivers and floods is found throughout the chapter, befitting those who lived beside one of the largest-flooding rivers in India. \textsanskrit{Bhadrāvudha} speaks of people having gathered from afar (Snp 5.13).

Announcing one of the most prominent themes of the whole collection, we are told that \textsanskrit{Bāvari} has gone to the south in search of “nothingness” (\textit{\textsanskrit{ākiñcañña}}). The latter term is ambiguous, and in the context of the discussions to follow, the ambiguity is surely intended. On its surface, it has the sense of “owning nothing”, the longing for simplicity felt by all true renunciants. But in the suttas, it is far more commonly used for the third of the four formless attainments, the “dimension of nothingness”.

Now, the key term \textit{\textsanskrit{kiñci}} is used in many different senses. In the questions that follow, it is sometimes used in reference to \textsanskrit{Nibbāna} (Snp 5.13:4.2) and sometimes more ambiguously. But the meditation state of “nothingness” is explicitly discussed in two sets of questions (Snp 5.7, Snp 5.15). This meditation was taught as the highest goal of the spiritual life by \textsanskrit{Āḷāra} \textsanskrit{Kālāma}, one of the Buddha’s former teachers (MN 26:15.13). I think it is likely that \textsanskrit{Bāvari} was at the very least from related circles of the Brahmanical renunciant community as \textsanskrit{Āḷāra} \textsanskrit{Kālāma}. Note that the distinction between \textsanskrit{Āḷāra} \textsanskrit{Kālāma} and Uddaka \textsanskrit{Rāmaputta} is that the former reached the meditation state that he taught, whereas Uddaka \textsanskrit{Rāmaputta} did not. It would seem that \textsanskrit{Āḷāra} \textsanskrit{Kālāma} was not alone and that there was an active community of Brahmanical meditators for whom the dimension of nothingness was their highest goal.

The Introduction is implying that \textsanskrit{Bāvari} traveled to the south in order to reach the most exalted of meditation states. Yet when we find him, he is engaged in performing a large sacrifice with the offerings from the prosperous local village. It’s tempting to think that these two things are in tension, but things are not so simple. The ur-text for Brahmanical contemplatives is the \textsanskrit{Bṛhadāraṇyaka} \textsanskrit{Upaniṣad}, which begins with an extended passage on the horse sacrifice. Indeed, it would seem that the overriding purpose of the text is to provide a unifying framework for ritual, sacrifice, and contemplative philosophy; it even makes room for sex magic. In modern Buddhism, it is true, some contemplatives spurn the performance of rituals. Others, though, are quite happy to perform them, seeing them as a genuine part of their spiritual community. So long as the sacrifices are not harmful, the Buddhist tradition has always found room to accommodate different practices.

In traditional societies, sacrifices often served as a means of redistributing wealth. The text implies this, for it says that it was with the revenues of a “prosperous” village that he performed his “great sacrifice”. The sacrifice itself was not in his hermitage, so it was presumably in or near the village. We learn in the next verse that he has distributed all his wealth, confirming to us that he is sincere in his aspiration to “own nothing”. In this respect, the offerings to Buddhist mendicants today are similar. When a \textsanskrit{Saṅgha} is well-practicing, excessive offerings may be shared with those in need, acting as an informal but pervasive method of redistributing wealth.

The drama starts when a visiting brahmin arrives. He already looks suspicious, his disheveled appearance painting him as the bad guy. Nonetheless, \textsanskrit{Bāvari} receives him with respect. But when he demands five hundred coins and doesn’t get it, he grows furious, and utters a curse: on the seventh day, \textsanskrit{Bāvari}’s head will explode in seven pieces.

The power of curses is widely respected in much of the world to this day. \textsanskrit{Bāvari} is psychologically distressed by the curse, giving up food, and being unable to enjoy \textit{\textsanskrit{jhāna}}. Psychologists have discussed the phenomenon sometimes called “voodoo death”, where a curse can even result in death. While there is no doubting the distress experienced by someone thus affected, psychologists differ as to whether death is caused by a psychosomatic response, or where it is simply that the person stops eating and drinking, likely dying of dehydration. Regardless, such a curse is a serious matter. Confronting an affected person directly, by simply dismissing it outright, is unlikely to have any effect; their belief in the curse will likely prove stronger than their belief in the one giving such advice.

It would seem that the deity who approaches \textsanskrit{Bāvari}—or if you prefer, the author of the story—understood this, as they didn’t simply reject the idea of a curse outright. People are complicated. It is entirely possible that \textsanskrit{Bāvari} was an advanced meditator, a dedicated scholar, and a teacher of wisdom and integrity, yet still fell prey to the fear of curses. Once in that state, it is hard to see reason. Rather than dismissing the notion of a curse, then, the deity criticizes the brahmin who made it, saying he has no real understanding of what he speaks. Humbly, the goddess admits that she too does not understand head-splitting. But there is someone who does: the Buddha. The deity eulogizes the Buddha in lofty verses that anticipate the devotional tenor of the closing passages.

\textsanskrit{Bāvari} called together his sixteen students—each of them accomplished meditators with their own students—and charge them with a mission: to seek out the Buddha at \textsanskrit{Sāvatthī} and ask about head-splitting. They are to recognize him through the supposed Brahmanical prophecy of the thirty-two marks of the Great Man. In a unique twist, \textsanskrit{Bāvari} enjoins them to ask their questions only in their minds: if he is the Buddha he will answer. This procedure was followed in the initial encounter, but the main sets of sixteen questions were spoken normally.

Nine places are mentioned in their long journey to \textsanskrit{Sāvatthī}. Their first stop was at the town of \textsanskrit{Patiṭṭhāna}, modern Paithan, which has an extensive history as a prominent center of both trade and religion. From there they headed north along the trade routes to Mahissati and then \textsanskrit{Ujjenī} in \textsanskrit{Avantī}, the gateway to the middle country. They made their way up to \textsanskrit{Sāvatthī}, near the modern Nepalese border in the north. From there they turned east and then south. The text does not explain their detour, but presumably, they arrived at \textsanskrit{Sāvatthī} only to discover that the Buddha was not there. Perhaps they expected to find the Buddha in his hometown of Kapilavatthu, but there too they met disappointment and headed south through \textsanskrit{Vajjī} and across the Ganges to \textsanskrit{Rājagaha}. They finally found the Buddha teaching at the \textsanskrit{Pāsāṇaka} shrine at the peak of one of the hills surrounding \textsanskrit{Rājagaha}. The text emphasizes how, even after such a long journey, they rushed up the mountain, eager to ask their questions.

One of the students, Ajita, takes the lead in asking questions with his mind, to the understandable perplexity of the audience. The Buddha gives a straightforward, if paradoxical, answer to the question of head-splitting. We would think of the “head” as the seat of knowledge, but the Buddha says, rather, that ignorance is the head, and knowledge is the head-splitter. The allusion is to dependent origination, where ignorance is the head of all the unfolding conditions that lead to suffering.

The “head” of ignorance is not “split” by mere theory, however. It only happens when accompanied by a holistic practice and development, here described as “faith, mindfulness, and immersion, and enthusiasm and energy”. It is surely no coincidence that these are essentially four of the five “faculties” (\textit{indriya}). The fifth is wisdom, which has already been described as the head-splitter. And these are the same five qualities that the Buddha said were possessed by \textsanskrit{Āḷāra} \textsanskrit{Kālāma}, the teacher and practitioner of the dimension of nothingness (MN 26:15.15). The Buddha is, then, not telling them anything new, but rather affirming that the development of these qualities, which they learned under their previous system, was still crucial. We shall see throughout these chapters that the Buddha is, as always, totally accepting of the good parts of the Brahmanical teachings, but unsparing in his rejection of the bad parts.

Satisfied, Ajita bows to the Buddha, who responds with a rather touching and personal blessing for him and his teacher, \textsanskrit{Bāvari}. Ajita then proceed to ask his questions, having been invited to do so by the Buddha himself.

The narrative offers a compelling story, humanizing \textsanskrit{Bāvari}, and setting in motion the events of the chapter. The story is meant to enliven and engage, to broaden the appeal of the beautiful, but sometimes esoteric, questions that follow. And in that, it succeeds.

Critics have been sometimes dismissive of this Introduction, but I believe it is a more careful and intelligent composition than has been previously recognized. We have already noticed that it introduces, almost casually, the idea of “nothingness” which is such a major theme in many of the questions to follow (Snp 5.5, Snp 5.6, Snp 5.11, Snp 5.12, Snp 5.15). Other themes are introduced here and drawn out in the questions, such as ignorance (Snp 5.2, Snp 5.14), the great man (Snp 5.3), sacrifice (Snp 5.4), the gods (Snp 5.16), and especially textual learning and tradition (Snp 5.5, Snp 5.6, Snp 5.8, Snp 5.9). This is more than a coincidence. The Introduction has been carefully composed to foreshadow the themes raised in the questions, giving the whole composition a high degree of unity. And it does that while at the same time telling a story that has quite a different flavor.

Once we have established the fact that the Introduction frequently employs foreshadowing, another seemingly incidental detail takes on a greater significance. \textsanskrit{Bāvari} is said to have settled down on the banks of the \textsanskrit{Godhāvarī}. This is a mighty river drawing from a massive catchment that is sometimes called the “Ganges of the south”. These days it is hedged in by dams, but it is said to have the largest flood flows in all of India.

The idea of the “flood” is the primary metaphor for fear in the \textsanskrit{Pārāyanavagga}, where it is mentioned fourteen times, as compared to four times in the \textsanskrit{Aṭṭhakavagga}. Floods are a common feature of life in many parts of India, including the Ganges basin, where the bulk of the early Buddhists lived. But here they take on a primal significance, haunting the minds of the brahmins. And if the flood is the primary image of fear and despair, the path of freedom is best represented as the safe passage to the “far shore”, which lends the entire collection its name.

Corresponding to the idea of the flood is the idea of the “middle” that serves as an island or refuge as the flood rises. Again this is much more common in the \textsanskrit{Pārāyanavagga}, where it appears eight times, whereas in the \textsanskrit{Aṭṭhakavagga} it appears only thrice.

The image of the “middle” has a double valence in these chapters. As the island in the middle of the flood, it is a metaphor for \textsanskrit{Nibbāna} (Snp 4.14:6.1, Snp 5.11:1.1). But it can also stand for a center between extremes that should not be grasped (Snp 4.15:15.3 = Snp 5.12:4.3, etc.).

This dual valence is reminiscent of the two senses of \textit{\textsanskrit{pāra}}, the “far shore”. As we have seen, in the Uragasutta (Snp 1.1), a wise mendicant should avoid both the near and the far shore. Here, as we conclude the collection, the Dhamma is said to lead from the near to the far shore, which stands for \textsanskrit{Nibbāna}. These are metaphors, and we should not push them too far.

\subsection*{The Questions of Ajita}

Ajita begins with a fundamental question (Snp 5.2): by what is the word shrouded? It’s a very general question, but one with rather specific connotations. Remember that Ajita has just asked about head-splitting, and the Buddha replied that “ignorance was the head”. The Buddha now responds that ignorance is also that which shrouds the world. The passages are linked. Since we know the Introduction was added later, it would seem that the form of the narrative was shaped in order to lead up to this smooth segue.

In speaking of ignorance as “shrouding”, the text echoes a common phrase from the prose. Sentient beings are said to be “shrouded by ignorance” (\textit{\textsanskrit{avijjānīvaraṇa}}) and fettered by craving. It’s tricky to capture in translation because \textit{\textsanskrit{nīvaraṇa}} is commonly translated as “hindrance” in the context of the well-known “five hindrances” that obstruct meditation. The root meaning hails back to the myth of \textsanskrit{Vṛṭra}, the cosmic serpent who wrapped the world in his coils, obscuring the light and trapping the world in its coils. In the context of the five hindrances, then, it suggests something that is holding you back, whereas, in the context of ignorance, it is blocking out the light. Ajita’s question, then, is asking for a psychological explanation in terms that spring ultimately from the Vedic mythology in which he has been schooled. Pop psychologists still invoke the same imagery today, speaking of the “dragons” that have to be overcome in one’s journey.

The third of Ajita’s questions receives perhaps the most unexpected answer. He asks what is the world’s “tar pit”. This evokes the tar that is laid as traps for unwary monkeys; once they lay their hand in it they are done for (SN 47.7:1.4). Here it is a term for the craving and attachment that binds people, just as ignorance is that which shrouds them. The Buddha says the tar pit is \textit{jappa}. Like the English word “prayer”, this means both “words uttered in supplication of a god” and “a deep longing”. The implication is that the Brahmanical rituals that they have devoted themselves to are not merely useless, but serve only to trap them in another form of desire.

Next Ajita brings up \textit{\textsanskrit{nīvaraṇa}} once more, this time in the positive sense of that which “blocks” the streams of craving, rather than being “shrouded” by ignorance. Here the translator faces a difficult choice: render \textit{\textsanskrit{nīvaraṇa}} consistently at the expense of comprehension or rendering for contextual comprehension at the expense of consistency. The Buddha says it is mindfulness that blocks the streams of craving; it eliminates the five hindrances through the practice of meditation. But it is wisdom that will “lock out” the streams, eliminating them forever.

The verses until now have been an extremely compressed summary of three of the four noble truths: the danger is suffering, it is caused by ignorance and craving, and mindfulness and wisdom show us the path out. Now Ajita asks a subtle question that gets to the nature of the remaining truth, \textsanskrit{Nibbāna}. Those things that lead us on the path, the purely wholesome qualities of wisdom and mindfulness; surely even these must cease when all “name and form” cease? Here Ajita shows that he is intent upon a final existential liberation, not merely the psychological ease of freedom from defilements.

Ajita’s former question had invoked the “streams” in the world. He was perhaps drawing upon the imagery in \textsanskrit{Chāndogya} \textsanskrit{Upaniṣad} 6.10.1, where the many rivers are said to run east or west, but all have arisen from the great ocean and will return to it. For the \textsanskrit{Upaniṣads}, this process of differentiation or individuation describes how we come to identify our “self” with “name and form”. Name and form are merely the ephemeral differentiated aspects of the phenomenal world in which we live, which emerged from the undefined (\textit{\textsanskrit{avyākṛta}}) reality before creation, the divine Brahman. But we have forgotten our divine nature and just see the aspects. Thus the \textsanskrit{Upaniṣadic} answer to this question would be that name and form cease when they return to the eternal divine Brahman that is their true Self, which ultimately is nothing but a sheer mass of consciousness.

The Buddha delays his response, drawing out his final revelation: that name and form will only cease when consciousness itself ceases. Here he stakes out his fundamental departure from the \textsanskrit{Upaniṣadic} teachings. He shows confidence in the students, and events show that his confidence was warranted. It is not easy to let go of such a profound and fundamental belief, yet it would seem that these brahmins, like \textsanskrit{Āḷāra} \textsanskrit{Kālāma} and Uddaka \textsanskrit{Rāmaputta} before them, had “little dust in their eyes” and were ready to take that final step.

Finally, Ajita returns to a practical question. Given that there are those who have realized this profound teaching and others who are already on the path, how do such people live? Once again, the Buddha responds with a summary of the whole path of practice: not greedy, unclouded in mind, always doing what is skillful, a mendicant would wander, mindful.

\subsection*{The Questions of Tissametteyya}

Tissametteyya’s questions concern the nature of the individual who is regarded as perfected (Snp 5.3). Most ancient Indian philosophies posited that our flawed humanity was not doomed to remain forever a work in progress, but that a person could transcend imperfections. Tissametteyya calls such a person a ”great man”. Just as Ajita echoed the topic of ignorance from the Introduction, here Tissametteyya echoes the idea of the great man, a scriptural concept used by \textsanskrit{Bāvari} to assess the Buddha. For the Buddha, the great man was the \textit{arahant} (“perfected one”, “worthy one”), who here fulfills Tissametteyya’s questions about the one who is content and tranquil.

As to the remainder of these verses, I have discussed them above in the context of the prose explanation at AN 6.61.

\subsection*{The Questions of \textsanskrit{Puṇṇaka}}

What \textsanskrit{Puṇṇaka} asks is something that, to a modern Buddhist, would appear to be a no-brainer (Snp 5.4). But to him, it speaks to a questioning of the very fundamentals of the spiritual path to which he had devoted his life. What, he wants to know, is the reason so many folks perform sacrifices to the gods?

The problem of sacrifice harrows our humanity. Students of religion, history, anthropology, mythology, and psychology have been asking the same question ever since. To a rational mind, it all seems so foolish, such an obvious waste of time and effort, and all too often, cruel beyond belief. Yet it did not seem like that to most of humanity throughout most of our history. And that includes \textsanskrit{Puṇṇaka}’s beloved and wise teacher, \textsanskrit{Bāvari}, who you will recall was said to have performed a great sacrifice. If we dismiss the sacrifice too blithely, we will miss a crucial dimension of our humanity.

The Buddha gives a straight answer: sacrifice stems from fear of old age, in hope of some (better or eternal) state of existence. We can expand on this by distinguishing two dimensions of the sacrifice. One we might call the “religious” or “supplicatory”, where people will make offerings to gods and beg favors of them. This approach is rational insofar as it extends to the gods the ordinary psychology of fellow humans: be nice to them and they will be nice to you. The other dimension is, I believe, older and deeper, and we might call it “magical”. It is not about supplication, but participation. It stems from the recognition that we are a part of this cycle of birth and death, and it enacts participation in the cycle through the sacrifice. From life there comes death, and from death new life springs. Thus we sacrifice, creating death in hope of bearing greater fruits in the future, just as a farmer will sacrifice the grains to the soil in hope of next year’s crop.

The Indian religions did not separate these two dimensions. When sacrifices are made to the gods, they are not just free to return the favor, but in a deep sense they are obliged by their \textit{dhamma}; a sacrifice is not so much begging as coercing the gods. The brahmins believed that they participated in this cycle as direct expressions of divinity, which made them more holy than the gods themselves. The words they speak and the actions they perform instantiate and reinforce the underlying creative and destructive forces that make the world.

This is why the sacrifice is not just for fools and charlatans, although there have always been plenty of them. For \textsanskrit{Puṇṇaka} and his companions, it is a profoundly meaningful gesture that affords them a primary place in the very fabric of being.

It took a lot for \textsanskrit{Puṇṇaka} to ask this question, and more to accept the Buddha’s answer. But it is not over. He goes on to ask where anyone thus seeking an escape from the cycle has even succeeded. But the Buddha says no. The very act of praying and hoping is what binds us to rebirth. For the Buddha, such religious aspirations are \textit{\textsanskrit{bhavataṇhā}}, “craving to continue existence”. They may be more elevated in aim than ordinary sense desires, but the outcome is still more transmigration.

But all is not lost. It is not that no one can escape, simply that prayer and sacrifice are not the means for escape. The Buddha speaks of one who has “assessed” the world. Such a one can find peace through letting go and cross over old age.

\subsection*{The Questions of \textsanskrit{Mettagū}}

\textsanskrit{Mettagū}’s question must have pleased the Buddha, as he cuts straight to the essence of the Buddha’s teachings: what is the origin of suffering? The Buddha answers that it is attachment, here using the term \textit{upadhi}, which we encountered in Snp 1.2.

When \textsanskrit{Mettagū} follows up by asking how to cross over the flood of transmigration, the Buddha first announces that he will proclaim a teaching that is “apparent in the present, not relying on tradition”, a claim he will repeat in his verses to Dhotaka (Snp 5.6:6.3). This is a recurring feature of this chapter: the brahmins were longing for something they could realize directly, and were, it would seem, disenchanted with relying always on tradition or hearsay. \textsanskrit{Bāvari} was said to be a master of the traditional forms of knowledge (\textit{\textsanskrit{itihāsa}}), while we shall see Hemaka complain that all his former knowledge was based on tradition (Snp 5.9:1.5), a complaint echoed by \textsanskrit{Piṅgiya} in the closing passages (Snp 5.19:5.4). In a way, this rejection of “hearsay” is the obverse of the theme so prominent in the \textsanskrit{Aṭṭhakavagga}, that one cannot purify another.

The Buddha then proclaims his liberating teaching: to let go of all attachments and uproot consciousness itself, so that one does not remain in existence. We have noted that the brahmins of this group, as \textit{\textsanskrit{jhāna}} practitioners, seem to have placed a special emphasis on the state of purified and expanded consciousness known as the “dimension of nothingness”; a little below, the Buddha hints at this with the word \textit{\textsanskrit{akiñcana}}. And we have further noted that this attainment was famously achieved by \textsanskrit{Āḷāra} \textsanskrit{Kālāma}, who taught that rebirth in the realm of nothingness was the highest goal of the spiritual life. This was the direct reason the Bodhisatta rejected that path (\textit{na \textsanskrit{nibbānāya} \textsanskrit{saṁvattati}, \textsanskrit{yāvadeva} \textsanskrit{ākiñcaññāyatanūpapattiyā}}).

This gives a layer of subtlety to the complaint about relying on traditional knowledge. It is not that their path had led nowhere; it had led to deep meditation. However such states, elevated though they are, are still impermanent and of themselves cannot be the final goal. The Brahmanical teaching, it would seem, was that such states were a temporary realization of the true nature of the self. Practicing them ensures that at death one’s \textit{atman} will be united fully and eternally with the cosmic divinity (\textit{brahman}). Thus even such advanced practitioners must rely on a metaphysical doctrine passed down through the tradition that foretells their future destiny, and cannot fully realize the truth in this life. And this, says the Buddha, is where his teaching is different.

Such is \textsanskrit{Mettagū}’s faith that even the revelation that he must give up all forms of consciousness does not deter him. He begs the Buddha for further instruction. The Buddha concludes by eulogizing the true “knowledge master” (\textit{\textsanskrit{vedagū}}), not merely one who has memorized the Vedas, but one with nothing, who does not proceed from one life to another.

\subsection*{The Questions of Dhotaka}

Dhotaka shows his devotion and eagerness to learn, so like his companions and so unlike the argumentative ascetics of the \textsanskrit{Aṭṭhakavagga} (Snp 5.6). He longs for the Buddha’s voice and declares his commitment to training for \textsanskrit{Nibbāna}, so the Buddha urges him to be alert, mindful, and energetic.

But it would seem that Dhotaka’s devotion tends towards excess, for he begs the Buddha to release him from his doubts. The Buddha demurs, explaining that no one can free someone who still has doubts. It is up to each of us to understand the Dhamma for ourselves.

Dhotaka further begs to practice, employing the striking image that he wishes to wander as “unimpeded as space”. The metaphor draws on the observation of the heavenly bodies wandering freely in space. The brilliant and unchanging nature of the stars, moon, and sun, lead to them being identified with divine beings. Dhotaka wishes to proceed “down here” just as they do “up there”.

\subsection*{The Questions of \textsanskrit{Upasīva}}

Like Dhotaka, \textsanskrit{Upasīva} humbly begs for help in crossing the flood (Snp 5.7). The Buddha immediately urges him to contemplate “nothingness”, and, being mindful, to cross the flood relying on “there is nothing”. This makes explicit the reference to the dimension of nothingness, which has been hinted at previously.

It is unusual, and probably unique, for the Buddha to immediately recommend such a specific and very advanced meditation attainment to a new student. It strongly implies that \textsanskrit{Upasīva} was already practicing this. The Buddha is reassuring him that the meditation he has already realized is not useless, but can be a foundation to realize \textsanskrit{Nibbāna}, so long as he lets go of all craving. So long, however, as he remains dependent on that meditation state, he will be reborn in the corresponding realm of rebirth. The sutta goes on to discuss what happens to such a person.

Echoing the \textsanskrit{Kalahavivādasutta} (Snp 4.11), \textsanskrit{Upasīva} continues from this already advanced point, pushing the Buddha to explain more about the state of nothingness, here described as “the ultimate liberation of perception”. That this refers to the dimension of nothingness appears to be confirmed in the \textsanskrit{Pañcattayasutta} (MN 102:4.4), where the Buddha declares that certain ascetics and brahmins practice the dimension of nothingness and declare that to be the “ultimate” of all perceptions. While it is not the highest of all the meditation states recognized by the Buddha, the higher states are those of “neither perception nor non-perception” or the “cessation of perception and feeling”. Thus the dimension of nothingness is the highest of all states of perception.

\textsanskrit{Upasīva} wants to know whether such a person remains would “remain” without “traveling on”. The Buddha answers that they would indeed remain in such a state. This is explained in the Niddesa as meaning that they would last the full span of life in that dimension.

Seemingly following up on the earlier revelation to \textsanskrit{Mettagū} that the final goal leads to the ending of consciousness, \textsanskrit{Upasīva} asks whether the consciousness of someone who “grows cool” after many years in such a state would “pass away”. By using the term “many years”, \textsanskrit{Upasīva} confirms that he is asking about someone reborn in the realm of nothingness, rather than someone in the corresponding state of meditation, which cannot, of course, last for many years. Rebirth in the realm of nothingness is said to last for 60,000 eons, after which an ordinary person will pass away and fall to one of the other realms, whereas a disciple of the Buddha will realize \textsanskrit{Nibbāna} in that very state (AN 3.116:3.3).

The word for “pass away” (\textit{cavetha}) is a standard term for someone who dies in one realm and is reborn in another. \textsanskrit{Upasīva} wants to know if someone who is freed will still be subject to such a fate.

The Buddha answers with his famous simile of the flame going out. When a flame is tossed by the wind, it is removed from the source of fuel that sustained it, and it simply goes out. The Buddha employs the idiom \textit{na upeti \textsanskrit{saṅkhaṁ}}, where \textit{\textsanskrit{saṅkha}} is to reckon or count. Elsewhere the same idiom is used as a stock phrase when the Buddha is comparing something vast, like the Himalayas, with something tiny, like a few pebbles, which are so small they “don’t count”. In the same way, a sage “goes out’ or “comes to an end” and no longer counts.

Here the Buddha uses the unusual term \textit{\textsanskrit{nāmakāya}}. This means neither \textit{\textsanskrit{nāmarūpa}} (“name and form”) nor \textit{\textsanskrit{manomayakāya}} (“mind-made body”). In DN 15 it refers to the mental phenomena responsible for designation. One is reborn in the dimension of nothingness due to letting go of any attachment to the physical (\textit{\textsanskrit{rūpa}}). Now the Buddha says that one in that realm also lets go of what is mental, i.e. the states of consciousness associated with that attainment.

The contrast between the Buddhist and Brahmanical views is well highlighted in comparison with a similar verse in Mundaka \textsanskrit{Upaniṣad} 3.2.8. Here is the Pali with translation.

\begin{verse}%
\textit{\textsanskrit{accī} \textsanskrit{yathā} \textsanskrit{vātavegena} \textsanskrit{khittā},}\\

As a flame tossed by a gust of wind,\\
\textit{\textsanskrit{atthaṁ} paleti na upeti \textsanskrit{saṅkhaṁ};}\\

comes to an end and no longer counts;\\
\textit{\textsanskrit{evaṁ} \textsanskrit{munī} \textsanskrit{nāmakāyā} vimutto,}\\

so too, a sage freed from mental phenomena\\
\textit{\textsanskrit{atthaṁ} paleti na upeti \textsanskrit{saṅkhaṁ}.}\\

comes to an end and no longer counts.

%
\end{verse}

And here is the Sanskrit.

\begin{verse}%
\textit{\textsanskrit{yathā} nadyaḥ \textsanskrit{syandamānāḥ} samudre}\\

As rivers flowing to the sea\\
\textit{astaṃ gacchanti \textsanskrit{nāmarūpe} \textsanskrit{vihāya}}\\

come to an end, giving up name and form;\\
\textit{\textsanskrit{tathā} \textsanskrit{vidvān} \textsanskrit{nāmarūpād} vimuktaḥ}\\

so too, a wise one freed from name and form,\\
\textit{\textsanskrit{parāt} paraṃ \textsanskrit{puruṣam} upaiti divyam}\\

enters the divine person beyond the beyond.

%
\end{verse}

The opening line replaces the Buddhist image of the flame going out with the \textsanskrit{Upaniṣadic} image of the water returning to the sea, an image that has been mistakenly associated with Buddhism ever since Sir Edwin Arnold used it to conclude his \textit{The Light of Asia}. The middle two lines are roughly similar. For the conclusion, the difference could not be starker. The Buddha speaks of the going out of the flame, and repeats the same image, emphasizing the fact that for him, to speak of extinguishment was the goal itself. The Brahmanical verse makes its different philosophy quite clear, speaking of the “divine person” into which the sage enters; in other words the cosmic divinity of the Brahman. Whether these two verses are historically related I cannot say. But the fact that the Pali verse is in the \textsanskrit{Pārāyana}, a dialogue with brahmins; and that the Mundaka uses the word \textit{para} here suggests to me that it may have been a direct answer to the Buddhist verse.

\textsanskrit{Upasīva} presses further, wanting to know whether that person ceases to exist, or whether they will exist eternally in a state of wellness. He is contrasting the two most common understandings of the fate of the individual: the annihilation of personal identity, or eternal life in a state of bliss. The prose suttas present this dichotomy as two “extremes” and the Buddha’s “middle teaching” of dependent origination as the way that avoids them (SN 12.47).

The Buddha explains why it is that such a person “does not count”. Normally, language relies on identifying certain characteristic features of the thing in question. We identify a person by the details of their face, or the clothes they wear, or their voice. Based on that, we assign them a name, and we implicitly assume that the same name applies to the same “person”. \textsanskrit{Nibbāna}, however, lacks any such identifying features. It has no “limit”, nothing by which it can be “measured” (\textit{\textsanskrit{pamāṇa}}). Here we find an early use of the word \textit{\textsanskrit{pamāṇa}} in the sense of “means by which something can be known”, which went on to become a central idea in later Indian philosophy. It is crucial to understand that for Indian philosophy, the word for “limit” and the word for “way of knowing” are the same. This is not dissimilar to the English word “define”. Something can only be known by its limits, and we can only speak in terms defined by those limits. But \textsanskrit{Nibbāna} has no such “measure” and escapes definition.

\subsection*{The Questions of Nanda}

Nanda wants to know if a “sage” (\textit{muni}) is defined by their knowledge or their conduct (Snp 5.8). The Buddha repeats a line from the \textsanskrit{Māgaṇḍiyasutta}, saying a sage is not described in terms of view, learning, or knowledge. (I have discussed this line in the introduction to the \textsanskrit{Aṭṭhakavagga}). Rather, they live aloof and free of attachments. Even those who are disciplined in themselves will not be free so long as they speak of purity in these terms. Only those who have given this all up have crossed the flood.

Nanda applauds the Buddha’s teachings and accepts his argument.

\subsection*{The Questions of Hemaka}

Hemaka complains that all he had learned before was but “the testament of hearsay”, thus echoing the critique of traditional knowledge that was expressed by the Buddha to both \textsanskrit{Mettagū} and Dhotaka (Snp 5.9). For Hemaka, this has led to nothing but endless speculation.

The Buddha gives a brief teaching, urging Hemaka to let go of all attachment to what is “seen, heard, thought, and cognized”, which as we have seen, includes all forms of traditional learning, devotional inspiration, and rational philosophy, as well as the revelations of deep meditation. The “imperishable” state of \textsanskrit{Nibbāna} is the ending of any desire for such things in this very life.

\subsection*{The Questions of Todeyya}

Todeyya shifts his concern from traditional forms of knowledge or conduct, and instead asks about the one who has let go of any sensual desire, craving, and doubt (Snp 5.10). He wants to know what manner of liberation they experience. The Buddha’s answer is very telling: liberation is just the ending of craving, nothing else. In other words, we do not practice to end craving so that we may attain something further. It is the end of craving itself that is the goal.

Todeyya asks another critical question of the one who is awakened: do they still hope? Are they still forming wisdom? The Buddha answers that they are indeed free of hope, that they possess wisdom, and are not forming wisdom.

A word on the word “hope” is probably in order. The Pali is \textit{\textsanskrit{āsā}}, which, of the many Pali words for “wish, desire” probably falls closest to the English idea of “hope”. The Buddha did not teach a Messianic religion, and his followers did not look forward to salvation in the future. We seek salvation in this very life, and in doing so we make kamma that will help to build a better future. For this reason, we find no parallel for the Christian concept of “hope” as a virtue in an absolute sense. The arahant has no need for hope, for they have already found the solace they seek.

Of course, it is still the case that the Buddhist path offers “hope” in the ordinary sense of having a reasonable expectation of a positive outcome. The story of \textsanskrit{Bāvari} offers a good example of this. He falls into despair due to the fear of the curse, and it is not until he is offered the hope that there is someone who can help him that he pulls himself back together. But hope is not offered as an absolute, as a solution in itself. It is also interesting that in this case, his despair is an intrinsic part of his path. Without the curse, he would never have looked for the Buddha’s teachings. But once he realizes the truth, there is no place for either hope or despair.

\subsection*{The Questions of Kappa}

In one of the most poignant and beloved exchanges of this whole chapter, Kappa speaks of the fears of aging and death as being like one who is stuck in the middle while a flood swells all around them (Snp 5.11). It is a terrifying image, especially in our age of climate catastrophe brought about by human greed and ignorance. As I write, my country of Australia is beset yet again by the latest in a series of floods that have swept the country, right after it was swept by fire. More storms are forecast this week, and countless Australians will be looking out on their own land as it is swallowed by the waters once more.

The Buddha gives Kappa an island as a refuge in the flood. It is the “isle of no return”, with no possessions and no attachments that is called \textsanskrit{Nibbāna}.

\subsection*{The Questions of \textsanskrit{Jatukaṇṇī}}

\textsanskrit{Jatukaṇṇī} again refers to the Buddha as the “one who has crossed the flood”, and asks of the state of peace (Snp 5.12). He speaks with deep humility of himself as “one of little wisdom”, a stark contrast with the ego-driven ascetics of the \textsanskrit{Aṭṭhakavagga}, or indeed, the prideful brahmins encountered in other discourses.

The Buddha begins by speaking of dispelling sensual desire since renunciation is the true sanctuary. This is a critical point in Buddhist philosophy. Renunciation is not about flagellation or self-denial, but about freeing oneself from addiction and finding a true place of safety and release.

The Buddha echoes the “neither taking up nor putting down” formulation so common in the \textsanskrit{Aṭṭhakavagga}. The state of peace, he goes on to say, comes when we let the past wither away, create nothing in the future, and do not attach to the middle. This too is an echo, recalling the teaching to Tissametteyya to understand the two extremes and not get stuck in the middle.

\subsection*{The Questions of \textsanskrit{Bhadrāvudha}}

\textsanskrit{Bhadrāvudha}’s verses are unique, for it appears that he begs, eloquently and devotedly, for a clear explanation, but does not actually ask a question (Snp 5.13). It feels to me like something is missing. Perhaps, on the other hand, he merely wanted a more full explanation of matters raised in the previous discussion.

In any case, that is what the Buddha does, urging the letting go of all attachments, for that is how \textsanskrit{Māra} latches on to people. To cling to attachments is to cling to death.

\subsection*{The Questions of Udaya}

Returning to the question of smashing ignorance—which was identified by the Buddha as the “head” that needed smashing—Udaya asks about liberation by enlightened wisdom (Snp 5.14). He addresses the Buddha as a “meditator” (\textit{\textsanskrit{jhāyī}}), which the Niddesa explains as meaning one who has attained the four jhanas.

The Buddha begins his answer with a truncated explanation of the five hindrances, although here only four are mentioned explicitly; doubt is omitted. It’s worth noting that here \textit{domanassa} must mean “aversion” rather than its normal “sadness”. It bears this meaning in similar contexts elsewhere, such as the \textit{\textsanskrit{abhijjhādomanassā}} (“desire and aversion”) that is abandoned due to sense restraint. In another slightly confusing usage, while other hindrances are said to be abandoned or dispelled, remorse is said to be “hindered”: the hindering of the hindrance. The Niddesa explains this as meaning the same as preventing, abandoning, or settling.

By giving up the five hindrances the entry into jhana is implied. The next verse goes straight to the “pure equanimity and mindfulness” of the fourth jhana. However, the verse cannot be speaking only of jhana, for it concludes by answering the question of the “liberation by enlightenment” which is arahantship. Thus the second line refers to the investigation of principles (\textit{dhammatakka}), which is said to “run out in front” (\textit{\textsanskrit{purejavaṁ}}). This unique term is a poetic variation of \textit{dhammavicaya}, which is the wisdom element of the seven awakening factors. In that context, \textit{dhammavicaya} rather unusually for a wisdom factor, is near the head of the list, preceded only by mindfulness. Normally the wisdom factor comes near the end. This is not the only context where wisdom comes first, though, for in the eightfold path “right view” comes first (MN 117:4.1). And, bringing the connection full circle, it is in the context of right view that we find the only other occurrence of the word \textit{\textsanskrit{purejavaṁ}}. There, the context is an extended metaphor of a chariot, with right view “running out in front” (SN 1.46:3.4) like the horses pulling a chariot. The idea, then, is not so much that \textit{dhammatakka} “precedes” the fourth jhana in time, but that it is a leading principle that drives the whole path.

The next verses offer another slightly unusual usage that challenges the translator. Delight is said to be the world’s fetter, while “thought”(\textit{vitakka}) is its “means of traveling about” (\textit{\textsanskrit{vicāraṇa}}). We commonly find \textit{vitakka} paired with \textit{\textsanskrit{vicāra}}, referring to two similar but complementary aspects of the mind, either “thought” and “consideration”, or in the context of deep meditation, where all mental factors have a much more refined meaning, “placing the mind and keeping it connected”. \textit{\textsanskrit{Vicāraṇa}} is the causative form of \textit{\textsanskrit{vicāra}}, explained by the commentary here as the “means of locomotion” and in the commentary to the same passage at SN 1.64:1.2 more concretely as “feet”.

In a rather abrupt transition, Udaya asks about the cessation of consciousness for a mindful practitioner. The Buddha answers that it is by not taking pleasure in any feelings, including the subtle “internal” feelings of deep meditation. This question would seem to follow on from that of Ajita at Snp 5.2:6.5. And in the Sanskrit version preserved in the \textsanskrit{Yogacārabhūmi}, that is exactly what they do. This appears to be a case where some verses have shifted place in the Pali tradition, a speculation supported by the fact that the preceding verses of Udaya are also found at SN 1.64 and with a slight variation at SN 1.65, where they appear by themselves. They are commented on here in the Niddesa, though, so if it is a corruption, it is an old one.

\subsection*{The Question of \textsanskrit{Posāla}}

\textsanskrit{Posāla} opens with an unusual praise of the Buddha, calling him the one who “reveals the past” (Snp 5.15). The Niddesa explains this doctrinally, by reference to the Buddha’s recollection of past lives. But this is not a topic that has been discussed with the sixteen brahmins. Perhaps it refers to the questions of \textsanskrit{Puṇṇaka}, where the Buddha claims that all those in the past who have offered sacrifices have not thereby attained freedom.

In any case, \textsanskrit{Posāla} goes on to ask about how someone who has reached the by-now familiar formless meditation on nothingness should practice for deeper insight.

The Buddha claims to have direct knowledge of all the stations of consciousness. This refers to the planes in which consciousness can be reborn, and echoes the Bodhisatta's dismissal of the practice of nothingness under \textsanskrit{Āḷāra} \textsanskrit{Kālāma} since it only led to rebirth in the corresponding dimension; see also the questions of \textsanskrit{Mettagū}. One who “remains” with that as their final goal is still attached to rebirth, even if very subtly. It is through insight into this that liberation from all attachments is found.

\subsection*{The Questions of \textsanskrit{Mogharājā}}

In another unusual opening, \textsanskrit{Mogharājā} claims to have already asked his question twice, to no avail (Snp 5.16). He persists, in the understanding that sometimes the Buddha answers on the third request. The Buddha does indeed answer, but no reason is given in the text as to why he made \textsanskrit{Mogharājā} wait. The Niddesa explains that the Buddha waited a while for \textsanskrit{Mogharājā}’s faculties to mature. This seems reasonable. It is tempting to see support for this in \textsanskrit{Mogharājā}’s name, which means “Fool of a King”. As if to ward off such an assumption, the introduction and closing verses qualify him as “\textsanskrit{Mogharājā} the intelligent (\textit{\textsanskrit{medhāvī}})”, which to a cynical mind might appear to protest too much.

His question, however, is far from foolish. He says that he has not heard the views of the Buddha on the different realms: this world, the next world, and the world of Brahma with its gods. The simple cosmological division of “this world and the next world” is a common way of speaking of the afterlife. The Brahmanical systems by this time had not settled on a coherent scheme for describing the afterlife, with multiple quite different explanations given in various scriptures. There was an emerging idea, however, that the goal of the spiritual life was not the rebirth in one or other of these worlds, but the escape from them all. To the brahmins, this escape was the realization of the divinity underlying all of these manifest worlds, that is, the cosmic creator \textsanskrit{Brahmā}, whose essence was the Self of pure consciousness in all beings. Here, however, the \textsanskrit{Brahmā} realm is conceived more concretely, as a divine plane inhabited by gods. Thus \textsanskrit{Mogharājā} is asking where the Buddha fits in with all this. These realms are all haunted by the specter of Death. How should one see them so as to escape his sight?

Much has been made in certain quarters about the difference between this passage, where the Buddha answered \textsanskrit{Mogharājā}’s question, and the Buddha’s refusal to answer Vacchagotta at SN 44.10. There, Vacchagotta asked whether the Self exists, or by implication, whether there is a Self that survives after death. Not receiving an answer, he walked away. This is adduced as evidence that the Buddha refused to directly say whether a Self existed or not but rather used not-self solely as a strategy. This argument flails from the beginning since the Buddha gives perfectly good reasons why he did not answer Vacchagotta. But the nail in the coffin for this speculation is that it relies on ignoring context. The reason \textsanskrit{Mogharājā} got his answer had nothing to do with the nature of his question. It was because he had the patience to wait around and ask three times.

The Buddha’s teachings are effective only insofar as they describe the world as it is. The contemplation of not-self is no exception: it is a successful strategy precisely because there is, in fact, no Self.

The Buddha answers by introducing the idea of emptiness. This famous Buddhist concept is found rather sparingly in the early texts, where it usually stands in close relation to the teaching of not-self, rather than being an all-encompassing philosophical perspective. This verse is the only time in the \textsanskrit{Pārāyanavagga} where the Buddha explicitly invokes the teaching on not-self. This reinforces the close connection between the idea of the Self and that of rebirth. For the brahmins, the Self (\textit{\textsanskrit{ātman}}) was that which is continuous from one life to the next, the unchanging core or essence of a person. The Buddha warns \textsanskrit{Mogharājā} against seeing rebirth in this way, for no such eternal essence can ever be found. The world is empty—there is no such thing as an eternal, unchanging essence. Rebirth is nothing but an ongoing flow of changing conditions.

\subsection*{The Questions of \textsanskrit{Piṅgiya}}

\textsanskrit{Piṅgiya}, after waiting all this time, concludes the round of questions (Snp 5.17). His thoughts are not the most challenging for the tradition or the intellect. But they are the most moving, as he speaks directly to the suffering of old age. Worried that he will die before understanding the truth, he asks the Buddha to explain the Dhamma so he can give up old age and rebirth.

The Buddha answers that it is negligent people who are harmed on account of forms. Here, “forms” mostly means the body, though as always it may have a somewhat broader sense too. The Buddha responds to \textsanskrit{Piṅgiya}’s emotive plea with a rhetorical flourish. The use of repeated \textit{\textsanskrit{rūpesu}} in passive constructions emphasizes how those who are negligent become subject to ongoing troubles and afflictions, lost in the middle of forms, and suffering on account of them. Trapped by their attachments, so long as they let their bodies bother them, they will keep on taking up new bodies. \textsanskrit{Piṅgiya} is urged to do the opposite: being diligent, let go of his form and don’t get reborn.

\textsanskrit{Piṅgiya} expresses his devotion and confidence in the Buddha and repeats his question once more. Here, it is not that he is still confused and must ask again, but because in such matters of emotional depth, sometimes we need to hear things more than once.

The Buddha responds by expressing the same idea differently, urging \textsanskrit{Piṅgiya} to observe how people become overwhelmed and mired in old age due to craving. By letting go of craving, he can escape the whole cycle.

\subsection*{Homage to the Way to the Far Shore}

The esteem with which these sixteen questions were regarded is underscored by the fact that they are concluded, not with a simple statement on how the mendicants applauded the teaching and practiced, but with two distinct verse passages. The first passage, the “Homage”, sums up the events of the whole chapter, and explains why it has the title “The Way to the Far Shore” (Snp 5.18). The next chapter offers a coda, showing what happened when one of the brahmins conveyed the experience back to \textsanskrit{Bāvari}. Few texts in early Buddhism are treated to such a lavish conclusion.

The Homage, as well as repeating the names and summing up events, confirms that all of the brahmins went on to ordain and practice under the Buddha. It also explains that the “supreme path”—the noble eightfold path—leads from the near shore to the far shore, which is what gives the work its name.

The Homage was added by redactors at some point, as were the Introduction and the coda on “Preserving the Way to the Far Shore” which is to follow. But each of these passages has a distinct voice. The Introduction is a dramatic narrative in a popular style. The Homage is a reverent summation in a devotional yet emotionally distant voice. The Preserving is highly personal, a first-person account of the emotional impact of the teachings. It seems likely that all three passages were composed distinctly. There is no particular logical connection between them. The two concluding passages presuppose the general idea of the Introduction, but do not confirm any more than that. Given that the Niddesa comments on the final two passages, but not on the Introduction, we should probably see these two passages as the basis upon which the Introduction was created, filled out with recollections from the oral tradition and a dose of imagination.

\subsection*{Preserving the Way to the Far Shore}

The purpose of this final chapter is lost without a clear understanding of the exact terms used. The title refers to \textit{\textsanskrit{anugīti}}, which, as indicated by the prefix \textit{anu-}, does not mean simply to “recite”, but rather, to “recite after”, to “keep on reciting”. It recalls the Buddha’s first sermon, where he said his teaching was not “learned from another” (\textit{an-anussuta}). His was a breakthrough based on his own insight. Now his disciples are committing themselves to continue reciting his teachings, creating a new kind of tradition.

The brahmins came from a background where to “keep on reciting” was to preserve an ancient sacred tradition. We frequently hear of how the brahmins “keep on reciting” (\textit{\textsanskrit{anugāyanti}}) the texts that were first uttered by the seers of the long past (AN 5.192:3.1).

Now \textsanskrit{Piṅgiya}, as an experienced Vedic reciter himself, announces that he will keep this new teaching alive in just the same way. It is obvious from the cultural background and the forms of the texts that the early Buddhists adopted much of the methodology for oral transmission from the brahmins. But it is rare to find such explicit evidence for a direct historical link between the oral recitation methods of the brahmins and the Buddhists.

\textsanskrit{Piṅgiya} speaks in the first person, maintaining the personal and emotional tone of his previous questions. The text does not name the person he is speaking to, who is addressed throughout simply as “brahmin”, but it is clearly meant to be \textsanskrit{Bāvari}.

Repeating the complaint found so often in the sixteen questions, \textsanskrit{Piṅgiya} compares his previous studies, which were merely tales of what was or would be based on hearsay, with the Buddha’s teaching, which is immediately effective in this very life. Thus he establishes the critical point of difference between the oral tradition of the Buddhists and that of the brahmins. The Buddha’s teaching is verifiable, it can be tested against one’s own direct experience.

I feel the need to point out here that I am not personally saying that the Brahmanical traditions lack experiential dimensions. I have no experience and hence no qualifications to speak on such matters. I am merely relating what the brahmins here believed. My own belief is that any religious path can offer some form of experiential solace to its followers, although it is not the same form of solace. Indeed, the complaint seems odd in light of the fact that the brahmins had clearly found a degree of solace through their meditation and practice of jhana. And it seems very likely that they participated in the same cultural milieu as the early \textsanskrit{Upaniṣads}, whose teachings have a strongly contemplative and experiential flavor. Perhaps the most characteristic idea of the early \textsanskrit{Upaniṣads} is the repeated phrase \textit{ya evam veda}, “one who knows this”, shifting focus from the correct performance of ritual to the correct understanding.

Perhaps the explanation for this lies in the relatively recent creation of the \textsanskrit{Upaniṣads}. On page 282 of his \textit{Buddhist and Vedic Studies}, Wijesekera discusses the way that texts are referred to according to Sanskrit grammarians. It is implied as early as \textsanskrit{Pāṇini}, a century or so after the Buddha, and confirmed by his commentator \textsanskrit{Katyāyana}, that the works of \textsanskrit{Yajñavalkya}, which are among the oldest of the contemplative passages in the \textsanskrit{Upaniṣads}, are distinguished as being “of recent origin, almost contemporaneous with ourselves”. It seems likely, then, that the textual studies of these brahmins consisted of the ancient Vedic hymns, as well as various legends and prophecies, ritual texts and the like, rather than the contemplative passages of the \textsanskrit{Upaniṣads}. At that time, the \textsanskrit{Upaniṣads} may have still been in formation, or of limited distribution and not regarded as canonical. The brahmins of the \textsanskrit{Pārāyana} were familiar with the ideas and practices in the \textsanskrit{Upaniṣads}, but perhaps not with the actual texts as we know them today.

The aspect of solace is here beautifully illuminated with the imagery of water, which in this conclusion is elevated from the fearful threat so often mentioned in the sixteen questions. Now \textsanskrit{Piṅgiya} is like a swan come home to a great river, like one who formerly was mired in the swamp, but now has seen the one who has crossed over the flood.

Such is \textsanskrit{Piṅgiya}’s praise that his respondent wonders how he can bear to be separated from him even for a moment. Assuming that this is \textsanskrit{Bāvari} speaking, it seems that \textsanskrit{Piṅgiya} has made the long journey back to the far south to convey the good news to his old teacher, a journey even more remarkable on account of \textsanskrit{Piṅgiya}’s advanced age. \textsanskrit{Piṅgiya} says that he does not dwell apart from his beloved Buddha for a single moment, for his mind never leaves him, and he remains in constant homage to the Buddha wherever he is.

At this point, the voice of the discussion changes, and it seems as if a new speaker is interjecting. The speaker is not named in the text, but \textsanskrit{Piṅgiya} refers to him as the “sage”, and he has knowledge of several of the Buddha’s disciples, so the Niddesa seems to be correct in identifying the new speaker with the Buddha himself. The commentary supplies a dramatic background in which the Buddha appears in a mind-made body just at this time.

The Buddha compares \textsanskrit{Piṅgiya} with three disciples who all found freedom through their devotion to faith. It is a curious selection. Vakkali makes sense, for the story of his devotion and ultimate demise is found at SN 22.8, and at AN 1.208:1.1 he is identified as the disciple most committed to faith. The others are obscure, and it is unclear why they are thus singled out. \textsanskrit{Bhadrāvudha} is one of the sixteen brahmins, but nothing further is found on him in the early texts. Gotama of \textsanskrit{Āḷavi} is, apparently, only mentioned here. It seems their stories were unknown to the commentary, too, for it glosses over them.

The Buddha’s words spur \textsanskrit{Piṅgiya} on to even greater devotion. He is not yet enlightened, but his final verse is one of the most inspiring in all the canon.

\begin{verse}%
Unfaltering, unshakable;\\

that to which there is no compare.\\

For sure I will go there, I have no doubt of that.\\

Remember me as one whose mind is made up.

%
\end{verse}

\section*{A Brief Translation History}

As one of the most beloved texts of the whole Buddhist tradition, the \textsanskrit{Suttanipāta} has a long history of modern studies and translations.

Thirty suttas translated by Sir Muthu Coomaraswamy were published by Trübner in 1874 under the title \textit{Sutta-\textsanskrit{Nipāta} or Dialogues and Discourses of Gotama Buddha}. He was a leading politician for the Sri Lankan Tamils and was the first Asian to be knighted. His translations, which are prefaced with a summary of the commentarial background, aim to be “thoroughly literal”. I can’t imagine what it would have been like for him to be publishing such a prestigious volume in Victorian England. But his Introduction reads like an extended and well-deserved flex, quoting liberally in French and German (in blackletter!), and discoursing eloquently on topics as wide-ranging as the true nature of \textsanskrit{Nibbāna} or the “fair women of the West, the high character of whose intellectual gifts is destined to leave quite as great a mark on the present progress of humanity as the many discoveries of science and art by which the age is distinguished” (p. ix). He does not argue for the particular antiquity of the \textsanskrit{Suttanipāta} but does mention that some of the experts in Sri Lanka say it contains some of the earliest Pali passages. Even in the mid-19th century, the Sri Lankan Pali pandits were taking up the challenge of the historical analysis of scripture. The Introduction contains several other passages that are historically interesting, such as his discussion of the relative merits of using Devanagari or Roman script for Pali, and his wish that international scholars will soon settle on a standard. Coomaraswamy promised a second volume to complete his translation, but his work was cut short by his tragically early death at 45, only a few years after publishing his translation.

The oldest complete English translation was that of the Danish scholar Viggo Fausböll, a leading pioneer of Pali and Indological studies, whose contributions earned him a knighthood in the Order of the Dannebrog in 1888. It was published simply under the title \textit{Sutta-\textsanskrit{Nipāta}} as volume X of the Sacred Books of the East series in 1881. This work established a standard of clarity and precision in translations that served well as a foundation for later scholars. Fausböll also edited the initial edition of the Pali text, which was published with the Pali Text Society in 1885; a new edition by Dines Anderson and Helmer Smith replaced it in 1913. Fausböll, in discussing the age of the text, says it “contains some remnants of Primitive Buddhism”, and he regards much of the \textsanskrit{Mahāvagga} and nearly all the \textsanskrit{Aṭṭhakavagga} as “very old”. Curiously, he does not single out the sixteen questions or the Rhinoceros Sutta, as do most later scholars. It is, however, a mark of his prudence that he makes no general claim for the whole of the \textsanskrit{Suttanipāta}.

The tradition of titled translators was continued fifty years later, in 1932, when another Indologist, Lord Chalmers, published a new translation for the Harvard University Press under the title \textit{Buddha’s Teachings: being the Sutta-\textsanskrit{Nipāta} or Discourse Collection}. Less prudent than Fausböll, he does not hesitate to assert that there is no older “book” than the \textsanskrit{Suttanipāta} and no older corpus of primitive Buddhist doctrine. In light of the proven lateness of certain portions, this seems excessive, but Chalmers is careful to note that “its materials are by no means of equal antiquity”. Chalmers made the unusual choice to render the text in metrical poetry. This gives a life and flavor to his translation, but it is less clear and transparent than Fausböll’s.

Until this time, there was no translation of this important text through the Pali Text Society. This was rectified in 1945, with the publication of E.M. Hare’s \textit{Woven Cadences}. Hare was a tea merchant from Norfolk who lived in Sri Lanka for three decades. Mentored by F.L. Woodward, he translated the \textsanskrit{Aṅguttaranikāya} as well. The unconventional translation of the title claims only a tenuous connection with the Pali—“woven” alludes to the idea that \textit{sutta} can mean “thread”, while a “cadence” is a “fall” in music, as a \textit{\textsanskrit{nipāta}} can be a fall—and sets the tone for the work as a whole. It was another attempt at a metrical translation, a style that is mostly remembered today as a curious experiment.

Twenty discourses were translated by John Ireland under the title \textit{The Discourse Collection} through the Buddhist Publications Society in Kandy in 1965. This brought a more plain and modern style to the text, in keeping with the approach developed by Ireland’s mentor, Venerable \textsanskrit{Ñāṇapoṇika}.

Students had to wait until 1984 for the Pali Text Society to produce an accurate modern translation. It came in a rather curious form. K.R. Norman’s translation was initially presented with translation alternates by I.B. Horner and Walpola Rahula. It seems the idea was to present a text simultaneously from the rigorous philologically-informed perspective of Norman, tempered with the more commentarial-leaning voices of Horner and Rahula. The result was clumsy to read, so the 1992 update abandoned the attempt and just gave Norman’s text. In his Preface, he notes that he began his translation in 1972, so it took twenty years to complete. It was accompanied by an extensive Introduction that gave a systematic presentation of relevant literary and background information, as well as extensive linguistic notes on every verse. Norman’s renderings are, as always, literal to a fault, and his text serves well as a reference work for serious students. As for dating, Norman agrees with the consensus that the \textsanskrit{Khaggavisāṇasutta}, \textsanskrit{Aṭṭhakavagga}, and \textsanskrit{Pārayanavagga} (except for the \textsanskrit{Vatthugāthā}) are the “oldest parts”. The other chapters contain texts of variable ages. He makes no claims about the date of the \textsanskrit{Suttanipāta} relative to the prose \textsanskrit{nikāyas}.

In 1985, Venerable Hammalawa Saddhatissa published his much-loved translation under the title \textit{The Sutta-\textsanskrit{Nipāta}} through Curzon. He aimed at capturing the spirit rather than the letter and doing so in a prose translation resulted in perhaps the most readable and engaging translation to date. It was this translation that first inspired my love of this text.

N.A. Jayawickrama culminated nearly 60 years of teaching Pali with his 2001 translation titled \textit{\textsanskrit{Suttanipāta}: Text and Translation} through the Post-Graduate Institute of \textsanskrit{Pāli} and Buddhist Studies at Kelaniya University. He drew on his extensive previous research in his indispensable series of articles for the \textsanskrit{Pāli} Buddhist Review from 1976–78, under the title \textit{A Critical Analysis of the \textsanskrit{Suttanipāta}}. These articles delve into many of the linguistic and formal puzzles posed by the text and are an inexhaustible mine of information. He deals at length with questions of dating, but his position may be summed up with his opening line, “The \textsanskrit{Suttanipāta} contains older and younger material side by side.” I have made extensive use of these articles for my research but unfortunately have not been able to read his translation. In this Introduction, I reference Jayawickrama’s articles by section number of \textit{A Critical Analysis of the \textsanskrit{Suttanipāta}} (CAS) as well as volume and page number of the Pali Buddhist Review (PBR).

Venerable Thanissaro has published a complete translation of the \textsanskrit{Suttanipāta}. I believe the suttas were published individually over time, but the full collection was completed no later than 2016. His long introductory essay draws together many of the themes of the text. His analysis is marred by the fact that he treats Indian “aesthetic theory” as a pre-existing body of axioms that guided the formation of the text, whereas such theories are first attested centuries later, and are not developed until more than a millennium later. On the question of dating, he points out that the existence of multiple metres in varied, sometimes innovative, forms “belies the idea, often advanced, that the style of the \textsanskrit{Suttanipāta} is consistently old, and therefore must represent an old stratum in the Pali Canon”.

A new translation under the title \textit{The Sutta \textsanskrit{Nipāta}} was made in 2015 by Laurence Khantipalo Mills with the aim of creating a more poetic rendering. Mills, who as the monk Khantipalo made major contributions to Pali studies, was in the final stages of his life, and he has since passed away. His last work was this translation, which he was making as a labor of love even as his body was falling apart. His students Michael Wells and Gary Dellora found his hand-written, incomplete manuscript, and contacted me to finish it. I edited the whole text with Laurence’s extensive notes, supplied a few new translations that were missing, and published it on SuttaCentral. I corrected some obvious mistakes in Laurence’s draft, but most of his words remain.

The most recent translation of the complete work is that by Bhikkhu Bodhi, published as \textit{The \textsanskrit{Suttanipāta}: An Ancient Collection of Buddha's Discourses Together With Its Commentaries} by Wisdom Publications in 2017. This monumental work includes the entire text with comprehensive critical apparatus, as well as a fairly complete translation of the traditional Pali commentary the \textsanskrit{Paramatthajotikā}, and relevant portions of the Niddesa. The author describes his translation as being a middle ground between the philological approach of Norman and the traditional approach of Jayawickrama; it mostly follows the commentarial interpretations, with the author noting exceptions to this. It is a highly literal translation, in some cases even more so than Norman’s, but still achieves greater readability. He does not discuss dating in detail, but says, “Linguistic and doctrinal evidence suggests that the \textsanskrit{Suttanipāta} took shape through a gradual process of accretion spread out over three or four centuries. … Several of its texts are considered to be among the most ancient specimens of Buddhist literature.”

Further translations have been made of portions of the \textsanskrit{Suttanipāta}, mostly focussing on the older sections. Ānandajoti Bhikkhu translated the \textsanskrit{Pārāyanavagga} in 1999 (revised 2007) under the title \textit{The Way to the Beyond}, as well as several individual suttas. Bhikkhu Varado translated the \textsanskrit{Aṭṭhakavagga} under the title \textit{The Group of Octads}. An engaging translation of the \textsanskrit{Aṭṭhakavagga} under the title \textit{The Way Things Really Are} by Lesley Fowler Lebkowicz and Tamara Ditrich with Primoz Pecenko dating from around 2006 is available online. Gil Fronsdale has translated the \textsanskrit{Aṭṭhakavagga} under the title \textit{The Buddha before Buddhism: Wisdom from the Early Teachings}.

Venerable \textsanskrit{Ñāṇadīpa} translated a selection of the oldest texts in the \textsanskrit{Suttanipāta} under the title \textit{The Silent Sages of Old}. It focused, as the title suggests, on the poems of the \textit{muni} or “sage”. This was published by Path Press Publications in 2018. This work brought to bear the Venerable’s deep familiarity with the text, informed by both his unparalleled knowledge of Pali verse, as well as his many years of dedicated practice in the forest, living the ideal of the \textit{muni}.

You may come across a translation of the \textsanskrit{Aṭṭhakavagga} by a certain ex-monk called \textsanskrit{Paññobhāsa}, but I would not recommend it. He is an antisemite and a Nazi sympathizer. The \textsanskrit{Aṭṭhakavagga} speaks eloquently of the harm that comes from attachment to hateful views, but it would seem that merely translating it is not enough to inoculate oneself against falling into the same trap.

As to my translation, it was prepared primarily from the \textsanskrit{Mahāsaṅgīti} edition, with occasional reference to the PTS and BJT editions. I leaned heavily on the solid work of Norman and Bodhi, and where relevant, that of \textsanskrit{Ñāṇadīpa} and Ānandajoti. I translated the \textsanskrit{Suttanipāta} last of all, having already completed all the other early sutta texts in Pali. Nonetheless, it continually posed challenges.

One distinguishing feature of my translation is that it aims at a reasonable degree of consistency with the rest of the canon. This enabled me to notice certain cases where readings in the \textsanskrit{Suttanipāta} could be better informed by readings elsewhere. In other cases, it has helped me notice the reverse: usages that are characteristic of texts here. I have discussed such cases, especially in the context of the \textsanskrit{Aṭṭhakavagga}.

\chapter*{Endnotes}
\addcontentsline{toc}{chapter}{Endnotes}
\markboth{Endnotes}{Endnotes}

\begin{enumerate}%
\item Warder, \textit{Pali Metre}, pp. 12–13. ↩

%
\item \textit{The Group of Discourses}, Introduction, §14. ↩

%
\item Snp 1.1, Snp 1.3. ↩

%
\item Snp 1.6, Snp 1.7. ↩

%
\item Snp 1.4, Snp 2.7. ↩

%
\item Snp 1.9, Snp 1.10, Snp 2.5. ↩

%
\item Snp 2.1, Snp 3.7. ↩

%
\item Snp 3.1, Snp 3.2, Snp 3.10, Snp 3.11, Snp 4.15. ↩

%
\item Snp 1.8, Snp 1.11, Snp 2.10. ↩

%
\item Snp 3.12, Snp 4.11. ↩

%
\item Snp 4.3, Snp 4.12, Snp 4.13. ↩

%
\item K.R. Norman, “The Origins of the \textsanskrit{Āryā} Metre”, in \textit{Buddhist Philosophy and Culture (Essays in honor of N.A. Jayawickrema)}, Colombo 1987, pp. 203–214. ↩

%
\item Warder, \textit{Pali Metre}, 10. ↩

%
\item CAS 17: PBR 2,1 1977, p. 18. ↩

%
\item BrhUp 4.4.8; cp. SN 12.65:7.1. ↩

%
\item BrhUp 4.3.35; cp. DN 16:2.25.11. ↩

%
\item BrhUp 4.3.15 ff., BrhUp 4.3.34; cp. DN 2:95.2. ↩

%
\item BrhUp 4.3.33; cp. AN 8.43. ↩

%
\item BrhUp 4.3.23 ff.; cp. Snp 4.12:10.1. ↩

%
\item BrhUp 4.3.36; cp. DN 1:3.73.4. ↩

%
\item BrhUp 4.4.12: \textit{\textsanskrit{ayamasmīti}}; cp. SN 35.248:3.1: \textit{\textsanskrit{ayamahamasmīti}}). ↩

%
\item BrhUp 4.4.14; cp. AN 4.171:5.2. ↩

%
\item BrhUp 4.4.21; cp. MN 19:8.8. ↩

%
\item (BrhUp 4.4.22; cp. Snp 1.3:1.3. ↩

%
\item BrhUp 4.4.25; cp. Ud 8.3:3.1, SN 6.13:3.4. ↩

%
\item BrhUp 4.4.23; cp. AN 8.19:15.2. ↩

%
\item BrhUp 4.3.22; cp. AN 11.9:7.3. ↩

%
\item BrhUp 4.4.13: \textit{tasya lokaḥ, sa u loka eva}; cp. SN 22.81:10.7: \textit{so \textsanskrit{attā} so loko}). ↩

%
\item DN 1 and DN 2, or MN 1 and MN 2. ↩

%
\item CAS 60: PBR 2,3 1977, p. 151. ↩

%
\item Richard Salomon, \textit{A \textsanskrit{Gāndhārī} Version of the Rhinoceros \textsanskrit{Sūtra}}, University of Washington, 2000, pp. 115–201. ↩

%
\item \textit{‘Wander Alone Like the Rhinoceros!’: The Solitary, Itinerant Renouncer in Ancient Indian \textsanskrit{Gāthā}-Poetry}. In: Larsson, S. and af Edholm, K. (eds.) \textit{Songs on the Road: Wandering Religious Poets in India, Tibet, and Japan}. Pp. 35–66. Stockholm: Stockholm University Press. ↩

%
\item \textit{Like the Rhinoceros, or Like Its Horn? The Problem of \textsanskrit{Khaggavisāṇa} Revisited}, Buddhist Studies Review, 31.2 (2014) 165–178. ↩

%
\item This line is in Pali at SN 6.3:7.3. ↩

%
\item CAS 63: PBR 2,3 1977, p. 154. ↩

%
\item T 211, pp. 575–609; translated by Charles Willemen as \textit{The Scriptural Text: Verses of the Doctrine, with Parables}, Numata Center for Buddhist Translation, 1999, pp. 215–219. ↩

%
\item Snp 2.7:2.1: \textit{\textsanskrit{saññatattā}}; Baudh 1.5.10.31: \textit{\textsanskrit{saṁyata} \textsanskrit{ātmā}}. ↩

%
\item Snp 2.7:2.1: \textit{tapassino}; Baudh 1.5.10.33: \textit{\textsanskrit{tapasvī}}. ↩

%
\item Snp 2.7:3.1: \textit{na \textsanskrit{pasū} \textsanskrit{brāhmaṇānāsuṁ}}; Baudh 1.10.18.4: \textit{\textsanskrit{viṭsvadhyayana}-\textsanskrit{yajanadāna}-\textsanskrit{kṛṣi}-\textsanskrit{vāṇijya}-\textsanskrit{paśupālana}}. ↩

%
\item Snp 2.7:3.2: \textit{\textsanskrit{sajjhāyadhanadhaññāsuṁ} \textsanskrit{brahmaṁ} \textsanskrit{nidhimapālayuṁ}}; Baudh 1.10.18.2 \textit{brahma vai svaṃ \textsanskrit{mahimānaṃ} \textsanskrit{brāhmaṇeṣv} \textsanskrit{adadhādadhyayana}}. ↩

%
\item Baudh 1.2.3.16, Baudh 2.10.18.4, Baudh 2.10.18.22: \textit{\textsanskrit{bhaikṣa}-\textsanskrit{arthī} \textsanskrit{grāmam} anvicchet}. ↩

%
\item Snp 2.7:4.1; Baudh 2.10.18.12: \textit{\textsanskrit{ayācitam} \textsanskrit{asaṁkḷptam}}. ↩

%
\item Snp 2.7:5.1: \textit{\textsanskrit{nānārattehi} vatthehi}; Baudh 1.6.13.9: \textit{citra-\textsanskrit{vāsasaś} citra-\textsanskrit{āsaṅgāvṛṣākapāv}}. ↩

%
\item Snp 2.7:9: \textit{\textsanskrit{aññatra} \textsanskrit{tamhā} \textsanskrit{samayā} \textsanskrit{utuveramaṇiṁ} pati \textsanskrit{antarā} \textsanskrit{methunaṁ} \textsanskrit{dhammaṁ} \textsanskrit{nāssu} gacchanti}; Baudh 4.1.19: \textit{\textsanskrit{ṛtau} na upaiti yo \textsanskrit{bhāryām} \textsanskrit{anṛtau} \textsanskrit{yaś} ca gacchati}. ↩

%
\item Snp 2.7:10.3: \textit{\textsanskrit{avihiṁsa}}. ↩

%
\item Baudh 1.5.8.2: \textit{\textsanskrit{ahiṃsayā} ca \textsanskrit{bhūtātmā} manaḥ satyena \textsanskrit{śudhyati}}. ↩

%
\item Baudh 2.10.18.2. ↩

%
\item Baudh 3.10.13. ↩

%
\item Baudh 3.3.19: \textit{na druhyed \textsanskrit{daṃśa}.\textsanskrit{maśakān} \textsanskrit{himavāṃs} \textsanskrit{tāpaso} bhavet}. ↩

%
\item Baudh 2.9.16.3: \textit{\textsanskrit{prajām} \textsanskrit{utpādayed} yuktaḥ sve sve \textsanskrit{varṇe}}. ↩

%
\item Rig Veda 1.164.40, Wilson translation. ↩

%
\item RV 10.86.13–4. ↩

%
\item RV 10.27.2, RV 10.28.3. ↩

%
\item RV 6.17.11, RV 5.29.7, RV 8.12.8. ↩

%
\item RV 8.43.11. ↩

%
\item RV 10.85.13. ↩

%
\item TaittB 3.9.8. ↩

%
\item AitB 3.4. ↩

%
\item BrhUp 6.4.18. ↩

%
\item ŚpB 13.6.2.18. ↩

%
\item ŚpB 13.5.4.24. ↩

%
\item ŚpB 13.7.1.13. ↩

%
\item ŚpB 13.5.4.27. ↩

%
\item ŚpB 13.4.1.13. ↩

%
\item ŚpB 13.4.2.10. ↩

%
\item ŚpB 13.5.4.7. ↩

%
\item ŚpB 13.8.4.10. ↩

%
\item ŚpB 13.5.4.15. ↩

%
\item ŚpB 13.6.2.13. ↩

%
\item Louis de la Vallée Poussin, review of C.A.F Rhys Davids’ edition of the \textsanskrit{Dukapaṭṭhāna}, \textit{Journal of the Royal Asiatic Society}, 1907, p. 453. ↩

%
\item \textit{The Arthapada Sutra}, Visva-Bharati, Santiniketan, 1951. ↩

%
\item Snp 4.3:1.3, Snp 4.3:8.1, Snp 4.9:12.4, Snp 4.13:3.3, Snp 4.13:17.1, Snp 4.13:3.2. ↩

%
\item Snp 4.3:2.2, Snp 4.13:19.2. ↩

%
\item Snp 4.11:15.4, Snp 4.12:1.2, Snp 4.12:2.2, Snp 4.12:2.4, Snp 4.12:4.2, Snp 4.12:8.2, Snp 4.12:11.3, Snp 4.13:4.4, Snp 4.13:9.4, Snp 4.13:15.4. ↩

%
\item Snp 4.3:2.4, Snp 4.4:1.3, Snp 4.4:1.4, Snp 4.4:2.2, Snp 4.5:4.2, Snp 4.5:5.2, Snp 4.9:5.1, Snp 4.9:5.4, Snp 4.9:6.1, Snp 4.9:6.4, Snp 4.12:1.3, Snp 4.13:14.1, Snp 4.13:17.2. ↩

%
\item Snp 4.3:5.1, Snp 4.3:7.2, Snp 4.4:6.4, Snp 4.4:7.1, Snp 4.5:4.1, Snp 4.5:7.2, Snp 4.5:7.4, Snp 4.5:8.1, Snp 4.9:4.1, Snp 4.10:13.4, Snp 4.12:9.3, Snp 4.13:8.2, Snp 4.13:16.2, Snp 4.13:17.1, Snp 4.13:20.4, Snp 4.15:11.3. ↩

%
\item Snp 4.4:6.4, Snp 4.5:7.4, Snp 4.13:4.4, Snp 4.14:4.4. ↩

%
\item Snp 4.3:5.2, Snp 4.4:7.1, Snp 4.5:8.1, Snp 4.9:10.3, Snp 4.10:2.5, Snp 4.10:12.3, Snp 4.13:16.2. In a positive sense “prioritize wisdom” at Snp 4.16:15.1. ↩

%
\item Snp 4.4:1.4, Snp 4.5:5.4, Snp 4.5:8.4, Snp 4.9:6.7, Snp 4.13:14.2. ↩

%
\item Snp 4.3:6.2, Snp 4.4:8.2, Snp 4.4:8.4; Snp 4.5:2.3, Snp 4.5:6.4, Snp 4.8:9.1, Snp 4.8:10.4, Snp 4.9:3.3, Snp 4.9:3.4; Snp 4.9:4.3, Snp 4.9:5.6, Snp 4.9:7.3, Snp 4.9:11.2, Snp 4.9:13.3, Snp 4.13:13.2, Snp 4.13:17.4; see also Snp 2.12:11.1. ↩

%
\item Snp 4.3:6.2, Snp 4.5:6.4, Snp 4.9:3.3, Snp 4.9:4.1, Snp 4.11:5.2, Snp 4.11:6.4, Snp 4.12:10.3, Snp 4.12:17.1, Snp 4.12:17.3, Snp 4.13:13.2. ↩

%
\item Snp 4.3:6.1, Snp 4.5:6.3, Snp 4.8:1.4, Snp 4.9:12.4, Snp 4.12:15.3, Snp 4.13:16.1, Snp 4.13:19.2. ↩

%
\item Snp 4.3:2.3, Snp 4.4:5.1, Snp 4.12:4.4, Snp 4.12:12.1, Snp 4.12:12.4, Snp 4.13:4.2, Snp 4.16:8.1. Also see \textit{atta} below. ↩

%
\item Snp 4.5:1.1, Snp 4.12:1.1, Snp 4.12:3.4, Snp 4.13:1.1. ↩

%
\item K.N. Jayatilleke, \textit{Early Buddhist Theory of Knowledge}, George Allen \& Unwin, 1963, §§ 68–74. ↩

%
\end{enumerate}

%
\chapter*{Acknowledgements}
\addcontentsline{toc}{chapter}{Acknowledgements}
\markboth{Acknowledgements}{Acknowledgements}

I remember with gratitude all those from whom I have learned the Dhamma, especially Ajahn Brahm and Bhikkhu Bodhi, the two monks who more than anyone else showed me the depth, meaning, and practical value of the Suttas.

Special thanks to Dustin and Keiko Cheah and family, who sponsored my stay in Qi Mei while I made this translation.

Thanks also for Blake Walshe, who provided essential software support for my translation work.

Throughout the process of translation, I have frequently sought feedback and suggestions from the community on the SuttaCentral community on our forum, “Discuss and Discover”. I want to thank all those who have made suggestions and contributed to my understanding, as well as to the moderators who have made the forum possible. My thinking around the economics of meat-eating in the Āmagandhasutta was prompted by the forum users Bhikkhu Khemaratana and Prajnadeva. A special thanks is due to \textsanskrit{Sabbamittā}, a true friend of all, who has tirelessly and precisely checked my work. 

Finally my everlasting thanks to all those people, far too many to mention, who have supported SuttaCentral, and those who have supported my life as a monastic. None of this would be possible without you.

%
\mainmatter%
\pagestyle{fancy}%
\addtocontents{toc}{\let\protect\contentsline\protect\nopagecontentsline}
\part*{Anthology of Discourses}
\addcontentsline{toc}{part}{Anthology of Discourses}
\markboth{}{}
\addtocontents{toc}{\let\protect\contentsline\protect\oldcontentsline}

%
\addtocontents{toc}{\let\protect\contentsline\protect\nopagecontentsline}
\chapter*{The Serpent Chapter}
\addcontentsline{toc}{chapter}{\tocchapterline{The Serpent Chapter}}
\addtocontents{toc}{\let\protect\contentsline\protect\oldcontentsline}

%
\section*{{\suttatitleacronym Snp 1.1}{\suttatitletranslation The Snake }{\suttatitleroot Uragasutta}}
\addcontentsline{toc}{section}{\tocacronym{Snp 1.1} \toctranslation{The Snake } \tocroot{Uragasutta}}
\markboth{The Snake }{Uragasutta}
\extramarks{Snp 1.1}{Snp 1.1}

\begin{verse}%
When\marginnote{1.1} anger surges, they drive it out, \\
as with medicine a snake’s spreading venom. \\
Such a mendicant sheds this world and the next, \\
as a snake its old worn-out skin. 

They’ve\marginnote{2.1} cut off greed entirely, \\
like a lotus plucked flower and stalk. \\
Such a mendicant sheds this world and the next, \\
as a snake its old worn-out skin. 

They’ve\marginnote{3.1} cut off craving entirely, \\
drying up that swift-flowing stream. \\
Such a mendicant sheds this world and the next, \\
as a snake its old worn-out skin. 

They’ve\marginnote{4.1} swept away conceit entirely, \\
as a fragile bridge of reeds by a great flood. \\
Such a mendicant sheds this world and the next, \\
as a snake its old worn-out skin. 

In\marginnote{5.1} future lives they find no substance, \\
as an inspector of fig trees finds no flower. \\
Such a mendicant sheds this world and the next, \\
as a snake its old worn-out skin. 

They\marginnote{6.1} hide no anger within, \\
gone beyond any kind of existence. \\
Such a mendicant sheds this world and the next, \\
as a snake its old worn-out skin. 

Their\marginnote{7.1} mental vibrations are cleared away,\footnote{Read \textit{sa ve}. } \\
internally clipped off entirely. \\
Such a mendicant sheds this world and the next, \\
as a snake its old worn-out skin. 

They\marginnote{8.1} have not run too far nor run back, \\
but have gone beyond all this proliferation. \\
Such a mendicant sheds this world and the next, \\
as a snake its old worn-out skin. 

They\marginnote{9.1} have not run too far nor run back, \\
for they know that nothing in the world \\>is what it seems.\footnote{This line and the next echo the practices of Jain ascetics: \textit{naehibhaddantiko, \textsanskrit{natiṭṭhabhaddantiko}, \textsanskrit{nābhihaṭaṁ}}. The sense of the first two terms is similar to \textit{\textsanskrit{avhāna}}, while the last of these terms is the same as \textit{\textsanskrit{abhihāra}} in the next line. } \\
Such a mendicant sheds this world and the next, \\
as a snake its old worn-out skin. 

They\marginnote{10.1} have not run too far nor run back, \\
knowing nothing is what it seems, free of greed. \\
Such a mendicant sheds this world and the next, \\
as a snake its old worn-out skin. 

They\marginnote{11.1} have not run too far nor run back, \\
knowing nothing is what it seems, free of lust. \\
Such a mendicant sheds this world and the next, \\
as a snake its old worn-out skin. 

They\marginnote{12.1} have not run too far nor run back, \\
knowing nothing is what it seems, free of hate. \\
Such a mendicant sheds this world and the next, \\
as a snake its old worn-out skin. 

They\marginnote{13.1} have not run too far nor run back, \\
knowing nothing is what it seems, free of delusion. \\
Such a mendicant sheds this world and the next, \\
as a snake its old worn-out skin. 

They\marginnote{14.1} have no underlying tendencies at all, \\
and are rid of unskillful roots, \\
Such a mendicant sheds this world and the next, \\
as a snake its old worn-out skin. 

They\marginnote{15.1} have nothing born of distress at all, \\
that might cause them to come back to this world. \\
Such a mendicant sheds this world and the next, \\
as a snake its old worn-out skin. 

They\marginnote{16.1} have nothing born of entanglement at all, \\
that would shackle them to a new life. \\
Such a mendicant sheds this world and the next, \\
as a snake its old worn-out skin. 

They’ve\marginnote{17.1} given up the five hindrances, \\
untroubled, rid of doubt, free of thorns. \\
Such a mendicant sheds this world and the next, \\
as a snake its old worn-out skin. 

%
\end{verse}

%
\section*{{\suttatitleacronym Snp 1.2}{\suttatitletranslation With Dhaniya the Cowherd }{\suttatitleroot Dhaniyasutta}}
\addcontentsline{toc}{section}{\tocacronym{Snp 1.2} \toctranslation{With Dhaniya the Cowherd } \tocroot{Dhaniyasutta}}
\markboth{With Dhaniya the Cowherd }{Dhaniyasutta}
\extramarks{Snp 1.2}{Snp 1.2}

\begin{verse}%
“I’ve\marginnote{1.1} boiled my rice and drawn my milk,” \\
\scspeaker{said Dhaniya the cowherd, }\\
“I stay with my family along the bank of the \textsanskrit{Mahī}. \\
My hut is roofed, my fire kindled: \\
so rain, sky, if you wish.” 

“I\marginnote{2.1} boil not with anger and have drawn out hard-heartedness,” \\
\scspeaker{said the Buddha, }\\
“I stay for one night along the bank of the \textsanskrit{Mahī}. \\
My hut is wide open, my fire is quenched: \\
so rain, sky, if you wish.” 

“No\marginnote{3.1} gadflies or mosquitoes are found,” \\
\scspeaker{said Dhaniya, }\\
“cows graze on the lush meadow grass. \\
They get by even when the rain comes: \\
so rain, sky, if you wish.” 

“I\marginnote{4.1} bound a raft and made it well,” \\
\scspeaker{said the Buddha,\footnote{Bodhi’s “should not behave rashly” and Norman’s “should not pursue his search for food inconsiderately” confuse a plain meaning. When on alms round a mendicant should not walk too fast, else the families have no time to ready the food. } }\\
“and with it I crossed over, went beyond, and dispelled the flood. \\
Now I have no need for a raft: \\
so rain, sky, if you wish.” 

“My\marginnote{5.1} wife is obedient, not wanton,” \\
\scspeaker{said Dhaniya, }\\
“long have we lived together happily. \\
I hear nothing bad about her: \\
so rain, sky, if you wish.” 

“My\marginnote{6.1} mind is obedient and freed,” \\
\scspeaker{said the Buddha, }\\
“long nurtured and well-tamed. \\
Nothing bad is found in me: \\
so rain, sky, if you wish.” 

“I\marginnote{7.1} am self-employed,” \\
\scspeaker{said Dhaniya, }\\
“and my healthy children likewise.\footnote{As above, \textit{\textsanskrit{uccāvaca}} probably means “diverse” practices rather than “high and low” per commentary followed by Bodhi and Norman. The only usage of \textit{\textsanskrit{uccāvaca}} in the context of practice is the “diverse duties” performed by mendicants for their fellows (eg. DN 33:3.3.17). } \\
I hear nothing bad about them: \\
so rain, sky, if you wish.” 

“I\marginnote{8.1} am no-one’s lackey,” \\
\scspeaker{said the Buddha, }\\
“with what I have earned I wander the world. \\
I have no need for wages: \\
so rain, sky, if you wish.” 

“I\marginnote{9.1} have heifers and sucklings,” \\
\scspeaker{said Dhaniya, }\\
“cows in calf and breeding cows. \\
I’ve also got a bull, leader of the herd here: \\
so rain, sky, if you wish.” 

“I\marginnote{10.1} have no heifers or sucklings,” \\
\scspeaker{said the Buddha, }\\
“no cows in calf or breeding cows. \\
I haven’t got a bull, leader of the herd here: \\
so rain, sky, if you wish.” 

“The\marginnote{11.1} stakes are driven in, unshakable,” \\
\scspeaker{said Dhaniya, }\\
“The grass halters are new and well-woven, \\
not even the sucklings can break them: \\
so rain, sky, if you wish.” 

“Like\marginnote{12.1} a bull I broke the bonds,” \\
\scspeaker{said the Buddha, }\\
“like an elephant I snapped the vine. \\
I will never lie in a womb again: \\
so rain, sky, if you wish.” 

Right\marginnote{13.1} then a thundercloud rained down, \\
soaking the uplands and valleys. \\
Hearing the sky rain down, \\
Dhaniya said this: 

“It\marginnote{14.1} is no small gain for us \\
that we have seen the Buddha. \\
We come to you for refuge, Seer. \\
O great sage, please be our Teacher. 

My\marginnote{15.1} wife and I, obedient, \\
shall lead the spiritual life under the Holy One\footnote{Norman’s proposal that this is not the Buddha is implausible. } \\
Gone beyond birth and death, \\
we shall make an end of suffering.” 

“Your\marginnote{16.1} children bring you delight!” \\
\scspeaker{said \textsanskrit{Māra} the Wicked, }\\
“Your cattle also bring you delight! \\
For attachments are a man’s delight; \\
without attachments there’s no delight.” 

“Your\marginnote{17.1} children bring you sorrow,” \\
\scspeaker{said the Buddha, }\\
“Your cattle also bring you sorrow. \\
For attachments are a man’s sorrow; \\
without attachments there are no sorrows.” 

%
\end{verse}

%
\section*{{\suttatitleacronym Snp 1.3}{\suttatitletranslation The Rhinoceros Horn }{\suttatitleroot Khaggavisāṇasutta}}
\addcontentsline{toc}{section}{\tocacronym{Snp 1.3} \toctranslation{The Rhinoceros Horn } \tocroot{Khaggavisāṇasutta}}
\markboth{The Rhinoceros Horn }{Khaggavisāṇasutta}
\extramarks{Snp 1.3}{Snp 1.3}

\begin{verse}%
When\marginnote{1.1} you’ve laid down arms toward all creatures, \\
not harming even a single one, \\
don’t wish for a child, let alone a companion: \\
live alone like a rhino’s horn. 

Those\marginnote{2.1} with close relationships have affection, \\
following which this pain arises. \\
Seeing this danger born of affection, \\
live alone like a rhino’s horn. 

When\marginnote{3.1} feelings for friends and loved ones \\
are tied up in selfish love, you miss out on the goal. \\
Seeing this peril in intimacy, \\
live alone like a rhino’s horn. 

As\marginnote{4.1} a spreading bamboo gets entangled, \\
so does concern for partners and children. \\
Like a bamboo shoot unimpeded, \\
live alone like a rhino’s horn. 

As\marginnote{5.1} a wild deer loose in the forest \\
grazes wherever it wants, \\
a smart person looking for freedom would \\
live alone like a rhino’s horn. 

When\marginnote{6.1} among friends, whether staying in place \\
or going on a journey, you’re always on call. \\
Looking for the uncoveted freedom, \\
live alone like a rhino’s horn. 

Among\marginnote{7.1} friends you have fun and games, \\
and for children you are full of love. \\
Though loathe to depart from those you hold dear, \\
live alone like a rhino’s horn. 

At\marginnote{8.1} ease in any quarter, unresisting, \\
content with whatever comes your way; \\
prevailing over adversities, dauntless, \\
live alone like a rhino’s horn. 

Even\marginnote{9.1} some renunciates are hard to please, \\
as are some layfolk dwelling at home. \\
Don’t worry about others’ children, \\
live alone like a rhino’s horn. 

Having\marginnote{10.1} shed the marks of the home life, \\
like the fallen leaves of the Shady Orchid Tree; \\
having cut the bonds of the home life, a hero would \\
live alone like a rhino’s horn. 

If\marginnote{11.1} you find an alert companion, \\
a wise and virtuous friend, \\
then, overcoming all adversities, \\
wander with them, joyful and mindful. 

If\marginnote{12.1} you find no alert companion, \\
no wise and virtuous friend, \\
then, like a king who flees his conquered realm, \\
wander alone like a tusker in the wilds. 

Clearly\marginnote{13.1} we praise the blessing of a friend, \\
it’s good to be with friends your equal or better. \\
but failing to find them, eating blamelessly, \\
live alone like a rhino’s horn. 

Though\marginnote{14.1} made of shining gold, well-finished by a smith, \\
when two bracelets share the same arm \\
they clash up against each other. Seeing this, \\
live alone like a rhino’s horn. 

Thinking,\marginnote{15.1} “So too, if I had a partner, \\
there’d be flattery or curses.”\footnote{This verse is quoted by the \textsanskrit{Kathāvatthu} (Kv 1.2:113.1) to refute the thesis that an arahant might fall away from their attainment. This is overlooked by both Bodhi and Norman, although it evidently underlies the commentary. } \\
Seeing this peril in the future, \\
live alone like a rhino’s horn. 

Sensual\marginnote{16.1} pleasures are diverse, sweet, delightful, \\
appearing in disguise they disturb the mind. \\
Seeing danger in the many kinds of sensual stimulation, \\
live alone like a rhino’s horn. 

This\marginnote{17.1} is a calamity, a boil, a disaster, \\
an illness, a dart, and a danger for me. \\
Seeing this peril in sensuality, \\
live alone like a rhino’s horn. 

Heat\marginnote{18.1} and cold, hunger and thirst, \\
wind and sun, flies and snakes: \\
having put up with all these things, \\
live alone like a rhino’s horn. 

As\marginnote{19.1} a full-grown elephant, lotus-eating, magnificent, \\
forsaking the herd,\footnote{Both Bodhi and Norman discuss this line, without any compelling conclusion. The problematic \textit{\textsanskrit{mutaṁ}} cannot mean “experienced” per Bodhi and Norman following the commentary. Since the \textsanskrit{Kathāvatthu} is by far the earliest source on this verse, I suggest we adopt its reading that the verse is about “falling away”, and amend \textit{\textsanskrit{mutaṁ}} to \textit{\textsanskrit{cutaṁ}}. } \\
stays where it wants in the forest, \\
live alone like a rhino’s horn. 

It’s\marginnote{20.1} impossible for one who delights in company \\
to experience even temporary freedom. \\
Heeding the speech of the Kinsman of the Sun, \\
live alone like a rhino’s horn. 

Thinking,\marginnote{21.1} “I am one who has left warped views behind, \\
has reached the sure way, has gained the path, \\
has given rise to knowledge, and needs no-one to guide me”, \\
live alone like a rhino’s horn. 

No\marginnote{22.1} greed, no guile, no thirst, no slur, \\
dross and delusion is smelted off;\footnote{The PTS edition, as does the BJT, has a reciter’s remark here identifying the speaker as the Buddha, and this is translated without comment by Norman and Bodhi. It is unusual, if not unique, to find such a mark in the middle of a series of verses by one speaker. The \textsanskrit{Mahāsaṅgīti} edition, following the VRI, lacks this remark and I have translated accordingly. } \\
free of hoping for anything in the world, \\
live alone like a rhino’s horn. 

Avoid\marginnote{23.1} a wicked companion, \\
blind to the good, habitually immoral. \\
One ought not befriend the heedless and hankering, but \\
live alone like a rhino’s horn. 

Spend\marginnote{24.1} time with a learned expert who has memorized the teachings, \\
an eloquent and uplifting friend. \\
When you understand the meanings and have dispelled doubt, \\
live alone like a rhino’s horn. 

When\marginnote{25.1} you realize that worldly fun and games \\
and pleasure are unsatisfying, disregarding them, \\
as one unadorned, a speaker of truth, \\
live alone like a rhino’s horn. 

Children,\marginnote{26.1} partner, father, mother, \\
wealth and grain and relatives: \\
having given up sensual pleasures to this extent, \\
live alone like a rhino’s horn. 

“This\marginnote{27.1} is a snare. Here there’s hardly any happiness, \\
little gratification, and it’s full of drawbacks. \\
It’s a hook.” Knowing this, a thoughtful person would \\
live alone like a rhino’s horn. 

Having\marginnote{28.1} burst apart the fetters, \\
like a fish that tears the net and swims free, \\
or a fire not returning to ground it has burned, \\
live alone like a rhino’s horn. 

Eyes\marginnote{29.1} downcast, not footloose, \\
senses guarded, mind protected, \\
uncorrupted, not burning with desire, \\
live alone like a rhino’s horn. 

Having\marginnote{30.1} shed the marks of the home life, \\
like the fallen leaves of the Shady Orchid Tree, \\
and gone forth in the ocher robe, \\
live alone like a rhino’s horn. 

Not\marginnote{31.1} wanton, nor rousing greed for tastes, \\
providing for no other, wandering indiscriminately for alms, \\
not attached to this family or that, \\
live alone like a rhino’s horn. 

When\marginnote{32.1} you’ve given up five mental obstacles, \\
and expelled all corruptions, \\
and cut off affection and hate, being independent, \\
live alone like a rhino’s horn. 

When\marginnote{33.1} you’ve put pleasure and pain behind you, \\
and former happiness and sadness, \\
and gained equanimity serene and pure, \\
live alone like a rhino’s horn. 

With\marginnote{34.1} energy roused to reach the ultimate goal, \\
not sluggish in mind or lazy, \\
vigorous, strong and powerful, \\
live alone like a rhino’s horn. 

Not\marginnote{35.1} neglecting retreat and absorption, \\
always living in line with the teachings, \\
comprehending the danger in rebirths, \\
live alone like a rhino’s horn. 

One\marginnote{36.1} whose aim is the ending of craving—\\
diligent, clever, learned, mindful, resolute—\\
who has assessed the teaching and is bound for awakening, should \\
live alone like a rhino’s horn. 

Like\marginnote{37.1} a lion not startled by sounds, \\
like wind not caught in a net, \\
like water not sticking to a lotus, \\
live alone like a rhino’s horn. 

Like\marginnote{38.1} the fierce-fanged lion, king of beasts, \\
who wanders as victor and master, \\
you should frequent remote lodgings, and \\
live alone like a rhino’s horn. 

In\marginnote{39.1} time, cultivate freedom through \\
love, compassion, rejoicing, and equanimity. \\
Not upset by anything in the world, \\
live alone like a rhino’s horn. 

Having\marginnote{40.1} given up greed, hate, and delusion, \\
having burst apart the fetters, \\
unafraid at the end of life, \\
live alone like a rhino’s horn. 

They\marginnote{41.1} befriend you and serve you for their own sake; \\
these days it’s hard to find friends lacking ulterior motive. \\
Impure folk cleverly profit themselves—\\
live alone like a rhino’s horn. 

%
\end{verse}

%
\section*{{\suttatitleacronym Snp 1.4}{\suttatitletranslation With Bhāradvāja the Farmer }{\suttatitleroot Kasibhāradvājasutta}}
\addcontentsline{toc}{section}{\tocacronym{Snp 1.4} \toctranslation{With Bhāradvāja the Farmer } \tocroot{Kasibhāradvājasutta}}
\markboth{With Bhāradvāja the Farmer }{Kasibhāradvājasutta}
\extramarks{Snp 1.4}{Snp 1.4}

\scevam{So\marginnote{1.1} I have heard. }At one time the Buddha was staying in the land of the Magadhans in the Southern Hills near the brahmin village of \textsanskrit{Ekanāḷa}. Now at that time the brahmin \textsanskrit{Bhāradvāja} the Farmer had harnessed around five hundred plows, it being the season for sowing. Then the Buddha robed up in the morning and, taking his bowl and robe, went to where \textsanskrit{Bhāradvāja} the Farmer was working. Now at that time \textsanskrit{Bhāradvāja} the Farmer was distributing food. Then the Buddha went to where the distribution was taking place and stood to one side. 

\textsanskrit{Bhāradvāja}\marginnote{2.1} the Farmer saw him standing for alms and said to him, “I plough and sow, ascetic, and then I eat. You too should plough and sow, then you may eat.” 

“I\marginnote{3.1} too plough and sow, brahmin, and then I eat.” “I don’t see Master Gotama with a yoke or plow or plowshare or goad or oxen, yet he says: ‘I too plough and sow, brahmin, and then I eat.’” 

Then\marginnote{4.1} \textsanskrit{Bhāradvāja} the Farmer addressed the Buddha in verse: 

\begin{verse}%
“You\marginnote{5.1} claim to be a farmer, \\
but I don’t see you farming. \\
Tell me your farming when asked, \\
so I can recognize your farming.” 

“Faith\marginnote{6.1} is my seed, austerity my rain, \\
and wisdom is my yoke and plough. \\
Conscience is my pole, mind my strap, \\
mindfulness my plowshare and goad. 

Guarded\marginnote{7.1} in body and speech, \\
I restrict my intake of food. \\
I use truth as my scythe, \\
and gentleness is my release. 

Energy\marginnote{8.1} is my beast of burden, \\
transporting me to a place of sanctuary. \\
It goes without turning back \\
where there is no sorrow. 

That’s\marginnote{9.1} how to do the farming \\
that has the Deathless as its fruit. \\
When you finish this farming \\
you’re released from all suffering.” 

%
\end{verse}

Then\marginnote{10.1} \textsanskrit{Bhāradvāja} the Farmer filled a large bronze dish with milk-rice and presented it to the Buddha: “Eat the milk-rice, Master Gotama, you are truly a farmer. For Master Gotama does the farming that has the Deathless as its fruit.” 

\begin{verse}%
“Food\marginnote{11.1} enchanted by a spell isn’t fit for me to eat. \\
That’s not the principle of those who see, brahmin. \\
The Buddhas reject things enchanted with spells. \\
Since there is such a principle, brahmin, that’s how they live. 

Serve\marginnote{12.1} with other food and drink \\
the consummate one, the great hermit, \\
with defilements ended and remorse stilled. \\
For he is the field for the seeker of merit.” 

%
\end{verse}

“Then,\marginnote{13.1} Master Gotama, to whom should I give the milk-rice?” “Brahmin, I don’t see anyone in this world—with its gods, \textsanskrit{Māras}, and \textsanskrit{Brahmās}, this population with its ascetics and brahmins, its gods and humans—who can properly digest this milk-rice, except for the Realized One or one of his disciples. Well then, brahmin, throw out the milk-rice where there is little that grows, or drop it into water that has no living creatures.” 

So\marginnote{14.1} \textsanskrit{Bhāradvāja} the Farmer dropped the milk-rice in water that had no living creatures. And when the milk-rice was placed in the water, it sizzled and hissed, steaming and fuming. Suppose there was an iron cauldron that had been heated all day. If you placed it in the water, it would sizzle and hiss, steaming and fuming. In the same way, when the milk-rice was placed in the water, it sizzled and hissed, steaming and fuming. 

Then\marginnote{15.1} \textsanskrit{Bhāradvāja} the Farmer, shocked and awestruck,  went up to the Buddha, bowed down with his head at the Buddha’s feet, and said, “Excellent, Master Gotama! Excellent! As if he were righting the overturned, or revealing the hidden, or pointing out the path to the lost, or lighting a lamp in the dark so people with good eyes can see what’s there, Master Gotama has made the teaching clear in many ways. I go for refuge to Master Gotama, to the teaching, and to the mendicant \textsanskrit{Saṅgha}. Sir, may I receive the going forth, the ordination in the Buddha’s presence?” 

And\marginnote{16.1} \textsanskrit{Bhāradvāja} the Farmer received the going forth, the ordination in the Buddha’s presence. Not long after his ordination, Venerable \textsanskrit{Bhāradvāja}, living alone, withdrawn, diligent, keen, and resolute, soon realized the supreme end of the spiritual path in this very life. He lived having achieved with his own insight the goal for which gentlemen rightly go forth from the lay life to homelessness. He understood: “Rebirth is ended; the spiritual journey has been completed; what had to be done has been done; there is no return to any state of existence.” And Venerable \textsanskrit{Bhāradvāja} became one of the perfected. 

%
\section*{{\suttatitleacronym Snp 1.5}{\suttatitletranslation With Cunda }{\suttatitleroot Cundasutta}}
\addcontentsline{toc}{section}{\tocacronym{Snp 1.5} \toctranslation{With Cunda } \tocroot{Cundasutta}}
\markboth{With Cunda }{Cundasutta}
\extramarks{Snp 1.5}{Snp 1.5}

\begin{verse}%
“I\marginnote{1.1} ask the sage abounding in wisdom,” \\
\scspeaker{said Cunda the smith, }\\
“the Buddha, master of the teaching, free of craving, \\
best of men, excellent charioteer, please tell me this: \\
how many ascetics are there in the world?” 

“There\marginnote{2.1} are four ascetics, not a fifth.” \\
\scspeaker{said the Buddha to Cunda, }\\
“Being asked to bear witness, I will explain them to you:\footnote{By forcing the commentary’s sense of \textit{kitti} into \textit{nigghosa}, Bodhi and Norman confuse a simple meaning. \textit{Nighosa} (more commonly spelled \textit{nigghosa}) just means “word, teaching, statement”, eg. \textit{tava \textsanskrit{sutvāna} \textsanskrit{nigghosaṁ}} “having heard your teaching”. } \\
the path-victor, the path-teacher, \\
the path-liver, and the path-wrecker.” 

“Who\marginnote{3.1} is a path-victor according to the Buddhas?” \\
\scspeaker{said Cunda the smith, }\\
“and how is one an unequaled path-explainer? \\
Tell me when asked about one who lives the path, \\
then declare the path-wrecker.” 

“Rid\marginnote{4.1} of doubt, free of thorns, \\
delighting in quenching, not fawning, \\
a guide for the world with its gods. \\
The Buddhas say one such is victor of the path. 

Knowing\marginnote{5.1} the ultimate as ultimate, \\
they explain and analyze the teaching right here. \\
That sage unstirred, with doubt cut off, \\
is the second mendicant, I say, the path-teacher. 

Living\marginnote{6.1} restrained and mindful on the path \\
of the well-taught passages of teaching,\footnote{SN 36.2:3.4 has \textit{virajjati}. See Bodhi’s note 1735 for the readings and commentary in both contexts. } \\
cultivating blameless states, \\
is the third mendicant, I say, the path-liver. 

Dressed\marginnote{7.1} like one true to their vows, \\
pushy, rude, a corrupter of families, \\
devious, unrestrained, chaff, \\
the path-wrecker’s life is a sham. 

A\marginnote{8.1} layperson who gets this, \\
a learned, wise noble disciple, \\
knows that ‘They are not all like that one’. \\
So when they see them they don’t lose their faith. \\
For how could one equate them—\\
the corrupt with the uncorrupt, the pure with the impure?” 

%
\end{verse}

%
\section*{{\suttatitleacronym Snp 1.6}{\suttatitletranslation Downfalls }{\suttatitleroot Parābhavasutta}}
\addcontentsline{toc}{section}{\tocacronym{Snp 1.6} \toctranslation{Downfalls } \tocroot{Parābhavasutta}}
\markboth{Downfalls }{Parābhavasutta}
\extramarks{Snp 1.6}{Snp 1.6}

\scevam{So\marginnote{1.1} I have heard. }At one time the Buddha was staying near \textsanskrit{Sāvatthī} in Jeta’s Grove, \textsanskrit{Anāthapiṇḍika}’s monastery. Then, late at night, a glorious deity, lighting up the entire Jeta’s Grove, went up to the Buddha, bowed, and stood to one side. Standing to one side, that deity addressed the Buddha in verse: 

\begin{verse}%
“We\marginnote{2.1} ask Gotama\footnote{I take \textit{\textsanskrit{ārambha}} here as having the same sense as \textit{\textsanskrit{samārambha}} at AN 4.195:6.2. But see too the sense of “transgression” at AN 5.142:6.4. } \\
about a man’s downfall. \\
We have come to ask you sir:\footnote{\textit{\textsanskrit{Saṅkhāya} \textsanskrit{sevī}} refers to the monastic practice of making use of requisites, including food, only after reflection on them. } \\
what leads to downfall?”\footnote{Text treats \textit{\textsanskrit{iñjita}} as \textit{eja} and the translation should reflect this. } 

“It’s\marginnote{3.1} easy to know success, \\
and downfall is just as easy. \\
One who loves the teaching succeeds,\footnote{In this unique formulation, \textit{\textsanskrit{upanijjhāyati}} should be understood in terms of its consistent usage in the Suttas: the excessive and obsessive focusing or gazing on something. It’s not just that people miss what is true, it is that they are looking intently in the wrong direction. } \\
but a hater of the teaching meets their downfall.” 

“We\marginnote{4.1} get what you’re saying, \\
this is the first downfall. \\
Tell us the second, Blessed One: \\
what leads to downfall?” 

“The\marginnote{5.1} bad are dear to him, \\
he has no love for the good. \\
He believes the teaching of the bad; \\
and that leads to his downfall.” 

“We\marginnote{6.1} get what you’re saying, \\
this is the second downfall. \\
Tell us the third, Blessed One: \\
what leads to downfall?” 

“Fond\marginnote{7.1} of sleep, fond of company, \\
a man who does no work; \\
he’s lazy, marked by anger, \\
and that leads to his downfall.” 

“We\marginnote{8.1} get what you’re saying, \\
this is the third downfall. \\
Tell us the fourth, Blessed One: \\
what leads to downfall?” 

“Though\marginnote{9.1} able, he does not look after \\
his mother and father \\
when elderly, past their prime, \\
and that leads to his downfall.” 

“We\marginnote{10.1} get what you’re saying, \\
this is the fourth downfall. \\
Tell us the fifth, Blessed One: \\
what leads to downfall?” 

“He\marginnote{11.1} deceives with lies \\
ascetics and brahmins \\
and other renunciates, \\
and that leads to his downfall.” 

“We\marginnote{12.1} get what you’re saying, \\
this is the fifth downfall. \\
Tell us the sixth, Blessed One: \\
what leads to downfall?” 

“A\marginnote{13.1} man with plenty of wealth—\\
gold and food—\\
eats delicacies alone, \\
and that leads to his downfall.” 

“We\marginnote{14.1} get what you’re saying, \\
this is the sixth downfall. \\
Tell us the seventh, Blessed One: \\
what leads to downfall?” 

“Vain\marginnote{15.1} of caste, wealth, \\
and clan, a man \\
looks down on his own family, \\
and that leads to his downfall.” 

“We\marginnote{16.1} get what you’re saying, \\
this is the seventh downfall. \\
Tell us the eighth, Blessed One: \\
what leads to downfall?” 

“In\marginnote{17.1} womanizing, drinking, \\
and gambling, a man \\
wastes all that he has earned, \\
and that leads to his downfall.” 

“We\marginnote{18.1} get what you’re saying, \\
this is the eighth downfall. \\
Tell us the ninth, Blessed One: \\
what leads to downfall?” 

“Not\marginnote{19.1} content with his own partners, \\
he debauches himself with prostitutes,\footnote{Read \textit{\textsanskrit{akovidā}} per SN 35.136:7.4. } \\
and with others’ partners, \\
and that leads to his downfall.” 

“We\marginnote{20.1} get what you’re saying, \\
this is the ninth downfall. \\
Tell us the tenth, Blessed One: \\
what leads to downfall?” 

“A\marginnote{21.1} man well past his prime \\
marries a girl with budding breasts; \\
he cannot sleep for jealousy, \\
and that leads to his downfall.” 

“We\marginnote{22.1} get what you’re saying, \\
this is the tenth downfall. \\
Tell us the eleventh, Blessed One: \\
what leads to downfall?” 

“He\marginnote{23.1} places in authority \\
a woman or a man \\
who’s a drunkard and a spender, \\
and that leads to his downfall.” 

“We\marginnote{24.1} get what you’re saying, \\
this is the eleventh downfall. \\
Tell us the twelfth, Blessed One: \\
what leads to downfall?” 

“A\marginnote{25.1} man of little wealth and strong craving, \\
born into an aristocratic family, \\
sets his sights on kingship, \\
and that leads to his downfall. 

Seeing\marginnote{26.1} these downfalls in the world, \\
an astute and noble person, \\
accomplished in vision, \\
will enjoy a world of grace.” 

%
\end{verse}

%
\section*{{\suttatitleacronym Snp 1.7}{\suttatitletranslation The Lowlife }{\suttatitleroot Vasalasutta}}
\addcontentsline{toc}{section}{\tocacronym{Snp 1.7} \toctranslation{The Lowlife } \tocroot{Vasalasutta}}
\markboth{The Lowlife }{Vasalasutta}
\extramarks{Snp 1.7}{Snp 1.7}

\scevam{So\marginnote{1.1} I have heard. }At one time the Buddha was staying near \textsanskrit{Sāvatthī} in Jeta’s Grove, \textsanskrit{Anāthapiṇḍika}’s monastery. Then the Buddha robed up in the morning and, taking his bowl and robe, entered \textsanskrit{Sāvatthī} for alms. Now at that time in the brahmin \textsanskrit{Bhāradvāja} the Fire-Worshipper’s home the sacred flame had been kindled and the oblation prepared. Wandering indiscriminately for almsfood in \textsanskrit{Sāvatthī}, the Buddha approached \textsanskrit{Bhāradvāja} the Fire-Worshiper’s house. 

\textsanskrit{Bhāradvāja}\marginnote{2.1} the Fire-Worshiper saw the Buddha coming off in the distance and said to him, “Stop right there, shaveling! Right there, fake ascetic! Right there, lowlife!” 

When\marginnote{3.1} he said this, the Buddha said to him, “But brahmin, do you know what is a lowlife or what are the qualities that make you a lowlife?” “No I do not, Master Gotama. Please, Master Gotama, teach me this matter so I can understand what is a lowlife or what are the qualities that make you a lowlife.” “Well then, brahmin, listen and pay close attention, I will speak.” “Yes sir,” \textsanskrit{Bhāradvāja} the Fire-Worshiper replied. The Buddha said this: 

\begin{verse}%
“Irritable\marginnote{4.1} and hostile, \\
wicked and offensive, \\
a man deficient in view, deceitful: \\
know him as a lowlife. 

He\marginnote{5.1} harms living creatures \\
born of womb or of egg, \\
and has no kindness for creatures: \\
know him as a lowlife. 

He\marginnote{6.1} destroys and devastates \\
villages and towns, \\
a notorious oppressor: \\
know him as a lowlife. 

Whether\marginnote{7.1} in village or wilderness, \\
he steals what belongs to others, \\
taking what has not been given: \\
know him as a lowlife. 

Having\marginnote{8.1} fallen into debt, \\
when pressed to pay up he flees, saying \\
‘I don’t owe you anything!’: \\
know him as a lowlife. 

Wanting\marginnote{9.1} some item or other, \\
he attacks a person in the street \\
and takes it: \\
know him as a lowlife. 

For\marginnote{10.1} his own sake or the sake of another, \\
or for the sake of wealth, a man \\
tells a lie when asked to bear witness: \\
know him as a lowlife. 

He\marginnote{11.1} is spied among the partners\footnote{Bodhi, Norman, and \textsanskrit{Ñāṇadīpa} all have “enter” here, while Thanissaro has “invades”. But this is a stock line, and it’s hard to read \textit{anveti} as anything other than “follows”. Niddesa (Mnd 1:70.2) has \textit{anveti anugacchati \textsanskrit{anvāyikaṁ} hoti}. The metaphor is not that water “enters” a boat, but that a leaky boat already contains water and takes it along (like a shadow or an ox-cart per the opening verses of the Dhammapada). That’s why it has to be baled out in the next verse. } \\
of relatives and friends, \\
by force or seduction: \\
know him as a lowlife. 

Though\marginnote{12.1} able, he does not look after \\
his mother and father \\
when elderly, past their prime: \\
know him as a lowlife. 

He\marginnote{13.1} hits or verbally abuses \\
his mother or father, \\
brother, sister, or mother-in-law: \\
know him as a lowlife. 

When\marginnote{14.1} asked about the good, \\
he teaches what is bad, \\
giving secretive advice: \\
know him as a lowlife. 

Having\marginnote{15.1} done a bad deed, he wishes, \\
‘May no-one find me out!’ \\
His deeds are underhand: \\
know him as a lowlife. 

When\marginnote{16.1} visiting another family \\
he eats their delicious food, \\
but does not return the honor: \\
know him as a lowlife. 

He\marginnote{17.1} deceives with lies \\
ascetics and brahmins \\
and other renunciates: \\
know him as a lowlife. 

When\marginnote{18.1} time comes to offer a meal \\
to brahmins or ascetics, \\
he abuses them and does not give: \\
know him as a lowlife. 

He\marginnote{19.1} talks about what never happened, \\
being wrapped up in delusion, \\
chasing after some item or other: \\
know him as a lowlife. 

He\marginnote{20.1} extols himself \\
and disparages others, \\
brought down by his pride: \\
know him as a lowlife. 

He’s\marginnote{21.1} a bully and a miser, \\
of wicked desires, stingy, and devious, \\
shameless, imprudent: \\
know him as a lowlife. 

He\marginnote{22.1} insults the Buddha \\
or his disciple, \\
whether lay or renunciate: \\
know him as a lowlife. 

He\marginnote{23.1} claims to be a perfected one, \\
when he really is no such thing. \\
In the world with its \textsanskrit{Brahmās}, \\
that crook is truly the lowest lowlife. \\
These who are called lowlifes \\
I have explained to you. 

You’re\marginnote{24.1} not a lowlife by birth, \\
nor by birth are you a brahmin. \\
You’re a lowlife by your deeds, \\
by deeds you’re a brahmin. 

And\marginnote{25.1} also you should know \\
according to this example. \\
Sopaka the outcaste’s son \\
became renowned as \textsanskrit{Mātaṅga}. 

\textsanskrit{Mātaṅga}\marginnote{26.1} achieved the highest fame \\
so very hard to find. \\
Lots of aristocrats and brahmins \\
came to serve him. 

He\marginnote{27.1} ascended the stainless highway \\
that leads to the heavens; \\
having discarded sensual desire, \\
he was reborn in a \textsanskrit{Brahmā} realm. \\
His birth did not prevent him \\
from rebirth in the \textsanskrit{Brahmā} realm. 

Those\marginnote{28.1} born in a brahmin family \\
who recite as kinsmen of the hymns, \\
are often discovered \\
in the midst of wicked deeds. 

Blameworthy\marginnote{29.1} in the present life, \\
and in the next, a bad destination. \\
Their birth does not prevent them \\
from blame or bad destiny. 

You’re\marginnote{30.1} not a lowlife by birth, \\
nor by birth are you a brahmin. \\
You’re a lowlife by your deeds, \\
by deeds you’re a brahmin.” 

%
\end{verse}

When\marginnote{31.1} he had spoken, the brahmin \textsanskrit{Bhāradvāja} the Fire-Worshiper said to the Buddha, “Excellent, Master Gotama! Excellent! … From this day forth, may Master Gotama remember me as a lay follower who has gone for refuge for life.” 

%
\section*{{\suttatitleacronym Snp 1.8}{\suttatitletranslation The Discourse on Love }{\suttatitleroot Mettasutta}}
\addcontentsline{toc}{section}{\tocacronym{Snp 1.8} \toctranslation{The Discourse on Love } \tocroot{Mettasutta}}
\markboth{The Discourse on Love }{Mettasutta}
\extramarks{Snp 1.8}{Snp 1.8}

\begin{verse}%
This\marginnote{1.1} is what should be done by those who are skilled in goodness, \\
and have known the state of peace. \\
Let them be able and upright, very upright, \\
easy to speak to, gentle and humble; 

content\marginnote{2.1} and unburdensome, \\
unbusied, living lightly, \\
alert, with senses calmed, \\
courteous, not fawning on families. 

Let\marginnote{3.1} them not do the slightest thing \\
that others might blame with reason. \\
May they be happy and safe! \\
May all beings be happy! 

Whatever\marginnote{4.1} living creatures there are \\
with not a one left out—\\
frail or firm, long or large, \\
medium, small, tiny or round, 

visible\marginnote{5.1} or invisible, \\
living far or near, \\
those born or to be born: \\
May all beings be happy! 

Let\marginnote{6.1} none turn from another, \\
nor look down on anyone anywhere. \\
Though provoked or aggrieved, \\
let them not wish pain on each other. 

Even\marginnote{7.1} as a mother would protect with her life \\
her child, her only child, \\
so too for all creatures \\
unfold a boundless heart. 

With\marginnote{8.1} love for the whole world, \\
unfold a boundless heart. \\
Above, below, all round, \\
unconstricted, without enemy or foe. 

When\marginnote{9.1} standing, walking, sitting, \\
or lying down while yet unweary, \\
keep this ever in mind; \\
for this, they say, is a holy abiding in this life. 

Avoiding\marginnote{10.1} harmful views, \\
virtuous, accomplished in insight, \\
with sensual desire dispelled, \\
they never return to a womb again.\footnote{\textit{Te \textsanskrit{duppamuñca}} has been taken by most translators as referring to people, in senses either active (Bodhi’s “let go with difficulty”) or passive (Norman’s “are hard to release”). However the phrase occurs in a very similar context at SN 3.10:5.2 and Dhp 346:2, where it refers to the chains that are hard to escape. } 

%
\end{verse}

%
\section*{{\suttatitleacronym Snp 1.9}{\suttatitletranslation With Hemavata }{\suttatitleroot Hemavatasutta}}
\addcontentsline{toc}{section}{\tocacronym{Snp 1.9} \toctranslation{With Hemavata } \tocroot{Hemavatasutta}}
\markboth{With Hemavata }{Hemavatasutta}
\extramarks{Snp 1.9}{Snp 1.9}

\begin{verse}%
“Today\marginnote{1.1} is the fifteenth day sabbath,” \\
\scspeaker{said \textsanskrit{Sātāgira}, the native spirit of mount \textsanskrit{Sātā}, }\\
“a holy night is at hand. \\
Come now, let us see Gotama, \\
the Teacher of peerless name.” 

“Isn’t\marginnote{2.1} his mind well-disposed,” \\
\scspeaker{said Hemavata, the native spirit of the Himalayas, }\\
“impartial towards all creatures? \\
And aren’t his thoughts under control \\
when it comes to likes and dislikes?” 

“His\marginnote{3.1} mind is well-disposed,” \\
\scspeaker{said \textsanskrit{Sātāgira}, }\\
“impartial towards all creatures. \\
His thoughts are under control \\
when it comes to his likes and dislikes.” 

“Doesn’t\marginnote{4.1} he not steal?” \\
\scspeaker{said Hemavata, }\\
“And doesn’t he harm not a creature? \\
Isn’t he far from negligence? \\
And doesn’t he not neglect absorption?” 

“He\marginnote{5.1} does not take what is not given,” \\
\scspeaker{said \textsanskrit{Sātāgira}, }\\
“and he harms not a creature. \\
He is far from negligence—\\
the Buddha does not neglect absorption.” 

“Doesn’t\marginnote{6.1} he avoid lying?” \\
\scspeaker{said Hemavata, }\\
“And doesn’t he not speak sharply? \\
Doesn’t he avoid divisive speech,\footnote{\textit{Jappa} means both to “incant” and to “long for”, and the English “pray’ has exactly the same connotations—for exactly the same reasons. They originated in begging favors of a god. } \\
as well as speaking nonsense?” 

“He\marginnote{7.1} does not lie,” \\
\scspeaker{said \textsanskrit{Sātāgira}, }\\
“nor does he speak sharply. \\
He avoids divisive speech, \\
and speaks words of wise counsel.”\footnote{The reference here is to the two ends of contact (AN 6.61) rather than the two extremes of views or paths. } 

“Doesn’t\marginnote{8.1} he find sensual pleasures unattractive?” \\
\scspeaker{said Hemavata, }\\
“And isn’t his mind unclouded? \\
Hasn’t he escaped delusion? \\
And isn’t he seer of truths?” 

“He\marginnote{9.1} does not find sensual pleasures attractive,” \\
\scspeaker{said \textsanskrit{Sātāgira}, }\\
“and his mind is unclouded. \\
He has escaped all delusion—\\
the Buddha is seer of truths.” 

“Isn’t\marginnote{10.1} he accomplished in knowledge?” \\
\scspeaker{said Hemavata, }\\
“And doesn’t he live a pure life? \\
Aren’t his defilements all ended? \\
Doesn’t he have no future lives?” 

“He\marginnote{11.1} is accomplished in knowledge,” \\
\scspeaker{said \textsanskrit{Sātāgira}, }\\
“and he does live a pure life. \\
His defilements are all ended, \\
there are no future lives for him.” 

“Accomplished\marginnote{12.1} is the sage’s mind \\
in action and in speech, \\
and he’s accomplished in knowledge and conduct \\
as per the teaching you praise.”\footnote{Bodhi, Norman, and \textsanskrit{Ñāṇadīpa}, apparently following the commentary, take \textit{vitareyya} as an optative. But the previous verse spoke of an arahant, and here the one who has fully understood perception must have already crossed over. \textit{Vitareyya \textsanskrit{oghaṁ}} occurs at Snp 3.5:13.2, where \textit{vitareyya} is clearly an absolutive, and is rendered as such by both Bodhi and Norman. } 

“Accomplished\marginnote{13.1} is the sage’s mind \\
in action and in speech, \\
and he’s accomplished in knowledge and conduct \\
as per the teaching you rejoice in. 

Accomplished\marginnote{14.1} is the sage’s mind \\
in action and in speech, \\
and he’s accomplished in knowledge and conduct: \\
come now, let us see Gotama.” 

“The\marginnote{15.1} hero so lean, with antelope calves, \\
not greedy, eating little, \\
the sage meditating alone in the forest, \\
come now, let us see Gotama. 

An\marginnote{16.1} elephant, wandering alone like a lion, \\
unconcerned for sensual pleasures, \\
let’s approach him and ask about \\
release from the snare of death.” 

“The\marginnote{17.1} communicator, the instructor, \\
who has gone beyond all things, \\
Awakened, beyond enmity and fear, \\
let us ask Gotama.” 

“What\marginnote{18.1} has the world arisen in?” \\
\scspeaker{said Hemavata, }\\
What does it get close to? \\
By grasping what \\
is the world troubled in what?” 

“The\marginnote{19.1} world’s arisen in six,” \\
\scspeaker{said the Buddha to Hemavata. }\\
“It gets close to six. \\
By grasping at these six, \\
the world’s troubled in six.” 

“What\marginnote{20.1} is that grasping \\
by which the world is troubled? \\
Tell us the exit when asked: \\
how is one released from all suffering?” 

“There\marginnote{21.1} are five kinds of sensual stimulation in the world, \\
and the mind is said to be the sixth. \\
When you’ve discarded desire for these, \\
you’re released from all suffering. 

This\marginnote{22.1} is the exit from the world, \\
explained in accord with the truth. \\
The way I’ve explained it is how \\
you’re released from all suffering.” 

“Who\marginnote{23.1} here crosses the flood, \\
Who crosses the deluge? \\
Who, not standing and unsupported, \\
does not sink in the deep?” 

“Someone\marginnote{24.1} who is always endowed with ethics, \\
wise and serene, \\
inwardly reflective, mindful, \\
crosses the flood so hard to cross. 

Someone\marginnote{25.1} who desists from sensual perception, \\
who has escaped all fetters, \\
and is finished with relishing of rebirth, \\
does not sink in the deep.” 

“Behold\marginnote{26.1} him of wisdom deep who sees the subtle meaning, \\
who has nothing, unattached to sensual life, \\
everywhere free, \\
the great hermit treading the holy road. 

Behold\marginnote{27.1} him of peerless name who sees the subtle meaning, \\
giver of wisdom, unattached to the realm of sensuality: \\
see him, the all-knower, so very intelligent, \\
the great hermit treading the noble road.” 

“It\marginnote{28.1} was a fine sight for us today, \\
a good dawn, a good rising, \\
to see the Awakened One, \\
the undefiled one who has crossed the flood. 

These\marginnote{29.1} thousand native spirits \\
powerful and glorious, \\
all go to you for refuge, \\
you are our supreme Teacher. 

We\marginnote{30.1} shall journey \\
village to village, peak to peak, \\
paying homage to the Buddha, \\
and the natural excellence of the teaching!” 

%
\end{verse}

%
\section*{{\suttatitleacronym Snp 1.10}{\suttatitletranslation With Āḷavaka }{\suttatitleroot Āḷavakasutta}}
\addcontentsline{toc}{section}{\tocacronym{Snp 1.10} \toctranslation{With Āḷavaka } \tocroot{Āḷavakasutta}}
\markboth{With Āḷavaka }{Āḷavakasutta}
\extramarks{Snp 1.10}{Snp 1.10}

\scevam{So\marginnote{1.1} I have heard. }At one time the Buddha was staying near \textsanskrit{Āḷavī} in the haunt of the native spirit \textsanskrit{Āḷavaka}. Then the native spirit \textsanskrit{Āḷavaka} went up to the Buddha, and said to him: “Get out, ascetic!” Saying, “All right, sir,” the Buddha went out. “Get in, ascetic!” Saying, “All right, sir,” the Buddha went in. 

For\marginnote{2.1} a second time … And for a third time the native spirit \textsanskrit{Āḷavaka} said to the Buddha, “Get out, ascetic!” Saying, “All right, sir,” the Buddha went out. “Get in, ascetic!” Saying, “All right, sir,” the Buddha went in. 

And\marginnote{3.1} for a fourth time the native spirit \textsanskrit{Āḷavaka} said to the Buddha, “Get out, ascetic!” “No, sir, I won’t get out. Do what you must.” 

“I\marginnote{4.1} will ask you a question, ascetic. If you don’t answer me, I’ll drive you insane, or explode your heart, or grab you by the feet and throw you to the far shore of the Ganges!” 

“I\marginnote{5.1} don’t see anyone in this world with its gods, \textsanskrit{Māras}, and \textsanskrit{Brahmās}, this population with its ascetics and brahmins, its gods and humans who could do that to me. But anyway, ask what you wish.” Then the native spirit \textsanskrit{Āḷavaka} addressed the Buddha in verse: 

\begin{verse}%
“What’s\marginnote{6.1} a person’s best wealth? \\
What brings happiness when practiced well? \\
What’s the sweetest taste of all? \\
The one who they say has the best life: how do they live?” 

“Faith\marginnote{7.1} here is a person’s best wealth. \\
the teaching brings happiness when practiced well. \\
Truth is the sweetest taste of all. \\
The one who they say has the best life lives by wisdom.” 

“How\marginnote{8.1} do you cross the flood? \\
How do you cross the deluge? \\
How do you get over suffering? \\
How do you get purified?” 

“By\marginnote{9.1} faith you cross the flood, \\
and by diligence the deluge. \\
By energy you get past suffering, \\
and you’re purified by wisdom.” 

“How\marginnote{10.1} do you get wisdom? \\
How do you earn wealth? \\
How do you get a good reputation? \\
How do you hold on to friends? \\
How do the departed not grieve \\
when passing from this world to the next?” 

“One\marginnote{11.1} who is diligent and discerning \\
gains wisdom by wanting to learn, \\
having faith in the perfected ones, \\
and the teaching for becoming extinguished. 

Being\marginnote{12.1} responsible, acting appropriately, \\
and working hard you earn wealth. \\
Truthfulness wins you a good reputation. \\
You hold on to friends by giving. 

A\marginnote{13.1} faithful householder \\
who has these four qualities \\
does not grieve after passing away: \\
truth, principle, steadfastness, and generosity. 

Go\marginnote{14.1} ahead, ask others as well, \\
there are many ascetics and brahmins. \\
See whether anything better is found \\
than truth, self-control, generosity, and patience.” 

“Why\marginnote{15.1} now would I question \\
the many ascetics and brahmins? \\
Today I understand \\
what’s good for the next life. 

It\marginnote{16.1} was truly for my benefit \\
that the Buddha came to stay at \textsanskrit{Āḷavī}. \\
Today I understand \\
where a gift is very fruitful. 

I\marginnote{17.1} myself will journey \\
village to village, town to town, \\
paying homage to the Buddha, \\
and the natural excellence of the teaching!” 

%
\end{verse}

%
\section*{{\suttatitleacronym Snp 1.11}{\suttatitletranslation Victory }{\suttatitleroot Vijayasutta}}
\addcontentsline{toc}{section}{\tocacronym{Snp 1.11} \toctranslation{Victory } \tocroot{Vijayasutta}}
\markboth{Victory }{Vijayasutta}
\extramarks{Snp 1.11}{Snp 1.11}

\begin{verse}%
Walking\marginnote{1.1} and standing, \\
sitting and lying down, \\
extending and contracting the limbs: \\
these are the movements of the body. 

Linked\marginnote{2.1} together by bones and sinews, \\
plastered over with flesh and hide, \\
and covered by the skin, \\
the body is not seen as it is. 

It’s\marginnote{3.1} full of guts and belly, \\
liver and bladder, \\
heart and lungs, \\
kidney and spleen, 

spit\marginnote{4.1} and snot, \\
sweat and fat, \\
blood and synovial fluid, \\
bile and grease. 

Then\marginnote{5.1} in nine streams \\
the filth is always flowing. \\
There is muck from the eyes, \\
wax from the ears, 

and\marginnote{6.1} snot from the nostrils. \\
The mouth sometimes vomits \\
bile and sometimes phlegm. \\
And from the body, sweat and waste. 

Then\marginnote{7.1} there is the hollow head \\
all filled with brains. \\
Governed by ignorance, \\
the fool thinks it’s lovely. 

And\marginnote{8.1} when it lies dead, \\
bloated and livid, \\
discarded in a charnel ground, \\
the relatives forget it. 

It’s\marginnote{9.1} devoured by dogs, \\
by jackals, wolves, and worms. \\
It’s devoured by crows and vultures, \\
and any other creatures there. 

A\marginnote{10.1} wise mendicant here, \\
having heard the Buddha’s words, \\
fully understands it, \\
for they see it as it is. 

“As\marginnote{11.1} this is, so is that, \\
as that is, so is this.” \\
They’d reject desire for the body \\
inside and out. 

That\marginnote{12.1} wise mendicant here \\
rid of desire and lust, \\
has found the deathless peace, \\
extinguishment, the imperishable state. 

This\marginnote{13.1} two-legged body is dirty and stinking, \\
full of different carcasses, \\
and oozing all over the place—\\
but still it is cherished! 

And\marginnote{14.1} if, on account of such a body, \\
someone prides themselves \\
or looks down on others—\\
what is that but a failure to see? 

%
\end{verse}

%
\section*{{\suttatitleacronym Snp 1.12}{\suttatitletranslation The Sage }{\suttatitleroot Munisutta}}
\addcontentsline{toc}{section}{\tocacronym{Snp 1.12} \toctranslation{The Sage } \tocroot{Munisutta}}
\markboth{The Sage }{Munisutta}
\extramarks{Snp 1.12}{Snp 1.12}

\begin{verse}%
Peril\marginnote{1.1} stems from intimacy, \\
dust comes from a home. \\
Freedom from home and intimacy: \\
that is the sage’s vision. 

Having\marginnote{2.1} cut down what’s grown, they wouldn’t replant, \\
nor would they nurture what’s growing. \\
That’s who they call a sage wandering alone, \\
the great hermit has seen the state of peace. 

Having\marginnote{3.1} assessed the fields and measured the seeds,\footnote{\textit{\textsanskrit{Duṭṭhamana}}, or more commonly \textit{\textsanskrit{paduṭṭhamana}}, means “malice” not “corruption”, as is found through the Suttas and confirmed by the Niddesa. } \\
they wouldn’t nurture them with moisture. \\
Truly that sage sees the utter ending of rebirth; \\
when logic’s left behind, judgments no longer apply.\footnote{Bodhi, Norman, and \textsanskrit{Ñāṇadīpa} all render \textit{chanda} as “desire”, and that is certainly the sense of \textit{\textsanskrit{chandānunīto}} at SN 35.94:6.4. Here, however, Niddesa glosses with synonyms for “view” and this seems like a more plausible sense. Note too the sense of \textit{\textsanskrit{niviṭṭha}} as “dogmatic”, which recurs through the \textsanskrit{Aṭṭhakavagga} (cp. \textit{abhinivesa}). } 

Understanding\marginnote{4.1} all the planes of rebirth, \\
not wanting a single one of them, \\
Truly that sage freed of greed \\
need not strive, for they have reached the far shore. 

The\marginnote{5.1} champion, all-knower, so very intelligent, \\
unsullied in the midst of all things, \\
has given up all, freed in the ending of craving: \\
that’s who the wise know as a sage. 

Strong\marginnote{6.1} in wisdom, with precepts and observances intact, \\
serene, loving absorption, mindful, \\
released from chains, kind, undefiled: \\
that’s who the wise know as a sage. 

The\marginnote{7.1} diligent sage wandering alone, \\
is unaffected by blame and praise—\\
like a lion not startled by sounds, \\
like wind not caught in a net, \\
like water not sticking to a lotus. \\
Leader of others, not by others led: \\
that’s who the wise know as a sage. 

Steady\marginnote{8.1} as a post in a bathing-place \\
when others speak endlessly against them, \\
freed of greed, with senses stilled: \\
that’s who the wise know as a sage. 

Steadfast,\marginnote{9.1} straight as a shuttle, \\
horrified by wicked deeds, \\
discerning the just and the unjust: \\
that’s who the wise know as a sage. 

Restrained,\marginnote{10.1} they do no evil, \\
young or middle-aged, the sage is self-controlled. \\
Irreproachable, he does not insult anyone: \\
that’s who the wise know as a sage. 

When\marginnote{11.1} one who lives on charity receives alms, \\
from the top, the middle, or the leftovers, \\
they think it unworthy to praise or put down: \\
that’s who the wise know as a sage. 

The\marginnote{12.1} sage lives refraining from sex, \\
even when young is not tied down, \\
refraining from indulgence and negligence, freed: \\
that’s who the wise know as a sage. 

Understanding\marginnote{13.1} the world, the seer of the ultimate goal, \\
the poised one who has crossed the flood and the ocean, \\
has cut the ties, unattached and undefiled: \\
that’s who the wise know as a sage. 

The\marginnote{14.1} two are not the same, far apart in lifestyle and conduct—\\
the householder providing for a wife, and the selfless one true to their vows. \\
The unrestrained householder kills other creatures, \\
while the restrained sage always protects living creatures. 

As\marginnote{15.1} the crested blue-necked peacock flying through the sky \\
never approaches the speed of the swan, \\
so the householder cannot compete with the mendicant, \\
the sage meditating secluded in the woods. 

%
\end{verse}

%
\addtocontents{toc}{\let\protect\contentsline\protect\nopagecontentsline}
\chapter*{The Lesser Chapter }
\addcontentsline{toc}{chapter}{\tocchapterline{The Lesser Chapter }}
\addtocontents{toc}{\let\protect\contentsline\protect\oldcontentsline}

%
\section*{{\suttatitleacronym Snp 2.1}{\suttatitletranslation Gems }{\suttatitleroot Ratanasutta}}
\addcontentsline{toc}{section}{\tocacronym{Snp 2.1} \toctranslation{Gems } \tocroot{Ratanasutta}}
\markboth{Gems }{Ratanasutta}
\extramarks{Snp 2.1}{Snp 2.1}

\begin{verse}%
Whatever\marginnote{1.1} beings have gathered here, \\
on the ground or in the sky: \\
may beings all be of happy heart, \\
and listen carefully to what is said. 

So\marginnote{2.1} pay heed, all you beings, \\
have love for humankind, \\
who day and night bring offerings; \\
please protect them diligently. 

There’s\marginnote{3.1} no wealth here or beyond, \\
no sublime gem in the heavens, \\
that equals the Realized One. \\
This sublime gem is in the Buddha: \\
by this truth, may you be well! 

Ending,\marginnote{4.1} dispassion, the undying, the sublime, \\
attained by the Sakyan Sage immersed in \textsanskrit{samādhi}; \\
there is nothing equal to that Dhamma. \\
This sublime gem is in the Dhamma: \\
by this truth, may you be well! 

The\marginnote{5.1} purity praised by the highest Buddha \\
is said to be the “immersion with immediate fruit”; \\
no equal to that immersion is found. \\
This sublime gem is in the Dhamma: \\
by this truth, may you be well! 

The\marginnote{6.1} eight individuals praised by the good, \\
are the four pairs of the Holy One’s disciples; \\
they are worthy of religious donations, \\
what’s given to them is very fruitful. \\
This sublime gem is in the \textsanskrit{Saṅgha}: \\
by this truth, may you be well! 

Dedicated\marginnote{7.1} to Gotama’s dispensation, \\
strong-minded, free of sense desire, \\
they’ve attained the goal, plunged into the deathless, \\
and enjoy the quenching they’ve freely gained. \\
This sublime gem is in the \textsanskrit{Saṅgha}: \\
by this truth, may you be well! 

As\marginnote{8.1} a well planted boundary-pillar \\
is not shaken by the four winds, \\
I say a good person is like this, \\
who sees the noble truths in experience. \\
This sublime gem is in the \textsanskrit{Saṅgha}: \\
by this truth, may you be well! 

Those\marginnote{9.1} who fathom the noble truths \\
taught by the one of deep wisdom, \\
do not take an eighth life, \\
even if they are hugely negligent. \\
This sublime gem is in the \textsanskrit{Saṅgha}: \\
by this truth, may you be well! 

When\marginnote{10.1} they attain to vision \\
they give up three things: \\
identity view, doubt, and any \\
attachment to precepts and observances. 

They’re\marginnote{11.1} freed from the four places of loss, \\
and unable to perform the six grave crimes. \\
This sublime gem is in the \textsanskrit{Saṅgha}: \\
by this truth, may you be well! 

Even\marginnote{12.1} if they do a bad deed \\
by body, speech, or mind, \\
they are unable to conceal it; \\
they say this inability applies to one who has seen the truth. \\
This sublime gem is in the \textsanskrit{Saṅgha}: \\
by this truth, may you be well! 

Like\marginnote{13.1} a tall forest tree crowned with flowers \\
in the first month of summer; \\
that’s how he taught the superb Dhamma, \\
leading to quenching, the ultimate benefit. \\
This sublime gem is in the Buddha: \\
by this truth, may you be well! 

The\marginnote{14.1} superb, knower of the superb, giver of the superb, bringer of the superb; \\
taught the superb Dhamma supreme. \\
This sublime gem is in the Buddha: \\
by this truth, may you be well! 

The\marginnote{15.1} old is ended, nothing new is produced. \\
their minds have no desire for future rebirth. \\
Withered are the seeds, there’s no desire for growth, \\
those wise ones are extinguished just like this lamp. \\
This sublime gem is in the \textsanskrit{Saṅgha}: \\
by this truth, may you be well! 

Whatever\marginnote{16.1} beings have gathered here, \\
on the ground or in the sky: \\
the Realized One is honored by gods and humans! \\
We bow to the Buddha! May you be safe! 

Whatever\marginnote{17.1} beings have gathered here, \\
on the ground or in the sky: \\
the Realized One is honored by gods and humans! \\
We bow to the Dhamma! May you be safe! 

Whatever\marginnote{18.1} beings have gathered here, \\
on the ground or in the sky: \\
the Realized One is honored by gods and humans! \\
We bow to the \textsanskrit{Saṅgha}! May you be safe! 

%
\end{verse}

%
\section*{{\suttatitleacronym Snp 2.2}{\suttatitletranslation Putrefaction }{\suttatitleroot Āmagandhasutta}}
\addcontentsline{toc}{section}{\tocacronym{Snp 2.2} \toctranslation{Putrefaction } \tocroot{Āmagandhasutta}}
\markboth{Putrefaction }{Āmagandhasutta}
\extramarks{Snp 2.2}{Snp 2.2}

\begin{verse}%
“The\marginnote{1.1} good eat properly obtained \\
millet, wild grains, broomcorn, \\
greens, tubers, and squashes. \\
They don’t lie to get what they want. 

But\marginnote{2.1} when you eat delicious food, \\
nicely cooked and prepared, and offered by others,\footnote{\textit{\textsanskrit{Samattāni}} is regularly used of vows “undertaken”, and given the following verse this is surely the meaning here, \emph{contra} Niddesa and commentary. See Snp 4.4:5.1, where \textit{\textsanskrit{samādāya}} is used in the same sense. } \\
enjoying a dish of fine rice, \\
Kassapa, you eat putrefaction. 

‘Putrefaction\marginnote{3.1} is not appropriate for me’—\\
so you said, kinsman of \textsanskrit{Brahmā}. \\
Yet here you are enjoying a dish of fine rice, \\
nicely cooked with the flesh of fowl. \\
I’m asking you this, Kassapa: \\
what do you take to be putrefaction?” 

“Killing\marginnote{4.1} living creatures, mutilation, murder, abduction; \\
stealing, lying, cheating and fraud, \\
learning crooked spells, adultery:\footnote{Here we find a sense of \textit{\textsanskrit{jānāti}} that doesn’t fit the English idea of “knowledge” as “true belief”. The idea is that they speak from their own (limited and opinionated) ideas. It’s not as pejorative as “opinion” nor as solid as “knowledge”, so I render as “notion”. We find this usage commonly in the final two chapters of Snp. } \\
this is putrefaction, not eating meat. 

People\marginnote{5.1} here with unbridled sensuality, \\
greedy for tastes, mixed up in impurity, \\
nihilists, immoral, intractable: \\
this is putrefaction, not eating meat. 

Brutal\marginnote{6.1} and rough backbiters, \\
pitiless and arrogant betrayers of friends, \\
misers who never give anything: \\
this is putrefaction, not eating meat. 

Anger,\marginnote{7.1} vanity, obstinacy, contrariness, \\
deceit, jealousy, boastfulness, \\
haughtiness, wicked associates: \\
this is putrefaction, not eating meat. 

The\marginnote{8.1} ill-behaved, debt-evaders, slanderers, \\
business cheats and con-artists, \\
vile men committing depravity: \\
this is putrefaction, not eating meat. 

People\marginnote{9.1} here who can’t stop harming living creatures, \\
taking from others, intent on hurting, \\
immoral, cruel, harsh, lacking regard for others: \\
this is putrefaction, not eating meat. 

Greedy,\marginnote{10.1} hostile, aggressive to others, \\
and addicted to evil—those beings pass into darkness, \\
falling headlong into hell: \\
this is putrefaction, not eating meat. 

Not\marginnote{11.1} fish or flesh or fasting, \\
being naked or shaven, or dreadlocks or dirt, \\
not rough hides or serving the sacred flame, \\
or the many austerities in the world aimed at immortality, \\
not hymns or oblations, sacrifices or seasonal observances, \\
will cleanse a mortal not free of doubt. 

Guarding\marginnote{12.1} the streams of sense impressions, wander with faculties conquered, \\
standing on the teaching, delighting in sincerity and gentleness. \\
The wise have escaped their chains and given up all pain; \\
they don’t cling to the seen and the heard.” 

The\marginnote{13.1} Buddha explained this matter to him again and again, \\
until the master of hymns understood it. \\
It was illustrated with colorful verses \\
by the sage free of putrefaction, unattached, hard to trace. 

Having\marginnote{14.1} heard the fine words of the Buddha, \\
that are free of putrefaction, getting rid of all suffering; \\
humbled, he bowed to the Realized One, \\
and right away begged to go forth. 

%
\end{verse}

%
\section*{{\suttatitleacronym Snp 2.3}{\suttatitletranslation Conscience }{\suttatitleroot Hirisutta}}
\addcontentsline{toc}{section}{\tocacronym{Snp 2.3} \toctranslation{Conscience } \tocroot{Hirisutta}}
\markboth{Conscience }{Hirisutta}
\extramarks{Snp 2.3}{Snp 2.3}

\begin{verse}%
Flouting\marginnote{1.1} conscience, loathing it, \\
saying “I’m on your side”, \\
but not following up in deeds—\\
know they’re not on your side. 

Some\marginnote{2.1} say nice things to their friends \\
without following it up. \\
The wise will recognize \\
one who talks without doing. 

No\marginnote{3.1} true friend relentlessly \\
suspects betrayal, looking for fault. \\
One on whom you rest, like a child on the breast, \\
is a true friend, not split from you by others. 

One\marginnote{4.1} whose reward is the fruit \\
of bearing the burden of service \\
develops a happy state, \\
producing joy and attracting praise. 

One\marginnote{5.1} who has drunk the nectar of seclusion \\
and the nectar of peace, \\
free of stress, free of evil, \\
drinks the joyous nectar of Dhamma. 

%
\end{verse}

%
\section*{{\suttatitleacronym Snp 2.4}{\suttatitletranslation Blessings }{\suttatitleroot Maṅgalasutta}}
\addcontentsline{toc}{section}{\tocacronym{Snp 2.4} \toctranslation{Blessings } \tocroot{Maṅgalasutta}}
\markboth{Blessings }{Maṅgalasutta}
\extramarks{Snp 2.4}{Snp 2.4}

\scevam{So\marginnote{1.1} I have heard. }At one time the Buddha was staying near \textsanskrit{Sāvatthī} in Jeta’s Grove, \textsanskrit{Anāthapiṇḍika}’s monastery. Then, late at night, a glorious deity, lighting up the entire Jeta’s Grove, went up to the Buddha, bowed, and stood to one side. Standing to one side, that deity addressed the Buddha in verse: 

\begin{verse}%
“Many\marginnote{2.1} gods and humans \\
have thought about blessings \\
desiring well-being: \\
declare the highest blessing.” 

“Not\marginnote{3.1} to fraternize with fools, \\
but to fraternize with the wise, \\
and honoring those worthy of honor: \\
this is the highest blessing. 

Living\marginnote{4.1} in a suitable region, \\
having made merit in the past, \\
being rightly resolved in oneself, \\
this is the highest blessing. 

Education\marginnote{5.1} and a craft, \\
discipline and training, \\
and well-spoken speech: \\
this is the highest blessing. 

Caring\marginnote{6.1} for mother and father, \\
kindness to children and partners, \\
and unstressful work: \\
this is the highest blessing. 

Giving\marginnote{7.1} and righteous conduct, \\
kindness to relatives, \\
blameless deeds: \\
this is the highest blessing. 

Desisting\marginnote{8.1} and abstaining from evil, \\
avoiding alcoholic drinks, \\
diligence in good qualities: \\
this is the highest blessing. 

Respect\marginnote{9.1} and humility, \\
contentment and gratitude, \\
and timely listening to the teaching: \\
this is the highest blessing. 

Patience,\marginnote{10.1} being easy to admonish, \\
the sight of ascetics, \\
and timely discussion of the teaching: \\
this is the highest blessing. 

Austerity\marginnote{11.1} and celibacy \\
seeing the noble truths, \\
and realization of extinguishment: \\
this is the highest blessing. 

Though\marginnote{12.1} touched by worldly things, \\
their mind does not tremble; \\
sorrowless, stainless, secure, \\
this is the highest blessing. 

Having\marginnote{13.1} completed these things, \\
undefeated everywhere; \\
everywhere they go in safety: \\
this is their highest blessing.” 

%
\end{verse}

%
\section*{{\suttatitleacronym Snp 2.5}{\suttatitletranslation With Spiky }{\suttatitleroot Sūcilomasutta}}
\addcontentsline{toc}{section}{\tocacronym{Snp 2.5} \toctranslation{With Spiky } \tocroot{Sūcilomasutta}}
\markboth{With Spiky }{Sūcilomasutta}
\extramarks{Snp 2.5}{Snp 2.5}

\scevam{So\marginnote{1.1} I have heard. }At one time the Buddha was staying near \textsanskrit{Gayā} on the cut-stone ledge in the haunt of Spiky the native spirit. Now at that time the native spirits Shaggy and Spiky were passing by not far from the Buddha. So Shaggy said to Spiky, “That’s an ascetic.” “That’s no ascetic, he’s a faker! I’ll soon find out whether he’s an ascetic or a faker.” 

Then\marginnote{2.1} Spiky went up to the Buddha and leaned up against his body, but the Buddha pulled away. Then Spiky said to the Buddha, “Are you afraid, ascetic?” “No, sir, I’m not afraid. But your touch is nasty.” 

“I\marginnote{3.1} will ask you a question, ascetic. If you don’t answer me, I’ll drive you insane, or explode your heart, or grab you by the feet and throw you to the far shore of the Ganges!” 

“I\marginnote{4.1} don’t see anyone in this world with its gods, \textsanskrit{Māras}, and \textsanskrit{Brahmās}, this population with its ascetics and brahmins, its gods and humans who could do that to me. But anyway, ask what you wish.” Then Spiky addressed the Buddha in verse: 

\begin{verse}%
“Where\marginnote{5.1} do greed and hate come from? \\
From where spring discontent, desire, and terror? \\
Where do the mind’s thoughts originate, \\
like a crow let loose by boys.” 

“Greed\marginnote{6.1} and hate come from here; \\
from here spring discontent, desire, and terror; \\
here’s where the mind’s thoughts originate, \\
like a crow let loose by boys. 

Born\marginnote{7.1} of affection, originating in oneself, \\
like the shoots from a banyan’s trunk; \\
the many kinds of attachment to sensual pleasures \\
are like camel’s foot creeper strung through the woods. 

Those\marginnote{8.1} who understand where they come from \\
get rid of them—listen up, spirit! \\
They cross this flood so hard to cross, \\
not crossed before, so as to not be reborn.” 

%
\end{verse}

%
\section*{{\suttatitleacronym Snp 2.6}{\suttatitletranslation A Righteous Life }{\suttatitleroot Kapilasutta (dhammacariyasutta)}}
\addcontentsline{toc}{section}{\tocacronym{Snp 2.6} \toctranslation{A Righteous Life } \tocroot{Kapilasutta (dhammacariyasutta)}}
\markboth{A Righteous Life }{Kapilasutta (dhammacariyasutta)}
\extramarks{Snp 2.6}{Snp 2.6}

\begin{verse}%
A\marginnote{1.1} righteous life, a spiritual life, \\
they call this the supreme treasure. \\
But if someone goes forth \\
from the lay life to homelessness 

who\marginnote{2.1} is of scurrilous character, \\
a beast and a bully, \\
their life gets worse, \\
as poison grows inside them. 

A\marginnote{3.1} mendicant who loves to argue, \\
wrapped in delusion, \\
doesn’t even know what’s been explained \\
in the Dhamma taught by the Buddha. 

Harassing\marginnote{4.1} those who are evolved, \\
governed by ignorance, \\
they don’t know that corruption \\
is the path that leads to hell. 

Entering\marginnote{5.1} the underworld, \\
passing from womb to womb, from darkness to darkness, \\
such a mendicant \\
falls into suffering after death. 

One\marginnote{6.1} such as that is \\
like a sewer \\
brimful with years of filth \\
for it’s hard to clean one full of grime. 

Mendicants,\marginnote{7.1} knowing that someone is like this, \\
attached to the lay life, \\
of wicked desires and wicked intent, \\
of bad behavior and alms-resort, 

then\marginnote{8.1} having gathered in harmony, \\
you should expel them. \\
Throw out the trash! \\
Get rid of the rubbish! 

And\marginnote{9.1} sweep away the scraps—\\
they’re not ascetics, they just think they are. \\
When you’ve thrown out those of wicked desires, \\
of bad behavior and alms-resort, 

dwell\marginnote{10.1} in communion, ever mindful, \\
the pure with the pure. \\
Then in harmony, alert, \\
you’ll make an end to suffering.” 

%
\end{verse}

%
\section*{{\suttatitleacronym Snp 2.7}{\suttatitletranslation Brahmanical Traditions }{\suttatitleroot Brāhmaṇadhammikasutta}}
\addcontentsline{toc}{section}{\tocacronym{Snp 2.7} \toctranslation{Brahmanical Traditions } \tocroot{Brāhmaṇadhammikasutta}}
\markboth{Brahmanical Traditions }{Brāhmaṇadhammikasutta}
\extramarks{Snp 2.7}{Snp 2.7}

\scevam{So\marginnote{1.1} I have heard. }At one time the Buddha was staying near \textsanskrit{Sāvatthī} in Jeta’s Grove, \textsanskrit{Anāthapiṇḍika}’s monastery. Then several old and well-to-do brahmins of Kosala—elderly and senior, who were advanced in years and had reached the final stage of life—went up to the Buddha, and exchanged greetings with him. When the greetings and polite conversation were over, they sat down to one side and said to the Buddha: “Master Gotama, are the ancient traditions of the brahmins seen these days among brahmins?” “No, brahmins, they are not.” “If you wouldn’t mind, Master Gotama, please teach us the ancient traditions of the brahmins.” “Well then, brahmins, listen and pay close attention, I will speak.” “Yes, sir,” they replied. The Buddha said this: 

\begin{verse}%
“The\marginnote{2.1} ancient hermits used to be \\
restrained and austere. \\
Having given up the five sensual titillations, \\
they lived for their own true good. 

Brahmins\marginnote{3.1} used to own no cattle, \\
nor gold or grain. \\
Chanting was their wealth and grain, \\
which they guarded as a gift from god. 

Food\marginnote{4.1} was prepared for them \\
and left beside their doors. \\
People believed that food prepared in faith \\
should be given to them. 

With\marginnote{5.1} colorful clothes, \\
clothes and bedding, \\
prosperous nations and countries \\
honored those brahmins. 

Brahmins\marginnote{6.1} used to be inviolable and \\
invincible, protected by principle. \\
No-one ever turned them away \\
from the doors of families. 

For\marginnote{7.1} forty-eight years \\
they led the spiritual life.\footnote{The problem addressed is one who is picking up one thing after another. It’s therefore best to translate \textit{atta} in an active sense, per Niddesa: \textit{\textsanskrit{gahaṇaṁ} \textsanskrit{muñcanā} samatikkanto}. } \\
The brahmins of old pursued \\
their quest for knowledge and conduct. 

Brahmins\marginnote{8.1} never transgressed with another,\footnote{This line refers to seeing a supposed saint or holy person, a sight that was believed to grant blessings. As Bodhi remarks, \textit{aroga} refers to the \textit{\textsanskrit{attā}}. However this does not necessarily mean an immaterial soul. Here what is seen is the soul as a physical form (\textit{\textsanskrit{rūpī} \textsanskrit{attā}}) manifesting purity and wellness. } \\
nor did they purchase a wife. \\
They lived together in love, \\
joining together by mutual consent. 

Brahmins\marginnote{9.1} never approached their wives for sex \\
during the time outside \\
the fertile half of the month \\
after menstruation. 

They\marginnote{10.1} praised celibacy and morality, \\
integrity, gentleness, and austerity, \\
gentleness and harmlessness, \\
and also patience. 

He\marginnote{11.1} who was supreme among them, \\
godlike, staunchly vigorous, \\
did not engage in sex \\
even in a dream. 

Training\marginnote{12.1} in line with their duties, \\
many smart people here \\
praised celibacy and morality, \\
and also patience. 

They\marginnote{13.1} begged for rice, \\
bedding, clothes, ghee, and oil. \\
Having collected them legitimately, \\
they arranged a sacrifice. 

But\marginnote{14.1} they slew no cows \\
while serving at the sacrifice. \\
Like a mother, father, or brother, \\
or some other relative, \\
cows are our best friends, \\
the fonts of medicine. 

They\marginnote{15.1} give food and health, \\
and beauty and happiness. \\
Knowing these benefits, \\
they slew no cows. 

The\marginnote{16.1} brahmins were delicate and tall, \\
beautiful and glorious. \\
They were keen on all the duties \\
required by their own traditions.\footnote{Here as in 4.3, \textit{\textsanskrit{ñāṇa}} is used in the sense of “notion”. Bodhi, Norman, and \textsanskrit{Ñāṇadīpa} all follow Niddesa in taking \textit{\textsanskrit{abhijāna}} in its usual sense of “understanding, direct knowing”. But here the secondary sense of “recall” fits better. Someone thinking about their vision of a holy man believes that there must be an observer of such purity, i.e. a self. } \\
So long as they continued in the world, \\
people flourished happily. 

But\marginnote{17.1} perversion crept into them \\
little by little when they saw \\
the splendor of the king \\
and the ladies in all their finery. 

Their\marginnote{18.1} chariots were harnessed with thoroughbreds, \\
well-made with bright canopies, \\
and their homes and houses were \\
neatly laid out in measured rows.\footnote{\textit{Pacceti} is a standard verb for someone who “believes” in a religious rite or idea. } 

They\marginnote{19.1} were lavished with herds of cattle, \\
and furnished with bevies of lovely ladies. \\
This extravagant human wealth \\
was coveted by the brahmins. 

They\marginnote{20.1} compiled hymns to that end,\footnote{Bodhi, Norman, and \textsanskrit{Ñāṇadīpa} all render \textit{\textsanskrit{saṅga}} in some variant of “tie, attachment”. Yet that fails to render the metaphor. \textit{\textsanskrit{Saṅga}} is used in the sense of a “snare” in which one may be caught, like a net (Snp 3.6:28.4). And a snare is indeed something that one may cross over or get past. } \\
approached King \textsanskrit{Okkāka} and said, \\
‘You have plenty of wealth and grain. \\
Sacrifice! For you have much treasure. \\
Sacrifice! For you have much wealth.’ 

Persuaded\marginnote{21.1} by the brahmins, \\
the king, chief of charioteers, performed \\
horse sacrifice, human sacrifice, \\
the sacrifices of the ‘stick-casting’, the ‘royal soma drinking’, and the ‘unbarred’. \\
When he had carried out these sacrifices, \\
he gave riches to the brahmins. 

There\marginnote{22.1} were cattle, bedding, and clothes, \\
and ladies in all their finery; \\
chariots harnessed with thoroughbreds, \\
well-made with bright canopies; 

and\marginnote{23.1} lovely homes, all \\
neatly laid out in measured rows. \\
Having furnished them with different grains, \\
he gave riches to the brahmins. 

When\marginnote{24.1} they got hold of that wealth, \\
they arranged to store it up. \\
Falling under the sway of desire, \\
their craving grew and grew. \\
They compiled hymns to that end, \\
approached King \textsanskrit{Okkāka} once more and said, 

‘Like\marginnote{25.1} water and earth, \\
gold, riches, and grain, \\
are cows for humankind, \\
as they are essential for creatures. \\
Sacrifice! For you have much treasure. \\
Sacrifice! For you have much wealth.’ 

Persuaded\marginnote{26.1} by the brahmins, \\
the king, chief of charioteers, \\
had many hundred thousand cows\footnote{The monkey grabs the branches, not the trunk, as a metaphor for missing the essence. And to drive home the pun, religious orders are called “branches”. } \\
slain at the sacrifice. 

Neither\marginnote{27.1} with feet nor with horns \\
do cows harm anyone at all. \\
Cows meek as lambs, \\
supply buckets of milk. \\
But taking them by the horns, \\
the king slew them with a sword. 

At\marginnote{28.1} that the gods and the ancestors, \\
with Indra, the titans and monsters, \\
roared out: ‘This is a crime against nature!’ \\
as the sword fell on the cows. 

There\marginnote{29.1} used to be three kinds of illness: \\
greed, starvation, and old age. \\
But due to the slaughter of cows, \\
this grew to be ninety-eight. 

This\marginnote{30.1} unnatural violence \\
has been passed down as an ancient custom. \\
Killing innocent creatures, \\
the sacrificers forsake righteousness. 

And\marginnote{31.1} that is how this mean old practice \\
was criticized by sensible people. \\
Wherever they see such a thing, \\
folk criticize the sacrificer. 

With\marginnote{32.1} righteousness gone, \\
merchants and workers were split, \\
as were many aristocrats, \\
and wives looked down on their husbands. 

Aristocrats\marginnote{33.1} and \textsanskrit{Brahmā}’s kinsmen \\
and others protected by their clan, \\
neglecting the lessons of ancestry, \\
fell under the sway of sensual pleasures.” 

%
\end{verse}

When\marginnote{34.1} he had spoken, those well-to-do brahmins said to the Buddha, “Excellent, Master Gotama! Excellent! … From this day forth, may Master Gotama remember us as lay followers who have gone for refuge for life.” 

%
\section*{{\suttatitleacronym Snp 2.8}{\suttatitletranslation The Boat }{\suttatitleroot Dhamma (nāvā) sutta}}
\addcontentsline{toc}{section}{\tocacronym{Snp 2.8} \toctranslation{The Boat } \tocroot{Dhamma (nāvā) sutta}}
\markboth{The Boat }{Dhamma (nāvā) sutta}
\extramarks{Snp 2.8}{Snp 2.8}

\begin{verse}%
Honor\marginnote{1.1} the person from whom you would learn the teaching, \\
as the gods honor Inda. \\
Then they will have confidence in you, \\
and being learned, they reveal the teaching. 

Heeding\marginnote{2.1} well, a wise pupil \\
practicing in line with that teaching \\
grows intelligent, discerning, and subtle \\
through diligently sticking close to such a person. 

But\marginnote{3.1} associating with a petty fool \\
who falls short of the goal, jealous, \\
then unable to discern the teaching in this life, \\
one proceeds to death still plagued by doubts. 

It’s\marginnote{4.1} like a man who has plunged into a river, \\
a rushing torrent in spate. \\
As they are swept away downstream, \\
how could they help others across? 

Just\marginnote{5.1} so, one unable to discern the teaching, \\
who hasn’t studied the meaning under the learned, \\
not knowing it oneself, still plagued by doubts, \\
how could they help others to contemplate? 

But\marginnote{6.1} one who has embarked on a strong boat \\
equipped with rudder and oar, \\
would bring many others across there \\
with skill, care, and intelligence. 

So\marginnote{7.1} too one who understands—a knowledge master, \\
evolved, learned, and unflappable—\\
can help others to contemplate, \\
so long as they are prepared to listen carefully. 

That’s\marginnote{8.1} why you should spend time with a good person, \\
intelligent and learned. \\
Having understood the meaning, putting it into practice, \\
one who has realized the teaching may find happiness. 

%
\end{verse}

%
\section*{{\suttatitleacronym Snp 2.9}{\suttatitletranslation What Morality? }{\suttatitleroot Kiṁsīlasutta}}
\addcontentsline{toc}{section}{\tocacronym{Snp 2.9} \toctranslation{What Morality? } \tocroot{Kiṁsīlasutta}}
\markboth{What Morality? }{Kiṁsīlasutta}
\extramarks{Snp 2.9}{Snp 2.9}

\begin{verse}%
“With\marginnote{1.1} what morality, what conduct, \\
fostering what deeds, \\
would a person lay the foundations right, \\
and reach the highest goal?” 

“Honoring\marginnote{2.1} elders without jealousy, \\
they’d know the time to visit their teachers. \\
Treasuring the chance for a Dhamma talk, \\
they’d listen carefully to the well-spoken words. 

At\marginnote{3.1} the right time, they’d humbly enter \\
the teachers’ presence, leaving obstinacy behind. \\
They’d call to mind and put into practice \\
the meaning, the teaching, self-control, and the spiritual life. 

Delighting\marginnote{4.1} in the teaching, enjoying the teaching, \\
standing on the teaching, investigating the teaching, \\
they’d never say anything that degraded the teaching, \\
but would be guided by genuine words well-spoken. 

Giving\marginnote{5.1} up mirth, prayer, weeping, ill will, \\
deception, fraud, greed, conceit, \\
aggression, crudeness, stains, and indulgence, \\
they’d wander free of vanity, steadfast. 

Understanding\marginnote{6.1} is the essence of well-spoken words, \\
stillness is the essence of learning and understanding. \\
Wisdom and learning do not flourish \\
in a hasty and negligent person. 

Those\marginnote{7.1} happy with the teaching proclaimed by the Noble One \\
are supreme in speech, mind, and deed. \\
Settled in peace, gentleness, and stillness, \\
they’ve realized the essence of learning and wisdom.” 

%
\end{verse}

%
\section*{{\suttatitleacronym Snp 2.10}{\suttatitletranslation Get Up! }{\suttatitleroot Uṭṭhānasutta}}
\addcontentsline{toc}{section}{\tocacronym{Snp 2.10} \toctranslation{Get Up! } \tocroot{Uṭṭhānasutta}}
\markboth{Get Up! }{Uṭṭhānasutta}
\extramarks{Snp 2.10}{Snp 2.10}

\begin{verse}%
Get\marginnote{1.1} up and meditate! \\
What’s the point in your sleeping? \\
How can the afflicted slumber \\
when injured by an arrow strike? 

Get\marginnote{2.1} up and meditate! \\
Train hard for peace! \\
The King of Death has caught you heedless—\\
don’t let him fool you under his sway. 

Needy\marginnote{3.1} gods and humans \\
are held back by clinging: \\
get over it. \\
Don’t let the moment pass you by. \\
For if you miss your moment \\
you’ll grieve when sent to hell. 

Negligence\marginnote{4.1} is always dust; \\
dust follows right behind negligence. \\
Through diligence and knowledge, \\
pluck out the dart from yourself. 

%
\end{verse}

%
\section*{{\suttatitleacronym Snp 2.11}{\suttatitletranslation With Rāhula }{\suttatitleroot Rāhulasutta}}
\addcontentsline{toc}{section}{\tocacronym{Snp 2.11} \toctranslation{With Rāhula } \tocroot{Rāhulasutta}}
\markboth{With Rāhula }{Rāhulasutta}
\extramarks{Snp 2.11}{Snp 2.11}

\begin{verse}%
“Does\marginnote{1.1} familiarity breed contempt, \\
even for the man of wisdom? \\
Do you honor he who holds aloft \\
the torch for all humanity?” 

“Familiarity\marginnote{2.1} breeds no contempt \\
for the man of wisdom. \\
I always honor he who holds aloft \\
the torch for all humanity.” 

“One\marginnote{3.1} who’s given up the five sensual stimulations, \\
so pleasing and delightful, \\
and who’s left the home life out of faith—\\
let them make an end to suffering! 

Mix\marginnote{4.1} with spiritual friends, \\
stay in remote lodgings, \\
secluded and quiet, \\
and eat in moderation. 

Robes,\marginnote{5.1} almsfood, \\
requisites and lodgings: \\
don’t crave such things; \\
don’t come back to this world again. 

Be\marginnote{6.1} restrained in the monastic code, \\
and the five sense faculties, \\
With mindfulness immersed in the body, \\
be full of disillusionment. 

Turn\marginnote{7.1} away from the feature of things \\
that’s attractive, provoking lust. \\
With mind unified and serene, \\
meditate on the ugly aspects of the body. 

Meditate\marginnote{8.1} on the signless, \\
give up the tendency to conceit; \\
and when you comprehend conceit, \\
you will live at peace.” 

%
\end{verse}

That\marginnote{9.1} is how the Buddha regularly advised Venerable \textsanskrit{Rāhula} with these verses. 

%
\section*{{\suttatitleacronym Snp 2.12}{\suttatitletranslation Vaṅgīsa and His Mentor Nigrodhakappa }{\suttatitleroot Nigrodhakappa (vaṅgīsa) sutta}}
\addcontentsline{toc}{section}{\tocacronym{Snp 2.12} \toctranslation{Vaṅgīsa and His Mentor Nigrodhakappa } \tocroot{Nigrodhakappa (vaṅgīsa) sutta}}
\markboth{Vaṅgīsa and His Mentor Nigrodhakappa }{Nigrodhakappa (vaṅgīsa) sutta}
\extramarks{Snp 2.12}{Snp 2.12}

\scevam{So\marginnote{1.1} I have heard. }At one time the Buddha was staying near \textsanskrit{Āḷavī}, at the \textsanskrit{Aggāḷava} Tree-shrine. Now at that time it was not long after Venerable \textsanskrit{Vaṅgīsa}’s mentor, the senior monk named Nigrodhakappa, had become extinguished. Then as \textsanskrit{Vaṅgīsa} was in private retreat this thought came to his mind: “Has my mentor become extinguished or not?” Then in the late afternoon, Venerable \textsanskrit{Vaṅgīsa} came out of retreat and went to the Buddha. He bowed, sat down to one side, and said to him: “Just now, sir, as I was in private retreat this thought came to mind. ‘Has my mentor become extinguished or not?’” Then Venerable \textsanskrit{Vaṅgīsa} got up from his seat, arranged his robe over one shoulder, raised his joined palms toward the Buddha, and addressed him in verse: 

\begin{verse}%
“I\marginnote{2.1} ask the teacher unrivaled in wisdom, \\
who has cut off all doubts in this very life: \\
a monk has died at \textsanskrit{Aggāḷava}, who was \\
well-known, famous, and quenched. 

Nigrodhakappa\marginnote{3.1} was his name; \\
it was given to that brahmin by you, Blessed One. \\
He wandered in your honor, yearning for freedom, \\
energetic, a resolute Seer of Truth. 

O\marginnote{4.1} Sakyan, all-seer, \\
we all wish to know about that disciple. \\
Our ears are eager to hear, \\
for you are the most excellent teacher. 

Cut\marginnote{5.1} off our doubt, declare this to us; \\
your wisdom is vast, tell us of his quenching! \\
All-seer, speak among us, \\
like the thousand-eyed Sakka in the midst of the gods! 

Whatever\marginnote{6.1} ties there are, or paths to delusion, \\
or things on the side of unknowing, or that are bases of doubt \\
vanish on reaching a Realized One, \\
for his eye is the best of all people’s. 

If\marginnote{7.1} no man were ever to disperse corruptions, \\
like the wind dispersing the clouds, \\
darkness would shroud the whole world; \\
not even brilliant men would shine. 

The\marginnote{8.1} wise are bringers of light; \\
my hero, that is what I think of you. \\
We’ve come for your discernment and knowledge: \\
here in this assembly, declare to us about \textsanskrit{Kappāyana}. 

Swiftly\marginnote{9.1} send forth your sweet, sweet voice, \\
like a goose stretching its neck, gently honking, \\
lucid-flowing, with lovely tone: \\
alert, we all listen to you. 

You\marginnote{10.1} have entirely abandoned birth and death; \\
restrained and pure, I urge you to speak the Dhamma! \\
For ordinary people have no wish-granter,\footnote{See \textit{\textsanskrit{uccāvacaṁ} \textsanskrit{vā} pana \textsanskrit{dassanāya} gacchati} at AN 6.30:2.2. In the current poem, it refers to someone who visits a variety of teachers or sects, as per Niddesa. } \\
but Realized Ones have a comprehensibility-granter.\footnote{Here the plural \textit{vedehi} stands for the \textit{\textsanskrit{tevijjā}}. } 

Your\marginnote{11.1} answer is definitive, and we will adopt it, \\
for you have perfect understanding. \\
We raise our joined palms one last time, \\
one of unrivaled wisdom, don’t deliberately confuse us. 

Knowing\marginnote{12.1} the noble teaching from top to bottom, \\
unrivaled hero, don’t deliberately confuse us. \\
As a man in the baking summer sun would long for water, \\
I long for your voice, so let the sound rain down. 

Surely\marginnote{13.1} \textsanskrit{Kappāyana} did not lead the spiritual life in vain? \\
Did he realize quenching, \\
or did he still have a remnant of defilement? \\
Let us hear what kind of liberation he had!” 

“He\marginnote{14.1} cut off craving for mind and body in this very life,” \\
\scspeaker{said the Buddha, }\\
“the river of darkness that had long lain within him. \\
He has entirely crossed over birth and death.” \\
So declared the Blessed One, the leader of the five. 

“Now\marginnote{15.1} that I have heard your words, \\
seventh of sages, I am confident. \\
My question, it seems, was not in vain, \\
the brahmin did not deceive me. 

As\marginnote{16.1} he said, so he did—\\
he was a disciple of the Buddha. \\
He cut the net of death the deceiver, \\
so extended and strong. 

Blessed\marginnote{17.1} One, \textsanskrit{Kappāyana} saw \\
the starting point of grasping. \\
He has indeed gone far beyond \\
Death’s domain so hard to pass.” 

%
\end{verse}

%
\section*{{\suttatitleacronym Snp 2.13}{\suttatitletranslation The Right Way to Wander }{\suttatitleroot Sammāparibbājanīyasutta}}
\addcontentsline{toc}{section}{\tocacronym{Snp 2.13} \toctranslation{The Right Way to Wander } \tocroot{Sammāparibbājanīyasutta}}
\markboth{The Right Way to Wander }{Sammāparibbājanīyasutta}
\extramarks{Snp 2.13}{Snp 2.13}

\begin{verse}%
“I\marginnote{1.1} ask the sage abounding in wisdom—\\
crossed-over, gone beyond, quenched, steadfast: \\
when a mendicant has left home, expelling sensuality, \\
what’s the right way to wander the world?” 

“When\marginnote{2.1} they’ve eradicated superstitions,” \\
\scspeaker{said the Buddha, }\\
“about celestial portents, dreams, or bodily marks; \\
with the stain of superstitions left behind, \\
they’d rightly wander the world. 

A\marginnote{3.1} mendicant ought dispel desire \\
for pleasures human or divine; \\
with rebirth transcended and truth comprehended, \\
they’d rightly wander the world. 

Putting\marginnote{4.1} divisiveness behind them, \\
a mendicant gives up anger and stinginess; \\
with favoring and opposing left behind, \\
they’d rightly wander the world. 

When\marginnote{5.1} the loved and the unloved are both left behind, \\
not grasping or dependent on anything; \\
freed from all things that fetter, \\
they’d rightly wander the world. 

Finding\marginnote{6.1} no substance in attachments, \\
rid of desire for things they’ve acquired, \\
independent, needing no-one to guide them, \\
they’d rightly wander the world. 

Not\marginnote{7.1} hostile in speech, mind, or deed, \\
they’ve rightly understood the teaching. \\
Aspiring to the state of quenching, \\
they’d rightly wander the world. 

Not\marginnote{8.1} pridefully thinking, ‘they bow to me’; \\
though reviled, they’d still stay in touch;\footnote{Note that \textit{\textsanskrit{samuggahīta}} is used here in the same sense as Snp 4.3:6.2 or Snp 2.12:11.1, i.e. the “adoption” of a theory or view. } \\
not besotted when getting food from others, \\
they’d rightly wander the world. 

When\marginnote{9.1} greed and craving to live again are cast off, \\
a mendicant refrains from violence and abduction; \\
rid of doubt, free of thorns, \\
they’d rightly wander the world. 

Knowing\marginnote{10.1} what is suitable for themselves, \\
a mendicant would hurt no-one in the world; \\
understanding the teaching in accord with reality, \\
they’d rightly wander the world. 

They\marginnote{11.1} have no underlying tendencies at all, \\
and are rid of unskillful roots; \\
free of hope, with no need for hope, \\
they’d rightly wander the world. 

Defilements\marginnote{12.1} ended, conceit given up, \\
beyond all manner of desire; \\
tamed, quenched, and steadfast, \\
they’d rightly wander the world. 

Faithful,\marginnote{13.1} learned, seer of the sure path, \\
the wise one takes no side among factions; \\
rid of greed, hate, and repulsion, \\
they’d rightly wander the world. 

A\marginnote{14.1} purified victor with veil drawn back, \\
among worldly things master, transcendent, stilled; \\
expert in knowledge of conditions’ cessation, \\
they’d rightly wander the world. 

They’re\marginnote{15.1} over speculating on the future or past, \\
and understand what it means to be pure; \\
freed from all the sense fields, \\
they’d rightly wander the world. 

The\marginnote{16.1} state of peace is understood, the truth is comprehended, \\
they’ve openly seen defilements cast off; \\
and with the ending of all attachments, \\
they’d rightly wander the world.” 

“Clearly,\marginnote{17.1} Blessed One, it is just as you say. \\
One who lives like this is a tamed mendicant, \\
beyond all fetters and yokes: \\
they’d rightly wander the world.” 

%
\end{verse}

%
\section*{{\suttatitleacronym Snp 2.14}{\suttatitletranslation With Dhammika }{\suttatitleroot Dhammikasutta}}
\addcontentsline{toc}{section}{\tocacronym{Snp 2.14} \toctranslation{With Dhammika } \tocroot{Dhammikasutta}}
\markboth{With Dhammika }{Dhammikasutta}
\extramarks{Snp 2.14}{Snp 2.14}

\scevam{So\marginnote{1.1} I have heard. }At one time the Buddha was staying near \textsanskrit{Sāvatthī} in Jeta’s Grove, \textsanskrit{Anāthapiṇḍika}’s monastery. Then the lay follower Dhammika, together with five hundred lay followers, went up to the Buddha, bowed, sat down to one side, and addressed him in verse: 

\begin{verse}%
“I\marginnote{2.1} ask you, Gotama, whose wisdom is vast: \\
what does one do to become a good disciple, \\
both one who has left the home, \\
and the lay followers staying at home? 

For\marginnote{3.1} you understand the course and destiny \\
of the world with all its gods. \\
There is no equal to you who sees the subtle meaning, \\
for you are the Buddha most excellent, they say. 

Having\marginnote{4.1} experienced all knowledge, \\
you explain the teaching out of compassion for beings. \\
All-seer, you have drawn back the veil, \\
and immaculate, you shine on the whole world. 

The\marginnote{5.1} dragon king \textsanskrit{Erāvaṇa}, hearing you called ‘Victor’, \\
came into your presence. \\
He consulted with you then, having heard your words, \\
left consoled, saying ‘Excellent!’ 

And\marginnote{6.1} King Kuvera \textsanskrit{Vessavaṇa} also \\
approached to ask about the teaching. \\
You also answered him, O wise one, \\
and hearing you he too was consoled. 

Those\marginnote{7.1} teachers of other paths given to debate, \\
whether \textsanskrit{Ājīvakas} or Jains, \\
all fail to overtake you in wisdom, \\
like a standing man next to a sprinter. 

Those\marginnote{8.1} brahmins given to debate, \\
some of whom are quite senior, \\
all end up beholden to you for the meaning, \\
and others too who think themselves debaters. 

So\marginnote{9.1} subtle and pleasant is the teaching \\
that is well proclaimed by you, Blessed One. \\
It’s all we long to hear. So when asked, \\
O Best of Buddhas, tell us! 

All\marginnote{10.1} these mendicants have gathered, \\
and the layfolk too are here to listen. \\
Let them hear the teaching the immaculate one discovered, \\
like gods listening to the fine words of \textsanskrit{Vāsava}.” 

“Listen\marginnote{11.1} to me, mendicants, I will educate you \\
in the cleansing teaching; all bear it in mind. \\
An intelligent person, seeing the meaning, \\
would adopt the deportment proper to a renunciate.\footnote{Reading \textit{yad-} here and below as \textit{yadi}. } 

No\marginnote{12.1} way would a mendicant go out at the wrong time; \\
at the right time, they’d walk the village for alms. \\
For chains bind one who wanders outside the right time, \\
which is why the Buddhas avoid it. 

Sights,\marginnote{13.1} sounds, tastes, smells, and touches, \\
which drive beings mad—\\
dispel desire for such things, \\
and enter for the morning meal at the right time. 

After\marginnote{14.1} receiving alms for the day, \\
on returning a mendicant would sit in private alone. \\
Inwardly reflective, they’d curb their mind \\
from outside things, keeping themselves collected. 

Should\marginnote{15.1} they converse with a disciple, \\
with anyone else, or with a mendicant, \\
they’d bring up only the sublime teaching, \\
not dividing or blaming. 

For\marginnote{16.1} some contend in debate, \\
but we praise not those of little wisdom. \\
In place after place they are bound in chains, \\
for they send their mind over there far away. 

Alms,\marginnote{17.1} a dwelling, a bed and seat, \\
and water for rinsing the dust from the cloak—\\
after hearing the teaching of the Holy One, \\
a disciple of splendid wisdom would use these after appraisal. 

That’s\marginnote{18.1} why, when it comes to alms and lodgings, \\
and water for rinsing the dust from the cloak, \\
a mendicant is unsullied in the midst of these things, \\
like a droplet on a lotus-leaf. 

Now\marginnote{19.1} I shall tell you the householder’s duty, \\
doing which one becomes a good disciple. \\
For one burdened with possessions does not get to realize \\
the whole of the mendicant’s practice. 

They’d\marginnote{20.1} not kill any creature, nor have them killed, \\
nor grant permission for others to kill. \\
They’ve laid aside violence towards all creatures \\
frail or firm that there are in the world. 

Next,\marginnote{21.1} a disciple would avoid knowingly \\
taking anything not given at all, \\
they’d not get others to do it, nor grant them permission to steal; \\
they’d avoid \emph{all} theft. 

A\marginnote{22.1} sensible person would avoid the unchaste life, \\
like a burning pit of coals. \\
But if unable to remain chaste, \\
they’d not transgress with another’s partner. 

In\marginnote{23.1} a council or assembly, \\
or one on one, they would not lie. \\
They’d not get others to lie, nor grant them permission to lie; \\
they’d avoid \emph{all} untruths. 

A\marginnote{24.1} householder espousing this teaching \\
would not consume liquor or drink. \\
They’d not get others to drink, nor grant them permission to drink; \\
knowing that ends in intoxication. 

For\marginnote{25.1} drunken fools do bad things, \\
and encourage other heedless folk. \\
Reject this field of demerit, \\
the maddening, deluding frolic of fools. 

You\marginnote{26.1} shouldn’t kill living creatures, or steal, \\
or lie, or drink alcohol. \\
Be celibate, refraining from sex, \\
and don’t eat at night, the wrong time. 

Not\marginnote{27.1} wearing garlands or applying perfumes, \\
you should sleep on a low bed, or a mat on the ground. \\
This is the eight-factored sabbath, they say, \\
explained by the Buddha, who has gone to suffering’s end. 

Then\marginnote{28.1} having rightly undertaken the sabbath \\
complete in all its eight factors \\
on the fourteenth, fifteenth, and eighth of the fortnight, \\
as well as on the fortnightly special displays, 

on\marginnote{29.1} the morning after the sabbath \\
a clever person, rejoicing with confident heart, \\
would distribute food and drink \\
to the mendicant \textsanskrit{Saṅgha} as is fitting. 

One\marginnote{30.1} should rightfully support one’s parents, \\
and undertake a legitimate business. \\
A diligent layperson observing these duties \\
ascends to the gods called Self-luminous.” 

%
\end{verse}

%
\addtocontents{toc}{\let\protect\contentsline\protect\nopagecontentsline}
\chapter*{The Great Chapter }
\addcontentsline{toc}{chapter}{\tocchapterline{The Great Chapter }}
\addtocontents{toc}{\let\protect\contentsline\protect\oldcontentsline}

%
\section*{{\suttatitleacronym Snp 3.1}{\suttatitletranslation Going Forth }{\suttatitleroot Pabbajjāsutta}}
\addcontentsline{toc}{section}{\tocacronym{Snp 3.1} \toctranslation{Going Forth } \tocroot{Pabbajjāsutta}}
\markboth{Going Forth }{Pabbajjāsutta}
\extramarks{Snp 3.1}{Snp 3.1}

\begin{verse}%
“I\marginnote{1.1} shall extol going forth \\
with the example of the seer, \\
the course of inquiry that led to \\
his choice to go forth. 

‘This\marginnote{2.1} life at home is cramped, \\
a realm of dirt.’ \\
‘The life of one gone forth is like an open space.’ \\
Seeing this, he went forth. 

Having\marginnote{3.1} gone forth, he shunned \\
bad deeds of body. \\
And leaving verbal misconduct behind, \\
he purified his livelihood. 

The\marginnote{4.1} Buddha went to \textsanskrit{Rājagaha}, \\
the Mountainfold of the Magadhans. \\
He betook himself for alms, \\
replete with excellent marks. 

\textsanskrit{Bimbisāra}\marginnote{5.1} saw him \\
while standing atop his longhouse. \\
Noticing that he was endowed with marks, \\
he said the following: 

‘Pay\marginnote{6.1} heed, sirs, to this one, \\
handsome, majestic, radiant;\footnote{\textit{Namati} is usually rendered with “incline” but I feel something stronger is meant here. } \\
accomplished in deportment, \\
he looks just a plough’s length in front. 

Eyes\marginnote{7.1} downcast, mindful, \\
unlike one from a low family. \\
Let the king’s messengers run out, \\
and find where the mendicant will go.’ 

The\marginnote{8.1} messengers sent out \\
followed right behind, thinking \\
‘Where will the mendicant go? \\
Where shall he find a place to stay?’ 

Wandering\marginnote{9.1} indiscriminately for alms, \\
sense-doors guarded and well restrained, \\
his bowl was quickly filled, \\
aware and mindful. 

Having\marginnote{10.1} wandered for alms, \\
the sage left the city. \\
He betook himself to Mount \textsanskrit{Paṇḍava}, \\
thinking, ‘Here is the place I shall stay.’ 

Seeing\marginnote{11.1} that he had arrived at a place to stay, \\
the messengers withdrew nearby,\footnote{The historical past tense here is unusual. It seems to be an allusion to legendary hermits of the past. } \\
but one of them returned \\
to inform the king. 

‘Great\marginnote{12.1} king, the mendicant \\
is on the east flank of Mount \textsanskrit{Paṇḍava}. \\
There he sits, like a tiger or a bull, \\
like a lion in a mountain cave.’ 

Hearing\marginnote{13.1} the messenger’s report, \\
the aristocrat set out \\
hurriedly in his fine chariot \\
towards Mount \textsanskrit{Paṇḍava}. 

He\marginnote{14.1} went as far as vehicles could go, \\
then dismounted from his chariot, \\
approached on foot, \\
and reaching him, drew near. 

Seated,\marginnote{15.1} the king greeted him \\
and made polite conversation. \\
When the courtesies were over, \\
he said the following: 

‘You\marginnote{16.1} are young, just a youth, \\
a lad in the prime of life. \\
You are endowed with beauty and stature, \\
like an aristocrat of good lineage 

in\marginnote{17.1} glory at the army’s head, \\
surrounded by a troop of elephants. \\
I offer you pleasures—enjoy them! \\
But please tell me your lineage by birth.’ 

‘Up\marginnote{18.1} north lies a nation, great king, \\
on the slope of the Himalayas, \\
full of wealth and strength, \\
led by one loyal to the Kosalans.\footnote{In the Suttas, \textit{virajjati} is a stock term in the process of liberation. It follows \textit{\textsanskrit{nibbidā}} (“disenchantment”) and precedes \textit{vimutti} (“freedom”). Being free, the arahant has gone beyond this stage. } 

They\marginnote{19.1} are of the Solar clan, \\
their lineage is the Sakyans. \\
I have gone forth from that family—\\
I do not yearn for sensual pleasure. 

Seeing\marginnote{20.1} the danger in sensual pleasures, \\
seeing renunciation as sanctuary, \\
I shall go on to strive; \\
that is where my mind delights.’” 

%
\end{verse}

%
\section*{{\suttatitleacronym Snp 3.2}{\suttatitletranslation Striving }{\suttatitleroot Padhānasutta}}
\addcontentsline{toc}{section}{\tocacronym{Snp 3.2} \toctranslation{Striving } \tocroot{Padhānasutta}}
\markboth{Striving }{Padhānasutta}
\extramarks{Snp 3.2}{Snp 3.2}

\begin{verse}%
“During\marginnote{1.1} my time of resolute striving \\
on the bank of the \textsanskrit{Nerañjara} River, \\
I was meditating very hard \\
for the sake of finding sanctuary. 

\textsanskrit{Namucī}\marginnote{2.1} approached, \\
speaking words of kindness: \\
‘You’re thin, discolored, \\
on the verge of death. 

Death\marginnote{3.1} has a thousand parts of you, \\
one fraction is left to life. \\
Live sir! Life is better! \\
Living, you can make merits. 

While\marginnote{4.1} leading the spiritual life \\
and serving the sacred flame, \\
you can pile up abundant merit—\\
so what will striving do for you? 

Hard\marginnote{5.1} to walk is the path of striving, \\
hard to do, a hard challenge to win.’” \\
These are the verses \textsanskrit{Māra} spoke \\
as he stood beside the Buddha. 

When\marginnote{6.1} \textsanskrit{Māra} had spoken in this way, \\
the Buddha said this: \\
“O Wicked One, kinsman of the negligent, \\
you’re here for your own purpose. 

I\marginnote{7.1} have no need for \\
the slightest bit of merit. \\
Those with need for merit \\
are fit for \textsanskrit{Māra} to address. 

I\marginnote{8.1} have faith and energy too, \\
and wisdom is found in me. \\
When I am so resolute, \\
why do you beg me to live? 

The\marginnote{9.1} rivers and streams \\
might be dried by the wind, \\
so why, when I am resolute, \\
should it not dry up my blood? 

And\marginnote{10.1} while the blood is drying up, \\
the bile and phlegm dry too. \\
And as my muscles waste away, \\
my mind grows more serene. \\
And all the stronger grow mindfulness \\
and wisdom and immersion. 

As\marginnote{11.1} I meditate like this, \\
having attained the supreme feeling,\footnote{\textit{Gedha} also at AN 5.103:3.2, where the sense is not “entanglement” but “cover from sight”. See Norman’s note on this verse for the apparent confusion between \textit{gedha}, \textit{rodha}, and the intermediary \textit{godha}. } \\
my mind has no interest in sensual pleasures: \\
behold a being’s purity! 

Sensual\marginnote{12.1} pleasures are your first army, \\
the second is called discontent, \\
hunger and thirst are the third, \\
and the fourth is said to be craving. 

Your\marginnote{13.1} fifth is dullness and drowsiness, \\
the sixth is said to be cowardice, \\
your seventh is doubt, \\
contempt and obstinacy are your eighth. 

Profit,\marginnote{14.1} praise, and honor, \\
and misbegotten fame; \\
the extolling of oneself \\
while scorning others. 

This\marginnote{15.1} is your army, \textsanskrit{Namucī}, \\
the strike force of the Dark One. \\
Only a hero can defeat it, \\
but in victory there lies bliss. 

Let\marginnote{16.1} me gird myself—\footnote{A line used of Devadatta at Iti 89:4.1. } \\
so what if I die!\footnote{Bodhi follows the commentary and Niddesa in rendering \textit{parikissati} as “afflicted”, as does Norman with “troubled”, both assuming a contracted \textit{parikilissati}. But \textit{parikissati} here is connected with wisdom (or lack thereof), not suffering. Surely we should look to such passages as AN 4.186:2.6, where \textit{parikassati} is the mind that “drags” a person around. \textit{Parikissati} is the passive form. } \\
I’d rather die in battle \\
than live on in defeat. 

Here\marginnote{17.1} some ascetics and brahmins \\
are swallowed up, not to be seen again. \\
They do not know the path \\
traveled by those true to their vows. 

Seeing\marginnote{18.1} \textsanskrit{Māra} ready on his mount, \\
surrounded by his bannered forces, \\
I shall meet them in battle—\\
they’ll never make me retreat! 

That\marginnote{19.1} army of yours has never been beaten \\
by the world with all its gods. \\
Yet I shall smash it with wisdom, \\
like an unfired pot with a stone.\footnote{At Pli Tv Pvr 15:7.2 \textit{\textsanskrit{pubbāpara}} is defined in terms of changing and inconsistent behavior. } 

Having\marginnote{20.1} brought my thoughts under control, \\
and established mindfulness well, \\
I shall wander from country to country, \\
guiding many disciples. 

Diligent\marginnote{21.1} and resolute, \\
following my instructions, \\
they will proceed despite your will, \\
to where there is no sorrow.” 

“For\marginnote{22.1} seven years I followed \\
step by step behind the Blessed One. \\
I found no vulnerability \\
in the mindful Awakened One. 

A\marginnote{23.1} crow once circled a stone \\
that looked like a lump of fat. \\
‘Perhaps I’ll find something tender,’ it thought, \\
‘perhaps there’s something tasty.’ 

But\marginnote{24.1} finding nothing tasty, \\
the crow left that place. \\
Like the crow that pecked the stone, \\
I leave Gotama disappointed.” 

So\marginnote{25.1} stricken with sorrow \\
that his harp dropped from his armpit, \\
that spirit, downcast, \\
vanished right there. 

%
\end{verse}

%
\section*{{\suttatitleacronym Snp 3.3}{\suttatitletranslation Well-Spoken Words }{\suttatitleroot Subhāsitasutta}}
\addcontentsline{toc}{section}{\tocacronym{Snp 3.3} \toctranslation{Well-Spoken Words } \tocroot{Subhāsitasutta}}
\markboth{Well-Spoken Words }{Subhāsitasutta}
\extramarks{Snp 3.3}{Snp 3.3}

\scevam{So\marginnote{1.1} I have heard. }At one time the Buddha was staying near \textsanskrit{Sāvatthī} in Jeta’s Grove, \textsanskrit{Anāthapiṇḍika}’s monastery. There the Buddha addressed the mendicants, “Mendicants!” “Venerable sir,” they replied. The Buddha said this: 

“Mendicants,\marginnote{2.1} speech that has four factors is well spoken, not poorly spoken. It’s blameless and is not criticized by sensible people. What four? It’s when a mendicant speaks well, not poorly; they speak on the teaching, not against the teaching; they speak pleasantly, not unpleasantly; and they speak truthfully, not falsely. Speech with these four factors is well spoken, not poorly spoken. It’s blameless and is not criticized by sensible people.” That is what the Buddha said. Then the Holy One, the Teacher, went on to say: 

\begin{verse}%
“Good\marginnote{3.1} people say that well-spoken words are foremost; \\
second, speak on the teaching, not against it; \\
third, speak pleasantly, not unpleasantly; \\
and fourth, speak truthfully, not falsely.” 

%
\end{verse}

Then\marginnote{4.1} Venerable \textsanskrit{Vaṅgīsa} got up from his seat, arranged his robe over one shoulder, raised his joined palms toward the Buddha, and said, “I feel inspired to speak, Blessed One! I feel inspired to speak, Holy One!” “Then speak as you feel inspired,” said the Buddha. Then \textsanskrit{Vaṅgīsa} extolled the Buddha in his presence with fitting verses: 

\begin{verse}%
“Speak\marginnote{5.1} only such words \\
that do not hurt yourself \\
nor harm others; \\
such speech is truly well spoken. 

Speak\marginnote{6.1} only pleasing words, \\
words gladly welcomed. \\
Pleasing words are those \\
that bring nothing bad to others. 

Truth\marginnote{7.1} itself is the undying word: \\
this is an eternal truth. \\
Good people say that the teaching and its meaning \\
are grounded in the truth. 

The\marginnote{8.1} words spoken by the Buddha \\
for realizing the sanctuary, extinguishment, \\
for the attainment of vision, \\
this really is the best kind of speech.” 

%
\end{verse}

%
\section*{{\suttatitleacronym Snp 3.4}{\suttatitletranslation With Bhāradvāja of Sundarikā on the Sacrificial Cake }{\suttatitleroot Pūraḷāsa (sundarikabhāradvāja) sutta}}
\addcontentsline{toc}{section}{\tocacronym{Snp 3.4} \toctranslation{With Bhāradvāja of Sundarikā on the Sacrificial Cake } \tocroot{Pūraḷāsa (sundarikabhāradvāja) sutta}}
\markboth{With Bhāradvāja of Sundarikā on the Sacrificial Cake }{Pūraḷāsa (sundarikabhāradvāja) sutta}
\extramarks{Snp 3.4}{Snp 3.4}

\scevam{So\marginnote{1.1} I have heard. }At one time the Buddha was staying in the Kosalan lands on the bank of the \textsanskrit{Sundarikā} river. Now at that time the brahmin \textsanskrit{Sundarikabhāradvāja} was serving the sacred flame and performing the fire sacrifice on the bank of the \textsanskrit{Sundarikā}. Then he looked all around the four quarters, wondering, “Now who might eat the leftovers of this offering?” He saw the Buddha meditating at the root of a certain tree with his robe pulled over his head. Taking the leftovers of the offering in his left hand and a pitcher in the right he approached the Buddha. 

When\marginnote{2.1} he heard \textsanskrit{Sundarikabhāradvāja}’s footsteps the Buddha uncovered his head. \textsanskrit{Sundarikabhāradvāja} thought, “This man is shaven, he is shaven!” And he wanted to turn back. But he thought, “Even some brahmins are shaven. Why don’t I go to him and ask about his birth?” Then the brahmin \textsanskrit{Sundarikabhāradvāja} went up to the Buddha, and said to him, “Sir, in what caste were you born?” 

Then\marginnote{3.1} the Buddha addressed \textsanskrit{Sundarikabhāradvāja} in verse: 

\begin{verse}%
“I\marginnote{4.1} am no brahmin, nor am I a prince, \\
nor merchant nor anything else. \\
Fully understanding the clan of ordinary people, \\
I wander in the world owning nothing, reflective. 

Clad\marginnote{5.1} in my cloak, I wander without home, \\
my hair shorn, quenched. \\
Since I’m unburdened by youngsters,\footnote{Norman has “good” for \textit{subha}, Bodhi “excellent”, but the Niddesa shows that the sense is “illuminating, bringing wisdom”. Compare the same phrase at Snp 4.13:16.3, where the context shows we are speaking of those whose doctrine is derived from meditative experience. Perhaps here (as in the \textsanskrit{Pārāyanavagga}) we are dealing with ascetics who rely on the \textit{subhavimutti}, the “beautiful release” (of \textsanskrit{jhānas}). } \\
it’s inappropriate to ask me about clan.” 

“Actually\marginnote{6.1} sir, when brahmins meet they politely \\
ask each other whether they are brahmins.” 

“Well,\marginnote{7.1} if you say that you’re a brahmin, \\
and that I am not, \\
I shall question you on the \textsanskrit{Gāyatrī} Mantra,\footnote{Here, as at Snp 4.12:15.3, \textit{puthu} emphasizes how they all have their own distinct view. } \\
with its three lines and twenty-four syllables.” 

“On\marginnote{8.1} what grounds have hermits and men, \\
aristocrats and brahmins here in the world \\
performed so many different sacrifices to the gods?” 

“During\marginnote{9.1} a sacrifice, should a past master, a knowledge master, \\
receive an oblation, it profits the donor, I say.” 

“Then\marginnote{10.1} clearly my oblation will be profitable,” \\
\scspeaker{said the brahmin, }\\
“since I have met such a knowledge master. \\
It’s because I’d never met anyone like you \\
that others ate the sacrificial cake.” 

“So\marginnote{11.1} then, brahmin, since you have approached me \\
as a seeker of the good, ask.\footnote{\textit{Dahanti} is to “regard” or “see”, per Niddesa and Commentary; Bodhi’s “accuse” seems hard to justify. } \\
Perhaps you may find here someone intelligent, \\
peaceful, unclouded, untroubled, with no need for hope.” 

“Master\marginnote{12.1} Gotama, I like to sacrifice \\
and wish to perform a sacrifice. Please advise me, \\
for I do not understand \\
where an oblation is profitable; tell me this.” 

%
\end{verse}

“Well\marginnote{13.1} then, brahmin, lend an ear, I will teach you the Dhamma. 

\begin{verse}%
Don’t\marginnote{14.1} ask about birth, ask about conduct; \\
for any wood can surely generate fire. \\
A steadfast sage, even though from a low class family, \\
is a thoroughbred checked by conscience. 

Tamed\marginnote{15.1} by truth, fulfilled by taming, \\
a complete knowledge master who has completed the spiritual journey—\\
that is where a brahmin seeking merit \\
should bestow a timely offering as sacrifice. 

Those\marginnote{16.1} who have left sensuality behind, wandering homeless, \\
self-controlled, straight as a shuttle—\\
that is where a brahmin seeking merit \\
should bestow a timely offering as sacrifice. 

Those\marginnote{17.1} freed of greed, with senses stilled, \\
like the moon released from the eclipse—\\
that is where a brahmin seeking merit \\
should bestow a timely offering as sacrifice. 

They\marginnote{18.1} wander the world unimpeded, \\
always mindful, calling nothing their own—\\
that is where a brahmin seeking merit \\
should bestow a timely offering as sacrifice. 

Having\marginnote{19.1} left sensuality behind, wandering triumphant, \\
knowing the end of rebirth and death, \\
extinguished and cool as a lake: \\
the Realized One is worthy of the sacrificial cake. 

Good\marginnote{20.1} among the good, far from the bad, \\
the Realized One has infinite wisdom. \\
Unsullied in this world and the next: \\
the Realized One is worthy of the sacrificial cake. 

In\marginnote{21.1} whom dwells no deceit or conceit, \\
rid of greed, unselfish, with no need for hope, \\
with anger eliminated, quenched, \\
a brahmin rid of sorrow’s stain: \\
the Realized One is worthy of the sacrificial cake. 

He\marginnote{22.1} has given up the mind’s home, \\
and has no possessions at all. \\
Not grasping to this world or the next: \\
the Realized One is worthy of the sacrificial cake. 

Serene,\marginnote{23.1} he has crossed the flood, \\
and has understood the teaching with ultimate view. \\
With defilements ended, bearing his final body: \\
the Realized One is worthy of the sacrificial cake. 

In\marginnote{24.1} whom desire to be reborn, and caustic speech \\
are cleared and ended, they are no more; \\
that knowledge master, everywhere free: \\
the Realized One is worthy of the sacrificial cake. 

He\marginnote{25.1} has escaped his chains, he’s chained no more, \\
among those caught in conceit he is free of conceit; \\
he has fully understood suffering with its field and ground: \\
the Realized One is worthy of the sacrificial cake. 

Not\marginnote{26.1} relying on hope, seeing seclusion, \\
well past the views proclaimed by others. \\
In him there are no supporting conditions at all: \\
the Realized One is worthy of the sacrificial cake. 

He\marginnote{27.1} has comprehended all things, high and low, \\
cleared them and ended them, so they are no more. \\
Peaceful, freed in the ending of grasping: \\
the Realized One is worthy of the sacrificial cake. 

He\marginnote{28.1} sees the utter ending of rebirth’s fetter, \\
and has swept away all manner of desire. \\
Pure, stainless, immaculate, flawless: \\
the Realized One is worthy of the sacrificial cake. 

Not\marginnote{29.1} seeing himself in terms of a self, \\
he is stilled, upright, and steadfast. \\
Imperturbable, kind, wishless: \\
the Realized One is worthy of the sacrificial cake. 

He\marginnote{30.1} harbors no delusions within at all, \\
he has insight into all things. \\
He bears his final body, \\
attained to the state of grace, the supreme awakening. \\
That’s how the purity of a spirit is defined: \\
the Realized One is worthy of the sacrificial cake.” 

“Let\marginnote{31.1} my oblation be a true offering, \\
since I have found such a knowledge master! \\
I see \textsanskrit{Brahmā} in person! Accept my offering, Blessed One: \\
please eat my sacrificial cake.” 

“Food\marginnote{32.1} enchanted by a spell isn’t fit for me to eat. \\
That’s not the principle of those who see, brahmin. \\
The Buddhas reject things enchanted with spells. \\
Since there is such a principle, brahmin, that’s how they live. 

Serve\marginnote{33.1} with other food and drink \\
the consummate one, the great hermit, \\
with defilements ended and remorse stilled. \\
For he is the field for the seeker of merit.” 

“Please,\marginnote{34.1} Blessed One, help me understand: \\
now that I have encountered your teaching, \\
when I look for someone during a sacrifice, \\
who should eat the religious donation of one like me?” 

“One\marginnote{35.1} who is rid of aggression, \\
whose mind is unclouded, \\
who is liberated from sensual pleasures, \\
and who has dispelled dullness. 

One\marginnote{36.1} who has erased boundaries and limits, \\
expert in birth and death, \\
a sage, blessed with sagacity. \\
When such a person comes to the sacrifice, 

get\marginnote{37.1} rid of your scowl! \\
Honor them with joined palms, \\
and venerate them with food and drink, \\
and in this way your religious donation will succeed.” 

“The\marginnote{38.1} Buddha is worthy of the sacrificial cake, \\
he is the supreme field of merit, \\
Recipient of gifts from the whole world, \\
what’s given to the worthy one is very fruitful.” 

%
\end{verse}

Then\marginnote{39.1} the brahmin \textsanskrit{Sundarikabhāradvāja} said to the Buddha, “Excellent, Master Gotama! Excellent! As if he were righting the overturned, or revealing the hidden, or pointing out the path to the lost, or lighting a lamp in the dark so people with good eyes can see what’s there, Master Gotama has made the teaching clear in many ways. I go for refuge to Master Gotama, to the teaching, and to the mendicant \textsanskrit{Saṅgha}. Sir, may I receive the going forth, the ordination in the Buddha’s presence?” And the brahmin Sundarika \textsanskrit{Bhāradvāja} received the going forth, the ordination in the Buddha’s presence. And soon after, he became one of the perfected. 

%
\section*{{\suttatitleacronym Snp 3.5}{\suttatitletranslation With Māgha }{\suttatitleroot Māghasutta}}
\addcontentsline{toc}{section}{\tocacronym{Snp 3.5} \toctranslation{With Māgha } \tocroot{Māghasutta}}
\markboth{With Māgha }{Māghasutta}
\extramarks{Snp 3.5}{Snp 3.5}

\scevam{So\marginnote{1.1} I have heard. }At one time the Buddha was staying near \textsanskrit{Rājagaha}, on the Vulture’s Peak Mountain. Then the brahmin student \textsanskrit{Māgha} approached the Buddha and exchanged greetings with him. When the greetings and polite conversation were over, he sat down to one side, and said to the Buddha: 

“I’m\marginnote{2.1} a giver, Master Gotama, a donor; I am bountiful and committed to charity. I seek wealth in a principled manner, and with that legitimate wealth I give to one person, to two, three, four, five, six, seven, eight, nine, ten, twenty, thirty, forty, fifty, a hundred people or even more. Giving and sacrificing like this, Master Gotama, do I accrue much merit?” 

“Indeed\marginnote{3.1} you do, student. A giver or donor who is bountiful and committed to charity, who seeks wealth in a principled manner, and with that legitimate wealth gives to one person, or up to a hundred people or even more, accrues much merit.” Then \textsanskrit{Māgha} addressed the Buddha in verse: 

\begin{verse}%
“I\marginnote{4.1} ask the bountiful Gotama,” \\
\scspeaker{said \textsanskrit{Māgha}, }\\
“wearing an ochre robe, wandering homeless. \\
Suppose a lay donor who is committed to charity \\
makes a sacrifice seeking merit, looking for merit. \\
Giving food and drink to others here, \\
how is their offering purifed?” 

“Suppose\marginnote{5.1} a lay donor who is committed to charity,” \\
\scspeaker{replied the Buddha, }\\
“makes a sacrifice seeking merit, looking for merit, \\
giving food and drink to others here: \\
such a one would succeed due to those who are worthy of donations.”\footnote{Here, \textit{yuga} is connected in the Niddesa with \textit{\textsanskrit{yugaggāha}} (“taking the reins”), which in turn is connected in Vb 17:75.2 with “causing disputes”. It seems to be an idiom for “issuing a challenge”. In any case, it has a stronger sense than just “met, encountered”. } 

“Suppose\marginnote{6.1} a lay donor who is committed to charity,” \\
\scspeaker{said \textsanskrit{Māgha}, }\\
“makes a sacrifice seeking merit, looking for merit, \\
giving food and drink to others here: \\
explain to me who is worthy of donations.” 

“Those\marginnote{7.1} who wander the world unattached, \\
consummate, restrained, owning nothing: \\
that is where a brahmin seeking merit \\
should bestow a timely offering as sacrifice. 

Those\marginnote{8.1} who have cut off all fetters and bonds, \\
tamed, liberated, untroubled, with no need for hope: \\
that is where a brahmin seeking merit \\
should bestow a timely offering as sacrifice. 

Those\marginnote{9.1} who are released from all fetters, \\
tamed, liberated, untroubled, with no need for hope: \\
that is where a brahmin seeking merit \\
should bestow a timely offering as sacrifice. 

Having\marginnote{10.1} given up greed, hate, and delusion, \\
with defilements ended, the spiritual journey completed: \\
that is where a brahmin seeking merit \\
should bestow a timely offering as sacrifice. 

Those\marginnote{11.1} in whom dwells no deceit or conceit, \\
with defilements ended, the spiritual journey completed: \\
that is where a brahmin seeking merit \\
should bestow a timely offering as sacrifice. 

Those\marginnote{12.1} rid of greed, unselfish, with no need for hope, \\
with defilements ended, the spiritual journey completed: \\
that is where a brahmin seeking merit \\
should bestow a timely offering as sacrifice. 

Those\marginnote{13.1} not fallen prey to cravings, \\
who, having crossed the flood, live unselfishly: \\
that is where a brahmin seeking merit \\
should bestow a timely offering as sacrifice. 

Those\marginnote{14.1} with no craving at all in the world \\
to any form of existence in this life or the next: \\
that is where a brahmin seeking merit \\
should bestow a timely offering as sacrifice. 

Those\marginnote{15.1} who have left sensuality behind, wandering homeless, \\
self-controlled, straight as a shuttle: \\
that is where a brahmin seeking merit \\
should bestow a timely offering as sacrifice. 

Those\marginnote{16.1} freed of greed, with senses stilled, \\
like the moon released from the eclipse: \\
that is where a brahmin seeking merit \\
should bestow a timely offering as sacrifice. 

Those\marginnote{17.1} peaceful ones free of greed and anger, \\
for whom there are no destinies, being rid of them in this life: \\
that is where a brahmin seeking merit \\
should bestow a timely offering as sacrifice. 

They’ve\marginnote{18.1} given up rebirth and death completely, \\
and have gone beyond all doubt: \\
that is where a brahmin seeking merit \\
should bestow a timely offering as sacrifice. 

Those\marginnote{19.1} who live as their own island, \\
everywhere free, owning nothing: \\
that is where a brahmin seeking merit \\
should bestow a timely offering as sacrifice. 

Those\marginnote{20.1} here who know this to be true: \\
‘This is my last life, there are no future lives’: \\
that is where a brahmin seeking merit \\
should bestow a timely offering as sacrifice. 

A\marginnote{21.1} knowledge master, loving absorption, mindful, \\
who has reached awakening and is a refuge for many: \\
that is where a brahmin seeking merit \\
should bestow a timely offering as sacrifice.” 

“Clearly\marginnote{22.1} my questions were not in vain!” \\
\scspeaker{said \textsanskrit{Māgha}, }\\
“The Buddha has explained to me who is worthy of donations. \\
You are the one here who knows this to be true, \\
for truly you understand this matter. 

Suppose\marginnote{23.1} a lay donor who is committed to charity \\
makes a sacrifice seeking merit, looking for merit, \\
giving food and drink to others here: \\
explain to me how to accomplish the sacrifice.” 

“Sacrifice,\marginnote{24.1} and while doing so,”\footnote{\textit{Sakkhasi} is 2nd future, \textit{\textsanskrit{sampayātave}} is infinitive. \textit{\textsanskrit{Sampāyati}} is a fairly common word, “answer a question”, usually in a negative sense, to be “stumped” by a question. Oddly, most translators (Bodhi, Norman, \textsanskrit{Ñāṇadīpa}, Thanissaro) don’t catch this sense. } \\
\scspeaker{replied the Buddha, }\\
“be clear and confident in every way. \\
Sacrifice is the ground standing upon which \\
the sacrificer sheds their flaws. 

One\marginnote{25.1} free of greed, rid of anger, \\
developing a heart of limitless love, \\
spreads that limitlessness in every direction, \\
ever diligent day and night.” 

“Who\marginnote{26.1} is purified, freed, awake?\footnote{\textsanskrit{Māra}’s daughters, the archetypal temptresses. Their failed seduction is told at SN 4.25. \textit{Arati} seems to be an error in the Pali tradition for \textit{rati}. } \\
How can one go to the \textsanskrit{Brahmā} realm oneself? \\
I do not know, so please tell me when asked, \\
for the Buddha is the \textsanskrit{Brahmā} I see in person today! \\
To us you are truly the equal of \textsanskrit{Brahmā}. \\
Splendid One, how is one reborn in the \textsanskrit{Brahmā} realm?” 

“One\marginnote{27.1} who accomplishes the sacrifice with three modes,” \\
\scspeaker{replied the Buddha, }\\
“such a one would succeed due to those who are worthy of donations. \\
Sacrificing like this, one rightly committed to charity \\
is reborn in the \textsanskrit{Brahmā} realm, I say.” 

%
\end{verse}

When\marginnote{28.1} he had spoken, the student \textsanskrit{Māgha} said to the Buddha, “Excellent, Master Gotama! Excellent! … From this day forth, may Master Gotama remember me as a lay follower who has gone for refuge for life.” 

%
\section*{{\suttatitleacronym Snp 3.6}{\suttatitletranslation With Sabhiya }{\suttatitleroot Sabhiyasutta}}
\addcontentsline{toc}{section}{\tocacronym{Snp 3.6} \toctranslation{With Sabhiya } \tocroot{Sabhiyasutta}}
\markboth{With Sabhiya }{Sabhiyasutta}
\extramarks{Snp 3.6}{Snp 3.6}

\scevam{So\marginnote{1.1} I have heard. }At one time the Buddha was staying near \textsanskrit{Rājagaha}, in the Bamboo Grove, the squirrels’ feeding ground. Now at that time the wanderer Sabhiya had been presented with several questions by a deity who was a former relative, saying: “Sabhiya, you should live the spiritual life with whatever ascetic or brahmin answers these questions.” 

Then\marginnote{2.1} Sabhiya, after learning those questions in the presence of that deity, approached those ascetics and brahmins who led an order and a community, and taught a community, who were well-known and famous religious founders, regarded as holy by many people. That is, \textsanskrit{Pūraṇa} Kassapa, Makkhali \textsanskrit{Gosāla}, \textsanskrit{Nigaṇṭha} \textsanskrit{Nāṭaputta}, \textsanskrit{Sañjaya} \textsanskrit{Belaṭṭhiputta}, Pakudha \textsanskrit{Kaccāyana}, and Ajita Kesakambala. And he asked them those questions, but they were stumped by them. Displaying annoyance, hate, and bitterness, they questioned Sabiya in return. 

Then\marginnote{3.1} Sabhiya thought, “Those famous ascetics and brahmins were stumped by my questions. Displaying annoyance, hate, and bitterness, they questioned me in return on that matter. Why don’t I return to a lesser life so I can enjoy sensual pleasures?” 

Then\marginnote{4.1} Sabhiya thought, “This ascetic Gotama also leads an order and a community, and teaches a community. He’s a well-known and famous religious founder, regarded as holy by many people. Why don’t I ask him this question?” 

Then\marginnote{5.1} he thought, “Even those ascetics and brahmins who are elderly and senior, who are advanced in years and have reached the final stage of life; who are senior, long standing, long gone forth; who lead an order and a community, and teach a community; who are well-known and famous religious founders, regarded as holy by many people—that is \textsanskrit{Pūraṇa} Kassapa and the rest—were stumped by my questions. They displayed annoyance, hate, and bitterness, and even questioned me in return. How can the ascetic Gotama possibly answer my questions, since he is so young in age and newly gone forth?” 

Then\marginnote{6.1} he thought, “An ascetic should not be looked down upon or disparaged because they are young. Though young, the ascetic Gotama has great psychic power and might. Why don’t I ask him this question?” 

Then\marginnote{7.1} Sabhiya set out for \textsanskrit{Rājagaha}. Traveling stage by stage, he came to \textsanskrit{Rājagaha}, the Bamboo Grove, the squirrels’ feeding ground. He went up to the Buddha and exchanged greetings with him. When the greetings and polite conversation were over, he sat down to one side, and addressed the Buddha in verse: 

\begin{verse}%
“I’ve\marginnote{8.1} come full of  doubts and uncertainties,” \\
\scspeaker{said Sabhiya, }\\
“wishing to ask some questions. \\
Please solve them for me. \\
Answer my questions in turn, in accordance with the truth.” 

“You\marginnote{9.1} have come from afar, Sabhiya,” \\
\scspeaker{said the Buddha, }\\
“wishing to ask some questions. \\
I shall solve them for you, \\
answering your questions in turn, in accordance with the truth. 

Ask\marginnote{10.1} me your question, Sabhiya, \\
whatever you want. \\
I’ll solve each and every \\
question you have.” 

%
\end{verse}

Then\marginnote{11.1} Sabhiya thought, “It’s incredible, it’s amazing! Where those other ascetics and brahmins didn’t even give me a chance, the Buddha has invited me.” Uplifted and elated, full of rapture and happiness, he asked this question. 

\begin{verse}%
“What\marginnote{12.1} must one attain to be called a mendicant?” \\
\scspeaker{said Sabhiya, }\\
“How is one ‘sweet’, how said to be ‘tamed’? \\
How is one declared to be ‘awakened’? \\
May the Buddha please answer my question.” 

“When\marginnote{13.1} by the path they have walked themselves,” \\
\scspeaker{said the Buddha to Sabhiya, }\\
“they reach quenching, with doubt overcome; \\
giving up desire to continue existence or to end it, \\
their journey complete, their rebirths ended: that is a mendicant. 

Equanimous\marginnote{14.1} towards everything, mindful, \\
they don’t harm anyone in the world. \\
An ascetic who has crossed over, unclouded, \\
not full of themselves, is sweet-natured. 

Their\marginnote{15.1} faculties have been developed \\
inside and out in the whole world. \\
Having pierced through this world and the next, \\
tamed, they bide their time. 

They\marginnote{16.1} have examined the aeons in their entirety,\footnote{The commentary is surely correct in glossing these lines as \textit{\textbf{\textsanskrit{kimevidaṁ}} \textsanskrit{imissā} \textsanskrit{dārikāya} \textbf{\textsanskrit{muttakarīsapuṇṇaṁ}} \textsanskrit{rūpaṁ}}. \textit{\textsanskrit{Idaṁ}} (neuter) refers to the body (\textit{\textsanskrit{rūpa}}) not to the girl. It’s a meaningful distinction; the Buddha was criticizing the nature of the  body, not shaming the person. } \\
and both sides of transmigration—passing away and rebirth. \\
Rid of dust, unblemished, purified: \\
the one they call ‘awakened’ has attained the end of rebirth.” 

%
\end{verse}

And\marginnote{17.1} then, having approved and agreed with what the Buddha said, uplifted and elated, full of rapture and happiness, Sabhiya asked another question: 

\begin{verse}%
“What\marginnote{18.1} must one attain to be called ‘brahmin’?” \\
\scspeaker{said Sabhiya. }\\
“Why is one an ‘ascetic’, and how a ‘bathed initiate’? \\
How is one declared to be a ‘giant’? \\
May the Buddha please answer my question.” 

“Having\marginnote{19.1} banished all bad things,” \\
\scspeaker{said the Buddha to Sabhiya, }\\
“immaculate, well-composed, steadfast, \\
consummate, they’ve left transmigration behind: \\
such an unattached one is called ‘brahmin’. 

A\marginnote{20.1} peaceful one who has given up good and evil, \\
stainless, understanding this world and the next, \\
gone beyond rebirth and death: \\
such an one is rightly called ‘ascetic’. 

Having\marginnote{21.1} washed off all bad things \\
inside and out in the whole world, \\
among gods and humans bound to creations, \\
the one they call ‘washed’ does not return to creation. 

They\marginnote{22.1} do nothing monstrous at all in the world, \\
discarding all fetters and bonds, \\
everywhere not stuck, freed: \\
such an one is rightly called ‘giant’.” 

%
\end{verse}

And\marginnote{23.1} then Sabhiya asked another question: 

\begin{verse}%
“Who\marginnote{24.1} is a ‘field-victor’ according to the Buddhas?”\footnote{\textit{Icche} is 1st optative. \textit{\textsanskrit{Naṁ}} is rendered as feminine (“her”) by Norman and Bodhi, but it agrees with \textit{\textsanskrit{idaṁ}} in the previous line, so it should be neuter (“it”). Again, the Buddha says nothing in criticism of the woman, only of the body. } \\
\scspeaker{said Sabhiya, }\\
“Why is one ‘skillful’, and how ‘a wise scholar’? \\
How is one declared to be a ‘sage’? \\
May the Buddha please answer my question.” 

“They\marginnote{25.1} are victorious over the fields of deeds in their entirety,”\footnote{The sense only emerges with the specific nuance of terms. \textit{Niccheyya} means to “judge, decide” not simply “consider”; and \textit{\textsanskrit{samuggahītaṁ}} means “[a view that is] adopted” (not “assumed” or “grasped”). Finally, the \textit{na} here applies, not just to \textit{\textsanskrit{idaṁ} \textsanskrit{vadāmi}}, but also to \textit{\textsanskrit{samuggahītaṁ}}, with which \textit{\textsanskrit{idaṁ}} agrees. } \\
\scspeaker{said the Buddha to Sabhiya, }\\
“the fields of gods, humans, and Brahmas; \\
released from the root bondage to all fields: \\
such a one is rightly called ‘field-victor’. 

They\marginnote{26.1} have examined the stockpiles of deeds in their entirety, \\
the stockpiles of gods, humans, and Brahmas; \\
released from the root bondage to all stockpiles: \\
such a one is rightly called ‘skillful’. 

They\marginnote{27.1} have examined whiteness\footnote{\textit{Suti} is what is “heard”, but with the extra emphasis on “heard via oral transmission of a sacred scripture”. I think the case should be read here as “instrumental of relation”, since the question was not, “how is peace attained”, but “how is it described”. } \\
both inside and out; understanding purity, \\
they have left dark and bright behind: \\
such an one is rightly called ‘a wise scholar’.”\footnote{\textit{Pi} here doesn’t qualify \textit{\textsanskrit{sīlabbata}} (per Norman), it co-ordinates with \textit{nopi tena}. I capture this with the neither/nor construction. } 

Understanding\marginnote{28.1} the nature of the bad and the good \\
inside and out in the whole world; \\
one worthy of honor by gods and humans, \\
who has escaped from the net and the snare: that is a sage.” 

%
\end{verse}

And\marginnote{29.1} then Sabhiya asked another question: 

\begin{verse}%
“What\marginnote{30.1} must one attain to be called ‘knowledge master’?” \\
\scspeaker{said Sabhiya, }\\
“Why is one ‘studied’, and how is one ‘heroic’? \\
How to gain the name ‘thoroughbred’? \\
May the Buddha please answer my question.” 

“They\marginnote{31.1} have examined knowledges in their entirety,” \\
\scspeaker{said the Buddha to Sabhiya, }\\
“those that are current among ascetics and brahmins; \\
rid of greed for all feelings, \\
having left all knowledges behind: that is a knowledge master. 

Having\marginnote{32.1} studied proliferation and name \& form \\
inside and out—the root of disease; \\
released from the root bondage to all disease: \\
such an one is rightly called ‘studied’. 

Refraining\marginnote{33.1} from all evil here, \\
heroic, he escapes from the suffering of hell;\footnote{\textit{Neyyo} has the sense “needing to be led”, for example a student who “requires education” before they can master a passage (AN 4.133:1.3). Here it refers to the arahant. } \\
he is heroic and energetic: \\
such an one is rightly called ‘hero’.\footnote{\textit{\textsanskrit{Anupanīto}} (“not led in”) relates to \textit{neyyo} in the previous line. \textit{\textsanskrit{Anupanīto}} has the sense of “one who has been educated (in recitation)” at MN 93:15.2. Compare English “brought into the fold”. The sense of “dogma” for \textit{nivesana} is established at Snp 4.3:6.3 and Snp 4.5:6.3. } 

Whoever’s\marginnote{34.1} bonds are cut, \\
the root of clinging inside and out; \\
released from the root bondage to all clinging: \\
such an one is rightly called ‘thoroughbred’.” 

%
\end{verse}

And\marginnote{35.1} then Sabhiya asked another question: 

\begin{verse}%
“What\marginnote{36.1} must one attain to be called ‘scholar’?” \\
\scspeaker{said Sabhiya, }\\
“Why is one ‘noble’, and how is one ‘well conducted’? \\
How to gain the name ‘wanderer’? \\
May the Buddha please answer my question.” 

“One\marginnote{37.1} who has learned every teaching,”\footnote{See SN 44.1, MN 72. } \\
\scspeaker{said the Buddha to Sabhiya, }\\
“and has known for themselves whatever is blameworthy and blameless in the world; \\
a champion, decided, liberated, \\
untroubled everywhere: they call them ‘scholar’. 

Having\marginnote{38.1} cut off defilements and attachments, \\
being wise, they enter no womb. \\
They’ve expelled the bog of the three perceptions,\footnote{Such as ignorance (AN 4.10:9.4) or craving (Dhp 342). } \\
the one they call ‘noble’ does not return to creation. 

One\marginnote{39.1} here who is accomplished and skillful in all forms of good conduct;\footnote{Agreeing with Norman that this is a rare instance of \textit{saddha} in the sense of wish, desire, per Sanskrit \textit{\textsanskrit{śraddhā}}, one sense of which is “appetite”. } \\
always understanding the teaching, \\
everywhere not stuck, freed in mind, \\
who has no repulsion: they are ‘well-conducted’. 

Avoiding\marginnote{40.1} any deed that results in suffering—\\
above, below, all round, between: \\
deceit and conceit, as well as greed and anger,\footnote{As in Snp 4.3:8.3 this makes better sense considered as an active process. } \\
they live full of wisdom. \\
They have made a limit on name \& form;\footnote{Given the significance of “views” in the \textsanskrit{Aṭṭhakavagga}, and that \textit{dhamma} is commonly used in the sense of “teaching”, I read this in view of such passages as MN 47:14.3: \textit{dhammesu \textsanskrit{niṭṭhaṁ} gacchati} “comes to a conclusion about the teachings”. } \\
the one they call a ‘wanderer’ has reached their destination.”\footnote{\textit{\textsanskrit{Samparāya}} means “in the next life”. The Niddesa’s gloss of “refuge, shelter” etc. is not meant to change this but to qualify it: people look for safety in the next life. } 

%
\end{verse}

And\marginnote{41.1} then, having approved and agreed with what the Buddha said, uplifted and elated, full of rapture and happiness, Sabhiya got up from his seat, arranged his robe over one shoulder, raised his joined palms toward the Buddha, and extolled the Buddha in his presence with fitting verses: 

\begin{verse}%
“O\marginnote{42.1} one of vast wisdom, there are three \& sixty opinions \\
based on the doctrines of ascetics: \\
they are expressions of perception, based on perception. \\
Having dispelled them all, you passed over the dark flood. 

You\marginnote{43.1} have gone to the end, gone beyond suffering, \\
you are perfected, a fully awakened Buddha; I think you have ended defilements. \\
Splendid, intelligent, abounding in wisdom, \\
ender of suffering—you brought me across! 

When\marginnote{44.1} you understood my uncertainty, \\
you brought me beyond doubt—homage to you! \\
A sage, accomplished in the ways of sagacity, \\
you are gentle, not hardhearted, O Kinsman of the Sun.\footnote{Here \textit{dvaya} obviously refers back to the pair of the previous lines and should not be overinterpreted as “duality”. } 

Any\marginnote{45.1} doubts that I once had, \\
you have answered for me, O Seer. \\
Clearly you are a sage, an Awakened One, \\
there are no hindrances in you. 

All\marginnote{46.1} your distress \\
is blown away and mown down. \\
Cooled, tamed, steadfast: \\
truth is your strength. 

O\marginnote{47.1} giant among giants, O great hero, \\
when you are speaking \\
all the gods rejoice, \\
including both \textsanskrit{Nārada} and Pabbata.\footnote{The past participle \textit{sameta} here should be read, as per Niddesa, similarly to \textit{\textsanskrit{paṭipanna}}, i.e. “engaged in the practice” rather than “completed the practice”. } 

Homage\marginnote{48.1} to you, O thoroughbred! \\
Homage to you, supreme among men! \\
In the world with its gods, \\
you have no counterpart. 

You\marginnote{49.1} are the Buddha, you are the Teacher, \\
you are the sage who has overcome \textsanskrit{Māra}; \\
you have cut off the underlying tendencies, \\
you’ve crossed over, and you bring humanity across. 

You\marginnote{50.1} have transcended attachments, \\
your defilements are shattered; \\
you are a lion, free of grasping, \\
with fear and dread given up. 

Like\marginnote{51.1} a graceful lotus \\
to which water does not stick, \\
so both good and evil \\
do not stick to you. \\
Stretch out your feet, great hero: \\
Sabhiya bows to the Teacher.” 

%
\end{verse}

Then\marginnote{52.1} the wanderer Sabhiya bowed with his head at the Buddha’s feet and said, “Excellent, sir! Excellent! … I go for refuge to the Buddha, to the teaching, and to the mendicant \textsanskrit{Saṅgha}. Sir, may I receive the going forth, the ordination in the Buddha’s presence?” 

“Sabhiya,\marginnote{53.1} if someone formerly ordained in another sect wishes to take the going forth, the ordination in this teaching and training, they must spend four months on probation. When four months have passed, if the mendicants are satisfied, they’ll give the going forth, the ordination into monkhood. However, I have recognized individual differences in this matter.” 

“Sir,\marginnote{54.1} if four months probation are required in such a case, I’ll spend four years on probation. When four years have passed, if the mendicants are satisfied, let them give me the going forth, the ordination into monkhood.” And the wanderer Sabhiya received the going forth, the ordination in the Buddha’s presence. And Venerable Sabhiya became one of the perfected. 

%
\section*{{\suttatitleacronym Snp 3.7}{\suttatitletranslation With Sela }{\suttatitleroot Selasutta}}
\addcontentsline{toc}{section}{\tocacronym{Snp 3.7} \toctranslation{With Sela } \tocroot{Selasutta}}
\markboth{With Sela }{Selasutta}
\extramarks{Snp 3.7}{Snp 3.7}

\scevam{So\marginnote{1.1} I have heard. }At one time the Buddha was wandering in the land of the Northern \textsanskrit{Āpaṇas} together with a large \textsanskrit{Saṅgha} of 1,250 mendicants when he arrived at a town of the Northern \textsanskrit{Āpaṇas} named \textsanskrit{Āpaṇa}. The matted-hair ascetic \textsanskrit{Keṇiya} heard: “It seems the ascetic Gotama—a Sakyan, gone forth from a Sakyan family—has arrived at \textsanskrit{Āpaṇa}, together with a large \textsanskrit{Saṅgha} of 1,250 mendicants. He has this good reputation: ‘That Blessed One is perfected, a fully awakened Buddha, accomplished in knowledge and conduct, holy, knower of the world, supreme guide for those who wish to train, teacher of gods and humans, awakened, blessed.’ He has realized with his own insight this world—with its gods, \textsanskrit{Māras} and \textsanskrit{Brahmās}, this population with its ascetics and brahmins, gods and humans—and he makes it known to others. He teaches Dhamma that’s good in the beginning, good in the middle, and good in the end, meaningful and well-phrased. And he reveals a spiritual practice that’s entirely full and pure. It’s good to see such perfected ones.” 

So\marginnote{2.1} \textsanskrit{Keṇiya} approached the Buddha and exchanged greetings with him. When the greetings and polite conversation were over, he sat down to one side. The Buddha educated, encouraged, fired up, and inspired him with a Dhamma talk. Then he said to the Buddha, “Would Master Gotama together with the mendicant \textsanskrit{Saṅgha} please accept tomorrow’s meal from me?” When he said this, the Buddha said to him, “The \textsanskrit{Saṅgha} is large, \textsanskrit{Keṇiya}; there are 1,250 mendicants. And you are devoted to the brahmins.” 

For\marginnote{3.1} a second time, \textsanskrit{Keṇiya} asked the Buddha to accept a meal offering. “Never mind that the \textsanskrit{Saṅgha} is large, with 1,250 mendicants, and that I am devoted to the brahmins. Would Master Gotama together with the mendicant \textsanskrit{Saṅgha} please accept tomorrow’s meal from me?” And for a second time, the Buddha gave the same reply. 

For\marginnote{4.1} a third time, \textsanskrit{Keṇiya} asked the Buddha to accept a meal offering. “Never mind that the \textsanskrit{Saṅgha} is large, with 1,250 mendicants, and that I am devoted to the brahmins. Would Master Gotama together with the mendicant \textsanskrit{Saṅgha} please accept tomorrow’s meal from me?” The Buddha consented in silence. Then, knowing that the Buddha had consented, \textsanskrit{Keṇiya} got up from his seat and went to his own hermitage. There he addressed his friends and colleagues, relatives and family members, “My friends and colleagues, relatives and family members: please listen! The ascetic Gotama together with the mendicant \textsanskrit{Saṅgha} has been invited by me for tomorrow’s meal. Please help me with the preparations.” “Yes, sir,” they replied. Some dug ovens, some chopped wood, some washed dishes, some set out a water jar, and some spread out seats. Meanwhile, \textsanskrit{Keṇiya} set up the pavilion himself. 

Now\marginnote{5.1} at that time the brahmin Sela was residing in \textsanskrit{Āpaṇa}. He had mastered the three Vedas, together with their vocabularies, ritual, phonology and etymology, and the testament as fifth. He knew philology and grammar, and was well versed in cosmology and the marks of a great man. And he was teaching three hundred students to recite the hymns. 

And\marginnote{6.1} at that time \textsanskrit{Keṇiya} was devoted to Sela. Then Sela, while going for a walk escorted by the three hundred students, approached \textsanskrit{Keṇiya}’s hermitage. He saw the preparations going on, and said to \textsanskrit{Keṇiya}, “\textsanskrit{Keṇiya}, is your son or daughter being married? Or are you setting up a big sacrifice? Or has King Seniya \textsanskrit{Bimbisāra} of Magadha been invited for tomorrow’s meal?” 

“There\marginnote{7.1} is no marriage, Sela, and the king is not coming. Rather, I am setting up a big sacrifice. The ascetic Gotama has arrived at \textsanskrit{Āpaṇa}, together with a large \textsanskrit{Saṅgha} of 1,250 mendicants. He has this good reputation: ‘That Blessed One is perfected, a fully awakened Buddha, accomplished in knowledge and conduct, holy, knower of the world, supreme guide for those who wish to train, teacher of gods and humans, awakened, blessed.’ He has been invited by me for tomorrow’s meal together with the mendicant \textsanskrit{Saṅgha}.” “Mister \textsanskrit{Keṇiya}, did you say ‘the awakened one’?” “I said ‘the awakened one’.” “Mister \textsanskrit{Keṇiya}, did you say ‘the awakened one’?” “I said ‘the awakened one’.” 

Then\marginnote{8.1} Sela thought, “It’s hard to even find the word ‘awakened one’ in the world. The thirty-two marks of a great man have been handed down in our hymns. A great man who possesses these has only two possible destinies, no other. If he stays at home he becomes a king, a wheel-turning monarch, a just and principled king. His dominion extends to all four sides, he achieves stability in the country, and he possesses the seven treasures. He has the following seven treasures: the wheel, the elephant, the horse, the jewel, the woman, the treasurer, and the counselor as the seventh treasure. He has over a thousand sons who are valiant and heroic, crushing the armies of his enemies. After conquering this land girt by sea, he reigns by principle, without rod or sword. But if he goes forth from the lay life to homelessness, he becomes a perfected one, a fully awakened Buddha, who draws back the veil from the world. “But \textsanskrit{Keṇiya}, where is the Blessed One at present, the perfected one, the fully awakened Buddha?” 

When\marginnote{9.1} he said this, \textsanskrit{Keṇiya} pointed with his right arm and said, “There, Mister Sela, at that line of blue forest.” Then Sela, together with his students, approached the Buddha. He said to his students, “Come quietly, gentlemen, tread gently. For the Buddhas are intimidating, like a lion living alone. When I’m consulting with the ascetic Gotama, don’t interrupt. Wait until I’ve finished speaking.” 

Then\marginnote{10.1} Sela went up to the Buddha, and exchanged greetings with him. When the greetings and polite conversation were over, he sat down to one side. and scrutinized the Buddha’s body for the thirty-two marks of a great man. He saw all of them except for two, which he had doubts about: whether the private parts were covered in a foreskin, and the largeness of the tongue. 

Then\marginnote{11.1} it occurred to the Buddha, “Sela sees all the marks except for two, which he has doubts about: whether the private parts are covered in a foreskin, and the largeness of the tongue.” The Buddha used his psychic power to will that Sela would see his private parts covered in a foreskin. And he stuck out his tongue and stroked back and forth on his ear holes and nostrils, and covered his entire forehead with his tongue. 

Then\marginnote{12.1} Sela thought, “The ascetic Gotama possesses the thirty-two marks completely, lacking none. But I don’t know whether or not he is an awakened one. I have heard that brahmins of the past who were elderly and senior, the teachers of teachers, said, ‘Those who are perfected ones, fully awakened Buddhas reveal themselves when praised.’ Why don’t I extoll him in his presence with fitting verses?” Then Sela extolled the Buddha in his presence with fitting verses: 

\begin{verse}%
“O\marginnote{13.1} Blessed One, your body’s perfect, \\
you’re radiant, handsome, lovely to behold; \\
golden colored, \\
with teeth so white; you’re strong. 

The\marginnote{14.1} characteristics \\
of a handsome man, \\
the marks of a great man, \\
are all found on your body. 

Your\marginnote{15.1} eyes are clear, your face is fair, \\
you’re formidable, upright, majestic. \\
In the midst of the \textsanskrit{Saṅgha} of ascetics, \\
you shine like the sun. 

You’re\marginnote{16.1} a mendicant fine to see, \\
with skin of golden sheen. \\
But with such excellent appearance, \\
what do you want with the ascetic life? 

You’re\marginnote{17.1} fit to be a king, \\
a wheel-turning monarch, chief of charioteers, \\
victorious in the four quarters, \\
lord of all India. 

Aristocrats,\marginnote{18.1} nobles, and kings \\
ought follow your rule. \\
Gotama, you should reign \\
as king of kings, lord of men!” 

“I\marginnote{19.1} am a king, Sela”, \\
\scspeaker{said the Buddha, }\\
“the supreme king of the teaching. \\
By the teaching I roll forth the wheel \\
which cannot be rolled back.” 

“You\marginnote{20.1} claim to be awakened,” \\
\scspeaker{said Sela the brahmin, }\\
“the supreme king of the teaching. \\
‘I roll forth the teaching’: \\
so you say, Gotama. 

Then\marginnote{21.1} who is your general, \\
the disciple who follows the Teacher’s way? \\
Who keeps rolling the wheel \\
of the teaching you rolled forth?” 

“By\marginnote{22.1} me the wheel was rolled forth,” \\
\scspeaker{said the Buddha, }\\
“the supreme wheel of the teaching. \\
\textsanskrit{Sāriputta}, taking after the Realized One, \\
keeps it rolling on. 

I\marginnote{23.1} have known what should be known, \\
and developed what should be developed, \\
and given up what should be given up: \\
and so, brahmin, I am a Buddha. 

Dispel\marginnote{24.1} your doubt in me—\\
make up your mind, brahmin! \\
The sight of a Buddha \\
is hard to find again. 

I\marginnote{25.1} am a Buddha, brahmin, \\
the supreme surgeon, \\
one of those whose appearance in the world \\
is hard to find again. 

Holy,\marginnote{26.1} unequaled, \\
crusher of \textsanskrit{Māra}’s army; \\
having subdued all my opponents, \\
I rejoice, fearing nothing from any quarter.” 

“Pay\marginnote{27.1} heed, sirs, to what \\
is spoken by the seer. \\
The surgeon, the great hero, \\
roars like a lion in the jungle. 

Holy,\marginnote{28.1} unequaled, \\
crusher of \textsanskrit{Māra}’s army; \\
who would not be inspired by him, \\
even one whose nature is dark? 

Those\marginnote{29.1} who wish may follow me; \\
those who don’t may go. \\
Right here, I’ll go forth in his presence, \\
the one of such splendid wisdom.” 

“Sir,\marginnote{30.1} if you like \\
in the teaching of the Buddha, \\
we’ll also go forth in his presence, \\
the one of such splendid wisdom.” 

“These\marginnote{31.1} three hundred brahmins \\
with joined palms held up, ask: \\
‘May we lead the spiritual life \\
in your presence, Blessed One?’” 

“The\marginnote{32.1} spiritual life is well explained,” \\
\scspeaker{said the Buddha, }\\
“visible in this very life, immediately effective. \\
Here the going forth isn’t in vain \\
for one who trains with diligence.” 

%
\end{verse}

And\marginnote{33.1} the brahmin Sela together with his assembly received the going forth, the ordination in the Buddha’s presence. And when the night had passed \textsanskrit{Keṇiya} had a variety of delicious foods prepared in his own home. Then he had the Buddha informed of the time, saying, “Itʼs time, Master Gotama, the meal is ready.” Then the Buddha robed up in the morning and, taking his bowl and robe, went to \textsanskrit{Keṇiya}’s hermitage, where he sat on the seat spread out, together with the \textsanskrit{Saṅgha} of mendicants. 

Then\marginnote{34.1} \textsanskrit{Keṇiya} served and satisfied the mendicant \textsanskrit{Saṅgha} headed by the Buddha with his own hands with a variety of delicious foods. When the Buddha had eaten and washed his hand and bowl, \textsanskrit{Keṇiya} took a low seat and sat to one side. The Buddha expressed his appreciation with these verses: 

\begin{verse}%
“The\marginnote{35.1} foremost of sacrifices is offering to the sacred flame; \\
the \textsanskrit{Gāyatrī} Mantra is the foremost of poetic meters; \\
of humans, the king is the foremost; \\
the ocean’s the foremost of rivers; 

the\marginnote{36.1} foremost of stars is the moon; \\
the sun is the foremost of lights; \\
for those who sacrifice seeking merit, \\
the \textsanskrit{Saṅgha} is the foremost.” 

%
\end{verse}

When\marginnote{37.1} the Buddha had expressed his appreciation to \textsanskrit{Keṇiya} the matted-hair ascetic with these verses, he got up from his seat and left. Then Venerable Sela and his assembly, living alone, withdrawn, diligent, keen, and resolute, soon realized the supreme end of the spiritual path in this very life. They lived having achieved with their own insight the goal for which gentlemen rightly go forth from the lay life to homelessness. And Venerable Sela together with his assembly became perfected. 

Then\marginnote{38.1} Sela with his assembly went to see the Buddha. He arranged his robe over one shoulder, raised his joined palms toward the Buddha, and said: 

\begin{verse}%
“This\marginnote{39.1} is the eighth day since \\
we went for refuge, O seer. \\
In these seven days, Blessed One, \\
we’ve become tamed in your teaching. 

You\marginnote{40.1} are the Buddha, you are the Teacher, \\
you are the sage who has overcome \textsanskrit{Māra}; \\
you have cut off the underlying tendencies, \\
you’ve crossed over, and you bring humanity across. 

You\marginnote{41.1} have transcended attachments, \\
your defilements are shattered; \\
you are a lion, free of grasping, \\
with fear and dread given up. 

These\marginnote{42.1} three hundred mendicants \\
stand with joined palms raised. \\
Stretch out your feet, great hero: \\
let these giants bow to the Teacher.” 

%
\end{verse}

%
\section*{{\suttatitleacronym Snp 3.8}{\suttatitletranslation The Dart }{\suttatitleroot Sallasutta}}
\addcontentsline{toc}{section}{\tocacronym{Snp 3.8} \toctranslation{The Dart } \tocroot{Sallasutta}}
\markboth{The Dart }{Sallasutta}
\extramarks{Snp 3.8}{Snp 3.8}

\begin{verse}%
Unforeseen\marginnote{1.1} and unknown\footnote{Following Niddesa. } \\
is the extent of this mortal life—\\
hard and short \\
and bound to pain. 

There\marginnote{2.1} is no way that \\
those born will not die. \\
On reaching old age death follows: \\
such is the nature of living creatures. 

As\marginnote{3.1} ripe fruit \\
are always in danger of falling,\footnote{From the following verses we can infer that these enigmatic lines refer to an advanced state of \textit{\textsanskrit{samādhi}}, probably the formless attainments. These are not “normal” as they have no sense-perception or defilements; they are not “distorted” as they are free of hindrances; they are not the non-percipient realm; and they do not perceive what has disappeared, namely the \textit{\textsanskrit{rūpa}} or the \textit{sukha} of lower absorptions. } \\
so mortals once born \\
are always in danger of death. 

As\marginnote{4.1} clay pots \\
made by a potter \\
all end up being broken, \\
so is the life of mortals. 

Young\marginnote{5.1} and old, \\
foolish and wise—\\
all go under the sway of death; \\
all are destined to die. 

When\marginnote{6.1} those overcome by death \\
leave this world for the next, \\
a father cannot protect his son, \\
nor relatives their kin. 

See\marginnote{7.1} how, while relatives look on, \\
wailing profusely, \\
mortals are led away one by one, \\
like a cow to the slaughter. 

And\marginnote{8.1} so the world is stricken \\
by old age and by death. \\
That is why the wise do not grieve, \\
for they understand the way of the world. 

For\marginnote{9.1} one whose path you do not know—\\
not whence they came nor where they went—\\
you lament in vain, \\
seeing neither end. 

If\marginnote{10.1} a bewildered person, \\
lamenting and self-harming, \\
could extract any good from that, \\
then those who see clearly would do the same. 

For\marginnote{11.1} not by weeping and wailing \\
will you find peace of heart. \\
It just gives rise to more suffering, \\
and distresses your body. 

Growing\marginnote{12.1} thin and pale, \\
you hurt yourself. \\
It does nothing to help the dead: \\
your lamentation is in vain. 

Unless\marginnote{13.1} a person gives up grief, \\
they fall into suffering all the more. \\
Bewailing those whose time has come, \\
you fall under the sway of grief. 

See,\marginnote{14.1} too, other folk departing \\
to fare after their deeds; \\
fallen under the sway of death, \\
beings flounder while still here. 

For\marginnote{15.1} whatever you imagine it is, \\
it turns out to be something else. \\
Such is separation: \\
see the way of the world! 

Even\marginnote{16.1} if a human lives \\
a hundred years or more, \\
they are parted from their family circle, \\
they leave this life behind. 

Therefore,\marginnote{17.1} having learned from the Perfected One, \\
dispel lamentation. \\
Seeing the dead and departed, think: \\
“I cannot escape this.”\footnote{In MN 18, we have the sequence perception, thought, proliferation, then \textit{\textsanskrit{papañcasaññāsaṅkhā}}. This suggests that \textit{\textsanskrit{papañca}} causes \textit{\textsanskrit{saṅkhā}}, as per Bodhi. } 

As\marginnote{18.1} one would extinguish \\
a blazing refuge with water, \\
so too a sage—a wise, \\
astute, and skilled person—\\
would swiftly blow away grief that comes up, \\
like the wind a tuft of cotton. 

One\marginnote{19.1} who seeks their own happiness \\
would pluck out the dart from themselves—\\
the wailing and moaning, \\
and sadness inside. 

With\marginnote{20.1} dart plucked out, unattached, \\
having found peace of mind, \\
overcoming all sorrow, \\
one is sorrowless and extinguished. 

%
\end{verse}

%
\section*{{\suttatitleacronym Snp 3.9}{\suttatitletranslation With Vāseṭṭha }{\suttatitleroot Vāseṭṭhasutta}}
\addcontentsline{toc}{section}{\tocacronym{Snp 3.9} \toctranslation{With Vāseṭṭha } \tocroot{Vāseṭṭhasutta}}
\markboth{With Vāseṭṭha }{Vāseṭṭhasutta}
\extramarks{Snp 3.9}{Snp 3.9}

\scevam{So\marginnote{1.1} I have heard. }At one time the Buddha was staying in a forest near \textsanskrit{Icchānaṅgala}. Now at that time several very well-known well-to-do brahmins were residing in \textsanskrit{Icchānaṅgala}. They included the brahmins \textsanskrit{Caṅkī}, \textsanskrit{Tārukkha}, \textsanskrit{Pokkharasāti}, \textsanskrit{Jāṇussoṇi}, Todeyya, and others. Then as the brahmin students \textsanskrit{Vāseṭṭha} and \textsanskrit{Bhāradvāja} were going for a walk they began to discuss the question: “How do you become a brahmin?” 

\textsanskrit{Bhāradvāja}\marginnote{2.1} said this: “When you’re well born on both your mother’s and father’s side, of pure descent, irrefutable and impeccable in questions of ancestry back to the seventh paternal generation—then you’re a brahmin.” 

\textsanskrit{Vāseṭṭha}\marginnote{3.1} said this: “When you’re ethical and accomplished in doing your duties—then you’re a brahmin.” But neither was able to persuade the other. 

So\marginnote{4.1} \textsanskrit{Vāseṭṭha} said to \textsanskrit{Bhāradvāja}, “Master \textsanskrit{Bhāradvāja}, the ascetic Gotama—a Sakyan, gone forth from a Sakyan family—is staying in a forest near \textsanskrit{Icchānaṅgala}. He has this good reputation: ‘That Blessed One is perfected, a fully awakened Buddha, accomplished in knowledge and conduct, holy, knower of the world, supreme guide for those who wish to train, teacher of gods and humans, awakened, blessed.’ Come, let’s go to see him and ask him about this matter. As he answers, so we’ll remember it.” “Yes, sir,” replied \textsanskrit{Bhāradvāja}. 

So\marginnote{5.1} they went to the Buddha and exchanged greetings with him. When the greetings and polite conversation were over, they sat down to one side, and \textsanskrit{Vāseṭṭha} addressed the Buddha in verse: 

\begin{verse}%
“We’re\marginnote{6.1} both authorized masters \\
of the three Vedas. \\
I’m a student of \textsanskrit{Pokkharasāti}, \\
and he of \textsanskrit{Tārukkha}. 

We’re\marginnote{7.1} fully qualified \\
in all the Vedic experts teach. \\
As philologists and grammarians, \\
we match our teachers in recitation. 

We\marginnote{8.1} have a dispute \\
regarding the question of ancestry. \\
For \textsanskrit{Bhāradvāja} says that \\
one is a brahmin due to birth, \\
but I declare it’s because of one’s actions. \\
Oh seer, know this as our debate. 

Since\marginnote{9.1} neither of us was able \\
to convince the other, \\
we’ve come to ask you, sir, \\
renowned as the awakened one. 

As\marginnote{10.1} people honor with joined palms \\
the moon on the cusp of waxing, \\
bowing, they revere \\
Gotama in the world. 

We\marginnote{11.1} ask this of Gotama, \\
the eye arisen in the world: \\
is one a brahmin due to birth, \\
or else because of actions? \\
We don’t know, please tell us, \\
so we can recognize a brahmin.” 

“I\marginnote{12.1} shall explain to you,” \\
\scspeaker{said the Buddha, }\\
“accurately and in sequence, \\
the taxonomy of living creatures, \\
for species are indeed diverse. 

Know\marginnote{13.1} the grass and trees, \\
though they lack self-awareness. \\
They’re defined by birth, \\
for species are indeed diverse. 

Next\marginnote{14.1} there are bugs and moths, \\
and so on, to ants and termites. \\
They’re defined by birth, \\
for species are indeed diverse. 

Know\marginnote{15.1} the quadrupeds, too, \\
both small and large. \\
They’re defined by birth, \\
for species are indeed diverse. 

Know,\marginnote{16.1} too, the long-backed snakes, \\
crawling on their bellies. \\
They’re defined by birth, \\
for species are indeed diverse. 

Next\marginnote{17.1} know the fish, \\
whose habitat is the water. \\
They’re defined by birth, \\
for species are indeed diverse. 

Next\marginnote{18.1} know the birds, \\
flying with wings as chariots. \\
They’re defined by birth, \\
for species are indeed diverse. 

While\marginnote{19.1} the differences between these species \\
are defined by birth, \\
the differences between humans \\
are not defined by birth. 

Not\marginnote{20.1} by hair nor by head, \\
not by ear nor by eye, \\
not by mouth nor by nose, \\
not by lips nor by eyebrow, 

not\marginnote{21.1} by shoulder nor by neck, \\
not by belly nor by back, \\
not by buttocks nor by breast, \\
not by groin nor by genitals, 

not\marginnote{22.1} by hands nor by feet, \\
not by fingers nor by nails, \\
not by knees nor by thighs, \\
not by color nor by voice: \\
none of these are defined by birth \\
as it is for other species. 

In\marginnote{23.1} individual human bodies \\
you can’t find such distinctions. \\
The distinctions among humans \\
are spoken of by convention. 

Anyone\marginnote{24.1} among humans \\
who lives off keeping cattle: \\
know them, \textsanskrit{Vāseṭṭha}, \\
as a farmer, not a brahmin. 

Anyone\marginnote{25.1} among humans \\
who lives off various professions: \\
know them, \textsanskrit{Vāseṭṭha}, \\
as a professional, not a brahmin. 

Anyone\marginnote{26.1} among humans \\
who lives off trade: \\
know them, \textsanskrit{Vāseṭṭha}, \\
as a trader, not a brahmin. 

Anyone\marginnote{27.1} among humans \\
who lives off serving others: \\
know them, \textsanskrit{Vāseṭṭha}, \\
as an employee, not a brahmin. 

Anyone\marginnote{28.1} among humans \\
who lives off stealing: \\
know them, \textsanskrit{Vāseṭṭha}, \\
as a bandit, not a brahmin. 

Anyone\marginnote{29.1} among humans \\
who lives off archery: \\
know them, \textsanskrit{Vāseṭṭha}, \\
as a soldier, not a brahmin. 

Anyone\marginnote{30.1} among humans \\
who lives off priesthood: \\
know them, \textsanskrit{Vāseṭṭha}, \\
as a sacrificer, not a brahmin. 

Anyone\marginnote{31.1} among humans \\
who taxes village and nation, \\
know them, \textsanskrit{Vāseṭṭha}, \\
as a ruler, not a brahmin. 

I\marginnote{32.1} don’t call someone a brahmin \\
after the mother or womb they came from. \\
If they still have attachments, \\
they’re just someone who says ‘sir’. \\
Having nothing, taking nothing: \\
that’s who I call a brahmin. 

Having\marginnote{33.1} cut off all fetters \\
they have no anxiety; \\
they’ve got over clinging, and are detached: \\
that’s who I call a brahmin. 

They’ve\marginnote{34.1} cut the strap and harness, \\
the reins and bridle too; \\
with cross-bar lifted, they’re awakened: \\
that’s who I call a brahmin. 

Abuse,\marginnote{35.1} killing, caging: \\
they endure these without anger. \\
Patience is their powerful army: \\
that’s who I call a brahmin. 

Not\marginnote{36.1} irritable or stuck up, \\
dutiful in precepts and observances, \\
tamed, bearing their final body: \\
that’s who I call a brahmin. 

Like\marginnote{37.1} water from a lotus leaf, \\
like a mustard seed off a pin-point, \\
sensual pleasures slip off them: \\
that’s who I call a brahmin. 

They\marginnote{38.1} understand for themselves \\
the end of suffering in this life; \\
with burden put down, detached: \\
that’s who I call a brahmin. 

Deep\marginnote{39.1} in wisdom, intelligent, \\
expert in the variety of paths; \\
arrived at the highest goal: \\
that’s who I call a brahmin. 

Socializing\marginnote{40.1} with neither \\
householders nor the homeless; \\
a migrant with no shelter, few in wishes: \\
that’s who I call a brahmin. 

They’ve\marginnote{41.1} laid aside violence \\
against creatures firm and frail; \\
not killing or making others kill: \\
that’s who I call a brahmin. 

Not\marginnote{42.1} fighting among those who fight, \\
extinguished among those who are armed, \\
not taking among those who take: \\
that’s who I call a brahmin. 

They’ve\marginnote{43.1} discarded greed and hate, \\
along with conceit and contempt, \\
like a mustard seed off the point of a pin: \\
that’s who I call a brahmin. 

The\marginnote{44.1} words they utter \\
are sweet, informative, and true, \\
and don’t offend anyone: \\
that’s who I call a brahmin. 

They\marginnote{45.1} don’t steal anything in the world, \\
long or short, \\
fine or coarse, beautiful or ugly: \\
that’s who I call a brahmin. 

They\marginnote{46.1} have no hope \\
for this world or the next. \\
with no need for hope, detached: \\
that’s who I call a brahmin. 

They\marginnote{47.1} have no clinging, \\
knowledge has freed them of indecision, \\
they’ve plunged right into the deathless: \\
that’s who I call a brahmin. 

They’ve\marginnote{48.1} escaped clinging \\
to both good and bad deeds; \\
sorrowless, stainless, pure: \\
that’s who I call a brahmin. 

Pure\marginnote{49.1} as the spotless moon, \\
clear and undisturbed, \\
they’ve ended delight and future lives: \\
that’s who I call a brahmin. 

They’ve\marginnote{50.1} got past this grueling swamp \\
of delusion, transmigration. \\
Meditating in stillness, free of indecision, \\
they have crossed over to the far shore. \\
They’re extinguished by not grasping: \\
that’s who I call a brahmin. 

They’ve\marginnote{51.1} given up sensual stimulations, \\
and have gone forth from lay life; \\
they’ve ended rebirth in the sensual realm: \\
that’s who I call a brahmin. 

They’ve\marginnote{52.1} given up craving, \\
and have gone forth from lay life; \\
they’ve ended craving to be reborn: \\
that’s who I call a brahmin. 

They’ve\marginnote{53.1} given up human bonds, \\
and gone beyond heavenly bonds; \\
detached from all attachments: \\
that’s who I call a brahmin. 

Giving\marginnote{54.1} up discontent and desire, \\
they’re cooled and free of attachments; \\
a hero, master of the whole world: \\
that’s who I call a brahmin. 

They\marginnote{55.1} know the passing away \\
and rebirth of all beings; \\
unattached, holy, awakened: \\
that’s who I call a brahmin. 

Gods,\marginnote{56.1} fairies, and humans \\
don’t know their destiny; \\
the perfected ones with defilements ended: \\
that’s who I call a brahmin. 

They\marginnote{57.1} have nothing before or after, \\
or even in between. \\
Having nothing, taking nothing: \\
that’s who I call a brahmin. 

Leader\marginnote{58.1} of the herd, excellent hero, \\
great hermit and victor; \\
unstirred, washed, awakened: \\
that’s who I call a brahmin. 

They\marginnote{59.1} know their past lives, \\
and sees heaven and places of loss, \\
and has attained the ending of rebirth, \\
that’s who I call a brahmin. 

For\marginnote{60.1} name and clan are formulated \\
as mere convention in the world. \\
Produced by mutual agreement, \\
they’re formulated for each individual. 

For\marginnote{61.1} a long time this misconception \\
has prejudiced those who don’t understand. \\
Ignorant, they declare \\
that one is a brahmin by birth. 

You’re\marginnote{62.1} not a brahmin by birth, \\
nor by birth a non-brahmin. \\
You’re a brahmin by your deeds, \\
and by deeds a non-brahmin. 

You’re\marginnote{63.1} a farmer by your deeds, \\
by deeds you’re a professional; \\
you’re a trader by your deeds, \\
by deeds are you an employee; 

you’re\marginnote{64.1} a bandit by your deeds, \\
by deeds you’re a soldier; \\
you’re a sacrificer by your deeds, \\
by deeds you’re a ruler. 

In\marginnote{65.1} this way the astute regard deeds \\
in accord with truth. \\
Seeing dependent origination, \\
they’re expert in deeds and their results. 

Deeds\marginnote{66.1} make the world go on, \\
deeds make people go on; \\
sentient beings are bound by deeds, \\
like a moving chariot’s linchpin. 

By\marginnote{67.1} austerity and spiritual practice, \\
by restraint and by self-control: \\
that’s how to become a brahmin, \\
this is the supreme brahmin. 

Accomplished\marginnote{68.1} in the three knowledges, \\
peaceful, with rebirth ended, \\
know them, \textsanskrit{Vāseṭṭha}, \\
as \textsanskrit{Brahmā} and Sakka to the wise.” 

%
\end{verse}

When\marginnote{69.1} he had spoken, \textsanskrit{Vāseṭṭha} and \textsanskrit{Bhāradvāja} said to him, “Excellent, Master Gotama! Excellent! … From this day forth, may Master Gotama remember us as lay followers who have gone for refuge for life.” 

%
\section*{{\suttatitleacronym Snp 3.10}{\suttatitletranslation With Kokālika }{\suttatitleroot Kokālikasutta}}
\addcontentsline{toc}{section}{\tocacronym{Snp 3.10} \toctranslation{With Kokālika } \tocroot{Kokālikasutta}}
\markboth{With Kokālika }{Kokālikasutta}
\extramarks{Snp 3.10}{Snp 3.10}

\scevam{So\marginnote{1.1} I have heard. }At one time the Buddha was staying near \textsanskrit{Sāvatthī} in Jeta’s Grove, \textsanskrit{Anāthapiṇḍika}’s monastery. Then the mendicant \textsanskrit{Kokālika} went up to the Buddha, bowed, sat down to one side, and said to him, “Sir, \textsanskrit{Sāriputta} and \textsanskrit{Moggallāna} have wicked desires. They’ve fallen under the sway of wicked desires.” 

When\marginnote{2.1} this was said, the Buddha said to \textsanskrit{Kokālika}, “Don’t say that, \textsanskrit{Kokālika}! Don’t say that, \textsanskrit{Kokālika}! Have confidence in \textsanskrit{Sāriputta} and \textsanskrit{Moggallāna}, they’re good monks.” 

For\marginnote{3.1} a second time … For a third time \textsanskrit{Kokālika} said to the Buddha, “Despite my faith and trust in the Buddha, \textsanskrit{Sāriputta} and \textsanskrit{Moggallāna} have wicked desires. They’ve fallen under the sway of wicked desires.” For a third time, the Buddha said to \textsanskrit{Kokālika}, “Don’t say that, \textsanskrit{Kokālika}! Don’t say that, \textsanskrit{Kokālika}! Have confidence in \textsanskrit{Sāriputta} and \textsanskrit{Moggallāna}, they’re good monks.” 

Then\marginnote{4.1} \textsanskrit{Kokālika} got up from his seat, bowed, and respectfully circled the Buddha, keeping him on his right, before leaving. Not long after he left his body erupted with boils the size of mustard seeds. The boils grew to the size of mung beans, then chickpeas, then jujube seeds, then jujubes, then myrobalans, then unripe wood apples, then ripe wood apples. Finally they burst open, and pus and blood oozed out. Then the mendicant \textsanskrit{Kokālika} died of that illness. He was reborn in the Pink Lotus hell because of his resentment for \textsanskrit{Sāriputta} and \textsanskrit{Moggallāna}. 

Then,\marginnote{5.1} late at night, the beautiful \textsanskrit{Brahmā} Sahampati, lighting up the entire Jeta’s Grove, went up to the Buddha, bowed, stood to one side, and said to him, “Sir, the mendicant \textsanskrit{Kokālika} has passed away. He was reborn in the Pink Lotus hell because of his resentment for \textsanskrit{Sāriputta} and \textsanskrit{Moggallāna}.” That’s what \textsanskrit{Brahmā} Sahampati said. Then he bowed and respectfully circled the Buddha, keeping him on his right side, before vanishing right there. 

Then,\marginnote{6.1} when the night had passed, the Buddha told the mendicants all that had happened. 

When\marginnote{7.1} he said this, one of the mendicants said to the Buddha, “Sir, how long is the life span in the Pink Lotus hell?” “It’s long, mendicant. It’s not easy to calculate how many years, how many hundreds or thousands or hundreds of thousands of years it lasts.” “But sir, is it possible to give a simile?” “It’s possible,” said the Buddha. 

“Suppose\marginnote{8.1} there was a Kosalan cartload of twenty bushels of sesame seed. And at the end of every hundred years someone would remove a single seed from it. By this means the Kosalan cartload of twenty bushels of sesame seed would run out faster than a single lifetime in the Abbuda hell. Now, twenty lifetimes in the Abbuda hell equal one lifetime in the Nirabbuda hell. Twenty lifetimes in the Nirabbuda hell equal one lifetime in the Ababa hell. Twenty lifetimes in the Ababa hell equal one lifetime in the \textsanskrit{Aṭaṭa} hell. Twenty lifetimes in the \textsanskrit{Aṭaṭa} hell equal one lifetime in the Ahaha hell. Twenty lifetimes in the Ahaha hell equal one lifetime in the Yellow Lotus hell. Twenty lifetimes in the Yellow Lotus hell equal one lifetime in the Sweet-Smelling hell. Twenty lifetimes in the Sweet-Smelling hell equal one lifetime in the Blue Water Lily hell. Twenty lifetimes in the Blue Water Lily hell equal one lifetime in the White Lotus hell. Twenty lifetimes in the White Lotus hell equal one lifetime in the Pink Lotus hell. The mendicant \textsanskrit{Kokālika} has been reborn in the Pink Lotus hell because of his resentment for \textsanskrit{Sāriputta} and \textsanskrit{Moggallāna}.” That is what the Buddha said. Then the Holy One, the Teacher, went on to say: 

\begin{verse}%
“A\marginnote{9.1} person is born \\
with an axe in their mouth. \\
A fool cuts themselves with it \\
when they say bad words. 

When\marginnote{10.1} you praise someone worthy of criticism, \\
or criticize someone worthy of praise, \\
you choose bad luck with your own mouth: \\
you’ll never find happiness that way. 

Bad\marginnote{11.1} luck at dice is a trivial thing, \\
if all you lose is your money \\
and all you own, even yourself. \\
What’s really terrible luck \\
is to hate the holy ones. 

For\marginnote{12.1} more than two quinquadecillion years, \\
and another five quattuordecillion years, \\
a slanderer of noble ones goes to hell, \\
having aimed bad words and thoughts at them. 

A\marginnote{13.1} liar goes to hell, \\
as does one who denies what they did. \\
Both are equal in the hereafter, \\
those men of base deeds. 

Whoever\marginnote{14.1} wrongs a man who has done no wrong, \\
a pure man who has not a blemish, \\
the evil backfires on the fool, \\
like fine dust thrown upwind. 

One\marginnote{15.1} addicted to the way of greed, \\
abuses others with their speech, \\
faithless, miserly, uncharitable, \\
stingy, addicted to backbiting. 

Foul-mouthed,\marginnote{16.1} divisive, ignoble,\footnote{Oddly, however, the disappearance of \textit{sukha} and \textit{dukkha} is not directly answered. It is, however, implied in the surmounting of 3rd \textit{\textsanskrit{jhāna}}. } \\
a life-destroyer, wicked, wrongdoer, \\
worst of men, cursed, base-born—\\
quiet now, for you are bound for hell.\footnote{Such as the Buddha’s former teachers, who aimed at rebirth in formless realms. See AN 10.29:20.1. } 

You\marginnote{17.1} stir up dust, causing harm,\footnote{How to distinguish this from the Buddha’s teaching? If, following Bodhi, we read \textit{samaya} as “attainment”, we must add in an extra layer of interpretation. If we follow Norman in accepting the (more common) sense of “time, occasion”, then the difference becomes clear: they believe that the end of things happens at some point in the future (probably at death), rather than the Buddha who spoke of realizing the truth now. } \\
when you, evildoer, malign the good. \\
Having done many bad deeds, \\
you’ll go to the pit for a long time. 

For\marginnote{18.1} no-one’s deeds are ever lost, \\
they return to their owner. \\
In the next life that stupid evildoer \\
sees suffering in themselves. 

They\marginnote{19.1} approach the place of impalement, \\
with its iron spikes, sharp blades, and iron stakes. \\
Then there is the food, which appropriately, \\
is like a red-hot iron ball. 

For\marginnote{20.1} the speakers speak not sweetly,\footnote{Here, as often in the \textsanskrit{Aṭṭhakavagga}, \textit{\textsanskrit{jānāti}} implies different ways of knowing, which may be right or wrong (to degrees). We can’t press “know” to serve this sense, per Bodhi, Norman, and \textsanskrit{Ñāṇadīpa}. I can have a different “understanding” to you, I can “see” it differently, but I can’t “know” it differently. } \\
they don’t hurry there, or find shelter. \\
They lie upon a spread of coals, \\
they enter a blazing mass of fire.\footnote{Preferring \textit{omako} over \textit{mago}, which seems a little harsh; \textit{omako} conforms to the principle of least meaning. } 

Wrapping\marginnote{21.1} them in a net, \\
they strike them there with iron hammers. \\
They come to blinding darkness, \\
which spreads about them like a fog. 

Next\marginnote{22.1} they enter a copper pot, \\
a blazing mass of fire. \\
There they roast for a long time, \\
writhing in the masses of fire. 

Then\marginnote{23.1} the evildoer roasts there \\
in a mixture of pus and blood. \\
No matter where they settle, \\
everything they touch there hurts them. 

The\marginnote{24.1} evildoer roasts in \\
worm-infested water. \\
There’s not even a shore to go to, \\
for all around are the same kind of pots. 

They\marginnote{25.1} enter the Wood of Sword-Leaves, \\
so sharp they cut their body to pieces. \\
Having grabbed the tongue with a hook, \\
they stab it, slashing back and forth. 

Then\marginnote{26.1} they approach the impassable  \textsanskrit{Vetaraṇi} River, \\
with its sharp blades, its razor blades. \\
Idiots fall into it, \\
the wicked who have done wicked deeds. 

There\marginnote{27.1} dogs all brown and spotted, \\
and raven flocks, and greedy jackals \\
devour them as they wail, \\
while hawks and crows attack them. 

Hard,\marginnote{28.1} alas, is the life here \\
that evildoers endure. \\
That’s why for the rest of this life \\
a person ought do their duty without fail. 

Experts\marginnote{29.1} have counted the loads of sesame\footnote{I think “this is correct” refers, not to the assertion of a view, but to the assertion that the other is a fool. Niddesa takes it to mean the assertion of one of the 62 views, but this seems to be reading into the text. It might mean that the Buddha does not make dogmatic assertions; but on the face of it, the Buddha is constantly saying that his teaching is correct. It is the four noble truths, after all. } \\
as compared to the Pink Lotus Hell. \\
They amount to 50,000,000 times 10,000, \\
plus another 12,000,000,000. 

As\marginnote{30.1} painful as life is said to be in hell, \\
that’s how long one must dwell there. \\
That’s why, for those who are pure, well-behaved, full of good qualities, \\
one should always guard one’s speech and mind.” 

%
\end{verse}

%
\section*{{\suttatitleacronym Snp 3.11}{\suttatitletranslation About Nālaka }{\suttatitleroot Nālakasutta}}
\addcontentsline{toc}{section}{\tocacronym{Snp 3.11} \toctranslation{About Nālaka } \tocroot{Nālakasutta}}
\markboth{About Nālaka }{Nālakasutta}
\extramarks{Snp 3.11}{Snp 3.11}

\begin{verse}%
The\marginnote{1.1} hermit Asita in his daily meditation \\
saw the bright-clad gods of the Thirty-Three\footnote{Here I take \textit{\textsanskrit{bālā}} as the quoted speech, as in the lines above and below. The commentary supplies the expected quotation indicator: \textit{‘\textsanskrit{bālo}’ti \textsanskrit{āhu}}. If we take \textit{\textsanskrit{bālā}} as subject of the verb here, per Bodhi, Norman, and \textsanskrit{Ñāṇadīpa}, then the Buddha is adopting the reductive language of those he seeks to counter, obscuring the sense of the verse. } \\
and their lord Sakka joyfully celebrating, \\
waving streamers in exuberant exaltation. 

Seeing\marginnote{2.1} the gods rejoicing, elated, \\
he paid respects and said this there: \\
“Why is the community of gods in such excellent spirits? \\
Why take up streamers and whirl them about? 

Even\marginnote{3.1} in the war with the demons, \\
when gods were victorious and demons defeated, \\
there was no such excitement. \\
What marvel have the celestials seen that they so rejoice? 

Shouting\marginnote{4.1} and singing and playing music, \\
they clap their hands and dance. \\
I ask you, dwellers on Mount Meru’s peak, \\
quickly dispel my doubt, good sirs!” 

“The\marginnote{5.1} being intent on awakening, a peerless gem, \\
has been born in the human realm for the sake of welfare and happiness, \\
in \textsanskrit{Lumbinī}, a village in the Sakyan land. \\
That’s why we’re so happy, in such excellent spirits. 

He\marginnote{6.1} is supreme among all beings, the best of people, \\
a bull among men, supreme among all creatures. \\
He will roll forth the wheel in the grove of the hermits, \\
roaring like a mighty lion, lord of beasts.” 

Hearing\marginnote{7.1} this, he swiftly descended \\
and right away approached Suddhodana’s home.\footnote{It’s important to maintain the past sense of \textit{\textsanskrit{akaṁsu}} here: a view is something they have constructed and built up over time; it has solidified and become part of them, which is why it’s so hard to give up. } \\
Seated there he said this to the Sakyans, \\
“Where is the boy? I too wish to see him!” 

Then\marginnote{8.1} the Sakyans showed their son to the one named Asita—\\
the boy shone like burning gold \\
well-wrought in the forge; \\
resplendent with glory, of peerless beauty. 

The\marginnote{9.1} boy beamed like crested flame, \\
pure as the moon, lord of stars traversing the sky, \\
blazing like the sun freed from the clouds after the rains;\footnote{With these lines the Buddha dismisses subjectivist notions of truth. } \\
seeing him, he was joyful, brimming with happiness. 

The\marginnote{10.1} celestials held up a parasol in the sky, \\
many-ribbed and thousand-circled; \\
and golden-handled chowries waved—\\
but none could see who held the chowries or the parasols. 

When\marginnote{11.1} the dreadlocked hermit who they called “Dark Splendor” \\
had seen the boy like a gold nugget on a cream rug\footnote{Read \textit{su \textsanskrit{tāni}}. } \\
with a white parasol held over his head, \\
he received him, elated and happy. 

Having\marginnote{12.1} received the Sakyan bull, \\
the seeker, master of marks and hymns, \\
lifted up his voice with confident heart: \\
“He is supreme, the best of men!” 

But\marginnote{13.1} then, remembering he would depart this world, \\
his spirits fell and his tears flowed. \\
Seeing the weeping hermit, the Sakyans said, \\
“Surely there will be no threat to the boy?” 

Seeing\marginnote{14.1} the crestfallen Sakyans, the hermit said, \\
“I do not forsee harm befall the boy, \\
and there will be no threat to him, \\
not in the least; set your minds at ease. 

This\marginnote{15.1} boy shall reach the highest awakening. \\
As one of perfectly purified vision, compassionate for the welfare of the many, \\
he shall roll forth the wheel of the teaching; \\
his spiritual path will become widespread. 

But\marginnote{16.1} I have not long left in this life, \\
I shall die before then. \\
I will never hear the teaching of the one who bore the unequaled burden.\footnote{The Niddesa glosses \textit{\textsanskrit{aññatra} \textsanskrit{saññāya} \textsanskrit{niccaggāhā}}, which is translated by Bodhi as “apart from the grasping of permanence by perception”. But this seems impossible to me: \textit{\textsanskrit{saññāya}} is an ablative constructed with \textit{\textsanskrit{aññatra}}, and \textit{\textsanskrit{niccāni}} surely agrees with \textit{\textsanskrit{saccāni}}. Further, the sense of this reading is a stretch, as we haven’t dealt with impermanence at all so far. I think the allusion is, rather, to the idea of \textit{\textsanskrit{dhammaniyāmatā}}, the “fixity” or “regularity” of the Dhamma as an “eternal truth”. } \\
That’s why I’m so upset and distraught—it’s a disaster for me!” 

Having\marginnote{17.1} brought abundant happiness to the Sakyans, \\
the spiritual seeker left the royal compound. \\
He had a nephew; and out of compassion \\
he encouraged him in the teaching of the one who bore the unequaled burden. 

“When\marginnote{18.1} you hear the voice of another saying ‘Buddha’—\footnote{This verse begins a rather striking pattern where the first couplet is essentially restated in the second couplet. } \\
one who has attained awakening and who reveals the foremost teaching—\footnote{Again the two couplets repeat the same meaning. } \\
go there and ask about his breakthrough;\footnote{This must refer back to the previous verse, where \textit{\textsanskrit{sayaṁ}} means “on one’s own account”. } \\
lead the spiritual life under that Blessed One.” 

Now,\marginnote{19.1} that \textsanskrit{Nālaka} had a store of accumulated merit; \\
so when instructed by one of such kindly intent,\footnote{Cf. \textit{Na, bhikkhave, \textsanskrit{dhammavādī} kenaci \textsanskrit{lokasmiṁ} vivadati}. } \\
with perfectly purified vision of the future, \\
he waited in hope for the Victor, guarding his senses.\footnote{If we render \textit{khema} as “security” and \textit{\textsanskrit{bhūmi}} as “stage” then we miss the metaphor. \textit{Abhi-√pas} is found at AN 3.39:11.4 and AN 5.57:15.4 where it has a similar sense of “looking forward to”. } 

When\marginnote{20.1} he heard of the Victor rolling forth the excellent wheel he went to him, \\
and seeing the leading hermit, he became confident. \\
The time of Asita’s instruction had arrived; \\
so he asked the excellent sage about the highest sagacity. 

%
\end{verse}

\scendsection{The introductory verses are finished. }

\begin{verse}%
“I\marginnote{21.1} now know that Asita’s words \\
have turned out to be true. \\
I ask you this, Gotama, \\
who has gone beyond all things: 

For\marginnote{22.1} one who has entered the homeless life, \\
seeking food on alms round, \\
when questioned, O sage, please tell me \\
of sagacity, the ultimate state.”\footnote{The Niddesa’s explanation of \textit{santi} as “view” is not plausible. I follow Norman. } 

“I\marginnote{23.1} shall school you in sagacity,” \\
\scspeaker{said the Buddha, }\\
“so difficult and challenging. \\
Come, I shall tell you all about it. \\
Brace yourself; stay strong! 

In\marginnote{24.1} the village, keep the same attitude \\
no matter if reviled or praised. \\
Guard against ill-tempered thoughts, \\
wander peaceful, not frantic. 

Many\marginnote{25.1} different things come up,\footnote{Reading \textit{\textsanskrit{tapūpanissāya}}; see Bodhi’s note 1961. Since in prose \textit{\textsanskrit{tapojigucchā}}usually occur together, I treat them as such here. } \\
like tongues of fire in a forest. \\
Women try to seduce a sage—\\
let them not seduce you! 

Refraining\marginnote{26.1} from sex, \\
having left behind sensual pleasures high and low, \\
don’t be hostile or attached \\
to living creatures firm or frail. 

‘As\marginnote{27.1} am I, so are they; \\
as are they, so am I’—\\
Treating others like oneself, \\
neither kill nor incite to kill. 

Leaving\marginnote{28.1} behind desire and greed \\
for what ordinary people are attached to,\footnote{This refers to the doctrine of \textit{\textsanskrit{saṁsārasuddhi}}, that the process of rebirth leads to release. See MN 102:11.3. \textit{Anutthunati} rather consistently means “bemoans”, and here I think it is used in a disparaging way of those who, suffering under self-inflicted mortification, moan that purity is just around the corner. } \\
a seer would set out to practice, \\
they’d cross over this abyss.\footnote{Here I follow \textsanskrit{Ñāṇadīpa}, who takes the whole verse as a conditional argument, whereas Bodhi and Norman treat each couplet separately. I think the verse is making the argument that each ascetic bases their teaching on their own experience, and so if their teachings were equally valid, then purity would be attained by many (mutually exclusive) means. } 

With\marginnote{29.1} empty stomach, taking limited food, \\
few in wishes, not greedy; \\
truly hungerless regarding all desires,\footnote{Cp. Dhp 165:5, where an apparently similar idea (\textit{\textsanskrit{suddhī} asuddhi \textsanskrit{paccattaṁ}}) is Buddhist. There, the emphasis is on the personal realization of the teachings, here on different, contradictory, realizations. } \\
desireless, one is quenched. 

Having\marginnote{30.1} wandered for alms, \\
they’d take themselves into the forest; \\
and nearing the foot of a tree, \\
the sage would take their seat. 

That\marginnote{31.1} wise one intent on absorption, \\
would delight within the forest. \\
They’d practice absorption at the foot of a tree, \\
filling themselves with bliss. 

Then,\marginnote{32.1} at the end of the night, \\
they’d take themselves into a village. \\
They’d not welcome being called,\footnote{This signifies that we are speaking of ascetics who are, in Jayatilleke’s terms, “experientialists”, relying on their meditation experiences to justify their views. The same phrase describes the Buddha at AN 4.24:9.3. } \\
nor offerings brought from the village. 

A\marginnote{33.1} sage who has come to a village \\
would not walk hastily among the families.\footnote{\textit{\textsanskrit{Atisitvā}} is normally used in the sense of someone who “overlooks” the Buddha to seek answers elsewhere. In AN 3.39:4.1 it’s used in the sense of “overlooking oneself”, which \textsanskrit{Ñāṇadīpa} takes as the relevant sense here. Either reading would make sense here. } \\
They’d not discuss their search for food, \\
nor would they speak suggestively. 

‘I\marginnote{34.1} got something, that’s good. \\
I got nothing, that’s fine.’ \\
Impartial in both cases, \\
they return right to the tree. 

Wandering\marginnote{35.1} with bowl in hand, \\
not dumb, but thought to be dumb, \\
they wouldn’t scorn a tiny gift, \\
nor look down upon the giver. 

For\marginnote{36.1} the practice has many aspects,\footnote{Here the ascetic appears to refer to their meditative attainment. Compare the use of \textit{nissita} at Snp 5.7. } \\
as explained by the Ascetic.\footnote{To render \textit{\textsanskrit{ñāṇa}} here as knowledge would be a mistake. } \\
They do not go to the far shore twice,\footnote{Here, as at Snp 4.3:2.2, \textit{chanda} is better read as “preference” than “desire”. } \\
nor having gone once do they fall away.\footnote{\textit{\textsanskrit{Mantā}} is explained by Niddesa (followed by Bodhi) as “wisdom” and treated as a truncated instrumental. It is elsewhere found as an agent noun in a positive sense, and is thus rendered by \textsanskrit{Ñāṇadīpa} as “deep thinker”. These renderings are certainly possible, but the context, as the next verses make clear, deal with someone who suffers conceit due to their learning and understanding. A \textit{\textsanskrit{mantā}} is one to whom others turn for advice or wise counsel. Furthermore, \textit{\textsanskrit{asmīti}} commonly follows nominatives in a similar sense, eg. \textit{\textsanskrit{seyyohamasmīti}}. Thus I take \textit{\textsanskrit{mantā} \textsanskrit{asmīti}} as a single phrase. } 

When\marginnote{37.1} a mendicant has no creeping, \\
and has cut the stream of craving, \\
and given up all the various duties, \\
no fever is found in them. 

I\marginnote{38.1} shall school you in sagacity.\footnote{Reading \textit{\textsanskrit{puṭṭho}} per variants, and contra most translators and the Niddesa. I find the renderings by Bodhi, Norman, and \textsanskrit{Ñāṇadīpa} to be borderline incomprehensible. Norman and \textsanskrit{Ñāṇadīpa} are surely incorrect in taking \textit{\textsanskrit{rūpa}} here as “forms”, but that is a tempting direction once we read \textit{\textsanskrit{phuṭṭho}}. The passage up till now has been about conceit that arises from knowledge. I think the sense here is the same as AN 9.14:11.11: \textit{\textsanskrit{Sādhu} kho \textsanskrit{tvaṁ}, samiddhi, \textsanskrit{puṭṭho} \textsanskrit{puṭṭho} vissajjesi, tena ca \textsanskrit{mā} \textsanskrit{maññī}} (“It’s good that you answered each question. But don’t get conceited because of that.”) } \\
Practice as if you were licking a razor’s edge. \\
With tongue pressed to the roof of your mouth, \\
be restrained regarding your stomach. 

Don’t\marginnote{39.1} be sluggish in mind, \\
nor think overly much. \\
Be free of putrefaction and unattached, \\
committed to the spiritual life. 

Train\marginnote{40.1} in a lonely seat, \\
attending closely to ascetics; \\
solitude is sagacity, they say. \\
If you welcome solitude, \\
you’ll light up the ten directions. 

Having\marginnote{41.1} heard the words of the wise,\footnote{The commentary explains \textit{majjhe} “in the middle” firstly as between top and bottom layers, or alternatively as in-between the mountains (i.e. land masses). While it’s possible that there was a folk belief that the center of the ocean was waveless, surely a striking feature of the ocean is how the surface movement conceals a stillness beneath. } \\
the meditators who’ve given up sensual desires, \\
a follower of mine would develop \\
conscience and faith all the more. 

Understand\marginnote{42.1} this by the way streams move \\
in clefts and crevices: \\
the little creeks flow on babbling, \\
while silent flow the great rivers. 

What\marginnote{43.1} is lacking, babbles; \\
what is full is at peace. \\
The fool is like a half-full pot; \\
the wise like a brimfull lake. 

When\marginnote{44.1} the Ascetic speaks much \\
it is relevant and meaningful: \\
knowing, he teaches the Dhamma; \\
knowing, he speaks much. 

But\marginnote{45.1} one who, knowing, is restrained, \\
knowing, does not speak much; \\
that sage is worthy of sagacity, \\
that sage has achieved sagacity.” 

%
\end{verse}

%
\section*{{\suttatitleacronym Snp 3.12}{\suttatitletranslation 3.12 Contemplating Pairs }{\suttatitleroot Dvayatānupassanāsutta}}
\addcontentsline{toc}{section}{\tocacronym{Snp 3.12} \toctranslation{3.12 Contemplating Pairs } \tocroot{Dvayatānupassanāsutta}}
\markboth{3.12 Contemplating Pairs }{Dvayatānupassanāsutta}
\extramarks{Snp 3.12}{Snp 3.12}

\scevam{So\marginnote{1.1} I have heard. }At one time the Buddha was staying near \textsanskrit{Sāvatthī} in the Eastern Monastery, the stilt longhouse of \textsanskrit{Migāra}’s mother. Now, at that time it was the sabbath—the full moon on the fifteenth day—and the Buddha was sitting in the open surrounded by the \textsanskrit{Saṅgha} of monks. Then the Buddha looked around the \textsanskrit{Saṅgha} of monks, who were so very silent. He addressed them: 

“Suppose,\marginnote{2.1} mendicants, they questioned you thus: ‘There are skillful teachings that are noble, emancipating, and lead to awakening. What is the real reason for listening to such teachings?’ You should answer: ‘Only so as to truly know the pairs of teachings.’ And what pairs do they speak of? 

‘This\marginnote{3.1} is suffering; this is the origin of suffering’: this is the first contemplation. ‘This is the cessation of suffering; this is the practice that leads to the cessation of suffering’: this is the second contemplation. When a mendicant meditates rightly contemplating a pair of teachings in this way—diligent, keen, and resolute—they can expect one of two results: enlightenment in the present life, or if there’s something left over, non-return.” 

That\marginnote{4.1} is what the Buddha said. Then the Holy One, the Teacher, went on to say: 

\begin{verse}%
“There\marginnote{5.1} are those who don’t understand suffering \\
and suffering’s cause, \\
and where all suffering \\
ceases with nothing left over. \\
And they don’t know the path \\
that leads to the stilling of suffering. 

They\marginnote{6.1} lack the heart’s release, \\
as well as the release by wisdom. \\
Unable to make an end, \\
they continue to be reborn and grow old. 

But\marginnote{7.1} there are those who understand suffering \\
and suffering’s cause, \\
and where all suffering \\
ceases with nothing left over. \\
And they understand the path \\
that leads to the stilling of suffering. 

They’re\marginnote{8.1} endowed with the heart’s release, \\
as well as the release by wisdom. \\
Able to make an end, \\
they don’t continue to be reborn and grow old.” 

%
\end{verse}

“Suppose,\marginnote{9.1} mendicants, they questioned you thus: ‘Could there be another way to contemplate the pairs?’ You should say, ‘There could.’ And how could there be? ‘All the suffering that originates is caused by attachment’: this is one contemplation. ‘With the utter cessation of attachment there is no origination of suffering’: this is the second contemplation. When a mendicant meditates in this way they can expect enlightenment or non-return.” Then the Teacher went on to say: 

\begin{verse}%
“Attachment\marginnote{10.1} is the source of suffering \\
in all its countless forms in the world. \\
When an ignorant person builds up attachments, \\
that idiot returns to suffering again and again. \\
So let one who understands not build up attachments, \\
contemplating the birth and origin of suffering.” 

%
\end{verse}

“Suppose,\marginnote{11.1} mendicants, they questioned you thus: ‘Could there be another way to contemplate the pairs?’ You should say, ‘There could.’ And how could there be? ‘All the suffering that originates is caused by ignorance’: this is one contemplation. ‘With the utter cessation of ignorance there is no origination of suffering’: this is the second contemplation. When a mendicant meditates in this way they can expect enlightenment or non-return.” Then the Teacher went on to say: 

\begin{verse}%
“Those\marginnote{12.1} who journey again and again, \\
transmigrating through birth and death; \\
they go from this state to another, \\
destined only for ignorance. 

For\marginnote{13.1} ignorance is the great delusion \\
because of which we have long transmigrated. \\
Those beings who have arrived at knowledge \\
do not proceed to a future life.” 

%
\end{verse}

“‘Could\marginnote{14.1} there be another way?’ … And how could there be? ‘All the suffering that originates is caused by choices’: this is one contemplation. ‘With the utter cessation of choices there is no origination of suffering’: this is the second contemplation. When a mendicant meditates in this way they can expect enlightenment or non-return.” Then the Teacher went on to say: 

\begin{verse}%
“All\marginnote{15.1} the suffering that originates \\
is caused by choices. \\
With the cessation of choices, \\
there is no origination of suffering. 

Knowing\marginnote{16.1} this danger, \\
that suffering is caused by choices; \\
through the stilling of all choices, \\
and the stopping of perceptions, \\
this is the way suffering ends. \\
For those who truly know this, 

rightly\marginnote{17.1} seeing, knowledge masters, \\
the astute, understanding rightly, \\
having overcome \textsanskrit{Māra}’s bonds, \\
do not proceed to a future life.” 

%
\end{verse}

“‘Could\marginnote{18.1} there be another way?’ … And how could there be? ‘All the suffering that originates is caused by consciousness’: this is one contemplation. ‘With the utter cessation of consciousness there is no origination of suffering’: this is the second contemplation. When a mendicant meditates in this way they can expect enlightenment or non-return.” Then the Teacher went on to say: 

\begin{verse}%
“All\marginnote{19.1} the suffering that originates \\
is caused by consciousness. \\
With the cessation of consciousness, \\
there is no origination of suffering. 

Knowing\marginnote{20.1} this danger, \\
that suffering is caused by consciousness, \\
with the stilling of consciousness a mendicant \\
is hungerless, extinguished.” 

%
\end{verse}

“‘Could\marginnote{21.1} there be another way?’ … And how could there be? ‘All the suffering that originates is caused by contact’: this is one contemplation. ‘With the utter cessation of contact there is no origination of suffering’: this is the second contemplation. When a mendicant meditates in this way they can expect enlightenment or non-return.” Then the Teacher went on to say: 

\begin{verse}%
“Those\marginnote{22.1} mired in contact, \\
swept down the stream of rebirths, \\
practicing the wrong way, \\
are far from the ending of fetters. 

But\marginnote{23.1} those who completely understand contact, \\
who, understanding, delight in peace; \\
by comprehending contact \\
they are hungerless, extinguished.” 

%
\end{verse}

“‘Could\marginnote{24.1} there be another way?’ … And how could there be? ‘All the suffering that originates is caused by feeling’: this is one contemplation. ‘With the utter cessation of feeling there is no origination of suffering’: this is the second contemplation. When a mendicant meditates in this way they can expect enlightenment or non-return.” Then the Teacher went on to say: 

\begin{verse}%
“Having\marginnote{25.1} known everything that is felt—\\
whether pleasure or pain, \\
as well as what’s neutral, \\
both internally and externally—

as\marginnote{26.1} suffering, \\
deceptive, falling apart; \\
they see it vanish with every touch: \\
that’s how they understand it.\footnote{This is the only mention in the Pali EBTs of the texts later known as the Arthavaveda. Niddesa explains it as casting harmful spells. } \\
With the ending of feelings, a mendicant \\
is hungerless, extinguished.” 

%
\end{verse}

“‘Could\marginnote{27.1} there be another way?’ … And how could there be? ‘All the suffering that originates is caused by craving’: this is one contemplation. ‘With the utter cessation of craving there is no origination of suffering’: this is the second contemplation. When a mendicant meditates in this way they can expect enlightenment or non-return.” Then the Teacher went on to say: 

\begin{verse}%
“Craving\marginnote{28.1} is a person’s partner \\
as they transmigrate on this long journey. \\
They go from this state to another, \\
but don’t escape transmigration. 

Knowing\marginnote{29.1} this danger, \\
that craving is the cause of suffering—\\
rid of craving, free of grasping, \\
a mendicant would wander mindful.” 

%
\end{verse}

“‘Could\marginnote{30.1} there be another way?’ … And how could there be? ‘All the suffering that originates is caused by grasping’: this is one contemplation. ‘With the utter cessation of grasping there is no origination of suffering’: this is the second contemplation. When a mendicant meditates in this way they can expect enlightenment or non-return.” Then the Teacher went on to say: 

\begin{verse}%
“Grasping\marginnote{31.1} is the cause of continued existence; \\
one who exists falls into suffering. \\
Death comes to those who are born—\\
this is the origination of suffering. 

That’s\marginnote{32.1} why with the end of grasping, \\
the astute, understanding rightly, \\
having directly known the end of rebirth, \\
do not proceed to a future life.” 

%
\end{verse}

“‘Could\marginnote{33.1} there be another way?’ … And how could there be? ‘All the suffering that originates is caused by instigating karma’: this is one contemplation.\footnote{PTS Dictionary suggests this should be read as the better-established \textit{\textsanskrit{viruddhagabbhakaraṇaṁ}} (DN 1:1.26.2). If it is an error in the reading, it predates the Niddesa. Nonetheless, I translate \textit{\textsanskrit{gabbhakaraṇa}} in a way that encompasses causing both pregnancy and abortion. } ‘With the utter cessation of instigation there is no origination of suffering’: this is the second contemplation. When a mendicant meditates in this way they can expect enlightenment or non-return.” Then the Teacher went on to say: 

\begin{verse}%
“All\marginnote{34.1} the suffering that originates \\
is caused by instigating karma. \\
With the cessation of instigation, \\
there is no origination of suffering. 

Knowing\marginnote{35.1} this danger, \\
that suffering is caused by instigating karma, \\
having given up all instigation, \\
one is freed with respects to instigation. 

For\marginnote{36.1} the mendicant with peaceful mind, \\
who has cut off craving for continued existence, \\
transmigration through births is finished; \\
there are no future lives for them.” 

%
\end{verse}

“‘Could\marginnote{37.1} there be another way?’ … And how could there be? ‘All the suffering that originates is caused by sustenance’: this is one contemplation. ‘With the utter cessation of sustenance there is no origination of suffering’: this is the second contemplation. When a mendicant meditates in this way they can expect enlightenment or non-return.” Then the Teacher went on to say: 

\begin{verse}%
“All\marginnote{38.1} the suffering that originates \\
is caused by sustenance. \\
With the cessation of sustenance, \\
there is no origination of suffering. 

Knowing\marginnote{39.1} this danger, \\
that suffering is caused by sustenance, \\
completely understanding all sustenance, \\
one is independent of all sustenance. 

Having\marginnote{40.1} rightly understood the state of health, \\
through the ending of defilements, \\
using after reflection, firm in principle,\footnote{The MS reading seems unlikely to me, these are not usually paired. More likely it’s an example of the \textsanskrit{Aṭṭhakavagga} idiom where \textit{puthu} refers to all the many different things said by ascetics. } \\
a knowledge master cannot be reckoned.” 

%
\end{verse}

“‘Could\marginnote{41.1} there be another way?’ … And how could there be? ‘All the suffering that originates is caused by perturbation’: this is one contemplation. ‘With the utter cessation of perturbation there is no origination of suffering’: this is the second contemplation. When a mendicant meditates in this way they can expect enlightenment or non-return.” Then the Teacher went on to say: 

\begin{verse}%
“All\marginnote{42.1} the suffering that originates \\
is caused by perturbation. \\
With the cessation of perturbation, \\
there is no origination of suffering. 

Knowing\marginnote{43.1} this danger, \\
that suffering is caused by perturbation, \\
that’s why, having relinquished perturbation,\footnote{\textit{Osita} (and \textit{\textsanskrit{osāna}} in the next line) has the root sense “lay to rest”. Originally it probably meant the place that one laid down one’s burdens at the end of the day (cf. \textit{khema}). From there we see two main applications, to “end” (cf. \textit{\textsanskrit{pariyosāna}}), or to “reside”. The fact that it follows right after the mention of finding “home” in the “world” in an unafflicted “quarter” shows that the sense “reside” applies. Nonetheless, both Norman and Bodhi translate the two occurrences as “unoccupied” and “at the end”, obscuring the fact that they are the same word in different form. Meanwhile \textsanskrit{Ñāṇadīpa} has “obstruct”. In this case the Niddesa’s explanation should be taken as a creative extension of the text. } \\
and stopped making karmic choices, \\
imperturbable, free of grasping, \\
a mendicant would wander mindful.” 

%
\end{verse}

“‘Could\marginnote{44.1} there be another way?’ … And how could there be? ‘For the dependent there is agitation’: this is the first contemplation. ‘For the independent there’s no agitation’: this is the second contemplation. When a mendicant meditates in this way they can expect enlightenment or non-return.” Then the Teacher went on to say: 

\begin{verse}%
“For\marginnote{45.1} the independent there’s no agitation. \\
The dependent, grasping, \\
goes from this state to another, \\
without escaping transmigration. 

Knowing\marginnote{46.1} this danger, \\
the great fear in dependencies, \\
independent, free of grasping, \\
a mendicant would wander mindful.” 

%
\end{verse}

“‘Could\marginnote{47.1} there be another way?’ … And how could there be? ‘Formless states are more peaceful than states of form’: this is the first contemplation. ‘Cessation is more peaceful than formless states’: this is the second contemplation. When a mendicant meditates in this way they can expect enlightenment or non-return.” Then the Teacher went on to say: 

\begin{verse}%
“There\marginnote{48.1} are beings in the realm of luminous form, \\
and others stuck in the formless. \\
Not understanding cessation, \\
they return in future lives. 

But\marginnote{49.1} the people who completely understand form, \\
not stuck in the formless, \\
released in cessation—\\
they are conquerors of death.” 

%
\end{verse}

“‘Could\marginnote{50.1} there be another way?’ … And how could there be? ‘What this world—with its gods, \textsanskrit{Māras}, and \textsanskrit{Brahmās}, this population with its ascetics and brahmins, its gods and humans—focuses on as true, the noble ones have clearly seen with right wisdom to be actually false’: this is the first contemplation.\footnote{Here, \textit{tveva} has an adversative sense, “but even then …”. This verse contrasts with the previous, a contrast lost in the translations by Norman, Bodhi, \textsanskrit{Ñāṇadīpa}, Thanissaro, Olendzki, Ireland, Mills, and Hare. Nor is the sense “settlement” for \textit{\textsanskrit{osāna}} found in DoP, CPD, or PTS Dictionary. So far as I know, the only source with the correct reading of these terms is the Digital \textsanskrit{Pāli} Dictionary. } ‘What this world focuses on as false, the noble ones have clearly seen with right wisdom to be actually true’: this is the second contemplation. When a mendicant meditates in this way they can expect enlightenment or non-return.” Then the Teacher went on to say: 

\begin{verse}%
“See\marginnote{51.1} how the world with its gods \\
imagines not-self to be self; \\
habituated to name and form, \\
imagining this is truth. 

For\marginnote{52.1} whatever you imagine it is, \\
it turns out to be something else. \\
And that is what is false in it, \\
for the ephemeral is deceptive by nature. 

Extinguishment\marginnote{53.1} has an undeceptive nature, \\
the noble ones know it as truth. \\
Having comprehended the truth, \\
they are hungerless, extinguished.” 

%
\end{verse}

“Suppose,\marginnote{54.1} mendicants, they questioned you thus: ‘Could there be another way to contemplate the pairs?’ You should say, ‘There could.’ And how could there be? ‘What this world—with its gods, \textsanskrit{Māras}, and \textsanskrit{Brahmās}, this population with its ascetics and brahmins, its gods and humans—focuses on as happiness, the noble ones have clearly seen with right wisdom to be actually suffering’: this is the first contemplation. ‘What this world focuses on as suffering, the noble ones have clearly seen with right wisdom to be actually happiness’: this is the second contemplation. When a mendicant meditates rightly contemplating a pair of teachings in this way—diligent, keen, and resolute—they can expect one of two results: enlightenment in the present life, or if there’s something left over, non-return. That is what the Buddha said. Then the Holy One, the Teacher, went on to say: 

\begin{verse}%
“Sights,\marginnote{55.1} sounds, tastes, smells, \\
touches, and thoughts, the lot of them—\\
they’re likable, desirable, and pleasurable \\
as long as you can say that they exist. 

For\marginnote{56.1} all the world with its gods, \\
this is what they agree is happiness. \\
And where they cease \\
is agreed on as suffering for them. 

The\marginnote{57.1} noble ones have seen as happiness \\
the ceasing of identity. \\
This insight by those who see \\
contradicts the whole world. 

What\marginnote{58.1} others say is happiness \\
the noble ones say is suffering. \\
What others say is suffering \\
the noble ones know as happiness. 

See,\marginnote{59.1} this teaching is hard to understand, \\
it confuses the ignorant. \\
There is darkness for the shrouded; \\
blackness for those who don’t see. 

But\marginnote{60.1} the good are open; \\
like light for those who see. \\
Though close, they do not understand, \\
those fools inexpert in the teaching.\footnote{Agreeing with Norman that this line is likely a reciter’s remark. However I take \textit{tattha} as the locative of reference, else it would be \textit{ettha}. The practical guidelines that follow show how to extract the dart. } 

They’re\marginnote{61.1} mired in desire to be reborn, \\
flowing along the stream of lives, \\
mired in \textsanskrit{Māra}’s sway: \\
this teaching isn’t easy for them to understand. 

Who,\marginnote{62.1} apart from the noble ones, \\
is qualified to understand this state? \\
Having rightly understood this state, \\
the undefiled become fully extinguished.” 

%
\end{verse}

That\marginnote{63.1} is what the Buddha said. Satisfied, the mendicants were happy with what the Buddha said. And while this discourse was being spoken, the minds of sixty mendicants were freed from defilements by not grasping. 

%
\addtocontents{toc}{\let\protect\contentsline\protect\nopagecontentsline}
\chapter*{The Chapter of Eights}
\addcontentsline{toc}{chapter}{\tocchapterline{The Chapter of Eights}}
\addtocontents{toc}{\let\protect\contentsline\protect\oldcontentsline}

%
\section*{{\suttatitleacronym Snp 4.1}{\suttatitletranslation Sensual Pleasures }{\suttatitleroot Kāmasutta}}
\addcontentsline{toc}{section}{\tocacronym{Snp 4.1} \toctranslation{Sensual Pleasures } \tocroot{Kāmasutta}}
\markboth{Sensual Pleasures }{Kāmasutta}
\extramarks{Snp 4.1}{Snp 4.1}

\begin{verse}%
If\marginnote{1.1} a mortal desires sensual pleasure \\
and their desire succeeds, \\
they definitely become elated, \\
having got what they want. 

But\marginnote{2.1} for that person in the throes of pleasure, \\
aroused by desire, \\
if those pleasures fade, \\
it hurts like an arrow’s strike. 

One\marginnote{3.1} who, being mindful, \\
avoids sensual pleasures \\
like side-stepping a snake’s head, \\
transcends attachment to the world. 

There\marginnote{4.1} are many objects of sensual desire: \\
fields, lands, and gold; cattle and horses; \\
slaves and servants; women and relatives. \\
When a man lusts over these, 

the\marginnote{5.1} weak overpower him \\
and adversities crush him. \\
Suffering follows him\footnote{Usually \textit{nigghosa} means “message, noise”, but here it is glossed in Niddesa with the opposite meaning, \textit{appasadde appanigghose}. } \\
like water in a leaky boat. 

That’s\marginnote{6.1} why a person, ever mindful, \\
should avoid sensual pleasures. \\
Give them up and cross the flood, \\
as a bailed-out boat reaches the far shore. 

%
\end{verse}

%
\section*{{\suttatitleacronym Snp 4.2}{\suttatitletranslation Eight on the Cave }{\suttatitleroot Guhaṭṭhakasutta}}
\addcontentsline{toc}{section}{\tocacronym{Snp 4.2} \toctranslation{Eight on the Cave } \tocroot{Guhaṭṭhakasutta}}
\markboth{Eight on the Cave }{Guhaṭṭhakasutta}
\extramarks{Snp 4.2}{Snp 4.2}

\begin{verse}%
Trapped\marginnote{1.1} in a cave, thickly overspread, \\
sunk in delusion they stay. \\
A person like this is far from seclusion, \\
for sensual pleasures in the world are not easy to give up. 

The\marginnote{2.1} chains of desire, the bonds of life’s pleasures \\
are hard to escape, for one cannot free another.\footnote{\textit{\textsanskrit{Pariyantacārī}} is read by Niddesa, and followed by Bodhi, Norman, and \textsanskrit{Ñāṇadīpa}, as “of bounded conduct”. However at DN 25:5.5 it means “living on the periphery”, which agrees with the current theme of the mendicant living remotely. } \\
Looking to the past or the future, \\
they pray for these pleasures or former ones.\footnote{\textit{\textsanskrit{Bahujāgar}’assa}. } 

Greedy,\marginnote{3.1} fixated, infatuated by sensual pleasures, \\
they are incorrigible, habitually immoral. \\
When led to suffering they lament, \\
“What will become of us when we pass away from here?” 

That’s\marginnote{4.1} why a person should train in this life: \\
should you know that anything in the world is wrong, \\
don’t act wrongly on account of that; \\
for the wise say this life is short. 

I\marginnote{5.1} see the world’s population floundering, \\
given to craving for future lives. \\
Base men wail in the jaws of death, \\
not rid of craving for life after life. 

See\marginnote{6.1} them flounder over belongings, \\
like fish in puddles of a dried-up stream. \\
Seeing this, live unselfishly, \\
forming no attachment to future lives. 

Rid\marginnote{7.1} of desire for both ends,\footnote{\textit{Vineyya} here is absolutive, not optative. } \\
having completely understood contact, free of greed, \\
doing nothing for which they’d blame themselves, \\
the wise don’t cling to the seen and the heard. 

Having\marginnote{8.1} completely understood perception and having crossed the flood,\footnote{I think the force of \textit{\textsanskrit{kālena}} here is not, “at the right time” (when is the wrong time?) but “before it’s too late”. } \\
the sage, not clinging to possessions, \\
with dart plucked out, living diligently, \\
does not long for this world or the next. 

%
\end{verse}

%
\section*{{\suttatitleacronym Snp 4.3}{\suttatitletranslation Eight on Malice }{\suttatitleroot Duṭṭhaṭṭhakasutta}}
\addcontentsline{toc}{section}{\tocacronym{Snp 4.3} \toctranslation{Eight on Malice } \tocroot{Duṭṭhaṭṭhakasutta}}
\markboth{Eight on Malice }{Duṭṭhaṭṭhakasutta}
\extramarks{Snp 4.3}{Snp 4.3}

\begin{verse}%
Some\marginnote{1.1} speak with malicious intent,\footnote{\textit{\textsanskrit{Ākiñcaññaṁ}} could mean either “owning nothing” (per commentary, followed by Norman) or the “dimension of nothingness” (suggested by Bodhi). Given that just a little below we hear of him performing an expensive sacrifice; and further, that the dimension of nothingness is an important part of the conversation with of his students \textsanskrit{Upasīva} and \textsanskrit{Posāla}, the latter seems more likely. Anyway, “aspiring to own nothing” is hardly a noteworthy trait among ascetics. } \\
while others speak set on truth. \\
When disputes come up a sage does not get involved, \\
which is why they’ve no barrenness at all. 

How\marginnote{2.1} can you transcend your own view \\
when you’re led by preference, dogmatic in belief?\footnote{Her gender is indicated below. } \\
Inventing your own undertakings,\footnote{Note that, in the EBTs, \textit{\textsanskrit{pathavimaṇḍala}} primarily refers to the region ruled by a Wheel-Turning Monarch. Thus it doesn’t mean “circle” in the sense of “the circle of the earth” but rather a “sphere of influence”, i.e. “territory”. } \\
you’d speak according to your notion.\footnote{\textit{\textsanskrit{Purā}} can be either “from the city” (Norman), or “formerly” (commentary, followed by Bodhi and Jayawickrama). I think the poet is deliberately echoing the opening line of the text, where it must mean “from the city”. } 

Some,\marginnote{3.1} unasked, tell others \\
of their own precepts and vows. \\
They have an ignoble nature, say the experts, \\
since they speak about themselves of their own accord. 

A\marginnote{4.1} mendicant, peaceful, quenched, \\
never boasts “thus am I” of their precepts. \\
They have a noble nature, say the experts, \\
not proud of anything in the world. 

For\marginnote{5.1} one who formulates and creates teachings, \\
and promotes them despite their defects, \\
if they see an advantage for themselves, \\
they become dependent on that, relying on unstable peace. 

It’s\marginnote{6.1} not easy to get over dogmatic views \\
adopted after judging among the teachings. \\
That’s why, among all these dogmas, a person \\
rejects one teaching and takes up another. 

The\marginnote{7.1} cleansed one has no formulated view \\
at all in the world about the different realms. \\
Having given up illusion and conceit, \\
by what path would they go? They are not involved. 

For\marginnote{8.1} one who is involved gets embroiled in disputes about teachings—\\
but how to dispute with the uninvolved? About what? \\
For picking up and putting down is not what they do;\footnote{\textit{Mandira} is unusual and probably a sign of lateness. Bodhi has “realm”, Norman “city”, but the normal meaning in Sanskrit is a “dwelling place”, and in the \textsanskrit{Jātakas} it is always used in the sense of a home. } \\
they have shaken off all views in this very life. 

%
\end{verse}

%
\section*{{\suttatitleacronym Snp 4.4}{\suttatitletranslation Eight on the Pure }{\suttatitleroot Suddhaṭṭhakasutta}}
\addcontentsline{toc}{section}{\tocacronym{Snp 4.4} \toctranslation{Eight on the Pure } \tocroot{Suddhaṭṭhakasutta}}
\markboth{Eight on the Pure }{Suddhaṭṭhakasutta}
\extramarks{Snp 4.4}{Snp 4.4}

\begin{verse}%
“I\marginnote{1.1} see a pure being of ultimate wellness;\footnote{The commentary gives the senses \textit{vigatadhuro} “free of burden” (followed by Norman) and \textit{\textsanskrit{appaṭimo}} “unrivalled” (followed by Bodhi), but at AN 3.20:2.1 it has the sense “indefatigable” and there seems no reason why it shouldn’t have the same meaning here. } \\
it is vision that grants a person purity.” \\
Recalling this notion of the ultimate,\footnote{Bodhi omits \textit{\textsanskrit{dhīrā}}. } \\
they believe in the notion that there is one who observes purity.\footnote{One of the many signs of lateness in this passage. } 

If\marginnote{2.1} a person were granted purity through vision, \\
or if by a notion they could give up suffering, \\
then one with attachments is purified by another: \\
their view betrays them as one who asserts thus. 

The\marginnote{3.1} brahmin speaks not of purity from another \\
in terms of what is seen, heard, or thought; or by precepts or vows. \\
They are unsullied in the midst of good and evil, \\
letting go what was picked up, without creating anything new here. 

Having\marginnote{4.1} let go the last they lay hold of the next; \\
following impulse, they don’t get past the snare.\footnote{Bodhi accepts the reading “former” but, given that there seems no evidence that the name was abandoned at this time, this seems unlikely. } \\
They grab on and let go like a monkey \\
grabbing and releasing a branch.\footnote{Bodhi and Norman have Setavya, but DN 23 shows it is feminine. } 

Having\marginnote{5.1} undertaken their own vows, a person \\
visits various teachers, being attached to perception.\footnote{See my remarks on \textit{mandira} above. \textsanskrit{Kusinārā} was famously \emph{not} a city, contra Bodhi and Norman. } \\
One who knows, having comprehended the truth through the knowledges,\footnote{Taking \textit{nivesana} here as “dogma”, the same sense as in the \textsanskrit{Aṭṭhakavagga}. Otherwise it might just mean “strong attachment”. } \\
does not visit various teachers, being of vast wisdom. 

They\marginnote{6.1} are remote from all things \\
seen, heard, or thought. \\
Seeing them living openly, \\
how could anyone in this world judge them? 

They\marginnote{7.1} don’t make things up or promote them, \\
or speak of the uttermost purity. \\
After untying the tight knot of grasping \\
they long for nothing in the world. 

The\marginnote{8.1} brahmin has stepped over the boundary; \\
knowing and seeing, they adopt nothing.\footnote{This is an abbreviated form of the phrase \textit{natthi \textsanskrit{kiñcī}’ti} that normally marks the dimension of nothingness. } \\
Neither in love with passion nor besotted by dispassion, \\
there is nothing here they adopt as the ultimate. 

%
\end{verse}

%
\section*{{\suttatitleacronym Snp 4.5}{\suttatitletranslation Eight on the Ultimate }{\suttatitleroot Paramaṭṭhakasutta}}
\addcontentsline{toc}{section}{\tocacronym{Snp 4.5} \toctranslation{Eight on the Ultimate } \tocroot{Paramaṭṭhakasutta}}
\markboth{Eight on the Ultimate }{Paramaṭṭhakasutta}
\extramarks{Snp 4.5}{Snp 4.5}

\begin{verse}%
If,\marginnote{1.1} maintaining that theirs is the “ultimate” view, \\
a person makes it out to be highest in the world;\footnote{Niddesa explains \textit{\textsanskrit{kathāhi}} as either “doubts” (hence a truncated form of \textit{\textsanskrit{kathaṅkathā}}), or “talks”; the former is followed by Bodhi, the latter by Norman and \textsanskrit{Ñāṇadīpa}. While “doubts” seems contextually more likely, elsewhere \textit{(\textsanskrit{paṭi})virato} is commonly used with speech and never, to my knowledge, with doubt. } \\
then they declare all others are “lesser”; \\
that’s why they’re not over disputes. 

If\marginnote{2.1} they see an advantage for themselves \\
in what’s seen, heard, or thought; or in precepts or vows, \\
in that case, having adopted that one alone, \\
they see all others as inferior. 

Experts\marginnote{3.1} say that, too, is a knot, \\
relying on which people see others as lesser. \\
That’s why a mendicant ought not rely \\
on what’s seen, heard, or thought, or on precepts and vows. 

Nor\marginnote{4.1} would they form a view about the world \\
through a notion or through precepts and vows. \\
They would never represent themselves as “equal”, \\
nor conceive themselves “worse” or “better”. 

What\marginnote{5.1} was picked up has been set down and is not grasped again; \\
they form no dependency even on notions. \\
They follow no side among the factions, \\
and believe in no view at all. 

One\marginnote{6.1} here who has no wish for either end—\\
for any form of existence in this life or the next—\\
has adopted no dogma at all \\
after judging among the teachings. 

For\marginnote{7.1} them not even the tiniest idea is formulated here \\
regarding what is seen, heard, or thought. \\
That brahmin does not grasp any view—\\
how could anyone in this world judge them? 

They\marginnote{8.1} don’t make things up or promote them, \\
and don’t subscribe to any of the doctrines. \\
The brahmin has no need to be led by precept or vow; \\
gone to the far shore, one such does not return. 

%
\end{verse}

%
\section*{{\suttatitleacronym Snp 4.6}{\suttatitletranslation Old Age }{\suttatitleroot Jarāsutta}}
\addcontentsline{toc}{section}{\tocacronym{Snp 4.6} \toctranslation{Old Age } \tocroot{Jarāsutta}}
\markboth{Old Age }{Jarāsutta}
\extramarks{Snp 4.6}{Snp 4.6}

\begin{verse}%
Short,\marginnote{1.1} alas, is this life; \\
you die before a hundred years. \\
Even if you live a little longer, \\
you still die of old age. 

People\marginnote{2.1} grieve over belongings, \\
yet there is no such thing as permanent possessions. \\
Separation is a fact of life; when you see this, \\
you wouldn’t stay living at home. 

Whatever\marginnote{3.1} a person thinks of as belonging to them, \\
that too is given up when they die. \\
Knowing this, an astute follower of mine \\
would not be bent on ownership.\footnote{Compare the stock idiom \textit{\textsanskrit{Saṅkhampi} na upeti upanidhimpi na upeti \textsanskrit{kalabhāgampi} na upeti} at eg. SN 20.2:1.7; also Snp 1.12:3.4. } 

Just\marginnote{4.1} as, upon awakening, a person does not see \\
what they encountered in a dream; \\
so too you do not see your loved ones \\
when they are dead and gone. 

You\marginnote{5.1} used to see and hear those folk, \\
and call them by their name. \\
Yet the name is all that’s left to tell \\
of a person when they’re gone. 

Those\marginnote{6.1} who are greedy for belongings \\
don’t give up sorrow, lamentation, and stinginess. \\
That’s why the sages, seers of sanctuary,\footnote{\textit{\textsanskrit{Nāmakāya}} appears only here and in DN 15, apparently in the same sense. It means neither \textit{\textsanskrit{nāmarūpa}} nor \textit{\textsanskrit{manomayakāya}}, but rather the “group” of “mental phenomena”. In this case it refers to one who is freed based on a formless attainment, for whom the \textit{\textsanskrit{rūpakāya}} has been formerly abandoned. } \\
left possessions behind and wandered. 

For\marginnote{7.1} a mendicant who lives withdrawn, \\
frequenting a secluded seat, \\
they say it’s fitting \\
to not show themselves in a home. 

The\marginnote{8.1} sage is independent everywhere, \\
they don’t form likes or dislikes. \\
Lamentation and stinginess \\
slip off them like water from a leaf. 

Like\marginnote{9.1} a droplet slips from a lotus-leaf, \\
like water from a lotus flower; \\
the sage doesn’t cling to that \\
which is seen or heard or thought. 

For\marginnote{10.1} the one who is cleansed does not conceive \\
in terms of things seen, heard, or thought. \\
They do not wish to be purified by another; \\
they are neither passionate nor growing dispassioned.\footnote{As at Snp 4.9:5.1, read as “instrumental of relation”. Curiously, the extraneous phrase \textit{\textsanskrit{sīlabbatena}} (not in Niddesa) appears where the reciter’s identification of the speaker should be. } 

%
\end{verse}

%
\section*{{\suttatitleacronym Snp 4.7}{\suttatitletranslation With Tissametteyya }{\suttatitleroot Tissametteyyasutta}}
\addcontentsline{toc}{section}{\tocacronym{Snp 4.7} \toctranslation{With Tissametteyya } \tocroot{Tissametteyyasutta}}
\markboth{With Tissametteyya }{Tissametteyyasutta}
\extramarks{Snp 4.7}{Snp 4.7}

\begin{verse}%
“When\marginnote{1.1} someone indulges in sex,” \\
\scspeaker{said Venerable Tissametteyya, }\\
“tell us, sir: what trouble befalls them? \\
After hearing your instruction, \\
we shall train in seclusion.” 

“When\marginnote{2.1} someone indulges in sex,” \\
\scspeaker{replied the Buddha, }\\
“they forget their instructions \\
and go the wrong way—\\
that is something ignoble in them. 

Someone\marginnote{3.1} who formerly lived alone \\
and then resorts to sex \\
is like a chariot careening off-track; \\
in the world they call them a low, ordinary person. 

Their\marginnote{4.1} former fame and reputation \\
also fall away. \\
Seeing this, they’d train \\
to give up sex. 

Oppressed\marginnote{5.1} by thoughts, \\
they brood like a wretch. \\
When they hear what others are saying, \\
such a person is embarrassed. 

Then\marginnote{6.1} they lash out with verbal daggers \\
when reproached by others. \\
This is their great blind spot;\footnote{Bodhi reads \textit{\textsanskrit{kappī}} (against the Niddesa) in the sense “manner”. But it seems unlikely Todeyya, or anyone really, would ask whether a sage merely \emph{seemed} wise. Surely the question must be whether a sage is someone who is still in the process of learning, a question that has been important to the Buddhist traditions and still is today. } \\
they sink to lies. 

They\marginnote{7.1} once were considered astute,\footnote{While the exact sense of the unique term \textit{sahajanetta} is open to interpretation, surely Norman’s “omniscient one” (following Niddesa) is not right. } \\
committed to the solitary life. \\
But then they indulged in sex, \\
dragged along by desire like an idiot.\footnote{The following line lacks a verb, and Norman, Bodhi, and \textsanskrit{Ñāṇadīpa} all construe it with \textit{abhibhuyya}: the sun “overcomes” the earth. But that doesn’t make sense. Rather, the idiom \textit{abhibhuyya iriyati} leans on \textit{irayati} in the sense “to keep going, to proceed”. The Buddha “proceeds” (i.e. keeps living) after mastering desires, like the sun “proceeds” shedding light on the earth. } 

Knowing\marginnote{8.1} this danger \\
in falling from a former state here,\footnote{Niddesa, followed by Bodhi, Norman, and \textsanskrit{Ñāṇadīpa}, takes \textit{vinaya} as optative. But where the same phrase occurs at Kp 9:10.3 and Snp 1.8:10.3 it is clearly absolutive. Moreover, the next includes the absolutive \textit{\textsanskrit{daṭṭhu}}, which elsewhere follows a similar line with an absolutive verb: \textit{\textsanskrit{kāmesvādīnavaṁ} \textsanskrit{disvā}}. } \\
a sage would firmly resolve to wander alone, \\
and would not resort to sex. 

They’d\marginnote{9.1} train themselves only in seclusion; \\
this, for the noble ones, is highest. \\
One who wouldn’t conceive themselves “best” due to that—\\
they have truly drawn near to extinguishment. 

People\marginnote{10.1} tied to sensual pleasures envy them: \\
the isolated, wandering sage \\
who has crossed the flood, \\
unconcerned for sensual pleasures. 

%
\end{verse}

%
\section*{{\suttatitleacronym Snp 4.8}{\suttatitletranslation With Pasūra }{\suttatitleroot Pasūrasutta}}
\addcontentsline{toc}{section}{\tocacronym{Snp 4.8} \toctranslation{With Pasūra } \tocroot{Pasūrasutta}}
\markboth{With Pasūra }{Pasūrasutta}
\extramarks{Snp 4.8}{Snp 4.8}

\begin{verse}%
“Here\marginnote{1.1} alone is purity,” they say, \\
denying that there is purification in other teachings. \\
Speaking of the beauty in that which they depend on,\footnote{As in the \textsanskrit{Aṭṭhakavagga}, it is essential that this phrase be rendered in an active present tense. It is not that nothing \emph{has ever} been taken up or put down, but that they are no longer actively engaging in taking up and putting down. } \\
each one is dogmatic about their own idiosyncratic interpretation.\footnote{Norman’s suggestion to read \textit{\textsanskrit{ādānasatte}} (against Niddesa) as locative singular would be tempting were it not that at Snp 5.13:4.3 and Thag 19.1:20.3 \textit{iti \textsanskrit{pekkhamāno}} qualifies the former part of the line. } 

Desiring\marginnote{2.1} debate, they plunge into an assembly, \\
where each takes the other as a fool.\footnote{This is an unusual idiom. } \\
Relying on others they state their contention, \\
desiring praise while claiming to be experts. 

Addicted\marginnote{3.1} to debating in the midst of the assembly, \\
their need for praise makes them nervous. \\
But when they’re repudiated they get embarrassed; \\
upset at criticism, they find fault in others. 

If\marginnote{4.1} their doctrine is said to be weak, \\
and judges declare it repudiated, \\
the loser weeps and wails, \\
moaning, “They beat me.” 

When\marginnote{5.1} these arguments come up among ascetics, \\
they get excited or dejected. \\
Seeing this, refrain from contention, \\
for the only purpose is praise and profit. 

But\marginnote{6.1} if, having declared their doctrine, \\
they are praised there in the midst of the assembly, \\
they laugh and proudly show off because of it, \\
having got what they wanted. 

Their\marginnote{7.1} pride is their downfall, \\
yet they speak from conceit and arrogance. \\
Seeing this, one ought not get into arguments, \\
for experts say this is no way to purity. 

As\marginnote{8.1} a warrior, after feasting on royal food, \\
goes roaring, wanting an opponent—\\
go off and find an opponent, \textsanskrit{Sūra}, \\
for here, as before, there is no-one to fight. 

When\marginnote{9.1} someone argues about a view they have adopted, \\
saying, “This is the only truth,” \\
say to them, “Here you’ll have no adversary \\
when a dispute has come up.” 

There\marginnote{10.1} are those who live far from the crowd, \\
not countering views with view. \\
Who is there to argue with you, \textsanskrit{Pasūra}, \\
among those who grasp nothing here as the highest? 

And\marginnote{11.1} so you come along speculating, \\
thinking up theories in your mind. \\
Now that you’ve challenged someone who’s cleansed,\footnote{See Snp 4.11. The question refers to a meditator who has attained the dimension of nothingness. } \\
you’ll not be able to respond.\footnote{The Niddesa explains, in line with the Buddha’s answer, that the question is about the kind of knowledge needed to guide one with such an attainment. Thus it must be dative, not genitive. } 

%
\end{verse}

%
\section*{{\suttatitleacronym Snp 4.9}{\suttatitletranslation With Māgaṇḍiya }{\suttatitleroot Māgaṇḍiyasutta}}
\addcontentsline{toc}{section}{\tocacronym{Snp 4.9} \toctranslation{With Māgaṇḍiya } \tocroot{Māgaṇḍiyasutta}}
\markboth{With Māgaṇḍiya }{Māgaṇḍiyasutta}
\extramarks{Snp 4.9}{Snp 4.9}

\begin{verse}%
“Even\marginnote{1.1} when I saw the sirens Craving, Delight, and Lust,\footnote{Read \textit{’\textsanskrit{dhimuttaṁ}} per Niddesa, see discussion in Bodhi. The sense is the same as eg. an4.125. This refers to someone, such as \textsanskrit{Āḷāra} \textsanskrit{Kālāma}, whose spiritual goal is rebirth in the dimension of nothingness. } \\
I had no desire for sex. \\
What is this body full of piss and shit?\footnote{To \textit{\textsanskrit{anugāyati}} is to keep on reciting, as the brahmins did with their hymns descended from seers of old (as eg. AN 5.192:3.1). Here, \textsanskrit{Piṅgiya}, as an experienced Vedic reciter himself, announces that he will keep the this teaching alive in just the same way. This is perhaps the most explicit evidence for a direct historical link between the oral recitation methods of the brahmins and the Buddhists. } \\
I wouldn’t even want to touch it with my foot.” 

“If\marginnote{2.1} you do not want a gem such as this, \\
a lady desired by many rulers of men, \\
then what kind of theory, precepts and vows, livelihood, \\
and rebirth in a new life do you assert?” 

“After\marginnote{3.1} judging among the teachings,” \\
\scspeaker{said the Buddha to \textsanskrit{Māgaṇḍiya}, }\\
“none have been adopted thinking, ‘I assert this.’ \\
Seeing views without adopting any, \\
searching, I saw inner peace.” 

“O\marginnote{4.1} sage, you speak of judgments you have formed,” \\
\scspeaker{said \textsanskrit{Māgaṇḍiya}, }\\
“without having adopted any of those views. \\
As to that matter of ‘inner peace’—\\
how is that described by the wise?” 

“Purity\marginnote{5.1} is neither spoken of in terms of view,” \\
\scspeaker{said the Buddha to \textsanskrit{Māgaṇḍiya}, }\\
“oral transmission, notion, and precepts and vows; \\
nor in terms of that without view, oral transmission, \\
notion, and precepts and vows. \\
Having relinquished these, not adopting them, \\
peaceful, independent, one would not pray to be reborn.” 

“It\marginnote{6.1} seems purity is neither spoken of in terms of view,” \\
\scspeaker{said \textsanskrit{Māgaṇḍiya}, }\\
“oral transmission, notion, and precepts and vows; \\
nor in terms of that without view, oral transmission, \\
notion, and precepts and vows. \\
If so, I think this teaching is sheer confusion; \\
for some believe in purity in terms of view.” 

“Continuing\marginnote{7.1} to question while relying on a view,” \\
\scspeaker{said the Buddha to \textsanskrit{Māgaṇḍiya}, }\\
“you’ve become confused by all you’ve adopted. \\
From this you’ve not glimpsed the slightest idea, \\
which is why you consider the teaching confused. 

If\marginnote{8.1} you think that ‘I’m equal, \\
special, or worse’, you’ll get into arguments. \\
Unwavering in the face of the three discriminations, \\
you’ll have no thought ‘I’m equal or special’. 

Why\marginnote{9.1} would that brahmin say, ‘It’s true’, \\
or with whom would they argue, ‘It’s false’? \\
There is no equal or unequal in them, \\
so who would they take on in debate? 

After\marginnote{10.1} leaving shelter to migrate unsettled, \\
a sage doesn’t get close to anyone in town. \\
Rid of sensual pleasures, expecting nothing, \\
they wouldn’t get in arguments with people. 

A\marginnote{11.1} spiritual giant would not take up for argument \\
the things in the world from which they live secluded. \\
As a prickly lotus born in the water \\
is unsullied by water and mud, \\
so the greedless sage, proponent of peace, \\
is unsmeared by sensuality and the world. 

A\marginnote{12.1} knowledge master does not become conceited \\
due to view or thought, for they do not identify with that. \\
They’ve no need for deeds or learning, \\
they’re not indoctrinated in dogmas. 

There\marginnote{13.1} are no ties for one detached from ideas; \\
there are no delusions for one freed by wisdom. \\
But those who have adopted ideas and views \\
wander the world causing conflict.” 

%
\end{verse}

%
\section*{{\suttatitleacronym Snp 4.10}{\suttatitletranslation Before the Breakup }{\suttatitleroot Purābhedasutta}}
\addcontentsline{toc}{section}{\tocacronym{Snp 4.10} \toctranslation{Before the Breakup } \tocroot{Purābhedasutta}}
\markboth{Before the Breakup }{Purābhedasutta}
\extramarks{Snp 4.10}{Snp 4.10}

\begin{verse}%
“Seeing\marginnote{1.1} how, behaving how, \\
is one said to be at peace? \\
When asked, Gotama, please tell me \\
about the ultimate person.” 

“Rid\marginnote{2.1} of craving before the breakup,” \\
\scspeaker{said the Buddha, }\\
“not dependent on the past, \\
unfathomable in the middle, \\
they are not governed by anything. 

Unangry,\marginnote{3.1} unafraid, \\
not boastful or regretful, \\
thoughtful in counsel, and stable—\\
truly that sage is controlled in speech. 

Rid\marginnote{4.1} of attachment to the future, \\
they don’t grieve for the past. \\
A seer of seclusion in the midst of contacts \\
is not led astray among views. 

Withdrawn,\marginnote{5.1} free of deceit, \\
they’re not envious or stingy, \\
nor rude or disgusting, \\
or given to slander. 

Not\marginnote{6.1} swept up in pleasures, \\
or given to arrogance, \\
they’re gentle and articulate, \\
neither hungering nor growing dispassionate. 

They\marginnote{7.1} don’t train out of desire for profit, \\
nor get annoyed at lack of profit. \\
Not hostile due to craving, \\
nor greedy for flavors, 

they\marginnote{8.1} are equanimous, ever mindful. \\
They never conceive themselves in the world \\
as equal, special, or less than; \\
for them there is no pride. 

They\marginnote{9.1} have no dependencies, \\
understanding the teaching, they are independent. \\
No craving is found in them \\
to continue existence or to end it. 

I\marginnote{10.1} declare them to be at peace, \\
unconcerned for sensual pleasures. \\
No ties are found in them, \\
they have crossed over clinging. 

They\marginnote{11.1} have no sons or cattle, \\
nor possess fields or lands. \\
No picking up or putting down \\
is to be found in them. 

That\marginnote{12.1} by which one might describe \\
an ordinary person or ascetics and brahmins \\
has no importance to them, \\
which is why they’re unaffected by words. 

Freed\marginnote{13.1} of greed, not stingy, \\
a sage doesn’t speak of themselves as being \\
among superiors, inferiors, or equals. \\
One not prone to creation does not return to creation. 

They\marginnote{14.1} who have nothing in the world of their own \\
do not grieve for that which is not, \\
or drift among the teachings; \\
that’s who is said to be at peace.” 

%
\end{verse}

%
\section*{{\suttatitleacronym Snp 4.11}{\suttatitletranslation Quarrels and Disputes }{\suttatitleroot Kalahavivādasutta}}
\addcontentsline{toc}{section}{\tocacronym{Snp 4.11} \toctranslation{Quarrels and Disputes } \tocroot{Kalahavivādasutta}}
\markboth{Quarrels and Disputes }{Kalahavivādasutta}
\extramarks{Snp 4.11}{Snp 4.11}

\begin{verse}%
“Where\marginnote{1.1} do quarrels and disputes come from? \\
And lamentation and sorrow, and stinginess? \\
What of conceit and arrogance, and slander too—\\
tell me please, where do they come from?” 

“Quarrels\marginnote{2.1} and disputes come from what we hold dear, \\
as do lamentation and sorrow, stinginess, \\
conceit and arrogance. \\
Quarrels and disputes are linked to stinginess, \\
and when disputes have arisen there is slander.” 

“So\marginnote{3.1} where do what we hold dear in the world spring from? \\
And the lusts that are loose in the world? \\
Where spring the hopes and aims \\
a man has for the next life?” 

“What\marginnote{4.1} we hold dear in the world spring from desire, \\
as do the lusts that are loose in the world. \\
From there spring the hopes and aims \\
a man has for the next life.” 

“So\marginnote{5.1} where does desire in the world spring from? \\
And judgments, too, where do they come from? \\
And anger, lies, and doubt, \\
and other things spoken of by the Ascetic?” 

“What\marginnote{6.1} they call pleasure and pain in the world—\\
based on that, desire comes about. \\
Seeing the appearance and disappearance of forms, \\
a person forms judgments in the world. 

Anger,\marginnote{7.1} lies, and doubt—\\
these things are, too, when that pair is present. \\
One who has doubts should train in the path of knowledge; \\
it is from knowledge that the Ascetic speaks of these things.” 

“Where\marginnote{8.1} do pleasure and pain spring from? \\
When what is absent do these things not occur? \\
And also, on the topic of appearance and disappearance—\\
tell me where they spring from.” 

“Pleasure\marginnote{9.1} and pain spring from contact; \\
when contact is absent they do not occur. \\
And on the topic of appearance and disappearance—\\
I tell you they spring from there.” 

“So\marginnote{10.1} where does contact in the world spring from? \\
And possessions, too, where do they come from? \\
When what is absent is there no possessiveness? \\
When what disappears do contacts not strike?” 

“Name\marginnote{11.1} and form cause contact; \\
possessions spring from wishing; \\
when wishing is absent there is no possessiveness; \\
when form disappears, contacts don’t strike.” 

“How\marginnote{12.1} to proceed so that form disappears? \\
And how do happiness and suffering disappear? \\
Tell me how they disappear; \\
I think we ought to know these things.” 

“Without\marginnote{13.1} normal perception or distorted perception; \\
not lacking perception, nor perceiving what has disappeared. \\
That’s how to proceed so that form disappears: \\
for concepts of identity due to proliferation spring from perception.” 

“Whatever\marginnote{14.1} I asked you have explained to me. \\
I ask you once more, please tell me this: \\
Do some astute folk here say that this is the extent \\
of purification of the spirit? \\
Or do they say it is something else?” 

“Some\marginnote{15.1} astute folk do say that this is the highest extent \\
of purification of the spirit. \\
But some of them, claiming to be experts, \\
speak of a time when nothing remains. 

Knowing\marginnote{16.1} that these states are dependent, \\
and knowing what they depend on, the inquiring sage, \\
having understood, is freed, and does not dispute. \\
The wise do not go on into life after life.” 

%
\end{verse}

%
\section*{{\suttatitleacronym Snp 4.12}{\suttatitletranslation The Shorter Discourse on Arrayed For Battle }{\suttatitleroot Cūḷabyūhasutta}}
\addcontentsline{toc}{section}{\tocacronym{Snp 4.12} \toctranslation{The Shorter Discourse on Arrayed For Battle } \tocroot{Cūḷabyūhasutta}}
\markboth{The Shorter Discourse on Arrayed For Battle }{Cūḷabyūhasutta}
\extramarks{Snp 4.12}{Snp 4.12}

\begin{verse}%
“Each\marginnote{1.1} maintaining their own view, \\
the experts disagree, arguing: \\
‘Whoever sees it this way has understood the teaching; \\
those who reject this are inadequate.’ 

So\marginnote{2.1} arguing, they quarrel, \\
saying, ‘The other is a fool, an amateur!’ \\
Which one of these speaks true, \\
for they all claim to be an expert?” 

“If\marginnote{3.1} not accepting another’s teaching \\
makes you a useless fool lacking wisdom, \\
then they’re all fools lacking wisdom, \\
for they all maintain their own view. 

But\marginnote{4.1} if having your own view is what makes you pristine—\\
pure in wisdom, expert and intelligent—\\
then none of them lack wisdom, \\
for such is the view they have all embraced. 

I\marginnote{5.1} do not say that it is correct \\
when they call each other fools. \\
Each has built up their own view to be the truth, \\
which is why they take the other as a fool.” 

“What\marginnote{6.1} some say is true and correct, \\
others say is hollow and false. \\
So arguing, they quarrel; \\
why don’t ascetics say the same thing?” 

“The\marginnote{7.1} truth is one, there is no second; \\
wise folk would not argue about this. \\
But those ascetics each boast of different truths; \\
that’s why they don’t say the same thing.” 

“But\marginnote{8.1} why do they speak of different truths, \\
these proponents who claim to be experts? \\
Are there really so many different truths, \\
or do they just follow their own lines of reasoning?” 

“No,\marginnote{9.1} there are not many different truths \\
that, apart from perception, are lasting in the world. \\
Having formed their reasoning regarding different views, \\
they say there are two things: true and false. 

The\marginnote{10.1} seen, heard, or thought, or precepts or vows— \\
based on these they show disdain. \\
Standing in judgment, they scoff, \\
saying, ‘The other is a fool, an amateur!’ 

They\marginnote{11.1} take the other as a fool on the same grounds \\
that they speak of themselves as an expert. \\
Claiming to be an expert on their own authority, \\
they disdain the other while saying the same thing. 

They\marginnote{12.1} are perfect, according to their own extreme view; \\
drunk on conceit, imagining themselves proficient. \\
They have anointed themselves in their own mind, \\
for such is the view they have embraced. 

If\marginnote{13.1} the word of your opponent makes you deficient, \\
then they too are lacking wisdom. \\
But if on your own authority you’re a knowledge master, a wise person, \\
then there are no fools among the ascetics. 

‘Those\marginnote{14.1} who proclaim a teaching other than this \\
have fallen short of purity, and are inadequate’: \\
so say each one of the sectarians, \\
for they are deeply attached to their own view. 

‘Here\marginnote{15.1} alone is purity,’ they say, \\
denying that there is purification in other teachings. \\
Thus each one of the sectarians, being dogmatic, \\
speaks forcefully within the context of their own journey. 

But\marginnote{16.1} in that case, so long as they are speaking forcefully of their own journey, \\
how can they take the other as a fool? \\
They are the ones who provoke conflict \\
when they call the other a fool with an impure teaching. 

Standing\marginnote{17.1} in judgment, measuring by their own standard, \\
they keep getting into disputes with the world. \\
But a person who has given up all judgments \\
creates no conflict in the world.” 

%
\end{verse}

%
\section*{{\suttatitleacronym Snp 4.13}{\suttatitletranslation The Longer Discourse on Arrayed for Battle }{\suttatitleroot Mahābyūhasutta}}
\addcontentsline{toc}{section}{\tocacronym{Snp 4.13} \toctranslation{The Longer Discourse on Arrayed for Battle } \tocroot{Mahābyūhasutta}}
\markboth{The Longer Discourse on Arrayed for Battle }{Mahābyūhasutta}
\extramarks{Snp 4.13}{Snp 4.13}

\begin{verse}%
“Regarding\marginnote{1.1} those who maintain their own view, \\
arguing that, ‘This is the only truth’: \\
are all of them subject only to criticism, \\
or do some also win praise for that?” 

“That\marginnote{2.1} is a small thing, insufficient for peace, \\
these two fruits of conflict, I say. \\
Seeing this, one ought not get into arguments, \\
looking for sanctuary in the land of no conflict. 

One\marginnote{3.1} who knows does not get involved \\
with any of the many different convictions. \\
Why would the uninvolved get involved, \\
since they do not believe based on the seen or the heard? 

Those\marginnote{4.1} who champion ethics speak of purity through self-control; \\
having undertaken a vow, they stick to it: \\
‘Let us train right here, then we will be pure.’ \\
Claiming to be experts, they are led on to future lives. 

If\marginnote{5.1} they fall away from their precepts and vows, \\
they tremble, having failed in their task. \\
They pray and long for purity, \\
like one who has lost their caravan while journeying far from home. 

But\marginnote{6.1} having given up all precepts and vows, \\
and these deeds blameworthy or blameless; \\
not longing for ‘purity’ or ‘impurity’, \\
live detached, fostering peace. 

Relying\marginnote{7.1} on mortification in disgust at sin, \\
or else on what is seen, heard, or thought, \\
they moan that purification comes through heading upstream, \\
not rid of craving for life after life. 

For\marginnote{8.1} one who longs there are prayers, \\
and trembling too over ideas they have formed. \\
But one here for whom there is no passing away or reappearing: \\
why would they tremble? For what would they pray?” 

“The\marginnote{9.1} very same teaching that some say is ‘ultimate’, \\
others say is inferior. \\
Which of these doctrines is true, \\
for they all claim to be an expert?” 

“They\marginnote{10.1} say their own teaching is perfect, \\
while the teaching of others is inferior. \\
So arguing, they quarrel, \\
each saying their own convictions are the truth. 

If\marginnote{11.1} you became inferior because someone else disparaged you, \\
no-one in any teaching would be distinguished. \\
For each of them says the other’s teaching is lacking, \\
while forcefully advocating their own. 

But\marginnote{12.1} if they honor their own teachings \\
just as they praise their own journeys, \\
then all doctrines would be equally valid, \\
and purity for them would be an individual matter. 

After\marginnote{13.1} judging among the teachings, a brahmin has adopted nothing \\
that requires interpretation by another. \\
That’s why they’ve gotten over disputes, \\
for they see no other doctrine as best. 

Saying,\marginnote{14.1} ‘I know, I see, that’s how it is’, \\
some believe that purity comes from view. \\
But if they’ve really seen, what use is that view to them? \\
Overlooking what matters, they say purity comes from another. 

When\marginnote{15.1} a person sees, they see name and form, \\
and having seen, they will know just these things. \\
Gladly let them see much or little, \\
for experts say this is no way to purity. 

It’s\marginnote{16.1} not easy to educate someone who is dogmatic, \\
promoting a view they have formulated. \\
Speaking of the beauty in that which they depend on, \\
they talk of purity in accord with what they saw there. 

The\marginnote{17.1} brahmin does not get involved with formulating and calculating; \\
they’re not followers of views, nor kinsmen of notions. \\
Having understood the many different convictions, \\
they look on when others grasp. 

Having\marginnote{18.1} untied the knots here in the world, \\
the sage takes no side among factions. \\
Peaceful among the peaceless, equanimous, \\
they don’t grasp when others grasp. 

Having\marginnote{19.1} given up former defilements, and not making new ones, \\
not swayed by preference, nor a proponent of dogma, \\
that wise one is released from views, \\
not clinging to the world, nor reproaching themselves. 

They\marginnote{20.1} are remote from all things \\
seen, heard, or thought. \\
With burden put down, the sage is released: \\
not formulating, not abstaining, not longing.” 

%
\end{verse}

%
\section*{{\suttatitleacronym Snp 4.14}{\suttatitletranslation Speedy }{\suttatitleroot Tuvaṭakasutta}}
\addcontentsline{toc}{section}{\tocacronym{Snp 4.14} \toctranslation{Speedy } \tocroot{Tuvaṭakasutta}}
\markboth{Speedy }{Tuvaṭakasutta}
\extramarks{Snp 4.14}{Snp 4.14}

\begin{verse}%
“Great\marginnote{1.1} hermit, I ask you, the Kinsman of the Sun, \\
about seclusion and the state of peace. \\
How, having seen, is a mendicant quenched, \\
not grasping anything in this world?” 

“They\marginnote{2.1} would cut off the idea, ‘I am the thinker,” \\
\scspeaker{said the Buddha, }\\
“which is the root of all concepts of identity due to proliferation. \\
Ever mindful, they would train to remove \\
any internal cravings. 

Regardless\marginnote{3.1} of what things they know, \\
whether internal or external, \\
they wouldn’t be proud because of that, \\
for that is not extinguishment, say the good. 

They\marginnote{4.1} wouldn’t let that make them conceited, \\
thinking themselves better or worse or alike. \\
When questioned in many ways, \\
they wouldn’t keep justifying themselves. 

A\marginnote{5.1} mendicant would find peace inside themselves, \\
and not seek peace from another. \\
For one at peace inside themselves, \\
there’s no picking up, whence putting down? 

Just\marginnote{6.1} as, in the mid-ocean deeps \\
no waves arise, it stays still; \\
so too one unstirred is still—\\
a mendicant would nurse no pride at all.” 

“He\marginnote{7.1} whose eyes are open has explained \\
the truth he witnessed, where adversities are removed. \\
Please now speak of the practice, sir, \\
the monastic code and immersion in \textsanskrit{samādhi}.” 

“With\marginnote{8.1} eyes not wanton, \\
they’d turn their ears from village gossip. \\
They wouldn’t be greedy for flavors, \\
nor possessive about anything in the world. 

Though\marginnote{9.1} struck by contacts, \\
a mendicant would not lament at all. \\
They wouldn’t pray for another life, \\
nor tremble in the face of dangers. 

When\marginnote{10.1} they receive food and drink, \\
edibles and clothes, \\
they wouldn’t store them up, \\
nor worry about not getting them. 

Meditative,\marginnote{11.1} not footloose, \\
they’d avoid remorse and not be negligent. \\
Then a mendicant would stay \\
in quiet places to sit and rest. 

They\marginnote{12.1} wouldn’t take much sleep, \\
but, being keen, would apply themselves to wakefulness. \\
They’d give up sloth, illusion, mirth, and play, \\
and sex and ornamentation. 

They\marginnote{13.1} wouldn’t cast \textsanskrit{Artharvaṇa} spells, interpret dreams \\
or omens, or practice astrology. \\
My followers would not decipher animal cries, practice healing, \\
or cast pregnancy spells. 

Not\marginnote{14.1} shaken by criticism, \\
a mendicant would not pride themselves when praised. \\
They’d reject greed and stinginess, \\
anger, and slander. 

They’d\marginnote{15.1} not stand for buying and selling; \\
a mendicant would not speak ill at all. \\
They wouldn’t linger in the village, \\
nor cajole people from desire for profit. 

A\marginnote{16.1} mendicant would be no boaster, \\
nor would they speak suggestively. \\
They wouldn’t train in impudence, \\
nor speak argumentatively. 

They\marginnote{17.1} wouldn’t be led into lying, \\
nor be deliberately devious. \\
And they’d never look down on another \\
because of livelihood, wisdom, or precepts and vows. 

Though\marginnote{18.1} provoked from hearing much talk \\
from ascetics saying all different things, \\
they wouldn’t react harshly, \\
for the virtuous do not retaliate. 

Having\marginnote{19.1} understood this teaching, \\
inquiring, a mendicant would always train mindfully. \\
Knowing extinguishment as peace, \\
they’d not be negligent in Gotama’s bidding. 

For\marginnote{20.1} he is the undefeated, the champion, \\
seer of the truth as witness, not by hearsay—\\
that’s why, being diligent, they would always train \\
respectfully in the Buddha’s teaching.” 

%
\end{verse}

%
\section*{{\suttatitleacronym Snp 4.15}{\suttatitletranslation Taking Up Arms }{\suttatitleroot Attadaṇḍasutta}}
\addcontentsline{toc}{section}{\tocacronym{Snp 4.15} \toctranslation{Taking Up Arms } \tocroot{Attadaṇḍasutta}}
\markboth{Taking Up Arms }{Attadaṇḍasutta}
\extramarks{Snp 4.15}{Snp 4.15}

\begin{verse}%
Peril\marginnote{1.1} stems from those who take up arms—\\
just look at people in conflict! \\
I shall extol how I came to be \\
stirred with a sense of urgency. 

I\marginnote{2.1} saw this population flounder, \\
like a fish in a little puddle. \\
Seeing them fight each other, \\
fear came upon me. 

The\marginnote{3.1} world around was hollow, \\
all directions were in turmoil. \\
Wanting a home for myself, \\
I saw nowhere unsettled. 

But\marginnote{4.1} even in their settlement they fight— \\
seeing that, I grew uneasy. \\
Then I saw a dart there, \\
so hard to see, stuck in the heart. 

When\marginnote{5.1} struck by that dart, \\
you run about in all directions. \\
But when that same dart has been plucked out, \\
you neither run about nor sink down. 

(On\marginnote{6.1} that topic, the trainings are recited.) \\
Whatever attachments there are in the world, \\
don’t pursue them. \\
Having pierced through sensual pleasures in every way, \\
train yourself for quenching. 

Be\marginnote{7.1} truthful, not rude, \\
free of deceit, and rid of slander; \\
without anger, a sage would cross over \\
the evils of greed and avarice. 

Prevail\marginnote{8.1} over sleepiness, sloth, and drowsiness, \\
don’t abide in negligence, \\
A person intent on quenching \\
would not stand for arrogance. 

Don’t\marginnote{9.1} be led into lying, \\
or get caught up in fondness for form. \\
Completely understand conceit, \\
and desist from hasty conduct. 

Don’t\marginnote{10.1} relish the old, \\
or welcome the new. \\
Don’t grieve for what is running out, \\
or get attached to things that pull you in. 

Greed,\marginnote{11.1} I say, is the great flood, \\
and longing is the current—\\
the basis, the compulsion, \\
the swamp of sensuality so hard to get past. 

The\marginnote{12.1} sage never strays from the truth; \\
the brahman stands firm on the shore. \\
Having given up everything, \\
they are said to be at peace. 

They\marginnote{13.1} have truly known, they’re a knowledge master, \\
understanding the teaching, they are independent. \\
They rightly proceed in the world, \\
not coveting anything here. 

One\marginnote{14.1} who has crossed over sensuality here, \\
the snare in the world so hard to get past, \\
grieves not, nor hopes; \\
they’ve cut the strings, they’re no longer bound. 

What\marginnote{15.1} came before, let wither away, \\
and after, let there be nothing. \\
If you don’t grasp at the middle, \\
you will live at peace. 

One\marginnote{16.1} who has no sense of ownership \\
in the whole realm of name and form, \\
does not grieve for that which is not, \\
they suffer no loss in the world. 

If\marginnote{17.1} you don’t think of anything \\
as belonging to yourself or others, \\
not finding anything to be ‘mine’, \\
you won’t grieve, thinking ‘I don’t have it’. 

Not\marginnote{18.1} bitter, not fawning, \\
unstirred, everywhere even; \\
when asked about one who is unshakable, \\
I declare that that is the benefit. 

For\marginnote{19.1} the unstirred who understand, \\
there’s no performance of deeds. \\
Desisting from instigation, \\
they see sanctuary everywhere. 

A\marginnote{20.1} sage doesn’t speak of themselves as being \\
among superiors, inferiors, or equals. \\
Peaceful, rid of stinginess, \\
they neither take nor reject. 

%
\end{verse}

%
\section*{{\suttatitleacronym Snp 4.16}{\suttatitletranslation With Sāriputta }{\suttatitleroot Sāriputtasutta}}
\addcontentsline{toc}{section}{\tocacronym{Snp 4.16} \toctranslation{With Sāriputta } \tocroot{Sāriputtasutta}}
\markboth{With Sāriputta }{Sāriputtasutta}
\extramarks{Snp 4.16}{Snp 4.16}

\begin{verse}%
“Never\marginnote{1.1} before have I seen,” \\
\scspeaker{said Venerable \textsanskrit{Sāriputta}, }\\
“or heard from anyone \\
about a teacher of such graceful speech, \\
come from Tusita heaven to lead a community. 

To\marginnote{2.1} all the world with its gods \\
he appears as a seer \\
who has dispelled all darkness, \\
and alone attained to bliss. 

On\marginnote{3.1} behalf of the many here still bound, \\
I have come in need with a question \\
to that Buddha, unattached and impartial, \\
free of deceit, come to lead a community. 

Suppose\marginnote{4.1} a mendicant who loathes attachment \\
frequents a lonely lodging—\\
the root of a tree, a charnel ground, \\
on mountains, or in caves. 

In\marginnote{5.1} these many different lodgings, \\
how many dangers are there \\
at which a mendicant in their silent lodging \\
ought not tremble? 

On\marginnote{6.1} their journey to the untrodden place, \\
how many adversities are there in the world \\
that must they overcome \\
in their remote lodging? 

What\marginnote{7.1} ways of speech should they have? \\
Where should they go for alms? \\
What precepts and vows \\
should a resolute mendicant uphold? 

Having\marginnote{8.1} undertaken what training, \\
unified, alert, and mindful, \\
would they purge their own stains, \\
like a smith smelting silver?” 

“If\marginnote{9.1} one who loathes attachment frequents a lonely lodging,” \\
\scspeaker{said the Buddha to \textsanskrit{Sāriputta}, }\\
“in their search for awakening—as accords with the teaching—\\
I shall tell you, as I understand it, \\
what is comfortable for them. 

A\marginnote{10.1} wise one, a mindful mendicant living on the periphery \\
should not be afraid of five perils: \\
flies, mosquitoes, snakes, \\
human contact, or four-legged creatures. 

Nor\marginnote{11.1} should they fear followers of other teachings, \\
even having seen the many threats they pose. \\
And then one seeking the good \\
should overcome any other adversities. 

Afflicted\marginnote{12.1} by illness and hunger, \\
they should endure cold and excessive heat. \\
Though afflicted by many such things, the homeless one \\
should exert energy, firmly striving. 

They\marginnote{13.1} must not steal or lie; \\
and should touch creatures firm or frail with love. \\
If they notice any clouding of the mind, \\
they should dispel it as \textsanskrit{Māra}’s ally. 

They\marginnote{14.1} must not fall under the sway of anger or arrogance; \\
having dug them out by the root, they would stand firm. \\
Then, withstanding likes and dislikes, \\
they would overcome. 

Putting\marginnote{15.1} wisdom in the foremost place, rejoicing in goodness, \\
they would put an end to those adversities. \\
They’d vanquish discontent in their remote lodging. \\
And they’d vanquish the four lamentations: 

‘What\marginnote{16.1} will I eat? Where will I eat? \\
Oh, I slept badly! Where will I sleep?’ \\
The trainee, the homeless migrant, \\
would dispel these lamentable thoughts. 

Receiving\marginnote{17.1} food and clothes in due season, \\
they would know moderation for the sake of contentment. \\
Guarded in these things, walking restrained in the village, \\
they wouldn’t speak harshly even when provoked. 

Eyes\marginnote{18.1} downcast, not footloose, \\
devoted to absorption, they’d be very wakeful. \\
Grounded in equanimity, serene, \\
they should cut off worrisome habits of thought. 

A\marginnote{19.1} mindful one should welcome repoach, \\
breaking up hard-heartedness towards their spiritual companions. \\
They may utter skillful speech, but not for too long, \\
and they shouldn’t provoke people to blame. 

And\marginnote{20.1} there are five more taints in the world, \\
for the removal of which the mindful one should train, \\
vanquishing desire for sights, \\
sounds, flavors, smells, and touches. 

Having\marginnote{21.1} removed desire for these things, \\
a mindful mendicant, their heart well freed, \\
rightly investigating the Dhamma in good time, \\
unified, would shatter the darkness.” 

%
\end{verse}

%
\addtocontents{toc}{\let\protect\contentsline\protect\nopagecontentsline}
\chapter*{The Chapter on the Way to the Beyond}
\addcontentsline{toc}{chapter}{\tocchapterline{The Chapter on the Way to the Beyond}}
\addtocontents{toc}{\let\protect\contentsline\protect\oldcontentsline}

%
\section*{{\suttatitleacronym Snp 5.1}{\suttatitletranslation Introductory Verses }{\suttatitleroot Vatthugāthā}}
\addcontentsline{toc}{section}{\tocacronym{Snp 5.1} \toctranslation{Introductory Verses } \tocroot{Vatthugāthā}}
\markboth{Introductory Verses }{Vatthugāthā}
\extramarks{Snp 5.1}{Snp 5.1}

\begin{verse}%
From\marginnote{1.1} the fair city of the Kosalans \\
to the southern region \\
came a brahmin expert in hymns, \\
aspiring to nothingness. 

In\marginnote{2.1} the domain of Assaka, \\
close by \textsanskrit{Aḷaka}, \\
he lived on the bank of the \textsanskrit{Godhāvarī} River, \\
getting by on gleanings and fruit. 

He\marginnote{3.1} was supported \\
by a prosperous village nearby. \\
With the revenue earned from there \\
he performed a great sacrifice. 

When\marginnote{4.1} he had completed the great sacrifice, \\
he returned to his hermitage once more. \\
Upon his return, \\
another brahmin arrived. 

Foot-sore\marginnote{5.1} and thirsty, \\
with grotty teeth and dusty head, \\
he approached the other \\
and asked for five hundred coins. 

When\marginnote{6.1} \textsanskrit{Bāvari} saw him, \\
he invited him to sit down, \\
asked of his happiness and well-being, \\
and said the following. 

“Whatever\marginnote{7.1} I had available to give, \\
I have already distributed. \\
Believe me, brahmin, \\
I don’t have five hundred coins.” 

“If,\marginnote{8.1} good sir, you do not \\
give me what I ask, \\
then on the seventh day, \\
let your head explode in seven!” 

After\marginnote{9.1} performing a ritual, \\
that charlatan uttered his dreadful curse. \\
When he heard these words, \\
\textsanskrit{Bāvari} became distressed. 

Not\marginnote{10.1} eating, he grew emaciated, \\
stricken by the dart of sorrow. \\
And in such a state of mind, \\
he could not enjoy absorption. 

Seeing\marginnote{11.1} him anxious and distraught, \\
a goddess wishing to help, \\
approached \textsanskrit{Bāvari} \\
and said the following. 

“That\marginnote{12.1} charlatan understands nothing \\
about the head, he only wants money. \\
When it comes to heads or head-splitting, \\
he has no knowledge at all.” 

“Madam,\marginnote{13.1} surely you must know—\\
please answer my question. \\
Let me hear what you say \\
about heads and head-splitting.” 

“I\marginnote{14.1} too do not know that, \\
I have no knowledge in that matter. \\
When it comes to heads or head-splitting, \\
it is the Victors who have vision.” 

“Then,\marginnote{15.1} in all this vast territory, \\
who exactly does know \\
about heads and head-splitting? \\
Please tell me, goddess.” 

“From\marginnote{16.1} the city of Kapilavatthu \\
the World Leader has gone forth. \\
He is a scion of King \textsanskrit{Okkāka}, \\
a Sakyan, and a beacon. 

For\marginnote{17.1} he, brahmin, is the Awakened One! \\
He has gone beyond all things; \\
he has attained to all knowledge and power; \\
he is the seer into all things, \\
he has attained the end of all deeds; \\
he is freed with the ending of attachments. 

That\marginnote{18.1} Buddha, the Blessed One in the world, \\
the Seer, teaches Dhamma. \\
Go to him and ask—\\
he will answer you.” 

When\marginnote{19.1} he heard the word “Buddha”, \\
\textsanskrit{Bāvari} was elated. \\
His sorrow faded, \\
and he was filled to brimming with joy. 

Uplifted,\marginnote{20.1} elated, and inspired, \\
\textsanskrit{Bāvari} questioned that goddess: \\
“But in what village or town, \\
or in what land is the protector of the world, \\
where we may go and pay respects \\
to the Awakened One, best of men?” 

“Near\marginnote{21.1} \textsanskrit{Sāvatthī}, the home of the Kosalans, is the Victor \\
abounding in wisdom, vast in intelligence. \\
That Sakyan is indefatigable, free of defilements, a bull among men: \\
he understands head-splitting. 

Therefore\marginnote{22.1} he addressed his pupils, \\
brahmins who had mastered the hymns: \\
“Come, students, I shall speak. \\
Listen to what I say. 

Today\marginnote{23.1} has arisen in the world \\
one whose appearance in the world \\
is hard to find again—\\
he is renowned as the Awakened One. \\
Quickly go to \textsanskrit{Sāvatthī} \\
and see the best of men.” 

“Brahmin,\marginnote{24.1} how exactly are we to know \\
the Buddha when we see him? \\
We don’t know, please tell us, \\
so we can recognize him.” 

“The\marginnote{25.1} marks of a great man \\
have been handed down in our hymns. \\
Thirty-two have been described, \\
complete and in order. 

One\marginnote{26.1} upon whose body is found \\
these marks of a great man \\
has two possible destinies, \\
there is no third. 

If\marginnote{27.1} he stays at home, \\
having conquered this land \\
without rod or sword, \\
he shall govern by principle. 

But\marginnote{28.1} if he goes forth \\
from the lay life to homelessness, \\
he becomes an Awakened One, a perfected one, \\
with veil drawn back, supreme. 

Ask\marginnote{29.1} him about my birth, clan, and marks, \\
my hymns and students; and further, \\
about heads and head-splitting—\\
but do so only in your mind! 

If\marginnote{30.1} he is the Buddha \\
of unobstructed vision, \\
he will answer with his voice \\
the questions in your mind.” 

Sixteen\marginnote{31.1} brahmin pupils \\
heard what \textsanskrit{Bāvari} said: \\
Ajita, Tissametteyya, \\
\textsanskrit{Puṇṇaka} and \textsanskrit{Mettagū}, 

Dhotaka\marginnote{32.1} and Upasiva, \\
Nanda and then Hemaka, \\
both Todeyya and Kappa, \\
and \textsanskrit{Jatukaṇṇī} the astute, 

\textsanskrit{Bhadrāvudha}\marginnote{33.1} and Udaya, \\
and the brahmin Posala, \\
\textsanskrit{Mogharājā} the intelligent, \\
and \textsanskrit{Piṅgiya} the great hermit. 

Each\marginnote{34.1} of them had their own following, \\
they were renowned the whole world over. \\
Those wise ones, meditators who love absorption, \\
were redolent with the potential of their past deeds. 

Having\marginnote{35.1} bowed to \textsanskrit{Bāvari}, \\
and circled him to his right, \\
they set out for the north, \\
with their dreadlocks and hides. 

First\marginnote{36.1} to \textsanskrit{Patiṭṭhāna} of \textsanskrit{Aḷaka}, \\
then on to the city of Mahissati; \\
to \textsanskrit{Ujjenī} and \textsanskrit{Gonaddhā}, \\
and Vedisa, and Vanasa. 

Then\marginnote{37.1} to Kosambi and \textsanskrit{Sāketa}, \\
and the supreme city of \textsanskrit{Sāvatthī}; \\
on they went to \textsanskrit{Setavyā} and Kapilavatthu, \\
and the homestead at \textsanskrit{Kusinārā}. 

To\marginnote{38.1} \textsanskrit{Pāvā} they went, and Bhoganagara, \\
and on to \textsanskrit{Vesālī} and the Magadhan city. \\
Finally they reached the \textsanskrit{Pāsāṇaka} shrine, \\
fair and delightful. 

Like\marginnote{39.1} a thirsty person to cool water, \\
like a merchant to great profit, \\
like a heat-struck person to shade, \\
they quickly climbed the mountain. 

At\marginnote{40.1} that time the Buddha \\
at the fore of the mendicant \textsanskrit{Saṅgha}, \\
was teaching the mendicants the Dhamma, \\
like a lion roaring in the jungle. 

Ajita\marginnote{41.1} saw the Buddha, \\
like the sun shining with a hundred rays, \\
like the moon on the fifteenth day \\
when it has come into its fullness. 

Then\marginnote{42.1} he saw his body, \\
complete in all features. \\
Thrilled, he stood to one side \\
and asked this question in his mind. 

“Speak\marginnote{43.1} about the brahmin’s birth; \\
of his clan; and his own marks; \\
what hymns is he proficient in; \\
and how many he teaches.” 

“His\marginnote{44.1} age is a hundred and twenty. \\
By clan he is a \textsanskrit{Bāvari}. \\
There are three marks on his body. \\
He is a master of the three Vedas, 

the\marginnote{45.1} teachings on the marks, the testaments, \\
the vocabularies, and the rituals. \\
He teaches five hundred, \\
and has reached proficiency in his own teaching.” 

“O\marginnote{46.1} supreme person, cutter of craving, \\
please reveal in detail \\
\textsanskrit{Bāvari}’s marks—\\
let us doubt no longer!” 

“He\marginnote{47.1} can cover his face with his tongue; \\
there is a tuft of hair between his eyebrows; \\
his private parts are concealed in a foreskin: \\
know them as this, young man.” 

Hearing\marginnote{48.1} the answers \\
without having heard any questions, \\
all the people, inspired, \\
with joined palms, wondered: 

“Who\marginnote{49.1} is it that asked a question with their mind? \\
Was it a god or \textsanskrit{Brahmā}? \\
Or Indra, \textsanskrit{Sujā}’s husband? \\
To whom does the Buddha reply?” 

“\textsanskrit{Bāvari}\marginnote{50.1} asks \\
about heads and head-splitting. \\
May the Buddha please answer, \\
and so, O hermit, dispel our doubt.” 

“Know\marginnote{51.1} ignorance as the head, \\
and knowledge as the head-splitter, \\
when joined with faith, mindfulness, and immersion, \\
and enthusiasm and energy.” 

At\marginnote{52.1} that, the brahmin student, \\
full of inspiration, \\
arranged his antelope-skin cloak over one shoulder, \\
and fell with his head to the Buddha’s feet. 

“Good\marginnote{53.1} sir, the brahmin \textsanskrit{Bāvari} \\
together with his pupils, \\
elated and happy, \\
bows to your feet, O seer!” 

“May\marginnote{54.1} the brahmin \textsanskrit{Bāvari} be happy, \\
together with his pupils. \\
And may you, too, be happy! \\
May you live long, young man. 

To\marginnote{55.1} \textsanskrit{Bāvari} and you all \\
I grant the opportunity to clear up all doubt. \\
Please ask \\
whatever you want.” 

Granted\marginnote{56.1} the opportunity by the Buddha, \\
they sat down with joined palms. \\
Ajita asked the Realized One \\
the first question right there. 

%
\end{verse}

\scendsection{The introductory verses are finished. }

%
\section*{{\suttatitleacronym Snp 5.2}{\suttatitletranslation The Questions of Ajita }{\suttatitleroot Ajitamāṇavapucchā}}
\addcontentsline{toc}{section}{\tocacronym{Snp 5.2} \toctranslation{The Questions of Ajita } \tocroot{Ajitamāṇavapucchā}}
\markboth{The Questions of Ajita }{Ajitamāṇavapucchā}
\extramarks{Snp 5.2}{Snp 5.2}

\begin{verse}%
“By\marginnote{1.1} what is the world shrouded?” \\
\scspeaker{said Venerable Ajita. }\\
“Why does it not shine? \\
Tell me, what is its tar pit? \\
What is its greatest fear?” 

“The\marginnote{2.1} world is shrouded in ignorance.” \\
\scspeaker{replied the Buddha. }\\
“Avarice and negligence make it not shine. \\
Prayer is its tar pit. \\
Suffering is its greatest fear.” 

“The\marginnote{3.1} streams flow everywhere,” \\
\scspeaker{said Venerable Ajita. }\\
“What is there to block them? \\
And tell me the restraint of streams—\\
by what are they locked out?” 

“The\marginnote{4.1} streams in the world,” \\
\scspeaker{replied the Buddha, }\\
“are blocked by mindfulness. \\
I tell you the restraint of streams—\\
they are locked out by wisdom.” 

“That\marginnote{5.1} wisdom and mindfulness,” \\
\scspeaker{said Venerable Ajita, }\\
“and that which is name and form, good sir; \\
when questioned, please tell me of this: \\
where does this all cease?” 

“This\marginnote{6.1} question which you have asked, \\
I shall answer you, Ajita. \\
Where name and form \\
cease with nothing left over—\\
with the cessation of consciousness, \\
that’s where they cease.” 

“There\marginnote{7.1} are those who have assessed the teaching, \\
and many kinds of trainees here. \\
Tell me about their behavior, good sir, \\
when asked, for you are alert.” 

“Not\marginnote{8.1} greedy for sensual pleasures, \\
their mind would be unclouded. \\
Skilled in all things, \\
a mendicant would wander mindful.” 

%
\end{verse}

%
\section*{{\suttatitleacronym Snp 5.3}{\suttatitletranslation The Questions of Tissametteyya }{\suttatitleroot Tissametteyyamāṇavapucchā}}
\addcontentsline{toc}{section}{\tocacronym{Snp 5.3} \toctranslation{The Questions of Tissametteyya } \tocroot{Tissametteyyamāṇavapucchā}}
\markboth{The Questions of Tissametteyya }{Tissametteyyamāṇavapucchā}
\extramarks{Snp 5.3}{Snp 5.3}

\begin{verse}%
“Who\marginnote{1.1} is content here in the world?” \\
\scspeaker{said Venerable Tissametteyya. }\\
“Who has no disturbances? \\
Who, having known both ends, \\
is not stuck in the middle? \\
Who do they say is a great man? \\
Who here has escaped the seamstress?” 

“Leading\marginnote{2.1} the spiritual life among sensual pleasures,” \\
\scspeaker{replied the Buddha, }\\
“rid of craving, ever mindful; \\
a mendicant who, after assessing, is quenched: \\
that’s who has no disturbances. 

That\marginnote{3.1} sage, having known both ends, \\
is not stuck in the middle. \\
He is a great man, I declare, \\
he has escaped the seamstress here.” 

%
\end{verse}

%
\section*{{\suttatitleacronym Snp 5.4}{\suttatitletranslation The Questions of Puṇṇaka }{\suttatitleroot Puṇṇakamāṇavapucchā}}
\addcontentsline{toc}{section}{\tocacronym{Snp 5.4} \toctranslation{The Questions of Puṇṇaka } \tocroot{Puṇṇakamāṇavapucchā}}
\markboth{The Questions of Puṇṇaka }{Puṇṇakamāṇavapucchā}
\extramarks{Snp 5.4}{Snp 5.4}

\begin{verse}%
“To\marginnote{1.1} the imperturbable, the seer of the root,” \\
\scspeaker{said Venerable \textsanskrit{Puṇṇaka}, }\\
“I have come in need with a question. \\
On what grounds have hermits and men, \\
aristocrats and brahmins here in the world \\
performed so many different sacrifices to the gods? \\
I ask you, Blessed One; please tell me this.” 

“Whatever\marginnote{2.1} hermits and men,” \\
\scspeaker{replied the Buddha, }\\
“aristocrats and brahmins here in the world \\
have performed so many different sacrifices to the gods: \\
all performed sacrifices bound to old age, \\
hoping for some state of existence.” 

“As\marginnote{3.1} to those hermits and men,” \\
\scspeaker{said Venerable \textsanskrit{Puṇṇaka}, }\\
“and aristocrats and brahmins here in the world \\
who have performed so many different sacrifices to the gods: \\
being diligent in the methods of sacrifice, \\
have they crossed over rebirth and old age, good sir? \\
I ask you, Blessed One; please tell me this.” 

“Hoping,\marginnote{4.1} invoking, praying, and worshiping,” \\
\scspeaker{replied the Buddha, }\\
“they pray for pleasure derived from profit. \\
Devoted to sacrifice, besotted by rebirth, \\
they’ve not crossed over rebirth and old age, I declare.” 

“If\marginnote{5.1} those devoted to sacrifice,” \\
\scspeaker{said Venerable \textsanskrit{Puṇṇaka}, }\\
“have not, by sacrificing, crossed over rebirth and old age, \\
then who exactly in the world of gods and humans \\
has crossed over rebirth and old age, good sir? \\
I ask you, Blessed One; please tell me this.” 

“Having\marginnote{6.1} assessed the world high and low,” \\
\scspeaker{replied the Buddha, }\\
“there is nothing in the world that disturbs them. \\
Peaceful, unclouded, untroubled, with no need for hope—\\
they’ve crossed over rebirth and old age, I declare.” 

%
\end{verse}

%
\section*{{\suttatitleacronym Snp 5.5}{\suttatitletranslation The Questions of Mettagū }{\suttatitleroot Mettagūmāṇavapucchā}}
\addcontentsline{toc}{section}{\tocacronym{Snp 5.5} \toctranslation{The Questions of Mettagū } \tocroot{Mettagūmāṇavapucchā}}
\markboth{The Questions of Mettagū }{Mettagūmāṇavapucchā}
\extramarks{Snp 5.5}{Snp 5.5}

\begin{verse}%
“I\marginnote{1.1} ask you, Blessed One; please tell me this,” \\
\scspeaker{said Venerable \textsanskrit{Mettagū}, }\\
“for I think you are a knowledge master, evolved. \\
Where do all these sufferings come from, \\
in all their countless forms in the world?” 

“You\marginnote{2.1} have rightly asked me of the origin of suffering,” \\
\scspeaker{replied the Buddha, }\\
“I shall tell you as I understand it. \\
Attachment is the source of suffering \\
in all its countless forms in the world. 

When\marginnote{3.1} an ignorant person builds up attachments, \\
that idiot returns to suffering again and again. \\
So let one who understands not build up attachments, \\
contemplating the birth and origin of suffering.” 

“Whatever\marginnote{4.1} I asked you have explained to me. \\
I ask you once more, please tell me this: \\
How do the wise cross the flood \\
of rebirth, old age, sorrow, and lamenting? \\
Please, sage, answer me clearly, \\
for truly you understand this matter.” 

“I\marginnote{5.1} shall extol a teaching to you,” \\
\scspeaker{replied the Buddha, }\\
“that is apparent in the present, not relying on tradition. \\
Having understood it, one who lives mindfully \\
may cross over clinging in the world.” 

“And\marginnote{6.1} I rejoice, great hermit, \\
in that supreme teaching, \\
having understood which, one who lives mindfully \\
may cross over clinging in the world.” 

“Once\marginnote{7.1} you’ve expelled relishing and dogmatism,” \\
\scspeaker{replied the Buddha, }\\
“regarding everything you are aware of—\\
above, below, all round, between— \\
having uprooted consciousness, don’t continue in existence. 

A\marginnote{8.1} mendicant who wanders meditating like this, \\
diligent and mindful, calling nothing their own, \\
would, being wise, give up the suffering \\
of rebirth, old age, sorrow and lamenting right here.” 

“I\marginnote{9.1} rejoice in the words of the great hermit! \\
You have expounded non-attachment well, Gotama. \\
Clearly the Buddha has given up suffering, \\
for truly you understand this matter. 

Surely\marginnote{10.1} those you’d regularly instruct \\
would also give up suffering. \\
Therefore, having met, I bow to you, O spiritual giant; \\
hopefully the Buddha may regularly instruct me.” 

“Any\marginnote{11.1} brahmin recognized as a knowledge master, \\
who has nothing, unattached to sensual life, \\
clearly has crossed this flood, \\
crossed to the far shore, kind, wishless. 

And\marginnote{12.1} a wise person here, a knowledge master, \\
having untied the bond to live after life, \\
free of craving, untroubled, with no need for hope, \\
has crossed over rebirth and old age, I declare.” 

%
\end{verse}

%
\section*{{\suttatitleacronym Snp 5.6}{\suttatitletranslation The Questions of Dhotaka }{\suttatitleroot Dhotakamāṇavapucchā}}
\addcontentsline{toc}{section}{\tocacronym{Snp 5.6} \toctranslation{The Questions of Dhotaka } \tocroot{Dhotakamāṇavapucchā}}
\markboth{The Questions of Dhotaka }{Dhotakamāṇavapucchā}
\extramarks{Snp 5.6}{Snp 5.6}

\begin{verse}%
“I\marginnote{1.1} ask you, Blessed One; please tell me this,” \\
\scspeaker{said Venerable Dhotaka, }\\
“I long for your voice, great hermit. \\
After hearing your message, \\
I shall train myself for quenching.” 

“Well\marginnote{2.1} then, be keen, alert,” \\
\scspeaker{replied the Buddha, }\\
“and mindful right here. \\
After hearing this message, go on \\
and train yourself for quenching.” 

“I\marginnote{3.1} see in the world of gods and humans \\
a brahmin travelling with nothing. \\
Therefore I bow to you, all-seer: \\
release me, Sakyan, from my doubts.” 

“I\marginnote{4.1} am not able to release anyone \\
in the world who has doubts, Dhotaka. \\
But when you understand the best of teachings, \\
you shall cross this flood.” 

“Teach\marginnote{5.1} me, brahmin, out of compassion, \\
the principle of seclusion so that I may understand. \\
I wish to practice right here, peaceful, independent, \\
as unimpeded as space.” 

“I\marginnote{6.1} shall extol that peace for you,” \\
\scspeaker{replied the Buddha, }\\
“that is apparent in the present, not relying on tradition. \\
Having understood it, one who lives mindfully \\
may cross over clinging in the world.” 

“And\marginnote{7.1} I rejoice, great hermit, \\
in that supreme peace, \\
having understood which, one who lives mindfully \\
may cross over clinging in the world.” 

“Once\marginnote{8.1} you have understood that everything,” \\
\scspeaker{replied the Buddha, }\\
“you are aware of in the world—\\
above, below, all round, between—\\
is a snare, don’t crave for life after life.” 

%
\end{verse}

%
\section*{{\suttatitleacronym Snp 5.7}{\suttatitletranslation The Questions of Upasiva }{\suttatitleroot Upasīvamāṇavapucchā}}
\addcontentsline{toc}{section}{\tocacronym{Snp 5.7} \toctranslation{The Questions of Upasiva } \tocroot{Upasīvamāṇavapucchā}}
\markboth{The Questions of Upasiva }{Upasīvamāṇavapucchā}
\extramarks{Snp 5.7}{Snp 5.7}

\begin{verse}%
“Alone\marginnote{1.1} and independent, O Sakyan,” \\
\scspeaker{said Venerable Upasiva, }\\
“I am not able to cross the great flood. \\
Tell me a support, All-seer, \\
depending on which I may cross this flood.” 

“Mindfully\marginnote{2.1} contemplating nothingness,” \\
\scspeaker{replied the Buddha, }\\
depending on the perception ‘there is nothing’, cross the flood. \\
Giving up sensual pleasures, refraining from chatter, \\
watch day and night for the ending of craving.” 

“One\marginnote{3.1} who is free of all sensual desires,” \\
\scspeaker{said Venerable Upasiva, }\\
“depending on nothingness, all else left behind, \\
freed in the ultimate liberation of perception: \\
would they remain there without travelling on?” 

“One\marginnote{4.1} free of all sensual desires,” \\
\scspeaker{replied the Buddha, }\\
“depending on nothingness, all else left behind, \\
freed in the ultimate liberation of perception: \\
they would remain there without travelling on.” 

“If\marginnote{5.1} they were to remain there without travelling on, \\
even for many years, All-seer, \\
and, growing cool right there, were freed, \\
would the consciousness of such a one pass away?” 

“As\marginnote{6.1} a flame tossed by a gust of wind,” \\
\scspeaker{replied the Buddha, }\\
“comes to an end and no longer counts; \\
so too, a sage freed from mental phenomena \\
comes to an end and no longer counts.” 

“One\marginnote{7.1} who has come to an end—do they not exist? \\
Or are they eternally well? \\
Please, sage, answer me clearly, \\
for truly you understand this matter.” 

“One\marginnote{8.1} who has come to an end cannot be measured,” \\
\scspeaker{replied the Buddha. }\\
“They have nothing by which one might describe them. \\
When all things have been eradicated, \\
eradicated, too, are all ways of speech.” 

%
\end{verse}

%
\section*{{\suttatitleacronym Snp 5.8}{\suttatitletranslation The Questions of Nanda }{\suttatitleroot Nandamāṇavapucchā}}
\addcontentsline{toc}{section}{\tocacronym{Snp 5.8} \toctranslation{The Questions of Nanda } \tocroot{Nandamāṇavapucchā}}
\markboth{The Questions of Nanda }{Nandamāṇavapucchā}
\extramarks{Snp 5.8}{Snp 5.8}

\begin{verse}%
“People\marginnote{1.1} say there are sages in the world,” \\
\scspeaker{said Venerable Nanda, }\\
“but how is this the case? \\
Is someone called a sage because of their knowledge, \\
or because of their way of life?” 

“Experts\marginnote{2.1} do not speak of a sage in terms of \\
view, oral transmission, or notion. \\
Those who are sages live far from the crowd, I say, \\
untroubled, with no need for hope.” 

“As\marginnote{3.1} to those ascetics and brahmins,” \\
\scspeaker{said Venerable Nanda, }\\
“who speak of purity in terms of what is seen or heard, \\
or in terms of precepts and vows, \\
or in terms of countless different things. \\
Living self-controlled in that matter, \\
have they crossed over rebirth and old age, good sir? \\
I ask you, Blessed One; please tell me this.” 

“As\marginnote{4.1} to those ascetics and brahmins,” \\
\scspeaker{replied the Buddha, }\\
“who speak of purity in terms of what is seen or heard, \\
or in terms of precepts and vows, \\
or in terms of countless different things. \\
Even though they live self-controlled in that matter, \\
they’ve not crossed over rebirth and old age, I declare.” 

“As\marginnote{5.1} to those ascetics and brahmins,” \\
\scspeaker{said Venerable Nanda, }\\
“who speak of purity in terms of what is seen or heard, \\
or in terms of precepts and vows, \\
or in terms of countless different things. \\
You say they have not crossed the flood, sage. \\
Then who exactly in the world of gods and humans \\
has crossed over rebirth and old age, good sir? \\
I ask you, Blessed One; please tell me this.” 

“I\marginnote{6.1} don’t say that all ascetics and brahmins,” \\
\scspeaker{replied the Buddha, }\\
“are shrouded by rebirth and old age. \\
There are those here who have given up all \\
that is seen, heard, and thought, and precepts and vows, \\
who have given up all the countless different things. \\
Fully understanding craving, free of defilements, \\
those people, I say, have crossed the flood.” 

“I\marginnote{7.1} rejoice in the words of the great hermit! \\
You have expounded non-attachment well, Gotama. \\
There are those here who have given up all \\
that is seen, heard, and thought, and precepts and vows, \\
who have given up all the countless different things. \\
Fully understanding craving, free of defilements, \\
those people, I agree, have crossed the flood.” 

%
\end{verse}

%
\section*{{\suttatitleacronym Snp 5.9}{\suttatitletranslation The Questions of Hemaka }{\suttatitleroot Hemakamāṇavapucchā}}
\addcontentsline{toc}{section}{\tocacronym{Snp 5.9} \toctranslation{The Questions of Hemaka } \tocroot{Hemakamāṇavapucchā}}
\markboth{The Questions of Hemaka }{Hemakamāṇavapucchā}
\extramarks{Snp 5.9}{Snp 5.9}

\begin{verse}%
“Those\marginnote{1.1} who have previously answered me,” \\
\scspeaker{said Venerable Hemaka, }\\
“before I encountered Gotama’s teaching, \\
said ‘thus it was’ or ‘so it shall be’. \\
All that was just the testament of hearsay; \\
all that just fostered speculation: \\
I found no delight in that. 

But\marginnote{2.1} you, sage, explain to me \\
the teaching that destroys craving. \\
Having understood it, one who lives mindfully \\
may cross over clinging in the world.” 

“The\marginnote{3.1} removal of desire and lust, Hemaka, \\
for what is seen, heard, thought, or cognized here; \\
for anything liked or disliked, \\
is extinguishment, the imperishable state. 

Those\marginnote{4.1} who have fully understood this, mindful, \\
are extinguished in this very life. \\
Always at peace, \\
they’ve crossed over clinging to the world.” 

%
\end{verse}

%
\section*{{\suttatitleacronym Snp 5.10}{\suttatitletranslation The Questions of Todeyya }{\suttatitleroot Todeyyamāṇavapucchā}}
\addcontentsline{toc}{section}{\tocacronym{Snp 5.10} \toctranslation{The Questions of Todeyya } \tocroot{Todeyyamāṇavapucchā}}
\markboth{The Questions of Todeyya }{Todeyyamāṇavapucchā}
\extramarks{Snp 5.10}{Snp 5.10}

\begin{verse}%
“In\marginnote{1.1} whom sensual pleasures do not dwell,” \\
\scspeaker{said Venerable Todeyya, }\\
“and for whom there is no craving, \\
and who has crossed over doubts—\\
of what kind is their liberation?” 

“In\marginnote{2.1} whom sensual pleasures do not dwell,” \\
\scspeaker{replied the Buddha, }\\
“and for whom there is no craving, \\
and who has crossed over doubts—\\
their liberation is none other than this.” 

“Are\marginnote{3.1} they free of hope, or are they still in need of hope? \\
Do they possess wisdom, or are they still forming wisdom? \\
O Sakyan, elucidate the sage to me, \\
so that I may understand, All-seer.” 

“They\marginnote{4.1} are free of hope, they are not in need of hope. \\
They possess wisdom, they are not still forming wisdom. \\
That, Todeyya, is how to understand a sage, \\
one who has nothing, unattached to sensual life.” 

%
\end{verse}

%
\section*{{\suttatitleacronym Snp 5.11}{\suttatitletranslation The Questions of Kappa }{\suttatitleroot Kappamāṇavapucchā}}
\addcontentsline{toc}{section}{\tocacronym{Snp 5.11} \toctranslation{The Questions of Kappa } \tocroot{Kappamāṇavapucchā}}
\markboth{The Questions of Kappa }{Kappamāṇavapucchā}
\extramarks{Snp 5.11}{Snp 5.11}

\begin{verse}%
“For\marginnote{1.1} those overwhelmed by old age and death,” \\
\scspeaker{said Venerable Kappa, }\\
“stuck mid-stream \\
as the terrifying flood arises, \\
tell me an island, good sir. \\
Explain to me an island \\
so that this may not occur again.” 

“For\marginnote{2.1} those overwhelmed by old age and death,” \\
\scspeaker{replied the Buddha, }\\
“stuck mid-stream \\
as the terrifying flood arises, \\
I shall tell you an island, Kappa. 

Having\marginnote{3.1} nothing, taking nothing: \\
this is the isle of no return. \\
I call it extinguishment, \\
the ending of old age and death. 

Those\marginnote{4.1} who have fully understood this, mindful, \\
are extinguished in this very life. \\
They don’t fall under \textsanskrit{Māra}’s sway, \\
nor are they his lackeys.” 

%
\end{verse}

%
\section*{{\suttatitleacronym Snp 5.12}{\suttatitletranslation The Questions of Jatukaṇṇī }{\suttatitleroot Jatukaṇṇimāṇavapucchā}}
\addcontentsline{toc}{section}{\tocacronym{Snp 5.12} \toctranslation{The Questions of Jatukaṇṇī } \tocroot{Jatukaṇṇimāṇavapucchā}}
\markboth{The Questions of Jatukaṇṇī }{Jatukaṇṇimāṇavapucchā}
\extramarks{Snp 5.12}{Snp 5.12}

\begin{verse}%
“Hearing\marginnote{1.1} of the hero with no desire for sensual pleasures,” \\
\scspeaker{said Venerable \textsanskrit{Jatukaṇṇī}, }\\
“who has passed over the flood, I’ve come with a question for that desireless one. \\
Tell me the state of peace, O natural visionary. \\
Tell me this, Blessed One, as it really is. 

For,\marginnote{2.1} having mastered sensual desires, the Blessed One proceeds, \\
as the blazing sun shines on the earth. \\
May you of vast wisdom explain the teaching \\
to me of little wisdom so that I may understand \\
the giving up of rebirth and old age here.” 

“With\marginnote{3.1} sensual desire dispelled,” \\
\scspeaker{replied the Buddha, }\\
“seeing renunciation as sanctuary, \\
don’t be taking up or putting down \\
anything at all. 

What\marginnote{4.1} came before, let wither away, \\
and after, let there be nothing. \\
If you don’t grasp at the middle, \\
you will live at peace. 

One\marginnote{5.1} rid of greed, brahmin, \\
for the whole realm of name and form, \\
has no defilements by which \\
they might fall under the sway of Death.” 

%
\end{verse}

%
\section*{{\suttatitleacronym Snp 5.13}{\suttatitletranslation The Questions of Bhadrāvudha }{\suttatitleroot Bhadrāvudhamāṇavapucchā}}
\addcontentsline{toc}{section}{\tocacronym{Snp 5.13} \toctranslation{The Questions of Bhadrāvudha } \tocroot{Bhadrāvudhamāṇavapucchā}}
\markboth{The Questions of Bhadrāvudha }{Bhadrāvudhamāṇavapucchā}
\extramarks{Snp 5.13}{Snp 5.13}

\begin{verse}%
“I\marginnote{1.1} have a request for you, the shelter-leaver, the craving-cutter, the imperturbable,” \\
\scspeaker{said Venerable \textsanskrit{Bhadrāvudha}, }\\
“the delight-leaver, the flood-crosser, the freed, \\
the formulation-leaver, the intelligent. \\
Many people have gathered from different lands 

wishing\marginnote{2.1} to hear your word, O hero. \\
After hearing the spiritual giant they will depart from here. \\
Please, sage, answer them clearly, \\
for truly you understand this matter.” 

“Dispel\marginnote{3.1} all craving for attachments,” \\
\scspeaker{replied the Buddha, }\\
“above, below, all round, between. \\
For whatever a person grasps in the world, \\
\textsanskrit{Māra} pursues them right there. 

So\marginnote{4.1} let a mindful mendicant who understands \\
not grasp anything in all the world, \\
observing that, in clinging to attachments, \\
these people cling to the domain of death.” 

%
\end{verse}

%
\section*{{\suttatitleacronym Snp 5.14}{\suttatitletranslation The Questions of Udaya }{\suttatitleroot Udayamāṇavapucchā}}
\addcontentsline{toc}{section}{\tocacronym{Snp 5.14} \toctranslation{The Questions of Udaya } \tocroot{Udayamāṇavapucchā}}
\markboth{The Questions of Udaya }{Udayamāṇavapucchā}
\extramarks{Snp 5.14}{Snp 5.14}

\begin{verse}%
“To\marginnote{1.1} the meditator, rid of hopes,” \\
\scspeaker{said Venerable Udaya, }\\
“who has completed the task, is free of defilements, \\
and has gone beyond all things, \\
I have come in need with a question. \\
Tell me the liberation by enlightenment, \\
the smashing of ignorance.” 

“The\marginnote{2.1} giving up of sensual desires,” \\
\scspeaker{replied the Buddha, }\\
“and aversions, both; \\
the dispelling of dullness, \\
and the cessation of remorse. 

Pure\marginnote{3.1} equanimity and mindfulness, \\
preceded by investigation of principles—\\
this, I declare, is liberation by enlightenment, \\
the smashing of ignorance.” 

“What\marginnote{4.1} fetters the world? \\
What explores it? \\
With the giving up of what \\
is extinguishment spoken of?” 

“Delight\marginnote{5.1} fetters the world. \\
Thought explores it. \\
With the giving up of craving \\
extinguishment is spoken of.” 

“For\marginnote{6.1} one living mindfully, \\
how does consciousness cease? \\
We’ve come to ask the Buddha; \\
let us hear what you say.” 

“Not\marginnote{7.1} taking pleasure in feeling \\
internally and externally—\\
for one living mindfully, \\
that’s how consciousness ceases.” 

%
\end{verse}

%
\section*{{\suttatitleacronym Snp 5.15}{\suttatitletranslation The Question of Posala }{\suttatitleroot Posālamāṇavapucchā}}
\addcontentsline{toc}{section}{\tocacronym{Snp 5.15} \toctranslation{The Question of Posala } \tocroot{Posālamāṇavapucchā}}
\markboth{The Question of Posala }{Posālamāṇavapucchā}
\extramarks{Snp 5.15}{Snp 5.15}

\begin{verse}%
“To\marginnote{1.1} the one who reveals the past,” \\
\scspeaker{said Venerable Posala, }\\
who is imperturbable, with doubts cut off, \\
and who has gone beyond all things, \\
I have come in need with a question. 

Consider\marginnote{2.1} one who perceives the disappearance of form, \\
who has entirely given up the body, \\
and who sees nothing at all \\
internally and externally. \\
I ask the Sakyan about knowledge for them; \\
how should one like that be guided?” 

“The\marginnote{3.1} Realized One directly knows,” \\
\scspeaker{said the Buddha, }\\
“all the planes of consciousness. \\
And he knows this one who remains, \\
committed to that as their final goal. 

Understanding\marginnote{4.1} that desire for rebirth \\
in the dimension of nothingness is a fetter, \\
and directly knowing what this really means, \\
one then sees that matter clearly. \\
That is the knowledge of reality for them, \\
the brahmin who has lived the life.” 

%
\end{verse}

%
\section*{{\suttatitleacronym Snp 5.16}{\suttatitletranslation The Questions of Mogharājā }{\suttatitleroot Mogharājamāṇavapucchā}}
\addcontentsline{toc}{section}{\tocacronym{Snp 5.16} \toctranslation{The Questions of Mogharājā } \tocroot{Mogharājamāṇavapucchā}}
\markboth{The Questions of Mogharājā }{Mogharājamāṇavapucchā}
\extramarks{Snp 5.16}{Snp 5.16}

\begin{verse}%
“Twice\marginnote{1.1} I have asked the Sakyan,” \\
\scspeaker{said Venerable \textsanskrit{Mogharājā}, }\\
“but you haven’t answered me, O Seer. \\
I have heard that the divine hermit \\
answers when questioned a third time. 

Regarding\marginnote{2.1} this world, the other world, \\
and the realm of \textsanskrit{Brahmā} with its gods, \\
I’m not familiar with the view \\
of the renowned Gotama. 

So\marginnote{3.1} I’ve come in need with a question \\
to the one of excellent vision. \\
How to look upon the world \\
so the King of Death won’t see you?” 

“Look\marginnote{4.1} upon the world as empty, \\
\textsanskrit{Mogharājā}, ever mindful. \\
Having uprooted the view of self, \\
you may thus cross over death. \\
That’s how to look upon the world \\
so the King of Death won’t see you.” 

%
\end{verse}

%
\section*{{\suttatitleacronym Snp 5.17}{\suttatitletranslation The Questions of Piṅgiya }{\suttatitleroot Piṅgiyamāṇavapucchā}}
\addcontentsline{toc}{section}{\tocacronym{Snp 5.17} \toctranslation{The Questions of Piṅgiya } \tocroot{Piṅgiyamāṇavapucchā}}
\markboth{The Questions of Piṅgiya }{Piṅgiyamāṇavapucchā}
\extramarks{Snp 5.17}{Snp 5.17}

\begin{verse}%
“I\marginnote{1.1} am old, feeble, and pallid,” \\
\scspeaker{said Venerable \textsanskrit{Piṅgiya}, }\\
“my eyes unclear, my hearing faint. \\
Don’t let stupid me perish meanwhile; \\
explain the teaching so that I may understand \\
the giving up of rebirth and old age here.” 

“Having\marginnote{2.1} seen those stricken by forms,” \\
\scspeaker{replied the Buddha, }\\
“negligent people afflicted by forms; \\
therefore, \textsanskrit{Piṅgiya}, being diligent, \\
give up form so as not to be reborn.” 

“The\marginnote{3.1} four quarters, the intermediate directions, \\
below, and above: in these ten directions \\
there’s nothing at all in the world \\
that you’ve not seen, heard, thought, or cognized. \\
Explain the teaching so that I may understand \\
the giving up of rebirth and old age here.” 

“Observing\marginnote{4.1} people sunk in craving,” \\
\scspeaker{replied the Buddha, }\\
“tormented, mired in old age; \\
therefore, \textsanskrit{Piṅgiya}, being diligent, \\
give up craving so as not to be reborn.” 

%
\end{verse}

%
\section*{{\suttatitleacronym Snp 5.18}{\suttatitletranslation Homage to the Way to the Beyond }{\suttatitleroot Pārāyanatthutigāthā}}
\addcontentsline{toc}{section}{\tocacronym{Snp 5.18} \toctranslation{Homage to the Way to the Beyond } \tocroot{Pārāyanatthutigāthā}}
\markboth{Homage to the Way to the Beyond }{Pārāyanatthutigāthā}
\extramarks{Snp 5.18}{Snp 5.18}

This\marginnote{1.1} was said by the Buddha while staying in the land of the Magadhans at the \textsanskrit{Pāsāṇake} shrine. When requested by the sixteen brahmin devotees, he answered their questions one by one. If you understand the meaning and the teaching of each of these questions, and practice accordingly, you may go right to the far shore of old age and death. These teachings are said to lead to the far shore, which is why the name of this exposition of the teaching is “The Way to the Beyond”. 

\begin{verse}%
Ajita,\marginnote{2.1} Tissametteyya, \\
\textsanskrit{Puṇṇaka} and \textsanskrit{Mettagū}, \\
Dhotaka and Upasiva, \\
Nanda and then Hemaka, 

both\marginnote{3.1} Todeyya and Kappa, \\
and \textsanskrit{Jatukaṇṇī} the astute, \\
\textsanskrit{Bhadrāvudha} and Udaya, \\
and the brahmin Posala, \\
\textsanskrit{Mogharājā} the intelligent, \\
and \textsanskrit{Piṅgiya} the great hermit: 

they\marginnote{4.1} approached the Buddha, \\
the hermit of consummate conduct. \\
Asking their subtle questions, \\
they came to the most excellent Buddha. 

The\marginnote{5.1} Buddha answered their questions \\
in accordance with truth. \\
The sage satisfied the brahmins \\
with his answers to their questions. 

Those\marginnote{6.1} who were satisfied by the all-seer, \\
the Buddha, Kinsman of the Sun, \\
led the spiritual life in his presence, \\
the one of such splendid wisdom. 

If\marginnote{7.1} you practice in accordance \\
with each of these questions \\
as taught by the Buddha, \\
you’ll go from the near shore to the far. 

Developing\marginnote{8.1} the supreme path, \\
you’ll go from the near shore to the far. \\
This path is for going to the far shore; \\
that’s why it’s called “The Way to the Beyond”. 

%
\end{verse}

%
\section*{{\suttatitleacronym Snp 5.19}{\suttatitletranslation Preserving the Way to the Beyond }{\suttatitleroot Pārāyanānugītigāthā}}
\addcontentsline{toc}{section}{\tocacronym{Snp 5.19} \toctranslation{Preserving the Way to the Beyond } \tocroot{Pārāyanānugītigāthā}}
\markboth{Preserving the Way to the Beyond }{Pārāyanānugītigāthā}
\extramarks{Snp 5.19}{Snp 5.19}

\begin{verse}%
“I\marginnote{1.1} shall keep reciting the Way to the Beyond,” \\
\scspeaker{said Venerable \textsanskrit{Piṅgiya}, }\\
“which was taught as it was seen \\
by the immaculate one of vast intelligence. \\
He is desireless, unentangled, a spiritual giant: \\
why would he speak falsely? 

Come,\marginnote{2.1} let me extol \\
in sweet words of praise \\
the one who’s given up stains and delusions, \\
conceit and contempt. 

The\marginnote{3.1} Buddha, all-seer, dispeler of darkness, \\
has gone to world’s end, beyond all rebirths; \\
he is free of defilements, and has given up all pain, \\
the rightly-named one, brahmin, is revered by me. 

Like\marginnote{4.1} a bird that flees a little copse, \\
to roost in a forest abounding in fruit, \\
I’ve left the near-sighted behind, \\
like a swan come to a great river. 

Those\marginnote{5.1} who explained to me previously, \\
before I encountered Gotama’s teaching, \\
said ‘thus it was’ or ‘so it shall be’. \\
All that was just the testament of hearsay; \\
all that just fostered speculation. 

Alone,\marginnote{6.1} the dispeler of darkness \\
is splendid, a beacon: \\
Gotama, vast in wisdom, \\
Gotama, vast in intelligence. 

He\marginnote{7.1} is the one who taught me Dhamma, \\
visible in this very life, immediately effective, \\
the untroubled, the end of craving, \\
to which there is no compare.” 

“Why\marginnote{8.1} would you dwell apart from him \\
even for an hour, \textsanskrit{Piṅgiya}? \\
From Gotama, vast in wisdom, \\
from Gotama, vast in intelligence? 

He\marginnote{9.1} is the one who taught you Dhamma, \\
visible in this very life, immediately effective, \\
the untroubled, the end of craving, \\
to which there is no compare.” 

“I\marginnote{10.1} never dwell apart from him, \\
not even for an hour, brahmin. \\
From Gotama, vast in wisdom, \\
from Gotama, vast in intelligence. 

He\marginnote{11.1} is the one who taught me Dhamma, \\
visible in this very life, immediately effective, \\
the untroubled, the end of craving, \\
to which there is no compare. 

Being\marginnote{12.1} diligent, I see him \\
in my mind’s eye day and night. \\
I spend the night in homage to him, \\
hence I think I dwell with him. 

My\marginnote{13.1} faith and joy and intent and mindfulness \\
never stray from Gotama’s teaching. \\
I bow to whatever direction \\
the one of vast wisdom heads. 

I’m\marginnote{14.1} old and feeble, \\
so my body cannot go there, \\
but I always travel in my thoughts, \\
for my mind, brahmin, is bound to him. 

Lying\marginnote{15.1} floundering in the mud, \\
I drifted from island to island. \\
Then I saw the Buddha, \\
the undefiled one who has crossed the flood.” 

“Just\marginnote{16.1} as Vakkali was committed to faith—\\
\textsanskrit{Bhadrāvudha} and Gotama of \textsanskrit{Āḷavī} too—\\
so too you should commit to faith. \\
You will go, \textsanskrit{Piṅgiya}, beyond the domain of death.” 

“My\marginnote{17.1} confidence grew \\
when I heard the word of the sage, \\
the Buddha with veil drawn back, \\
so kind and eloquent. 

Having\marginnote{18.1} directly known all about the gods, \\
he understands all top to bottom, \\
the teacher who settles all questions \\
for those who admit their doubts. 

Unfaltering,\marginnote{19.1} unshakable; \\
that to which there is no compare. \\
For sure I will go there, I have no doubt of that. \\
Remember me as one whose mind is made up.” 

%
\end{verse}

\scendbook{The Anthology of Discourses is completed. }

%
\backmatter%
%
\chapter*{Colophon}
\addcontentsline{toc}{chapter}{Colophon}
\markboth{Colophon}{Colophon}

\section*{The Translator}

Bhikkhu Sujato was born as Anthony Aidan Best on 4/11/1966 in Perth, Western Australia. He grew up in the pleasant suburbs of Mt Lawley and Attadale alongside his sister Nicola, who was the good child. His mother, Margaret Lorraine Huntsman née Pinder, said “he’ll either be a priest or a poet”, while his father, Anthony Thomas Best, advised him to “never do anything for money”. He attended Aquinas College, a Catholic school, where he decided to become an atheist. At the University of WA he studied philosophy, aiming to learn what he wanted to do with his life. Finding that what he wanted to do was play guitar, he dropped out. His main band was named Martha’s Vineyard, which achieved modest success in the indie circuit. Then it broke up, because everyone thought they personally were reason for the success, which, oddly enough, turns out not to have been the case. 

A seemingly random encounter with a roadside joey took him to Thailand, where he entered his first meditation retreat at Wat Ram Poeng, Chieng Mai in 1992. He decided to devote himself to the Buddha’s path, and took full ordination in Wat Pa Nanachat in 1994, where his teachers were Ajahn Pasanno and Ajahn Jayasaro. In 1997 he returned to Perth to study with Ajahn Brahm at Bodhinyana Monastery. 

He spent several years practicing in seclusion in Malaysia and Thailand before establishing Santi Forest Monastery in Bundanoon, NSW, in 2003. There he was instrumental in supporting the establishment of the Theravada bhikkhuni order in Australia and advocating for women’s rights. He continues to teach in Australia and globally, with a special concern for the moral implications of climate change and other forms of environmental destruction. He has published a series of books of original and groundbreaking research on early Buddhism. 

In 2005 he founded SuttaCentral together with Rod Bucknell and John Kelly. In 2015, seeing the need for a complete, accurate, plain English translation of the Pali texts, he undertook the task, spending nearly three years in isolation on the isle of Qi Mei off the coast of the nation of Taiwan. He completed the four main \textsanskrit{Nikāyas} in 2018, and the early books of the Khuddaka \textsanskrit{Nikāya} were complete by 2021. All this work is dedicated to the public domain and is entirely free of copyright encumbrance. 

In 2019 he returned to Sydney where, together with Bhikkhu Akaliko, he established Lokanta Vihara (The Monastery at the End of the World). 

\section*{Creation Process}

Translated from the Pali. Primary source was the \textsanskrit{Mahāsaṅgīti} edition, with reference to several English translations, especially those of K.R. Norman, Bhikkhu Bodhi, and Bhikkhu \textsanskrit{Ñāṇadīpa}.

\section*{The Translation}

This translation aims to retain the directness and urgency of the \textsanskrit{Suttanipāta}. The \textsanskrit{Suttanipāta} includes verses from the earliest and latest periods within the period encompassed by the early texts, and so it covers a challenging variety of styles and themes, by turns fierce, devotional, or incisive. In several portions, most notably the \textsanskrit{Aṭṭhakavagga}, there are a range of highly specific usages that demand careful attention.

\section*{About SuttaCentral}

SuttaCentral publishes early Buddhist texts. Since 2005 we have provided root texts in Pali, Chinese, Sanskrit, Tibetan, and other languages, parallels between these texts, and translations in many modern languages. We build on the work of generations of scholars, and offer our contribution freely.

SuttaCentral is driven by volunteer contributions, and in addition we employ professional developers. We offer a sponsorship program for high quality translations from the original languages. Financial support for SuttaCentral is handled by the SuttaCentral Development Trust, a charitable trust registered in Australia.

\section*{About Bilara}

“Bilara” means “cat” in Pali, and it is the name of our Computer Assisted Translation (CAT) software. Bilara is a web app that enables translators to translate early Buddhist texts into their own language. These translations are published on SuttaCentral with the root text and translation side by side.

\section*{About SuttaCentral Editions}

The SuttaCentral Editions project makes high quality books from selected Bilara translations. These are published in formats including HTML, EPUB, PDF, and print.

If you want to print any of our Editions, please let us know and we will help prepare a file to your specifications.

%
\end{document}