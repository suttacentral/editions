\documentclass[12pt,openany]{book}%
\usepackage{lastpage}%
%
\usepackage[inner=1in, outer=1in, top=.7in, bottom=1in, papersize={6in,9in}, headheight=13pt]{geometry}
\usepackage{polyglossia}
\usepackage[12pt]{moresize}
\usepackage{soul}%
\usepackage{microtype}
\usepackage{tocbasic}
\usepackage{realscripts}
\usepackage{epigraph}%
\usepackage{setspace}%
\usepackage{sectsty}
\usepackage{fontspec}
\usepackage{marginnote}
\usepackage[bottom]{footmisc}
\usepackage{enumitem}
\usepackage{fancyhdr}
\usepackage{extramarks}
\usepackage{graphicx}
\usepackage{verse}
\usepackage{relsize}
\usepackage{etoolbox}
\usepackage[a-3u]{pdfx}

\hypersetup{
colorlinks=true,
urlcolor=black,
linkcolor=black,
citecolor=black
}

% use a small amount of tracking on small caps
\SetTracking[ spacing = {25*,166, } ]{ encoding = *, shape = sc }{ 25 }

% add a blank page
\newcommand{\blankpage}{
\newpage
\thispagestyle{empty}
\mbox{}
\newpage
}

% define languages
\setdefaultlanguage[]{english}
\setotherlanguage[script=Latin]{sanskrit}

%\usepackage{pagegrid}
%\pagegridsetup{top-left, step=.25in}

% define fonts
% use if arno sanskrit is unavailable
%\setmainfont{Gentium Plus}
%\newfontfamily\Semiboldsubheadfont[]{Gentium Plus}
%\newfontfamily\Semiboldnormalfont[]{Gentium Plus}
%\newfontfamily\Lightfont[]{Gentium Plus}
%\newfontfamily\Marginalfont[]{Gentium Plus}
%\newfontfamily\Allsmallcapsfont[RawFeature=+c2sc]{Gentium Plus}
%\newfontfamily\Noligaturefont[Renderer=Basic]{Gentium Plus}
%\newfontfamily\Noligaturecaptionfont[Renderer=Basic]{Gentium Plus}
%\newfontfamily\Fleuronfont[Ornament=1]{Gentium Plus}

% use if arno sanskrit is available. display is applied to \chapter and \part, subhead to \section and \subsection. When specifying semibold, the italic must be defined.
\setmainfont[Numbers=OldStyle]{Arno Pro}
\newfontfamily\Semibolddisplayfont[BoldItalicFont = Arno Pro Semibold Italic Display]{Arno Pro Semibold Display} %
\newfontfamily\Semiboldsubheadfont[BoldItalicFont = Arno Pro Semibold Italic Subhead]{Arno Pro Semibold Subhead}
\newfontfamily\Semiboldnormalfont[BoldItalicFont = Arno Pro Semibold Italic]{Arno Pro Semibold}
\newfontfamily\Marginalfont[RawFeature=+subs]{Arno Pro Regular}
\newfontfamily\Allsmallcapsfont[RawFeature=+c2sc]{Arno Pro}
\newfontfamily\Noligaturefont[Renderer=Basic]{Arno Pro}
\newfontfamily\Noligaturecaptionfont[Renderer=Basic]{Arno Pro Caption}

% chinese fonts
\newfontfamily\cjk{Noto Serif TC}
\newcommand*{\langlzh}[1]{\cjk{#1}\normalfont}%

% logo
\newfontfamily\Logofont{sclogo.ttf}
\newcommand*{\sclogo}[1]{\large\Logofont{#1}}

% use subscript numerals for margin notes
\renewcommand*{\marginfont}{\Marginalfont}

% ensure margin notes have consistent vertical alignment
\renewcommand*{\marginnotevadjust}{-.17em}

% use compact lists
\setitemize{noitemsep,leftmargin=1em}
\setenumerate{noitemsep,leftmargin=1em}
\setdescription{noitemsep, style=unboxed, leftmargin=0em}

% style ToC
\DeclareTOCStyleEntries[
  raggedentrytext,
  linefill=\hfill,
  pagenumberwidth=.5in,
  pagenumberformat=\normalfont,
  entryformat=\normalfont
]{tocline}{chapter,section}


  \setlength\topsep{0pt}%
  \setlength\parskip{0pt}%

% define new \centerpars command for use in ToC. This ensures centering, proper wrapping, and no page break after
\def\startcenter{%
  \par
  \begingroup
  \leftskip=0pt plus 1fil
  \rightskip=\leftskip
  \parindent=0pt
  \parfillskip=0pt
}
\def\stopcenter{%
  \par
  \endgroup
}
\long\def\centerpars#1{\startcenter#1\stopcenter}

% redefine part, so that it adds a toc entry without page number
\let\oldcontentsline\contentsline
\newcommand{\nopagecontentsline}[3]{\oldcontentsline{#1}{#2}{}}

    \makeatletter
\renewcommand*\l@part[2]{%
  \ifnum \c@tocdepth >-2\relax
    \addpenalty{-\@highpenalty}%
    \addvspace{0em \@plus\p@}%
    \setlength\@tempdima{3em}%
    \begingroup
      \parindent \z@ \rightskip \@pnumwidth
      \parfillskip -\@pnumwidth
      {\leavevmode
       \setstretch{.85}\large\scshape\centerpars{#1}\vspace*{-1em}\llap{#2}}\par
       \nobreak
         \global\@nobreaktrue
         \everypar{\global\@nobreakfalse\everypar{}}%
    \endgroup
  \fi}
\makeatother

\makeatletter
\def\@pnumwidth{2em}
\makeatother

% define new sectioning command, which is only used in volumes where the pannasa is found in some parts but not others, especially in an and sn

\newcommand*{\pannasa}[1]{\clearpage\thispagestyle{empty}\begin{center}\vspace*{14em}\setstretch{.85}\huge\itshape\scshape\MakeLowercase{#1}\end{center}}

    \makeatletter
\newcommand*\l@pannasa[2]{%
  \ifnum \c@tocdepth >-2\relax
    \addpenalty{-\@highpenalty}%
    \addvspace{.5em \@plus\p@}%
    \setlength\@tempdima{3em}%
    \begingroup
      \parindent \z@ \rightskip \@pnumwidth
      \parfillskip -\@pnumwidth
      {\leavevmode
       \setstretch{.85}\large\itshape\scshape\lowercase{\centerpars{#1}}\vspace*{-1em}\llap{#2}}\par
       \nobreak
         \global\@nobreaktrue
         \everypar{\global\@nobreakfalse\everypar{}}%
    \endgroup
  \fi}
\makeatother

% don't put page number on first page of toc (relies on etoolbox)
\patchcmd{\chapter}{plain}{empty}{}{}

% global line height
\setstretch{1.05}

% allow linebreak after em-dash
\catcode`\—=13
\protected\def—{\unskip\textemdash\allowbreak}

% style headings with secsty. chapter and section are defined per-edition
\partfont{\setstretch{.85}\normalfont\centering\textsc}
\subsectionfont{\setstretch{.85}\Semiboldsubheadfont}%
\subsubsectionfont{\setstretch{.85}\Semiboldnormalfont}

% style elements of suttatitle
\newcommand*{\suttatitleacronym}[1]{\smaller[2]{#1}\vspace*{.3em}}
\newcommand*{\suttatitletranslation}[1]{\linebreak{#1}}
\newcommand*{\suttatitleroot}[1]{\linebreak\smaller[2]\itshape{#1}}

\DeclareTOCStyleEntries[
  indent=3.3em,
  dynindent,
  beforeskip=.2em plus -2pt minus -1pt,
]{tocline}{section}

\DeclareTOCStyleEntries[
  indent=0em,
  dynindent,
  beforeskip=.4em plus -2pt minus -1pt,
]{tocline}{chapter}

\newcommand*{\tocacronym}[1]{\hspace*{-3.3em}{#1}\quad}
\newcommand*{\toctranslation}[1]{#1}
\newcommand*{\tocroot}[1]{(\textit{#1})}
\newcommand*{\tocchapterline}[1]{\bfseries\itshape{#1}}


% redefine paragraph and subparagraph headings to not be inline
\makeatletter
% Change the style of paragraph headings %
\renewcommand\paragraph{\@startsection{paragraph}{4}{\z@}%
            {-2.5ex\@plus -1ex \@minus -.25ex}%
            {1.25ex \@plus .25ex}%
            {\noindent\Semiboldnormalfont\normalsize}}

% Change the style of subparagraph headings %
\renewcommand\subparagraph{\@startsection{subparagraph}{5}{\z@}%
            {-2.5ex\@plus -1ex \@minus -.25ex}%
            {1.25ex \@plus .25ex}%
            {\noindent\Semiboldnormalfont\small}}
\makeatother

% use etoolbox to suppress page numbers on \part
\patchcmd{\part}{\thispagestyle{plain}}{\thispagestyle{empty}}
  {}{\errmessage{Cannot patch \string\part}}

% and to reduce margins on quotation
\patchcmd{\quotation}{\rightmargin}{\leftmargin 1.2em \rightmargin}{}{}
\AtBeginEnvironment{quotation}{\small}

% titlepage
\newcommand*{\titlepageTranslationTitle}[1]{{\begin{center}\begin{large}{#1}\end{large}\end{center}}}
\newcommand*{\titlepageCreatorName}[1]{{\begin{center}\begin{normalsize}{#1}\end{normalsize}\end{center}}}

% halftitlepage
\newcommand*{\halftitlepageTranslationTitle}[1]{\setstretch{2.5}{\begin{Huge}\uppercase{\so{#1}}\end{Huge}}}
\newcommand*{\halftitlepageTranslationSubtitle}[1]{\setstretch{1.2}{\begin{large}{#1}\end{large}}}
\newcommand*{\halftitlepageFleuron}[1]{{\begin{large}\Fleuronfont{{#1}}\end{large}}}
\newcommand*{\halftitlepageByline}[1]{{\begin{normalsize}\textit{{#1}}\end{normalsize}}}
\newcommand*{\halftitlepageCreatorName}[1]{{\begin{LARGE}{\textsc{#1}}\end{LARGE}}}
\newcommand*{\halftitlepageVolumeNumber}[1]{{\begin{normalsize}{\Allsmallcapsfont{\textsc{#1}}}\end{normalsize}}}
\newcommand*{\halftitlepageVolumeAcronym}[1]{{\begin{normalsize}{#1}\end{normalsize}}}
\newcommand*{\halftitlepageVolumeTranslationTitle}[1]{{\begin{Large}{\textsc{#1}}\end{Large}}}
\newcommand*{\halftitlepageVolumeRootTitle}[1]{{\begin{normalsize}{\Allsmallcapsfont{\textsc{\itshape #1}}}\end{normalsize}}}
\newcommand*{\halftitlepagePublisher}[1]{{\begin{large}{\Noligaturecaptionfont\textsc{#1}}\end{large}}}

% epigraph
\renewcommand{\epigraphflush}{center}
\renewcommand*{\epigraphwidth}{.85\textwidth}
\newcommand*{\epigraphTranslatedTitle}[1]{\vspace*{.5em}\footnotesize\textsc{#1}\\}%
\newcommand*{\epigraphRootTitle}[1]{\footnotesize\textit{#1}\\}%
\newcommand*{\epigraphReference}[1]{\footnotesize{#1}}%

% custom commands for html styling classes
\newcommand*{\scnamo}[1]{\begin{center}\textit{#1}\end{center}}
\newcommand*{\scendsection}[1]{\begin{center}\textit{#1}\end{center}}
\newcommand*{\scendsutta}[1]{\begin{center}\textit{#1}\end{center}}
\newcommand*{\scendbook}[1]{\begin{center}\uppercase{#1}\end{center}}
\newcommand*{\scendkanda}[1]{\begin{center}\textbf{#1}\end{center}}
\newcommand*{\scend}[1]{\begin{center}\textit{#1}\end{center}}
\newcommand*{\scuddanaintro}[1]{\textit{#1}}
\newcommand*{\scendvagga}[1]{\begin{center}\textbf{#1}\end{center}}
\newcommand*{\scrule}[1]{\textbf{#1}}
\newcommand*{\scadd}[1]{\textit{#1}}
\newcommand*{\scevam}[1]{\textsc{#1}}
\newcommand*{\scspeaker}[1]{\hspace{2em}\textit{#1}}
\newcommand*{\scbyline}[1]{\begin{flushright}\textit{#1}\end{flushright}\bigskip}

% custom command for thematic break = hr
\newcommand*{\thematicbreak}{\begin{center}\rule[.5ex]{6em}{.4pt}\begin{normalsize}\quad\Fleuronfont{•}\quad\end{normalsize}\rule[.5ex]{6em}{.4pt}\end{center}}

% manage and style page header and footer. "fancy" has header and footer, "plain" has footer only

\pagestyle{fancy}
\fancyhf{}
\fancyfoot[RE,LO]{\thepage}
\fancyfoot[LE,RO]{\footnotesize\lastleftxmark}
\fancyhead[CE]{\setstretch{.85}\Noligaturefont\MakeLowercase{\textsc{\firstrightmark}}}
\fancyhead[CO]{\setstretch{.85}\Noligaturefont\MakeLowercase{\textsc{\firstleftmark}}}
\renewcommand{\headrulewidth}{0pt}
\fancypagestyle{plain}{ %
\fancyhf{} % remove everything
\fancyfoot[RE,LO]{\thepage}
\fancyfoot[LE,RO]{\footnotesize\lastleftxmark}
\renewcommand{\headrulewidth}{0pt}
\renewcommand{\footrulewidth}{0pt}}

% style footnotes
\setlength{\skip\footins}{1em}

\makeatletter
\newcommand{\@makefntextcustom}[1]{%
    \parindent 0em%
    \thefootnote.\enskip #1%
}
\renewcommand{\@makefntext}[1]{\@makefntextcustom{#1}}
\makeatother

% hang quotes (requires microtype)
\microtypesetup{
  protrusion = true,
  expansion  = true,
  tracking   = true,
  factor     = 1000,
  patch      = all,
  final
}

% Custom protrusion rules to allow hanging punctuation
\SetProtrusion
{ encoding = *}
{
% char   right left
  {-} = {    , 500 },
  % Double Quotes
  \textquotedblleft
      = {1000,     },
  \textquotedblright
      = {    , 1000},
  \quotedblbase
      = {1000,     },
  % Single Quotes
  \textquoteleft
      = {1000,     },
  \textquoteright
      = {    , 1000},
  \quotesinglbase
      = {1000,     }
}

% make latex use actual font em for parindent, not Computer Modern Roman
\AtBeginDocument{\setlength{\parindent}{1em}}%
%

% Default values; a bit sloppier than normal
\tolerance 1414
\hbadness 1414
\emergencystretch 1.5em
\hfuzz 0.3pt
\clubpenalty = 10000
\widowpenalty = 10000
\displaywidowpenalty = 10000
\hfuzz \vfuzz
 \raggedbottom%

\title{Anthology of Discourses}
\author{Bhikkhu Sujato}
\date{}%
% define a different fleuron for each edition
\newfontfamily\Fleuronfont[Ornament=11]{Arno Pro}

% Define heading styles per edition for chapter and section. Suttatitle can be either of these, depending on the volume. 

\let\oldfrontmatter\frontmatter
\renewcommand{\frontmatter}{%
\chapterfont{\setstretch{.85}\normalfont\centering}%
\sectionfont{\setstretch{.85}\Semiboldsubheadfont}%
\oldfrontmatter}

\let\oldmainmatter\mainmatter
\renewcommand{\mainmatter}{%
\chapterfont{\setstretch{.85}\normalfont\centering}%
\sectionfont{\setstretch{.85}\normalfont\centering}%
\oldmainmatter}

\let\oldbackmatter\backmatter
\renewcommand{\backmatter}{%
\chapterfont{\setstretch{.85}\normalfont\centering}%
\sectionfont{\setstretch{.85}\Semiboldsubheadfont}%
\oldbackmatter}
%
%
\begin{document}%
\normalsize%
\frontmatter%
\setlength{\parindent}{0cm}

\pagestyle{empty}

\maketitle

\blankpage%
\begin{center}

\vspace*{2.2em}

\halftitlepageTranslationTitle{Anthology of Discourses}

\vspace*{1em}

\halftitlepageTranslationSubtitle{A refreshing translation of the Suttanipāta}

\vspace*{2em}

\halftitlepageFleuron{•}

\vspace*{2em}

\halftitlepageByline{translated and introduced by}

\vspace*{.5em}

\halftitlepageCreatorName{Bhikkhu Sujato}

\vspace*{4em}

\halftitlepageVolumeNumber{}

\smallskip

\halftitlepageVolumeAcronym{Snp}

\smallskip

\halftitlepageVolumeTranslationTitle{}

\smallskip

\halftitlepageVolumeRootTitle{}

\vspace*{\fill}

\sclogo{0}
 \halftitlepagePublisher{SuttaCentral}

\end{center}

\newpage
%
\setstretch{1.05}

\begin{footnotesize}

\textit{Anthology of Discourses} is a translation of the Suttanipāta by Bhikkhu Sujato.

\medskip

Creative Commons Zero (CC0)

To the extent possible under law, Bhikkhu Sujato has waived all copyright and related or neighboring rights to \textit{Anthology of Discourses}.

\medskip

This work is published from Australia.

\begin{center}
\textit{This translation is an expression of an ancient spiritual text that has been passed down by the Buddhist tradition for the benefit of all sentient beings. It is dedicated to the public domain via Creative Commons Zero (CC0). You are encouraged to copy, reproduce, adapt, alter, or otherwise make use of this translation. The translator respectfully requests that any use be in accordance with the values and principles of the Buddhist community.}
\end{center}

\medskip

\begin{description}
    \item[Web publication date] 2021
    \item[This edition] 2022-11-03 09:42:05
    \item[Publication type] paperback
    \item[Edition] ed5
    \item[Number of volumes] 1
    \item[Publication ISBN] 978-1-76132-005-7
    \item[Publication URL] https://suttacentral.net/editions/snp/en/sujato
    \item[Source URL] https://github.com/suttacentral/bilara-data/tree/published/translation/en/sujato/sutta/kn/snp
    \item[Publication number] scpub17
\end{description}

\medskip

Published by SuttaCentral

\medskip

\textit{SuttaCentral,\\
c/o Alwis \& Alwis Pty Ltd\\
Kaurna Country,\\
Suite 12,\\
198 Greenhill Road,\\
Eastwood,\\
SA 5063,\\
Australia}

\end{footnotesize}

\newpage

\setlength{\parindent}{1.5em}%%
\newpage

\vspace*{\fill}

\begin{center}
\epigraph{The little creeks flow on babbling,\\
while silent flow the great rivers.}
{
\epigraphTranslatedTitle{About \textsanskrit{Nālaka}}
\epigraphRootTitle{\textsanskrit{Nālakasutta}}
\epigraphReference{Sutta \textsanskrit{Nipāta} 3.11:42}
}
\end{center}

\vspace*{2in}

\vspace*{\fill}

\blankpage%

\setlength{\parindent}{1em}
%
\tableofcontents
\newpage
\pagestyle{fancy}
%
\chapter*{The SuttaCentral Editions Series}
\addcontentsline{toc}{chapter}{The SuttaCentral Editions Series}
\markboth{The SuttaCentral Editions Series}{The SuttaCentral Editions Series}

Since 2005 SuttaCentral has provided access to the texts, translations, and parallels of early Buddhist texts. In 2018 we started creating and publishing our own translations of these seminal spiritual classics. The “Editions” series now makes selected translations available as books in various forms, including print, PDF, and EPUB.

Editions are selected from our most complete, well-crafted, and reliable translations. They aim to bring these texts to a wider audience in forms that reward mindful reading. Care is taken with every detail of the production, and we aim to meet or exceed professional best standards in every way. These are the core scriptures underlying the entire Buddhist tradition, and we believe that they deserve to be preserved and made available in highest quality without compromise.

SuttaCentral is a charitable organization. Our work is accomplished by volunteers and through the generosity of our donors. Everything we create is offered to all of humanity free of any copyright or licensing restrictions. 

%
\chapter*{Preface to the Sutta \textsanskrit{Nipāta}}
\addcontentsline{toc}{chapter}{Preface to the Sutta \textsanskrit{Nipāta}}
\markboth{Preface to the Sutta \textsanskrit{Nipāta}}{Preface to the Sutta \textsanskrit{Nipāta}}

To be written

%
\chapter*{Introduction to Anthology of Discourses}
\addcontentsline{toc}{chapter}{Introduction to Anthology of Discourses}
\markboth{Introduction to Anthology of Discourses}{Introduction to Anthology of Discourses}

\scbyline{Bhikkhu Sujato, 2022}

To be written

%
\chapter*{Acknowledgements}
\addcontentsline{toc}{chapter}{Acknowledgements}
\markboth{Acknowledgements}{Acknowledgements}

I remember with gratitude all those from whom I have learned the Dhamma, especially Ajahn Brahm and Bhikkhu Bodhi, the two monks who more than anyone else showed me the depth, meaning, and practical value of the Suttas.

Special thanks to Dustin and Keiko Cheah and family, who sponsored my stay in Qi Mei while I made this translation.

Thanks also for Blake Walshe, who provided essential software support for my translation work.

Throughout the process of translation, I have frequently sought feedback and suggestions from the community on the SuttaCentral community on our forum, “Discuss and Discover”. I want to thank all those who have made suggestions and contributed to my understanding, as well as to the moderators who have made the forum possible. My thinking around the economics of meat-eating in the Āmagandhasutta was prompted by the forum users Bhikkhu Khemaratana and Prajnadeva. A special thanks is due to \textsanskrit{Sabbamittā}, a true friend of all, who has tirelessly and precisely checked my work. 

Finally my everlasting thanks to all those people, far too many to mention, who have supported SuttaCentral, and those who have supported my life as a monastic. None of this would be possible without you.

%
\mainmatter%
\pagestyle{fancy}%
\addtocontents{toc}{\let\protect\contentsline\protect\nopagecontentsline}
\part*{Anthology of Discourses}
\addcontentsline{toc}{part}{Anthology of Discourses}
\markboth{}{}
\addtocontents{toc}{\let\protect\contentsline\protect\oldcontentsline}

%
\addtocontents{toc}{\let\protect\contentsline\protect\nopagecontentsline}
\chapter*{The Serpent Chapter}
\addcontentsline{toc}{chapter}{\tocchapterline{The Serpent Chapter}}
\addtocontents{toc}{\let\protect\contentsline\protect\oldcontentsline}

%
\section*{{\suttatitleacronym Snp 1.1}{\suttatitletranslation The Snake }{\suttatitleroot Uragasutta}}
\addcontentsline{toc}{section}{\tocacronym{Snp 1.1} \toctranslation{The Snake } \tocroot{Uragasutta}}
\markboth{The Snake }{Uragasutta}
\extramarks{Snp 1.1}{Snp 1.1}

\begin{verse}%
When\marginnote{1.1} anger surges, they drive it out, \\
as with medicine a snake’s spreading venom. \\
Such a mendicant sheds this world and the next, \\
as a snake its old worn-out skin. 

They’ve\marginnote{2.1} cut off greed entirely, \\
like a lotus plucked flower and stalk. \\
Such a mendicant sheds this world and the next, \\
as a snake its old worn-out skin. 

They’ve\marginnote{3.1} cut off craving entirely, \\
drying up that swift-flowing stream. \\
Such a mendicant sheds this world and the next, \\
as a snake its old worn-out skin. 

They’ve\marginnote{4.1} swept away conceit entirely, \\
as a fragile bridge of reeds by a great flood. \\
Such a mendicant sheds this world and the next, \\
as a snake its old worn-out skin. 

In\marginnote{5.1} future lives they find no substance, \\
as an inspector of fig trees finds no flower. \\
Such a mendicant sheds this world and the next, \\
as a snake its old worn-out skin. 

They\marginnote{6.1} hide no anger within, \\
gone beyond any kind of existence. \\
Such a mendicant sheds this world and the next, \\
as a snake its old worn-out skin. 

Their\marginnote{7.1} mental vibrations are cleared away,\footnote{Both Bodhi and Norman implicitly accept the commentarial reading here, i.e. that this refers to “wrong thoughts”. However the identical line at Ud 6.7:4.1 clearly refers to \textsanskrit{jhāna}. Both also follow the commentary in reading \textit{\textsanskrit{vidhūpitā}} as “burning up”, but \textit{\textsanskrit{vidhūpa}} refers to clearing the mist, smoke, or incense. } \\
internally clipped off entirely. \\
Such a mendicant sheds this world and the next, \\
as a snake its old worn-out skin. 

They\marginnote{8.1} have not run too far nor run back, \\
but have gone beyond all this proliferation. \\
Such a mendicant sheds this world and the next, \\
as a snake its old worn-out skin. 

They\marginnote{9.1} have not run too far nor run back, \\
for they know that nothing in the world \\>is what it seems.\footnote{Bodhi, Norman, and Thanissaro all render \textit{vitatha} as “unreal”. Normally it’s used in the context of “true” speech or the “reality” or “unerringness” of say dependent origination. To say the whole world is “unreal” seems overly idealistic for early Buddhism, so I translate in accord with the apparent sense. } \\
Such a mendicant sheds this world and the next, \\
as a snake its old worn-out skin. 

They\marginnote{10.1} have not run too far nor run back, \\
knowing nothing is what it seems, free of greed. \\
Such a mendicant sheds this world and the next, \\
as a snake its old worn-out skin. 

They\marginnote{11.1} have not run too far nor run back, \\
knowing nothing is what it seems, free of lust. \\
Such a mendicant sheds this world and the next, \\
as a snake its old worn-out skin. 

They\marginnote{12.1} have not run too far nor run back, \\
knowing nothing is what it seems, free of hate. \\
Such a mendicant sheds this world and the next, \\
as a snake its old worn-out skin. 

They\marginnote{13.1} have not run too far nor run back, \\
knowing nothing is what it seems, free of delusion. \\
Such a mendicant sheds this world and the next, \\
as a snake its old worn-out skin. 

They\marginnote{14.1} have no underlying tendencies at all, \\
and are rid of unskillful roots, \\
Such a mendicant sheds this world and the next, \\
as a snake its old worn-out skin. 

They\marginnote{15.1} have nothing born of distress at all, \\
that might cause them to come back to this world. \\
Such a mendicant sheds this world and the next, \\
as a snake its old worn-out skin. 

They\marginnote{16.1} have nothing born of entanglement at all, \\
that would shackle them to a new life. \\
Such a mendicant sheds this world and the next, \\
as a snake its old worn-out skin. 

They’ve\marginnote{17.1} given up the five hindrances, \\
untroubled, rid of doubt, free of thorns. \\
Such a mendicant sheds this world and the next, \\
as a snake its old worn-out skin. 

%
\end{verse}

%
\section*{{\suttatitleacronym Snp 1.2}{\suttatitletranslation With Dhaniya the Cowherd }{\suttatitleroot Dhaniyasutta}}
\addcontentsline{toc}{section}{\tocacronym{Snp 1.2} \toctranslation{With Dhaniya the Cowherd } \tocroot{Dhaniyasutta}}
\markboth{With Dhaniya the Cowherd }{Dhaniyasutta}
\extramarks{Snp 1.2}{Snp 1.2}

\begin{verse}%
“I’ve\marginnote{1.1} boiled my rice and drawn my milk,” \\
\scspeaker{said Dhaniya the cowherd, }\\
“I stay with my family along the bank of the \textsanskrit{Mahī}. \\
My hut is roofed, my fire kindled: \\
so rain, sky, if you wish.” 

“I\marginnote{2.1} boil not with anger and have drawn out hard-heartedness,” \\
\scspeaker{said the Buddha, }\\
“I stay for one night along the bank of the \textsanskrit{Mahī}. \\
My hut is wide open, my fire is quenched: \\
so rain, sky, if you wish.” 

“No\marginnote{3.1} gadflies or mosquitoes are found,” \\
\scspeaker{said Dhaniya, }\\
“cows graze on the lush meadow grass. \\
They get by even when the rain comes: \\
so rain, sky, if you wish.” 

“I\marginnote{4.1} bound a raft and made it well,” \\
\scspeaker{said the Buddha,\footnote{Norman points out that this verse is not a proper response to the previous and suggests that a verse has been lost. I agree, but I disagree there is a contradiction in the sense. Rather, the verse tells a story over time. I translate to bring out the meaning. } }\\
“and with it I crossed over, went beyond, and dispelled the flood. \\
Now I have no need for a raft: \\
so rain, sky, if you wish.” 

“My\marginnote{5.1} wife is obedient, not wanton,” \\
\scspeaker{said Dhaniya, }\\
“long have we lived together happily. \\
I hear nothing bad about her: \\
so rain, sky, if you wish.” 

“My\marginnote{6.1} mind is obedient and freed,” \\
\scspeaker{said the Buddha, }\\
“long nurtured and well-tamed. \\
Nothing bad is found in me: \\
so rain, sky, if you wish.” 

“I\marginnote{7.1} am self-employed,” \\
\scspeaker{said Dhaniya, }\\
“and my healthy children likewise.\footnote{I’m not convinced by the commentary’s “not separated”. It doesn’t seem to occur in this meaning elsewhere, and lacks a connection with the opening line. Surely the whole verse must be about the children, which requires that there be a comparison made with the first line. The point is, I think, that the children follow the father and have become self-sufficient. This also echoes better with the Buddha’s response. } \\
I hear nothing bad about them: \\
so rain, sky, if you wish.” 

“I\marginnote{8.1} am no-one’s lackey,” \\
\scspeaker{said the Buddha, }\\
“with what I have earned I wander the world. \\
I have no need for wages: \\
so rain, sky, if you wish.” 

“I\marginnote{9.1} have heifers and sucklings,” \\
\scspeaker{said Dhaniya, }\\
“cows in calf and breeding cows. \\
I’ve also got a bull, leader of the herd here: \\
so rain, sky, if you wish.” 

“I\marginnote{10.1} have no heifers or sucklings,” \\
\scspeaker{said the Buddha, }\\
“no cows in calf or breeding cows. \\
I haven’t got a bull, leader of the herd here: \\
so rain, sky, if you wish.” 

“The\marginnote{11.1} stakes are driven in, unshakable,” \\
\scspeaker{said Dhaniya, }\\
“The grass halters are new and well-woven, \\
not even the sucklings can break them: \\
so rain, sky, if you wish.” 

“Like\marginnote{12.1} a bull I broke the bonds,” \\
\scspeaker{said the Buddha, }\\
“like an elephant I snapped the vine. \\
I will never lie in a womb again: \\
so rain, sky, if you wish.” 

Right\marginnote{13.1} then a thundercloud rained down, \\
soaking the uplands and valleys. \\
Hearing the sky rain down, \\
Dhaniya said this: 

“It\marginnote{14.1} is no small gain for us \\
that we have seen the Buddha. \\
We come to you for refuge, Seer. \\
O great sage, please be our Teacher. 

My\marginnote{15.1} wife and I, obedient, \\
shall lead the spiritual life under the Holy One\footnote{Norman and Bodhi read the middle form \textit{\textsanskrit{carāmase}} as imperative, apparently, but that doesn’t make sense to me; in this context optative would make more sense, but I am not aware of this form. Rather, I take it as the present indicative in the sense of definite future. } \\
Gone beyond birth and death, \\
we shall make an end of suffering.” 

“Your\marginnote{16.1} children bring you delight!” \\
\scspeaker{said \textsanskrit{Māra} the Wicked, }\\
“Your cattle also bring you delight! \\
For attachments are a man’s delight; \\
without attachments there’s no delight.” 

“Your\marginnote{17.1} children bring you sorrow,” \\
\scspeaker{said the Buddha, }\\
“Your cattle also bring you sorrow. \\
For attachments are a man’s sorrow; \\
without attachments there are no sorrows.” 

%
\end{verse}

%
\section*{{\suttatitleacronym Snp 1.3}{\suttatitletranslation The Rhinoceros Horn }{\suttatitleroot Khaggavisāṇasutta}}
\addcontentsline{toc}{section}{\tocacronym{Snp 1.3} \toctranslation{The Rhinoceros Horn } \tocroot{Khaggavisāṇasutta}}
\markboth{The Rhinoceros Horn }{Khaggavisāṇasutta}
\extramarks{Snp 1.3}{Snp 1.3}

\begin{verse}%
When\marginnote{1.1} you’ve laid down arms toward all creatures, \\
not harming even a single one, \\
don’t wish for a child, let alone a companion: \\
live alone like a rhino’s horn. 

Those\marginnote{2.1} with close relationships have affection, \\
following which this pain arises. \\
Seeing this danger born of affection, \\
live alone like a rhino’s horn. 

When\marginnote{3.1} feelings for friends and loved ones \\
are tied up in selfish love, you miss out on the goal. \\
Seeing this peril in intimacy, \\
live alone like a rhino’s horn. 

As\marginnote{4.1} a spreading bamboo gets entangled, \\
so does concern for partners and children. \\
Like a bamboo shoot unimpeded, \\
live alone like a rhino’s horn. 

As\marginnote{5.1} a wild deer loose in the forest \\
grazes wherever it wants, \\
a smart person looking for freedom would \\
live alone like a rhino’s horn. 

When\marginnote{6.1} among friends, whether staying in place \\
or going on a journey, you’re always on call. \\
Looking for the uncoveted freedom, \\
live alone like a rhino’s horn. 

Among\marginnote{7.1} friends you have fun and games, \\
and for children you are full of love. \\
Though loathe to depart from those you hold dear, \\
live alone like a rhino’s horn. 

At\marginnote{8.1} ease in any quarter, unresisting, \\
content with whatever comes your way; \\
prevailing over adversities, dauntless, \\
live alone like a rhino’s horn. 

Even\marginnote{9.1} some renunciates are hard to please, \\
as are some layfolk dwelling at home. \\
Don’t worry about others’ children, \\
live alone like a rhino’s horn. 

Having\marginnote{10.1} shed the marks of the home life, \\
like the fallen leaves of the Shady Orchid Tree; \\
having cut the bonds of the home life, a hero would \\
live alone like a rhino’s horn. 

If\marginnote{11.1} you find an alert companion, \\
a wise and virtuous friend, \\
then, overcoming all adversities, \\
wander with them, joyful and mindful. 

If\marginnote{12.1} you find no alert companion, \\
no wise and virtuous friend, \\
then, like a king who flees his conquered realm, \\
wander alone like a tusker in the wilds. 

Clearly\marginnote{13.1} we praise the blessing of a friend, \\
it’s good to be with friends your equal or better. \\
but failing to find them, eating blamelessly, \\
live alone like a rhino’s horn. 

Though\marginnote{14.1} made of shining gold, well-finished by a smith, \\
when two bracelets share the same arm \\
they clash up against each other. Seeing this, \\
live alone like a rhino’s horn. 

Thinking,\marginnote{15.1} “So too, if I had a partner, \\
there’d be flattery or curses.”\footnote{The exact nuance of this line is hard to pin down. Later Pali texts equate \textit{\textsanskrit{abhilāpa}} with “naming”, but it doesn’t seem to occur in any early texts with this sense. However \textit{\textsanskrit{lapanā}} is used in the sense of “flattery” in eg. MN 117:29. } \\
Seeing this peril in the future, \\
live alone like a rhino’s horn. 

Sensual\marginnote{16.1} pleasures are diverse, sweet, delightful, \\
appearing in disguise they disturb the mind. \\
Seeing danger in the many kinds of sensual stimulation, \\
live alone like a rhino’s horn. 

This\marginnote{17.1} is a calamity, a boil, a disaster, \\
an illness, a dart, and a danger for me. \\
Seeing this peril in sensuality, \\
live alone like a rhino’s horn. 

Heat\marginnote{18.1} and cold, hunger and thirst, \\
wind and sun, flies and snakes: \\
having put up with all these things, \\
live alone like a rhino’s horn. 

As\marginnote{19.1} a full-grown elephant, lotus-eating, magnificent, \\
forsaking the herd,\footnote{\textit{\textsanskrit{Padumī}} is uncertain, see Bodhi’s note 475. But other Sanskrit sources attribute the epithet to the elephant’s predeliction to bathing and eating in lotus ponds, which would seem much more likely. } \\
stays where it wants in the forest, \\
live alone like a rhino’s horn. 

It’s\marginnote{20.1} impossible for one who delights in company \\
to experience even temporary freedom. \\
Heeding the speech of the Kinsman of the Sun, \\
live alone like a rhino’s horn. 

Thinking,\marginnote{21.1} “I am one who has left warped views behind, \\
has reached the sure way, has gained the path, \\
has given rise to knowledge, and needs no-one to guide me”, \\
live alone like a rhino’s horn. 

No\marginnote{22.1} greed, no guile, no thirst, no slur, \\
dross and delusion is smelted off;\footnote{The reference is to the smelting of gold, AN 3.101:1.9. } \\
free of hoping for anything in the world, \\
live alone like a rhino’s horn. 

Avoid\marginnote{23.1} a wicked companion, \\
blind to the good, habitually immoral. \\
One ought not befriend the heedless and hankering, but \\
live alone like a rhino’s horn. 

Spend\marginnote{24.1} time with a learned expert who has memorized the teachings, \\
an eloquent and uplifting friend. \\
When you understand the meanings and have dispelled doubt, \\
live alone like a rhino’s horn. 

When\marginnote{25.1} you realize that worldly fun and games \\
and pleasure are unsatisfying, disregarding them, \\
as one unadorned, a speaker of truth, \\
live alone like a rhino’s horn. 

Children,\marginnote{26.1} partner, father, mother, \\
wealth and grain and relatives: \\
having given up sensual pleasures to this extent, \\
live alone like a rhino’s horn. 

“This\marginnote{27.1} is a snare. Here there’s hardly any happiness, \\
little gratification, and it’s full of drawbacks. \\
It’s a hook.” Knowing this, a thoughtful person would \\
live alone like a rhino’s horn. 

Having\marginnote{28.1} burst apart the fetters, \\
like a fish that tears the net and swims free, \\
or a fire not returning to ground it has burned, \\
live alone like a rhino’s horn. 

Eyes\marginnote{29.1} downcast, not footloose, \\
senses guarded, mind protected, \\
uncorrupted, not burning with desire, \\
live alone like a rhino’s horn. 

Having\marginnote{30.1} shed the marks of the home life, \\
like the fallen leaves of the Shady Orchid Tree, \\
and gone forth in the ocher robe, \\
live alone like a rhino’s horn. 

Not\marginnote{31.1} wanton, nor rousing greed for tastes, \\
providing for no other, wandering indiscriminately for alms, \\
not attached to this family or that, \\
live alone like a rhino’s horn. 

When\marginnote{32.1} you’ve given up five mental obstacles, \\
and expelled all corruptions, \\
and cut off affection and hate, being independent, \\
live alone like a rhino’s horn. 

When\marginnote{33.1} you’ve put pleasure and pain behind you, \\
and former happiness and sadness, \\
and gained equanimity serene and pure, \\
live alone like a rhino’s horn. 

With\marginnote{34.1} energy roused to reach the ultimate goal, \\
not sluggish in mind or lazy, \\
vigorous, strong and powerful, \\
live alone like a rhino’s horn. 

Not\marginnote{35.1} neglecting retreat and absorption, \\
always living in line with the teachings, \\
comprehending the danger in rebirths, \\
live alone like a rhino’s horn. 

One\marginnote{36.1} whose aim is the ending of craving—\\
diligent, clever, learned, mindful, resolute—\\
who has assessed the teaching and is bound for awakening, should \\
live alone like a rhino’s horn. 

Like\marginnote{37.1} a lion not startled by sounds, \\
like wind not caught in a net, \\
like water not sticking to a lotus, \\
live alone like a rhino’s horn. 

Like\marginnote{38.1} the fierce-fanged lion, king of beasts, \\
who wanders as victor and master, \\
you should frequent remote lodgings, and \\
live alone like a rhino’s horn. 

In\marginnote{39.1} time, cultivate freedom through \\
love, compassion, rejoicing, and equanimity. \\
Not upset by anything in the world, \\
live alone like a rhino’s horn. 

Having\marginnote{40.1} given up greed, hate, and delusion, \\
having burst apart the fetters, \\
unafraid at the end of life, \\
live alone like a rhino’s horn. 

They\marginnote{41.1} befriend you and serve you for their own sake; \\
these days it’s hard to find friends lacking ulterior motive. \\
Impure folk cleverly profit themselves—\\
live alone like a rhino’s horn. 

%
\end{verse}

%
\section*{{\suttatitleacronym Snp 1.4}{\suttatitletranslation With Bhāradvāja the Farmer }{\suttatitleroot Kasibhāradvājasutta}}
\addcontentsline{toc}{section}{\tocacronym{Snp 1.4} \toctranslation{With Bhāradvāja the Farmer } \tocroot{Kasibhāradvājasutta}}
\markboth{With Bhāradvāja the Farmer }{Kasibhāradvājasutta}
\extramarks{Snp 1.4}{Snp 1.4}

\scevam{So\marginnote{1.1} I have heard. }At one time the Buddha was staying in the land of the Magadhans in the Southern Hills near the brahmin village of \textsanskrit{Ekanāḷa}. Now at that time the brahmin \textsanskrit{Bhāradvāja} the Farmer had harnessed around five hundred plows, it being the season for sowing. Then the Buddha robed up in the morning and, taking his bowl and robe, went to where \textsanskrit{Bhāradvāja} the Farmer was working. Now at that time \textsanskrit{Bhāradvāja} the Farmer was distributing food. Then the Buddha went to where the distribution was taking place and stood to one side. 

\textsanskrit{Bhāradvāja}\marginnote{2.1} the Farmer saw him standing for alms and said to him, “I plough and sow, ascetic, and then I eat. You too should plough and sow, then you may eat.” 

“I\marginnote{3.1} too plough and sow, brahmin, and then I eat.” “I don’t see Master Gotama with a yoke or plow or plowshare or goad or oxen, yet he says: ‘I too plough and sow, brahmin, and then I eat.’” 

Then\marginnote{4.1} \textsanskrit{Bhāradvāja} the Farmer addressed the Buddha in verse: 

\begin{verse}%
“You\marginnote{5.1} claim to be a farmer, \\
but I don’t see you farming. \\
Tell me your farming when asked, \\
so I can recognize your farming.” 

“Faith\marginnote{6.1} is my seed, austerity my rain, \\
and wisdom is my yoke and plough. \\
Conscience is my pole, mind my strap, \\
mindfulness my plowshare and goad. 

Guarded\marginnote{7.1} in body and speech, \\
I restrict my intake of food. \\
I use truth as my scythe, \\
and gentleness is my release. 

Energy\marginnote{8.1} is my beast of burden, \\
transporting me to a place of sanctuary. \\
It goes without turning back \\
where there is no sorrow. 

That’s\marginnote{9.1} how to do the farming \\
that has the Deathless as its fruit. \\
When you finish this farming \\
you’re released from all suffering.” 

%
\end{verse}

Then\marginnote{10.1} \textsanskrit{Bhāradvāja} the Farmer filled a large bronze dish with milk-rice and presented it to the Buddha: “Eat the milk-rice, Master Gotama, you are truly a farmer. For Master Gotama does the farming that has the Deathless as its fruit.” 

\begin{verse}%
“Food\marginnote{11.1} enchanted by a spell isn’t fit for me to eat. \\
That’s not the principle of those who see, brahmin. \\
The Buddhas reject things enchanted with spells. \\
Since there is such a principle, brahmin, that’s how they live. 

Serve\marginnote{12.1} with other food and drink \\
the consummate one, the great hermit, \\
with defilements ended and remorse stilled. \\
For he is the field for the seeker of merit.” 

%
\end{verse}

“Then,\marginnote{13.1} Master Gotama, to whom should I give the milk-rice?” “Brahmin, I don’t see anyone in this world—with its gods, \textsanskrit{Māras}, and \textsanskrit{Brahmās}, this population with its ascetics and brahmins, its gods and humans—who can properly digest this milk-rice, except for the Realized One or one of his disciples. Well then, brahmin, throw out the milk-rice where there is little that grows, or drop it into water that has no living creatures.” 

So\marginnote{14.1} \textsanskrit{Bhāradvāja} the Farmer dropped the milk-rice in water that had no living creatures. And when the milk-rice was placed in the water, it sizzled and hissed, steaming and fuming. Suppose there was an iron cauldron that had been heated all day. If you placed it in the water, it would sizzle and hiss, steaming and fuming. In the same way, when the milk-rice was placed in the water, it sizzled and hissed, steaming and fuming. 

Then\marginnote{15.1} \textsanskrit{Bhāradvāja} the Farmer, shocked and awestruck,  went up to the Buddha, bowed down with his head at the Buddha’s feet, and said, “Excellent, Master Gotama! Excellent! As if he were righting the overturned, or revealing the hidden, or pointing out the path to the lost, or lighting a lamp in the dark so people with good eyes can see what’s there, Master Gotama has made the teaching clear in many ways. I go for refuge to Master Gotama, to the teaching, and to the mendicant \textsanskrit{Saṅgha}. Sir, may I receive the going forth, the ordination in the Buddha’s presence?” 

And\marginnote{16.1} \textsanskrit{Bhāradvāja} the Farmer received the going forth, the ordination in the Buddha’s presence. Not long after his ordination, Venerable \textsanskrit{Bhāradvāja}, living alone, withdrawn, diligent, keen, and resolute, soon realized the supreme end of the spiritual path in this very life. He lived having achieved with his own insight the goal for which gentlemen rightly go forth from the lay life to homelessness. He understood: “Rebirth is ended; the spiritual journey has been completed; what had to be done has been done; there is no return to any state of existence.” And Venerable \textsanskrit{Bhāradvāja} became one of the perfected. 

%
\section*{{\suttatitleacronym Snp 1.5}{\suttatitletranslation With Cunda }{\suttatitleroot Cundasutta}}
\addcontentsline{toc}{section}{\tocacronym{Snp 1.5} \toctranslation{With Cunda } \tocroot{Cundasutta}}
\markboth{With Cunda }{Cundasutta}
\extramarks{Snp 1.5}{Snp 1.5}

\begin{verse}%
“I\marginnote{1.1} ask the sage abounding in wisdom,” \\
\scspeaker{said Cunda the smith, }\\
“the Buddha, master of the teaching, free of craving, \\
best of men, excellent charioteer, please tell me this: \\
how many ascetics are there in the world?” 

“There\marginnote{2.1} are four ascetics, not a fifth.” \\
\scspeaker{said the Buddha to Cunda, }\\
“Being asked to bear witness, I will explain them to you:\footnote{\textit{\textsanskrit{Sakkhipuṭṭho}} is only elsewhere used in the context of being asked to bear witness in court. I think the context is relevant here, as the Buddha is offering personal testimony on a matter that otherwise he may not speak. } \\
the path-victor, the path-teacher, \\
the path-liver, and the path-wrecker.” 

“Who\marginnote{3.1} is a path-victor according to the Buddhas?” \\
\scspeaker{said Cunda the smith, }\\
“and how is one an unequaled path-explainer? \\
Tell me when asked about one who lives the path, \\
then declare the path-wrecker.” 

“Rid\marginnote{4.1} of doubt, free of thorns, \\
delighting in quenching, not fawning, \\
a guide for the world with its gods. \\
The Buddhas say one such is victor of the path. 

Knowing\marginnote{5.1} the ultimate as ultimate, \\
they explain and analyze the teaching right here. \\
That sage unstirred, with doubt cut off, \\
is the second mendicant, I say, the path-teacher. 

Living\marginnote{6.1} restrained and mindful on the path \\
of the well-taught passages of teaching,\footnote{Norman and Bodhi both read \textit{pada} as “way”, but it surely means “passage” as eg. AN 4.191:1.7. } \\
cultivating blameless states, \\
is the third mendicant, I say, the path-liver. 

Dressed\marginnote{7.1} like one true to their vows, \\
pushy, rude, a corrupter of families, \\
devious, unrestrained, chaff, \\
the path-wrecker’s life is a sham. 

A\marginnote{8.1} layperson who gets this, \\
a learned, wise noble disciple, \\
knows that ‘They are not all like that one’. \\
So when they see them they don’t lose their faith. \\
For how could one equate them—\\
the corrupt with the uncorrupt, the pure with the impure?” 

%
\end{verse}

%
\section*{{\suttatitleacronym Snp 1.6}{\suttatitletranslation Downfalls }{\suttatitleroot Parābhavasutta}}
\addcontentsline{toc}{section}{\tocacronym{Snp 1.6} \toctranslation{Downfalls } \tocroot{Parābhavasutta}}
\markboth{Downfalls }{Parābhavasutta}
\extramarks{Snp 1.6}{Snp 1.6}

\scevam{So\marginnote{1.1} I have heard. }At one time the Buddha was staying near \textsanskrit{Sāvatthī} in Jeta’s Grove, \textsanskrit{Anāthapiṇḍika}’s monastery. Then, late at night, a glorious deity, lighting up the entire Jeta’s Grove, went up to the Buddha, bowed, and stood to one side. Standing to one side, that deity addressed the Buddha in verse: 

\begin{verse}%
“We\marginnote{2.1} ask Gotama\footnote{Several of the examples are gendered so I use “man”. } \\
about a man’s downfall. \\
We have come to ask you sir:\footnote{Reading \textit{\textsanskrit{bhavantaṁ}} which is the regular form used with \textit{Gotama}. } \\
what leads to downfall?”\footnote{“Cause” or “reason” are not wrong, but they miss the force of the metaphor, which is about going from one state to another. } 

“It’s\marginnote{3.1} easy to know success, \\
and downfall is just as easy. \\
One who loves the teaching succeeds,\footnote{\textit{\textsanskrit{Dhammakāmo}} is apparently always used in this sense. } \\
but a hater of the teaching meets their downfall.” 

“We\marginnote{4.1} get what you’re saying, \\
this is the first downfall. \\
Tell us the second, Blessed One: \\
what leads to downfall?” 

“The\marginnote{5.1} bad are dear to him, \\
he has no love for the good. \\
He believes the teaching of the bad; \\
and that leads to his downfall.” 

“We\marginnote{6.1} get what you’re saying, \\
this is the second downfall. \\
Tell us the third, Blessed One: \\
what leads to downfall?” 

“Fond\marginnote{7.1} of sleep, fond of company, \\
a man who does no work; \\
he’s lazy, marked by anger, \\
and that leads to his downfall.” 

“We\marginnote{8.1} get what you’re saying, \\
this is the third downfall. \\
Tell us the fourth, Blessed One: \\
what leads to downfall?” 

“Though\marginnote{9.1} able, he does not look after \\
his mother and father \\
when elderly, past their prime, \\
and that leads to his downfall.” 

“We\marginnote{10.1} get what you’re saying, \\
this is the fourth downfall. \\
Tell us the fifth, Blessed One: \\
what leads to downfall?” 

“He\marginnote{11.1} deceives with lies \\
ascetics and brahmins \\
and other renunciates, \\
and that leads to his downfall.” 

“We\marginnote{12.1} get what you’re saying, \\
this is the fifth downfall. \\
Tell us the sixth, Blessed One: \\
what leads to downfall?” 

“A\marginnote{13.1} man with plenty of wealth—\\
gold and food—\\
eats delicacies alone, \\
and that leads to his downfall.” 

“We\marginnote{14.1} get what you’re saying, \\
this is the sixth downfall. \\
Tell us the seventh, Blessed One: \\
what leads to downfall?” 

“Vain\marginnote{15.1} of caste, wealth, \\
and clan, a man \\
looks down on his own family, \\
and that leads to his downfall.” 

“We\marginnote{16.1} get what you’re saying, \\
this is the seventh downfall. \\
Tell us the eighth, Blessed One: \\
what leads to downfall?” 

“In\marginnote{17.1} womanizing, drinking, \\
and gambling, a man \\
wastes all that he has earned, \\
and that leads to his downfall.” 

“We\marginnote{18.1} get what you’re saying, \\
this is the eighth downfall. \\
Tell us the ninth, Blessed One: \\
what leads to downfall?” 

“Not\marginnote{19.1} content with his own partners, \\
he debauches himself with prostitutes,\footnote{Both Norman and Bodhi read \textit{dissati}, I read \textit{dussati}. } \\
and with others’ partners, \\
and that leads to his downfall.” 

“We\marginnote{20.1} get what you’re saying, \\
this is the ninth downfall. \\
Tell us the tenth, Blessed One: \\
what leads to downfall?” 

“A\marginnote{21.1} man well past his prime \\
marries a girl with budding breasts; \\
he cannot sleep for jealousy, \\
and that leads to his downfall.” 

“We\marginnote{22.1} get what you’re saying, \\
this is the tenth downfall. \\
Tell us the eleventh, Blessed One: \\
what leads to downfall?” 

“He\marginnote{23.1} places in authority \\
a woman or a man \\
who’s a drunkard and a spender, \\
and that leads to his downfall.” 

“We\marginnote{24.1} get what you’re saying, \\
this is the eleventh downfall. \\
Tell us the twelfth, Blessed One: \\
what leads to downfall?” 

“A\marginnote{25.1} man of little wealth and strong craving, \\
born into an aristocratic family, \\
sets his sights on kingship, \\
and that leads to his downfall. 

Seeing\marginnote{26.1} these downfalls in the world, \\
an astute and noble person, \\
accomplished in vision, \\
will enjoy a world of grace.” 

%
\end{verse}

%
\section*{{\suttatitleacronym Snp 1.7}{\suttatitletranslation The Lowlife }{\suttatitleroot Vasalasutta}}
\addcontentsline{toc}{section}{\tocacronym{Snp 1.7} \toctranslation{The Lowlife } \tocroot{Vasalasutta}}
\markboth{The Lowlife }{Vasalasutta}
\extramarks{Snp 1.7}{Snp 1.7}

\scevam{So\marginnote{1.1} I have heard. }At one time the Buddha was staying near \textsanskrit{Sāvatthī} in Jeta’s Grove, \textsanskrit{Anāthapiṇḍika}’s monastery. Then the Buddha robed up in the morning and, taking his bowl and robe, entered \textsanskrit{Sāvatthī} for alms. Now at that time in the brahmin \textsanskrit{Bhāradvāja} the Fire-Worshipper’s home the sacred flame had been kindled and the oblation prepared. Wandering indiscriminately for almsfood in \textsanskrit{Sāvatthī}, the Buddha approached \textsanskrit{Bhāradvāja} the Fire-Worshiper’s house. 

\textsanskrit{Bhāradvāja}\marginnote{2.1} the Fire-Worshiper saw the Buddha coming off in the distance and said to him, “Stop right there, shaveling! Right there, fake ascetic! Right there, lowlife!” 

When\marginnote{3.1} he said this, the Buddha said to him, “But brahmin, do you know what is a lowlife or what are the qualities that make you a lowlife?” “No I do not, Master Gotama. Please, Master Gotama, teach me this matter so I can understand what is a lowlife or what are the qualities that make you a lowlife.” “Well then, brahmin, listen and pay close attention, I will speak.” “Yes sir,” \textsanskrit{Bhāradvāja} the Fire-Worshiper replied. The Buddha said this: 

\begin{verse}%
“Irritable\marginnote{4.1} and hostile, \\
wicked and offensive, \\
a man deficient in view, deceitful: \\
know him as a lowlife. 

He\marginnote{5.1} harms living creatures \\
born of womb or of egg, \\
and has no kindness for creatures: \\
know him as a lowlife. 

He\marginnote{6.1} destroys and devastates \\
villages and towns, \\
a notorious oppressor: \\
know him as a lowlife. 

Whether\marginnote{7.1} in village or wilderness, \\
he steals what belongs to others, \\
taking what has not been given: \\
know him as a lowlife. 

Having\marginnote{8.1} fallen into debt, \\
when pressed to pay up he flees, saying \\
‘I don’t owe you anything!’: \\
know him as a lowlife. 

Wanting\marginnote{9.1} some item or other, \\
he attacks a person in the street \\
and takes it: \\
know him as a lowlife. 

For\marginnote{10.1} his own sake or the sake of another, \\
or for the sake of wealth, a man \\
tells a lie when asked to bear witness: \\
know him as a lowlife. 

He\marginnote{11.1} is spied among the partners\footnote{Both Norman and Bodhi add a term suggesting transgression here, in line with the commentary and the apparent sense. I don’t, as I suspect the text is corrupt. Could \textit{diss} be \textit{duss}? } \\
of relatives and friends, \\
by force or seduction: \\
know him as a lowlife. 

Though\marginnote{12.1} able, he does not look after \\
his mother and father \\
when elderly, past their prime: \\
know him as a lowlife. 

He\marginnote{13.1} hits or verbally abuses \\
his mother or father, \\
brother, sister, or mother-in-law: \\
know him as a lowlife. 

When\marginnote{14.1} asked about the good, \\
he teaches what is bad, \\
giving secretive advice: \\
know him as a lowlife. 

Having\marginnote{15.1} done a bad deed, he wishes, \\
‘May no-one find me out!’ \\
His deeds are underhand: \\
know him as a lowlife. 

When\marginnote{16.1} visiting another family \\
he eats their delicious food, \\
but does not return the honor: \\
know him as a lowlife. 

He\marginnote{17.1} deceives with lies \\
ascetics and brahmins \\
and other renunciates: \\
know him as a lowlife. 

When\marginnote{18.1} time comes to offer a meal \\
to brahmins or ascetics, \\
he abuses them and does not give: \\
know him as a lowlife. 

He\marginnote{19.1} talks about what never happened, \\
being wrapped up in delusion, \\
chasing after some item or other: \\
know him as a lowlife. 

He\marginnote{20.1} extols himself \\
and disparages others, \\
brought down by his pride: \\
know him as a lowlife. 

He’s\marginnote{21.1} a bully and a miser, \\
of wicked desires, stingy, and devious, \\
shameless, imprudent: \\
know him as a lowlife. 

He\marginnote{22.1} insults the Buddha \\
or his disciple, \\
whether lay or renunciate: \\
know him as a lowlife. 

He\marginnote{23.1} claims to be a perfected one, \\
when he really is no such thing. \\
In the world with its \textsanskrit{Brahmās}, \\
that crook is truly the lowest lowlife. \\
These who are called lowlifes \\
I have explained to you. 

You’re\marginnote{24.1} not a lowlife by birth, \\
nor by birth are you a brahmin. \\
You’re a lowlife by your deeds, \\
by deeds you’re a brahmin. 

And\marginnote{25.1} also you should know \\
according to this example. \\
Sopaka the outcaste’s son \\
became renowned as \textsanskrit{Mātaṅga}. 

\textsanskrit{Mātaṅga}\marginnote{26.1} achieved the highest fame \\
so very hard to find. \\
Lots of aristocrats and brahmins \\
came to serve him. 

He\marginnote{27.1} ascended the stainless highway \\
that leads to the heavens; \\
having discarded sensual desire, \\
he was reborn in a \textsanskrit{Brahmā} realm. \\
His birth did not prevent him \\
from rebirth in the \textsanskrit{Brahmā} realm. 

Those\marginnote{28.1} born in a brahmin family \\
who recite as kinsmen of the hymns, \\
are often discovered \\
in the midst of wicked deeds. 

Blameworthy\marginnote{29.1} in the present life, \\
and in the next, a bad destination. \\
Their birth does not prevent them \\
from blame or bad destiny. 

You’re\marginnote{30.1} not a lowlife by birth, \\
nor by birth are you a brahmin. \\
You’re a lowlife by your deeds, \\
by deeds you’re a brahmin.” 

%
\end{verse}

When\marginnote{31.1} he had spoken, the brahmin \textsanskrit{Bhāradvāja} the Fire-Worshiper said to the Buddha, “Excellent, Master Gotama! Excellent! … From this day forth, may Master Gotama remember me as a lay follower who has gone for refuge for life.” 

%
\section*{{\suttatitleacronym Snp 1.8}{\suttatitletranslation The Discourse on Love }{\suttatitleroot Mettasutta}}
\addcontentsline{toc}{section}{\tocacronym{Snp 1.8} \toctranslation{The Discourse on Love } \tocroot{Mettasutta}}
\markboth{The Discourse on Love }{Mettasutta}
\extramarks{Snp 1.8}{Snp 1.8}

\begin{verse}%
This\marginnote{1.1} is what should be done by those who are skilled in goodness, \\
and have known the state of peace. \\
Let them be able and upright, very upright, \\
easy to speak to, gentle and humble; 

content\marginnote{2.1} and unburdensome, \\
unbusied, living lightly, \\
alert, with senses calmed, \\
courteous, not fawning on families. 

Let\marginnote{3.1} them not do the slightest thing \\
that others might blame with reason. \\
May they be happy and safe! \\
May all beings be happy! 

Whatever\marginnote{4.1} living creatures there are \\
with not a one left out—\\
frail or firm, long or large, \\
medium, small, tiny or round, 

visible\marginnote{5.1} or invisible, \\
living far or near, \\
those born or to be born: \\
May all beings be happy! 

Let\marginnote{6.1} none turn from another, \\
nor look down on anyone anywhere. \\
Though provoked or aggrieved, \\
let them not wish pain on each other. 

Even\marginnote{7.1} as a mother would protect with her life \\
her child, her only child, \\
so too for all creatures \\
unfold a boundless heart. 

With\marginnote{8.1} love for the whole world, \\
unfold a boundless heart. \\
Above, below, all round, \\
unconstricted, without enemy or foe. 

When\marginnote{9.1} standing, walking, sitting, \\
or lying down while yet unweary, \\
keep this ever in mind; \\
for this, they say, is a holy abiding in this life. 

Avoiding\marginnote{10.1} harmful views, \\
virtuous, accomplished in insight, \\
with sensual desire dispelled, \\
they never return to a womb again.\footnote{Note that \textit{\textsanskrit{jātu}} here is not connected to \textit{\textsanskrit{jāti}} “birth”. Rather, \textit{na hi \textsanskrit{jātu}} is an idiom meaning “absolutely not”. Commentary: \textit{ekaṃseneva}. } 

%
\end{verse}

%
\section*{{\suttatitleacronym Snp 1.9}{\suttatitletranslation With Hemavata }{\suttatitleroot Hemavatasutta}}
\addcontentsline{toc}{section}{\tocacronym{Snp 1.9} \toctranslation{With Hemavata } \tocroot{Hemavatasutta}}
\markboth{With Hemavata }{Hemavatasutta}
\extramarks{Snp 1.9}{Snp 1.9}

\begin{verse}%
“Today\marginnote{1.1} is the fifteenth day sabbath,” \\
\scspeaker{said \textsanskrit{Sātāgira}, the native spirit of mount \textsanskrit{Sātā}, }\\
“a holy night is at hand. \\
Come now, let us see Gotama, \\
the Teacher of peerless name.” 

“Isn’t\marginnote{2.1} his mind well-disposed,” \\
\scspeaker{said Hemavata, the native spirit of the Himalayas, }\\
“impartial towards all creatures? \\
And aren’t his thoughts under control \\
when it comes to likes and dislikes?” 

“His\marginnote{3.1} mind is well-disposed,” \\
\scspeaker{said \textsanskrit{Sātāgira}, }\\
“impartial towards all creatures. \\
His thoughts are under control \\
when it comes to his likes and dislikes.” 

“Doesn’t\marginnote{4.1} he not steal?” \\
\scspeaker{said Hemavata, }\\
“And doesn’t he harm not a creature? \\
Isn’t he far from negligence? \\
And doesn’t he not neglect absorption?” 

“He\marginnote{5.1} does not take what is not given,” \\
\scspeaker{said \textsanskrit{Sātāgira}, }\\
“and he harms not a creature. \\
He is far from negligence—\\
the Buddha does not neglect absorption.” 

“Doesn’t\marginnote{6.1} he avoid lying?” \\
\scspeaker{said Hemavata, }\\
“And doesn’t he not speak sharply? \\
Doesn’t he avoid divisive speech,\footnote{Norman has “untruth”, Bodhi “destructive”. Neither notices \textit{\textsanskrit{vebhūtiyaṁ}} at  DN 30:2.21.2 and DN 28:11.2, where “divisive” fits all cases. } \\
as well as speaking nonsense?” 

“He\marginnote{7.1} does not lie,” \\
\scspeaker{said \textsanskrit{Sātāgira}, }\\
“nor does he speak sharply. \\
He avoids divisive speech, \\
and speaks words of wise counsel.”\footnote{Lit. “the counselor speaks meaningfully.” } 

“Doesn’t\marginnote{8.1} he find sensual pleasures unattractive?” \\
\scspeaker{said Hemavata, }\\
“And isn’t his mind unclouded? \\
Hasn’t he escaped delusion? \\
And isn’t he seer of truths?” 

“He\marginnote{9.1} does not find sensual pleasures attractive,” \\
\scspeaker{said \textsanskrit{Sātāgira}, }\\
“and his mind is unclouded. \\
He has escaped all delusion—\\
the Buddha is seer of truths.” 

“Isn’t\marginnote{10.1} he accomplished in knowledge?” \\
\scspeaker{said Hemavata, }\\
“And doesn’t he live a pure life? \\
Aren’t his defilements all ended? \\
Doesn’t he have no future lives?” 

“He\marginnote{11.1} is accomplished in knowledge,” \\
\scspeaker{said \textsanskrit{Sātāgira}, }\\
“and he does live a pure life. \\
His defilements are all ended, \\
there are no future lives for him.” 

“Accomplished\marginnote{12.1} is the sage’s mind \\
in action and in speech, \\
and he’s accomplished in knowledge and conduct \\
as per the teaching you praise.”\footnote{Read \textit{\textsanskrit{pasaṁsasi}} per PTS. } 

“Accomplished\marginnote{13.1} is the sage’s mind \\
in action and in speech, \\
and he’s accomplished in knowledge and conduct \\
as per the teaching you rejoice in. 

Accomplished\marginnote{14.1} is the sage’s mind \\
in action and in speech, \\
and he’s accomplished in knowledge and conduct: \\
come now, let us see Gotama.” 

“The\marginnote{15.1} hero so lean, with antelope calves, \\
not greedy, eating little, \\
the sage meditating alone in the forest, \\
come now, let us see Gotama. 

An\marginnote{16.1} elephant, wandering alone like a lion, \\
unconcerned for sensual pleasures, \\
let’s approach him and ask about \\
release from the snare of death.” 

“The\marginnote{17.1} communicator, the instructor, \\
who has gone beyond all things, \\
Awakened, beyond enmity and fear, \\
let us ask Gotama.” 

“What\marginnote{18.1} has the world arisen in?” \\
\scspeaker{said Hemavata, }\\
What does it get close to? \\
By grasping what \\
is the world troubled in what?” 

“The\marginnote{19.1} world’s arisen in six,” \\
\scspeaker{said the Buddha to Hemavata. }\\
“It gets close to six. \\
By grasping at these six, \\
the world’s troubled in six.” 

“What\marginnote{20.1} is that grasping \\
by which the world is troubled? \\
Tell us the exit when asked: \\
how is one released from all suffering?” 

“There\marginnote{21.1} are five kinds of sensual stimulation in the world, \\
and the mind is said to be the sixth. \\
When you’ve discarded desire for these, \\
you’re released from all suffering. 

This\marginnote{22.1} is the exit from the world, \\
explained in accord with the truth. \\
The way I’ve explained it is how \\
you’re released from all suffering.” 

“Who\marginnote{23.1} here crosses the flood, \\
Who crosses the deluge? \\
Who, not standing and unsupported, \\
does not sink in the deep?” 

“Someone\marginnote{24.1} who is always endowed with ethics, \\
wise and serene, \\
inwardly reflective, mindful, \\
crosses the flood so hard to cross. 

Someone\marginnote{25.1} who desists from sensual perception, \\
who has escaped all fetters, \\
and is finished with relishing of rebirth, \\
does not sink in the deep.” 

“Behold\marginnote{26.1} him of wisdom deep who sees the subtle meaning, \\
who has nothing, unattached to sensual life, \\
everywhere free, \\
the great hermit treading the holy road. 

Behold\marginnote{27.1} him of peerless name who sees the subtle meaning, \\
giver of wisdom, unattached to the realm of sensuality: \\
see him, the all-knower, so very intelligent, \\
the great hermit treading the noble road.” 

“It\marginnote{28.1} was a fine sight for us today, \\
a good dawn, a good rising, \\
to see the Awakened One, \\
the undefiled one who has crossed the flood. 

These\marginnote{29.1} thousand native spirits \\
powerful and glorious, \\
all go to you for refuge, \\
you are our supreme Teacher. 

We\marginnote{30.1} shall journey \\
village to village, peak to peak, \\
paying homage to the Buddha, \\
and the natural excellence of the teaching!” 

%
\end{verse}

%
\section*{{\suttatitleacronym Snp 1.10}{\suttatitletranslation With Āḷavaka }{\suttatitleroot Āḷavakasutta}}
\addcontentsline{toc}{section}{\tocacronym{Snp 1.10} \toctranslation{With Āḷavaka } \tocroot{Āḷavakasutta}}
\markboth{With Āḷavaka }{Āḷavakasutta}
\extramarks{Snp 1.10}{Snp 1.10}

\scevam{So\marginnote{1.1} I have heard. }At one time the Buddha was staying near \textsanskrit{Āḷavī} in the haunt of the native spirit \textsanskrit{Āḷavaka}. Then the native spirit \textsanskrit{Āḷavaka} went up to the Buddha, and said to him: “Get out, ascetic!” Saying, “All right, sir,” the Buddha went out. “Get in, ascetic!” Saying, “All right, sir,” the Buddha went in. 

For\marginnote{2.1} a second time … And for a third time the native spirit \textsanskrit{Āḷavaka} said to the Buddha, “Get out, ascetic!” Saying, “All right, sir,” the Buddha went out. “Get in, ascetic!” Saying, “All right, sir,” the Buddha went in. 

And\marginnote{3.1} for a fourth time the native spirit \textsanskrit{Āḷavaka} said to the Buddha, “Get out, ascetic!” “No, sir, I won’t get out. Do what you must.” 

“I\marginnote{4.1} will ask you a question, ascetic. If you don’t answer me, I’ll drive you insane, or explode your heart, or grab you by the feet and throw you to the far shore of the Ganges!” 

“I\marginnote{5.1} don’t see anyone in this world with its gods, \textsanskrit{Māras}, and \textsanskrit{Brahmās}, this population with its ascetics and brahmins, its gods and humans who could do that to me. But anyway, ask what you wish.” Then the native spirit \textsanskrit{Āḷavaka} addressed the Buddha in verse: 

\begin{verse}%
“What’s\marginnote{6.1} a person’s best wealth? \\
What brings happiness when practiced well? \\
What’s the sweetest taste of all? \\
The one who they say has the best life: how do they live?” 

“Faith\marginnote{7.1} here is a person’s best wealth. \\
the teaching brings happiness when practiced well. \\
Truth is the sweetest taste of all. \\
The one who they say has the best life lives by wisdom.” 

“How\marginnote{8.1} do you cross the flood? \\
How do you cross the deluge? \\
How do you get over suffering? \\
How do you get purified?” 

“By\marginnote{9.1} faith you cross the flood, \\
and by diligence the deluge. \\
By energy you get past suffering, \\
and you’re purified by wisdom.” 

“How\marginnote{10.1} do you get wisdom? \\
How do you earn wealth? \\
How do you get a good reputation? \\
How do you hold on to friends? \\
How do the departed not grieve \\
when passing from this world to the next?” 

“One\marginnote{11.1} who is diligent and discerning \\
gains wisdom by wanting to learn, \\
having faith in the perfected ones, \\
and the teaching for becoming extinguished. 

Being\marginnote{12.1} responsible, acting appropriately, \\
and working hard you earn wealth. \\
Truthfulness wins you a good reputation. \\
You hold on to friends by giving. 

A\marginnote{13.1} faithful householder \\
who has these four qualities \\
does not grieve after passing away: \\
truth, principle, steadfastness, and generosity. 

Go\marginnote{14.1} ahead, ask others as well, \\
there are many ascetics and brahmins. \\
See whether anything better is found \\
than truth, self-control, generosity, and patience.” 

“Why\marginnote{15.1} now would I question \\
the many ascetics and brahmins? \\
Today I understand \\
what’s good for the next life. 

It\marginnote{16.1} was truly for my benefit \\
that the Buddha came to stay at \textsanskrit{Āḷavī}. \\
Today I understand \\
where a gift is very fruitful. 

I\marginnote{17.1} myself will journey \\
village to village, town to town, \\
paying homage to the Buddha, \\
and the natural excellence of the teaching!” 

%
\end{verse}

%
\section*{{\suttatitleacronym Snp 1.11}{\suttatitletranslation Victory }{\suttatitleroot Vijayasutta}}
\addcontentsline{toc}{section}{\tocacronym{Snp 1.11} \toctranslation{Victory } \tocroot{Vijayasutta}}
\markboth{Victory }{Vijayasutta}
\extramarks{Snp 1.11}{Snp 1.11}

\begin{verse}%
Walking\marginnote{1.1} and standing, \\
sitting and lying down, \\
extending and contracting the limbs: \\
these are the movements of the body. 

Linked\marginnote{2.1} together by bones and sinews, \\
plastered over with flesh and hide, \\
and covered by the skin, \\
the body is not seen as it is. 

It’s\marginnote{3.1} full of guts and belly, \\
liver and bladder, \\
heart and lungs, \\
kidney and spleen, 

spit\marginnote{4.1} and snot, \\
sweat and fat, \\
blood and synovial fluid, \\
bile and grease. 

Then\marginnote{5.1} in nine streams \\
the filth is always flowing. \\
There is muck from the eyes, \\
wax from the ears, 

and\marginnote{6.1} snot from the nostrils. \\
The mouth sometimes vomits \\
bile and sometimes phlegm. \\
And from the body, sweat and waste. 

Then\marginnote{7.1} there is the hollow head \\
all filled with brains. \\
Governed by ignorance, \\
the fool thinks it’s lovely. 

And\marginnote{8.1} when it lies dead, \\
bloated and livid, \\
discarded in a charnel ground, \\
the relatives forget it. 

It’s\marginnote{9.1} devoured by dogs, \\
by jackals, wolves, and worms. \\
It’s devoured by crows and vultures, \\
and any other creatures there. 

A\marginnote{10.1} wise mendicant here, \\
having heard the Buddha’s words, \\
fully understands it, \\
for they see it as it is. 

“As\marginnote{11.1} this is, so is that, \\
as that is, so is this.” \\
They’d reject desire for the body \\
inside and out. 

That\marginnote{12.1} wise mendicant here \\
rid of desire and lust, \\
has found the deathless peace, \\
extinguishment, the imperishable state. 

This\marginnote{13.1} two-legged body is dirty and stinking, \\
full of different carcasses, \\
and oozing all over the place—\\
but still it is cherished! 

And\marginnote{14.1} if, on account of such a body, \\
someone prides themselves \\
or looks down on others—\\
what is that but a failure to see? 

%
\end{verse}

%
\section*{{\suttatitleacronym Snp 1.12}{\suttatitletranslation The Sage }{\suttatitleroot Munisutta}}
\addcontentsline{toc}{section}{\tocacronym{Snp 1.12} \toctranslation{The Sage } \tocroot{Munisutta}}
\markboth{The Sage }{Munisutta}
\extramarks{Snp 1.12}{Snp 1.12}

\begin{verse}%
Peril\marginnote{1.1} stems from intimacy, \\
dust comes from a home. \\
Freedom from home and intimacy: \\
that is the sage’s vision. 

Having\marginnote{2.1} cut down what’s grown, they wouldn’t replant, \\
nor would they nurture what’s growing. \\
That’s who they call a sage wandering alone, \\
the great hermit has seen the state of peace. 

Having\marginnote{3.1} assessed the fields and measured the seeds,\footnote{The commentary, followed by Norman, Bodhi, and \textsanskrit{Ñāṇadīpa}, all take \textit{\textsanskrit{pamāya}} here in the sense of “crushed” (Sanskrit: \textit{\textsanskrit{pramṛṇati}}). However, \textit{\textsanskrit{pamāya}} occurs at Snp 4.12:17.1, where, being beside \textit{vinicchaye} (“judge, assess”), it is clearly an absolutive of \textit{\textsanskrit{pamiṇāti}} “having measured”. Here too it sits beside a word (\textit{\textsanskrit{saṅkhāya}}) having the sense to reckon or calculate. In the simile of rebirth and the field, the “seed” is consciousness, which belonging to the first noble truth is not “crushed” but “fully known”. If the seeds are already “crushed” then choosing not to water them (in the next line) is a tad redundant. } \\
they wouldn’t nurture them with moisture. \\
Truly that sage sees the utter ending of rebirth; \\
when logic’s left behind, judgments no longer apply.\footnote{Compare the stock idiom \textit{\textsanskrit{Saṅkhampi} na upeti upanidhimpi na upeti \textsanskrit{kalabhāgampi} na upeti} at eg. SN 20.2:1.7; also Snp 5.7:6.3. } 

Understanding\marginnote{4.1} all the planes of rebirth, \\
not wanting a single one of them, \\
Truly that sage freed of greed \\
need not strive, for they have reached the far shore. 

The\marginnote{5.1} champion, all-knower, so very intelligent, \\
unsullied in the midst of all things, \\
has given up all, freed in the ending of craving: \\
that’s who the wise know as a sage. 

Strong\marginnote{6.1} in wisdom, with precepts and observances intact, \\
serene, loving absorption, mindful, \\
released from chains, kind, undefiled: \\
that’s who the wise know as a sage. 

The\marginnote{7.1} diligent sage wandering alone, \\
is unaffected by blame and praise—\\
like a lion not startled by sounds, \\
like wind not caught in a net, \\
like water not sticking to a lotus. \\
Leader of others, not by others led: \\
that’s who the wise know as a sage. 

Steady\marginnote{8.1} as a post in a bathing-place \\
when others speak endlessly against them, \\
freed of greed, with senses stilled: \\
that’s who the wise know as a sage. 

Steadfast,\marginnote{9.1} straight as a shuttle, \\
horrified by wicked deeds, \\
discerning the just and the unjust: \\
that’s who the wise know as a sage. 

Restrained,\marginnote{10.1} they do no evil, \\
young or middle-aged, the sage is self-controlled. \\
Irreproachable, he does not insult anyone: \\
that’s who the wise know as a sage. 

When\marginnote{11.1} one who lives on charity receives alms, \\
from the top, the middle, or the leftovers, \\
they think it unworthy to praise or put down: \\
that’s who the wise know as a sage. 

The\marginnote{12.1} sage lives refraining from sex, \\
even when young is not tied down, \\
refraining from indulgence and negligence, freed: \\
that’s who the wise know as a sage. 

Understanding\marginnote{13.1} the world, the seer of the ultimate goal, \\
the poised one who has crossed the flood and the ocean, \\
has cut the ties, unattached and undefiled: \\
that’s who the wise know as a sage. 

The\marginnote{14.1} two are not the same, far apart in lifestyle and conduct—\\
the householder providing for a wife, and the selfless one true to their vows. \\
The unrestrained householder kills other creatures, \\
while the restrained sage always protects living creatures. 

As\marginnote{15.1} the crested blue-necked peacock flying through the sky \\
never approaches the speed of the swan, \\
so the householder cannot compete with the mendicant, \\
the sage meditating secluded in the woods. 

%
\end{verse}

%
\addtocontents{toc}{\let\protect\contentsline\protect\nopagecontentsline}
\chapter*{The Lesser Chapter }
\addcontentsline{toc}{chapter}{\tocchapterline{The Lesser Chapter }}
\addtocontents{toc}{\let\protect\contentsline\protect\oldcontentsline}

%
\section*{{\suttatitleacronym Snp 2.1}{\suttatitletranslation Gems }{\suttatitleroot Ratanasutta}}
\addcontentsline{toc}{section}{\tocacronym{Snp 2.1} \toctranslation{Gems } \tocroot{Ratanasutta}}
\markboth{Gems }{Ratanasutta}
\extramarks{Snp 2.1}{Snp 2.1}

\begin{verse}%
Whatever\marginnote{1.1} beings have gathered here, \\
on the ground or in the sky: \\
may beings all be of happy heart, \\
and listen carefully to what is said. 

So\marginnote{2.1} pay heed, all you beings, \\
have love for humankind, \\
who day and night bring offerings; \\
please protect them diligently. 

There’s\marginnote{3.1} no wealth here or beyond, \\
no sublime gem in the heavens, \\
that equals the Realized One. \\
This sublime gem is in the Buddha: \\
by this truth, may you be well! 

Ending,\marginnote{4.1} dispassion, the undying, the sublime, \\
attained by the Sakyan Sage immersed in \textsanskrit{samādhi}; \\
there is nothing equal to that Dhamma. \\
This sublime gem is in the Dhamma: \\
by this truth, may you be well! 

The\marginnote{5.1} purity praised by the highest Buddha \\
is said to be the “immersion with immediate fruit”; \\
no equal to that immersion is found. \\
This sublime gem is in the Dhamma: \\
by this truth, may you be well! 

The\marginnote{6.1} eight individuals praised by the good, \\
are the four pairs of the Holy One’s disciples; \\
they are worthy of religious donations, \\
what’s given to them is very fruitful. \\
This sublime gem is in the \textsanskrit{Saṅgha}: \\
by this truth, may you be well! 

Dedicated\marginnote{7.1} to Gotama’s dispensation, \\
strong-minded, free of sense desire, \\
they’ve attained the goal, plunged into the deathless, \\
and enjoy the quenching they’ve freely gained. \\
This sublime gem is in the \textsanskrit{Saṅgha}: \\
by this truth, may you be well! 

As\marginnote{8.1} a well planted boundary-pillar \\
is not shaken by the four winds, \\
I say a good person is like this, \\
who sees the noble truths in experience. \\
This sublime gem is in the \textsanskrit{Saṅgha}: \\
by this truth, may you be well! 

Those\marginnote{9.1} who fathom the noble truths \\
taught by the one of deep wisdom, \\
do not take an eighth life, \\
even if they are hugely negligent. \\
This sublime gem is in the \textsanskrit{Saṅgha}: \\
by this truth, may you be well! 

When\marginnote{10.1} they attain to vision \\
they give up three things: \\
identity view, doubt, and any \\
attachment to precepts and observances. 

They’re\marginnote{11.1} freed from the four places of loss, \\
and unable to perform the six grave crimes. \\
This sublime gem is in the \textsanskrit{Saṅgha}: \\
by this truth, may you be well! 

Even\marginnote{12.1} if they do a bad deed \\
by body, speech, or mind, \\
they are unable to conceal it; \\
they say this inability applies to one who has seen the truth. \\
This sublime gem is in the \textsanskrit{Saṅgha}: \\
by this truth, may you be well! 

Like\marginnote{13.1} a tall forest tree crowned with flowers \\
in the first month of summer; \\
that’s how he taught the superb Dhamma, \\
leading to quenching, the ultimate benefit. \\
This sublime gem is in the Buddha: \\
by this truth, may you be well! 

The\marginnote{14.1} superb, knower of the superb, giver of the superb, bringer of the superb; \\
taught the superb Dhamma supreme. \\
This sublime gem is in the Buddha: \\
by this truth, may you be well! 

The\marginnote{15.1} old is ended, nothing new is produced. \\
their minds have no desire for future rebirth. \\
Withered are the seeds, there’s no desire for growth, \\
those wise ones are extinguished just like this lamp. \\
This sublime gem is in the \textsanskrit{Saṅgha}: \\
by this truth, may you be well! 

Whatever\marginnote{16.1} beings have gathered here, \\
on the ground or in the sky: \\
the Realized One is honored by gods and humans! \\
We bow to the Buddha! May you be safe! 

Whatever\marginnote{17.1} beings have gathered here, \\
on the ground or in the sky: \\
the Realized One is honored by gods and humans! \\
We bow to the Dhamma! May you be safe! 

Whatever\marginnote{18.1} beings have gathered here, \\
on the ground or in the sky: \\
the Realized One is honored by gods and humans! \\
We bow to the \textsanskrit{Saṅgha}! May you be safe! 

%
\end{verse}

%
\section*{{\suttatitleacronym Snp 2.2}{\suttatitletranslation Putrefaction }{\suttatitleroot Āmagandhasutta}}
\addcontentsline{toc}{section}{\tocacronym{Snp 2.2} \toctranslation{Putrefaction } \tocroot{Āmagandhasutta}}
\markboth{Putrefaction }{Āmagandhasutta}
\extramarks{Snp 2.2}{Snp 2.2}

\begin{verse}%
“The\marginnote{1.1} good eat properly obtained \\
millet, wild grains, broomcorn, \\
greens, tubers, and squashes. \\
They don’t lie to get what they want. 

But\marginnote{2.1} when you eat delicious food, \\
nicely cooked and prepared, and offered by others,\footnote{\textit{\textsanskrit{Dinnaṁ}} and \textit{\textsanskrit{payataṁ}} are synonyms. } \\
enjoying a dish of fine rice, \\
Kassapa, you eat putrefaction. 

‘Putrefaction\marginnote{3.1} is not appropriate for me’—\\
so you said, kinsman of \textsanskrit{Brahmā}. \\
Yet here you are enjoying a dish of fine rice, \\
nicely cooked with the flesh of fowl. \\
I’m asking you this, Kassapa: \\
what do you take to be putrefaction?” 

“Killing\marginnote{4.1} living creatures, mutilation, murder, abduction; \\
stealing, lying, cheating and fraud, \\
learning crooked spells, adultery:\footnote{Read \textit{kujja}. } \\
this is putrefaction, not eating meat. 

People\marginnote{5.1} here with unbridled sensuality, \\
greedy for tastes, mixed up in impurity, \\
nihilists, immoral, intractable: \\
this is putrefaction, not eating meat. 

Brutal\marginnote{6.1} and rough backbiters, \\
pitiless and arrogant betrayers of friends, \\
misers who never give anything: \\
this is putrefaction, not eating meat. 

Anger,\marginnote{7.1} vanity, obstinacy, contrariness, \\
deceit, jealousy, boastfulness, \\
haughtiness, wicked associates: \\
this is putrefaction, not eating meat. 

The\marginnote{8.1} ill-behaved, debt-evaders, slanderers, \\
business cheats and con-artists, \\
vile men committing depravity: \\
this is putrefaction, not eating meat. 

People\marginnote{9.1} here who can’t stop harming living creatures, \\
taking from others, intent on hurting, \\
immoral, cruel, harsh, lacking regard for others: \\
this is putrefaction, not eating meat. 

Greedy,\marginnote{10.1} hostile, aggressive to others, \\
and addicted to evil—those beings pass into darkness, \\
falling headlong into hell: \\
this is putrefaction, not eating meat. 

Not\marginnote{11.1} fish or flesh or fasting, \\
being naked or shaven, or dreadlocks or dirt, \\
not rough hides or serving the sacred flame, \\
or the many austerities in the world aimed at immortality, \\
not hymns or oblations, sacrifices or seasonal observances, \\
will cleanse a mortal not free of doubt. 

Guarding\marginnote{12.1} the streams of sense impressions, wander with faculties conquered, \\
standing on the teaching, delighting in sincerity and gentleness. \\
The wise have escaped their chains and given up all pain; \\
they don’t cling to the seen and the heard.” 

The\marginnote{13.1} Buddha explained this matter to him again and again, \\
until the master of hymns understood it. \\
It was illustrated with colorful verses \\
by the sage free of putrefaction, unattached, hard to trace. 

Having\marginnote{14.1} heard the fine words of the Buddha, \\
that are free of putrefaction, getting rid of all suffering; \\
humbled, he bowed to the Realized One, \\
and right away begged to go forth. 

%
\end{verse}

%
\section*{{\suttatitleacronym Snp 2.3}{\suttatitletranslation Conscience }{\suttatitleroot Hirisutta}}
\addcontentsline{toc}{section}{\tocacronym{Snp 2.3} \toctranslation{Conscience } \tocroot{Hirisutta}}
\markboth{Conscience }{Hirisutta}
\extramarks{Snp 2.3}{Snp 2.3}

\begin{verse}%
Flouting\marginnote{1.1} conscience, loathing it, \\
saying “I’m on your side”, \\
but not following up in deeds—\\
know they’re not on your side. 

Some\marginnote{2.1} say nice things to their friends \\
without following it up. \\
The wise will recognize \\
one who talks without doing. 

No\marginnote{3.1} true friend relentlessly \\
suspects betrayal, looking for fault. \\
One on whom you rest, like a child on the breast, \\
is a true friend, not split from you by others. 

One\marginnote{4.1} whose reward is the fruit \\
of bearing the burden of service \\
develops a happy state, \\
producing joy and attracting praise. 

One\marginnote{5.1} who has drunk the nectar of seclusion \\
and the nectar of peace, \\
free of stress, free of evil, \\
drinks the joyous nectar of Dhamma. 

%
\end{verse}

%
\section*{{\suttatitleacronym Snp 2.4}{\suttatitletranslation Blessings }{\suttatitleroot Maṅgalasutta}}
\addcontentsline{toc}{section}{\tocacronym{Snp 2.4} \toctranslation{Blessings } \tocroot{Maṅgalasutta}}
\markboth{Blessings }{Maṅgalasutta}
\extramarks{Snp 2.4}{Snp 2.4}

\scevam{So\marginnote{1.1} I have heard. }At one time the Buddha was staying near \textsanskrit{Sāvatthī} in Jeta’s Grove, \textsanskrit{Anāthapiṇḍika}’s monastery. Then, late at night, a glorious deity, lighting up the entire Jeta’s Grove, went up to the Buddha, bowed, and stood to one side. Standing to one side, that deity addressed the Buddha in verse: 

\begin{verse}%
“Many\marginnote{2.1} gods and humans \\
have thought about blessings \\
desiring well-being: \\
declare the highest blessing.” 

“Not\marginnote{3.1} to fraternize with fools, \\
but to fraternize with the wise, \\
and honoring those worthy of honor: \\
this is the highest blessing. 

Living\marginnote{4.1} in a suitable region, \\
having made merit in the past, \\
being rightly resolved in oneself, \\
this is the highest blessing. 

Education\marginnote{5.1} and a craft, \\
discipline and training, \\
and well-spoken speech: \\
this is the highest blessing. 

Caring\marginnote{6.1} for mother and father, \\
kindness to children and partners, \\
and unstressful work: \\
this is the highest blessing. 

Giving\marginnote{7.1} and righteous conduct, \\
kindness to relatives, \\
blameless deeds: \\
this is the highest blessing. 

Desisting\marginnote{8.1} and abstaining from evil, \\
avoiding alcoholic drinks, \\
diligence in good qualities: \\
this is the highest blessing. 

Respect\marginnote{9.1} and humility, \\
contentment and gratitude, \\
and timely listening to the teaching: \\
this is the highest blessing. 

Patience,\marginnote{10.1} being easy to admonish, \\
the sight of ascetics, \\
and timely discussion of the teaching: \\
this is the highest blessing. 

Austerity\marginnote{11.1} and celibacy \\
seeing the noble truths, \\
and realization of extinguishment: \\
this is the highest blessing. 

Though\marginnote{12.1} touched by worldly things, \\
their mind does not tremble; \\
sorrowless, stainless, secure, \\
this is the highest blessing. 

Having\marginnote{13.1} completed these things, \\
undefeated everywhere; \\
everywhere they go in safety: \\
this is their highest blessing.” 

%
\end{verse}

%
\section*{{\suttatitleacronym Snp 2.5}{\suttatitletranslation With Spiky }{\suttatitleroot Sūcilomasutta}}
\addcontentsline{toc}{section}{\tocacronym{Snp 2.5} \toctranslation{With Spiky } \tocroot{Sūcilomasutta}}
\markboth{With Spiky }{Sūcilomasutta}
\extramarks{Snp 2.5}{Snp 2.5}

\scevam{So\marginnote{1.1} I have heard. }At one time the Buddha was staying near \textsanskrit{Gayā} on the cut-stone ledge in the haunt of Spiky the native spirit. Now at that time the native spirits Shaggy and Spiky were passing by not far from the Buddha. So Shaggy said to Spiky, “That’s an ascetic.” “That’s no ascetic, he’s a faker! I’ll soon find out whether he’s an ascetic or a faker.” 

Then\marginnote{2.1} Spiky went up to the Buddha and leaned up against his body, but the Buddha pulled away. Then Spiky said to the Buddha, “Are you afraid, ascetic?” “No, sir, I’m not afraid. But your touch is nasty.” 

“I\marginnote{3.1} will ask you a question, ascetic. If you don’t answer me, I’ll drive you insane, or explode your heart, or grab you by the feet and throw you to the far shore of the Ganges!” 

“I\marginnote{4.1} don’t see anyone in this world with its gods, \textsanskrit{Māras}, and \textsanskrit{Brahmās}, this population with its ascetics and brahmins, its gods and humans who could do that to me. But anyway, ask what you wish.” Then Spiky addressed the Buddha in verse: 

\begin{verse}%
“Where\marginnote{5.1} do greed and hate come from? \\
From where spring discontent, desire, and terror? \\
Where do the mind’s thoughts originate, \\
like a crow let loose by boys.” 

“Greed\marginnote{6.1} and hate come from here; \\
from here spring discontent, desire, and terror; \\
here’s where the mind’s thoughts originate, \\
like a crow let loose by boys. 

Born\marginnote{7.1} of affection, originating in oneself, \\
like the shoots from a banyan’s trunk; \\
the many kinds of attachment to sensual pleasures \\
are like camel’s foot creeper strung through the woods. 

Those\marginnote{8.1} who understand where they come from \\
get rid of them—listen up, spirit! \\
They cross this flood so hard to cross, \\
not crossed before, so as to not be reborn.” 

%
\end{verse}

%
\section*{{\suttatitleacronym Snp 2.6}{\suttatitletranslation A Righteous Life }{\suttatitleroot Kapilasutta (dhammacariyasutta)}}
\addcontentsline{toc}{section}{\tocacronym{Snp 2.6} \toctranslation{A Righteous Life } \tocroot{Kapilasutta (dhammacariyasutta)}}
\markboth{A Righteous Life }{Kapilasutta (dhammacariyasutta)}
\extramarks{Snp 2.6}{Snp 2.6}

\begin{verse}%
A\marginnote{1.1} righteous life, a spiritual life, \\
they call this the supreme treasure. \\
But if someone goes forth \\
from the lay life to homelessness 

who\marginnote{2.1} is of scurrilous character, \\
a beast and a bully, \\
their life gets worse, \\
as poison grows inside them. 

A\marginnote{3.1} mendicant who loves to argue, \\
wrapped in delusion, \\
doesn’t even know what’s been explained \\
in the Dhamma taught by the Buddha. 

Harassing\marginnote{4.1} those who are evolved, \\
governed by ignorance, \\
they don’t know that corruption \\
is the path that leads to hell. 

Entering\marginnote{5.1} the underworld, \\
passing from womb to womb, from darkness to darkness, \\
such a mendicant \\
falls into suffering after death. 

One\marginnote{6.1} such as that is \\
like a sewer \\
brimful with years of filth \\
for it’s hard to clean one full of grime. 

Mendicants,\marginnote{7.1} knowing that someone is like this, \\
attached to the lay life, \\
of wicked desires and wicked intent, \\
of bad behavior and alms-resort, 

then\marginnote{8.1} having gathered in harmony, \\
you should expel them. \\
Throw out the trash! \\
Get rid of the rubbish! 

And\marginnote{9.1} sweep away the scraps—\\
they’re not ascetics, they just think they are. \\
When you’ve thrown out those of wicked desires, \\
of bad behavior and alms-resort, 

dwell\marginnote{10.1} in communion, ever mindful, \\
the pure with the pure. \\
Then in harmony, alert, \\
you’ll make an end to suffering.” 

%
\end{verse}

%
\section*{{\suttatitleacronym Snp 2.7}{\suttatitletranslation Brahmanical Traditions }{\suttatitleroot Brāhmaṇadhammikasutta}}
\addcontentsline{toc}{section}{\tocacronym{Snp 2.7} \toctranslation{Brahmanical Traditions } \tocroot{Brāhmaṇadhammikasutta}}
\markboth{Brahmanical Traditions }{Brāhmaṇadhammikasutta}
\extramarks{Snp 2.7}{Snp 2.7}

\scevam{So\marginnote{1.1} I have heard. }At one time the Buddha was staying near \textsanskrit{Sāvatthī} in Jeta’s Grove, \textsanskrit{Anāthapiṇḍika}’s monastery. Then several old and well-to-do brahmins of Kosala—elderly and senior, who were advanced in years and had reached the final stage of life—went up to the Buddha, and exchanged greetings with him. When the greetings and polite conversation were over, they sat down to one side and said to the Buddha: “Master Gotama, are the ancient traditions of the brahmins seen these days among brahmins?” “No, brahmins, they are not.” “If you wouldn’t mind, Master Gotama, please teach us the ancient traditions of the brahmins.” “Well then, brahmins, listen and pay close attention, I will speak.” “Yes, sir,” they replied. The Buddha said this: 

\begin{verse}%
“The\marginnote{2.1} ancient hermits used to be \\
restrained and austere. \\
Having given up the five sensual titillations, \\
they lived for their own true good. 

Brahmins\marginnote{3.1} used to own no cattle, \\
nor gold or grain. \\
Chanting was their wealth and grain, \\
which they guarded as a gift from god. 

Food\marginnote{4.1} was prepared for them \\
and left beside their doors. \\
People believed that food prepared in faith \\
should be given to them. 

With\marginnote{5.1} colorful clothes, \\
clothes and bedding, \\
prosperous nations and countries \\
honored those brahmins. 

Brahmins\marginnote{6.1} used to be inviolable and \\
invincible, protected by principle. \\
No-one ever turned them away \\
from the doors of families. 

For\marginnote{7.1} forty-eight years \\
they led the spiritual life.\footnote{\textit{\textsanskrit{Komāra}} is hypermetrical and has probably been inserted by analogy with AN 5.192:5.4. } \\
The brahmins of old pursued \\
their quest for knowledge and conduct. 

Brahmins\marginnote{8.1} never transgressed with another,\footnote{According to Baudh 2.1.2.13, \textit{\textsanskrit{agamyā} \textsanskrit{gamanaṁ}} means not transgressing with women considered inappropriate, such as the female friend of a male or female teacher. It doesn’t mean “outside caste”. } \\
nor did they purchase a wife. \\
They lived together in love, \\
joining together by mutual consent. 

Brahmins\marginnote{9.1} never approached their wives for sex \\
during the time outside \\
the fertile half of the month \\
after menstruation. 

They\marginnote{10.1} praised celibacy and morality, \\
integrity, gentleness, and austerity, \\
gentleness and harmlessness, \\
and also patience. 

He\marginnote{11.1} who was supreme among them, \\
godlike, staunchly vigorous, \\
did not engage in sex \\
even in a dream. 

Training\marginnote{12.1} in line with their duties, \\
many smart people here \\
praised celibacy and morality, \\
and also patience. 

They\marginnote{13.1} begged for rice, \\
bedding, clothes, ghee, and oil. \\
Having collected them legitimately, \\
they arranged a sacrifice. 

But\marginnote{14.1} they slew no cows \\
while serving at the sacrifice. \\
Like a mother, father, or brother, \\
or some other relative, \\
cows are our best friends, \\
the fonts of medicine. 

They\marginnote{15.1} give food and health, \\
and beauty and happiness. \\
Knowing these benefits, \\
they slew no cows. 

The\marginnote{16.1} brahmins were delicate and tall, \\
beautiful and glorious. \\
They were keen on all the duties \\
required by their own traditions.\footnote{\textit{\textsanskrit{Kiccākicca}} means “all kinds of duties, various business”, not “what is to be done and not done” (per both Norman and Bodhi). See eg. Thag 16.10:20.2. } \\
So long as they continued in the world, \\
people flourished happily. 

But\marginnote{17.1} perversion crept into them \\
little by little when they saw \\
the splendor of the king \\
and the ladies in all their finery. 

Their\marginnote{18.1} chariots were harnessed with thoroughbreds, \\
well-made with bright canopies, \\
and their homes and houses were \\
neatly laid out in measured rows.\footnote{A line elsewhere only used to describe hell. Perhaps the Buddha was no fan of the suburbs. } 

They\marginnote{19.1} were lavished with herds of cattle, \\
and furnished with bevies of lovely ladies. \\
This extravagant human wealth \\
was coveted by the brahmins. 

They\marginnote{20.1} compiled hymns to that end,\footnote{I’m not entirely convinced that \textit{\textsanskrit{ganthetvā}} means “composed” here. It’s the only early usage in this sense, and the reading is derived from the highly polemical commentary. It may mean just that they “put together” i.e. “selected” favorable passages. This would be less nasty to the brahmins and more historically plausible (as we know that the Vedas are, in fact, old.) } \\
approached King \textsanskrit{Okkāka} and said, \\
‘You have plenty of wealth and grain. \\
Sacrifice! For you have much treasure. \\
Sacrifice! For you have much wealth.’ 

Persuaded\marginnote{21.1} by the brahmins, \\
the king, chief of charioteers, performed \\
horse sacrifice, human sacrifice, \\
the sacrifices of the ‘stick-casting’, the ‘royal soma drinking’, and the ‘unbarred’. \\
When he had carried out these sacrifices, \\
he gave riches to the brahmins. 

There\marginnote{22.1} were cattle, bedding, and clothes, \\
and ladies in all their finery; \\
chariots harnessed with thoroughbreds, \\
well-made with bright canopies; 

and\marginnote{23.1} lovely homes, all \\
neatly laid out in measured rows. \\
Having furnished them with different grains, \\
he gave riches to the brahmins. 

When\marginnote{24.1} they got hold of that wealth, \\
they arranged to store it up. \\
Falling under the sway of desire, \\
their craving grew and grew. \\
They compiled hymns to that end, \\
approached King \textsanskrit{Okkāka} once more and said, 

‘Like\marginnote{25.1} water and earth, \\
gold, riches, and grain, \\
are cows for humankind, \\
as they are essential for creatures. \\
Sacrifice! For you have much treasure. \\
Sacrifice! For you have much wealth.’ 

Persuaded\marginnote{26.1} by the brahmins, \\
the king, chief of charioteers, \\
had many hundred thousand cows\footnote{No doubt an exaggeration, but sacrifices on this scale have been performed in modern times. } \\
slain at the sacrifice. 

Neither\marginnote{27.1} with feet nor with horns \\
do cows harm anyone at all. \\
Cows meek as lambs, \\
supply buckets of milk. \\
But taking them by the horns, \\
the king slew them with a sword. 

At\marginnote{28.1} that the gods and the ancestors, \\
with Indra, the titans and monsters, \\
roared out: ‘This is a crime against nature!’ \\
as the sword fell on the cows. 

There\marginnote{29.1} used to be three kinds of illness: \\
greed, starvation, and old age. \\
But due to the slaughter of cows, \\
this grew to be ninety-eight. 

This\marginnote{30.1} unnatural violence \\
has been passed down as an ancient custom. \\
Killing innocent creatures, \\
the sacrificers forsake righteousness. 

And\marginnote{31.1} that is how this mean old practice \\
was criticized by sensible people. \\
Wherever they see such a thing, \\
folk criticize the sacrificer. 

With\marginnote{32.1} righteousness gone, \\
merchants and workers were split, \\
as were many aristocrats, \\
and wives looked down on their husbands. 

Aristocrats\marginnote{33.1} and \textsanskrit{Brahmā}’s kinsmen \\
and others protected by their clan, \\
neglecting the lessons of ancestry, \\
fell under the sway of sensual pleasures.” 

%
\end{verse}

When\marginnote{34.1} he had spoken, those well-to-do brahmins said to the Buddha, “Excellent, Master Gotama! Excellent! … From this day forth, may Master Gotama remember us as lay followers who have gone for refuge for life.” 

%
\section*{{\suttatitleacronym Snp 2.8}{\suttatitletranslation The Boat }{\suttatitleroot Dhamma (nāvā) sutta}}
\addcontentsline{toc}{section}{\tocacronym{Snp 2.8} \toctranslation{The Boat } \tocroot{Dhamma (nāvā) sutta}}
\markboth{The Boat }{Dhamma (nāvā) sutta}
\extramarks{Snp 2.8}{Snp 2.8}

\begin{verse}%
Honor\marginnote{1.1} the person from whom you would learn the teaching, \\
as the gods honor Inda. \\
Then they will have confidence in you, \\
and being learned, they reveal the teaching. 

Heeding\marginnote{2.1} well, a wise pupil \\
practicing in line with that teaching \\
grows intelligent, discerning, and subtle \\
through diligently sticking close to such a person. 

But\marginnote{3.1} associating with a petty fool \\
who falls short of the goal, jealous, \\
then unable to discern the teaching in this life, \\
one proceeds to death still plagued by doubts. 

It’s\marginnote{4.1} like a man who has plunged into a river, \\
a rushing torrent in spate. \\
As they are swept away downstream, \\
how could they help others across? 

Just\marginnote{5.1} so, one unable to discern the teaching, \\
who hasn’t studied the meaning under the learned, \\
not knowing it oneself, still plagued by doubts, \\
how could they help others to contemplate? 

But\marginnote{6.1} one who has embarked on a strong boat \\
equipped with rudder and oar, \\
would bring many others across there \\
with skill, care, and intelligence. 

So\marginnote{7.1} too one who understands—a knowledge master, \\
evolved, learned, and unflappable—\\
can help others to contemplate, \\
so long as they are prepared to listen carefully. 

That’s\marginnote{8.1} why you should spend time with a good person, \\
intelligent and learned. \\
Having understood the meaning, putting it into practice, \\
one who has realized the teaching may find happiness. 

%
\end{verse}

%
\section*{{\suttatitleacronym Snp 2.9}{\suttatitletranslation What Morality? }{\suttatitleroot Kiṁsīlasutta}}
\addcontentsline{toc}{section}{\tocacronym{Snp 2.9} \toctranslation{What Morality? } \tocroot{Kiṁsīlasutta}}
\markboth{What Morality? }{Kiṁsīlasutta}
\extramarks{Snp 2.9}{Snp 2.9}

\begin{verse}%
“With\marginnote{1.1} what morality, what conduct, \\
fostering what deeds, \\
would a person lay the foundations right, \\
and reach the highest goal?” 

“Honoring\marginnote{2.1} elders without jealousy, \\
they’d know the time to visit their teachers. \\
Treasuring the chance for a Dhamma talk, \\
they’d listen carefully to the well-spoken words. 

At\marginnote{3.1} the right time, they’d humbly enter \\
the teachers’ presence, leaving obstinacy behind. \\
They’d call to mind and put into practice \\
the meaning, the teaching, self-control, and the spiritual life. 

Delighting\marginnote{4.1} in the teaching, enjoying the teaching, \\
standing on the teaching, investigating the teaching, \\
they’d never say anything that degraded the teaching, \\
but would be guided by genuine words well-spoken. 

Giving\marginnote{5.1} up mirth, prayer, weeping, ill will, \\
deception, fraud, greed, conceit, \\
aggression, crudeness, stains, and indulgence, \\
they’d wander free of vanity, steadfast. 

Understanding\marginnote{6.1} is the essence of well-spoken words, \\
stillness is the essence of learning and understanding. \\
Wisdom and learning do not flourish \\
in a hasty and negligent person. 

Those\marginnote{7.1} happy with the teaching proclaimed by the Noble One \\
are supreme in speech, mind, and deed. \\
Settled in peace, gentleness, and stillness, \\
they’ve realized the essence of learning and wisdom.” 

%
\end{verse}

%
\section*{{\suttatitleacronym Snp 2.10}{\suttatitletranslation Get Up! }{\suttatitleroot Uṭṭhānasutta}}
\addcontentsline{toc}{section}{\tocacronym{Snp 2.10} \toctranslation{Get Up! } \tocroot{Uṭṭhānasutta}}
\markboth{Get Up! }{Uṭṭhānasutta}
\extramarks{Snp 2.10}{Snp 2.10}

\begin{verse}%
Get\marginnote{1.1} up and meditate! \\
What’s the point in your sleeping? \\
How can the afflicted slumber \\
when injured by an arrow strike? 

Get\marginnote{2.1} up and meditate! \\
Train hard for peace! \\
The King of Death has caught you heedless—\\
don’t let him fool you under his sway. 

Needy\marginnote{3.1} gods and humans \\
are held back by clinging: \\
get over it. \\
Don’t let the moment pass you by. \\
For if you miss your moment \\
you’ll grieve when sent to hell. 

Negligence\marginnote{4.1} is always dust; \\
dust follows right behind negligence. \\
Through diligence and knowledge, \\
pluck out the dart from yourself. 

%
\end{verse}

%
\section*{{\suttatitleacronym Snp 2.11}{\suttatitletranslation With Rāhula }{\suttatitleroot Rāhulasutta}}
\addcontentsline{toc}{section}{\tocacronym{Snp 2.11} \toctranslation{With Rāhula } \tocroot{Rāhulasutta}}
\markboth{With Rāhula }{Rāhulasutta}
\extramarks{Snp 2.11}{Snp 2.11}

\begin{verse}%
“Does\marginnote{1.1} familiarity breed contempt, \\
even for the man of wisdom? \\
Do you honor he who holds aloft \\
the torch for all humanity?” 

“Familiarity\marginnote{2.1} breeds no contempt \\
for the man of wisdom. \\
I always honor he who holds aloft \\
the torch for all humanity.” 

“One\marginnote{3.1} who’s given up the five sensual stimulations, \\
so pleasing and delightful, \\
and who’s left the home life out of faith—\\
let them make an end to suffering! 

Mix\marginnote{4.1} with spiritual friends, \\
stay in remote lodgings, \\
secluded and quiet, \\
and eat in moderation. 

Robes,\marginnote{5.1} almsfood, \\
requisites and lodgings: \\
don’t crave such things; \\
don’t come back to this world again. 

Be\marginnote{6.1} restrained in the monastic code, \\
and the five sense faculties, \\
With mindfulness immersed in the body, \\
be full of disillusionment. 

Turn\marginnote{7.1} away from the feature of things \\
that’s attractive, provoking lust. \\
With mind unified and serene, \\
meditate on the ugly aspects of the body. 

Meditate\marginnote{8.1} on the signless, \\
give up the tendency to conceit; \\
and when you comprehend conceit, \\
you will live at peace.” 

%
\end{verse}

That\marginnote{9.1} is how the Buddha regularly advised Venerable \textsanskrit{Rāhula} with these verses. 

%
\section*{{\suttatitleacronym Snp 2.12}{\suttatitletranslation Vaṅgīsa and His Mentor Nigrodhakappa }{\suttatitleroot Nigrodhakappa (vaṅgīsa) sutta}}
\addcontentsline{toc}{section}{\tocacronym{Snp 2.12} \toctranslation{Vaṅgīsa and His Mentor Nigrodhakappa } \tocroot{Nigrodhakappa (vaṅgīsa) sutta}}
\markboth{Vaṅgīsa and His Mentor Nigrodhakappa }{Nigrodhakappa (vaṅgīsa) sutta}
\extramarks{Snp 2.12}{Snp 2.12}

\scevam{So\marginnote{1.1} I have heard. }At one time the Buddha was staying near \textsanskrit{Āḷavī}, at the \textsanskrit{Aggāḷava} Tree-shrine. Now at that time it was not long after Venerable \textsanskrit{Vaṅgīsa}’s mentor, the senior monk named Nigrodhakappa, had become extinguished. Then as \textsanskrit{Vaṅgīsa} was in private retreat this thought came to his mind: “Has my mentor become extinguished or not?” Then in the late afternoon, Venerable \textsanskrit{Vaṅgīsa} came out of retreat and went to the Buddha. He bowed, sat down to one side, and said to him: “Just now, sir, as I was in private retreat this thought came to mind. ‘Has my mentor become extinguished or not?’” Then Venerable \textsanskrit{Vaṅgīsa} got up from his seat, arranged his robe over one shoulder, raised his joined palms toward the Buddha, and addressed him in verse: 

\begin{verse}%
“I\marginnote{2.1} ask the teacher unrivaled in wisdom, \\
who has cut off all doubts in this very life: \\
a monk has died at \textsanskrit{Aggāḷava}, who was \\
well-known, famous, and quenched. 

Nigrodhakappa\marginnote{3.1} was his name; \\
it was given to that brahmin by you, Blessed One. \\
He wandered in your honor, yearning for freedom, \\
energetic, a resolute Seer of Truth. 

O\marginnote{4.1} Sakyan, all-seer, \\
we all wish to know about that disciple. \\
Our ears are eager to hear, \\
for you are the most excellent teacher. 

Cut\marginnote{5.1} off our doubt, declare this to us; \\
your wisdom is vast, tell us of his quenching! \\
All-seer, speak among us, \\
like the thousand-eyed Sakka in the midst of the gods! 

Whatever\marginnote{6.1} ties there are, or paths to delusion, \\
or things on the side of unknowing, or that are bases of doubt \\
vanish on reaching a Realized One, \\
for his eye is the best of all people’s. 

If\marginnote{7.1} no man were ever to disperse corruptions, \\
like the wind dispersing the clouds, \\
darkness would shroud the whole world; \\
not even brilliant men would shine. 

The\marginnote{8.1} wise are bringers of light; \\
my hero, that is what I think of you. \\
We’ve come for your discernment and knowledge: \\
here in this assembly, declare to us about \textsanskrit{Kappāyana}. 

Swiftly\marginnote{9.1} send forth your sweet, sweet voice, \\
like a goose stretching its neck, gently honking, \\
lucid-flowing, with lovely tone: \\
alert, we all listen to you. 

You\marginnote{10.1} have entirely abandoned birth and death; \\
restrained and pure, I urge you to speak the Dhamma! \\
For ordinary people have no wish-granter,\footnote{\textit{\textsanskrit{Kāmakāro}}: “wish-maker”. At Kv 23.3 \textit{\textsanskrit{issariyakāmakārikā}} means a “sovereign act of will”. Commentary here says ordinary people may not simply know or say what they want. } \\
but Realized Ones have a comprehensibility-granter.\footnote{\textit{\textsanskrit{Saṅkheyyakāro}}: “comprehensibility-maker”. \textit{\textsanskrit{Saṅkheyya}} is always used in the sense of “calculable, comprehensible”, not “after comprehension”. The point, in line with the previous and succeeding verses, is that the Buddha explains things in a way that makes them clear. It’s a clever pair of lines, best served by sticking close to the literal form. } 

Your\marginnote{11.1} answer is definitive, and we will adopt it, \\
for you have perfect understanding. \\
We raise our joined palms one last time, \\
one of unrivaled wisdom, don’t deliberately confuse us. 

Knowing\marginnote{12.1} the noble teaching from top to bottom, \\
unrivaled hero, don’t deliberately confuse us. \\
As a man in the baking summer sun would long for water, \\
I long for your voice, so let the sound rain down. 

Surely\marginnote{13.1} \textsanskrit{Kappāyana} did not lead the spiritual life in vain? \\
Did he realize quenching, \\
or did he still have a remnant of defilement? \\
Let us hear what kind of liberation he had!” 

“He\marginnote{14.1} cut off craving for mind and body in this very life,” \\
\scspeaker{said the Buddha, }\\
“the river of darkness that had long lain within him. \\
He has entirely crossed over birth and death.” \\
So declared the Blessed One, the leader of the five. 

“Now\marginnote{15.1} that I have heard your words, \\
seventh of sages, I am confident. \\
My question, it seems, was not in vain, \\
the brahmin did not deceive me. 

As\marginnote{16.1} he said, so he did—\\
he was a disciple of the Buddha. \\
He cut the net of death the deceiver, \\
so extended and strong. 

Blessed\marginnote{17.1} One, \textsanskrit{Kappāyana} saw \\
the starting point of grasping. \\
He has indeed gone far beyond \\
Death’s domain so hard to pass.” 

%
\end{verse}

%
\section*{{\suttatitleacronym Snp 2.13}{\suttatitletranslation The Right Way to Wander }{\suttatitleroot Sammāparibbājanīyasutta}}
\addcontentsline{toc}{section}{\tocacronym{Snp 2.13} \toctranslation{The Right Way to Wander } \tocroot{Sammāparibbājanīyasutta}}
\markboth{The Right Way to Wander }{Sammāparibbājanīyasutta}
\extramarks{Snp 2.13}{Snp 2.13}

\begin{verse}%
“I\marginnote{1.1} ask the sage abounding in wisdom—\\
crossed-over, gone beyond, quenched, steadfast: \\
when a mendicant has left home, expelling sensuality, \\
what’s the right way to wander the world?” 

“When\marginnote{2.1} they’ve eradicated superstitions,” \\
\scspeaker{said the Buddha, }\\
“about celestial portents, dreams, or bodily marks; \\
with the stain of superstitions left behind, \\
they’d rightly wander the world. 

A\marginnote{3.1} mendicant ought dispel desire \\
for pleasures human or divine; \\
with rebirth transcended and truth comprehended, \\
they’d rightly wander the world. 

Putting\marginnote{4.1} divisiveness behind them, \\
a mendicant gives up anger and stinginess; \\
with favoring and opposing left behind, \\
they’d rightly wander the world. 

When\marginnote{5.1} the loved and the unloved are both left behind, \\
not grasping or dependent on anything; \\
freed from all things that fetter, \\
they’d rightly wander the world. 

Finding\marginnote{6.1} no substance in attachments, \\
rid of desire for things they’ve acquired, \\
independent, needing no-one to guide them, \\
they’d rightly wander the world. 

Not\marginnote{7.1} hostile in speech, mind, or deed, \\
they’ve rightly understood the teaching. \\
Aspiring to the state of quenching, \\
they’d rightly wander the world. 

Not\marginnote{8.1} pridefully thinking, ‘they bow to me’; \\
though reviled, they’d still stay in touch;\footnote{Neither Norman nor Bodhi note \textit{sandhiyatimeva} in identical context at AN 3.132:3.2. There Bodhi has “remains on friendly terms”, I have “stay in touch”. I assume there is a confusion of the negative which is required to give the correct sense. If I’m right, the corruption must predate the commentary. } \\
not besotted when getting food from others, \\
they’d rightly wander the world. 

When\marginnote{9.1} greed and craving to live again are cast off, \\
a mendicant refrains from violence and abduction; \\
rid of doubt, free of thorns, \\
they’d rightly wander the world. 

Knowing\marginnote{10.1} what is suitable for themselves, \\
a mendicant would hurt no-one in the world; \\
understanding the teaching in accord with reality, \\
they’d rightly wander the world. 

They\marginnote{11.1} have no underlying tendencies at all, \\
and are rid of unskillful roots; \\
free of hope, with no need for hope, \\
they’d rightly wander the world. 

Defilements\marginnote{12.1} ended, conceit given up, \\
beyond all manner of desire; \\
tamed, quenched, and steadfast, \\
they’d rightly wander the world. 

Faithful,\marginnote{13.1} learned, seer of the sure path, \\
the wise one takes no side among factions; \\
rid of greed, hate, and repulsion, \\
they’d rightly wander the world. 

A\marginnote{14.1} purified victor with veil drawn back, \\
among worldly things master, transcendent, stilled; \\
expert in knowledge of conditions’ cessation, \\
they’d rightly wander the world. 

They’re\marginnote{15.1} over speculating on the future or past, \\
and understand what it means to be pure; \\
freed from all the sense fields, \\
they’d rightly wander the world. 

The\marginnote{16.1} state of peace is understood, the truth is comprehended, \\
they’ve openly seen defilements cast off; \\
and with the ending of all attachments, \\
they’d rightly wander the world.” 

“Clearly,\marginnote{17.1} Blessed One, it is just as you say. \\
One who lives like this is a tamed mendicant, \\
beyond all fetters and yokes: \\
they’d rightly wander the world.” 

%
\end{verse}

%
\section*{{\suttatitleacronym Snp 2.14}{\suttatitletranslation With Dhammika }{\suttatitleroot Dhammikasutta}}
\addcontentsline{toc}{section}{\tocacronym{Snp 2.14} \toctranslation{With Dhammika } \tocroot{Dhammikasutta}}
\markboth{With Dhammika }{Dhammikasutta}
\extramarks{Snp 2.14}{Snp 2.14}

\scevam{So\marginnote{1.1} I have heard. }At one time the Buddha was staying near \textsanskrit{Sāvatthī} in Jeta’s Grove, \textsanskrit{Anāthapiṇḍika}’s monastery. Then the lay follower Dhammika, together with five hundred lay followers, went up to the Buddha, bowed, sat down to one side, and addressed him in verse: 

\begin{verse}%
“I\marginnote{2.1} ask you, Gotama, whose wisdom is vast: \\
what does one do to become a good disciple, \\
both one who has left the home, \\
and the lay followers staying at home? 

For\marginnote{3.1} you understand the course and destiny \\
of the world with all its gods. \\
There is no equal to you who sees the subtle meaning, \\
for you are the Buddha most excellent, they say. 

Having\marginnote{4.1} experienced all knowledge, \\
you explain the teaching out of compassion for beings. \\
All-seer, you have drawn back the veil, \\
and immaculate, you shine on the whole world. 

The\marginnote{5.1} dragon king \textsanskrit{Erāvaṇa}, hearing you called ‘Victor’, \\
came into your presence. \\
He consulted with you then, having heard your words, \\
left consoled, saying ‘Excellent!’ 

And\marginnote{6.1} King Kuvera \textsanskrit{Vessavaṇa} also \\
approached to ask about the teaching. \\
You also answered him, O wise one, \\
and hearing you he too was consoled. 

Those\marginnote{7.1} teachers of other paths given to debate, \\
whether \textsanskrit{Ājīvakas} or Jains, \\
all fail to overtake you in wisdom, \\
like a standing man next to a sprinter. 

Those\marginnote{8.1} brahmins given to debate, \\
some of whom are quite senior, \\
all end up beholden to you for the meaning, \\
and others too who think themselves debaters. 

So\marginnote{9.1} subtle and pleasant is the teaching \\
that is well proclaimed by you, Blessed One. \\
It’s all we long to hear. So when asked, \\
O Best of Buddhas, tell us! 

All\marginnote{10.1} these mendicants have gathered, \\
and the layfolk too are here to listen. \\
Let them hear the teaching the immaculate one discovered, \\
like gods listening to the fine words of \textsanskrit{Vāsava}.” 

“Listen\marginnote{11.1} to me, mendicants, I will educate you \\
in the cleansing teaching; all bear it in mind. \\
An intelligent person, seeing the meaning, \\
would adopt the deportment proper to a renunciate.\footnote{\textit{Atthadaso}: Bodhi follows comm in “seeing the good”, Norman has “seeing the goal”. I usually have “seeing the meaning”, and I think that is supported here. It is at the start of the practice, after several lines talking about the teaching. The point is, I think, that a reflective person doesn't just hear the teaching, but understands the point of it and puts it into practice. } 

No\marginnote{12.1} way would a mendicant go out at the wrong time; \\
at the right time, they’d walk the village for alms. \\
For chains bind one who wanders outside the right time, \\
which is why the Buddhas avoid it. 

Sights,\marginnote{13.1} sounds, tastes, smells, and touches, \\
which drive beings mad—\\
dispel desire for such things, \\
and enter for the morning meal at the right time. 

After\marginnote{14.1} receiving alms for the day, \\
on returning a mendicant would sit in private alone. \\
Inwardly reflective, they’d curb their mind \\
from outside things, keeping themselves collected. 

Should\marginnote{15.1} they converse with a disciple, \\
with anyone else, or with a mendicant, \\
they’d bring up only the sublime teaching, \\
not dividing or blaming. 

For\marginnote{16.1} some contend in debate, \\
but we praise not those of little wisdom. \\
In place after place they are bound in chains, \\
for they send their mind over there far away. 

Alms,\marginnote{17.1} a dwelling, a bed and seat, \\
and water for rinsing the dust from the cloak—\\
after hearing the teaching of the Holy One, \\
a disciple of splendid wisdom would use these after appraisal. 

That’s\marginnote{18.1} why, when it comes to alms and lodgings, \\
and water for rinsing the dust from the cloak, \\
a mendicant is unsullied in the midst of these things, \\
like a droplet on a lotus-leaf. 

Now\marginnote{19.1} I shall tell you the householder’s duty, \\
doing which one becomes a good disciple. \\
For one burdened with possessions does not get to realize \\
the whole of the mendicant’s practice. 

They’d\marginnote{20.1} not kill any creature, nor have them killed, \\
nor grant permission for others to kill. \\
They’ve laid aside violence towards all creatures \\
frail or firm that there are in the world. 

Next,\marginnote{21.1} a disciple would avoid knowingly \\
taking anything not given at all, \\
they’d not get others to do it, nor grant them permission to steal; \\
they’d avoid \emph{all} theft. 

A\marginnote{22.1} sensible person would avoid the unchaste life, \\
like a burning pit of coals. \\
But if unable to remain chaste, \\
they’d not transgress with another’s partner. 

In\marginnote{23.1} a council or assembly, \\
or one on one, they would not lie. \\
They’d not get others to lie, nor grant them permission to lie; \\
they’d avoid \emph{all} untruths. 

A\marginnote{24.1} householder espousing this teaching \\
would not consume liquor or drink. \\
They’d not get others to drink, nor grant them permission to drink; \\
knowing that ends in intoxication. 

For\marginnote{25.1} drunken fools do bad things, \\
and encourage other heedless folk. \\
Reject this field of demerit, \\
the maddening, deluding frolic of fools. 

You\marginnote{26.1} shouldn’t kill living creatures, or steal, \\
or lie, or drink alcohol. \\
Be celibate, refraining from sex, \\
and don’t eat at night, the wrong time. 

Not\marginnote{27.1} wearing garlands or applying perfumes, \\
you should sleep on a low bed, or a mat on the ground. \\
This is the eight-factored sabbath, they say, \\
explained by the Buddha, who has gone to suffering’s end. 

Then\marginnote{28.1} having rightly undertaken the sabbath \\
complete in all its eight factors \\
on the fourteenth, fifteenth, and eighth of the fortnight, \\
as well as on the fortnightly special displays, 

on\marginnote{29.1} the morning after the sabbath \\
a clever person, rejoicing with confident heart, \\
would distribute food and drink \\
to the mendicant \textsanskrit{Saṅgha} as is fitting. 

One\marginnote{30.1} should rightfully support one’s parents, \\
and undertake a legitimate business. \\
A diligent layperson observing these duties \\
ascends to the gods called Self-luminous.” 

%
\end{verse}

%
\addtocontents{toc}{\let\protect\contentsline\protect\nopagecontentsline}
\chapter*{The Great Chapter }
\addcontentsline{toc}{chapter}{\tocchapterline{The Great Chapter }}
\addtocontents{toc}{\let\protect\contentsline\protect\oldcontentsline}

%
\section*{{\suttatitleacronym Snp 3.1}{\suttatitletranslation Going Forth }{\suttatitleroot Pabbajjāsutta}}
\addcontentsline{toc}{section}{\tocacronym{Snp 3.1} \toctranslation{Going Forth } \tocroot{Pabbajjāsutta}}
\markboth{Going Forth }{Pabbajjāsutta}
\extramarks{Snp 3.1}{Snp 3.1}

\begin{verse}%
“I\marginnote{1.1} shall extol going forth \\
with the example of the seer, \\
the course of inquiry that led to \\
his choice to go forth. 

‘This\marginnote{2.1} life at home is cramped, \\
a realm of dirt.’ \\
‘The life of one gone forth is like an open space.’ \\
Seeing this, he went forth. 

Having\marginnote{3.1} gone forth, he shunned \\
bad deeds of body. \\
And leaving verbal misconduct behind, \\
he purified his livelihood. 

The\marginnote{4.1} Buddha went to \textsanskrit{Rājagaha}, \\
the Mountainfold of the Magadhans. \\
He betook himself for alms, \\
replete with excellent marks. 

\textsanskrit{Bimbisāra}\marginnote{5.1} saw him \\
while standing atop his longhouse. \\
Noticing that he was endowed with marks, \\
he said the following: 

‘Pay\marginnote{6.1} heed, sirs, to this one, \\
handsome, majestic, radiant;\footnote{\textit{Suci} here refers to his appearance, as agreed by the commentary, so “pure” is not very useful. } \\
accomplished in deportment, \\
he looks just a plough’s length in front. 

Eyes\marginnote{7.1} downcast, mindful, \\
unlike one from a low family. \\
Let the king’s messengers run out, \\
and find where the mendicant will go.’ 

The\marginnote{8.1} messengers sent out \\
followed right behind, thinking \\
‘Where will the mendicant go? \\
Where shall he find a place to stay?’ 

Wandering\marginnote{9.1} indiscriminately for alms, \\
sense-doors guarded and well restrained, \\
his bowl was quickly filled, \\
aware and mindful. 

Having\marginnote{10.1} wandered for alms, \\
the sage left the city. \\
He betook himself to Mount \textsanskrit{Paṇḍava}, \\
thinking, ‘Here is the place I shall stay.’ 

Seeing\marginnote{11.1} that he had arrived at a place to stay, \\
the messengers withdrew nearby,\footnote{Reading \textit{tato}. } \\
but one of them returned \\
to inform the king. 

‘Great\marginnote{12.1} king, the mendicant \\
is on the east flank of Mount \textsanskrit{Paṇḍava}. \\
There he sits, like a tiger or a bull, \\
like a lion in a mountain cave.’ 

Hearing\marginnote{13.1} the messenger’s report, \\
the aristocrat set out \\
hurriedly in his fine chariot \\
towards Mount \textsanskrit{Paṇḍava}. 

He\marginnote{14.1} went as far as vehicles could go, \\
then dismounted from his chariot, \\
approached on foot, \\
and reaching him, drew near. 

Seated,\marginnote{15.1} the king greeted him \\
and made polite conversation. \\
When the courtesies were over, \\
he said the following: 

‘You\marginnote{16.1} are young, just a youth, \\
a lad in the prime of life. \\
You are endowed with beauty and stature, \\
like an aristocrat of good lineage 

in\marginnote{17.1} glory at the army’s head, \\
surrounded by a troop of elephants. \\
I offer you pleasures—enjoy them! \\
But please tell me your lineage by birth.’ 

‘Up\marginnote{18.1} north lies a nation, great king, \\
on the slope of the Himalayas, \\
full of wealth and strength, \\
led by one loyal to the Kosalans.\footnote{Following the suggestion in PTS dict under \textit{niketa} that it is related to \textit{ketu}. } 

They\marginnote{19.1} are of the Solar clan, \\
their lineage is the Sakyans. \\
I have gone forth from that family—\\
I do not yearn for sensual pleasure. 

Seeing\marginnote{20.1} the danger in sensual pleasures, \\
seeing renunciation as sanctuary, \\
I shall go on to strive; \\
that is where my mind delights.’” 

%
\end{verse}

%
\section*{{\suttatitleacronym Snp 3.2}{\suttatitletranslation Striving }{\suttatitleroot Padhānasutta}}
\addcontentsline{toc}{section}{\tocacronym{Snp 3.2} \toctranslation{Striving } \tocroot{Padhānasutta}}
\markboth{Striving }{Padhānasutta}
\extramarks{Snp 3.2}{Snp 3.2}

\begin{verse}%
“During\marginnote{1.1} my time of resolute striving \\
on the bank of the \textsanskrit{Nerañjara} River, \\
I was meditating very hard \\
for the sake of finding sanctuary. 

\textsanskrit{Namucī}\marginnote{2.1} approached, \\
speaking words of kindness: \\
‘You’re thin, discolored, \\
on the verge of death. 

Death\marginnote{3.1} has a thousand parts of you, \\
one fraction is left to life. \\
Live sir! Life is better! \\
Living, you can make merits. 

While\marginnote{4.1} leading the spiritual life \\
and serving the sacred flame, \\
you can pile up abundant merit—\\
so what will striving do for you? 

Hard\marginnote{5.1} to walk is the path of striving, \\
hard to do, a hard challenge to win.’” \\
These are the verses \textsanskrit{Māra} spoke \\
as he stood beside the Buddha. 

When\marginnote{6.1} \textsanskrit{Māra} had spoken in this way, \\
the Buddha said this: \\
“O Wicked One, kinsman of the negligent, \\
you’re here for your own purpose. 

I\marginnote{7.1} have no need for \\
the slightest bit of merit. \\
Those with need for merit \\
are fit for \textsanskrit{Māra} to address. 

I\marginnote{8.1} have faith and energy too, \\
and wisdom is found in me. \\
When I am so resolute, \\
why do you beg me to live? 

The\marginnote{9.1} rivers and streams \\
might be dried by the wind, \\
so why, when I am resolute, \\
should it not dry up my blood? 

And\marginnote{10.1} while the blood is drying up, \\
the bile and phlegm dry too. \\
And as my muscles waste away, \\
my mind grows more serene. \\
And all the stronger grow mindfulness \\
and wisdom and immersion. 

As\marginnote{11.1} I meditate like this, \\
having attained the supreme feeling,\footnote{While this whole sutta flirts with the idea of self-mortification, the commentary, followed by Bodhi, takes it to the next level here, identifying \textit{\textsanskrit{vedanā}} as extreme suffering. No doubt in the previous verse such suffering might apply, but we have just been told that the faculties, including \textit{\textsanskrit{samādhi}} have grown strong. The next line indicates freedom from sensuality. All this agrees with the normal Sutta description of the feeling in \textsanskrit{jhāna}. } \\
my mind has no interest in sensual pleasures: \\
behold a being’s purity! 

Sensual\marginnote{12.1} pleasures are your first army, \\
the second is called discontent, \\
hunger and thirst are the third, \\
and the fourth is said to be craving. 

Your\marginnote{13.1} fifth is dullness and drowsiness, \\
the sixth is said to be cowardice, \\
your seventh is doubt, \\
contempt and obstinacy are your eighth. 

Profit,\marginnote{14.1} praise, and honor, \\
and misbegotten fame; \\
the extolling of oneself \\
while scorning others. 

This\marginnote{15.1} is your army, \textsanskrit{Namucī}, \\
the strike force of the Dark One. \\
Only a hero can defeat it, \\
but in victory there lies bliss. 

Let\marginnote{16.1} me gird myself—\footnote{The meaning of these lines is quite obscure. The \textit{\textsanskrit{muñja}} was used for the brahmin’s girdle, and as such became an epithet of Vishnu and Shiva. The sense of “girdle” in English conveys the idea of preparing oneself for a righteous challenge, “gird thy loins”. I take \textit{parihare} as optative. } \\
so what if I die!\footnote{\textit{Dhiratthu} is a curse, lit. “damn my life”. But it's hard to render as a literal curse without sounding like it’s weird, swearing, or casting a spell. } \\
I’d rather die in battle \\
than live on in defeat. 

Here\marginnote{17.1} some ascetics and brahmins \\
are swallowed up, not to be seen again. \\
They do not know the path \\
traveled by those true to their vows. 

Seeing\marginnote{18.1} \textsanskrit{Māra} ready on his mount, \\
surrounded by his bannered forces, \\
I shall meet them in battle—\\
they’ll never make me retreat! 

That\marginnote{19.1} army of yours has never been beaten \\
by the world with all its gods. \\
Yet I shall smash it with wisdom, \\
like an unfired pot with a stone.\footnote{And unfired pot also used as simile for weakness at MN 122:27.1. } 

Having\marginnote{20.1} brought my thoughts under control, \\
and established mindfulness well, \\
I shall wander from country to country, \\
guiding many disciples. 

Diligent\marginnote{21.1} and resolute, \\
following my instructions, \\
they will proceed despite your will, \\
to where there is no sorrow.” 

“For\marginnote{22.1} seven years I followed \\
step by step behind the Blessed One. \\
I found no vulnerability \\
in the mindful Awakened One. 

A\marginnote{23.1} crow once circled a stone \\
that looked like a lump of fat. \\
‘Perhaps I’ll find something tender,’ it thought, \\
‘perhaps there’s something tasty.’ 

But\marginnote{24.1} finding nothing tasty, \\
the crow left that place. \\
Like the crow that pecked the stone, \\
I leave Gotama disappointed.” 

So\marginnote{25.1} stricken with sorrow \\
that his harp dropped from his armpit, \\
that spirit, downcast, \\
vanished right there. 

%
\end{verse}

%
\section*{{\suttatitleacronym Snp 3.3}{\suttatitletranslation Well-Spoken Words }{\suttatitleroot Subhāsitasutta}}
\addcontentsline{toc}{section}{\tocacronym{Snp 3.3} \toctranslation{Well-Spoken Words } \tocroot{Subhāsitasutta}}
\markboth{Well-Spoken Words }{Subhāsitasutta}
\extramarks{Snp 3.3}{Snp 3.3}

\scevam{So\marginnote{1.1} I have heard. }At one time the Buddha was staying near \textsanskrit{Sāvatthī} in Jeta’s Grove, \textsanskrit{Anāthapiṇḍika}’s monastery. There the Buddha addressed the mendicants, “Mendicants!” “Venerable sir,” they replied. The Buddha said this: 

“Mendicants,\marginnote{2.1} speech that has four factors is well spoken, not poorly spoken. It’s blameless and is not criticized by sensible people. What four? It’s when a mendicant speaks well, not poorly; they speak on the teaching, not against the teaching; they speak pleasantly, not unpleasantly; and they speak truthfully, not falsely. Speech with these four factors is well spoken, not poorly spoken. It’s blameless and is not criticized by sensible people.” That is what the Buddha said. Then the Holy One, the Teacher, went on to say: 

\begin{verse}%
“Good\marginnote{3.1} people say that well-spoken words are foremost; \\
second, speak on the teaching, not against it; \\
third, speak pleasantly, not unpleasantly; \\
and fourth, speak truthfully, not falsely.” 

%
\end{verse}

Then\marginnote{4.1} Venerable \textsanskrit{Vaṅgīsa} got up from his seat, arranged his robe over one shoulder, raised his joined palms toward the Buddha, and said, “I feel inspired to speak, Blessed One! I feel inspired to speak, Holy One!” “Then speak as you feel inspired,” said the Buddha. Then \textsanskrit{Vaṅgīsa} extolled the Buddha in his presence with fitting verses: 

\begin{verse}%
“Speak\marginnote{5.1} only such words \\
that do not hurt yourself \\
nor harm others; \\
such speech is truly well spoken. 

Speak\marginnote{6.1} only pleasing words, \\
words gladly welcomed. \\
Pleasing words are those \\
that bring nothing bad to others. 

Truth\marginnote{7.1} itself is the undying word: \\
this is an eternal truth. \\
Good people say that the teaching and its meaning \\
are grounded in the truth. 

The\marginnote{8.1} words spoken by the Buddha \\
for realizing the sanctuary, extinguishment, \\
for the attainment of vision, \\
this really is the best kind of speech.” 

%
\end{verse}

%
\section*{{\suttatitleacronym Snp 3.4}{\suttatitletranslation With Bhāradvāja of Sundarikā on the Sacrificial Cake }{\suttatitleroot Pūraḷāsa (sundarikabhāradvāja) sutta}}
\addcontentsline{toc}{section}{\tocacronym{Snp 3.4} \toctranslation{With Bhāradvāja of Sundarikā on the Sacrificial Cake } \tocroot{Pūraḷāsa (sundarikabhāradvāja) sutta}}
\markboth{With Bhāradvāja of Sundarikā on the Sacrificial Cake }{Pūraḷāsa (sundarikabhāradvāja) sutta}
\extramarks{Snp 3.4}{Snp 3.4}

\scevam{So\marginnote{1.1} I have heard. }At one time the Buddha was staying in the Kosalan lands on the bank of the \textsanskrit{Sundarikā} river. Now at that time the brahmin \textsanskrit{Sundarikabhāradvāja} was serving the sacred flame and performing the fire sacrifice on the bank of the \textsanskrit{Sundarikā}. Then he looked all around the four quarters, wondering, “Now who might eat the leftovers of this offering?” He saw the Buddha meditating at the root of a certain tree with his robe pulled over his head. Taking the leftovers of the offering in his left hand and a pitcher in the right he approached the Buddha. 

When\marginnote{2.1} he heard \textsanskrit{Sundarikabhāradvāja}’s footsteps the Buddha uncovered his head. \textsanskrit{Sundarikabhāradvāja} thought, “This man is shaven, he is shaven!” And he wanted to turn back. But he thought, “Even some brahmins are shaven. Why don’t I go to him and ask about his birth?” Then the brahmin \textsanskrit{Sundarikabhāradvāja} went up to the Buddha, and said to him, “Sir, in what caste were you born?” 

Then\marginnote{3.1} the Buddha addressed \textsanskrit{Sundarikabhāradvāja} in verse: 

\begin{verse}%
“I\marginnote{4.1} am no brahmin, nor am I a prince, \\
nor merchant nor anything else. \\
Fully understanding the clan of ordinary people, \\
I wander in the world owning nothing, reflective. 

Clad\marginnote{5.1} in my cloak, I wander without home, \\
my hair shorn, quenched. \\
Since I’m unburdened by youngsters,\footnote{Norman and Bodhi both follow comm. here without comment: \textit{manussehi}. But \textit{\textsanskrit{māṇava}} is not used elsewhere in this sense (PTS Dict’s ref to Thig 7.1 notwithstanding), nor does such a sense seem to be attested in the Sanskrit dictionaries. The context is the Buddha’s unattachment to “clan” and it surely has the sense here of “youngling”, either children or students. } \\
it’s inappropriate to ask me about clan.” 

“Actually\marginnote{6.1} sir, when brahmins meet they politely \\
ask each other whether they are brahmins.” 

“Well,\marginnote{7.1} if you say that you’re a brahmin, \\
and that I am not, \\
I shall question you on the \textsanskrit{Gāyatrī} Mantra,\footnote{It’s a translation: we should use the normal word used these days. } \\
with its three lines and twenty-four syllables.” 

“On\marginnote{8.1} what grounds have hermits and men, \\
aristocrats and brahmins here in the world \\
performed so many different sacrifices to the gods?” 

“During\marginnote{9.1} a sacrifice, should a past master, a knowledge master, \\
receive an oblation, it profits the donor, I say.” 

“Then\marginnote{10.1} clearly my oblation will be profitable,” \\
\scspeaker{said the brahmin, }\\
“since I have met such a knowledge master. \\
It’s because I’d never met anyone like you \\
that others ate the sacrificial cake.” 

“So\marginnote{11.1} then, brahmin, since you have approached me \\
as a seeker of the good, ask.\footnote{Norman and Bodhi both have “approach and ask”, but \textit{\textsanskrit{upasaṅkamma}} is absolutive: he is already there. } \\
Perhaps you may find here someone intelligent, \\
peaceful, unclouded, untroubled, with no need for hope.” 

“Master\marginnote{12.1} Gotama, I like to sacrifice \\
and wish to perform a sacrifice. Please advise me, \\
for I do not understand \\
where an oblation is profitable; tell me this.” 

%
\end{verse}

“Well\marginnote{13.1} then, brahmin, lend an ear, I will teach you the Dhamma. 

\begin{verse}%
Don’t\marginnote{14.1} ask about birth, ask about conduct; \\
for any wood can surely generate fire. \\
A steadfast sage, even though from a low class family, \\
is a thoroughbred checked by conscience. 

Tamed\marginnote{15.1} by truth, fulfilled by taming, \\
a complete knowledge master who has completed the spiritual journey—\\
that is where a brahmin seeking merit \\
should bestow a timely offering as sacrifice. 

Those\marginnote{16.1} who have left sensuality behind, wandering homeless, \\
self-controlled, straight as a shuttle—\\
that is where a brahmin seeking merit \\
should bestow a timely offering as sacrifice. 

Those\marginnote{17.1} freed of greed, with senses stilled, \\
like the moon released from the eclipse—\\
that is where a brahmin seeking merit \\
should bestow a timely offering as sacrifice. 

They\marginnote{18.1} wander the world unimpeded, \\
always mindful, calling nothing their own—\\
that is where a brahmin seeking merit \\
should bestow a timely offering as sacrifice. 

Having\marginnote{19.1} left sensuality behind, wandering triumphant, \\
knowing the end of rebirth and death, \\
extinguished and cool as a lake: \\
the Realized One is worthy of the sacrificial cake. 

Good\marginnote{20.1} among the good, far from the bad, \\
the Realized One has infinite wisdom. \\
Unsullied in this world and the next: \\
the Realized One is worthy of the sacrificial cake. 

In\marginnote{21.1} whom dwells no deceit or conceit, \\
rid of greed, unselfish, with no need for hope, \\
with anger eliminated, quenched, \\
a brahmin rid of sorrow’s stain: \\
the Realized One is worthy of the sacrificial cake. 

He\marginnote{22.1} has given up the mind’s home, \\
and has no possessions at all. \\
Not grasping to this world or the next: \\
the Realized One is worthy of the sacrificial cake. 

Serene,\marginnote{23.1} he has crossed the flood, \\
and has understood the teaching with ultimate view. \\
With defilements ended, bearing his final body: \\
the Realized One is worthy of the sacrificial cake. 

In\marginnote{24.1} whom desire to be reborn, and caustic speech \\
are cleared and ended, they are no more; \\
that knowledge master, everywhere free: \\
the Realized One is worthy of the sacrificial cake. 

He\marginnote{25.1} has escaped his chains, he’s chained no more, \\
among those caught in conceit he is free of conceit; \\
he has fully understood suffering with its field and ground: \\
the Realized One is worthy of the sacrificial cake. 

Not\marginnote{26.1} relying on hope, seeing seclusion, \\
well past the views proclaimed by others. \\
In him there are no supporting conditions at all: \\
the Realized One is worthy of the sacrificial cake. 

He\marginnote{27.1} has comprehended all things, high and low, \\
cleared them and ended them, so they are no more. \\
Peaceful, freed in the ending of grasping: \\
the Realized One is worthy of the sacrificial cake. 

He\marginnote{28.1} sees the utter ending of rebirth’s fetter, \\
and has swept away all manner of desire. \\
Pure, stainless, immaculate, flawless: \\
the Realized One is worthy of the sacrificial cake. 

Not\marginnote{29.1} seeing himself in terms of a self, \\
he is stilled, upright, and steadfast. \\
Imperturbable, kind, wishless: \\
the Realized One is worthy of the sacrificial cake. 

He\marginnote{30.1} harbors no delusions within at all, \\
he has insight into all things. \\
He bears his final body, \\
attained to the state of grace, the supreme awakening. \\
That’s how the purity of a spirit is defined: \\
the Realized One is worthy of the sacrificial cake.” 

“Let\marginnote{31.1} my oblation be a true offering, \\
since I have found such a knowledge master! \\
I see \textsanskrit{Brahmā} in person! Accept my offering, Blessed One: \\
please eat my sacrificial cake.” 

“Food\marginnote{32.1} enchanted by a spell isn’t fit for me to eat. \\
That’s not the principle of those who see, brahmin. \\
The Buddhas reject things enchanted with spells. \\
Since there is such a principle, brahmin, that’s how they live. 

Serve\marginnote{33.1} with other food and drink \\
the consummate one, the great hermit, \\
with defilements ended and remorse stilled. \\
For he is the field for the seeker of merit.” 

“Please,\marginnote{34.1} Blessed One, help me understand: \\
now that I have encountered your teaching, \\
when I look for someone during a sacrifice, \\
who should eat the religious donation of one like me?” 

“One\marginnote{35.1} who is rid of aggression, \\
whose mind is unclouded, \\
who is liberated from sensual pleasures, \\
and who has dispelled dullness. 

One\marginnote{36.1} who has erased boundaries and limits, \\
expert in birth and death, \\
a sage, blessed with sagacity. \\
When such a person comes to the sacrifice, 

get\marginnote{37.1} rid of your scowl! \\
Honor them with joined palms, \\
and venerate them with food and drink, \\
and in this way your religious donation will succeed.” 

“The\marginnote{38.1} Buddha is worthy of the sacrificial cake, \\
he is the supreme field of merit, \\
Recipient of gifts from the whole world, \\
what’s given to the worthy one is very fruitful.” 

%
\end{verse}

Then\marginnote{39.1} the brahmin \textsanskrit{Sundarikabhāradvāja} said to the Buddha, “Excellent, Master Gotama! Excellent! As if he were righting the overturned, or revealing the hidden, or pointing out the path to the lost, or lighting a lamp in the dark so people with good eyes can see what’s there, Master Gotama has made the teaching clear in many ways. I go for refuge to Master Gotama, to the teaching, and to the mendicant \textsanskrit{Saṅgha}. Sir, may I receive the going forth, the ordination in the Buddha’s presence?” And the brahmin Sundarika \textsanskrit{Bhāradvāja} received the going forth, the ordination in the Buddha’s presence. And soon after, he became one of the perfected. 

%
\section*{{\suttatitleacronym Snp 3.5}{\suttatitletranslation With Māgha }{\suttatitleroot Māghasutta}}
\addcontentsline{toc}{section}{\tocacronym{Snp 3.5} \toctranslation{With Māgha } \tocroot{Māghasutta}}
\markboth{With Māgha }{Māghasutta}
\extramarks{Snp 3.5}{Snp 3.5}

\scevam{So\marginnote{1.1} I have heard. }At one time the Buddha was staying near \textsanskrit{Rājagaha}, on the Vulture’s Peak Mountain. Then the brahmin student \textsanskrit{Māgha} approached the Buddha and exchanged greetings with him. When the greetings and polite conversation were over, he sat down to one side, and said to the Buddha: 

“I’m\marginnote{2.1} a giver, Master Gotama, a donor; I am bountiful and committed to charity. I seek wealth in a principled manner, and with that legitimate wealth I give to one person, to two, three, four, five, six, seven, eight, nine, ten, twenty, thirty, forty, fifty, a hundred people or even more. Giving and sacrificing like this, Master Gotama, do I accrue much merit?” 

“Indeed\marginnote{3.1} you do, student. A giver or donor who is bountiful and committed to charity, who seeks wealth in a principled manner, and with that legitimate wealth gives to one person, or up to a hundred people or even more, accrues much merit.” Then \textsanskrit{Māgha} addressed the Buddha in verse: 

\begin{verse}%
“I\marginnote{4.1} ask the bountiful Gotama,” \\
\scspeaker{said \textsanskrit{Māgha}, }\\
“wearing an ochre robe, wandering homeless. \\
Suppose a lay donor who is committed to charity \\
makes a sacrifice seeking merit, looking for merit. \\
Giving food and drink to others here, \\
how is their offering purifed?” 

“Suppose\marginnote{5.1} a lay donor who is committed to charity,” \\
\scspeaker{replied the Buddha, }\\
“makes a sacrifice seeking merit, looking for merit, \\
giving food and drink to others here: \\
such a one would succeed due to those who are worthy of donations.”\footnote{Bodhi translates \textit{\textsanskrit{ārādhaye}} as simple present tense here (“succeeds”) and in the identical line below as imperative (“should accomplish”). Norman has the optative mood in both cases, which is surely correct. } 

“Suppose\marginnote{6.1} a lay donor who is committed to charity,” \\
\scspeaker{said \textsanskrit{Māgha}, }\\
“makes a sacrifice seeking merit, looking for merit, \\
giving food and drink to others here: \\
explain to me who is worthy of donations.” 

“Those\marginnote{7.1} who wander the world unattached, \\
consummate, restrained, owning nothing: \\
that is where a brahmin seeking merit \\
should bestow a timely offering as sacrifice. 

Those\marginnote{8.1} who have cut off all fetters and bonds, \\
tamed, liberated, untroubled, with no need for hope: \\
that is where a brahmin seeking merit \\
should bestow a timely offering as sacrifice. 

Those\marginnote{9.1} who are released from all fetters, \\
tamed, liberated, untroubled, with no need for hope: \\
that is where a brahmin seeking merit \\
should bestow a timely offering as sacrifice. 

Having\marginnote{10.1} given up greed, hate, and delusion, \\
with defilements ended, the spiritual journey completed: \\
that is where a brahmin seeking merit \\
should bestow a timely offering as sacrifice. 

Those\marginnote{11.1} in whom dwells no deceit or conceit, \\
with defilements ended, the spiritual journey completed: \\
that is where a brahmin seeking merit \\
should bestow a timely offering as sacrifice. 

Those\marginnote{12.1} rid of greed, unselfish, with no need for hope, \\
with defilements ended, the spiritual journey completed: \\
that is where a brahmin seeking merit \\
should bestow a timely offering as sacrifice. 

Those\marginnote{13.1} not fallen prey to cravings, \\
who, having crossed the flood, live unselfishly: \\
that is where a brahmin seeking merit \\
should bestow a timely offering as sacrifice. 

Those\marginnote{14.1} with no craving at all in the world \\
to any form of existence in this life or the next: \\
that is where a brahmin seeking merit \\
should bestow a timely offering as sacrifice. 

Those\marginnote{15.1} who have left sensuality behind, wandering homeless, \\
self-controlled, straight as a shuttle: \\
that is where a brahmin seeking merit \\
should bestow a timely offering as sacrifice. 

Those\marginnote{16.1} freed of greed, with senses stilled, \\
like the moon released from the eclipse: \\
that is where a brahmin seeking merit \\
should bestow a timely offering as sacrifice. 

Those\marginnote{17.1} peaceful ones free of greed and anger, \\
for whom there are no destinies, being rid of them in this life: \\
that is where a brahmin seeking merit \\
should bestow a timely offering as sacrifice. 

They’ve\marginnote{18.1} given up rebirth and death completely, \\
and have gone beyond all doubt: \\
that is where a brahmin seeking merit \\
should bestow a timely offering as sacrifice. 

Those\marginnote{19.1} who live as their own island, \\
everywhere free, owning nothing: \\
that is where a brahmin seeking merit \\
should bestow a timely offering as sacrifice. 

Those\marginnote{20.1} here who know this to be true: \\
‘This is my last life, there are no future lives’: \\
that is where a brahmin seeking merit \\
should bestow a timely offering as sacrifice. 

A\marginnote{21.1} knowledge master, loving absorption, mindful, \\
who has reached awakening and is a refuge for many: \\
that is where a brahmin seeking merit \\
should bestow a timely offering as sacrifice.” 

“Clearly\marginnote{22.1} my questions were not in vain!” \\
\scspeaker{said \textsanskrit{Māgha}, }\\
“The Buddha has explained to me who is worthy of donations. \\
You are the one here who knows this to be true, \\
for truly you understand this matter. 

Suppose\marginnote{23.1} a lay donor who is committed to charity \\
makes a sacrifice seeking merit, looking for merit, \\
giving food and drink to others here: \\
explain to me how to accomplish the sacrifice.” 

“Sacrifice,\marginnote{24.1} and while doing so,”\footnote{The “speaker’s mark” is in MS treated as part of the line, which is unusual if not unique. According to Norman’s note here however, we should consider this line a seven-syllable sloka (perhaps restoring the variant \textit{ca} to make it eight). I have adjusted the Pali punctuation accordingly. } \\
\scspeaker{replied the Buddha, }\\
“be clear and confident in every way. \\
Sacrifice is the ground standing upon which \\
the sacrificer sheds their flaws. 

One\marginnote{25.1} free of greed, rid of anger, \\
developing a heart of limitless love, \\
spreads that limitlessness in every direction, \\
ever diligent day and night.” 

“Who\marginnote{26.1} is purified, freed, awake?\footnote{\textit{Bajjhati} (“bound”) seems odd to me in this line. I think it’s more likely these are three semi-synonyms. As it stands, the Buddha does not actually answer the question, since he only discusses the positive side. PTS notes a reading \textit{bujjhati} in the commentary, which I follow. However the commentary reads \textit{bajjhati} in the VRI edition, and the comment itself supports this, since it treats it as a bad thing. If it is a mistake, then, it is an old one. } \\
How can one go to the \textsanskrit{Brahmā} realm oneself? \\
I do not know, so please tell me when asked, \\
for the Buddha is the \textsanskrit{Brahmā} I see in person today! \\
To us you are truly the equal of \textsanskrit{Brahmā}. \\
Splendid One, how is one reborn in the \textsanskrit{Brahmā} realm?” 

“One\marginnote{27.1} who accomplishes the sacrifice with three modes,” \\
\scspeaker{replied the Buddha, }\\
“such a one would succeed due to those who are worthy of donations. \\
Sacrificing like this, one rightly committed to charity \\
is reborn in the \textsanskrit{Brahmā} realm, I say.” 

%
\end{verse}

When\marginnote{28.1} he had spoken, the student \textsanskrit{Māgha} said to the Buddha, “Excellent, Master Gotama! Excellent! … From this day forth, may Master Gotama remember me as a lay follower who has gone for refuge for life.” 

%
\section*{{\suttatitleacronym Snp 3.6}{\suttatitletranslation With Sabhiya }{\suttatitleroot Sabhiyasutta}}
\addcontentsline{toc}{section}{\tocacronym{Snp 3.6} \toctranslation{With Sabhiya } \tocroot{Sabhiyasutta}}
\markboth{With Sabhiya }{Sabhiyasutta}
\extramarks{Snp 3.6}{Snp 3.6}

\scevam{So\marginnote{1.1} I have heard. }At one time the Buddha was staying near \textsanskrit{Rājagaha}, in the Bamboo Grove, the squirrels’ feeding ground. Now at that time the wanderer Sabhiya had been presented with several questions by a deity who was a former relative, saying: “Sabhiya, you should live the spiritual life with whatever ascetic or brahmin answers these questions.” 

Then\marginnote{2.1} Sabhiya, after learning those questions in the presence of that deity, approached those ascetics and brahmins who led an order and a community, and taught a community, who were well-known and famous religious founders, regarded as holy by many people. That is, \textsanskrit{Pūraṇa} Kassapa, Makkhali \textsanskrit{Gosāla}, \textsanskrit{Nigaṇṭha} \textsanskrit{Nāṭaputta}, \textsanskrit{Sañjaya} \textsanskrit{Belaṭṭhiputta}, Pakudha \textsanskrit{Kaccāyana}, and Ajita Kesakambala. And he asked them those questions, but they were stumped by them. Displaying annoyance, hate, and bitterness, they questioned Sabiya in return. 

Then\marginnote{3.1} Sabhiya thought, “Those famous ascetics and brahmins were stumped by my questions. Displaying annoyance, hate, and bitterness, they questioned me in return on that matter. Why don’t I return to a lesser life so I can enjoy sensual pleasures?” 

Then\marginnote{4.1} Sabhiya thought, “This ascetic Gotama also leads an order and a community, and teaches a community. He’s a well-known and famous religious founder, regarded as holy by many people. Why don’t I ask him this question?” 

Then\marginnote{5.1} he thought, “Even those ascetics and brahmins who are elderly and senior, who are advanced in years and have reached the final stage of life; who are senior, long standing, long gone forth; who lead an order and a community, and teach a community; who are well-known and famous religious founders, regarded as holy by many people—that is \textsanskrit{Pūraṇa} Kassapa and the rest—were stumped by my questions. They displayed annoyance, hate, and bitterness, and even questioned me in return. How can the ascetic Gotama possibly answer my questions, since he is so young in age and newly gone forth?” 

Then\marginnote{6.1} he thought, “An ascetic should not be looked down upon or disparaged because they are young. Though young, the ascetic Gotama has great psychic power and might. Why don’t I ask him this question?” 

Then\marginnote{7.1} Sabhiya set out for \textsanskrit{Rājagaha}. Traveling stage by stage, he came to \textsanskrit{Rājagaha}, the Bamboo Grove, the squirrels’ feeding ground. He went up to the Buddha and exchanged greetings with him. When the greetings and polite conversation were over, he sat down to one side, and addressed the Buddha in verse: 

\begin{verse}%
“I’ve\marginnote{8.1} come full of  doubts and uncertainties,” \\
\scspeaker{said Sabhiya, }\\
“wishing to ask some questions. \\
Please solve them for me. \\
Answer my questions in turn, in accordance with the truth.” 

“You\marginnote{9.1} have come from afar, Sabhiya,” \\
\scspeaker{said the Buddha, }\\
“wishing to ask some questions. \\
I shall solve them for you, \\
answering your questions in turn, in accordance with the truth. 

Ask\marginnote{10.1} me your question, Sabhiya, \\
whatever you want. \\
I’ll solve each and every \\
question you have.” 

%
\end{verse}

Then\marginnote{11.1} Sabhiya thought, “It’s incredible, it’s amazing! Where those other ascetics and brahmins didn’t even give me a chance, the Buddha has invited me.” Uplifted and elated, full of rapture and happiness, he asked this question. 

\begin{verse}%
“What\marginnote{12.1} must one attain to be called a mendicant?” \\
\scspeaker{said Sabhiya, }\\
“How is one ‘sweet’, how said to be ‘tamed’? \\
How is one declared to be ‘awakened’? \\
May the Buddha please answer my question.” 

“When\marginnote{13.1} by the path they have walked themselves,” \\
\scspeaker{said the Buddha to Sabhiya, }\\
“they reach quenching, with doubt overcome; \\
giving up desire to continue existence or to end it, \\
their journey complete, their rebirths ended: that is a mendicant. 

Equanimous\marginnote{14.1} towards everything, mindful, \\
they don’t harm anyone in the world. \\
An ascetic who has crossed over, unclouded, \\
not full of themselves, is sweet-natured. 

Their\marginnote{15.1} faculties have been developed \\
inside and out in the whole world. \\
Having pierced through this world and the next, \\
tamed, they bide their time. 

They\marginnote{16.1} have examined the aeons in their entirety,\footnote{Comm allows \textit{kappa} in the sense of both “mental activity” and “aeon” here. For the sake of consistency with later uses of the term, it might seem prudent to use “mental activity” here, and most translators have done so. Yet the context in this verse suggests it is about the aeons of transmigration. } \\
and both sides of transmigration—passing away and rebirth. \\
Rid of dust, unblemished, purified: \\
the one they call ‘awakened’ has attained the end of rebirth.” 

%
\end{verse}

And\marginnote{17.1} then, having approved and agreed with what the Buddha said, uplifted and elated, full of rapture and happiness, Sabhiya asked another question: 

\begin{verse}%
“What\marginnote{18.1} must one attain to be called ‘brahmin’?” \\
\scspeaker{said Sabhiya. }\\
“Why is one an ‘ascetic’, and how a ‘bathed initiate’? \\
How is one declared to be a ‘giant’? \\
May the Buddha please answer my question.” 

“Having\marginnote{19.1} banished all bad things,” \\
\scspeaker{said the Buddha to Sabhiya, }\\
“immaculate, well-composed, steadfast, \\
consummate, they’ve left transmigration behind: \\
such an unattached one is called ‘brahmin’. 

A\marginnote{20.1} peaceful one who has given up good and evil, \\
stainless, understanding this world and the next, \\
gone beyond rebirth and death: \\
such an one is rightly called ‘ascetic’. 

Having\marginnote{21.1} washed off all bad things \\
inside and out in the whole world, \\
among gods and humans bound to creations, \\
the one they call ‘washed’ does not return to creation. 

They\marginnote{22.1} do nothing monstrous at all in the world, \\
discarding all fetters and bonds, \\
everywhere not stuck, freed: \\
such an one is rightly called ‘giant’.” 

%
\end{verse}

And\marginnote{23.1} then Sabhiya asked another question: 

\begin{verse}%
“Who\marginnote{24.1} is a ‘field-victor’ according to the Buddhas?”\footnote{I think Norman’s reconstruction of this to root \textit{\textsanskrit{ñā}} (followed by Bodhi) is too speculative. The word makes fine sense as is, and is more metaphorically connected with the sense of \textit{khetta}. } \\
\scspeaker{said Sabhiya, }\\
“Why is one ‘skillful’, and how ‘a wise scholar’? \\
How is one declared to be a ‘sage’? \\
May the Buddha please answer my question.” 

“They\marginnote{25.1} are victorious over the fields of deeds in their entirety,”\footnote{Reading \textit{vijeyya}; both \textit{viceyya} and \textit{vijeyya} are accepted in the commentary. } \\
\scspeaker{said the Buddha to Sabhiya, }\\
“the fields of gods, humans, and Brahmas; \\
released from the root bondage to all fields: \\
such a one is rightly called ‘field-victor’. 

They\marginnote{26.1} have examined the stockpiles of deeds in their entirety, \\
the stockpiles of gods, humans, and Brahmas; \\
released from the root bondage to all stockpiles: \\
such a one is rightly called ‘skillful’. 

They\marginnote{27.1} have examined whiteness\footnote{The commentary has led translators astray here. \textit{\textsanskrit{Paṇḍarāni}} means not “senses” per commentary (followed without remark by Norman); nor “translucencies” per Bodhi; nor “white flowers” per Thanissaro. In the EBTs, \textit{\textsanskrit{paṇḍara}} simply means “white, pure” and is a synonym for \textit{suddhi} in the next line. A true \textit{\textsanskrit{paṇḍita}} is pure both inside and out. } \\
both inside and out; understanding purity, \\
they have left dark and bright behind: \\
such an one is rightly called ‘a wise scholar’.”\footnote{Using ‘wise scholar’ rather than my normal ‘astute’ to capture, albeit lamely, a pun with \textit{\textsanskrit{paṇḍara}}. } 

Understanding\marginnote{28.1} the nature of the bad and the good \\
inside and out in the whole world; \\
one worthy of honor by gods and humans, \\
who has escaped from the net and the snare: that is a sage.” 

%
\end{verse}

And\marginnote{29.1} then Sabhiya asked another question: 

\begin{verse}%
“What\marginnote{30.1} must one attain to be called ‘knowledge master’?” \\
\scspeaker{said Sabhiya, }\\
“Why is one ‘studied’, and how is one ‘heroic’? \\
How to gain the name ‘thoroughbred’? \\
May the Buddha please answer my question.” 

“They\marginnote{31.1} have examined knowledges in their entirety,” \\
\scspeaker{said the Buddha to Sabhiya, }\\
“those that are current among ascetics and brahmins; \\
rid of greed for all feelings, \\
having left all knowledges behind: that is a knowledge master. 

Having\marginnote{32.1} studied proliferation and name \& form \\
inside and out—the root of disease; \\
released from the root bondage to all disease: \\
such an one is rightly called ‘studied’. 

Refraining\marginnote{33.1} from all evil here, \\
heroic, he escapes from the suffering of hell;\footnote{Commentarial “abode of energy” for \textit{\textsanskrit{viriyavāso}} doesn’t sound right to me, though followed by Bodhi and Norman. I read \textit{\textsanskrit{viriyavā} so}. } \\
he is heroic and energetic: \\
such an one is rightly called ‘hero’.\footnote{Reading \textit{\textsanskrit{vīro}}. } 

Whoever’s\marginnote{34.1} bonds are cut, \\
the root of clinging inside and out; \\
released from the root bondage to all clinging: \\
such an one is rightly called ‘thoroughbred’.” 

%
\end{verse}

And\marginnote{35.1} then Sabhiya asked another question: 

\begin{verse}%
“What\marginnote{36.1} must one attain to be called ‘scholar’?” \\
\scspeaker{said Sabhiya, }\\
“Why is one ‘noble’, and how is one ‘well conducted’? \\
How to gain the name ‘wanderer’? \\
May the Buddha please answer my question.” 

“One\marginnote{37.1} who has learned every teaching,”\footnote{Earlier translations misconstrued this verse. The gerunds \textit{\textsanskrit{sutvā}} and \textit{\textsanskrit{abhiññāya}} apply to their respective following accusatives \textit{\textsanskrit{sabbadhammaṁ}} and \textit{\textsanskrit{sāvajjānavajjaṁ}}, not both to \textit{\textsanskrit{sabbadhammaṁ}} per Norman and Bodhi. \textit{Dhamma} means “teachings” here, not “phenomena” per Bodhi. The sense is that to be a “scholar” it is not enough to have “learned” the teachings, one must know for oneself. The commentary well explains this sense, but it doesn’t come through in the translations. Thanissaro attempts this sense, but only by adding an unsupported reading of \textit{\textsanskrit{abhiññāya}} as dative. } \\
\scspeaker{said the Buddha to Sabhiya, }\\
“and has known for themselves whatever is blameworthy and blameless in the world; \\
a champion, decided, liberated, \\
untroubled everywhere: they call them ‘scholar’. 

Having\marginnote{38.1} cut off defilements and attachments, \\
being wise, they enter no womb. \\
They’ve expelled the bog of the three perceptions,\footnote{The three perceptions are sensuality, malice, and cruelty. Since \textit{\textsanskrit{paṅka}} is regularly said to be a term for \textit{\textsanskrit{kāma}}, it seems we should read “the bog of the three perceptions” rather than “the three perceptions and the bog” per Bodhi and Thanissaro. } \\
the one they call ‘noble’ does not return to creation. 

One\marginnote{39.1} here who is accomplished and skillful in all forms of good conduct;\footnote{It seems to me that the syntax reads more cleanly if the nominatives in this line and the next are read together; and the clauses following \textit{sabb-} are likewise read in parallel. } \\
always understanding the teaching, \\
everywhere not stuck, freed in mind, \\
who has no repulsion: they are ‘well-conducted’. 

Avoiding\marginnote{40.1} any deed that results in suffering—\\
above, below, all round, between: \\
deceit and conceit, as well as greed and anger,\footnote{Following Norman and Bodhi in reading \textit{\textsanskrit{parivajjayitā}} here. } \\
they live full of wisdom. \\
They have made a limit on name \& form;\footnote{I think Bodhi, Norman, and Thanissaro have missed two related points. First, \textit{pariyanta} means “limit, boundary” not “end”; and second, \textit{\textsanskrit{nāmarūpa}} is not ended in this life but rather “limited”. An arahant still has name \& form but only until \textit{\textsanskrit{parinibbāna}}; it is “limited”. This further highlights the rather awkward inclusion of \textit{\textsanskrit{nāmarūpa}} in the list of defilements, which are ended. I think we should, rather, include the defilements with the “deeds causing suffering”. This has the further advantage of dividing this long verse into 4 + 2 lines rather than 3 + 3. } \\
the one they call a ‘wanderer’ has reached their destination.”\footnote{I think \textit{pattipatta} is a play on words here. } 

%
\end{verse}

And\marginnote{41.1} then, having approved and agreed with what the Buddha said, uplifted and elated, full of rapture and happiness, Sabhiya got up from his seat, arranged his robe over one shoulder, raised his joined palms toward the Buddha, and extolled the Buddha in his presence with fitting verses: 

\begin{verse}%
“O\marginnote{42.1} one of vast wisdom, there are three \& sixty opinions \\
based on the doctrines of ascetics: \\
they are expressions of perception, based on perception. \\
Having dispelled them all, you passed over the dark flood. 

You\marginnote{43.1} have gone to the end, gone beyond suffering, \\
you are perfected, a fully awakened Buddha; I think you have ended defilements. \\
Splendid, intelligent, abounding in wisdom, \\
ender of suffering—you brought me across! 

When\marginnote{44.1} you understood my uncertainty, \\
you brought me beyond doubt—homage to you! \\
A sage, accomplished in the ways of sagacity, \\
you are gentle, not hardhearted, O Kinsman of the Sun.\footnote{Oddly, at AN 5.38:5.4 we find \textit{sorata} with \textit{sakhila} in a similar sense to \textit{akhila} here. } 

Any\marginnote{45.1} doubts that I once had, \\
you have answered for me, O Seer. \\
Clearly you are a sage, an Awakened One, \\
there are no hindrances in you. 

All\marginnote{46.1} your distress \\
is blown away and mown down. \\
Cooled, tamed, steadfast: \\
truth is your strength. 

O\marginnote{47.1} giant among giants, O great hero, \\
when you are speaking \\
all the gods rejoice, \\
including both \textsanskrit{Nārada} and Pabbata.\footnote{The commentary says these were two groups of wise devas. But it seems it refers to the legendary sages \textsanskrit{Nārada} and his nephew Parvata, whose friendship, estrangement, and reconciliation is told in the \textsanskrit{Viṣṇu} \textsanskrit{Purāṇa}. \textsanskrit{Nārada} is known for his learning and wisdom, which supports the commentary to an extent. Why these two are invoked here is a mystery; presumably they were important to Sabhiya. They occur in a similar context, but with many more gods, in Ja 547. } 

Homage\marginnote{48.1} to you, O thoroughbred! \\
Homage to you, supreme among men! \\
In the world with its gods, \\
you have no counterpart. 

You\marginnote{49.1} are the Buddha, you are the Teacher, \\
you are the sage who has overcome \textsanskrit{Māra}; \\
you have cut off the underlying tendencies, \\
you’ve crossed over, and you bring humanity across. 

You\marginnote{50.1} have transcended attachments, \\
your defilements are shattered; \\
you are a lion, free of grasping, \\
with fear and dread given up. 

Like\marginnote{51.1} a graceful lotus \\
to which water does not stick, \\
so both good and evil \\
do not stick to you. \\
Stretch out your feet, great hero: \\
Sabhiya bows to the Teacher.” 

%
\end{verse}

Then\marginnote{52.1} the wanderer Sabhiya bowed with his head at the Buddha’s feet and said, “Excellent, sir! Excellent! … I go for refuge to the Buddha, to the teaching, and to the mendicant \textsanskrit{Saṅgha}. Sir, may I receive the going forth, the ordination in the Buddha’s presence?” 

“Sabhiya,\marginnote{53.1} if someone formerly ordained in another sect wishes to take the going forth, the ordination in this teaching and training, they must spend four months on probation. When four months have passed, if the mendicants are satisfied, they’ll give the going forth, the ordination into monkhood. However, I have recognized individual differences in this matter.” 

“Sir,\marginnote{54.1} if four months probation are required in such a case, I’ll spend four years on probation. When four years have passed, if the mendicants are satisfied, let them give me the going forth, the ordination into monkhood.” And the wanderer Sabhiya received the going forth, the ordination in the Buddha’s presence. And Venerable Sabhiya became one of the perfected. 

%
\section*{{\suttatitleacronym Snp 3.7}{\suttatitletranslation With Sela }{\suttatitleroot Selasutta}}
\addcontentsline{toc}{section}{\tocacronym{Snp 3.7} \toctranslation{With Sela } \tocroot{Selasutta}}
\markboth{With Sela }{Selasutta}
\extramarks{Snp 3.7}{Snp 3.7}

\scevam{So\marginnote{1.1} I have heard. }At one time the Buddha was wandering in the land of the Northern \textsanskrit{Āpaṇas} together with a large \textsanskrit{Saṅgha} of 1,250 mendicants when he arrived at a town of the Northern \textsanskrit{Āpaṇas} named \textsanskrit{Āpaṇa}. The matted-hair ascetic \textsanskrit{Keṇiya} heard: “It seems the ascetic Gotama—a Sakyan, gone forth from a Sakyan family—has arrived at \textsanskrit{Āpaṇa}, together with a large \textsanskrit{Saṅgha} of 1,250 mendicants. He has this good reputation: ‘That Blessed One is perfected, a fully awakened Buddha, accomplished in knowledge and conduct, holy, knower of the world, supreme guide for those who wish to train, teacher of gods and humans, awakened, blessed.’ He has realized with his own insight this world—with its gods, \textsanskrit{Māras} and \textsanskrit{Brahmās}, this population with its ascetics and brahmins, gods and humans—and he makes it known to others. He teaches Dhamma that’s good in the beginning, good in the middle, and good in the end, meaningful and well-phrased. And he reveals a spiritual practice that’s entirely full and pure. It’s good to see such perfected ones.” 

So\marginnote{2.1} \textsanskrit{Keṇiya} approached the Buddha and exchanged greetings with him. When the greetings and polite conversation were over, he sat down to one side. The Buddha educated, encouraged, fired up, and inspired him with a Dhamma talk. Then he said to the Buddha, “Would Master Gotama together with the mendicant \textsanskrit{Saṅgha} please accept tomorrow’s meal from me?” When he said this, the Buddha said to him, “The \textsanskrit{Saṅgha} is large, \textsanskrit{Keṇiya}; there are 1,250 mendicants. And you are devoted to the brahmins.” 

For\marginnote{3.1} a second time, \textsanskrit{Keṇiya} asked the Buddha to accept a meal offering. “Never mind that the \textsanskrit{Saṅgha} is large, with 1,250 mendicants, and that I am devoted to the brahmins. Would Master Gotama together with the mendicant \textsanskrit{Saṅgha} please accept tomorrow’s meal from me?” And for a second time, the Buddha gave the same reply. 

For\marginnote{4.1} a third time, \textsanskrit{Keṇiya} asked the Buddha to accept a meal offering. “Never mind that the \textsanskrit{Saṅgha} is large, with 1,250 mendicants, and that I am devoted to the brahmins. Would Master Gotama together with the mendicant \textsanskrit{Saṅgha} please accept tomorrow’s meal from me?” The Buddha consented in silence. Then, knowing that the Buddha had consented, \textsanskrit{Keṇiya} got up from his seat and went to his own hermitage. There he addressed his friends and colleagues, relatives and family members, “My friends and colleagues, relatives and family members: please listen! The ascetic Gotama together with the mendicant \textsanskrit{Saṅgha} has been invited by me for tomorrow’s meal. Please help me with the preparations.” “Yes, sir,” they replied. Some dug ovens, some chopped wood, some washed dishes, some set out a water jar, and some spread out seats. Meanwhile, \textsanskrit{Keṇiya} set up the pavilion himself. 

Now\marginnote{5.1} at that time the brahmin Sela was residing in \textsanskrit{Āpaṇa}. He had mastered the three Vedas, together with their vocabularies, ritual, phonology and etymology, and the testament as fifth. He knew philology and grammar, and was well versed in cosmology and the marks of a great man. And he was teaching three hundred students to recite the hymns. 

And\marginnote{6.1} at that time \textsanskrit{Keṇiya} was devoted to Sela. Then Sela, while going for a walk escorted by the three hundred students, approached \textsanskrit{Keṇiya}’s hermitage. He saw the preparations going on, and said to \textsanskrit{Keṇiya}, “\textsanskrit{Keṇiya}, is your son or daughter being married? Or are you setting up a big sacrifice? Or has King Seniya \textsanskrit{Bimbisāra} of Magadha been invited for tomorrow’s meal?” 

“There\marginnote{7.1} is no marriage, Sela, and the king is not coming. Rather, I am setting up a big sacrifice. The ascetic Gotama has arrived at \textsanskrit{Āpaṇa}, together with a large \textsanskrit{Saṅgha} of 1,250 mendicants. He has this good reputation: ‘That Blessed One is perfected, a fully awakened Buddha, accomplished in knowledge and conduct, holy, knower of the world, supreme guide for those who wish to train, teacher of gods and humans, awakened, blessed.’ He has been invited by me for tomorrow’s meal together with the mendicant \textsanskrit{Saṅgha}.” “Mister \textsanskrit{Keṇiya}, did you say ‘the awakened one’?” “I said ‘the awakened one’.” “Mister \textsanskrit{Keṇiya}, did you say ‘the awakened one’?” “I said ‘the awakened one’.” 

Then\marginnote{8.1} Sela thought, “It’s hard to even find the word ‘awakened one’ in the world. The thirty-two marks of a great man have been handed down in our hymns. A great man who possesses these has only two possible destinies, no other. If he stays at home he becomes a king, a wheel-turning monarch, a just and principled king. His dominion extends to all four sides, he achieves stability in the country, and he possesses the seven treasures. He has the following seven treasures: the wheel, the elephant, the horse, the jewel, the woman, the treasurer, and the counselor as the seventh treasure. He has over a thousand sons who are valiant and heroic, crushing the armies of his enemies. After conquering this land girt by sea, he reigns by principle, without rod or sword. But if he goes forth from the lay life to homelessness, he becomes a perfected one, a fully awakened Buddha, who draws back the veil from the world. “But \textsanskrit{Keṇiya}, where is the Blessed One at present, the perfected one, the fully awakened Buddha?” 

When\marginnote{9.1} he said this, \textsanskrit{Keṇiya} pointed with his right arm and said, “There, Mister Sela, at that line of blue forest.” Then Sela, together with his students, approached the Buddha. He said to his students, “Come quietly, gentlemen, tread gently. For the Buddhas are intimidating, like a lion living alone. When I’m consulting with the ascetic Gotama, don’t interrupt. Wait until I’ve finished speaking.” 

Then\marginnote{10.1} Sela went up to the Buddha, and exchanged greetings with him. When the greetings and polite conversation were over, he sat down to one side. and scrutinized the Buddha’s body for the thirty-two marks of a great man. He saw all of them except for two, which he had doubts about: whether the private parts were covered in a foreskin, and the largeness of the tongue. 

Then\marginnote{11.1} it occurred to the Buddha, “Sela sees all the marks except for two, which he has doubts about: whether the private parts are covered in a foreskin, and the largeness of the tongue.” The Buddha used his psychic power to will that Sela would see his private parts covered in a foreskin. And he stuck out his tongue and stroked back and forth on his ear holes and nostrils, and covered his entire forehead with his tongue. 

Then\marginnote{12.1} Sela thought, “The ascetic Gotama possesses the thirty-two marks completely, lacking none. But I don’t know whether or not he is an awakened one. I have heard that brahmins of the past who were elderly and senior, the teachers of teachers, said, ‘Those who are perfected ones, fully awakened Buddhas reveal themselves when praised.’ Why don’t I extoll him in his presence with fitting verses?” Then Sela extolled the Buddha in his presence with fitting verses: 

\begin{verse}%
“O\marginnote{13.1} Blessed One, your body’s perfect, \\
you’re radiant, handsome, lovely to behold; \\
golden colored, \\
with teeth so white; you’re strong. 

The\marginnote{14.1} characteristics \\
of a handsome man, \\
the marks of a great man, \\
are all found on your body. 

Your\marginnote{15.1} eyes are clear, your face is fair, \\
you’re formidable, upright, majestic. \\
In the midst of the \textsanskrit{Saṅgha} of ascetics, \\
you shine like the sun. 

You’re\marginnote{16.1} a mendicant fine to see, \\
with skin of golden sheen. \\
But with such excellent appearance, \\
what do you want with the ascetic life? 

You’re\marginnote{17.1} fit to be a king, \\
a wheel-turning monarch, chief of charioteers, \\
victorious in the four quarters, \\
lord of all India. 

Aristocrats,\marginnote{18.1} nobles, and kings \\
ought follow your rule. \\
Gotama, you should reign \\
as king of kings, lord of men!” 

“I\marginnote{19.1} am a king, Sela”, \\
\scspeaker{said the Buddha, }\\
“the supreme king of the teaching. \\
By the teaching I roll forth the wheel \\
which cannot be rolled back.” 

“You\marginnote{20.1} claim to be awakened,” \\
\scspeaker{said Sela the brahmin, }\\
“the supreme king of the teaching. \\
‘I roll forth the teaching’: \\
so you say, Gotama. 

Then\marginnote{21.1} who is your general, \\
the disciple who follows the Teacher’s way? \\
Who keeps rolling the wheel \\
of the teaching you rolled forth?” 

“By\marginnote{22.1} me the wheel was rolled forth,” \\
\scspeaker{said the Buddha, }\\
“the supreme wheel of the teaching. \\
\textsanskrit{Sāriputta}, taking after the Realized One, \\
keeps it rolling on. 

I\marginnote{23.1} have known what should be known, \\
and developed what should be developed, \\
and given up what should be given up: \\
and so, brahmin, I am a Buddha. 

Dispel\marginnote{24.1} your doubt in me—\\
make up your mind, brahmin! \\
The sight of a Buddha \\
is hard to find again. 

I\marginnote{25.1} am a Buddha, brahmin, \\
the supreme surgeon, \\
one of those whose appearance in the world \\
is hard to find again. 

Holy,\marginnote{26.1} unequaled, \\
crusher of \textsanskrit{Māra}’s army; \\
having subdued all my opponents, \\
I rejoice, fearing nothing from any quarter.” 

“Pay\marginnote{27.1} heed, sirs, to what \\
is spoken by the seer. \\
The surgeon, the great hero, \\
roars like a lion in the jungle. 

Holy,\marginnote{28.1} unequaled, \\
crusher of \textsanskrit{Māra}’s army; \\
who would not be inspired by him, \\
even one whose nature is dark? 

Those\marginnote{29.1} who wish may follow me; \\
those who don’t may go. \\
Right here, I’ll go forth in his presence, \\
the one of such splendid wisdom.” 

“Sir,\marginnote{30.1} if you like \\
in the teaching of the Buddha, \\
we’ll also go forth in his presence, \\
the one of such splendid wisdom.” 

“These\marginnote{31.1} three hundred brahmins \\
with joined palms held up, ask: \\
‘May we lead the spiritual life \\
in your presence, Blessed One?’” 

“The\marginnote{32.1} spiritual life is well explained,” \\
\scspeaker{said the Buddha, }\\
“visible in this very life, immediately effective. \\
Here the going forth isn’t in vain \\
for one who trains with diligence.” 

%
\end{verse}

And\marginnote{33.1} the brahmin Sela together with his assembly received the going forth, the ordination in the Buddha’s presence. And when the night had passed \textsanskrit{Keṇiya} had a variety of delicious foods prepared in his own home. Then he had the Buddha informed of the time, saying, “Itʼs time, Master Gotama, the meal is ready.” Then the Buddha robed up in the morning and, taking his bowl and robe, went to \textsanskrit{Keṇiya}’s hermitage, where he sat on the seat spread out, together with the \textsanskrit{Saṅgha} of mendicants. 

Then\marginnote{34.1} \textsanskrit{Keṇiya} served and satisfied the mendicant \textsanskrit{Saṅgha} headed by the Buddha with his own hands with a variety of delicious foods. When the Buddha had eaten and washed his hand and bowl, \textsanskrit{Keṇiya} took a low seat and sat to one side. The Buddha expressed his appreciation with these verses: 

\begin{verse}%
“The\marginnote{35.1} foremost of sacrifices is offering to the sacred flame; \\
the \textsanskrit{Gāyatrī} Mantra is the foremost of poetic meters; \\
of humans, the king is the foremost; \\
the ocean’s the foremost of rivers; 

the\marginnote{36.1} foremost of stars is the moon; \\
the sun is the foremost of lights; \\
for those who sacrifice seeking merit, \\
the \textsanskrit{Saṅgha} is the foremost.” 

%
\end{verse}

When\marginnote{37.1} the Buddha had expressed his appreciation to \textsanskrit{Keṇiya} the matted-hair ascetic with these verses, he got up from his seat and left. Then Venerable Sela and his assembly, living alone, withdrawn, diligent, keen, and resolute, soon realized the supreme end of the spiritual path in this very life. They lived having achieved with their own insight the goal for which gentlemen rightly go forth from the lay life to homelessness. And Venerable Sela together with his assembly became perfected. 

Then\marginnote{38.1} Sela with his assembly went to see the Buddha. He arranged his robe over one shoulder, raised his joined palms toward the Buddha, and said: 

\begin{verse}%
“This\marginnote{39.1} is the eighth day since \\
we went for refuge, O seer. \\
In these seven days, Blessed One, \\
we’ve become tamed in your teaching. 

You\marginnote{40.1} are the Buddha, you are the Teacher, \\
you are the sage who has overcome \textsanskrit{Māra}; \\
you have cut off the underlying tendencies, \\
you’ve crossed over, and you bring humanity across. 

You\marginnote{41.1} have transcended attachments, \\
your defilements are shattered; \\
you are a lion, free of grasping, \\
with fear and dread given up. 

These\marginnote{42.1} three hundred mendicants \\
stand with joined palms raised. \\
Stretch out your feet, great hero: \\
let these giants bow to the Teacher.” 

%
\end{verse}

%
\section*{{\suttatitleacronym Snp 3.8}{\suttatitletranslation The Dart }{\suttatitleroot Sallasutta}}
\addcontentsline{toc}{section}{\tocacronym{Snp 3.8} \toctranslation{The Dart } \tocroot{Sallasutta}}
\markboth{The Dart }{Sallasutta}
\extramarks{Snp 3.8}{Snp 3.8}

\begin{verse}%
Unforeseen\marginnote{1.1} and unknown\footnote{I can’t find any translation or note that captures this line properly. A \textit{nimitta} is a sign or portent, and is commonly used in the context of omens, prophecy, or fortune-telling. Here it means that there is no portent or omen by which one can predict the extent of life. } \\
is the extent of this mortal life—\\
hard and short \\
and bound to pain. 

There\marginnote{2.1} is no way that \\
those born will not die. \\
On reaching old age death follows: \\
such is the nature of living creatures. 

As\marginnote{3.1} ripe fruit \\
are always in danger of falling,\footnote{Accepting \textit{\textsanskrit{niccaṁ}} per Norman and Bodhi. } \\
so mortals once born \\
are always in danger of death. 

As\marginnote{4.1} clay pots \\
made by a potter \\
all end up being broken, \\
so is the life of mortals. 

Young\marginnote{5.1} and old, \\
foolish and wise—\\
all go under the sway of death; \\
all are destined to die. 

When\marginnote{6.1} those overcome by death \\
leave this world for the next, \\
a father cannot protect his son, \\
nor relatives their kin. 

See\marginnote{7.1} how, while relatives look on, \\
wailing profusely, \\
mortals are led away one by one, \\
like a cow to the slaughter. 

And\marginnote{8.1} so the world is stricken \\
by old age and by death. \\
That is why the wise do not grieve, \\
for they understand the way of the world. 

For\marginnote{9.1} one whose path you do not know—\\
not whence they came nor where they went—\\
you lament in vain, \\
seeing neither end. 

If\marginnote{10.1} a bewildered person, \\
lamenting and self-harming, \\
could extract any good from that, \\
then those who see clearly would do the same. 

For\marginnote{11.1} not by weeping and wailing \\
will you find peace of heart. \\
It just gives rise to more suffering, \\
and distresses your body. 

Growing\marginnote{12.1} thin and pale, \\
you hurt yourself. \\
It does nothing to help the dead: \\
your lamentation is in vain. 

Unless\marginnote{13.1} a person gives up grief, \\
they fall into suffering all the more. \\
Bewailing those whose time has come, \\
you fall under the sway of grief. 

See,\marginnote{14.1} too, other folk departing \\
to fare after their deeds; \\
fallen under the sway of death, \\
beings flounder while still here. 

For\marginnote{15.1} whatever you imagine it is, \\
it turns out to be something else. \\
Such is separation: \\
see the way of the world! 

Even\marginnote{16.1} if a human lives \\
a hundred years or more, \\
they are parted from their family circle, \\
they leave this life behind. 

Therefore,\marginnote{17.1} having learned from the Perfected One, \\
dispel lamentation. \\
Seeing the dead and departed, think: \\
“I cannot escape this.”\footnote{Commentary says this means, “I cannot bring back the dead” followed by Norman, Bodhi, and Thanissaro. But I think that reading is suspect. I think the text elides a second negative and should read: \textit{na eso \textsanskrit{alabbhā}}, literally “This is not not-to-be-gotten by me”, i.e. “I cannot escape this”. } 

As\marginnote{18.1} one would extinguish \\
a blazing refuge with water, \\
so too a sage—a wise, \\
astute, and skilled person—\\
would swiftly blow away grief that comes up, \\
like the wind a tuft of cotton. 

One\marginnote{19.1} who seeks their own happiness \\
would pluck out the dart from themselves—\\
the wailing and moaning, \\
and sadness inside. 

With\marginnote{20.1} dart plucked out, unattached, \\
having found peace of mind, \\
overcoming all sorrow, \\
one is sorrowless and extinguished. 

%
\end{verse}

%
\section*{{\suttatitleacronym Snp 3.9}{\suttatitletranslation With Vāseṭṭha }{\suttatitleroot Vāseṭṭhasutta}}
\addcontentsline{toc}{section}{\tocacronym{Snp 3.9} \toctranslation{With Vāseṭṭha } \tocroot{Vāseṭṭhasutta}}
\markboth{With Vāseṭṭha }{Vāseṭṭhasutta}
\extramarks{Snp 3.9}{Snp 3.9}

\scevam{So\marginnote{1.1} I have heard. }At one time the Buddha was staying in a forest near \textsanskrit{Icchānaṅgala}. Now at that time several very well-known well-to-do brahmins were residing in \textsanskrit{Icchānaṅgala}. They included the brahmins \textsanskrit{Caṅkī}, \textsanskrit{Tārukkha}, \textsanskrit{Pokkharasāti}, \textsanskrit{Jāṇussoṇi}, Todeyya, and others. Then as the brahmin students \textsanskrit{Vāseṭṭha} and \textsanskrit{Bhāradvāja} were going for a walk they began to discuss the question: “How do you become a brahmin?” 

\textsanskrit{Bhāradvāja}\marginnote{2.1} said this: “When you’re well born on both your mother’s and father’s side, of pure descent, irrefutable and impeccable in questions of ancestry back to the seventh paternal generation—then you’re a brahmin.” 

\textsanskrit{Vāseṭṭha}\marginnote{3.1} said this: “When you’re ethical and accomplished in doing your duties—then you’re a brahmin.” But neither was able to persuade the other. 

So\marginnote{4.1} \textsanskrit{Vāseṭṭha} said to \textsanskrit{Bhāradvāja}, “Master \textsanskrit{Bhāradvāja}, the ascetic Gotama—a Sakyan, gone forth from a Sakyan family—is staying in a forest near \textsanskrit{Icchānaṅgala}. He has this good reputation: ‘That Blessed One is perfected, a fully awakened Buddha, accomplished in knowledge and conduct, holy, knower of the world, supreme guide for those who wish to train, teacher of gods and humans, awakened, blessed.’ Come, let’s go to see him and ask him about this matter. As he answers, so we’ll remember it.” “Yes, sir,” replied \textsanskrit{Bhāradvāja}. 

So\marginnote{5.1} they went to the Buddha and exchanged greetings with him. When the greetings and polite conversation were over, they sat down to one side, and \textsanskrit{Vāseṭṭha} addressed the Buddha in verse: 

\begin{verse}%
“We’re\marginnote{6.1} both authorized masters \\
of the three Vedas. \\
I’m a student of \textsanskrit{Pokkharasāti}, \\
and he of \textsanskrit{Tārukkha}. 

We’re\marginnote{7.1} fully qualified \\
in all the Vedic experts teach. \\
As philologists and grammarians, \\
we match our teachers in recitation. 

We\marginnote{8.1} have a dispute \\
regarding the question of ancestry. \\
For \textsanskrit{Bhāradvāja} says that \\
one is a brahmin due to birth, \\
but I declare it’s because of one’s actions. \\
Oh seer, know this as our debate. 

Since\marginnote{9.1} neither of us was able \\
to convince the other, \\
we’ve come to ask you, sir, \\
renowned as the awakened one. 

As\marginnote{10.1} people honor with joined palms \\
the moon on the cusp of waxing, \\
bowing, they revere \\
Gotama in the world. 

We\marginnote{11.1} ask this of Gotama, \\
the eye arisen in the world: \\
is one a brahmin due to birth, \\
or else because of actions? \\
We don’t know, please tell us, \\
so we can recognize a brahmin.” 

“I\marginnote{12.1} shall explain to you,” \\
\scspeaker{said the Buddha, }\\
“accurately and in sequence, \\
the taxonomy of living creatures, \\
for species are indeed diverse. 

Know\marginnote{13.1} the grass and trees, \\
though they lack self-awareness. \\
They’re defined by birth, \\
for species are indeed diverse. 

Next\marginnote{14.1} there are bugs and moths, \\
and so on, to ants and termites. \\
They’re defined by birth, \\
for species are indeed diverse. 

Know\marginnote{15.1} the quadrupeds, too, \\
both small and large. \\
They’re defined by birth, \\
for species are indeed diverse. 

Know,\marginnote{16.1} too, the long-backed snakes, \\
crawling on their bellies. \\
They’re defined by birth, \\
for species are indeed diverse. 

Next\marginnote{17.1} know the fish, \\
whose habitat is the water. \\
They’re defined by birth, \\
for species are indeed diverse. 

Next\marginnote{18.1} know the birds, \\
flying with wings as chariots. \\
They’re defined by birth, \\
for species are indeed diverse. 

While\marginnote{19.1} the differences between these species \\
are defined by birth, \\
the differences between humans \\
are not defined by birth. 

Not\marginnote{20.1} by hair nor by head, \\
not by ear nor by eye, \\
not by mouth nor by nose, \\
not by lips nor by eyebrow, 

not\marginnote{21.1} by shoulder nor by neck, \\
not by belly nor by back, \\
not by buttocks nor by breast, \\
not by groin nor by genitals, 

not\marginnote{22.1} by hands nor by feet, \\
not by fingers nor by nails, \\
not by knees nor by thighs, \\
not by color nor by voice: \\
none of these are defined by birth \\
as it is for other species. 

In\marginnote{23.1} individual human bodies \\
you can’t find such distinctions. \\
The distinctions among humans \\
are spoken of by convention. 

Anyone\marginnote{24.1} among humans \\
who lives off keeping cattle: \\
know them, \textsanskrit{Vāseṭṭha}, \\
as a farmer, not a brahmin. 

Anyone\marginnote{25.1} among humans \\
who lives off various professions: \\
know them, \textsanskrit{Vāseṭṭha}, \\
as a professional, not a brahmin. 

Anyone\marginnote{26.1} among humans \\
who lives off trade: \\
know them, \textsanskrit{Vāseṭṭha}, \\
as a trader, not a brahmin. 

Anyone\marginnote{27.1} among humans \\
who lives off serving others: \\
know them, \textsanskrit{Vāseṭṭha}, \\
as an employee, not a brahmin. 

Anyone\marginnote{28.1} among humans \\
who lives off stealing: \\
know them, \textsanskrit{Vāseṭṭha}, \\
as a bandit, not a brahmin. 

Anyone\marginnote{29.1} among humans \\
who lives off archery: \\
know them, \textsanskrit{Vāseṭṭha}, \\
as a soldier, not a brahmin. 

Anyone\marginnote{30.1} among humans \\
who lives off priesthood: \\
know them, \textsanskrit{Vāseṭṭha}, \\
as a sacrificer, not a brahmin. 

Anyone\marginnote{31.1} among humans \\
who taxes village and nation, \\
know them, \textsanskrit{Vāseṭṭha}, \\
as a ruler, not a brahmin. 

I\marginnote{32.1} don’t call someone a brahmin \\
after the mother or womb they came from. \\
If they still have attachments, \\
they’re just someone who says ‘sir’. \\
Having nothing, taking nothing: \\
that’s who I call a brahmin. 

Having\marginnote{33.1} cut off all fetters \\
they have no anxiety; \\
they’ve got over clinging, and are detached: \\
that’s who I call a brahmin. 

They’ve\marginnote{34.1} cut the strap and harness, \\
the reins and bridle too; \\
with cross-bar lifted, they’re awakened: \\
that’s who I call a brahmin. 

Abuse,\marginnote{35.1} killing, caging: \\
they endure these without anger. \\
Patience is their powerful army: \\
that’s who I call a brahmin. 

Not\marginnote{36.1} irritable or stuck up, \\
dutiful in precepts and observances, \\
tamed, bearing their final body: \\
that’s who I call a brahmin. 

Like\marginnote{37.1} water from a lotus leaf, \\
like a mustard seed off a pin-point, \\
sensual pleasures slip off them: \\
that’s who I call a brahmin. 

They\marginnote{38.1} understand for themselves \\
the end of suffering in this life; \\
with burden put down, detached: \\
that’s who I call a brahmin. 

Deep\marginnote{39.1} in wisdom, intelligent, \\
expert in the variety of paths; \\
arrived at the highest goal: \\
that’s who I call a brahmin. 

Socializing\marginnote{40.1} with neither \\
householders nor the homeless; \\
a migrant with no shelter, few in wishes: \\
that’s who I call a brahmin. 

They’ve\marginnote{41.1} laid aside violence \\
against creatures firm and frail; \\
not killing or making others kill: \\
that’s who I call a brahmin. 

Not\marginnote{42.1} fighting among those who fight, \\
extinguished among those who are armed, \\
not taking among those who take: \\
that’s who I call a brahmin. 

They’ve\marginnote{43.1} discarded greed and hate, \\
along with conceit and contempt, \\
like a mustard seed off the point of a pin: \\
that’s who I call a brahmin. 

The\marginnote{44.1} words they utter \\
are sweet, informative, and true, \\
and don’t offend anyone: \\
that’s who I call a brahmin. 

They\marginnote{45.1} don’t steal anything in the world, \\
long or short, \\
fine or coarse, beautiful or ugly: \\
that’s who I call a brahmin. 

They\marginnote{46.1} have no hope \\
for this world or the next. \\
with no need for hope, detached: \\
that’s who I call a brahmin. 

They\marginnote{47.1} have no clinging, \\
knowledge has freed them of indecision, \\
they’ve plunged right into the deathless: \\
that’s who I call a brahmin. 

They’ve\marginnote{48.1} escaped clinging \\
to both good and bad deeds; \\
sorrowless, stainless, pure: \\
that’s who I call a brahmin. 

Pure\marginnote{49.1} as the spotless moon, \\
clear and undisturbed, \\
they’ve ended delight and future lives: \\
that’s who I call a brahmin. 

They’ve\marginnote{50.1} got past this grueling swamp \\
of delusion, transmigration. \\
Meditating in stillness, free of indecision, \\
they have crossed over to the far shore. \\
They’re extinguished by not grasping: \\
that’s who I call a brahmin. 

They’ve\marginnote{51.1} given up sensual stimulations, \\
and have gone forth from lay life; \\
they’ve ended rebirth in the sensual realm: \\
that’s who I call a brahmin. 

They’ve\marginnote{52.1} given up craving, \\
and have gone forth from lay life; \\
they’ve ended craving to be reborn: \\
that’s who I call a brahmin. 

They’ve\marginnote{53.1} given up human bonds, \\
and gone beyond heavenly bonds; \\
detached from all attachments: \\
that’s who I call a brahmin. 

Giving\marginnote{54.1} up discontent and desire, \\
they’re cooled and free of attachments; \\
a hero, master of the whole world: \\
that’s who I call a brahmin. 

They\marginnote{55.1} know the passing away \\
and rebirth of all beings; \\
unattached, holy, awakened: \\
that’s who I call a brahmin. 

Gods,\marginnote{56.1} fairies, and humans \\
don’t know their destiny; \\
the perfected ones with defilements ended: \\
that’s who I call a brahmin. 

They\marginnote{57.1} have nothing before or after, \\
or even in between. \\
Having nothing, taking nothing: \\
that’s who I call a brahmin. 

Leader\marginnote{58.1} of the herd, excellent hero, \\
great hermit and victor; \\
unstirred, washed, awakened: \\
that’s who I call a brahmin. 

They\marginnote{59.1} know their past lives, \\
and sees heaven and places of loss, \\
and has attained the ending of rebirth, \\
that’s who I call a brahmin. 

For\marginnote{60.1} name and clan are formulated \\
as mere convention in the world. \\
Produced by mutual agreement, \\
they’re formulated for each individual. 

For\marginnote{61.1} a long time this misconception \\
has prejudiced those who don’t understand. \\
Ignorant, they declare \\
that one is a brahmin by birth. 

You’re\marginnote{62.1} not a brahmin by birth, \\
nor by birth a non-brahmin. \\
You’re a brahmin by your deeds, \\
and by deeds a non-brahmin. 

You’re\marginnote{63.1} a farmer by your deeds, \\
by deeds you’re a professional; \\
you’re a trader by your deeds, \\
by deeds are you an employee; 

you’re\marginnote{64.1} a bandit by your deeds, \\
by deeds you’re a soldier; \\
you’re a sacrificer by your deeds, \\
by deeds you’re a ruler. 

In\marginnote{65.1} this way the astute regard deeds \\
in accord with truth. \\
Seeing dependent origination, \\
they’re expert in deeds and their results. 

Deeds\marginnote{66.1} make the world go on, \\
deeds make people go on; \\
sentient beings are bound by deeds, \\
like a moving chariot’s linchpin. 

By\marginnote{67.1} austerity and spiritual practice, \\
by restraint and by self-control: \\
that’s how to become a brahmin, \\
this is the supreme brahmin. 

Accomplished\marginnote{68.1} in the three knowledges, \\
peaceful, with rebirth ended, \\
know them, \textsanskrit{Vāseṭṭha}, \\
as \textsanskrit{Brahmā} and Sakka to the wise.” 

%
\end{verse}

When\marginnote{69.1} he had spoken, \textsanskrit{Vāseṭṭha} and \textsanskrit{Bhāradvāja} said to him, “Excellent, Master Gotama! Excellent! … From this day forth, may Master Gotama remember us as lay followers who have gone for refuge for life.” 

%
\section*{{\suttatitleacronym Snp 3.10}{\suttatitletranslation With Kokālika }{\suttatitleroot Kokālikasutta}}
\addcontentsline{toc}{section}{\tocacronym{Snp 3.10} \toctranslation{With Kokālika } \tocroot{Kokālikasutta}}
\markboth{With Kokālika }{Kokālikasutta}
\extramarks{Snp 3.10}{Snp 3.10}

\scevam{So\marginnote{1.1} I have heard. }At one time the Buddha was staying near \textsanskrit{Sāvatthī} in Jeta’s Grove, \textsanskrit{Anāthapiṇḍika}’s monastery. Then the mendicant \textsanskrit{Kokālika} went up to the Buddha, bowed, sat down to one side, and said to him, “Sir, \textsanskrit{Sāriputta} and \textsanskrit{Moggallāna} have wicked desires. They’ve fallen under the sway of wicked desires.” 

When\marginnote{2.1} this was said, the Buddha said to \textsanskrit{Kokālika}, “Don’t say that, \textsanskrit{Kokālika}! Don’t say that, \textsanskrit{Kokālika}! Have confidence in \textsanskrit{Sāriputta} and \textsanskrit{Moggallāna}, they’re good monks.” 

For\marginnote{3.1} a second time … For a third time \textsanskrit{Kokālika} said to the Buddha, “Despite my faith and trust in the Buddha, \textsanskrit{Sāriputta} and \textsanskrit{Moggallāna} have wicked desires. They’ve fallen under the sway of wicked desires.” For a third time, the Buddha said to \textsanskrit{Kokālika}, “Don’t say that, \textsanskrit{Kokālika}! Don’t say that, \textsanskrit{Kokālika}! Have confidence in \textsanskrit{Sāriputta} and \textsanskrit{Moggallāna}, they’re good monks.” 

Then\marginnote{4.1} \textsanskrit{Kokālika} got up from his seat, bowed, and respectfully circled the Buddha, keeping him on his right, before leaving. Not long after he left his body erupted with boils the size of mustard seeds. The boils grew to the size of mung beans, then chickpeas, then jujube seeds, then jujubes, then myrobalans, then unripe wood apples, then ripe wood apples. Finally they burst open, and pus and blood oozed out. Then the mendicant \textsanskrit{Kokālika} died of that illness. He was reborn in the Pink Lotus hell because of his resentment for \textsanskrit{Sāriputta} and \textsanskrit{Moggallāna}. 

Then,\marginnote{5.1} late at night, the beautiful \textsanskrit{Brahmā} Sahampati, lighting up the entire Jeta’s Grove, went up to the Buddha, bowed, stood to one side, and said to him, “Sir, the mendicant \textsanskrit{Kokālika} has passed away. He was reborn in the Pink Lotus hell because of his resentment for \textsanskrit{Sāriputta} and \textsanskrit{Moggallāna}.” That’s what \textsanskrit{Brahmā} Sahampati said. Then he bowed and respectfully circled the Buddha, keeping him on his right side, before vanishing right there. 

Then,\marginnote{6.1} when the night had passed, the Buddha told the mendicants all that had happened. 

When\marginnote{7.1} he said this, one of the mendicants said to the Buddha, “Sir, how long is the life span in the Pink Lotus hell?” “It’s long, mendicant. It’s not easy to calculate how many years, how many hundreds or thousands or hundreds of thousands of years it lasts.” “But sir, is it possible to give a simile?” “It’s possible,” said the Buddha. 

“Suppose\marginnote{8.1} there was a Kosalan cartload of twenty bushels of sesame seed. And at the end of every hundred years someone would remove a single seed from it. By this means the Kosalan cartload of twenty bushels of sesame seed would run out faster than a single lifetime in the Abbuda hell. Now, twenty lifetimes in the Abbuda hell equal one lifetime in the Nirabbuda hell. Twenty lifetimes in the Nirabbuda hell equal one lifetime in the Ababa hell. Twenty lifetimes in the Ababa hell equal one lifetime in the \textsanskrit{Aṭaṭa} hell. Twenty lifetimes in the \textsanskrit{Aṭaṭa} hell equal one lifetime in the Ahaha hell. Twenty lifetimes in the Ahaha hell equal one lifetime in the Yellow Lotus hell. Twenty lifetimes in the Yellow Lotus hell equal one lifetime in the Sweet-Smelling hell. Twenty lifetimes in the Sweet-Smelling hell equal one lifetime in the Blue Water Lily hell. Twenty lifetimes in the Blue Water Lily hell equal one lifetime in the White Lotus hell. Twenty lifetimes in the White Lotus hell equal one lifetime in the Pink Lotus hell. The mendicant \textsanskrit{Kokālika} has been reborn in the Pink Lotus hell because of his resentment for \textsanskrit{Sāriputta} and \textsanskrit{Moggallāna}.” That is what the Buddha said. Then the Holy One, the Teacher, went on to say: 

\begin{verse}%
“A\marginnote{9.1} person is born \\
with an axe in their mouth. \\
A fool cuts themselves with it \\
when they say bad words. 

When\marginnote{10.1} you praise someone worthy of criticism, \\
or criticize someone worthy of praise, \\
you choose bad luck with your own mouth: \\
you’ll never find happiness that way. 

Bad\marginnote{11.1} luck at dice is a trivial thing, \\
if all you lose is your money \\
and all you own, even yourself. \\
What’s really terrible luck \\
is to hate the holy ones. 

For\marginnote{12.1} more than two quinquadecillion years, \\
and another five quattuordecillion years, \\
a slanderer of noble ones goes to hell, \\
having aimed bad words and thoughts at them. 

A\marginnote{13.1} liar goes to hell, \\
as does one who denies what they did. \\
Both are equal in the hereafter, \\
those men of base deeds. 

Whoever\marginnote{14.1} wrongs a man who has done no wrong, \\
a pure man who has not a blemish, \\
the evil backfires on the fool, \\
like fine dust thrown upwind. 

One\marginnote{15.1} addicted to the way of greed, \\
abuses others with their speech, \\
faithless, miserly, uncharitable, \\
stingy, addicted to backbiting. 

Foul-mouthed,\marginnote{16.1} divisive, ignoble,\footnote{I can’t find any evidence to support the commentary’s gloss as “liar”. I assume it is shorthand for \textit{\textsanskrit{vebhūtiya}} which is a synonym of \textit{\textsanskrit{pesuṇiya}} (DN 30:2.21.2 and DN 28:11.2). } \\
a life-destroyer, wicked, wrongdoer, \\
worst of men, cursed, base-born—\\
quiet now, for you are bound for hell.\footnote{I can’t find any explanation for the shift to direct address in second person here. Perhaps these verses were inserted from a separate source. } 

You\marginnote{17.1} stir up dust, causing harm,\footnote{Norman and Bodhi both accept commentary’s \textit{kilesarajaṃ attani pakkhipasi}, but this seems too narrow; their deeds harmed others too. } \\
when you, evildoer, malign the good. \\
Having done many bad deeds, \\
you’ll go to the pit for a long time. 

For\marginnote{18.1} no-one’s deeds are ever lost, \\
they return to their owner. \\
In the next life that stupid evildoer \\
sees suffering in themselves. 

They\marginnote{19.1} approach the place of impalement, \\
with its iron spikes, sharp blades, and iron stakes. \\
Then there is the food, which appropriately, \\
is like a red-hot iron ball. 

For\marginnote{20.1} the speakers speak not sweetly,\footnote{Commentary says the hell-keepers are the subject here, hell-dwellers the subject of the next line. But this seems forced and unnecessary. Those going to hell are screaming, not uttering sweet words. } \\
they don’t hurry there, or find shelter. \\
They lie upon a spread of coals, \\
they enter a blazing mass of fire.\footnote{Read \textit{aggini-\textsanskrit{samaṁ} \textsanskrit{jalitaṁ}}, as at Snp 3.10:22.2 below. } 

Wrapping\marginnote{21.1} them in a net, \\
they strike them there with iron hammers. \\
They come to blinding darkness, \\
which spreads about them like a fog. 

Next\marginnote{22.1} they enter a copper pot, \\
a blazing mass of fire. \\
There they roast for a long time, \\
writhing in the masses of fire. 

Then\marginnote{23.1} the evildoer roasts there \\
in a mixture of pus and blood. \\
No matter where they settle, \\
everything they touch there hurts them. 

The\marginnote{24.1} evildoer roasts in \\
worm-infested water. \\
There’s not even a shore to go to, \\
for all around are the same kind of pots. 

They\marginnote{25.1} enter the Wood of Sword-Leaves, \\
so sharp they cut their body to pieces. \\
Having grabbed the tongue with a hook, \\
they stab it, slashing back and forth. 

Then\marginnote{26.1} they approach the impassable  \textsanskrit{Vetaraṇi} River, \\
with its sharp blades, its razor blades. \\
Idiots fall into it, \\
the wicked who have done wicked deeds. 

There\marginnote{27.1} dogs all brown and spotted, \\
and raven flocks, and greedy jackals \\
devour them as they wail, \\
while hawks and crows attack them. 

Hard,\marginnote{28.1} alas, is the life here \\
that evildoers endure. \\
That’s why for the rest of this life \\
a person ought do their duty without fail. 

Experts\marginnote{29.1} have counted the loads of sesame\footnote{The commentary says that these two verses are absent from the old “Great Commentary”. Bodhi takes this as a sign they were a later addition. Norman thinks otherwise, arguing that these verses answer the question asked in the prose introduction. But that question has already been answered in the prose, so I don’t think this is a compelling reason. Norman also argues on the basis of metre, which I’m not competent to assess. On the whole, however, I tend to agree with Bodhi here. } \\
as compared to the Pink Lotus Hell. \\
They amount to 50,000,000 times 10,000, \\
plus another 12,000,000,000. 

As\marginnote{30.1} painful as life is said to be in hell, \\
that’s how long one must dwell there. \\
That’s why, for those who are pure, well-behaved, full of good qualities, \\
one should always guard one’s speech and mind.” 

%
\end{verse}

%
\section*{{\suttatitleacronym Snp 3.11}{\suttatitletranslation About Nālaka }{\suttatitleroot Nālakasutta}}
\addcontentsline{toc}{section}{\tocacronym{Snp 3.11} \toctranslation{About Nālaka } \tocroot{Nālakasutta}}
\markboth{About Nālaka }{Nālakasutta}
\extramarks{Snp 3.11}{Snp 3.11}

\begin{verse}%
The\marginnote{1.1} hermit Asita in his daily meditation \\
saw the bright-clad gods of the Thirty-Three\footnote{Bodhi and Norman follow the PTS reading \textit{sakkacca} here, but it seems uncharacteristic to me. Why mention they have just honored Indra? On the other hand, it’s common to express “a group with its leader”. } \\
and their lord Sakka joyfully celebrating, \\
waving streamers in exuberant exaltation. 

Seeing\marginnote{2.1} the gods rejoicing, elated, \\
he paid respects and said this there: \\
“Why is the community of gods in such excellent spirits? \\
Why take up streamers and whirl them about? 

Even\marginnote{3.1} in the war with the demons, \\
when gods were victorious and demons defeated, \\
there was no such excitement. \\
What marvel have the celestials seen that they so rejoice? 

Shouting\marginnote{4.1} and singing and playing music, \\
they clap their hands and dance. \\
I ask you, dwellers on Mount Meru’s peak, \\
quickly dispel my doubt, good sirs!” 

“The\marginnote{5.1} being intent on awakening, a peerless gem, \\
has been born in the human realm for the sake of welfare and happiness, \\
in \textsanskrit{Lumbinī}, a village in the Sakyan land. \\
That’s why we’re so happy, in such excellent spirits. 

He\marginnote{6.1} is supreme among all beings, the best of people, \\
a bull among men, supreme among all creatures. \\
He will roll forth the wheel in the grove of the hermits, \\
roaring like a mighty lion, lord of beasts.” 

Hearing\marginnote{7.1} this, he swiftly descended \\
and right away approached Suddhodana’s home.\footnote{Notice the use of the humble \textit{bhavana} rather than the royal \textit{\textsanskrit{pasāda}}. There is an almost complete lack of royal language in this text, the sole exception being the \textit{antepura} “royal compound” in Snp 3.11:17.2. } \\
Seated there he said this to the Sakyans, \\
“Where is the boy? I too wish to see him!” 

Then\marginnote{8.1} the Sakyans showed their son to the one named Asita—\\
the boy shone like burning gold \\
well-wrought in the forge; \\
resplendent with glory, of peerless beauty. 

The\marginnote{9.1} boy beamed like crested flame, \\
pure as the moon, lord of stars traversing the sky, \\
blazing like the sun freed from the clouds after the rains;\footnote{Here, as so often, “autumn” (per Bodhi and Norman) conveys quite the wrong impression. Autumn is the time of gathering clouds, \textit{sarada} is the time after the rains. } \\
seeing him, he was joyful, brimming with happiness. 

The\marginnote{10.1} celestials held up a parasol in the sky, \\
many-ribbed and thousand-circled; \\
and golden-handled chowries waved—\\
but none could see who held the chowries or the parasols. 

When\marginnote{11.1} the dreadlocked hermit who they called “Dark Splendor” \\
had seen the boy like a gold nugget on a cream rug\footnote{Here the commentary glosses \textit{\textsanskrit{paṇḍu}} as \textit{ratta} (“red”), followed by both Norman (“pale red”) and Bodhi. But \textit{\textsanskrit{paṇḍu}} means “pale, white, cream, yellowish” and I can’t see anywhere in Pali or Sanskrit to suggest a meaning “red”. Given that it’s the standard color of a luxurious rug, perhaps it is due to a change in fashions; in the Buddha’s day luxury was cream-colored, but later it became red. } \\
with a white parasol held over his head, \\
he received him, elated and happy. 

Having\marginnote{12.1} received the Sakyan bull, \\
the seeker, master of marks and hymns, \\
lifted up his voice with confident heart: \\
“He is supreme, the best of men!” 

But\marginnote{13.1} then, remembering he would depart this world, \\
his spirits fell and his tears flowed. \\
Seeing the weeping hermit, the Sakyans said, \\
“Surely there will be no threat to the boy?” 

Seeing\marginnote{14.1} the crestfallen Sakyans, the hermit said, \\
“I do not forsee harm befall the boy, \\
and there will be no threat to him, \\
not in the least; set your minds at ease. 

This\marginnote{15.1} boy shall reach the highest awakening. \\
As one of perfectly purified vision, compassionate for the welfare of the many, \\
he shall roll forth the wheel of the teaching; \\
his spiritual path will become widespread. 

But\marginnote{16.1} I have not long left in this life, \\
I shall die before then. \\
I will never hear the teaching of the one who bore the unequaled burden.\footnote{Neither Bodhi’s “fortitude” nor Norman’s “peerless one” quite capture the force of \textit{asamadhura}. The implication is that, as the forger of the path, the Buddha carries a burden greater than any other. } \\
That’s why I’m so upset and distraught—it’s a disaster for me!” 

Having\marginnote{17.1} brought abundant happiness to the Sakyans, \\
the spiritual seeker left the royal compound. \\
He had a nephew; and out of compassion \\
he encouraged him in the teaching of the one who bore the unequaled burden. 

“When\marginnote{18.1} you hear the voice of another saying ‘Buddha’—\footnote{Note the idiom \textit{parato \textsanskrit{ghosaṁ}} here. } \\
one who has attained awakening and who reveals the foremost teaching—\footnote{This is a succinct definition of \textit{buddha}, not an alternate thing that you might hear people say, per Bodhi and Norman. } \\
go there and ask about his breakthrough;\footnote{Both Norman and Bodhi have “doctrine” here, while commentary is silent. I’m not aware of \textit{samaya} in the sense of “doctrine” in the early texts, nor is it listed in the senses of \textit{samaya} in the commentarial analysis in the \textsanskrit{Dhammasaṅgaṇī} commentary, \textsanskrit{Kāmāvacarakusala}. Given that the concern in previous lines is the notion of an awakened one, surely we have here an abbreviated \textit{abhisamaya} i.e. “breakthrough, enlightenment experience”. } \\
lead the spiritual life under that Blessed One.” 

Now,\marginnote{19.1} that \textsanskrit{Nālaka} had a store of accumulated merit; \\
so when instructed by one of such kindly intent,\footnote{I find it curious that this epithet is used of both the Buddha and Asita. } \\
with perfectly purified vision of the future, \\
he waited in hope for the Victor, guarding his senses.\footnote{Possibly the only usage of “hope” in a spiritual sense in the EBTs. } 

When\marginnote{20.1} he heard of the Victor rolling forth the excellent wheel he went to him, \\
and seeing the leading hermit, he became confident. \\
The time of Asita’s instruction had arrived; \\
so he asked the excellent sage about the highest sagacity. 

%
\end{verse}

\scendsection{The introductory verses are finished. }

\begin{verse}%
“I\marginnote{21.1} now know that Asita’s words \\
have turned out to be true. \\
I ask you this, Gotama, \\
who has gone beyond all things: 

For\marginnote{22.1} one who has entered the homeless life, \\
seeking food on alms round, \\
when questioned, O sage, please tell me \\
of sagacity, the ultimate state.”\footnote{Bodhi accepts the commentarial gloss of \textit{pada} as \textit{\textsanskrit{paṭipadā}} here, but I find his reasoning unconvincing. Yes, the text speaks of practice, but this correlates to the first part of the answer. Here, at the end of the question, it relates to the end of the text, which speaks of the state of sagacity. } 

“I\marginnote{23.1} shall school you in sagacity,” \\
\scspeaker{said the Buddha, }\\
“so difficult and challenging. \\
Come, I shall tell you all about it. \\
Brace yourself; stay strong! 

In\marginnote{24.1} the village, keep the same attitude \\
no matter if reviled or praised. \\
Guard against ill-tempered thoughts, \\
wander peaceful, not frantic. 

Many\marginnote{25.1} different things come up,\footnote{Note that, despite the etymology, \textit{\textsanskrit{uccāvaca}} seems to be used more in the sense of “diversity” rather than “high and low”; eg. the colors of a rainbow. Here all things are like tongues of flame, i.e. there are many different ones, but they are not treated as better or worse. } \\
like tongues of fire in a forest. \\
Women try to seduce a sage—\\
let them not seduce you! 

Refraining\marginnote{26.1} from sex, \\
having left behind sensual pleasures high and low, \\
don’t be hostile or attached \\
to living creatures firm or frail. 

‘As\marginnote{27.1} am I, so are they; \\
as are they, so am I’—\\
Treating others like oneself, \\
neither kill nor incite to kill. 

Leaving\marginnote{28.1} behind desire and greed \\
for what ordinary people are attached to,\footnote{As other instances of this line make clear, the referent of \textit{yattha} is \textit{\textsanskrit{kāmā}}, i.e. the pleasures of the senses, not the desire and greed (of the previous line). Bodhi gets this nuance right, Norman does not. } \\
a seer would set out to practice, \\
they’d cross over this abyss.\footnote{Bodhi has “inferno”, Norman “hell” but the normal sense of “abyss” better fits the metaphor of “crossing over”. } 

With\marginnote{29.1} empty stomach, taking limited food, \\
few in wishes, not greedy; \\
truly hungerless regarding all desires,\footnote{Read \textit{sa ve}. } \\
desireless, one is quenched. 

Having\marginnote{30.1} wandered for alms, \\
they’d take themselves into the forest; \\
and nearing the foot of a tree, \\
the sage would take their seat. 

That\marginnote{31.1} wise one intent on absorption, \\
would delight within the forest. \\
They’d practice absorption at the foot of a tree, \\
filling themselves with bliss. 

Then,\marginnote{32.1} at the end of the night, \\
they’d take themselves into a village. \\
They’d not welcome being called,\footnote{This line and the next echo the practices of Jain ascetics: \textit{naehibhaddantiko, \textsanskrit{natiṭṭhabhaddantiko}, \textsanskrit{nābhihaṭaṁ}}. The sense of the first two terms is similar to \textit{\textsanskrit{avhāna}}, while the last of these terms is the same as \textit{\textsanskrit{abhihāra}} in the next line. } \\
nor offerings brought from the village. 

A\marginnote{33.1} sage who has come to a village \\
would not walk hastily among the families.\footnote{Bodhi’s “should not behave rashly” and Norman’s “should not pursue his search for food inconsiderately” confuse a plain meaning. When on alms round a mendicant should not walk too fast, else the families have no time to ready the food. } \\
They’d not discuss their search for food, \\
nor would they speak suggestively. 

‘I\marginnote{34.1} got something, that’s good. \\
I got nothing, that’s fine.’ \\
Impartial in both cases, \\
they return right to the tree. 

Wandering\marginnote{35.1} with bowl in hand, \\
not dumb, but thought to be dumb, \\
they wouldn’t scorn a tiny gift, \\
nor look down upon the giver. 

For\marginnote{36.1} the practice has many aspects,\footnote{As above, \textit{\textsanskrit{uccāvaca}} probably means “diverse” practices rather than “high and low” per commentary followed by Bodhi and Norman. The only usage of \textit{\textsanskrit{uccāvaca}} in the context of practice is the “diverse duties” performed by mendicants for their fellows (eg. DN 33:3.3.17). } \\
as explained by the Ascetic.\footnote{Norman’s proposal that this is not the Buddha is implausible. } \\
They do not go to the far shore twice,\footnote{This verse is quoted by the \textsanskrit{Kathāvatthu} (Kv 1.2:113.1) to refute the thesis that an arahant might fall away from their attainment. This is overlooked by both Bodhi and Norman, although it evidently underlies the commentary. } \\
nor having gone once do they fall away.\footnote{Both Bodhi and Norman discuss this line, without any compelling conclusion. The problematic \textit{\textsanskrit{mutaṁ}} cannot mean “experienced” per Bodhi and Norman following the commentary. Since the \textsanskrit{Kathāvatthu} is by far the earliest source on this verse, I suggest we adopt its reading that the verse is about “falling away”, and amend \textit{\textsanskrit{mutaṁ}} to \textit{\textsanskrit{cutaṁ}}. } 

When\marginnote{37.1} a mendicant has no creeping, \\
and has cut the stream of craving, \\
and given up all the various duties, \\
no fever is found in them. 

I\marginnote{38.1} shall school you in sagacity.\footnote{The PTS edition, as does the BJT, has a reciter’s remark here identifying the speaker as the Buddha, and this is translated without comment by Norman and Bodhi. It is unusual, if not unique, to find such a mark in the middle of a series of verses by one speaker. The \textsanskrit{Mahāsaṅgīti} edition, following the VRI, lacks this remark and I have translated accordingly. } \\
Practice as if you were licking a razor’s edge. \\
With tongue pressed to the roof of your mouth, \\
be restrained regarding your stomach. 

Don’t\marginnote{39.1} be sluggish in mind, \\
nor think overly much. \\
Be free of putrefaction and unattached, \\
committed to the spiritual life. 

Train\marginnote{40.1} in a lonely seat, \\
attending closely to ascetics; \\
solitude is sagacity, they say. \\
If you welcome solitude, \\
you’ll light up the ten directions. 

Having\marginnote{41.1} heard the words of the wise,\footnote{By forcing the commentary’s sense of \textit{kitti} into \textit{nigghosa}, Bodhi and Norman confuse a simple meaning. \textit{Nighosa} (more commonly spelled \textit{nigghosa}) just means “word, teaching, statement”, eg. \textit{tava \textsanskrit{sutvāna} \textsanskrit{nigghosaṁ}} “having heard your teaching”. } \\
the meditators who’ve given up sensual desires, \\
a follower of mine would develop \\
conscience and faith all the more. 

Understand\marginnote{42.1} this by the way streams move \\
in clefts and crevices: \\
the little creeks flow on babbling, \\
while silent flow the great rivers. 

What\marginnote{43.1} is lacking, babbles; \\
what is full is at peace. \\
The fool is like a half-full pot; \\
the wise like a brimfull lake. 

When\marginnote{44.1} the Ascetic speaks much \\
it is relevant and meaningful: \\
knowing, he teaches the Dhamma; \\
knowing, he speaks much. 

But\marginnote{45.1} one who, knowing, is restrained, \\
knowing, does not speak much; \\
that sage is worthy of sagacity, \\
that sage has achieved sagacity.” 

%
\end{verse}

%
\section*{{\suttatitleacronym Snp 3.12}{\suttatitletranslation 3.12 Contemplating Pairs }{\suttatitleroot Dvayatānupassanāsutta}}
\addcontentsline{toc}{section}{\tocacronym{Snp 3.12} \toctranslation{3.12 Contemplating Pairs } \tocroot{Dvayatānupassanāsutta}}
\markboth{3.12 Contemplating Pairs }{Dvayatānupassanāsutta}
\extramarks{Snp 3.12}{Snp 3.12}

\scevam{So\marginnote{1.1} I have heard. }At one time the Buddha was staying near \textsanskrit{Sāvatthī} in the Eastern Monastery, the stilt longhouse of \textsanskrit{Migāra}’s mother. Now, at that time it was the sabbath—the full moon on the fifteenth day—and the Buddha was sitting in the open surrounded by the \textsanskrit{Saṅgha} of monks. Then the Buddha looked around the \textsanskrit{Saṅgha} of monks, who were so very silent. He addressed them: 

“Suppose,\marginnote{2.1} mendicants, they questioned you thus: ‘There are skillful teachings that are noble, emancipating, and lead to awakening. What is the real reason for listening to such teachings?’ You should answer: ‘Only so as to truly know the pairs of teachings.’ And what pairs do they speak of? 

‘This\marginnote{3.1} is suffering; this is the origin of suffering’: this is the first contemplation. ‘This is the cessation of suffering; this is the practice that leads to the cessation of suffering’: this is the second contemplation. When a mendicant meditates rightly contemplating a pair of teachings in this way—diligent, keen, and resolute—they can expect one of two results: enlightenment in the present life, or if there’s something left over, non-return.” 

That\marginnote{4.1} is what the Buddha said. Then the Holy One, the Teacher, went on to say: 

\begin{verse}%
“There\marginnote{5.1} are those who don’t understand suffering \\
and suffering’s cause, \\
and where all suffering \\
ceases with nothing left over. \\
And they don’t know the path \\
that leads to the stilling of suffering. 

They\marginnote{6.1} lack the heart’s release, \\
as well as the release by wisdom. \\
Unable to make an end, \\
they continue to be reborn and grow old. 

But\marginnote{7.1} there are those who understand suffering \\
and suffering’s cause, \\
and where all suffering \\
ceases with nothing left over. \\
And they understand the path \\
that leads to the stilling of suffering. 

They’re\marginnote{8.1} endowed with the heart’s release, \\
as well as the release by wisdom. \\
Able to make an end, \\
they don’t continue to be reborn and grow old.” 

%
\end{verse}

“Suppose,\marginnote{9.1} mendicants, they questioned you thus: ‘Could there be another way to contemplate the pairs?’ You should say, ‘There could.’ And how could there be? ‘All the suffering that originates is caused by attachment’: this is one contemplation. ‘With the utter cessation of attachment there is no origination of suffering’: this is the second contemplation. When a mendicant meditates in this way they can expect enlightenment or non-return.” Then the Teacher went on to say: 

\begin{verse}%
“Attachment\marginnote{10.1} is the source of suffering \\
in all its countless forms in the world. \\
When an ignorant person builds up attachments, \\
that idiot returns to suffering again and again. \\
So let one who understands not build up attachments, \\
contemplating the birth and origin of suffering.” 

%
\end{verse}

“Suppose,\marginnote{11.1} mendicants, they questioned you thus: ‘Could there be another way to contemplate the pairs?’ You should say, ‘There could.’ And how could there be? ‘All the suffering that originates is caused by ignorance’: this is one contemplation. ‘With the utter cessation of ignorance there is no origination of suffering’: this is the second contemplation. When a mendicant meditates in this way they can expect enlightenment or non-return.” Then the Teacher went on to say: 

\begin{verse}%
“Those\marginnote{12.1} who journey again and again, \\
transmigrating through birth and death; \\
they go from this state to another, \\
destined only for ignorance. 

For\marginnote{13.1} ignorance is the great delusion \\
because of which we have long transmigrated. \\
Those beings who have arrived at knowledge \\
do not proceed to a future life.” 

%
\end{verse}

“‘Could\marginnote{14.1} there be another way?’ … And how could there be? ‘All the suffering that originates is caused by choices’: this is one contemplation. ‘With the utter cessation of choices there is no origination of suffering’: this is the second contemplation. When a mendicant meditates in this way they can expect enlightenment or non-return.” Then the Teacher went on to say: 

\begin{verse}%
“All\marginnote{15.1} the suffering that originates \\
is caused by choices. \\
With the cessation of choices, \\
there is no origination of suffering. 

Knowing\marginnote{16.1} this danger, \\
that suffering is caused by choices; \\
through the stilling of all choices, \\
and the stopping of perceptions, \\
this is the way suffering ends. \\
For those who truly know this, 

rightly\marginnote{17.1} seeing, knowledge masters, \\
the astute, understanding rightly, \\
having overcome \textsanskrit{Māra}’s bonds, \\
do not proceed to a future life.” 

%
\end{verse}

“‘Could\marginnote{18.1} there be another way?’ … And how could there be? ‘All the suffering that originates is caused by consciousness’: this is one contemplation. ‘With the utter cessation of consciousness there is no origination of suffering’: this is the second contemplation. When a mendicant meditates in this way they can expect enlightenment or non-return.” Then the Teacher went on to say: 

\begin{verse}%
“All\marginnote{19.1} the suffering that originates \\
is caused by consciousness. \\
With the cessation of consciousness, \\
there is no origination of suffering. 

Knowing\marginnote{20.1} this danger, \\
that suffering is caused by consciousness, \\
with the stilling of consciousness a mendicant \\
is hungerless, extinguished.” 

%
\end{verse}

“‘Could\marginnote{21.1} there be another way?’ … And how could there be? ‘All the suffering that originates is caused by contact’: this is one contemplation. ‘With the utter cessation of contact there is no origination of suffering’: this is the second contemplation. When a mendicant meditates in this way they can expect enlightenment or non-return.” Then the Teacher went on to say: 

\begin{verse}%
“Those\marginnote{22.1} mired in contact, \\
swept down the stream of rebirths, \\
practicing the wrong way, \\
are far from the ending of fetters. 

But\marginnote{23.1} those who completely understand contact, \\
who, understanding, delight in peace; \\
by comprehending contact \\
they are hungerless, extinguished.” 

%
\end{verse}

“‘Could\marginnote{24.1} there be another way?’ … And how could there be? ‘All the suffering that originates is caused by feeling’: this is one contemplation. ‘With the utter cessation of feeling there is no origination of suffering’: this is the second contemplation. When a mendicant meditates in this way they can expect enlightenment or non-return.” Then the Teacher went on to say: 

\begin{verse}%
“Having\marginnote{25.1} known everything that is felt—\\
whether pleasure or pain, \\
as well as what’s neutral, \\
both internally and externally—

as\marginnote{26.1} suffering, \\
deceptive, falling apart; \\
they see it vanish with every touch: \\
that’s how they understand it.\footnote{SN 36.2:3.4 has \textit{virajjati}. See Bodhi’s note 1735 for the readings and commentary in both contexts. } \\
With the ending of feelings, a mendicant \\
is hungerless, extinguished.” 

%
\end{verse}

“‘Could\marginnote{27.1} there be another way?’ … And how could there be? ‘All the suffering that originates is caused by craving’: this is one contemplation. ‘With the utter cessation of craving there is no origination of suffering’: this is the second contemplation. When a mendicant meditates in this way they can expect enlightenment or non-return.” Then the Teacher went on to say: 

\begin{verse}%
“Craving\marginnote{28.1} is a person’s partner \\
as they transmigrate on this long journey. \\
They go from this state to another, \\
but don’t escape transmigration. 

Knowing\marginnote{29.1} this danger, \\
that craving is the cause of suffering—\\
rid of craving, free of grasping, \\
a mendicant would wander mindful.” 

%
\end{verse}

“‘Could\marginnote{30.1} there be another way?’ … And how could there be? ‘All the suffering that originates is caused by grasping’: this is one contemplation. ‘With the utter cessation of grasping there is no origination of suffering’: this is the second contemplation. When a mendicant meditates in this way they can expect enlightenment or non-return.” Then the Teacher went on to say: 

\begin{verse}%
“Grasping\marginnote{31.1} is the cause of continued existence; \\
one who exists falls into suffering. \\
Death comes to those who are born—\\
this is the origination of suffering. 

That’s\marginnote{32.1} why with the end of grasping, \\
the astute, understanding rightly, \\
having directly known the end of rebirth, \\
do not proceed to a future life.” 

%
\end{verse}

“‘Could\marginnote{33.1} there be another way?’ … And how could there be? ‘All the suffering that originates is caused by instigating karma’: this is one contemplation.\footnote{I take \textit{\textsanskrit{ārambha}} here as having the same sense as \textit{\textsanskrit{samārambha}} at AN 4.195:6.2. But see too the sense of “transgression” at AN 5.142:6.4. } ‘With the utter cessation of instigation there is no origination of suffering’: this is the second contemplation. When a mendicant meditates in this way they can expect enlightenment or non-return.” Then the Teacher went on to say: 

\begin{verse}%
“All\marginnote{34.1} the suffering that originates \\
is caused by instigating karma. \\
With the cessation of instigation, \\
there is no origination of suffering. 

Knowing\marginnote{35.1} this danger, \\
that suffering is caused by instigating karma, \\
having given up all instigation, \\
one is freed with respects to instigation. 

For\marginnote{36.1} the mendicant with peaceful mind, \\
who has cut off craving for continued existence, \\
transmigration through births is finished; \\
there are no future lives for them.” 

%
\end{verse}

“‘Could\marginnote{37.1} there be another way?’ … And how could there be? ‘All the suffering that originates is caused by sustenance’: this is one contemplation. ‘With the utter cessation of sustenance there is no origination of suffering’: this is the second contemplation. When a mendicant meditates in this way they can expect enlightenment or non-return.” Then the Teacher went on to say: 

\begin{verse}%
“All\marginnote{38.1} the suffering that originates \\
is caused by sustenance. \\
With the cessation of sustenance, \\
there is no origination of suffering. 

Knowing\marginnote{39.1} this danger, \\
that suffering is caused by sustenance, \\
completely understanding all sustenance, \\
one is independent of all sustenance. 

Having\marginnote{40.1} rightly understood the state of health, \\
through the ending of defilements, \\
using after reflection, firm in principle,\footnote{\textit{\textsanskrit{Saṅkhāya} \textsanskrit{sevī}} refers to the monastic practice of making use of requisites, including food, only after reflection on them. } \\
a knowledge master cannot be reckoned.” 

%
\end{verse}

“‘Could\marginnote{41.1} there be another way?’ … And how could there be? ‘All the suffering that originates is caused by perturbation’: this is one contemplation. ‘With the utter cessation of perturbation there is no origination of suffering’: this is the second contemplation. When a mendicant meditates in this way they can expect enlightenment or non-return.” Then the Teacher went on to say: 

\begin{verse}%
“All\marginnote{42.1} the suffering that originates \\
is caused by perturbation. \\
With the cessation of perturbation, \\
there is no origination of suffering. 

Knowing\marginnote{43.1} this danger, \\
that suffering is caused by perturbation, \\
that’s why, having relinquished perturbation,\footnote{Text treats \textit{\textsanskrit{iñjita}} as \textit{eja} and the translation should reflect this. } \\
and stopped making karmic choices, \\
imperturbable, free of grasping, \\
a mendicant would wander mindful.” 

%
\end{verse}

“‘Could\marginnote{44.1} there be another way?’ … And how could there be? ‘For the dependent there is agitation’: this is the first contemplation. ‘For the independent there’s no agitation’: this is the second contemplation. When a mendicant meditates in this way they can expect enlightenment or non-return.” Then the Teacher went on to say: 

\begin{verse}%
“For\marginnote{45.1} the independent there’s no agitation. \\
The dependent, grasping, \\
goes from this state to another, \\
without escaping transmigration. 

Knowing\marginnote{46.1} this danger, \\
the great fear in dependencies, \\
independent, free of grasping, \\
a mendicant would wander mindful.” 

%
\end{verse}

“‘Could\marginnote{47.1} there be another way?’ … And how could there be? ‘Formless states are more peaceful than states of form’: this is the first contemplation. ‘Cessation is more peaceful than formless states’: this is the second contemplation. When a mendicant meditates in this way they can expect enlightenment or non-return.” Then the Teacher went on to say: 

\begin{verse}%
“There\marginnote{48.1} are beings in the realm of luminous form, \\
and others stuck in the formless. \\
Not understanding cessation, \\
they return in future lives. 

But\marginnote{49.1} the people who completely understand form, \\
not stuck in the formless, \\
released in cessation—\\
they are conquerors of death.” 

%
\end{verse}

“‘Could\marginnote{50.1} there be another way?’ … And how could there be? ‘What this world—with its gods, \textsanskrit{Māras}, and \textsanskrit{Brahmās}, this population with its ascetics and brahmins, its gods and humans—focuses on as true, the noble ones have clearly seen with right wisdom to be actually false’: this is the first contemplation.\footnote{In this unique formulation, \textit{\textsanskrit{upanijjhāyati}} should be understood in terms of its consistent usage in the Suttas: the excessive and obsessive focusing or gazing on something. It’s not just that people miss what is true, it is that they are looking intently in the wrong direction. } ‘What this world focuses on as false, the noble ones have clearly seen with right wisdom to be actually true’: this is the second contemplation. When a mendicant meditates in this way they can expect enlightenment or non-return.” Then the Teacher went on to say: 

\begin{verse}%
“See\marginnote{51.1} how the world with its gods \\
imagines not-self to be self; \\
habituated to name and form, \\
imagining this is truth. 

For\marginnote{52.1} whatever you imagine it is, \\
it turns out to be something else. \\
And that is what is false in it, \\
for the ephemeral is deceptive by nature. 

Extinguishment\marginnote{53.1} has an undeceptive nature, \\
the noble ones know it as truth. \\
Having comprehended the truth, \\
they are hungerless, extinguished.” 

%
\end{verse}

“Suppose,\marginnote{54.1} mendicants, they questioned you thus: ‘Could there be another way to contemplate the pairs?’ You should say, ‘There could.’ And how could there be? ‘What this world—with its gods, \textsanskrit{Māras}, and \textsanskrit{Brahmās}, this population with its ascetics and brahmins, its gods and humans—focuses on as happiness, the noble ones have clearly seen with right wisdom to be actually suffering’: this is the first contemplation. ‘What this world focuses on as suffering, the noble ones have clearly seen with right wisdom to be actually happiness’: this is the second contemplation. When a mendicant meditates rightly contemplating a pair of teachings in this way—diligent, keen, and resolute—they can expect one of two results: enlightenment in the present life, or if there’s something left over, non-return. That is what the Buddha said. Then the Holy One, the Teacher, went on to say: 

\begin{verse}%
“Sights,\marginnote{55.1} sounds, tastes, smells, \\
touches, and thoughts, the lot of them—\\
they’re likable, desirable, and pleasurable \\
as long as you can say that they exist. 

For\marginnote{56.1} all the world with its gods, \\
this is what they agree is happiness. \\
And where they cease \\
is agreed on as suffering for them. 

The\marginnote{57.1} noble ones have seen as happiness \\
the ceasing of identity. \\
This insight by those who see \\
contradicts the whole world. 

What\marginnote{58.1} others say is happiness \\
the noble ones say is suffering. \\
What others say is suffering \\
the noble ones know as happiness. 

See,\marginnote{59.1} this teaching is hard to understand, \\
it confuses the ignorant. \\
There is darkness for the shrouded; \\
blackness for those who don’t see. 

But\marginnote{60.1} the good are open; \\
like light for those who see. \\
Though close, they do not understand, \\
those fools inexpert in the teaching.\footnote{Read \textit{\textsanskrit{akovidā}} per SN 35.136:7.4. } 

They’re\marginnote{61.1} mired in desire to be reborn, \\
flowing along the stream of lives, \\
mired in \textsanskrit{Māra}’s sway: \\
this teaching isn’t easy for them to understand. 

Who,\marginnote{62.1} apart from the noble ones, \\
is qualified to understand this state? \\
Having rightly understood this state, \\
the undefiled become fully extinguished.” 

%
\end{verse}

That\marginnote{63.1} is what the Buddha said. Satisfied, the mendicants were happy with what the Buddha said. And while this discourse was being spoken, the minds of sixty mendicants were freed from defilements by not grasping. 

%
\addtocontents{toc}{\let\protect\contentsline\protect\nopagecontentsline}
\chapter*{The Chapter of Eights}
\addcontentsline{toc}{chapter}{\tocchapterline{The Chapter of Eights}}
\addtocontents{toc}{\let\protect\contentsline\protect\oldcontentsline}

%
\section*{{\suttatitleacronym Snp 4.1}{\suttatitletranslation Sensual Pleasures }{\suttatitleroot Kāmasutta}}
\addcontentsline{toc}{section}{\tocacronym{Snp 4.1} \toctranslation{Sensual Pleasures } \tocroot{Kāmasutta}}
\markboth{Sensual Pleasures }{Kāmasutta}
\extramarks{Snp 4.1}{Snp 4.1}

\begin{verse}%
If\marginnote{1.1} a mortal desires sensual pleasure \\
and their desire succeeds, \\
they definitely become elated, \\
having got what they want. 

But\marginnote{2.1} for that person in the throes of pleasure, \\
aroused by desire, \\
if those pleasures fade, \\
it hurts like an arrow’s strike. 

One\marginnote{3.1} who, being mindful, \\
avoids sensual pleasures \\
like side-stepping a snake’s head, \\
transcends attachment to the world. 

There\marginnote{4.1} are many objects of sensual desire: \\
fields, lands, and gold; cattle and horses; \\
slaves and servants; women and relatives. \\
When a man lusts over these, 

the\marginnote{5.1} weak overpower him \\
and adversities crush him. \\
Suffering follows him\footnote{Bodhi, Norman, and \textsanskrit{Ñāṇadīpa} all have “enter” here, while Thanissaro has “invades”. But this is a stock line, and it’s hard to read \textit{anveti} as anything other than “follows”. Niddesa (Mnd 1:70.2) has \textit{anveti anugacchati \textsanskrit{anvāyikaṁ} hoti}. The metaphor is not that water “enters” a boat, but that a leaky boat already contains water and takes it along (like a shadow or an ox-cart per the opening verses of the Dhammapada). That’s why it has to be baled out in the next verse. } \\
like water in a leaky boat. 

That’s\marginnote{6.1} why a person, ever mindful, \\
should avoid sensual pleasures. \\
Give them up and cross the flood, \\
as a bailed-out boat reaches the far shore. 

%
\end{verse}

%
\section*{{\suttatitleacronym Snp 4.2}{\suttatitletranslation Eight on the Cave }{\suttatitleroot Guhaṭṭhakasutta}}
\addcontentsline{toc}{section}{\tocacronym{Snp 4.2} \toctranslation{Eight on the Cave } \tocroot{Guhaṭṭhakasutta}}
\markboth{Eight on the Cave }{Guhaṭṭhakasutta}
\extramarks{Snp 4.2}{Snp 4.2}

\begin{verse}%
Trapped\marginnote{1.1} in a cave, thickly overspread, \\
sunk in delusion they stay. \\
A person like this is far from seclusion, \\
for sensual pleasures in the world are not easy to give up. 

The\marginnote{2.1} chains of desire, the bonds of life’s pleasures \\
are hard to escape, for one cannot free another.\footnote{\textit{Te \textsanskrit{duppamuñca}} has been taken by most translators as referring to people, in senses either active (Bodhi’s “let go with difficulty”) or passive (Norman’s “are hard to release”). However the phrase occurs in a very similar context at SN 3.10:5.2 and Dhp 346:2, where it refers to the chains that are hard to escape. } \\
Looking to the past or the future, \\
they pray for these pleasures or former ones.\footnote{\textit{Jappa} means both to “incant” and to “long for”, and the English “pray’ has exactly the same connotations—for exactly the same reasons. They originated in begging favors of a god. } 

Greedy,\marginnote{3.1} fixated, infatuated by sensual pleasures, \\
they are incorrigible, habitually immoral. \\
When led to suffering they lament, \\
“What will become of us when we pass away from here?” 

That’s\marginnote{4.1} why a person should train in this life: \\
should you know that anything in the world is wrong, \\
don’t act wrongly on account of that; \\
for the wise say this life is short. 

I\marginnote{5.1} see the world’s population floundering, \\
given to craving for future lives. \\
Base men wail in the jaws of death, \\
not rid of craving for life after life. 

See\marginnote{6.1} them flounder over belongings, \\
like fish in puddles of a dried-up stream. \\
Seeing this, live unselfishly, \\
forming no attachment to future lives. 

Rid\marginnote{7.1} of desire for both ends,\footnote{The reference here is to the two ends of contact (AN 6.61) rather than the two extremes of views or paths. } \\
having completely understood contact, free of greed, \\
doing nothing for which they’d blame themselves, \\
the wise don’t cling to the seen and the heard. 

Having\marginnote{8.1} completely understood perception and having crossed the flood,\footnote{Bodhi, Norman, and \textsanskrit{Ñāṇadīpa}, apparently following the commentary, take \textit{vitareyya} as an optative. But the previous verse spoke of an arahant, and here the one who has fully understood perception must have already crossed over. \textit{Vitareyya \textsanskrit{oghaṁ}} occurs at Snp 3.5:13.2, where \textit{vitareyya} is clearly an absolutive, and is rendered as such by both Bodhi and Norman. } \\
the sage, not clinging to possessions, \\
with dart plucked out, living diligently, \\
does not long for this world or the next. 

%
\end{verse}

%
\section*{{\suttatitleacronym Snp 4.3}{\suttatitletranslation Eight on Malice }{\suttatitleroot Duṭṭhaṭṭhakasutta}}
\addcontentsline{toc}{section}{\tocacronym{Snp 4.3} \toctranslation{Eight on Malice } \tocroot{Duṭṭhaṭṭhakasutta}}
\markboth{Eight on Malice }{Duṭṭhaṭṭhakasutta}
\extramarks{Snp 4.3}{Snp 4.3}

\begin{verse}%
Some\marginnote{1.1} speak with malicious intent,\footnote{\textit{\textsanskrit{Duṭṭhamana}}, or more commonly \textit{\textsanskrit{paduṭṭhamana}}, means “malice” not “corruption”, as is found through the Suttas and confirmed by the Niddesa. } \\
while others speak set on truth. \\
When disputes come up a sage does not get involved, \\
which is why they’ve no barrenness at all. 

How\marginnote{2.1} can you transcend your own view \\
when you’re led by preference, dogmatic in belief?\footnote{Bodhi, Norman, and \textsanskrit{Ñāṇadīpa} all render \textit{chanda} as “desire”, and that is certainly the sense of \textit{\textsanskrit{chandānunīto}} at SN 35.94:6.4. Here, however, Niddesa glosses with synonyms for “view” and this seems like a more plausible sense. Note too the sense of \textit{\textsanskrit{niviṭṭha}} as “dogmatic”, which recurs through the \textsanskrit{Aṭṭhakavagga} (cp. \textit{abhinivesa}). } \\
Inventing your own undertakings,\footnote{\textit{\textsanskrit{Samattāni}} is regularly used of vows “undertaken”, and given the following verse this is surely the meaning here, \emph{contra} Niddesa and commentary. See Snp 4.4:5.1, where \textit{\textsanskrit{samādāya}} is used in the same sense. } \\
you’d speak according to your notion.\footnote{Here we find a sense of \textit{\textsanskrit{jānāti}} that doesn’t fit the English idea of “knowledge” as “true belief”. The idea is that they speak from their own (limited and opinionated) ideas. It’s not as pejorative as “opinion” nor as solid as “knowledge”, so I render as “notion”. We find this usage commonly in the final two chapters of Snp. } 

Some,\marginnote{3.1} unasked, tell others \\
of their own precepts and vows. \\
They have an ignoble nature, say the experts, \\
since they speak about themselves of their own accord. 

A\marginnote{4.1} mendicant, peaceful, quenched, \\
never boasts “thus am I” of their precepts. \\
They have a noble nature, say the experts, \\
not proud of anything in the world. 

For\marginnote{5.1} one who formulates and creates teachings, \\
and promotes them despite their defects, \\
if they see an advantage for themselves, \\
they become dependent on that, relying on unstable peace. 

It’s\marginnote{6.1} not easy to get over dogmatic views \\
adopted after judging among the teachings. \\
That’s why, among all these dogmas, a person \\
rejects one teaching and takes up another. 

The\marginnote{7.1} cleansed one has no formulated view \\
at all in the world about the different realms. \\
Having given up illusion and conceit, \\
by what path would they go? They are not involved. 

For\marginnote{8.1} one who is involved gets embroiled in disputes about teachings—\\
but how to dispute with the uninvolved? About what? \\
For picking up and putting down is not what they do;\footnote{The problem addressed is one who is picking up one thing after another. It’s therefore best to translate \textit{atta} in an active sense, per Niddesa: \textit{\textsanskrit{gahaṇaṁ} \textsanskrit{muñcanā} samatikkanto}. } \\
they have shaken off all views in this very life. 

%
\end{verse}

%
\section*{{\suttatitleacronym Snp 4.4}{\suttatitletranslation Eight on the Pure }{\suttatitleroot Suddhaṭṭhakasutta}}
\addcontentsline{toc}{section}{\tocacronym{Snp 4.4} \toctranslation{Eight on the Pure } \tocroot{Suddhaṭṭhakasutta}}
\markboth{Eight on the Pure }{Suddhaṭṭhakasutta}
\extramarks{Snp 4.4}{Snp 4.4}

\begin{verse}%
“I\marginnote{1.1} see a pure being of ultimate wellness;\footnote{This line refers to seeing a supposed saint or holy person, a sight that was believed to grant blessings. As Bodhi remarks, \textit{aroga} refers to the \textit{\textsanskrit{attā}}. However this does not necessarily mean an immaterial soul. Here what is seen is the soul as a physical form (\textit{\textsanskrit{rūpī} \textsanskrit{attā}}) manifesting purity and wellness. } \\
it is vision that grants a person purity.” \\
Recalling this notion of the ultimate,\footnote{Here as in 4.3, \textit{\textsanskrit{ñāṇa}} is used in the sense of “notion”. Bodhi, Norman, and \textsanskrit{Ñāṇadīpa} all follow Niddesa in taking \textit{\textsanskrit{abhijāna}} in its usual sense of “understanding, direct knowing”. But here the secondary sense of “recall” fits better. Someone thinking about their vision of a holy man believes that there must be an observer of such purity, i.e. a self. } \\
they believe in the notion that there is one who observes purity.\footnote{\textit{Pacceti} is a standard verb for someone who “believes” in a religious rite or idea. } 

If\marginnote{2.1} a person were granted purity through vision, \\
or if by a notion they could give up suffering, \\
then one with attachments is purified by another: \\
their view betrays them as one who asserts thus. 

The\marginnote{3.1} brahmin speaks not of purity from another \\
in terms of what is seen, heard, or thought; or by precepts or vows. \\
They are unsullied in the midst of good and evil, \\
letting go what was picked up, without creating anything new here. 

Having\marginnote{4.1} let go the last they lay hold of the next; \\
following impulse, they don’t get past the snare.\footnote{Bodhi, Norman, and \textsanskrit{Ñāṇadīpa} all render \textit{\textsanskrit{saṅga}} in some variant of “tie, attachment”. Yet that fails to render the metaphor. \textit{\textsanskrit{Saṅga}} is used in the sense of a “snare” in which one may be caught, like a net (Snp 3.6:28.4). And a snare is indeed something that one may cross over or get past. } \\
They grab on and let go like a monkey \\
grabbing and releasing a branch.\footnote{The monkey grabs the branches, not the trunk, as a metaphor for missing the essence. And to drive home the pun, religious orders are called “branches”. } 

Having\marginnote{5.1} undertaken their own vows, a person \\
visits various teachers, being attached to perception.\footnote{See \textit{\textsanskrit{uccāvacaṁ} \textsanskrit{vā} pana \textsanskrit{dassanāya} gacchati} at AN 6.30:2.2. In the current poem, it refers to someone who visits a variety of teachers or sects, as per Niddesa. } \\
One who knows, having comprehended the truth through the knowledges,\footnote{Here the plural \textit{vedehi} stands for the \textit{\textsanskrit{tevijjā}}. } \\
does not visit various teachers, being of vast wisdom. 

They\marginnote{6.1} are remote from all things \\
seen, heard, or thought. \\
Seeing them living openly, \\
how could anyone in this world judge them? 

They\marginnote{7.1} don’t make things up or promote them, \\
or speak of the uttermost purity. \\
After untying the tight knot of grasping \\
they long for nothing in the world. 

The\marginnote{8.1} brahmin has stepped over the boundary; \\
knowing and seeing, they adopt nothing.\footnote{Note that \textit{\textsanskrit{samuggahīta}} is used here in the same sense as Snp 4.3:6.2 or Snp 2.12:11.1, i.e. the “adoption” of a theory or view. } \\
Neither in love with passion nor besotted by dispassion, \\
there is nothing here they adopt as the ultimate. 

%
\end{verse}

%
\section*{{\suttatitleacronym Snp 4.5}{\suttatitletranslation Eight on the Ultimate }{\suttatitleroot Paramaṭṭhakasutta}}
\addcontentsline{toc}{section}{\tocacronym{Snp 4.5} \toctranslation{Eight on the Ultimate } \tocroot{Paramaṭṭhakasutta}}
\markboth{Eight on the Ultimate }{Paramaṭṭhakasutta}
\extramarks{Snp 4.5}{Snp 4.5}

\begin{verse}%
If,\marginnote{1.1} maintaining that theirs is the “ultimate” view, \\
a person makes it out to be highest in the world;\footnote{Reading \textit{yad-} here and below as \textit{yadi}. } \\
then they declare all others are “lesser”; \\
that’s why they’re not over disputes. 

If\marginnote{2.1} they see an advantage for themselves \\
in what’s seen, heard, or thought; or in precepts or vows, \\
in that case, having adopted that one alone, \\
they see all others as inferior. 

Experts\marginnote{3.1} say that, too, is a knot, \\
relying on which people see others as lesser. \\
That’s why a mendicant ought not rely \\
on what’s seen, heard, or thought, or on precepts and vows. 

Nor\marginnote{4.1} would they form a view about the world \\
through a notion or through precepts and vows. \\
They would never represent themselves as “equal”, \\
nor conceive themselves “worse” or “better”. 

What\marginnote{5.1} was picked up has been set down and is not grasped again; \\
they form no dependency even on notions. \\
They follow no side among the factions, \\
and believe in no view at all. 

One\marginnote{6.1} here who has no wish for either end—\\
for any form of existence in this life or the next—\\
has adopted no dogma at all \\
after judging among the teachings. 

For\marginnote{7.1} them not even the tiniest idea is formulated here \\
regarding what is seen, heard, or thought. \\
That brahmin does not grasp any view—\\
how could anyone in this world judge them? 

They\marginnote{8.1} don’t make things up or promote them, \\
and don’t subscribe to any of the doctrines. \\
The brahmin has no need to be led by precept or vow; \\
gone to the far shore, one such does not return. 

%
\end{verse}

%
\section*{{\suttatitleacronym Snp 4.6}{\suttatitletranslation Old Age }{\suttatitleroot Jarāsutta}}
\addcontentsline{toc}{section}{\tocacronym{Snp 4.6} \toctranslation{Old Age } \tocroot{Jarāsutta}}
\markboth{Old Age }{Jarāsutta}
\extramarks{Snp 4.6}{Snp 4.6}

\begin{verse}%
Short,\marginnote{1.1} alas, is this life; \\
you die before a hundred years. \\
Even if you live a little longer, \\
you still die of old age. 

People\marginnote{2.1} grieve over belongings, \\
yet there is no such thing as permanent possessions. \\
Separation is a fact of life; when you see this, \\
you wouldn’t stay living at home. 

Whatever\marginnote{3.1} a person thinks of as belonging to them, \\
that too is given up when they die. \\
Knowing this, an astute follower of mine \\
would not be bent on ownership.\footnote{\textit{Namati} is usually rendered with “incline” but I feel something stronger is meant here. } 

Just\marginnote{4.1} as, upon awakening, a person does not see \\
what they encountered in a dream; \\
so too you do not see your loved ones \\
when they are dead and gone. 

You\marginnote{5.1} used to see and hear those folk, \\
and call them by their name. \\
Yet the name is all that’s left to tell \\
of a person when they’re gone. 

Those\marginnote{6.1} who are greedy for belongings \\
don’t give up sorrow, lamentation, and stinginess. \\
That’s why the sages, seers of sanctuary,\footnote{The historical past tense here is unusual. It seems to be an allusion to legendary hermits of the past. } \\
left possessions behind and wandered. 

For\marginnote{7.1} a mendicant who lives withdrawn, \\
frequenting a secluded seat, \\
they say it’s fitting \\
to not show themselves in a home. 

The\marginnote{8.1} sage is independent everywhere, \\
they don’t form likes or dislikes. \\
Lamentation and stinginess \\
slip off them like water from a leaf. 

Like\marginnote{9.1} a droplet slips from a lotus-leaf, \\
like water from a lotus flower; \\
the sage doesn’t cling to that \\
which is seen or heard or thought. 

For\marginnote{10.1} the one who is cleansed does not conceive \\
in terms of things seen, heard, or thought. \\
They do not wish to be purified by another; \\
they are neither passionate nor growing dispassioned.\footnote{In the Suttas, \textit{virajjati} is a stock term in the process of liberation. It follows \textit{\textsanskrit{nibbidā}} (“disenchantment”) and precedes \textit{vimutti} (“freedom”). Being free, the arahant has gone beyond this stage. } 

%
\end{verse}

%
\section*{{\suttatitleacronym Snp 4.7}{\suttatitletranslation With Tissametteyya }{\suttatitleroot Tissametteyyasutta}}
\addcontentsline{toc}{section}{\tocacronym{Snp 4.7} \toctranslation{With Tissametteyya } \tocroot{Tissametteyyasutta}}
\markboth{With Tissametteyya }{Tissametteyyasutta}
\extramarks{Snp 4.7}{Snp 4.7}

\begin{verse}%
“When\marginnote{1.1} someone indulges in sex,” \\
\scspeaker{said Venerable Tissametteyya, }\\
“tell us, sir: what trouble befalls them? \\
After hearing your instruction, \\
we shall train in seclusion.” 

“When\marginnote{2.1} someone indulges in sex,” \\
\scspeaker{replied the Buddha, }\\
“they forget their instructions \\
and go the wrong way—\\
that is something ignoble in them. 

Someone\marginnote{3.1} who formerly lived alone \\
and then resorts to sex \\
is like a chariot careening off-track; \\
in the world they call them a low, ordinary person. 

Their\marginnote{4.1} former fame and reputation \\
also fall away. \\
Seeing this, they’d train \\
to give up sex. 

Oppressed\marginnote{5.1} by thoughts, \\
they brood like a wretch. \\
When they hear what others are saying, \\
such a person is embarrassed. 

Then\marginnote{6.1} they lash out with verbal daggers \\
when reproached by others. \\
This is their great blind spot;\footnote{\textit{Gedha} also at AN 5.103:3.2, where the sense is not “entanglement” but “cover from sight”. See Norman’s note on this verse for the apparent confusion between \textit{gedha}, \textit{rodha}, and the intermediary \textit{godha}. } \\
they sink to lies. 

They\marginnote{7.1} once were considered astute,\footnote{A line used of Devadatta at Iti 89:4.1. } \\
committed to the solitary life. \\
But then they indulged in sex, \\
dragged along by desire like an idiot.\footnote{Bodhi follows the commentary and Niddesa in rendering \textit{parikissati} as “afflicted”, as does Norman with “troubled”, both assuming a contracted \textit{parikilissati}. But \textit{parikissati} here is connected with wisdom (or lack thereof), not suffering. Surely we should look to such passages as AN 4.186:2.6, where \textit{parikassati} is the mind that “drags” a person around. \textit{Parikissati} is the passive form. } 

Knowing\marginnote{8.1} this danger \\
in falling from a former state here,\footnote{At Pli Tv Pvr 15:7.2 \textit{\textsanskrit{pubbāpara}} is defined in terms of changing and inconsistent behavior. } \\
a sage would firmly resolve to wander alone, \\
and would not resort to sex. 

They’d\marginnote{9.1} train themselves only in seclusion; \\
this, for the noble ones, is highest. \\
One who wouldn’t conceive themselves “best” due to that—\\
they have truly drawn near to extinguishment. 

People\marginnote{10.1} tied to sensual pleasures envy them: \\
the isolated, wandering sage \\
who has crossed the flood, \\
unconcerned for sensual pleasures. 

%
\end{verse}

%
\section*{{\suttatitleacronym Snp 4.8}{\suttatitletranslation With Pasūra }{\suttatitleroot Pasūrasutta}}
\addcontentsline{toc}{section}{\tocacronym{Snp 4.8} \toctranslation{With Pasūra } \tocroot{Pasūrasutta}}
\markboth{With Pasūra }{Pasūrasutta}
\extramarks{Snp 4.8}{Snp 4.8}

\begin{verse}%
“Here\marginnote{1.1} alone is purity,” they say, \\
denying that there is purification in other teachings. \\
Speaking of the beauty in that which they depend on,\footnote{Norman has “good” for \textit{subha}, Bodhi “excellent”, but the Niddesa shows that the sense is “illuminating, bringing wisdom”. Compare the same phrase at Snp 4.13:16.3, where the context shows we are speaking of those whose doctrine is derived from meditative experience. Perhaps here (as in the \textsanskrit{Pārāyanavagga}) we are dealing with ascetics who rely on the \textit{subhavimutti}, the “beautiful release” (of \textsanskrit{jhānas}). } \\
each one is dogmatic about their own idiosyncratic interpretation.\footnote{Here, as at Snp 4.12:15.3, \textit{puthu} emphasizes how they all have their own distinct view. } 

Desiring\marginnote{2.1} debate, they plunge into an assembly, \\
where each takes the other as a fool.\footnote{\textit{Dahanti} is to “regard” or “see”, per Niddesa and Commentary; Bodhi’s “accuse” seems hard to justify. } \\
Relying on others they state their contention, \\
desiring praise while claiming to be experts. 

Addicted\marginnote{3.1} to debating in the midst of the assembly, \\
their need for praise makes them nervous. \\
But when they’re repudiated they get embarrassed; \\
upset at criticism, they find fault in others. 

If\marginnote{4.1} their doctrine is said to be weak, \\
and judges declare it repudiated, \\
the loser weeps and wails, \\
moaning, “They beat me.” 

When\marginnote{5.1} these arguments come up among ascetics, \\
they get excited or dejected. \\
Seeing this, refrain from contention, \\
for the only purpose is praise and profit. 

But\marginnote{6.1} if, having declared their doctrine, \\
they are praised there in the midst of the assembly, \\
they laugh and proudly show off because of it, \\
having got what they wanted. 

Their\marginnote{7.1} pride is their downfall, \\
yet they speak from conceit and arrogance. \\
Seeing this, one ought not get into arguments, \\
for experts say this is no way to purity. 

As\marginnote{8.1} a warrior, after feasting on royal food, \\
goes roaring, wanting an opponent—\\
go off and find an opponent, \textsanskrit{Sūra}, \\
for here, as before, there is no-one to fight. 

When\marginnote{9.1} someone argues about a view they have adopted, \\
saying, “This is the only truth,” \\
say to them, “Here you’ll have no adversary \\
when a dispute has come up.” 

There\marginnote{10.1} are those who live far from the crowd, \\
not countering views with view. \\
Who is there to argue with you, \textsanskrit{Pasūra}, \\
among those who grasp nothing here as the highest? 

And\marginnote{11.1} so you come along speculating, \\
thinking up theories in your mind. \\
Now that you’ve challenged someone who’s cleansed,\footnote{Here, \textit{yuga} is connected in the Niddesa with \textit{\textsanskrit{yugaggāha}} (“taking the reins”), which in turn is connected in Vb 17:75.2 with “causing disputes”. It seems to be an idiom for “issuing a challenge”. In any case, it has a stronger sense than just “met, encountered”. } \\
you’ll not be able to respond.\footnote{\textit{Sakkhasi} is 2nd future, \textit{\textsanskrit{sampayātave}} is infinitive. \textit{\textsanskrit{Sampāyati}} is a fairly common word, “answer a question”, usually in a negative sense, to be “stumped” by a question. Oddly, most translators (Bodhi, Norman, \textsanskrit{Ñāṇadīpa}, Thanissaro) don’t catch this sense. } 

%
\end{verse}

%
\section*{{\suttatitleacronym Snp 4.9}{\suttatitletranslation With Māgaṇḍiya }{\suttatitleroot Māgaṇḍiyasutta}}
\addcontentsline{toc}{section}{\tocacronym{Snp 4.9} \toctranslation{With Māgaṇḍiya } \tocroot{Māgaṇḍiyasutta}}
\markboth{With Māgaṇḍiya }{Māgaṇḍiyasutta}
\extramarks{Snp 4.9}{Snp 4.9}

\begin{verse}%
“Even\marginnote{1.1} when I saw the sirens Craving, Delight, and Lust,\footnote{\textsanskrit{Māra}’s daughters, the archetypal temptresses. Their failed seduction is told at SN 4.25. \textit{Arati} seems to be an error in the Pali tradition for \textit{rati}. } \\
I had no desire for sex. \\
What is this body full of piss and shit?\footnote{The commentary is surely correct in glossing these lines as \textit{\textbf{\textsanskrit{kimevidaṁ}} \textsanskrit{imissā} \textsanskrit{dārikāya} \textbf{\textsanskrit{muttakarīsapuṇṇaṁ}} \textsanskrit{rūpaṁ}}. \textit{\textsanskrit{Idaṁ}} (neuter) refers to the body (\textit{\textsanskrit{rūpa}}) not to the girl. It’s a meaningful distinction; the Buddha was criticizing the nature of the  body, not shaming the person. } \\
I wouldn’t even want to touch it with my foot.”\footnote{\textit{Icche} is 1st optative. \textit{\textsanskrit{Naṁ}} is rendered as feminine (“her”) by Norman and Bodhi, but it agrees with \textit{\textsanskrit{idaṁ}} in the previous line, so it should be neuter (“it”). Again, the Buddha says nothing in criticism of the woman, only of the body. } 

“If\marginnote{2.1} you do not want a gem such as this, \\
a lady desired by many rulers of men, \\
then what kind of theory, precepts and vows, livelihood, \\
and rebirth in a new life do you assert?” 

“After\marginnote{3.1} judging among the teachings,”\footnote{The sense only emerges with the specific nuance of terms. \textit{Niccheyya} means to “judge, decide” not simply “consider”; and \textit{\textsanskrit{samuggahītaṁ}} means “[a view that is] adopted” (not “assumed” or “grasped”). Finally, the \textit{na} here applies, not just to \textit{\textsanskrit{idaṁ} \textsanskrit{vadāmi}}, but also to \textit{\textsanskrit{samuggahītaṁ}}, with which \textit{\textsanskrit{idaṁ}} agrees. } \\
\scspeaker{said the Buddha to \textsanskrit{Māgaṇḍiya}, }\\
“none have been adopted thinking, ‘I assert this.’ \\
Seeing views without adopting any, \\
searching, I saw inner peace.” 

“O\marginnote{4.1} sage, you speak of judgments you have formed,” \\
\scspeaker{said \textsanskrit{Māgaṇḍiya}, }\\
“without having adopted any of those views. \\
As to that matter of ‘inner peace’—\\
how is that described by the wise?” 

“Purity\marginnote{5.1} is neither spoken of in terms of view,”\footnote{\textit{Suti} is what is “heard”, but with the extra emphasis on “heard via oral transmission of a sacred scripture”. I think the case should be read here as “instrumental of relation”, since the question was not, “how is peace attained”, but “how is it described”. } \\
\scspeaker{said the Buddha to \textsanskrit{Māgaṇḍiya}, }\\
“oral transmission, notion, and precepts and vows;\footnote{\textit{Pi} here doesn’t qualify \textit{\textsanskrit{sīlabbata}} (per Norman), it co-ordinates with \textit{nopi tena}. I capture this with the neither/nor construction. } \\
nor in terms of that without view, oral transmission, \\
notion, and precepts and vows. \\
Having relinquished these, not adopting them, \\
peaceful, independent, one would not pray to be reborn.” 

“It\marginnote{6.1} seems purity is neither spoken of in terms of view,” \\
\scspeaker{said \textsanskrit{Māgaṇḍiya}, }\\
“oral transmission, notion, and precepts and vows; \\
nor in terms of that without view, oral transmission, \\
notion, and precepts and vows. \\
If so, I think this teaching is sheer confusion; \\
for some believe in purity in terms of view.” 

“Continuing\marginnote{7.1} to question while relying on a view,” \\
\scspeaker{said the Buddha to \textsanskrit{Māgaṇḍiya}, }\\
“you’ve become confused by all you’ve adopted. \\
From this you’ve not glimpsed the slightest idea, \\
which is why you consider the teaching confused. 

If\marginnote{8.1} you think that ‘I’m equal, \\
special, or worse’, you’ll get into arguments. \\
Unwavering in the face of the three discriminations, \\
you’ll have no thought ‘I’m equal or special’. 

Why\marginnote{9.1} would that brahmin say, ‘It’s true’, \\
or with whom would they argue, ‘It’s false’? \\
There is no equal or unequal in them, \\
so who would they take on in debate? 

After\marginnote{10.1} leaving shelter to migrate unsettled, \\
a sage doesn’t get close to anyone in town. \\
Rid of sensual pleasures, expecting nothing, \\
they wouldn’t get in arguments with people. 

A\marginnote{11.1} spiritual giant would not take up for argument \\
the things in the world from which they live secluded. \\
As a prickly lotus born in the water \\
is unsullied by water and mud, \\
so the greedless sage, proponent of peace, \\
is unsmeared by sensuality and the world. 

A\marginnote{12.1} knowledge master does not become conceited \\
due to view or thought, for they do not identify with that. \\
They’ve no need for deeds or learning,\footnote{\textit{Neyyo} has the sense “needing to be led”, for example a student who “requires education” before they can master a passage (AN 4.133:1.3). Here it refers to the arahant. } \\
they’re not indoctrinated in dogmas.\footnote{\textit{\textsanskrit{Anupanīto}} (“not led in”) relates to \textit{neyyo} in the previous line. \textit{\textsanskrit{Anupanīto}} has the sense of “one who has been educated (in recitation)” at MN 93:15.2. Compare English “brought into the fold”. The sense of “dogma” for \textit{nivesana} is established at Snp 4.3:6.3 and Snp 4.5:6.3. } 

There\marginnote{13.1} are no ties for one detached from ideas; \\
there are no delusions for one freed by wisdom. \\
But those who have adopted ideas and views \\
wander the world causing conflict.” 

%
\end{verse}

%
\section*{{\suttatitleacronym Snp 4.10}{\suttatitletranslation Before the Breakup }{\suttatitleroot Purābhedasutta}}
\addcontentsline{toc}{section}{\tocacronym{Snp 4.10} \toctranslation{Before the Breakup } \tocroot{Purābhedasutta}}
\markboth{Before the Breakup }{Purābhedasutta}
\extramarks{Snp 4.10}{Snp 4.10}

\begin{verse}%
“Seeing\marginnote{1.1} how, behaving how, \\
is one said to be at peace? \\
When asked, Gotama, please tell me \\
about the ultimate person.” 

“Rid\marginnote{2.1} of craving before the breakup,” \\
\scspeaker{said the Buddha, }\\
“not dependent on the past, \\
unfathomable in the middle,\footnote{See SN 44.1, MN 72. } \\
they are not governed by anything.\footnote{Such as ignorance (AN 4.10:9.4) or craving (Dhp 342). } 

Unangry,\marginnote{3.1} unafraid, \\
not boastful or regretful, \\
thoughtful in counsel, and stable—\\
truly that sage is controlled in speech. 

Rid\marginnote{4.1} of attachment to the future, \\
they don’t grieve for the past. \\
A seer of seclusion in the midst of contacts \\
is not led astray among views. 

Withdrawn,\marginnote{5.1} free of deceit, \\
they’re not envious or stingy, \\
nor rude or disgusting, \\
or given to slander. 

Not\marginnote{6.1} swept up in pleasures, \\
or given to arrogance, \\
they’re gentle and articulate, \\
neither hungering nor growing dispassionate.\footnote{Agreeing with Norman that this is a rare instance of \textit{saddha} in the sense of wish, desire, per Sanskrit \textit{\textsanskrit{śraddhā}}, one sense of which is “appetite”. } 

They\marginnote{7.1} don’t train out of desire for profit, \\
nor get annoyed at lack of profit. \\
Not hostile due to craving, \\
nor greedy for flavors, 

they\marginnote{8.1} are equanimous, ever mindful. \\
They never conceive themselves in the world \\
as equal, special, or less than; \\
for them there is no pride. 

They\marginnote{9.1} have no dependencies, \\
understanding the teaching, they are independent. \\
No craving is found in them \\
to continue existence or to end it. 

I\marginnote{10.1} declare them to be at peace, \\
unconcerned for sensual pleasures. \\
No ties are found in them, \\
they have crossed over clinging. 

They\marginnote{11.1} have no sons or cattle, \\
nor possess fields or lands. \\
No picking up or putting down\footnote{As in Snp 4.3:8.3 this makes better sense considered as an active process. } \\
is to be found in them. 

That\marginnote{12.1} by which one might describe \\
an ordinary person or ascetics and brahmins \\
has no importance to them, \\
which is why they’re unaffected by words. 

Freed\marginnote{13.1} of greed, not stingy, \\
a sage doesn’t speak of themselves as being \\
among superiors, inferiors, or equals. \\
One not prone to creation does not return to creation. 

They\marginnote{14.1} who have nothing in the world of their own \\
do not grieve for that which is not, \\
or drift among the teachings;\footnote{Given the significance of “views” in the \textsanskrit{Aṭṭhakavagga}, and that \textit{dhamma} is commonly used in the sense of “teaching”, I read this in view of such passages as MN 47:14.3: \textit{dhammesu \textsanskrit{niṭṭhaṁ} gacchati} “comes to a conclusion about the teachings”. } \\
that’s who is said to be at peace.” 

%
\end{verse}

%
\section*{{\suttatitleacronym Snp 4.11}{\suttatitletranslation Quarrels and Disputes }{\suttatitleroot Kalahavivādasutta}}
\addcontentsline{toc}{section}{\tocacronym{Snp 4.11} \toctranslation{Quarrels and Disputes } \tocroot{Kalahavivādasutta}}
\markboth{Quarrels and Disputes }{Kalahavivādasutta}
\extramarks{Snp 4.11}{Snp 4.11}

\begin{verse}%
“Where\marginnote{1.1} do quarrels and disputes come from? \\
And lamentation and sorrow, and stinginess? \\
What of conceit and arrogance, and slander too—\\
tell me please, where do they come from?” 

“Quarrels\marginnote{2.1} and disputes come from what we hold dear, \\
as do lamentation and sorrow, stinginess, \\
conceit and arrogance. \\
Quarrels and disputes are linked to stinginess, \\
and when disputes have arisen there is slander.” 

“So\marginnote{3.1} where do what we hold dear in the world spring from? \\
And the lusts that are loose in the world? \\
Where spring the hopes and aims \\
a man has for the next life?”\footnote{\textit{\textsanskrit{Samparāya}} means “in the next life”. The Niddesa’s gloss of “refuge, shelter” etc. is not meant to change this but to qualify it: people look for safety in the next life. } 

“What\marginnote{4.1} we hold dear in the world spring from desire, \\
as do the lusts that are loose in the world. \\
From there spring the hopes and aims \\
a man has for the next life.” 

“So\marginnote{5.1} where does desire in the world spring from? \\
And judgments, too, where do they come from? \\
And anger, lies, and doubt, \\
and other things spoken of by the Ascetic?” 

“What\marginnote{6.1} they call pleasure and pain in the world—\\
based on that, desire comes about. \\
Seeing the appearance and disappearance of forms, \\
a person forms judgments in the world. 

Anger,\marginnote{7.1} lies, and doubt—\\
these things are, too, when that pair is present.\footnote{Here \textit{dvaya} obviously refers back to the pair of the previous lines and should not be overinterpreted as “duality”. } \\
One who has doubts should train in the path of knowledge; \\
it is from knowledge that the Ascetic speaks of these things.” 

“Where\marginnote{8.1} do pleasure and pain spring from? \\
When what is absent do these things not occur? \\
And also, on the topic of appearance and disappearance—\\
tell me where they spring from.” 

“Pleasure\marginnote{9.1} and pain spring from contact; \\
when contact is absent they do not occur. \\
And on the topic of appearance and disappearance—\\
I tell you they spring from there.” 

“So\marginnote{10.1} where does contact in the world spring from? \\
And possessions, too, where do they come from? \\
When what is absent is there no possessiveness? \\
When what disappears do contacts not strike?” 

“Name\marginnote{11.1} and form cause contact; \\
possessions spring from wishing; \\
when wishing is absent there is no possessiveness; \\
when form disappears, contacts don’t strike.” 

“How\marginnote{12.1} to proceed so that form disappears?\footnote{The past participle \textit{sameta} here should be read, as per Niddesa, similarly to \textit{\textsanskrit{paṭipanna}}, i.e. “engaged in the practice” rather than “completed the practice”. } \\
And how do happiness and suffering disappear? \\
Tell me how they disappear; \\
I think we ought to know these things.” 

“Without\marginnote{13.1} normal perception or distorted perception;\footnote{Following Niddesa. } \\
not lacking perception, nor perceiving what has disappeared.\footnote{From the following verses we can infer that these enigmatic lines refer to an advanced state of \textit{\textsanskrit{samādhi}}, probably the formless attainments. These are not “normal” as they have no sense-perception or defilements; they are not “distorted” as they are free of hindrances; they are not the non-percipient realm; and they do not perceive what has disappeared, namely the \textit{\textsanskrit{rūpa}} or the \textit{sukha} of lower absorptions. } \\
That’s how to proceed so that form disappears: \\
for concepts of identity due to proliferation spring from perception.”\footnote{In MN 18, we have the sequence perception, thought, proliferation, then \textit{\textsanskrit{papañcasaññāsaṅkhā}}. This suggests that \textit{\textsanskrit{papañca}} causes \textit{\textsanskrit{saṅkhā}}, as per Bodhi. } 

“Whatever\marginnote{14.1} I asked you have explained to me.\footnote{Oddly, however, the disappearance of \textit{sukha} and \textit{dukkha} is not directly answered. It is, however, implied in the surmounting of 3rd \textit{\textsanskrit{jhāna}}. } \\
I ask you once more, please tell me this: \\
Do some astute folk here say that this is the extent \\
of purification of the spirit? \\
Or do they say it is something else?” 

“Some\marginnote{15.1} astute folk do say that this is the highest extent\footnote{Such as the Buddha’s former teachers, who aimed at rebirth in formless realms. See AN 10.29:20.1. } \\
of purification of the spirit. \\
But some of them, claiming to be experts,\footnote{How to distinguish this from the Buddha’s teaching? If, following Bodhi, we read \textit{samaya} as “attainment”, we must add in an extra layer of interpretation. If we follow Norman in accepting the (more common) sense of “time, occasion”, then the difference becomes clear: they believe that the end of things happens at some point in the future (probably at death), rather than the Buddha who spoke of realizing the truth now. } \\
speak of a time when nothing remains. 

Knowing\marginnote{16.1} that these states are dependent, \\
and knowing what they depend on, the inquiring sage, \\
having understood, is freed, and does not dispute. \\
The wise do not go on into life after life.” 

%
\end{verse}

%
\section*{{\suttatitleacronym Snp 4.12}{\suttatitletranslation The Shorter Discourse on Arrayed For Battle }{\suttatitleroot Cūḷabyūhasutta}}
\addcontentsline{toc}{section}{\tocacronym{Snp 4.12} \toctranslation{The Shorter Discourse on Arrayed For Battle } \tocroot{Cūḷabyūhasutta}}
\markboth{The Shorter Discourse on Arrayed For Battle }{Cūḷabyūhasutta}
\extramarks{Snp 4.12}{Snp 4.12}

\begin{verse}%
“Each\marginnote{1.1} maintaining their own view, \\
the experts disagree, arguing: \\
‘Whoever sees it this way has understood the teaching;\footnote{Here, as often in the \textsanskrit{Aṭṭhakavagga}, \textit{\textsanskrit{jānāti}} implies different ways of knowing, which may be right or wrong (to degrees). We can’t press “know” to serve this sense, per Bodhi, Norman, and \textsanskrit{Ñāṇadīpa}. I can have a different “understanding” to you, I can “see” it differently, but I can’t “know” it differently. } \\
those who reject this are inadequate.’ 

So\marginnote{2.1} arguing, they quarrel, \\
saying, ‘The other is a fool, an amateur!’ \\
Which one of these speaks true, \\
for they all claim to be an expert?” 

“If\marginnote{3.1} not accepting another’s teaching \\
makes you a useless fool lacking wisdom,\footnote{Preferring \textit{omako} over \textit{mago}, which seems a little harsh; \textit{omako} conforms to the principle of least meaning. } \\
then they’re all fools lacking wisdom, \\
for they all maintain their own view. 

But\marginnote{4.1} if having your own view is what makes you pristine—\\
pure in wisdom, expert and intelligent—\\
then none of them lack wisdom, \\
for such is the view they have all embraced. 

I\marginnote{5.1} do not say that it is correct\footnote{I think “this is correct” refers, not to the assertion of a view, but to the assertion that the other is a fool. Niddesa takes it to mean the assertion of one of the 62 views, but this seems to be reading into the text. It might mean that the Buddha does not make dogmatic assertions; but on the face of it, the Buddha is constantly saying that his teaching is correct. It is the four noble truths, after all. } \\
when they call each other fools.\footnote{Here I take \textit{\textsanskrit{bālā}} as the quoted speech, as in the lines above and below. The commentary supplies the expected quotation indicator: \textit{‘\textsanskrit{bālo}’ti \textsanskrit{āhu}}. If we take \textit{\textsanskrit{bālā}} as subject of the verb here, per Bodhi, Norman, and \textsanskrit{Ñāṇadīpa}, then the Buddha is adopting the reductive language of those he seeks to counter, obscuring the sense of the verse. } \\
Each has built up their own view to be the truth,\footnote{It’s important to maintain the past sense of \textit{\textsanskrit{akaṁsu}} here: a view is something they have constructed and built up over time; it has solidified and become part of them, which is why it’s so hard to give up. } \\
which is why they take the other as a fool.” 

“What\marginnote{6.1} some say is true and correct, \\
others say is hollow and false. \\
So arguing, they quarrel; \\
why don’t ascetics say the same thing?” 

“The\marginnote{7.1} truth is one, there is no second; \\
wise folk would not argue about this.\footnote{With these lines the Buddha dismisses subjectivist notions of truth. } \\
But those ascetics each boast of different truths; \\
that’s why they don’t say the same thing.” 

“But\marginnote{8.1} why do they speak of different truths, \\
these proponents who claim to be experts? \\
Are there really so many different truths,\footnote{Read \textit{su \textsanskrit{tāni}}. } \\
or do they just follow their own lines of reasoning?” 

“No,\marginnote{9.1} there are not many different truths \\
that, apart from perception, are lasting in the world.\footnote{The Niddesa glosses \textit{\textsanskrit{aññatra} \textsanskrit{saññāya} \textsanskrit{niccaggāhā}}, which is translated by Bodhi as “apart from the grasping of permanence by perception”. But this seems impossible to me: \textit{\textsanskrit{saññāya}} is an ablative constructed with \textit{\textsanskrit{aññatra}}, and \textit{\textsanskrit{niccāni}} surely agrees with \textit{\textsanskrit{saccāni}}. Further, the sense of this reading is a stretch, as we haven’t dealt with impermanence at all so far. I think the allusion is, rather, to the idea of \textit{\textsanskrit{dhammaniyāmatā}}, the “fixity” or “regularity” of the Dhamma as an “eternal truth”. } \\
Having formed their reasoning regarding different views, \\
they say there are two things: true and false. 

The\marginnote{10.1} seen, heard, or thought, or precepts or vows—\footnote{This verse begins a rather striking pattern where the first couplet is essentially restated in the second couplet. } \\
based on these they show disdain. \\
Standing in judgment, they scoff, \\
saying, ‘The other is a fool, an amateur!’ 

They\marginnote{11.1} take the other as a fool on the same grounds\footnote{Again the two couplets repeat the same meaning. } \\
that they speak of themselves as an expert. \\
Claiming to be an expert on their own authority, \\
they disdain the other while saying the same thing. 

They\marginnote{12.1} are perfect, according to their own extreme view; \\
drunk on conceit, imagining themselves proficient. \\
They have anointed themselves in their own mind, \\
for such is the view they have embraced. 

If\marginnote{13.1} the word of your opponent makes you deficient, \\
then they too are lacking wisdom. \\
But if on your own authority you’re a knowledge master, a wise person,\footnote{This must refer back to the previous verse, where \textit{\textsanskrit{sayaṁ}} means “on one’s own account”. } \\
then there are no fools among the ascetics. 

‘Those\marginnote{14.1} who proclaim a teaching other than this \\
have fallen short of purity, and are inadequate’: \\
so say each one of the sectarians, \\
for they are deeply attached to their own view. 

‘Here\marginnote{15.1} alone is purity,’ they say, \\
denying that there is purification in other teachings. \\
Thus each one of the sectarians, being dogmatic, \\
speaks forcefully within the context of their own journey. 

But\marginnote{16.1} in that case, so long as they are speaking forcefully of their own journey, \\
how can they take the other as a fool? \\
They are the ones who provoke conflict \\
when they call the other a fool with an impure teaching. 

Standing\marginnote{17.1} in judgment, measuring by their own standard, \\
they keep getting into disputes with the world.\footnote{Cf. \textit{Na, bhikkhave, \textsanskrit{dhammavādī} kenaci \textsanskrit{lokasmiṁ} vivadati}. } \\
But a person who has given up all judgments \\
creates no conflict in the world.” 

%
\end{verse}

%
\section*{{\suttatitleacronym Snp 4.13}{\suttatitletranslation The Longer Discourse on Arrayed for Battle }{\suttatitleroot Mahābyūhasutta}}
\addcontentsline{toc}{section}{\tocacronym{Snp 4.13} \toctranslation{The Longer Discourse on Arrayed for Battle } \tocroot{Mahābyūhasutta}}
\markboth{The Longer Discourse on Arrayed for Battle }{Mahābyūhasutta}
\extramarks{Snp 4.13}{Snp 4.13}

\begin{verse}%
“Regarding\marginnote{1.1} those who maintain their own view, \\
arguing that, ‘This is the only truth’: \\
are all of them subject only to criticism, \\
or do some also win praise for that?” 

“That\marginnote{2.1} is a small thing, insufficient for peace, \\
these two fruits of conflict, I say. \\
Seeing this, one ought not get into arguments, \\
looking for sanctuary in the land of no conflict.\footnote{If we render \textit{khema} as “security” and \textit{\textsanskrit{bhūmi}} as “stage” then we miss the metaphor. \textit{Abhi-√pas} is found at AN 3.39:11.4 and AN 5.57:15.4 where it has a similar sense of “looking forward to”. } 

One\marginnote{3.1} who knows does not get involved \\
with any of the many different convictions. \\
Why would the uninvolved get involved, \\
since they do not believe based on the seen or the heard? 

Those\marginnote{4.1} who champion ethics speak of purity through self-control; \\
having undertaken a vow, they stick to it: \\
‘Let us train right here, then we will be pure.’ \\
Claiming to be experts, they are led on to future lives. 

If\marginnote{5.1} they fall away from their precepts and vows, \\
they tremble, having failed in their task. \\
They pray and long for purity, \\
like one who has lost their caravan while journeying far from home. 

But\marginnote{6.1} having given up all precepts and vows, \\
and these deeds blameworthy or blameless; \\
not longing for ‘purity’ or ‘impurity’, \\
live detached, fostering peace.\footnote{The Niddesa’s explanation of \textit{santi} as “view” is not plausible. I follow Norman. } 

Relying\marginnote{7.1} on mortification in disgust at sin,\footnote{Reading \textit{\textsanskrit{tapūpanissāya}}; see Bodhi’s note 1961. Since in prose \textit{\textsanskrit{tapojigucchā}}usually occur together, I treat them as such here. } \\
or else on what is seen, heard, or thought, \\
they moan that purification comes through heading upstream,\footnote{This refers to the doctrine of \textit{\textsanskrit{saṁsārasuddhi}}, that the process of rebirth leads to release. See MN 102:11.3. \textit{Anutthunati} rather consistently means “bemoans”, and here I think it is used in a disparaging way of those who, suffering under self-inflicted mortification, moan that purity is just around the corner. } \\
not rid of craving for life after life. 

For\marginnote{8.1} one who longs there are prayers, \\
and trembling too over ideas they have formed. \\
But one here for whom there is no passing away or reappearing: \\
why would they tremble? For what would they pray?” 

“The\marginnote{9.1} very same teaching that some say is ‘ultimate’, \\
others say is inferior. \\
Which of these doctrines is true, \\
for they all claim to be an expert?” 

“They\marginnote{10.1} say their own teaching is perfect, \\
while the teaching of others is inferior. \\
So arguing, they quarrel, \\
each saying their own convictions are the truth. 

If\marginnote{11.1} you became inferior because someone else disparaged you, \\
no-one in any teaching would be distinguished. \\
For each of them says the other’s teaching is lacking, \\
while forcefully advocating their own. 

But\marginnote{12.1} if they honor their own teachings\footnote{Here I follow \textsanskrit{Ñāṇadīpa}, who takes the whole verse as a conditional argument, whereas Bodhi and Norman treat each couplet separately. I think the verse is making the argument that each ascetic bases their teaching on their own experience, and so if their teachings were equally valid, then purity would be attained by many (mutually exclusive) means. } \\
just as they praise their own journeys, \\
then all doctrines would be equally valid, \\
and purity for them would be an individual matter.\footnote{Cp. Dhp 165:5, where an apparently similar idea (\textit{\textsanskrit{suddhī} asuddhi \textsanskrit{paccattaṁ}}) is Buddhist. There, the emphasis is on the personal realization of the teachings, here on different, contradictory, realizations. } 

After\marginnote{13.1} judging among the teachings, a brahmin has adopted nothing \\
that requires interpretation by another. \\
That’s why they’ve gotten over disputes, \\
for they see no other doctrine as best. 

Saying,\marginnote{14.1} ‘I know, I see, that’s how it is’,\footnote{This signifies that we are speaking of ascetics who are, in Jayatilleke’s terms, “experientialists”, relying on their meditation experiences to justify their views. The same phrase describes the Buddha at AN 4.24:9.3. } \\
some believe that purity comes from view. \\
But if they’ve really seen, what use is that view to them? \\
Overlooking what matters, they say purity comes from another.\footnote{\textit{\textsanskrit{Atisitvā}} is normally used in the sense of someone who “overlooks” the Buddha to seek answers elsewhere. In AN 3.39:4.1 it’s used in the sense of “overlooking oneself”, which \textsanskrit{Ñāṇadīpa} takes as the relevant sense here. Either reading would make sense here. } 

When\marginnote{15.1} a person sees, they see name and form, \\
and having seen, they will know just these things. \\
Gladly let them see much or little, \\
for experts say this is no way to purity. 

It’s\marginnote{16.1} not easy to educate someone who is dogmatic, \\
promoting a view they have formulated. \\
Speaking of the beauty in that which they depend on,\footnote{Here the ascetic appears to refer to their meditative attainment. Compare the use of \textit{nissita} at Snp 5.7. } \\
they talk of purity in accord with what they saw there. 

The\marginnote{17.1} brahmin does not get involved with formulating and calculating; \\
they’re not followers of views, nor kinsmen of notions.\footnote{To render \textit{\textsanskrit{ñāṇa}} here as knowledge would be a mistake. } \\
Having understood the many different convictions, \\
they look on when others grasp. 

Having\marginnote{18.1} untied the knots here in the world, \\
the sage takes no side among factions. \\
Peaceful among the peaceless, equanimous, \\
they don’t grasp when others grasp. 

Having\marginnote{19.1} given up former defilements, and not making new ones, \\
not swayed by preference, nor a proponent of dogma,\footnote{Here, as at Snp 4.3:2.2, \textit{chanda} is better read as “preference” than “desire”. } \\
that wise one is released from views, \\
not clinging to the world, nor reproaching themselves. 

They\marginnote{20.1} are remote from all things \\
seen, heard, or thought. \\
With burden put down, the sage is released: \\
not formulating, not abstaining, not longing.” 

%
\end{verse}

%
\section*{{\suttatitleacronym Snp 4.14}{\suttatitletranslation Speedy }{\suttatitleroot Tuvaṭakasutta}}
\addcontentsline{toc}{section}{\tocacronym{Snp 4.14} \toctranslation{Speedy } \tocroot{Tuvaṭakasutta}}
\markboth{Speedy }{Tuvaṭakasutta}
\extramarks{Snp 4.14}{Snp 4.14}

\begin{verse}%
“Great\marginnote{1.1} hermit, I ask you, the Kinsman of the Sun, \\
about seclusion and the state of peace. \\
How, having seen, is a mendicant quenched, \\
not grasping anything in this world?” 

“They\marginnote{2.1} would cut off the idea, ‘I am the thinker,” \\
\scspeaker{said the Buddha, }\\
“which is the root of all concepts of identity due to proliferation.\footnote{\textit{\textsanskrit{Mantā}} is explained by Niddesa (followed by Bodhi) as “wisdom” and treated as a truncated instrumental. It is elsewhere found as an agent noun in a positive sense, and is thus rendered by \textsanskrit{Ñāṇadīpa} as “deep thinker”. These renderings are certainly possible, but the context, as the next verses make clear, deal with someone who suffers conceit due to their learning and understanding. A \textit{\textsanskrit{mantā}} is one to whom others turn for advice or wise counsel. Furthermore, \textit{\textsanskrit{asmīti}} commonly follows nominatives in a similar sense, eg. \textit{\textsanskrit{seyyohamasmīti}}. Thus I take \textit{\textsanskrit{mantā} \textsanskrit{asmīti}} as a single phrase. } \\
Ever mindful, they would train to remove \\
any internal cravings. 

Regardless\marginnote{3.1} of what things they know, \\
whether internal or external, \\
they wouldn’t be proud because of that, \\
for that is not extinguishment, say the good. 

They\marginnote{4.1} wouldn’t let that make them conceited, \\
thinking themselves better or worse or alike. \\
When questioned in many ways,\footnote{Reading \textit{\textsanskrit{puṭṭho}} per variants, and contra most translators and the Niddesa. I find the renderings by Bodhi, Norman, and \textsanskrit{Ñāṇadīpa} to be borderline incomprehensible. Norman and \textsanskrit{Ñāṇadīpa} are surely incorrect in taking \textit{\textsanskrit{rūpa}} here as “forms”, but that is a tempting direction once we read \textit{\textsanskrit{phuṭṭho}}. The passage up till now has been about conceit that arises from knowledge. I think the sense here is the same as AN 9.14:11.11: \textit{\textsanskrit{Sādhu} kho \textsanskrit{tvaṁ}, samiddhi, \textsanskrit{puṭṭho} \textsanskrit{puṭṭho} vissajjesi, tena ca \textsanskrit{mā} \textsanskrit{maññī}} (“It’s good that you answered each question. But don’t get conceited because of that.”) } \\
they wouldn’t keep justifying themselves. 

A\marginnote{5.1} mendicant would find peace inside themselves, \\
and not seek peace from another. \\
For one at peace inside themselves, \\
there’s no picking up, whence putting down? 

Just\marginnote{6.1} as, in the mid-ocean deeps\footnote{The commentary explains \textit{majjhe} “in the middle” firstly as between top and bottom layers, or alternatively as in-between the mountains (i.e. land masses). While it’s possible that there was a folk belief that the center of the ocean was waveless, surely a striking feature of the ocean is how the surface movement conceals a stillness beneath. } \\
no waves arise, it stays still; \\
so too one unstirred is still—\\
a mendicant would nurse no pride at all.” 

“He\marginnote{7.1} whose eyes are open has explained \\
the truth he witnessed, where adversities are removed. \\
Please now speak of the practice, sir, \\
the monastic code and immersion in \textsanskrit{samādhi}.” 

“With\marginnote{8.1} eyes not wanton, \\
they’d turn their ears from village gossip. \\
They wouldn’t be greedy for flavors, \\
nor possessive about anything in the world. 

Though\marginnote{9.1} struck by contacts, \\
a mendicant would not lament at all. \\
They wouldn’t pray for another life, \\
nor tremble in the face of dangers. 

When\marginnote{10.1} they receive food and drink, \\
edibles and clothes, \\
they wouldn’t store them up, \\
nor worry about not getting them. 

Meditative,\marginnote{11.1} not footloose, \\
they’d avoid remorse and not be negligent. \\
Then a mendicant would stay \\
in quiet places to sit and rest. 

They\marginnote{12.1} wouldn’t take much sleep, \\
but, being keen, would apply themselves to wakefulness. \\
They’d give up sloth, illusion, mirth, and play, \\
and sex and ornamentation. 

They\marginnote{13.1} wouldn’t cast \textsanskrit{Artharvaṇa} spells, interpret dreams\footnote{This is the only mention in the Pali EBTs of the texts later known as the Arthavaveda. Niddesa explains it as casting harmful spells. } \\
or omens, or practice astrology. \\
My followers would not decipher animal cries, practice healing,\footnote{PTS Dictionary suggests this should be read as the better-established \textit{\textsanskrit{viruddhagabbhakaraṇaṁ}} (DN 1:1.26.2). If it is an error in the reading, it predates the Niddesa. Nonetheless, I translate \textit{\textsanskrit{gabbhakaraṇa}} in a way that encompasses causing both pregnancy and abortion. } \\
or cast pregnancy spells. 

Not\marginnote{14.1} shaken by criticism, \\
a mendicant would not pride themselves when praised. \\
They’d reject greed and stinginess, \\
anger, and slander. 

They’d\marginnote{15.1} not stand for buying and selling; \\
a mendicant would not speak ill at all. \\
They wouldn’t linger in the village, \\
nor cajole people from desire for profit. 

A\marginnote{16.1} mendicant would be no boaster, \\
nor would they speak suggestively. \\
They wouldn’t train in impudence, \\
nor speak argumentatively. 

They\marginnote{17.1} wouldn’t be led into lying, \\
nor be deliberately devious. \\
And they’d never look down on another \\
because of livelihood, wisdom, or precepts and vows. 

Though\marginnote{18.1} provoked from hearing much talk \\
from ascetics saying all different things,\footnote{The MS reading seems unlikely to me, these are not usually paired. More likely it’s an example of the \textsanskrit{Aṭṭhakavagga} idiom where \textit{puthu} refers to all the many different things said by ascetics. } \\
they wouldn’t react harshly, \\
for the virtuous do not retaliate. 

Having\marginnote{19.1} understood this teaching, \\
inquiring, a mendicant would always train mindfully. \\
Knowing extinguishment as peace, \\
they’d not be negligent in Gotama’s bidding. 

For\marginnote{20.1} he is the undefeated, the champion, \\
seer of the truth as witness, not by hearsay—\\
that’s why, being diligent, they would always train \\
respectfully in the Buddha’s teaching.” 

%
\end{verse}

%
\section*{{\suttatitleacronym Snp 4.15}{\suttatitletranslation Taking Up Arms }{\suttatitleroot Attadaṇḍasutta}}
\addcontentsline{toc}{section}{\tocacronym{Snp 4.15} \toctranslation{Taking Up Arms } \tocroot{Attadaṇḍasutta}}
\markboth{Taking Up Arms }{Attadaṇḍasutta}
\extramarks{Snp 4.15}{Snp 4.15}

\begin{verse}%
Peril\marginnote{1.1} stems from those who take up arms—\\
just look at people in conflict! \\
I shall extol how I came to be \\
stirred with a sense of urgency. 

I\marginnote{2.1} saw this population flounder, \\
like a fish in a little puddle. \\
Seeing them fight each other, \\
fear came upon me. 

The\marginnote{3.1} world around was hollow, \\
all directions were in turmoil. \\
Wanting a home for myself, \\
I saw nowhere unsettled.\footnote{\textit{Osita} (and \textit{\textsanskrit{osāna}} in the next line) has the root sense “lay to rest”. Originally it probably meant the place that one laid down one’s burdens at the end of the day (cf. \textit{khema}). From there we see two main applications, to “end” (cf. \textit{\textsanskrit{pariyosāna}}), or to “reside”. The fact that it follows right after the mention of finding “home” in the “world” in an unafflicted “quarter” shows that the sense “reside” applies. Nonetheless, both Norman and Bodhi translate the two occurrences as “unoccupied” and “at the end”, obscuring the fact that they are the same word in different form. Meanwhile \textsanskrit{Ñāṇadīpa} has “obstruct”. In this case the Niddesa’s explanation should be taken as a creative extension of the text. } 

But\marginnote{4.1} even in their settlement they fight—\footnote{Here, \textit{tveva} has an adversative sense, “but even then …”. This verse contrasts with the previous, a contrast lost in the translations by Norman, Bodhi, \textsanskrit{Ñāṇadīpa}, Thanissaro, Olendzki, Ireland, Mills, and Hare. Nor is the sense “settlement” for \textit{\textsanskrit{osāna}} found in DoP, CPD, or PTS Dictionary. So far as I know, the only source with the correct reading of these terms is the Digital \textsanskrit{Pāli} Dictionary. } \\
seeing that, I grew uneasy. \\
Then I saw a dart there, \\
so hard to see, stuck in the heart. 

When\marginnote{5.1} struck by that dart, \\
you run about in all directions. \\
But when that same dart has been plucked out, \\
you neither run about nor sink down. 

(On\marginnote{6.1} that topic, the trainings are recited.)\footnote{Agreeing with Norman that this line is likely a reciter’s remark. However I take \textit{tattha} as the locative of reference, else it would be \textit{ettha}. The practical guidelines that follow show how to extract the dart. } \\
Whatever attachments there are in the world, \\
don’t pursue them. \\
Having pierced through sensual pleasures in every way, \\
train yourself for quenching. 

Be\marginnote{7.1} truthful, not rude, \\
free of deceit, and rid of slander; \\
without anger, a sage would cross over \\
the evils of greed and avarice. 

Prevail\marginnote{8.1} over sleepiness, sloth, and drowsiness, \\
don’t abide in negligence, \\
A person intent on quenching \\
would not stand for arrogance. 

Don’t\marginnote{9.1} be led into lying, \\
or get caught up in fondness for form. \\
Completely understand conceit, \\
and desist from hasty conduct. 

Don’t\marginnote{10.1} relish the old, \\
or welcome the new. \\
Don’t grieve for what is running out, \\
or get attached to things that pull you in. 

Greed,\marginnote{11.1} I say, is the great flood, \\
and longing is the current—\\
the basis, the compulsion, \\
the swamp of sensuality so hard to get past. 

The\marginnote{12.1} sage never strays from the truth; \\
the brahman stands firm on the shore. \\
Having given up everything, \\
they are said to be at peace. 

They\marginnote{13.1} have truly known, they’re a knowledge master, \\
understanding the teaching, they are independent. \\
They rightly proceed in the world, \\
not coveting anything here. 

One\marginnote{14.1} who has crossed over sensuality here, \\
the snare in the world so hard to get past, \\
grieves not, nor hopes; \\
they’ve cut the strings, they’re no longer bound. 

What\marginnote{15.1} came before, let wither away, \\
and after, let there be nothing. \\
If you don’t grasp at the middle, \\
you will live at peace. 

One\marginnote{16.1} who has no sense of ownership \\
in the whole realm of name and form, \\
does not grieve for that which is not, \\
they suffer no loss in the world. 

If\marginnote{17.1} you don’t think of anything \\
as belonging to yourself or others, \\
not finding anything to be ‘mine’, \\
you won’t grieve, thinking ‘I don’t have it’. 

Not\marginnote{18.1} bitter, not fawning, \\
unstirred, everywhere even; \\
when asked about one who is unshakable, \\
I declare that that is the benefit. 

For\marginnote{19.1} the unstirred who understand, \\
there’s no performance of deeds. \\
Desisting from instigation, \\
they see sanctuary everywhere. 

A\marginnote{20.1} sage doesn’t speak of themselves as being \\
among superiors, inferiors, or equals. \\
Peaceful, rid of stinginess, \\
they neither take nor reject. 

%
\end{verse}

%
\section*{{\suttatitleacronym Snp 4.16}{\suttatitletranslation With Sāriputta }{\suttatitleroot Sāriputtasutta}}
\addcontentsline{toc}{section}{\tocacronym{Snp 4.16} \toctranslation{With Sāriputta } \tocroot{Sāriputtasutta}}
\markboth{With Sāriputta }{Sāriputtasutta}
\extramarks{Snp 4.16}{Snp 4.16}

\begin{verse}%
“Never\marginnote{1.1} before have I seen,” \\
\scspeaker{said Venerable \textsanskrit{Sāriputta}, }\\
“or heard from anyone \\
about a teacher of such graceful speech, \\
come from Tusita heaven to lead a community. 

To\marginnote{2.1} all the world with its gods \\
he appears as a seer \\
who has dispelled all darkness, \\
and alone attained to bliss. 

On\marginnote{3.1} behalf of the many here still bound, \\
I have come in need with a question \\
to that Buddha, unattached and impartial, \\
free of deceit, come to lead a community. 

Suppose\marginnote{4.1} a mendicant who loathes attachment \\
frequents a lonely lodging—\\
the root of a tree, a charnel ground, \\
on mountains, or in caves. 

In\marginnote{5.1} these many different lodgings, \\
how many dangers are there \\
at which a mendicant in their silent lodging \\
ought not tremble?\footnote{Usually \textit{nigghosa} means “message, noise”, but here it is glossed in Niddesa with the opposite meaning, \textit{appasadde appanigghose}. } 

On\marginnote{6.1} their journey to the untrodden place, \\
how many adversities are there in the world \\
that must they overcome \\
in their remote lodging? 

What\marginnote{7.1} ways of speech should they have? \\
Where should they go for alms? \\
What precepts and vows \\
should a resolute mendicant uphold? 

Having\marginnote{8.1} undertaken what training, \\
unified, alert, and mindful, \\
would they purge their own stains, \\
like a smith smelting silver?” 

“If\marginnote{9.1} one who loathes attachment frequents a lonely lodging,” \\
\scspeaker{said the Buddha to \textsanskrit{Sāriputta}, }\\
“in their search for awakening—as accords with the teaching—\\
I shall tell you, as I understand it, \\
what is comfortable for them. 

A\marginnote{10.1} wise one, a mindful mendicant living on the periphery \\
should not be afraid of five perils:\footnote{\textit{\textsanskrit{Pariyantacārī}} is read by Niddesa, and followed by Bodhi, Norman, and \textsanskrit{Ñāṇadīpa}, as “of bounded conduct”. However at DN 25:5.5 it means “living on the periphery”, which agrees with the current theme of the mendicant living remotely. } \\
flies, mosquitoes, snakes, \\
human contact, or four-legged creatures. 

Nor\marginnote{11.1} should they fear followers of other teachings, \\
even having seen the many threats they pose. \\
And then one seeking the good \\
should overcome any other adversities. 

Afflicted\marginnote{12.1} by illness and hunger, \\
they should endure cold and excessive heat. \\
Though afflicted by many such things, the homeless one \\
should exert energy, firmly striving. 

They\marginnote{13.1} must not steal or lie; \\
and should touch creatures firm or frail with love. \\
If they notice any clouding of the mind, \\
they should dispel it as \textsanskrit{Māra}’s ally. 

They\marginnote{14.1} must not fall under the sway of anger or arrogance; \\
having dug them out by the root, they would stand firm. \\
Then, withstanding likes and dislikes, \\
they would overcome. 

Putting\marginnote{15.1} wisdom in the foremost place, rejoicing in goodness, \\
they would put an end to those adversities. \\
They’d vanquish discontent in their remote lodging. \\
And they’d vanquish the four lamentations: 

‘What\marginnote{16.1} will I eat? Where will I eat? \\
Oh, I slept badly! Where will I sleep?’ \\
The trainee, the homeless migrant, \\
would dispel these lamentable thoughts. 

Receiving\marginnote{17.1} food and clothes in due season, \\
they would know moderation for the sake of contentment. \\
Guarded in these things, walking restrained in the village, \\
they wouldn’t speak harshly even when provoked. 

Eyes\marginnote{18.1} downcast, not footloose, \\
devoted to absorption, they’d be very wakeful.\footnote{\textit{\textsanskrit{Bahujāgar}’assa}. } \\
Grounded in equanimity, serene, \\
they should cut off worrisome habits of thought. 

A\marginnote{19.1} mindful one should welcome repoach, \\
breaking up hard-heartedness towards their spiritual companions. \\
They may utter skillful speech, but not for too long, \\
and they shouldn’t provoke people to blame. 

And\marginnote{20.1} there are five more taints in the world, \\
for the removal of which the mindful one should train, \\
vanquishing desire for sights, \\
sounds, flavors, smells, and touches. 

Having\marginnote{21.1} removed desire for these things,\footnote{\textit{Vineyya} here is absolutive, not optative. } \\
a mindful mendicant, their heart well freed, \\
rightly investigating the Dhamma in good time,\footnote{I think the force of \textit{\textsanskrit{kālena}} here is not, “at the right time” (when is the wrong time?) but “before it’s too late”. } \\
unified, would shatter the darkness.” 

%
\end{verse}

%
\addtocontents{toc}{\let\protect\contentsline\protect\nopagecontentsline}
\chapter*{The Chapter on the Way to the Beyond}
\addcontentsline{toc}{chapter}{\tocchapterline{The Chapter on the Way to the Beyond}}
\addtocontents{toc}{\let\protect\contentsline\protect\oldcontentsline}

%
\section*{{\suttatitleacronym Snp 5.1}{\suttatitletranslation Introductory Verses }{\suttatitleroot Vatthugāthā}}
\addcontentsline{toc}{section}{\tocacronym{Snp 5.1} \toctranslation{Introductory Verses } \tocroot{Vatthugāthā}}
\markboth{Introductory Verses }{Vatthugāthā}
\extramarks{Snp 5.1}{Snp 5.1}

\begin{verse}%
From\marginnote{1.1} the fair city of the Kosalans \\
to the southern region \\
came a brahmin expert in hymns,\footnote{\textit{\textsanskrit{Ākiñcaññaṁ}} could mean either “owning nothing” (per commentary, followed by Norman) or the “dimension of nothingness” (suggested by Bodhi). Given that just a little below we hear of him performing an expensive sacrifice; and further, that the dimension of nothingness is an important part of the conversation with of his students \textsanskrit{Upasīva} and \textsanskrit{Posāla}, the latter seems more likely. Anyway, “aspiring to own nothing” is hardly a noteworthy trait among ascetics. } \\
aspiring to nothingness. 

In\marginnote{2.1} the domain of Assaka, \\
close by \textsanskrit{Aḷaka}, \\
he lived on the bank of the \textsanskrit{Godhāvarī} River, \\
getting by on gleanings and fruit. 

He\marginnote{3.1} was supported \\
by a prosperous village nearby. \\
With the revenue earned from there \\
he performed a great sacrifice. 

When\marginnote{4.1} he had completed the great sacrifice, \\
he returned to his hermitage once more. \\
Upon his return, \\
another brahmin arrived. 

Foot-sore\marginnote{5.1} and thirsty, \\
with grotty teeth and dusty head, \\
he approached the other \\
and asked for five hundred coins. 

When\marginnote{6.1} \textsanskrit{Bāvari} saw him, \\
he invited him to sit down, \\
asked of his happiness and well-being, \\
and said the following. 

“Whatever\marginnote{7.1} I had available to give, \\
I have already distributed. \\
Believe me, brahmin, \\
I don’t have five hundred coins.” 

“If,\marginnote{8.1} good sir, you do not \\
give me what I ask, \\
then on the seventh day, \\
let your head explode in seven!” 

After\marginnote{9.1} performing a ritual, \\
that charlatan uttered his dreadful curse. \\
When he heard these words, \\
\textsanskrit{Bāvari} became distressed. 

Not\marginnote{10.1} eating, he grew emaciated, \\
stricken by the dart of sorrow. \\
And in such a state of mind, \\
he could not enjoy absorption. 

Seeing\marginnote{11.1} him anxious and distraught, \\
a goddess wishing to help,\footnote{Her gender is indicated below. } \\
approached \textsanskrit{Bāvari} \\
and said the following. 

“That\marginnote{12.1} charlatan understands nothing \\
about the head, he only wants money. \\
When it comes to heads or head-splitting, \\
he has no knowledge at all.” 

“Madam,\marginnote{13.1} surely you must know—\\
please answer my question. \\
Let me hear what you say \\
about heads and head-splitting.” 

“I\marginnote{14.1} too do not know that, \\
I have no knowledge in that matter. \\
When it comes to heads or head-splitting, \\
it is the Victors who have vision.” 

“Then,\marginnote{15.1} in all this vast territory, \\
who exactly does know\footnote{Note that, in the EBTs, \textit{\textsanskrit{pathavimaṇḍala}} primarily refers to the region ruled by a Wheel-Turning Monarch. Thus it doesn’t mean “circle” in the sense of “the circle of the earth” but rather a “sphere of influence”, i.e. “territory”. } \\
about heads and head-splitting? \\
Please tell me, goddess.” 

“From\marginnote{16.1} the city of Kapilavatthu\footnote{\textit{\textsanskrit{Purā}} can be either “from the city” (Norman), or “formerly” (commentary, followed by Bodhi and Jayawickrama). I think the poet is deliberately echoing the opening line of the text, where it must mean “from the city”. } \\
the World Leader has gone forth. \\
He is a scion of King \textsanskrit{Okkāka}, \\
a Sakyan, and a beacon. 

For\marginnote{17.1} he, brahmin, is the Awakened One! \\
He has gone beyond all things; \\
he has attained to all knowledge and power; \\
he is the seer into all things, \\
he has attained the end of all deeds; \\
he is freed with the ending of attachments. 

That\marginnote{18.1} Buddha, the Blessed One in the world, \\
the Seer, teaches Dhamma. \\
Go to him and ask—\\
he will answer you.” 

When\marginnote{19.1} he heard the word “Buddha”, \\
\textsanskrit{Bāvari} was elated. \\
His sorrow faded, \\
and he was filled to brimming with joy. 

Uplifted,\marginnote{20.1} elated, and inspired, \\
\textsanskrit{Bāvari} questioned that goddess: \\
“But in what village or town, \\
or in what land is the protector of the world, \\
where we may go and pay respects \\
to the Awakened One, best of men?” 

“Near\marginnote{21.1} \textsanskrit{Sāvatthī}, the home of the Kosalans, is the Victor\footnote{\textit{Mandira} is unusual and probably a sign of lateness. Bodhi has “realm”, Norman “city”, but the normal meaning in Sanskrit is a “dwelling place”, and in the \textsanskrit{Jātakas} it is always used in the sense of a home. } \\
abounding in wisdom, vast in intelligence. \\
That Sakyan is indefatigable, free of defilements, a bull among men:\footnote{The commentary gives the senses \textit{vigatadhuro} “free of burden” (followed by Norman) and \textit{\textsanskrit{appaṭimo}} “unrivalled” (followed by Bodhi), but at AN 3.20:2.1 it has the sense “indefatigable” and there seems no reason why it shouldn’t have the same meaning here. } \\
he understands head-splitting. 

Therefore\marginnote{22.1} he addressed his pupils, \\
brahmins who had mastered the hymns: \\
“Come, students, I shall speak. \\
Listen to what I say. 

Today\marginnote{23.1} has arisen in the world \\
one whose appearance in the world \\
is hard to find again—\\
he is renowned as the Awakened One. \\
Quickly go to \textsanskrit{Sāvatthī} \\
and see the best of men.” 

“Brahmin,\marginnote{24.1} how exactly are we to know \\
the Buddha when we see him? \\
We don’t know, please tell us, \\
so we can recognize him.” 

“The\marginnote{25.1} marks of a great man \\
have been handed down in our hymns. \\
Thirty-two have been described, \\
complete and in order. 

One\marginnote{26.1} upon whose body is found \\
these marks of a great man \\
has two possible destinies, \\
there is no third. 

If\marginnote{27.1} he stays at home, \\
having conquered this land \\
without rod or sword, \\
he shall govern by principle. 

But\marginnote{28.1} if he goes forth \\
from the lay life to homelessness, \\
he becomes an Awakened One, a perfected one, \\
with veil drawn back, supreme. 

Ask\marginnote{29.1} him about my birth, clan, and marks, \\
my hymns and students; and further, \\
about heads and head-splitting—\\
but do so only in your mind! 

If\marginnote{30.1} he is the Buddha \\
of unobstructed vision, \\
he will answer with his voice \\
the questions in your mind.” 

Sixteen\marginnote{31.1} brahmin pupils \\
heard what \textsanskrit{Bāvari} said: \\
Ajita, Tissametteyya, \\
\textsanskrit{Puṇṇaka} and \textsanskrit{Mettagū}, 

Dhotaka\marginnote{32.1} and Upasiva, \\
Nanda and then Hemaka, \\
both Todeyya and Kappa, \\
and \textsanskrit{Jatukaṇṇī} the astute, 

\textsanskrit{Bhadrāvudha}\marginnote{33.1} and Udaya, \\
and the brahmin Posala, \\
\textsanskrit{Mogharājā} the intelligent, \\
and \textsanskrit{Piṅgiya} the great hermit. 

Each\marginnote{34.1} of them had their own following, \\
they were renowned the whole world over. \\
Those wise ones, meditators who love absorption,\footnote{Bodhi omits \textit{\textsanskrit{dhīrā}}. } \\
were redolent with the potential of their past deeds.\footnote{One of the many signs of lateness in this passage. } 

Having\marginnote{35.1} bowed to \textsanskrit{Bāvari}, \\
and circled him to his right, \\
they set out for the north, \\
with their dreadlocks and hides. 

First\marginnote{36.1} to \textsanskrit{Patiṭṭhāna} of \textsanskrit{Aḷaka}, \\
then on to the city of Mahissati;\footnote{Bodhi accepts the reading “former” but, given that there seems no evidence that the name was abandoned at this time, this seems unlikely. } \\
to \textsanskrit{Ujjenī} and \textsanskrit{Gonaddhā}, \\
and Vedisa, and Vanasa. 

Then\marginnote{37.1} to Kosambi and \textsanskrit{Sāketa}, \\
and the supreme city of \textsanskrit{Sāvatthī}; \\
on they went to \textsanskrit{Setavyā} and Kapilavatthu,\footnote{Bodhi and Norman have Setavya, but DN 23 shows it is feminine. } \\
and the homestead at \textsanskrit{Kusinārā}.\footnote{See my remarks on \textit{mandira} above. \textsanskrit{Kusinārā} was famously \emph{not} a city, contra Bodhi and Norman. } 

To\marginnote{38.1} \textsanskrit{Pāvā} they went, and Bhoganagara, \\
and on to \textsanskrit{Vesālī} and the Magadhan city. \\
Finally they reached the \textsanskrit{Pāsāṇaka} shrine, \\
fair and delightful. 

Like\marginnote{39.1} a thirsty person to cool water, \\
like a merchant to great profit, \\
like a heat-struck person to shade, \\
they quickly climbed the mountain. 

At\marginnote{40.1} that time the Buddha \\
at the fore of the mendicant \textsanskrit{Saṅgha}, \\
was teaching the mendicants the Dhamma, \\
like a lion roaring in the jungle. 

Ajita\marginnote{41.1} saw the Buddha, \\
like the sun shining with a hundred rays, \\
like the moon on the fifteenth day \\
when it has come into its fullness. 

Then\marginnote{42.1} he saw his body, \\
complete in all features. \\
Thrilled, he stood to one side \\
and asked this question in his mind. 

“Speak\marginnote{43.1} about the brahmin’s birth; \\
of his clan; and his own marks; \\
what hymns is he proficient in; \\
and how many he teaches.” 

“His\marginnote{44.1} age is a hundred and twenty. \\
By clan he is a \textsanskrit{Bāvari}. \\
There are three marks on his body. \\
He is a master of the three Vedas, 

the\marginnote{45.1} teachings on the marks, the testaments, \\
the vocabularies, and the rituals. \\
He teaches five hundred, \\
and has reached proficiency in his own teaching.” 

“O\marginnote{46.1} supreme person, cutter of craving, \\
please reveal in detail \\
\textsanskrit{Bāvari}’s marks—\\
let us doubt no longer!” 

“He\marginnote{47.1} can cover his face with his tongue; \\
there is a tuft of hair between his eyebrows; \\
his private parts are concealed in a foreskin: \\
know them as this, young man.” 

Hearing\marginnote{48.1} the answers \\
without having heard any questions, \\
all the people, inspired, \\
with joined palms, wondered: 

“Who\marginnote{49.1} is it that asked a question with their mind? \\
Was it a god or \textsanskrit{Brahmā}? \\
Or Indra, \textsanskrit{Sujā}’s husband? \\
To whom does the Buddha reply?” 

“\textsanskrit{Bāvari}\marginnote{50.1} asks \\
about heads and head-splitting. \\
May the Buddha please answer, \\
and so, O hermit, dispel our doubt.” 

“Know\marginnote{51.1} ignorance as the head, \\
and knowledge as the head-splitter, \\
when joined with faith, mindfulness, and immersion, \\
and enthusiasm and energy.” 

At\marginnote{52.1} that, the brahmin student, \\
full of inspiration, \\
arranged his antelope-skin cloak over one shoulder, \\
and fell with his head to the Buddha’s feet. 

“Good\marginnote{53.1} sir, the brahmin \textsanskrit{Bāvari} \\
together with his pupils, \\
elated and happy, \\
bows to your feet, O seer!” 

“May\marginnote{54.1} the brahmin \textsanskrit{Bāvari} be happy, \\
together with his pupils. \\
And may you, too, be happy! \\
May you live long, young man. 

To\marginnote{55.1} \textsanskrit{Bāvari} and you all \\
I grant the opportunity to clear up all doubt. \\
Please ask \\
whatever you want.” 

Granted\marginnote{56.1} the opportunity by the Buddha, \\
they sat down with joined palms. \\
Ajita asked the Realized One \\
the first question right there. 

%
\end{verse}

\scendsection{The introductory verses are finished. }

%
\section*{{\suttatitleacronym Snp 5.2}{\suttatitletranslation The Questions of Ajita }{\suttatitleroot Ajitamāṇavapucchā}}
\addcontentsline{toc}{section}{\tocacronym{Snp 5.2} \toctranslation{The Questions of Ajita } \tocroot{Ajitamāṇavapucchā}}
\markboth{The Questions of Ajita }{Ajitamāṇavapucchā}
\extramarks{Snp 5.2}{Snp 5.2}

\begin{verse}%
“By\marginnote{1.1} what is the world shrouded?” \\
\scspeaker{said Venerable Ajita. }\\
“Why does it not shine? \\
Tell me, what is its tar pit? \\
What is its greatest fear?” 

“The\marginnote{2.1} world is shrouded in ignorance.” \\
\scspeaker{replied the Buddha. }\\
“Avarice and negligence make it not shine. \\
Prayer is its tar pit. \\
Suffering is its greatest fear.” 

“The\marginnote{3.1} streams flow everywhere,” \\
\scspeaker{said Venerable Ajita. }\\
“What is there to block them? \\
And tell me the restraint of streams—\\
by what are they locked out?” 

“The\marginnote{4.1} streams in the world,” \\
\scspeaker{replied the Buddha, }\\
“are blocked by mindfulness. \\
I tell you the restraint of streams—\\
they are locked out by wisdom.” 

“That\marginnote{5.1} wisdom and mindfulness,” \\
\scspeaker{said Venerable Ajita, }\\
“and that which is name and form, good sir; \\
when questioned, please tell me of this: \\
where does this all cease?” 

“This\marginnote{6.1} question which you have asked, \\
I shall answer you, Ajita. \\
Where name and form \\
cease with nothing left over—\\
with the cessation of consciousness, \\
that’s where they cease.” 

“There\marginnote{7.1} are those who have assessed the teaching, \\
and many kinds of trainees here. \\
Tell me about their behavior, good sir, \\
when asked, for you are alert.” 

“Not\marginnote{8.1} greedy for sensual pleasures, \\
their mind would be unclouded. \\
Skilled in all things, \\
a mendicant would wander mindful.” 

%
\end{verse}

%
\section*{{\suttatitleacronym Snp 5.3}{\suttatitletranslation The Questions of Tissametteyya }{\suttatitleroot Tissametteyyamāṇavapucchā}}
\addcontentsline{toc}{section}{\tocacronym{Snp 5.3} \toctranslation{The Questions of Tissametteyya } \tocroot{Tissametteyyamāṇavapucchā}}
\markboth{The Questions of Tissametteyya }{Tissametteyyamāṇavapucchā}
\extramarks{Snp 5.3}{Snp 5.3}

\begin{verse}%
“Who\marginnote{1.1} is content here in the world?” \\
\scspeaker{said Venerable Tissametteyya. }\\
“Who has no disturbances? \\
Who, having known both ends, \\
is not stuck in the middle? \\
Who do they say is a great man? \\
Who here has escaped the seamstress?” 

“Leading\marginnote{2.1} the spiritual life among sensual pleasures,” \\
\scspeaker{replied the Buddha, }\\
“rid of craving, ever mindful; \\
a mendicant who, after assessing, is quenched: \\
that’s who has no disturbances. 

That\marginnote{3.1} sage, having known both ends, \\
is not stuck in the middle. \\
He is a great man, I declare, \\
he has escaped the seamstress here.” 

%
\end{verse}

%
\section*{{\suttatitleacronym Snp 5.4}{\suttatitletranslation The Questions of Puṇṇaka }{\suttatitleroot Puṇṇakamāṇavapucchā}}
\addcontentsline{toc}{section}{\tocacronym{Snp 5.4} \toctranslation{The Questions of Puṇṇaka } \tocroot{Puṇṇakamāṇavapucchā}}
\markboth{The Questions of Puṇṇaka }{Puṇṇakamāṇavapucchā}
\extramarks{Snp 5.4}{Snp 5.4}

\begin{verse}%
“To\marginnote{1.1} the imperturbable, the seer of the root,” \\
\scspeaker{said Venerable \textsanskrit{Puṇṇaka}, }\\
“I have come in need with a question. \\
On what grounds have hermits and men, \\
aristocrats and brahmins here in the world \\
performed so many different sacrifices to the gods? \\
I ask you, Blessed One; please tell me this.” 

“Whatever\marginnote{2.1} hermits and men,” \\
\scspeaker{replied the Buddha, }\\
“aristocrats and brahmins here in the world \\
have performed so many different sacrifices to the gods: \\
all performed sacrifices bound to old age, \\
hoping for some state of existence.” 

“As\marginnote{3.1} to those hermits and men,” \\
\scspeaker{said Venerable \textsanskrit{Puṇṇaka}, }\\
“and aristocrats and brahmins here in the world \\
who have performed so many different sacrifices to the gods: \\
being diligent in the methods of sacrifice, \\
have they crossed over rebirth and old age, good sir? \\
I ask you, Blessed One; please tell me this.” 

“Hoping,\marginnote{4.1} invoking, praying, and worshiping,” \\
\scspeaker{replied the Buddha, }\\
“they pray for pleasure derived from profit. \\
Devoted to sacrifice, besotted by rebirth, \\
they’ve not crossed over rebirth and old age, I declare.” 

“If\marginnote{5.1} those devoted to sacrifice,” \\
\scspeaker{said Venerable \textsanskrit{Puṇṇaka}, }\\
“have not, by sacrificing, crossed over rebirth and old age, \\
then who exactly in the world of gods and humans \\
has crossed over rebirth and old age, good sir? \\
I ask you, Blessed One; please tell me this.” 

“Having\marginnote{6.1} assessed the world high and low,” \\
\scspeaker{replied the Buddha, }\\
“there is nothing in the world that disturbs them. \\
Peaceful, unclouded, untroubled, with no need for hope—\\
they’ve crossed over rebirth and old age, I declare.” 

%
\end{verse}

%
\section*{{\suttatitleacronym Snp 5.5}{\suttatitletranslation The Questions of Mettagū }{\suttatitleroot Mettagūmāṇavapucchā}}
\addcontentsline{toc}{section}{\tocacronym{Snp 5.5} \toctranslation{The Questions of Mettagū } \tocroot{Mettagūmāṇavapucchā}}
\markboth{The Questions of Mettagū }{Mettagūmāṇavapucchā}
\extramarks{Snp 5.5}{Snp 5.5}

\begin{verse}%
“I\marginnote{1.1} ask you, Blessed One; please tell me this,” \\
\scspeaker{said Venerable \textsanskrit{Mettagū}, }\\
“for I think you are a knowledge master, evolved. \\
Where do all these sufferings come from, \\
in all their countless forms in the world?” 

“You\marginnote{2.1} have rightly asked me of the origin of suffering,” \\
\scspeaker{replied the Buddha, }\\
“I shall tell you as I understand it. \\
Attachment is the source of suffering \\
in all its countless forms in the world. 

When\marginnote{3.1} an ignorant person builds up attachments, \\
that idiot returns to suffering again and again. \\
So let one who understands not build up attachments, \\
contemplating the birth and origin of suffering.” 

“Whatever\marginnote{4.1} I asked you have explained to me. \\
I ask you once more, please tell me this: \\
How do the wise cross the flood \\
of rebirth, old age, sorrow, and lamenting? \\
Please, sage, answer me clearly, \\
for truly you understand this matter.” 

“I\marginnote{5.1} shall extol a teaching to you,” \\
\scspeaker{replied the Buddha, }\\
“that is apparent in the present, not relying on tradition. \\
Having understood it, one who lives mindfully \\
may cross over clinging in the world.” 

“And\marginnote{6.1} I rejoice, great hermit, \\
in that supreme teaching, \\
having understood which, one who lives mindfully \\
may cross over clinging in the world.” 

“Once\marginnote{7.1} you’ve expelled relishing and dogmatism,” \\
\scspeaker{replied the Buddha, }\\
“regarding everything you are aware of—\\
above, below, all round, between—\footnote{Taking \textit{nivesana} here as “dogma”, the same sense as in the \textsanskrit{Aṭṭhakavagga}. Otherwise it might just mean “strong attachment”. } \\
having uprooted consciousness, don’t continue in existence. 

A\marginnote{8.1} mendicant who wanders meditating like this, \\
diligent and mindful, calling nothing their own, \\
would, being wise, give up the suffering \\
of rebirth, old age, sorrow and lamenting right here.” 

“I\marginnote{9.1} rejoice in the words of the great hermit! \\
You have expounded non-attachment well, Gotama. \\
Clearly the Buddha has given up suffering, \\
for truly you understand this matter. 

Surely\marginnote{10.1} those you’d regularly instruct \\
would also give up suffering. \\
Therefore, having met, I bow to you, O spiritual giant; \\
hopefully the Buddha may regularly instruct me.” 

“Any\marginnote{11.1} brahmin recognized as a knowledge master, \\
who has nothing, unattached to sensual life, \\
clearly has crossed this flood, \\
crossed to the far shore, kind, wishless. 

And\marginnote{12.1} a wise person here, a knowledge master, \\
having untied the bond to live after life, \\
free of craving, untroubled, with no need for hope, \\
has crossed over rebirth and old age, I declare.” 

%
\end{verse}

%
\section*{{\suttatitleacronym Snp 5.6}{\suttatitletranslation The Questions of Dhotaka }{\suttatitleroot Dhotakamāṇavapucchā}}
\addcontentsline{toc}{section}{\tocacronym{Snp 5.6} \toctranslation{The Questions of Dhotaka } \tocroot{Dhotakamāṇavapucchā}}
\markboth{The Questions of Dhotaka }{Dhotakamāṇavapucchā}
\extramarks{Snp 5.6}{Snp 5.6}

\begin{verse}%
“I\marginnote{1.1} ask you, Blessed One; please tell me this,” \\
\scspeaker{said Venerable Dhotaka, }\\
“I long for your voice, great hermit. \\
After hearing your message, \\
I shall train myself for quenching.” 

“Well\marginnote{2.1} then, be keen, alert,” \\
\scspeaker{replied the Buddha, }\\
“and mindful right here. \\
After hearing this message, go on \\
and train yourself for quenching.” 

“I\marginnote{3.1} see in the world of gods and humans \\
a brahmin travelling with nothing. \\
Therefore I bow to you, all-seer: \\
release me, Sakyan, from my doubts.” 

“I\marginnote{4.1} am not able to release anyone \\
in the world who has doubts, Dhotaka. \\
But when you understand the best of teachings, \\
you shall cross this flood.” 

“Teach\marginnote{5.1} me, brahmin, out of compassion, \\
the principle of seclusion so that I may understand. \\
I wish to practice right here, peaceful, independent, \\
as unimpeded as space.” 

“I\marginnote{6.1} shall extol that peace for you,” \\
\scspeaker{replied the Buddha, }\\
“that is apparent in the present, not relying on tradition. \\
Having understood it, one who lives mindfully \\
may cross over clinging in the world.” 

“And\marginnote{7.1} I rejoice, great hermit, \\
in that supreme peace, \\
having understood which, one who lives mindfully \\
may cross over clinging in the world.” 

“Once\marginnote{8.1} you have understood that everything,” \\
\scspeaker{replied the Buddha, }\\
“you are aware of in the world—\\
above, below, all round, between—\\
is a snare, don’t crave for life after life.” 

%
\end{verse}

%
\section*{{\suttatitleacronym Snp 5.7}{\suttatitletranslation The Questions of Upasiva }{\suttatitleroot Upasīvamāṇavapucchā}}
\addcontentsline{toc}{section}{\tocacronym{Snp 5.7} \toctranslation{The Questions of Upasiva } \tocroot{Upasīvamāṇavapucchā}}
\markboth{The Questions of Upasiva }{Upasīvamāṇavapucchā}
\extramarks{Snp 5.7}{Snp 5.7}

\begin{verse}%
“Alone\marginnote{1.1} and independent, O Sakyan,” \\
\scspeaker{said Venerable Upasiva, }\\
“I am not able to cross the great flood. \\
Tell me a support, All-seer, \\
depending on which I may cross this flood.” 

“Mindfully\marginnote{2.1} contemplating nothingness,” \\
\scspeaker{replied the Buddha, }\\
depending on the perception ‘there is nothing’, cross the flood.\footnote{This is an abbreviated form of the phrase \textit{natthi \textsanskrit{kiñcī}’ti} that normally marks the dimension of nothingness. } \\
Giving up sensual pleasures, refraining from chatter,\footnote{Niddesa explains \textit{\textsanskrit{kathāhi}} as either “doubts” (hence a truncated form of \textit{\textsanskrit{kathaṅkathā}}), or “talks”; the former is followed by Bodhi, the latter by Norman and \textsanskrit{Ñāṇadīpa}. While “doubts” seems contextually more likely, elsewhere \textit{(\textsanskrit{paṭi})virato} is commonly used with speech and never, to my knowledge, with doubt. } \\
watch day and night for the ending of craving.” 

“One\marginnote{3.1} who is free of all sensual desires,” \\
\scspeaker{said Venerable Upasiva, }\\
“depending on nothingness, all else left behind, \\
freed in the ultimate liberation of perception: \\
would they remain there without travelling on?” 

“One\marginnote{4.1} free of all sensual desires,” \\
\scspeaker{replied the Buddha, }\\
“depending on nothingness, all else left behind, \\
freed in the ultimate liberation of perception: \\
they would remain there without travelling on.” 

“If\marginnote{5.1} they were to remain there without travelling on, \\
even for many years, All-seer, \\
and, growing cool right there, were freed, \\
would the consciousness of such a one pass away?” 

“As\marginnote{6.1} a flame tossed by a gust of wind,” \\
\scspeaker{replied the Buddha, }\\
“comes to an end and no longer counts;\footnote{Compare the stock idiom \textit{\textsanskrit{Saṅkhampi} na upeti upanidhimpi na upeti \textsanskrit{kalabhāgampi} na upeti} at eg. SN 20.2:1.7; also Snp 1.12:3.4. } \\
so too, a sage freed from mental phenomena\footnote{\textit{\textsanskrit{Nāmakāya}} appears only here and in DN 15, apparently in the same sense. It means neither \textit{\textsanskrit{nāmarūpa}} nor \textit{\textsanskrit{manomayakāya}}, but rather the “group” of “mental phenomena”. In this case it refers to one who is freed based on a formless attainment, for whom the \textit{\textsanskrit{rūpakāya}} has been formerly abandoned. } \\
comes to an end and no longer counts.” 

“One\marginnote{7.1} who has come to an end—do they not exist? \\
Or are they eternally well? \\
Please, sage, answer me clearly, \\
for truly you understand this matter.” 

“One\marginnote{8.1} who has come to an end cannot be measured,” \\
\scspeaker{replied the Buddha. }\\
“They have nothing by which one might describe them. \\
When all things have been eradicated, \\
eradicated, too, are all ways of speech.” 

%
\end{verse}

%
\section*{{\suttatitleacronym Snp 5.8}{\suttatitletranslation The Questions of Nanda }{\suttatitleroot Nandamāṇavapucchā}}
\addcontentsline{toc}{section}{\tocacronym{Snp 5.8} \toctranslation{The Questions of Nanda } \tocroot{Nandamāṇavapucchā}}
\markboth{The Questions of Nanda }{Nandamāṇavapucchā}
\extramarks{Snp 5.8}{Snp 5.8}

\begin{verse}%
“People\marginnote{1.1} say there are sages in the world,” \\
\scspeaker{said Venerable Nanda, }\\
“but how is this the case? \\
Is someone called a sage because of their knowledge, \\
or because of their way of life?” 

“Experts\marginnote{2.1} do not speak of a sage in terms of\footnote{As at Snp 4.9:5.1, read as “instrumental of relation”. Curiously, the extraneous phrase \textit{\textsanskrit{sīlabbatena}} (not in Niddesa) appears where the reciter’s identification of the speaker should be. } \\
view, oral transmission, or notion. \\
Those who are sages live far from the crowd, I say, \\
untroubled, with no need for hope.” 

“As\marginnote{3.1} to those ascetics and brahmins,” \\
\scspeaker{said Venerable Nanda, }\\
“who speak of purity in terms of what is seen or heard, \\
or in terms of precepts and vows, \\
or in terms of countless different things. \\
Living self-controlled in that matter, \\
have they crossed over rebirth and old age, good sir? \\
I ask you, Blessed One; please tell me this.” 

“As\marginnote{4.1} to those ascetics and brahmins,” \\
\scspeaker{replied the Buddha, }\\
“who speak of purity in terms of what is seen or heard, \\
or in terms of precepts and vows, \\
or in terms of countless different things. \\
Even though they live self-controlled in that matter, \\
they’ve not crossed over rebirth and old age, I declare.” 

“As\marginnote{5.1} to those ascetics and brahmins,” \\
\scspeaker{said Venerable Nanda, }\\
“who speak of purity in terms of what is seen or heard, \\
or in terms of precepts and vows, \\
or in terms of countless different things. \\
You say they have not crossed the flood, sage. \\
Then who exactly in the world of gods and humans \\
has crossed over rebirth and old age, good sir? \\
I ask you, Blessed One; please tell me this.” 

“I\marginnote{6.1} don’t say that all ascetics and brahmins,” \\
\scspeaker{replied the Buddha, }\\
“are shrouded by rebirth and old age. \\
There are those here who have given up all \\
that is seen, heard, and thought, and precepts and vows, \\
who have given up all the countless different things. \\
Fully understanding craving, free of defilements, \\
those people, I say, have crossed the flood.” 

“I\marginnote{7.1} rejoice in the words of the great hermit! \\
You have expounded non-attachment well, Gotama. \\
There are those here who have given up all \\
that is seen, heard, and thought, and precepts and vows, \\
who have given up all the countless different things. \\
Fully understanding craving, free of defilements, \\
those people, I agree, have crossed the flood.” 

%
\end{verse}

%
\section*{{\suttatitleacronym Snp 5.9}{\suttatitletranslation The Questions of Hemaka }{\suttatitleroot Hemakamāṇavapucchā}}
\addcontentsline{toc}{section}{\tocacronym{Snp 5.9} \toctranslation{The Questions of Hemaka } \tocroot{Hemakamāṇavapucchā}}
\markboth{The Questions of Hemaka }{Hemakamāṇavapucchā}
\extramarks{Snp 5.9}{Snp 5.9}

\begin{verse}%
“Those\marginnote{1.1} who have previously answered me,” \\
\scspeaker{said Venerable Hemaka, }\\
“before I encountered Gotama’s teaching, \\
said ‘thus it was’ or ‘so it shall be’. \\
All that was just the testament of hearsay; \\
all that just fostered speculation: \\
I found no delight in that. 

But\marginnote{2.1} you, sage, explain to me \\
the teaching that destroys craving. \\
Having understood it, one who lives mindfully \\
may cross over clinging in the world.” 

“The\marginnote{3.1} removal of desire and lust, Hemaka, \\
for what is seen, heard, thought, or cognized here; \\
for anything liked or disliked, \\
is extinguishment, the imperishable state. 

Those\marginnote{4.1} who have fully understood this, mindful, \\
are extinguished in this very life. \\
Always at peace, \\
they’ve crossed over clinging to the world.” 

%
\end{verse}

%
\section*{{\suttatitleacronym Snp 5.10}{\suttatitletranslation The Questions of Todeyya }{\suttatitleroot Todeyyamāṇavapucchā}}
\addcontentsline{toc}{section}{\tocacronym{Snp 5.10} \toctranslation{The Questions of Todeyya } \tocroot{Todeyyamāṇavapucchā}}
\markboth{The Questions of Todeyya }{Todeyyamāṇavapucchā}
\extramarks{Snp 5.10}{Snp 5.10}

\begin{verse}%
“In\marginnote{1.1} whom sensual pleasures do not dwell,” \\
\scspeaker{said Venerable Todeyya, }\\
“and for whom there is no craving, \\
and who has crossed over doubts—\\
of what kind is their liberation?” 

“In\marginnote{2.1} whom sensual pleasures do not dwell,” \\
\scspeaker{replied the Buddha, }\\
“and for whom there is no craving, \\
and who has crossed over doubts—\\
their liberation is none other than this.” 

“Are\marginnote{3.1} they free of hope, or are they still in need of hope? \\
Do they possess wisdom, or are they still forming wisdom?\footnote{Bodhi reads \textit{\textsanskrit{kappī}} (against the Niddesa) in the sense “manner”. But it seems unlikely Todeyya, or anyone really, would ask whether a sage merely \emph{seemed} wise. Surely the question must be whether a sage is someone who is still in the process of learning, a question that has been important to the Buddhist traditions and still is today. } \\
O Sakyan, elucidate the sage to me, \\
so that I may understand, All-seer.” 

“They\marginnote{4.1} are free of hope, they are not in need of hope. \\
They possess wisdom, they are not still forming wisdom. \\
That, Todeyya, is how to understand a sage, \\
one who has nothing, unattached to sensual life.” 

%
\end{verse}

%
\section*{{\suttatitleacronym Snp 5.11}{\suttatitletranslation The Questions of Kappa }{\suttatitleroot Kappamāṇavapucchā}}
\addcontentsline{toc}{section}{\tocacronym{Snp 5.11} \toctranslation{The Questions of Kappa } \tocroot{Kappamāṇavapucchā}}
\markboth{The Questions of Kappa }{Kappamāṇavapucchā}
\extramarks{Snp 5.11}{Snp 5.11}

\begin{verse}%
“For\marginnote{1.1} those overwhelmed by old age and death,” \\
\scspeaker{said Venerable Kappa, }\\
“stuck mid-stream \\
as the terrifying flood arises, \\
tell me an island, good sir. \\
Explain to me an island \\
so that this may not occur again.” 

“For\marginnote{2.1} those overwhelmed by old age and death,” \\
\scspeaker{replied the Buddha, }\\
“stuck mid-stream \\
as the terrifying flood arises, \\
I shall tell you an island, Kappa. 

Having\marginnote{3.1} nothing, taking nothing: \\
this is the isle of no return. \\
I call it extinguishment, \\
the ending of old age and death. 

Those\marginnote{4.1} who have fully understood this, mindful, \\
are extinguished in this very life. \\
They don’t fall under \textsanskrit{Māra}’s sway, \\
nor are they his lackeys.” 

%
\end{verse}

%
\section*{{\suttatitleacronym Snp 5.12}{\suttatitletranslation The Questions of Jatukaṇṇī }{\suttatitleroot Jatukaṇṇimāṇavapucchā}}
\addcontentsline{toc}{section}{\tocacronym{Snp 5.12} \toctranslation{The Questions of Jatukaṇṇī } \tocroot{Jatukaṇṇimāṇavapucchā}}
\markboth{The Questions of Jatukaṇṇī }{Jatukaṇṇimāṇavapucchā}
\extramarks{Snp 5.12}{Snp 5.12}

\begin{verse}%
“Hearing\marginnote{1.1} of the hero with no desire for sensual pleasures,” \\
\scspeaker{said Venerable \textsanskrit{Jatukaṇṇī}, }\\
“who has passed over the flood, I’ve come with a question for that desireless one. \\
Tell me the state of peace, O natural visionary.\footnote{While the exact sense of the unique term \textit{sahajanetta} is open to interpretation, surely Norman’s “omniscient one” (following Niddesa) is not right. } \\
Tell me this, Blessed One, as it really is. 

For,\marginnote{2.1} having mastered sensual desires, the Blessed One proceeds,\footnote{The following line lacks a verb, and Norman, Bodhi, and \textsanskrit{Ñāṇadīpa} all construe it with \textit{abhibhuyya}: the sun “overcomes” the earth. But that doesn’t make sense. Rather, the idiom \textit{abhibhuyya iriyati} leans on \textit{irayati} in the sense “to keep going, to proceed”. The Buddha “proceeds” (i.e. keeps living) after mastering desires, like the sun “proceeds” shedding light on the earth. } \\
as the blazing sun shines on the earth. \\
May you of vast wisdom explain the teaching \\
to me of little wisdom so that I may understand \\
the giving up of rebirth and old age here.” 

“With\marginnote{3.1} sensual desire dispelled,”\footnote{Niddesa, followed by Bodhi, Norman, and \textsanskrit{Ñāṇadīpa}, takes \textit{vinaya} as optative. But where the same phrase occurs at Kp 9:10.3 and Snp 1.8:10.3 it is clearly absolutive. Moreover, the next includes the absolutive \textit{\textsanskrit{daṭṭhu}}, which elsewhere follows a similar line with an absolutive verb: \textit{\textsanskrit{kāmesvādīnavaṁ} \textsanskrit{disvā}}. } \\
\scspeaker{replied the Buddha, }\\
“seeing renunciation as sanctuary, \\
don’t be taking up or putting down\footnote{As in the \textsanskrit{Aṭṭhakavagga}, it is essential that this phrase be rendered in an active present tense. It is not that nothing \emph{has ever} been taken up or put down, but that they are no longer actively engaging in taking up and putting down. } \\
anything at all. 

What\marginnote{4.1} came before, let wither away, \\
and after, let there be nothing. \\
If you don’t grasp at the middle, \\
you will live at peace. 

One\marginnote{5.1} rid of greed, brahmin, \\
for the whole realm of name and form, \\
has no defilements by which \\
they might fall under the sway of Death.” 

%
\end{verse}

%
\section*{{\suttatitleacronym Snp 5.13}{\suttatitletranslation The Questions of Bhadrāvudha }{\suttatitleroot Bhadrāvudhamāṇavapucchā}}
\addcontentsline{toc}{section}{\tocacronym{Snp 5.13} \toctranslation{The Questions of Bhadrāvudha } \tocroot{Bhadrāvudhamāṇavapucchā}}
\markboth{The Questions of Bhadrāvudha }{Bhadrāvudhamāṇavapucchā}
\extramarks{Snp 5.13}{Snp 5.13}

\begin{verse}%
“I\marginnote{1.1} have a request for you, the shelter-leaver, the craving-cutter, the imperturbable,” \\
\scspeaker{said Venerable \textsanskrit{Bhadrāvudha}, }\\
“the delight-leaver, the flood-crosser, the freed, \\
the formulation-leaver, the intelligent. \\
Many people have gathered from different lands 

wishing\marginnote{2.1} to hear your word, O hero. \\
After hearing the spiritual giant they will depart from here. \\
Please, sage, answer them clearly, \\
for truly you understand this matter.” 

“Dispel\marginnote{3.1} all craving for attachments,” \\
\scspeaker{replied the Buddha, }\\
“above, below, all round, between. \\
For whatever a person grasps in the world, \\
\textsanskrit{Māra} pursues them right there. 

So\marginnote{4.1} let a mindful mendicant who understands \\
not grasp anything in all the world, \\
observing that, in clinging to attachments,\footnote{Norman’s suggestion to read \textit{\textsanskrit{ādānasatte}} (against Niddesa) as locative singular would be tempting were it not that at Snp 5.13:4.3 and Thag 19.1:20.3 \textit{iti \textsanskrit{pekkhamāno}} qualifies the former part of the line. } \\
these people cling to the domain of death.” 

%
\end{verse}

%
\section*{{\suttatitleacronym Snp 5.14}{\suttatitletranslation The Questions of Udaya }{\suttatitleroot Udayamāṇavapucchā}}
\addcontentsline{toc}{section}{\tocacronym{Snp 5.14} \toctranslation{The Questions of Udaya } \tocroot{Udayamāṇavapucchā}}
\markboth{The Questions of Udaya }{Udayamāṇavapucchā}
\extramarks{Snp 5.14}{Snp 5.14}

\begin{verse}%
“To\marginnote{1.1} the meditator, rid of hopes,” \\
\scspeaker{said Venerable Udaya, }\\
“who has completed the task, is free of defilements, \\
and has gone beyond all things, \\
I have come in need with a question. \\
Tell me the liberation by enlightenment, \\
the smashing of ignorance.” 

“The\marginnote{2.1} giving up of sensual desires,” \\
\scspeaker{replied the Buddha, }\\
“and aversions, both; \\
the dispelling of dullness, \\
and the cessation of remorse. 

Pure\marginnote{3.1} equanimity and mindfulness, \\
preceded by investigation of principles—\\
this, I declare, is liberation by enlightenment, \\
the smashing of ignorance.” 

“What\marginnote{4.1} fetters the world? \\
What explores it? \\
With the giving up of what \\
is extinguishment spoken of?” 

“Delight\marginnote{5.1} fetters the world. \\
Thought explores it. \\
With the giving up of craving \\
extinguishment is spoken of.” 

“For\marginnote{6.1} one living mindfully, \\
how does consciousness cease? \\
We’ve come to ask the Buddha; \\
let us hear what you say.” 

“Not\marginnote{7.1} taking pleasure in feeling \\
internally and externally—\\
for one living mindfully, \\
that’s how consciousness ceases.” 

%
\end{verse}

%
\section*{{\suttatitleacronym Snp 5.15}{\suttatitletranslation The Question of Posala }{\suttatitleroot Posālamāṇavapucchā}}
\addcontentsline{toc}{section}{\tocacronym{Snp 5.15} \toctranslation{The Question of Posala } \tocroot{Posālamāṇavapucchā}}
\markboth{The Question of Posala }{Posālamāṇavapucchā}
\extramarks{Snp 5.15}{Snp 5.15}

\begin{verse}%
“To\marginnote{1.1} the one who reveals the past,”\footnote{This is an unusual idiom. } \\
\scspeaker{said Venerable Posala, }\\
who is imperturbable, with doubts cut off, \\
and who has gone beyond all things, \\
I have come in need with a question. 

Consider\marginnote{2.1} one who perceives the disappearance of form,\footnote{See Snp 4.11. The question refers to a meditator who has attained the dimension of nothingness. } \\
who has entirely given up the body, \\
and who sees nothing at all \\
internally and externally. \\
I ask the Sakyan about knowledge for them;\footnote{The Niddesa explains, in line with the Buddha’s answer, that the question is about the kind of knowledge needed to guide one with such an attainment. Thus it must be dative, not genitive. } \\
how should one like that be guided?” 

“The\marginnote{3.1} Realized One directly knows,” \\
\scspeaker{said the Buddha, }\\
“all the planes of consciousness. \\
And he knows this one who remains, \\
committed to that as their final goal.\footnote{Read \textit{’\textsanskrit{dhimuttaṁ}} per Niddesa, see discussion in Bodhi. The sense is the same as eg. an4.125. This refers to someone, such as \textsanskrit{Āḷāra} \textsanskrit{Kālāma}, whose spiritual goal is rebirth in the dimension of nothingness. } 

Understanding\marginnote{4.1} that desire for rebirth \\
in the dimension of nothingness is a fetter, \\
and directly knowing what this really means, \\
one then sees that matter clearly. \\
That is the knowledge of reality for them, \\
the brahmin who has lived the life.” 

%
\end{verse}

%
\section*{{\suttatitleacronym Snp 5.16}{\suttatitletranslation The Questions of Mogharājā }{\suttatitleroot Mogharājamāṇavapucchā}}
\addcontentsline{toc}{section}{\tocacronym{Snp 5.16} \toctranslation{The Questions of Mogharājā } \tocroot{Mogharājamāṇavapucchā}}
\markboth{The Questions of Mogharājā }{Mogharājamāṇavapucchā}
\extramarks{Snp 5.16}{Snp 5.16}

\begin{verse}%
“Twice\marginnote{1.1} I have asked the Sakyan,” \\
\scspeaker{said Venerable \textsanskrit{Mogharājā}, }\\
“but you haven’t answered me, O Seer. \\
I have heard that the divine hermit \\
answers when questioned a third time. 

Regarding\marginnote{2.1} this world, the other world, \\
and the realm of \textsanskrit{Brahmā} with its gods, \\
I’m not familiar with the view \\
of the renowned Gotama. 

So\marginnote{3.1} I’ve come in need with a question \\
to the one of excellent vision. \\
How to look upon the world \\
so the King of Death won’t see you?” 

“Look\marginnote{4.1} upon the world as empty, \\
\textsanskrit{Mogharājā}, ever mindful. \\
Having uprooted the view of self, \\
you may thus cross over death. \\
That’s how to look upon the world \\
so the King of Death won’t see you.” 

%
\end{verse}

%
\section*{{\suttatitleacronym Snp 5.17}{\suttatitletranslation The Questions of Piṅgiya }{\suttatitleroot Piṅgiyamāṇavapucchā}}
\addcontentsline{toc}{section}{\tocacronym{Snp 5.17} \toctranslation{The Questions of Piṅgiya } \tocroot{Piṅgiyamāṇavapucchā}}
\markboth{The Questions of Piṅgiya }{Piṅgiyamāṇavapucchā}
\extramarks{Snp 5.17}{Snp 5.17}

\begin{verse}%
“I\marginnote{1.1} am old, feeble, and pallid,” \\
\scspeaker{said Venerable \textsanskrit{Piṅgiya}, }\\
“my eyes unclear, my hearing faint. \\
Don’t let stupid me perish meanwhile; \\
explain the teaching so that I may understand \\
the giving up of rebirth and old age here.” 

“Having\marginnote{2.1} seen those stricken by forms,” \\
\scspeaker{replied the Buddha, }\\
“negligent people afflicted by forms; \\
therefore, \textsanskrit{Piṅgiya}, being diligent, \\
give up form so as not to be reborn.” 

“The\marginnote{3.1} four quarters, the intermediate directions, \\
below, and above: in these ten directions \\
there’s nothing at all in the world \\
that you’ve not seen, heard, thought, or cognized. \\
Explain the teaching so that I may understand \\
the giving up of rebirth and old age here.” 

“Observing\marginnote{4.1} people sunk in craving,” \\
\scspeaker{replied the Buddha, }\\
“tormented, mired in old age; \\
therefore, \textsanskrit{Piṅgiya}, being diligent, \\
give up craving so as not to be reborn.” 

%
\end{verse}

%
\section*{{\suttatitleacronym Snp 5.18}{\suttatitletranslation Homage to the Way to the Beyond }{\suttatitleroot Pārāyanatthutigāthā}}
\addcontentsline{toc}{section}{\tocacronym{Snp 5.18} \toctranslation{Homage to the Way to the Beyond } \tocroot{Pārāyanatthutigāthā}}
\markboth{Homage to the Way to the Beyond }{Pārāyanatthutigāthā}
\extramarks{Snp 5.18}{Snp 5.18}

This\marginnote{1.1} was said by the Buddha while staying in the land of the Magadhans at the \textsanskrit{Pāsāṇake} shrine. When requested by the sixteen brahmin devotees, he answered their questions one by one. If you understand the meaning and the teaching of each of these questions, and practice accordingly, you may go right to the far shore of old age and death. These teachings are said to lead to the far shore, which is why the name of this exposition of the teaching is “The Way to the Beyond”. 

\begin{verse}%
Ajita,\marginnote{2.1} Tissametteyya, \\
\textsanskrit{Puṇṇaka} and \textsanskrit{Mettagū}, \\
Dhotaka and Upasiva, \\
Nanda and then Hemaka, 

both\marginnote{3.1} Todeyya and Kappa, \\
and \textsanskrit{Jatukaṇṇī} the astute, \\
\textsanskrit{Bhadrāvudha} and Udaya, \\
and the brahmin Posala, \\
\textsanskrit{Mogharājā} the intelligent, \\
and \textsanskrit{Piṅgiya} the great hermit: 

they\marginnote{4.1} approached the Buddha, \\
the hermit of consummate conduct. \\
Asking their subtle questions, \\
they came to the most excellent Buddha. 

The\marginnote{5.1} Buddha answered their questions \\
in accordance with truth. \\
The sage satisfied the brahmins \\
with his answers to their questions. 

Those\marginnote{6.1} who were satisfied by the all-seer, \\
the Buddha, Kinsman of the Sun, \\
led the spiritual life in his presence, \\
the one of such splendid wisdom. 

If\marginnote{7.1} you practice in accordance \\
with each of these questions \\
as taught by the Buddha, \\
you’ll go from the near shore to the far. 

Developing\marginnote{8.1} the supreme path, \\
you’ll go from the near shore to the far. \\
This path is for going to the far shore; \\
that’s why it’s called “The Way to the Beyond”. 

%
\end{verse}

%
\section*{{\suttatitleacronym Snp 5.19}{\suttatitletranslation Preserving the Way to the Beyond }{\suttatitleroot Pārāyanānugītigāthā}}
\addcontentsline{toc}{section}{\tocacronym{Snp 5.19} \toctranslation{Preserving the Way to the Beyond } \tocroot{Pārāyanānugītigāthā}}
\markboth{Preserving the Way to the Beyond }{Pārāyanānugītigāthā}
\extramarks{Snp 5.19}{Snp 5.19}

\begin{verse}%
“I\marginnote{1.1} shall keep reciting the Way to the Beyond,”\footnote{To \textit{\textsanskrit{anugāyati}} is to keep on reciting, as the brahmins did with their hymns descended from seers of old (as eg. AN 5.192:3.1). Here, \textsanskrit{Piṅgiya}, as an experienced Vedic reciter himself, announces that he will keep the this teaching alive in just the same way. This is perhaps the most explicit evidence for a direct historical link between the oral recitation methods of the brahmins and the Buddhists. } \\
\scspeaker{said Venerable \textsanskrit{Piṅgiya}, }\\
“which was taught as it was seen \\
by the immaculate one of vast intelligence. \\
He is desireless, unentangled, a spiritual giant: \\
why would he speak falsely? 

Come,\marginnote{2.1} let me extol \\
in sweet words of praise \\
the one who’s given up stains and delusions, \\
conceit and contempt. 

The\marginnote{3.1} Buddha, all-seer, dispeler of darkness, \\
has gone to world’s end, beyond all rebirths; \\
he is free of defilements, and has given up all pain, \\
the rightly-named one, brahmin, is revered by me. 

Like\marginnote{4.1} a bird that flees a little copse, \\
to roost in a forest abounding in fruit, \\
I’ve left the near-sighted behind, \\
like a swan come to a great river. 

Those\marginnote{5.1} who explained to me previously, \\
before I encountered Gotama’s teaching, \\
said ‘thus it was’ or ‘so it shall be’. \\
All that was just the testament of hearsay; \\
all that just fostered speculation. 

Alone,\marginnote{6.1} the dispeler of darkness \\
is splendid, a beacon: \\
Gotama, vast in wisdom, \\
Gotama, vast in intelligence. 

He\marginnote{7.1} is the one who taught me Dhamma, \\
visible in this very life, immediately effective, \\
the untroubled, the end of craving, \\
to which there is no compare.” 

“Why\marginnote{8.1} would you dwell apart from him \\
even for an hour, \textsanskrit{Piṅgiya}? \\
From Gotama, vast in wisdom, \\
from Gotama, vast in intelligence? 

He\marginnote{9.1} is the one who taught you Dhamma, \\
visible in this very life, immediately effective, \\
the untroubled, the end of craving, \\
to which there is no compare.” 

“I\marginnote{10.1} never dwell apart from him, \\
not even for an hour, brahmin. \\
From Gotama, vast in wisdom, \\
from Gotama, vast in intelligence. 

He\marginnote{11.1} is the one who taught me Dhamma, \\
visible in this very life, immediately effective, \\
the untroubled, the end of craving, \\
to which there is no compare. 

Being\marginnote{12.1} diligent, I see him \\
in my mind’s eye day and night. \\
I spend the night in homage to him, \\
hence I think I dwell with him. 

My\marginnote{13.1} faith and joy and intent and mindfulness \\
never stray from Gotama’s teaching. \\
I bow to whatever direction \\
the one of vast wisdom heads. 

I’m\marginnote{14.1} old and feeble, \\
so my body cannot go there, \\
but I always travel in my thoughts, \\
for my mind, brahmin, is bound to him. 

Lying\marginnote{15.1} floundering in the mud, \\
I drifted from island to island. \\
Then I saw the Buddha, \\
the undefiled one who has crossed the flood.” 

“Just\marginnote{16.1} as Vakkali was committed to faith—\\
\textsanskrit{Bhadrāvudha} and Gotama of \textsanskrit{Āḷavī} too—\\
so too you should commit to faith. \\
You will go, \textsanskrit{Piṅgiya}, beyond the domain of death.” 

“My\marginnote{17.1} confidence grew \\
when I heard the word of the sage, \\
the Buddha with veil drawn back, \\
so kind and eloquent. 

Having\marginnote{18.1} directly known all about the gods, \\
he understands all top to bottom, \\
the teacher who settles all questions \\
for those who admit their doubts. 

Unfaltering,\marginnote{19.1} unshakable; \\
that to which there is no compare. \\
For sure I will go there, I have no doubt of that. \\
Remember me as one whose mind is made up.” 

%
\end{verse}

\scendbook{The Anthology of Discourses is completed. }

%
\backmatter%
%
\chapter*{Colophon}
\addcontentsline{toc}{chapter}{Colophon}
\markboth{Colophon}{Colophon}

\section*{The Translator}

Bhikkhu Sujato was born as Anthony Aidan Best on 4/11/1966 in Perth, Western Australia. He grew up in the pleasant suburbs of Mt Lawley and Attadale alongside his sister Nicola, who was the good child. His mother, Margaret Lorraine Huntsman née Pinder, said “he’ll either be a priest or a poet”, while his father, Anthony Thomas Best, advised him to “never do anything for money”. He attended Aquinas College, a Catholic school, where he decided to become an atheist. At the University of WA he studied philosophy, aiming to learn what he wanted to do with his life. Finding that what he wanted to do was play guitar, he dropped out. His main band was named Martha’s Vineyard, which achieved modest success in the indie circuit. Then it broke up, because everyone thought they personally were reason for the success, which, oddly enough, turns out not to have been the case. 

A seemingly random encounter with a roadside joey took him to Thailand, where he entered his first meditation retreat at Wat Ram Poeng, Chieng Mai in 1992. He decided to devote himself to the Buddha’s path, and took full ordination in Wat Pa Nanachat in 1994, where his teachers were Ajahn Pasanno and Ajahn Jayasaro. In 1997 he returned to Perth to study with Ajahn Brahm at Bodhinyana Monastery. 

He spent several years practicing in seclusion in Malaysia and Thailand before establishing Santi Forest Monastery in Bundanoon, NSW, in 2003. There he was instrumental in supporting the establishment of the Theravada bhikkhuni order in Australia and advocating for women’s rights. He continues to teach in Australia and globally, with a special concern for the moral implications of climate change and other forms of environmental destruction. He has published a series of books of original and groundbreaking research on early Buddhism. 

In 2005 he founded SuttaCentral together with Rod Bucknell and John Kelly. In 2015, seeing the need for a complete, accurate, plain English translation of the Pali texts, he undertook the task, spending nearly three years in isolation on the isle of Qi Mei off the coast of the nation of Taiwan. He completed the four main \textsanskrit{Nikāyas} in 2018, and the early books of the Khuddaka \textsanskrit{Nikāya} were complete by 2021. All this work is dedicated to the public domain and is entirely free of copyright encumbrance. 

In 2019 he returned to Sydney where, together with Bhikkhu Akaliko, he established Lokanta Vihara (The Monastery at the End of the World). 

\section*{Creation Process}

Translated from the Pali. Primary source was the \textsanskrit{Mahāsaṅgīti} edition, with reference to several English translations, especially those of K.R. Norman, Bhikkhu Bodhi, and Bhikkhu \textsanskrit{Ñāṇadīpa}.

\section*{The Translation}

This translation aims to retain the directness and urgency of the \textsanskrit{Suttanipāta}. The \textsanskrit{Suttanipāta} includes verses from the earliest and latest periods within the period encompassed by the early texts, and so it covers a challenging variety of styles and themes, by turns fierce, devotional, or incisive. In several portions, most notably the \textsanskrit{Aṭṭhakavagga}, there are a range of highly specific usages that demand careful attention.

\section*{About SuttaCentral}

SuttaCentral publishes early Buddhist texts. Since 2005 we have provided root texts in Pali, Chinese, Sanskrit, Tibetan, and other languages, parallels between these texts, and translations in many modern languages. We build on the work of generations of scholars, and offer our contribution freely.

SuttaCentral is driven by volunteer contributions, and in addition we employ professional developers. We offer a sponsorship program for high quality translations from the original languages. Financial support for SuttaCentral is handled by the SuttaCentral Development Trust, a charitable trust registered in Australia.

\section*{About Bilara}

“Bilara” means “cat” in Pali, and it is the name of our Computer Assisted Translation (CAT) software. Bilara is a web app that enables translators to translate early Buddhist texts into their own language. These translations are published on SuttaCentral with the root text and translation side by side.

\section*{About SuttaCentral Editions}

The SuttaCentral Editions project makes high quality books from selected Bilara translations. These are published in formats including HTML, EPUB, PDF, and print.

If you want to print any of our Editions, please let us know and we will help prepare a file to your specifications.

%
\end{document}