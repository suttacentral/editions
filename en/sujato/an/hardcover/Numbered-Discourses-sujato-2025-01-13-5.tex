\documentclass[12pt,openany]{book}%
\usepackage{lastpage}%
%
\usepackage{ragged2e}
\usepackage{verse}
\usepackage[a-3u]{pdfx}
\usepackage[inner=1in, outer=1in, top=.7in, bottom=1in, papersize={6in,9in}, headheight=13pt]{geometry}
\usepackage{polyglossia}
\usepackage[12pt]{moresize}
\usepackage{soul}%
\usepackage{microtype}
\usepackage{tocbasic}
\usepackage{realscripts}
\usepackage{epigraph}%
\usepackage{setspace}%
\usepackage{sectsty}
\usepackage{fontspec}
\usepackage{marginnote}
\usepackage[bottom]{footmisc}
\usepackage{enumitem}
\usepackage{fancyhdr}
\usepackage{emptypage}
\usepackage{extramarks}
\usepackage{graphicx}
\usepackage{relsize}
\usepackage{etoolbox}

% improve ragged right headings by suppressing hyphenation and orphans. spaceskip plus and minus adjust interword spacing; increase rightskip stretch to make it want to push a word on the first line(s) to the next line; reduce parfillskip stretch to make line length more equal . spacefillskip and xspacefillskip can be deleted to use defaults.
\protected\def\BalancedRagged{
\leftskip     0pt
\rightskip    0pt plus 10em
\spaceskip=1\fontdimen2\font plus .5\fontdimen3\font minus 1.5\fontdimen4\font
\xspaceskip=1\fontdimen2\font plus 1\fontdimen3\font minus 1\fontdimen4\font
\parfillskip  0pt plus 15em
\relax
}

\hypersetup{
colorlinks=true,
urlcolor=black,
linkcolor=black,
citecolor=black,
allcolors=black
}

% use a small amount of tracking on small caps
\SetTracking[ spacing = {25*,166, } ]{ encoding = *, shape = sc }{ 25 }

% add a blank page
\newcommand{\blankpage}{
\newpage
\thispagestyle{empty}
\mbox{}
\newpage
}

% define languages
\setdefaultlanguage[]{english}
\setotherlanguage[script=Latin]{sanskrit}

%\usepackage{pagegrid}
%\pagegridsetup{top-left, step=.25in}

% define fonts
% use if arno sanskrit is unavailable
%\setmainfont{Gentium Plus}
%\newfontfamily\Marginalfont[]{Gentium Plus}
%\newfontfamily\Allsmallcapsfont[RawFeature=+c2sc]{Gentium Plus}
%\newfontfamily\Noligaturefont[Renderer=Basic]{Gentium Plus}
%\newfontfamily\Noligaturecaptionfont[Renderer=Basic]{Gentium Plus}
%\newfontfamily\Fleuronfont[Ornament=1]{Gentium Plus}

% use if arno sanskrit is available. display is applied to \chapter and \part, subhead to \section and \subsection.
\setmainfont[
  FontFace={sb}{n}{Font = {Arno Pro Semibold}},
  FontFace={sb}{it}{Font = {Arno  Pro Semibold Italic}}
]{Arno Pro}

% create commands for using semibold
\DeclareRobustCommand{\sbseries}{\fontseries{sb}\selectfont}
\DeclareTextFontCommand{\textsb}{\sbseries}

\newfontfamily\Marginalfont[RawFeature=+subs]{Arno Pro Regular}
\newfontfamily\Allsmallcapsfont[RawFeature=+c2sc]{Arno Pro}
\newfontfamily\Noligaturefont[Renderer=Basic]{Arno Pro}
\newfontfamily\Noligaturecaptionfont[Renderer=Basic]{Arno Pro Caption}

% chinese fonts
\newfontfamily\cjk{Noto Serif TC}
\newcommand*{\langlzh}[1]{\cjk{#1}\normalfont}%

% logo
\newfontfamily\Logofont{sclogo.ttf}
\newcommand*{\sclogo}[1]{\large\Logofont{#1}}

% use subscript numerals for margin notes
\renewcommand*{\marginfont}{\Marginalfont}

% ensure margin notes have consistent vertical alignment
\renewcommand*{\marginnotevadjust}{-.17em}

% use compact lists
\setitemize{noitemsep,leftmargin=1em}
\setenumerate{noitemsep,leftmargin=1em}
\setdescription{noitemsep, style=unboxed, leftmargin=1em}

% style ToC
\DeclareTOCStyleEntries[
  raggedentrytext,
  linefill=\hfill,
  pagenumberwidth=.5in,
  pagenumberformat=\normalfont,
  entryformat=\normalfont
]{tocline}{chapter,section}


  \setlength\topsep{0pt}%
  \setlength\parskip{0pt}%

% define new \centerpars command for use in ToC. This ensures centering, proper wrapping, and no page break after
\def\startcenter{%
  \par
  \begingroup
  \leftskip=0pt plus 1fil
  \rightskip=\leftskip
  \parindent=0pt
  \parfillskip=0pt
}
\def\stopcenter{%
  \par
  \endgroup
}
\long\def\centerpars#1{\startcenter#1\stopcenter}

% redefine part, so that it adds a toc entry without page number
\let\oldcontentsline\contentsline
\newcommand{\nopagecontentsline}[3]{\oldcontentsline{#1}{#2}{}}

    \makeatletter
\renewcommand*\l@part[2]{%
  \ifnum \c@tocdepth >-2\relax
    \addpenalty{-\@highpenalty}%
    \addvspace{0em \@plus\p@}%
    \setlength\@tempdima{3em}%
    \begingroup
      \parindent \z@ \rightskip \@pnumwidth
      \parfillskip -\@pnumwidth
      {\leavevmode
       \setstretch{.85}\large\scshape\centerpars{#1}\vspace*{-1em}\llap{#2}}\par
       \nobreak
         \global\@nobreaktrue
         \everypar{\global\@nobreakfalse\everypar{}}%
    \endgroup
  \fi}
\makeatother

\makeatletter
\def\@pnumwidth{2em}
\makeatother

% define new sectioning command, which is only used in volumes where the pannasa is found in some parts but not others, especially in an and sn

\newcommand*{\pannasa}[1]{\clearpage\thispagestyle{empty}\begin{center}\vspace*{14em}\setstretch{.85}\huge\itshape\scshape\MakeLowercase{#1}\end{center}}

    \makeatletter
\newcommand*\l@pannasa[2]{%
  \ifnum \c@tocdepth >-2\relax
    \addpenalty{-\@highpenalty}%
    \addvspace{.5em \@plus\p@}%
    \setlength\@tempdima{3em}%
    \begingroup
      \parindent \z@ \rightskip \@pnumwidth
      \parfillskip -\@pnumwidth
      {\leavevmode
       \setstretch{.85}\large\itshape\scshape\lowercase{\centerpars{#1}}\vspace*{-1em}\llap{#2}}\par
       \nobreak
         \global\@nobreaktrue
         \everypar{\global\@nobreakfalse\everypar{}}%
    \endgroup
  \fi}
\makeatother

% don't put page number on first page of toc (relies on etoolbox)
\patchcmd{\chapter}{plain}{empty}{}{}

% global line height
\setstretch{1.05}

% allow linebreak after em-dash
\catcode`\—=13
\protected\def—{\unskip\textemdash\allowbreak}

% style headings with secsty. chapter and section are defined per-edition
\partfont{\setstretch{.85}\normalfont\centering\textsc}
\subsectionfont{\setstretch{.95}\normalfont\BalancedRagged}%
\subsubsectionfont{\setstretch{1}\normalfont\itshape\BalancedRagged}

% style elements of suttatitle
\newcommand*{\suttatitleacronym}[1]{\smaller[2]{#1}\vspace*{.3em}}
\newcommand*{\suttatitletranslation}[1]{\linebreak{#1}}
\newcommand*{\suttatitleroot}[1]{\linebreak\smaller[2]\itshape{#1}}

\DeclareTOCStyleEntries[
  indent=3.3em,
  dynindent,
  beforeskip=.2em plus -2pt minus -1pt,
]{tocline}{section}

\DeclareTOCStyleEntries[
  indent=0em,
  dynindent,
  beforeskip=.4em plus -2pt minus -1pt,
]{tocline}{chapter}

\newcommand*{\tocacronym}[1]{\hspace*{-3.3em}{#1}\quad}
\newcommand*{\toctranslation}[1]{#1}
\newcommand*{\tocroot}[1]{(\textit{#1})}
\newcommand*{\tocchapterline}[1]{\bfseries\itshape{#1}}


% redefine paragraph and subparagraph headings to not be inline
\makeatletter
% Change the style of paragraph headings %
\renewcommand\paragraph{\@startsection{paragraph}{4}{\z@}%
            {-2.5ex\@plus -1ex \@minus -.25ex}%
            {1.25ex \@plus .25ex}%
            {\noindent\normalfont\itshape\small}}

% Change the style of subparagraph headings %
\renewcommand\subparagraph{\@startsection{subparagraph}{5}{\z@}%
            {-2.5ex\@plus -1ex \@minus -.25ex}%
            {1.25ex \@plus .25ex}%
            {\noindent\normalfont\itshape\footnotesize}}
\makeatother

% use etoolbox to suppress page numbers on \part
\patchcmd{\part}{\thispagestyle{plain}}{\thispagestyle{empty}}
  {}{\errmessage{Cannot patch \string\part}}

% and to reduce margins on quotation
\patchcmd{\quotation}{\rightmargin}{\leftmargin 1.2em \rightmargin}{}{}
\AtBeginEnvironment{quotation}{\small}

% titlepage
\newcommand*{\titlepageTranslationTitle}[1]{{\begin{center}\begin{large}{#1}\end{large}\end{center}}}
\newcommand*{\titlepageCreatorName}[1]{{\begin{center}\begin{normalsize}{#1}\end{normalsize}\end{center}}}

% halftitlepage
\newcommand*{\halftitlepageTranslationTitle}[1]{\setstretch{2.5}{\begin{Huge}\uppercase{\so{#1}}\end{Huge}}}
\newcommand*{\halftitlepageTranslationSubtitle}[1]{\setstretch{1.2}{\begin{large}{#1}\end{large}}}
\newcommand*{\halftitlepageFleuron}[1]{{\begin{large}\Fleuronfont{{#1}}\end{large}}}
\newcommand*{\halftitlepageByline}[1]{{\begin{normalsize}\textit{{#1}}\end{normalsize}}}
\newcommand*{\halftitlepageCreatorName}[1]{{\begin{LARGE}{\textsc{#1}}\end{LARGE}}}
\newcommand*{\halftitlepageVolumeNumber}[1]{{\begin{normalsize}{\Allsmallcapsfont{\textsc{#1}}}\end{normalsize}}}
\newcommand*{\halftitlepageVolumeAcronym}[1]{{\begin{normalsize}{#1}\end{normalsize}}}
\newcommand*{\halftitlepageVolumeTranslationTitle}[1]{{\begin{Large}{\textsc{#1}}\end{Large}}}
\newcommand*{\halftitlepageVolumeRootTitle}[1]{{\begin{normalsize}{\Allsmallcapsfont{\textsc{\itshape #1}}}\end{normalsize}}}
\newcommand*{\halftitlepagePublisher}[1]{{\begin{large}{\Noligaturecaptionfont\textsc{#1}}\end{large}}}

% epigraph
\renewcommand{\epigraphflush}{center}
\renewcommand*{\epigraphwidth}{.85\textwidth}
\newcommand*{\epigraphTranslatedTitle}[1]{\vspace*{.5em}\footnotesize\textsc{#1}\\}%
\newcommand*{\epigraphRootTitle}[1]{\footnotesize\textit{#1}\\}%
\newcommand*{\epigraphReference}[1]{\footnotesize{#1}}%

% map
\newsavebox\IBox

% custom commands for html styling classes
\newcommand*{\scnamo}[1]{\begin{Center}\textit{#1}\end{Center}\bigskip}
\newcommand*{\scendsection}[1]{\begin{Center}\begin{small}\textit{#1}\end{small}\end{Center}\addvspace{1em}}
\newcommand*{\scendsutta}[1]{\begin{Center}\textit{#1}\end{Center}\addvspace{1em}}
\newcommand*{\scendbook}[1]{\bigskip\begin{Center}\uppercase{#1}\end{Center}\addvspace{1em}}
\newcommand*{\scendkanda}[1]{\begin{Center}\textbf{#1}\end{Center}\addvspace{1em}} % use for ending vinaya rule sections and also samyuttas %
\newcommand*{\scend}[1]{\begin{Center}\begin{small}\textit{#1}\end{small}\end{Center}\addvspace{1em}}
\newcommand*{\scendvagga}[1]{\begin{Center}\textbf{#1}\end{Center}\addvspace{1em}}
\newcommand*{\scrule}[1]{\textsb{#1}}
\newcommand*{\scadd}[1]{\textit{#1}}
\newcommand*{\scevam}[1]{\textsc{#1}}
\newcommand*{\scspeaker}[1]{\hspace{2em}\textit{#1}}
\newcommand*{\scbyline}[1]{\begin{flushright}\textit{#1}\end{flushright}\bigskip}
\newcommand*{\scexpansioninstructions}[1]{\begin{small}\textit{#1}\end{small}}
\newcommand*{\scuddanaintro}[1]{\medskip\noindent\begin{footnotesize}\textit{#1}\end{footnotesize}\smallskip}

\newenvironment{scuddana}{%
\setlength{\stanzaskip}{.5\baselineskip}%
  \vspace{-1em}\begin{verse}\begin{footnotesize}%
}{%
\end{footnotesize}\end{verse}
}%

% custom command for thematic break = hr
\newcommand*{\thematicbreak}{\begin{center}\rule[.5ex]{6em}{.4pt}\begin{normalsize}\quad\Fleuronfont{•}\quad\end{normalsize}\rule[.5ex]{6em}{.4pt}\end{center}}

% manage and style page header and footer. "fancy" has header and footer, "plain" has footer only

\pagestyle{fancy}
\fancyhf{}
\fancyfoot[RE,LO]{\thepage}
\fancyfoot[LE,RO]{\footnotesize\lastleftxmark}
\fancyhead[CE]{\setstretch{.85}\Noligaturefont\MakeLowercase{\textsc{\firstrightmark}}}
\fancyhead[CO]{\setstretch{.85}\Noligaturefont\MakeLowercase{\textsc{\firstleftmark}}}
\renewcommand{\headrulewidth}{0pt}
\fancypagestyle{plain}{ %
\fancyhf{} % remove everything
\fancyfoot[RE,LO]{\thepage}
\fancyfoot[LE,RO]{\footnotesize\lastleftxmark}
\renewcommand{\headrulewidth}{0pt}
\renewcommand{\footrulewidth}{0pt}}
\fancypagestyle{plainer}{ %
\fancyhf{} % remove everything
\fancyfoot[RE,LO]{\thepage}
\renewcommand{\headrulewidth}{0pt}
\renewcommand{\footrulewidth}{0pt}}

% style footnotes
\setlength{\skip\footins}{1em}

\makeatletter
\newcommand{\@makefntextcustom}[1]{%
    \parindent 0em%
    \thefootnote.\enskip #1%
}
\renewcommand{\@makefntext}[1]{\@makefntextcustom{#1}}
\makeatother

% hang quotes (requires microtype)
\microtypesetup{
  protrusion = true,
  expansion  = true,
  tracking   = true,
  factor     = 1000,
  patch      = all,
  final
}

% Custom protrusion rules to allow hanging punctuation
\SetProtrusion
{ encoding = *}
{
% char   right left
  {-} = {    , 500 },
  % Double Quotes
  \textquotedblleft
      = {1000,     },
  \textquotedblright
      = {    , 1000},
  \quotedblbase
      = {1000,     },
  % Single Quotes
  \textquoteleft
      = {1000,     },
  \textquoteright
      = {    , 1000},
  \quotesinglbase
      = {1000,     }
}

% make latex use actual font em for parindent, not Computer Modern Roman
\AtBeginDocument{\setlength{\parindent}{1em}}%
%

% Default values; a bit sloppier than normal
\tolerance 1414
\hbadness 1414
\emergencystretch 1.5em
\hfuzz 0.3pt
\clubpenalty = 10000
\widowpenalty = 10000
\displaywidowpenalty = 10000
\hfuzz \vfuzz
 \raggedbottom%

\title{Numbered Discourses}
\author{Bhikkhu Sujato}
\date{}%
% define a different fleuron for each edition
\newfontfamily\Fleuronfont[Ornament=18]{Arno Pro}

% Define heading styles per edition for chapter and section. Suttatitle can be either of these, depending on the volume. 

\let\oldfrontmatter\frontmatter
\renewcommand{\frontmatter}{%
\chapterfont{\setstretch{.85}\normalfont\centering}%
\sectionfont{\setstretch{.85}\normalfont\BalancedRagged}%
\oldfrontmatter}

\let\oldmainmatter\mainmatter
\renewcommand{\mainmatter}{%
\chapterfont{\setstretch{.85}\normalfont\centering}%
\sectionfont{\setstretch{.85}\normalfont\centering}%
\oldmainmatter}

\let\oldbackmatter\backmatter
\renewcommand{\backmatter}{%
\chapterfont{\setstretch{.85}\normalfont\centering}%
\sectionfont{\setstretch{.85}\normalfont\BalancedRagged}%
\pagestyle{plainer}%
\oldbackmatter}
%
%
\begin{document}%
\normalsize%
\frontmatter%
\setlength{\parindent}{0cm}

\pagestyle{empty}

\maketitle

\blankpage%
\begin{center}

\vspace*{2.2em}

\halftitlepageTranslationTitle{Numbered Discourses}

\vspace*{1em}

\halftitlepageTranslationSubtitle{A sensible translation of the Aṅguttara Nikāya}

\vspace*{2em}

\halftitlepageFleuron{•}

\vspace*{2em}

\halftitlepageByline{translated and introduced by}

\vspace*{.5em}

\halftitlepageCreatorName{Bhikkhu Sujato}

\vspace*{4em}

\halftitlepageVolumeNumber{Volume 5}

\smallskip

\halftitlepageVolumeAcronym{AN 10–11}

\smallskip

\halftitlepageVolumeTranslationTitle{}

\smallskip

\halftitlepageVolumeRootTitle{}

\vspace*{\fill}

\sclogo{0}
 \halftitlepagePublisher{SuttaCentral}

\end{center}

\newpage
%
\setstretch{1.05}

\begin{footnotesize}

\textit{Numbered Discourses} is a translation of the Aṅguttaranikāya by Bhikkhu Sujato.

\medskip

Creative Commons Zero (CC0)

To the extent possible under law, Bhikkhu Sujato has waived all copyright and related or neighboring rights to \textit{Numbered Discourses}.

\medskip

This work is published from Australia.

\begin{center}
\textit{This translation is an expression of an ancient spiritual text that has been passed down by the Buddhist tradition for the benefit of all sentient beings. It is dedicated to the public domain via Creative Commons Zero (CC0). You are encouraged to copy, reproduce, adapt, alter, or otherwise make use of this translation. The translator respectfully requests that any use be in accordance with the values and principles of the Buddhist community.}
\end{center}

\medskip

\begin{description}
    \item[Web publication date] 2018
    \item[This edition] 2025-01-13 01:01:43
    \item[Publication type] hardcover
    \item[Edition] ed3
    \item[Number of volumes] 5
    \item[Publication ISBN] 978-1-76132-029-3
    \item[Volume ISBN] 978-1-76132-034-7
    \item[Publication URL] \href{https://suttacentral.net/editions/an/en/sujato}{https://suttacentral.net/editions/an/en/sujato}
    \item[Source URL] \href{https://github.com/suttacentral/bilara-data/tree/published/translation/en/sujato/sutta/an}{https://github.com/suttacentral/bilara-data/tree/published/translation/en/sujato/sutta/an}
    \item[Publication number] scpub5
\end{description}

\medskip

Map of Jambudīpa is by Jonas David Mitja Lang, and is released by him under Creative Commons Zero (CC0).

\medskip

Published by SuttaCentral

\medskip

\textit{SuttaCentral,\\
c/o Alwis \& Alwis Pty Ltd\\
Kaurna Country,\\
Suite 12,\\
198 Greenhill Road,\\
Eastwood,\\
SA 5063,\\
Australia}

\end{footnotesize}

\newpage

\setlength{\parindent}{1em}%%
\tableofcontents
\newpage
\pagestyle{fancy}
%
\mainmatter%
\pagestyle{fancy}%
\addtocontents{toc}{\let\protect\contentsline\protect\nopagecontentsline}
\part*{The Book of the Tens }
\addcontentsline{toc}{part}{The Book of the Tens }
\markboth{}{}
\addtocontents{toc}{\let\protect\contentsline\protect\oldcontentsline}

%
%
\addtocontents{toc}{\let\protect\contentsline\protect\nopagecontentsline}
\pannasa{The First Fifty }
\addcontentsline{toc}{pannasa}{The First Fifty }
\markboth{}{}
\addtocontents{toc}{\let\protect\contentsline\protect\oldcontentsline}

%
\addtocontents{toc}{\let\protect\contentsline\protect\nopagecontentsline}
\chapter*{The Chapter on Benefits }
\addcontentsline{toc}{chapter}{\tocchapterline{The Chapter on Benefits }}
\addtocontents{toc}{\let\protect\contentsline\protect\oldcontentsline}

%
\section*{{\suttatitleacronym AN 10.1}{\suttatitletranslation What’s the Purpose? }{\suttatitleroot Kimatthiyasutta}}
\addcontentsline{toc}{section}{\tocacronym{AN 10.1} \toctranslation{What’s the Purpose? } \tocroot{Kimatthiyasutta}}
\markboth{What’s the Purpose? }{Kimatthiyasutta}
\extramarks{AN 10.1}{AN 10.1}

\scevam{So\marginnote{1.1} I have heard. }At one time the Buddha was staying near \textsanskrit{Sāvatthī} in Jeta’s Grove, \textsanskrit{Anāthapiṇḍika}’s monastery. Then Venerable Ānanda went up to the Buddha, bowed, sat down to one side, and said to him: 

“Sir,\marginnote{2.1} what is the purpose and benefit of skillful ethics?” 

“Ānanda,\marginnote{2.2} having no regrets is the purpose and benefit of skillful ethics.” 

“But\marginnote{3.1} what is the purpose and benefit of having no regrets?” 

“Joy\marginnote{3.2} is the purpose and benefit of having no regrets.” 

“But\marginnote{4.1} what is the purpose and benefit of joy?” 

“Rapture\marginnote{4.2} …” 

“But\marginnote{5.1} what is the purpose and benefit of rapture?” 

“Tranquility\marginnote{5.2} …” 

“But\marginnote{6.1} what is the purpose and benefit of tranquility?” 

“Bliss\marginnote{6.2} …” 

“But\marginnote{7.1} what is the purpose and benefit of bliss?” 

“Immersion\marginnote{7.2} …” 

“But\marginnote{8.1} what is the purpose and benefit of immersion?” 

“Truly\marginnote{8.2} knowing and seeing …” 

“But\marginnote{9.1} what is the purpose and benefit of truly knowing and seeing?” 

“Disillusionment\marginnote{9.2} and dispassion …” 

“But\marginnote{10.1} what is the purpose and benefit of disillusionment and dispassion?” 

“Knowledge\marginnote{10.2} and vision of freedom is the purpose and benefit of disillusionment and dispassion. 

So,\marginnote{11.1} Ānanda, the purpose and benefit of skillful ethics is not having regrets. Joy is the purpose and benefit of not having regrets. Rapture is the purpose and benefit of joy. Tranquility is the purpose and benefit of rapture. Bliss is the purpose and benefit of tranquility. Immersion is the purpose and benefit of bliss. Truly knowing and seeing is the purpose and benefit of immersion. Disillusionment and dispassion is the purpose and benefit of truly knowing and seeing. Knowledge and vision of freedom is the purpose and benefit of disillusionment and dispassion. So, Ānanda, skillful ethics progressively lead up to the highest.” 

%
\section*{{\suttatitleacronym AN 10.2}{\suttatitletranslation Making a Wish }{\suttatitleroot Cetanākaraṇīyasutta}}
\addcontentsline{toc}{section}{\tocacronym{AN 10.2} \toctranslation{Making a Wish } \tocroot{Cetanākaraṇīyasutta}}
\markboth{Making a Wish }{Cetanākaraṇīyasutta}
\extramarks{AN 10.2}{AN 10.2}

“Mendicants,\marginnote{1.1} an ethical person, who has fulfilled ethical conduct, need not make a wish: ‘May I have no regrets!’ It’s only natural that an ethical person has no regrets. When you have no regrets you need not make a wish: ‘May I feel joy!’ It’s only natural that joy springs up when you have no regrets. When you feel joy you need not make a wish: ‘May I experience rapture!’ It’s only natural that rapture arises when you’re joyful. When your mind is full of rapture you need not make a wish: ‘May my body become tranquil!’ It’s only natural that your body becomes tranquil when your mind is full of rapture. When your body is tranquil you need not make a wish: ‘May I feel bliss!’ It’s only natural to feel bliss when your body is tranquil. When you feel bliss you need not make a wish: ‘May my mind be immersed in \textsanskrit{samādhi}!’ It’s only natural for the mind to be immersed in \textsanskrit{samādhi} when you feel bliss. When your mind is immersed in \textsanskrit{samādhi} you need not make a wish: ‘May I truly know and see!’ It’s only natural to truly know and see when your mind is immersed in \textsanskrit{samādhi}. When you truly know and see you need not make a wish: ‘May I become disillusioned and dispassionate!’ It’s only natural to become disillusioned and dispassionate when you truly know and see. When you’re disillusioned and dispassionate you need not make a wish: ‘May I realize the knowledge and vision of freedom!’ It’s only natural to realize the knowledge and vision of freedom when you’re disillusioned and dispassionate. 

And\marginnote{2.1} so, mendicants, the knowledge and vision of freedom is the purpose and benefit of disillusionment and dispassion. Disillusionment and dispassion is the purpose and benefit of truly knowing and seeing. Truly knowing and seeing is the purpose and benefit of immersion. Immersion is the purpose and benefit of bliss. Bliss is the purpose and benefit of tranquility. Tranquility is the purpose and benefit of rapture. Rapture is the purpose and benefit of joy. Joy is the purpose and benefit of not having regrets. Not having regrets is the purpose and benefit of skillful ethics. And so, mendicants, good qualities flow on and fill up from one to the other, for going from the near shore to the far shore.” 

%
\section*{{\suttatitleacronym AN 10.3}{\suttatitletranslation Vital Conditions (1st) }{\suttatitleroot Paṭhamaupanisasutta}}
\addcontentsline{toc}{section}{\tocacronym{AN 10.3} \toctranslation{Vital Conditions (1st) } \tocroot{Paṭhamaupanisasutta}}
\markboth{Vital Conditions (1st) }{Paṭhamaupanisasutta}
\extramarks{AN 10.3}{AN 10.3}

“Mendicants,\marginnote{1.1} an unethical person, who lacks ethics, has destroyed a vital condition for having no regrets. When there are regrets, one who has regrets has destroyed a vital condition for joy. When there is no joy, one who lacks joy has destroyed a vital condition for rapture. When there is no rapture, one who lacks rapture has destroyed a vital condition for tranquility. When there is no tranquility, one who lacks tranquility has destroyed a vital condition for bliss. When there is no bliss, one who lacks bliss has destroyed a vital condition for right immersion. When there is no right immersion, one who lacks right immersion has destroyed a vital condition for true knowledge and vision. When there is no true knowledge and vision, one who lacks true knowledge and vision has destroyed a vital condition for disillusionment and dispassion. When there is no disillusionment and dispassion, one who lacks disillusionment and dispassion has destroyed a vital condition for knowledge and vision of freedom. 

Suppose\marginnote{1.10} there was a tree that lacked branches and foliage. Its shoots, bark, softwood, and heartwood would not grow to fullness. 

In\marginnote{1.11} the same way, an unethical person, who lacks ethics, has destroyed a vital condition for having no regrets. When there are regrets, one who has regrets has destroyed a vital condition for joy. … One who lacks disillusionment and dispassion has destroyed a vital condition for knowledge and vision of freedom. 

An\marginnote{2.1} ethical person, who has fulfilled ethics, has fulfilled a vital condition for not having regrets. When there are no regrets, one who has no regrets has fulfilled a vital condition for joy. When there is joy, one who has fulfilled joy has fulfilled a vital condition for rapture. When there is rapture, one who has fulfilled rapture has fulfilled a vital condition for tranquility. When there is tranquility, one who has fulfilled tranquility has fulfilled a vital condition for bliss. When there is bliss, one who has fulfilled bliss has fulfilled a vital condition for right immersion. When there is right immersion, one who has fulfilled right immersion has fulfilled a vital condition for true knowledge and vision. When there is true knowledge and vision, one who has fulfilled true knowledge and vision has fulfilled a vital condition for disillusionment and dispassion. When there is disillusionment and dispassion, one who has fulfilled disillusionment and dispassion has fulfilled a vital condition for knowledge and vision of freedom. 

Suppose\marginnote{2.10} there was a tree that was complete with branches and foliage. Its shoots, bark, softwood, and heartwood would grow to fullness. 

In\marginnote{2.11} the same way, an ethical person, who has fulfilled ethics, has fulfilled a vital condition for not having regrets. When there are no regrets, one who has no regrets has fulfilled a vital condition for joy. … One who has fulfilled disillusionment and dispassion has fulfilled a vital condition for knowledge and vision of freedom.” 

%
\section*{{\suttatitleacronym AN 10.4}{\suttatitletranslation Vital Conditions (2nd) }{\suttatitleroot Dutiyaupanisasutta}}
\addcontentsline{toc}{section}{\tocacronym{AN 10.4} \toctranslation{Vital Conditions (2nd) } \tocroot{Dutiyaupanisasutta}}
\markboth{Vital Conditions (2nd) }{Dutiyaupanisasutta}
\extramarks{AN 10.4}{AN 10.4}

There\marginnote{1.1} Venerable \textsanskrit{Sāriputta} addressed the mendicants … “Reverends, an unethical person, who lacks ethics, has destroyed a vital condition for having no regrets. When there are regrets, one who has regrets has destroyed a vital condition for joy. … One who lacks disillusionment and dispassion has destroyed a vital condition for knowledge and vision of freedom. Suppose there was a tree that lacked branches and foliage. Its shoots, bark, softwood, and heartwood would not grow to fullness. In the same way, an unethical person, who lacks ethics, has destroyed a vital condition for having no regrets. When there are regrets, one who has regrets has destroyed a vital condition for joy. … One who lacks disillusionment and dispassion has destroyed a vital condition for knowledge and vision of freedom. 

An\marginnote{2.1} ethical person, who has fulfilled ethics, has fulfilled a vital condition for not having regrets. When there are no regrets, one who has no regrets has fulfilled a vital condition for joy. … One who has fulfilled disillusionment and dispassion has fulfilled a vital condition for knowledge and vision of freedom. Suppose there was a tree that was complete with branches and foliage. Its shoots, bark, softwood, and heartwood would grow to fullness. In the same way, an ethical person, who has fulfilled ethics, has fulfilled a vital condition for not having regrets. When there are no regrets, one who has no regrets has fulfilled a vital condition for joy. … One who has fulfilled disillusionment and dispassion has fulfilled a vital condition for knowledge and vision of freedom.” 

%
\section*{{\suttatitleacronym AN 10.5}{\suttatitletranslation Vital Conditions (3rd) }{\suttatitleroot Tatiyaupanisasutta}}
\addcontentsline{toc}{section}{\tocacronym{AN 10.5} \toctranslation{Vital Conditions (3rd) } \tocroot{Tatiyaupanisasutta}}
\markboth{Vital Conditions (3rd) }{Tatiyaupanisasutta}
\extramarks{AN 10.5}{AN 10.5}

There\marginnote{1.1} Venerable Ānanda addressed the mendicants … “Reverends, an unethical person, who lacks ethics, has destroyed a vital condition for having no regrets. When there are regrets, one who has regrets has destroyed a vital condition for joy. When there is no joy, one who lacks joy has destroyed a vital condition for rapture. When there is no rapture, one who lacks rapture has destroyed a vital condition for tranquility. When there is no tranquility, one who lacks tranquility has destroyed a vital condition for bliss. When there is no bliss, one who lacks bliss has destroyed a vital condition for right immersion. When there is no right immersion, one who lacks right immersion has destroyed a vital condition for true knowledge and vision. When there is no true knowledge and vision, one who lacks true knowledge and vision has destroyed a vital condition for disillusionment and dispassion. When there is no disillusionment and dispassion, one who lacks disillusionment and dispassion has destroyed a vital condition for knowledge and vision of freedom. 

Suppose\marginnote{1.11} there was a tree that lacked branches and foliage. Its shoots, bark, softwood, and heartwood would not grow to fullness. 

In\marginnote{1.12} the same way, an unethical person, who lacks ethics, has destroyed a vital condition for having no regrets. When there are regrets, one who has regrets has destroyed a vital condition for joy. … One who lacks disillusionment and dispassion has destroyed a vital condition for knowledge and vision of freedom. 

An\marginnote{2.1} ethical person, who has fulfilled ethics, has fulfilled a vital condition for not having regrets. When there are no regrets, one who has no regrets has fulfilled a vital condition for joy. When there is joy, one who has fulfilled joy has fulfilled a vital condition for rapture. When there is rapture, one who has fulfilled rapture has fulfilled a vital condition for tranquility. When there is tranquility, one who has fulfilled tranquility has fulfilled a vital condition for bliss. When there is bliss, one who has fulfilled bliss has fulfilled a vital condition for right immersion. When there is right immersion, one who has fulfilled right immersion has fulfilled a vital condition for true knowledge and vision. When there is true knowledge and vision, one who has fulfilled true knowledge and vision has fulfilled a vital condition for disillusionment and dispassion. When there is disillusionment and dispassion, one who has fulfilled disillusionment and dispassion has fulfilled a vital condition for knowledge and vision of freedom. 

Suppose\marginnote{2.10} there was a tree that was complete with branches and foliage. Its shoots, bark, softwood, and heartwood would grow to fullness. 

In\marginnote{2.11} the same way, an ethical person, who has fulfilled ethics, has fulfilled a vital condition for not having regrets. When there are no regrets, one who has no regrets has fulfilled a vital condition for joy. … One who has fulfilled disillusionment and dispassion has fulfilled a vital condition for knowledge and vision of freedom.” 

%
\section*{{\suttatitleacronym AN 10.6}{\suttatitletranslation Immersion }{\suttatitleroot Samādhisutta}}
\addcontentsline{toc}{section}{\tocacronym{AN 10.6} \toctranslation{Immersion } \tocroot{Samādhisutta}}
\markboth{Immersion }{Samādhisutta}
\extramarks{AN 10.6}{AN 10.6}

Then\marginnote{1.1} Venerable Ānanda went up to the Buddha, bowed, sat down to one side, and said to him: 

“Could\marginnote{1.2} it be, sir, that a mendicant might gain a state of immersion like this? They wouldn’t perceive earth in earth, water in water, fire in fire, or air in air. And they wouldn’t perceive the dimension of infinite space in the dimension of infinite space, the dimension of infinite consciousness in the dimension of infinite consciousness, the dimension of nothingness in the dimension of nothingness, or the dimension of neither perception nor non-perception in the dimension of neither perception nor non-perception. And they wouldn’t perceive this world in this world, or the other world in the other world. And yet they would still perceive.” 

“It\marginnote{1.3} could be, Ānanda, that a mendicant might gain a state of immersion like this. They wouldn’t perceive earth in earth, water in water, fire in fire, or air in air. And they wouldn’t perceive the dimension of infinite space in the dimension of infinite space, the dimension of infinite consciousness in the dimension of infinite consciousness, the dimension of nothingness in the dimension of nothingness, or the dimension of neither perception nor non-perception in the dimension of neither perception nor non-perception. And they wouldn’t perceive this world in this world, or the other world in the other world. And yet they would still perceive.” 

“But\marginnote{2.1} how could this be, sir?” 

“Ānanda,\marginnote{3.1} it’s when a mendicant perceives: ‘This is peaceful; this is sublime—that is, the stilling of all activities, the letting go of all attachments, the ending of craving, fading away, cessation, extinguishment.’ 

That’s\marginnote{3.3} how a mendicant might gain a state of immersion like this. They wouldn’t perceive earth in earth, water in water, fire in fire, or air in air. And they wouldn’t perceive the dimension of infinite space in the dimension of infinite space, the dimension of infinite consciousness in the dimension of infinite consciousness, the dimension of nothingness in the dimension of nothingness, or the dimension of neither perception nor non-perception in the dimension of neither perception nor non-perception. And they wouldn’t perceive this world in this world, or the other world in the other world. And yet they would still perceive.” 

%
\section*{{\suttatitleacronym AN 10.7}{\suttatitletranslation Sāriputta }{\suttatitleroot Sāriputtasutta}}
\addcontentsline{toc}{section}{\tocacronym{AN 10.7} \toctranslation{Sāriputta } \tocroot{Sāriputtasutta}}
\markboth{Sāriputta }{Sāriputtasutta}
\extramarks{AN 10.7}{AN 10.7}

Then\marginnote{1.1} Venerable Ānanda went up to Venerable \textsanskrit{Sāriputta}, and exchanged greetings with him. When the greetings and polite conversation were over, he sat down to one side and said to \textsanskrit{Sāriputta}: 

“Could\marginnote{2.1} it be, reverend \textsanskrit{Sāriputta}, that a mendicant might gain a state of immersion like this? They wouldn’t perceive earth in earth, water in water, fire in fire, or air in air. And they wouldn’t perceive the dimension of infinite space in the dimension of infinite space, the dimension of infinite consciousness in the dimension of infinite consciousness, the dimension of nothingness in the dimension of nothingness, or the dimension of neither perception nor non-perception in the dimension of neither perception nor non-perception. And they wouldn’t perceive this world in this world, or the other world in the other world. And yet they would still perceive.” 

“It\marginnote{3.1} could be, Reverend Ānanda.” 

“But\marginnote{4.1} how could this be?” 

“Reverend\marginnote{4.2} Ānanda, this one time I was staying right here at \textsanskrit{Sāvatthī} in the Dark Forest. There I gained a state of immersion like this. I didn’t perceive earth in earth, water in water, fire in fire, or air in air. And I didn’t perceive the dimension of infinite space in the dimension of infinite space, the dimension of infinite consciousness in the dimension of infinite consciousness, the dimension of nothingness in the dimension of nothingness, or the dimension of neither perception nor non-perception in the dimension of neither perception nor non-perception. And I didn’t perceive this world in this world, or the other world in the other world. And yet I still perceived.” 

“But\marginnote{5.1} at that time what did Reverend \textsanskrit{Sāriputta} perceive?” 

“One\marginnote{5.2} perception arose in me and another perception ceased: ‘The cessation of continued existence is extinguishment. The cessation of continued existence is extinguishment.’ Suppose there was a burning pile of twigs. One flame would arise and another would cease. In the same way, one perception arose in me and another perception ceased: ‘The cessation of continued existence is extinguishment. The cessation of continued existence is extinguishment.’ At that time I perceived that the cessation of continued existence is extinguishment.” 

%
\section*{{\suttatitleacronym AN 10.8}{\suttatitletranslation Inspiring All Around: the Absorptions }{\suttatitleroot Jhānasutta}}
\addcontentsline{toc}{section}{\tocacronym{AN 10.8} \toctranslation{Inspiring All Around: the Absorptions } \tocroot{Jhānasutta}}
\markboth{Inspiring All Around: the Absorptions }{Jhānasutta}
\extramarks{AN 10.8}{AN 10.8}

“Mendicants,\marginnote{1.1} a mendicant is faithful but not ethical. So they’re incomplete in that respect, and should fulfill it, thinking: ‘How can I become faithful and ethical?’ When the mendicant is faithful and ethical, they’re complete in that respect. 

A\marginnote{2.1} mendicant is faithful and ethical, but not learned. … they’re not a Dhamma speaker … they don’t frequent assemblies … they don’t teach Dhamma to the assembly with assurance … they’re not an expert in the monastic law … they don’t stay in the wilderness, in remote lodgings … they don’t get the four absorptions—blissful meditations in this life that belong to the higher mind—when they want, without trouble or difficulty … they don’t realize the undefiled freedom of heart and freedom by wisdom in this very life, and live having realized it with their own insight due to the ending of defilements. So they’re incomplete in that respect, and should fulfill it, thinking: ‘How can I become faithful, ethical, and learned, a Dhamma speaker, one who frequents assemblies, one who teaches Dhamma to the assembly with assurance, an expert in the training, one who lives in the wilderness, in remote lodgings, one who gets the four absorptions when they want, and one who lives having realized the ending of defilements?’ 

When\marginnote{3.1} they’re faithful, ethical, and learned, a Dhamma speaker, one who frequents assemblies, one who teaches Dhamma to the assembly with assurance, an expert in the training, one who lives in the wilderness, in remote lodgings, one who gets the four absorptions when they want, and one who lives having realized the ending of defilements, they’re complete in that respect. A mendicant who has these ten qualities is impressive all around, and is complete in every respect.” 

%
\section*{{\suttatitleacronym AN 10.9}{\suttatitletranslation Inspiring All Around: the Peaceful Liberations }{\suttatitleroot Santavimokkhasutta}}
\addcontentsline{toc}{section}{\tocacronym{AN 10.9} \toctranslation{Inspiring All Around: the Peaceful Liberations } \tocroot{Santavimokkhasutta}}
\markboth{Inspiring All Around: the Peaceful Liberations }{Santavimokkhasutta}
\extramarks{AN 10.9}{AN 10.9}

“A\marginnote{1.1} mendicant is faithful, but not ethical. … they’re not learned. … they’re not a Dhamma speaker … they don’t frequent assemblies … they don’t teach Dhamma to the assembly with assurance … they’re not an expert in the training … they don’t stay in the wilderness, in remote lodgings … they don’t have direct meditative experience of the peaceful liberations that are formless, transcending form … they don’t realize the undefiled freedom of heart and freedom by wisdom in this very life, and live having realized it with their own insight due to the ending of defilements. So they’re incomplete in that respect, and should fulfill it, thinking: ‘How can I become faithful, ethical, and learned, a Dhamma speaker, one who frequents assemblies, one who teaches Dhamma to the assembly with assurance, an expert in the training, one who lives in the wilderness, in remote lodgings, one who gets the formless liberations, and one who lives having realized the ending of defilements?’ 

When\marginnote{2.1} they’re faithful, ethical, and learned, a Dhamma speaker, one who frequents assemblies, one who teaches Dhamma to the assembly with assurance, an expert in the monastic law, one who lives in the wilderness, in remote lodgings, one who gets the formless liberations, and one who lives having realized the ending of defilements, they’re complete in that respect. A mendicant who has these ten qualities is impressive all around, and is complete in every respect.” 

%
\section*{{\suttatitleacronym AN 10.10}{\suttatitletranslation Inspiring All Around: the Three Knowledges }{\suttatitleroot Vijjāsutta}}
\addcontentsline{toc}{section}{\tocacronym{AN 10.10} \toctranslation{Inspiring All Around: the Three Knowledges } \tocroot{Vijjāsutta}}
\markboth{Inspiring All Around: the Three Knowledges }{Vijjāsutta}
\extramarks{AN 10.10}{AN 10.10}

“A\marginnote{1.1} mendicant is faithful, but not ethical. So they’re incomplete in that respect, and should fulfill it, thinking: ‘How can I become faithful and ethical?’ When the mendicant is faithful and ethical, they’re complete in that respect. 

A\marginnote{2.1} mendicant is faithful and ethical, but not learned … they’re not a Dhamma speaker … they don’t frequent assemblies … they don’t teach Dhamma to the assembly with assurance … they’re not an expert in the monastic law … they don’t recollect their many kinds of past lives … they don’t, with clairvoyance that is purified and superhuman, see sentient beings passing away and being reborn … they don’t realize the undefiled freedom of heart and freedom by wisdom in this very life, and live having realized it with their own insight due to the ending of defilements. So they’re incomplete in that respect, and should fulfill it, thinking: ‘How can I become faithful, ethical, and learned, a Dhamma speaker, one who frequents assemblies, one who teaches Dhamma to the assembly with assurance, an expert in the training, one who recollects their many kinds of past lives, one who with clairvoyance that surpasses the human sees sentient beings passing away and being reborn, and one who lives having realized the ending of defilements?’ 

When\marginnote{3.1} they are faithful, ethical, and learned, a Dhamma speaker, one who frequents assemblies, one who teaches Dhamma to the assembly with assurance, an expert in the training, one who recollects their many kinds of past lives, one who with clairvoyance that surpasses the human sees sentient beings passing away and being reborn, and one who lives having realized the ending of defilements, they’re complete in that respect. A mendicant who has these ten qualities is impressive all around, and is complete in every respect.” 

%
\addtocontents{toc}{\let\protect\contentsline\protect\nopagecontentsline}
\chapter*{The Chapter on a Protector }
\addcontentsline{toc}{chapter}{\tocchapterline{The Chapter on a Protector }}
\addtocontents{toc}{\let\protect\contentsline\protect\oldcontentsline}

%
\section*{{\suttatitleacronym AN 10.11}{\suttatitletranslation Lodgings }{\suttatitleroot Senāsanasutta}}
\addcontentsline{toc}{section}{\tocacronym{AN 10.11} \toctranslation{Lodgings } \tocroot{Senāsanasutta}}
\markboth{Lodgings }{Senāsanasutta}
\extramarks{AN 10.11}{AN 10.11}

“Mendicants,\marginnote{1.1} a mendicant with five factors, using and frequenting lodgings with five factors, will soon realize the undefiled freedom of heart and freedom by wisdom in this very life, and live having realized it with their own insight due to the ending of defilements. 

And\marginnote{2.1} how does a mendicant have five factors? It’s when a noble disciple has faith in the Realized One’s awakening: ‘That Blessed One is perfected, a fully awakened Buddha, accomplished in knowledge and conduct, holy, knower of the world, supreme guide for those who wish to train, teacher of gods and humans, awakened, blessed.’ They are rarely ill or unwell. Their stomach digests well, being neither too hot nor too cold, but just right, and fit for meditation. They’re not devious or deceitful. They reveal themselves honestly to the Teacher or sensible spiritual companions. They live with energy roused up for giving up unskillful qualities and embracing skillful qualities. They’re strong, staunchly vigorous, not slacking off when it comes to developing skillful qualities. They’re wise. They have the wisdom of arising and passing away which is noble, penetrative, and leads to the complete ending of suffering. That’s how a mendicant has five factors. 

And\marginnote{3.1} how does a lodging have five factors? It’s when a lodging is neither too far nor too near, but convenient for coming and going. It’s not bothered by people by day, and at night it’s quiet and still. There’s little disturbance from flies, mosquitoes, wind, sun, and reptiles. While staying in that lodging the necessities of life—robes, almsfood, lodgings, and medicines and supplies for the sick—are easy to come by. And in that lodging there are several senior mendicants who are very learned, inheritors of the heritage, who have memorized the teachings, the monastic law, and the outlines. From time to time they go up to those mendicants and ask them questions: ‘Why, sir, does it say this? What does that mean?’ Those venerables clarify what is unclear, reveal what is obscure, and dispel doubt regarding the many doubtful matters. That’s how a lodging has five factors. A mendicant with five factors, using and frequenting lodgings with five factors, will soon realize the undefiled freedom of heart and freedom by wisdom in this very life, and live having realized it with their own insight due to the ending of defilements.” 

%
\section*{{\suttatitleacronym AN 10.12}{\suttatitletranslation Five Factors }{\suttatitleroot Pañcaṅgasutta}}
\addcontentsline{toc}{section}{\tocacronym{AN 10.12} \toctranslation{Five Factors } \tocroot{Pañcaṅgasutta}}
\markboth{Five Factors }{Pañcaṅgasutta}
\extramarks{AN 10.12}{AN 10.12}

“Mendicants,\marginnote{1.1} in this teaching and training a mendicant who has given up five factors and possesses five factors is called consummate, accomplished, a supreme person. 

And\marginnote{1.2} how has a mendicant given up five factors? It’s when a mendicant has given up sensual desire, ill will, dullness and drowsiness, restlessness and remorse, and doubt. That’s how a mendicant has given up five factors. 

And\marginnote{2.1} how does a mendicant have five factors? It’s when a mendicant has the entire spectrum of an adept’s ethics, immersion, wisdom, freedom, and knowledge and vision of freedom. That’s how a mendicant has five factors. 

In\marginnote{3.1} this teaching and training a mendicant who has given up five factors and possesses five factors is called consummate, accomplished, a supreme person. 

\begin{verse}%
Sensual\marginnote{4.1} desire, ill will, \\
dullness and drowsiness, \\
restlessness, and doubt \\
are not found in a mendicant at all. 

One\marginnote{5.1} like this is accomplished \\
in an adept’s ethics, \\
an adept’s immersion, \\
and freedom and knowledge. 

Possessing\marginnote{6.1} these five factors, \\
and rid of five factors, \\
in this teaching and training \\
they’re called ‘consummate’.” 

%
\end{verse}

%
\section*{{\suttatitleacronym AN 10.13}{\suttatitletranslation Fetters }{\suttatitleroot Saṁyojanasutta}}
\addcontentsline{toc}{section}{\tocacronym{AN 10.13} \toctranslation{Fetters } \tocroot{Saṁyojanasutta}}
\markboth{Fetters }{Saṁyojanasutta}
\extramarks{AN 10.13}{AN 10.13}

“Mendicants,\marginnote{1.1} there are ten fetters. What ten? The five lower fetters and the five higher fetters. What are the five lower fetters? Substantialist view, doubt, misapprehension of precepts and observances, sensual desire, and ill will. These are the five lower fetters. 

What\marginnote{2.1} are the five higher fetters? Desire for rebirth in the realm of luminous form, desire for rebirth in the formless realm, conceit, restlessness, and ignorance. These are the five higher fetters. These are the ten fetters.” 

%
\section*{{\suttatitleacronym AN 10.14}{\suttatitletranslation Hard-heartedness }{\suttatitleroot Cetokhilasutta}}
\addcontentsline{toc}{section}{\tocacronym{AN 10.14} \toctranslation{Hard-heartedness } \tocroot{Cetokhilasutta}}
\markboth{Hard-heartedness }{Cetokhilasutta}
\extramarks{AN 10.14}{AN 10.14}

“Mendicants,\marginnote{1.1} a monk or nun who has not given up five kinds of hard-heartedness and has not severed five shackles of the heart can expect decline, not growth, in skillful qualities, whether by day or by night. 

What\marginnote{2.1} are the five kinds of hard-heartedness they haven’t given up? 

Firstly,\marginnote{2.2} a mendicant has doubts about the Teacher. They’re uncertain, undecided, and lacking confidence. This being so, their mind doesn’t incline toward keenness, commitment, persistence, and striving. This is the first kind of hard-heartedness they haven’t given up. 

Furthermore,\marginnote{3.1} a mendicant has doubts about the teaching … the \textsanskrit{Saṅgha} … the training … A mendicant is angry and upset with their spiritual companions, resentful and closed off. This being so, their mind doesn’t incline toward keenness, commitment, persistence, and striving. This is the fifth kind of hard-heartedness they haven’t given up. These are the five kinds of hard-heartedness they haven’t given up. 

What\marginnote{4.1} are the five shackles of the heart they haven’t severed? Firstly, a mendicant isn’t free of greed, desire, fondness, thirst, passion, and craving for sensual pleasures. This being so, their mind doesn’t incline toward keenness, commitment, persistence, and striving. This is the first shackle of the heart they haven’t severed. 

Furthermore,\marginnote{5.1} a mendicant isn’t free of greed for the body … They’re not free of greed for form … They eat as much as they like until their belly is full, then indulge in the pleasures of sleeping, lying down, and drowsing … They lead the spiritual life wishing to be reborn in one of the orders of gods: ‘By this precept or observance or fervent austerity or spiritual life, may I become one of the gods!’ This being so, their mind doesn’t incline toward keenness, commitment, persistence, and striving. This is the fifth shackle of the heart they haven’t severed. These are the five shackles of the heart they haven’t severed. 

A\marginnote{6.1} monk or nun who has not given up these five kinds of hard-heartedness and has not severed these five shackles of the heart can expect decline, not growth, in skillful qualities, whether by day or by night. 

It’s\marginnote{7.1} like the moon in the waning fortnight. Whether by day or by night, its beauty, roundness, light, and diameter and circumference only decline. In the same way, monk or nun who has not given up these five kinds of hard-heartedness and has not severed these five shackles of the heart can expect decline, not growth, in skillful qualities, whether by day or by night. 

A\marginnote{8.1} monk or nun who has given up five kinds of hard-heartedness and has severed five shackles of the heart can expect growth, not decline, in skillful qualities, whether by day or by night. 

What\marginnote{9.1} are the five kinds of hard-heartedness they’ve given up? Firstly, a mendicant has no doubts about the Teacher. They’re not uncertain, undecided, or lacking confidence. This being so, their mind inclines toward keenness, commitment, persistence, and striving. This is the first kind of hard-heartedness they’ve given up. 

Furthermore,\marginnote{10.1} a mendicant has no doubts about the teaching … the \textsanskrit{Saṅgha} … the training … A mendicant is not angry and upset with their spiritual companions, not resentful or closed off. This being so, their mind inclines toward keenness, commitment, persistence, and striving. This is the fifth kind of hard-heartedness they’ve given up. These are the five kinds of hard-heartedness they’ve given up. 

What\marginnote{11.1} are the five shackles of the heart they’ve severed? Firstly, a mendicant is rid of greed, desire, fondness, thirst, passion, and craving for sensual pleasures. This being so, their mind inclines toward keenness, commitment, persistence, and striving. This is the first shackle of the heart they’ve severed. 

Furthermore,\marginnote{12.1} a mendicant is rid of greed for the body … They’re rid of greed for form … They don’t eat as much as they like until their belly is full, then indulge in the pleasures of sleeping, lying down, and drowsing … They don’t lead the spiritual life wishing to be reborn in one of the orders of gods: ‘By this precept or observance or fervent austerity or spiritual life, may I become one of the gods!’ This being so, their mind inclines toward keenness, commitment, persistence, and striving. This is the fifth shackle of the heart they’ve severed. These are the five shackles of the heart they’ve severed. 

A\marginnote{13.1} monk or nun who has given up these five kinds of hard-heartedness and has severed these five shackles of the heart can expect growth, not decline, in skillful qualities, whether by day or by night. 

It’s\marginnote{14.1} like the moon in the waxing fortnight. Whether by day or by night, its beauty, roundness, light, and diameter and circumference only grow. In the same way, a monk or nun who has given up these five kinds of hard-heartedness and has severed these five shackles of the heart can expect growth, not decline, in skillful qualities, whether by day or by night.” 

%
\section*{{\suttatitleacronym AN 10.15}{\suttatitletranslation Diligence }{\suttatitleroot Appamādasutta}}
\addcontentsline{toc}{section}{\tocacronym{AN 10.15} \toctranslation{Diligence } \tocroot{Appamādasutta}}
\markboth{Diligence }{Appamādasutta}
\extramarks{AN 10.15}{AN 10.15}

“Mendicants,\marginnote{1.1} the Realized One, the perfected one, the fully awakened Buddha, is said to be the best of all sentient beings—be they footless, with two feet, four feet, or many feet; with form or formless; with perception or without perception or with neither perception nor non-perception. In the same way, all skillful qualities are rooted in diligence and meet at diligence, and diligence is said to be the best of them. 

The\marginnote{2.1} footprints of all creatures that walk can fit inside an elephant’s footprint, so an elephant’s footprint is said to be the biggest of them all. In the same way, all skillful qualities are rooted in diligence and meet at diligence, and diligence is said to be the best of them. 

The\marginnote{3.1} rafters of a bungalow all lean to the peak, slope to the peak, and meet at the peak, so the peak is said to be the topmost of them all. In the same way, all skillful qualities are rooted in diligence and meet at diligence, and diligence is said to be the best of them. 

Of\marginnote{4.1} all kinds of fragrant root, spikenard is said to be the best. In the same way … 

Of\marginnote{5.1} all kinds of fragrant heartwood, red sandalwood is said to be the best. In the same way … 

Of\marginnote{6.1} all kinds of fragrant flower, jasmine is said to be the best. In the same way … 

All\marginnote{7.1} lesser kings are vassals of a wheel-turning monarch, so the wheel-turning monarch is said to be the foremost of them all. In the same way … 

The\marginnote{8.1} radiance of all the stars is not worth a sixteenth part of the moon’s radiance, so the moon’s radiance is said to be the best of them all. In the same way … 

In\marginnote{9.1} the autumn the heavens are clear and cloudless. And as the sun is rising to the firmament, having dispelled all the darkness of space, it shines and glows and radiates. In the same way … 

All\marginnote{10.1} the great rivers—that is, the Ganges, \textsanskrit{Yamunā}, \textsanskrit{Aciravatī}, \textsanskrit{Sarabhū}, and \textsanskrit{Mahī}—flow, slant, slope, and incline towards the ocean, and the ocean is said to be the greatest of them. In the same way, all skillful qualities are rooted in diligence and meet at diligence, and diligence is said to be the best of them.” 

%
\section*{{\suttatitleacronym AN 10.16}{\suttatitletranslation Worthy of Offerings Dedicated to the Gods }{\suttatitleroot Āhuneyyasutta}}
\addcontentsline{toc}{section}{\tocacronym{AN 10.16} \toctranslation{Worthy of Offerings Dedicated to the Gods } \tocroot{Āhuneyyasutta}}
\markboth{Worthy of Offerings Dedicated to the Gods }{Āhuneyyasutta}
\extramarks{AN 10.16}{AN 10.16}

“Mendicants,\marginnote{1.1} these ten people are worthy of offerings dedicated to the gods, worthy of hospitality, worthy of a religious donation, worthy of greeting with joined palms, and are the supreme field of merit for the world. What ten? A Realized One, a perfected one, a fully awakened Buddha; an Independent Buddha; one freed both ways; one freed by wisdom; a direct witness; one attained to view; one freed by faith; a follower by faith; a follower of teachings; a lamb of the flock. These are the ten people who are worthy of offerings dedicated to the gods, worthy of hospitality, worthy of a religious donation, worthy of greeting with joined palms, and are the supreme field of merit for the world.” 

%
\section*{{\suttatitleacronym AN 10.17}{\suttatitletranslation A Protector (1st) }{\suttatitleroot Paṭhamanāthasutta}}
\addcontentsline{toc}{section}{\tocacronym{AN 10.17} \toctranslation{A Protector (1st) } \tocroot{Paṭhamanāthasutta}}
\markboth{A Protector (1st) }{Paṭhamanāthasutta}
\extramarks{AN 10.17}{AN 10.17}

“Mendicants,\marginnote{1.1} you should live with a protector, not without one. Living without a protector is suffering. There are ten qualities that serve as protector. What ten? Firstly, a mendicant is ethical, restrained in the monastic code, conducting themselves well and resorting for alms in suitable places. Seeing danger in the slightest fault, they keep the rules they’ve undertaken. This is a quality that serves as protector. 

Furthermore,\marginnote{2.1} a mendicant is very learned, remembering and keeping what they’ve learned. These teachings are good in the beginning, good in the middle, and good in the end, meaningful and well-phrased, describing a spiritual practice that’s entirely full and pure. They are very learned in such teachings, remembering them, rehearsing them, mentally scrutinizing them, and comprehending them theoretically. This too is a quality that serves as protector. 

Furthermore,\marginnote{3.1} a mendicant has good friends, companions, and associates. This too is a quality that serves as protector. 

Furthermore,\marginnote{4.1} a mendicant is easy to admonish, having qualities that make them easy to admonish. They’re patient, and take instruction respectfully. This too is a quality that serves as protector. 

Furthermore,\marginnote{5.1} a mendicant is deft and tireless in a diverse spectrum of duties for their spiritual companions, understanding how to go about things in order to complete and organize the work. This too is a quality that serves as protector. 

Furthermore,\marginnote{6.1} a mendicant loves the teachings and is a delight to converse with, being full of joy in the teaching and training. This too is a quality that serves as protector. 

Furthermore,\marginnote{7.1} a mendicant lives with energy roused up for giving up unskillful qualities and embracing skillful qualities. They are strong, staunchly vigorous, not slacking off when it comes to developing skillful qualities. This too is a quality that serves as protector. 

Furthermore,\marginnote{8.1} a mendicant is content with any kind of robes, almsfood, lodgings, and medicines and supplies for the sick. This too is a quality that serves as protector. 

Furthermore,\marginnote{9.1} a mendicant is mindful. They have utmost mindfulness and alertness, and can remember and recall what was said and done long ago. This too is a quality that serves as protector. 

Furthermore,\marginnote{10.1} a mendicant is wise. They have the wisdom of arising and passing away which is noble, penetrative, and leads to the complete ending of suffering. This too is a quality that serves as protector. 

You\marginnote{11.1} should live with a protector, not without one. Living without a protector is suffering. These are the ten qualities that serve as protector.” 

%
\section*{{\suttatitleacronym AN 10.18}{\suttatitletranslation A Protector (2nd) }{\suttatitleroot Dutiyanāthasutta}}
\addcontentsline{toc}{section}{\tocacronym{AN 10.18} \toctranslation{A Protector (2nd) } \tocroot{Dutiyanāthasutta}}
\markboth{A Protector (2nd) }{Dutiyanāthasutta}
\extramarks{AN 10.18}{AN 10.18}

\scevam{So\marginnote{1.1} I have heard. }At one time the Buddha was staying near \textsanskrit{Sāvatthī} in Jeta’s Grove, \textsanskrit{Anāthapiṇḍika}’s monastery. There the Buddha addressed the mendicants, “Mendicants!” 

“Venerable\marginnote{1.5} sir,” they replied. The Buddha said this: 

“Mendicants,\marginnote{2.1} you should live with a protector, not without one. Living without a protector is suffering. There are ten qualities that serve as protector. What ten? Firstly, a mendicant is ethical, restrained in the monastic code, conducting themselves well and resorting for alms in suitable places. Seeing danger in the slightest fault, they keep the rules they’ve undertaken. Knowing this, the mendicants—whether senior, middle, or junior—think that mendicant is worth advising and instructing. Being treated with such kindness by the senior, middle, and junior mendicants, that mendicant can expect only growth, not decline. This is a quality that serves as protector. 

Furthermore,\marginnote{3.1} a mendicant is very learned, remembering and keeping what they’ve learned. These teachings are good in the beginning, good in the middle, and good in the end, meaningful and well-phrased, describing a spiritual practice that’s entirely full and pure. They are very learned in such teachings, remembering them, reinforcing them by recitation, mentally scrutinizing them, and comprehending them theoretically. Knowing this, the mendicants—whether senior, middle, or junior—think that mendicant is worth advising and instructing. Being treated with such kindness by the senior, middle, and junior mendicants, that mendicant can expect only growth, not decline. This too is a quality that serves as protector. 

Furthermore,\marginnote{4.1} a mendicant has good friends, companions, and associates. Knowing this, the mendicants—whether senior, middle, or junior—think that mendicant is worth advising and instructing. Being treated with such kindness by the senior, middle, and junior mendicants, that mendicant can expect only growth, not decline. This too is a quality that serves as protector. 

Furthermore,\marginnote{5.1} a mendicant is easy to admonish, having qualities that make them easy to admonish. They’re patient, and take instruction respectfully. Knowing this, the mendicants—whether senior, middle, or junior—think that mendicant is worth advising and instructing. Being treated with such kindness by the senior, middle, and junior mendicants, that mendicant can expect only growth, not decline. This too is a quality that serves as protector. 

Furthermore,\marginnote{6.1} a mendicant is deft and tireless in a diverse spectrum of duties for their spiritual companions, understanding how to go about things in order to complete and organize the work. Knowing this, the mendicants—whether senior, middle, or junior—think that mendicant is worth advising and instructing. Being treated with such kindness by the senior, middle, and junior mendicants, that mendicant can expect only growth, not decline. This too is a quality that serves as protector. 

Furthermore,\marginnote{7.1} a mendicant loves the teachings and is a delight to converse with, being full of joy in the teaching and training. Knowing this, the mendicants—whether senior, middle, or junior—think that mendicant is worth advising and instructing. Being treated with such kindness by the senior, middle, and junior mendicants, that mendicant can expect only growth, not decline. This too is a quality that serves as protector. 

Furthermore,\marginnote{8.1} a mendicant lives with energy roused up for giving up unskillful qualities and embracing skillful qualities. They are strong, staunchly vigorous, not slacking off when it comes to developing skillful qualities. Knowing this, the mendicants—whether senior, middle, or junior—think that mendicant is worth advising and instructing. Being treated with such kindness by the senior, middle, and junior mendicants, that mendicant can expect only growth, not decline. This too is a quality that serves as protector. 

Furthermore,\marginnote{9.1} a mendicant is content with any kind of robes, almsfood, lodgings, and medicines and supplies for the sick. Knowing this, the mendicants—whether senior, middle, or junior—think that mendicant is worth advising and instructing. Being treated with such kindness by the senior, middle, and junior mendicants, that mendicant can expect only growth, not decline. This too is a quality that serves as protector. 

Furthermore,\marginnote{10.1} a mendicant is mindful. They have utmost mindfulness and alertness, and can remember and recall what was said and done long ago. Knowing this, the mendicants—whether senior, middle, or junior—think that mendicant is worth advising and instructing. Being treated with such kindness by the senior, middle, and junior mendicants, that mendicant can expect only growth, not decline. This too is a quality that serves as protector. 

Furthermore,\marginnote{11.1} a mendicant is wise. They have the wisdom of arising and passing away which is noble, penetrative, and leads to the complete ending of suffering. Knowing this, the mendicants—whether senior, middle, or junior—think that mendicant is worth advising and instructing. Being treated with such kindness by the senior, middle, and junior mendicants, that mendicant can expect only growth, not decline. This too is a quality that serves as protector. 

You\marginnote{12.1} should live with a protector, not without one. Living without a protector is suffering. These are the ten qualities that serve as protector.” 

That\marginnote{12.4} is what the Buddha said. Satisfied, the mendicants approved what the Buddha said. 

%
\section*{{\suttatitleacronym AN 10.19}{\suttatitletranslation Abodes of the Noble Ones (1st) }{\suttatitleroot Paṭhamaariyāvāsasutta}}
\addcontentsline{toc}{section}{\tocacronym{AN 10.19} \toctranslation{Abodes of the Noble Ones (1st) } \tocroot{Paṭhamaariyāvāsasutta}}
\markboth{Abodes of the Noble Ones (1st) }{Paṭhamaariyāvāsasutta}
\extramarks{AN 10.19}{AN 10.19}

“There\marginnote{1.1} are these ten abodes of the noble ones in which the noble ones of the past, present, and future abide. What ten? A mendicant has given up five factors, is endowed with six factors, has a single guard, has four supports, has eliminated idiosyncratic interpretations of the truth, has totally given up searching, has pure intentions, has stilled the physical process, and is well freed in mind and well freed by wisdom. These are the ten abodes of the noble ones in which the noble ones of the past, present, and future abide.” 

%
\section*{{\suttatitleacronym AN 10.20}{\suttatitletranslation Abodes of the Noble Ones (2nd) }{\suttatitleroot Dutiyaariyāvāsasutta}}
\addcontentsline{toc}{section}{\tocacronym{AN 10.20} \toctranslation{Abodes of the Noble Ones (2nd) } \tocroot{Dutiyaariyāvāsasutta}}
\markboth{Abodes of the Noble Ones (2nd) }{Dutiyaariyāvāsasutta}
\extramarks{AN 10.20}{AN 10.20}

At\marginnote{1.1} one time the Buddha was staying in the land of the Kurus, near the Kuru town named \textsanskrit{Kammāsadamma}. There the Buddha addressed the mendicants: 

“There\marginnote{2.1} are these ten abodes of the noble ones in which the noble ones of the past, present, and future abide. What ten? A mendicant has given up five factors, possesses six factors, has a single guard, has four supports, has eliminated idiosyncratic interpretations of the truth, has totally given up searching, has unsullied intentions, has stilled the physical process, and is well freed in mind and well freed by wisdom. 

And\marginnote{3.1} how has a mendicant given up five factors? It’s when a mendicant has given up sensual desire, ill will, dullness and drowsiness, restlessness and remorse, and doubt. That’s how a mendicant has given up five factors. 

And\marginnote{4.1} how does a mendicant possess six factors? It’s when a mendicant, seeing a sight with their eyes, is neither happy nor sad. They remain equanimous, mindful and aware. Hearing a sound with their ears … Smelling an odor with their nose … Tasting a flavor with their tongue … 

Feeling\marginnote{4.6} a touch with their body … Knowing an idea with their mind, they’re neither happy nor sad. They remain equanimous, mindful and aware. That’s how a mendicant possesses six factors. 

And\marginnote{5.1} how does a mendicant have a single guard? It’s when a mendicant’s heart is guarded by mindfulness. That’s how a mendicant has a single guard. 

And\marginnote{6.1} how does a mendicant have four supports? After appraisal, a mendicant uses some things, endures some things, avoids some things, and gets rid of some things. That’s how a mendicant has four supports. 

And\marginnote{7.1} how has a mendicant eliminated idiosyncratic interpretations of the truth? Different ascetics and brahmins have different idiosyncratic interpretations of the truth. For example: the cosmos is eternal, or not eternal, or finite, or infinite; the soul and the body are the same thing, or they are different things; after death, a realized one still exists, or no longer exists, or both still exists and no longer exists, or neither still exists nor no longer exists. A mendicant has dispelled, eliminated, thrown out, rejected, let go of, given up, and relinquished all these. That’s how a mendicant has eliminated idiosyncratic interpretations of the truth. 

And\marginnote{8.1} how has a mendicant totally given up searching? It’s when they’ve given up searching for sensual pleasures, for continued existence, and for a spiritual life. That’s how a mendicant has totally given up searching. 

And\marginnote{9.1} how does a mendicant have unsullied intentions? It’s when a mendicant has given up intentions of sensuality, malice, and cruelty. That’s how a mendicant has unsullied intentions. 

And\marginnote{10.1} how has a mendicant stilled the physical process? It’s when, giving up pleasure and pain, and ending former happiness and sadness, they enter and remain in the fourth absorption, without pleasure or pain, with pure equanimity and mindfulness. That’s how a mendicant has stilled the physical process. 

And\marginnote{11.1} how is a mendicant well freed in mind? It’s when a mendicant’s mind is freed from greed, hate, and delusion. That’s how a mendicant is well freed in mind. 

And\marginnote{12.1} how is a mendicant well freed by wisdom? It’s when a mendicant understands: ‘I’ve given up greed, hate, and delusion, cut them off at the root, made them like a palm stump, obliterated them, so they’re unable to arise in the future.’ That’s how a mendicant’s mind is well freed by wisdom. 

Mendicants,\marginnote{13.1} whether in the past, future, or present, all noble ones abide in these same ten abodes of the noble ones. These are the ten abodes of the noble ones in which the noble ones of the past, present, and future abide.” 

%
\addtocontents{toc}{\let\protect\contentsline\protect\nopagecontentsline}
\chapter*{The Great Chapter }
\addcontentsline{toc}{chapter}{\tocchapterline{The Great Chapter }}
\addtocontents{toc}{\let\protect\contentsline\protect\oldcontentsline}

%
\section*{{\suttatitleacronym AN 10.21}{\suttatitletranslation The Lion’s Roar }{\suttatitleroot Sīhanādasutta}}
\addcontentsline{toc}{section}{\tocacronym{AN 10.21} \toctranslation{The Lion’s Roar } \tocroot{Sīhanādasutta}}
\markboth{The Lion’s Roar }{Sīhanādasutta}
\extramarks{AN 10.21}{AN 10.21}

“Mendicants,\marginnote{1.1} towards evening the lion, king of beasts, emerges from his den, yawns, looks all around the four quarters, and roars his lion’s roar three times. Then he sets out on the hunt. Why is that? ‘May I not injure any little creatures on unclear ground.’ 

‘Lion’\marginnote{2.1} is a term for the Realized One, the perfected one, the fully awakened Buddha. When the Realized One teaches Dhamma to an assembly, this is his lion’s roar. 

The\marginnote{3.1} Realized One possesses ten powers of a Realized One. With these he claims the bull’s place, roars his lion’s roar in the assemblies, and turns the divine wheel. What ten? Firstly, the Realized One truly understands the possible as possible and the impossible as impossible. Since he truly understands this, this is a power of the Realized One. Relying on this he claims the bull’s place, roars his lion’s roar in the assemblies, and turns the divine wheel. 

Furthermore,\marginnote{4.1} the Realized One truly understands the result of deeds undertaken in the past, future, and present in terms of grounds and causes. Since he truly understands this, this is a power of the Realized One. … 

Furthermore,\marginnote{5.1} the Realized One truly understands where all paths of practice lead. Since he truly understands this, this is a power of the Realized One. … 

Furthermore,\marginnote{6.1} the Realized One truly understands the world with its many and diverse elements. Since he truly understands this, this is a power of the Realized One. … 

Furthermore,\marginnote{7.1} the Realized One truly understands the diverse convictions of sentient beings. Since he truly understands this, this is a power of the Realized One. … 

Furthermore,\marginnote{8.1} the Realized One truly understands the faculties of other sentient beings and other individuals after comprehending them with his mind. Since he truly understands this, this is a power of the Realized One. … 

Furthermore,\marginnote{9.1} the Realized One truly understands corruption, cleansing, and emergence regarding the absorptions, liberations, immersions, and attainments. Since he truly understands this, this is a power of the Realized One. … 

Furthermore,\marginnote{10.1} the Realized One recollects many kinds of past lives. That is: one, two, three, four, five, ten, twenty, thirty, forty, fifty, a hundred, a thousand, a hundred thousand rebirths; many eons of the world contracting, many eons of the world expanding, many eons of the world contracting and expanding. He remembers: ‘There, I was named this, my clan was that, I looked like this, and that was my food. This was how I felt pleasure and pain, and that was how my life ended. When I passed away from that place I was reborn somewhere else. There, too, I was named this, my clan was that, I looked like this, and that was my food. This was how I felt pleasure and pain, and that was how my life ended. When I passed away from that place I was reborn here.’ Thus he recollects his many past lives, with features and details. Since he truly understands this, this is a power of the Realized One. … 

Furthermore,\marginnote{11.1} with clairvoyance that is purified and superhuman, the Realized One sees sentient beings passing away and being reborn—inferior and superior, beautiful and ugly, in a good place or a bad place. He understands how sentient beings are reborn according to their deeds. ‘These dear beings did bad things by way of body, speech, and mind. They denounced the noble ones; they had wrong view; and they chose to act out of that wrong view. When their body breaks up, after death, they’re reborn in a place of loss, a bad place, the underworld, hell. These dear beings, however, did good things by way of body, speech, and mind. They never denounced the noble ones; they had right view; and they chose to act out of that right view. When their body breaks up, after death, they’re reborn in a good place, a heavenly realm.’ And so, with clairvoyance that is purified and superhuman, he sees sentient beings passing away and being reborn—inferior and superior, beautiful and ugly, in a good place or a bad place. He understands how sentient beings are reborn according to their deeds. Since he truly understands this, this is a power of the Realized One. … 

Furthermore,\marginnote{12.1} the Realized One has realized the undefiled freedom of heart and freedom by wisdom in this very life, and lives having realized it with his own insight due to the ending of defilements. Since he truly understands this, this is a power of the Realized One. … 

These\marginnote{13.1} are the ten powers of a Realized One that the Realized One possesses. With these he claims the bull’s place, roars his lion’s roar in the assemblies, and turns the divine wheel.” 

%
\section*{{\suttatitleacronym AN 10.22}{\suttatitletranslation Hypotheses }{\suttatitleroot Adhivuttipadasutta}}
\addcontentsline{toc}{section}{\tocacronym{AN 10.22} \toctranslation{Hypotheses } \tocroot{Adhivuttipadasutta}}
\markboth{Hypotheses }{Adhivuttipadasutta}
\extramarks{AN 10.22}{AN 10.22}

Then\marginnote{1.1} Venerable Ānanda went up to the Buddha, bowed, and sat down to one side. The Buddha said to him: 

“Ānanda,\marginnote{2.1} I claim to be assured regarding the teachings that lead to realizing by insight the various different hypotheses. So I am able to teach the Dhamma in appropriate ways to different people. Practicing accordingly, when something exists they’ll know it exists. When it doesn’t exist they’ll know it doesn’t exist. When something is inferior they’ll know it’s inferior. When it’s superior they’ll know it’s superior. When something is not supreme they’ll know it’s not supreme. When it is supreme they’ll know it’s supreme. And they will know or see or realize it in whatever way it should be known or seen or realized. This is possible. But this is the unsurpassable knowledge, that is: truly knowing each and every case. And Ānanda, I say that there is no other knowledge better or finer than this. 

The\marginnote{3.1} Realized One possesses ten powers of a Realized One. With these he claims the bull’s place, roars his lion’s roar in the assemblies, and turns the divine wheel. What ten? Firstly, the Realized One truly understands the possible as possible, and the impossible as impossible. Since he truly understands this, this is a power of the Realized One. Relying on this he claims the bull’s place, roars his lion’s roar in the assemblies, and turns the divine wheel. 

Furthermore,\marginnote{4.1} the Realized One truly understands the result of deeds undertaken in the past, future, and present in terms of grounds and causes. Since he truly understands this, this is a power of the Realized One. … 

Furthermore,\marginnote{5.1} the Realized One truly understands where all paths of practice lead. Since he truly understands this, this is a power of the Realized One. … 

Furthermore,\marginnote{6.1} the Realized One truly understands the world with its many and diverse elements. Since he truly understands this, this is a power of the Realized One. … 

Furthermore,\marginnote{7.1} the Realized One truly understands the diverse convictions of sentient beings. Since he truly understands this, this is a power of the Realized One. … 

Furthermore,\marginnote{8.1} the Realized One truly understands the faculties of other sentient beings and other individuals after comprehending them with his mind. Since he truly understands this, this is a power of the Realized One. … 

Furthermore,\marginnote{9.1} the Realized One truly understands corruption, cleansing, and emergence regarding the absorptions, liberations, immersions, and attainments. Since he truly understands this, this is a power of the Realized One. … 

Furthermore,\marginnote{10.1} the Realized One recollects many kinds of past lives, with features and details. Since he truly understands this, this is a power of the Realized One. … 

Furthermore,\marginnote{11.1} with clairvoyance that is purified and superhuman, the Realized One sees sentient beings passing away and being reborn—inferior and superior, beautiful and ugly, in a good place or a bad place. He understands how sentient beings are reborn according to their deeds. Since he truly understands this, this is a power of the Realized One. … 

Furthermore,\marginnote{12.1} the Realized One has realized the undefiled freedom of heart and freedom by wisdom in this very life, and lives having realized it with his own insight due to the ending of defilements. Since he truly understands this, this is a power of the Realized One. … 

These\marginnote{13.1} are the ten powers of a Realized One that the Realized One possesses. With these he claims the bull’s place, roars his lion’s roar in the assemblies, and turns the divine wheel.” 

%
\section*{{\suttatitleacronym AN 10.23}{\suttatitletranslation Body }{\suttatitleroot Kāyasutta}}
\addcontentsline{toc}{section}{\tocacronym{AN 10.23} \toctranslation{Body } \tocroot{Kāyasutta}}
\markboth{Body }{Kāyasutta}
\extramarks{AN 10.23}{AN 10.23}

“Mendicants,\marginnote{1.1} there are things that should be given up by the body, not by speech. There are things that should be given up by speech, not by the body. There are things that should be given up neither by the body, nor by speech, but by seeing again and again with wisdom. 

And\marginnote{2.1} what are the things that should be given up by the body, not by speech? It’s when a mendicant has committed a certain unskillful offense by way of body. After examination, sensible spiritual companions say this to them: ‘Venerable, you’ve committed a certain unskillful offense by way of body. Please give up that bad bodily conduct and develop good bodily conduct.’ When spoken to by their sensible spiritual companions they give up that bad bodily conduct and develop good bodily conduct. These are the things that should be given up by the body, not by speech. 

And\marginnote{3.1} what are the things that should be given up by speech, not by the body? It’s when a mendicant has committed a certain unskillful offense by way of speech. After examination, sensible spiritual companions say this to them: ‘Venerable, you’ve committed a certain unskillful offense by way of speech. Please give up that bad verbal conduct and develop good verbal conduct.’ When spoken to by their sensible spiritual companions they give up that bad verbal conduct and develop good verbal conduct. These are the things that should be given up by speech, not by the body. 

And\marginnote{4.1} what are the things that should be given up neither by the body, nor by speech, but by seeing again and again with wisdom? Greed … hate … delusion … anger … acrimony … disdain … contempt … and stinginess are things that should be given up neither by the body, nor by speech, but by seeing again and again with wisdom. 

Nasty\marginnote{5.1} jealousy should be given up neither by the body, nor by speech, but by seeing again and again with wisdom. And what is nasty jealousy? It’s when a householder or their child is prospering in money, grain, silver, or gold. And a bondservant or dependent thinks: ‘Oh, may that householder or their child not prosper in money, grain, silver, or gold!’ Or an ascetic or brahmin receives robes, almsfood, lodgings, and medicines and supplies for the sick. And some other ascetic or brahmin thinks: ‘Oh, may that ascetic or brahmin not receive robes, almsfood, lodgings, and medicines and supplies for the sick.’ This is called nasty jealousy. 

Corrupt\marginnote{6.1} wishes should be given up neither by the body, nor by speech, but by seeing again and again with wisdom. And what are corrupt wishes? It’s when a faithless person wishes to be known as faithful. An unethical person wishes to be known as ethical. An unlearned person wishes to be known as learned. A lover of company wishes to be known as secluded. A lazy person wishes to be known as energetic. An unmindful person wishes to be known as mindful. A person without immersion wishes to be known as having immersion. A witless person wishes to be known as wise. A person who has not ended the defilements wishes to be known as having ended the defilements. These are called corrupt wishes. Corrupt wishes should be given up neither by the body, nor by speech, but by seeing again and again with wisdom. 

Suppose\marginnote{7.1} that greed masters that mendicant and keeps going. Or that hate … delusion … anger … acrimony … disdain … contempt … stinginess … nasty jealousy … or corrupt wishes master them and keep going. You should know of them: ‘This venerable does not have the understanding that would eliminate greed, so greed masters them and keeps going. They don’t have the understanding that would eliminate hate … delusion … anger … acrimony … disdain … contempt … stinginess … nasty jealousy … or corrupt wishes, so corrupt wishes master them and keep going.’ 

Suppose\marginnote{8.1} that greed does not master that mendicant and keep going. Or that hate … delusion … anger … acrimony … disdain … contempt … stinginess … nasty jealousy … or corrupt wishes don’t master that mendicant and keep going. You should know of them: ‘This venerable has the understanding that eliminates greed, so greed doesn’t master them and keep going. They have the understanding that eliminates hate … delusion … anger … acrimony … disdain … contempt … stinginess … nasty jealousy … and corrupt wishes, so corrupt wishes don’t master them and keep going.’” 

%
\section*{{\suttatitleacronym AN 10.24}{\suttatitletranslation By Mahācunda }{\suttatitleroot Mahācundasutta}}
\addcontentsline{toc}{section}{\tocacronym{AN 10.24} \toctranslation{By Mahācunda } \tocroot{Mahācundasutta}}
\markboth{By Mahācunda }{Mahācundasutta}
\extramarks{AN 10.24}{AN 10.24}

At\marginnote{1.1} one time Venerable \textsanskrit{Mahācunda} was staying in the land of the \textsanskrit{Cetīs} at \textsanskrit{Sahajāti}. There he addressed the mendicants: “Reverends, mendicants!” 

“Reverend,”\marginnote{1.4} they replied. Venerable \textsanskrit{Mahācunda} said this: 

“Reverends,\marginnote{2.1} a mendicant who makes a declaration of knowledge says: ‘I know this teaching, I see this teaching.’ Suppose that greed masters that mendicant and keeps going. Or that hate … delusion … anger … acrimony … disdain … contempt … stinginess … nasty jealousy … or corrupt wishes master that mendicant and keep going. You should know of them: ‘This venerable does not have the understanding that would eliminate greed, so greed masters them and keeps going. They don’t have the understanding that would eliminate hate … delusion … anger … acrimony … disdain … contempt … stinginess … nasty jealousy … or corrupt wishes, so corrupt wishes master them and keep going.’ 

A\marginnote{3.1} mendicant who makes a declaration of development says: ‘I am developed in physical endurance, ethics, mind, and wisdom.’ Suppose that greed masters that mendicant and keeps going. Or that hate … delusion … anger … acrimony … disdain … contempt … stinginess … nasty jealousy … or corrupt wishes master that mendicant and keep going. You should know of them: ‘This venerable does not have the understanding that would eliminate greed, so greed masters them and keeps going. They don’t have the understanding that would eliminate hate … delusion … anger … acrimony … disdain … contempt … stinginess … nasty jealousy … or corrupt wishes, so corrupt wishes master them and keep going.’ 

A\marginnote{4.1} mendicant who makes a declaration of both knowledge and development says: ‘I know this teaching, I see this teaching. And I am developed in physical endurance, ethics, mind, and wisdom.’ Suppose that greed masters that mendicant and keeps going. Or that hate … delusion … anger … acrimony … disdain … contempt … stinginess … nasty jealousy … or corrupt wishes master that mendicant and keep going. You should know of them: ‘This venerable does not have the understanding that would eliminate greed, so greed masters them and keeps going. They don’t have the understanding that would eliminate hate … delusion … anger … acrimony … disdain … contempt … stinginess … nasty jealousy … or corrupt wishes, so corrupt wishes master them and keep going.’ 

Suppose\marginnote{5.1} a poor, needy, and penniless person was to declare themselves to be rich, affluent, and wealthy. But when it came time to make a payment they weren’t able to come up with any money, grain, silver, or gold. They would know of them: ‘This person declares themselves to be rich, affluent, and wealthy, but they are in fact poor, penniless, and needy.’ Why is that? Because when it came time to make a payment they weren’t able to come up with any money, grain, silver, or gold. 

In\marginnote{6.1} the same way, a mendicant who makes a declaration of knowledge and development says: ‘I know this teaching, I see this teaching. And I am developed in physical endurance, ethics, mind, and wisdom.’ Suppose that greed masters that mendicant and keeps going. Or that hate … delusion … anger … acrimony … disdain … contempt … stinginess … nasty jealousy … or corrupt wishes master that mendicant and keep going. You should know of them: ‘This venerable does not have the understanding that would eliminate greed, so greed masters them and keeps going. They don’t have the understanding that would eliminate hate … delusion … anger … acrimony … disdain … contempt … stinginess … nasty jealousy … or corrupt wishes, so corrupt wishes master them and keep going.’ 

A\marginnote{7.1} mendicant who makes a declaration of knowledge says: ‘I know this teaching, I see this teaching.’ Suppose that greed does not master that mendicant and keep going. Or that hate … delusion … anger … acrimony … disdain … contempt … stinginess … nasty jealousy … or corrupt wishes don’t master that mendicant and keep going. You should know of them: ‘This venerable has the understanding that eliminates greed, so greed doesn’t master them and keep going. They have the understanding that eliminates hate … delusion … anger … acrimony … disdain … contempt … stinginess … nasty jealousy … and corrupt wishes, so corrupt wishes don’t master them and keep going.’ 

A\marginnote{8.1} mendicant who makes a declaration of development says: ‘I am developed in physical endurance, ethics, mind, and wisdom.’ Suppose that greed does not master that mendicant and keep going. Or that hate … delusion … anger … acrimony … disdain … contempt … stinginess … nasty jealousy … or corrupt wishes don’t master that mendicant and keep going. You should know of them: ‘This venerable has the understanding that eliminates greed, so greed doesn’t master them and keep going. They have the understanding that eliminates hate … delusion … anger … acrimony … disdain … contempt … stinginess … nasty jealousy … and corrupt wishes, so corrupt wishes don’t master them and keep going.’ 

A\marginnote{9.1} mendicant who makes a declaration of both knowledge and development says: ‘I know this teaching, I see this teaching. And I am developed in physical endurance, ethics, mind, and wisdom.’ Suppose that greed does not master that mendicant and keep going. Or that hate … delusion … anger … acrimony … disdain … contempt … stinginess … nasty jealousy … or corrupt wishes don’t master that mendicant and keep going. You should know of them: ‘This venerable has the understanding that eliminates greed, so greed doesn’t master them and keep going. They have the understanding that eliminates hate … delusion … anger … acrimony … disdain … contempt … stinginess … nasty jealousy … and corrupt wishes, so corrupt wishes don’t master them and keep going.’ 

Suppose\marginnote{10.1} a rich, affluent, and wealthy person was to declare themselves to be rich, affluent, and wealthy. And whenever it came time to make a payment they were able to come up with the money, grain, silver, or gold. They would know of them: ‘This person declares themselves to be rich, affluent, and wealthy, and they are in fact rich, affluent, and wealthy.’ Why is that? Because when it came time to make a payment they were able to come up with the money, grain, silver, or gold. 

In\marginnote{11.1} the same way, a mendicant who makes a declaration of knowledge and development says: ‘I know this teaching, I see this teaching. And I am developed in physical endurance, ethics, mind, and wisdom.’ Suppose that greed does not master that mendicant and keep going. Or that hate … delusion … anger … acrimony … disdain … contempt … stinginess … nasty jealousy … or corrupt wishes don’t master that mendicant and keep going. You should know of them: ‘This venerable has the understanding that eliminates greed, so greed doesn’t master them and keep going. They have the understanding that eliminates hate … delusion … anger … acrimony … disdain … contempt … stinginess … nasty jealousy … and corrupt wishes, so corrupt wishes don’t master them and keep going.’” 

%
\section*{{\suttatitleacronym AN 10.25}{\suttatitletranslation Meditation on Universals }{\suttatitleroot Kasiṇasutta}}
\addcontentsline{toc}{section}{\tocacronym{AN 10.25} \toctranslation{Meditation on Universals } \tocroot{Kasiṇasutta}}
\markboth{Meditation on Universals }{Kasiṇasutta}
\extramarks{AN 10.25}{AN 10.25}

“Mendicants,\marginnote{1.1} there are these ten universal dimensions of meditation. What ten? Someone perceives the meditation on universal earth above, below, across, undivided and limitless. They perceive the meditation on universal water … the meditation on universal fire … the meditation on universal air … the meditation on universal blue … the meditation on universal yellow … the meditation on universal red … the meditation on universal white … the meditation on universal space … They perceive the meditation on universal consciousness above, below, across, undivided and limitless. These are the ten universal dimensions of meditation.” 

%
\section*{{\suttatitleacronym AN 10.26}{\suttatitletranslation With Kāḷī }{\suttatitleroot Kāḷīsutta}}
\addcontentsline{toc}{section}{\tocacronym{AN 10.26} \toctranslation{With Kāḷī } \tocroot{Kāḷīsutta}}
\markboth{With Kāḷī }{Kāḷīsutta}
\extramarks{AN 10.26}{AN 10.26}

At\marginnote{1.1} one time Venerable \textsanskrit{Mahākaccāna} was staying in the land of the Avantis near Kuraraghara on Steep Mountain. 

Then\marginnote{1.2} the laywoman \textsanskrit{Kāḷī} of Kurughara went up to Venerable \textsanskrit{Mahākaccāna}, bowed, sat down to one side, and said to him, “Sir, this was said by the Buddha in ‘The Maidens’ Questions’: 

\begin{verse}%
‘I’ve\marginnote{2.1} reached the goal, peace of heart. \\
Having conquered the army \\>of the likable and pleasant, \\
alone, practicing absorption, I awakened to bliss. \\
That’s why I don’t get too close to people, \\
and no-one gets too close to me.’ 

%
\end{verse}

How\marginnote{3.1} should we see the detailed meaning of the Buddha’s brief statement?” 

“Sister,\marginnote{4.1} some ascetics and brahmins regard the attainment of the meditation on universal earth to be the ultimate. Thinking ‘this is the goal’, they are reborn. The Buddha directly knew the extent to which the attainment of the meditation on universal earth was the ultimate. Directly knowing this he saw the beginning, the drawback, and the escape. And he saw the knowledge and vision of what is the path and what is not the path. Because he saw the beginning, the drawback, and the escape, and he saw the knowledge and vision of what is the path and what is not the path, he knew that he had reached the goal, peace of heart. 

Some\marginnote{5.1} ascetics and brahmins regard the attainment of the meditation on universal water to be the ultimate. Thinking ‘this is the goal’, they are reborn. … Some ascetics and brahmins regard the attainment of the meditation on universal fire … universal air … universal blue … universal yellow … universal red … universal white … universal space … universal consciousness to be the ultimate. Thinking ‘this is the goal’, they are reborn. The Buddha directly knew the extent to which the attainment of the meditation on universal consciousness was the ultimate. Directly knowing this he saw the beginning, the drawback, and the escape. And he saw the knowledge and vision of what is the path and what is not the path. Because he saw the beginning, the drawback, and the escape, and he saw the knowledge and vision of what is the path and what is not the path, he knew that he had reached the goal, peace of heart. 

So,\marginnote{5.13} sister, that’s how to understand the detailed meaning of what the Buddha said in brief in ‘The Maiden’s Questions’: 

\begin{verse}%
‘I’ve\marginnote{6.1} reached the goal, peace of heart. \\
Having conquered the army \\>of the likable and pleasant, \\
alone, practicing absorption, I awakened to bliss. \\
That’s why I don’t get too close to people, \\
and no-one gets too close to me.’” 

%
\end{verse}

%
\section*{{\suttatitleacronym AN 10.27}{\suttatitletranslation The Great Questions (1st) }{\suttatitleroot Paṭhamamahāpañhāsutta}}
\addcontentsline{toc}{section}{\tocacronym{AN 10.27} \toctranslation{The Great Questions (1st) } \tocroot{Paṭhamamahāpañhāsutta}}
\markboth{The Great Questions (1st) }{Paṭhamamahāpañhāsutta}
\extramarks{AN 10.27}{AN 10.27}

At\marginnote{1.1} one time the Buddha was staying near \textsanskrit{Sāvatthī} in Jeta’s Grove, \textsanskrit{Anāthapiṇḍika}’s monastery. 

Then\marginnote{1.2} several mendicants robed up in the morning and, taking their bowls and robes, entered \textsanskrit{Sāvatthī} for alms. Then it occurred to him, “It’s too early to wander for alms in \textsanskrit{Sāvatthī}. Why don’t we visit the monastery of the wanderers of other religions?” 

Then\marginnote{2.1} they went to the monastery of the wanderers of other religions and exchanged greetings with the wanderers there. When the greetings and polite conversation were over, they sat down to one side. The wanderers said to them: 

“Reverends,\marginnote{3.1} the ascetic Gotama teaches his disciples like this: ‘Please, mendicants, directly know all things. Meditate having directly known all things.’ We too teach our disciples: ‘Please, reverends, directly know all things. Live having directly known all things.’ What, then, is the difference between the ascetic Gotama’s teaching and instruction and ours?” 

Those\marginnote{4.1} mendicants neither approved nor dismissed that statement of the wanderers of other religions. They got up from their seat, thinking, “We will learn the meaning of this statement from the Buddha himself.” 

Then,\marginnote{5.1} after the meal, when they returned from almsround, they went up to the Buddha, bowed, sat down to one side, and told him what had happened. 

“Mendicants,\marginnote{9.1} when wanderers of other religions say this, you should say to them: ‘One thing: question, summary recital, and answer. Two … three … four … five … six … seven … eight … nine … ten things: question, summary recital, and answer.’ Questioned like this, the wanderers of other religions would be stumped, and, in addition, would get frustrated. Why is that? Because they’re out of their element. I don’t see anyone in this world—with its gods, \textsanskrit{Māras}, and Divinities, this population with its ascetics and brahmins, its gods and humans—who could provide a satisfying answer to these questions except for the Realized One or his disciple or someone who has heard it from them. 

‘One\marginnote{10.1} thing: question, summary recital, and answer.’ That’s what I said, but why did I say it? Becoming completely disillusioned, dispassionate, and freed regarding one thing, seeing its limits and fully comprehending its meaning, a mendicant makes an end of suffering in this very life. What one thing? ‘All sentient beings are sustained by food.’ Becoming completely disillusioned, dispassionate, and freed regarding this one thing, seeing its limits and fully comprehending its meaning, a mendicant makes an end of suffering in this very life. ‘One thing: question, summary recital, and answer.’ That’s what I said, and this is why I said it. 

What\marginnote{11.1} two? Name and form. … 

What\marginnote{12.1} three? Three feelings. … 

What\marginnote{13.1} four? The four foods. … 

What\marginnote{14.1} five? The five grasping aggregates. … 

What\marginnote{15.1} six? The six interior sense fields. … 

What\marginnote{16.1} seven? The seven planes of consciousness. … 

What\marginnote{17.1} eight? The eight worldly conditions. … 

What\marginnote{18.1} nine? The nine abodes of sentient beings. … 

‘Ten\marginnote{19.1} things: question, summary recital, and answer.’ That’s what I said, but why did I say it? Becoming completely disillusioned, dispassionate, and freed regarding ten things, seeing their limits and fully comprehending their meaning, a mendicant makes an end of suffering in this very life. What ten? The ten ways of performing unskillful deeds. Becoming completely disillusioned, dispassionate, and freed regarding these ten things, seeing their limits and fully comprehending their meaning, a mendicant makes an end of suffering in this very life. ‘Ten things: question, summary recital, and answer.’ That’s what I said, and this is why I said it.” 

%
\section*{{\suttatitleacronym AN 10.28}{\suttatitletranslation The Great Questions (2nd) }{\suttatitleroot Dutiyamahāpañhāsutta}}
\addcontentsline{toc}{section}{\tocacronym{AN 10.28} \toctranslation{The Great Questions (2nd) } \tocroot{Dutiyamahāpañhāsutta}}
\markboth{The Great Questions (2nd) }{Dutiyamahāpañhāsutta}
\extramarks{AN 10.28}{AN 10.28}

At\marginnote{1.1} one time the Buddha was staying near \textsanskrit{Kajaṅgalā} in a bamboo grove. Then several lay followers of \textsanskrit{Kajaṅgalā} went to the nun \textsanskrit{Kajaṅgalikā}, bowed, sat down to one side, and said to her: 

“Ma’am,\marginnote{2.1} this was said by the Buddha in ‘The Great Questions’: ‘One thing: question, summary recital, and answer. Two … three … four … five … six … seven … eight … nine … ten things: question, summary recital, and answer.’ How should we see the detailed meaning of the Buddha’s brief statement?” 

“Good\marginnote{3.1} people, I haven’t heard and learned this in the presence of the Buddha or from esteemed mendicants. But as to how it seems to me, listen and apply your mind well, I will speak.” 

“Yes,\marginnote{3.4} ma’am,” replied the lay followers. The nun \textsanskrit{Kajaṅgalikā} said this: 

“‘One\marginnote{4.1} thing: question, summary recital, and answer.’ That’s what the Buddha said, but why did he say it? Becoming completely disillusioned, dispassionate, and freed regarding one thing, seeing its limits and fully comprehending its meaning, a mendicant makes an end of suffering in this very life. What one thing? ‘All sentient beings are sustained by food.’ Becoming completely disillusioned, dispassionate, and freed regarding this one thing, seeing its limits and fully comprehending its meaning, a mendicant makes an end of suffering in this very life. ‘One thing: question, summary recital, and answer.’ That’s what the Buddha said, and this is why he said it. 

What\marginnote{5.1} two? Name and form. … What three? Three feelings. … 

With\marginnote{6.1} a mind well developed in four things—seeing their limits and fully comprehending their meaning—a mendicant makes an end of suffering in this very life. What four? The four kinds of mindfulness meditation. … With a mind well developed in these four things—seeing their limits and fully fathoming their meaning—a mendicant makes an end of suffering in this very life. … 

What\marginnote{7.1} five? The five faculties. … What six? The six elements of escape. … What seven? The seven awakening factors. … What eight? The noble eightfold path. … 

Becoming\marginnote{8.1} completely disillusioned, dispassionate, and freed regarding nine things, seeing their limits and fully comprehending their meaning, a mendicant makes an end of suffering in this very life. What nine? The nine abodes of sentient beings. Becoming completely disillusioned, dispassionate, and freed regarding these nine things, seeing their limits and fully comprehending their meaning, a mendicant makes an end of suffering in this very life. 

‘Ten\marginnote{9.1} things: question, summary recital, and answer.’ That’s what the Buddha said, but why did he say it? Becoming well developed in ten things—seeing their limits and fully fathoming their meaning—a mendicant makes an end of suffering in this very life. What ten? The ten ways of performing skillful deeds. With a mind well developed in these ten things—seeing their limits and fully fathoming their meaning—a mendicant makes an end of suffering in this very life. ‘Ten things: question, summary recital, and answer.’ That’s what the Buddha said, and this is why he said it. 

That’s\marginnote{10.1} how I understand the detailed meaning of what the Buddha said in brief in ‘The Great Questions’. If you wish, you may go to the Buddha and ask him about this. You should remember it in line with the Buddha’s answer.” 

“Yes,\marginnote{10.6} ma’am,” replied those lay followers, approving and agreeing with what the nun \textsanskrit{Kajaṅgalikā} said. Then they got up from their seat, bowed, and respectfully circled her, keeping her on their right. Then they went to the Buddha, bowed, sat down to one side, and informed the Buddha of all they had discussed. 

“Good,\marginnote{11.1} good, householders. The nun \textsanskrit{Kajaṅgalikā} is astute, she has great wisdom. If you came to me and asked this question, I would answer it in exactly the same way as the nun \textsanskrit{Kajaṅgalikā}. That is what it means, and that’s how you should remember it.” 

%
\section*{{\suttatitleacronym AN 10.29}{\suttatitletranslation Kosala (1st) }{\suttatitleroot Paṭhamakosalasutta}}
\addcontentsline{toc}{section}{\tocacronym{AN 10.29} \toctranslation{Kosala (1st) } \tocroot{Paṭhamakosalasutta}}
\markboth{Kosala (1st) }{Paṭhamakosalasutta}
\extramarks{AN 10.29}{AN 10.29}

“As\marginnote{1.1} far as \textsanskrit{Kāsi} and Kosala extend, and as far as the dominion of King Pasenadi of Kosala extends, King Pasenadi is said to be the foremost. But even King Pasenadi decays and perishes. 

Seeing\marginnote{1.3} this, a learned noble disciple grows disillusioned with it. Their desire fades away even for the foremost, let alone the inferior. 

A\marginnote{2.1} galaxy extends a thousand times as far as the moon and sun revolve and the shining ones light up the quarters. In that galaxy there are a thousand moons, a thousand suns, a thousand Sinerus king of mountains, a thousand Black Plum Tree Lands, a thousand Western Continents, a thousand Northern Continents, a thousand Eastern Continents, four thousand oceans, four thousand great kings, a thousand realms of the gods of the four great kings, a thousand realms of the gods of the thirty-three, of the gods of Yama, of the joyful gods, of the gods who love to imagine, of the gods who control what is imagined by others, and a thousand realms of divinity. As far as the galaxy extends, the Great Divinity is said to be the foremost. But even the Great Divinity decays and perishes. 

Seeing\marginnote{2.5} this, a learned noble disciple grows disillusioned with it. Their desire fades away even for the foremost, let alone the inferior. 

There\marginnote{3.1} comes a time when this cosmos contracts. As it contracts, most sentient beings migrate to the realm of streaming radiance. There they are mind-made, feeding on rapture, self-luminous, wandering in midair, steadily glorious, and they remain like that for a very long time. When the cosmos is contracting, the gods of streaming radiance are said to be the foremost. But even the gods of streaming radiance decay and perish. 

Seeing\marginnote{3.6} this, a learned noble disciple grows disillusioned with it. Their desire fades away even for the foremost, let alone the inferior. 

There\marginnote{4.1} are these ten universal dimensions of meditation. What ten? Someone perceives the meditation on universal earth above, below, across, undivided and limitless. They perceive the meditation on universal water … the meditation on universal fire … the meditation on universal air … the meditation on universal blue … the meditation on universal yellow … the meditation on universal red … the meditation on universal white … the meditation on universal space … They perceive the meditation on universal consciousness above, below, across, undivided and limitless. These are the ten universal dimensions of meditation. 

The\marginnote{5.1} best of these ten universal dimensions of meditation is when someone perceives the meditation on universal consciousness above, below, across, undivided and limitless. Some sentient beings perceive like this. But even the sentient beings who perceive like this decay and perish. 

Seeing\marginnote{5.4} this, a learned noble disciple grows disillusioned with it. Their desire fades away even for the foremost, let alone the inferior. 

There\marginnote{6.1} are these eight dimensions of mastery. What eight? Perceiving form internally, someone sees forms externally, limited, both pretty and ugly. Mastering them, they perceive: ‘I know and see.’ This is the first dimension of mastery. 

Perceiving\marginnote{7.1} form internally, someone sees forms externally, limitless, both pretty and ugly. Mastering them, they perceive: ‘I know and see.’ This is the second dimension of mastery. 

Not\marginnote{8.1} perceiving form internally, someone sees forms externally, limited, both pretty and ugly. Mastering them, they perceive: ‘I know and see.’ This is the third dimension of mastery. 

Not\marginnote{9.1} perceiving form internally, someone sees forms externally, limitless, both pretty and ugly. Mastering them, they perceive: ‘I know and see.’ This is the fourth dimension of mastery. 

Not\marginnote{10.1} perceiving form internally, someone sees forms externally, blue, with blue color and blue appearance. They’re like a flax flower that’s blue, with blue color and blue appearance. Or a cloth from Varanasi that’s smoothed on both sides, blue, with blue color and blue appearance. In the same way, not perceiving form internally, someone sees forms externally, blue, with blue color and blue appearance. Mastering them, they perceive: ‘I know and see.’ This is the fifth dimension of mastery. 

Not\marginnote{11.1} perceiving form internally, someone sees forms externally, yellow, with yellow color and yellow appearance. They’re like a champak flower that’s yellow, with yellow color and yellow appearance. Or a cloth from Varanasi that’s smoothed on both sides, yellow, with yellow color and yellow appearance. In the same way, not perceiving form internally, someone sees forms externally, yellow, with yellow color and yellow appearance. Mastering them, they perceive: ‘I know and see.’ This is the sixth dimension of mastery. 

Not\marginnote{12.1} perceiving form internally, someone sees forms externally, red, with red color and red appearance. They’re like a scarlet mallow flower that’s red, with red color and red appearance. Or a cloth from Varanasi that’s smoothed on both sides, red, with red color and red appearance. In the same way, not perceiving form internally, someone sees forms externally, red, with red color and red appearance. Mastering them, they perceive: ‘I know and see.’ This is the seventh dimension of mastery. 

Not\marginnote{13.1} perceiving form internally, someone sees forms externally, white, with white color and white appearance. They’re like the morning star that’s white, with white color and white appearance. Or a cloth from Varanasi that’s smoothed on both sides, white, with white color and white appearance. In the same way, not perceiving form internally, someone sees forms externally, white, with white color and white appearance. Mastering them, they perceive: ‘I know and see.’ This is the eighth dimension of mastery. These are the eight dimensions of mastery. 

The\marginnote{14.1} best of these dimensions of mastery is when someone, not perceiving form internally, sees forms externally, white, with white color and white appearance. Mastering them, they perceive: ‘I know and see.’ Some sentient beings perceive like this. But even the sentient beings who perceive like this decay and perish. 

Seeing\marginnote{14.5} this, a learned noble disciple grows disillusioned with it. Their desire fades away even for the foremost, let alone the inferior. 

There\marginnote{15.1} are four ways of practice. What four? 

\begin{enumerate}%
\item Painful practice with slow insight, %
\item painful practice with swift insight, %
\item pleasant practice with slow insight, and %
\item pleasant practice with swift insight. %
\end{enumerate}

These\marginnote{15.7} are the four ways of practice. 

The\marginnote{16.1} best of these four ways of practice is the pleasant practice with swift insight. Some sentient beings practice like this. But even the sentient beings who practice like this decay and perish. 

Seeing\marginnote{16.4} this, a learned noble disciple grows disillusioned with it. Their desire fades away even for the foremost, let alone the inferior. 

There\marginnote{17.1} are these four perceptions. What four? 

\begin{enumerate}%
\item One person perceives the limited. %
\item One person perceives the expansive. %
\item One person perceives the limitless. %
\item One person, aware that ‘there is nothing at all’, perceives the dimension of nothingness. %
\end{enumerate}

These\marginnote{17.7} are the four perceptions. 

The\marginnote{18.1} best of these four perceptions is when a person, aware that ‘there is nothing at all’, perceives the dimension of nothingness. Some sentient beings perceive like this. But even the sentient beings who perceive like this decay and perish. 

Seeing\marginnote{18.4} this, a learned noble disciple grows disillusioned with it. Their desire fades away even for the foremost, let alone the inferior. 

This\marginnote{19.1} is the best of the convictions of outsiders, that is: ‘I might not be, and it might not be mine. I will not be, and it will not be mine.’ When someone has such a view, you can expect that they will not be attracted to continued existence, and they will not be repulsed by the cessation of continued existence. Some sentient beings have such a view. But even the sentient beings who have views like this decay and perish. 

Seeing\marginnote{19.7} this, a learned noble disciple grows disillusioned with it. Their desire fades away even for the foremost, let alone the inferior. 

There\marginnote{20.1} are some ascetics and brahmins who advocate the ultimate purity of the spirit. This is the best of the advocates of the ultimate purity of the spirit, that is, when someone, going totally beyond the dimension of nothingness, enters and remains in the dimension of neither perception nor non-perception. They teach Dhamma in order to directly know and realize this. Some sentient beings have such a doctrine. But even the sentient beings who have such a doctrine decay and perish. 

Seeing\marginnote{20.6} this, a learned noble disciple grows disillusioned with it. Their desire fades away even for the foremost, let alone the inferior. 

There\marginnote{21.1} are some ascetics and brahmins who advocate ultimate extinguishment in this very life. This is the best of those who advocate extinguishment in this very life, that is, liberation by not grasping after truly understanding the origin, ending, gratification, drawback, and escape of the six fields of contact. Though I state and assert this, certain ascetics and brahmins misrepresent me with the incorrect, hollow, false, untruthful claim: ‘The ascetic Gotama doesn’t advocate the complete understanding of sensual pleasures, forms, or feelings.’ But I do advocate the complete understanding of sensual pleasures, forms, and feelings. And I advocate full extinguishment by not grasping in this very life, wishless, quenched, and cooled.” 

%
\section*{{\suttatitleacronym AN 10.30}{\suttatitletranslation Kosala (2nd) }{\suttatitleroot Dutiyakosalasutta}}
\addcontentsline{toc}{section}{\tocacronym{AN 10.30} \toctranslation{Kosala (2nd) } \tocroot{Dutiyakosalasutta}}
\markboth{Kosala (2nd) }{Dutiyakosalasutta}
\extramarks{AN 10.30}{AN 10.30}

At\marginnote{1.1} one time the Buddha was staying near \textsanskrit{Sāvatthī} in Jeta’s Grove, \textsanskrit{Anāthapiṇḍika}’s monastery. 

Now\marginnote{1.2} at that time King Pasenadi of Kosala returned from combat after winning a battle and succeeding in his objective. Then King Pasenadi of Kosala went to the monastery. He went by carriage as far as the terrain allowed, then descended and entered the monastery on foot. 

At\marginnote{1.5} that time several mendicants were walking mindfully in the open air. Then King Pasenadi of Kosala went up to them and said, “Sirs, where is the Blessed One at present, the perfected one, the fully awakened Buddha? For I want to see the Buddha.” 

“Great\marginnote{1.9} king, that’s his dwelling, with the door closed. Approach it quietly, without hurrying; go onto the porch, clear your throat, and knock on the door-panel. The Buddha will open the door.” 

So\marginnote{2.1} the king approached the Buddha’s dwelling, cleared his throat and knocked on the door-panel, and the Buddha opened the door. Then King Pasenadi entered the Buddha’s dwelling. He bowed with his head at the Buddha’s feet, caressing them and covering them with kisses, and pronounced his name: “Sir, I am Pasenadi, king of Kosala! I am Pasenadi, king of Kosala!” 

“But\marginnote{3.1} great king, for what reason do you demonstrate such utmost devotion for this body, conveying your manifest love?” 

“Sir,\marginnote{3.2} it’s because of my gratitude and thanks for the Buddha that I demonstrate such utmost devotion, conveying my manifest love. 

The\marginnote{4.1} Buddha is practicing for the welfare and happiness of the people. He has established many people in the noble system, that is, the principles of goodness and skillfulness. This is a reason that I demonstrate such utmost devotion for the Buddha, conveying my manifest love. 

Furthermore,\marginnote{5.1} the Buddha is ethical, possessing ethical conduct that is mature, noble, and skillful. This is another reason that I demonstrate such utmost devotion for the Buddha, conveying my manifest love. 

Furthermore,\marginnote{6.1} the Buddha lives in the wilderness, frequenting remote lodgings in the wilderness and the forest. This is another reason that I demonstrate such utmost devotion for the Buddha, conveying my manifest love. 

Furthermore,\marginnote{7.1} the Buddha is content with any kind of robes, almsfood, lodgings, and medicines and supplies for the sick. This is another reason that I demonstrate such utmost devotion for the Buddha, conveying my manifest love. 

Furthermore,\marginnote{8.1} the Buddha is worthy of offerings dedicated to the gods, worthy of hospitality, worthy of a religious donation, worthy of greeting with joined palms, and is the supreme field of merit for the world. This is another reason that I demonstrate such utmost devotion for the Buddha, conveying my manifest love. 

Furthermore,\marginnote{9.1} the Buddha gets to take part in talk about self-effacement that helps open the heart, when he wants, without trouble or difficulty. That is, talk about fewness of wishes, contentment, seclusion, aloofness, arousing energy, ethics, immersion, wisdom, freedom, and the knowledge and vision of freedom. This is another reason that I demonstrate such utmost devotion for the Buddha, conveying my manifest love. 

Furthermore,\marginnote{10.1} the Buddha gets the four absorptions—blissful meditations in this life that belong to the higher mind—when he wants, without trouble or difficulty. This is another reason that I demonstrate such utmost devotion for the Buddha, conveying my manifest love. 

Furthermore,\marginnote{11.1} the Buddha recollects many kinds of past lives. That is: one, two, three, four, five, ten, twenty, thirty, forty, fifty, a hundred, a thousand, a hundred thousand rebirths; many eons of the world contracting, many eons of the world expanding, many eons of the world contracting and expanding. He remembers: ‘There, I was named this, my clan was that, I looked like this, and that was my food. This was how I felt pleasure and pain, and that was how my life ended. When I passed away from that place I was reborn somewhere else. There, too, I was named this, my clan was that, I looked like this, and that was my food. This was how I felt pleasure and pain, and that was how my life ended. When I passed away from that place I was reborn here.’ And so he recollects his many kinds of past lives, with features and details. This is another reason that I demonstrate such utmost devotion for the Buddha, conveying my manifest love. 

Furthermore,\marginnote{12.1} with clairvoyance that is purified and superhuman, the Buddha sees sentient beings passing away and being reborn—inferior and superior, beautiful and ugly, in a good place or a bad place. He understands how sentient beings are reborn according to their deeds. ‘These dear beings did bad things by way of body, speech, and mind. They denounced the noble ones; they had wrong view; and they chose to act out of that wrong view. When their body breaks up, after death, they’re reborn in a place of loss, a bad place, the underworld, hell. These dear beings, however, did good things by way of body, speech, and mind. They never denounced the noble ones; they had right view; and they chose to act out of that right view. When their body breaks up, after death, they’re reborn in a good place, a heavenly realm.’ He understands how sentient beings are reborn according to their deeds. This is another reason that I demonstrate such utmost devotion for the Buddha, conveying my manifest love. 

Furthermore,\marginnote{13.1} the Buddha has realized the undefiled freedom of heart and freedom by wisdom in this very life, and lives having realized it with his own insight due to the ending of defilements. This is another reason that I demonstrate such utmost devotion for the Buddha, conveying my manifest love. 

Well,\marginnote{14.1} now, sir, I must go. I have many duties, and much to do.” 

“Please,\marginnote{14.3} great king, go at your convenience.” Then King Pasenadi got up from his seat, bowed, and respectfully circled the Buddha, keeping him on his right, before leaving. 

%
\addtocontents{toc}{\let\protect\contentsline\protect\nopagecontentsline}
\chapter*{The Chapter with Upāli }
\addcontentsline{toc}{chapter}{\tocchapterline{The Chapter with Upāli }}
\addtocontents{toc}{\let\protect\contentsline\protect\oldcontentsline}

%
\section*{{\suttatitleacronym AN 10.31}{\suttatitletranslation With Upāli }{\suttatitleroot Upālisutta}}
\addcontentsline{toc}{section}{\tocacronym{AN 10.31} \toctranslation{With Upāli } \tocroot{Upālisutta}}
\markboth{With Upāli }{Upālisutta}
\extramarks{AN 10.31}{AN 10.31}

Then\marginnote{1.1} Venerable \textsanskrit{Upāli} went up to the Buddha, bowed, sat down to one side, and said to him: 

“Sir,\marginnote{1.2} for how many reasons did the Realized One lay down training rules for his disciples and recite the monastic code?” 

“\textsanskrit{Upāli},\marginnote{2.1} the Realized One laid down training rules for his disciples and recited the monastic code for ten reasons. What ten? For the well-being of the \textsanskrit{Saṅgha} and for the comfort of the \textsanskrit{Saṅgha}. For keeping difficult persons in check and for the comfort of good-hearted mendicants. For restraining defilements that affect this life and protecting against defilements that affect lives to come. For inspiring confidence in those without it, and increasing confidence in those who have it. For the continuation of the true teaching and the support of the training. The Realized One laid down training rules for his disciples and recited the monastic code for these ten reasons.” 

%
\section*{{\suttatitleacronym AN 10.32}{\suttatitletranslation Suspending the Recitation of the Monastic Code }{\suttatitleroot Pātimokkhaṭṭhapanāsutta}}
\addcontentsline{toc}{section}{\tocacronym{AN 10.32} \toctranslation{Suspending the Recitation of the Monastic Code } \tocroot{Pātimokkhaṭṭhapanāsutta}}
\markboth{Suspending the Recitation of the Monastic Code }{Pātimokkhaṭṭhapanāsutta}
\extramarks{AN 10.32}{AN 10.32}

“Sir,\marginnote{1.1} how many grounds are there to suspend the recitation of the monastic code?” 

“\textsanskrit{Upāli},\marginnote{1.2} there are ten grounds to suspend the recitation of the monastic code. What ten? A mendicant who has committed an expulsion offense is sitting in the assembly. A discussion about whether someone has committed an expulsion offense is unfinished. A person who is not fully ordained is sitting in the assembly. A discussion about whether someone is fully ordained or not is unfinished. Someone who has resigned the training is sitting in the assembly. A discussion about whether someone has rejected the training or not is unfinished. A eunuch is sitting in the assembly. A discussion about whether someone is a eunuch is unfinished. A raper of nuns is sitting in the assembly. A discussion about whether or not someone is a raper of nuns is unfinished. These are the ten grounds to suspend the recitation of the monastic code.” 

%
\section*{{\suttatitleacronym AN 10.33}{\suttatitletranslation A Judge }{\suttatitleroot Ubbāhikāsutta}}
\addcontentsline{toc}{section}{\tocacronym{AN 10.33} \toctranslation{A Judge } \tocroot{Ubbāhikāsutta}}
\markboth{A Judge }{Ubbāhikāsutta}
\extramarks{AN 10.33}{AN 10.33}

“Sir,\marginnote{1.1} how many qualities should a mendicant have to be deemed a judge?” 

“\textsanskrit{Upāli},\marginnote{1.2} a mendicant should have ten qualities to be deemed a judge. What ten? It’s when a mendicant is ethical, restrained in the monastic code, conducting themselves well and resorting for alms in suitable places. Seeing danger in the slightest fault, they keep the rules they’ve undertaken. They’re very learned, remembering and keeping what they’ve learned. These teachings are good in the beginning, good in the middle, and good in the end, meaningful and well-phrased, describing a spiritual practice that’s entirely full and pure. They are very learned in such teachings, remembering them, rehearsing them, mentally scrutinizing them, and comprehending them theoretically. Both monastic codes have been passed down to them in detail, well analyzed, well mastered, well evaluated in both the rules and accompanying material. They’re firm and unfaltering in the training. When there are opposing parties, they’re able to persuade, advocate, and convince them, make them see the other side and trust each other. They’re skilled in raising and settling disciplinary issues. They know what a disciplinary issue is. They know how a disciplinary issue originates. They know how a disciplinary issue ceases. They know the practical way leading to the cessation of a disciplinary issue. A mendicant should have these ten qualities to be deemed a judge.” 

%
\section*{{\suttatitleacronym AN 10.34}{\suttatitletranslation Ordination }{\suttatitleroot Upasampadāsutta}}
\addcontentsline{toc}{section}{\tocacronym{AN 10.34} \toctranslation{Ordination } \tocroot{Upasampadāsutta}}
\markboth{Ordination }{Upasampadāsutta}
\extramarks{AN 10.34}{AN 10.34}

“Sir,\marginnote{1.1} how many qualities should a mendicant have to give ordination?” 

“\textsanskrit{Upāli},\marginnote{1.2} a mendicant should have ten qualities to give ordination. What ten? It’s when a mendicant is ethical, restrained in the monastic code, conducting themselves well and resorting for alms in suitable places. Seeing danger in the slightest fault, they keep the rules they’ve undertaken. They’re very learned, remembering and keeping what they’ve learned. These teachings are good in the beginning, good in the middle, and good in the end, meaningful and well-phrased, describing a spiritual practice that’s entirely full and pure. They are very learned in such teachings, remembering them, rehearsing them, mentally scrutinizing them, and comprehending them theoretically. Both monastic codes have been passed down to them in detail, well analyzed, well mastered, well evaluated in both the rules and accompanying material. They’re able to care for the sick or get someone else to do so. They’re able to settle dissatisfaction or get someone else to do so. They’re able to dispel remorse when it has come up. They’re able to rationally dissuade someone from misconceptions that come up. They’re able to encourage someone in the higher ethics, the higher mind, and the higher wisdom. A mendicant should have these ten qualities to give ordination.” 

%
\section*{{\suttatitleacronym AN 10.35}{\suttatitletranslation Dependence }{\suttatitleroot Nissayasutta}}
\addcontentsline{toc}{section}{\tocacronym{AN 10.35} \toctranslation{Dependence } \tocroot{Nissayasutta}}
\markboth{Dependence }{Nissayasutta}
\extramarks{AN 10.35}{AN 10.35}

“Sir,\marginnote{1.1} how many qualities should a mendicant have to give dependence?” 

“\textsanskrit{Upāli},\marginnote{1.2} a mendicant should have ten qualities to give dependence. What ten? It’s when a mendicant is ethical … They’re learned … Both monastic codes have been passed down to them in detail, well analyzed, well mastered, well evaluated in both the rules and accompanying material. They’re able to care for the sick or get someone else to do so. They’re able to settle dissatisfaction or get someone else to do so. They’re able to dispel remorse when it has come up. They’re able to rationally dissuade someone from misconceptions that come up. They’re able to encourage someone in the higher ethics, the higher mind, and the higher wisdom. A mendicant should have these ten qualities to give dependence.” 

%
\section*{{\suttatitleacronym AN 10.36}{\suttatitletranslation A Novice }{\suttatitleroot Sāmaṇerasutta}}
\addcontentsline{toc}{section}{\tocacronym{AN 10.36} \toctranslation{A Novice } \tocroot{Sāmaṇerasutta}}
\markboth{A Novice }{Sāmaṇerasutta}
\extramarks{AN 10.36}{AN 10.36}

“Sir,\marginnote{1.1} how many qualities should a mendicant have to be attended on by a novice?” 

“\textsanskrit{Upāli},\marginnote{1.2} a mendicant should have ten qualities to be attended on by a novice. What ten? It’s when a mendicant is ethical … They’re learned … Both monastic codes have been passed down to them in detail, well analyzed, well mastered, well evaluated in both the rules and accompanying material. They’re able to care for the sick or get someone else to do so. They’re able to settle dissatisfaction or get someone else to do so. They’re able to dispel remorse when it has come up. They’re able to rationally dissuade someone from misconceptions that come up. They’re able to encourage someone in the higher ethics, the higher mind, and the higher wisdom. A mendicant should have these ten qualities to be attended on by a novice.” 

%
\section*{{\suttatitleacronym AN 10.37}{\suttatitletranslation Schism in the Saṅgha }{\suttatitleroot Saṁghabhedasutta}}
\addcontentsline{toc}{section}{\tocacronym{AN 10.37} \toctranslation{Schism in the Saṅgha } \tocroot{Saṁghabhedasutta}}
\markboth{Schism in the Saṅgha }{Saṁghabhedasutta}
\extramarks{AN 10.37}{AN 10.37}

“Sir,\marginnote{1.1} they speak of ‘schism in the \textsanskrit{Saṅgha}’. How is schism in the \textsanskrit{Saṅgha} defined?” 

“\textsanskrit{Upāli},\marginnote{1.3} it’s when a mendicant explains what is not the teaching as the teaching, and what is the teaching as not the teaching. They explain what is not the training as the training, and what is the training as not the training. They explain what was not spoken and stated by the Realized One as spoken and stated by the Realized One, and what was spoken and stated by the Realized One as not spoken and stated by the Realized One. They explain what was not practiced by the Realized One as practiced by the Realized One, and what was practiced by the Realized One as not practiced by the Realized One. They explain what was not prescribed by the Realized One as prescribed by the Realized One, and what was prescribed by the Realized One as not prescribed by the Realized One. On these ten grounds they split off and go their own way. They perform legal acts autonomously and recite the monastic code autonomously. That is how schism in the \textsanskrit{Saṅgha} is defined.” 

%
\section*{{\suttatitleacronym AN 10.38}{\suttatitletranslation Harmony in the Saṅgha }{\suttatitleroot Saṁghasāmaggīsutta}}
\addcontentsline{toc}{section}{\tocacronym{AN 10.38} \toctranslation{Harmony in the Saṅgha } \tocroot{Saṁghasāmaggīsutta}}
\markboth{Harmony in the Saṅgha }{Saṁghasāmaggīsutta}
\extramarks{AN 10.38}{AN 10.38}

“Sir,\marginnote{1.1} they speak of ‘harmony in the \textsanskrit{Saṅgha}’. How is harmony in the \textsanskrit{Saṅgha} defined?” 

“\textsanskrit{Upāli},\marginnote{1.3} it’s when a mendicant explains what is not the teaching as not the teaching, and what is the teaching as the teaching. They explain what is not the training as not the training, and what is the training as the training. They explain what was not spoken and stated by the Realized One as not spoken and stated by the Realized One, and what was spoken and stated by the Realized One as spoken and stated by the Realized One. They explain what was not practiced by the Realized One as not practiced by the Realized One, and what was practiced by the Realized One as practiced by the Realized One. They explain what was not prescribed by the Realized One as not prescribed by the Realized One, and what was prescribed by the Realized One as prescribed by the Realized One. On these ten grounds they don’t split off and go their own way. They don’t perform legal acts autonomously or recite the monastic code autonomously. That is how harmony in the \textsanskrit{Saṅgha} is defined.” 

%
\section*{{\suttatitleacronym AN 10.39}{\suttatitletranslation With Ānanda (1st) }{\suttatitleroot Paṭhamaānandasutta}}
\addcontentsline{toc}{section}{\tocacronym{AN 10.39} \toctranslation{With Ānanda (1st) } \tocroot{Paṭhamaānandasutta}}
\markboth{With Ānanda (1st) }{Paṭhamaānandasutta}
\extramarks{AN 10.39}{AN 10.39}

Then\marginnote{1.1} Venerable Ānanda went up to the Buddha, bowed, sat down to one side, and said to the Buddha: 

“Sir,\marginnote{1.2} they speak of ‘schism in the \textsanskrit{Saṅgha}’. How is schism in the \textsanskrit{Saṅgha} defined?” 

“Ānanda,\marginnote{1.4} it’s when a mendicant explains what is not the teaching as the teaching … and what was prescribed by the Realized One as not prescribed by the Realized One. On these ten grounds they split off and go their own way. They perform legal acts autonomously and recite the monastic code autonomously. That is how schism in the \textsanskrit{Saṅgha} is defined.” 

“But\marginnote{2.1} sir, what does someone who has split a harmonious \textsanskrit{Saṅgha} create?” 

“They\marginnote{2.2} create an sin that lasts for an eon.” 

“But\marginnote{2.3} sir, what is the iniquity that lasts for an eon?” 

“They\marginnote{2.4} burn in hell for an eon, Ānanda. 

\begin{verse}%
A\marginnote{3.1} schismatic remains for the eon \\
in a place of loss, in hell. \\
Taking a stand against the teaching, \\
favoring factions, they ruin their sanctuary. \\
After causing schism in a harmonious \textsanskrit{Saṅgha}, \\
they burn in hell for an eon.” 

%
\end{verse}

%
\section*{{\suttatitleacronym AN 10.40}{\suttatitletranslation With Ānanda (2nd) }{\suttatitleroot Dutiyaānandasutta}}
\addcontentsline{toc}{section}{\tocacronym{AN 10.40} \toctranslation{With Ānanda (2nd) } \tocroot{Dutiyaānandasutta}}
\markboth{With Ānanda (2nd) }{Dutiyaānandasutta}
\extramarks{AN 10.40}{AN 10.40}

“Sir,\marginnote{1.1} they speak of ‘harmony in the \textsanskrit{Saṅgha}’. How is harmony in the \textsanskrit{Saṅgha} defined?” 

“Ānanda,\marginnote{1.3} it’s when a mendicant explains what is not the teaching as not the teaching, and what is the teaching as the teaching. They explain what is not the training as not the training, and what is the training as the training. They explain what was not spoken and stated by the Realized One as not spoken and stated by the Realized One, and what was spoken and stated by the Realized One as spoken and stated by the Realized One. They explain what was not practiced by the Realized One as not practiced by the Realized One, and what was practiced by the Realized One as practiced by the Realized One. They explain what was not prescribed by the Realized One as not prescribed by the Realized One, and what was prescribed by the Realized One as prescribed by the Realized One. On these ten grounds they don’t split off and go their own way. They don’t perform legal acts autonomously or recite the monastic code autonomously. That is how harmony in the \textsanskrit{Saṅgha} is defined.” 

“But\marginnote{2.1} sir, what does someone who has created harmony in a schismatic \textsanskrit{Saṅgha} create?” 

“They\marginnote{2.2} create divine merit.” 

“But\marginnote{2.3} what is divine merit?” 

“They\marginnote{2.4} rejoice in heaven for an eon, Ānanda. 

\begin{verse}%
A\marginnote{3.1} \textsanskrit{Saṅgha} in harmony is happy, \\
as is support for those in harmony. \\
Taking a stand on the teaching, \\
favoring harmony, they ruin no sanctuary. \\
After creating harmony in the \textsanskrit{Saṅgha}, \\
they rejoice in heaven for an eon.” 

%
\end{verse}

%
\addtocontents{toc}{\let\protect\contentsline\protect\nopagecontentsline}
\chapter*{The Chapter on Abuse }
\addcontentsline{toc}{chapter}{\tocchapterline{The Chapter on Abuse }}
\addtocontents{toc}{\let\protect\contentsline\protect\oldcontentsline}

%
\section*{{\suttatitleacronym AN 10.41}{\suttatitletranslation Arguments }{\suttatitleroot Vivādasutta}}
\addcontentsline{toc}{section}{\tocacronym{AN 10.41} \toctranslation{Arguments } \tocroot{Vivādasutta}}
\markboth{Arguments }{Vivādasutta}
\extramarks{AN 10.41}{AN 10.41}

Then\marginnote{1.1} Venerable \textsanskrit{Upāli} went up to the Buddha, bowed, sat down to one side, and said to him: 

“What\marginnote{1.2} is the cause, sir, what is the reason, why arguments, quarrels, and disputes arise in the \textsanskrit{Saṅgha}, and the mendicants don’t live happily?” 

“\textsanskrit{Upāli},\marginnote{1.3} it’s when a mendicant explains what is not the teaching as the teaching, and what is the teaching as not the teaching. They explain what is not the training as the training, and what is the training as not the training. They explain what was not spoken and stated by the Realized One as spoken and stated by the Realized One, and what was spoken and stated by the Realized One as not spoken and stated by the Realized One. They explain what was not practiced by the Realized One as practiced by the Realized One, and what was practiced by the Realized One as not practiced by the Realized One. They explain what was not prescribed by the Realized One as prescribed by the Realized One, and what was prescribed by the Realized One as not prescribed by the Realized One. This is the cause, this is the reason why arguments, quarrels, and disputes arise in the \textsanskrit{Saṅgha}, and the mendicants don’t live happily.” 

%
\section*{{\suttatitleacronym AN 10.42}{\suttatitletranslation Roots of Arguments (1st) }{\suttatitleroot Paṭhamavivādamūlasutta}}
\addcontentsline{toc}{section}{\tocacronym{AN 10.42} \toctranslation{Roots of Arguments (1st) } \tocroot{Paṭhamavivādamūlasutta}}
\markboth{Roots of Arguments (1st) }{Paṭhamavivādamūlasutta}
\extramarks{AN 10.42}{AN 10.42}

“Sir,\marginnote{1.1} how many roots of arguments are there?” 

“\textsanskrit{Upāli},\marginnote{1.2} there are ten roots of arguments. What ten? It’s when a mendicant explains what is not the teaching as the teaching, and what is the teaching as not the teaching. They explain what is not the training as the training, and what is the training as not the training. They explain what was not spoken and stated by the Realized One as spoken and stated by the Realized One, and what was spoken and stated by the Realized One as not spoken and stated by the Realized One. They explain what was not practiced by the Realized One as practiced by the Realized One, and what was practiced by the Realized One as not practiced by the Realized One. They explain what was not prescribed by the Realized One as prescribed by the Realized One, and what was prescribed by the Realized One as not prescribed by the Realized One. These are the ten roots of arguments.” 

%
\section*{{\suttatitleacronym AN 10.43}{\suttatitletranslation Roots of Arguments (2nd) }{\suttatitleroot Dutiyavivādamūlasutta}}
\addcontentsline{toc}{section}{\tocacronym{AN 10.43} \toctranslation{Roots of Arguments (2nd) } \tocroot{Dutiyavivādamūlasutta}}
\markboth{Roots of Arguments (2nd) }{Dutiyavivādamūlasutta}
\extramarks{AN 10.43}{AN 10.43}

“Sir,\marginnote{1.1} how many roots of arguments are there?” 

“\textsanskrit{Upāli},\marginnote{1.2} there are ten roots of arguments. What ten? It’s when a mendicant explains what is not an offense as an offense, and what is an offense as not an offense. They explain a light offense as a serious offense, and a serious offense as a light offense. They explain an offense committed with corrupt intention as an offense not committed with corrupt intention, and an offense not committed with corrupt intention as an offense committed with corrupt intention. They explain an offense requiring rehabilitation as an offense not requiring rehabilitation, and an offense not requiring rehabilitation as an offense requiring rehabilitation. They explain an offense with redress as an offense without redress, and an offense without redress as an offense with redress. These are the ten roots of arguments.” 

%
\section*{{\suttatitleacronym AN 10.44}{\suttatitletranslation At Kusinārā }{\suttatitleroot Kusinārasutta}}
\addcontentsline{toc}{section}{\tocacronym{AN 10.44} \toctranslation{At Kusinārā } \tocroot{Kusinārasutta}}
\markboth{At Kusinārā }{Kusinārasutta}
\extramarks{AN 10.44}{AN 10.44}

At\marginnote{1.1} one time the Buddha was staying near \textsanskrit{Kusinārā}, in the Forest of Offerings. There the Buddha addressed the mendicants, “Mendicants!” 

“Venerable\marginnote{1.4} sir,” they replied. The Buddha said this: 

“Mendicants,\marginnote{2.1} a mendicant who wants to accuse another should first check five things in themselves and establish five things in themselves. What five things should they check in themselves? A mendicant who wants to accuse another should check this: ‘Is my bodily behavior pure? Do I have pure bodily behavior that is impeccable and irreproachable? Is this thing found in me or not?’ If it’s not, there will be people who say: ‘Come on, venerable, train your own bodily behavior first.’ 

Furthermore,\marginnote{3.1} a mendicant who wants to accuse another should check this: ‘Is my verbal behavior pure? Do I have pure verbal behavior that is impeccable and irreproachable? Is this thing found in me or not?’ If it’s not, there will be people who say: ‘Come on, venerable, train your own verbal behavior first.’ 

Furthermore,\marginnote{4.1} a mendicant who wants to accuse another should check this: ‘Is my heart established in love for my spiritual companions, without resentment? Is this thing found in me or not?’ If it’s not, there will be people who say: ‘Come on, venerable, establish your heart in love for your spiritual companions first.’ 

Furthermore,\marginnote{5.1} a mendicant who wants to accuse another should check this: ‘Am I very learned, remembering and keeping what I’ve learned? These teachings are good in the beginning, good in the middle, and good in the end, meaningful and well-phrased, describing a spiritual practice that’s entirely full and pure. Am I very learned in such teachings, remembering them, rehearsing them, mentally scrutinizing them, and comprehending them theoretically? Is this thing found in me or not?’ If it’s not, there will be people who say: ‘Come on, venerable, memorize the scriptures first.’ 

Furthermore,\marginnote{6.1} a mendicant who wants to accuse another should check this: ‘Have both monastic codes been passed down to me in detail, well analyzed, well mastered, and well evaluated in both the rules and accompanying material? Is this thing found in me or not?’ If it’s not, and if they are unable to respond when asked “Venerable, where was this spoken by the Buddha?” there will be people who say: ‘Come on, venerable, train in the monastic code first.’ These are the five things they should check in themselves. 

What\marginnote{7.1} five things should they establish in themselves? ‘I will speak at the right time, not at the wrong time. I will speak truthfully, not falsely. I will speak gently, not harshly. I will speak beneficially, not harmfully. I will speak lovingly, not from secret hate.’ These are the five things they should establish in themselves. A mendicant who wants to accuse another should first check these five things in themselves and establish these five things in themselves.” 

%
\section*{{\suttatitleacronym AN 10.45}{\suttatitletranslation Entering a Royal Compound }{\suttatitleroot Rājantepurappavesanasutta}}
\addcontentsline{toc}{section}{\tocacronym{AN 10.45} \toctranslation{Entering a Royal Compound } \tocroot{Rājantepurappavesanasutta}}
\markboth{Entering a Royal Compound }{Rājantepurappavesanasutta}
\extramarks{AN 10.45}{AN 10.45}

“Mendicants,\marginnote{1.1} there are ten drawbacks to entering a royal compound. What ten? 

Firstly,\marginnote{1.3} a king is sitting with his chief queen when a monk enters. When the queen sees the monk she smiles, or when the monk sees the queen he smiles. So the king thinks: ‘They’ve done it for sure, or they will do it.’ This is the first drawback of entering a royal compound. 

Furthermore,\marginnote{2.1} a king has many duties, and much to do. He has sex with one of the women but doesn’t remember. She gets pregnant from that. So the king thinks: ‘No-one else has entered here, except for that monk. Could this be the monk’s doing?’ This is the second drawback of entering a royal compound. 

Furthermore,\marginnote{3.1} a gem is lost somewhere in the royal compound. So the king thinks: ‘No-one else has entered here, except for that monk. Could this be the monk’s doing?’ This is the third drawback of entering a royal compound. 

Furthermore,\marginnote{4.1} secret deliberations in the royal compound are leaked outside. So the king thinks: ‘No-one else has entered here, except for that monk. Could this be the monk’s doing?’ This is the fourth drawback of entering a royal compound. 

Furthermore,\marginnote{5.1} in a royal compound, a father longs for their son, or a son longs for his father. They think: ‘No-one else has entered here, except for that monk. Could this be the monk’s doing?’ This is the fifth drawback of entering a royal compound. 

Furthermore,\marginnote{6.1} the king promotes someone to a higher position. Those who are upset by this think: ‘The king is close to that monk. Could this be the monk’s doing?’ This is the sixth drawback of entering a royal compound. 

Furthermore,\marginnote{7.1} the king demotes someone to a lower position. Those who are upset by this think: ‘The king is close to that monk. Could this be the monk’s doing?’ This is the seventh drawback of entering a royal compound. 

Furthermore,\marginnote{8.1} the king dispatches the army at the wrong time. Those who are upset by this think: ‘The king is close to that monk. Could this be the monk’s doing?’ This is the eighth drawback of entering a royal compound. 

Furthermore,\marginnote{9.1} the king dispatches the army at the right time, but orders it to turn back while still on the road. Those who are upset by this think: ‘The king is close to that monk. Could this be the monk’s doing?’ This is the ninth drawback of entering a royal compound. 

Furthermore,\marginnote{10.1} in the royal compound there is a trampling of elephants, horses, and chariots, as well as arousing sights, sounds, smells, tastes, and touches that do not befit a monk. This is the tenth drawback of entering a royal compound. 

These\marginnote{10.3} are the ten drawbacks of entering a royal compound.” 

%
\section*{{\suttatitleacronym AN 10.46}{\suttatitletranslation With the Sakyans }{\suttatitleroot Sakkasutta}}
\addcontentsline{toc}{section}{\tocacronym{AN 10.46} \toctranslation{With the Sakyans } \tocroot{Sakkasutta}}
\markboth{With the Sakyans }{Sakkasutta}
\extramarks{AN 10.46}{AN 10.46}

At\marginnote{1.1} one time the Buddha was staying in the land of the Sakyans, near Kapilavatthu in the Banyan Tree Monastery. Then on the sabbath several Sakyan lay followers went up to the Buddha, bowed, and sat down to one side. The Buddha said to them: 

“Sakyans,\marginnote{1.3} do you observe the sabbath with its eight factors?” 

“Sir,\marginnote{1.4} sometimes we do, sometimes we don’t.” 

“That’s\marginnote{1.5} your loss, Sakyans, it’s your misfortune. In this life with its fear of sorrow and death, you sometimes keep the sabbath and you sometimes don’t. 

What\marginnote{2.1} do you think, Sakyans? Take a man who earns half a dollar for an honest day’s work. Is this enough to call him a deft and industrious man?” 

“Yes,\marginnote{2.4} sir.” 

“What\marginnote{3.1} do you think, Sakyans? Take a man who earns a dollar for an honest day’s work. Is this enough to call him a deft and industrious man?” 

“Yes,\marginnote{3.4} sir.” 

“What\marginnote{4.1} do you think, Sakyans? Take a man who, for an honest day’s work, earns two dollars … three dollars … four dollars … five dollars … six dollars … seven dollars … eight dollars … nine dollars … ten dollars … twenty dollars … thirty dollars … forty dollars … fifty dollars … a hundred dollars. Is this enough to call him a deft and industrious man?” 

“Yes,\marginnote{4.17} sir.” 

“What\marginnote{5.1} do you think, Sakyans? Suppose that man earned a hundred or a thousand dollars every day and saved it all up. If he lived for a hundred years, would he not accumulate a large mass of wealth?” 

“Yes,\marginnote{5.3} sir.” 

“What\marginnote{6.1} do you think, Sakyans? Would that man, on account of that wealth, experience perfect happiness for a single day or night, or even half a day or night?” 

“No,\marginnote{6.3} sir.” 

“Why\marginnote{6.4} is that?” 

“Because\marginnote{6.5} sensual pleasures, sir, are impermanent, hollow, false, and deceptive.” 

“But\marginnote{7.1} take one of my disciples who lives diligent, keen, and resolute for ten years, practicing in line with my instructions. They can experience perfect happiness for a hundred years, ten thousand years, or a hundred thousand years. And they could become a once-returner or a non-returner, or unfailingly a stream-enterer. Let alone ten years, take one of my disciples who lives diligent, keen, and resolute for nine years … eight years … seven years … six years … five years … four years … three years … two years … one year … ten months … nine months … eight months … seven months … six months … five months … four months … three months … two months … one month … a fortnight … ten days … nine days … eight days … seven days … six days … five days … four days … three days … two days … 

Let\marginnote{12.9} alone two days, take one of my disciples who lives diligent, keen, and resolute for one day, practicing in line with my instructions. They can experience perfect happiness for a hundred years, ten thousand years, or a hundred thousand years. And they could become a once-returner or a non-returner, or unfailingly a stream-enterer. It’s your loss, Sakyans, it’s your misfortune. In this life with its fear of sorrow and death, you sometimes keep the sabbath and you sometimes don’t.” 

“Well,\marginnote{12.11} sir, from this day forth we will observe the sabbath with its eight factors.” 

%
\section*{{\suttatitleacronym AN 10.47}{\suttatitletranslation With Mahāli }{\suttatitleroot Mahālisutta}}
\addcontentsline{toc}{section}{\tocacronym{AN 10.47} \toctranslation{With Mahāli } \tocroot{Mahālisutta}}
\markboth{With Mahāli }{Mahālisutta}
\extramarks{AN 10.47}{AN 10.47}

At\marginnote{1.1} one time the Buddha was staying near \textsanskrit{Vesālī}, at the Great Wood, in the hall with the peaked roof. Then \textsanskrit{Mahāli} the Licchavi went up to the Buddha, bowed, sat down to one side, and said to him: 

“What\marginnote{1.3} is the cause, sir, what is the reason for doing bad deeds, for performing bad deeds?” 

“Greed\marginnote{1.4} is a cause, \textsanskrit{Mahāli}, greed is a reason for doing bad deeds, for performing bad deeds. Hate is a cause of bad deeds … Delusion is a cause of bad deeds … Irrational application of mind is a cause of bad deeds … A wrongly directed mind is a cause of bad deeds … This is the cause, \textsanskrit{Mahāli}, this is the reason for doing bad deeds, for performing bad deeds.” 

“What\marginnote{2.1} is the cause, sir, what is the reason for doing good deeds, for performing good deeds?” 

“Contentment\marginnote{2.2} is a cause, \textsanskrit{Mahāli}, contentment is a reason for doing good deeds, for performing good deeds. Love is a cause of good deeds … Understanding is a cause of good deeds … Rational application of mind is a cause of good deeds … A rightly directed mind is a cause of good deeds … This is the cause, \textsanskrit{Mahāli}, this is the reason for doing good deeds, for performing good deeds. If these ten things were not found in the world, we wouldn’t see either unprincipled and immoral conduct, or principled and moral conduct. But since these ten things are found in the world, we see both unprincipled and immoral conduct, and principled and moral conduct.” 

%
\section*{{\suttatitleacronym AN 10.48}{\suttatitletranslation Ten Regular Reflections for a Renunciate }{\suttatitleroot Pabbajitaabhiṇhasutta}}
\addcontentsline{toc}{section}{\tocacronym{AN 10.48} \toctranslation{Ten Regular Reflections for a Renunciate } \tocroot{Pabbajitaabhiṇhasutta}}
\markboth{Ten Regular Reflections for a Renunciate }{Pabbajitaabhiṇhasutta}
\extramarks{AN 10.48}{AN 10.48}

“Mendicants,\marginnote{1.1} one who has gone forth should often review these ten things. What ten? 

One\marginnote{2.1} who has gone forth should often review this: ‘I have secured freedom from class.’ 

‘My\marginnote{3.1} livelihood is tied up with others.’ 

‘My\marginnote{4.1} behavior should be different.’ 

‘I\marginnote{5.1} hope there’s no reason to blame myself when it comes to ethical conduct?’ 

‘I\marginnote{6.1} hope that, after examination, sensible spiritual companions don’t reproach any aspect of my ethics?’ 

‘I\marginnote{7.1} must be parted and separated from all I hold dear and beloved.’ 

‘I\marginnote{8.1} am the owner of my deeds and heir to my deeds. Deeds are my womb, my relative, and my refuge. 

I\marginnote{9.1} shall be the heir of whatever deeds I do, whether good or bad.’ 

‘As\marginnote{10.1} the days and nights flit by, what sort of person am I becoming?’ 

‘Do\marginnote{11.1} I love to stay in empty huts?’ 

‘Do\marginnote{12.1} I have any superhuman distinctions in knowledge and vision worthy of the noble ones, so that when my spiritual companions question me on my deathbed I will not be embarrassed?’ 

One\marginnote{13.1} who has gone forth should often review these ten things.” 

%
\section*{{\suttatitleacronym AN 10.49}{\suttatitletranslation Existing Because of the Body }{\suttatitleroot Sarīraṭṭhadhammasutta}}
\addcontentsline{toc}{section}{\tocacronym{AN 10.49} \toctranslation{Existing Because of the Body } \tocroot{Sarīraṭṭhadhammasutta}}
\markboth{Existing Because of the Body }{Sarīraṭṭhadhammasutta}
\extramarks{AN 10.49}{AN 10.49}

“Mendicants,\marginnote{1.1} these ten things exist because of the body. What ten? Cold, heat, hunger, thirst, feces, urine, restraint of body, speech, and livelihood, and the will to live that leads to future lives. These ten things exist because of the body.” 

%
\section*{{\suttatitleacronym AN 10.50}{\suttatitletranslation Arguments }{\suttatitleroot Bhaṇḍanasutta}}
\addcontentsline{toc}{section}{\tocacronym{AN 10.50} \toctranslation{Arguments } \tocroot{Bhaṇḍanasutta}}
\markboth{Arguments }{Bhaṇḍanasutta}
\extramarks{AN 10.50}{AN 10.50}

At\marginnote{1.1} one time the Buddha was staying near \textsanskrit{Sāvatthī} in Jeta’s Grove, \textsanskrit{Anāthapiṇḍika}’s monastery. Now at that time, after the meal, on return from almsround, several mendicants sat together in the assembly hall. They were arguing, quarreling, and disputing, wounding each other with barbed words. 

Then\marginnote{2.1} in the late afternoon, the Buddha came out of retreat and went to the assembly hall. He sat down on the seat spread out, and addressed the mendicants: “Mendicants, what were you sitting talking about just now? What conversation was left unfinished?” 

“Sir,\marginnote{3.1} after the meal, on return from almsround, we sat together in the assembly hall, arguing, quarreling, and disputing, wounding each other with barbed words.” 

“Mendicants,\marginnote{3.2} this is not appropriate for you gentlemen who have gone forth out of faith from the lay life to homelessness. 

There\marginnote{4.1} are ten warm-hearted qualities that make for fondness and respect, conducing to inclusion, harmony, and unity, without quarreling. What ten? Firstly, a mendicant is ethical, restrained in the monastic code, conducting themselves well and resorting for alms in suitable places. Seeing danger in the slightest fault, they keep the rules they’ve undertaken. When a mendicant is ethical, this warm-hearted quality makes for fondness and respect, conducing to inclusion, harmony, and unity, without quarreling. 

Furthermore,\marginnote{5.1} a mendicant is very learned, remembering and keeping what they’ve learned. These teachings are good in the beginning, good in the middle, and good in the end, meaningful and well-phrased, describing a spiritual practice that’s entirely full and pure. They are very learned in such teachings, remembering them, rehearsing them, mentally scrutinizing them, and comprehending them theoretically. … 

Furthermore,\marginnote{6.1} a mendicant has good friends, companions, and associates. … 

Furthermore,\marginnote{7.1} a mendicant is easy to admonish, having qualities that make them easy to admonish. They’re patient, and take instruction respectfully. … 

Furthermore,\marginnote{8.1} a mendicant is deft and tireless in a diverse spectrum of duties for their spiritual companions, understanding how to go about things in order to complete and organize the work. … 

Furthermore,\marginnote{9.1} a mendicant loves the teachings and is a delight to converse with, being full of joy in the teaching and training. … 

Furthermore,\marginnote{10.1} a mendicant lives with energy roused up for giving up unskillful qualities and embracing skillful qualities. They are strong, staunchly vigorous, not slacking off when it comes to developing skillful qualities. … 

Furthermore,\marginnote{11.1} a mendicant is content with any kind of robes, almsfood, lodgings, and medicines and supplies for the sick. … 

Furthermore,\marginnote{12.1} a mendicant is mindful. They have utmost mindfulness and alertness, and can remember and recall what was said and done long ago. … 

Furthermore,\marginnote{13.1} a mendicant is wise. They have the wisdom of arising and passing away which is noble, penetrative, and leads to the complete ending of suffering. When a mendicant is wise, this warm-hearted quality makes for fondness and respect, helping the \textsanskrit{Saṅgha} to live in harmony and unity, without quarreling. 

These\marginnote{14.1} ten warm-hearted qualities make for fondness and respect, conducing to inclusion, harmony, and unity, without quarreling.” 

%
\addtocontents{toc}{\let\protect\contentsline\protect\nopagecontentsline}
\pannasa{The Second Fifty }
\addcontentsline{toc}{pannasa}{The Second Fifty }
\markboth{}{}
\addtocontents{toc}{\let\protect\contentsline\protect\oldcontentsline}

%
\addtocontents{toc}{\let\protect\contentsline\protect\nopagecontentsline}
\chapter*{The Chapter on Your Own Mind }
\addcontentsline{toc}{chapter}{\tocchapterline{The Chapter on Your Own Mind }}
\addtocontents{toc}{\let\protect\contentsline\protect\oldcontentsline}

%
\section*{{\suttatitleacronym AN 10.51}{\suttatitletranslation Your Own Mind }{\suttatitleroot Sacittasutta}}
\addcontentsline{toc}{section}{\tocacronym{AN 10.51} \toctranslation{Your Own Mind } \tocroot{Sacittasutta}}
\markboth{Your Own Mind }{Sacittasutta}
\extramarks{AN 10.51}{AN 10.51}

At\marginnote{1.1} one time the Buddha was staying near \textsanskrit{Sāvatthī} in Jeta’s Grove, \textsanskrit{Anāthapiṇḍika}’s monastery. There the Buddha addressed the mendicants, “Mendicants!” 

“Venerable\marginnote{1.4} sir,” they replied. The Buddha said this: 

“Mendicants,\marginnote{2.1} if a mendicant isn’t skilled in the ways of another’s mind, then they should train themselves: ‘I will be skilled in the ways of my own mind.’ 

And\marginnote{3.1} how is a mendicant skilled in the ways of their own mind? Suppose there was a woman or man who was young, youthful, and fond of adornments, and they check their own reflection in a clean bright mirror or a clear bowl of water. If they see any dirt or blemish there, they’d try to remove it. But if they don’t see any dirt or blemish there, they’re happy with that, as they’ve got all they wished for: ‘How fortunate that I’m clean!’ In the same way, checking is very helpful for a mendicant’s skillful qualities. ‘Am I often covetous or not? Am I often malicious or not? Am I often overcome with dullness and drowsiness or not? Am I often restless or not? Am I often doubtful or not? Am I often irritable or not? Am I often corrupted in mind or not? Am I often disturbed in body or not? Am I often energetic or not? Am I often immersed in \textsanskrit{samādhi} or not?’ 

Suppose\marginnote{4.1} that, upon checking, a mendicant knows this: ‘I am often covetous, malicious, overcome with dullness and drowsiness, restless, doubtful, irritable, defiled in mind, disturbed in body, lazy, and not immersed in \textsanskrit{samādhi}.’ In order to give up those bad, unskillful qualities, they should apply intense enthusiasm, effort, zeal, vigor, perseverance, mindfulness, and situational awareness. Suppose your clothes or head were on fire. In order to extinguish it, you’d apply intense enthusiasm, effort, zeal, vigor, perseverance, mindfulness, and situational awareness. In the same way, in order to give up those bad, unskillful qualities, that mendicant should apply intense enthusiasm … 

But\marginnote{5.1} suppose that, upon checking, a mendicant knows this: ‘I am often content, kind-hearted, free of dullness and drowsiness, calm, confident, loving, pure in mind, undisturbed in body, energetic, and immersed in \textsanskrit{samādhi}.’ Grounded on those skillful qualities, they should practice meditation further to end the defilements.” 

%
\section*{{\suttatitleacronym AN 10.52}{\suttatitletranslation With Sāriputta }{\suttatitleroot Sāriputtasutta}}
\addcontentsline{toc}{section}{\tocacronym{AN 10.52} \toctranslation{With Sāriputta } \tocroot{Sāriputtasutta}}
\markboth{With Sāriputta }{Sāriputtasutta}
\extramarks{AN 10.52}{AN 10.52}

There\marginnote{1.1} \textsanskrit{Sāriputta} addressed the mendicants: “Reverends, mendicants!” 

“Reverend,”\marginnote{1.3} they replied. \textsanskrit{Sāriputta} said this: 

“Reverends,\marginnote{2.1} if a mendicant isn’t skilled in the ways of another’s mind, then they should train themselves: ‘I will be skilled in the ways of my own mind.’ 

And\marginnote{3.1} how is a mendicant skilled in the ways of their own mind? Suppose there was a woman or man who was young, youthful, and fond of adornments, and they check their own reflection in a clean bright mirror or a clear bowl of water. If they see any dirt or blemish there, they’d try to remove it. But if they don’t see any dirt or blemish there, they’re happy with that, as they’ve got all they wished for: ‘How fortunate that I’m clean!’ 

In\marginnote{4.1} the same way, checking is very helpful for a mendicant’s skillful qualities. ‘Am I often covetous or not? Am I often malicious or not? Am I often overcome with dullness and drowsiness or not? Am I often restless or not? Am I often doubtful or not? Am I often irritable or not? Am I often defiled in mind or not? Am I often disturbed in body or not? Am I often energetic or not? Am I often immersed in \textsanskrit{samādhi} or not?’ 

Suppose\marginnote{5.1} that, upon checking, a mendicant knows this: ‘I am often covetous, malicious, overcome with dullness and drowsiness, restless, doubtful, angry, defiled in mind, disturbed in body, lazy, and not immersed in \textsanskrit{samādhi}.’ In order to give up those bad, unskillful qualities, they should apply intense enthusiasm, effort, zeal, vigor, perseverance, mindfulness, and situational awareness. Suppose your clothes or head were on fire. In order to extinguish it, you’d apply intense enthusiasm, effort, zeal, vigor, perseverance, mindfulness, and situational awareness. In the same way, in order to give up those bad, unskillful qualities, that mendicant should apply intense enthusiasm … 

But\marginnote{6.1} suppose that, upon checking, a mendicant knows this: ‘I am often content, kind-hearted, rid of dullness and drowsiness, calm, confident, loving, pure in mind, undisturbed in body, energetic, and immersed in \textsanskrit{samādhi}.’ Grounded on those skillful qualities, they should practice meditation further to end the defilements.” 

%
\section*{{\suttatitleacronym AN 10.53}{\suttatitletranslation Stagnation }{\suttatitleroot Ṭhitisutta}}
\addcontentsline{toc}{section}{\tocacronym{AN 10.53} \toctranslation{Stagnation } \tocroot{Ṭhitisutta}}
\markboth{Stagnation }{Ṭhitisutta}
\extramarks{AN 10.53}{AN 10.53}

“Mendicants,\marginnote{1.1} I don’t praise stagnation in skillful qualities, let alone decline. I praise growth in skillful qualities, not stagnation or decline. 

And\marginnote{2.1} how is there decline in skillful qualities, not stagnation or growth? It’s when a mendicant has a certain degree of faith, ethics, generosity, wisdom, and eloquence. Those qualities neither stagnate nor grow in them. I call this decline in skillful qualities, not stagnation or growth. This is how there’s decline in skillful qualities, not stagnation or growth. 

And\marginnote{3.1} how is there stagnation in skillful qualities, not decline or growth? It’s when a mendicant has a certain degree of faith, ethics, generosity, wisdom, and eloquence. Those qualities neither decline nor grow in them. I call this stagnation in skillful qualities, not decline or growth. This is how there’s stagnation in skillful qualities, not decline or growth. 

And\marginnote{4.1} how is there growth in skillful qualities, not stagnation or decline? It’s when a mendicant has a certain degree of faith, ethics, generosity, wisdom, and eloquence. Those qualities neither stagnate nor decline in them. I call this growth in skillful qualities, not stagnation or decline. This is how there’s growth in skillful qualities, not stagnation or decline. 

If\marginnote{5.1} a mendicant isn’t skilled in the ways of another’s mind, then they should train themselves: ‘I will be skilled in the ways of my own mind.’ 

And\marginnote{6.1} how is a mendicant skilled in the ways of their own mind? Suppose there was a woman or man who was young, youthful, and fond of adornments, and they check their own reflection in a clean bright mirror or a clear bowl of water. If they see any dirt or blemish there, they’d try to remove it. But if they don’t see any dirt or blemish there, they’re happy with that, as they’ve got all they wished for: ‘How fortunate that I’m clean!’ In the same way, checking is very helpful for a mendicant’s skillful qualities. ‘Am I often covetous or not? Am I often malicious or not? Am I often overcome with dullness and drowsiness or not? Am I often restless or not? Am I often doubtful or not? Am I often irritable or not? Am I often defiled in mind or not? Am I often disturbed in body or not? Am I often energetic or not? Am I often immersed in \textsanskrit{samādhi} or not?’ 

Suppose\marginnote{7.1} that, upon checking, a mendicant knows this: ‘I am often covetous, malicious, overcome with dullness and drowsiness, restless, doubtful, irritable, defiled in mind, disturbed in body, lazy, and not immersed in \textsanskrit{samādhi}.’ In order to give up those bad, unskillful qualities, they should apply intense enthusiasm, effort, zeal, vigor, perseverance, mindfulness, and situational awareness. Suppose your clothes or head were on fire. In order to extinguish it, you’d apply intense enthusiasm, effort, zeal, vigor, perseverance, mindfulness, and situational awareness. In the same way, in order to give up those bad, unskillful qualities, that mendicant should apply intense enthusiasm … 

But\marginnote{8.1} suppose that, upon checking, a mendicant knows this: ‘I am often content, kind-hearted, rid of dullness and drowsiness, calm, confident, loving, pure in mind, undisturbed in body, energetic, and immersed in \textsanskrit{samādhi}.’ Grounded on those skillful qualities, they should practice meditation further to end the defilements.” 

%
\section*{{\suttatitleacronym AN 10.54}{\suttatitletranslation Serenity }{\suttatitleroot Samathasutta}}
\addcontentsline{toc}{section}{\tocacronym{AN 10.54} \toctranslation{Serenity } \tocroot{Samathasutta}}
\markboth{Serenity }{Samathasutta}
\extramarks{AN 10.54}{AN 10.54}

“Mendicants,\marginnote{1.1} if a mendicant isn’t skilled in the ways of another’s mind, then they should train themselves: ‘I will be skilled in the ways of my own mind.’ 

And\marginnote{2.1} how is a mendicant skilled in the ways of their own mind? Suppose there was a woman or man who was young, youthful, and fond of adornments, and they check their own reflection in a clean bright mirror or a clear bowl of water. If they see any dirt or blemish there, they’d try to remove it. But if they don’t see any dirt or blemish there, they’re happy with that, as they’ve got all they wished for: ‘How fortunate that I’m clean!’ In the same way, checking is very helpful for a mendicant’s skillful qualities. ‘Do I have internal serenity of heart or not? Do I have the higher wisdom of discernment of principles or not?’ 

Suppose\marginnote{3.1} that, upon checking, a mendicant knows this: ‘I have serenity but not discernment.’ Grounded on serenity, they should practice meditation to get discernment. After some time they have both serenity and discernment. 

But\marginnote{4.1} suppose that, upon checking, a mendicant knows this: ‘I have discernment but not serenity.’ Grounded on discernment, they should practice meditation to get serenity. After some time they have both serenity and discernment. 

But\marginnote{5.1} suppose that, upon checking, a mendicant knows this: ‘I have neither serenity nor discernment.’ In order to get those skillful qualities, they should apply intense enthusiasm, effort, zeal, vigor, perseverance, mindfulness, and situational awareness. Suppose your clothes or head were on fire. In order to extinguish it, you’d apply intense enthusiasm, effort, zeal, vigor, perseverance, mindfulness, and situational awareness. In the same way, in order to get those skillful qualities, that mendicant should apply intense enthusiasm … After some time they have both serenity and discernment. 

But\marginnote{6.1} suppose that, upon checking, a mendicant knows this: ‘I have both serenity and discernment.’ Grounded on those skillful qualities, they should practice meditation further to end the defilements. 

I\marginnote{7.1} say that there are two kinds of robes: those you should wear, and those you shouldn’t wear. I say that there are two kinds of almsfood: that which you should eat, and that which you shouldn’t eat. I say that there are two kinds of lodging: those you should frequent, and those you shouldn’t frequent. I say that there are two kinds of village or town: those you should frequent, and those you shouldn’t frequent. I say that there are two kinds of country: those you should frequent, and those you shouldn’t frequent. I say that there are two kinds of people: those you should frequent, and those you shouldn’t frequent. 

‘I\marginnote{8.1} say that there are two kinds of robes: those you should wear, and those you shouldn’t wear.’ That’s what I said, but why did I say it? Well, should you know of a robe: ‘When I wear this robe, unskillful qualities grow, and skillful qualities decline.’ You should not wear that kind of robe. Whereas, should you know of a robe: ‘When I wear this robe, unskillful qualities decline, and skillful qualities grow.’ You should wear that kind of robe. ‘I say that there are two kinds of robes: those you should wear, and those you shouldn’t wear.’ That’s what I said, and this is why I said it. 

‘I\marginnote{9.1} say that there are two kinds of almsfood: that which you should eat, and that which you shouldn’t eat.’ That’s what I said, but why did I say it? Well, should you know of almsfood: ‘When I eat this almsfood, unskillful qualities grow, and skillful qualities decline.’ You should not eat that kind of almsfood. Whereas, should you know of almsfood: ‘When I eat this almsfood, unskillful qualities decline, and skillful qualities grow.’ You should eat that kind of almsfood. ‘I say that there are two kinds of almsfood: that which you should eat, and that which you shouldn’t eat.’ That’s what I said, and this is why I said it. 

‘I\marginnote{10.1} say that there are two kinds of lodging: those you should frequent, and those you shouldn’t frequent.’ That’s what I said, but why did I say it? Well, should you know of a lodging: ‘When I frequent this lodging, unskillful qualities grow, and skillful qualities decline.’ You should not frequent that kind of lodging. Whereas, should you know of a lodging: ‘When I frequent this lodging, unskillful qualities decline, and skillful qualities grow.’ You should frequent that kind of lodging. ‘I say that there are two kinds of lodging: those you should frequent, and those you shouldn’t frequent.’ That’s what I said, and this is why I said it. 

‘I\marginnote{11.1} say that there are two kinds of village or town: those you should frequent, and those you shouldn’t frequent.’ That’s what I said, but why did I say it? Well, should you know of a village or town: ‘When I frequent this village or town, unskillful qualities grow, and skillful qualities decline.’ You should not frequent that kind of village or town. Whereas, should you know of a village or town: ‘When I frequent this village or town, unskillful qualities decline, and skillful qualities grow.’ You should frequent that kind of village or town. ‘I say that there are two kinds of village or town: those you should frequent, and those you shouldn’t frequent.’ That’s what I said, and this is why I said it. 

‘I\marginnote{12.1} say that there are two kinds of country: those you should frequent, and those you shouldn’t frequent.’ That’s what I said, but why did I say it? Well, should you know of a country: ‘When I frequent this country, unskillful qualities grow, and skillful qualities decline.’ You should not frequent that kind of country. Whereas, should you know of a country: ‘When I frequent this country, unskillful qualities decline, and skillful qualities grow.’ You should frequent that kind of country. ‘I say that there are two kinds of country: those you should frequent, and those you shouldn’t frequent.’ That’s what I said, and this is why I said it. 

‘I\marginnote{13.1} say that there are two kinds of people: those you should frequent, and those you shouldn’t frequent.’ That’s what I said, but why did I say it? Well, should you know of a person: ‘When I frequent this person, unskillful qualities grow, and skillful qualities decline.’ You should not frequent that kind of person. Whereas, should you know of a person: ‘When I frequent this person, unskillful qualities decline, and skillful qualities grow.’ You should frequent that kind of person. ‘I say that there are two kinds of people: those you should frequent, and those you shouldn’t frequent.’ That’s what I said, and this is why I said it.” 

%
\section*{{\suttatitleacronym AN 10.55}{\suttatitletranslation Decline }{\suttatitleroot Parihānasutta}}
\addcontentsline{toc}{section}{\tocacronym{AN 10.55} \toctranslation{Decline } \tocroot{Parihānasutta}}
\markboth{Decline }{Parihānasutta}
\extramarks{AN 10.55}{AN 10.55}

There\marginnote{1.1} \textsanskrit{Sāriputta} addressed the mendicants: “Reverends, mendicants!” 

“Reverend,”\marginnote{1.3} they replied. \textsanskrit{Sāriputta} said this: 

“Reverends,\marginnote{2.1} they speak of a person liable to decline, and one not liable to decline. But how did the Buddha define a person liable to decline, and one not liable to decline?” 

“Reverend,\marginnote{2.4} we would travel a long way to learn the meaning of this statement in the presence of Venerable \textsanskrit{Sāriputta}. May Venerable \textsanskrit{Sāriputta} himself please clarify the meaning of this. The mendicants will listen and remember it.” 

“Then\marginnote{3.1} listen and apply your mind well, I will speak.” 

“Yes,\marginnote{3.2} reverend,” they replied. \textsanskrit{Sāriputta} said this: 

“How\marginnote{4.1} did the Buddha define a person liable to decline? It’s when a mendicant doesn’t get to hear a teaching they haven’t heard before. They forget those teachings they have heard. They don’t keep exercising the teachings with which they are already familiar. And they don’t come to understand what they haven’t understood before. That’s how the Buddha defined a person liable to decline. 

And\marginnote{5.1} how did the Buddha define a person not liable to decline? It’s when a mendicant gets to hear a teaching they haven’t heard before. They remember those teachings they have heard. They keep exercising the teachings with which they are already familiar. And they come to understand what they haven’t understood before. That’s how the Buddha defined a person not liable to decline. 

If\marginnote{6.1} a mendicant isn’t skilled in the ways of another’s mind, then they should train themselves: ‘I will be skilled in the ways of my own mind.’ 

And\marginnote{7.1} how is a mendicant skilled in the ways of their own mind? Suppose there was a woman or man who was young, youthful, and fond of adornments, and they check their own reflection in a clean bright mirror or a clear bowl of water. If they see any dirt or blemish there, they’d try to remove it. But if they don’t see any dirt or blemish there, they’re happy with that, as they’ve got all they wished for: ‘How fortunate that I’m clean!’ 

In\marginnote{7.5} the same way, checking is very helpful for a mendicant’s skillful qualities. ‘Is contentment often found in me or not? Is kind-heartedness often found in me or not? Is freedom from dullness and drowsiness often found in me or not? Is calm often found in me or not? Is confidence often found in me or not? Is love often found in me or not? Is purity of mind often found in me or not? Is internal joy with the teaching found in me or not? Is internal serenity of heart found in me or not? Is the higher wisdom of discernment of principles found in me or not?’ 

Suppose\marginnote{8.1} a mendicant, while checking, doesn’t see any of these skillful qualities in themselves. In order to get them they should apply intense enthusiasm, effort, zeal, vigor, perseverance, mindfulness, and situational awareness. Suppose your clothes or head were on fire. In order to extinguish it, you’d apply intense enthusiasm, effort, zeal, vigor, perseverance, mindfulness, and situational awareness. In the same way, they should apply intense enthusiasm to get those skillful qualities … 

Suppose\marginnote{9.1} a mendicant, while checking, sees some of these skillful qualities in themselves, but doesn’t see others. Grounded on the skillful qualities they see, they should apply intense enthusiasm, effort, zeal, vigor, perseverance, mindfulness, and situational awareness in order to get the skillful qualities they don’t see. Suppose your clothes or head were on fire. In order to extinguish it, you’d apply intense enthusiasm, effort, zeal, vigor, perseverance, mindfulness, and situational awareness. In the same way, grounded on the skillful qualities they see, they should apply intense enthusiasm to get those skillful qualities they don’t see. 

But\marginnote{10.1} suppose a mendicant, while checking, sees all of these skillful qualities in themselves. Grounded on all these skillful qualities they should practice meditation further to end the defilements.” 

%
\section*{{\suttatitleacronym AN 10.56}{\suttatitletranslation Perceptions (1st) }{\suttatitleroot Paṭhamasaññāsutta}}
\addcontentsline{toc}{section}{\tocacronym{AN 10.56} \toctranslation{Perceptions (1st) } \tocroot{Paṭhamasaññāsutta}}
\markboth{Perceptions (1st) }{Paṭhamasaññāsutta}
\extramarks{AN 10.56}{AN 10.56}

“Mendicants,\marginnote{1.1} these ten perceptions, when developed and cultivated, are very fruitful and beneficial. They culminate in freedom from death and end in freedom from death. What ten? The perceptions of ugliness, death, repulsiveness of food, dissatisfaction with the whole world, impermanence, suffering in impermanence, and not-self in suffering, giving up, fading away, and cessation. These ten perceptions, when developed and cultivated, are very fruitful and beneficial. They culminate in freedom from death and end in freedom from death.” 

%
\section*{{\suttatitleacronym AN 10.57}{\suttatitletranslation Perceptions (2nd) }{\suttatitleroot Dutiyasaññāsutta}}
\addcontentsline{toc}{section}{\tocacronym{AN 10.57} \toctranslation{Perceptions (2nd) } \tocroot{Dutiyasaññāsutta}}
\markboth{Perceptions (2nd) }{Dutiyasaññāsutta}
\extramarks{AN 10.57}{AN 10.57}

“Mendicants,\marginnote{1.1} these ten perceptions, when developed and cultivated, are very fruitful and beneficial. They culminate in freedom from death and end in freedom from death. What ten? The perceptions of impermanence, not-self, death, repulsiveness of food, dissatisfaction with the whole world, a skeleton, a worm-infested corpse, a livid corpse, a split open corpse, and a bloated corpse. These ten perceptions, when developed and cultivated, are very fruitful and beneficial. They culminate in freedom from death and end in freedom from death.” 

%
\section*{{\suttatitleacronym AN 10.58}{\suttatitletranslation Rooted }{\suttatitleroot Mūlakasutta}}
\addcontentsline{toc}{section}{\tocacronym{AN 10.58} \toctranslation{Rooted } \tocroot{Mūlakasutta}}
\markboth{Rooted }{Mūlakasutta}
\extramarks{AN 10.58}{AN 10.58}

“Mendicants,\marginnote{1.1} if wanderers of other religions were to ask: ‘Reverends, all things have what as their root? What produces them? What is their origin? What is their meeting place? What is their chief? What is their ruler? What is their overseer? What is their core? What is their culmination? What is their final end?’ How would you answer them?” 

“Our\marginnote{1.3} teachings are rooted in the Buddha. He is our guide and our refuge. Sir, may the Buddha himself please clarify the meaning of this. The mendicants will listen and remember it.” 

“Well\marginnote{2.1} then, mendicants, listen and apply your mind well, I will speak.” 

“Yes,\marginnote{2.2} sir,” they replied. The Buddha said this: 

“Mendicants,\marginnote{3.1} if wanderers of other religions were to ask: ‘Reverends, all things have what as their root? What produces them? What is their origin? What is their meeting place? What is their chief? What is their ruler? What is their overseer? What is their core? What is their culmination? What is their final end?’ You should answer them: ‘Reverends, all things are rooted in desire. They are produced by application of mind. Contact is their origin. Feeling is their meeting place. Immersion is their chief. Mindfulness is their ruler. Wisdom is their overseer. Freedom is their core. They culminate in freedom from death. And extinguishment is their final end.’ When questioned by wanderers of other religions, that’s how you should answer them.” 

%
\section*{{\suttatitleacronym AN 10.59}{\suttatitletranslation Going Forth }{\suttatitleroot Pabbajjāsutta}}
\addcontentsline{toc}{section}{\tocacronym{AN 10.59} \toctranslation{Going Forth } \tocroot{Pabbajjāsutta}}
\markboth{Going Forth }{Pabbajjāsutta}
\extramarks{AN 10.59}{AN 10.59}

“So\marginnote{1.1} you should train like this: ‘Our minds will be consolidated as they were when we went forth, and arisen bad unskillful qualities will not occupy our minds. Our minds will be consolidated in the perceptions of impermanence, not-self, ugliness, and drawbacks. Knowing what is fair and unfair in the world, our minds will be consolidated in that perception. Knowing continued existence and ending of existence in the world, our minds will be consolidated in that perception. Knowing the origination and ending of the world, our minds will be consolidated in that perception. Our minds will be consolidated in the perceptions of giving up, fading away, and cessation.’ That’s how you should train. 

When\marginnote{2.1} your minds are consolidated in these ten perceptions, you can expect one of two results: enlightenment in this very life, or if there’s something left over, non-return.” 

%
\section*{{\suttatitleacronym AN 10.60}{\suttatitletranslation With Girimānanda }{\suttatitleroot Girimānandasutta}}
\addcontentsline{toc}{section}{\tocacronym{AN 10.60} \toctranslation{With Girimānanda } \tocroot{Girimānandasutta}}
\markboth{With Girimānanda }{Girimānandasutta}
\extramarks{AN 10.60}{AN 10.60}

At\marginnote{1.1} one time the Buddha was staying near \textsanskrit{Sāvatthī} in Jeta’s Grove, \textsanskrit{Anāthapiṇḍika}’s monastery. Now at that time Venerable \textsanskrit{Girimānanda} was sick, suffering, gravely ill. Then Venerable Ānanda went up to the Buddha, bowed, sat down to one side, and said to him: 

“Sir,\marginnote{2.1} Venerable \textsanskrit{Girimānanda} is sick, suffering, gravely ill. Sir, please go to Venerable \textsanskrit{Girimānanda} out of sympathy.” 

“Ānanda,\marginnote{2.3} if you were to recite to the mendicant \textsanskrit{Girimānanda} these ten perceptions, it’s possible that after hearing them his illness will die down on the spot. 

What\marginnote{3.1} ten? The perceptions of impermanence, not-self, ugliness, drawbacks, giving up, fading away, cessation, dissatisfaction with the whole world, impermanence of all conditions, and mindfulness of breathing. 

And\marginnote{4.1} what is the perception of impermanence? It’s when a mendicant has gone to a wilderness, or to the root of a tree, or to an empty hut, and reflects like this: ‘Form, feeling, perception, choices, and consciousness are impermanent.’ And so they meditate observing impermanence in the five grasping aggregates. This is called the perception of impermanence. 

And\marginnote{5.1} what is the perception of not-self? It’s when a mendicant has gone to a wilderness, or to the root of a tree, or to an empty hut, and reflects like this: ‘The eye and sights, ear and sounds, nose and smells, tongue and tastes, body and touches, and mind and ideas are not-self.’ And so they meditate observing not-self in the six interior and exterior sense fields. This is called the perception of not-self. 

And\marginnote{6.1} what is the perception of ugliness? It’s when a mendicant examines their own body up from the soles of the feet and down from the tips of the hairs, wrapped in skin and full of many kinds of filth. ‘In this body there is head hair, body hair, nails, teeth, skin, flesh, sinews, bones, bone marrow, kidneys, heart, liver, diaphragm, spleen, lungs, intestines, mesentery, undigested food, feces, bile, phlegm, pus, blood, sweat, fat, tears, grease, saliva, snot, synovial fluid, urine.’ And so they meditate observing ugliness in this body. This is called the perception of ugliness. 

And\marginnote{7.1} what is the perception of drawbacks? It’s when a mendicant has gone to a wilderness, or to the root of a tree, or to an empty hut, and reflects like this: ‘This body has much suffering and many drawbacks. For this body is beset with many kinds of affliction, such as the following. Diseases of the eye, inner ear, nose, tongue, body, head, outer ear, mouth, teeth, and lips. Cough, asthma, catarrh, inflammation, fever, stomach ache, fainting, dysentery, gastric pain, cholera, leprosy, boils, eczema, tuberculosis, epilepsy, herpes, itch, scabs, smallpox, scabies, hemorrhage, diabetes, piles, pimples, and ulcers. Afflictions stemming from disorders of bile, phlegm, wind, or their conjunction. Afflictions caused by change in weather, by not taking care of yourself, by overexertion, or as the result of past deeds. Cold, heat, hunger, thirst, defecation, and urination.’ And so they meditate observing drawbacks in this body. This is called the perception of drawbacks. 

And\marginnote{8.1} what is the perception of giving up? It’s when a mendicant doesn’t tolerate a sensual, malicious, or cruel thought that has arisen, and they don’t tolerate any bad, unskillful qualities that have arisen, but give them up, get rid of them, eliminate them, and obliterate them. This is called the perception of giving up. 

And\marginnote{9.1} what is the perception of fading away? It’s when a mendicant has gone to a wilderness, or to the root of a tree, or to an empty hut, and reflects like this: ‘This is peaceful; this is sublime—that is, the stilling of all activities, the letting go of all attachments, the ending of craving, fading away, extinguishment.’ This is called the perception of fading away. 

And\marginnote{10.1} what is the perception of cessation? It’s when a mendicant has gone to a wilderness, or to the root of a tree, or to an empty hut, and reflects like this: ‘This is peaceful; this is sublime—that is, the stilling of all activities, the letting go of all attachments, the ending of craving, cessation, extinguishment.’ This is called the perception of cessation. 

And\marginnote{11.1} what is the perception of dissatisfaction with the whole world? It’s when a mendicant lives giving up and not grasping on to the attraction and grasping to the world, the mental fixation, insistence, and underlying tendencies. This is called the perception of dissatisfaction with the whole world. 

And\marginnote{12.1} what is the perception of the impermanence of all conditions? It’s when a mendicant is horrified, repelled, and disgusted with all conditions. This is called the perception of the impermanence of all conditions. 

And\marginnote{13.1} what is mindfulness of breathing? It’s when a mendicant has gone to a wilderness, or to the root of a tree, or to an empty hut, sits down cross-legged, sets their body straight, and establishes mindfulness in their presence. Just mindful, they breathe in. Mindful, they breathe out. Breathing in heavily they know: ‘I’m breathing in heavily.’ Breathing out heavily they know: ‘I’m breathing out heavily.’ When breathing in lightly they know: ‘I’m breathing in lightly.’ Breathing out lightly they know: ‘I’m breathing out lightly.’ They practice like this: ‘I’ll breathe in experiencing the whole body.’ They practice like this: ‘I’ll breathe out experiencing the whole body.’They practice like this: ‘I’ll breathe in stilling the physical process.’ They practice like this: ‘I’ll breathe out stilling the physical process.’ They practice like this: ‘I’ll breathe in experiencing rapture.’ They practice like this: ‘I’ll breathe out experiencing rapture.’ They practice like this: ‘I’ll breathe in experiencing bliss.’ They practice like this: ‘I’ll breathe out experiencing bliss.’ They practice like this: ‘I’ll breathe in experiencing mental processes.’ They practice like this: ‘I’ll breathe out experiencing mental processes.’They practice like this: ‘I’ll breathe in stilling mental processes.’ They practice like this: ‘I’ll breathe out stilling mental processes.’They practice like this: ‘I’ll breathe in experiencing the mind.’ They practice like this: ‘I’ll breathe out experiencing the mind.’ They practice like this: ‘I’ll breathe in gladdening the mind.’ They practice like this: ‘I’ll breathe out gladdening the mind.’ They practice like this: ‘I’ll breathe in immersing the mind in \textsanskrit{samādhi}.’ They practice like this: ‘I’ll breathe out immersing the mind in \textsanskrit{samādhi}.’ They practice like this: ‘I’ll breathe in freeing the mind.’ They practice like this: ‘I’ll breathe out freeing the mind.’ They practice like this: ‘I’ll breathe in observing impermanence.’ They practice like this: ‘I’ll breathe out observing impermanence.’ They practice like this: ‘I’ll breathe in observing fading away.’ They practice like this: ‘I’ll breathe out observing fading away.’ They practice like this: ‘I’ll breathe in observing cessation.’ They practice like this: ‘I’ll breathe out observing cessation.’ They practice like this: ‘I’ll breathe in observing letting go.’ They practice like this: ‘I’ll breathe out observing letting go.’ This is called mindfulness of breathing. 

If\marginnote{14.1} you were to recite to the mendicant \textsanskrit{Girimānanda} these ten perceptions, it’s possible that after hearing them his illness will die down on the spot.” 

Then\marginnote{15.1} Ānanda, having learned these ten perceptions from the Buddha himself, went to \textsanskrit{Girimānanda} and recited them. Then after \textsanskrit{Girimānanda} heard these ten perceptions his illness died down on the spot. And that’s how he recovered from that illness. 

%
\addtocontents{toc}{\let\protect\contentsline\protect\nopagecontentsline}
\chapter*{The Chapter on Pairs }
\addcontentsline{toc}{chapter}{\tocchapterline{The Chapter on Pairs }}
\addtocontents{toc}{\let\protect\contentsline\protect\oldcontentsline}

%
\section*{{\suttatitleacronym AN 10.61}{\suttatitletranslation Ignorance }{\suttatitleroot Avijjāsutta}}
\addcontentsline{toc}{section}{\tocacronym{AN 10.61} \toctranslation{Ignorance } \tocroot{Avijjāsutta}}
\markboth{Ignorance }{Avijjāsutta}
\extramarks{AN 10.61}{AN 10.61}

“Mendicants,\marginnote{1.1} it is said that no first point of ignorance is evident, before which there was no ignorance, and afterwards it came to be. And yet it is evident that there is a specific condition for ignorance. 

I\marginnote{2.1} say that ignorance is fueled by something, it’s not unfueled. And what is the fuel for ignorance? You should say: ‘The five hindrances.’ I say that the five hindrances are fueled by something, they’re not unfueled. And what is the fuel for the five hindrances? You should say: ‘The three kinds of misconduct.’ I say that the three kinds of misconduct are fueled by something, they’re not unfueled. And what is the fuel for the three kinds of misconduct? You should say: ‘Lack of sense restraint.’ I say that lack of sense restraint is fueled by something, it’s not unfueled. And what is the fuel for lack of sense restraint? You should say: ‘Lack of mindfulness and situational awareness.’ I say that lack of mindfulness and situational awareness is fueled by something, it’s not unfueled. And what is the fuel for lack of mindfulness and situational awareness? You should say: ‘Irrational application of mind.’ I say that irrational application of mind is fueled by something, it’s not unfueled. And what is the fuel for irrational application of mind? You should say: ‘Lack of faith.’ I say that lack of faith is fueled by something, it’s not unfueled. And what is the fuel for lack of faith? You should say: ‘Listening to an untrue teaching.’ I say that listening to an untrue teaching is fueled by something, it’s not unfueled. And what is the fuel for listening to an untrue teaching? You should say: ‘Associating with untrue persons.’ 

In\marginnote{3.1} this way, when the factor of associating with untrue persons is fulfilled, it fulfills the factor of listening to an untrue teaching. When the factor of listening to an untrue teaching is fulfilled, it fulfills the factor of lack of faith … irrational application of mind … lack of mindfulness and situational awareness … lack of sense restraint …the three kinds of misconduct … the five hindrances. When the five hindrances are fulfilled, they fulfill ignorance. That’s the fuel for ignorance, and that’s how it’s fulfilled. 

It’s\marginnote{4.1} like when the heavens rain heavily on a mountain top, and the water flows downhill to fill the hollows, crevices, and creeks. As they become full, they fill up the pools. The pools fill up the lakes, the lakes fill up the streams, and the streams fill up the rivers. And as the rivers become full, they fill up the ocean. That’s the fuel for the ocean, and that’s how it’s filled up. 

In\marginnote{5.1} the same way, when the factor of associating with untrue persons is fulfilled, it fulfills the factor of listening to an untrue teaching. When the factor of listening to an untrue teaching is fulfilled, it fulfills the factor of lack of faith … irrational application of mind … lack of mindfulness and situational awareness … lack of sense restraint …the three kinds of misconduct … the five hindrances. When the five hindrances are fulfilled, they fulfill ignorance. That’s the fuel for ignorance, and that’s how it’s fulfilled. 

I\marginnote{6.1} say that knowledge and freedom are fueled by something, they’re not unfueled. And what is the fuel for knowledge and freedom? You should say: ‘The seven awakening factors.’ I say that the seven awakening factors are fueled by something, they’re not unfueled. And what is the fuel for the seven awakening factors? You should say: ‘The four kinds of mindfulness meditation.’ I say that the four kinds of mindfulness meditation are fueled by something, they’re not unfueled. And what is the fuel for the four kinds of mindfulness meditation? You should say: ‘The three kinds of good conduct.’ I say that the three kinds of good conduct are fueled by something, they’re not unfueled. And what is the fuel for the three kinds of good conduct? You should say: ‘Sense restraint.’ I say that sense restraint is fueled by something, it’s not unfueled. And what is the fuel for sense restraint? You should say: ‘Mindfulness and situational awareness.’ I say that mindfulness and situational awareness is fueled by something, it’s not unfueled. And what is the fuel for mindfulness and situational awareness? You should say: ‘Rational application of mind.’ I say that rational application of mind is fueled by something, it’s not unfueled. And what is the fuel for rational application of mind? You should say: ‘Faith.’ I say that faith is fueled by something, it’s not unfueled. And what is the fuel for faith? You should say: ‘Listening to the true teaching.’ I say that listening to the true teaching is fueled by something, it’s not unfueled. And what is the fuel for listening to the true teaching? You should say: ‘Associating with true persons.’ 

In\marginnote{7.1} this way, when the factor of associating with true persons is fulfilled, it fulfills the factor of listening to the true teaching. When the factor of listening to the true teaching is fulfilled, it fulfills the factor of faith … rational application of mind … mindfulness and situational awareness … sense restraint …the three kinds of good conduct … the four kinds of mindfulness meditation … the seven awakening factors. When the seven awakening factors are fulfilled, they fulfill knowledge and freedom. That’s the fuel for knowledge and freedom, and that’s how it’s fulfilled. 

It’s\marginnote{8.1} like when the heavens rain heavily on a mountain top, and the water flows downhill to fill the hollows, crevices, and creeks. As they become full, they fill up the pools. The pools fill up the lakes, the lakes fill up the streams, and the streams fill up the rivers. And as the rivers become full, they fill up the ocean. That’s the fuel for the ocean, and that’s how it’s filled up. 

In\marginnote{9.1} the same way, when the factor of associating with true persons is fulfilled, it fulfills the factor of listening to the true teaching. When the factor of listening to the true teaching is fulfilled, it fulfills the factor of faith … rational application of mind … mindfulness and situational awareness … sense restraint …the three kinds of good conduct … the four kinds of mindfulness meditation … the seven awakening factors. When the seven awakening factors are fulfilled, they fulfill knowledge and freedom. That’s the fuel for knowledge and freedom, and that’s how it’s fulfilled.” 

%
\section*{{\suttatitleacronym AN 10.62}{\suttatitletranslation Craving }{\suttatitleroot Taṇhāsutta}}
\addcontentsline{toc}{section}{\tocacronym{AN 10.62} \toctranslation{Craving } \tocroot{Taṇhāsutta}}
\markboth{Craving }{Taṇhāsutta}
\extramarks{AN 10.62}{AN 10.62}

“Mendicants,\marginnote{1.1} it is said that no first point of craving for continued existence is evident, before which there was no craving for continued existence, and afterwards it came to be. And yet it is evident that there is a specific condition for craving for continued existence. 

I\marginnote{2.1} say that craving for continued existence is fueled by something, it’s not unfueled. And what is the fuel for craving for continued existence? You should say: ‘Ignorance.’ 

I\marginnote{2.4} say that ignorance is fueled by something, it’s not unfueled. And what is the fuel for ignorance? You should say: ‘The five hindrances.’ 

I\marginnote{2.7} say that the five hindrances are fueled by something, they’re not unfueled. And what is the fuel for the five hindrances? You should say: ‘The three kinds of misconduct.’ 

I\marginnote{2.10} say that the three kinds of misconduct are fueled by something, they’re not unfueled. And what is the fuel for the three kinds of misconduct? You should say: ‘Lack of sense restraint.’ 

I\marginnote{2.13} say that lack of sense restraint is fueled by something, it’s not unfueled. And what is the fuel for lack of sense restraint? You should say: ‘Lack of mindfulness and situational awareness.’ 

I\marginnote{2.16} say that lack of mindfulness and situational awareness is fueled by something, it’s not unfueled. And what is the fuel for lack of mindfulness and situational awareness? You should say: ‘Irrational application of mind.’ 

I\marginnote{2.19} say that irrational application of mind is fueled by something, it’s not unfueled. And what is the fuel for irrational application of mind? You should say: ‘Lack of faith.’ 

I\marginnote{2.22} say that lack of faith is fueled by something, it’s not unfueled. And what is the fuel for lack of faith? You should say: ‘Listening to an untrue teaching.’ 

I\marginnote{2.25} say that listening to an untrue teaching is fueled by something, it’s not unfueled. And what is the fuel for listening to an untrue teaching? You should say: ‘Associating with untrue persons.’ 

In\marginnote{3.1} this way, when the factor of associating with untrue persons is fulfilled, it fulfills the factor of listening to an untrue teaching. When the factor of listening to an untrue teaching is fulfilled, it fulfills the factor of lack of faith … irrational application of mind … lack of mindfulness and situational awareness … lack of sense restraint …the three kinds of misconduct … the five hindrances … ignorance. When ignorance is fulfilled, it fulfills craving for continued existence. That’s the fuel for craving for continued existence, and that’s how it’s fulfilled. 

It’s\marginnote{4.1} like when the heavens rain heavily on a mountain top, and the water flows downhill to fill the hollows, crevices, and creeks. As they become full, they fill up the pools. The pools fill up the lakes, the lakes fill up the streams, and the streams fill up the rivers. And as the rivers become full, they fill up the ocean. That’s the fuel for the ocean, and that’s how it’s filled up. 

In\marginnote{5.1} the same way, when the factor of associating with untrue persons is fulfilled, it fulfills the factor of listening to an untrue teaching. When the factor of listening to an untrue teaching is fulfilled, it fulfills the factor of lack of faith … irrational application of mind … lack of mindfulness and situational awareness … lack of sense restraint …the three kinds of misconduct … the five hindrances … ignorance. When ignorance is fulfilled, it fulfills craving for continued existence. That’s the fuel for craving for continued existence, and that’s how it’s fulfilled. 

I\marginnote{6.1} say that knowledge and freedom are fueled by something, they’re not unfueled. And what is the fuel for knowledge and freedom? You should say: ‘The seven awakening factors.’ 

I\marginnote{6.4} say that the seven awakening factors are fueled by something, they’re not unfueled. And what is the fuel for the seven awakening factors? You should say: ‘The four kinds of mindfulness meditation.’ 

I\marginnote{6.7} say that the four kinds of mindfulness meditation are fueled by something, they’re not unfueled. And what is the fuel for the four kinds of mindfulness meditation? You should say: ‘The three kinds of good conduct.’ 

I\marginnote{6.10} say that the three kinds of good conduct are fueled by something, they’re not unfueled. And what is the fuel for the three kinds of good conduct? You should say: ‘Sense restraint.’ 

I\marginnote{6.13} say that sense restraint is fueled by something, it’s not unfueled. And what is the fuel for sense restraint? You should say: ‘Mindfulness and situational awareness.’ 

I\marginnote{6.16} say that mindfulness and situational awareness is fueled by something, it’s not unfueled. And what is the fuel for mindfulness and situational awareness? You should say: ‘Rational application of mind.’ 

I\marginnote{6.19} say that rational application of mind is fueled by something, it’s not unfueled. And what is the fuel for rational application of mind? You should say: ‘Faith.’ 

I\marginnote{6.22} say that faith is fueled by something, it’s not unfueled. And what is the fuel for faith? You should say: ‘Listening to the true teaching.’ 

I\marginnote{6.25} say that listening to the true teaching is fueled by something, it’s not unfueled. And what is the fuel for listening to the true teaching? You should say: ‘Associating with true persons.’ 

In\marginnote{7.1} this way, when the factor of associating with true persons is fulfilled, it fulfills the factor of listening to the true teaching. When the factor of listening to the true teaching is fulfilled, it fulfills the factor of faith … rational application of mind … mindfulness and situational awareness … sense restraint …the three kinds of good conduct … the four kinds of mindfulness meditation … the seven awakening factors. When the seven awakening factors are fulfilled, they fulfill knowledge and freedom. That’s the fuel for knowledge and freedom, and that’s how it’s fulfilled. 

It’s\marginnote{8.1} like when the heavens rain heavily on a mountain top, and the water flows downhill to fill the hollows, crevices, and creeks. As they become full, they fill up the pools. The pools fill up the lakes, the lakes fill up the streams, and the streams fill up the rivers. And as the rivers become full, they fill up the ocean. That’s the fuel for the ocean, and that’s how it’s filled up. In this way, when the factor of associating with true persons is fulfilled, it fulfills the factor of listening to the true teaching. When the factor of listening to the true teaching is fulfilled, it fulfills the factor of faith … rational application of mind … mindfulness and situational awareness … sense restraint …the three kinds of good conduct … the four kinds of mindfulness meditation … the seven awakening factors. When the seven awakening factors are fulfilled, they fulfill knowledge and freedom. That’s the fuel for knowledge and freedom, and that’s how it’s fulfilled.” 

%
\section*{{\suttatitleacronym AN 10.63}{\suttatitletranslation Come to a Conclusion }{\suttatitleroot Niṭṭhaṅgatasutta}}
\addcontentsline{toc}{section}{\tocacronym{AN 10.63} \toctranslation{Come to a Conclusion } \tocroot{Niṭṭhaṅgatasutta}}
\markboth{Come to a Conclusion }{Niṭṭhaṅgatasutta}
\extramarks{AN 10.63}{AN 10.63}

“Mendicants,\marginnote{1.1} all those who have come to a conclusion about me are accomplished in view. Of those who are accomplished in view, five conclude their path in this realm, and five conclude their path after leaving this realm behind. Which five conclude their path in this realm? The one who has seven rebirths at most, the one who goes from family to family, the one-seeder, the once returner, and the one who is perfected in this very life. These five conclude their path in this realm. Which five conclude their path after leaving this realm behind? The one who is extinguished between one life and the next, the one who is extinguished upon landing, the one who is extinguished without extra effort, the one who is extinguished with extra effort, and the one who heads upstream, going to the \textsanskrit{Akaniṭṭha} realm. These five conclude their path after leaving this realm behind. All those who have come to a conclusion about me are accomplished in view. Of those who are accomplished in view, these five conclude their path in this realm, and these five conclude their path after leaving this realm behind.” 

%
\section*{{\suttatitleacronym AN 10.64}{\suttatitletranslation Experiential Confidence }{\suttatitleroot Aveccappasannasutta}}
\addcontentsline{toc}{section}{\tocacronym{AN 10.64} \toctranslation{Experiential Confidence } \tocroot{Aveccappasannasutta}}
\markboth{Experiential Confidence }{Aveccappasannasutta}
\extramarks{AN 10.64}{AN 10.64}

“Mendicants,\marginnote{1.1} all those who have experiential confidence in me have entered the stream. Of those who have entered the stream, five conclude their path in this realm, and five conclude their path after leaving this realm behind. Which five conclude their path in this realm? The one who has seven rebirths at most, the one who goes from family to family, the one-seeder, the once returner, and the one who is perfected in this very life. These five conclude their path in this realm. Which five conclude their path after leaving this realm behind? The one who is extinguished between one life and the next, the one who is extinguished upon landing, the one who is extinguished without extra effort, the one who is extinguished with extra effort, and the one who heads upstream, going to the \textsanskrit{Akaniṭṭha} realm. These five conclude their path after leaving this realm behind. All those who have experiential confidence in me have entered the stream. Of those who have entered the stream, these five conclude their path in this realm, and these five conclude their path after leaving this realm behind.” 

%
\section*{{\suttatitleacronym AN 10.65}{\suttatitletranslation Happiness (1st) }{\suttatitleroot Paṭhamasukhasutta}}
\addcontentsline{toc}{section}{\tocacronym{AN 10.65} \toctranslation{Happiness (1st) } \tocroot{Paṭhamasukhasutta}}
\markboth{Happiness (1st) }{Paṭhamasukhasutta}
\extramarks{AN 10.65}{AN 10.65}

At\marginnote{1.1} one time Venerable \textsanskrit{Sāriputta} was staying in the land of the Magadhans near the little village of \textsanskrit{Nālaka}. Then the wanderer \textsanskrit{Sāmaṇḍakāni} went up to Venerable \textsanskrit{Sāriputta} and exchanged greetings with him. When the greetings and polite conversation were over, \textsanskrit{Sāmaṇḍakāni} sat down to one side, and said to \textsanskrit{Sāriputta}: 

“Reverend\marginnote{2.1} \textsanskrit{Sāriputta}, what is happiness and what is suffering?” 

“Rebirth\marginnote{2.2} is suffering, reverend, no rebirth is happiness. When there is rebirth, you can expect this kind of suffering. Cold, heat, hunger, thirst, defecation, and urination. Contact with fire, clubs, and knives. And relatives and friends get together and annoy you. When there is rebirth, this is the kind of suffering you can expect. When there is no rebirth, you can expect this kind of happiness. No cold, heat, hunger, thirst, defecation, or urination. No contact with fire, clubs, or knives. And relatives and friends don’t get together and annoy you. When there is no rebirth, this is the kind of happiness you can expect.” 

%
\section*{{\suttatitleacronym AN 10.66}{\suttatitletranslation Happiness (2nd) }{\suttatitleroot Dutiyasukhasutta}}
\addcontentsline{toc}{section}{\tocacronym{AN 10.66} \toctranslation{Happiness (2nd) } \tocroot{Dutiyasukhasutta}}
\markboth{Happiness (2nd) }{Dutiyasukhasutta}
\extramarks{AN 10.66}{AN 10.66}

At\marginnote{1.1} one time Venerable \textsanskrit{Sāriputta} was staying in the land of the Magadhans near the little village of \textsanskrit{Nālaka}. Then the wanderer \textsanskrit{Sāmaṇḍakāni} went up to Venerable \textsanskrit{Sāriputta} and exchanged greetings with him. When the greetings and polite conversation were over, \textsanskrit{Sāmaṇḍakāni} sat down to one side and said to \textsanskrit{Sāriputta}: 

“Reverend\marginnote{2.1} \textsanskrit{Sāriputta}, in this teaching and training, what is happiness and what is suffering?” 

“Reverend,\marginnote{2.2} in this teaching and training dissatisfaction is suffering and satisfaction is happiness. When you’re dissatisfied, you can expect this kind of suffering. You find no happiness or pleasure while walking … standing … sitting … or lying down … or when in a village … a wilderness … at the root of a tree … an empty hut … the open air … or when among the mendicants. When you’re dissatisfied, this is the kind of suffering you can expect. 

When\marginnote{3.1} you’re satisfied, you can expect this kind of happiness. You find happiness or pleasure while walking … standing … sitting … or lying down … or when in a village … a wilderness … at the root of a tree … an empty hut … the open air … or when among the mendicants. When you’re satisfied, this is the kind of happiness you can expect.” 

%
\section*{{\suttatitleacronym AN 10.67}{\suttatitletranslation At Naḷakapāna (1st) }{\suttatitleroot Paṭhamanaḷakapānasutta}}
\addcontentsline{toc}{section}{\tocacronym{AN 10.67} \toctranslation{At Naḷakapāna (1st) } \tocroot{Paṭhamanaḷakapānasutta}}
\markboth{At Naḷakapāna (1st) }{Paṭhamanaḷakapānasutta}
\extramarks{AN 10.67}{AN 10.67}

At\marginnote{1.1} one time the Buddha was wandering in the land of the Kosalans together with a large \textsanskrit{Saṅgha} of mendicants when he arrived at a town of the Kosalans named \textsanskrit{Naḷakapāna}. There the Buddha stayed near \textsanskrit{Naḷakapāna} in the grove of flame-of-the-forest trees. Now, at that time it was the sabbath, and the Buddha was sitting surrounded by a \textsanskrit{Saṅgha} of monks. The Buddha spent much of the night educating, encouraging, firing up, and inspiring the mendicants with a Dhamma talk. Then he looked around the \textsanskrit{Saṅgha} of mendicants, who were so very silent. He addressed Venerable \textsanskrit{Sāriputta}: 

“\textsanskrit{Sāriputta},\marginnote{2.1} the \textsanskrit{Saṅgha} of mendicants is rid of dullness and drowsiness. Give them some Dhamma talk as you feel inspired. My back is sore, I’ll stretch it.” 

“Yes,\marginnote{2.5} sir,” \textsanskrit{Sāriputta} replied. 

And\marginnote{3.1} then the Buddha spread out his outer robe folded in four and laid down in the lion’s posture—on the right side, placing one foot on top of the other—mindful and aware, and focused on the time of getting up. There \textsanskrit{Sāriputta} addressed the mendicants: “Reverends, mendicants!” 

“Reverend,”\marginnote{3.4} they replied. \textsanskrit{Sāriputta} said this: 

“Reverends,\marginnote{4.1} whoever has no faith, conscience, prudence, energy, and wisdom when it comes to skillful qualities can expect decline, not growth, in skillful qualities, whether by day or by night. It’s like the moon in the waning fortnight. Whether by day or by night, its beauty, roundness, light, and diameter and circumference only decline. In the same way, whoever has no faith, conscience, prudence, energy, and wisdom when it comes to skillful qualities can expect decline, not growth, in skillful qualities, whether by day or by night. 

A\marginnote{5.1} faithless individual is in decline. An individual with no conscience is in decline. An imprudent individual is in decline. A lazy individual is in decline. A witless individual is in decline. An irritable individual is in decline. An acrimonious individual is in decline. An individual with corrupt wishes is in decline. An individual with bad friends is in decline. An individual with wrong view is in decline. 

Whoever\marginnote{6.1} has faith, conscience, prudence, energy, and wisdom when it comes to skillful qualities can expect growth, not decline, in skillful qualities, whether by day or by night. It’s like the moon in the waxing fortnight. Whether by day or by night, its beauty, roundness, light, and diameter and circumference only grow. In the same way, whoever has faith, conscience, prudence, energy, and wisdom when it comes to skillful qualities can expect growth, not decline, in skillful qualities, whether by day or by night. 

A\marginnote{7.1} faithful individual doesn’t decline. An individual with a conscience doesn’t decline. A prudent individual doesn’t decline. An energetic individual doesn’t decline. A wise individual doesn’t decline. A loving individual doesn’t decline. A kind individual doesn’t decline. An individual with few desires doesn’t decline. An individual with good friends doesn’t decline. An individual with right view doesn’t decline.” 

Then\marginnote{8.1} the Buddha got up and said to Venerable \textsanskrit{Sāriputta}: 

“Good,\marginnote{8.2} good, \textsanskrit{Sāriputta}! Whoever has no faith, conscience, prudence, energy, and wisdom when it comes to skillful qualities can expect decline, not growth, in skillful qualities, whether by day or by night. It’s like the moon in the waning fortnight. Whether by day or by night, its beauty, roundness, light, and diameter and circumference only decline. In the same way, whoever has no faith, conscience, prudence, energy, and wisdom when it comes to skillful qualities can expect decline, not growth, in skillful qualities, whether by day or by night. 

A\marginnote{9.1} faithless individual is in decline. An individual with no conscience … imprudent … lazy … witless … irritable … acrimonious … with corrupt wishes … bad friends … An individual with wrong view is in decline. 

Whoever\marginnote{10.1} has faith, conscience, prudence, energy, and wisdom when it comes to skillful qualities can expect growth, not decline, in skillful qualities, whether by day or by night. It’s like the moon in the waxing fortnight. Whether by day or by night, its beauty, roundness, light, and diameter and circumference only grow. In the same way, whoever has faith, conscience, prudence, energy, and wisdom when it comes to skillful qualities can expect growth, not decline, in skillful qualities, whether by day or by night. 

A\marginnote{11.1} faithful individual doesn’t decline. A conscientious individual … prudent … energetic … wise … loving … kind … with few desires … good friends … An individual with right view doesn’t decline.” 

%
\section*{{\suttatitleacronym AN 10.68}{\suttatitletranslation At Naḷakapāna (2nd) }{\suttatitleroot Dutiyanaḷakapānasutta}}
\addcontentsline{toc}{section}{\tocacronym{AN 10.68} \toctranslation{At Naḷakapāna (2nd) } \tocroot{Dutiyanaḷakapānasutta}}
\markboth{At Naḷakapāna (2nd) }{Dutiyanaḷakapānasutta}
\extramarks{AN 10.68}{AN 10.68}

At\marginnote{1.1} one time the Buddha stayed near \textsanskrit{Naḷakapāna} in the grove of flame-of-the-forest trees. 

Now,\marginnote{1.2} at that time it was the sabbath, and the Buddha was sitting surrounded by a \textsanskrit{Saṅgha} of monks. The Buddha spent much of the night educating, encouraging, firing up, and inspiring the mendicants with a Dhamma talk. Then he looked around the \textsanskrit{Saṅgha} of mendicants, who were so very silent. He addressed Venerable \textsanskrit{Sāriputta}, “\textsanskrit{Sāriputta}, the \textsanskrit{Saṅgha} of mendicants is rid of dullness and drowsiness. Give them some Dhamma talk as you feel inspired. My back is sore, I’ll stretch it.” 

“Yes,\marginnote{2.5} sir,” \textsanskrit{Sāriputta} replied. 

And\marginnote{3.1} then the Buddha spread out his outer robe folded in four and laid down in the lion’s posture—on the right side, placing one foot on top of the other—mindful and aware, and focused on the time of getting up. 

There\marginnote{3.2} \textsanskrit{Sāriputta} addressed the mendicants: “Reverends, mendicants!” 

“Reverend,”\marginnote{3.4} they replied. \textsanskrit{Sāriputta} said this: 

“Reverends,\marginnote{4.1} whoever has no faith, conscience, prudence, energy, and wisdom; who doesn’t want to listen, doesn’t memorize the teachings, examine their meaning, or practice accordingly, and is not diligent when it comes to skillful qualities can expect decline, not growth, in skillful qualities, whether by day or by night. It’s like the moon in the waning fortnight. Whether by day or by night, its beauty, roundness, light, and diameter and circumference only decline. In the same way, whoever has no faith, conscience, prudence, energy, and wisdom; who doesn’t want to listen, doesn’t memorize the teachings, examine their meaning, or practice accordingly, and is negligent when it comes to skillful qualities can expect decline, not growth, in skillful qualities, whether by day or by night. 

Whoever\marginnote{5.1} has faith, conscience, prudence, energy, and wisdom; who wants to listen, memorizes the teachings, examines their meaning, and practices accordingly, and is diligent when it comes to skillful qualities can expect growth, not decline, in skillful qualities, whether by day or by night. It’s like the moon in the waxing fortnight. Whether by day or by night, its beauty, roundness, light, and diameter and circumference only grow. In the same way, whoever has faith, conscience, prudence, energy, and wisdom; who wants to listen, memorizes the teachings, examines their meaning, and practices accordingly, and is diligent when it comes to skillful qualities can expect growth, not decline, in skillful qualities, whether by day or by night.” 

Then\marginnote{6.1} the Buddha got up and said to Venerable \textsanskrit{Sāriputta}: 

“Good,\marginnote{6.2} good, \textsanskrit{Sāriputta}! Whoever has no faith, conscience, prudence, energy, and wisdom; who doesn’t want to listen, doesn’t memorize the teachings, examine their meaning, or practice accordingly, and is negligent when it comes to skillful qualities can expect decline, not growth, in skillful qualities, whether by day or by night. It’s like the moon in the waning fortnight. Whether by day or by night, its beauty, roundness, light, and diameter and circumference only decline. In the same way, whoever has no faith, conscience, prudence, energy, and wisdom; who doesn’t want to listen, doesn’t memorize the teachings, examine their meaning, or practice accordingly, and is negligent when it comes to skillful qualities can expect decline, not growth, in skillful qualities, whether by day or by night. 

Whoever\marginnote{7.1} has faith, conscience, prudence, energy, and wisdom; who wants to listen, memorizes the teachings, examines their meaning, and practices accordingly, and is diligent when it comes to skillful qualities can expect growth, not decline, in skillful qualities, whether by day or by night. It’s like the moon in the waxing fortnight. Whether by day or by night, its beauty, roundness, light, and diameter and circumference only grow. In the same way, whoever has faith, conscience, prudence, energy, and wisdom; who wants to listen, memorizes the teachings, examines their meaning, and practices accordingly, and is diligent when it comes to skillful qualities can expect growth, not decline, in skillful qualities, whether by day or by night.” 

%
\section*{{\suttatitleacronym AN 10.69}{\suttatitletranslation Topics of Discussion (1st) }{\suttatitleroot Paṭhamakathāvatthusutta}}
\addcontentsline{toc}{section}{\tocacronym{AN 10.69} \toctranslation{Topics of Discussion (1st) } \tocroot{Paṭhamakathāvatthusutta}}
\markboth{Topics of Discussion (1st) }{Paṭhamakathāvatthusutta}
\extramarks{AN 10.69}{AN 10.69}

At\marginnote{1.1} one time the Buddha was staying near \textsanskrit{Sāvatthī} in Jeta’s Grove, \textsanskrit{Anāthapiṇḍika}’s monastery. Now at that time, after the meal, on return from almsround, several mendicants sat together in the assembly hall. They engaged in all kinds of low talk, such as talk about kings, bandits, and ministers; talk about armies, threats, and wars; talk about food, drink, clothes, and beds; talk about garlands and fragrances; talk about family, vehicles, villages, towns, cities, and countries; talk about women and heroes; street talk and well talk; talk about the departed; motley talk; tales of land and sea; and talk about being reborn in this or that place. 

Then\marginnote{2.1} in the late afternoon, the Buddha came out of retreat and went to the assembly hall, where he sat on the seat spread out and addressed the mendicants: “Mendicants, what were you sitting talking about just now? What conversation was left unfinished?” 

And\marginnote{3.1} they told him what had happened. 

“Mendicants,\marginnote{3.3} it is not appropriate for you gentlemen who have gone forth out of faith from the lay life to homelessness to engage in these kinds of low talk. 

There\marginnote{4.1} are, mendicants, these ten topics of discussion. What ten? Talk about fewness of wishes, contentment, seclusion, aloofness, arousing energy, ethics, immersion, wisdom, freedom, and the knowledge and vision of freedom. These are the ten topics of discussion. 

Mendicants,\marginnote{5.1} if you bring up these topics of conversation again and again then your glory could surpass even the sun and moon, so mighty and powerful, let alone the wanderers of other religions.” 

%
\section*{{\suttatitleacronym AN 10.70}{\suttatitletranslation Topics of Discussion (2nd) }{\suttatitleroot Dutiyakathāvatthusutta}}
\addcontentsline{toc}{section}{\tocacronym{AN 10.70} \toctranslation{Topics of Discussion (2nd) } \tocroot{Dutiyakathāvatthusutta}}
\markboth{Topics of Discussion (2nd) }{Dutiyakathāvatthusutta}
\extramarks{AN 10.70}{AN 10.70}

At\marginnote{1.1} one time the Buddha was staying near \textsanskrit{Sāvatthī} in Jeta’s Grove, \textsanskrit{Anāthapiṇḍika}’s monastery. 

Now\marginnote{1.2} at that time, after the meal, on return from almsround, several mendicants sat together in the assembly hall. They engaged in all kinds of low talk, such as talk about kings, bandits, and chief ministers; talk about armies, threats, and wars; talk about food, drink, clothes, and beds; talk about garlands and fragrances; talk about family, vehicles, villages, towns, cities, and nations; talk about women and heroes; street talk and well talk; talk about the departed; miscellaneous talk; tales of land and sea; and talk about being reborn in this or that state of existence. 

“Mendicants,\marginnote{2.1} there are ten grounds for praise. What ten? It’s when a mendicant personally has few wishes, and speaks to the mendicants on having few wishes. This is a ground for praise. 

A\marginnote{3.1} mendicant personally is content, and speaks to the mendicants on contentment. This is a ground for praise. 

A\marginnote{4.1} mendicant personally is secluded, and speaks to the mendicants on seclusion. This is a ground for praise. 

A\marginnote{5.1} mendicant personally doesn’t mix closely with others, and speaks to the mendicants on not mixing closely with others. This is a ground for praise. 

A\marginnote{6.1} mendicant personally is energetic, and speaks to the mendicants on rousing energy. This is a ground for praise. 

A\marginnote{7.1} mendicant personally is accomplished in ethics, and speaks to the mendicants on being accomplished in ethics. This is a ground for praise. 

A\marginnote{8.1} mendicant personally is accomplished in immersion, and speaks to the mendicants on being accomplished in immersion. This is a ground for praise. 

A\marginnote{9.1} mendicant personally is accomplished in wisdom, and speaks to the mendicants on being accomplished in wisdom. This is a ground for praise. 

A\marginnote{10.1} mendicant personally is accomplished in freedom, and speaks to the mendicants on being accomplished in freedom. This is a ground for praise. 

A\marginnote{11.1} mendicant personally is accomplished in the knowledge and vision of freedom, and speaks to the mendicants on being accomplished in the knowledge and vision of freedom. This is a ground for praise. 

These\marginnote{12.1} are the ten grounds for praise.” 

%
\addtocontents{toc}{\let\protect\contentsline\protect\nopagecontentsline}
\chapter*{The Chapter on If You Want }
\addcontentsline{toc}{chapter}{\tocchapterline{The Chapter on If You Want }}
\addtocontents{toc}{\let\protect\contentsline\protect\oldcontentsline}

%
\section*{{\suttatitleacronym AN 10.71}{\suttatitletranslation One Might Wish }{\suttatitleroot Ākaṅkhasutta}}
\addcontentsline{toc}{section}{\tocacronym{AN 10.71} \toctranslation{One Might Wish } \tocroot{Ākaṅkhasutta}}
\markboth{One Might Wish }{Ākaṅkhasutta}
\extramarks{AN 10.71}{AN 10.71}

At\marginnote{1.1} one time the Buddha was staying near \textsanskrit{Sāvatthī} in Jeta’s Grove, \textsanskrit{Anāthapiṇḍika}’s monastery. There the Buddha addressed the mendicants, “Mendicants!” 

“Venerable\marginnote{1.4} sir,” they replied. The Buddha said this: 

“Mendicants,\marginnote{2.1} live by the ethical precepts and the monastic code. Live restrained in the monastic code, conducting yourselves well and seeking alms in suitable places. Seeing danger in the slightest fault, keep the rules you’ve undertaken. 

A\marginnote{3.1} mendicant might wish: ‘May I be liked and approved by my spiritual companions, respected and admired.’ So let them fulfill their precepts, be committed to inner serenity of the heart, not neglect absorption, be endowed with discernment, and frequent empty huts. 

A\marginnote{4.1} mendicant might wish: ‘May I receive robes, almsfood, lodgings, and medicines and supplies for the sick.’ So let them fulfill their precepts, be committed to inner serenity of the heart, not neglect absorption, be endowed with discernment, and frequent empty huts. 

A\marginnote{5.1} mendicant might wish: ‘May the services of those whose robes, almsfood, lodgings, and medicines and supplies for the sick I enjoy be very fruitful and beneficial for them.’ So let them fulfill their precepts … 

A\marginnote{6.1} mendicant might wish: ‘When deceased family and relatives who have passed away recollect me with a confident mind, may this be very fruitful and beneficial for them.’ So let them fulfill their precepts … 

A\marginnote{7.1} mendicant might wish: ‘May I be content with any kind of robes, almsfood, lodgings, and medicines and supplies for the sick.’ So let them fulfill their precepts … 

A\marginnote{8.1} mendicant might wish: ‘May I endure cold, heat, hunger, and thirst. May I endure the touch of flies, mosquitoes, wind, sun, and reptiles. May I endure rude and unwelcome criticism. And may I put up with physical pain—sharp, severe, acute, unpleasant, disagreeable, and life-threatening.’ So let them fulfill their precepts … 

A\marginnote{9.1} mendicant might wish: ‘May I prevail over desire and discontent, and may desire and discontent not prevail over me. May I live having mastered desire and discontent whenever they have arisen.’ So let them fulfill their precepts … 

A\marginnote{10.1} mendicant might wish: ‘May I prevail over fear and dread, and may fear and dread not prevail over me. May I live having mastered fear and dread whenever they arise.’ So let them fulfill their precepts … 

A\marginnote{11.1} mendicant might wish: ‘May I get the four absorptions—blissful meditations in this life that belong to the higher mind—when I want, without trouble or difficulty.’ So let them fulfill their precepts … 

A\marginnote{12.1} mendicant might wish: ‘May I realize the undefiled freedom of heart and freedom by wisdom in this very life, and live having realized it with my own insight due to the ending of defilements.’ So let them fulfill their precepts, be committed to inner serenity of the heart, not neglect absorption, be endowed with discernment, and frequent empty huts. 

‘Live\marginnote{13.1} by the ethical precepts and the monastic code. Live restrained in the monastic code, conducting yourselves well and resorting for alms in suitable places. Seeing danger in the slightest fault, keep the rules you’ve undertaken.’ That’s what I said, and this is why I said it.” 

%
\section*{{\suttatitleacronym AN 10.72}{\suttatitletranslation Thorns }{\suttatitleroot Kaṇṭakasutta}}
\addcontentsline{toc}{section}{\tocacronym{AN 10.72} \toctranslation{Thorns } \tocroot{Kaṇṭakasutta}}
\markboth{Thorns }{Kaṇṭakasutta}
\extramarks{AN 10.72}{AN 10.72}

At\marginnote{1.1} one time the Buddha was staying near \textsanskrit{Vesālī}, at the Great Wood, in the hall with the peaked roof, together with several well-known senior disciples. They included Venerables \textsanskrit{Cāla}, \textsanskrit{Upacāla}, \textsanskrit{Kakkaṭa}, \textsanskrit{Kaṭimbha}, \textsanskrit{Kaṭa}, \textsanskrit{Kaṭissaṅga}, and other well-known senior disciples. 

Now\marginnote{2.1} at that time several well-known Licchavis plunged deep into the Great Wood to see the Buddha. Driving a succession of fine carriages, they made a dreadful racket. Then those venerables thought: 

“These\marginnote{2.3} several well-known Licchavis have plunged deep into the Great Wood to see the Buddha. Driving a succession of fine carriages, they’re making a dreadful racket. But the Buddha has said that sound is a thorn to absorption. Let’s go to the \textsanskrit{Gosiṅga} Sal Wood. There we can meditate comfortably, free of noise and crowds.” Then those venerables went to the \textsanskrit{Gosiṅga} Sal Wood, where they meditated comfortably, free of noise and crowds. 

Then\marginnote{3.1} the Buddha said to the mendicants: 

“Mendicants,\marginnote{3.2} where are \textsanskrit{Cāla}, \textsanskrit{Upacāla}, \textsanskrit{Kakkaṭa}, \textsanskrit{Kaṭimbha}, \textsanskrit{Kaṭa}, and \textsanskrit{Kaṭissaṅga}? Where have these senior disciples gone?” 

And\marginnote{4.1} the mendicants told him what had happened. 

“Good,\marginnote{5.1} good, mendicants! It’s just as those great disciples have so rightly explained. I have said that sound is a thorn to absorption. 

Mendicants,\marginnote{6.1} there are these ten thorns. What ten? Relishing company is a thorn for someone who loves seclusion. Focusing on the beautiful feature of things is a thorn for someone pursuing the meditation on ugliness. Seeing shows is a thorn to someone restraining the senses. Lingering in the neighborhood of females is a thorn to celibacy. Sound is a thorn to the first absorption. Placing the mind and keeping it connected are a thorn to the second absorption. Rapture is a thorn to the third absorption. Breathing is a thorn to the fourth absorption. Perception and feeling are a thorn to the attainment of the cessation of perception and feeling. Greed, hate, and delusion are thorns. 

Mendicants,\marginnote{7.1} live free of thorns! Live rid of thorns! Mendicants, live free of thorns and rid of thorns! The perfected ones live free of thorns, rid of thorns, free and rid of thorns.” 

%
\section*{{\suttatitleacronym AN 10.73}{\suttatitletranslation Likable }{\suttatitleroot Iṭṭhadhammasutta}}
\addcontentsline{toc}{section}{\tocacronym{AN 10.73} \toctranslation{Likable } \tocroot{Iṭṭhadhammasutta}}
\markboth{Likable }{Iṭṭhadhammasutta}
\extramarks{AN 10.73}{AN 10.73}

“Mendicants,\marginnote{1.1} these ten likable, desirable, and agreeable things are rare in the world. What ten? Wealth, beauty, health, ethical conduct, the spiritual life, friends, learning, wisdom, good qualities, and heaven are likable, desirable, and agreeable things that are rare in the world. 

Ten\marginnote{2.1} things are roadblocks for these ten likable, desirable, and agreeable things that are rare in the world. Sloth and lack of initiative are a roadblock for wealth. Lack of adornment and decoration are a roadblock for beauty. Unsuitable activity is a roadblock for health. Bad friendship is a roadblock for ethical conduct. Lack of sense restraint is a roadblock for the spiritual life. Dishonesty is a roadblock for friends. Not reciting is a roadblock for learning. Not wanting to listen and ask questions are roadblocks for wisdom. Lack of commitment and reviewing are roadblocks for good qualities. Wrong practice hinders heaven. These ten things are roadblocks for these ten likable, desirable, and agreeable things that are rare in the world. 

Ten\marginnote{3.1} things nourish these ten likable, desirable, and agreeable things that are rare in the world. Application and initiative nourish wealth. Adornment and decoration nourish beauty. Suitable activity nourishes health. Good friendship nourishes ethical conduct. Sense restraint nourishes the spiritual life. Honesty nourishes friends. Reciting nourishes learning. Eagerness to listen and ask questions nourishes wisdom. Commitment and reviewing nourish good qualities. Right practice nourishes heaven. These ten things nourish these ten likable, desirable, and agreeable things that are rare in the world.” 

%
\section*{{\suttatitleacronym AN 10.74}{\suttatitletranslation Growth }{\suttatitleroot Vaḍḍhisutta}}
\addcontentsline{toc}{section}{\tocacronym{AN 10.74} \toctranslation{Growth } \tocroot{Vaḍḍhisutta}}
\markboth{Growth }{Vaḍḍhisutta}
\extramarks{AN 10.74}{AN 10.74}

“Mendicants,\marginnote{1.1} a noble disciple who grows in ten ways grows nobly, taking on what is essential and excellent in this life. What ten? He grows in fields and lands, money and grain, wives and children, in bondservants, workers, and staff, and in livestock. And he grows in faith, ethics, learning, generosity, and wisdom. A noble disciple who grows in ten ways grows nobly, taking on what is essential and excellent in this life. 

\begin{verse}%
Someone\marginnote{2.1} who grows in money and grain, \\
in wives, children, and livestock, \\
is wealthy, famous, and respected \\
by relatives and friends, and even by royals. 

When\marginnote{3.1} someone grows in faith and ethics, \\
wisdom, and both generosity and learning—\\
a good man such as he sees clearly, \\
and in this very life he grows in both ways.” 

%
\end{verse}

%
\section*{{\suttatitleacronym AN 10.75}{\suttatitletranslation With Migasālā }{\suttatitleroot Migasālāsutta}}
\addcontentsline{toc}{section}{\tocacronym{AN 10.75} \toctranslation{With Migasālā } \tocroot{Migasālāsutta}}
\markboth{With Migasālā }{Migasālāsutta}
\extramarks{AN 10.75}{AN 10.75}

At\marginnote{1.1} one time the Buddha was staying near \textsanskrit{Sāvatthī} in Jeta’s Grove, \textsanskrit{Anāthapiṇḍika}’s monastery. Then Venerable Ānanda robed up in the morning and, taking his bowl and robe, went to the home of the laywoman \textsanskrit{Migasālā}, where he sat on the seat spread out. Then the laywoman \textsanskrit{Migasālā} went up to Ānanda, bowed, sat down to one side, and said to him: 

“Honorable\marginnote{2.1} Ānanda, how on earth are we supposed to understand the teaching taught by the Buddha, when the chaste and the unchaste are both reborn in exactly the same place in the next life? My father \textsanskrit{Purāṇa} was celibate, set apart, avoiding the vulgar act of sex. When he passed away the Buddha declared that, since he was a once-returner, he was reborn in the host of joyful gods. But my uncle Isidatta was not celibate; he lived content with his wife. When he passed away the Buddha also declared that, since he was a once-returner, he was reborn in the host of joyful gods. 

How\marginnote{3.1} on earth are we supposed to understand the teaching taught by the Buddha, when the chaste and the unchaste are both reborn in exactly the same place in the next life?” 

“You’re\marginnote{3.2} right, sister, but that’s how the Buddha declared it.” 

Then\marginnote{4.1} Ānanda, after receiving almsfood at \textsanskrit{Migasālā}’s home, rose from his seat and left. Then after the meal, on his return from almsround, Ānanda went to the Buddha, bowed, sat down to one side, and told him what had happened. 

“Ānanda,\marginnote{8.1} who is this laywoman \textsanskrit{Migasālā}, a foolish incompetent aunty, with an aunty’s wit? And who is it that knows how to assess individuals? 

These\marginnote{9.1} ten people are found in the world. What ten? Take a certain person who is unethical. And they don’t truly understand the freedom of heart and freedom by wisdom where that unethical conduct ceases without anything left over. And they’ve not listened or learned or comprehended theoretically or found even temporary freedom. When their body breaks up, after death, they’re headed for a lower place, not a higher. They’re going to a lower place, not a higher. 

Take\marginnote{10.1} a certain person who is unethical. But they truly understand the freedom of heart and freedom by wisdom where that unethical conduct ceases without anything left over. And they have listened and learned and comprehended theoretically and found at least temporary freedom. When their body breaks up, after death, they’re headed for a higher place, not a lower. They’re going to a higher place, not a lower. 

Judgmental\marginnote{11.1} people compare them, saying: ‘This one has just the same qualities as the other, so why is one worse and one better?’ This will be for their lasting harm and suffering. 

In\marginnote{12.1} this case, the person who is unethical, but truly understands the freedom of heart … and has listened and learned and comprehended theoretically and found at least temporary freedom is better and finer than the other person. Why is that? Because the stream of the teaching carries them along. But who knows the difference between them except a Realized One? So, Ānanda, don’t be judgmental about people. Don’t pass judgment on people. Those who pass judgment on people harm themselves. I, or someone like me, may pass judgment on people. 

Take\marginnote{13.1} a certain person who is ethical. But they don’t truly understand the freedom of heart and freedom by wisdom where that ethical conduct ceases without anything left over. And they’ve not listened or learned or comprehended theoretically or found even temporary freedom. When their body breaks up, after death, they’re headed for a lower place, not a higher. They’re going to a lower place, not a higher. 

Take\marginnote{14.1} a certain person who is ethical. And they truly understand the freedom of heart and freedom by wisdom where that ethical conduct ceases without anything left over. And they’ve listened and learned and comprehended theoretically and found at least temporary freedom. When their body breaks up, after death, they’re headed for a higher place, not a lower. They’re going to a higher place, not a lower. 

Judgmental\marginnote{15.1} people compare them … I, or someone like me, may pass judgment on people. 

Take\marginnote{16.1} a certain person who is very lustful. And they don’t truly understand the freedom of heart and freedom by wisdom where that lust ceases without anything left over. And they’ve not listened or learned or comprehended theoretically or found even temporary freedom. When their body breaks up, after death, they’re headed for a lower place, not a higher. They’re going to a lower place, not a higher. 

Take\marginnote{17.1} a certain person who is very lustful. But they truly understand the freedom of heart and freedom by wisdom where that lust ceases without anything left over. And they’ve listened and learned and comprehended theoretically and found at least temporary freedom. When their body breaks up, after death, they’re headed for a higher place, not a lower. They’re going to a higher place, not a lower. 

Judgmental\marginnote{18.1} people compare them … I, or someone like me, may pass judgment on people. 

Take\marginnote{19.1} a certain person who is irritable. And they don’t truly understand the freedom of heart and freedom by wisdom where that anger ceases without anything left over. And they’ve not listened or learned or comprehended theoretically or found even temporary freedom. When their body breaks up, after death, they’re headed for a lower place, not a higher. They’re going to a lower place, not a higher. 

Take\marginnote{20.1} a certain person who is irritable. But they truly understand the freedom of heart and freedom by wisdom where that anger ceases without anything left over. And they’ve listened and learned and comprehended theoretically and found at least temporary freedom. When their body breaks up, after death, they’re headed for a higher place, not a lower. They’re going to a higher place, not a lower. 

Judgmental\marginnote{21.1} people compare them … I, or someone like me, may pass judgment on people. 

Take\marginnote{22.1} a certain person who is restless. And they don’t truly understand the freedom of heart and freedom by wisdom where that restlessness ceases without anything left over. And they’ve not listened or learned or comprehended theoretically or found even temporary freedom. When their body breaks up, after death, they’re headed for a lower place, not a higher. They’re going to a lower place, not a higher. 

Take\marginnote{23.1} a certain person who is restless. But they truly understand the freedom of heart and freedom by wisdom where that restlessness ceases without anything left over. And they’ve listened and learned and comprehended theoretically and found at least temporary freedom. When their body breaks up, after death, they’re headed for a higher place, not a lower. They’re going to a higher place, not a lower. 

Judgmental\marginnote{24.1} people compare them, saying: ‘This one has just the same qualities as the other, so why is one worse and one better?’ This will be for their lasting harm and suffering. 

In\marginnote{25.1} this case the person who is restless, but truly understands the freedom of heart … and has listened and learned and comprehended theoretically and found at least temporary freedom is better and finer than the other person. Why is that? Because the stream of the teaching carries them along. But who knows the difference between them except a Realized One? So, Ānanda, don’t be judgmental about people. Don’t pass judgment on people. Those who pass judgment on people harm themselves. I, or someone like me, may pass judgment on people. 

Who\marginnote{26.1} is this laywoman \textsanskrit{Migasālā}, a foolish incompetent aunty, with an aunty’s wit? And who is it that knows how to assess individuals? These ten people are found in the world. 

If\marginnote{27.1} Isidatta had achieved \textsanskrit{Purāṇa}’s level of ethical conduct, \textsanskrit{Purāṇa} could not have even known Isidatta’s destination. And if \textsanskrit{Purāṇa} had achieved Isidatta’s level of wisdom, Isidatta could not have even known \textsanskrit{Purāṇa}’s destination. So both individuals were lacking in one respect.” 

%
\section*{{\suttatitleacronym AN 10.76}{\suttatitletranslation Three Things }{\suttatitleroot Tayodhammasutta}}
\addcontentsline{toc}{section}{\tocacronym{AN 10.76} \toctranslation{Three Things } \tocroot{Tayodhammasutta}}
\markboth{Three Things }{Tayodhammasutta}
\extramarks{AN 10.76}{AN 10.76}

“Mendicants,\marginnote{1.1} if three things were not found, the Realized One, the perfected one, the fully awakened Buddha would not arise in the world, and the teaching and training proclaimed by the Realized One would not shine in the world. What three? Rebirth, old age, and death. If these three things were not found, the Realized One, the perfected one, the fully awakened Buddha would not arise in the world, and the teaching and training proclaimed by the Realized One would not shine in the world. But since these three things are found, the Realized One, the perfected one, the fully awakened Buddha arises in the world, and the teaching and training proclaimed by the Realized One shines in the world. 

Without\marginnote{2.1} giving up three things you can’t give up rebirth, old age, and death. What three? Greed, hate, and delusion. Without giving up these three things you can’t give up rebirth, old age, and death. 

Without\marginnote{3.1} giving up three things you can’t give up greed, hate, and delusion. What three? Substantialist view, doubt, and misapprehension of precepts and observances. Without giving up these three things you can’t give up greed, hate, and delusion. 

Without\marginnote{4.1} giving up three things you can’t give up substantialist view, doubt, and misapprehension of precepts and observances. What three? Irrational application of mind, following a wrong path, and mental sluggishness. Without giving up these three things you can’t give up substantialist view, doubt, and misapprehension of precepts and observances. 

Without\marginnote{5.1} giving up three things you can’t give up irrational application of mind, following a wrong path, and mental sluggishness. What three? Unmindfulness, lack of situational awareness, and scattered mind. Without giving up these three things you can’t give up irrational application of mind, following a wrong path, and mental sluggishness. 

Without\marginnote{6.1} giving up three things you can’t give up unmindfulness, lack of situational awareness, and scattered mind. What three? Not wanting to see the noble ones, not wanting to hear the teaching of the noble ones, and a fault-finding mind. Without giving up these three things you can’t give up unmindfulness, lack of situational awareness, and scattered mind. 

Without\marginnote{7.1} giving up three things you can’t give up not wanting to see the noble ones, not wanting to hear the teaching of the noble ones, and a fault-finding mind. What three? Restlessness, lack of restraint, and unethical conduct. Without giving up these three things you can’t give up not wanting to see the noble ones, not wanting to hear the teaching of the noble ones, and a fault-finding mind. 

Without\marginnote{8.1} giving up three things you can’t give up restlessness, lack of restraint, and unethical conduct. What three? Faithlessness, uncharitableness, and laziness. Without giving up these three things you can’t give up restlessness, lack of restraint, and unethical conduct. 

Without\marginnote{9.1} giving up three things you can’t give up faithlessness, uncharitableness, and laziness. What three? Disregard, being hard to admonish, and having bad friends. Without giving up these three things you can’t give up faithlessness, uncharitableness, and laziness. 

Without\marginnote{10.1} giving up three things you can’t give up disregard, being hard to admonish, and having bad friends. What three? Lack of conscience, imprudence, and negligence. Without giving up these three things you can’t give up disregard, being hard to admonish, and having bad friends. 

Mendicants,\marginnote{11.1} someone who lacks conscience and prudence is negligent. When you’re negligent you can’t give up disregard, being hard to admonish, and having bad friends. When you’ve got bad friends you can’t give up faithlessness, uncharitableness, and laziness. When you’re lazy you can’t give up restlessness, lack of restraint, and unethical conduct. When you’re unethical you can’t give up not wanting to see the noble ones, not wanting to hear the teaching of the noble ones, and a fault-finding mind. When you’ve got a fault-finding mind you can’t give up unmindfulness, lack of situational awareness, and a scattered mind. When your mind is scattered you can’t give up irrational application of mind, following a wrong path, and mental sluggishness. When your mind is sluggish you can’t give up substantialist view, doubt, and misapprehension of precepts and observances. When you have doubts you can’t give up greed, hate, and delusion. Without giving up greed, hate, and delusion you can’t give up rebirth, old age, and death. 

After\marginnote{12.1} giving up three things you can give up rebirth, old age, and death. What three? Greed, hate, and delusion. After giving up these three things you can give up rebirth, old age, and death. 

After\marginnote{13.1} giving up three things you can give up greed, hate, and delusion. What three? Substantialist view, doubt, and misapprehension of precepts and observances. After giving up these three things you can give up greed, hate, and delusion. 

After\marginnote{14.1} giving up three things you can give up substantialist view, doubt, and misapprehension of precepts and observances. What three? Irrational application of mind, following a wrong path, and mental sluggishness. After giving up these three things you can give up substantialist view, doubt, and misapprehension of precepts and observances. 

After\marginnote{15.1} giving up three things you can give up irrational application of mind, following a wrong path, and mental sluggishness. What three? Unmindfulness, lack of situational awareness, and a scattered mind. After giving up these three things you can give up irrational application of mind, following a wrong path, and mental sluggishness. 

After\marginnote{16.1} giving up three things you can give up unmindfulness, lack of situational awareness, and scattered mind. What three? Not wanting to see the noble ones, not wanting to hear the teaching of the noble ones, and a fault-finding mind. After giving up these three things you can give up unmindfulness, lack of situational awareness, and scattered mind. 

After\marginnote{17.1} giving up three things you can give up not wanting to see the noble ones, not wanting to hear the teaching of the noble ones, and a fault-finding mind. What three? Restlessness, lack of restraint, and unethical conduct. After giving up these three things you can give up not wanting to see the noble ones, not wanting to hear the teaching of the noble ones, and a fault-finding mind. 

After\marginnote{18.1} giving up three things you can give up restlessness, lack of restraint, and unethical conduct. What three? Faithlessness, uncharitableness, and laziness. After giving up these three things you can give up restlessness, lack of restraint, and unethical conduct. 

After\marginnote{19.1} giving up three things you can give up faithlessness, uncharitableness, and laziness. What three? Disregard, being hard to admonish, and having bad friends. After giving up these three things you can give up faithlessness, uncharitableness, and laziness. 

After\marginnote{20.1} giving up three things you can give up disregard, being hard to admonish, and having bad friends. What three? Lack of conscience, imprudence, and negligence. After giving up these three things you can give up disregard, being hard to admonish, and having bad friends. 

Mendicants,\marginnote{21.1} someone who has conscience and prudence is diligent. When you’re diligent you can give up disregard, being hard to admonish, and having bad friends. When you’ve got good friends you can give up faithlessness, uncharitableness, and laziness. When you’re energetic you can give up restlessness, lack of restraint, and unethical conduct. When you’re ethical you can give up not wanting to see the noble ones, not wanting to hear the teaching of the noble ones, and a fault-finding mind. When you don’t have a fault-finding mind you can give up unmindfulness, lack of situational awareness, and a scattered mind. When your mind isn’t scattered you can give up irrational application of mind, following a wrong path, and mental sluggishness. When your mind isn’t sluggish you can give up substantialist view, doubt, and misapprehension of precepts and observances. When you have no doubts you can give up greed, hate, and delusion. After giving up greed, hate, and delusion you can give up rebirth, old age, and death.” 

%
\section*{{\suttatitleacronym AN 10.77}{\suttatitletranslation A Crow }{\suttatitleroot Kākasutta}}
\addcontentsline{toc}{section}{\tocacronym{AN 10.77} \toctranslation{A Crow } \tocroot{Kākasutta}}
\markboth{A Crow }{Kākasutta}
\extramarks{AN 10.77}{AN 10.77}

“Mendicants,\marginnote{1.1} a crow has ten bad qualities. What ten? They’re rude and impudent, gluttonous and voracious, cruel and pitiless, weak and raucous, unmindful and acquisitive. A crow has these ten bad qualities. In the same way, a bad mendicant has these ten bad qualities. What ten? They’re rude and impudent, gluttonous and voracious, cruel and pitiless, weak and raucous, unmindful and acquisitive. A bad mendicant has these ten bad qualities.” 

%
\section*{{\suttatitleacronym AN 10.78}{\suttatitletranslation Jains }{\suttatitleroot Nigaṇṭhasutta}}
\addcontentsline{toc}{section}{\tocacronym{AN 10.78} \toctranslation{Jains } \tocroot{Nigaṇṭhasutta}}
\markboth{Jains }{Nigaṇṭhasutta}
\extramarks{AN 10.78}{AN 10.78}

“Mendicants,\marginnote{1.1} Jain ascetics have ten bad qualities. What ten? They’re faithless and unethical, without conscience or prudence, and devoted to untrue persons. They glorify themselves and put others down. They’re attached to their own views, holding them tight, and refusing to let go. They’re deceptive, with corrupt wishes and bad friends. Jain ascetics have these ten bad qualities.” 

%
\section*{{\suttatitleacronym AN 10.79}{\suttatitletranslation Grounds for Resentment }{\suttatitleroot Āghātavatthusutta}}
\addcontentsline{toc}{section}{\tocacronym{AN 10.79} \toctranslation{Grounds for Resentment } \tocroot{Āghātavatthusutta}}
\markboth{Grounds for Resentment }{Āghātavatthusutta}
\extramarks{AN 10.79}{AN 10.79}

“Mendicants,\marginnote{1.1} there are ten grounds for resentment. What ten? Thinking: ‘They did wrong to me,’ you harbor resentment. Thinking: ‘They are doing wrong to me’ … ‘They will do wrong to me’ … ‘They did wrong by someone I love’ … ‘They are doing wrong by someone I love’ … ‘They will do wrong by someone I love’ … ‘They helped someone I dislike’ … ‘They are helping someone I dislike’ … Thinking: ‘They will help someone I dislike,’ you harbor resentment. You get angry for no reason. These are the ten grounds for resentment.” 

%
\section*{{\suttatitleacronym AN 10.80}{\suttatitletranslation Getting Rid of Resentment }{\suttatitleroot Āghātapaṭivinayasutta}}
\addcontentsline{toc}{section}{\tocacronym{AN 10.80} \toctranslation{Getting Rid of Resentment } \tocroot{Āghātapaṭivinayasutta}}
\markboth{Getting Rid of Resentment }{Āghātapaṭivinayasutta}
\extramarks{AN 10.80}{AN 10.80}

“Mendicants,\marginnote{1.1} there are these ten methods to get rid of resentment. What ten? Thinking: ‘They harmed me, but what can I possibly do?’ you get rid of resentment. Thinking: ‘They are harming me …’ … ‘They will harm me …’ … ‘They harmed someone I love …’ … ‘They are harming someone I love …’ ‘They will harm someone I love …’ … They helped someone I dislike …’ … ‘They are helping someone I dislike …’ … Thinking: ‘They will help someone I dislike, but what can I possibly do?’ you get rid of resentment. And you don’t get angry for no reason. These are the ten ways of getting rid of resentment.” 

%
\addtocontents{toc}{\let\protect\contentsline\protect\nopagecontentsline}
\chapter*{The Chapter on Senior Mendicants }
\addcontentsline{toc}{chapter}{\tocchapterline{The Chapter on Senior Mendicants }}
\addtocontents{toc}{\let\protect\contentsline\protect\oldcontentsline}

%
\section*{{\suttatitleacronym AN 10.81}{\suttatitletranslation With Bāhuna }{\suttatitleroot Vāhanasutta}}
\addcontentsline{toc}{section}{\tocacronym{AN 10.81} \toctranslation{With Bāhuna } \tocroot{Vāhanasutta}}
\markboth{With Bāhuna }{Vāhanasutta}
\extramarks{AN 10.81}{AN 10.81}

At\marginnote{1.1} one time the Buddha was staying near \textsanskrit{Campā} on the banks of the \textsanskrit{Gaggarā} Lotus Pond. Then Venerable \textsanskrit{Bāhuna} went up to the Buddha, bowed, sat down to one side, and said to him: 

“Sir,\marginnote{1.3} how many things has the Realized One escaped from, so that he lives detached, liberated, his mind free of limits?” 

“\textsanskrit{Bāhuna},\marginnote{2.1} the Realized One has escaped from ten things, so that he lives detached, liberated, his mind free of limits. What ten? Form … feeling … perception … choices … consciousness … rebirth … old age … death … suffering … defilements … Suppose there was a blue water lily, or a pink or white lotus. Though it sprouted and grew in the water, it would rise up above the water and stand with no water clinging to it. In the same way, the Realized One has escaped from ten things, so that he lives detached, liberated, his mind free of limits.” 

%
\section*{{\suttatitleacronym AN 10.82}{\suttatitletranslation With Ānanda }{\suttatitleroot Ānandasutta}}
\addcontentsline{toc}{section}{\tocacronym{AN 10.82} \toctranslation{With Ānanda } \tocroot{Ānandasutta}}
\markboth{With Ānanda }{Ānandasutta}
\extramarks{AN 10.82}{AN 10.82}

Then\marginnote{1.1} Venerable Ānanda went up to the Buddha, bowed, and sat down to one side. The Buddha said to him: 

“Ānanda,\marginnote{2.1} it is quite impossible for a faithless mendicant to achieve growth, improvement, or maturity in this teaching and training. 

It\marginnote{3.1} is quite impossible for a mendicant who is unethical … 

unlearned\marginnote{4.1} … 

hard\marginnote{5.1} to admonish … 

with\marginnote{6.1} bad friends … 

lazy\marginnote{7.1} … 

unmindful\marginnote{8.1} … 

lacking\marginnote{9.1} contentment … 

of\marginnote{10.1} corrupt wishes … 

of\marginnote{11.1} wrong view to achieve growth, improvement, or maturity in this teaching and training. 

It\marginnote{12.1} is quite impossible for a mendicant with these ten qualities to achieve growth, improvement, or maturity in this teaching and training. 

It\marginnote{13.1} is quite possible for a faithful mendicant to achieve growth, improvement, or maturity in this teaching and training. 

It\marginnote{14.1} is quite possible for a mendicant who is ethical … 

a\marginnote{15.1} learned memorizer … 

easy\marginnote{16.1} to admonish … 

with\marginnote{17.1} good friends … 

energetic\marginnote{18.1} … 

mindful\marginnote{19.1} … 

contented\marginnote{20.1} … 

of\marginnote{21.1} few desires … 

of\marginnote{22.1} right view to achieve growth, improvement, or maturity in this teaching and training. 

It\marginnote{23.1} is quite possible for a mendicant with these ten qualities to achieve growth, improvement, or maturity in this teaching and training.” 

%
\section*{{\suttatitleacronym AN 10.83}{\suttatitletranslation With Puṇṇiya }{\suttatitleroot Puṇṇiyasutta}}
\addcontentsline{toc}{section}{\tocacronym{AN 10.83} \toctranslation{With Puṇṇiya } \tocroot{Puṇṇiyasutta}}
\markboth{With Puṇṇiya }{Puṇṇiyasutta}
\extramarks{AN 10.83}{AN 10.83}

Then\marginnote{1.1} Venerable \textsanskrit{Puṇṇiya} went up to the Buddha, bowed, sat down to one side, and said to him: 

“Sir,\marginnote{1.2} what is the cause, what is the reason why sometimes the Realized One feels inspired to teach, and other times not?” 

“\textsanskrit{Puṇṇiya},\marginnote{2.1} when a mendicant has faith but doesn’t approach, the Realized One doesn’t feel inspired to teach. But when a mendicant has faith and approaches, the Realized One feels inspired to teach. 

When\marginnote{3.1} a mendicant has faith and approaches, but doesn’t pay homage … they pay homage, but don’t ask questions … they ask questions, but don’t actively listen to the teaching … they actively listen to the teaching, but don’t remember the teaching they’ve heard … they remember the teaching they’ve heard, but don’t reflect on the meaning of the teachings they’ve remembered … they reflect on the meaning of the teachings they’ve remembered, but, not having understood the meaning and the teaching, they don’t practice accordingly … they practice accordingly, but they’re not a good speaker and do not enunciate well. Their voice is not polished, clear, articulate, and doesn’t express the meaning … They’re a good speaker who enunciates well, but they don’t educate, encourage, fire up, and inspire their spiritual companions. The Realized One doesn’t feel inspired to teach. 

But\marginnote{4.1} when a mendicant has faith, approaches, pays homage, asks questions, actively listen to the teachings, remembers the teachings, reflects on the meaning, practices accordingly, has a good voice, and encourages their spiritual companions, the Realized One feels inspired to teach. When someone has these ten qualities, the Realized One feels totally inspired to teach.” 

%
\section*{{\suttatitleacronym AN 10.84}{\suttatitletranslation Declaration }{\suttatitleroot Byākaraṇasutta}}
\addcontentsline{toc}{section}{\tocacronym{AN 10.84} \toctranslation{Declaration } \tocroot{Byākaraṇasutta}}
\markboth{Declaration }{Byākaraṇasutta}
\extramarks{AN 10.84}{AN 10.84}

There\marginnote{1.1} Venerable \textsanskrit{Mahāmoggallāna} addressed the mendicants: “Reverends, mendicants!” 

“Reverend,”\marginnote{1.3} they replied. Venerable \textsanskrit{Mahāmoggallāna} said this: 

“Take\marginnote{2.1} a mendicant who declares enlightenment: ‘I understand: “Rebirth is ended, the spiritual journey has been completed, what had to be done has been done, there is nothing further for this place.”’ They’re pursued, pressed, and grilled by the Realized One, or by one of his disciples who has the absorptions, and is skilled in attainments, in the minds of others, and in the ways of another’s mind. Grilled in this way they get stuck or lose their way. They fall to ruin and disaster. 

The\marginnote{3.1} Realized One or one of his disciples comprehends their mind and investigates: ‘Why does this venerable declare enlightenment, saying: 

“I\marginnote{3.3} understand: ‘Rebirth is ended, the spiritual journey has been completed, what had to be done has been done, there is nothing further for this place.’?”’ 

They\marginnote{4.1} understand: 

‘This\marginnote{5.1} venerable gets irritable, and often lives with a heart full of anger. But being full of anger means decline in the teaching and training proclaimed by the Realized One. 

This\marginnote{6.1} venerable is acrimonious … 

prone\marginnote{7.1} to disdain … 

contemptuous\marginnote{8.1} … 

jealous\marginnote{9.1} … 

stingy\marginnote{10.1} … 

devious\marginnote{11.1} … 

deceitful\marginnote{12.1} … 

This\marginnote{13.1} venerable has corrupt wishes, and often lives with a heart full of desire. But being full of desire means decline in the teaching and training proclaimed by the Realized One. 

When\marginnote{14.1} there is still more to be done, this venerable stopped half-way after achieving some insignificant distinction. But stopping half-way means decline in the teaching and training proclaimed by the Realized One.’ 

It\marginnote{15.1} is quite impossible for a mendicant to achieve growth, improvement, or maturity in this teaching and training without giving up these ten qualities. It is quite possible for a mendicant to achieve growth, improvement, or maturity in this teaching and training after giving up these ten qualities.” 

%
\section*{{\suttatitleacronym AN 10.85}{\suttatitletranslation A Boaster }{\suttatitleroot Katthīsutta}}
\addcontentsline{toc}{section}{\tocacronym{AN 10.85} \toctranslation{A Boaster } \tocroot{Katthīsutta}}
\markboth{A Boaster }{Katthīsutta}
\extramarks{AN 10.85}{AN 10.85}

At\marginnote{1.1} one time Venerable \textsanskrit{Mahācunda} was staying in the land of the \textsanskrit{Cetīs} at \textsanskrit{Sahajāti}. There he addressed the mendicants: “Reverends, mendicants!” 

“Reverend,”\marginnote{1.4} they replied. Venerable \textsanskrit{Mahācunda} said this: 

“Take\marginnote{2.1} a mendicant who boasts and brags about their achievements: ‘I enter and emerge from the first absorption, the second absorption, the third absorption, and the fourth absorption. And I enter and emerge from the dimensions of infinite space, infinite consciousness, nothingness, and neither perception nor non-perception. And I enter and emerge from the cessation of perception and feeling.’ 

They’re\marginnote{3.1} pursued, pressed, and grilled by the Realized One, or by one of his disciples who has the absorptions, and is skilled in attainments, in the minds of others, and in the ways of another’s mind. Grilled in this way they get stuck or lose their way. They fall to ruin and disaster. 

The\marginnote{4.1} Realized One or one of his disciples comprehends their mind and investigates: ‘Why does this venerable boast and brag about their achievements, saying, “I enter and emerge from the first absorption … and the cessation of perception and feeling.”’ 

They\marginnote{5.1} understand, ‘For a long time this venerable’s deeds have been broken, tainted, spotty, and marred. Their deeds and behavior are inconsistent. This venerable is unethical, and unethical conduct means decline in the teaching and training proclaimed by the Realized One. 

This\marginnote{7.1} venerable is unfaithful, and lack of faith means decline … 

This\marginnote{8.1} venerable is unlearned and unpracticed, and lack of learning means decline … 

This\marginnote{9.1} venerable is hard to admonish, and being hard to admonish means decline … 

This\marginnote{10.1} venerable has bad friends, and bad friends mean decline … 

This\marginnote{11.1} venerable is lazy, and laziness means decline … 

This\marginnote{12.1} venerable is unmindful, and unmindfulness means decline … 

This\marginnote{13.1} venerable is deceptive, and deceitfulness means decline … 

This\marginnote{14.1} venerable is burdensome, and being burdensome means decline … 

This\marginnote{15.1} venerable is witless, and lack of wisdom means decline in the teaching and training proclaimed by the Realized One.’ 

Suppose\marginnote{16.1} one friend was to say to another: ‘My dear friend, when you need money for some payment, just ask me and I’ll give it.’ Then when some payment falls due, that friend says to their friend: ‘I need some money, my dear friend. Give me some.’ They’d say: ‘Well then, my dear friend, dig here.’ So they dig there, but don’t find anything. They’d say: ‘You lied to me, my dear friend, you spoke hollow words when you told me to dig here.’ They’d say: ‘My dear friend, I didn’t lie or speak hollow words. Well then, dig here.’ So they dig there as well, but don’t find anything. They’d say: ‘You lied to me, my dear friend, you spoke hollow words when you said dig here.’ They’d say: ‘My dear friend, I didn’t lie or speak hollow words. Well then, dig here.’ So they dig there as well, but don’t find anything. They’d say: ‘You lied to me, my dear friend, you spoke hollow words when you said dig here.’ They’d say: ‘My dear friend, I didn’t lie or speak hollow words. But I had gone mad, I was out of my mind.’ 

In\marginnote{17.1} the same way, take a mendicant who boasts and brags about their achievements: ‘I enter and emerge from the first absorption … and the cessation of perception and feeling.’ 

They’re\marginnote{18.1} pursued, pressed, and grilled by the Realized One, or by one of his disciples … Grilled in this way they get stuck or lose their way. They fall to ruin and disaster. 

The\marginnote{19.1} Realized One or one of his disciples comprehends their mind and investigates: ‘Why does this venerable boast and brag about their achievements, saying, “I enter and emerge from the first absorption … and the cessation of perception and feeling.”’ 

They\marginnote{20.1} understand, ‘For a long time this venerable’s deeds have been broken, tainted, spotty, and marred. Their deeds and behavior are inconsistent. This venerable is unethical, and unethical conduct means decline in the teaching and training proclaimed by the Realized One. 

This\marginnote{22.1} venerable is unfaithful … 

unlearned\marginnote{23.1} and unpracticed … 

hard\marginnote{24.1} to admonish … 

with\marginnote{25.1} bad friends … 

lazy\marginnote{26.1} … 

unmindful\marginnote{27.1} … 

deceptive\marginnote{28.1} … 

burdensome\marginnote{29.1} … 

This\marginnote{30.1} venerable is witless, and lack of wisdom means decline in the teaching and training proclaimed by the Realized One.’ 

It\marginnote{31.1} is quite impossible for a mendicant to achieve growth, improvement, or maturity in this teaching and training without giving up these ten qualities. It is quite possible for a mendicant to achieve growth, improvement, or maturity in this teaching and training after giving up these ten qualities.” 

%
\section*{{\suttatitleacronym AN 10.86}{\suttatitletranslation Overestimation }{\suttatitleroot Adhimānasutta}}
\addcontentsline{toc}{section}{\tocacronym{AN 10.86} \toctranslation{Overestimation } \tocroot{Adhimānasutta}}
\markboth{Overestimation }{Adhimānasutta}
\extramarks{AN 10.86}{AN 10.86}

At\marginnote{1.1} one time Venerable \textsanskrit{Mahākassapa} was staying near \textsanskrit{Rājagaha}, in the Bamboo Grove, the squirrels’ feeding ground. There he addressed the mendicants: “Reverends, mendicants!” 

“Reverend,”\marginnote{1.4} they replied. Venerable \textsanskrit{Mahākassapa} said this: 

“Take\marginnote{2.1} a mendicant who declares enlightenment: ‘I understand: “Rebirth is ended, the spiritual journey has been completed, what had to be done has been done, there is nothing further for this place.”’ They’re pursued, pressed, and grilled by the Realized One, or by one of his disciples who has the absorptions, and is skilled in attainments, in the minds of others, and in the ways of another’s mind. Grilled in this way they get stuck or lose their way. They fall to ruin and disaster. 

The\marginnote{3.1} Realized One or one of his disciples comprehends their mind and investigates: ‘Why does this venerable declare enlightenment, saying, “I understand: ‘Rebirth is ended, the spiritual journey has been completed, what had to be done has been done, there is nothing further for this place.’?”’ 

They\marginnote{4.1} understand, ‘This venerable overestimates themselves and takes that to be the truth. They perceive that they’ve attained what they haven’t attained, done what they haven’t done, and achieved what they haven’t achieved. And they declare enlightenment out of overestimation: 

“I\marginnote{5.3} understand: ‘Rebirth is ended, the spiritual journey has been completed, what had to be done has been done, there is nothing further for this place.’”’ 

The\marginnote{6.1} Realized One or one of his disciples comprehends their mind and investigates: ‘Why does this venerable overestimate themselves and take that to be the truth? Why do they perceive that they’ve attained what they haven’t attained, done what they haven’t done, and achieved what they haven’t achieved? And why do they declare enlightenment out of overestimation: 

“I\marginnote{6.4} understand: ‘Rebirth is ended, the spiritual journey has been completed, what had to be done has been done, there is nothing further for this place.’”’ 

They\marginnote{7.1} understand, ‘This venerable is very learned, remembering and keeping what they’ve learned. These teachings are good in the beginning, good in the middle, and good in the end, meaningful and well-phrased, describing a spiritual practice that’s entirely full and pure. They are very learned in such teachings, remembering them, rehearsing them, mentally scrutinizing them, and comprehending them theoretically. Therefore this venerable overestimates themselves and takes that to be the truth. …’ 

They\marginnote{9.1} understand, ‘This venerable is covetous, and often lives with a heart full covetousness. Being full of covetousness means decline in the teaching and training proclaimed by the Realized One. 

This\marginnote{11.1} venerable has ill will … 

dullness\marginnote{12.1} and drowsiness … 

restlessness\marginnote{13.1} … 

doubt\marginnote{14.1} … 

This\marginnote{15.1} venerable relishes work. They love it and like to relish it … 

This\marginnote{16.1} venerable relishes talk … 

sleep\marginnote{17.1} … 

company\marginnote{18.1} … 

When\marginnote{19.1} there is still more to be done, this venerable stopped half-way after achieving some insignificant distinction. Stopping half-way means decline in the teaching and training proclaimed by the Realized One.’ 

It\marginnote{20.1} is quite impossible for a mendicant to achieve growth, improvement, or maturity in this teaching and training without giving up these ten qualities. It is quite possible for a mendicant to achieve growth, improvement, or maturity in this teaching and training after giving up these ten qualities.” 

%
\section*{{\suttatitleacronym AN 10.87}{\suttatitletranslation Disciplinary Issues }{\suttatitleroot Nappiyasutta}}
\addcontentsline{toc}{section}{\tocacronym{AN 10.87} \toctranslation{Disciplinary Issues } \tocroot{Nappiyasutta}}
\markboth{Disciplinary Issues }{Nappiyasutta}
\extramarks{AN 10.87}{AN 10.87}

There\marginnote{1.1} the Buddha addressed the mendicants concerning the mendicant Kalandaka: 

“Mendicants!”\marginnote{1.2} 

“Venerable\marginnote{1.3} sir,” they replied. The Buddha said this: 

“Firstly,\marginnote{2.1} a mendicant raises disciplinary issues and doesn’t praise the settlement of disciplinary issues. This quality doesn’t conduce to fondness, respect, esteem, harmony, and unity. 

Furthermore,\marginnote{3.1} a mendicant doesn’t want to train, and doesn’t praise taking up the training. … 

Furthermore,\marginnote{4.1} a mendicant has corrupt wishes, and doesn’t praise getting rid of wishes. … 

Furthermore,\marginnote{5.1} a mendicant is irritable, and doesn’t praise getting rid of anger. … 

Furthermore,\marginnote{6.1} a mendicant denigrates others, and doesn’t praise getting rid of denigration. … 

Furthermore,\marginnote{7.1} a mendicant is devious, and doesn’t praise getting rid of deviousness. … 

Furthermore,\marginnote{8.1} a mendicant is deceitful, and doesn’t praise getting rid of deceitfulness. … 

Furthermore,\marginnote{9.1} a mendicant doesn’t pay attention to the teachings, and doesn’t praise attending to the teachings. … 

Furthermore,\marginnote{10.1} a mendicant is not in retreat, and doesn’t praise retreat. … 

Furthermore,\marginnote{11.1} a mendicant is inhospitable to their spiritual companions, and doesn’t praise hospitality. This quality doesn’t conduce to fondness, respect, esteem, harmony, and unity. 

Even\marginnote{12.1} though a mendicant such as this might wish: ‘If only my spiritual companions would honor, respect, esteem, and venerate me!’ Still they don’t honor, respect, esteem, and venerate them. Why is that? Because their sensible spiritual companions see that they haven’t given up those bad unskillful qualities. 

Suppose\marginnote{13.1} a wild colt was to wish: ‘If only the humans would put me in a thoroughbred’s place, feed me a thoroughbred’s food, and give me a thoroughbred’s grooming.’ Still the humans wouldn’t put them in a thoroughbred’s place, feed them a thoroughbred’s food, or give them a thoroughbred’s grooming. Why is that? Because sensible humans see that they haven’t given up their tricks, bluffs, ruses, and feints. In the same way, even though a mendicant such as this might wish: ‘If only my spiritual companions would honor, respect, esteem, and venerate me!’ Still they don’t honor, respect, esteem, and venerate them. Why is that? Because their sensible spiritual companions see that they haven’t given up those bad unskillful qualities. 

Next,\marginnote{14.1} a mendicant doesn’t raise disciplinary issues and praises the settlement of disciplinary issues. This quality conduces to fondness, respect, esteem, harmony, and unity. 

Furthermore,\marginnote{15.1} a mendicant wants to train, and praises taking up the training. … 

Furthermore,\marginnote{16.1} a mendicant has few desires, and praises getting rid of desires. … 

Furthermore,\marginnote{17.1} a mendicant is not irritable, and praises getting rid of anger. … 

Furthermore,\marginnote{18.1} a mendicant doesn’t denigrate others, and praises getting rid of denigration. … 

Furthermore,\marginnote{19.1} a mendicant isn’t devious, and praises getting rid of deviousness. … 

Furthermore,\marginnote{20.1} a mendicant isn’t deceitful, and praises getting rid of deceitfulness. … 

Furthermore,\marginnote{21.1} a mendicant pays attention to the teachings, and praises attending to the teachings. … 

Furthermore,\marginnote{22.1} a mendicant is in retreat, and praises retreat. … 

Furthermore,\marginnote{23.1} a mendicant is hospitable to their spiritual companions, and praises hospitality. This quality conduces to fondness, respect, esteem, harmony, and unity. 

Even\marginnote{24.1} though a mendicant such as this might never wish: ‘If only my spiritual companions would honor, respect, esteem, and venerate me!’ Still they honor, respect, esteem, and venerate them. Why is that? Because their sensible spiritual companions see that they’ve given up those bad unskillful qualities. 

Suppose\marginnote{25.1} a fine thoroughbred never wished: ‘If only the humans would put me in a thoroughbred’s place, feed me a thoroughbred’s food, and give me a thoroughbred’s grooming.’ Still the humans would put them in a thoroughbred’s place, feed them a thoroughbred’s food, and give them a thoroughbred’s grooming. Why is that? Because sensible humans see that they’ve given up their tricks, bluffs, ruses, and feints. 

In\marginnote{26.1} the same way, even though a mendicant such as this might never wish: ‘If only my spiritual companions would honor, respect, esteem, and venerate me!’ Still they honor, respect, esteem, and venerate them. Why is that? Because their sensible spiritual companions see that they’ve given up those bad unskillful qualities.” 

%
\section*{{\suttatitleacronym AN 10.88}{\suttatitletranslation An Abuser }{\suttatitleroot Akkosakasutta}}
\addcontentsline{toc}{section}{\tocacronym{AN 10.88} \toctranslation{An Abuser } \tocroot{Akkosakasutta}}
\markboth{An Abuser }{Akkosakasutta}
\extramarks{AN 10.88}{AN 10.88}

“Mendicants,\marginnote{1.1} any mendicant who abuses and insults their spiritual companions, denouncing the noble ones, will, without a doubt, fall into one or other of these ten disasters. What ten? They don’t achieve the unachieved. What they have achieved falls away. They don’t refine their good qualities. They overestimate their good qualities, or lead the spiritual life dissatisfied, or commit a corrupt offense, or contract a severe illness, or go mad and lose their mind. They feel lost when they die. And when their body breaks up, after death, they are reborn in a place of loss, a bad place, the underworld, hell. Any mendicant who abuses and insults their spiritual companions, denouncing the noble ones, will, without a doubt, fall into one or other of these ten disasters.” 

%
\section*{{\suttatitleacronym AN 10.89}{\suttatitletranslation With Kokālika }{\suttatitleroot Kokālikasutta}}
\addcontentsline{toc}{section}{\tocacronym{AN 10.89} \toctranslation{With Kokālika } \tocroot{Kokālikasutta}}
\markboth{With Kokālika }{Kokālikasutta}
\extramarks{AN 10.89}{AN 10.89}

Then\marginnote{1.1} the mendicant \textsanskrit{Kokālika} went up to the Buddha, bowed, sat down to one side, and said to him, “Sir, \textsanskrit{Sāriputta} and \textsanskrit{Moggallāna} have corrupt wishes. They’ve fallen under the sway of corrupt wishes.” 

“Don’t\marginnote{1.3} say that, \textsanskrit{Kokālika}! Don’t say that, \textsanskrit{Kokālika}! Have confidence in \textsanskrit{Sāriputta} and \textsanskrit{Moggallāna}, they’re good monks.” 

For\marginnote{2.1} a second time \textsanskrit{Kokālika} said to the Buddha, “Despite my faith and trust in the Buddha, \textsanskrit{Sāriputta} and \textsanskrit{Moggallāna} have corrupt wishes. They’ve fallen under the sway of corrupt wishes.” 

“Don’t\marginnote{2.3} say that, \textsanskrit{Kokālika}! Don’t say that, \textsanskrit{Kokālika}! Have confidence in \textsanskrit{Sāriputta} and \textsanskrit{Moggallāna}, they’re good monks.” 

For\marginnote{3.1} a third time \textsanskrit{Kokālika} said to the Buddha, “Despite my faith and trust in the Buddha, \textsanskrit{Sāriputta} and \textsanskrit{Moggallāna} have corrupt wishes. They’ve fallen under the sway of corrupt wishes.” 

“Don’t\marginnote{3.3} say that, \textsanskrit{Kokālika}! Don’t say that, \textsanskrit{Kokālika}! Have confidence in \textsanskrit{Sāriputta} and \textsanskrit{Moggallāna}, they’re good monks.” 

Then\marginnote{4.1} \textsanskrit{Kokālika} got up from his seat, bowed, and respectfully circled the Buddha, keeping him on his right, before leaving. Not long after he left his body erupted with boils the size of mustard seeds. The boils grew to the size of mung beans, then chickpeas, then jujube seeds, then jujubes, then myrobalans, then unripe wood apples, then ripe wood apples. Finally they burst open, and pus and blood oozed out. He just laid down on banana leaves like a poisoned fish. 

Then\marginnote{5.1} Tudu the independent divinity went to \textsanskrit{Kokālika}, and standing in the air he said to him, “\textsanskrit{Kokālika}, have confidence in \textsanskrit{Sāriputta} and \textsanskrit{Moggallāna}, they’re good monks.” 

“Who\marginnote{5.4} are you, reverend?” 

“I\marginnote{5.5} am Tudu the independent divinity.” 

“Didn’t\marginnote{5.6} the Buddha declare you a non-returner? So what exactly are you doing back here? See how far you have strayed!” 

Then\marginnote{6.1} Tudu addressed \textsanskrit{Kokālika} in verse: 

\begin{verse}%
“A\marginnote{7.1} person is born \\
with an axe in their mouth. \\
A fool cuts themselves with it \\
when they say bad words. 

When\marginnote{8.1} you praise someone worthy of criticism, \\
or criticize someone worthy of praise, \\
you choose bad luck with your own mouth: \\
you’ll never find happiness that way. 

Bad\marginnote{9.1} luck at dice is a trivial thing, \\
if all you lose is your money \\
and all you own, even yourself. \\
What’s really terrible luck \\
is to hate the holy ones. 

For\marginnote{10.1} more than two quinquadecillion years, \\
and another five quattuordecillion years, \\
a slanderer of noble ones goes to hell, \\
having aimed bad words and thoughts at them.” 

%
\end{verse}

Then\marginnote{11.1} the mendicant \textsanskrit{Kokālika} died of that illness. He was reborn in the Pink Lotus hell because of his resentment for \textsanskrit{Sāriputta} and \textsanskrit{Moggallāna}. 

Then,\marginnote{12.1} late at night, the beautiful divinity Sahampati, lighting up the entire Jeta’s Grove, went up to the Buddha, bowed, stood to one side, and said to him, “Sir, the mendicant \textsanskrit{Kokālika} has passed away. He was reborn in the pink lotus hell because of his resentment for \textsanskrit{Sāriputta} and \textsanskrit{Moggallāna}.” 

That’s\marginnote{12.4} what the divinity Sahampati said. Then he bowed and respectfully circled the Buddha, keeping him on his right side, before vanishing right there. 

Then,\marginnote{13.1} when the night had passed, the Buddha told the mendicants all that had happened. 

When\marginnote{14.1} he said this, one of the mendicants asked the Buddha, “Sir, how long is the lifespan in the Pink Lotus hell?” 

“It’s\marginnote{14.3} long, mendicant. It’s not easy to calculate how many years, how many hundreds or thousands or hundreds of thousands of years it lasts.” 

“But\marginnote{15.1} sir, is it possible to give a simile?” 

“It’s\marginnote{15.2} possible,” said the Buddha. 

“Suppose\marginnote{15.3} there was a Kosalan cartload of twenty bushels of sesame seed. And at the end of every hundred years someone would remove a single seed from it. By this means the Kosalan cartload of twenty bushels of sesame seed would run out faster than a single lifetime in the Abbuda hell. Now, twenty lifetimes in the Abbuda hell equal one lifetime in the Nirabbuda hell. Twenty lifetimes in the Nirabbuda hell equal one lifetime in the Ababa hell. Twenty lifetimes in the Ababa hell equal one lifetime in the \textsanskrit{Aṭaṭa} hell. Twenty lifetimes in the \textsanskrit{Aṭaṭa} hell equal one lifetime in the Ahaha hell. Twenty lifetimes in the Ahaha hell equal one lifetime in the Yellow Lotus hell. Twenty lifetimes in the Yellow Lotus hell equal one lifetime in the Sweet-Smelling hell. Twenty lifetimes in the Sweet-Smelling hell equal one lifetime in the Blue Water Lily hell. Twenty lifetimes in the Blue Water Lily hell equal one lifetime in the White Lotus hell. Twenty lifetimes in the White Lotus hell equal one lifetime in the Pink Lotus hell. The mendicant \textsanskrit{Kokālika} has been reborn in the Pink Lotus hell because of his resentment for \textsanskrit{Sāriputta} and \textsanskrit{Moggallāna}.” 

That\marginnote{15.15} is what the Buddha said. Then the Holy One, the Teacher, went on to say: 

\begin{verse}%
“A\marginnote{16.1} person is born \\
with an axe in their mouth. \\
A fool cuts themselves with it \\
when they say bad words. 

When\marginnote{17.1} you praise someone worthy of criticism, \\
or criticize someone worthy of praise, \\
you choose bad luck with your own mouth: \\
you’ll never find happiness that way. 

Bad\marginnote{18.1} luck at dice is a trivial thing, \\
if all you lose is your money \\
and all you own, even yourself. \\
What’s really terrible luck \\
is to hate the holy ones. 

For\marginnote{19.1} more than two quinquadecillion years, \\
and another five quattuordecillion years, \\
a slanderer of noble ones goes to hell, \\
having aimed bad words and thoughts at them.” 

%
\end{verse}

%
\section*{{\suttatitleacronym AN 10.90}{\suttatitletranslation The Powers of One Who has Ended Defilements }{\suttatitleroot Khīṇāsavabalasutta}}
\addcontentsline{toc}{section}{\tocacronym{AN 10.90} \toctranslation{The Powers of One Who has Ended Defilements } \tocroot{Khīṇāsavabalasutta}}
\markboth{The Powers of One Who has Ended Defilements }{Khīṇāsavabalasutta}
\extramarks{AN 10.90}{AN 10.90}

Then\marginnote{1.1} Venerable \textsanskrit{Sāriputta} went up to the Buddha, bowed, and sat down to one side. The Buddha said to him: 

“\textsanskrit{Sāriputta},\marginnote{1.2} how many powers does a mendicant who has ended the defilements have that qualify them to claim: ‘My defilements have ended.’” 

“Sir,\marginnote{2.1} a mendicant who has ended the defilements has ten powers that qualify them to claim: ‘My defilements have ended.’ What ten? Firstly, a mendicant with defilements ended has clearly seen with right wisdom all conditions as truly impermanent. This is a power that a mendicant who has ended the defilements relies on to claim: ‘My defilements have ended.’ 

Furthermore,\marginnote{3.1} a mendicant with defilements ended has clearly seen with right wisdom that sensual pleasures are truly like a pit of glowing coals. This is a power that a mendicant who has ended the defilements relies on to claim: ‘My defilements have ended.’ 

Furthermore,\marginnote{4.1} the mind of a mendicant with defilements ended slants, slopes, and inclines to seclusion. They’re withdrawn, loving renunciation, and have totally eliminated defiling influences. This is a power that a mendicant who has ended the defilements relies on to claim: ‘My defilements have ended.’ 

Furthermore,\marginnote{5.1} a mendicant with defilements ended has well developed the four kinds of mindfulness meditation. This is a power that a mendicant who has ended the defilements relies on to claim: ‘My defilements have ended.’ 

Furthermore,\marginnote{6.1} a mendicant with defilements ended has well developed the four right efforts. … the four bases of psychic power … the five faculties … the five powers … the seven awakening factors … the noble eightfold path. This is a power that a mendicant who has ended the defilements relies on to claim: ‘My defilements have ended.’ 

A\marginnote{7.1} mendicant who has ended the defilements has these ten powers that qualify them to claim: ‘My defilements have ended.’” 

%
\addtocontents{toc}{\let\protect\contentsline\protect\nopagecontentsline}
\chapter*{The Chapter with Upāli }
\addcontentsline{toc}{chapter}{\tocchapterline{The Chapter with Upāli }}
\addtocontents{toc}{\let\protect\contentsline\protect\oldcontentsline}

%
\section*{{\suttatitleacronym AN 10.91}{\suttatitletranslation Pleasure Seekers }{\suttatitleroot Kāmabhogīsutta}}
\addcontentsline{toc}{section}{\tocacronym{AN 10.91} \toctranslation{Pleasure Seekers } \tocroot{Kāmabhogīsutta}}
\markboth{Pleasure Seekers }{Kāmabhogīsutta}
\extramarks{AN 10.91}{AN 10.91}

At\marginnote{1.1} one time the Buddha was staying near \textsanskrit{Sāvatthī} in Jeta’s Grove, \textsanskrit{Anāthapiṇḍika}’s monastery. Then the householder \textsanskrit{Anāthapiṇḍika} went up to the Buddha, bowed, and sat down to one side. Seated to one side, the Buddha said to the householder \textsanskrit{Anāthapiṇḍika}: 

“These\marginnote{2.1} ten pleasure seekers are found in the world. What ten? First, a pleasure seeker seeks wealth using illegitimate, coercive means. They don’t make themselves happy and pleased, nor share it and make merit. 

Next,\marginnote{3.1} a pleasure seeker seeks wealth using illegitimate, coercive means. They make themselves happy and pleased, but don’t share it and make merit. 

Next,\marginnote{4.1} a pleasure seeker seeks wealth using illegitimate, coercive means. They make themselves happy and pleased, and they share it and make merit. 

Next,\marginnote{5.1} a pleasure seeker seeks wealth using means both legitimate and illegitimate, and coercive and non-coercive. They don’t make themselves happy and pleased, nor share it and make merit. 

Next,\marginnote{6.1} a pleasure seeker seeks wealth using means both legitimate and illegitimate, and coercive and non-coercive. They make themselves happy and pleased, but don't share it and make merit. 

Next,\marginnote{7.1} a pleasure seeker seeks wealth using means both legitimate and illegitimate, and coercive and non-coercive. They make themselves happy and pleased, and they share it and make merit. 

Next,\marginnote{8.1} a pleasure seeker seeks wealth using legitimate, non-coercive means. They don’t make themselves happy and pleased, nor share it and make merit. 

Next,\marginnote{9.1} a pleasure seeker seeks wealth using legitimate, non-coercive means. They make themselves happy and pleased, but don’t share it and make merit. 

Next,\marginnote{10.1} a pleasure seeker seeks wealth using legitimate, non-coercive means. They make themselves happy and pleased, and they share it and make merit. But they enjoy that wealth tied, infatuated, attached, blind to the drawbacks, and not understanding the escape. 

Next,\marginnote{11.1} a pleasure seeker seeks wealth using legitimate, non-coercive means. They make themselves happy and pleased, and they share it and make merit. And they enjoy that wealth untied, uninfatuated, unattached, seeing the drawbacks, and understanding the escape. 

Now,\marginnote{12.1} consider the pleasure seeker who seeks wealth using illegitimate, coercive means, and who doesn’t make themselves happy and pleased, nor share it and make merit. They may be criticized on three grounds. They seek for wealth using illegitimate, coercive means. This is the first ground for criticism. They don’t make themselves happy and pleased. This is the second ground for criticism. They don’t share it and make merit. This is the third ground for criticism. This pleasure seeker may be criticized on these three grounds. 

Now,\marginnote{13.1} consider the pleasure seeker who seeks wealth using illegitimate, coercive means, and who makes themselves happy and pleased, but doesn’t share it and make merit. They may be criticized on two grounds, and praised on one. They seek for wealth using illegitimate, coercive means. This is the first ground for criticism. They make themselves happy and pleased. This is the one ground for praise. They don’t share it and make merit. This is the second ground for criticism. This pleasure seeker may be criticized on these two grounds, and praised on this one. 

Now,\marginnote{14.1} consider the pleasure seeker who seeks wealth using illegitimate, coercive means, and who makes themselves happy and pleased, and shares it and makes merit. They may be criticized on one ground, and praised on two. They seek for wealth using illegitimate, coercive means. This is the one ground for criticism. They make themselves happy and pleased. This is the first ground for praise. They share it and make merit. This is the second ground for praise. This pleasure seeker may be criticized on this one ground, and praised on these two. 

Now,\marginnote{15.1} consider the pleasure seeker who seeks wealth using means both legitimate and illegitimate, and coercive and non-coercive, and who doesn’t make themselves happy and pleased, nor share it and make merit. They may be praised on one ground, and criticized on three. They seek for wealth using legitimate, non-coercive means. This is the one ground for praise. They seek for wealth using illegitimate, coercive means. This is the first ground for criticism. They don’t make themselves happy and pleased. This is the second ground for criticism. They don’t share it and make merit. This is the third ground for criticism. This pleasure seeker may be praised on this one ground, and criticized on these three. 

Now,\marginnote{16.1} consider the pleasure seeker who seeks wealth using means both legitimate and illegitimate, and coercive and non-coercive, and who makes themselves happy and pleased, but doesn’t share it and make merit. They may be praised on two grounds, and criticized on two. They seek for wealth using legitimate, non-coercive means. This is the first ground for praise. They seek for wealth using illegitimate, coercive means. This is the first ground for criticism. They make themselves happy and pleased. This is the second ground for praise. They don’t share it and make merit. This is the second ground for criticism. This pleasure seeker may be praised on these two grounds, and criticized on these two. 

Now,\marginnote{17.1} consider the pleasure seeker who seeks wealth using means both legitimate and illegitimate, and coercive and non-coercive, and who makes themselves happy and pleased, and shares it and make merit. They may be praised on three grounds, and criticized on one. They seek for wealth using legitimate, non-coercive means. This is the first ground for praise. They seek for wealth using illegitimate, coercive means. This is the one ground for criticism. They make themselves happy and pleased. This is the second ground for praise. They share it and make merit. This is the third ground for praise. This pleasure seeker may be praised on these three grounds, and criticized on this one. 

Now,\marginnote{18.1} consider the pleasure seeker who seeks wealth using legitimate, non-coercive means, and who doesn’t make themselves happy and pleased, nor share it and make merit. They may be praised on one ground and criticized on two. They seek for wealth using legitimate, non-coercive means. This is the one ground for praise. They don’t make themselves happy and pleased. This is the first ground for criticism. They don’t share it and make merit. This is the second ground for criticism. This pleasure seeker may be praised on this one ground, and criticized on these two. 

Now,\marginnote{19.1} consider the pleasure seeker who seeks wealth using legitimate, non-coercive means, and who makes themselves happy and pleased, but doesn’t share it and make merit. They may be praised on two grounds and criticized on one. They seek for wealth using legitimate, non-coercive means. This is the first ground for praise. They make themselves happy and pleased. This is the second ground for praise. They don’t share it and make merit. This is the one ground for criticism. This pleasure seeker may be praised on these two grounds, and criticized on this one. 

Now,\marginnote{20.1} consider the pleasure seeker who seeks wealth using legitimate, non-coercive means, and who makes themselves happy and pleased, and shares it and makes merit. But they enjoy that wealth tied, infatuated, attached, blind to the drawbacks, and not understanding the escape. They may be praised on three grounds and criticized on one. They seek for wealth using legitimate, non-coercive means. This is the first ground for praise. They make themselves happy and pleased. This is the second ground for praise. They share it and make merit. This is the third ground for praise. They enjoy that wealth tied, infatuated, attached, blind to the drawbacks, and not understanding the escape. This is the one ground for criticism. This pleasure seeker may be praised on these three grounds, and criticized on this one. 

Now,\marginnote{21.1} consider the pleasure seeker who seeks wealth using legitimate, non-coercive means, and who makes themselves happy and pleased, and shares it and makes merit. And they enjoy that wealth untied, uninfatuated, unattached, seeing the drawbacks, and understanding the escape. They may be praised on four grounds. They seek for wealth using legitimate, non-coercive means. This is the first ground for praise. They make themselves happy and pleased. This is the second ground for praise. They share it and make merit. This is the third ground for praise. They enjoy that wealth untied, uninfatuated, unattached, seeing the drawbacks, and understanding the escape. This is the fourth ground for praise. This pleasure seeker may be praised on these four grounds. 

These\marginnote{22.1} are the ten pleasure seekers found in the world. The pleasure seeker who seeks wealth using legitimate, non-coercive means, who makes themselves happy and pleased, and shares it and makes merit, and who uses that wealth untied, uninfatuated, unattached, seeing the drawbacks, and understanding the escape is the foremost, best, chief, highest, and finest of the ten. From a cow comes milk, from milk comes curds, from curds come butter, from butter comes ghee, and from ghee comes cream of ghee. And the cream of ghee is said to be the best of these. 

In\marginnote{23.1} the same way, the pleasure seeker who seeks wealth using legitimate, non-coercive means, who makes themselves happy and pleased, and shares it and makes merit, and who uses that wealth untied, uninfatuated, unattached, seeing the drawbacks, and understanding the escape is the foremost, best, chief, highest, and finest of the ten.” 

%
\section*{{\suttatitleacronym AN 10.92}{\suttatitletranslation Dangers }{\suttatitleroot Bhayasutta}}
\addcontentsline{toc}{section}{\tocacronym{AN 10.92} \toctranslation{Dangers } \tocroot{Bhayasutta}}
\markboth{Dangers }{Bhayasutta}
\extramarks{AN 10.92}{AN 10.92}

Then\marginnote{1.1} the householder \textsanskrit{Anāthapiṇḍika} went up to the Buddha, bowed, and sat down to one side. The Buddha said to him: 

“Householder,\marginnote{2.1} when a noble disciple has quelled five dangers and threats, has the four factors of stream-entry, and has clearly seen and comprehended the noble system with wisdom, they may, if they wish, declare of themselves: ‘I’ve finished with rebirth in hell, the animal realm, and the ghost realm. I’ve finished with all places of loss, bad places, the underworld. I am a stream-enterer! I’m not liable to be reborn in the underworld, and am bound for awakening.’ 

What\marginnote{3.1} are the five dangers and threats they have quelled? Anyone who kills living creatures creates dangers and threats both in this life and in lives to come, and experiences mental pain and sadness. Anyone who refrains from killing living creatures creates no dangers and threats either in this life or in lives to come, and doesn’t experience mental pain and sadness. So that danger and threat is quelled for anyone who refrains from killing living creatures. 

Anyone\marginnote{4.1} who steals … Anyone who commits sexual misconduct … Anyone who lies … Anyone who consumes beer, wine, and liquor intoxicants creates dangers and threats both in this life and in lives to come, and experiences mental pain and sadness. Anyone who refrains from consuming beer, wine, and liquor intoxicants creates no dangers and threats either in this life or in lives to come, and doesn’t experience mental pain and sadness. So that danger and threat is quelled for anyone who refrains from consuming beer, wine, and liquor intoxicants. These are the five dangers and threats they have quelled. 

What\marginnote{5.1} are the four factors of stream-entry that they have? It’s when a noble disciple has experiential confidence in the Buddha: ‘That Blessed One is perfected, a fully awakened Buddha, accomplished in knowledge and conduct, holy, knower of the world, supreme guide for those who wish to train, teacher of gods and humans, awakened, blessed.’ They have experiential confidence in the teaching: ‘The teaching is well explained by the Buddha—apparent in the present life, immediately effective, inviting inspection, relevant, so that sensible people can know it for themselves.’ They have experiential confidence in the \textsanskrit{Saṅgha}: ‘The \textsanskrit{Saṅgha} of the Buddha’s disciples is practicing the way that’s good, sincere, systematic, and proper. It consists of the four pairs, the eight individuals. This is the \textsanskrit{Saṅgha} of the Buddha’s disciples that is worthy of offerings dedicated to the gods, worthy of hospitality, worthy of a religious donation, worthy of greeting with joined palms, and is the supreme field of merit for the world.’ And a noble disciple’s ethical conduct is loved by the noble ones, unbroken, impeccable, spotless, and unmarred, liberating, praised by sensible people, not mistaken, and leading to immersion. These are the four factors of stream-entry that they have. 

And\marginnote{6.1} what is the noble system that they have clearly seen and comprehended with wisdom? It’s when a noble disciple reflects: ‘When this exists, that is; due to the arising of this, that arises. When this doesn’t exist, that is not; due to the cessation of this, that ceases. That is: Ignorance is a condition for choices. Choices are a condition for consciousness. Consciousness is a condition for name and form. Name and form are conditions for the six sense fields. The six sense fields are conditions for contact. Contact is a condition for feeling. Feeling is a condition for craving. Craving is a condition for grasping. Grasping is a condition for continued existence. Continued existence is a condition for rebirth. Rebirth is a condition for old age and death, sorrow, lamentation, pain, sadness, and distress to come to be. That is how this entire mass of suffering originates. When ignorance fades away and ceases with nothing left over, choices cease. When choices cease, consciousness ceases. When consciousness ceases, name and form cease. When name and form cease, the six sense fields cease. When the six sense fields cease, contact ceases. When contact ceases, feeling ceases. When feeling ceases, craving ceases. When craving ceases, grasping ceases. When grasping ceases, continued existence ceases. When continued existence ceases, rebirth ceases. When rebirth ceases, old age and death, sorrow, lamentation, pain, sadness, and distress cease. That is how this entire mass of suffering ceases.’ This is the noble system that they have clearly seen and comprehended with wisdom. 

When\marginnote{7.1} a noble disciple has quelled five dangers and threats, has the four factors of stream-entry, and has clearly seen and comprehended the noble cycle with wisdom, they may, if they wish, declare of themselves: ‘I’ve finished with rebirth in hell, the animal realm, and the ghost realm. I’ve finished with all places of loss, bad places, the underworld. I am a stream-enterer! I’m not liable to be reborn in the underworld, and am bound for awakening.’” 

%
\section*{{\suttatitleacronym AN 10.93}{\suttatitletranslation What Is Your View? }{\suttatitleroot Kiṁdiṭṭhikasutta}}
\addcontentsline{toc}{section}{\tocacronym{AN 10.93} \toctranslation{What Is Your View? } \tocroot{Kiṁdiṭṭhikasutta}}
\markboth{What Is Your View? }{Kiṁdiṭṭhikasutta}
\extramarks{AN 10.93}{AN 10.93}

At\marginnote{1.1} one time the Buddha was staying near \textsanskrit{Sāvatthī} in Jeta’s Grove, \textsanskrit{Anāthapiṇḍika}’s monastery. 

Then\marginnote{1.2} the householder \textsanskrit{Anāthapiṇḍika} left \textsanskrit{Sāvatthī} in the middle of the day to see the Buddha. Then it occurred to him, “It’s the wrong time to see the Buddha, as he’s in retreat. And it’s the wrong time to see the esteemed mendicants, as they’re in retreat. Why don’t I visit the monastery of the wanderers of other religions?” 

Then\marginnote{2.1} he went to the monastery of the wanderers of other religions. Now at that time, the wanderers of other religions had come together, making an uproar, a dreadful racket as they sat and talked about all kinds of low topics. 

They\marginnote{2.3} saw \textsanskrit{Anāthapiṇḍika} coming off in the distance, and stopped each other, saying, “Be quiet, good sirs, don’t make a sound. The householder \textsanskrit{Anāthapiṇḍika}, a disciple of the ascetic Gotama, is coming into our monastery. He is included among the white-clothed lay disciples of the ascetic Gotama, who is residing in \textsanskrit{Sāvatthī}. Such venerables like the quiet, are educated to be quiet, and praise the quiet. Hopefully if he sees that our assembly is quiet he’ll see fit to approach.” 

Then\marginnote{3.1} those wanderers of other religions fell silent. Then \textsanskrit{Anāthapiṇḍika} went up to them, and exchanged greetings with those wanderers. When the greetings and polite conversation were over, he sat down to one side. The wanderers said to him, “Tell us, householder, what is the view of the ascetic Gotama?” 

“Sirs,\marginnote{3.5} I don’t know all his views.” 

“Well\marginnote{4.1} then, since it seems you don’t know all the views of the ascetic Gotama, tell us, what are the views of the mendicants?” 

“Sirs,\marginnote{4.3} I don’t know all the mendicants’ views.” 

“Well\marginnote{5.1} then, since it seems you don’t know all the views of the ascetic Gotama or of the mendicants, tell us, householder, what is your view?” 

“Sirs,\marginnote{5.3} it’s not hard for me to explain what my views are. But please, let the venerables explain their own convictions first. Afterwards it won’t be hard for me to explain my views.” 

When\marginnote{6.1} he said this, one of the wanderers said to him, “The cosmos is eternal. This is the only truth, anything else is futile. That’s my view, householder.” 

Another\marginnote{7.1} wanderer said, “The cosmos is not eternal. This is the only truth, anything else is futile. That’s my view, householder.” 

Another\marginnote{8.1} wanderer said, “The cosmos is finite …” … “The cosmos is infinite …” … “The soul and the body are the same thing …” … “The soul and the body are different things …” … “A realized one still exists after death …” … “A realized one no longer exists after death …” … “A realized one both still exists and no longer exists after death …” … “A Realized One neither exists nor doesn’t exist after death. This is the only truth, anything else is futile. That’s my view, householder.” 

When\marginnote{9.1} this was said, \textsanskrit{Anāthapiṇḍika} said this, “Sirs, regarding the venerable who said this: ‘The cosmos is eternal. This is the only truth, anything else is futile. That’s my view, householder.’ This view of his has either arisen from his own irrational application of mind, or is conditioned by what someone else says. But that view is created, conditioned, chosen, dependently originated. Anything that is created, conditioned, chosen, and dependently originated is impermanent. And what’s impermanent is suffering. What he clings to and holds to is just suffering. 

Regarding\marginnote{10.1} the venerable who said this: ‘The cosmos is not eternal. This is the only truth, anything else is futile. That’s my view, householder.’ This view of his has either arisen from his own irrational application of mind, or is conditioned by what someone else says. But that view is created, conditioned, chosen, dependently originated. Anything that is created, conditioned, chosen, and dependently originated is impermanent. And what’s impermanent is suffering. What he clings to and holds to is just suffering. 

Regarding\marginnote{11.1} the venerable who said this: ‘The cosmos is finite …’ … ‘The cosmos is infinite …’ … ‘The soul and the body are the same thing …’ … ‘The soul and the body are different things …’ … ‘A realized one still exists after death …’ … ‘A realized one no longer exists after death …’ … ‘A realized one both still exists and no longer exists after death …’ … ‘A Realized One neither exists nor doesn’t exist after death. This is the only truth, anything else is futile. That’s my view, householder.’ This view of his has either arisen from his own irrational application of mind, or is conditioned by what someone else says. But that view is created, conditioned, chosen, dependently originated. Anything that is created, conditioned, chosen, and dependently originated is impermanent. And what’s impermanent is suffering. What he clings to and holds to is just suffering.” 

When\marginnote{12.1} he said this the wanderers said to him, “Householder, we’ve each explained our own convictions. Tell us, householder, what is your view?” 

“Sirs,\marginnote{12.4} anything that is created, conditioned, chosen, and dependently originated is impermanent. And what’s impermanent is suffering. And what’s suffering is not mine, I am not this, this is not my self. That’s my view, sirs.” 

“Householder,\marginnote{13.1} anything that is created, conditioned, chosen, and dependently originated is impermanent. And what’s impermanent is suffering. What you cling to and hold to is just suffering.” 

“Sirs,\marginnote{14.1} anything that is created, conditioned, chosen, and dependently originated is impermanent. And what’s impermanent is suffering. And I’ve truly seen clearly with right wisdom that what’s suffering is not mine, I am not this, it’s not my self. And I truly understand the escape beyond that.” 

When\marginnote{15.1} this was said, those wanderers sat silent, dismayed, shoulders drooping, downcast, depressed, with nothing to say. Seeing this, \textsanskrit{Anāthapiṇḍika} got up from his seat. He went to the Buddha, bowed, sat down to one side, and informed the Buddha of all they had discussed. 

“Good,\marginnote{15.3} good, householder! That’s how you should legitimately and completely refute those futile men from time to time.” 

Then\marginnote{16.1} the Buddha educated, encouraged, fired up, and inspired the householder \textsanskrit{Anāthapiṇḍika} with a Dhamma talk, after which \textsanskrit{Anāthapiṇḍika} got up from his seat, bowed, and respectfully circled the Buddha before leaving. 

Then,\marginnote{17.1} not long after \textsanskrit{Anāthapiṇḍika} had left, the Buddha addressed the mendicants: “Mendicants, even a mendicant who has ordained for a hundred years in this teaching and training would legitimately and completely refute those wanderers of other religions just as the householder \textsanskrit{Anāthapiṇḍika} did.” 

%
\section*{{\suttatitleacronym AN 10.94}{\suttatitletranslation With Vajjiyamāhita }{\suttatitleroot Vajjiyamāhitasutta}}
\addcontentsline{toc}{section}{\tocacronym{AN 10.94} \toctranslation{With Vajjiyamāhita } \tocroot{Vajjiyamāhitasutta}}
\markboth{With Vajjiyamāhita }{Vajjiyamāhitasutta}
\extramarks{AN 10.94}{AN 10.94}

At\marginnote{1.1} one time the Buddha was staying near \textsanskrit{Campā} on the banks of the \textsanskrit{Gaggarā} Lotus Pond. 

Then\marginnote{1.2} the householder \textsanskrit{Vajjiyamāhita} left \textsanskrit{Sāvatthī} in the middle of the day to see the Buddha. Then it occurred to him, “It’s the wrong time to see the Buddha, as he’s in retreat. And it’s the wrong time to see the esteemed mendicants, as they’re in retreat. Why don’t I visit the monastery of the wanderers of other religions?” 

Then\marginnote{2.1} he went to the monastery of the wanderers of other religions. Now at that time, the wanderers of other religions had come together, making an uproar, a dreadful racket as they sat and talked about all kinds of low topics. 

They\marginnote{3.1} saw \textsanskrit{Vajjiyamāhita} coming off in the distance, and stopped each other, saying, “Be quiet, good sirs, don’t make a sound. The householder \textsanskrit{Vajjiyamāhita}, a disciple of the ascetic Gotama, is coming into our monastery. He is included among the white-clothed lay disciples of the ascetic Gotama, who is residing near \textsanskrit{Campā}. Such venerables like the quiet, are educated to be quiet, and praise the quiet. Hopefully if he sees that our assembly is quiet he’ll see fit to approach.” 

Then\marginnote{4.1} those wanderers of other religions fell silent. Then \textsanskrit{Vajjiyamāhita} went up to them, and exchanged greetings with the wanderers there. When the greetings and polite conversation were over, he sat down to one side. The wanderers said to him: 

“Is\marginnote{4.4} it really true, householder? Does the ascetic Gotama criticize all forms of mortification? Does he categorically condemn and denounce those fervent mortifiers who live rough?” 

“No,\marginnote{4.5} sirs, the Buddha does not criticize all forms of mortification. Nor does he categorically condemn and denounce those fervent mortifiers who live rough. The Buddha criticizes where it is due, and praises where it is due. In doing so he is one who speaks after analyzing the question, and is not one-sided on this point.” 

When\marginnote{5.1} he said this, one of the wanderers said to him, “Hold on, householder! That ascetic Gotama who you praise is an exterminator who refrains from making statements.” 

“On\marginnote{5.3} this point, also, I reasonably respond to the venerables. The Buddha has stated ‘This is skillful’ and ‘This is unskillful’. So when it comes to what is skillful and unskillful the Buddha makes a statement. He is not an exterminator who refrains from making statements.” 

When\marginnote{6.1} this was said, those wanderers sat silent, dismayed, shoulders drooping, downcast, depressed, with nothing to say. Seeing this, \textsanskrit{Vajjiyamāhita} got up from his seat. He went to the Buddha, bowed, sat down to one side, and informed the Buddha of all they had discussed. 

“Good,\marginnote{7.1} good, householder! That’s how you should legitimately and completely refute those futile men from time to time. Householder, I don’t say that all mortifications should be undergone. But I don’t say that no mortifications should be undergone. I don’t say that all observances should be undertaken. But I don’t say that no observances should be undertaken. I don’t say that all efforts should be tried. But I don’t say that no efforts should be tried. I don’t say that everything should be given up. But I don’t say that nothing should be given up. I don’t say that you should be liberated with all kinds of freedom. But I don’t say that you should not be liberated with any kind of freedom. 

When\marginnote{8.1} undergoing certain mortifications, unskillful qualities grow while skillful qualities decline. I say that you shouldn’t undergo those mortifications. When undergoing certain mortifications, unskillful qualities decline while skillful qualities grow. I say that you should undergo those mortifications. 

When\marginnote{9.1} undertaking certain observances, unskillful qualities grow while skillful qualities decline. I say that you shouldn’t undertake those observances. When undertaking certain observances, unskillful qualities decline while skillful qualities grow. I say that you should undertake those observances. 

When\marginnote{10.1} trying certain efforts, unskillful qualities grow while skillful qualities decline. I say that you shouldn’t try those efforts. When trying certain efforts, unskillful qualities decline while skillful qualities grow. I say that you should try those efforts. 

When\marginnote{11.1} giving up certain things, unskillful qualities grow while skillful qualities decline. I say that you shouldn’t give up those things. When giving up certain things, unskillful qualities decline while skillful qualities grow. I say that you should give up those things. 

When\marginnote{12.1} being liberated with certain kinds of freedom, unskillful qualities grow while skillful qualities decline. I say that you shouldn’t be liberated with those kinds of freedom. When being liberated with certain kinds of freedom, unskillful qualities decline while skillful qualities grow. I say that you should be liberated with those kinds of freedom.” 

After\marginnote{13.1} \textsanskrit{Vajjiyamāhita} had been educated, encouraged, fired up, and inspired with a Dhamma talk by the Buddha, he got up from his seat, bowed, and respectfully circled the Buddha before leaving. 

Then,\marginnote{14.1} not long after \textsanskrit{Vajjiyamāhita} had left, the Buddha addressed the mendicants: “Mendicants, even a mendicant who for a long time has had little dust in their eye in this teaching and training would legitimately and completely refute those wanderers of other religions just as the householder \textsanskrit{Vajjiyamāhita} did.” 

%
\section*{{\suttatitleacronym AN 10.95}{\suttatitletranslation With Uttiya }{\suttatitleroot Uttiyasutta}}
\addcontentsline{toc}{section}{\tocacronym{AN 10.95} \toctranslation{With Uttiya } \tocroot{Uttiyasutta}}
\markboth{With Uttiya }{Uttiyasutta}
\extramarks{AN 10.95}{AN 10.95}

Then\marginnote{1.1} the wanderer Uttiya went up to the Buddha, and exchanged greetings with him. 

When\marginnote{1.2} the greetings and polite conversation were over, he sat down to one side and said to the Buddha, “Mister Gotama, is this right: ‘The cosmos is eternal. This is the only truth, anything else is futile’?” 

“This\marginnote{1.4} has not been declared by me, Uttiya.” 

“Then\marginnote{2.1} is this right: ‘The cosmos is not eternal. This is the only truth, anything else is futile’?” 

“This\marginnote{2.2} has not been declared by me, Uttiya.” 

“Then\marginnote{3.1} is this right: ‘The cosmos is finite …’ … ‘The cosmos is infinite …’ … ‘The soul and the body are the same thing …’ … ‘The soul and the body are different things …’ … ‘A realized one still exists after death …’ … ‘A realized one no longer exists after death …’ … ‘A realized one both still exists and no longer exists after death …’ … ‘A Realized One neither exists nor doesn’t exist after death. This is the only truth, anything else is futile’?” 

“This\marginnote{3.9} has not been declared by me, Uttiya.” 

“When\marginnote{4.1} asked about all these points, Mister Gotama says that they have not been declared by him. 

So\marginnote{6.1} what exactly has been declared by Mister Gotama?” 

“Uttiya,\marginnote{7.1} I teach my disciples from my own insight in order to purify sentient beings, to get past sorrow and crying, to make an end of pain and sadness, to end the cycle of suffering, and to realize extinguishment.” 

“But\marginnote{8.1} when Mister Gotama teaches in this way, is the whole world saved, or half, or a third?” But when he said this, the Buddha kept silent. 

Then\marginnote{9.1} Venerable Ānanda thought, “The wanderer Uttiya must not get the harmful misconception: ‘When the ascetic Gotama was asked this all-important question he falters without answering. He just can’t do it!’ That would be for his lasting harm and suffering.” 

Then\marginnote{10.1} Ānanda said to the wanderer Uttiya, “Well then, Reverend Uttiya, I shall give you a simile. For by means of a simile some sensible people understand the meaning of what is said. Suppose there was a king’s frontier citadel with fortified embankments, ramparts, and arches, and a single gate. And it has a gatekeeper who is astute, competent, and clever. He keeps strangers out and lets known people in. As he walks around the patrol path, he doesn’t see a hole or cleft in the wall, not even one big enough for a cat to slip out. He doesn’t know how many creatures enter or leave the citadel. But he does know that whatever sizable creatures enter or leave the citadel, all of them do so via this gate. 

In\marginnote{11.1} the same way, it’s not the Realized One’s concern whether the whole world is saved by this, or half, or a third. But the Realized One knows that whoever is saved from the world—whether in the past, the future, or the present—all have given up the five hindrances, corruptions of the heart that weaken wisdom. They have firmly established their mind in the four kinds of mindfulness meditation. And they have truly developed the seven awakening factors. That’s how they’re saved from the world, in the past, future, or present. Uttiya, you were just asking the Buddha the same question as before in a different way. That’s why he didn’t answer.” 

%
\section*{{\suttatitleacronym AN 10.96}{\suttatitletranslation With Kokanada }{\suttatitleroot Kokanudasutta}}
\addcontentsline{toc}{section}{\tocacronym{AN 10.96} \toctranslation{With Kokanada } \tocroot{Kokanudasutta}}
\markboth{With Kokanada }{Kokanudasutta}
\extramarks{AN 10.96}{AN 10.96}

At\marginnote{1.1} one time Venerable Ānanda was staying near \textsanskrit{Rājagaha} in the Hot Springs Monastery. Then Ānanda rose at the crack of dawn and went to the hot springs to bathe. When he had bathed and emerged from the water he stood in one robe drying his limbs. The wanderer Kokanada also rose at the crack of dawn and went to the hot springs to bathe. 

He\marginnote{2.1} saw Ānanda coming off in the distance and said to him, “Who’s here, reverend?” 

“I’m\marginnote{2.4} a mendicant, reverend.” 

“Of\marginnote{3.1} which mendicants?” 

“Of\marginnote{3.2} the ascetics who follow the Sakyan.” 

“I’d\marginnote{4.1} like to ask the venerable about a certain point, if you’d take the time to answer.” 

“Ask,\marginnote{4.2} reverend. When I’ve heard it I’ll know.” 

“Is\marginnote{5.1} this your view: ‘The cosmos is eternal. This is the only truth, anything else is futile’?” 

“That’s\marginnote{5.2} not my view, reverend.” 

“Then\marginnote{6.1} is this your view: ‘The cosmos is not eternal. This is the only truth, anything else is futile’?” 

“That’s\marginnote{6.2} not my view, reverend.” 

“Then\marginnote{7.1} is this your view: ‘The cosmos is finite …’ … ‘The cosmos is infinite …’ … ‘The soul and the body are the same thing …’ … ‘The soul and the body are different things …’ … ‘A realized one still exists after death …’ … ‘A realized one no longer exists after death …’ … ‘A realized one both still exists and no longer exists after death …’ … ‘A Realized One neither exists nor doesn’t exist after death. This is the only truth, anything else is futile’?” 

“That’s\marginnote{7.9} not my view, reverend.” 

“Then,\marginnote{8.1} sir, do you neither know nor see?” 

“That’s\marginnote{8.2} not so, reverend. I do know and see.” 

“When\marginnote{9.1} asked about all these points, you say that’s not your view. 

Yet\marginnote{12.1} when asked whether you neither know nor see, you say, ‘That’s not so, reverend. I do know and see.’ How then should we see the meaning of this statement?” 

“‘The\marginnote{13.1} cosmos is eternal. This is the only truth, anything else is futile:’ that’s a misconception. ‘The cosmos is not eternal. This is the only truth, anything else is futile:’ that’s a misconception. ‘The cosmos is finite …’ … ‘The cosmos is infinite …’ … ‘The soul and the body are the same thing …’ … ‘The soul and the body are different things …’ … ‘A realized one still exists after death …’ … ‘A realized one no longer exists after death …’ … ‘A realized one both still exists and no longer exists after death …’ … ‘A Realized One neither exists nor doesn’t exist after death. This is the only truth, anything else is futile:’ that’s a misconception. 

I\marginnote{14.1} know and see the scope of convictions, the scope of grounds for views, fixation on views, obsession with views, the origin of views, and the uprooting of views. Knowing and seeing thus, why should I say: ‘I neither know nor see?’ I do know and see.” 

“What\marginnote{15.1} is the venerable’s name? And how are you known among your spiritual companions?” 

“Reverend,\marginnote{15.2} my name is Ānanda. And that’s how I’m known among my spiritual companions.” 

“Goodness!\marginnote{15.4} I had no idea I was consulting such a great tutor as Venerable Ānanda! If I had known who you were, I wouldn’t have said so much. May Venerable Ānanda please forgive me.” 

%
\section*{{\suttatitleacronym AN 10.97}{\suttatitletranslation Worthy of Offerings Dedicated to the Gods }{\suttatitleroot Āhuneyyasutta}}
\addcontentsline{toc}{section}{\tocacronym{AN 10.97} \toctranslation{Worthy of Offerings Dedicated to the Gods } \tocroot{Āhuneyyasutta}}
\markboth{Worthy of Offerings Dedicated to the Gods }{Āhuneyyasutta}
\extramarks{AN 10.97}{AN 10.97}

“Mendicants,\marginnote{1.1} a mendicant with ten qualities is worthy of offerings dedicated to the gods, worthy of hospitality, worthy of a religious donation, worthy of veneration with joined palms, and is the supreme field of merit for the world. What ten? 

It’s\marginnote{2.2} when a mendicant is ethical, restrained in the monastic code, conducting themselves well and resorting for alms in suitable places. Seeing danger in the slightest fault, they keep the rules they’ve undertaken. 

They’re\marginnote{3.1} very learned, remembering and keeping what they’ve learned. These teachings are good in the beginning, good in the middle, and good in the end, meaningful and well-phrased, describing a spiritual practice that’s entirely full and pure. They are very learned in such teachings, remembering them, rehearsing them, mentally scrutinizing them, and comprehending them theoretically. 

They\marginnote{4.1} have good friends, companions, and associates. 

They\marginnote{5.1} have right view, possessing right perspective. 

They\marginnote{6.1} wield the many kinds of psychic power: multiplying themselves and becoming one again; appearing and disappearing; going unobstructed through a wall, a rampart, or a mountain as if through space; diving in and out of the earth as if it were water; walking on water as if it were earth; flying cross-legged through the sky like a bird; touching and stroking with the hand the sun and moon, so mighty and powerful. They control the body as far as the realm of divinity. 

With\marginnote{7.1} clairaudience that is purified and superhuman, they hear both kinds of sounds, human and heavenly, whether near or far. 

They\marginnote{8.1} understand the minds of other beings and individuals, having comprehended them with their own mind. They understand mind with greed as ‘mind with greed’, and mind without greed as ‘mind without greed’. They understand mind with hate … mind without hate … mind with delusion … mind without delusion … constricted mind … scattered mind … expansive mind … unexpansive mind … mind that is not supreme … mind that is supreme … mind immersed in \textsanskrit{samādhi} … mind not immersed in \textsanskrit{samādhi} … freed mind … They understand unfreed mind as ‘unfreed mind’. 

They\marginnote{9.1} recollect many kinds of past lives, that is, one, two, three, four, five, ten, twenty, thirty, forty, fifty, a hundred, a thousand, a hundred thousand rebirths; many eons of the world contracting, many eons of the world expanding, many eons of the world contracting and expanding. They remember: ‘There, I was named this, my clan was that, I looked like this, and that was my food. This was how I felt pleasure and pain, and that was how my life ended. When I passed away from that place I was reborn somewhere else. There, too, I was named this, my clan was that, I looked like this, and that was my food. This was how I felt pleasure and pain, and that was how my life ended. When I passed away from that place I was reborn here.’ Thus they recollect their many past lives, with features and details. 

With\marginnote{10.1} clairvoyance that is purified and superhuman, they see sentient beings passing away and being reborn—inferior and superior, beautiful and ugly, in a good place or a bad place. They understand how sentient beings are reborn according to their deeds. ‘These dear beings did bad things by way of body, speech, and mind. They denounced the noble ones; they had wrong view; and they chose to act out of that wrong view. When their body breaks up, after death, they’re reborn in a place of loss, a bad place, the underworld, hell. These dear beings, however, did good things by way of body, speech, and mind. They never denounced the noble ones; they had right view; and they chose to act out of that right view. When their body breaks up, after death, they’re reborn in a good place, a heavenly realm.’ And so, with clairvoyance that is purified and superhuman, they see sentient beings passing away and being reborn—inferior and superior, beautiful and ugly, in a good place or a bad place. They understand how sentient beings are reborn according to their deeds. 

They\marginnote{11.1} realize the undefiled freedom of heart and freedom by wisdom in this very life. And they live having realized it with their own insight due to the ending of defilements. 

A\marginnote{12.1} mendicant with these ten qualities is worthy of offerings dedicated to the gods, worthy of hospitality, worthy of a religious donation, worthy of veneration with joined palms, and is the supreme field of merit for the world.” 

%
\section*{{\suttatitleacronym AN 10.98}{\suttatitletranslation A Senior Mendicant }{\suttatitleroot Therasutta}}
\addcontentsline{toc}{section}{\tocacronym{AN 10.98} \toctranslation{A Senior Mendicant } \tocroot{Therasutta}}
\markboth{A Senior Mendicant }{Therasutta}
\extramarks{AN 10.98}{AN 10.98}

“Mendicants,\marginnote{1.1} a senior mendicant with ten qualities lives comfortably in whatever region they live. What ten? 

They\marginnote{1.3} are senior and have long gone forth. 

They’re\marginnote{1.4} ethical, restrained in the monastic code, conducting themselves well and resorting for alms in suitable places. Seeing danger in the slightest fault, they keep the rules they’ve undertaken. 

They’re\marginnote{1.5} very learned, remembering and keeping what they’ve learned. These teachings are good in the beginning, good in the middle, and good in the end, meaningful and well-phrased, describing a spiritual practice that’s entirely full and pure. They are very learned in such teachings, remembering them, reinforcing them by recitation, mentally scrutinizing them, and comprehending them theoretically. 

Both\marginnote{1.6} monastic codes have been passed down to them in detail, well analyzed, well mastered, well evaluated in both the rules and accompanying material. 

They’re\marginnote{1.7} skilled in raising and settling disciplinary issues. 

They\marginnote{1.8} love the teachings and are a delight to converse with, being full of joy in the teaching and training. 

They’re\marginnote{1.9} content with any kind of robes, almsfood, lodgings, and medicines and supplies for the sick. 

They\marginnote{1.10} look impressive when going out and coming back, and are well restrained when sitting in an inhabited area. 

They\marginnote{1.11} get the four absorptions—blissful meditations in this life that belong to the higher mind—when they want, without trouble or difficulty. 

They\marginnote{1.12} realize the undefiled freedom of heart and freedom by wisdom in this very life, and they live having realized it with their own insight due to the ending of defilements. 

A\marginnote{1.13} senior mendicant with these ten qualities lives comfortably in whatever region they live.” 

%
\section*{{\suttatitleacronym AN 10.99}{\suttatitletranslation With Upāli }{\suttatitleroot Upālisutta}}
\addcontentsline{toc}{section}{\tocacronym{AN 10.99} \toctranslation{With Upāli } \tocroot{Upālisutta}}
\markboth{With Upāli }{Upālisutta}
\extramarks{AN 10.99}{AN 10.99}

Then\marginnote{1.1} Venerable \textsanskrit{Upāli} went up to the Buddha, bowed, sat down to one side, and said to him, “Sir, I wish to frequent remote lodgings in the wilderness and the forest.” 

“\textsanskrit{Upāli},\marginnote{2.1} remote lodgings in the wilderness and the forest are challenging. It’s hard to maintain seclusion and hard to find joy in it. Staying alone, the forests seem to rob the mind of a mendicant who isn’t immersed in \textsanskrit{samādhi}. If someone should say this, ‘Though I don’t have immersion, I’m going to frequent remote lodgings in the wilderness and the forest.’ You can expect that they’ll sink down or float away. 

Suppose\marginnote{3.1} there was a large lake, and along comes a bull elephant with a height of seven or eight cubits. He’d think, ‘Why don’t I plunge into this lake and play around while washing my ears and back? When I’ve bathed and drunk, I’ll emerge from the water and go wherever I want.’ And that’s just what he does. Why is that? Because his large life-form finds a footing in the depths. 

Then\marginnote{4.1} along comes a rabbit or a cat. They’d think, ‘What difference is there between me and a bull elephant? Why don’t I plunge into this lake and play around while washing my ears and back? When I’ve bathed and drunk, I’ll emerge from the water and go wherever I want.’ They jump into the lake rashly, without thinking. You can expect that they’ll sink down or float away. Why is that? Because their little life-form finds no footing in the depths. If someone should say this, ‘Though I don’t have immersion, I’m going to frequent remote lodgings in the wilderness and the forest.’ You can expect that they’ll sink down or float away. 

Suppose\marginnote{5.1} there was a little baby boy playing in his own urine and feces. What do you think, \textsanskrit{Upāli}? Isn’t that a totally foolish game?” 

“Yes,\marginnote{5.4} sir.” 

“After\marginnote{6.1} some time that boy grows up and his faculties mature. He accordingly plays childish games such as toy plows, tip-cat, somersaults, pinwheels, toy measures, toy carts, and toy bows. What do you think, \textsanskrit{Upāli}? Aren’t such games better than what he did before?” 

“Yes,\marginnote{6.5} sir.” 

“After\marginnote{7.1} some time that boy grows up and his faculties mature further. He accordingly amuses himself, supplied and provided with the five kinds of sensual stimulation. Sights known by the eye, which are likable, desirable, agreeable, pleasant, sensual, and arousing. Sounds known by the ear … Smells known by the nose … Tastes known by the tongue … Touches known by the body, which are likable, desirable, agreeable, pleasant, sensual, and arousing. What do you think, \textsanskrit{Upāli}? Aren’t such games better than what he did before?” 

“Yes,\marginnote{7.10} sir.” 

“But\marginnote{8.1} then a Realized One arises in the world, perfected, a fully awakened Buddha, accomplished in knowledge and conduct, holy, knower of the world, supreme guide for those who wish to train, teacher of gods and humans, awakened, blessed. He has realized with his own insight this world—with its gods, \textsanskrit{Māras}, and divinities, this population with its ascetics and brahmins, gods and humans—and he makes it known to others. He proclaims a teaching that is good in the beginning, good in the middle, and good in the end, meaningful and well-phrased. And he reveals a spiritual practice that’s entirely full and pure. 

A\marginnote{9.1} householder hears that teaching, or a householder’s child, or someone reborn in a good family. They gain faith in the Realized One and reflect, ‘Living in a house is cramped and dirty, but the life of one gone forth is wide open. It’s not easy for someone living at home to lead the spiritual life utterly full and pure, like a polished shell. Why don’t I shave off my hair and beard, dress in ocher robes, and go forth from the lay life to homelessness?’ 

After\marginnote{10.1} some time they give up a large or small fortune, and a large or small family circle. They shave off hair and beard, dress in ocher robes, and go forth from the lay life to homelessness. 

Once\marginnote{11.1} they’ve gone forth, they take up the training and livelihood of the mendicants. They give up killing living creatures, renouncing the rod and the sword. They’re scrupulous and kind, living full of sympathy for all living beings. 

They\marginnote{12.1} give up stealing. They take only what’s given, and expect only what’s given. They keep themselves clean by not thieving. 

They\marginnote{13.1} give up unchastity. They are celibate, set apart, avoiding the vulgar act of sex. 

They\marginnote{14.1} give up lying. They speak the truth and stick to the truth. They’re honest and dependable, and don’t trick the world with their words. 

They\marginnote{15.1} give up divisive speech. They don’t repeat in one place what they heard in another so as to divide people against each other. Instead, they reconcile those who are divided, supporting unity, delighting in harmony, loving harmony, speaking words that promote harmony. 

They\marginnote{16.1} give up harsh speech. They speak in a way that’s mellow, pleasing to the ear, lovely, going to the heart, polite, likable and agreeable to the people. 

They\marginnote{17.1} give up talking nonsense. Their words are timely, true, and meaningful, in line with the teaching and training. They say things at the right time which are valuable, reasonable, succinct, and beneficial. 

They\marginnote{18.1} refrain from injuring plants and seeds. They eat in one part of the day, abstaining from eating at night and food at the wrong time. They refrain from seeing shows of dancing, singing, and music . They refrain from beautifying and adorning themselves with garlands, fragrance, and makeup. They refrain from high and luxurious beds. They refrain from receiving gold and currency, raw grains, raw meat, women and girls, male and female bondservants, goats and sheep, chickens and pigs, elephants, cows, horses, and mares, and fields and land. They refrain from running errands and messages; buying and selling; falsifying weights, metals, or measures; bribery, fraud, cheating, and duplicity; mutilation, murder, abduction, banditry, plunder, and violence. 

They’re\marginnote{19.1} content with robes to look after the body and almsfood to look after the belly. Wherever they go, they set out taking only these things. They’re like a bird: wherever it flies, wings are its only burden. In the same way, a mendicant is content with robes to look after the body and almsfood to look after the belly. Wherever they go, they set out taking only these things. When they have this entire spectrum of noble ethics, they experience a blameless happiness inside themselves. 

When\marginnote{20.1} they see a sight with their eyes, they don’t get caught up in the features and details. If the faculty of sight were left unrestrained, bad unskillful qualities of covetousness and displeasure would become overwhelming. For this reason, they practice restraint, protecting the faculty of sight, and achieving restraint over it. When they hear a sound with their ears … When they smell an odor with their nose … When they taste a flavor with their tongue … When they feel a touch with their body … When they know an idea with their mind, they don’t get caught up in the features and details. If the faculty of mind were left unrestrained, bad unskillful qualities of covetousness and displeasure would become overwhelming. For this reason, they practice restraint, protecting the faculty of mind, and achieving its restraint. When they have this noble sense restraint, they experience an unsullied bliss inside themselves. 

They\marginnote{21.1} act with situational awareness when going out and coming back; when looking ahead and aside; when bending and extending the limbs; when bearing the outer robe, bowl and robes; when eating, drinking, chewing, and tasting; when urinating and defecating; when walking, standing, sitting, sleeping, waking, speaking, and keeping silent. 

When\marginnote{22.1} they have this entire spectrum of noble ethics, this noble sense restraint, and this noble mindfulness and situational awareness, they frequent a secluded lodging—a wilderness, the root of a tree, a hill, a ravine, a mountain cave, a charnel ground, a forest, the open air, a heap of straw. Gone to a wilderness, or to the root of a tree, or to an empty hut, they sit down cross-legged, set their body straight, and establish mindfulness in their presence. 

Giving\marginnote{23.1} up covetousness for the world, they meditate with a heart rid of covetousness, cleansing the mind of covetousness. Giving up ill will and malevolence, they meditate with a mind rid of ill will, full of sympathy for all living beings, cleansing the mind of ill will. Giving up dullness and drowsiness, they meditate with a mind rid of dullness and drowsiness, perceiving light, mindful and aware, cleansing the mind of dullness and drowsiness. Giving up restlessness and remorse, they meditate without restlessness, their mind peaceful inside, cleansing the mind of restlessness and remorse. Giving up doubt, they meditate having gone beyond doubt, not undecided about skillful qualities, cleansing the mind of doubt. 

They\marginnote{24.1} give up these five hindrances, corruptions of the heart that weaken wisdom. Then, quite secluded from sensual pleasures, secluded from unskillful qualities, they enter and remain in the first absorption, which has the rapture and bliss born of seclusion, while placing the mind and keeping it connected. What do you think, \textsanskrit{Upāli}? Isn’t this state better than what they had before?” 

“Yes,\marginnote{24.5} sir.” 

“When\marginnote{25.1} my disciples see this quality inside themselves they frequent remote lodgings in the wilderness and the forest. But so far they haven’t achieved their own goal. 

Furthermore,\marginnote{26.1} as the placing of the mind and keeping it connected are stilled, a mendicant enters and remains in the second absorption, which has the rapture and bliss born of immersion, with internal clarity and mind at one, without placing the mind and keeping it connected. What do you think, \textsanskrit{Upāli}? Isn’t this state better than what they had before?” 

“Yes,\marginnote{26.4} sir.” 

“When\marginnote{27.1} my disciples see this quality inside themselves they frequent remote lodgings in the wilderness and the forest. But so far they haven’t achieved their own goal. 

Furthermore,\marginnote{28.1} with the fading away of rapture, a mendicant enters and remains in the third absorption. They meditate with equanimity, mindful and aware, personally experiencing the bliss of which the noble ones declare, ‘Equanimous and mindful, one meditates in bliss.’ What do you think, \textsanskrit{Upāli}? Isn’t this state better than what they had before?” 

“Yes,\marginnote{28.4} sir.” 

“When\marginnote{29.1} my disciples see this quality inside themselves they frequent remote lodgings in the wilderness and the forest. But so far they haven’t achieved their own goal. 

Furthermore,\marginnote{30.1} giving up pleasure and pain, and ending former happiness and sadness, a mendicant enters and remains in the fourth absorption, without pleasure or pain, with pure equanimity and mindfulness. …” … 

“Furthermore,\marginnote{31.1} going totally beyond perceptions of form, with the ending of perceptions of impingement, not focusing on perceptions of diversity, aware that ‘space is infinite’, a mendicant enters and remains in the dimension of infinite space. What do you think, \textsanskrit{Upāli}? Isn’t this state better than what they had before?” 

“Yes,\marginnote{31.4} sir.” 

“When\marginnote{32.1} my disciples see this quality inside themselves they frequent remote lodgings in the wilderness and the forest. But so far they haven’t achieved their own goal. 

Furthermore,\marginnote{33.1} going totally beyond the dimension of infinite space, aware that ‘consciousness is infinite’, a mendicant enters and remains in the dimension of infinite consciousness. …” … 

“Going\marginnote{34.1} totally beyond the dimension of infinite consciousness, aware that ‘there is nothing at all’, they enter and remain in the dimension of nothingness. …” … 

“Going\marginnote{35.1} totally beyond the dimension of nothingness, aware that ‘this is peaceful, this is sublime’, they enter and remain in the dimension of neither perception nor non-perception. What do you think, \textsanskrit{Upāli}? Isn’t this state better than what they had before?” 

“Yes,\marginnote{35.4} sir.” 

“When\marginnote{36.1} my disciples see this quality inside themselves they frequent remote lodgings in the wilderness and the forest. But so far they haven’t achieved their own goal. 

Furthermore,\marginnote{37.1} going totally beyond the dimension of neither perception nor non-perception, they enter and remain in the cessation of perception and feeling. And, having seen with wisdom, their defilements come to an end. What do you think, \textsanskrit{Upāli}? Isn’t this state better than what they had before?” 

“Yes,\marginnote{37.4} sir.” 

“When\marginnote{38.1} my disciples see this quality inside themselves they frequent remote lodgings in the wilderness and the forest. And they have achieved their own goal. Come on, \textsanskrit{Upāli}, stay with the \textsanskrit{Saṅgha}. If you stay with the \textsanskrit{Saṅgha} you’ll be comfortable.” 

%
\section*{{\suttatitleacronym AN 10.100}{\suttatitletranslation Cannot }{\suttatitleroot Abhabbasutta}}
\addcontentsline{toc}{section}{\tocacronym{AN 10.100} \toctranslation{Cannot } \tocroot{Abhabbasutta}}
\markboth{Cannot }{Abhabbasutta}
\extramarks{AN 10.100}{AN 10.100}

“Mendicants,\marginnote{1.1} without giving up ten things you can’t realize perfection. What ten? Greed, hate, delusion, anger, acrimony, disdain, contempt, jealousy, stinginess, and conceit. Without giving up these ten things you can’t realize perfection. 

After\marginnote{2.1} giving up ten things you can realize perfection. What ten? Greed, hate, delusion, anger, acrimony, disdain, contempt, jealousy, stinginess, and conceit. After giving up these ten things you can realize perfection.” 

%
\addtocontents{toc}{\let\protect\contentsline\protect\nopagecontentsline}
\pannasa{The Third Fifty }
\addcontentsline{toc}{pannasa}{The Third Fifty }
\markboth{}{}
\addtocontents{toc}{\let\protect\contentsline\protect\oldcontentsline}

%
\addtocontents{toc}{\let\protect\contentsline\protect\nopagecontentsline}
\chapter*{The Chapter on Perceptions for Ascetics }
\addcontentsline{toc}{chapter}{\tocchapterline{The Chapter on Perceptions for Ascetics }}
\addtocontents{toc}{\let\protect\contentsline\protect\oldcontentsline}

%
\section*{{\suttatitleacronym AN 10.101}{\suttatitletranslation Perceptions for Ascetics }{\suttatitleroot Samaṇasaññāsutta}}
\addcontentsline{toc}{section}{\tocacronym{AN 10.101} \toctranslation{Perceptions for Ascetics } \tocroot{Samaṇasaññāsutta}}
\markboth{Perceptions for Ascetics }{Samaṇasaññāsutta}
\extramarks{AN 10.101}{AN 10.101}

“Mendicants,\marginnote{1.1} when these three perceptions for ascetics are developed and cultivated they fulfill seven things. What three? ‘I have secured freedom from class.’ ‘My livelihood is tied up with others.’ ‘My behavior should be different.’ When these three perceptions for ascetics are developed and cultivated they fulfill seven things. 

What\marginnote{2.1} seven? Their deeds and behavior are always consistent with the precepts. They’re content, kind-hearted, and humble. They want to train. They use the necessities of life after reflecting on their purpose. They’re energetic. When those three perceptions for ascetics are developed and cultivated they fulfill these seven things.” 

%
\section*{{\suttatitleacronym AN 10.102}{\suttatitletranslation Awakening Factors }{\suttatitleroot Bojjhaṅgasutta}}
\addcontentsline{toc}{section}{\tocacronym{AN 10.102} \toctranslation{Awakening Factors } \tocroot{Bojjhaṅgasutta}}
\markboth{Awakening Factors }{Bojjhaṅgasutta}
\extramarks{AN 10.102}{AN 10.102}

“Mendicants,\marginnote{1.1} when the seven awakening factors are developed and cultivated they fulfill three knowledges. What seven? The awakening factors of mindfulness, investigation of principles, energy, rapture, tranquility, immersion, and equanimity. When these seven awakening factors are developed and cultivated they fulfill three knowledges. What three? It’s when a mendicant recollects their many kinds of past lives. That is: one, two, three, four, five, ten, twenty, thirty, forty, fifty, a hundred, a thousand, a hundred thousand rebirths; many eons of the world contracting, many eons of the world expanding, many eons of the world contracting and expanding. They recollect their many kinds of past lives, with features and details. With clairvoyance that is purified and surpasses the human, they understand how sentient beings are reborn according to their deeds. They realize the undefiled freedom of heart and freedom by wisdom in this very life, and live having realized it with their own insight due to the ending of defilements. When those seven awakening factors are developed and cultivated they fulfill these three knowledges.” 

%
\section*{{\suttatitleacronym AN 10.103}{\suttatitletranslation The Wrong Way }{\suttatitleroot Micchattasutta}}
\addcontentsline{toc}{section}{\tocacronym{AN 10.103} \toctranslation{The Wrong Way } \tocroot{Micchattasutta}}
\markboth{The Wrong Way }{Micchattasutta}
\extramarks{AN 10.103}{AN 10.103}

“Mendicants,\marginnote{1.1} relying on the wrong way leads to failure, not success. And how does relying on the wrong way lead to failure, not success? Wrong view gives rise to wrong thought. Wrong thought gives rise to wrong speech. Wrong speech gives rise to wrong action. Wrong action gives rise to wrong livelihood. Wrong livelihood gives rise to wrong effort. Wrong effort gives rise to wrong mindfulness. Wrong mindfulness gives rise to wrong immersion. Wrong immersion gives rise to wrong knowledge. Wrong knowledge gives rise to wrong freedom. That’s how relying on the wrong way leads to failure, not success. 

Relying\marginnote{2.1} on the right way leads to success, not failure. And how does relying on the right way lead to success, not failure? Right view gives rise to right thought. Right thought gives rise to right speech. Right speech gives rise to right action. Right action gives rise to right livelihood. Right livelihood gives rise to right effort. Right effort gives rise to right mindfulness. Right mindfulness gives rise to right immersion. Right immersion gives rise to right knowledge. Right knowledge gives rise to right freedom. That’s how relying on the right way leads to success, not failure.” 

%
\section*{{\suttatitleacronym AN 10.104}{\suttatitletranslation A Seed }{\suttatitleroot Bījasutta}}
\addcontentsline{toc}{section}{\tocacronym{AN 10.104} \toctranslation{A Seed } \tocroot{Bījasutta}}
\markboth{A Seed }{Bījasutta}
\extramarks{AN 10.104}{AN 10.104}

“Mendicants,\marginnote{1.1} consider an individual who has wrong view, thought, speech, action, livelihood, effort, mindfulness, immersion, knowledge, and freedom. Whatever bodily, verbal, or mental deeds they undertake in line with that view, their intentions, aims, wishes, and choices all lead to what is unlikable, undesirable, disagreeable, harmful, and suffering. Why is that? Because their view is bad. 

Suppose\marginnote{2.1} a seed of neem, angled gourd, or bitter gourd was planted in moist earth. Whatever nutrients it takes up from the earth and water would lead to its bitter, acerbic, and unpleasant taste. Why is that? Because the seed is bad. In the same way, consider an individual who has wrong view, thought, speech, action, livelihood, effort, mindfulness, immersion, knowledge, and freedom. Whatever bodily, verbal, or mental deeds they undertake in line with that view, their intentions, aims, wishes, and choices all lead to what is unlikable, undesirable, disagreeable, harmful, and suffering. Why is that? Because their view is bad. 

Consider\marginnote{3.1} an individual who has right view, thought, speech, action, livelihood, effort, mindfulness, immersion, knowledge, and freedom. Whatever bodily, verbal, or mental deeds they undertake in line with that view, their intentions, aims, wishes, and choices all lead to what is likable, desirable, agreeable, beneficial, and pleasant. Why is that? Because their view is good. 

Suppose\marginnote{4.1} a seed of sugar cane, fine rice, or grape was planted in moist earth. Whatever nutrients it takes up from the earth and water would lead to its sweet, pleasant, and delicious taste. Why is that? Because the seed is fine. In the same way, consider a person who has right view, thought, speech, action, livelihood, effort, mindfulness, immersion, knowledge, and freedom. Whatever bodily, verbal, or mental deeds they undertake in line with that view, their intentions, aims, wishes, and choices all lead to what is likable, desirable, agreeable, beneficial, and pleasant. Why is that? Because their view is good.” 

%
\section*{{\suttatitleacronym AN 10.105}{\suttatitletranslation Knowledge }{\suttatitleroot Vijjāsutta}}
\addcontentsline{toc}{section}{\tocacronym{AN 10.105} \toctranslation{Knowledge } \tocroot{Vijjāsutta}}
\markboth{Knowledge }{Vijjāsutta}
\extramarks{AN 10.105}{AN 10.105}

“Mendicants,\marginnote{1.1} ignorance precedes the attainment of unskillful qualities, with lack of conscience and prudence following along. An ignoramus, sunk in ignorance, gives rise to wrong view. Wrong view gives rise to wrong thought. Wrong thought gives rise to wrong speech. Wrong speech gives rise to wrong action. Wrong action gives rise to wrong livelihood. Wrong livelihood gives rise to wrong effort. Wrong effort gives rise to wrong mindfulness. Wrong mindfulness gives rise to wrong immersion. Wrong immersion gives rise to wrong knowledge. Wrong knowledge gives rise to wrong freedom. 

Knowledge\marginnote{2.1} precedes the attainment of skillful qualities, with conscience and prudence following along. A sage, firm in knowledge, gives rise to right view. Right view gives rise to right thought. Right thought gives rise to right speech. Right speech gives rise to right action. Right action gives rise to right livelihood. Right livelihood gives rise to right effort. Right effort gives rise to right mindfulness. Right mindfulness gives rise to right immersion. Right immersion gives rise to right knowledge. Right knowledge gives rise to right freedom.” 

%
\section*{{\suttatitleacronym AN 10.106}{\suttatitletranslation Wearing Away }{\suttatitleroot Nijjarasutta}}
\addcontentsline{toc}{section}{\tocacronym{AN 10.106} \toctranslation{Wearing Away } \tocroot{Nijjarasutta}}
\markboth{Wearing Away }{Nijjarasutta}
\extramarks{AN 10.106}{AN 10.106}

“Mendicants,\marginnote{1.1} there are these ten grounds for wearing away. What ten? 

For\marginnote{1.3} one of right view, wrong view is worn away. And the many bad, unskillful qualities that arise because of wrong view are worn away. And because of right view, many skillful qualities are fully developed. 

For\marginnote{2.1} one of right thought, wrong thought is worn away. And the many bad, unskillful qualities that arise because of wrong thought are worn away. And because of right thought, many skillful qualities are fully developed. 

For\marginnote{3.1} one of right speech, wrong speech is worn away. And the many bad, unskillful qualities that arise because of wrong speech are worn away. And because of right speech, many skillful qualities are fully developed. 

For\marginnote{4.1} one of right action, wrong action is worn away. And the many bad, unskillful qualities that arise because of wrong action are worn away. And because of right action, many skillful qualities are fully developed. 

For\marginnote{5.1} one of right livelihood, wrong livelihood is worn away. And the many bad, unskillful qualities that arise because of wrong livelihood are worn away. And because of right livelihood, many skillful qualities are fully developed. 

For\marginnote{6.1} one of right effort, wrong effort is worn away. And the many bad, unskillful qualities that arise because of wrong effort are worn away. And because of right effort, many skillful qualities are fully developed. 

For\marginnote{7.1} one of right mindfulness, wrong mindfulness is worn away. And the many bad, unskillful qualities that arise because of wrong mindfulness are worn away. And because of right mindfulness, many skillful qualities are fully developed. 

For\marginnote{8.1} one of right immersion, wrong immersion is worn away. And the many bad, unskillful qualities that arise because of wrong immersion are worn away. And because of right immersion, many skillful qualities are fully developed. 

For\marginnote{9.1} one of right knowledge, wrong knowledge is worn away. And the many bad, unskillful qualities that arise because of wrong knowledge are worn away. And because of right knowledge, many skillful qualities are fully developed. 

For\marginnote{10.1} one of right freedom, wrong freedom is worn away. And the many bad, unskillful qualities that arise because of wrong freedom are worn away. And because of right freedom, many skillful qualities are fully developed. 

These\marginnote{11.1} are the ten grounds for wearing away.” 

%
\section*{{\suttatitleacronym AN 10.107}{\suttatitletranslation The Bone-Washing Ceremony }{\suttatitleroot Dhovanasutta}}
\addcontentsline{toc}{section}{\tocacronym{AN 10.107} \toctranslation{The Bone-Washing Ceremony } \tocroot{Dhovanasutta}}
\markboth{The Bone-Washing Ceremony }{Dhovanasutta}
\extramarks{AN 10.107}{AN 10.107}

“Mendicants,\marginnote{1.1} in the southern lands there is a ceremony named ‘bone-washing’. There they have food, drink, snacks, meals, refreshments, and beverages, as well as dancing, singing, and music. There is such a washing, I don’t deny it. But that washing is low, crude, ordinary, ignoble, and pointless. It doesn’t lead to disillusionment, dispassion, cessation, peace, insight, awakening, and extinguishment. 

I\marginnote{2.1} will teach a noble washing that leads solely to disillusionment, dispassion, cessation, peace, insight, awakening, and extinguishment. Relying on that washing, sentient beings who are liable to rebirth, old age, and death, to sorrow, lamentation, pain, sadness, and distress are freed from all these things. Listen and apply your mind well, I will speak.” 

“Yes,\marginnote{2.3} sir,” they replied. The Buddha said this: 

“And\marginnote{3.1} what is that noble washing? 

For\marginnote{4.1} one of right view, wrong view is washed away. And the many bad, unskillful qualities that arise because of wrong view are washed away. And because of right view, many skillful qualities are fully developed. 

For\marginnote{5.1} one of right thought, wrong thought is washed away. … For one of right speech, wrong speech is washed away. … For one of right action, wrong action is washed away. … For one of right livelihood, wrong livelihood is washed away. … For one of right effort, wrong effort is washed away. … For one of right mindfulness, wrong mindfulness is washed away. … For one of right immersion, wrong immersion is washed away. … For one of right knowledge, wrong knowledge is washed away. … 

For\marginnote{6.1} one of right freedom, wrong freedom is washed away. And the many bad, unskillful qualities that arise because of wrong freedom are washed away. And because of right freedom, many skillful qualities are fully developed. This is the noble washing that leads solely to disillusionment, dispassion, cessation, peace, insight, awakening, and extinguishment. Relying on this washing, sentient beings who are liable to rebirth, old age, and death, to sorrow, lamentation, pain, sadness, and distress are freed from all these things.” 

%
\section*{{\suttatitleacronym AN 10.108}{\suttatitletranslation Doctors }{\suttatitleroot Tikicchakasutta}}
\addcontentsline{toc}{section}{\tocacronym{AN 10.108} \toctranslation{Doctors } \tocroot{Tikicchakasutta}}
\markboth{Doctors }{Tikicchakasutta}
\extramarks{AN 10.108}{AN 10.108}

“Mendicants,\marginnote{1.1} doctors prescribe a purgative for eliminating illnesses stemming from disorders of bile, phlegm, and wind. There is such a purgative, I don’t deny it. But this kind of purgative sometimes works and sometimes fails. 

I\marginnote{2.1} will teach a noble purgative that works without fail. Relying on that purgative, sentient beings who are liable to rebirth, old age, and death, to sorrow, lamentation, pain, sadness, and distress are freed from all these things. Listen and apply your mind well, I will speak.” 

“Yes,\marginnote{2.3} sir,” they replied. The Buddha said this: 

“And\marginnote{3.1} what is the noble purgative that works without fail? 

For\marginnote{4.1} one of right view, wrong view is purged. And the many bad, unskillful qualities produced by wrong view are purged. And because of right view, many skillful qualities are fully developed. 

For\marginnote{5.1} one of right thought, wrong thought is purged. … For one of right speech, wrong speech is purged. … For one of right action, wrong action is purged. … For one of right livelihood, wrong livelihood is purged. … For one of right effort, wrong effort is purged. For one of right mindfulness, wrong mindfulness is purged. … For one of right immersion, wrong immersion is purged. … For one of right knowledge, wrong knowledge is purged. … 

For\marginnote{6.1} one of right freedom, wrong freedom is purged. And the many bad, unskillful qualities produced by wrong freedom are purged. And because of right freedom, many skillful qualities are fully developed. This is the noble purgative that works without fail. Relying on this purgative, sentient beings who are liable to rebirth, old age, and death, to sorrow, lamentation, pain, sadness, and distress are freed from all these things.” 

%
\section*{{\suttatitleacronym AN 10.109}{\suttatitletranslation Emetic }{\suttatitleroot Vamanasutta}}
\addcontentsline{toc}{section}{\tocacronym{AN 10.109} \toctranslation{Emetic } \tocroot{Vamanasutta}}
\markboth{Emetic }{Vamanasutta}
\extramarks{AN 10.109}{AN 10.109}

“Mendicants,\marginnote{1.1} doctors prescribe an emetic for eliminating illnesses stemming from disorders of bile, phlegm, and wind. There is such an emetic, I don’t deny it. But this kind of emetic sometimes works and sometimes fails. 

I\marginnote{2.1} will teach a noble emetic that works without fail. Relying on that emetic, sentient beings who are liable to rebirth, old age, and death, to sorrow, lamentation, pain, sadness, and distress are freed from all these things. Listen and apply your mind well, I will speak. … 

And\marginnote{3.1} what is that noble emetic that works without fail? 

For\marginnote{4.1} one of right view, wrong view is vomited up. And the many bad, unskillful qualities produced by wrong view are vomited up. And because of right view, many skillful qualities are fully developed. 

For\marginnote{5.1} one of right thought, wrong thought is vomited up. … For one of right speech, wrong speech is vomited up. … For one of right action, wrong action is vomited up. … For one of right livelihood, wrong livelihood is vomited up. … For one of right effort, wrong effort is vomited up. … For one of right mindfulness, wrong mindfulness is vomited up. … For one of right immersion, wrong immersion is vomited up. … For one of right knowledge, wrong knowledge is vomited up. … 

For\marginnote{6.1} one of right freedom, wrong freedom is vomited up. And the many bad, unskillful qualities produced by wrong freedom are vomited up. And because of right freedom, many skillful qualities are fully developed. This is the noble emetic that works without fail. Relying on this emetic, sentient beings who are liable to rebirth, old age, and death, to sorrow, lamentation, pain, sadness, and distress are freed from all these things.” 

%
\section*{{\suttatitleacronym AN 10.110}{\suttatitletranslation Blown Away }{\suttatitleroot Niddhamanīyasutta}}
\addcontentsline{toc}{section}{\tocacronym{AN 10.110} \toctranslation{Blown Away } \tocroot{Niddhamanīyasutta}}
\markboth{Blown Away }{Niddhamanīyasutta}
\extramarks{AN 10.110}{AN 10.110}

“Mendicants,\marginnote{1.1} these ten qualities should be blown away. What ten? For one of right view, wrong view is blown away. And the many bad, unskillful qualities produced by wrong view are blown away. And because of right view, many skillful qualities are fully developed. 

For\marginnote{2.1} one of right thought, wrong thought is blown away. … For one of right speech, wrong speech is blown away. … For one of right action, wrong action is blown away. … For one of right livelihood, wrong livelihood is blown away. … For one of right effort, wrong effort is blown away. … For one of right mindfulness, wrong mindfulness is blown away. … For one of right immersion, wrong immersion is blown away. … For one of right knowledge, wrong knowledge is blown away. … 

For\marginnote{3.1} one of right freedom, wrong freedom is blown away. And the many bad, unskillful qualities produced by wrong freedom are blown away. And because of right freedom, many skillful qualities are fully developed. These are the ten qualities that should be blown away.” 

%
\section*{{\suttatitleacronym AN 10.111}{\suttatitletranslation An Adept (1st) }{\suttatitleroot Paṭhamaasekhasutta}}
\addcontentsline{toc}{section}{\tocacronym{AN 10.111} \toctranslation{An Adept (1st) } \tocroot{Paṭhamaasekhasutta}}
\markboth{An Adept (1st) }{Paṭhamaasekhasutta}
\extramarks{AN 10.111}{AN 10.111}

Then\marginnote{1.1} a mendicant went up to the Buddha, bowed, sat down to one side, and said to him: 

“Sir,\marginnote{2.1} they speak of this person called ‘an adept’. How is an adept mendicant defined?” 

“Mendicant,\marginnote{2.3} it’s when a mendicant has an adept’s right view, right thought, right speech, right action, right livelihood, right effort, right mindfulness, right immersion, right knowledge, and right freedom. That’s how a mendicant is an adept.” 

%
\section*{{\suttatitleacronym AN 10.112}{\suttatitletranslation An Adept (2nd) }{\suttatitleroot Dutiyaasekhasutta}}
\addcontentsline{toc}{section}{\tocacronym{AN 10.112} \toctranslation{An Adept (2nd) } \tocroot{Dutiyaasekhasutta}}
\markboth{An Adept (2nd) }{Dutiyaasekhasutta}
\extramarks{AN 10.112}{AN 10.112}

“Mendicants,\marginnote{1.1} there are ten qualities of an adept. What ten? An adept's right view, right thought, right speech, right action, right livelihood, right effort, right mindfulness, right immersion, right knowledge, and right freedom. These are the ten qualities of an adept.” 

%
\addtocontents{toc}{\let\protect\contentsline\protect\nopagecontentsline}
\chapter*{The Chapter on the Ceremony of Descent }
\addcontentsline{toc}{chapter}{\tocchapterline{The Chapter on the Ceremony of Descent }}
\addtocontents{toc}{\let\protect\contentsline\protect\oldcontentsline}

%
\section*{{\suttatitleacronym AN 10.113}{\suttatitletranslation Bad Principles (1st) }{\suttatitleroot Paṭhamaadhammasutta}}
\addcontentsline{toc}{section}{\tocacronym{AN 10.113} \toctranslation{Bad Principles (1st) } \tocroot{Paṭhamaadhammasutta}}
\markboth{Bad Principles (1st) }{Paṭhamaadhammasutta}
\extramarks{AN 10.113}{AN 10.113}

“Mendicants,\marginnote{1.1} you should know bad principles with bad results. And you should know good principles with good results. Knowing these things, your practice should follow the good principles with good results. 

And\marginnote{2.1} what are bad principles with bad results? Wrong view, wrong thought, wrong speech, wrong action, wrong livelihood, wrong effort, wrong mindfulness, wrong immersion, wrong knowledge, and wrong freedom. These are called bad principles with bad results. 

And\marginnote{3.1} what are good principles with good results? Right view, right thought, right speech, right action, right livelihood, right effort, right mindfulness, right immersion, right knowledge, and right freedom. These are called good principles with good results. 

‘You\marginnote{4.1} should know bad principles with bad results. And you should know good principles with good results. Knowing these things, your practice should follow the good principles with good results.’ That’s what I said, and this is why I said it.” 

%
\section*{{\suttatitleacronym AN 10.114}{\suttatitletranslation Bad Principles (2nd) }{\suttatitleroot Dutiyaadhammasutta}}
\addcontentsline{toc}{section}{\tocacronym{AN 10.114} \toctranslation{Bad Principles (2nd) } \tocroot{Dutiyaadhammasutta}}
\markboth{Bad Principles (2nd) }{Dutiyaadhammasutta}
\extramarks{AN 10.114}{AN 10.114}

“Mendicants,\marginnote{1.1} you should know bad principles and good principles. And you should know bad results and good results. Knowing these things, your practice should follow the good principles with good results. 

So\marginnote{2.1} what are bad principles? What are good principles? What are bad results? And what are good results? 

Wrong\marginnote{3.1} view is a bad principle. Right view is a good principle. And the many bad, unskillful qualities produced by wrong view are bad results. And the many skillful qualities fully developed because of right view are good results. 

Wrong\marginnote{4.1} thought is a bad principle. Right thought is a good principle. And the many bad, unskillful qualities produced by wrong thought are bad results. And the many skillful qualities fully developed because of right thought are good results. 

Wrong\marginnote{5.1} speech is a bad principle. Right speech is a good principle. And the many bad, unskillful qualities produced by wrong speech are bad results. And the many skillful qualities fully developed because of right speech are good results. 

Wrong\marginnote{6.1} action is a bad principle. Right action is a good principle. And the many bad, unskillful qualities produced by wrong action are bad results. And the many skillful qualities fully developed because of right action are good results. 

Wrong\marginnote{7.1} livelihood is a bad principle. Right livelihood is a good principle. And the many bad, unskillful qualities produced by wrong livelihood are bad results. And the many skillful qualities fully developed because of right livelihood are good results. 

Wrong\marginnote{8.1} effort is a bad principle. Right effort is a good principle. And the many bad, unskillful qualities produced by wrong effort are bad results. And the many skillful qualities fully developed because of right effort are good results. 

Wrong\marginnote{9.1} mindfulness is a bad principle. Right mindfulness is a good principle. And the many bad, unskillful qualities produced by wrong mindfulness are bad results. And the many skillful qualities fully developed because of right mindfulness are good results. 

Wrong\marginnote{10.1} immersion is a bad principle. Right immersion is a good principle. And the many bad, unskillful qualities produced by wrong immersion are bad results. And the many skillful qualities fully developed because of right immersion are good results. 

Wrong\marginnote{11.1} knowledge is a bad principle. Right knowledge is a good principle. And the many bad, unskillful qualities produced by wrong knowledge are bad results. And the many skillful qualities fully developed because of right knowledge are good results. 

Wrong\marginnote{12.1} freedom is a bad principle. Right freedom is a good principle. And the many bad, unskillful qualities produced by wrong freedom are bad results. And the many skillful qualities fully developed because of right freedom are good results. 

‘You\marginnote{13.1} should know bad principles and good principles. And you should know bad results and good results. Knowing these things, your practice should follow the good principles with good results.’ That’s what I said, and this is why I said it.” 

%
\section*{{\suttatitleacronym AN 10.115}{\suttatitletranslation Bad Principles (3rd) }{\suttatitleroot Tatiyaadhammasutta}}
\addcontentsline{toc}{section}{\tocacronym{AN 10.115} \toctranslation{Bad Principles (3rd) } \tocroot{Tatiyaadhammasutta}}
\markboth{Bad Principles (3rd) }{Tatiyaadhammasutta}
\extramarks{AN 10.115}{AN 10.115}

“Mendicants,\marginnote{1.1} you should know bad principles and good principles. And you should know bad results and good results. Knowing these things, your practice should follow the good principles with good results.” 

That\marginnote{1.4} is what the Buddha said. When he had spoken, the Holy One got up from his seat and entered his dwelling. 

Soon\marginnote{2.1} after the Buddha left, those mendicants considered, “The Buddha gave this brief summary recital, then entered his dwelling without explaining the meaning in detail. Who can explain in detail the meaning of this brief summary recital given by the Buddha?” 

Then\marginnote{3.1} they considered, “This Venerable Ānanda is praised by the Buddha and esteemed by his sensible spiritual companions. He is capable of explaining in detail the meaning of this brief summary recital given by the Buddha. Let’s go to him, and ask him about this matter. As he answers, so we’ll remember it.” 

Then\marginnote{4.1} those mendicants went to Ānanda, and exchanged greetings with him. When the greetings and polite conversation were over, they sat down to one side. They told him what had happened, and said, “May Venerable Ānanda please explain this.” 

“Reverends,\marginnote{8.1} suppose there was a person in need of heartwood. And while wandering in search of heartwood he’d come across a large tree standing with heartwood. But he’d pass over the roots and trunk, imagining that the heartwood should be sought in the branches and leaves. Such is the consequence for the venerables. Though you were face to face with the Buddha, you overlooked him, imagining that you should ask me about this matter. For he is the Buddha, the one who knows and sees. He is vision, he is knowledge, he is the manifestation of principle, he is the manifestation of divinity. He is the teacher, the proclaimer, the elucidator of meaning, the bestower of freedom from death, the lord of truth, the Realized One. That was the time to approach the Buddha and ask about this matter. You should have remembered it in line with the Buddha’s answer.” 

“Certainly\marginnote{9.1} he is the Buddha, the one who knows and sees. He is vision, he is knowledge, he is the manifestation of principle, he is the manifestation of divinity. He is the teacher, the proclaimer, the elucidator of meaning, the bestower of freedom from death, the lord of truth, the Realized One. That was the time to approach the Buddha and ask about this matter. We should have remembered it in line with the Buddha’s answer. Still, Venerable Ānanda is praised by the Buddha and esteemed by his sensible spiritual companions. You are capable of explaining in detail the meaning of this brief summary recital given by the Buddha. Please explain this, if it’s no trouble.” 

“Then\marginnote{10.1} listen and apply your mind well, I will speak.” 

“Yes,\marginnote{10.2} reverend,” they replied. Ānanda said this: 

“Reverends,\marginnote{11.1} the Buddha gave this brief summary recital, then entered his dwelling without explaining the meaning in detail: ‘You should know bad principles and good principles. And you should know bad results and good results. Knowing these things, your practice should follow the good principles with good results.’ 

So\marginnote{12.1} what are bad principles? What are good principles? What are bad results? And what are good results? 

Wrong\marginnote{13.1} view is a bad principle. Right view is a good principle. And the many bad, unskillful qualities produced by wrong view are bad results. And the many skillful qualities fully developed because of right view are good results. 

Wrong\marginnote{14.1} thought is a bad principle. Right thought is a good principle. … Wrong speech is a bad principle. Right speech is a good principle. … Wrong action is a bad principle. Right action is a good principle. … Wrong livelihood is a bad principle. Right livelihood is a good principle. … Wrong effort is a bad principle. Right effort is a good principle. … Wrong mindfulness is a bad principle. Right mindfulness is a good principle. … Wrong immersion is a bad principle. Right immersion is a good principle. … Wrong knowledge is a bad principle. Right knowledge is a good principle. … 

Wrong\marginnote{15.1} freedom is a bad principle. Right freedom is a good principle. And the many bad, unskillful qualities produced by wrong freedom are bad results. And the many skillful qualities fully developed because of right freedom are good results. 

The\marginnote{16.1} Buddha gave this brief summary recital, then entered his dwelling without explaining the meaning in detail: ‘You should know bad principles and good principles … and practice accordingly.’ And this is how I understand the detailed meaning of this summary recital. If you wish, you may go to the Buddha and ask him about this. You should remember it in line with the Buddha’s answer.” 

“Yes,\marginnote{17.1} reverend,” said those mendicants, approving and agreeing with what Ānanda said. Then they rose from their seats and went to the Buddha, bowed, sat down to one side, and told him what had happened. Then they said: 

“Sir,\marginnote{21.1} we went to Ānanda and asked him about this matter. And Ānanda clearly explained the meaning to us in this manner, with these words and phrases.” 

“Good,\marginnote{22.1} good, mendicants! Ānanda is astute, he has great wisdom. If you came to me and asked this question, I would answer it in exactly the same way as Ānanda. That is what it means, and that’s how you should remember it.” 

%
\section*{{\suttatitleacronym AN 10.116}{\suttatitletranslation With Ajita }{\suttatitleroot Ajitasutta}}
\addcontentsline{toc}{section}{\tocacronym{AN 10.116} \toctranslation{With Ajita } \tocroot{Ajitasutta}}
\markboth{With Ajita }{Ajitasutta}
\extramarks{AN 10.116}{AN 10.116}

Then\marginnote{1.1} the wanderer Ajita went up to the Buddha, and exchanged greetings with him. When the greetings and polite conversation were over, he sat down to one side and said to the Buddha, “Mister Gotama, we have a spiritual companion called ‘Astute’. He has worked out around five hundred arguments by which followers of other religions will know when they’ve been refuted.” 

Then\marginnote{3.1} the Buddha said to the mendicants, “Mendicants, do you remember this Astute’s points?” 

“Now\marginnote{3.3} is the time, Blessed One! Now is the time, Holy One! Let the Buddha speak and the mendicants will remember it.” 

“Well\marginnote{4.1} then, mendicants, listen and apply your mind well, I will speak.” 

“Yes,\marginnote{4.2} sir,” they replied. The Buddha said this: 

“Mendicants,\marginnote{5.1} take a certain person who rebuts and quashes unprincipled statements with unprincipled statements. This delights an unprincipled assembly, who make a dreadful racket: ‘He is truly astute! He is truly astute!’ 

Another\marginnote{6.1} person rebuts and quashes principled statements with unprincipled statements. This delights an unprincipled assembly, who make a dreadful racket: ‘He is truly astute! He is truly astute!’ 

Another\marginnote{7.1} person rebuts and quashes principled and unprincipled statements with unprincipled statements. This delights an unprincipled assembly, who make a dreadful racket: ‘He is truly astute! He is truly astute!’ 

Mendicants,\marginnote{8.1} you should know bad principles and good principles. And you should know bad results and good results. Knowing these things, your practice should follow the good principles with good results. 

So\marginnote{9.1} what are bad principles? What are good principles? What are bad results? And what are good results? Wrong view is a bad principle. Right view is a good principle. And the many bad, unskillful qualities produced by wrong view are bad results. And the many skillful qualities fully developed because of right view are good results. 

Wrong\marginnote{10.1} thought is a bad principle. Right thought is a good principle. … Wrong speech is a bad principle. Right speech is a good principle. … Wrong action is a bad principle. Right action is a good principle. … Wrong livelihood is a bad principle. Right livelihood is a good principle. … Wrong effort is a bad principle. Right effort is a good principle. … Wrong mindfulness is a bad principle. Right mindfulness is a good principle. … Wrong immersion is a bad principle. Right immersion is a good principle. … Wrong knowledge is a bad principle. Right knowledge is a good principle. … 

Wrong\marginnote{11.1} freedom is a bad principle. Right freedom is a good principle. And the many bad, unskillful qualities produced by wrong freedom are bad results. And the many skillful qualities fully developed because of right freedom are good results. 

‘You\marginnote{12.1} should know bad principles and good principles. And you should know bad results and good results. Knowing these things, your practice should follow the good principles with good results.’ That’s what I said, and this is why I said it.” 

%
\section*{{\suttatitleacronym AN 10.117}{\suttatitletranslation With Saṅgārava }{\suttatitleroot Saṅgāravasutta}}
\addcontentsline{toc}{section}{\tocacronym{AN 10.117} \toctranslation{With Saṅgārava } \tocroot{Saṅgāravasutta}}
\markboth{With Saṅgārava }{Saṅgāravasutta}
\extramarks{AN 10.117}{AN 10.117}

Then\marginnote{1.1} \textsanskrit{Saṅgārava} the brahmin went up to the Buddha, and exchanged greetings with him. When the greetings and polite conversation were over, he sat down to one side and said to the Buddha: 

“Mister\marginnote{1.3} Gotama, what is the near shore? And what is the far shore?” 

“Wrong\marginnote{1.4} view is the near shore, brahmin, and right view is the far shore. Wrong thought is the near shore, and right thought is the far shore. Wrong speech is the near shore, and right speech is the far shore. Wrong action is the near shore, and right action is the far shore. Wrong livelihood is the near shore, and right livelihood is the far shore. Wrong effort is the near shore, and right effort is the far shore. Wrong mindfulness is the near shore, and right mindfulness is the far shore. Wrong immersion is the near shore, and right immersion is the far shore. Wrong knowledge is the near shore, and right knowledge is the far shore. Wrong freedom is the near shore, and right freedom is the far shore. This is the near shore, and this is the far shore. 

\begin{verse}%
Few\marginnote{2.1} are those among humans \\
who cross to the far shore. \\
The rest just run around \\
on the near shore. 

When\marginnote{3.1} the teaching is well explained, \\
those who practice accordingly \\
are the ones who will cross over \\
Death’s dominion so hard to pass. 

Rid\marginnote{4.1} of dark qualities, \\
an astute person should develop the bright. \\
Leaving home behind \\
for the seclusion so hard to enjoy, 

find\marginnote{5.1} delight there, \\
having left behind sensual pleasures. \\
With no possessions, an astute person \\
would cleanse themselves of mental corruptions. 

Those\marginnote{6.1} whose minds are rightly developed \\
in the awakening factors; \\
who, letting go of attachments, \\
delight in not grasping: \\
with defilements ended, brilliant, \\
they are quenched in this world.” 

%
\end{verse}

%
\section*{{\suttatitleacronym AN 10.118}{\suttatitletranslation The Near Shore }{\suttatitleroot Orimatīrasutta}}
\addcontentsline{toc}{section}{\tocacronym{AN 10.118} \toctranslation{The Near Shore } \tocroot{Orimatīrasutta}}
\markboth{The Near Shore }{Orimatīrasutta}
\extramarks{AN 10.118}{AN 10.118}

“Mendicants,\marginnote{1.1} I will teach you the near shore and the far shore. Listen and apply your mind well, I will speak.” 

“Yes,\marginnote{1.3} sir,” they replied. The Buddha said this: 

“And\marginnote{2.1} what, mendicants, is the near shore? What is the far shore? Wrong view is the near shore, and right view is the far shore. … Wrong freedom is the near shore, and right freedom is the far shore. This is the near shore, and this is the far shore. 

\begin{verse}%
Few\marginnote{3.1} are those among humans \\
who cross to the far shore. \\
The rest just run \\
around on the near shore. 

When\marginnote{4.1} the teaching is well explained, \\
those who practice accordingly \\
are the ones who will cross over \\
Death’s dominion so hard to pass. 

Rid\marginnote{5.1} of dark qualities, \\
an astute person should develop the bright. \\
Leaving home behind \\
for the seclusion so hard to enjoy, 

you\marginnote{6.1} should try to find delight there, \\
having left behind sensual pleasures. \\
With no possessions, an astute person \\
should cleanse themselves of mental corruptions. 

And\marginnote{7.1} those whose minds are rightly developed \\
in the awakening factors; \\
letting go of attachments, \\
they delight in not grasping. \\
With defilements ended, brilliant, \\
they are quenched in this world.” 

%
\end{verse}

%
\section*{{\suttatitleacronym AN 10.119}{\suttatitletranslation The Ceremony of Descent (1st) }{\suttatitleroot Paṭhamapaccorohaṇīsutta}}
\addcontentsline{toc}{section}{\tocacronym{AN 10.119} \toctranslation{The Ceremony of Descent (1st) } \tocroot{Paṭhamapaccorohaṇīsutta}}
\markboth{The Ceremony of Descent (1st) }{Paṭhamapaccorohaṇīsutta}
\extramarks{AN 10.119}{AN 10.119}

Now,\marginnote{1.1} at that time it was the sabbath. The brahmin \textsanskrit{Jānussoṇi} had bathed his head and dressed in a new pair of linen robes. Holding a handful of fresh grass, he stood to one side not far from the Buddha. 

The\marginnote{2.1} Buddha saw him, and said, “Brahmin, why have you bathed your head and dressed in a new pair of linen robes? Why are you standing to one side holding a handful of fresh grass? What’s going on today with the brahmin clan?” 

“Mister\marginnote{2.5} Gotama, today is the ceremony of descent for the brahmin clan.” 

“But\marginnote{3.1} how do the brahmins observe the ceremony of descent?” 

“Well,\marginnote{3.2} Mister Gotama, on the sabbath the brahmins bathe their heads and dress in a new pair of linen robes. They make a heap of fresh cow dung and spread it with green grass. Then they make their beds between the boundary and the fire chamber. That night they rise three times and worship the fire with joined palms: ‘We descend, lord! We descend, lord!’ And they serve the fire with abundant ghee, oil, and butter. And when the night has passed they serve the brahmins with delicious fresh and cooked foods. That’s how the brahmins observe the ceremony of descent.” 

“The\marginnote{4.1} ceremony of descent observed by the brahmins is quite different from that observed in the training of the Noble One.” 

“But\marginnote{4.2} Mister Gotama, how is the ceremony of descent observed in the training of the Noble One? Mister Gotama, please teach me this.” 

“Well\marginnote{5.1} then, brahmin, listen and apply your mind well, I will speak.” 

“Yes\marginnote{5.2} sir,” \textsanskrit{Jānussoṇi} replied. The Buddha said this: 

“It’s\marginnote{6.1} when a noble disciple reflects: ‘Wrong view has a bad result in both this life and the next.’ Reflecting like this, they give up wrong view, they descend from wrong view. 

‘Wrong\marginnote{7.1} thought has a bad result in both this life and the next.’ Reflecting like this, they give up wrong thought, they descend from wrong thought. 

‘Wrong\marginnote{8.1} speech has a bad result in both this life and the next.’ Reflecting like this, they give up wrong speech, they descend from wrong speech. 

‘Wrong\marginnote{9.1} action has a bad result in both this life and the next.’ Reflecting like this, they give up wrong action, they descend from wrong action. 

‘Wrong\marginnote{10.1} livelihood has a bad result in both this life and the next.’ Reflecting like this, they give up wrong livelihood, they descend from wrong livelihood. 

‘Wrong\marginnote{11.1} effort has a bad result in both this life and the next.’ Reflecting like this, they give up wrong effort, they descend from wrong effort. 

‘Wrong\marginnote{12.1} mindfulness has a bad result in both this life and the next.’ Reflecting like this, they give up wrong mindfulness, they descend from wrong mindfulness. 

‘Wrong\marginnote{13.1} immersion has a bad result in both this life and the next.’ Reflecting like this, they give up wrong immersion, they descend from wrong immersion. 

‘Wrong\marginnote{14.1} knowledge has a bad result in both this life and the next.’ Reflecting like this, they give up wrong knowledge, they descend from wrong knowledge. 

‘Wrong\marginnote{15.1} freedom has a bad result in both this life and the next.’ Reflecting like this, they give up wrong freedom, they descend from wrong freedom. This is the ceremony of descent in the training of the Noble One.” 

“The\marginnote{16.1} ceremony of descent observed by the brahmins is quite different from that observed in the training of the Noble One. And, Mister Gotama, the ceremony of descent observed by the brahmins is not worth a sixteenth part of a master of the ceremony of descent observed in the training of the Noble One. Excellent, Mister Gotama! … From this day forth, may Mister Gotama remember me as a lay follower who has gone for refuge for life.” 

%
\section*{{\suttatitleacronym AN 10.120}{\suttatitletranslation The Ceremony of Descent (2nd) }{\suttatitleroot Dutiyapaccorohaṇīsutta}}
\addcontentsline{toc}{section}{\tocacronym{AN 10.120} \toctranslation{The Ceremony of Descent (2nd) } \tocroot{Dutiyapaccorohaṇīsutta}}
\markboth{The Ceremony of Descent (2nd) }{Dutiyapaccorohaṇīsutta}
\extramarks{AN 10.120}{AN 10.120}

“Mendicants,\marginnote{1.1} I will teach you the noble descent. Listen and apply your mind well, I will speak. … And what is the noble descent? It’s when a noble disciple reflects: ‘Wrong view has a bad result in both this life and the next.’ Reflecting like this, they give up wrong view, they descend from wrong view. ‘Wrong thought has a bad result …’ … ‘Wrong speech …’ … ‘Wrong action …’ … ‘Wrong livelihood …’ … ‘Wrong effort …’ … ‘Wrong mindfulness …’ … ‘Wrong immersion …’ … ‘Wrong knowledge …’ … ‘Wrong freedom has a bad result in both this life and the next.’ Reflecting like this, they give up wrong freedom, they descend from wrong freedom. This is called the noble descent.” 

%
\section*{{\suttatitleacronym AN 10.121}{\suttatitletranslation Forerunner }{\suttatitleroot Pubbaṅgamasutta}}
\addcontentsline{toc}{section}{\tocacronym{AN 10.121} \toctranslation{Forerunner } \tocroot{Pubbaṅgamasutta}}
\markboth{Forerunner }{Pubbaṅgamasutta}
\extramarks{AN 10.121}{AN 10.121}

“Mendicants,\marginnote{1.1} the dawn is the forerunner and precursor of the sunrise. In the same way right view is the forerunner and precursor of skillful qualities. Right view gives rise to right thought. Right thought gives rise to right speech. Right speech gives rise to right action. Right action gives rise to right livelihood. Right livelihood gives rise to right effort. Right effort gives rise to right mindfulness. Right mindfulness gives rise to right immersion. Right immersion gives rise to right knowledge. Right knowledge gives rise to right freedom.” 

%
\section*{{\suttatitleacronym AN 10.122}{\suttatitletranslation The Ending of Defilements }{\suttatitleroot Āsavakkhayasutta}}
\addcontentsline{toc}{section}{\tocacronym{AN 10.122} \toctranslation{The Ending of Defilements } \tocroot{Āsavakkhayasutta}}
\markboth{The Ending of Defilements }{Āsavakkhayasutta}
\extramarks{AN 10.122}{AN 10.122}

“Mendicants,\marginnote{1.1} these ten things, when developed and cultivated, lead to the ending of defilements. What ten? Right view, right thought, right speech, right action, right livelihood, right effort, right mindfulness, right immersion, right knowledge, and right freedom. These ten things, when developed and cultivated, lead to the ending of defilements.” 

%
\addtocontents{toc}{\let\protect\contentsline\protect\nopagecontentsline}
\chapter*{The Chapter on Purified }
\addcontentsline{toc}{chapter}{\tocchapterline{The Chapter on Purified }}
\addtocontents{toc}{\let\protect\contentsline\protect\oldcontentsline}

%
\section*{{\suttatitleacronym AN 10.123}{\suttatitletranslation First }{\suttatitleroot Paṭhamasutta}}
\addcontentsline{toc}{section}{\tocacronym{AN 10.123} \toctranslation{First } \tocroot{Paṭhamasutta}}
\markboth{First }{Paṭhamasutta}
\extramarks{AN 10.123}{AN 10.123}

“Mendicants,\marginnote{1.1} these ten things are not purified and cleansed apart from the Holy One’s training. What ten? Right view, right thought, right speech, right action, right livelihood, right effort, right mindfulness, right immersion, right knowledge, and right freedom. These ten things are not purified and cleansed apart from the Holy One’s training.” 

%
\section*{{\suttatitleacronym AN 10.124}{\suttatitletranslation Second }{\suttatitleroot Dutiyasutta}}
\addcontentsline{toc}{section}{\tocacronym{AN 10.124} \toctranslation{Second } \tocroot{Dutiyasutta}}
\markboth{Second }{Dutiyasutta}
\extramarks{AN 10.124}{AN 10.124}

“Mendicants,\marginnote{1.1} these ten things don’t arise apart from the Holy One’s training. What ten? Right view, right thought, right speech, right action, right livelihood, right effort, right mindfulness, right immersion, right knowledge, and right freedom. These are the ten things that don’t arise apart from the Holy One’s training.” 

%
\section*{{\suttatitleacronym AN 10.125}{\suttatitletranslation Third }{\suttatitleroot Tatiyasutta}}
\addcontentsline{toc}{section}{\tocacronym{AN 10.125} \toctranslation{Third } \tocroot{Tatiyasutta}}
\markboth{Third }{Tatiyasutta}
\extramarks{AN 10.125}{AN 10.125}

“Mendicants,\marginnote{1.1} these ten things are not very fruitful and beneficial apart from the Holy One’s training. What ten? Right view, right thought, right speech, right action, right livelihood, right effort, right mindfulness, right immersion, right knowledge, and right freedom. These are the ten things that are not very fruitful and beneficial apart from the Holy One’s training.” 

%
\section*{{\suttatitleacronym AN 10.126}{\suttatitletranslation Fourth }{\suttatitleroot Catutthasutta}}
\addcontentsline{toc}{section}{\tocacronym{AN 10.126} \toctranslation{Fourth } \tocroot{Catutthasutta}}
\markboth{Fourth }{Catutthasutta}
\extramarks{AN 10.126}{AN 10.126}

“Mendicants,\marginnote{1.1} these ten things don’t culminate in the removal of greed, hate, and delusion apart from the Holy One’s training. What ten? Right view, right thought, right speech, right action, right livelihood, right effort, right mindfulness, right immersion, right knowledge, and right freedom. These are the ten things that don’t culminate in the removal of greed, hate, and delusion apart from the Holy One’s training.” 

%
\section*{{\suttatitleacronym AN 10.127}{\suttatitletranslation Fifth }{\suttatitleroot Pañcamasutta}}
\addcontentsline{toc}{section}{\tocacronym{AN 10.127} \toctranslation{Fifth } \tocroot{Pañcamasutta}}
\markboth{Fifth }{Pañcamasutta}
\extramarks{AN 10.127}{AN 10.127}

“Mendicants,\marginnote{1.1} these ten things don’t lead solely to disillusionment, dispassion, cessation, peace, insight, awakening, and extinguishment apart from the Holy One’s training. What ten? Right view, right thought, right speech, right action, right livelihood, right effort, right mindfulness, right immersion, right knowledge, and right freedom. These are the ten things that don’t lead solely to disillusionment, dispassion, cessation, peace, insight, awakening, and extinguishment apart from the Holy One’s training.” 

%
\section*{{\suttatitleacronym AN 10.128}{\suttatitletranslation Sixth }{\suttatitleroot Chaṭṭhasutta}}
\addcontentsline{toc}{section}{\tocacronym{AN 10.128} \toctranslation{Sixth } \tocroot{Chaṭṭhasutta}}
\markboth{Sixth }{Chaṭṭhasutta}
\extramarks{AN 10.128}{AN 10.128}

“Mendicants,\marginnote{1.1} these ten things don’t arise to be developed and cultivated apart from the Holy One’s training. What ten? Right view, right thought, right speech, right action, right livelihood, right effort, right mindfulness, right immersion, right knowledge, and right freedom. These are the ten things that don’t arise to be developed and cultivated apart from the Holy One’s training.” 

%
\section*{{\suttatitleacronym AN 10.129}{\suttatitletranslation Seventh }{\suttatitleroot Sattamasutta}}
\addcontentsline{toc}{section}{\tocacronym{AN 10.129} \toctranslation{Seventh } \tocroot{Sattamasutta}}
\markboth{Seventh }{Sattamasutta}
\extramarks{AN 10.129}{AN 10.129}

“Mendicants,\marginnote{1.1} these ten things when developed and cultivated are not very fruitful and beneficial apart from the Holy One’s training. What ten? Right view, right thought, right speech, right action, right livelihood, right effort, right mindfulness, right immersion, right knowledge, and right freedom. These are the ten things that when developed and cultivated are not very fruitful and beneficial apart from the Holy One’s training.” 

%
\section*{{\suttatitleacronym AN 10.130}{\suttatitletranslation Eighth }{\suttatitleroot Aṭṭhamasutta}}
\addcontentsline{toc}{section}{\tocacronym{AN 10.130} \toctranslation{Eighth } \tocroot{Aṭṭhamasutta}}
\markboth{Eighth }{Aṭṭhamasutta}
\extramarks{AN 10.130}{AN 10.130}

“Mendicants,\marginnote{1.1} these ten things when developed and cultivated don’t culminate in the removal of greed, hate, and delusion apart from the Holy One’s training. What ten? Right view, right thought, right speech, right action, right livelihood, right effort, right mindfulness, right immersion, right knowledge, and right freedom. These are the ten things that when developed and cultivated don’t culminate in the removal of greed, hate, and delusion apart from the Holy One’s training.” 

%
\section*{{\suttatitleacronym AN 10.131}{\suttatitletranslation Ninth }{\suttatitleroot Navamasutta}}
\addcontentsline{toc}{section}{\tocacronym{AN 10.131} \toctranslation{Ninth } \tocroot{Navamasutta}}
\markboth{Ninth }{Navamasutta}
\extramarks{AN 10.131}{AN 10.131}

“Mendicants,\marginnote{1.1} these ten things when developed and cultivated don’t lead solely to disillusionment, dispassion, cessation, peace, insight, awakening, and extinguishment apart from the Holy One’s training. What ten? Right view, right thought, right speech, right action, right livelihood, right effort, right mindfulness, right immersion, right knowledge, and right freedom. These are the ten things that when developed and cultivated don’t lead solely to disillusionment, dispassion, cessation, peace, insight, awakening, and extinguishment apart from the Holy One’s training.” 

%
\section*{{\suttatitleacronym AN 10.132}{\suttatitletranslation Tenth }{\suttatitleroot Dasamasutta}}
\addcontentsline{toc}{section}{\tocacronym{AN 10.132} \toctranslation{Tenth } \tocroot{Dasamasutta}}
\markboth{Tenth }{Dasamasutta}
\extramarks{AN 10.132}{AN 10.132}

“Mendicants,\marginnote{1.1} there are ten wrong ways. What ten? Wrong view, wrong thought, wrong speech, wrong action, wrong livelihood, wrong effort, wrong mindfulness, wrong immersion, wrong knowledge, and wrong freedom. These are the ten wrong ways.” 

%
\section*{{\suttatitleacronym AN 10.133}{\suttatitletranslation Eleventh }{\suttatitleroot Ekādasamasutta}}
\addcontentsline{toc}{section}{\tocacronym{AN 10.133} \toctranslation{Eleventh } \tocroot{Ekādasamasutta}}
\markboth{Eleventh }{Ekādasamasutta}
\extramarks{AN 10.133}{AN 10.133}

“Mendicants,\marginnote{1.1} there are ten right ways. What ten? Right view, right thought, right speech, right action, right livelihood, right effort, right mindfulness, right immersion, right knowledge, and right freedom. These are the ten right ways.” 

%
\addtocontents{toc}{\let\protect\contentsline\protect\nopagecontentsline}
\chapter*{The Chapter on Good }
\addcontentsline{toc}{chapter}{\tocchapterline{The Chapter on Good }}
\addtocontents{toc}{\let\protect\contentsline\protect\oldcontentsline}

%
\section*{{\suttatitleacronym AN 10.134}{\suttatitletranslation Good }{\suttatitleroot Sādhusutta}}
\addcontentsline{toc}{section}{\tocacronym{AN 10.134} \toctranslation{Good } \tocroot{Sādhusutta}}
\markboth{Good }{Sādhusutta}
\extramarks{AN 10.134}{AN 10.134}

“Mendicants,\marginnote{1.1} I will teach you what is good and what is not good. Listen and apply your mind well, I will speak.” 

“Yes,\marginnote{1.3} sir,” they replied. The Buddha said this: 

“And\marginnote{2.1} what, mendicants, is not good? Wrong view, wrong thought, wrong speech, wrong action, wrong livelihood, wrong effort, wrong mindfulness, wrong immersion, wrong knowledge, and wrong freedom. This is called what is not good. And what is good? Right view, right thought, right speech, right action, right livelihood, right effort, right mindfulness, right immersion, right knowledge, and right freedom. This is called what is good.” 

%
\section*{{\suttatitleacronym AN 10.135}{\suttatitletranslation The Teaching of the Noble Ones }{\suttatitleroot Ariyadhammasutta}}
\addcontentsline{toc}{section}{\tocacronym{AN 10.135} \toctranslation{The Teaching of the Noble Ones } \tocroot{Ariyadhammasutta}}
\markboth{The Teaching of the Noble Ones }{Ariyadhammasutta}
\extramarks{AN 10.135}{AN 10.135}

“Mendicants,\marginnote{1.1} I will teach you the teaching of the noble ones, and what is not the teaching of the noble ones. … And what is not the teaching of the noble ones? Wrong view, wrong thought, wrong speech, wrong action, wrong livelihood, wrong effort, wrong mindfulness, wrong immersion, wrong knowledge, and wrong freedom. This is called what is not the teaching of the noble ones. And what is the teaching of the noble ones? Right view, right thought, right speech, right action, right livelihood, right effort, right mindfulness, right immersion, right knowledge, and right freedom. This is called the teaching of the noble ones.” 

%
\section*{{\suttatitleacronym AN 10.136}{\suttatitletranslation Unskillful }{\suttatitleroot Akusalasutta}}
\addcontentsline{toc}{section}{\tocacronym{AN 10.136} \toctranslation{Unskillful } \tocroot{Akusalasutta}}
\markboth{Unskillful }{Akusalasutta}
\extramarks{AN 10.136}{AN 10.136}

“I\marginnote{1.1} will teach you the skillful and the unskillful … And what is the unskillful? Wrong view, wrong thought, wrong speech, wrong action, wrong livelihood, wrong effort, wrong mindfulness, wrong immersion, wrong knowledge, and wrong freedom. This is called the unskillful. And what is the skillful? Right view, right thought, right speech, right action, right livelihood, right effort, right mindfulness, right immersion, right knowledge, and right freedom. This is called the skillful.” 

%
\section*{{\suttatitleacronym AN 10.137}{\suttatitletranslation Beneficial }{\suttatitleroot Atthasutta}}
\addcontentsline{toc}{section}{\tocacronym{AN 10.137} \toctranslation{Beneficial } \tocroot{Atthasutta}}
\markboth{Beneficial }{Atthasutta}
\extramarks{AN 10.137}{AN 10.137}

“I\marginnote{1.1} will teach you the beneficial and the harmful. … And what is the harmful? Wrong view, wrong thought, wrong speech, wrong action, wrong livelihood, wrong effort, wrong mindfulness, wrong immersion, wrong knowledge, and wrong freedom. This is called the harmful. And what is the beneficial? Right view, right thought, right speech, right action, right livelihood, right effort, right mindfulness, right immersion, right knowledge, and right freedom. This is called the beneficial.” 

%
\section*{{\suttatitleacronym AN 10.138}{\suttatitletranslation The Teaching }{\suttatitleroot Dhammasutta}}
\addcontentsline{toc}{section}{\tocacronym{AN 10.138} \toctranslation{The Teaching } \tocroot{Dhammasutta}}
\markboth{The Teaching }{Dhammasutta}
\extramarks{AN 10.138}{AN 10.138}

“I\marginnote{1.1} will teach you what is the teaching and what is not the teaching. … And what is not the teaching? Wrong view, wrong thought, wrong speech, wrong action, wrong livelihood, wrong effort, wrong mindfulness, wrong immersion, wrong knowledge, and wrong freedom. This is called what is not the teaching. And what is the teaching? Right view, right thought, right speech, right action, right livelihood, right effort, right mindfulness, right immersion, right knowledge, and right freedom. This is called the teaching.” 

%
\section*{{\suttatitleacronym AN 10.139}{\suttatitletranslation Defiled }{\suttatitleroot Sāsavasutta}}
\addcontentsline{toc}{section}{\tocacronym{AN 10.139} \toctranslation{Defiled } \tocroot{Sāsavasutta}}
\markboth{Defiled }{Sāsavasutta}
\extramarks{AN 10.139}{AN 10.139}

“I\marginnote{1.1} will teach you the defiled principle and the undefiled. … And what is the defiled principle? Wrong view, wrong thought, wrong speech, wrong action, wrong livelihood, wrong effort, wrong mindfulness, wrong immersion, wrong knowledge, and wrong freedom. This is called the defiled principle. And what is the undefiled principle? Right view, right thought, right speech, right action, right livelihood, right effort, right mindfulness, right immersion, right knowledge, and right freedom. This is called the undefiled principle.” 

%
\section*{{\suttatitleacronym AN 10.140}{\suttatitletranslation Blameworthy }{\suttatitleroot Sāvajjasutta}}
\addcontentsline{toc}{section}{\tocacronym{AN 10.140} \toctranslation{Blameworthy } \tocroot{Sāvajjasutta}}
\markboth{Blameworthy }{Sāvajjasutta}
\extramarks{AN 10.140}{AN 10.140}

“I\marginnote{1.1} will teach you the blameworthy principle and the blameless principle. … And what is the blameworthy principle? Wrong view, wrong thought, wrong speech, wrong action, wrong livelihood, wrong effort, wrong mindfulness, wrong immersion, wrong knowledge, and wrong freedom. This is called the blameworthy principle. And what is the blameless principle? Right view, right thought, right speech, right action, right livelihood, right effort, right mindfulness, right immersion, right knowledge, and right freedom. This is called the blameless principle.” 

%
\section*{{\suttatitleacronym AN 10.141}{\suttatitletranslation Mortifying }{\suttatitleroot Tapanīyasutta}}
\addcontentsline{toc}{section}{\tocacronym{AN 10.141} \toctranslation{Mortifying } \tocroot{Tapanīyasutta}}
\markboth{Mortifying }{Tapanīyasutta}
\extramarks{AN 10.141}{AN 10.141}

“I\marginnote{1.1} will teach you the mortifying principle and the unmortifying. … And what is the mortifying principle? Wrong view, wrong thought, wrong speech, wrong action, wrong livelihood, wrong effort, wrong mindfulness, wrong immersion, wrong knowledge, and wrong freedom. This is called the mortifying principle. And what is the unmortifying principle? Right view, right thought, right speech, right action, right livelihood, right effort, right mindfulness, right immersion, right knowledge, and right freedom. This is called the unmortifying principle.” 

%
\section*{{\suttatitleacronym AN 10.142}{\suttatitletranslation Accumulation }{\suttatitleroot Ācayagāmisutta}}
\addcontentsline{toc}{section}{\tocacronym{AN 10.142} \toctranslation{Accumulation } \tocroot{Ācayagāmisutta}}
\markboth{Accumulation }{Ācayagāmisutta}
\extramarks{AN 10.142}{AN 10.142}

“I\marginnote{1.1} will teach you the principle that leads to accumulation and that which leads to dispersal. … And what is the principle that leads to accumulation? Wrong view, wrong thought, wrong speech, wrong action, wrong livelihood, wrong effort, wrong mindfulness, wrong immersion, wrong knowledge, and wrong freedom. This is called the principle that leads to accumulation. And what is the principle that leads to dispersal? Right view, right thought, right speech, right action, right livelihood, right effort, right mindfulness, right immersion, right knowledge, and right freedom. This is called the principle that leads to dispersal.” 

%
\section*{{\suttatitleacronym AN 10.143}{\suttatitletranslation With Suffering as Outcome }{\suttatitleroot Dukkhudrayasutta}}
\addcontentsline{toc}{section}{\tocacronym{AN 10.143} \toctranslation{With Suffering as Outcome } \tocroot{Dukkhudrayasutta}}
\markboth{With Suffering as Outcome }{Dukkhudrayasutta}
\extramarks{AN 10.143}{AN 10.143}

“I\marginnote{1.1} will teach you the principle that has suffering as outcome, and that which has happiness as outcome. … And what is the principle whose outcome is suffering? Wrong view, wrong thought, wrong speech, wrong action, wrong livelihood, wrong effort, wrong mindfulness, wrong immersion, wrong knowledge, and wrong freedom. This is the principle whose outcome is suffering. And what is the principle whose outcome is happiness? Right view, right thought, right speech, right action, right livelihood, right effort, right mindfulness, right immersion, right knowledge, and right freedom. This is the principle whose outcome is happiness.” 

%
\section*{{\suttatitleacronym AN 10.144}{\suttatitletranslation Result in Suffering }{\suttatitleroot Dukkhavipākasutta}}
\addcontentsline{toc}{section}{\tocacronym{AN 10.144} \toctranslation{Result in Suffering } \tocroot{Dukkhavipākasutta}}
\markboth{Result in Suffering }{Dukkhavipākasutta}
\extramarks{AN 10.144}{AN 10.144}

“I\marginnote{1.1} will teach you the principle that results in suffering and that which results in happiness. … And what principle results in suffering? Wrong view, wrong thought, wrong speech, wrong action, wrong livelihood, wrong effort, wrong mindfulness, wrong immersion, wrong knowledge, and wrong freedom. This is called the principle that results in suffering. And what principle results in happiness? Right view, right thought, right speech, right action, right livelihood, right effort, right mindfulness, right immersion, right knowledge, and right freedom. This is called the principle that results in happiness.” 

%
\addtocontents{toc}{\let\protect\contentsline\protect\nopagecontentsline}
\chapter*{The Chapter on the Noble Path }
\addcontentsline{toc}{chapter}{\tocchapterline{The Chapter on the Noble Path }}
\addtocontents{toc}{\let\protect\contentsline\protect\oldcontentsline}

%
\section*{{\suttatitleacronym AN 10.145}{\suttatitletranslation The Noble Path }{\suttatitleroot Ariyamaggasutta}}
\addcontentsline{toc}{section}{\tocacronym{AN 10.145} \toctranslation{The Noble Path } \tocroot{Ariyamaggasutta}}
\markboth{The Noble Path }{Ariyamaggasutta}
\extramarks{AN 10.145}{AN 10.145}

“I\marginnote{1.1} will teach you the noble path and the ignoble path. … And what is the ignoble path? Wrong view, wrong thought, wrong speech, wrong action, wrong livelihood, wrong effort, wrong mindfulness, wrong immersion, wrong knowledge, and wrong freedom. This is called the ignoble path. And what is the noble path? Right view, right thought, right speech, right action, right livelihood, right effort, right mindfulness, right immersion, right knowledge, and right freedom. This is called the noble path.” 

%
\section*{{\suttatitleacronym AN 10.146}{\suttatitletranslation The Dark Path }{\suttatitleroot Kaṇhamaggasutta}}
\addcontentsline{toc}{section}{\tocacronym{AN 10.146} \toctranslation{The Dark Path } \tocroot{Kaṇhamaggasutta}}
\markboth{The Dark Path }{Kaṇhamaggasutta}
\extramarks{AN 10.146}{AN 10.146}

“I\marginnote{1.1} will teach you the dark path and the bright path. … And what is the dark path? Wrong view, wrong thought, wrong speech, wrong action, wrong livelihood, wrong effort, wrong mindfulness, wrong immersion, wrong knowledge, and wrong freedom. This is called the dark path. And what is the bright path? Right view, right thought, right speech, right action, right livelihood, right effort, right mindfulness, right immersion, right knowledge, and right freedom. This is called the bright path.” 

%
\section*{{\suttatitleacronym AN 10.147}{\suttatitletranslation The True Teaching }{\suttatitleroot Saddhammasutta}}
\addcontentsline{toc}{section}{\tocacronym{AN 10.147} \toctranslation{The True Teaching } \tocroot{Saddhammasutta}}
\markboth{The True Teaching }{Saddhammasutta}
\extramarks{AN 10.147}{AN 10.147}

“I\marginnote{1.1} will teach you what is the true teaching and what is not the true teaching. … And what is not the true teaching? Wrong view, wrong thought, wrong speech, wrong action, wrong livelihood, wrong effort, wrong mindfulness, wrong immersion, wrong knowledge, and wrong freedom. This is called what is not the true teaching. And what is the true teaching? Right view, right thought, right speech, right action, right livelihood, right effort, right mindfulness, right immersion, right knowledge, and right freedom. This is called the true teaching.” 

%
\section*{{\suttatitleacronym AN 10.148}{\suttatitletranslation The Teaching of the True Persons }{\suttatitleroot Sappurisadhammasutta}}
\addcontentsline{toc}{section}{\tocacronym{AN 10.148} \toctranslation{The Teaching of the True Persons } \tocroot{Sappurisadhammasutta}}
\markboth{The Teaching of the True Persons }{Sappurisadhammasutta}
\extramarks{AN 10.148}{AN 10.148}

“Mendicants,\marginnote{1.1} I will teach you the teaching of the true persons and the teaching of the untrue persons. … And what is the teaching of the untrue persons? Wrong view, wrong thought, wrong speech, wrong action, wrong livelihood, wrong effort, wrong mindfulness, wrong immersion, wrong knowledge, and wrong freedom. This is the teaching of the untrue persons. And what is the teaching of the true persons? Right view, right thought, right speech, right action, right livelihood, right effort, right mindfulness, right immersion, right knowledge, and right freedom. This is the teaching of the true persons.” 

%
\section*{{\suttatitleacronym AN 10.149}{\suttatitletranslation Should Be Activated }{\suttatitleroot Uppādetabbasutta}}
\addcontentsline{toc}{section}{\tocacronym{AN 10.149} \toctranslation{Should Be Activated } \tocroot{Uppādetabbasutta}}
\markboth{Should Be Activated }{Uppādetabbasutta}
\extramarks{AN 10.149}{AN 10.149}

“I\marginnote{1.1} will teach you the principle to activate and the principle not to activate. … And what is the principle not to activate? Wrong view, wrong thought, wrong speech, wrong action, wrong livelihood, wrong effort, wrong mindfulness, wrong immersion, wrong knowledge, and wrong freedom. This is called the principle not to activate. And what is the principle to activate? Right view, right thought, right speech, right action, right livelihood, right effort, right mindfulness, right immersion, right knowledge, and right freedom. This is called the principle to activate.” 

%
\section*{{\suttatitleacronym AN 10.150}{\suttatitletranslation Should Be Cultivated }{\suttatitleroot Āsevitabbasutta}}
\addcontentsline{toc}{section}{\tocacronym{AN 10.150} \toctranslation{Should Be Cultivated } \tocroot{Āsevitabbasutta}}
\markboth{Should Be Cultivated }{Āsevitabbasutta}
\extramarks{AN 10.150}{AN 10.150}

“I\marginnote{1.1} will teach you the principle to cultivate and the principle not to cultivate. … And what is the principle not to cultivate? Wrong view, wrong thought, wrong speech, wrong action, wrong livelihood, wrong effort, wrong mindfulness, wrong immersion, wrong knowledge, and wrong freedom. This is called the principle not to cultivate. And what is the principle to cultivate? Right view, right thought, right speech, right action, right livelihood, right effort, right mindfulness, right immersion, right knowledge, and right freedom. This is called the principle to cultivate.” 

%
\section*{{\suttatitleacronym AN 10.151}{\suttatitletranslation Should Be Developed }{\suttatitleroot Bhāvetabbasutta}}
\addcontentsline{toc}{section}{\tocacronym{AN 10.151} \toctranslation{Should Be Developed } \tocroot{Bhāvetabbasutta}}
\markboth{Should Be Developed }{Bhāvetabbasutta}
\extramarks{AN 10.151}{AN 10.151}

“I\marginnote{1.1} will teach you the principle to develop and the principle not to develop. … And what is the principle not to develop? Wrong view, wrong thought, wrong speech, wrong action, wrong livelihood, wrong effort, wrong mindfulness, wrong immersion, wrong knowledge, and wrong freedom. This is called the principle not to develop. And what is the principle to develop? Right view, right thought, right speech, right action, right livelihood, right effort, right mindfulness, right immersion, right knowledge, and right freedom. This is called the principle to develop.” 

%
\section*{{\suttatitleacronym AN 10.152}{\suttatitletranslation Should Be Made Much Of }{\suttatitleroot Bahulīkātabbasutta}}
\addcontentsline{toc}{section}{\tocacronym{AN 10.152} \toctranslation{Should Be Made Much Of } \tocroot{Bahulīkātabbasutta}}
\markboth{Should Be Made Much Of }{Bahulīkātabbasutta}
\extramarks{AN 10.152}{AN 10.152}

“I\marginnote{1.1} will teach you the principle to make much of and the principle not to make much of. … And what is the principle not to make much of? Wrong view, wrong thought, wrong speech, wrong action, wrong livelihood, wrong effort, wrong mindfulness, wrong immersion, wrong knowledge, and wrong freedom. This is called the principle not to make much of. And what is the principle to make much of? Right view, right thought, right speech, right action, right livelihood, right effort, right mindfulness, right immersion, right knowledge, and right freedom. This is called the principle to make much of.” 

%
\section*{{\suttatitleacronym AN 10.153}{\suttatitletranslation Should Be Recollected }{\suttatitleroot Anussaritabbasutta}}
\addcontentsline{toc}{section}{\tocacronym{AN 10.153} \toctranslation{Should Be Recollected } \tocroot{Anussaritabbasutta}}
\markboth{Should Be Recollected }{Anussaritabbasutta}
\extramarks{AN 10.153}{AN 10.153}

“I\marginnote{1.1} will teach you the principle to recollect and the principle not to recollect. … And what is the principle not to recollect? Wrong view, wrong thought, wrong speech, wrong action, wrong livelihood, wrong effort, wrong mindfulness, wrong immersion, wrong knowledge, and wrong freedom. This is called the principle not to recollect. And what is the principle to recollect? Right view, right thought, right speech, right action, right livelihood, right effort, right mindfulness, right immersion, right knowledge, and right freedom. This is called the principle to recollect.” 

%
\section*{{\suttatitleacronym AN 10.154}{\suttatitletranslation Should Be Realized }{\suttatitleroot Sacchikātabbasutta}}
\addcontentsline{toc}{section}{\tocacronym{AN 10.154} \toctranslation{Should Be Realized } \tocroot{Sacchikātabbasutta}}
\markboth{Should Be Realized }{Sacchikātabbasutta}
\extramarks{AN 10.154}{AN 10.154}

“I\marginnote{1.1} will teach you the principle to realize and the principle not to realize. … And what is the principle not to realize? Wrong view, wrong thought, wrong speech, wrong action, wrong livelihood, wrong effort, wrong mindfulness, wrong immersion, wrong knowledge, and wrong freedom. This is called the principle not to realize. And what is the principle to realize? Right view, right thought, right speech, right action, right livelihood, right effort, right mindfulness, right immersion, right knowledge, and right freedom. This is called the principle to realize.” 

%
\addtocontents{toc}{\let\protect\contentsline\protect\nopagecontentsline}
\pannasa{The Fourth Fifty }
\addcontentsline{toc}{pannasa}{The Fourth Fifty }
\markboth{}{}
\addtocontents{toc}{\let\protect\contentsline\protect\oldcontentsline}

%
\addtocontents{toc}{\let\protect\contentsline\protect\nopagecontentsline}
\chapter*{The Chapter on Persons }
\addcontentsline{toc}{chapter}{\tocchapterline{The Chapter on Persons }}
\addtocontents{toc}{\let\protect\contentsline\protect\oldcontentsline}

%
\section*{{\suttatitleacronym AN 10.155}{\suttatitletranslation You Should Associate }{\suttatitleroot Sevitabbasutta}}
\addcontentsline{toc}{section}{\tocacronym{AN 10.155} \toctranslation{You Should Associate } \tocroot{Sevitabbasutta}}
\markboth{You Should Associate }{Sevitabbasutta}
\extramarks{AN 10.155}{AN 10.155}

“Mendicants,\marginnote{1.1} you should not associate with a person who has ten qualities. What ten? Wrong view, wrong thought, wrong speech, wrong action, wrong livelihood, wrong effort, wrong mindfulness, wrong immersion, wrong knowledge, and wrong freedom. You should not associate with a person who has these ten qualities. 

You\marginnote{2.1} should associate with a person who has ten qualities. What ten? Right view, right thought, right speech, right action, right livelihood, right effort, right mindfulness, right immersion, right knowledge, and right freedom. You should associate with a person who has these ten qualities.” 

%
\section*{{\suttatitleacronym AN 10.156–166}{\suttatitletranslation Frequenting, Etc. }{\suttatitleroot Bhajitabbādisutta}}
\addcontentsline{toc}{section}{\tocacronym{AN 10.156–166} \toctranslation{Frequenting, Etc. } \tocroot{Bhajitabbādisutta}}
\markboth{Frequenting, Etc. }{Bhajitabbādisutta}
\extramarks{AN 10.156–166}{AN 10.156–166}

“Mendicants,\marginnote{1.1} you should not frequent a person who has ten qualities. … you should frequent … you should not pay homage … you should pay homage … you should not venerate … you should venerate … you should not praise … you should praise … you should not respect … you should respect … you should not revere … you should revere … is not a success … is a success … is not pure … is pure … does not win over conceit … wins over conceit … does not grow in wisdom … grows in wisdom … 

creates\marginnote{2.1} much wickedness … creates much merit. What ten? Right view, right thought, right speech, right action, right livelihood, right effort, right mindfulness, right immersion, right knowledge, and right freedom. A person who has these ten qualities creates much merit.” 

%
\addtocontents{toc}{\let\protect\contentsline\protect\nopagecontentsline}
\chapter*{The Chapter with Jānussoṇi }
\addcontentsline{toc}{chapter}{\tocchapterline{The Chapter with Jānussoṇi }}
\addtocontents{toc}{\let\protect\contentsline\protect\oldcontentsline}

%
\section*{{\suttatitleacronym AN 10.167}{\suttatitletranslation The Brahmin Ceremony of Descent }{\suttatitleroot Brāhmaṇapaccorohaṇīsutta}}
\addcontentsline{toc}{section}{\tocacronym{AN 10.167} \toctranslation{The Brahmin Ceremony of Descent } \tocroot{Brāhmaṇapaccorohaṇīsutta}}
\markboth{The Brahmin Ceremony of Descent }{Brāhmaṇapaccorohaṇīsutta}
\extramarks{AN 10.167}{AN 10.167}

Now,\marginnote{1.1} at that time it was the sabbath. The brahmin \textsanskrit{Jānussoṇi} had bathed his head and dressed in a new pair of linen robes. Holding a handful of fresh grass, he stood to one side not far from the Buddha. 

The\marginnote{2.1} Buddha saw him, and said, “Brahmin, why have you bathed your head and dressed in a new pair of linen robes? Why are you standing to one side holding a handful of fresh grass? What’s going on today with the brahmin clan?” 

“Mister\marginnote{2.5} Gotama, today is the ceremony of descent for the brahmin clan.” 

“But\marginnote{3.1} how do the brahmins observe the ceremony of descent?” 

“Well,\marginnote{3.2} Mister Gotama, on the sabbath the brahmins bathe their heads and dress in a new pair of linen robes. They make a heap of fresh cow dung and spread it with green grass. Then they make their beds between the boundary and the fire chamber. That night they rise three times and worship the fire with joined palms: ‘We descend, lord! We descend, lord!’ And they serve the fire with abundant ghee, oil, and butter. And when the night has passed they serve the brahmins with delicious fresh and cooked foods. That’s how the brahmins observe the ceremony of descent.” 

“The\marginnote{4.1} ceremony of descent observed by the brahmins is quite different from that observed in the training of the Noble One.” 

“But\marginnote{4.2} Mister Gotama, how is the ceremony of descent observed in the training of the Noble One? Mister Gotama, please teach me this.” 

“Well\marginnote{5.1} then, brahmin, listen and apply your mind well, I will speak.” 

“Yes\marginnote{5.2} sir,” \textsanskrit{Jānussoṇi} replied. The Buddha said this: 

“It’s\marginnote{6.1} when a noble disciple reflects: ‘Killing living creatures has a bad result in the present life and in lives to come.’ Reflecting like this, they give up killing living creatures, they descend from killing living creatures. 

…\marginnote{7.1} ‘Stealing has a bad result in the present life and in lives to come.’ Reflecting like this, they give up stealing, they descend from stealing. 

…\marginnote{8.1} ‘Sexual misconduct has a bad result in the present life and in lives to come.’ Reflecting like this, they give up sexual misconduct, they descend from sexual misconduct. 

…\marginnote{9.1} ‘Lying has a bad result in the present life and in lives to come.’ Reflecting like this, they give up lying, they descend from lying. 

…\marginnote{10.1} ‘Divisive speech has a bad result in the present life and in lives to come.’ Reflecting like this, they give up divisive speech, they descend from divisive speech. 

…\marginnote{11.1} ‘Harsh speech has a bad result in the present life and in lives to come.’ Reflecting like this, they give up harsh speech, they descend from harsh speech. 

…\marginnote{12.1} ‘Talking nonsense has a bad result in the present life and in lives to come.’ Reflecting like this, they give up talking nonsense, they descend from talking nonsense. 

…\marginnote{13.1} ‘Covetousness has a bad result in the present life and in lives to come.’ Reflecting like this, they give up covetousness, they descend from covetousness. 

…\marginnote{14.1} ‘Ill will has a bad result in the present life and in lives to come.’ Reflecting like this, they give up ill will, they descend from ill will. 

‘Wrong\marginnote{15.1} view has a bad result in the present life and in lives to come.’ Reflecting like this, they give up wrong view, they descend from wrong view. This is the ceremony of descent in the training of the Noble One.” 

“The\marginnote{16.1} ceremony of descent observed by the brahmins is quite different from that observed in the training of the Noble One. And, Mister Gotama, the ceremony of descent observed by the brahmins is not worth a sixteenth part of the ceremony of descent observed in the training of the Noble One. Excellent, Mister Gotama, excellent! … From this day forth, may Mister Gotama remember me as a lay follower who has gone for refuge for life.” 

%
\section*{{\suttatitleacronym AN 10.168}{\suttatitletranslation The Noble Descent }{\suttatitleroot Ariyapaccorohaṇīsutta}}
\addcontentsline{toc}{section}{\tocacronym{AN 10.168} \toctranslation{The Noble Descent } \tocroot{Ariyapaccorohaṇīsutta}}
\markboth{The Noble Descent }{Ariyapaccorohaṇīsutta}
\extramarks{AN 10.168}{AN 10.168}

“Mendicants,\marginnote{1.1} I will teach you the noble descent. Listen and apply your mind well, I will speak.” 

“Yes,\marginnote{1.3} sir,” they replied. The Buddha said this: 

“And\marginnote{2.1} what, mendicants, is the noble descent? It’s when a noble disciple reflects: ‘Killing living creatures has a bad result in the present life and in lives to come.’ Reflecting like this, they give up killing living creatures, they descend from killing living creatures. 

…\marginnote{3.1} ‘Stealing has a bad result in the present life and in lives to come.’ Reflecting like this, they give up stealing, they descend from stealing. 

…\marginnote{4.1} ‘Sexual misconduct has a bad result …’ … they descend from sexual misconduct. 

…\marginnote{5.1} ‘Lying has a bad result …’ … they descend from lying. 

…\marginnote{6.1} ‘Divisive speech has a bad result …’ … they descend from divisive speech. 

…\marginnote{7.1} ‘Harsh speech has a bad result …’ … they descend from harsh speech. 

…\marginnote{8.1} ‘Talking nonsense has a bad result …’ … they descend from talking nonsense. 

…\marginnote{9.1} ‘Covetousness has a bad result …’ … they descend from covetousness. 

…\marginnote{10.1} ‘Ill will has a bad result …’ … they descend from ill will. 

…\marginnote{11.1} ‘Wrong view has a bad result both in the present life and in lives to come.’ Reflecting like this, they give up wrong view, they descend from wrong view. This is called the noble descent.” 

%
\section*{{\suttatitleacronym AN 10.169}{\suttatitletranslation With Saṅgārava }{\suttatitleroot Saṅgāravasutta}}
\addcontentsline{toc}{section}{\tocacronym{AN 10.169} \toctranslation{With Saṅgārava } \tocroot{Saṅgāravasutta}}
\markboth{With Saṅgārava }{Saṅgāravasutta}
\extramarks{AN 10.169}{AN 10.169}

Then\marginnote{1.1} \textsanskrit{Saṅgārava} the brahmin went up to the Buddha, and exchanged greetings with him. When the greetings and polite conversation were over, he sat down to one side and said to the Buddha: 

“Mister\marginnote{2.1} Gotama, what is the near shore? And what is the far shore?” 

“Killing\marginnote{2.2} living creatures is the near shore, brahmin, and not killing living creatures is the far shore. Stealing is the near shore, and not stealing is the far shore. Sexual misconduct is the near shore, and avoiding sexual misconduct is the far shore. Lying is the near shore, and not lying is the far shore. Divisive speech is the near shore, and avoiding divisive speech is the far shore. Harsh speech is the near shore, and avoiding harsh speech is the far shore. Talking nonsense is the near shore, and avoiding talking nonsense is the far shore. Covetousness is the near shore, and contentment is the far shore. Ill will is the near shore, and good will is the far shore. Wrong view is the near shore, and right view is the far shore. This is the near shore, and this is the far shore. 

\begin{verse}%
Few\marginnote{3.1} are those among humans \\
who cross to the far shore. \\
The rest just run \\
around on the near shore. 

When\marginnote{4.1} the teaching is well explained, \\
those who practice accordingly \\
are the ones who will cross over \\
Death’s dominion so hard to pass. 

Rid\marginnote{5.1} of dark qualities, \\
an astute person should develop the bright. \\
Leaving home behind \\
for the seclusion so hard to enjoy, 

you\marginnote{6.1} should try to find delight there, \\
having left behind sensual pleasures. \\
With no possessions, an astute person \\
should cleanse themselves of mental corruptions. 

And\marginnote{7.1} those whose minds are rightly developed \\
in the awakening factors; \\
letting go of attachments, \\
they delight in not grasping. \\
With defilements ended, brilliant, \\
they are quenched in this world.” 

%
\end{verse}

%
\section*{{\suttatitleacronym AN 10.170}{\suttatitletranslation The Near Shore }{\suttatitleroot Orimasutta}}
\addcontentsline{toc}{section}{\tocacronym{AN 10.170} \toctranslation{The Near Shore } \tocroot{Orimasutta}}
\markboth{The Near Shore }{Orimasutta}
\extramarks{AN 10.170}{AN 10.170}

“Mendicants,\marginnote{1.1} I will teach you the near shore and the far shore. Listen and apply your mind well, I will speak. … And what, mendicants, is the near shore? What is the far shore? Killing living creatures is the near shore, mendicants, and not killing living creatures is the far shore. Stealing is the near shore, and not stealing is the far shore. Sexual misconduct is the near shore, and avoiding sexual misconduct is the far shore. Lying is the near shore, and not lying is the far shore. Divisive speech is the near shore, and avoiding divisive speech is the far shore. Harsh speech is the near shore, and avoiding harsh speech is the far shore. Talking nonsense is the near shore, and avoiding talking nonsense is the far shore. Covetousness is the near shore, and contentment is the far shore. Ill will is the near shore, and good will is the far shore. Wrong view is the near shore, and right view is the far shore. This is the near shore, and this is the far shore. 

\begin{verse}%
Few\marginnote{2.1} are those among humans \\
who cross to the far shore. \\
The rest just run \\
around on the near shore. 

When\marginnote{3.1} the teaching is well explained, \\
those who practice accordingly \\
are the ones who will cross over \\
Death’s dominion so hard to pass. 

Rid\marginnote{4.1} of dark qualities, \\
an astute person should develop the bright. \\
Leaving home behind \\
for the seclusion so hard to enjoy, 

you\marginnote{5.1} should try to find delight there, \\
having left behind sensual pleasures. \\
With no possessions, an astute person \\
should cleanse themselves of mental corruptions. 

And\marginnote{6.1} those whose minds are rightly developed \\
in the awakening factors; \\
letting go of attachments, \\
they delight in not grasping. \\
With defilements ended, brilliant, \\
they are quenched in this world.” 

%
\end{verse}

%
\section*{{\suttatitleacronym AN 10.171}{\suttatitletranslation Bad Principles (1st) }{\suttatitleroot Paṭhamaadhammasutta}}
\addcontentsline{toc}{section}{\tocacronym{AN 10.171} \toctranslation{Bad Principles (1st) } \tocroot{Paṭhamaadhammasutta}}
\markboth{Bad Principles (1st) }{Paṭhamaadhammasutta}
\extramarks{AN 10.171}{AN 10.171}

“Mendicants,\marginnote{1.1} you should know bad principles with bad results. And you should know good principles with good results. Knowing these things, your practice should follow the good principles with good results. 

And\marginnote{2.1} what are bad principles with bad results? Killing living creatures, stealing, and sexual misconduct; speech that’s false, divisive, harsh, or nonsensical; covetousness, ill will, and wrong view. These are called bad principles with bad results. 

And\marginnote{3.1} what are good principles with good results? Avoiding killing living creatures, stealing, and sexual misconduct; avoiding speech that’s false, divisive, harsh, or nonsensical; contentment, good will, and right view. These are called good principles with good results. 

‘You\marginnote{4.1} should know bad principles with bad results. And you should know good principles with good results. Knowing these things, your practice should follow the good principles with good results.’ That’s what I said, and this is why I said it.” 

%
\section*{{\suttatitleacronym AN 10.172}{\suttatitletranslation Bad Principles (2nd) }{\suttatitleroot Dutiyaadhammasutta}}
\addcontentsline{toc}{section}{\tocacronym{AN 10.172} \toctranslation{Bad Principles (2nd) } \tocroot{Dutiyaadhammasutta}}
\markboth{Bad Principles (2nd) }{Dutiyaadhammasutta}
\extramarks{AN 10.172}{AN 10.172}

“Mendicants,\marginnote{1.1} you should know bad principles and good principles. And you should know bad results and good results. Knowing these things, your practice should follow the good principles with good results.” 

That\marginnote{1.4} is what the Buddha said. When he had spoken, the Holy One got up from his seat and entered his dwelling. 

Soon\marginnote{2.1} after the Buddha left, those mendicants considered, “The Buddha gave this brief summary recital, then entered his dwelling without explaining the meaning in detail. ‘You should know bad principles and good principles. And you should know bad results and good results. Knowing these things, your practice should follow the good principles with good results.’ Who can explain in detail the meaning of this brief summary recital given by the Buddha?” 

Then\marginnote{3.1} those mendicants thought, “This Venerable \textsanskrit{Mahākaccāna} is praised by the Buddha and esteemed by his sensible spiritual companions. He is capable of explaining in detail the meaning of this brief summary recital given by the Buddha. Let’s go to him, and ask him about this matter. As he answers, so we’ll remember it.” 

Then\marginnote{4.1} those mendicants went to \textsanskrit{Mahākaccāna}, and exchanged greetings with him. When the greetings and polite conversation were over, they sat down to one side. They told him what had happened, and said, “May Venerable \textsanskrit{Mahākaccāna} please explain this.” 

“Reverends,\marginnote{8.1} suppose there was a person in need of heartwood. And while wandering in search of heartwood he’d come across a large tree standing with heartwood. But he’d pass over the roots and trunk, imagining that the heartwood should be sought in the branches and leaves. Such is the consequence for the venerables. Though you were face to face with the Buddha, you overlooked him, imagining that you should ask me about this matter. For he is the Buddha, the one who knows and sees. He is vision, he is knowledge, he is the manifestation of principle, he is the manifestation of divinity. He is the teacher, the proclaimer, the elucidator of meaning, the bestower of freedom from death, the lord of truth, the Realized One. That was the time to approach the Buddha and ask about this matter. You should have remembered it in line with the Buddha’s answer.” 

“Certainly\marginnote{9.1} he is the Buddha, the one who knows and sees. He is vision, he is knowledge, he is the manifestation of principle, he is the manifestation of divinity. He is the teacher, the proclaimer, the elucidator of meaning, the bestower of freedom from death, the lord of truth, the Realized One. That was the time to approach the Buddha and ask about this matter. We should have remembered it in line with the Buddha’s answer. Still, Venerable \textsanskrit{Mahākaccāna} is praised by the Buddha and esteemed by his sensible spiritual companions. He is capable of explaining in detail the meaning of this brief summary recital given by the Buddha. Please explain this, if it’s no trouble.” 

“Well\marginnote{10.1} then, reverends, listen and apply your mind well, I will speak.” 

“Yes,\marginnote{10.2} reverend,” they replied. \textsanskrit{Mahākaccāna} said this: 

“Reverends,\marginnote{11.1} the Buddha gave this brief summary recital, then entered his dwelling without explaining the meaning in detail: ‘You should know bad principles and good principles … and practice accordingly.’ 

So\marginnote{12.1} what are bad principles? What are good principles? What are bad results? And what are good results? Killing living creatures is a bad principle. Not killing living creatures is a good principle. And the many bad, unskillful qualities produced by killing living creatures are bad results. And the many skillful qualities fully developed because of not killing living creatures are good results. 

Stealing\marginnote{13.1} is a bad principle. Not stealing is a good principle. And the many bad, unskillful qualities produced by stealing are bad results. And the many skillful qualities fully developed because of not stealing are good results. 

Sexual\marginnote{14.1} misconduct is a bad principle. Avoiding sexual misconduct is a good principle. And the many bad, unskillful qualities produced by sexual misconduct are bad results. And the many skillful qualities fully developed because of avoiding sexual misconduct are good results. 

Lying\marginnote{15.1} is a bad principle. Not lying is a good principle. And the many bad, unskillful qualities produced by lying are bad results. And the many skillful qualities fully developed because of not lying are good results. 

Divisive\marginnote{16.1} speech is a bad principle. Avoiding divisive speech is a good principle. And the many bad, unskillful qualities produced by divisive speech are bad results. And the many skillful qualities fully developed because of avoiding divisive speech are good results. 

Harsh\marginnote{17.1} speech is a bad principle. Avoiding harsh speech is a good principle. And the many bad, unskillful qualities produced by harsh speech are bad results. And the many skillful qualities fully developed because of avoiding harsh speech are good results. 

Talking\marginnote{18.1} nonsense is a bad principle. Avoiding talking nonsense is a good principle. And the many bad, unskillful qualities produced by talking nonsense are bad results. And the many skillful qualities fully developed because of avoiding talking nonsense are good results. 

Covetousness\marginnote{19.1} is a bad principle. Contentment is a good principle. And the many bad, unskillful qualities produced by covetousness are bad results. And the many skillful qualities fully developed because of contentment are good results. 

Ill\marginnote{20.1} will is a bad principle. Good will is a good principle. And the many bad, unskillful qualities produced by ill will are bad results. And the many skillful qualities fully developed because of good will are good results. 

Wrong\marginnote{21.1} view is a bad principle. Right view is a good principle. And the many bad, unskillful qualities produced by wrong view are bad results. And the many skillful qualities fully developed because of right view are good results. 

The\marginnote{22.1} Buddha gave this brief summary recital, then entered his dwelling without explaining the meaning in detail: ‘You should know bad principles and good principles … and practice accordingly.’ And this is how I understand the detailed meaning of this summary recital. If you wish, you may go to the Buddha and ask him about this. You should remember it in line with the Buddha’s answer.” 

“Yes,\marginnote{23.1} reverend,” said those mendicants, approving and agreeing with what \textsanskrit{Mahākaccāna} said. Then they rose from their seats and went to the Buddha, bowed, sat down to one side, and told him what had happened. Then they said: 

“Sir,\marginnote{27.1} we went to \textsanskrit{Mahākaccāna} and asked him about this matter. And \textsanskrit{Mahākaccāna} clearly explained the meaning to us in this manner, with these words and phrases.” 

“Good,\marginnote{28.1} good, mendicants! \textsanskrit{Mahākaccāna} is astute, he has great wisdom. If you came to me and asked this question, I would answer it in exactly the same way as \textsanskrit{Mahākaccāna}. That is what it means, and that’s how you should remember it.” 

%
\section*{{\suttatitleacronym AN 10.173}{\suttatitletranslation Bad Principles (3rd) }{\suttatitleroot Tatiyaadhammasutta}}
\addcontentsline{toc}{section}{\tocacronym{AN 10.173} \toctranslation{Bad Principles (3rd) } \tocroot{Tatiyaadhammasutta}}
\markboth{Bad Principles (3rd) }{Tatiyaadhammasutta}
\extramarks{AN 10.173}{AN 10.173}

“Mendicants,\marginnote{1.1} you should know bad principles and good principles. And you should know bad results and good results. Knowing these things, your practice should follow the good principles with good results. 

So\marginnote{2.1} what are bad principles? What are good principles? What are bad results? And what are good results? Killing living creatures is a bad principle. Not killing living creatures is a good principle. And the many bad, unskillful qualities produced by killing living creatures are bad results. And the many skillful qualities fully developed because of not killing living creatures are good results. 

Stealing\marginnote{3.1} is a bad principle. Not stealing is a good principle. … Sexual misconduct is a bad principle. Avoiding sexual misconduct is a good principle. … Lying is a bad principle. Not lying is a good principle. … Divisive speech is a bad principle. Avoiding divisive speech is a good principle. … Harsh speech is a bad principle. Avoiding harsh speech is a good principle. … Talking nonsense is a bad principle. Avoiding talking nonsense is a good principle. … Covetousness is a bad principle. Contentment is a good principle. … Ill will is a bad principle. Good will is a good principle. … 

Wrong\marginnote{4.1} view is a bad principle. Right view is a good principle. And the many bad, unskillful qualities produced by wrong view are bad results. And the many skillful qualities fully developed because of right view are good results. 

‘You\marginnote{5.1} should know bad principles and good principles. And you should know bad results and good results. Knowing these things, your practice should follow the good principles with good results.’ That’s what I said, and this is why I said it.” 

%
\section*{{\suttatitleacronym AN 10.174}{\suttatitletranslation Sources of Deeds }{\suttatitleroot Kammanidānasutta}}
\addcontentsline{toc}{section}{\tocacronym{AN 10.174} \toctranslation{Sources of Deeds } \tocroot{Kammanidānasutta}}
\markboth{Sources of Deeds }{Kammanidānasutta}
\extramarks{AN 10.174}{AN 10.174}

“Mendicants,\marginnote{1.1} I say that killing living creatures is threefold: caused by greed, hate, or delusion. 

I\marginnote{2.1} say that stealing is threefold: caused by greed, hate, or delusion. 

I\marginnote{3.1} say that sexual misconduct is threefold: caused by greed, hate, or delusion. 

I\marginnote{4.1} say that lying is threefold: caused by greed, hate, or delusion. 

I\marginnote{5.1} say that divisive speech is threefold: caused by greed, hate, or delusion. 

I\marginnote{6.1} say that harsh speech is threefold: caused by greed, hate, or delusion. 

I\marginnote{7.1} say that talking nonsense is threefold: caused by greed, hate, or delusion. 

I\marginnote{8.1} say that covetousness is threefold: caused by greed, hate, or delusion. 

I\marginnote{9.1} say that ill will is threefold: caused by greed, hate, or delusion. 

I\marginnote{10.1} say that wrong view is threefold: caused by greed, hate, or delusion. And so greed, hate, and delusion are sources and origins for deeds. With the ending of greed, hate, and delusion, the sources of deeds are ended.” 

%
\section*{{\suttatitleacronym AN 10.175}{\suttatitletranslation The Bypass }{\suttatitleroot Parikkamanasutta}}
\addcontentsline{toc}{section}{\tocacronym{AN 10.175} \toctranslation{The Bypass } \tocroot{Parikkamanasutta}}
\markboth{The Bypass }{Parikkamanasutta}
\extramarks{AN 10.175}{AN 10.175}

“Mendicants,\marginnote{1.1} this teaching provides a bypass, it doesn’t lack a bypass. And how does this teaching provide a bypass, not lacking a bypass? Not killing living creatures bypasses killing living creatures. Not stealing bypasses stealing. Avoiding sexual misconduct bypasses sexual misconduct. Not lying bypasses lying. Avoiding divisive speech bypasses divisive speech. Avoiding harsh speech bypasses harsh speech. Avoiding talking nonsense bypasses talking nonsense. Contentment bypasses covetousness. Good will bypasses ill will. Right view bypasses wrong view. That’s how this teaching provides a bypass, it doesn’t lack a bypass.” 

%
\section*{{\suttatitleacronym AN 10.176}{\suttatitletranslation With Cunda }{\suttatitleroot Cundasutta}}
\addcontentsline{toc}{section}{\tocacronym{AN 10.176} \toctranslation{With Cunda } \tocroot{Cundasutta}}
\markboth{With Cunda }{Cundasutta}
\extramarks{AN 10.176}{AN 10.176}

\scevam{So\marginnote{1.1} I have heard. }At one time the Buddha was staying near \textsanskrit{Pāvā} in Cunda the smith’s mango grove. Then Cunda the smith went to the Buddha, bowed, and sat down to one side. The Buddha said to him, “Cunda, whose purity do you believe in?” 

“Sir,\marginnote{1.5} I believe in the purity advocated by the western brahmins draped with moss who carry pitchers, serve the sacred flame, and immerse themselves in water.” 

“But\marginnote{2.1} Cunda, what kind of purity do these western brahmins advocate?” 

“The\marginnote{2.2} western brahmins encourage their disciples like this: ‘Please, good people, rising early you should stroke the earth from your bed. If you don’t stroke the earth, stroke fresh cow dung. If you don’t stroke fresh cow dung, stroke green grass. If you don’t stroke green grass, serve the sacred flame. If you don’t serve the sacred flame, revere the sun with joined palms. If you don’t revere the sun with joined palms, immerse yourself in water three times, including the evening.’ The western brahmins advocate this kind of purity.” 

“The\marginnote{3.1} purity advocated by the western brahmins is quite different from that in the training of the Noble One.” 

“But\marginnote{3.2} what, sir, is purity in the training of the Noble One? Sir, please teach me this.” 

“Well\marginnote{4.1} then, brahmin, listen and apply your mind well, I will speak.” 

“Yes,\marginnote{4.2} sir,” Cunda replied. The Buddha said this: 

“Cunda,\marginnote{5.1} impurity is threefold by way of body, fourfold by way of speech, and threefold by way of mind. 

And\marginnote{6.1} how is impurity threefold by way of body? It’s when a certain person kills living creatures. They’re violent, bloody-handed, a hardened killer, merciless to living beings. 

They\marginnote{7.1} steal. With the intention to commit theft, they take the wealth or belongings of others from village or wilderness. 

They\marginnote{8.1} commit sexual misconduct. They have sexual relations with women who have their mother, father, both mother and father, brother, sister, relatives, or clan as guardian. They have sexual relations with a woman who is protected on principle, or who has a husband, or whose violation is punishable by law, or even one who has been garlanded as a token of betrothal. 

This\marginnote{9.1} is the threefold impurity by way of body. 

And\marginnote{10.1} how is impurity fourfold by way of speech? It’s when a certain person lies. They’re summoned to a council, an assembly, a family meeting, a guild, or to the royal court, and asked to bear witness: ‘Please, mister, say what you know.’ Not knowing, they say ‘I know.’ Knowing, they say ‘I don’t know.’ Not seeing, they say ‘I see.’ And seeing, they say ‘I don’t see.’ So they deliberately lie for the sake of themselves or another, or for some trivial worldly reason. 

They\marginnote{11.1} speak divisively. They repeat in one place what they heard in another so as to divide people against each other. And so they divide those who are harmonious, supporting division, delighting in division, loving division, speaking words that promote division. 

They\marginnote{12.1} speak harshly. They use the kinds of words that are cruel, nasty, hurtful, offensive, bordering on anger, not leading to immersion. 

They\marginnote{13.1} talk nonsense. Their speech is untimely, and is neither factual nor beneficial. It has nothing to do with the teaching or the training. Their words have no value, and are untimely, unreasonable, rambling, and pointless. This is the fourfold impurity by way of speech. 

And\marginnote{14.1} how is impurity threefold by way of mind? It’s when a certain person is covetous. They covet the wealth and belongings of others: ‘Oh, if only their belongings were mine!’ 

They\marginnote{15.1} have ill will and malicious intentions: ‘May these sentient beings be killed, slaughtered, slain, destroyed, or annihilated!’ 

They\marginnote{16.1} have wrong view. Their perspective is distorted: ‘There’s no meaning in giving, sacrifice, or offerings. There’s no fruit or result of good and bad deeds. There’s no afterlife. There’s no such thing as mother and father, or beings that are reborn spontaneously. And there’s no ascetic or brahmin who is rightly comported and rightly practiced, and who describes the afterlife after realizing it with their own insight.’ This is the threefold impurity by way of mind. 

These\marginnote{17.1} are the ten ways of doing unskillful deeds. When you have these ten ways of doing unskillful deeds, then if you rise early, whether or not you stroke the earth from your bed, you’re still impure. 

Whether\marginnote{18.1} or not you stroke fresh cow dung, you’re still impure. 

Whether\marginnote{19.1} or not you stroke green grass, you’re still impure. 

Whether\marginnote{20.1} or not you serve the sacred flame, you’re still impure. 

Whether\marginnote{21.1} or not you revere the sun with joined palms, you’re still impure. 

Whether\marginnote{22.1} or not you immerse yourself in water three times, you’re still impure. Why is that? These ten ways of doing unskillful deeds are impure and make things impure. 

It’s\marginnote{23.1} because of those who do these ten kinds of unskillful deeds that hell, the animal realm, the ghost realm, or any other bad places are found. 

Cunda,\marginnote{24.1} purity is threefold by way of body, fourfold by way of speech, and threefold by way of mind. 

And\marginnote{25.1} how is purity threefold by way of body? It’s when a certain person gives up killing living creatures. They renounce the rod and the sword. They’re scrupulous and kind, living full of sympathy for all living beings. 

They\marginnote{26.1} give up stealing. They don’t, with the intention to commit theft, take the wealth or belongings of others from village or wilderness. 

They\marginnote{27.1} give up sexual misconduct. They don’t have sexual relations with women who have their mother, father, both mother and father, brother, sister, relatives, or clan as guardian. They don’t have sexual relations with a woman who is protected on principle, or who has a husband, or whose violation is punishable by law, or even one who has been garlanded as a token of betrothal. 

This\marginnote{28.1} is the threefold purity by way of body. 

And\marginnote{29.1} how is purity fourfold by way of speech? It’s when a certain person gives up lying. They’re summoned to a council, an assembly, a family meeting, a guild, or to the royal court, and asked to bear witness: ‘Please, mister, say what you know.’ Not knowing, they say ‘I don’t know.’ Knowing, they say ‘I know.’ Not seeing, they say ‘I don’t see.’ And seeing, they say ‘I see.’ So they don’t deliberately lie for the sake of themselves or another, or for some trivial worldly reason. 

They\marginnote{30.1} give up divisive speech. They don’t repeat in one place what they heard in another so as to divide people against each other. Instead, they reconcile those who are divided, supporting unity, delighting in harmony, loving harmony, speaking words that promote harmony. 

They\marginnote{31.1} give up harsh speech. They speak in a way that’s mellow, pleasing to the ear, lovely, going to the heart, polite, likable and agreeable to the people. 

They\marginnote{32.1} give up talking nonsense. Their words are timely, true, and meaningful, in line with the teaching and training. They say things at the right time which are valuable, reasonable, succinct, and beneficial. 

This\marginnote{33.1} is the fourfold purity by way of speech. 

And\marginnote{34.1} how is purity threefold by way of mind? It’s when a certain person is content. They don’t covet the wealth and belongings of others: ‘Oh, if only their belongings were mine!’ 

They\marginnote{35.1} have a kind heart and loving intentions: ‘May these sentient beings live free of enmity and ill will, untroubled and happy!’ 

They\marginnote{36.1} have right view, an undistorted perspective: ‘There is meaning in giving, sacrifice, and offerings. There are fruits and results of good and bad deeds. There is an afterlife. There are such things as mother and father, and beings that are reborn spontaneously. And there are ascetics and brahmins who are rightly comported and rightly practiced, and who describe the afterlife after realizing it with their own insight.’ 

This\marginnote{37.1} is the threefold purity by way of mind. 

These\marginnote{38.1} are the ten ways of doing skillful deeds. When you have these ten ways of doing skillful deeds, then if you rise early, whether or not you stroke the earth from your bed, you’re still pure. 

Whether\marginnote{39.1} or not you stroke fresh cow dung, you’re still pure. 

Whether\marginnote{40.1} or not you stroke green grass, you’re still pure. 

Whether\marginnote{41.1} or not you serve the sacred flame, you’re still pure. 

Whether\marginnote{42.1} or not you revere the sun with joined palms, you’re still pure. 

Whether\marginnote{43.1} or not you immerse yourself in water three times, you’re still pure. Why is that? These ten ways of doing skillful deeds are pure and make things pure. 

It’s\marginnote{44.1} because of those who do these ten kinds of skillful deeds that gods, humans, or any other good places are found.” 

When\marginnote{45.1} he said this, Cunda the smith said to the Buddha, “Excellent, sir! Excellent! … From this day forth, may the Buddha remember me as a lay follower who has gone for refuge for life.” 

%
\section*{{\suttatitleacronym AN 10.177}{\suttatitletranslation With Jānussoṇi }{\suttatitleroot Jāṇussoṇisutta}}
\addcontentsline{toc}{section}{\tocacronym{AN 10.177} \toctranslation{With Jānussoṇi } \tocroot{Jāṇussoṇisutta}}
\markboth{With Jānussoṇi }{Jāṇussoṇisutta}
\extramarks{AN 10.177}{AN 10.177}

Then\marginnote{1.1} the brahmin \textsanskrit{Jānussoṇi} went up to the Buddha, and exchanged greetings with him. 

When\marginnote{1.2} the greetings and polite conversation were over, he sat down to one side and said to the Buddha, “We who are known as brahmins give gifts and perform memorial rites for the dead: ‘May this gift aid my departed relatives and kin. May they partake of this gift.’ But does this gift really aid departed relatives and kin? Do they actually partake of it?” 

“It\marginnote{2.6} aids them if the conditions are right, brahmin, but not if the conditions are wrong.” 

“Then,\marginnote{3.1} Mister Gotama, what are the right and wrong conditions?” 

“Brahmin,\marginnote{3.2} take someone who kills living creatures, steals, and commits sexual misconduct. They use speech that’s false, divisive, harsh, or nonsensical. And they’re covetous, malicious, with wrong view. When their body breaks up, after death, they’re reborn in hell. There they survive feeding on the food of the hell beings. The conditions there are wrong, so the gift does not aid the one who lives there. 

Take\marginnote{4.1} someone else who kills living creatures … and has wrong view. When their body breaks up, after death, they’re reborn in the animal realm. There they survive feeding on the food of the beings in the animal realm. The conditions there too are wrong, so the gift does not aid the one who lives there. 

Take\marginnote{5.1} someone else who doesn’t kill living creatures, steal, commit sexual misconduct, or use speech that’s false, divisive, harsh, or nonsensical. They're contented, kind-hearted, and have right view. When their body breaks up, after death, they’re reborn in the human realm. There they survive feeding on human food. The conditions there too are wrong, so the gift does not aid the one who lives there. 

Take\marginnote{6.1} someone else who doesn’t kill living creatures … and has right view. When their body breaks up, after death, they’re reborn in the company of the gods. There they survive feeding on the food of the gods. The conditions there too are wrong, so the gift does not aid the one who lives there. 

Take\marginnote{7.1} someone else who kills living creatures … and has wrong view. When their body breaks up, after death, they’re reborn in the ghost realm. There they survive feeding on the food of the beings in the ghost realm. Or else they survive feeding on what friends and colleagues, relatives and kin provide them with from here. The conditions there are right, so the gift aids the one who lives there.” 

“But\marginnote{8.1} Mister Gotama, who partakes of that gift if the departed relative is not reborn in that place?” 

“Other\marginnote{8.2} departed relatives reborn there will partake of that gift.” 

“But\marginnote{9.1} who partakes of the gift when neither that relative nor other relatives have been reborn in that place?” 

“It’s\marginnote{9.2} impossible, brahmin, it cannot happen that that place is vacant of departed relatives in all this long time. It’s never fruitless for the donor.” 

“Does\marginnote{10.1} Mister Gotama propose this even when the conditions are wrong?” 

“I\marginnote{10.2} propose this even when the conditions are wrong. Take someone who kills living creatures, steals, and commits sexual misconduct. They use speech that’s false, divisive, harsh, or nonsensical. And they’re covetous, malicious, with wrong view. They give to ascetics or brahmins such things as food, drink, clothing, vehicles; garlands, fragrance, and makeup; and bed, house, and lighting. When their body breaks up, after death, they’re reborn in the company of elephants. There they get to have food and drink, garlands and various adornments. 

Since\marginnote{11.1} in this life they killed living creatures … and had wrong view, they were reborn in the company of elephants. Since they gave to ascetics or brahmins … they get to have food and drink, garlands and various adornments. 

Take\marginnote{12.1} someone else who kills living creatures … and has wrong view. They give to ascetics or brahmins … When their body breaks up, after death, they’re reborn in the company of horses. … cattle … dogs. There they get to have food and drink, garlands and various adornments. 

Since\marginnote{13.1} in this life they killed living creatures … and had wrong view, they were reborn in the company of dogs. Since they gave to ascetics or brahmins … they get to have food and drink, garlands and various adornments. 

Take\marginnote{14.1} someone else who doesn’t kill living creatures, steal, or commit sexual misconduct. They don’t use speech that’s false, divisive, harsh, or nonsensical. And they’re contented, kind-hearted, with right view. They give to ascetics or brahmins … When their body breaks up, after death, they’re reborn in the human realm. There they get to have the five kinds of human sensual stimulation. 

Since\marginnote{15.1} in this life they didn’t kill living creatures … and had right view, they were reborn in the company of humans. Since they gave to ascetics or brahmins … they get to have the five kinds of human sensual stimulation. 

Take\marginnote{16.1} someone else who doesn’t kill living creatures … and has right view. They give to ascetics or brahmins … When their body breaks up, after death, they’re reborn in the company of the gods. There they get to have the five kinds of heavenly sensual stimulation. 

Since\marginnote{17.1} in this life they didn’t kill living creatures … and had right view, they were reborn in the company of the gods. Since they gave to ascetics or brahmins … they get to have the five kinds of heavenly sensual stimulation. It’s never fruitless for the donor.” 

“It’s\marginnote{18.1} incredible, Mister Gotama, it’s amazing, This is quite enough to justify giving gifts and performing memorial rites for the dead, since it’s never fruitless for the donor.” 

“That’s\marginnote{18.3} so true, brahmin. It’s never fruitless for the donor.” 

“Excellent,\marginnote{19.1} Mister Gotama! Excellent! … From this day forth, may Mister Gotama remember me as a lay follower who has gone for refuge for life.” 

%
\addtocontents{toc}{\let\protect\contentsline\protect\nopagecontentsline}
\chapter*{The Chapter on Good }
\addcontentsline{toc}{chapter}{\tocchapterline{The Chapter on Good }}
\addtocontents{toc}{\let\protect\contentsline\protect\oldcontentsline}

%
\section*{{\suttatitleacronym AN 10.178}{\suttatitletranslation Good }{\suttatitleroot Sādhusutta}}
\addcontentsline{toc}{section}{\tocacronym{AN 10.178} \toctranslation{Good } \tocroot{Sādhusutta}}
\markboth{Good }{Sādhusutta}
\extramarks{AN 10.178}{AN 10.178}

“Mendicants,\marginnote{1.1} I will teach you what is good and what is not good. Listen and apply your mind well, I will speak.” 

“Yes,\marginnote{1.3} sir,” they replied. The Buddha said this: 

“And\marginnote{2.1} what, mendicants, is not good? Killing living creatures, stealing, and sexual misconduct; speech that’s false, divisive, harsh, or nonsensical; covetousness, ill will, and wrong view. This is called what is not good. 

And\marginnote{3.1} what is good? Avoiding killing living creatures, stealing, and sexual misconduct; avoiding speech that’s false, divisive, harsh, or nonsensical; contentment, good will, and right view. This is called what is good.” 

%
\section*{{\suttatitleacronym AN 10.179}{\suttatitletranslation The Teaching of the Noble Ones }{\suttatitleroot Ariyadhammasutta}}
\addcontentsline{toc}{section}{\tocacronym{AN 10.179} \toctranslation{The Teaching of the Noble Ones } \tocroot{Ariyadhammasutta}}
\markboth{The Teaching of the Noble Ones }{Ariyadhammasutta}
\extramarks{AN 10.179}{AN 10.179}

“Mendicants,\marginnote{1.1} I will teach you the teaching of the noble ones, and what is not the teaching of the noble ones. Listen and apply your mind well, I will speak. … And what is not the teaching of the noble ones? Killing living creatures … wrong view. This is called what is not the teaching of the noble ones. 

And\marginnote{2.1} what is the teaching of the noble ones? Not killing living creatures … right view. This is called the teaching of the noble ones.” 

%
\section*{{\suttatitleacronym AN 10.180}{\suttatitletranslation Skillful }{\suttatitleroot Kusalasutta}}
\addcontentsline{toc}{section}{\tocacronym{AN 10.180} \toctranslation{Skillful } \tocroot{Kusalasutta}}
\markboth{Skillful }{Kusalasutta}
\extramarks{AN 10.180}{AN 10.180}

“I\marginnote{1.1} will teach you the skillful and the unskillful … And what is the unskillful? Killing living creatures … wrong view. This is called the unskillful. 

And\marginnote{2.1} what is the skillful? Not killing living creatures … right view. This is called the skillful.” 

%
\section*{{\suttatitleacronym AN 10.181}{\suttatitletranslation Beneficial }{\suttatitleroot Atthasutta}}
\addcontentsline{toc}{section}{\tocacronym{AN 10.181} \toctranslation{Beneficial } \tocroot{Atthasutta}}
\markboth{Beneficial }{Atthasutta}
\extramarks{AN 10.181}{AN 10.181}

“I\marginnote{1.1} will teach you the beneficial and the harmful. … And what is the harmful? Killing living creatures … wrong view. This is called the harmful. 

And\marginnote{2.1} what is the beneficial? Not killing living creatures … right view. This is called the beneficial.” 

%
\section*{{\suttatitleacronym AN 10.182}{\suttatitletranslation The Teaching }{\suttatitleroot Dhammasutta}}
\addcontentsline{toc}{section}{\tocacronym{AN 10.182} \toctranslation{The Teaching } \tocroot{Dhammasutta}}
\markboth{The Teaching }{Dhammasutta}
\extramarks{AN 10.182}{AN 10.182}

“I\marginnote{1.1} will teach you what is the teaching and what is not the teaching. … And what is not the teaching? Killing living creatures … wrong view. This is called what is not the teaching. 

And\marginnote{2.1} what is the teaching? Not killing living creatures … right view. This is called the teaching.” 

%
\section*{{\suttatitleacronym AN 10.183}{\suttatitletranslation Defiled }{\suttatitleroot Āsavasutta}}
\addcontentsline{toc}{section}{\tocacronym{AN 10.183} \toctranslation{Defiled } \tocroot{Āsavasutta}}
\markboth{Defiled }{Āsavasutta}
\extramarks{AN 10.183}{AN 10.183}

“I\marginnote{1.1} will teach you the defiled principle and the undefiled. … And what is the defiled principle? Killing living creatures … wrong view. This is called the defiled principle. 

And\marginnote{2.1} what is the undefiled principle? Not killing living creatures … right view. This is called the undefiled principle.” 

%
\section*{{\suttatitleacronym AN 10.184}{\suttatitletranslation Blameworthy }{\suttatitleroot Vajjasutta}}
\addcontentsline{toc}{section}{\tocacronym{AN 10.184} \toctranslation{Blameworthy } \tocroot{Vajjasutta}}
\markboth{Blameworthy }{Vajjasutta}
\extramarks{AN 10.184}{AN 10.184}

“I\marginnote{1.1} will teach you the blameworthy principle and the blameless. … And what is the blameworthy principle? Killing living creatures … wrong view. This is called the blameworthy principle. 

And\marginnote{2.1} what is the blameless principle? Not killing living creatures … right view. This is called the blameless principle.” 

%
\section*{{\suttatitleacronym AN 10.185}{\suttatitletranslation Mortifying }{\suttatitleroot Tapanīyasutta}}
\addcontentsline{toc}{section}{\tocacronym{AN 10.185} \toctranslation{Mortifying } \tocroot{Tapanīyasutta}}
\markboth{Mortifying }{Tapanīyasutta}
\extramarks{AN 10.185}{AN 10.185}

“I\marginnote{1.1} will teach you the mortifying principle and the unmortifying. … And what is the mortifying principle? Killing living creatures … wrong view. This is called the mortifying principle. 

And\marginnote{2.1} what is the unmortifying principle? Not killing living creatures … right view. This is called the unmortifying principle.” 

%
\section*{{\suttatitleacronym AN 10.186}{\suttatitletranslation Leading to Accumulation }{\suttatitleroot Ācayagāmisutta}}
\addcontentsline{toc}{section}{\tocacronym{AN 10.186} \toctranslation{Leading to Accumulation } \tocroot{Ācayagāmisutta}}
\markboth{Leading to Accumulation }{Ācayagāmisutta}
\extramarks{AN 10.186}{AN 10.186}

“I\marginnote{1.1} will teach you the principle that leads to accumulation and that which leads to dispersal. … And what is the principle that leads to accumulation? Killing living creatures … wrong view. This is called the principle that leads to accumulation. 

And\marginnote{2.1} what is the principle that leads to dispersal? Not killing living creatures … right view. This is called the principle that leads to dispersal.” 

%
\section*{{\suttatitleacronym AN 10.187}{\suttatitletranslation With Suffering as Outcome }{\suttatitleroot Dukkhudrayasutta}}
\addcontentsline{toc}{section}{\tocacronym{AN 10.187} \toctranslation{With Suffering as Outcome } \tocroot{Dukkhudrayasutta}}
\markboth{With Suffering as Outcome }{Dukkhudrayasutta}
\extramarks{AN 10.187}{AN 10.187}

“I\marginnote{1.1} will teach you the principle that has suffering as outcome, and that which has happiness as outcome. … And what is the principle whose outcome is suffering? Killing living creatures … wrong view. This is the principle whose outcome is suffering. 

And\marginnote{2.1} what is the principle whose outcome is happiness? Not killing living creatures … right view. This is the principle whose outcome is happiness.” 

%
\section*{{\suttatitleacronym AN 10.188}{\suttatitletranslation Result }{\suttatitleroot Vipākasutta}}
\addcontentsline{toc}{section}{\tocacronym{AN 10.188} \toctranslation{Result } \tocroot{Vipākasutta}}
\markboth{Result }{Vipākasutta}
\extramarks{AN 10.188}{AN 10.188}

“I\marginnote{1.1} will teach you the principle that results in suffering and that which results in happiness. … And what is the principle that results in suffering? Killing living creatures … wrong view. This is called the principle that results in suffering. 

And\marginnote{2.1} what is the principle that results in happiness? Not killing living creatures … right view. This is called the principle that results in happiness.” 

%
\addtocontents{toc}{\let\protect\contentsline\protect\nopagecontentsline}
\chapter*{The Chapter on the Noble Path }
\addcontentsline{toc}{chapter}{\tocchapterline{The Chapter on the Noble Path }}
\addtocontents{toc}{\let\protect\contentsline\protect\oldcontentsline}

%
\section*{{\suttatitleacronym AN 10.189}{\suttatitletranslation The Noble Path }{\suttatitleroot Ariyamaggasutta}}
\addcontentsline{toc}{section}{\tocacronym{AN 10.189} \toctranslation{The Noble Path } \tocroot{Ariyamaggasutta}}
\markboth{The Noble Path }{Ariyamaggasutta}
\extramarks{AN 10.189}{AN 10.189}

“I\marginnote{1.1} will teach you the noble path and the ignoble path. … And what is the ignoble path? Killing living creatures … wrong view. This is called the ignoble path. 

And\marginnote{2.1} what is the noble path? Not killing living creatures … right view. This is called the noble path.” 

%
\section*{{\suttatitleacronym AN 10.190}{\suttatitletranslation The Dark Path }{\suttatitleroot Kaṇhamaggasutta}}
\addcontentsline{toc}{section}{\tocacronym{AN 10.190} \toctranslation{The Dark Path } \tocroot{Kaṇhamaggasutta}}
\markboth{The Dark Path }{Kaṇhamaggasutta}
\extramarks{AN 10.190}{AN 10.190}

“I\marginnote{1.1} will teach you the dark path and the bright path. … And what is the dark path? Killing living creatures … wrong view. This is called the dark path. 

And\marginnote{2.1} what is the bright path? Not killing living creatures … right view. This is called the bright path.” 

%
\section*{{\suttatitleacronym AN 10.191}{\suttatitletranslation The True Teaching }{\suttatitleroot Saddhammasutta}}
\addcontentsline{toc}{section}{\tocacronym{AN 10.191} \toctranslation{The True Teaching } \tocroot{Saddhammasutta}}
\markboth{The True Teaching }{Saddhammasutta}
\extramarks{AN 10.191}{AN 10.191}

“I\marginnote{1.1} will teach you what is the true teaching and what is not the true teaching. … And what is not the true teaching? Killing living creatures … wrong view. This is called what is not the true teaching. 

And\marginnote{2.1} what is the true teaching? Not killing living creatures … right view. This is called the true teaching.” 

%
\section*{{\suttatitleacronym AN 10.192}{\suttatitletranslation The Teaching of the True Persons }{\suttatitleroot Sappurisadhammasutta}}
\addcontentsline{toc}{section}{\tocacronym{AN 10.192} \toctranslation{The Teaching of the True Persons } \tocroot{Sappurisadhammasutta}}
\markboth{The Teaching of the True Persons }{Sappurisadhammasutta}
\extramarks{AN 10.192}{AN 10.192}

“Mendicants,\marginnote{1.1} I will teach you the teaching of the true persons and the teaching of the untrue persons. And what is the teaching of the untrue persons? Killing living creatures … wrong view. This is the teaching of the untrue persons. 

And\marginnote{2.1} what is the teaching of the true persons? Not killing living creatures … right view. This is the teaching of the true persons.” 

%
\section*{{\suttatitleacronym AN 10.193}{\suttatitletranslation Principles That Should Be Activated }{\suttatitleroot Uppādetabbadhammasutta}}
\addcontentsline{toc}{section}{\tocacronym{AN 10.193} \toctranslation{Principles That Should Be Activated } \tocroot{Uppādetabbadhammasutta}}
\markboth{Principles That Should Be Activated }{Uppādetabbadhammasutta}
\extramarks{AN 10.193}{AN 10.193}

“I\marginnote{1.1} will teach you the principle to activate and the principle not to activate. … And what is the principle not to activate? Killing living creatures … wrong view. This is called the principle not to activate. 

And\marginnote{2.1} what is the principle to activate? Not killing living creatures … right view. This is called the principle to activate.” 

%
\section*{{\suttatitleacronym AN 10.194}{\suttatitletranslation Principles That Should Be Cultivated }{\suttatitleroot Āsevitabbadhammasutta}}
\addcontentsline{toc}{section}{\tocacronym{AN 10.194} \toctranslation{Principles That Should Be Cultivated } \tocroot{Āsevitabbadhammasutta}}
\markboth{Principles That Should Be Cultivated }{Āsevitabbadhammasutta}
\extramarks{AN 10.194}{AN 10.194}

“I\marginnote{1.1} will teach you the principle to cultivate and the principle not to cultivate. … And what is the principle not to cultivate? Killing living creatures … wrong view. This is called the principle not to cultivate. 

And\marginnote{2.1} what is the principle to cultivate? Not killing living creatures … right view. This is called the principle to cultivate.” 

%
\section*{{\suttatitleacronym AN 10.195}{\suttatitletranslation Principles That Should Be Developed }{\suttatitleroot Bhāvetabbadhammasutta}}
\addcontentsline{toc}{section}{\tocacronym{AN 10.195} \toctranslation{Principles That Should Be Developed } \tocroot{Bhāvetabbadhammasutta}}
\markboth{Principles That Should Be Developed }{Bhāvetabbadhammasutta}
\extramarks{AN 10.195}{AN 10.195}

“I\marginnote{1.1} will teach you the principle to develop and the principle not to develop. … Listen and apply your mind well, I will speak. And what is the principle not to develop? Killing living creatures … wrong view. This is called the principle not to develop. 

And\marginnote{2.1} what is the principle to develop? Not killing living creatures … right view. This is called the principle to develop.” 

%
\section*{{\suttatitleacronym AN 10.196}{\suttatitletranslation Principles That Should Be Made Much Of }{\suttatitleroot Bahulīkātabbasutta}}
\addcontentsline{toc}{section}{\tocacronym{AN 10.196} \toctranslation{Principles That Should Be Made Much Of } \tocroot{Bahulīkātabbasutta}}
\markboth{Principles That Should Be Made Much Of }{Bahulīkātabbasutta}
\extramarks{AN 10.196}{AN 10.196}

“I\marginnote{1.1} will teach you the principle to make much of and the principle not to make much of. … And what is the principle not to make much of? Killing living creatures … wrong view. This is called the principle not to make much of. 

And\marginnote{2.1} what is the principle to make much of? Not killing living creatures … right view. This is called the principle to make much of.” 

%
\section*{{\suttatitleacronym AN 10.197}{\suttatitletranslation Should Be Recollected }{\suttatitleroot Anussaritabbasutta}}
\addcontentsline{toc}{section}{\tocacronym{AN 10.197} \toctranslation{Should Be Recollected } \tocroot{Anussaritabbasutta}}
\markboth{Should Be Recollected }{Anussaritabbasutta}
\extramarks{AN 10.197}{AN 10.197}

“I\marginnote{1.1} will teach you the principle to recollect and the principle not to recollect. … And what is the principle not to recollect? Killing living creatures … wrong view. This is called the principle not to recollect. 

And\marginnote{2.1} what is the principle to recollect? Not killing living creatures … right view. This is called the principle to recollect.” 

%
\section*{{\suttatitleacronym AN 10.198}{\suttatitletranslation Should Be Realized }{\suttatitleroot Sacchikātabbasutta}}
\addcontentsline{toc}{section}{\tocacronym{AN 10.198} \toctranslation{Should Be Realized } \tocroot{Sacchikātabbasutta}}
\markboth{Should Be Realized }{Sacchikātabbasutta}
\extramarks{AN 10.198}{AN 10.198}

“I\marginnote{1.1} will teach you the principle to realize and the principle not to realize. … And what is the principle not to realize? Killing living creatures … wrong view. This is called the principle not to realize. 

And\marginnote{2.1} what is the principle to realize? Not killing living creatures … right view. This is called the principle to realize.” 

%
\addtocontents{toc}{\let\protect\contentsline\protect\nopagecontentsline}
\chapter*{Another Chapter on Persons }
\addcontentsline{toc}{chapter}{\tocchapterline{Another Chapter on Persons }}
\addtocontents{toc}{\let\protect\contentsline\protect\oldcontentsline}

%
\section*{{\suttatitleacronym AN 10.199–210}{\suttatitletranslation Should Not Associate, Etc. }{\suttatitleroot Aparapuggalavagga}}
\addcontentsline{toc}{section}{\tocacronym{AN 10.199–210} \toctranslation{Should Not Associate, Etc. } \tocroot{Aparapuggalavagga}}
\markboth{Should Not Associate, Etc. }{Aparapuggalavagga}
\extramarks{AN 10.199–210}{AN 10.199–210}

“Mendicants,\marginnote{1.1} you should not associate with a person who has ten qualities. What ten? They kill living creatures, steal, and commit sexual misconduct. They use speech that’s false, divisive, harsh, or nonsensical. And they’re covetous, malicious, with wrong view. You should not associate with a person who has these ten qualities. 

You\marginnote{2.1} should associate with a person who has ten qualities. What ten? They don’t kill living creatures, steal, or commit sexual misconduct. They don’t use speech that’s false, divisive, harsh, or nonsensical. They’re contented, kind-hearted, with right view. You should associate with a person who has these ten qualities.” 

“Mendicants,\marginnote{3.1} you should not frequent a person who has ten qualities … You should frequent …” “You should not pay homage … You should pay homage …” “You should not venerate … You should venerate …” “You should not praise … You should praise …” “You should not revere … You should revere …” “You should not defer to … You should defer to …” “A person is not a success … A person is a success …” “A person is not pure … A person is pure …” “A person does not win over conceit … A person wins over conceit …” “A person does not grow in wisdom … A person grows in wisdom …” 

“A\marginnote{4.1} person who has these ten qualities creates much wickedness. … A person who has these ten qualities creates much merit. What ten? They don’t kill living creatures, steal, or commit sexual misconduct. They don’t use speech that’s false, divisive, harsh, or nonsensical. They’re contented, kind-hearted, with right view. A person who has these ten qualities creates much merit.” 

%
\addtocontents{toc}{\let\protect\contentsline\protect\nopagecontentsline}
\pannasa{The Fifth Fifty }
\addcontentsline{toc}{pannasa}{The Fifth Fifty }
\markboth{}{}
\addtocontents{toc}{\let\protect\contentsline\protect\oldcontentsline}

%
\addtocontents{toc}{\let\protect\contentsline\protect\nopagecontentsline}
\chapter*{The Chapter on the Body Born of Deeds }
\addcontentsline{toc}{chapter}{\tocchapterline{The Chapter on the Body Born of Deeds }}
\addtocontents{toc}{\let\protect\contentsline\protect\oldcontentsline}

%
\section*{{\suttatitleacronym AN 10.211}{\suttatitletranslation Heaven and Hell (1st) }{\suttatitleroot Paṭhamanirayasaggasutta}}
\addcontentsline{toc}{section}{\tocacronym{AN 10.211} \toctranslation{Heaven and Hell (1st) } \tocroot{Paṭhamanirayasaggasutta}}
\markboth{Heaven and Hell (1st) }{Paṭhamanirayasaggasutta}
\extramarks{AN 10.211}{AN 10.211}

“Someone\marginnote{1.1} with ten qualities is cast down to hell. What ten? It’s when a certain person kills living creatures. They’re violent, bloody-handed, a hardened killer, merciless to living beings. 

They\marginnote{2.1} steal. With the intention to commit theft, they take the wealth or belongings of others from village or wilderness. 

They\marginnote{3.1} commit sexual misconduct. They have sexual relations with women who have their mother, father, both mother and father, brother, sister, relatives, or clan as guardian. They have sexual relations with a woman who is protected on principle, or who has a husband, or whose violation is punishable by law, or even one who has been garlanded as a token of betrothal. 

They\marginnote{4.1} lie. They’re summoned to a council, an assembly, a family meeting, a guild, or to the royal court, and asked to bear witness: ‘Please, mister, say what you know.’ Not knowing, they say ‘I know.’ Knowing, they say ‘I don’t know.’ Not seeing, they say ‘I see.’ And seeing, they say ‘I don’t see.’ So they deliberately lie for the sake of themselves or another, or for some trivial worldly reason. 

They\marginnote{5.1} speak divisively. They repeat in one place what they heard in another so as to divide people against each other. And so they divide those who are harmonious, supporting division, delighting in division, loving division, speaking words that promote division. 

They\marginnote{6.1} speak harshly. They use the kinds of words that are cruel, nasty, hurtful, offensive, bordering on anger, not leading to immersion. 

They\marginnote{7.1} talk nonsense. Their speech is untimely, and is neither factual nor beneficial. It has nothing to do with the teaching or the training. Their words have no value, and are untimely, unreasonable, rambling, and pointless. 

They’re\marginnote{8.1} covetous. They covet the wealth and belongings of others: ‘Oh, if only their belongings were mine!’ 

They\marginnote{9.1} have ill will and malicious intentions: ‘May these sentient beings be killed, slaughtered, slain, destroyed, or annihilated!’ 

They\marginnote{10.1} have wrong view. Their perspective is distorted: ‘There’s no meaning in giving, sacrifice, or offerings. There’s no fruit or result of good and bad deeds. There’s no afterlife. There’s no such thing as mother and father, or beings that are reborn spontaneously. And there’s no ascetic or brahmin who is rightly comported and rightly practiced, and who describes the afterlife after realizing it with their own insight.’ Someone with these ten qualities is cast down to hell. 

Someone\marginnote{12.1} with ten qualities is raised up to heaven. What ten? It’s when a certain person gives up killing living creatures. They renounce the rod and the sword. They’re scrupulous and kind, living full of sympathy for all living beings. 

They\marginnote{13.1} give up stealing. They don’t, with the intention to commit theft, take the wealth or belongings of others from village or wilderness. 

They\marginnote{14.1} give up sexual misconduct. They don’t have sex with women who have their mother, father, both mother and father, brother, sister, relatives, or clan as guardian. They don’t have sex with a woman who is protected on principle, or who has a husband, or whose violation is punishable by law, or even one who has been garlanded as a token of betrothal. 

They\marginnote{15.1} give up lying. They’re summoned to a council, an assembly, a family meeting, a guild, or to the royal court, and asked to bear witness: ‘Please, mister, say what you know.’ Not knowing, they say ‘I don’t know.’ Knowing, they say ‘I know.’ Not seeing, they say ‘I don’t see.’ And seeing, they say ‘I see.’ So they don’t deliberately lie for the sake of themselves or another, or for some trivial worldly reason. 

They\marginnote{16.1} give up divisive speech. They don’t repeat in one place what they heard in another so as to divide people against each other. Instead, they reconcile those who are divided, supporting unity, delighting in harmony, loving harmony, speaking words that promote harmony. 

They\marginnote{17.1} give up harsh speech. They speak in a way that’s mellow, pleasing to the ear, lovely, going to the heart, polite, likable and agreeable to the people. 

They\marginnote{18.1} give up talking nonsense. Their words are timely, true, and meaningful, in line with the teaching and training. They say things at the right time which are valuable, reasonable, succinct, and beneficial. 

They’re\marginnote{19.1} content. They don’t covet the wealth and belongings of others: ‘Oh, if only their belongings were mine!’ 

They\marginnote{20.1} have a kind heart and loving intentions: ‘May these sentient beings live free of enmity and ill will, untroubled and happy!’ 

They\marginnote{21.1} have right view, an undistorted perspective: ‘There is meaning in giving, sacrifice, and offerings. There are fruits and results of good and bad deeds. There is an afterlife. There are such things as mother and father, and beings that are reborn spontaneously. And there are ascetics and brahmins who are rightly comported and rightly practiced, and who describe the afterlife after realizing it with their own insight.’ 

Someone\marginnote{22.1} with these ten qualities is raised up to heaven.” 

%
\section*{{\suttatitleacronym AN 10.212}{\suttatitletranslation Heaven and Hell (2nd) }{\suttatitleroot Dutiyanirayasaggasutta}}
\addcontentsline{toc}{section}{\tocacronym{AN 10.212} \toctranslation{Heaven and Hell (2nd) } \tocroot{Dutiyanirayasaggasutta}}
\markboth{Heaven and Hell (2nd) }{Dutiyanirayasaggasutta}
\extramarks{AN 10.212}{AN 10.212}

“Someone\marginnote{1.1} with ten qualities is cast down to hell. What ten? It’s when a certain person kills living creatures. They’re violent, bloody-handed, a hardened killer, merciless to living beings. 

They\marginnote{2.1} steal. … They commit sexual misconduct. … They lie. … They speak divisively. … They speak harshly. … They indulge in talking nonsense. … They’re covetous. … They have cruel intentions. … They have wrong view. … Someone with these ten qualities is cast down to hell. 

Someone\marginnote{3.1} with ten qualities is raised up to heaven. What ten? It’s when a certain person gives up killing living creatures. They renounce the rod and the sword. They’re scrupulous and kind, living full of sympathy for all living beings. 

They\marginnote{4.1} give up stealing. … They give up sexual misconduct. … They give up lying. … They give up divisive speech. … They give up harsh speech. … They give up talking nonsense. … They’re content. … They’re kind hearted. … They have right view. … Someone with these ten qualities is raised up to heaven.” 

%
\section*{{\suttatitleacronym AN 10.213}{\suttatitletranslation A Female }{\suttatitleroot Mātugāmasutta}}
\addcontentsline{toc}{section}{\tocacronym{AN 10.213} \toctranslation{A Female } \tocroot{Mātugāmasutta}}
\markboth{A Female }{Mātugāmasutta}
\extramarks{AN 10.213}{AN 10.213}

“A\marginnote{1.1} female with ten qualities is cast down to hell. What ten? She kills living creatures. … She steals. … She commits sexual misconduct. … She lies. … She speaks divisively. … She speaks harshly. … She indulges in talking nonsense. … She’s covetous. … She has cruel intentions. … She has wrong view. … A female with these ten qualities is cast down to hell. 

A\marginnote{2.1} female with ten qualities is raised up to heaven. What ten? She doesn’t kill living creatures. … She doesn’t steal. … She doesn’t commit sexual misconduct. … She doesn’t lie. … She doesn’t speak divisively. … She doesn’t speak harshly. … She doesn’t indulge in talking nonsense. … She’s content. … She’s kind hearted. … She has right view. … A female with these ten qualities is raised up to heaven.” 

%
\section*{{\suttatitleacronym AN 10.214}{\suttatitletranslation A Laywoman }{\suttatitleroot Upāsikāsutta}}
\addcontentsline{toc}{section}{\tocacronym{AN 10.214} \toctranslation{A Laywoman } \tocroot{Upāsikāsutta}}
\markboth{A Laywoman }{Upāsikāsutta}
\extramarks{AN 10.214}{AN 10.214}

“A\marginnote{1.1} laywoman with ten qualities is cast down to hell. What ten? She kills living creatures. … She has wrong view. … A laywoman with these ten qualities is cast down to hell. 

A\marginnote{2.1} laywoman with ten qualities is raised up to heaven. What ten? She doesn’t kill living creatures. … She has right view. … A laywoman with these ten qualities is raised up to heaven.” 

%
\section*{{\suttatitleacronym AN 10.215}{\suttatitletranslation Assured }{\suttatitleroot Visāradasutta}}
\addcontentsline{toc}{section}{\tocacronym{AN 10.215} \toctranslation{Assured } \tocroot{Visāradasutta}}
\markboth{Assured }{Visāradasutta}
\extramarks{AN 10.215}{AN 10.215}

“A\marginnote{1.1} laywoman living at home with these ten qualities is not self-assured. What ten? She kills living creatures. … She has wrong view. … A laywoman living at home with these ten qualities is not self-assured. 

A\marginnote{2.1} laywoman living at home with these ten qualities is self-assured. What ten? She doesn’t kill living creatures. … She has right view. … A laywoman living at home with these ten qualities is self-assured.” 

%
\section*{{\suttatitleacronym AN 10.216}{\suttatitletranslation Creepy Creatures }{\suttatitleroot Saṁsappanīyasutta}}
\addcontentsline{toc}{section}{\tocacronym{AN 10.216} \toctranslation{Creepy Creatures } \tocroot{Saṁsappanīyasutta}}
\markboth{Creepy Creatures }{Saṁsappanīyasutta}
\extramarks{AN 10.216}{AN 10.216}

“Mendicants,\marginnote{1.1} I will teach you an exposition of the teaching on creepy creatures. Listen and apply your mind well, I will speak.” 

“Yes,\marginnote{1.3} sir,” they replied. The Buddha said this: 

“What\marginnote{2.1} is the exposition of the teaching on creepy creatures? Sentient beings are the owners of their deeds and heir to their deeds. Deeds are their womb, their relative, and their refuge. They shall be the heir of whatever deeds they do, whether good or bad. 

Take\marginnote{3.1} a certain person who kills living creatures. They’re violent, bloody-handed, a hardened killer, merciless to living beings. They’re creepy in body, speech, and mind. Doing crooked deeds by way of body, speech, and mind, their destiny and rebirth are crooked. 

Someone\marginnote{4.1} whose destiny and rebirth is crooked is reborn in one of two places, I say: in an exclusively painful hell, or among the species of creepy animals. And what are the species of creepy animals? Snakes, scorpions, centipedes, mongooses, cats, mice, owls, or whatever other species of animal that creep away when they see humans. This is how a being is born from a being. For your deeds determine your rebirth, and when you’re reborn contacts strike you. This is why I say that sentient beings are heirs to their deeds. 

Take\marginnote{5.1} someone else who steals … commits sexual misconduct … lies … speaks divisively … speaks harshly … indulges in talking nonsense … is covetous … has cruel intentions … has wrong view … They’re creepy in body, speech, and mind. Doing crooked deeds by way of body, speech, and mind, their destiny and rebirth are crooked. 

Someone\marginnote{6.1} whose destiny and rebirth is crooked is reborn in one of two places, I say: in an exclusively painful hell, or among the species of creepy animals. And what are the species of creepy animals? Snakes, scorpions, centipedes, mongooses, cats, mice, owls, or whatever other species of animal that creep away when they see humans. This is how a being is born from a being. For your deeds determine your rebirth, and when you’re reborn contacts strike you. This is why I say that sentient beings are heirs to their deeds. Sentient beings are the owners of their deeds and heir to their deeds. Deeds are their womb, their relative, and their refuge. They shall be the heir of whatever deeds they do, whether good or bad. 

Take\marginnote{7.1} a certain person who gives up killing living creatures. They renounce the rod and the sword. They’re scrupulous and kind, living full of sympathy for all living beings. They’re not creepy in body, speech, and mind. Doing virtuous deeds by way of body, speech, and mind, their destiny and rebirth is virtuous. 

Someone\marginnote{8.1} whose destiny and rebirth is virtuous is reborn in one of two places, I say: in a heaven of perfect happiness, or in an eminent well-to-do family of aristocrats, brahmins, or householders—rich, affluent, and wealthy, with lots of gold and silver, lots of property and assets, and lots of money and grain. This is how a being is born from a being. For your deeds determine your rebirth, and when you’re reborn contacts strike you. This is why I say that sentient beings are heirs to their deeds. 

Take\marginnote{9.1} someone else who gives up stealing … sexual misconduct … lying … divisive speech … harsh speech … talking nonsense … They’re content … kind hearted … they have right view … They’re not creepy in body, speech, and mind. Doing virtuous deeds by way of body, speech, and mind, their destiny and rebirth is virtuous. 

Someone\marginnote{10.1} whose destiny and rebirth is virtuous is reborn in one of two places, I say: in a heaven of perfect happiness, or in an eminent well-to-do family of aristocrats, brahmins, or householders—rich, affluent, and wealthy, with lots of gold and silver, lots of property and assets, and lots of money and grain. This is how a being is born from a being. For your deeds determine your rebirth, and when you’re reborn contacts strike you. This is why I say that sentient beings are heirs to their deeds. 

Sentient\marginnote{11.1} beings are the owners of their deeds and heir to their deeds. Deeds are their womb, their relative, and their refuge. They shall be the heir of whatever deeds they do, whether good or bad. This is the exposition of the teaching on creepy creatures.” 

%
\section*{{\suttatitleacronym AN 10.217}{\suttatitletranslation Intentional (1st) }{\suttatitleroot Paṭhamasañcetanikasutta}}
\addcontentsline{toc}{section}{\tocacronym{AN 10.217} \toctranslation{Intentional (1st) } \tocroot{Paṭhamasañcetanikasutta}}
\markboth{Intentional (1st) }{Paṭhamasañcetanikasutta}
\extramarks{AN 10.217}{AN 10.217}

“Mendicants,\marginnote{1.1} I don’t say that intentional deeds that have been performed and accumulated are eliminated without being experienced. And that may be in this very life, or in the next life, or in some subsequent period. And I don’t say that suffering is ended without experiencing intentional deeds that have been performed and accumulated. 

Now,\marginnote{2.1} there are three kinds of corruption and failure of bodily action that have unskillful intention, with suffering as their outcome and result. There are four kinds of corruption and failure of verbal action that have unskillful intention, with suffering as their outcome and result. There are three kinds of corruption and failure of mental action that have unskillful intention, with suffering as their outcome and result. 

And\marginnote{3.1} what are the three kinds of corruption and failure of bodily action? It’s when a certain person kills living creatures. They’re violent, bloody-handed, a hardened killer, merciless to living beings. 

They\marginnote{4.1} steal. With the intention to commit theft, they take the wealth or belongings of others from village or wilderness. 

They\marginnote{5.1} commit sexual misconduct. They have sex with women who have their mother, father, both mother and father, brother, sister, relatives, or clan as guardian. They have sex with a woman who is protected on principle, or who has a husband, or whose violation is punishable by law, or even one who has been garlanded as a token of betrothal. 

These\marginnote{6.1} are the three kinds of corruption and failure of bodily action. 

And\marginnote{7.1} what are the four kinds of corruption and failure of verbal action? It’s when a certain person lies. They’re summoned to a council, an assembly, a family meeting, a guild, or to the royal court, and asked to bear witness: ‘Please, mister, say what you know.’ Not knowing, they say ‘I know.’ Knowing, they say ‘I don’t know.’ Not seeing, they say ‘I see.’ And seeing, they say ‘I don’t see.’ So they deliberately lie for the sake of themselves or another, or for some trivial worldly reason. 

They\marginnote{8.1} speak divisively. They repeat in one place what they heard in another so as to divide people against each other. And so they divide those who are harmonious, supporting division, delighting in division, loving division, speaking words that promote division. 

They\marginnote{9.1} speak harshly. They use the kinds of words that are cruel, nasty, hurtful, offensive, bordering on anger, not leading to immersion. 

They\marginnote{10.1} indulge in talking nonsense. Their speech is untimely, and is neither factual nor beneficial. It has nothing to do with the teaching or the training. Their words have no value, and are untimely, unreasonable, rambling, and pointless. 

These\marginnote{11.1} are the four kinds of corruption and failure of verbal action. 

And\marginnote{12.1} what are the three kinds of corruption and failure of mental action? It’s when someone is covetous. They covet the wealth and belongings of others: ‘Oh, if only their belongings were mine!’ 

They\marginnote{13.1} have ill will and malicious intentions: ‘May these sentient beings be killed, slaughtered, slain, destroyed, or annihilated!’ 

They\marginnote{14.1} have wrong view. Their perspective is distorted: ‘There’s no meaning in giving, sacrifice, or offerings. There’s no fruit or result of good and bad deeds. There’s no afterlife. There’s no such thing as mother and father, or beings that are reborn spontaneously. And there’s no ascetic or brahmin who is rightly comported and rightly practiced, and who describes the afterlife after realizing it with their own insight.’ 

These\marginnote{15.1} are the three kinds of corruption and failure of mental action. 

When\marginnote{16.1} their body breaks up, after death, sentient beings are reborn in a place of loss, a bad place, the underworld, hell because of these three kinds of corruption and failure of bodily action, these four kinds of corruption and failure of verbal action, or these three kinds of corruption and failure of mental action that have unskillful intention, with suffering as their outcome and result. 

It’s\marginnote{17.1} like throwing unfailing dice: they always fall the right side up. In the same way, when their body breaks up, after death, sentient beings are reborn in a place of loss, a bad place, the underworld, hell because of these three kinds of corruption and failure of bodily action, these four kinds of corruption and failure of verbal action, or these three kinds of corruption and failure of mental action that have unskillful intention, with suffering as their outcome and result. 

I\marginnote{18.1} don’t say that intentional deeds that have been performed and accumulated are eliminated without being experienced. And that may be in this very life, or in the next life, or in some subsequent period. And I don’t say that suffering is ended without experiencing intentional deeds that have been performed and accumulated. 

Now,\marginnote{19.1} there are three kinds of successful bodily action that have skillful intention, with happiness as their outcome and result. There are four kinds of successful verbal action that have skillful intention, with happiness as their outcome and result. There are three kinds of successful mental action that have skillful intention, with happiness as their outcome and result. 

And\marginnote{20.1} what are the three kinds of successful bodily action? It’s when a certain person gives up killing living creatures. They renounce the rod and the sword. They’re scrupulous and kind, living full of sympathy for all living beings. 

They\marginnote{21.1} don’t steal. They don’t, with the intention to commit theft, take the wealth or belongings of others from village or wilderness. 

They\marginnote{22.1} give up sexual misconduct. They don’t have sex with women who have their mother, father, both mother and father, brother, sister, relatives, or clan as guardian. They don’t have sex with a woman who is protected on principle, or who has a husband, or whose violation is punishable by law, or even one who has been garlanded as a token of betrothal. 

These\marginnote{23.1} are the three kinds of successful bodily action. 

And\marginnote{24.1} what are the four kinds of successful verbal action? It’s when a certain person gives up lying. They’re summoned to a council, an assembly, a family meeting, a guild, or to the royal court, and asked to bear witness: ‘Please, mister, say what you know.’ Not knowing, they say ‘I don’t know.’ Knowing, they say ‘I know.’ Not seeing, they say ‘I don’t see.’ And seeing, they say ‘I see.’ They don’t deliberately lie for the sake of themselves or another, or for some trivial worldly reason. 

They\marginnote{25.1} give up divisive speech. They don’t repeat in one place what they heard in another so as to divide people against each other. Instead, they reconcile those who are divided, supporting unity, delighting in harmony, loving harmony, speaking words that promote harmony. 

They\marginnote{26.1} give up harsh speech. They speak in a way that’s mellow, pleasing to the ear, lovely, going to the heart, polite, likable and agreeable to the people. 

They\marginnote{27.1} give up talking nonsense. Their words are timely, true, and meaningful, in line with the teaching and training. They say things at the right time which are valuable, reasonable, succinct, and beneficial. 

These\marginnote{28.1} are the four kinds of successful verbal action. 

And\marginnote{29.1} what are the three kinds of successful mental action? It’s when someone is content. They don’t covet the wealth and belongings of others: ‘Oh, if only their belongings were mine!’ 

They\marginnote{30.1} have a kind heart and loving intentions: ‘May these sentient beings live free of enmity and ill will, untroubled and happy!’ 

They\marginnote{31.1} have right view, an undistorted perspective: ‘There is meaning in giving, sacrifice, and offerings. There are fruits and results of good and bad deeds. There is an afterlife. There are such things as mother and father, and beings that are reborn spontaneously. And there are ascetics and brahmins who are rightly comported and rightly practiced, and who describe the afterlife after realizing it with their own insight.’ 

These\marginnote{32.1} are the three kinds of successful mental action. 

When\marginnote{33.1} their body breaks up, after death, sentient beings are reborn in a good place, in heaven because of these three kinds of successful bodily action, these four kinds of successful verbal action, or these three kinds of successful mental action that have skillful intention, with happiness as their outcome and result. 

It’s\marginnote{34.1} like throwing unfaling dice: they always fall the right side up. In the same way, when their body breaks up, after death, sentient beings are reborn in a good place, in heaven because of these three kinds of successful bodily action, these four kinds of successful verbal action, or these three kinds of successful mental action that have skillful intention, with happiness as their outcome and result. I don’t say that intentional deeds that have been performed and accumulated are eliminated without being experienced. And that may be in this very life, or in the next life, or in some subsequent period. And I don’t say that suffering is ended without experiencing intentional deeds that have been performed and accumulated.” 

%
\section*{{\suttatitleacronym AN 10.218}{\suttatitletranslation Intentional (2nd) }{\suttatitleroot Dutiyasañcetanikasutta}}
\addcontentsline{toc}{section}{\tocacronym{AN 10.218} \toctranslation{Intentional (2nd) } \tocroot{Dutiyasañcetanikasutta}}
\markboth{Intentional (2nd) }{Dutiyasañcetanikasutta}
\extramarks{AN 10.218}{AN 10.218}

“Mendicants,\marginnote{1.1} I don’t say that intentional deeds that have been performed and accumulated are eliminated without being experienced. And that may be in this very life, or in the next life, or in some subsequent period. And I don’t say that suffering is ended without experiencing intentional deeds that have been performed and accumulated. 

Now,\marginnote{2.1} there are three kinds of corruption and failure of bodily action that have unskillful intention, with suffering as their outcome and result. There are four kinds of corruption and failure of verbal action that have unskillful intention, with suffering as their outcome and result. There are three kinds of corruption and failure of mental action that have unskillful intention, with suffering as their outcome and result. 

And\marginnote{3.1} what are the three kinds of corruption and failure of bodily action? … These are the three kinds of corruption and failure of bodily action. 

And\marginnote{4.1} what are the four kinds of corruption and failure of verbal action? … These are the four kinds of corruption and failure of verbal action. 

And\marginnote{5.1} what are the three kinds of corruption and failure of mental action? … These are the three kinds of corruption and failure of mental action. 

When\marginnote{6.1} their body breaks up, after death, sentient beings are reborn in a place of loss, a bad place, the underworld, hell because of these three kinds of corruption and failure of bodily action, these four kinds of corruption and failure of verbal action, or these three kinds of corruption and failure of mental action that have unskillful intention, with suffering as their outcome and result. 

I\marginnote{7.1} don’t say that intentional deeds that have been performed and accumulated are eliminated without being experienced. And that may be in this very life, or in the next life, or in some subsequent period. And I don’t say that suffering is ended without experiencing intentional deeds that have been performed and accumulated. 

Now,\marginnote{8.1} there are three kinds of successful bodily action that have skillful intention, with happiness as their outcome and result. There are four kinds of successful verbal action that have skillful intention, with happiness as their outcome and result. There are three kinds of successful mental action that have skillful intention, with happiness as their outcome and result. 

And\marginnote{9.1} what are the three kinds of successful bodily action? … These are the three kinds of successful bodily action. 

And\marginnote{10.1} what are the four kinds of successful verbal action? … These are the four kinds of successful verbal action. 

And\marginnote{11.1} what are the three kinds of successful mental action? … These are the three kinds of successful mental action. 

When\marginnote{12.1} their body breaks up, after death, sentient beings are reborn in a good place, in heaven because of these three kinds of successful bodily action, these four kinds of successful verbal action, or these three kinds of successful mental action that have skillful intention, with happiness as their outcome and result. …” 

%
\section*{{\suttatitleacronym AN 10.219}{\suttatitletranslation The Body Born of Deeds }{\suttatitleroot Karajakāyasutta}}
\addcontentsline{toc}{section}{\tocacronym{AN 10.219} \toctranslation{The Body Born of Deeds } \tocroot{Karajakāyasutta}}
\markboth{The Body Born of Deeds }{Karajakāyasutta}
\extramarks{AN 10.219}{AN 10.219}

“Mendicants,\marginnote{1.1} I don’t say that intentional deeds that have been performed and accumulated are eliminated without being experienced. And that may be in this very life, or in the next life, or in some subsequent period. And I don’t say that suffering is ended without experiencing intentional deeds that have been performed and accumulated. 

That\marginnote{2.1} noble disciple is rid of desire, rid of ill will, unconfused, aware, and mindful. They meditate spreading a heart full of love to one direction, and to the second, and to the third, and to the fourth. In the same way above, below, across, everywhere, all around, they spread a heart full of love to the whole world—abundant, expansive, limitless, free of enmity and ill will. 

They\marginnote{3.1} understand: ‘Formerly my mind was limited and undeveloped. Now it’s limitless and well developed. Whatever limited deeds I’ve done don’t remain or persist there.’ 

What\marginnote{4.1} do you think, mendicants? Suppose a child had developed the heart’s release by love from their childhood on. Would they still do any bad deed?” 

“No,\marginnote{4.3} sir.” 

“Not\marginnote{5.1} doing any bad deed, would they still experience any suffering?” 

“No,\marginnote{5.2} sir. For if they don’t do any bad deed, from where would suffering afflict them?” 

“This\marginnote{6.1} heart’s release by love should be developed by women or men. For neither women nor men take this body with them when they go. The mind is what’s inside mortal beings. They understand: ‘Whatever bad deeds I have done in the past with this deed-born body I will experience here. It will not follow me to my next life.’ The heart’s release by love developed in this way leads to non-return for a wise mendicant here who has not penetrated to a higher freedom. 

They\marginnote{7.1} meditate spreading a heart full of compassion … They meditate spreading a heart full of rejoicing … They meditate spreading a heart full of equanimity to one direction, and to the second, and to the third, and to the fourth. In the same way above, below, across, everywhere, all around, they spread a heart full of equanimity to the whole world—abundant, expansive, limitless, free of enmity and ill will. 

They\marginnote{8.1} understand: ‘Formerly my mind was limited and undeveloped. Now it’s limitless and well developed. Whatever limited deeds I’ve done don’t remain or persist there.’ 

What\marginnote{9.1} do you think, mendicants? Suppose a child had developed the heart’s release by equanimity from their childhood on. Would they still do any bad deed?” 

“No,\marginnote{9.3} sir.” 

“Not\marginnote{10.1} doing any bad deed, would they still experience any suffering?” 

“No,\marginnote{10.2} sir. For if they don’t do any bad deed, from where would suffering afflict them?” 

“This\marginnote{11.1} heart’s release by equanimity should be developed by women or men. For neither women nor men take this body with them when they go. The mind is what’s inside mortal beings. They understand: ‘Whatever bad deeds I have done in the past with this deed-born body I will experience here. It will not follow me to my next life.’ The heart’s release by equanimity developed in this way leads to non-return for a wise mendicant here who has not penetrated to a higher freedom.” 

%
\section*{{\suttatitleacronym AN 10.220}{\suttatitletranslation Unprincipled Conduct }{\suttatitleroot Adhammacariyāsutta}}
\addcontentsline{toc}{section}{\tocacronym{AN 10.220} \toctranslation{Unprincipled Conduct } \tocroot{Adhammacariyāsutta}}
\markboth{Unprincipled Conduct }{Adhammacariyāsutta}
\extramarks{AN 10.220}{AN 10.220}

Then\marginnote{1.1} a certain brahmin went up to the Buddha and exchanged greetings with him. When the greetings and polite conversation were over, he sat down to one side and said to the Buddha: 

“What\marginnote{1.3} is the cause, Mister Gotama, what is the reason why some sentient beings, when their body breaks up, after death, are reborn in a place of loss, a bad place, the underworld, hell?” 

“Unprincipled\marginnote{1.4} and immoral conduct is the reason why some sentient beings, when their body breaks up, after death, are reborn in a place of loss, a bad place, the underworld, hell.” 

“But\marginnote{2.1} what is the cause, Mister Gotama, what is the reason why some sentient beings, when their body breaks up, after death, are reborn in a good place, a heavenly realm?” 

“Principled\marginnote{2.2} and moral conduct is the reason why some sentient beings, when their body breaks up, after death, are reborn in a good place, a heavenly realm.” 

“I\marginnote{3.1} don’t understand the detailed meaning of what Mister Gotama has said in brief. Please, Mister Gotama, teach me this matter so I can understand the detailed meaning.” 

“Well\marginnote{3.3} then, brahmin, listen and apply your mind well, I will speak.” 

“Yes,\marginnote{3.4} sir,” the brahmin replied. The Buddha said this: 

“Brahmin,\marginnote{4.1} unprincipled and immoral conduct is threefold by way of body, fourfold by way of speech, and threefold by way of mind. 

And\marginnote{5.1} how is unprincipled and immoral conduct threefold by way of body? … That’s how unprincipled and immoral conduct is threefold by way of body. 

And\marginnote{6.1} how is unprincipled and immoral conduct fourfold by way of speech? … That’s how unprincipled and immoral conduct is fourfold by way of speech. 

And\marginnote{7.1} how is unprincipled and immoral conduct threefold by way of mind? … That’s how unprincipled and immoral conduct is threefold by way of mind. That’s how unprincipled and immoral conduct is the reason why some sentient beings, when their body breaks up, after death, are reborn in a place of loss, a bad place, the underworld, hell. 

Principled\marginnote{8.1} and moral conduct is threefold by way of body, fourfold by way of speech, and threefold by way of mind. 

And\marginnote{9.1} how is principled and moral conduct threefold by way of body? … That’s how principled and moral conduct is threefold by way of body. 

And\marginnote{10.1} how is principled and moral conduct fourfold by way of speech? … That’s how principled and moral conduct is fourfold by way of speech. 

And\marginnote{11.1} how is principled and moral conduct threefold by way of mind? … That’s how principled and moral conduct is threefold by way of mind. That’s how principled and moral conduct is the reason why some sentient beings, when their body breaks up, after death, are reborn in a good place, a heavenly realm.” 

“Excellent,\marginnote{12.1} Mister Gotama! Excellent! … From this day forth, may Mister Gotama remember me as a lay follower who has gone for refuge for life.” 

%
\addtocontents{toc}{\let\protect\contentsline\protect\nopagecontentsline}
\chapter*{The Chapter on Similarity }
\addcontentsline{toc}{chapter}{\tocchapterline{The Chapter on Similarity }}
\addtocontents{toc}{\let\protect\contentsline\protect\oldcontentsline}

%
\section*{{\suttatitleacronym AN 10.221}{\suttatitletranslation Untitled Discourse on Ten Qualities }{\suttatitleroot \textasciitilde }}
\addcontentsline{toc}{section}{\tocacronym{AN 10.221} \toctranslation{Untitled Discourse on Ten Qualities } \tocroot{\textasciitilde }}
\markboth{Untitled Discourse on Ten Qualities }{\textasciitilde }
\extramarks{AN 10.221}{AN 10.221}

“Someone\marginnote{1.1} with ten qualities is cast down to hell. What ten? They kill living creatures, steal, and commit sexual misconduct. They use speech that’s false, divisive, harsh, or nonsensical. And they’re covetous, malicious, with wrong view. Someone with these ten qualities is cast down to hell. 

Someone\marginnote{2.1} with ten qualities is raised up to heaven. What ten? They don’t kill living creatures, steal, or commit sexual misconduct. They don’t use speech that’s false, divisive, harsh, or nonsensical. They’re contented, kind-hearted, with right view. Someone with these ten qualities is raised up to heaven.” 

%
\section*{{\suttatitleacronym AN 10.222}{\suttatitletranslation Untitled Discourse on Twenty Qualities }{\suttatitleroot \textasciitilde }}
\addcontentsline{toc}{section}{\tocacronym{AN 10.222} \toctranslation{Untitled Discourse on Twenty Qualities } \tocroot{\textasciitilde }}
\markboth{Untitled Discourse on Twenty Qualities }{\textasciitilde }
\extramarks{AN 10.222}{AN 10.222}

“Someone\marginnote{1.1} with twenty qualities is cast down to hell. What twenty? They kill living creatures, steal, and commit sexual misconduct. They use speech that’s false, divisive, harsh, or nonsensical. They’re covetous, malicious, with wrong view. And they encourage others to do these things. Someone with these twenty qualities is cast down to hell. 

Someone\marginnote{2.1} with twenty qualities is raised up to heaven. What twenty? They don’t kill living creatures, steal, or commit sexual misconduct. They don’t use speech that’s false, divisive, harsh, or nonsensical. They’re contented, kind-hearted, with right view. And they encourage others to do these things. Someone with these twenty qualities is raised up to heaven.” 

%
\section*{{\suttatitleacronym AN 10.223}{\suttatitletranslation Untitled Discourse on Thirty Qualities }{\suttatitleroot \textasciitilde }}
\addcontentsline{toc}{section}{\tocacronym{AN 10.223} \toctranslation{Untitled Discourse on Thirty Qualities } \tocroot{\textasciitilde }}
\markboth{Untitled Discourse on Thirty Qualities }{\textasciitilde }
\extramarks{AN 10.223}{AN 10.223}

“Someone\marginnote{1.1} with thirty qualities is cast down to hell. What thirty? They kill living creatures, steal, and commit sexual misconduct. They use speech that’s false, divisive, harsh, or nonsensical. They’re covetous, malicious, with wrong view. They encourage others to do these things. And they approve of these things. Someone with these thirty qualities is cast down to hell. 

Someone\marginnote{2.1} with thirty qualities is raised up to heaven. What thirty? They don’t kill living creatures, steal, or commit sexual misconduct. They don’t use speech that’s false, divisive, harsh, or nonsensical. They’re contented, kind-hearted, with right view. They encourage others to do these things. And they approve of these things. Someone with these thirty qualities is raised up to heaven.” 

%
\section*{{\suttatitleacronym AN 10.224}{\suttatitletranslation Untitled Discourse on Forty Qualities }{\suttatitleroot \textasciitilde }}
\addcontentsline{toc}{section}{\tocacronym{AN 10.224} \toctranslation{Untitled Discourse on Forty Qualities } \tocroot{\textasciitilde }}
\markboth{Untitled Discourse on Forty Qualities }{\textasciitilde }
\extramarks{AN 10.224}{AN 10.224}

“Someone\marginnote{1.1} with forty qualities is cast down to hell. What forty? They kill living creatures, steal, and commit sexual misconduct. They use speech that’s false, divisive, harsh, or nonsensical. They’re covetous, malicious, with wrong view. They encourage others to do these things. They approve of these things. And they praise these things. Someone with these forty qualities is cast down to hell. 

Someone\marginnote{2.1} with forty qualities is raised up to heaven. What forty? They don’t kill living creatures, steal, or commit sexual misconduct. They don’t use speech that’s false, divisive, harsh, or nonsensical. They’re contented, kind-hearted, with right view. They encourage others to do these things. They approve of these things. And they praise these things. Someone with these forty qualities is raised up to heaven.” 

%
\section*{{\suttatitleacronym AN 10.225–228}{\suttatitletranslation Untitled Discourses on Ten to Forty Qualities (1st) }{\suttatitleroot \textasciitilde }}
\addcontentsline{toc}{section}{\tocacronym{AN 10.225–228} \toctranslation{Untitled Discourses on Ten to Forty Qualities (1st) } \tocroot{\textasciitilde }}
\markboth{Untitled Discourses on Ten to Forty Qualities (1st) }{\textasciitilde }
\extramarks{AN 10.225–228}{AN 10.225–228}

“Someone\marginnote{1.1} with ten qualities keeps themselves broken and damaged … keeps themselves intact and unscathed … twenty … thirty … forty …” 

%
\section*{{\suttatitleacronym AN 10.229–232}{\suttatitletranslation Untitled Discourses on Ten to Forty Qualities (2nd) }{\suttatitleroot \textasciitilde }}
\addcontentsline{toc}{section}{\tocacronym{AN 10.229–232} \toctranslation{Untitled Discourses on Ten to Forty Qualities (2nd) } \tocroot{\textasciitilde }}
\markboth{Untitled Discourses on Ten to Forty Qualities (2nd) }{\textasciitilde }
\extramarks{AN 10.229–232}{AN 10.229–232}

“When\marginnote{1.1} they have ten qualities, some people, when their body breaks up, after death, are reborn in a place of loss, a bad place, the underworld, hell. … some people, when their body breaks up, after death, are reborn in a good place, a heavenly realm … twenty … thirty … forty …” 

%
\section*{{\suttatitleacronym AN 10.233–236}{\suttatitletranslation Untitled Discourses on Ten to Forty Qualities (3rd) }{\suttatitleroot \textasciitilde }}
\addcontentsline{toc}{section}{\tocacronym{AN 10.233–236} \toctranslation{Untitled Discourses on Ten to Forty Qualities (3rd) } \tocroot{\textasciitilde }}
\markboth{Untitled Discourses on Ten to Forty Qualities (3rd) }{\textasciitilde }
\extramarks{AN 10.233–236}{AN 10.233–236}

“A\marginnote{1.1} fool is known by ten qualities … astute person … twenty … thirty … forty …” 

%
\addtocontents{toc}{\let\protect\contentsline\protect\nopagecontentsline}
\chapter*{Abbreviated Texts Beginning With Greed }
\addcontentsline{toc}{chapter}{\tocchapterline{Abbreviated Texts Beginning With Greed }}
\addtocontents{toc}{\let\protect\contentsline\protect\oldcontentsline}

%
\section*{{\suttatitleacronym AN 10.237}{\suttatitletranslation Untitled Discourse on Greed (1st) }{\suttatitleroot \textasciitilde }}
\addcontentsline{toc}{section}{\tocacronym{AN 10.237} \toctranslation{Untitled Discourse on Greed (1st) } \tocroot{\textasciitilde }}
\markboth{Untitled Discourse on Greed (1st) }{\textasciitilde }
\extramarks{AN 10.237}{AN 10.237}

“For\marginnote{1.1} insight into greed, ten things should be developed. What ten? The perceptions of ugliness, death, repulsiveness of food, dissatisfaction with the whole world, impermanence, suffering in impermanence, and not-self in suffering, giving up, fading away, and cessation. For insight into greed, these ten things should be developed.” 

%
\section*{{\suttatitleacronym AN 10.238}{\suttatitletranslation Untitled Discourse on Greed (2nd) }{\suttatitleroot \textasciitilde }}
\addcontentsline{toc}{section}{\tocacronym{AN 10.238} \toctranslation{Untitled Discourse on Greed (2nd) } \tocroot{\textasciitilde }}
\markboth{Untitled Discourse on Greed (2nd) }{\textasciitilde }
\extramarks{AN 10.238}{AN 10.238}

“For\marginnote{1.1} insight into greed, ten things should be developed. What ten? The perceptions of impermanence, not-self, death, repulsiveness of food, dissatisfaction with the whole world, a skeleton, a worm-infested corpse, a livid corpse, a split open corpse, and a bloated corpse. For insight into greed, these ten things should be developed.” 

%
\section*{{\suttatitleacronym AN 10.239}{\suttatitletranslation Untitled Discourse on Greed (3rd) }{\suttatitleroot \textasciitilde }}
\addcontentsline{toc}{section}{\tocacronym{AN 10.239} \toctranslation{Untitled Discourse on Greed (3rd) } \tocroot{\textasciitilde }}
\markboth{Untitled Discourse on Greed (3rd) }{\textasciitilde }
\extramarks{AN 10.239}{AN 10.239}

“For\marginnote{1.1} insight into greed, ten things should be developed. What ten? Right view, right thought, right speech, right action, right livelihood, right effort, right mindfulness, right immersion, right knowledge, and right freedom. For insight into greed, these ten things should be developed.” 

%
\section*{{\suttatitleacronym AN 10.240–266}{\suttatitletranslation Untitled Discourses on Greed }{\suttatitleroot \textasciitilde }}
\addcontentsline{toc}{section}{\tocacronym{AN 10.240–266} \toctranslation{Untitled Discourses on Greed } \tocroot{\textasciitilde }}
\markboth{Untitled Discourses on Greed }{\textasciitilde }
\extramarks{AN 10.240–266}{AN 10.240–266}

“For\marginnote{1.1} the complete understanding of greed … complete ending … giving up … ending … vanishing … fading away … cessation … giving away … letting go … these ten things should be developed.” 

%
\section*{{\suttatitleacronym AN 10.267–746}{\suttatitletranslation Untitled Discourses on Hate, Etc. }{\suttatitleroot \textasciitilde }}
\addcontentsline{toc}{section}{\tocacronym{AN 10.267–746} \toctranslation{Untitled Discourses on Hate, Etc. } \tocroot{\textasciitilde }}
\markboth{Untitled Discourses on Hate, Etc. }{\textasciitilde }
\extramarks{AN 10.267–746}{AN 10.267–746}

“Of\marginnote{1.1} hate … delusion … anger … acrimony … disdain … contempt … jealousy … stinginess … deceitfulness … deviousness … obstinacy … aggression … conceit … arrogance … vanity … for the complete understanding of negligence … complete ending … giving up … ending … vanishing … fading away … cessation … giving away … letting go of negligence … these ten things should be developed.” 

\scendbook{The Book of the Tens is finished. }

%
\addtocontents{toc}{\let\protect\contentsline\protect\nopagecontentsline}
\part*{The Book of the Elevens }
\addcontentsline{toc}{part}{The Book of the Elevens }
\markboth{}{}
\addtocontents{toc}{\let\protect\contentsline\protect\oldcontentsline}

%
%
\addtocontents{toc}{\let\protect\contentsline\protect\nopagecontentsline}
\pannasa{The First Fifty }
\addcontentsline{toc}{pannasa}{The First Fifty }
\markboth{}{}
\addtocontents{toc}{\let\protect\contentsline\protect\oldcontentsline}

%
\addtocontents{toc}{\let\protect\contentsline\protect\nopagecontentsline}
\chapter*{The Chapter on Dependence }
\addcontentsline{toc}{chapter}{\tocchapterline{The Chapter on Dependence }}
\addtocontents{toc}{\let\protect\contentsline\protect\oldcontentsline}

%
\section*{{\suttatitleacronym AN 11.1}{\suttatitletranslation What’s the Purpose? }{\suttatitleroot Kimatthiyasutta}}
\addcontentsline{toc}{section}{\tocacronym{AN 11.1} \toctranslation{What’s the Purpose? } \tocroot{Kimatthiyasutta}}
\markboth{What’s the Purpose? }{Kimatthiyasutta}
\extramarks{AN 11.1}{AN 11.1}

\scevam{So\marginnote{1.1} I have heard. }At one time the Buddha was staying near \textsanskrit{Sāvatthī} in Jeta’s Grove, \textsanskrit{Anāthapiṇḍika}’s monastery. 

Then\marginnote{1.3} Venerable Ānanda went up to the Buddha, bowed, sat down to one side, and said to him: 

“Sir,\marginnote{1.4} what is the purpose and benefit of skillful ethics?” 

“Ānanda,\marginnote{1.5} having no regrets is the purpose and benefit of skillful ethics.” 

“But\marginnote{2.1} what is the purpose and benefit of having no regrets?” 

“Joy\marginnote{2.2} is the purpose and benefit of having no regrets.” 

“But\marginnote{3.1} what is the purpose and benefit of joy?” 

“Rapture\marginnote{3.2} …” 

“But\marginnote{4.1} what is the purpose and benefit of rapture?” 

“Tranquility\marginnote{4.2} …” 

“But\marginnote{5.1} what is the purpose and benefit of tranquility?” 

“Bliss\marginnote{5.2} …” 

“But\marginnote{6.1} what is the purpose and benefit of bliss?” 

“Immersion\marginnote{6.2} …” 

“But\marginnote{7.1} what is the purpose and benefit of immersion?” 

“Truly\marginnote{7.2} knowing and seeing …” 

“But\marginnote{8.1} what is the purpose and benefit of truly knowing and seeing?” 

“Disillusionment\marginnote{8.2} …” 

“But\marginnote{9.1} what is the purpose and benefit of disillusionment?” 

“Dispassion\marginnote{9.2} …” 

“But\marginnote{10.1} what is the purpose and benefit of dispassion?” 

“Knowledge\marginnote{10.2} and vision of freedom is the purpose and benefit of dispassion. 

So,\marginnote{11.1} Ānanda, the purpose and benefit of skillful ethics is not having regrets. Joy is the purpose and benefit of not having regrets. Rapture is the purpose and benefit of joy. Tranquility is the purpose and benefit of rapture. Bliss is the purpose and benefit of tranquility. Immersion is the purpose and benefit of bliss. Truly knowing and seeing is the purpose and benefit of immersion. Disillusionment is the purpose and benefit of truly knowing and seeing. Dispassion is the purpose and benefit of disillusionment. And knowledge and vision of freedom is the purpose and benefit of dispassion. So, Ānanda, skillful ethics progressively lead up to the highest.” 

%
\section*{{\suttatitleacronym AN 11.2}{\suttatitletranslation Making a Wish }{\suttatitleroot Cetanākaraṇīyasutta}}
\addcontentsline{toc}{section}{\tocacronym{AN 11.2} \toctranslation{Making a Wish } \tocroot{Cetanākaraṇīyasutta}}
\markboth{Making a Wish }{Cetanākaraṇīyasutta}
\extramarks{AN 11.2}{AN 11.2}

“Mendicants,\marginnote{1.1} an ethical person, who has fulfilled ethical conduct, need not make a wish: ‘May I have no regrets!’ It’s only natural that an ethical person has no regrets. 

When\marginnote{2.1} you have no regrets you need not make a wish: ‘May I feel joy!’ It’s only natural that joy springs up when you have no regrets. 

When\marginnote{3.1} you feel joy you need not make a wish: ‘May I experience rapture!’ It’s only natural that rapture arises when you’re joyful. 

When\marginnote{4.1} your mind is full of rapture you need not make a wish: ‘May my body become tranquil!’ It’s only natural that your body becomes tranquil when your mind is full of rapture. 

When\marginnote{5.1} your body is tranquil you need not make a wish: ‘May I feel bliss!’ It’s only natural to feel bliss when your body is tranquil. 

When\marginnote{6.1} you feel bliss you need not make a wish: ‘May my mind be immersed in \textsanskrit{samādhi}!’ It’s only natural for the mind to become immersed in \textsanskrit{samādhi} when you feel bliss. 

When\marginnote{7.1} your mind is immersed in \textsanskrit{samādhi} you need not make a wish: ‘May I truly know and see!’ It’s only natural to truly know and see when your mind is immersed in \textsanskrit{samādhi}. 

When\marginnote{8.1} you truly know and see you need not make a wish: ‘May I grow disillusioned!’ It’s only natural to grow disillusioned when you truly know and see. 

When\marginnote{9.1} you’re disillusioned you need not make a wish: ‘May I become dispassionate!’ It’s only natural to grow dispassionate when you’re disillusioned. 

When\marginnote{10.1} you’re dispassionate you need not make a wish: ‘May I realize the knowledge and vision of freedom!’ It’s only natural to realize the knowledge and vision of freedom when you’re dispassionate. 

And\marginnote{11.1} so, mendicants, the knowledge and vision of freedom is the purpose and benefit of dispassion. Dispassion is the purpose and benefit of disillusionment. Disillusionment is the purpose and benefit of truly knowing and seeing. Truly knowing and seeing is the purpose and benefit of immersion. Immersion is the purpose and benefit of bliss. Bliss is the purpose and benefit of tranquility. Tranquility is the purpose and benefit of rapture. Rapture is the purpose and benefit of joy. Joy is the purpose and benefit of not having regrets. Not having regrets is the purpose and benefit of skillful ethics. And so, mendicants, good qualities flow on and fill up from one to the other, for going from the near shore to the far shore.” 

%
\section*{{\suttatitleacronym AN 11.3}{\suttatitletranslation Vital Conditions (1st) }{\suttatitleroot Paṭhamaupanisāsutta}}
\addcontentsline{toc}{section}{\tocacronym{AN 11.3} \toctranslation{Vital Conditions (1st) } \tocroot{Paṭhamaupanisāsutta}}
\markboth{Vital Conditions (1st) }{Paṭhamaupanisāsutta}
\extramarks{AN 11.3}{AN 11.3}

“Mendicants,\marginnote{1.1} an unethical person, who lacks ethics, has destroyed a vital condition for having no regrets. When there are regrets, one who has regrets has destroyed a vital condition for joy. When there is no joy, one who lacks joy has destroyed a vital condition for rapture. When there is no rapture, one who lacks rapture has destroyed a vital condition for tranquility. When there is no tranquility, one who lacks tranquility has destroyed a vital condition for bliss. When there is no bliss, one who lacks bliss has destroyed a vital condition for right immersion. When there is no right immersion, one who lacks right immersion has destroyed a vital condition for true knowledge and vision. When there is no true knowledge and vision, one who lacks true knowledge and vision has destroyed a vital condition for disillusionment. When there is no disillusionment, one who lacks disillusionment has destroyed a vital condition for dispassion. When there is no dispassion, one who lacks dispassion has destroyed a vital condition for knowledge and vision of freedom. 

Suppose\marginnote{2.1} there was a tree that lacked branches and foliage. Its shoots, bark, softwood, and heartwood would not grow to fullness. 

In\marginnote{2.2} the same way, an unethical person, who lacks ethics, has destroyed a vital condition for having no regrets. When there are regrets, one who has regrets has destroyed a vital condition for joy. … When there is no dispassion, one who lacks dispassion has destroyed a vital condition for knowledge and vision of freedom. 

An\marginnote{3.1} ethical person, who has fulfilled ethics, has fulfilled a vital condition for not having regrets. When there are no regrets, one who has no regrets has fulfilled a vital condition for joy. When there is joy, one who has fulfilled joy has fulfilled a vital condition for rapture. When there is rapture, one who has fulfilled rapture has fulfilled a vital condition for tranquility. When there is tranquility, one who has fulfilled tranquility has fulfilled a vital condition for bliss. When there is bliss, one who has fulfilled bliss has fulfilled a vital condition for right immersion. When there is right immersion, one who has fulfilled right immersion has fulfilled a vital condition for true knowledge and vision. When there is true knowledge and vision, one who has fulfilled true knowledge and vision has fulfilled a vital condition for disillusionment. When there is disillusionment, one who has fulfilled disillusionment has fulfilled a vital condition for dispassion. When there is dispassion, one who has fulfilled dispassion has fulfilled a vital condition for knowledge and vision of freedom. 

Suppose\marginnote{4.1} there was a tree that was complete with branches and foliage. Its shoots, bark, softwood, and heartwood would grow to fullness. 

In\marginnote{4.2} the same way, an ethical person, who has fulfilled ethics, has fulfilled a vital condition for not having regrets. When there are no regrets, one who has no regrets has fulfilled a vital condition for joy. … When there is dispassion, one who has fulfilled dispassion has fulfilled a vital condition for knowledge and vision of freedom.” 

%
\section*{{\suttatitleacronym AN 11.4}{\suttatitletranslation Vital Conditions (2nd) }{\suttatitleroot Dutiyaupanisāsutta}}
\addcontentsline{toc}{section}{\tocacronym{AN 11.4} \toctranslation{Vital Conditions (2nd) } \tocroot{Dutiyaupanisāsutta}}
\markboth{Vital Conditions (2nd) }{Dutiyaupanisāsutta}
\extramarks{AN 11.4}{AN 11.4}

There\marginnote{1.1} Venerable \textsanskrit{Sāriputta} addressed the mendicants: “Reverends, mendicants!” 

“Reverend,”\marginnote{1.3} they replied. \textsanskrit{Sāriputta} said this: 

“An\marginnote{2.1} unethical person, who lacks ethics, has destroyed a vital condition for not having regrets. When there are regrets, one who has regrets has destroyed a vital condition for joy. When there is no joy, one who lacks joy has destroyed a vital condition for rapture. When there is no rapture, one who lacks rapture has destroyed a vital condition for tranquility. When there is no tranquility, one who lacks tranquility has destroyed a vital condition for bliss. When there is no bliss, one who lacks bliss has destroyed a vital condition for right immersion. When there is no right immersion, one who lacks right immersion has destroyed a vital condition for true knowledge and vision. When there is no true knowledge and vision, one who lacks true knowledge and vision has destroyed a vital condition for disillusionment. When there is no disillusionment, one who lacks disillusionment has destroyed a vital condition for dispassion. When there is no dispassion, one who lacks dispassion has destroyed a vital condition for knowledge and vision of freedom. 

Suppose\marginnote{3.1} there was a tree that lacked branches and foliage. Its shoots, bark, softwood, and heartwood would not grow to fullness. 

In\marginnote{3.2} the same way, an unethical person, who lacks ethics, has destroyed a vital condition for having no regrets. When there are regrets, one who has regrets has destroyed a vital condition for joy. … When there is dispassion, one who lacks dispassion has destroyed a vital condition for knowledge and vision of freedom. 

An\marginnote{4.1} ethical person, who has fulfilled ethics, has fulfilled a vital condition for not having regrets. When there are no regrets, one who has no regrets has fulfilled a vital condition for joy. When there is joy, one who has fulfilled joy has fulfilled a vital condition for rapture. When there is rapture, one who has fulfilled rapture has fulfilled a vital condition for tranquility. When there is tranquility, one who has fulfilled tranquility has fulfilled a vital condition for bliss. When there is bliss, one who has fulfilled bliss has fulfilled a vital condition for right immersion. When there is right immersion, one who has fulfilled right immersion has fulfilled a vital condition for true knowledge and vision. When there is true knowledge and vision, one who has fulfilled true knowledge and vision has fulfilled a vital condition for disillusionment. When there is disillusionment, one who has fulfilled disillusionment has fulfilled a vital condition for dispassion. When there is dispassion, one who has fulfilled dispassion has fulfilled a vital condition for knowledge and vision of freedom. 

Suppose\marginnote{5.1} there was a tree that was complete with branches and foliage. Its shoots, bark, softwood, and heartwood would grow to fullness. 

In\marginnote{5.2} the same way, an ethical person, who has fulfilled ethics, has fulfilled a vital condition for not having regrets. When there are no regrets, one who has no regrets has fulfilled a vital condition for joy. … When there is dispassion, one who has fulfilled dispassion has fulfilled a vital condition for knowledge and vision of freedom.” 

%
\section*{{\suttatitleacronym AN 11.5}{\suttatitletranslation Vital Conditions (3rd) }{\suttatitleroot Tatiyaupanisāsutta}}
\addcontentsline{toc}{section}{\tocacronym{AN 11.5} \toctranslation{Vital Conditions (3rd) } \tocroot{Tatiyaupanisāsutta}}
\markboth{Vital Conditions (3rd) }{Tatiyaupanisāsutta}
\extramarks{AN 11.5}{AN 11.5}

There\marginnote{1.1} Venerable Ānanda addressed the mendicants … 

“An\marginnote{1.2} unethical person, who lacks ethics, has destroyed a vital condition for not having regrets. When there are regrets, one who has regrets has destroyed a vital condition for joy. When there is no joy, one who lacks joy has destroyed a vital condition for rapture. When there is no rapture, one who lacks rapture has destroyed a vital condition for tranquility. When there is no tranquility, one who lacks tranquility has destroyed a vital condition for bliss. When there is no bliss, one who lacks bliss has destroyed a vital condition for right immersion. When there is no right immersion, one who lacks right immersion has destroyed a vital condition for true knowledge and vision. When there is no true knowledge and vision, one who lacks true knowledge and vision has destroyed a vital condition for disillusionment. When there is no disillusionment, one who lacks disillusionment has destroyed a vital condition for dispassion. When there is no dispassion, one who lacks dispassion has destroyed a vital condition for knowledge and vision of freedom. 

Suppose\marginnote{2.1} there was a tree that lacked branches and foliage. Its shoots, bark, softwood, and heartwood would not grow to fullness. 

In\marginnote{2.2} the same way, an unethical person, who lacks ethics, has destroyed a vital condition for having no regrets. When there are regrets, one who has regrets has destroyed a vital condition for joy. … When there is no dispassion, one who lacks dispassion has destroyed a vital condition for knowledge and vision of freedom. 

An\marginnote{3.1} ethical person, who has fulfilled ethics, has fulfilled a vital condition for not having regrets. When there are no regrets, one who has no regrets has fulfilled a vital condition for joy. When there is joy, one who has fulfilled joy has fulfilled a vital condition for rapture. When there is rapture, one who has fulfilled rapture has fulfilled a vital condition for tranquility. When there is tranquility, one who has fulfilled tranquility has fulfilled a vital condition for bliss. When there is bliss, one who has fulfilled bliss has fulfilled a vital condition for right immersion. When there is right immersion, one who has fulfilled right immersion has fulfilled a vital condition for true knowledge and vision. When there is true knowledge and vision, one who has fulfilled true knowledge and vision has fulfilled a vital condition for disillusionment. When there is disillusionment, one who has fulfilled disillusionment has fulfilled a vital condition for dispassion. When there is dispassion, one who has fulfilled dispassion has fulfilled a vital condition for knowledge and vision of freedom. 

Suppose\marginnote{4.1} there was a tree that was complete with branches and foliage. Its shoots, bark, softwood, and heartwood would grow to fullness. 

In\marginnote{4.2} the same way, an ethical person, who has fulfilled ethics, has fulfilled a vital condition for not having regrets. When there are no regrets, one who has no regrets has fulfilled a vital condition for joy. … When there is dispassion, one who has fulfilled dispassion has fulfilled a vital condition for knowledge and vision of freedom.” 

%
\section*{{\suttatitleacronym AN 11.6}{\suttatitletranslation Disasters }{\suttatitleroot Byasanasutta}}
\addcontentsline{toc}{section}{\tocacronym{AN 11.6} \toctranslation{Disasters } \tocroot{Byasanasutta}}
\markboth{Disasters }{Byasanasutta}
\extramarks{AN 11.6}{AN 11.6}

“Mendicants,\marginnote{1.1} any mendicant who abuses and insults their spiritual companions, denouncing the noble ones, will, without a doubt, fall into one or other of these eleven disasters. What eleven? 

They\marginnote{2.2} don’t achieve the unachieved. What they have achieved falls away. They don’t refine their good qualities. They overestimate their good qualities. Or they lead the spiritual life dissatisfied. Or they commit a corrupt offense. Or they resign the training and return to a lesser life. Or they contract a severe illness. Or they go mad and lose their mind. They feel lost when they die. And when their body breaks up, after death, they are reborn in a place of loss, a bad place, the underworld, hell. 

Any\marginnote{2.13} mendicant who abuses and insults their spiritual companions, denouncing the noble ones, will, without a doubt, fall into one or other of these eleven disasters. 

Any\marginnote{3.1} mendicant who does not abuse and insult their spiritual companions, denouncing the noble ones, will, without a doubt, not fall into one or other of these eleven disasters. What eleven? 

They\marginnote{4.2} don’t achieve the unachieved. What they have achieved falls away. They don’t refine their good qualities. They overestimate their good qualities. Or they lead the spiritual life dissatisfied. Or they commit one of the corrupt offenses. Or they resign the training and return to a lesser life. Or they contract a severe illness. Or they go mad and lose their mind. They feel lost when they die. And when their body breaks up, after death, they are reborn in a place of loss, a bad place, the underworld, hell. 

Any\marginnote{4.13} mendicant who does not abuse and insult their spiritual companions, denouncing the noble ones, will, without a doubt, not fall into one or other of these eleven disasters.” 

%
\section*{{\suttatitleacronym AN 11.7}{\suttatitletranslation Percipient }{\suttatitleroot Saññāsutta}}
\addcontentsline{toc}{section}{\tocacronym{AN 11.7} \toctranslation{Percipient } \tocroot{Saññāsutta}}
\markboth{Percipient }{Saññāsutta}
\extramarks{AN 11.7}{AN 11.7}

Then\marginnote{1.1} Venerable Ānanda went up to the Buddha, bowed, sat down to one side, and said to him: 

“Could\marginnote{2.1} it be, sir, that a mendicant might gain a state of immersion like this? They wouldn’t perceive earth in earth, water in water, fire in fire, or air in air. And they wouldn’t perceive the dimension of infinite space in the dimension of infinite space, the dimension of infinite consciousness in the dimension of infinite consciousness, the dimension of nothingness in the dimension of nothingness, or the dimension of neither perception nor non-perception in the dimension of neither perception nor non-perception. They wouldn’t perceive this world in this world, or the other world in the other world. And they wouldn’t perceive what is seen, heard, thought, known, attained, sought, or explored by the mind. And yet they would still perceive.” 

“It\marginnote{3.1} could be, Ānanda, that a mendicant might gain a state of immersion like this. They wouldn’t perceive earth in earth, water in water, fire in fire, or air in air. And they wouldn’t perceive the dimension of infinite space in the dimension of infinite space, the dimension of infinite consciousness in the dimension of infinite consciousness, the dimension of nothingness in the dimension of nothingness, or the dimension of neither perception nor non-perception in the dimension of neither perception nor non-perception. They wouldn’t perceive this world in this world, or the other world in the other world. And they wouldn’t perceive what is seen, heard, thought, known, attained, sought, or explored by the mind. And yet they would still perceive.” 

“But\marginnote{4.1} how could this be, sir?” 

“Ānanda,\marginnote{5.1} it’s when a mendicant perceives: ‘This is peaceful; this is sublime—that is, the stilling of all activities, the letting go of all attachments, the ending of craving, fading away, cessation, extinguishment.’ 

That’s\marginnote{5.3} how a mendicant might gain a state of immersion like this. They wouldn’t perceive earth in earth, water in water, fire in fire, or air in air. And they wouldn’t perceive the dimension of infinite space in the dimension of infinite space, the dimension of infinite consciousness in the dimension of infinite consciousness, the dimension of nothingness in the dimension of nothingness, or the dimension of neither perception nor non-perception in the dimension of neither perception nor non-perception. They wouldn’t perceive this world in this world, or the other world in the other world. And they wouldn’t perceive what is seen, heard, thought, known, attained, sought, or explored by the mind. And yet they would still perceive.” 

And\marginnote{6.1} then Ānanda approved and agreed with what the Buddha said. He got up from his seat, bowed, and respectfully circled the Buddha, keeping him on his right. Then he went up to Venerable \textsanskrit{Sāriputta}, and exchanged greetings with him. When the greetings and polite conversation were over, he sat down to one side and said to \textsanskrit{Sāriputta}: 

“Could\marginnote{7.1} it be, reverend \textsanskrit{Sāriputta}, that a mendicant might gain a state of immersion like this? They wouldn’t perceive earth in earth … And they wouldn’t perceive what is seen, heard, thought, known, attained, sought, or explored by the mind. And yet they would still perceive.” 

“It\marginnote{7.2} could be, Reverend Ānanda.” 

“But\marginnote{8.1} how could this be?” 

“Ānanda,\marginnote{9.1} it’s when a mendicant perceives: ‘This is peaceful; this is sublime—that is, the stilling of all activities, the letting go of all attachments, the ending of craving, fading away, cessation, extinguishment.’ That’s how a mendicant might gain a state of immersion like this. They wouldn’t perceive earth in earth … And they wouldn’t perceive what is seen, heard, thought, known, attained, sought, or explored by the mind. And yet they would still perceive.” 

“It’s\marginnote{10.1} incredible, it’s amazing! How the meaning and the phrasing of the teacher and the disciple fit together and agree without conflict when it comes to the chief matter! Just now I went to the Buddha and asked him about this matter. And the Buddha explained it to me in this manner, with these words and phrases, just like Venerable \textsanskrit{Sāriputta}. It’s incredible, it’s amazing! How the meaning and the phrasing of the teacher and the disciple fit together and agree without conflict when it comes to the chief matter!” 

%
\section*{{\suttatitleacronym AN 11.8}{\suttatitletranslation Focus }{\suttatitleroot Manasikārasutta}}
\addcontentsline{toc}{section}{\tocacronym{AN 11.8} \toctranslation{Focus } \tocroot{Manasikārasutta}}
\markboth{Focus }{Manasikārasutta}
\extramarks{AN 11.8}{AN 11.8}

Then\marginnote{1.1} Venerable Ānanda went up to the Buddha, bowed, sat down to one side, and said to him: 

“Could\marginnote{2.1} it be, sir, that a mendicant might gain a state of immersion like this. They wouldn’t focus on the eye or sights, ear or sounds, nose or smells, tongue or tastes, or body or touches. They wouldn’t focus on earth in earth, water in water, fire in fire, or air in air. And they wouldn’t focus on the dimension of infinite space in the dimension of infinite space, the dimension of infinite consciousness in the dimension of infinite consciousness, the dimension of nothingness in the dimension of nothingness, or the dimension of neither perception nor non-perception in the dimension of neither perception nor non-perception. They wouldn’t focus on this world in this world, or the other world in the other world. And they wouldn’t focus on what is seen, heard, thought, known, attained, sought, or explored by the mind. Yet they would focus?” 

“It\marginnote{3.1} could be, Ānanda.” 

“But\marginnote{4.1} how could this be?” 

“Ānanda,\marginnote{5.1} it’s when a mendicant focuses thus: ‘This is peaceful; this is sublime—that is, the stilling of all activities, the letting go of all attachments, the ending of craving, fading away, cessation, extinguishment.’ 

That’s\marginnote{5.3} how a mendicant might gain a state of immersion like this. They wouldn’t focus on the eye or sights, ear or sounds, nose or smells, tongue or tastes, or body or touches. … And they wouldn’t focus on what is seen, heard, thought, known, attained, sought, or explored by the mind. Yet they would focus.” 

%
\section*{{\suttatitleacronym AN 11.9}{\suttatitletranslation With Sandha }{\suttatitleroot Saddhasutta}}
\addcontentsline{toc}{section}{\tocacronym{AN 11.9} \toctranslation{With Sandha } \tocroot{Saddhasutta}}
\markboth{With Sandha }{Saddhasutta}
\extramarks{AN 11.9}{AN 11.9}

At\marginnote{1.1} one time the Buddha was staying at \textsanskrit{Ñātika} in the brick house. 

Then\marginnote{1.2} Venerable Sandha went up to the Buddha, bowed, and sat down to one side. The Buddha said to him: 

“Sandha,\marginnote{2.1} meditate like a thoroughbred, not like a wild colt. 

And\marginnote{2.3} how does a wild colt meditate? A wild colt, tied up by the feeding trough, meditates: ‘Fodder, fodder!’ Why is that? Because it doesn’t occur to the wild colt tied up by the feeding trough: ‘What task will the horse trainer have me do today? How should I respond?’ Tied up by the feeding trough they just meditate: ‘Fodder, fodder!’ 

In\marginnote{2.9} the same way, take a certain wild person who has gone to the forest, the root of a tree, or an empty hut. Their heart is overcome and mired in sensual desire, and they don’t truly understand the escape from sensual desire that has arisen. Harboring sensual desire within they meditate and concentrate and contemplate and ruminate. Their heart is overcome by ill will … dullness and drowsiness … restlessness and remorse … doubt … Harboring doubt within they meditate and concentrate and contemplate and ruminate. They meditate dependent on earth, water, fire, and air. They meditate dependent on the dimension of infinite space, infinite consciousness, nothingness, or neither perception nor non-perception. They meditate dependent on this world or the other world. They meditate dependent on what is seen, heard, thought, known, attained, sought, or explored by the mind. That’s how a wild colt meditates. 

And\marginnote{3.1} how does a thoroughbred meditate? A fine thoroughbred, tied up by the feeding trough, doesn’t meditate: ‘Fodder, fodder!’ Why is that? Because it occurs to the fine thoroughbred tied up by the feeding trough: ‘What task will the horse trainer have me do today? How should I respond?’ Tied up by the feeding trough they don’t meditate: ‘Fodder, fodder!’ For that fine thoroughbred regards the use of the goad as a debt, a bond, a loss, a misfortune. 

In\marginnote{3.8} the same way, take a certain fine thoroughbred person who has gone to the forest, the root of a tree, or an empty hut. Their heart is not overcome and mired in sensual desire, and they truly understand the escape from sensual desire that has arisen. Their heart is not overcome by ill will … dullness and drowsiness … restlessness and remorse … doubt … They don’t meditate dependent on earth, water, fire, and air. They don’t meditate dependent on the dimension of infinite space, infinite consciousness, nothingness, or neither perception nor non-perception. They don’t meditate dependent on this world or the other world. They don’t meditate dependent on what is seen, heard, thought, known, attained, sought, or explored by the mind. Yet they do meditate. 

When\marginnote{3.15} a fine thoroughbred meditates like this, the gods together with Indra, the Divinity, and the Progenitor worship them from afar: 

\begin{verse}%
‘Homage\marginnote{4.1} to you, O thoroughbred! \\
Homage to you, supreme among men! \\
We don’t understand \\
the basis of your absorption.’” 

%
\end{verse}

When\marginnote{5.1} he said this, Venerable Sandha asked the Buddha, “But sir, how does that fine thoroughbred meditate?” 

“Sandha,\marginnote{7.1} for a fine thoroughbred person, the perception of earth has vanished in relation to earth. The perception of water … fire … air has vanished in relation to air. The perception of the dimension of infinite space has vanished in relation to the dimension of infinite space. The perception of the dimension of infinite consciousness … nothingness … neither perception nor non-perception has vanished in relation to the dimension of neither perception nor non-perception. The perception of this world has vanished in relation to this world. The perception of the other world has vanished in relation to the other world. And the perception of what is seen, heard, thought, known, attained, sought, or explored by the mind has vanished. That’s how that fine thoroughbred person doesn’t meditate dependent on earth, water, fire, and air. They don’t meditate dependent on the dimension of infinite space, infinite consciousness, nothingness, or neither perception nor non-perception. They don’t meditate dependent on this world or the other world. They don’t meditate dependent on what is seen, heard, thought, known, attained, sought, or explored by the mind. Yet they do meditate. 

When\marginnote{7.5} a fine thoroughbred person meditates like this, the gods together with Indra, the Divinity, and the Progenitor worship them from afar: 

\begin{verse}%
‘Homage\marginnote{8.1} to you, O thoroughbred! \\
Homage to you, supreme among men! \\
We don’t understand \\
the basis of your absorption.’” 

%
\end{verse}

%
\section*{{\suttatitleacronym AN 11.10}{\suttatitletranslation At the Peacocks’ Feeding Ground }{\suttatitleroot Moranivāpasutta}}
\addcontentsline{toc}{section}{\tocacronym{AN 11.10} \toctranslation{At the Peacocks’ Feeding Ground } \tocroot{Moranivāpasutta}}
\markboth{At the Peacocks’ Feeding Ground }{Moranivāpasutta}
\extramarks{AN 11.10}{AN 11.10}

At\marginnote{1.1} one time the Buddha was staying near \textsanskrit{Rājagaha}, at the monastery of the wanderers in the peacocks’ feeding ground. There the Buddha addressed the mendicants, “Mendicants!” 

“Venerable\marginnote{1.4} sir,” they replied. The Buddha said this: 

“Mendicants,\marginnote{2.1} a mendicant who has three qualities has reached the ultimate end, the ultimate sanctuary from the yoke, the ultimate spiritual life, the ultimate goal. They are best among gods and humans. What three? The entire spectrum of an adept’s ethics, immersion, and wisdom. A mendicant with these three qualities has reached the ultimate end, the ultimate sanctuary from the yoke, the ultimate spiritual life, the ultimate goal. They are best among gods and humans. 

A\marginnote{3.1} mendicant who has another three qualities has reached the ultimate end, the ultimate sanctuary from the yoke, the ultimate spiritual life, the ultimate goal. They are best among gods and humans. What three? A demonstration of psychic power, a demonstration of revealing, and a demonstration of instruction. A mendicant with these three qualities has reached the ultimate end, the ultimate sanctuary from the yoke, the ultimate spiritual life, the ultimate goal. They are best among gods and humans. 

A\marginnote{4.1} mendicant who has another three qualities has reached the ultimate end, the ultimate sanctuary from the yoke, the ultimate spiritual life, the ultimate goal. They are best among gods and humans. What three? Right view, right knowledge, and right freedom. A mendicant with these three qualities has reached the ultimate end, the ultimate sanctuary from the yoke, the ultimate spiritual life, the ultimate goal. They are best among gods and humans. 

A\marginnote{5.1} mendicant who has two qualities has reached the ultimate end, the ultimate sanctuary from the yoke, the ultimate spiritual life, the ultimate goal. They are best among gods and humans. What two? Knowledge and conduct. A mendicant with these two qualities has reached the ultimate end, the ultimate sanctuary from the yoke, the ultimate spiritual life, the ultimate goal. They are best among gods and humans. The divinity \textsanskrit{Sanaṅkumāra} also spoke this verse: 

\begin{verse}%
‘The\marginnote{6.1} aristocrat is best among people \\
who take clan as the standard. \\
But one accomplished in knowledge and conduct \\
is first among gods and humans.’ 

%
\end{verse}

Now,\marginnote{7.1} that verse spoken by the divinity \textsanskrit{Sanaṅkumāra} is well spoken, not poorly spoken. It’s beneficial, not pointless, and I agree with it. I also say: 

\begin{verse}%
‘The\marginnote{8.1} aristocrat is best among people \\
who take clan as the standard. \\
But one accomplished in knowledge and conduct \\
is first among gods and humans.’” 

%
\end{verse}

%
\addtocontents{toc}{\let\protect\contentsline\protect\nopagecontentsline}
\chapter*{The Chapter on Recollection }
\addcontentsline{toc}{chapter}{\tocchapterline{The Chapter on Recollection }}
\addtocontents{toc}{\let\protect\contentsline\protect\oldcontentsline}

%
\section*{{\suttatitleacronym AN 11.11}{\suttatitletranslation With Mahānāma (1st) }{\suttatitleroot Paṭhamamahānāmasutta}}
\addcontentsline{toc}{section}{\tocacronym{AN 11.11} \toctranslation{With Mahānāma (1st) } \tocroot{Paṭhamamahānāmasutta}}
\markboth{With Mahānāma (1st) }{Paṭhamamahānāmasutta}
\extramarks{AN 11.11}{AN 11.11}

At\marginnote{1.1} one time the Buddha was staying in the land of the Sakyans, near Kapilavatthu in the Banyan Tree Monastery. 

At\marginnote{1.2} that time several mendicants were making a robe for the Buddha, thinking that when his robe was finished and the three months of the rains residence had passed the Buddha would set out wandering. 

\textsanskrit{Mahānāma}\marginnote{1.4} the Sakyan heard about this. He went up to the Buddha, bowed, sat down to one side, and said to him: 

“Sir,\marginnote{2.2} I have heard that several mendicants are making a robe for the Buddha, thinking that when his robe was finished and the three months of the rains residence had passed the Buddha would set out wandering. Now, we spend our life in various ways. Which of these should we practice?” 

“Good,\marginnote{3.1} good, \textsanskrit{Mahānāma}! It’s appropriate that gentlemen such as you come to me and ask: ‘We spend our life in various ways. Which of these should we practice?’ The faithful succeed, not the faithless. The energetic succeed, not the lazy. The mindful succeed, not the unmindful. Those with immersion succeed, not those without immersion. The wise succeed, not the witless. 

When\marginnote{3.9} you’re grounded on these five things, go on to develop six further things. 

Firstly,\marginnote{3.10} you should recollect the Realized One: ‘That Blessed One is perfected, a fully awakened Buddha, accomplished in knowledge and conduct, holy, knower of the world, supreme guide for those who wish to train, teacher of gods and humans, awakened, blessed.’ When a noble disciple recollects the Realized One their mind is not full of greed, hate, and delusion. At that time their mind is unswerving, based on the Realized One. A noble disciple whose mind is unswerving finds inspiration in the meaning and the teaching, and finds joy connected with the teaching. When they’re joyful, rapture springs up. When the mind is full of rapture, the body becomes tranquil. When the body is tranquil, they feel bliss. And when they’re blissful, the mind becomes immersed in \textsanskrit{samādhi}. This is called a noble disciple who lives in balance among people who are unbalanced, and lives untroubled among people who are troubled. They’ve entered the stream of the teaching and developed the recollection of the Buddha. 

Furthermore,\marginnote{4.1} you should recollect the teaching: ‘The teaching is well explained by the Buddha—apparent in the present life, immediately effective, inviting inspection, relevant, so that sensible people can know it for themselves.’ When a noble disciple recollects the teaching their mind is not full of greed, hate, and delusion. … This is called a noble disciple who lives in balance among people who are unbalanced, and lives untroubled among people who are troubled. They’ve entered the stream of the teaching and developed the recollection of the teaching. 

Furthermore,\marginnote{5.1} you should recollect the \textsanskrit{Saṅgha}: ‘The \textsanskrit{Saṅgha} of the Buddha’s disciples is practicing the way that’s good, sincere, systematic, and proper. It consists of the four pairs, the eight individuals. This is the \textsanskrit{Saṅgha} of the Buddha’s disciples that is worthy of offerings dedicated to the gods, worthy of hospitality, worthy of a religious donation, worthy of greeting with joined palms, and is the supreme field of merit for the world.’ When a noble disciple recollects the \textsanskrit{Saṅgha} their mind is not full of greed, hate, and delusion. … This is called a noble disciple who lives in balance among people who are unbalanced, and lives untroubled among people who are troubled. They’ve entered the stream of the teaching and developed the recollection of the \textsanskrit{Saṅgha}. 

Furthermore,\marginnote{6.1} you should recollect your own ethical conduct, which is intact, impeccable, spotless, and unmarred, liberating, praised by sensible people, not mistaken, and leading to immersion. When a noble disciple recollects their ethical conduct their mind is not full of greed, hate, and delusion. … This is called a noble disciple who lives in balance among people who are unbalanced, and lives untroubled among people who are troubled. They’ve entered the stream of the teaching and developed the recollection of their ethical conduct. 

Furthermore,\marginnote{7.1} you should recollect your own generosity: ‘I’m so fortunate, so very fortunate. Among people with hearts full of the stain of stinginess I live at home rid of stinginess, freely generous, open-handed, loving to let go, committed to charity, loving to give and to share.’ When a noble disciple recollects their own generosity their mind is not full of greed, hate, and delusion. … This is called a noble disciple who lives in balance among people who are unbalanced, and lives untroubled among people who are troubled. They’ve entered the stream of the teaching and developed the recollection of generosity. 

Furthermore,\marginnote{8.1} you should recollect the deities: ‘There are the gods of the four great kings, the gods of the thirty-three, the gods of Yama, the joyful gods, the gods who love to imagine, the gods who control what is imagined by others, the gods of the Divinity’s host, and gods even higher than these. When those deities passed away from here, they were reborn there because of their faith, ethics, learning, generosity, and wisdom. I, too, have the same kind of faith, ethics, learning, generosity, and wisdom.’ When a noble disciple recollects the faith, ethics, learning, generosity, and wisdom of both themselves and the deities their mind is not full of greed, hate, and delusion. At that time their mind is unswerving, based on the deities. A noble disciple whose mind is unswerving finds inspiration in the meaning and the teaching, and finds joy connected with the teaching. When they’re joyful, rapture springs up. When the mind is full of rapture, the body becomes tranquil. When the body is tranquil, they feel bliss. And when they’re blissful, the mind becomes immersed in \textsanskrit{samādhi}. This is called a noble disciple who lives in balance among people who are unbalanced, and lives untroubled among people who are troubled. They’ve entered the stream of the teaching and developed the recollection of the deities.” 

%
\section*{{\suttatitleacronym AN 11.12}{\suttatitletranslation With Mahānāma (2nd) }{\suttatitleroot Dutiyamahānāmasutta}}
\addcontentsline{toc}{section}{\tocacronym{AN 11.12} \toctranslation{With Mahānāma (2nd) } \tocroot{Dutiyamahānāmasutta}}
\markboth{With Mahānāma (2nd) }{Dutiyamahānāmasutta}
\extramarks{AN 11.12}{AN 11.12}

At\marginnote{1.1} one time the Buddha was staying in the land of the Sakyans, near Kapilavatthu in the Banyan Tree Monastery. Now at that time \textsanskrit{Mahānāma} the Sakyan had recently recovered from an illness. At that time several mendicants were making a robe for the Buddha … 

\textsanskrit{Mahānāma}\marginnote{2.1} the Sakyan heard about this. He went up to the Buddha, bowed, sat down to one side, and said to him: 

“Sir,\marginnote{2.5} I have heard that several mendicants are making a robe for the Buddha, thinking that when his robe was finished and the three months of the rains residence had passed the Buddha would set out wandering. Now, we spend our life in various ways. Which of these should we practice?” 

“Good,\marginnote{3.1} good, \textsanskrit{Mahānāma}! It’s appropriate that gentlemen such as you come to me and ask: ‘We spend our life in various ways. Which of these should we practice?’ The faithful succeed, not the faithless. The energetic succeed, not the lazy. The mindful succeed, not the unmindful. Those with immersion succeed, not those without immersion. The wise succeed, not the witless. When you’re grounded on these five things, go on to develop six further things. 

Firstly,\marginnote{4.1} you should recollect the Realized One: ‘That Blessed One is perfected, a fully awakened Buddha, accomplished in knowledge and conduct, holy, knower of the world, supreme guide for those who wish to train, teacher of gods and humans, awakened, blessed.’ When a noble disciple recollects the Realized One their mind is not full of greed, hate, and delusion. At that time their mind is unswerving, based on the Realized One. A noble disciple whose mind is unswerving finds inspiration in the meaning and the teaching, and finds joy connected with the teaching. When they’re joyful, rapture springs up. When the mind is full of rapture, the body becomes tranquil. When the body is tranquil, they feel bliss. And when they’re blissful, the mind becomes immersed in \textsanskrit{samādhi}. You should develop this recollection of the Buddha while walking, standing, sitting, lying down, while working, and while at home with your children. 

Furthermore,\marginnote{5.1} you should recollect the teaching … the \textsanskrit{Saṅgha} … your own ethical conduct … your own generosity … the deities … When a noble disciple recollects the faith, ethics, learning, generosity, and wisdom of both themselves and the deities their mind is not full of greed, hate, and delusion. At that time their mind is unswerving, based on the deities. A noble disciple whose mind is unswerving finds inspiration in the meaning and the teaching, and finds joy connected with the teaching. When they’re joyful, rapture springs up. When the mind is full of rapture, the body becomes tranquil. When the body is tranquil, they feel bliss. And when they’re blissful, the mind becomes immersed in \textsanskrit{samādhi}. You should develop this recollection of the deities while walking, standing, sitting, lying down, while working, and while at home with your children.” 

%
\section*{{\suttatitleacronym AN 11.13}{\suttatitletranslation With Nandiya }{\suttatitleroot Nandiyasutta}}
\addcontentsline{toc}{section}{\tocacronym{AN 11.13} \toctranslation{With Nandiya } \tocroot{Nandiyasutta}}
\markboth{With Nandiya }{Nandiyasutta}
\extramarks{AN 11.13}{AN 11.13}

At\marginnote{1.1} one time the Buddha was staying in the land of the Sakyans, near Kapilavatthu in the Banyan Tree Monastery. 

Now\marginnote{1.2} at that time the Buddha wanted to commence the rains residence at \textsanskrit{Sāvatthī}. 

Nandiya\marginnote{2.1} the Sakyan heard about this, and thought, “Why don’t I also commence the rains residence at \textsanskrit{Sāvatthī}. There I can focus on my work and from time to time get to see the Buddha.” 

So\marginnote{3.1} the Buddha commenced the rains residence in \textsanskrit{Sāvatthī}, and so did Nandiya. There he focused on his work and from time to time got to see the Buddha. 

At\marginnote{3.4} that time several mendicants were making a robe for the Buddha, thinking that when his robe was finished and the three months of the rains residence had passed the Buddha would set out wandering. 

Nandiya\marginnote{4.1} the Sakyan heard about this. He went up to the Buddha, bowed, sat down to one side, and said to him: 

“Sir,\marginnote{4.5} I have heard that several mendicants are making a robe for the Buddha, thinking that when his robe was finished and the three months of the rains residence had passed the Buddha would set out wandering. Now, we spend our life in various ways. Which of these should we practice?” 

“Good,\marginnote{5.1} good Nandiya! It’s appropriate that gentlemen such as you come to me and ask: ‘We spend our life in various ways. Which of these should we practice?’ The faithful succeed, not the faithless. The ethical succeed, not the unethical. The energetic succeed, not the lazy. The mindful succeed, not the unmindful. Those with immersion succeed, not those without immersion. The wise succeed, not the witless. When you’re grounded on these six things, go on to establish mindfulness on five further things internally. 

Firstly,\marginnote{6.1} you should recollect the Realized One: ‘That Blessed One is perfected, a fully awakened Buddha, accomplished in knowledge and conduct, holy, knower of the world, supreme guide for those who wish to train, teacher of gods and humans, awakened, blessed.’ In this way you should establish mindfulness internally based on the Realized One. 

Furthermore,\marginnote{7.1} you should recollect the teaching: ‘The teaching is well explained by the Buddha—apparent in the present life, immediately effective, inviting inspection, relevant, so that sensible people can know it for themselves.’ In this way you should establish mindfulness internally based on the teaching. 

Furthermore,\marginnote{8.1} you should recollect your good friends: ‘I’m fortunate, so very fortunate, to have good friends who advise and instruct me out of kindness and sympathy.’ In this way you should establish mindfulness internally based on good friends. 

Furthermore,\marginnote{9.1} you should recollect your own generosity: ‘I’m so fortunate, so very fortunate. Among people with hearts full of the stain of stinginess I live at home rid of stinginess, freely generous, open-handed, loving to let go, committed to charity, loving to give and to share.’ In this way you should establish mindfulness internally based on generosity. 

Furthermore,\marginnote{10.1} you should recollect the deities: ‘There are deities who, surpassing the company of deities that consume solid food, are reborn in a certain host of mind-made deities. They don’t see in themselves anything more to do, or anything that needs improvement.’ An irreversibly freed mendicant doesn’t see in themselves anything more to do, or anything that needs improvement. In the same way, Nandiya, there are deities who, surpassing the company of deities that consume solid food, are reborn in a certain host of mind-made deities. They don’t see in themselves anything more to do, or anything that needs improvement. In this way you should establish mindfulness internally based on the deities. 

A\marginnote{11.1} noble disciple who has these eleven qualities gives up bad, unskillful qualities and doesn’t cling to them. It’s like when a pot full of water is tipped over, so the water drains out and doesn’t go back in. Suppose there was an uncontrolled fire. It advances burning up dry woodlands and doesn’t go back over what it has burned. In the same way, a noble disciple who has these eleven qualities gives up bad, unskillful qualities and doesn’t cling to them.” 

%
\section*{{\suttatitleacronym AN 11.14}{\suttatitletranslation With Subhūti }{\suttatitleroot Subhūtisutta}}
\addcontentsline{toc}{section}{\tocacronym{AN 11.14} \toctranslation{With Subhūti } \tocroot{Subhūtisutta}}
\markboth{With Subhūti }{Subhūtisutta}
\extramarks{AN 11.14}{AN 11.14}

And\marginnote{1.1} then Venerable \textsanskrit{Subhūti} together with the mendicant Saddha went up to the Buddha, bowed, and sat down to one side. The Buddha said to him, “\textsanskrit{Subhūti}, what is the name of this mendicant?” 

“Sir,\marginnote{1.3} the name of this mendicant is Saddha. He is the son of the layman Sudatta, and has gone forth out of faith from the lay life to homelessness.” 

“Well,\marginnote{2.1} I hope this mendicant Saddha exhibits the various evidences of faith.” 

“Now\marginnote{2.2} is the time, Blessed One! Now is the time, Holy One! Let the Buddha to speak on the evidence of faith. Now I will find out whether or not this mendicant Saddha exhibits the various evidences of faith.” 

“Well\marginnote{3.1} then, \textsanskrit{Subhūti}, listen and apply your mind well, I will speak.” 

“Yes,\marginnote{3.2} sir,” \textsanskrit{Subhūti} replied. The Buddha said this: 

“Firstly,\marginnote{4.1} a mendicant is ethical, restrained in the monastic code, conducting themselves well and resorting for alms in suitable places. Seeing danger in the slightest fault, they keep the rules they’ve undertaken. When a mendicant is ethical, this is evidence of faith. 

Furthermore,\marginnote{5.1} a mendicant is very learned, remembering and keeping what they’ve learned. These teachings are good in the beginning, good in the middle, and good in the end, meaningful and well-phrased, describing a spiritual practice that’s entirely full and pure. They are very learned in such teachings, remembering them, rehearsing them, mentally scrutinizing them, and comprehending them theoretically. When a mendicant is learned, this is evidence of faith. 

Furthermore,\marginnote{6.1} a mendicant has good friends, companions, and associates. When a mendicant has good friends, this is evidence of faith. 

Furthermore,\marginnote{7.1} a mendicant is easy to admonish, having qualities that make them easy to admonish. They’re patient, and take instruction respectfully. When a mendicant is easy to admonish, this is evidence of faith. 

Furthermore,\marginnote{8.1} a mendicant is deft and tireless in a diverse spectrum of duties for their spiritual companions, understanding how to go about things in order to complete and organize the work. When a mendicant is deft and tireless in a diverse spectrum of duties, this is evidence of faith. 

Furthermore,\marginnote{9.1} a mendicant loves the teachings and is a delight to converse with, being full of joy in the teaching and training. When a mendicant loves the teachings, this is evidence of faith. 

Furthermore,\marginnote{10.1} a mendicant lives with energy roused up for giving up unskillful qualities and embracing skillful qualities. They are strong, staunchly vigorous, not slacking off when it comes to developing skillful qualities. When a mendicant is energetic, this is evidence of faith. 

Furthermore,\marginnote{11.1} a mendicant gets the four absorptions—blissful meditations in this life that belong to the higher mind—when they want, without trouble or difficulty. When a mendicant gets the four absorptions, this is evidence of faith. 

Furthermore,\marginnote{12.1} a mendicant recollects many kinds of past lives. That is: one, two, three, four, five, ten, twenty, thirty, forty, fifty, a hundred, a thousand, a hundred thousand rebirths; many eons of the world contracting, many eons of the world expanding, many eons of the world contracting and expanding. They remember: ‘There, I was named this, my clan was that, I looked like this, and that was my food. This was how I felt pleasure and pain, and that was how my life ended. When I passed away from that place I was reborn somewhere else. There, too, I was named this, my clan was that, I looked like this, and that was my food. This was how I felt pleasure and pain, and that was how my life ended. When I passed away from that place I was reborn here.’ And so they recollect their many kinds of past lives, with features and details. When a mendicant recollects many kinds of past lives, this is evidence of faith. 

Furthermore,\marginnote{13.1} with clairvoyance that is purified and superhuman, a mendicant sees sentient beings passing away and being reborn—inferior and superior, beautiful and ugly, in a good place or a bad place. They understand how sentient beings are reborn according to their deeds. ‘These dear beings did bad things by way of body, speech, and mind. They denounced the noble ones; they had wrong view; and they chose to act out of that wrong view. When their body breaks up, after death, they’re reborn in a place of loss, a bad place, the underworld, hell. These dear beings, however, did good things by way of body, speech, and mind. They never denounced the noble ones; they had right view; and they chose to act out of that right view. When their body breaks up, after death, they’re reborn in a good place, a heavenly realm.’ And so, with clairvoyance that is purified and superhuman, they see sentient beings passing away and being reborn—inferior and superior, beautiful and ugly, in a good place or a bad place. They understand how sentient beings are reborn according to their deeds. When a mendicant has clairvoyance that is purified and superhuman, this is evidence of faith. 

Furthermore,\marginnote{14.1} a mendicant has realized the undefiled freedom of heart and freedom by wisdom in this very life, and lives having realized it with their own insight due to the ending of defilements. When a mendicant has ended the defilements, this is evidence of faith.” 

When\marginnote{15.1} he said this, Venerable \textsanskrit{Subhūti} said to the Buddha: 

“Sir,\marginnote{15.2} the various evidences of faith for a faithful person that the Buddha speaks of are found in this mendicant; he does exhibit them. 

This\marginnote{16.1} mendicant is ethical … 

This\marginnote{17.1} mendicant is learned … 

This\marginnote{18.1} mendicant has good friends … 

This\marginnote{19.1} mendicant is easy to admonish … 

This\marginnote{20.1} mendicant is deft and tireless in a diverse spectrum of duties … 

This\marginnote{21.1} mendicant loves the teachings … 

This\marginnote{22.1} mendicant is energetic … 

This\marginnote{23.1} mendicant gets the four absorptions … 

This\marginnote{24.1} mendicant recollects their many kinds of past lives … 

This\marginnote{25.1} mendicant has clairvoyance that is purified and surpasses the human … 

This\marginnote{26.1} mendicant has ended the defilements … 

The\marginnote{26.2} various evidences of faith for a faithful person that the Buddha speaks of are found in this mendicant; he does exhibit them.” 

“Good,\marginnote{27.1} good, \textsanskrit{Subhūti}! So, \textsanskrit{Subhūti}, you should live together with this mendicant Saddha. And when you want to see the Realized One, you should come together with him.” 

%
\section*{{\suttatitleacronym AN 11.15}{\suttatitletranslation The Benefits of Love }{\suttatitleroot Mettāsutta}}
\addcontentsline{toc}{section}{\tocacronym{AN 11.15} \toctranslation{The Benefits of Love } \tocroot{Mettāsutta}}
\markboth{The Benefits of Love }{Mettāsutta}
\extramarks{AN 11.15}{AN 11.15}

“Mendicants,\marginnote{1.1} you can expect eleven benefits when the heart’s release by love has been cultivated, developed, and practiced, made a vehicle and a basis, kept up, consolidated, and properly implemented. 

What\marginnote{2.1} eleven? You sleep at ease. You wake happily. You don’t have bad dreams. Humans love you. Non-humans love you. Deities protect you. You can’t be harmed by fire, poison, or blade. Your mind quickly enters immersion. Your face is clear and bright. You don’t feel lost when you die. If you don’t penetrate any higher, you’ll be reborn in a realm of divinity. You can expect eleven benefits when the heart’s release by love has been cultivated, developed, and practiced, made a vehicle and a basis, kept up, consolidated, and properly implemented.” 

%
\section*{{\suttatitleacronym AN 11.16}{\suttatitletranslation The Wealthy Citizen }{\suttatitleroot Aṭṭhakanāgarasutta}}
\addcontentsline{toc}{section}{\tocacronym{AN 11.16} \toctranslation{The Wealthy Citizen } \tocroot{Aṭṭhakanāgarasutta}}
\markboth{The Wealthy Citizen }{Aṭṭhakanāgarasutta}
\extramarks{AN 11.16}{AN 11.16}

At\marginnote{1.1} one time Venerable Ānanda was staying near \textsanskrit{Vesālī} in the little village of Beluva. 

Now\marginnote{1.2} at that time the householder Dasama, a wealthy citizen, had arrived at \textsanskrit{Pāṭaliputta} on some business. He went to the Chicken Monastery, approached a certain mendicant, and said to him, “Sir, where is Venerable Ānanda now staying? For I want to see him.” 

“Householder,\marginnote{2.4} Venerable Ānanda is staying near \textsanskrit{Vesālī} in the little village of Beluva.” 

Then\marginnote{3.1} the householder Dasama, having concluded his business there, went to the little village of Beluva in \textsanskrit{Vesālī} to see Ānanda. He bowed, sat down to one side, and said to Ānanda: 

“Honorable\marginnote{3.2} Ānanda, is there one thing that has been rightly explained by the Blessed One—who knows and sees, the perfected one, the fully awakened Buddha—practicing which a diligent, keen, and resolute mendicant’s mind is freed, their defilements are ended, and they arrive at the supreme sanctuary from the yoke?” 

“There\marginnote{3.3} is, householder.” 

“And\marginnote{4.1} what is that one thing?” 

“Householder,\marginnote{4.2} it’s when a mendicant, quite secluded from sensual pleasures, secluded from unskillful qualities, enters and remains in the first absorption, which has the rapture and bliss born of seclusion, while placing the mind and keeping it connected. Then they reflect: ‘Even this first absorption is produced by choices and intentions.’ They understand: ‘But whatever is produced by choices and intentions is impermanent and liable to cessation.’ Abiding in that they attain the ending of defilements. If they don’t attain the ending of defilements, with the ending of the five lower fetters they’re reborn spontaneously, because of their passion and love for that meditation. They are extinguished there, and are not liable to return from that world. This is one thing that has been rightly explained by the Blessed One—who knows and sees, the perfected one, the fully awakened Buddha—practicing which a diligent, keen, and resolute mendicant’s mind is freed, their defilements are ended, and they arrive at the supreme sanctuary from the yoke. 

Furthermore,\marginnote{5.1} as the placing of the mind and keeping it connected are stilled, they enter and remain in the second absorption … third absorption … fourth absorption. Then they reflect: ‘Even this fourth absorption is produced by choices and intentions.’ They understand: ‘But whatever is produced by choices and intentions is impermanent and liable to cessation.’ Abiding in that they attain the ending of defilements. If they don’t attain the ending of defilements, with the ending of the five lower fetters they’re reborn spontaneously, because of their passion and love for that meditation. They are extinguished there, and are not liable to return from that world. This too is one thing that has been rightly explained by the Blessed One—who knows and sees, the perfected one, the fully awakened Buddha—practicing which a diligent, keen, and resolute mendicant’s mind is freed, their defilements are ended, and they arrive at the supreme sanctuary from the yoke. 

Furthermore,\marginnote{6.1} a mendicant meditates spreading a heart full of love to one direction, and to the second, and to the third, and to the fourth. In the same way above, below, across, everywhere, all around, they spread a heart full of love to the whole world—abundant, expansive, limitless, free of enmity and ill will. Then they reflect: ‘Even this heart’s release by love is produced by choices and intentions.’ They understand: ‘But whatever is produced by choices and intentions is impermanent and liable to cessation.’ Abiding in that they attain the ending of defilements. If they don’t attain the ending of defilements, with the ending of the five lower fetters they’re reborn spontaneously, because of their passion and love for that meditation. They are extinguished there, and are not liable to return from that world. This too is one thing that has been rightly explained by the Blessed One … 

Furthermore,\marginnote{7.1} a mendicant meditates spreading a heart full of compassion … They meditate spreading a heart full of rejoicing … They meditate spreading a heart full of equanimity to one direction, and to the second, and to the third, and to the fourth. In the same way above, below, across, everywhere, all around, they spread a heart full of equanimity to the whole world—abundant, expansive, limitless, free of enmity and ill will. Then they reflect: ‘Even this heart’s release by equanimity is produced by choices and intentions.’ They understand: ‘But whatever is produced by choices and intentions is impermanent and liable to cessation.’ Abiding in that they attain the ending of defilements. If they don’t attain the ending of defilements, with the ending of the five lower fetters they’re reborn spontaneously, because of their passion and love for that meditation. They are extinguished there, and are not liable to return from that world. This too is one thing that has been rightly explained by the Blessed One … 

Furthermore,\marginnote{8.1} a mendicant, going totally beyond perceptions of form, with the ending of perceptions of impingement, not focusing on perceptions of diversity, aware that ‘space is infinite’, enters and remains in the dimension of infinite space. Then they reflect: ‘Even this attainment of the dimension of infinite space is produced by choices and intentions.’ They understand: ‘But whatever is produced by choices and intentions is impermanent and liable to cessation.’ Abiding in that they attain the ending of defilements. If they don’t attain the ending of defilements, with the ending of the five lower fetters they’re reborn spontaneously, because of their passion and love for that meditation. They are extinguished there, and are not liable to return from that world. This too is one thing that has been rightly explained by the Blessed One … 

Furthermore,\marginnote{9.1} a mendicant, going totally beyond the dimension of infinite space, aware that ‘consciousness is infinite’, enters and remains in the dimension of infinite consciousness. … Going totally beyond the dimension of infinite consciousness, aware that ‘there is nothing at all’, they enter and remain in the dimension of nothingness. … Then they reflect: ‘Even this attainment of the dimension of nothingness is produced by choices and intentions.’ They understand: ‘But whatever is produced by choices and intentions is impermanent and liable to cessation.’ Abiding in that they attain the ending of defilements. If they don’t attain the ending of defilements, with the ending of the five lower fetters they’re reborn spontaneously, because of their passion and love for that meditation. They are extinguished there, and are not liable to return from that world. This too is one thing that has been rightly explained by the Blessed One—who knows and sees, the perfected one, the fully awakened Buddha—practicing which a diligent, keen, and resolute mendicant’s mind is freed, their defilements are ended, and they reach the supreme sanctuary from the yoke.” 

When\marginnote{10.1} he said this, the householder Dasama said to Venerable Ānanda: 

“Honorable\marginnote{10.2} Ānanda, suppose a person was looking for an entrance to a hidden treasure. And all at once they’d come across eleven entrances! In the same way, I was searching for the door to freedom from death. And all at once I found eleven doors to freedom from death for cultivation. Suppose a person had a house with eleven doors. If the house caught fire they’d be able to flee to safety through any one of those doors. In the same way, I’m able to flee to safety through any one of these eleven doors to freedom from death. Sir, those of other religions will seek a fee for the tutor. Why shouldn’t I make an offering to Venerable Ānanda?” 

Then\marginnote{11.1} the householder Dasama, having assembled the \textsanskrit{Saṅgha} from \textsanskrit{Vesālī} and \textsanskrit{Pāṭaliputta}, served and satisfied them with his own hands with delicious fresh and cooked foods. He clothed each and every mendicant in a pair of garments, with a set of three robes for Ānanda. And he had a dwelling worth five hundred built for Ānanda. 

%
\section*{{\suttatitleacronym AN 11.17}{\suttatitletranslation The Cowherd }{\suttatitleroot Gopālasutta}}
\addcontentsline{toc}{section}{\tocacronym{AN 11.17} \toctranslation{The Cowherd } \tocroot{Gopālasutta}}
\markboth{The Cowherd }{Gopālasutta}
\extramarks{AN 11.17}{AN 11.17}

“Mendicants,\marginnote{1.1} a cowherd with eleven factors can’t maintain and expand a herd of cattle. What eleven? It’s when a cowherd doesn’t know form, is unskilled in characteristics, doesn’t pick out flies’ eggs, doesn’t dress wounds, doesn’t spread smoke, doesn’t know the ford, doesn’t know satisfaction, doesn’t know the trail, is not skilled in pastures, milks dry, and doesn’t show extra respect to the bulls who are fathers and leaders of the herd. A cowherd with these eleven factors can’t maintain and expand a herd of cattle. 

In\marginnote{2.1} the same way, a mendicant with eleven qualities can’t achieve growth, improvement, or maturity in this teaching and training. What eleven? It’s when a mendicant doesn’t know form, is unskilled in characteristics, doesn’t pick out flies’ eggs, doesn’t dress wounds, doesn’t spread smoke, doesn’t know the ford, doesn’t know satisfaction, doesn’t know the trail, is not skilled in pastures, milks dry, and doesn’t show extra respect to senior mendicants of long standing, long gone forth, fathers and leaders of the \textsanskrit{Saṅgha}. 

And\marginnote{3.1} how does a mendicant not know form? It’s when a mendicant doesn’t truly understand that all form is the four principal states, or form derived from the four principal states. That’s how a mendicant doesn’t know form. 

And\marginnote{4.1} how is a mendicant not skilled in characteristics? It’s when a mendicant doesn’t understand that a fool is characterized by their deeds, and an astute person is characterized by their deeds. That’s how a mendicant isn’t skilled in characteristics. 

And\marginnote{5.1} how does a mendicant not pick out flies’ eggs? It’s when a mendicant tolerates a sensual, malicious, or cruel thought that has arisen. They don’t give it up, get rid of it, eliminate it, and obliterate it. They tolerate any bad, unskillful qualities that have arisen. They don’t give them up, get rid of them, eliminate them, and obliterate them. That’s how a mendicant doesn’t pick out flies’ eggs. 

And\marginnote{6.1} how does a mendicant not dress wounds? When a mendicant sees a sight with their eyes, they get caught up in the features and details. Since the faculty of sight is left unrestrained, bad unskillful qualities of covetousness and displeasure become overwhelming. They don’t practice restraint, they don’t protect the faculty of sight, and they don’t achieve its restraint. When they hear a sound with their ears … When they smell an odor with their nose … When they taste a flavor with their tongue … When they feel a touch with their body … When they know an idea with their mind, they get caught up in the features and details. Since the faculty of the mind is left unrestrained, bad unskillful qualities of covetousness and displeasure become overwhelming. They don’t practice restraint, they don’t protect the faculty of the mind, and they don’t achieve its restraint. That’s how a mendicant doesn’t dress wounds. 

And\marginnote{7.1} how does a mendicant not spread smoke? It’s when a mendicant doesn’t teach others the Dhamma in detail as they learned and memorized it. That’s how a mendicant doesn’t spread smoke. 

And\marginnote{8.1} how does a mendicant not know the ford? It’s when a mendicant doesn’t from time to time go up to those mendicants who are very learned—inheritors of the heritage, who have memorized the teachings, the monastic law, and the outlines—and ask them questions: ‘Why, sir, does it say this? What does that mean?’ Those venerables don’t clarify what is unclear, reveal what is obscure, and dispel doubt regarding the many doubtful matters. That’s how a mendicant doesn’t know the ford. 

And\marginnote{9.1} how does a mendicant not know satisfaction? It’s when a mendicant, when the teaching and training proclaimed by the Realized One are being taught, finds no inspiration in the meaning and the teaching, and finds no joy connected with the teaching. That’s how a mendicant doesn’t know satisfaction. 

And\marginnote{10.1} how does a mendicant not know the trail? It’s when a mendicant doesn’t truly understand the noble eightfold path. That’s how a mendicant doesn’t know the trail. 

And\marginnote{11.1} how is a mendicant not skilled in pastures? It’s when a mendicant doesn’t truly understand the four kinds of mindfulness meditation. That’s how a mendicant is not skilled in pastures. 

And\marginnote{12.1} how does a mendicant milk dry? It’s when a mendicant is invited by a householder to accept robes, almsfood, lodgings, and medicines and supplies for the sick. But they don’t know moderation in accepting. That’s how a mendicant milks dry. 

And\marginnote{13.1} how does a mendicant not show extra respect to senior mendicants of long standing, long gone forth, fathers and leaders of the \textsanskrit{Saṅgha}? It’s when a mendicant doesn’t consistently treat senior mendicants of long standing, long gone forth, fathers and leaders of the \textsanskrit{Saṅgha} with kindness by way of body, speech, and mind, both in public and in private. That’s how a mendicant doesn’t show extra respect to senior mendicants of long standing, long gone forth, fathers and leaders of the \textsanskrit{Saṅgha}. 

A\marginnote{14.1} mendicant with these eleven qualities can’t achieve growth, improvement, or maturity in this teaching and training. 

A\marginnote{15.1} cowherd with eleven factors can maintain and expand a herd of cattle. What eleven? It’s when a cowherd knows form, is skilled in characteristics, picks out flies’ eggs, dresses wounds, spreads smoke, knows the ford, knows satisfaction, knows the trail, is skilled in pastures, doesn’t milk dry, and shows extra respect to the bulls who are fathers and leaders of the herd. A cowherd with these eleven factors can maintain and expand a herd of cattle. 

In\marginnote{16.1} the same way, a mendicant with eleven qualities can achieve growth, improvement, and maturity in this teaching and training. What eleven? It’s when a mendicant knows form, is skilled in characteristics, picks out flies’ eggs, dresses wounds, spreads smoke, knows the ford, knows satisfaction, knows the trail, is skilled in pastures, doesn’t milk dry, and shows extra respect to senior mendicants of long standing, long gone forth, fathers and leaders of the \textsanskrit{Saṅgha}. 

And\marginnote{17.1} how does a mendicant know form? It’s when a mendicant truly understands that all form is the four principal states, or form derived from the four principal states. That’s how a mendicant knows form. 

And\marginnote{18.1} how is a mendicant skilled in characteristics? It’s when a mendicant understands that a fool is characterized by their deeds, and an astute person is characterized by their deeds. That’s how a mendicant is skilled in characteristics. 

And\marginnote{19.1} how does a mendicant pick out flies’ eggs? It’s when a mendicant doesn’t tolerate a sensual, malicious, or cruel thought that has arisen, but gives it up, gets rid of it, eliminates it, and exterminates it. They don’t tolerate any bad, unskillful qualities that have arisen, but give them up, get rid of them, eliminate them, and obliterate them. That’s how a mendicant picks out flies’ eggs. 

And\marginnote{20.1} how does a mendicant dress wounds? When a mendicant sees a sight with their eyes, they don’t get caught up in the features and details. If the faculty of sight were left unrestrained, bad unskillful qualities of covetousness and displeasure would become overwhelming. For this reason, they practice restraint, protecting the faculty of sight, and achieving its restraint. When they hear a sound with their ears … When they smell an odor with their nose … When they taste a flavor with their tongue … When they feel a touch with their body … When they know an idea with their mind, they don’t get caught up in the features and details. If the faculty of mind were left unrestrained, bad unskillful qualities of covetousness and displeasure would become overwhelming. For this reason, they practice restraint, protecting the faculty of mind, and achieving its restraint. That’s how a mendicant dresses wounds. 

And\marginnote{21.1} how does a mendicant spread smoke? It’s when a mendicant teaches others the Dhamma in detail as they learned and memorized it. That’s how a mendicant spreads smoke. 

And\marginnote{22.1} how does a mendicant know the ford? It’s when from time to time a mendicant goes up to those mendicants who are very learned—inheritors of the heritage, who have memorized the teachings, the monastic law, and the outlines—and asks them questions: ‘Why, sir, does it say this? What does that mean?’ Those venerables clarify what is unclear, reveal what is obscure, and dispel doubt regarding the many doubtful matters. That’s how a mendicant knows the ford. 

And\marginnote{23.1} how does a mendicant know satisfaction? It’s when a mendicant, when the teaching and training proclaimed by the Realized One are being taught, finds inspiration in the meaning and the teaching, and finds joy connected with the teaching. That’s how a mendicant knows satisfaction. 

And\marginnote{24.1} how does a mendicant know the trail? It’s when a mendicant truly understands the noble eightfold path. That’s how a mendicant knows the trail. 

And\marginnote{25.1} how is a mendicant skilled in pastures? It’s when a mendicant truly understands the four kinds of mindfulness meditation. That’s how a mendicant is skilled in pastures. 

And\marginnote{26.1} how does a mendicant not milk dry? It’s when a mendicant is invited by a householder to accept robes, almsfood, lodgings, and medicines and supplies for the sick. And that mendicant knows moderation in accepting. That’s how a mendicant doesn’t milk dry. 

And\marginnote{27.1} how does a mendicant show extra respect to senior mendicants of long standing, long gone forth, fathers and leaders of the \textsanskrit{Saṅgha}? It’s when a mendicant consistently treats senior mendicants of long standing, long gone forth, fathers and leaders of the \textsanskrit{Saṅgha} with kindness by way of body, speech, and mind, both in public and in private. That’s how a mendicant shows extra respect to senior mendicants of long standing, long gone forth, fathers and leaders of the \textsanskrit{Saṅgha}. 

A\marginnote{28.1} mendicant with these eleven qualities can achieve growth, improvement, or maturity in this teaching and training.” 

%
\section*{{\suttatitleacronym AN 11.18}{\suttatitletranslation Immersion (1st) }{\suttatitleroot Paṭhamasamādhisutta}}
\addcontentsline{toc}{section}{\tocacronym{AN 11.18} \toctranslation{Immersion (1st) } \tocroot{Paṭhamasamādhisutta}}
\markboth{Immersion (1st) }{Paṭhamasamādhisutta}
\extramarks{AN 11.18}{AN 11.18}

And\marginnote{1.1} then several mendicants went up to the Buddha, bowed, sat down to one side, and said to him: 

“Could\marginnote{2.1} it be, sir, that a mendicant might gain a state of immersion like this? They wouldn’t perceive earth in earth, water in water, fire in fire, or air in air. And they wouldn’t perceive the dimension of infinite space in the dimension of infinite space, the dimension of infinite consciousness in the dimension of infinite consciousness, the dimension of nothingness in the dimension of nothingness, or the dimension of neither perception nor non-perception in the dimension of neither perception nor non-perception. They wouldn’t perceive this world in this world, or the other world in the other world. And they wouldn’t perceive what is seen, heard, thought, known, attained, sought, or explored by the mind. And yet they would still perceive.” 

“It\marginnote{3.1} could be, mendicants.” 

“But\marginnote{4.1} how could this be?” 

“It’s\marginnote{5.1} when a mendicant perceives: ‘This is peaceful; this is sublime—that is, the stilling of all activities, the letting go of all attachments, the ending of craving, fading away, cessation, extinguishment.’ That’s how a mendicant might gain a state of immersion like this. They wouldn’t perceive earth in earth, water in water, fire in fire, or air in air. And they wouldn’t perceive the dimension of infinite space in the dimension of infinite space, the dimension of infinite consciousness in the dimension of infinite consciousness, the dimension of nothingness in the dimension of nothingness, or the dimension of neither perception nor non-perception in the dimension of neither perception nor non-perception. They wouldn’t perceive this world in this world, or the other world in the other world. And they wouldn’t perceive what is seen, heard, thought, known, attained, sought, or explored by the mind. And yet they would still perceive.” 

%
\section*{{\suttatitleacronym AN 11.19}{\suttatitletranslation Immersion (2nd) }{\suttatitleroot Dutiyasamādhisutta}}
\addcontentsline{toc}{section}{\tocacronym{AN 11.19} \toctranslation{Immersion (2nd) } \tocroot{Dutiyasamādhisutta}}
\markboth{Immersion (2nd) }{Dutiyasamādhisutta}
\extramarks{AN 11.19}{AN 11.19}

There\marginnote{1.1} the Buddha addressed the mendicants, “Mendicants!” 

“Venerable\marginnote{1.3} sir,” they replied. The Buddha said this: 

“Could\marginnote{2.1} it be, mendicants, that a mendicant might gain a state of immersion like this? They wouldn’t perceive earth in earth, water in water, fire in fire, or air in air. And they wouldn’t perceive the dimension of infinite space in the dimension of infinite space, the dimension of infinite consciousness in the dimension of infinite consciousness, the dimension of nothingness in the dimension of nothingness, or the dimension of neither perception nor non-perception in the dimension of neither perception nor non-perception. They wouldn’t perceive this world in this world, or the other world in the other world. And they wouldn’t perceive what is seen, heard, thought, known, attained, sought, or explored by the mind. And yet they would still perceive.” 

“Our\marginnote{2.2} teachings are rooted in the Buddha. He is our guide and our refuge. Sir, may the Buddha himself please clarify the meaning of this. The mendicants will listen and remember it.” 

“Well\marginnote{3.1} then, mendicants, listen and apply your mind well, I will speak.” 

“Yes,\marginnote{3.2} sir,” they replied. The Buddha said this: 

“A\marginnote{4.1} mendicant could gain such a state of immersion.” 

“But\marginnote{5.1} how could this be?” 

“It’s\marginnote{6.1} when a mendicant perceives: ‘This is peaceful; this is sublime—that is, the stilling of all activities, the letting go of all attachments, the ending of craving, fading away, cessation, extinguishment.’ That’s how a mendicant might gain a state of immersion like this. They wouldn’t perceive earth in earth, water in water, fire in fire, or air in air. And they wouldn’t perceive the dimension of infinite space in the dimension of infinite space, the dimension of infinite consciousness in the dimension of infinite consciousness, the dimension of nothingness in the dimension of nothingness, or the dimension of neither perception nor non-perception in the dimension of neither perception nor non-perception. They wouldn’t perceive this world in this world, or the other world in the other world. And they wouldn’t perceive what is seen, heard, thought, known, attained, sought, or explored by the mind. And yet they would still perceive.” 

%
\section*{{\suttatitleacronym AN 11.20}{\suttatitletranslation Immersion (3rd) }{\suttatitleroot Tatiyasamādhisutta}}
\addcontentsline{toc}{section}{\tocacronym{AN 11.20} \toctranslation{Immersion (3rd) } \tocroot{Tatiyasamādhisutta}}
\markboth{Immersion (3rd) }{Tatiyasamādhisutta}
\extramarks{AN 11.20}{AN 11.20}

And\marginnote{1.1} then several mendicants went up to Venerable \textsanskrit{Sāriputta}, and exchanged greetings with him. When the greetings and polite conversation were over, they sat down to one side and said to him: 

“Could\marginnote{2.1} it be, reverend, that a mendicant might gain a state of immersion like this? They wouldn’t perceive earth in earth, water in water, fire in fire, or air in air. And they wouldn’t perceive the dimension of infinite space in the dimension of infinite space, the dimension of infinite consciousness in the dimension of infinite consciousness, the dimension of nothingness in the dimension of nothingness, or the dimension of neither perception nor non-perception in the dimension of neither perception nor non-perception. They wouldn’t perceive this world in this world, or the other world in the other world. And they wouldn’t perceive what is seen, heard, thought, known, attained, sought, or explored by the mind. And yet they would still perceive.” 

“It\marginnote{2.2} could be, reverends.” 

“But\marginnote{3.1} how could this be?” 

“It’s\marginnote{4.1} when a mendicant perceives: ‘This is peaceful; this is sublime—that is, the stilling of all activities, the letting go of all attachments, the ending of craving, fading away, cessation, extinguishment.’ That’s how a mendicant might gain a state of immersion like this. They wouldn’t perceive earth in earth, water in water, fire in fire, or air in air. And they wouldn’t perceive the dimension of infinite space in the dimension of infinite space, the dimension of infinite consciousness in the dimension of infinite consciousness, the dimension of nothingness in the dimension of nothingness, or the dimension of neither perception nor non-perception in the dimension of neither perception nor non-perception. They wouldn’t perceive this world in this world, or the other world in the other world. And they wouldn’t perceive what is seen, heard, thought, known, attained, sought, or explored by the mind. And yet they would still perceive.” 

%
\section*{{\suttatitleacronym AN 11.21}{\suttatitletranslation Immersion (4th) }{\suttatitleroot Catutthasamādhisutta}}
\addcontentsline{toc}{section}{\tocacronym{AN 11.21} \toctranslation{Immersion (4th) } \tocroot{Catutthasamādhisutta}}
\markboth{Immersion (4th) }{Catutthasamādhisutta}
\extramarks{AN 11.21}{AN 11.21}

There\marginnote{1.1} \textsanskrit{Sāriputta} addressed the mendicants: 

“Could\marginnote{1.2} it be, reverends, that a mendicant might gain a state of immersion like this? They wouldn’t perceive earth in earth, water in water, fire in fire, or air in air. And they wouldn’t perceive the dimension of infinite space in the dimension of infinite space, the dimension of infinite consciousness in the dimension of infinite consciousness, the dimension of nothingness in the dimension of nothingness, or the dimension of neither perception nor non-perception in the dimension of neither perception nor non-perception. They wouldn’t perceive this world in this world, or the other world in the other world. And they wouldn’t perceive what is seen, heard, thought, known, attained, sought, or explored by the mind. And yet they would still perceive.” 

“Reverend,\marginnote{2.1} we would travel a long way to learn the meaning of this statement in the presence of Venerable \textsanskrit{Sāriputta}. May Venerable \textsanskrit{Sāriputta} himself please clarify the meaning of this. The mendicants will listen and remember it.” 

“Then\marginnote{3.1} listen and apply your mind well, I will speak.” 

“Yes,\marginnote{3.2} friend,” they replied. \textsanskrit{Sāriputta} said this: 

“A\marginnote{4.1} mendicant could gain such a state of immersion.” 

“But\marginnote{5.1} how could this be?” 

“It’s\marginnote{6.1} when a mendicant perceives: ‘This is peaceful; this is sublime—that is, the stilling of all activities, the letting go of all attachments, the ending of craving, fading away, cessation, extinguishment.’ That’s how a mendicant might gain a state of immersion like this. They wouldn’t perceive earth in earth, water in water, fire in fire, or air in air. And they wouldn’t perceive the dimension of infinite space in the dimension of infinite space, the dimension of infinite consciousness in the dimension of infinite consciousness, the dimension of nothingness in the dimension of nothingness, or the dimension of neither perception nor non-perception in the dimension of neither perception nor non-perception. They wouldn’t perceive this world in this world, or the other world in the other world. And they wouldn’t perceive what is seen, heard, thought, known, attained, sought, or explored by the mind. And yet they would still perceive.” 

%
\addtocontents{toc}{\let\protect\contentsline\protect\nopagecontentsline}
\chapter*{The Chapter on Similarity }
\addcontentsline{toc}{chapter}{\tocchapterline{The Chapter on Similarity }}
\addtocontents{toc}{\let\protect\contentsline\protect\oldcontentsline}

%
\section*{{\suttatitleacronym AN 11.22–29}{\suttatitletranslation Untitled Discourses on the Eye }{\suttatitleroot \textasciitilde }}
\addcontentsline{toc}{section}{\tocacronym{AN 11.22–29} \toctranslation{Untitled Discourses on the Eye } \tocroot{\textasciitilde }}
\markboth{Untitled Discourses on the Eye }{\textasciitilde }
\extramarks{AN 11.22–29}{AN 11.22–29}

“Mendicants,\marginnote{1.1} a cowherd with eleven factors can’t maintain and expand a herd of cattle. What eleven? It’s when a cowherd doesn’t know form, is unskilled in characteristics, doesn’t pick out flies’ eggs, doesn’t dress wounds, doesn’t spread smoke, doesn’t know the ford, doesn’t know satisfaction, doesn’t know the trail, is not skilled in pastures, milks dry, and doesn’t show extra respect to the bulls who are fathers and leaders of the herd. A cowherd with these eleven factors can’t maintain and expand a herd of cattle. 

In\marginnote{2.1} the same way, a mendicant with eleven qualities can’t meditate observing impermanence in the eye … 

suffering\marginnote{1.1} … 

not-self\marginnote{1.1} … 

ending\marginnote{1.1} … 

vanishing\marginnote{1.1} … 

fading\marginnote{1.1} away … 

cessation\marginnote{1.1} … 

letting\marginnote{1.1} go …” 

%
\section*{{\suttatitleacronym AN 11.30–69}{\suttatitletranslation Untitled Discourses on the Ear, Etc. }{\suttatitleroot \textasciitilde }}
\addcontentsline{toc}{section}{\tocacronym{AN 11.30–69} \toctranslation{Untitled Discourses on the Ear, Etc. } \tocroot{\textasciitilde }}
\markboth{Untitled Discourses on the Ear, Etc. }{\textasciitilde }
\extramarks{AN 11.30–69}{AN 11.30–69}

…\marginnote{1.1} “… ear … nose … tongue … body … mind …” 

%
\section*{{\suttatitleacronym AN 11.70–117}{\suttatitletranslation Untitled Discourses on Sights, Etc. }{\suttatitleroot \textasciitilde }}
\addcontentsline{toc}{section}{\tocacronym{AN 11.70–117} \toctranslation{Untitled Discourses on Sights, Etc. } \tocroot{\textasciitilde }}
\markboth{Untitled Discourses on Sights, Etc. }{\textasciitilde }
\extramarks{AN 11.70–117}{AN 11.70–117}

…\marginnote{1.1} “… sights … sounds … smells … tastes … touches … ideas …” 

%
\section*{{\suttatitleacronym AN 11.118–165}{\suttatitletranslation Untitled Discourses on Eye Consciousness, Etc. }{\suttatitleroot \textasciitilde }}
\addcontentsline{toc}{section}{\tocacronym{AN 11.118–165} \toctranslation{Untitled Discourses on Eye Consciousness, Etc. } \tocroot{\textasciitilde }}
\markboth{Untitled Discourses on Eye Consciousness, Etc. }{\textasciitilde }
\extramarks{AN 11.118–165}{AN 11.118–165}

…\marginnote{1.1} “… eye consciousness … ear consciousness … nose consciousness … tongue consciousness … body consciousness … mind consciousness. …” 

%
\section*{{\suttatitleacronym AN 11.166–213}{\suttatitletranslation Untitled Discourses on Eye Contact, Etc. }{\suttatitleroot \textasciitilde }}
\addcontentsline{toc}{section}{\tocacronym{AN 11.166–213} \toctranslation{Untitled Discourses on Eye Contact, Etc. } \tocroot{\textasciitilde }}
\markboth{Untitled Discourses on Eye Contact, Etc. }{\textasciitilde }
\extramarks{AN 11.166–213}{AN 11.166–213}

…\marginnote{1.1} “… eye contact … ear contact … nose contact … tongue contact … body contact … mind contact. …” 

%
\section*{{\suttatitleacronym AN 11.214–261}{\suttatitletranslation Untitled Discourses on Feeling Born of Eye Contact, Etc. }{\suttatitleroot \textasciitilde }}
\addcontentsline{toc}{section}{\tocacronym{AN 11.214–261} \toctranslation{Untitled Discourses on Feeling Born of Eye Contact, Etc. } \tocroot{\textasciitilde }}
\markboth{Untitled Discourses on Feeling Born of Eye Contact, Etc. }{\textasciitilde }
\extramarks{AN 11.214–261}{AN 11.214–261}

…\marginnote{1.1} “… feeling born of eye contact … feeling born of ear contact … feeling born of nose contact … feeling born of tongue contact … feeling born of body contact … feeling born of mind contact … 

%
\section*{{\suttatitleacronym AN 11.262–309}{\suttatitletranslation Untitled Discourses on Perception of Sights, Etc. }{\suttatitleroot \textasciitilde }}
\addcontentsline{toc}{section}{\tocacronym{AN 11.262–309} \toctranslation{Untitled Discourses on Perception of Sights, Etc. } \tocroot{\textasciitilde }}
\markboth{Untitled Discourses on Perception of Sights, Etc. }{\textasciitilde }
\extramarks{AN 11.262–309}{AN 11.262–309}

…\marginnote{1.1} “… perception of sights … perception of sounds … perception of smells … perception of tastes … perception of touches … perception of ideas. …” 

%
\section*{{\suttatitleacronym AN 11.310–357}{\suttatitletranslation Untitled Discourses on Intention Regarding Sights, Etc. }{\suttatitleroot \textasciitilde }}
\addcontentsline{toc}{section}{\tocacronym{AN 11.310–357} \toctranslation{Untitled Discourses on Intention Regarding Sights, Etc. } \tocroot{\textasciitilde }}
\markboth{Untitled Discourses on Intention Regarding Sights, Etc. }{\textasciitilde }
\extramarks{AN 11.310–357}{AN 11.310–357}

…\marginnote{1.1} “… intention regarding sights … intention regarding sounds … intention regarding smells … intention regarding tastes … intention regarding touches … intention regarding ideas. …” 

%
\section*{{\suttatitleacronym AN 11.358–405}{\suttatitletranslation Untitled Discourses on Craving For Sights, Etc. }{\suttatitleroot \textasciitilde }}
\addcontentsline{toc}{section}{\tocacronym{AN 11.358–405} \toctranslation{Untitled Discourses on Craving For Sights, Etc. } \tocroot{\textasciitilde }}
\markboth{Untitled Discourses on Craving For Sights, Etc. }{\textasciitilde }
\extramarks{AN 11.358–405}{AN 11.358–405}

…\marginnote{1.1} “… Craving for sights … craving for sounds … craving for smells … craving for tastes … craving for touches … craving for ideas. …” 

%
\section*{{\suttatitleacronym AN 11.406–453}{\suttatitletranslation Untitled Discourses on Thoughts About Sights, Etc. }{\suttatitleroot \textasciitilde }}
\addcontentsline{toc}{section}{\tocacronym{AN 11.406–453} \toctranslation{Untitled Discourses on Thoughts About Sights, Etc. } \tocroot{\textasciitilde }}
\markboth{Untitled Discourses on Thoughts About Sights, Etc. }{\textasciitilde }
\extramarks{AN 11.406–453}{AN 11.406–453}

…\marginnote{1.1} “… thoughts about sights … thoughts about sounds … thoughts about smells … thoughts about tastes … thoughts about touches … thoughts about ideas. …” 

%
\section*{{\suttatitleacronym AN 11.454–501}{\suttatitletranslation Untitled Discourses on Considerations Regarding Sights, Etc. }{\suttatitleroot \textasciitilde }}
\addcontentsline{toc}{section}{\tocacronym{AN 11.454–501} \toctranslation{Untitled Discourses on Considerations Regarding Sights, Etc. } \tocroot{\textasciitilde }}
\markboth{Untitled Discourses on Considerations Regarding Sights, Etc. }{\textasciitilde }
\extramarks{AN 11.454–501}{AN 11.454–501}

…\marginnote{1.1} “… considerations regarding sights … considerations regarding sounds … considerations regarding smells … considerations regarding tastes … considerations regarding touches … meditate observing impermanence in considerations about ideas … meditate observing suffering … meditate observing not-self … meditate observing ending … meditate observing vanishing … meditate observing fading away … meditate observing cessation … meditate observing letting go. …” 

%
\section*{{\suttatitleacronym AN 11.502–981}{\suttatitletranslation Untitled Discourses on the Eye, Etc. }{\suttatitleroot \textasciitilde }}
\addcontentsline{toc}{section}{\tocacronym{AN 11.502–981} \toctranslation{Untitled Discourses on the Eye, Etc. } \tocroot{\textasciitilde }}
\markboth{Untitled Discourses on the Eye, Etc. }{\textasciitilde }
\extramarks{AN 11.502–981}{AN 11.502–981}

“Mendicants,\marginnote{1.1} a cowherd with eleven factors can maintain and expand a herd of cattle. What eleven? It’s when a cowherd knows form … 

In\marginnote{2.1} the same way, a mendicant with eleven qualities can meditate observing impermanence in the eye … meditate observing letting go. …” 

%
\addtocontents{toc}{\let\protect\contentsline\protect\nopagecontentsline}
\chapter*{Abbreviated Texts Beginning With Greed }
\addcontentsline{toc}{chapter}{\tocchapterline{Abbreviated Texts Beginning With Greed }}
\addtocontents{toc}{\let\protect\contentsline\protect\oldcontentsline}

%
\section*{{\suttatitleacronym AN 11.982}{\suttatitletranslation Untitled Discourse on Greed }{\suttatitleroot \textasciitilde }}
\addcontentsline{toc}{section}{\tocacronym{AN 11.982} \toctranslation{Untitled Discourse on Greed } \tocroot{\textasciitilde }}
\markboth{Untitled Discourse on Greed }{\textasciitilde }
\extramarks{AN 11.982}{AN 11.982}

“For\marginnote{1.1} insight into greed, eleven things should be developed. What eleven? The first, second, third, and fourth absorptions; the heart’s releases by love, compassion, rejoicing, and equanimity; the dimensions of infinite space, infinite consciousness, and nothingness. For insight into greed, these eleven things should be developed.” 

%
\section*{{\suttatitleacronym AN 11.983–991}{\suttatitletranslation Untitled Discourses on Greed }{\suttatitleroot \textasciitilde }}
\addcontentsline{toc}{section}{\tocacronym{AN 11.983–991} \toctranslation{Untitled Discourses on Greed } \tocroot{\textasciitilde }}
\markboth{Untitled Discourses on Greed }{\textasciitilde }
\extramarks{AN 11.983–991}{AN 11.983–991}

“For\marginnote{1.1} the complete understanding of greed … complete ending … giving up … ending … vanishing … fading away … cessation … giving away … letting go … these eleven things should be developed.” 

%
\section*{{\suttatitleacronym AN 11.992–1151}{\suttatitletranslation Untitled Discourses on Hate, Etc. }{\suttatitleroot \textasciitilde }}
\addcontentsline{toc}{section}{\tocacronym{AN 11.992–1151} \toctranslation{Untitled Discourses on Hate, Etc. } \tocroot{\textasciitilde }}
\markboth{Untitled Discourses on Hate, Etc. }{\textasciitilde }
\extramarks{AN 11.992–1151}{AN 11.992–1151}

“Of\marginnote{1.1} hate … delusion … anger … acrimony … disdain … contempt … jealousy … stinginess … deceitfulness … deviousness … obstinacy … aggression … conceit … arrogance … vanity … for insight into negligence … complete understanding … complete ending … giving up … ending … vanishing … fading away … cessation … giving away … For the letting go of negligence, these eleven things should be developed.” 

That\marginnote{2.1} is what the Buddha said. Satisfied, the mendicants approved what the Buddha said. 

\scendbook{The Book of the Elevens is finished. }

\scendbook{The Numbered Discourses are completed. }

%
\backmatter%
\chapter*{Colophon}
\addcontentsline{toc}{chapter}{Colophon}
\markboth{Colophon}{Colophon}

\section*{The Translator}

Bhikkhu Sujato was born as Anthony Aidan Best on 4/11/1966 in Perth, Western Australia. He grew up in the pleasant suburbs of Mt Lawley and Attadale alongside his sister Nicola, who was the good child. His mother, Margaret Lorraine Huntsman née Pinder, said “he’ll either be a priest or a poet”, while his father, Anthony Thomas Best, advised him to “never do anything for money”. He attended Aquinas College, a Catholic school, where he decided to become an atheist. At the University of WA he studied philosophy, aiming to learn what he wanted to do with his life. Finding that what he wanted to do was play guitar, he dropped out. His main band was named Martha’s Vineyard, which achieved modest success in the indie circuit. 

A seemingly random encounter with a roadside joey took him to Thailand, where he entered his first meditation retreat at Wat Ram Poeng, Chieng Mai in 1992. Feeling the call to the Buddha’s path, he took full ordination in Wat Pa Nanachat in 1994, where his teachers were Ajahn Pasanno and Ajahn Jayasaro. In 1997 he returned to Perth to study with Ajahn Brahm at Bodhinyana Monastery. 

He spent several years practicing in seclusion in Malaysia and Thailand before establishing Santi Forest Monastery in Bundanoon, NSW, in 2003. There he was instrumental in supporting the establishment of the Theravada bhikkhuni order in Australia and advocating for women’s rights. He continues to teach in Australia and globally, with a special concern for the moral implications of climate change and other forms of environmental destruction. He has published a series of books of original and groundbreaking research on early Buddhism. 

In 2005 he founded SuttaCentral together with Rod Bucknell and John Kelly. In 2015, seeing the need for a complete, accurate, plain English translation of the Pali texts, he undertook the task, spending nearly three years in isolation on the isle of Qi Mei off the coast of the nation of Taiwan. He completed the four main \textsanskrit{Nikāyas} in 2018, and the early books of the Khuddaka \textsanskrit{Nikāya} were complete by 2021. All this work is dedicated to the public domain and is entirely free of copyright encumbrance. 

In 2019 he returned to Sydney where he established Lokanta Vihara (The Monastery at the End of the World). 

\section*{Creation Process}

Primary source was the digital \textsanskrit{Mahāsaṅgīti} edition of the Pali \textsanskrit{Tipiṭaka}. Translated from the Pali, with reference to several English translations, especially those of Bhikkhu Bodhi.

\section*{The Translation}

This translation was part of a project to translate the four Pali \textsanskrit{Nikāyas} with the following aims: plain, approachable English; consistent terminology; accurate rendition of the Pali; free of copyright. It was made during 2016–2018 while Bhikkhu Sujato was staying in Qimei, Taiwan.

\section*{About SuttaCentral}

SuttaCentral publishes early Buddhist texts. Since 2005 we have provided root texts in Pali, Chinese, Sanskrit, Tibetan, and other languages, parallels between these texts, and translations in many modern languages. Building on the work of generations of scholars, we offer our contribution freely.

SuttaCentral is driven by volunteer contributions, and in addition we employ professional developers. We offer a sponsorship program for high quality translations from the original languages. Financial support for SuttaCentral is handled by the SuttaCentral Development Trust, a charitable trust registered in Australia.

\section*{About Bilara}

“Bilara” means “cat” in Pali, and it is the name of our Computer Assisted Translation (CAT) software. Bilara is a web app that enables translators to translate early Buddhist texts into their own language. These translations are published on SuttaCentral with the root text and translation side by side.

\section*{About SuttaCentral Editions}

The SuttaCentral Editions project makes high quality books from selected Bilara translations. These are published in formats including HTML, EPUB, PDF, and print.

You are welcome to print any of our Editions.

%
\end{document}