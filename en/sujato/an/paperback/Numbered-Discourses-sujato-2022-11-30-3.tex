\documentclass[12pt,openany]{book}%
\usepackage{lastpage}%
%
\usepackage[inner=1in, outer=1in, top=.7in, bottom=1in, papersize={6in,9in}, headheight=13pt]{geometry}
\usepackage{polyglossia}
\usepackage[12pt]{moresize}
\usepackage{soul}%
\usepackage{microtype}
\usepackage{tocbasic}
\usepackage{realscripts}
\usepackage{epigraph}%
\usepackage{setspace}%
\usepackage{sectsty}
\usepackage{fontspec}
\usepackage{marginnote}
\usepackage[bottom]{footmisc}
\usepackage{enumitem}
\usepackage{fancyhdr}
\usepackage{extramarks}
\usepackage{graphicx}
\usepackage{verse}
\usepackage{relsize}
\usepackage{etoolbox}
\usepackage[a-3u]{pdfx}

\hypersetup{
colorlinks=true,
urlcolor=black,
linkcolor=black,
citecolor=black
}

% use a small amount of tracking on small caps
\SetTracking[ spacing = {25*,166, } ]{ encoding = *, shape = sc }{ 25 }

% add a blank page
\newcommand{\blankpage}{
\newpage
\thispagestyle{empty}
\mbox{}
\newpage
}

% define languages
\setdefaultlanguage[]{english}
\setotherlanguage[script=Latin]{sanskrit}

%\usepackage{pagegrid}
%\pagegridsetup{top-left, step=.25in}

% define fonts
% use if arno sanskrit is unavailable
%\setmainfont{Gentium Plus}
%\newfontfamily\Semiboldsubheadfont[]{Gentium Plus}
%\newfontfamily\Semiboldnormalfont[]{Gentium Plus}
%\newfontfamily\Lightfont[]{Gentium Plus}
%\newfontfamily\Marginalfont[]{Gentium Plus}
%\newfontfamily\Allsmallcapsfont[RawFeature=+c2sc]{Gentium Plus}
%\newfontfamily\Noligaturefont[Renderer=Basic]{Gentium Plus}
%\newfontfamily\Noligaturecaptionfont[Renderer=Basic]{Gentium Plus}
%\newfontfamily\Fleuronfont[Ornament=1]{Gentium Plus}

% use if arno sanskrit is available. display is applied to \chapter and \part, subhead to \section and \subsection. When specifying semibold, the italic must be defined.
\setmainfont[Numbers=OldStyle]{Arno Pro}
\newfontfamily\Semibolddisplayfont[BoldItalicFont = Arno Pro Semibold Italic Display]{Arno Pro Semibold Display} %
\newfontfamily\Semiboldsubheadfont[BoldItalicFont = Arno Pro Semibold Italic Subhead]{Arno Pro Semibold Subhead}
\newfontfamily\Semiboldnormalfont[BoldItalicFont = Arno Pro Semibold Italic]{Arno Pro Semibold}
\newfontfamily\Marginalfont[RawFeature=+subs]{Arno Pro Regular}
\newfontfamily\Allsmallcapsfont[RawFeature=+c2sc]{Arno Pro}
\newfontfamily\Noligaturefont[Renderer=Basic]{Arno Pro}
\newfontfamily\Noligaturecaptionfont[Renderer=Basic]{Arno Pro Caption}

% chinese fonts
\newfontfamily\cjk{Noto Serif TC}
\newcommand*{\langlzh}[1]{\cjk{#1}\normalfont}%

% logo
\newfontfamily\Logofont{sclogo.ttf}
\newcommand*{\sclogo}[1]{\large\Logofont{#1}}

% use subscript numerals for margin notes
\renewcommand*{\marginfont}{\Marginalfont}

% ensure margin notes have consistent vertical alignment
\renewcommand*{\marginnotevadjust}{-.17em}

% use compact lists
\setitemize{noitemsep,leftmargin=1em}
\setenumerate{noitemsep,leftmargin=1em}
\setdescription{noitemsep, style=unboxed, leftmargin=0em}

% style ToC
\DeclareTOCStyleEntries[
  raggedentrytext,
  linefill=\hfill,
  pagenumberwidth=.5in,
  pagenumberformat=\normalfont,
  entryformat=\normalfont
]{tocline}{chapter,section}


  \setlength\topsep{0pt}%
  \setlength\parskip{0pt}%

% define new \centerpars command for use in ToC. This ensures centering, proper wrapping, and no page break after
\def\startcenter{%
  \par
  \begingroup
  \leftskip=0pt plus 1fil
  \rightskip=\leftskip
  \parindent=0pt
  \parfillskip=0pt
}
\def\stopcenter{%
  \par
  \endgroup
}
\long\def\centerpars#1{\startcenter#1\stopcenter}

% redefine part, so that it adds a toc entry without page number
\let\oldcontentsline\contentsline
\newcommand{\nopagecontentsline}[3]{\oldcontentsline{#1}{#2}{}}

    \makeatletter
\renewcommand*\l@part[2]{%
  \ifnum \c@tocdepth >-2\relax
    \addpenalty{-\@highpenalty}%
    \addvspace{0em \@plus\p@}%
    \setlength\@tempdima{3em}%
    \begingroup
      \parindent \z@ \rightskip \@pnumwidth
      \parfillskip -\@pnumwidth
      {\leavevmode
       \setstretch{.85}\large\scshape\centerpars{#1}\vspace*{-1em}\llap{#2}}\par
       \nobreak
         \global\@nobreaktrue
         \everypar{\global\@nobreakfalse\everypar{}}%
    \endgroup
  \fi}
\makeatother

\makeatletter
\def\@pnumwidth{2em}
\makeatother

% define new sectioning command, which is only used in volumes where the pannasa is found in some parts but not others, especially in an and sn

\newcommand*{\pannasa}[1]{\clearpage\thispagestyle{empty}\begin{center}\vspace*{14em}\setstretch{.85}\huge\itshape\scshape\MakeLowercase{#1}\end{center}}

    \makeatletter
\newcommand*\l@pannasa[2]{%
  \ifnum \c@tocdepth >-2\relax
    \addpenalty{-\@highpenalty}%
    \addvspace{.5em \@plus\p@}%
    \setlength\@tempdima{3em}%
    \begingroup
      \parindent \z@ \rightskip \@pnumwidth
      \parfillskip -\@pnumwidth
      {\leavevmode
       \setstretch{.85}\large\itshape\scshape\lowercase{\centerpars{#1}}\vspace*{-1em}\llap{#2}}\par
       \nobreak
         \global\@nobreaktrue
         \everypar{\global\@nobreakfalse\everypar{}}%
    \endgroup
  \fi}
\makeatother

% don't put page number on first page of toc (relies on etoolbox)
\patchcmd{\chapter}{plain}{empty}{}{}

% global line height
\setstretch{1.05}

% allow linebreak after em-dash
\catcode`\—=13
\protected\def—{\unskip\textemdash\allowbreak}

% style headings with secsty. chapter and section are defined per-edition
\partfont{\setstretch{.85}\normalfont\centering\textsc}
\subsectionfont{\setstretch{.85}\Semiboldsubheadfont}%
\subsubsectionfont{\setstretch{.85}\Semiboldnormalfont}

% style elements of suttatitle
\newcommand*{\suttatitleacronym}[1]{\smaller[2]{#1}\vspace*{.3em}}
\newcommand*{\suttatitletranslation}[1]{\linebreak{#1}}
\newcommand*{\suttatitleroot}[1]{\linebreak\smaller[2]\itshape{#1}}

\DeclareTOCStyleEntries[
  indent=3.3em,
  dynindent,
  beforeskip=.2em plus -2pt minus -1pt,
]{tocline}{section}

\DeclareTOCStyleEntries[
  indent=0em,
  dynindent,
  beforeskip=.4em plus -2pt minus -1pt,
]{tocline}{chapter}

\newcommand*{\tocacronym}[1]{\hspace*{-3.3em}{#1}\quad}
\newcommand*{\toctranslation}[1]{#1}
\newcommand*{\tocroot}[1]{(\textit{#1})}
\newcommand*{\tocchapterline}[1]{\bfseries\itshape{#1}}


% redefine paragraph and subparagraph headings to not be inline
\makeatletter
% Change the style of paragraph headings %
\renewcommand\paragraph{\@startsection{paragraph}{4}{\z@}%
            {-2.5ex\@plus -1ex \@minus -.25ex}%
            {1.25ex \@plus .25ex}%
            {\noindent\Semiboldnormalfont\normalsize}}

% Change the style of subparagraph headings %
\renewcommand\subparagraph{\@startsection{subparagraph}{5}{\z@}%
            {-2.5ex\@plus -1ex \@minus -.25ex}%
            {1.25ex \@plus .25ex}%
            {\noindent\Semiboldnormalfont\small}}
\makeatother

% use etoolbox to suppress page numbers on \part
\patchcmd{\part}{\thispagestyle{plain}}{\thispagestyle{empty}}
  {}{\errmessage{Cannot patch \string\part}}

% and to reduce margins on quotation
\patchcmd{\quotation}{\rightmargin}{\leftmargin 1.2em \rightmargin}{}{}
\AtBeginEnvironment{quotation}{\small}

% titlepage
\newcommand*{\titlepageTranslationTitle}[1]{{\begin{center}\begin{large}{#1}\end{large}\end{center}}}
\newcommand*{\titlepageCreatorName}[1]{{\begin{center}\begin{normalsize}{#1}\end{normalsize}\end{center}}}

% halftitlepage
\newcommand*{\halftitlepageTranslationTitle}[1]{\setstretch{2.5}{\begin{Huge}\uppercase{\so{#1}}\end{Huge}}}
\newcommand*{\halftitlepageTranslationSubtitle}[1]{\setstretch{1.2}{\begin{large}{#1}\end{large}}}
\newcommand*{\halftitlepageFleuron}[1]{{\begin{large}\Fleuronfont{{#1}}\end{large}}}
\newcommand*{\halftitlepageByline}[1]{{\begin{normalsize}\textit{{#1}}\end{normalsize}}}
\newcommand*{\halftitlepageCreatorName}[1]{{\begin{LARGE}{\textsc{#1}}\end{LARGE}}}
\newcommand*{\halftitlepageVolumeNumber}[1]{{\begin{normalsize}{\Allsmallcapsfont{\textsc{#1}}}\end{normalsize}}}
\newcommand*{\halftitlepageVolumeAcronym}[1]{{\begin{normalsize}{#1}\end{normalsize}}}
\newcommand*{\halftitlepageVolumeTranslationTitle}[1]{{\begin{Large}{\textsc{#1}}\end{Large}}}
\newcommand*{\halftitlepageVolumeRootTitle}[1]{{\begin{normalsize}{\Allsmallcapsfont{\textsc{\itshape #1}}}\end{normalsize}}}
\newcommand*{\halftitlepagePublisher}[1]{{\begin{large}{\Noligaturecaptionfont\textsc{#1}}\end{large}}}

% epigraph
\renewcommand{\epigraphflush}{center}
\renewcommand*{\epigraphwidth}{.85\textwidth}
\newcommand*{\epigraphTranslatedTitle}[1]{\vspace*{.5em}\footnotesize\textsc{#1}\\}%
\newcommand*{\epigraphRootTitle}[1]{\footnotesize\textit{#1}\\}%
\newcommand*{\epigraphReference}[1]{\footnotesize{#1}}%

% custom commands for html styling classes
\newcommand*{\scnamo}[1]{\begin{center}\textit{#1}\end{center}}
\newcommand*{\scendsection}[1]{\begin{center}\textit{#1}\end{center}}
\newcommand*{\scendsutta}[1]{\begin{center}\textit{#1}\end{center}}
\newcommand*{\scendbook}[1]{\begin{center}\uppercase{#1}\end{center}}
\newcommand*{\scendkanda}[1]{\begin{center}\textbf{#1}\end{center}}
\newcommand*{\scend}[1]{\begin{center}\textit{#1}\end{center}}
\newcommand*{\scuddanaintro}[1]{\textit{#1}}
\newcommand*{\scendvagga}[1]{\begin{center}\textbf{#1}\end{center}}
\newcommand*{\scrule}[1]{\textbf{#1}}
\newcommand*{\scadd}[1]{\textit{#1}}
\newcommand*{\scevam}[1]{\textsc{#1}}
\newcommand*{\scspeaker}[1]{\hspace{2em}\textit{#1}}
\newcommand*{\scbyline}[1]{\begin{flushright}\textit{#1}\end{flushright}\bigskip}

% custom command for thematic break = hr
\newcommand*{\thematicbreak}{\begin{center}\rule[.5ex]{6em}{.4pt}\begin{normalsize}\quad\Fleuronfont{•}\quad\end{normalsize}\rule[.5ex]{6em}{.4pt}\end{center}}

% manage and style page header and footer. "fancy" has header and footer, "plain" has footer only

\pagestyle{fancy}
\fancyhf{}
\fancyfoot[RE,LO]{\thepage}
\fancyfoot[LE,RO]{\footnotesize\lastleftxmark}
\fancyhead[CE]{\setstretch{.85}\Noligaturefont\MakeLowercase{\textsc{\firstrightmark}}}
\fancyhead[CO]{\setstretch{.85}\Noligaturefont\MakeLowercase{\textsc{\firstleftmark}}}
\renewcommand{\headrulewidth}{0pt}
\fancypagestyle{plain}{ %
\fancyhf{} % remove everything
\fancyfoot[RE,LO]{\thepage}
\fancyfoot[LE,RO]{\footnotesize\lastleftxmark}
\renewcommand{\headrulewidth}{0pt}
\renewcommand{\footrulewidth}{0pt}}

% style footnotes
\setlength{\skip\footins}{1em}

\makeatletter
\newcommand{\@makefntextcustom}[1]{%
    \parindent 0em%
    \thefootnote.\enskip #1%
}
\renewcommand{\@makefntext}[1]{\@makefntextcustom{#1}}
\makeatother

% hang quotes (requires microtype)
\microtypesetup{
  protrusion = true,
  expansion  = true,
  tracking   = true,
  factor     = 1000,
  patch      = all,
  final
}

% Custom protrusion rules to allow hanging punctuation
\SetProtrusion
{ encoding = *}
{
% char   right left
  {-} = {    , 500 },
  % Double Quotes
  \textquotedblleft
      = {1000,     },
  \textquotedblright
      = {    , 1000},
  \quotedblbase
      = {1000,     },
  % Single Quotes
  \textquoteleft
      = {1000,     },
  \textquoteright
      = {    , 1000},
  \quotesinglbase
      = {1000,     }
}

% make latex use actual font em for parindent, not Computer Modern Roman
\AtBeginDocument{\setlength{\parindent}{1em}}%
%

% Default values; a bit sloppier than normal
\tolerance 1414
\hbadness 1414
\emergencystretch 1.5em
\hfuzz 0.3pt
\clubpenalty = 10000
\widowpenalty = 10000
\displaywidowpenalty = 10000
\hfuzz \vfuzz
 \raggedbottom%

\title{Numbered Discourses}
\author{Bhikkhu Sujato}
\date{}%
% define a different fleuron for each edition
\newfontfamily\Fleuronfont[Ornament=18]{Arno Pro}

% Define heading styles per edition for chapter and section. Suttatitle can be either of these, depending on the volume. 

\let\oldfrontmatter\frontmatter
\renewcommand{\frontmatter}{%
\chapterfont{\setstretch{.85}\normalfont\centering}%
\sectionfont{\setstretch{.85}\Semiboldsubheadfont}%
\oldfrontmatter}

\let\oldmainmatter\mainmatter
\renewcommand{\mainmatter}{%
\chapterfont{\setstretch{.85}\normalfont\centering}%
\sectionfont{\setstretch{.85}\normalfont\centering}%
\oldmainmatter}

\let\oldbackmatter\backmatter
\renewcommand{\backmatter}{%
\chapterfont{\setstretch{.85}\normalfont\centering}%
\sectionfont{\setstretch{.85}\Semiboldsubheadfont}%
\oldbackmatter}
%
%
\begin{document}%
\normalsize%
\frontmatter%
\setlength{\parindent}{0cm}

\pagestyle{empty}

\maketitle

\blankpage%
\begin{center}

\vspace*{2.2em}

\halftitlepageTranslationTitle{Numbered Discourses}

\vspace*{1em}

\halftitlepageTranslationSubtitle{A sensible translation of the Aṅguttara Nikāya}

\vspace*{2em}

\halftitlepageFleuron{•}

\vspace*{2em}

\halftitlepageByline{translated and introduced by}

\vspace*{.5em}

\halftitlepageCreatorName{Bhikkhu Sujato}

\vspace*{4em}

\halftitlepageVolumeNumber{Volume 3}

\smallskip

\halftitlepageVolumeAcronym{AN 5–6}

\smallskip

\halftitlepageVolumeTranslationTitle{}

\smallskip

\halftitlepageVolumeRootTitle{}

\vspace*{\fill}

\sclogo{0}
 \halftitlepagePublisher{SuttaCentral}

\end{center}

\newpage
%
\setstretch{1.05}

\begin{footnotesize}

\textit{Numbered Discourses} is a translation of the Aṅguttaranikāya by Bhikkhu Sujato.

\medskip

Creative Commons Zero (CC0)

To the extent possible under law, Bhikkhu Sujato has waived all copyright and related or neighboring rights to \textit{Numbered Discourses}.

\medskip

This work is published from Australia.

\begin{center}
\textit{This translation is an expression of an ancient spiritual text that has been passed down by the Buddhist tradition for the benefit of all sentient beings. It is dedicated to the public domain via Creative Commons Zero (CC0). You are encouraged to copy, reproduce, adapt, alter, or otherwise make use of this translation. The translator respectfully requests that any use be in accordance with the values and principles of the Buddhist community.}
\end{center}

\medskip

\begin{description}
    \item[Web publication date] 2018
    \item[This edition] 2022-11-30 08:48:21
    \item[Publication type] paperback
    \item[Edition] ed5
    \item[Number of volumes] 5
    \item[Publication ISBN] 978-1-76132-037-8
    \item[Publication URL] https://suttacentral.net/editions/an/en/sujato
    \item[Source URL] https://github.com/suttacentral/bilara-data/tree/published/translation/en/sujato/sutta/an
    \item[Publication number] scpub5
\end{description}

\medskip

Published by SuttaCentral

\medskip

\textit{SuttaCentral,\\
c/o Alwis \& Alwis Pty Ltd\\
Kaurna Country,\\
Suite 12,\\
198 Greenhill Road,\\
Eastwood,\\
SA 5063,\\
Australia}

\end{footnotesize}

\newpage

\setlength{\parindent}{1.5em}%%
\tableofcontents
\newpage
\pagestyle{fancy}
%
\mainmatter%
\pagestyle{fancy}%
\addtocontents{toc}{\let\protect\contentsline\protect\nopagecontentsline}
\part*{The Book of the Fives }
\addcontentsline{toc}{part}{The Book of the Fives }
\markboth{}{}
\addtocontents{toc}{\let\protect\contentsline\protect\oldcontentsline}

%
%
\addtocontents{toc}{\let\protect\contentsline\protect\nopagecontentsline}
\pannasa{The First Fifty }
\addcontentsline{toc}{pannasa}{The First Fifty }
\markboth{}{}
\addtocontents{toc}{\let\protect\contentsline\protect\oldcontentsline}

%
\addtocontents{toc}{\let\protect\contentsline\protect\nopagecontentsline}
\chapter*{The Chapter on Powers of a Trainee }
\addcontentsline{toc}{chapter}{\tocchapterline{The Chapter on Powers of a Trainee }}
\addtocontents{toc}{\let\protect\contentsline\protect\oldcontentsline}

%
\section*{{\suttatitleacronym AN 5.1}{\suttatitletranslation In Brief }{\suttatitleroot Saṁkhittasutta}}
\addcontentsline{toc}{section}{\tocacronym{AN 5.1} \toctranslation{In Brief } \tocroot{Saṁkhittasutta}}
\markboth{In Brief }{Saṁkhittasutta}
\extramarks{AN 5.1}{AN 5.1}

\scevam{So\marginnote{1.1} I have heard. }At one time the Buddha was staying near \textsanskrit{Sāvatthī} in Jeta’s Grove, \textsanskrit{Anāthapiṇḍika}’s monastery. There the Buddha addressed the mendicants, “Mendicants!” 

“Venerable\marginnote{1.5} sir,” they replied. The Buddha said this: 

“Mendicants,\marginnote{2.1} there are these five powers of a trainee. What five? The powers of faith, conscience, prudence, energy, and wisdom. These are the five powers of a trainee. 

So\marginnote{3.1} you should train like this: ‘We will have the trainee’s powers of faith, conscience, prudence, energy, and wisdom.’ That’s how you should train.” 

That\marginnote{3.4} is what the Buddha said. Satisfied, the mendicants were happy with what the Buddha said. 

%
\section*{{\suttatitleacronym AN 5.2}{\suttatitletranslation In Detail }{\suttatitleroot Vitthatasutta}}
\addcontentsline{toc}{section}{\tocacronym{AN 5.2} \toctranslation{In Detail } \tocroot{Vitthatasutta}}
\markboth{In Detail }{Vitthatasutta}
\extramarks{AN 5.2}{AN 5.2}

“Mendicants,\marginnote{1.1} there are these five powers of a trainee. What five? The powers of faith, conscience, prudence, energy, and wisdom. 

And\marginnote{2.1} what is the power of faith? It’s when a noble disciple has faith in the Realized One’s awakening: ‘That Blessed One is perfected, a fully awakened Buddha, accomplished in knowledge and conduct, holy, knower of the world, supreme guide for those who wish to train, teacher of gods and humans, awakened, blessed.’ This is called the power of faith. 

And\marginnote{3.1} what is the power of conscience? It’s when a noble disciple has a conscience. They’re conscientious about bad conduct by way of body, speech, and mind, and conscientious about having any bad, unskillful qualities. This is called the power of conscience. 

And\marginnote{4.1} what is the power of prudence? It’s when a noble disciple is prudent. They’re prudent when it comes to bad conduct by way of body, speech, and mind, and prudent when it comes to acquiring any bad, unskillful qualities. This is called the power of prudence. 

And\marginnote{5.1} what is the power of energy? It’s when a mendicant lives with energy roused up for giving up unskillful qualities and embracing skillful qualities. They’re strong, staunchly vigorous, not slacking off when it comes to developing skillful qualities. This is called the power of energy. 

And\marginnote{6.1} what is the power of wisdom? It’s when a noble disciple is wise. They have the wisdom of arising and passing away which is noble, penetrative, and leads to the complete ending of suffering. This is called the power of wisdom. These are the five powers of a trainee. 

So\marginnote{7.1} you should train like this: ‘We will have the trainee’s powers of faith, conscience, prudence, energy, and wisdom.’ That’s how you should train.” 

%
\section*{{\suttatitleacronym AN 5.3}{\suttatitletranslation Suffering }{\suttatitleroot Dukkhasutta}}
\addcontentsline{toc}{section}{\tocacronym{AN 5.3} \toctranslation{Suffering } \tocroot{Dukkhasutta}}
\markboth{Suffering }{Dukkhasutta}
\extramarks{AN 5.3}{AN 5.3}

“Mendicants,\marginnote{1.1} when a mendicant has five qualities they live unhappily in the present life—with distress, anguish, and fever—and when the body breaks up, after death, they can expect a bad rebirth. What five? It’s when a mendicant is faithless, shameless, imprudent, lazy, and witless. When a mendicant has these five qualities they live unhappily in the present life—with distress, anguish, and fever—and when the body breaks up, after death, they can expect a bad rebirth. 

When\marginnote{2.1} a mendicant has five qualities they live happily in the present life—without distress, anguish, or fever—and when the body breaks up, after death, they can expect a good rebirth. What five? It’s when a mendicant is faithful, conscientious, prudent, energetic, and wise. When a mendicant has these five qualities they live happily in the present life—without distress, anguish, or fever—and when the body breaks up, after death, they can expect a good rebirth.” 

%
\section*{{\suttatitleacronym AN 5.4}{\suttatitletranslation Cast Down }{\suttatitleroot Yathābhatasutta}}
\addcontentsline{toc}{section}{\tocacronym{AN 5.4} \toctranslation{Cast Down } \tocroot{Yathābhatasutta}}
\markboth{Cast Down }{Yathābhatasutta}
\extramarks{AN 5.4}{AN 5.4}

“Mendicants,\marginnote{1.1} a mendicant with five qualities is cast down to hell. What five? It’s when a mendicant is faithless, shameless, imprudent, lazy, and witless. A mendicant with these five qualities is cast down to hell. 

A\marginnote{2.1} mendicant with five qualities is raised up to heaven. What five? It’s when a mendicant is faithful, conscientious, prudent, energetic, and wise. A mendicant with these five qualities is raised up to heaven.” 

%
\section*{{\suttatitleacronym AN 5.5}{\suttatitletranslation Disrobing }{\suttatitleroot Sikkhāsutta}}
\addcontentsline{toc}{section}{\tocacronym{AN 5.5} \toctranslation{Disrobing } \tocroot{Sikkhāsutta}}
\markboth{Disrobing }{Sikkhāsutta}
\extramarks{AN 5.5}{AN 5.5}

“Mendicants,\marginnote{1.1} any monk or nun who resigns the training and returns to a lesser life deserves rebuke and criticism on five legitimate grounds in the present life. What five? ‘You had no faith, conscience, prudence, energy, or wisdom regarding skillful qualities.’ Any monk or nun who resigns the training and returns to a lesser life deserves rebuke and criticism on these five legitimate grounds in the present life. 

Any\marginnote{2.1} monk or nun who lives the full and pure spiritual life in pain and sadness, weeping, with tearful face, deserves praise on five legitimate grounds in the present life. What five? ‘You had faith, conscience, prudence, energy, and wisdom regarding skillful qualities.’ Any monk or nun who lives the full and pure spiritual life in pain and sadness, weeping, with tearful face, deserves praise on these five legitimate grounds in the present life.” 

%
\section*{{\suttatitleacronym AN 5.6}{\suttatitletranslation Becoming }{\suttatitleroot Samāpattisutta}}
\addcontentsline{toc}{section}{\tocacronym{AN 5.6} \toctranslation{Becoming } \tocroot{Samāpattisutta}}
\markboth{Becoming }{Samāpattisutta}
\extramarks{AN 5.6}{AN 5.6}

“Mendicants,\marginnote{1.1} you don’t become unskillful as long as faith is established in skillful qualities. But when faith vanishes and faithlessness takes over, you become unskillful. 

You\marginnote{2.1} don’t become unskillful as long as conscience … prudence … energy … wisdom is established in skillful qualities. 

But\marginnote{5.1} when wisdom vanishes and witlessness takes over, you become unskillful.” 

%
\section*{{\suttatitleacronym AN 5.7}{\suttatitletranslation Sensual Pleasures }{\suttatitleroot Kāmasutta}}
\addcontentsline{toc}{section}{\tocacronym{AN 5.7} \toctranslation{Sensual Pleasures } \tocroot{Kāmasutta}}
\markboth{Sensual Pleasures }{Kāmasutta}
\extramarks{AN 5.7}{AN 5.7}

“Mendicants,\marginnote{1.1} sentient beings are mostly charmed by sensual pleasures. When a gentleman has abandoned the scythe and flail and gone forth from the lay life to homelessness, they’re qualified to be called ‘a faithful renunciate from a good family’. Why is that? Because a youth can get sensual pleasures of this kind or that. Now, all sensual pleasures are just reckoned as ‘sensual pleasures’, regardless of whether they’re inferior, average, or superior. 

Suppose\marginnote{1.6} there was a little baby boy who, because of his nurse’s negligence, puts a stick or stone in his mouth. The nurse would very quickly notice and try to take it out. If that didn’t work, she’d cradle his head with her left hand, and take it out using a hooked finger of her right hand, even if it drew blood. Why is that? I admit she’d know, ‘This will distress the child, there’s no denying.’ Still, it should be done by a nurse who wants what’s best for him, out of kindness and compassion. And when the boy has grown up and has enough sense, his nurse would not worry about him, thinking: ‘The boy can look after himself. He won’t be negligent.’ 

In\marginnote{2.1} the same way, I still need to look after a mendicant who hasn’t finished developing faith, conscience, prudence, energy, and wisdom regarding skillful qualities. But when a mendicant has finished developing faith, conscience, prudence, energy, and wisdom regarding skillful qualities, I need not be concerned, thinking: ‘They can look after themselves. They won’t be negligent.’” 

%
\section*{{\suttatitleacronym AN 5.8}{\suttatitletranslation Failure }{\suttatitleroot Cavanasutta}}
\addcontentsline{toc}{section}{\tocacronym{AN 5.8} \toctranslation{Failure } \tocroot{Cavanasutta}}
\markboth{Failure }{Cavanasutta}
\extramarks{AN 5.8}{AN 5.8}

“Mendicants,\marginnote{1.1} a mendicant with five qualities fails, and doesn’t establish themselves in the true teaching. What five? A mendicant who is faithless … shameless … imprudent … lazy … witless fails, and doesn’t establish themselves in the true teaching. A mendicant with these five qualities fails, and doesn’t establish themselves in the true teaching. 

A\marginnote{2.1} mendicant with five qualities doesn’t fail, and establishes themselves in the true teaching. What five? A mendicant who is faithful … conscientious … prudent … energetic … wise doesn’t fail, and establishes themselves in the true teaching. A mendicant with these five qualities doesn’t fail, and establishes themselves in the true teaching.” 

%
\section*{{\suttatitleacronym AN 5.9}{\suttatitletranslation Disrespect (1st) }{\suttatitleroot Paṭhamaagāravasutta}}
\addcontentsline{toc}{section}{\tocacronym{AN 5.9} \toctranslation{Disrespect (1st) } \tocroot{Paṭhamaagāravasutta}}
\markboth{Disrespect (1st) }{Paṭhamaagāravasutta}
\extramarks{AN 5.9}{AN 5.9}

“Mendicants,\marginnote{1.1} a disrespectful and irreverent mendicant with five qualities fails, and doesn’t establish themselves in the true teaching. What five? A disrespectful and irreverent mendicant who is faithless … shameless … imprudent … lazy … witless fails, and doesn’t establish themselves in the true teaching. A disrespectful and irreverent mendicant with these five qualities fails, and doesn’t establish themselves in the true teaching. 

A\marginnote{2.1} respectful and reverent mendicant with five qualities doesn’t fail, and establishes themselves in the true teaching. What five? A respectful and reverent mendicant who is faithful … conscientious … prudent … energetic … wise doesn’t fail, and establishes themselves in the true teaching. A respectful and reverent mendicant with these five qualities doesn’t fail, and establishes themselves in the true teaching.” 

%
\section*{{\suttatitleacronym AN 5.10}{\suttatitletranslation Disrespect (2nd) }{\suttatitleroot Dutiyaagāravasutta}}
\addcontentsline{toc}{section}{\tocacronym{AN 5.10} \toctranslation{Disrespect (2nd) } \tocroot{Dutiyaagāravasutta}}
\markboth{Disrespect (2nd) }{Dutiyaagāravasutta}
\extramarks{AN 5.10}{AN 5.10}

“Mendicants,\marginnote{1.1} a disrespectful and irreverent mendicant with five qualities can’t achieve growth, improvement, or maturity in this teaching and training. What five? A disrespectful and irreverent mendicant who is faithless … shameless … imprudent … lazy … witless can’t achieve growth, improvement, or maturity in this teaching and training. A disrespectful and irreverent mendicant with these five qualities can’t achieve growth, improvement, or maturity in this teaching and training. 

A\marginnote{2.1} respectful and reverent mendicant with five qualities can achieve growth, improvement, and maturity in this teaching and training. What five? A respectful and reverent mendicant who is faithful … conscientious … prudent … energetic … wise can achieve growth, improvement, and maturity in this teaching and training. A respectful and reverent mendicant with these five qualities can achieve growth, improvement, and maturity in this teaching and training.” 

%
\addtocontents{toc}{\let\protect\contentsline\protect\nopagecontentsline}
\chapter*{The Chapter on Powers }
\addcontentsline{toc}{chapter}{\tocchapterline{The Chapter on Powers }}
\addtocontents{toc}{\let\protect\contentsline\protect\oldcontentsline}

%
\section*{{\suttatitleacronym AN 5.11}{\suttatitletranslation Not Learned From Anyone Else }{\suttatitleroot Ananussutasutta}}
\addcontentsline{toc}{section}{\tocacronym{AN 5.11} \toctranslation{Not Learned From Anyone Else } \tocroot{Ananussutasutta}}
\markboth{Not Learned From Anyone Else }{Ananussutasutta}
\extramarks{AN 5.11}{AN 5.11}

“I\marginnote{1.1} claim to have attained perfection and consummation of insight regarding principles not learned before from another. The Realized One has five powers of a Realized One. With these he claims the bull’s place, roars his lion’s roar in the assemblies, and turns the holy wheel. What five? The powers of faith, conscience, prudence, energy, and wisdom. These are the five powers of a Realized One. With these he claims the bull’s place, roars his lion’s roar in the assemblies, and turns the holy wheel.” 

%
\section*{{\suttatitleacronym AN 5.12}{\suttatitletranslation Peak }{\suttatitleroot Kūṭasutta}}
\addcontentsline{toc}{section}{\tocacronym{AN 5.12} \toctranslation{Peak } \tocroot{Kūṭasutta}}
\markboth{Peak }{Kūṭasutta}
\extramarks{AN 5.12}{AN 5.12}

“Mendicants,\marginnote{1.1} there are these five powers of a trainee. What five? The powers of faith, conscience, prudence, energy, and wisdom. These are the five powers of a trainee. Of these five powers of a trainee, the power of wisdom is the chief. It holds and binds everything together. 

It’s\marginnote{2.1} like a bungalow. The roof-peak is the chief point, which holds and binds everything together. In the same way, of these five powers of a trainee, the power of wisdom is the chief. It holds and binds everything together. 

So\marginnote{3.1} you should train like this: ‘We will have the trainee’s powers of faith, conscience, prudence, energy, and wisdom.’ That’s how you should train.” 

%
\section*{{\suttatitleacronym AN 5.13}{\suttatitletranslation In Brief }{\suttatitleroot Saṁkhittasutta}}
\addcontentsline{toc}{section}{\tocacronym{AN 5.13} \toctranslation{In Brief } \tocroot{Saṁkhittasutta}}
\markboth{In Brief }{Saṁkhittasutta}
\extramarks{AN 5.13}{AN 5.13}

“Mendicants,\marginnote{1.1} there are these five powers. What five? The powers of faith, energy, mindfulness, immersion, and wisdom. These are the five powers.” 

%
\section*{{\suttatitleacronym AN 5.14}{\suttatitletranslation In Detail }{\suttatitleroot Vitthatasutta}}
\addcontentsline{toc}{section}{\tocacronym{AN 5.14} \toctranslation{In Detail } \tocroot{Vitthatasutta}}
\markboth{In Detail }{Vitthatasutta}
\extramarks{AN 5.14}{AN 5.14}

“Mendicants,\marginnote{1.1} there are these five powers. What five? The powers of faith, energy, mindfulness, immersion, and wisdom. 

And\marginnote{2.1} what is the power of faith? It’s when a noble disciple has faith in the Realized One’s awakening: ‘That Blessed One is perfected, a fully awakened Buddha, accomplished in knowledge and conduct, holy, knower of the world, supreme guide for those who wish to train, teacher of gods and humans, awakened, blessed.’ This is called the power of faith. 

And\marginnote{3.1} what is the power of energy? It’s when a mendicant lives with energy roused up for giving up unskillful qualities and embracing skillful qualities. They’re strong, staunchly vigorous, not slacking off when it comes to developing skillful qualities. This is called the power of energy. 

And\marginnote{4.1} what is the power of mindfulness? It’s when a noble disciple is mindful. They have utmost mindfulness and alertness, and can remember and recall what was said and done long ago. This is called the power of mindfulness. 

And\marginnote{5.1} what is the power of immersion? It’s when a mendicant, quite secluded from sensual pleasures, secluded from unskillful qualities, enters and remains in the first absorption, which has the rapture and bliss born of seclusion, while placing the mind and keeping it connected. As the placing of the mind and keeping it connected are stilled, they enter and remain in the second absorption, which has the rapture and bliss born of immersion, with internal clarity and confidence, and unified mind, without placing the mind and keeping it connected. And with the fading away of rapture, they enter and remain in the third absorption, where they meditate with equanimity, mindful and aware, personally experiencing the bliss of which the noble ones declare, ‘Equanimous and mindful, one meditates in bliss.’ Giving up pleasure and pain, and ending former happiness and sadness, they enter and remain in the fourth absorption, without pleasure or pain, with pure equanimity and mindfulness. This is called the power of immersion. 

And\marginnote{6.1} what is the power of wisdom? It’s when a noble disciple is wise. They have the wisdom of arising and passing away which is noble, penetrative, and leads to the complete ending of suffering. This is called the power of wisdom. 

These\marginnote{6.4} are the five powers.” 

%
\section*{{\suttatitleacronym AN 5.15}{\suttatitletranslation Should Be Seen }{\suttatitleroot Daṭṭhabbasutta}}
\addcontentsline{toc}{section}{\tocacronym{AN 5.15} \toctranslation{Should Be Seen } \tocroot{Daṭṭhabbasutta}}
\markboth{Should Be Seen }{Daṭṭhabbasutta}
\extramarks{AN 5.15}{AN 5.15}

“Mendicants,\marginnote{1.1} there are these five powers. What five? The powers of faith, energy, mindfulness, immersion, and wisdom. 

And\marginnote{1.4} where should the power of faith be seen? In the four factors of stream-entry. 

And\marginnote{1.7} where should the power of energy be seen? In the four right efforts. 

And\marginnote{1.10} where should the power of mindfulness be seen? In the four kinds of mindfulness meditation. 

And\marginnote{1.13} where should the power of immersion be seen? In the four absorptions. 

And\marginnote{1.16} where should the power of wisdom be seen? In the four noble truths. 

These\marginnote{1.19} are the five powers.” 

%
\section*{{\suttatitleacronym AN 5.16}{\suttatitletranslation The Peak, Again }{\suttatitleroot Punakūṭasutta}}
\addcontentsline{toc}{section}{\tocacronym{AN 5.16} \toctranslation{The Peak, Again } \tocroot{Punakūṭasutta}}
\markboth{The Peak, Again }{Punakūṭasutta}
\extramarks{AN 5.16}{AN 5.16}

“Mendicants,\marginnote{1.1} there are these five powers. What five? The powers of faith, energy, mindfulness, immersion, and wisdom. These are the five powers. Of these five powers, the power of wisdom is the chief. It holds and binds everything together. It’s like a bungalow. The roof-peak is the chief point, which holds and binds everything together. In the same way, of these five powers, the power of wisdom is the chief. It holds and binds everything together.” 

%
\section*{{\suttatitleacronym AN 5.17}{\suttatitletranslation One’s Own Welfare }{\suttatitleroot Paṭhamahitasutta}}
\addcontentsline{toc}{section}{\tocacronym{AN 5.17} \toctranslation{One’s Own Welfare } \tocroot{Paṭhamahitasutta}}
\markboth{One’s Own Welfare }{Paṭhamahitasutta}
\extramarks{AN 5.17}{AN 5.17}

“Mendicants,\marginnote{1.1} a mendicant with five qualities is practicing for their own welfare, but not that of others. What five? It’s when a mendicant is personally accomplished in ethics, immersion, wisdom, freedom, and the knowledge and vision of freedom. But they don’t encourage others in these qualities. A mendicant with these five qualities is practicing for their own welfare, but not that of others.” 

%
\section*{{\suttatitleacronym AN 5.18}{\suttatitletranslation Welfare of Others (2nd) }{\suttatitleroot Dutiyahitasutta}}
\addcontentsline{toc}{section}{\tocacronym{AN 5.18} \toctranslation{Welfare of Others (2nd) } \tocroot{Dutiyahitasutta}}
\markboth{Welfare of Others (2nd) }{Dutiyahitasutta}
\extramarks{AN 5.18}{AN 5.18}

“Mendicants,\marginnote{1.1} a mendicant with five qualities is practicing for the welfare of others, but not their own. What five? It’s when a mendicant is not personally accomplished in ethics, immersion, wisdom, freedom, or the knowledge and vision of freedom. But they encourage others in these qualities. A mendicant with these five qualities is practicing for the welfare of others, but not their own.” 

%
\section*{{\suttatitleacronym AN 5.19}{\suttatitletranslation The Welfare of Neither }{\suttatitleroot Tatiyahitasutta}}
\addcontentsline{toc}{section}{\tocacronym{AN 5.19} \toctranslation{The Welfare of Neither } \tocroot{Tatiyahitasutta}}
\markboth{The Welfare of Neither }{Tatiyahitasutta}
\extramarks{AN 5.19}{AN 5.19}

“Mendicants,\marginnote{1.1} a mendicant with five qualities is practicing neither for their own welfare, nor that of others. What five? It’s when a mendicant is not personally accomplished in ethics, immersion, wisdom, freedom, or the knowledge and vision of freedom. Nor do they encourage others in these qualities. A mendicant with these five qualities is practicing neither for their own welfare, nor that of others.” 

%
\section*{{\suttatitleacronym AN 5.20}{\suttatitletranslation The Welfare of Both }{\suttatitleroot Catutthahitasutta}}
\addcontentsline{toc}{section}{\tocacronym{AN 5.20} \toctranslation{The Welfare of Both } \tocroot{Catutthahitasutta}}
\markboth{The Welfare of Both }{Catutthahitasutta}
\extramarks{AN 5.20}{AN 5.20}

“Mendicants,\marginnote{1.1} a mendicant with five qualities is practicing for both their own welfare and that of others. What five? It’s when a mendicant is personally accomplished in ethics, immersion, wisdom, freedom, and the knowledge and vision of freedom. And they encourage others in these qualities. A mendicant with these five qualities is practicing both for their own welfare and that of others.” 

%
\addtocontents{toc}{\let\protect\contentsline\protect\nopagecontentsline}
\chapter*{The Chapter on Five Factors }
\addcontentsline{toc}{chapter}{\tocchapterline{The Chapter on Five Factors }}
\addtocontents{toc}{\let\protect\contentsline\protect\oldcontentsline}

%
\section*{{\suttatitleacronym AN 5.21}{\suttatitletranslation Disrespect (1st) }{\suttatitleroot Paṭhamaagāravasutta}}
\addcontentsline{toc}{section}{\tocacronym{AN 5.21} \toctranslation{Disrespect (1st) } \tocroot{Paṭhamaagāravasutta}}
\markboth{Disrespect (1st) }{Paṭhamaagāravasutta}
\extramarks{AN 5.21}{AN 5.21}

“Mendicants,\marginnote{1.1} it’s simply impossible for a disrespectful and irreverent mendicant with incompatible lifestyle to fulfill the practice dealing with supplementary regulations regarding their spiritual companions. Without fulfilling the practice dealing with supplementary regulations, it’s impossible to fulfill the practice of a trainee. Without fulfilling the practice of a trainee, it’s impossible to fulfill ethics. Without fulfilling ethics, it’s impossible to fulfill right view. Without fulfilling right view, it’s impossible to fulfill right immersion. 

But\marginnote{2.1} it is possible for a respectful and reverent mendicant with compatible lifestyle to fulfill the practice dealing with supplementary regulations regarding their spiritual companions. Having fulfilled the practice dealing with supplementary regulations, it’s possible to fulfill the practice of a trainee. Having fulfilled the practice of a trainee, it’s possible to fulfill ethics. Having fulfilled ethics, it’s possible to fulfill right view. Having fulfilled right view, it’s possible to fulfill right immersion.” 

%
\section*{{\suttatitleacronym AN 5.22}{\suttatitletranslation Disrespect (2nd) }{\suttatitleroot Dutiyaagāravasutta}}
\addcontentsline{toc}{section}{\tocacronym{AN 5.22} \toctranslation{Disrespect (2nd) } \tocroot{Dutiyaagāravasutta}}
\markboth{Disrespect (2nd) }{Dutiyaagāravasutta}
\extramarks{AN 5.22}{AN 5.22}

“Mendicants,\marginnote{1.1} it’s simply impossible for a disrespectful and irreverent mendicant with incompatible lifestyle to fulfill the practice dealing with supplementary regulations regarding their spiritual companions. Without fulfilling the practice dealing with supplementary regulations, it’s impossible to fulfill the practice of a trainee. Without fulfilling the practice of a trainee, it’s impossible to fulfill the entire spectrum of ethics. Without fulfilling the entire spectrum of ethics, it’s impossible to fulfill the entire spectrum of immersion. Without fulfilling the entire spectrum of immersion, it’s impossible to fulfill the entire spectrum of wisdom. 

But\marginnote{2.1} it is possible for a respectful and reverent mendicant with compatible lifestyle to fulfill the practice dealing with supplementary regulations regarding their spiritual companions. Having fulfilled the practice dealing with supplementary regulations, it’s possible to fulfill the practice of a trainee. Having fulfilled the practice of a trainee, it’s possible to fulfill the entire spectrum of ethics. Having fulfilled the entire spectrum of ethics, it’s possible to fulfill the entire spectrum of immersion. Having fulfilled the entire spectrum of immersion, it’s possible to fulfill the entire spectrum of wisdom.” 

%
\section*{{\suttatitleacronym AN 5.23}{\suttatitletranslation Corruptions }{\suttatitleroot Upakkilesasutta}}
\addcontentsline{toc}{section}{\tocacronym{AN 5.23} \toctranslation{Corruptions } \tocroot{Upakkilesasutta}}
\markboth{Corruptions }{Upakkilesasutta}
\extramarks{AN 5.23}{AN 5.23}

“Mendicants,\marginnote{1.1} there are these five corruptions of gold. When gold is corrupted by these it’s not pliable, workable, or radiant, but is brittle and not completely ready for working. What five? Iron, copper, tin, lead, and silver. When gold is corrupted by these five corruptions it’s not pliable, workable, or radiant, but is brittle and not completely ready for working. 

But\marginnote{1.5} when gold is free of these five corruptions it becomes pliable, workable, and radiant, not brittle, and ready to be worked. Then the goldsmith can successfully create any kind of ornament they want, whether a ring, earrings, a necklace, or a golden garland. 

In\marginnote{2.1} the same way, there are these five corruptions of the mind. When the mind is corrupted by these it’s not pliable, workable, or radiant. It’s brittle, and not completely immersed in \textsanskrit{samādhi} for the ending of defilements. What five? Sensual desire, ill will, dullness and drowsiness, restlessness and remorse, and doubt. These are the five corruptions of the mind. When the mind is corrupted by these it’s not pliable, workable, or radiant. It’s brittle, and not completely immersed in \textsanskrit{samādhi} for the ending of defilements. 

But\marginnote{2.5} when the mind is free of these five corruptions it’s pliable, workable, and radiant. It’s not brittle, and is completely immersed in \textsanskrit{samādhi} for the ending of defilements. You become capable of realizing anything that can be realized by insight to which you extend the mind, in each and every case. 

If\marginnote{3.1} you wish: ‘May I wield the many kinds of psychic power—multiplying myself and becoming one again; appearing and disappearing; going unimpeded through a wall, a rampart, or a mountain as if through space; diving in and out of the earth as if it were water; walking on water as if it were earth; flying cross-legged through the sky like a bird; touching and stroking with the hand the sun and moon, so mighty and powerful, controlling the body as far as the \textsanskrit{Brahmā} realm.’ You’re capable of realizing it, in each and every case. 

If\marginnote{4.1} you wish: ‘With clairaudience that is purified and superhuman, may I hear both kinds of sounds, human and divine, whether near or far.’ You’re capable of realizing it, in each and every case. 

If\marginnote{5.1} you wish: ‘May I understand the minds of other beings and individuals, having comprehended them with my mind. May I understand mind with greed as “mind with greed”, and mind without greed as “mind without greed”; mind with hate as “mind with hate”, and mind without hate as “mind without hate”; mind with delusion as “mind with delusion”, and mind without delusion as “mind without delusion”; constricted mind as “constricted mind”, and scattered mind as “scattered mind”; expansive mind as “expansive mind”, and unexpansive mind as “unexpansive mind”; mind that is not supreme as “mind that is not supreme”, and mind that is supreme as “mind that is supreme”; mind immersed in \textsanskrit{samādhi} as “mind immersed in \textsanskrit{samādhi}”, and mind not immersed in \textsanskrit{samādhi} as “mind not immersed in \textsanskrit{samādhi}”; freed mind as “freed mind”, and unfreed mind as “unfreed mind”.’ You’re capable of realizing it, in each and every case. 

If\marginnote{6.1} you wish: ‘May I recollect many kinds of past lives. That is: one, two, three, four, five, ten, twenty, thirty, forty, fifty, a hundred, a thousand, a hundred thousand rebirths; many eons of the world contracting, many eons of the world expanding, many eons of the world contracting and expanding. May I remember: “There, I was named this, my clan was that, I looked like this, and that was my food. This was how I felt pleasure and pain, and that was how my life ended. When I passed away from that place I was reborn somewhere else. There, too, I was named this, my clan was that, I looked like this, and that was my food. This was how I felt pleasure and pain, and that was how my life ended. When I passed away from that place I was reborn here.” May I recollect my many past lives, with features and details.’ You’re capable of realizing it, in each and every case. 

If\marginnote{7.1} you wish: ‘With clairvoyance that is purified and superhuman, may I see sentient beings passing away and being reborn—inferior and superior, beautiful and ugly, in a good place or a bad place—and understand how sentient beings are reborn according to their deeds: “These dear beings did bad things by way of body, speech, and mind. They spoke ill of the noble ones; they had wrong view; and they acted out of that wrong view. When their body breaks up, after death, they’re reborn in a place of loss, a bad place, the underworld, hell. These dear beings, however, did good things by way of body, speech, and mind. They never spoke ill of the noble ones; they had right view; and they acted out of that right view. When their body breaks up, after death, they’re reborn in a good place, a heavenly realm.” And so, with clairvoyance that is purified and superhuman, may I see sentient beings passing away and being reborn—inferior and superior, beautiful and ugly, in a good place or a bad place. And may I understand how sentient beings are reborn according to their deeds.’ You’re capable of realizing it, in each and every case. 

If\marginnote{8.1} you wish: ‘May I realize the undefiled freedom of heart and freedom by wisdom in this very life, and live having realized it with my own insight due to the ending of defilements.’ You’re capable of realizing it, in each and every case.” 

%
\section*{{\suttatitleacronym AN 5.24}{\suttatitletranslation Unethical }{\suttatitleroot Dussīlasutta}}
\addcontentsline{toc}{section}{\tocacronym{AN 5.24} \toctranslation{Unethical } \tocroot{Dussīlasutta}}
\markboth{Unethical }{Dussīlasutta}
\extramarks{AN 5.24}{AN 5.24}

“Mendicants,\marginnote{1.1} an unethical person, who lacks ethics, has destroyed a vital condition for right immersion. When there is no right immersion, one who lacks right immersion has destroyed a vital condition for true knowledge and vision. When there is no true knowledge and vision, one who lacks true knowledge and vision has destroyed a vital condition for disillusionment and dispassion. When there is no disillusionment and dispassion, one who lacks disillusionment and dispassion has destroyed a vital condition for knowledge and vision of freedom. 

Suppose\marginnote{1.5} there was a tree that lacked branches and foliage. Its shoots, bark, softwood, and heartwood would not grow to fullness. 

In\marginnote{1.7} the same way, an unethical person, who lacks ethics, has destroyed a vital condition for right immersion. When there is no right immersion, one who lacks right immersion has destroyed a vital condition for true knowledge and vision. When there is no true knowledge and vision, one who lacks true knowledge and vision has destroyed a vital condition for disillusionment and dispassion. When there is no disillusionment and dispassion, one who lacks disillusionment and dispassion has destroyed a vital condition for knowledge and vision of freedom. 

An\marginnote{2.1} ethical person, who has fulfilled ethics, has fulfilled a vital condition for right immersion. When there is right immersion, one who has fulfilled right immersion has fulfilled a vital condition for true knowledge and vision. When there is true knowledge and vision, one who has fulfilled true knowledge and vision has fulfilled a vital condition for disillusionment and dispassion. When there is disillusionment and dispassion, one who has fulfilled disillusionment and dispassion has fulfilled a vital condition for knowledge and vision of freedom. 

Suppose\marginnote{2.5} there was a tree that was complete with branches and foliage. Its shoots, bark, softwood, and heartwood would all grow to fullness. 

In\marginnote{2.6} the same way, an ethical person, who has fulfilled ethics, has fulfilled a vital condition for right immersion. When there is right immersion, one who has fulfilled right immersion has fulfilled a vital condition for true knowledge and vision. When there is true knowledge and vision, one who has fulfilled true knowledge and vision has fulfilled a vital condition for disillusionment and dispassion. When there is disillusionment and dispassion, one who has fulfilled disillusionment and dispassion has fulfilled a vital condition for knowledge and vision of freedom.” 

%
\section*{{\suttatitleacronym AN 5.25}{\suttatitletranslation Supported }{\suttatitleroot Anuggahitasutta}}
\addcontentsline{toc}{section}{\tocacronym{AN 5.25} \toctranslation{Supported } \tocroot{Anuggahitasutta}}
\markboth{Supported }{Anuggahitasutta}
\extramarks{AN 5.25}{AN 5.25}

“Mendicants,\marginnote{1.1} when right view is supported by five factors it has freedom of heart and freedom by wisdom as its fruit and benefit. 

What\marginnote{2.1} five? It’s when right view is supported by ethics, learning, discussion, serenity, and discernment. When right view is supported by these five factors it has freedom of heart and freedom by wisdom as its fruit and benefit.” 

%
\section*{{\suttatitleacronym AN 5.26}{\suttatitletranslation Opportunities for Freedom }{\suttatitleroot Vimuttāyatanasutta}}
\addcontentsline{toc}{section}{\tocacronym{AN 5.26} \toctranslation{Opportunities for Freedom } \tocroot{Vimuttāyatanasutta}}
\markboth{Opportunities for Freedom }{Vimuttāyatanasutta}
\extramarks{AN 5.26}{AN 5.26}

“Mendicants,\marginnote{1.1} there are these five opportunities for freedom. If a mendicant stays diligent, keen, and resolute at these times, their mind is freed, their defilements are ended, and they arrive at the supreme sanctuary. What five? 

Firstly,\marginnote{2.2} the Teacher or a respected spiritual companion teaches Dhamma to a mendicant. That mendicant feels inspired by the meaning and the teaching in that Dhamma, no matter how the Teacher or a respected spiritual companion teaches it. Feeling inspired, joy springs up. Being joyful, rapture springs up. When the mind is full of rapture, the body becomes tranquil. When the body is tranquil, one feels bliss. And when blissful, the mind becomes immersed in \textsanskrit{samādhi}. This is the first opportunity for freedom. If a mendicant stays diligent, keen, and resolute at this time, their mind is freed, their defilements are ended, and they arrive at the supreme sanctuary. 

Furthermore,\marginnote{3.1} it may be that neither the Teacher nor a respected spiritual companion teaches Dhamma to a mendicant. But the mendicant teaches Dhamma in detail to others as they learned and memorized it. That mendicant feels inspired by the meaning and the teaching in that Dhamma, no matter how they teach it in detail to others as they learned and memorized it. Feeling inspired, joy springs up. Being joyful, rapture springs up. When the mind is full of rapture, the body becomes tranquil. When the body is tranquil, one feels bliss. And when blissful, the mind becomes immersed in \textsanskrit{samādhi}. This is the second opportunity for freedom. … 

Furthermore,\marginnote{4.1} it may be that neither the Teacher nor … the mendicant teaches Dhamma. But the mendicant recites the teaching in detail as they learned and memorized it. That mendicant feels inspired by the meaning and the teaching in that Dhamma, no matter how they recite it in detail as they learned and memorized it. Feeling inspired, joy springs up. Being joyful, rapture springs up. When the mind is full of rapture, the body becomes tranquil. When the body is tranquil, one feels bliss. And when blissful, the mind becomes immersed in \textsanskrit{samādhi}. This is the third opportunity for freedom. … 

Furthermore,\marginnote{5.1} it may be that neither the Teacher nor … the mendicant teaches Dhamma … nor does the mendicant recite the teaching. But the mendicant thinks about and considers the teaching in their heart, examining it with the mind as they learned and memorized it. That mendicant feels inspired by the meaning and the teaching in that Dhamma, no matter how they think about and consider it in their heart, examining it with the mind as they learned and memorized it. Feeling inspired, joy springs up. Being joyful, rapture springs up. When the mind is full of rapture, the body becomes tranquil. When the body is tranquil, one feels bliss. And when blissful, the mind becomes immersed in \textsanskrit{samādhi}. This is the fourth opportunity for freedom. … 

Furthermore,\marginnote{6.1} it may be that neither the Teacher nor … the mendicant teaches Dhamma … nor does the mendicant recite the teaching … or think about it. But a meditation subject as a foundation of immersion is properly grasped, attended, borne in mind, and comprehended with wisdom. That mendicant feels inspired by the meaning and the teaching in that Dhamma, no matter how a meditation subject as a foundation of immersion is properly grasped, attended, borne in mind, and comprehended with wisdom. Feeling inspired, joy springs up. Being joyful, rapture springs up. When the mind is full of rapture, the body becomes tranquil. When the body is tranquil, one feels bliss. And when blissful, the mind becomes immersed in \textsanskrit{samādhi}. This is the fifth opportunity for freedom. … 

These\marginnote{7.1} are the five opportunities for freedom. If a mendicant stays diligent, keen, and resolute at these times, their mind is freed, their defilements are ended, and they arrive at the supreme sanctuary.” 

%
\section*{{\suttatitleacronym AN 5.27}{\suttatitletranslation Immersion }{\suttatitleroot Samādhisutta}}
\addcontentsline{toc}{section}{\tocacronym{AN 5.27} \toctranslation{Immersion } \tocroot{Samādhisutta}}
\markboth{Immersion }{Samādhisutta}
\extramarks{AN 5.27}{AN 5.27}

“Mendicants,\marginnote{1.1} develop limitless immersion, alert and mindful. When you develop limitless immersion, alert and mindful, five knowledges arise for you personally. What five? 

‘This\marginnote{1.4} immersion is blissful now, and results in bliss in the future.’ … 

‘This\marginnote{1.5} immersion is noble and spiritual.’ … 

‘This\marginnote{1.6} immersion is not cultivated by sinners.’ … 

‘This\marginnote{1.7} immersion is peaceful and sublime and tranquil and unified, not held in place by forceful suppression.’ … 

‘I\marginnote{1.8} mindfully enter into and emerge from this immersion.’ … 

Develop\marginnote{2.1} limitless immersion, alert and mindful. When you develop limitless immersion, alert and mindful, these five knowledges arise for you personally.” 

%
\section*{{\suttatitleacronym AN 5.28}{\suttatitletranslation With Five Factors }{\suttatitleroot Pañcaṅgikasutta}}
\addcontentsline{toc}{section}{\tocacronym{AN 5.28} \toctranslation{With Five Factors } \tocroot{Pañcaṅgikasutta}}
\markboth{With Five Factors }{Pañcaṅgikasutta}
\extramarks{AN 5.28}{AN 5.28}

“Mendicants,\marginnote{1.1} I will teach you how to develop noble right immersion with five factors. Listen and pay close attention, I will speak.” 

“Yes,\marginnote{1.3} sir,” they replied. The Buddha said this: 

“And\marginnote{2.1} how do you develop noble right immersion with five factors? 

Firstly,\marginnote{2.2} a mendicant, quite secluded from sensual pleasures, secluded from unskillful qualities, enters and remains in the first absorption. It has the rapture and bliss born of seclusion, while placing the mind and keeping it connected. They drench, steep, fill, and spread their body with rapture and bliss born of seclusion. There’s no part of the body that’s not spread with rapture and bliss born of seclusion. It’s like when a deft bathroom attendant or their apprentice pours bath powder into a bronze dish, sprinkling it little by little with water. They knead it until the ball of bath powder is soaked and saturated with moisture, spread through inside and out; yet no moisture oozes out. In the same way, a mendicant drenches, steeps, fills, and spreads their body with rapture and bliss born of seclusion. There’s no part of the body that’s not spread with rapture and bliss born of seclusion. This is the first way to develop noble right immersion with five factors. 

Furthermore,\marginnote{3.1} as the placing of the mind and keeping it connected are stilled, a mendicant enters and remains in the second absorption. It has the rapture and bliss born of immersion, with internal clarity and confidence, and unified mind, without placing the mind and keeping it connected. They drench, steep, fill, and spread their body with rapture and bliss born of immersion. There’s no part of the body that’s not spread with rapture and bliss born of immersion. It’s like a deep lake fed by spring water. There’s no inlet to the east, west, north, or south, and no rainfall to replenish it from time to time. But the stream of cool water welling up in the lake drenches, steeps, fills, and spreads throughout the lake. There’s no part of the lake that’s not spread through with cool water. In the same way, a mendicant drenches, steeps, fills, and spreads their body with rapture and bliss born of immersion. There’s no part of the body that’s not spread with rapture and bliss born of immersion. This is the second way to develop noble right immersion with five factors. 

Furthermore,\marginnote{4.1} with the fading away of rapture, a mendicant enters and remains in the third absorption. They meditate with equanimity, mindful and aware, personally experiencing the bliss of which the noble ones declare, ‘Equanimous and mindful, one meditates in bliss.’ They drench, steep, fill, and spread their body with bliss free of rapture. There’s no part of the body that’s not spread with bliss free of rapture. It’s like a pool with blue water lilies, or pink or white lotuses. Some of them sprout and grow in the water without rising above it, thriving underwater. From the tip to the root they’re drenched, steeped, filled, and soaked with cool water. There’s no part of them that’s not spread through with cool water. In the same way, a mendicant drenches, steeps, fills, and spreads their body with bliss free of rapture. There’s no part of the body that’s not spread with bliss free of rapture. This is the third way to develop noble right immersion with five factors. 

Furthermore,\marginnote{5.1} giving up pleasure and pain, and ending former happiness and sadness, a mendicant enters and remains in the fourth absorption. It is without pleasure or pain, with pure equanimity and mindfulness. They sit spreading their body through with pure bright mind. There’s no part of the body that’s not spread with pure bright mind. It’s like someone sitting wrapped from head to foot with white cloth. There’s no part of the body that’s not spread over with white cloth. In the same way, they sit spreading their body through with pure bright mind. There’s no part of the body that’s not spread with pure bright mind. This is the fourth way to develop noble right immersion with five factors. 

Furthermore,\marginnote{6.1} the meditation that is a foundation for reviewing is properly grasped, attended, borne in mind, and comprehended with wisdom by a mendicant. It’s like when someone views someone else. Someone standing might view someone sitting, or someone sitting might view someone lying down. In the same way, the meditation that is a foundation for reviewing is properly grasped, attended, borne in mind, and comprehended with wisdom by a mendicant. This is the fifth way to develop noble right immersion with five factors. 

When\marginnote{7.1} the noble right immersion with five factors is cultivated in this way, a mendicant becomes capable of realizing anything that can be realized by insight to which they extend the mind, in each and every case. 

Suppose\marginnote{8.1} a water jar was placed on a stand, full to the brim so a crow could drink from it. If a strong man was to tip it any which way, would water pour out?” 

“Yes,\marginnote{8.3} sir.” 

“In\marginnote{8.4} the same way, when noble right immersion with five factors is cultivated in this way, a mendicant becomes capable of realizing anything that can be realized by insight to which they extend the mind, in each and every case. 

Suppose\marginnote{9.1} there was a square, walled lotus pond on level ground, full to the brim so a crow could drink from it. If a strong man was to open the wall on any side, would water pour out?” 

“Yes,\marginnote{9.3} sir.” 

“In\marginnote{9.4} the same way, when noble right immersion with five factors is cultivated in this way, a mendicant becomes capable of realizing anything that can be realized by insight to which they extend the mind, in each and every case. 

Suppose\marginnote{10.1} a chariot stood harnessed to thoroughbreds at a level crossroads, with a goad ready. Then a deft horse trainer, a master charioteer, might mount the chariot, taking the reins in his right hand and goad in the left. He’d drive out and back wherever he wishes, whenever he wishes. In the same way, when noble right immersion with five factors is cultivated in this way, a mendicant becomes capable of realizing anything that can be realized by insight to which they extend the mind, in each and every case. 

If\marginnote{11.1} you wish: ‘May I wield the many kinds of psychic power: multiplying myself and becoming one again … controlling the body as far as the \textsanskrit{Brahmā} realm.’ You’re capable of realizing it, in each and every case. 

If\marginnote{12.1} you wish: ‘With clairaudience that is purified and superhuman, may I hear both kinds of sounds, human and divine, whether near or far.’ You’re capable of realizing it, in each and every case. 

If\marginnote{13.1} you wish: ‘May I understand the minds of other beings and individuals, having comprehended them with my mind. May I understand mind with greed as “mind with greed”, and mind without greed as “mind without greed”; mind with hate as “mind with hate”, and mind without hate as “mind without hate”; mind with delusion as “mind with delusion”, and mind without delusion as “mind without delusion”; constricted mind as “constricted mind”, and scattered mind as “scattered mind”; expansive mind as “expansive mind”, and unexpansive mind as “unexpansive mind”; mind that is not supreme as “mind that is not supreme”, and mind that is supreme as “mind that is supreme”; mind immersed in \textsanskrit{samādhi} as “mind immersed in \textsanskrit{samādhi}”, and mind not immersed in \textsanskrit{samādhi} as “mind not immersed in \textsanskrit{samādhi}”; freed mind as “freed mind”, and unfreed mind as “unfreed mind”.’ You’re capable of realizing it, in each and every case. 

If\marginnote{14.1} you wish: ‘May I recollect many kinds of past lives, with features and details.’ You’re capable of realizing it, in each and every case. 

If\marginnote{15.1} you wish: ‘With clairvoyance that is purified and superhuman, may I see sentient beings passing away and being reborn according to their deeds.’ You’re capable of realizing it, in each and every case. 

If\marginnote{16.1} you wish: ‘May I realize the undefiled freedom of heart and freedom by wisdom in this very life, and live having realized it with my own insight due to the ending of defilements.’ You’re capable of realizing it, in each and every case.” 

%
\section*{{\suttatitleacronym AN 5.29}{\suttatitletranslation Walking Meditation }{\suttatitleroot Caṅkamasutta}}
\addcontentsline{toc}{section}{\tocacronym{AN 5.29} \toctranslation{Walking Meditation } \tocroot{Caṅkamasutta}}
\markboth{Walking Meditation }{Caṅkamasutta}
\extramarks{AN 5.29}{AN 5.29}

“Mendicants,\marginnote{1.1} there are five benefits of walking meditation. What five? You get fit for traveling, fit for striving in meditation, and healthy. What’s eaten, drunk, chewed, and tasted is properly digested. And immersion gained while walking lasts long. These are the five benefits of walking meditation.” 

%
\section*{{\suttatitleacronym AN 5.30}{\suttatitletranslation With Nāgita }{\suttatitleroot Nāgitasutta}}
\addcontentsline{toc}{section}{\tocacronym{AN 5.30} \toctranslation{With Nāgita } \tocroot{Nāgitasutta}}
\markboth{With Nāgita }{Nāgitasutta}
\extramarks{AN 5.30}{AN 5.30}

\scevam{So\marginnote{1.1} I have heard. }At one time the Buddha was wandering in the land of the Kosalans together with a large \textsanskrit{Saṅgha} of mendicants when he arrived at a village of the Kosalan brahmins named \textsanskrit{Icchānaṅgala}. He stayed in a forest near \textsanskrit{Icchānaṅgala}. The brahmins and householders of \textsanskrit{Icchānaṅgala} heard: 

“It\marginnote{1.5} seems the ascetic Gotama—a Sakyan, gone forth from a Sakyan family—has arrived at \textsanskrit{Icchānaṅgala}. He is staying in a forest near \textsanskrit{Icchānaṅgala}. He has this good reputation: ‘That Blessed One is perfected, a fully awakened Buddha, accomplished in knowledge and conduct, holy, knower of the world, supreme guide for those who wish to train, teacher of gods and humans, awakened, blessed.’ He has realized with his own insight this world—with its gods, \textsanskrit{Māras} and \textsanskrit{Brahmās}, this population with its ascetics and brahmins, gods and humans—and he makes it known to others. He teaches Dhamma that’s good in the beginning, good in the middle, and good in the end, meaningful and well-phrased. And he reveals a spiritual practice that’s entirely full and pure. It’s good to see such perfected ones.” 

Then,\marginnote{1.11} when the night had passed, they took many different foods and went to the forest near \textsanskrit{Icchānaṅgala}, where they stood outside the gates making a dreadful racket. 

Now,\marginnote{2.1} at that time Venerable \textsanskrit{Nāgita} was the Buddha’s attendant. Then the Buddha said to \textsanskrit{Nāgita}, “\textsanskrit{Nāgita}, who’s making that dreadful racket? You’d think it was fishermen hauling in a catch!” 

“Sir,\marginnote{2.4} it’s these brahmins and householders of \textsanskrit{Icchānaṅgala}. They’ve brought many different foods, and they’re standing outside the gates wanting to offer it specially to the Buddha and the mendicant \textsanskrit{Saṅgha}.” 

“\textsanskrit{Nāgita},\marginnote{2.5} may I never become famous. May fame not come to me. There are those who can’t get the bliss of renunciation, the bliss of seclusion, the bliss of peace, the bliss of awakening when they want, without trouble or difficulty like I can. Let them enjoy the filthy, lazy pleasure of possessions, honor, and popularity.” 

“Sir,\marginnote{3.1} may the Blessed One please relent now! May the Holy One relent! Now is the time for the Buddha to relent. Wherever the Buddha now goes, the brahmins and householders will incline the same way, as will the people of town and country. It’s like when it rains heavily and the water flows downhill. In the same way, wherever the Buddha now goes, the brahmins and householders will incline the same way, as will the people of town and country. Why is that? Because of the Buddha’s ethics and wisdom.” 

“\textsanskrit{Nāgita},\marginnote{4.1} may I never become famous. May fame not come to me. There are those who can’t get the bliss of renunciation, the bliss of seclusion, the bliss of peace, the bliss of awakening when they want, without trouble or difficulty like I can. Let them enjoy the filthy, lazy pleasure of possessions, honor, and popularity. 

What\marginnote{4.4} you eat, drink, chew, and taste ends up as excrement and urine. This is its outcome. 

When\marginnote{4.6} loved ones decay and perish, sorrow, lamentation, pain, sadness, and distress arise. This is its outcome. 

When\marginnote{4.8} you pursue meditation on the feature of ugliness, revulsion at the feature of beauty becomes stabilized. This is its outcome. 

When\marginnote{4.10} you meditate observing impermanence in the six fields of contact, revulsion at contact becomes stabilized. This is its outcome. 

When\marginnote{4.12} you meditate observing rise and fall in the five grasping aggregates, revulsion at grasping becomes stabilized. This is its outcome.” 

%
\addtocontents{toc}{\let\protect\contentsline\protect\nopagecontentsline}
\chapter*{The Chapter with Sumanā }
\addcontentsline{toc}{chapter}{\tocchapterline{The Chapter with Sumanā }}
\addtocontents{toc}{\let\protect\contentsline\protect\oldcontentsline}

%
\section*{{\suttatitleacronym AN 5.31}{\suttatitletranslation With Sumanā }{\suttatitleroot Sumanasutta}}
\addcontentsline{toc}{section}{\tocacronym{AN 5.31} \toctranslation{With Sumanā } \tocroot{Sumanasutta}}
\markboth{With Sumanā }{Sumanasutta}
\extramarks{AN 5.31}{AN 5.31}

At\marginnote{1.1} one time the Buddha was staying near \textsanskrit{Sāvatthī} in Jeta’s Grove, \textsanskrit{Anāthapiṇḍika}’s monastery. Then Princess \textsanskrit{Sumanā}, escorted by five hundred chariots and five hundred royal maidens, went up to the Buddha, bowed, sat down to one side, and said to him: 

“Sir,\marginnote{2.1} suppose there were two disciples equal in faith, ethics, and wisdom. One is a giver, one is not. When their body breaks up, after death, they’re reborn in a good place, a heavenly realm. When they have become gods, would there be any distinction or difference between them?” 

“There\marginnote{3.1} would be, \textsanskrit{Sumanā},” said the Buddha. 

“As\marginnote{3.2} a god, the one who was a giver would surpass the other in five respects: divine lifespan, beauty, happiness, fame, and sovereignty. As a god, the one who was a giver would surpass the other in these five respects.” 

“But\marginnote{4.1} sir, if they pass away from there and come back to this state of existence as human beings, would there still be any distinction or difference between them?” 

“There\marginnote{4.2} would be, \textsanskrit{Sumanā},” said the Buddha. 

“As\marginnote{4.3} a human being, the one who was a giver would surpass the other in five respects: human lifespan, beauty, happiness, fame, and sovereignty. As a human being, the one who was a giver would surpass the other in these five respects.” 

“But\marginnote{5.1} sir, if they both go forth from the lay life to homelessness, would there still be any distinction or difference between them?” 

“There\marginnote{5.2} would be, \textsanskrit{Sumanā},” said the Buddha. 

“As\marginnote{5.3} a renunciate, the one who was a giver would surpass the other in five respects. They’d usually use only what they’ve been invited to accept—robes, almsfood, lodgings, and medicines and supplies for the sick—rarely using them without invitation. When living with other spiritual practitioners, they usually treat them agreeably by way of body, speech, and mind, rarely disagreeably. As a renunciate, the one who was a giver would surpass the other in these five respects.” 

“But\marginnote{6.1} sir, if they both attain perfection, as perfected ones would there still be any distinction or difference between them?” 

“In\marginnote{6.2} that case, I say there is no difference between the freedom of one and the freedom of the other.” 

“It’s\marginnote{7.1} incredible, sir, it’s amazing! Just this much is quite enough to justify giving gifts and making merit. For merit is helpful for those who have become gods, human beings, and renunciates.” 

“That’s\marginnote{7.4} so true, \textsanskrit{Sumanā}. It’s quite enough to justify giving gifts and making merit. For merit is helpful for those who have become gods, human beings, and renunciates.” 

That\marginnote{8.1} is what the Buddha said. Then the Holy One, the Teacher, went on to say: 

\begin{verse}%
“The\marginnote{9.1} moon so immaculate, \\
journeying across the dimension of space; \\
outshines with its radiance \\
all the world’s stars. 

So\marginnote{10.1} too, a faithful individual, \\
perfect in ethics, \\
outshines with their generosity \\
all the world’s stingy people. 

The\marginnote{11.1} thundering rain cloud, \\
its hundred peaks wreathed in lightning, \\
pours down over the rich earth, \\
soaking the uplands and valleys. 

Even\marginnote{12.1} so, an astute person accomplished in vision, \\
a disciple of the fully awakened Buddha, \\
surpasses a stingy person \\
in five respects: 

long\marginnote{13.1} life and fame, \\
beauty and happiness. \\
Lavished with riches, \\
they depart to rejoice in heaven.” 

%
\end{verse}

%
\section*{{\suttatitleacronym AN 5.32}{\suttatitletranslation With Cundī }{\suttatitleroot Cundīsutta}}
\addcontentsline{toc}{section}{\tocacronym{AN 5.32} \toctranslation{With Cundī } \tocroot{Cundīsutta}}
\markboth{With Cundī }{Cundīsutta}
\extramarks{AN 5.32}{AN 5.32}

At\marginnote{1.1} one time the Buddha was staying near \textsanskrit{Rājagaha}, in the Bamboo Grove, the squirrels’ feeding ground. Then Princess \textsanskrit{Cundī}, escorted by five hundred chariots and five hundred royal maidens, went up to the Buddha, bowed, sat down to one side, and said to him: 

“Sir,\marginnote{2.1} my brother, Prince Cunda, says this: ‘Take a woman or man who goes for refuge to the Buddha, the teaching, and the \textsanskrit{Saṅgha}, and doesn’t kill living creatures, steal, commit sexual misconduct, lie, or take alcoholic drinks that cause negligence. Only then do they get reborn in a good place, not a bad place, when their body breaks up, after death.’ And so I ask the Buddha: Sir, what kind of teacher should you have confidence in so as to be reborn in a good place, not a bad place, when the body breaks up, after death? Sir, what kind of teaching should you have confidence in so as to be reborn in a good place, not a bad place, when the body breaks up, after death? Sir, what kind of \textsanskrit{Saṅgha} should you have confidence in so as to be reborn in a good place, not a bad place, when the body breaks up, after death? Sir, what kind of ethics should you fulfill so as to be reborn in a good place, not a bad place, when the body breaks up, after death?” 

“\textsanskrit{Cundī},\marginnote{3.1} the Realized One, the perfected one, the fully awakened Buddha, is said to be the best of all sentient beings—be they footless, with two feet, four feet, or many feet; with form or formless; with perception or without perception or with neither perception nor non-perception. Those who have confidence in the Buddha have confidence in the best. Having confidence in the best, the result is the best. 

The\marginnote{4.1} noble eightfold path is said to be the best of all conditioned things. Those who have confidence in the noble eightfold path have confidence in the best. Having confidence in the best, the result is the best. 

Fading\marginnote{5.1} away is said to be the best of all things whether conditioned or unconditioned. That is, the quelling of vanity, the removing of thirst, the uprooting of clinging, the breaking of the round, the ending of craving, fading away, cessation, extinguishment. Those who have confidence in the teaching of fading away have confidence in the best. Having confidence in the best, the result is the best. 

The\marginnote{6.1} \textsanskrit{Saṅgha} of the Realized One’s disciples is said to be the best of all communities and groups. It consists of the four pairs, the eight individuals. This is the \textsanskrit{Saṅgha} of the Buddha’s disciples that is worthy of offerings dedicated to the gods, worthy of hospitality, worthy of a religious donation, worthy of greeting with joined palms, and is the supreme field of merit for the world. Those who have confidence in the \textsanskrit{Saṅgha} have confidence in the best. Having confidence in the best, the result is the best. 

The\marginnote{7.1} ethical conduct loved by the noble ones is said to be the best of all ethics. It is unbroken, impeccable, spotless, and unmarred, liberating, praised by sensible people, not mistaken, and leading to immersion. Those who fulfill the ethics loved by the noble ones fulfill the best. Fulfilling the best, the result is the best. 

\begin{verse}%
For\marginnote{8.1} those who, knowing the best teaching, \\
base their confidence on the best—\\
confident in the best Awakened One, \\
supremely worthy of a religious donation; 

confident\marginnote{9.1} in the best teaching, \\
the bliss of fading and stilling; \\
confident in the best \textsanskrit{Saṅgha}, \\
the supreme field of merit—

giving\marginnote{10.1} gifts to the best, \\
the best of merit grows: \\
the best lifespan, beauty, \\
fame, reputation, happiness, and strength. 

An\marginnote{11.1} intelligent person gives to the best, \\
settled on the best teaching. \\
When they become a god or human, \\
they rejoice at reaching the best.” 

%
\end{verse}

%
\section*{{\suttatitleacronym AN 5.33}{\suttatitletranslation With Uggaha }{\suttatitleroot Uggahasutta}}
\addcontentsline{toc}{section}{\tocacronym{AN 5.33} \toctranslation{With Uggaha } \tocroot{Uggahasutta}}
\markboth{With Uggaha }{Uggahasutta}
\extramarks{AN 5.33}{AN 5.33}

At\marginnote{1.1} one time the Buddha was staying near Bhaddiya, in \textsanskrit{Jātiyā} Wood. 

Then\marginnote{1.2} Uggaha, \textsanskrit{Meṇḍaka}’s grandson, went up to the Buddha, bowed, sat down to one side, and said to him, “Sir, may the Buddha please accept tomorrow’s meal from me, together with three other monks.” The Buddha consented in silence. Then, knowing that the Buddha had consented, Uggaha got up from his seat, bowed, and respectfully circled the Buddha, keeping him on his right, before leaving. 

Then\marginnote{3.1} when the night had passed, the Buddha robed up in the morning and, taking his bowl and robe, went to Uggaha’s home, where he sat on the seat spread out. Then Uggaha served and satisfied the Buddha with his own hands with a variety of delicious foods. 

When\marginnote{3.3} the Buddha had eaten and washed his hand and bowl, Uggaha sat down to one side, and said to him, “Sir, these girls of mine will be going to their husbands’ families. May the Buddha please advise and instruct them. It will be for their lasting welfare and happiness.” 

Then\marginnote{4.1} the Buddha said to those girls: 

“So,\marginnote{4.2} girls, you should train like this: ‘Our parents will give us to a husband wanting what’s best, out of kindness and compassion. We will get up before him and go to bed after him, and be obliging, behaving nicely and speaking politely.’ That’s how you should train. 

So,\marginnote{5.1} girls, you should train like this: ‘Those our husband respects—mother and father, ascetics and brahmins—we will honor, respect, revere, and venerate, and serve with a seat and a drink when they come as guests.’ That’s how you should train. 

So,\marginnote{6.1} girls, you should train like this: ‘We will be skilled and tireless in doing domestic duties for our husband, such as knitting and sewing. We will have an understanding of how to go about things in order to complete and organize the work.’ That’s how you should train. 

So,\marginnote{7.1} girls, you should train like this: ‘We will know what work our husband’s domestic bondservants, workers, and staff have completed, and what they’ve left incomplete. We will know who is sick, and who is fit or unwell. We will distribute to each a fair portion of various foods.’ That’s how you should train. 

So,\marginnote{8.1} girls, you should train like this: ‘We will ensure that any income our husbands earn is guarded and protected, whether money, grain, silver, or gold. We will not overspend, steal, waste, or lose it.’ That’s how you should train. When they have these five qualities, females—when their body breaks up, after death—are reborn in company with the Gods of the Lovable Host. 

\begin{verse}%
She’d\marginnote{9.1} never look down on her husband, \\
who’s always eager to work hard, \\
always looking after her, \\
and bringing whatever she wants. 

And\marginnote{10.1} a good woman never scolds her husband \\
with jealous words. \\
Being astute, she reveres \\
those respected by her husband. 

She\marginnote{11.1} gets up early, works tirelessly, \\
and manages the domestic help. \\
She’s loveable to her husband, \\
and preserves his wealth. 

A\marginnote{12.1} lady who fulfills these duties \\
according to her husband’s desire, \\
is reborn among the gods \\
called ‘Loveable’.” 

%
\end{verse}

%
\section*{{\suttatitleacronym AN 5.34}{\suttatitletranslation With General Sīha }{\suttatitleroot Sīhasenāpatisutta}}
\addcontentsline{toc}{section}{\tocacronym{AN 5.34} \toctranslation{With General Sīha } \tocroot{Sīhasenāpatisutta}}
\markboth{With General Sīha }{Sīhasenāpatisutta}
\extramarks{AN 5.34}{AN 5.34}

At\marginnote{1.1} one time the Buddha was staying near \textsanskrit{Vesālī}, at the Great Wood, in the hall with the peaked roof. 

Then\marginnote{1.2} General \textsanskrit{Sīha} went up to the Buddha, bowed, sat down to one side, and asked him, “Sir, can you point out a fruit of giving that’s apparent in the present life?” 

“I\marginnote{2.1} can, \textsanskrit{Sīha},” said the Buddha. 

“A\marginnote{2.2} giver, a donor is dear and beloved to many people. This is a fruit of giving that’s apparent in the present life. 

Furthermore,\marginnote{3.1} good people associate with a giver. This is another fruit of giving that’s apparent in the present life. 

Furthermore,\marginnote{4.1} a giver gains a good reputation. This is another fruit of giving that’s apparent in the present life. 

Furthermore,\marginnote{5.1} a giver enters any kind of assembly bold and assured, whether it’s an assembly of aristocrats, brahmins, householders, or ascetics. This is another fruit of giving that’s apparent in the present life. 

Furthermore,\marginnote{6.1} when a giver’s body breaks up, after death, they’re reborn in a good place, a heavenly realm. This is a fruit of giving to do with lives to come.” 

When\marginnote{7.1} he said this, General \textsanskrit{Sīha} said to the Buddha, “When it comes to those four fruits of giving that are apparent in the present life, I don’t have to rely on faith in the Buddha, for I know them too. I’m a giver, a donor, and am dear and beloved to many people. I’m a giver, and good people associate with me. I’m a giver, and I have this good reputation: ‘General \textsanskrit{Sīha} gives, serves, and attends on the \textsanskrit{Saṅgha}.’ I’m a giver, and I enter any kind of assembly bold and assured, whether it’s an assembly of aristocrats, brahmins, householders, or ascetics. When it comes to these four fruits of giving that are apparent in the present life, I don’t have to rely on faith in the Buddha, for I know them too. But when the Buddha says: ‘When a giver’s body breaks up, after death, they’re reborn in a good place, a heavenly realm.’ I don’t know this, so I have to rely on faith in the Buddha.” 

“That’s\marginnote{7.10} so true, \textsanskrit{Sīha}! That’s so true! When a giver’s body breaks up, after death, they’re reborn in a good place, a heavenly realm. 

\begin{verse}%
Giving,\marginnote{8.1} you’re loved and befriended by many people. \\
You get a good reputation, and your fame grows. \\
A generous man enters an assembly \\
bold and assured. 

So\marginnote{9.1} an astute person, seeking happiness, would give gifts, \\
having driven out the stain of stinginess. \\
They live long in the heaven of the Three and Thirty, \\
enjoying the company of the gods. 

Having\marginnote{10.1} taken the opportunity to do good, when they pass from here \\
they wander radiant in the Garden of Delight. \\
There they delight, rejoice, and enjoy themselves, \\
provided with the five kinds of sensual stimulation. \\
Having practiced the word of the unattached, the poised, \\
disciples of the Holy One rejoice in heaven.” 

%
\end{verse}

%
\section*{{\suttatitleacronym AN 5.35}{\suttatitletranslation The Benefits of Giving }{\suttatitleroot Dānānisaṁsasutta}}
\addcontentsline{toc}{section}{\tocacronym{AN 5.35} \toctranslation{The Benefits of Giving } \tocroot{Dānānisaṁsasutta}}
\markboth{The Benefits of Giving }{Dānānisaṁsasutta}
\extramarks{AN 5.35}{AN 5.35}

“Mendicants,\marginnote{1.1} there are five benefits of giving. What five? A giver, a donor is dear and beloved by many people. Good people associate with them. They get a good reputation. They don’t neglect a layperson’s duties. When their body breaks up, after death, they’re reborn in a good place, a heavenly realm. These are the five benefits of giving. 

\begin{verse}%
Giving,\marginnote{2.1} one is loved, \\
and follows the way of the good. \\
The good, disciplined spiritual practitioners \\
associate with you. 

They\marginnote{3.1} teach you the Dhamma \\
that dispels all suffering. \\
Having understood this teaching, \\
the undefiled become fully extinguished.” 

%
\end{verse}

%
\section*{{\suttatitleacronym AN 5.36}{\suttatitletranslation Timely Gifts }{\suttatitleroot Kāladānasutta}}
\addcontentsline{toc}{section}{\tocacronym{AN 5.36} \toctranslation{Timely Gifts } \tocroot{Kāladānasutta}}
\markboth{Timely Gifts }{Kāladānasutta}
\extramarks{AN 5.36}{AN 5.36}

“Mendicants,\marginnote{1.1} there are these five timely gifts. What five? A gift to a visitor. A gift to someone setting out on a journey. A gift to someone who is sick. A gift at a time of famine. Presenting the freshly harvested grains and fruits first to those who are ethical. These are the five timely gifts. 

\begin{verse}%
The\marginnote{2.1} wise give at the right time, \\
being bountiful and rid of stinginess. \\
A religious donation at the right time \\
to the noble ones, upright and poised, 

given\marginnote{3.1} with a clear and confident mind, \\
is indeed abundant. \\
Those who rejoice at that, \\
or do other services, \\
don’t miss out on the offering; \\
they too have a share in the merit. 

So\marginnote{4.1} you should give without holding back, \\
where a gift is very fruitful. \\
The good deeds of sentient beings \\
support them in the next world.” 

%
\end{verse}

%
\section*{{\suttatitleacronym AN 5.37}{\suttatitletranslation Food }{\suttatitleroot Bhojanasutta}}
\addcontentsline{toc}{section}{\tocacronym{AN 5.37} \toctranslation{Food } \tocroot{Bhojanasutta}}
\markboth{Food }{Bhojanasutta}
\extramarks{AN 5.37}{AN 5.37}

“Mendicants,\marginnote{1.1} when a giver gives food, they give the recipients five things. What five? Long life, beauty, happiness, strength, and eloquence. 

Giving\marginnote{1.4} long life, they have long life as a god or human. 

Giving\marginnote{1.5} beauty, they have beauty as a god or human. 

Giving\marginnote{1.6} happiness, they have happiness as a god or human. 

Giving\marginnote{1.7} strength, they have strength as a god or human. 

Giving\marginnote{1.8} eloquence, they are eloquent as a god or human. 

When\marginnote{1.9} a giver gives food, they give the recipients five things. 

\begin{verse}%
A\marginnote{2.1} wise one is a giver of life, strength, \\
beauty, and eloquence. \\
An intelligent giver of happiness \\
gains happiness in return. 

Giving\marginnote{3.1} life, strength, beauty, \\
happiness, and eloquence, \\
they’re long-lived and famous \\
wherever they’re reborn.” 

%
\end{verse}

%
\section*{{\suttatitleacronym AN 5.38}{\suttatitletranslation Faith }{\suttatitleroot Saddhasutta}}
\addcontentsline{toc}{section}{\tocacronym{AN 5.38} \toctranslation{Faith } \tocroot{Saddhasutta}}
\markboth{Faith }{Saddhasutta}
\extramarks{AN 5.38}{AN 5.38}

“Mendicants,\marginnote{1.1} a faithful gentleman gets five benefits. What five? The good persons in the world show compassion first to the faithful, not so much to the unfaithful. They first approach the faithful, not so much the unfaithful. They first receive alms from the faithful, not so much the unfaithful. They first teach Dhamma to the faithful, not so much the unfaithful. When their body breaks up, after death, the faithful are reborn in a good place, a heavenly realm. A faithful gentleman gets these five benefits. 

Suppose\marginnote{2.1} there was a great banyan tree at a level crossroads. It would become a refuge for birds from all around. In the same way, a faithful gentleman becomes a refuge for many people—monks, nuns, laywomen, and laymen. 

\begin{verse}%
With\marginnote{3.1} its branches, leaves, and fruit, \\
a great tree with its strong trunk, \\
firmly-rooted and fruit-bearing, \\
supports many birds. 

It’s\marginnote{4.1} a lovely place, \\
frequented by the sky-soarers. \\
Those that need shade go in the shade, \\
those that need fruit enjoy the fruit. 

So\marginnote{5.1} too, a faithful individual \\
is perfect in ethics, \\
humble and amenable, \\
sweet, friendly, and tender. 

Those\marginnote{6.1} free of greed, freed of hate, \\
free of delusion, undefiled, \\
fields of merit for the world, \\
associate with such a person. 

They\marginnote{7.1} teach them the Dhamma, \\
that dispels all suffering. \\
Having understood this teaching, \\
the undefiled become fully extinguished.” 

%
\end{verse}

%
\section*{{\suttatitleacronym AN 5.39}{\suttatitletranslation A Child }{\suttatitleroot Puttasutta}}
\addcontentsline{toc}{section}{\tocacronym{AN 5.39} \toctranslation{A Child } \tocroot{Puttasutta}}
\markboth{A Child }{Puttasutta}
\extramarks{AN 5.39}{AN 5.39}

“Mendicants,\marginnote{1.1} parents see five reasons to wish for the birth of a child in the family. What five? Since we looked after them, they’ll look after us. They’ll do their duty for us. The family traditions will last. They’ll take care of the inheritance. Or else when we have passed away they’ll give an offering on our behalf. Parents see these five reasons to wish for the birth of a child in the family. 

\begin{verse}%
Seeing\marginnote{2.1} five reasons, \\
astute people wish for a child. \\
Since we looked after them, they’ll look after us. \\
They’ll do their duty for us. 

The\marginnote{3.1} family traditions will last. \\
They’ll take care of the inheritance. \\
Or else when we have passed away \\
they’ll give an offering on our behalf. 

Seeing\marginnote{4.1} these five reasons \\
astute people wish for a child. \\
And so good people, \\
grateful and thankful, 

look\marginnote{5.1} after their parents, \\
remembering past deeds. \\
They do for their parents, \\
as their parents did for them in the past. 

Following\marginnote{6.1} their advice, looking after those who raised them, \\
the family traditions are not lost. \\
Faithful, accomplished in ethics, \\
such a child is praiseworthy.” 

%
\end{verse}

%
\section*{{\suttatitleacronym AN 5.40}{\suttatitletranslation Great Sal Trees }{\suttatitleroot Mahāsālaputtasutta}}
\addcontentsline{toc}{section}{\tocacronym{AN 5.40} \toctranslation{Great Sal Trees } \tocroot{Mahāsālaputtasutta}}
\markboth{Great Sal Trees }{Mahāsālaputtasutta}
\extramarks{AN 5.40}{AN 5.40}

“Mendicants,\marginnote{1.1} great sal trees grow in five ways supported by the Himalayas, the king of mountains. What five? The branches, leaves, and foliage; the bark; the shoots; the softwood; and the hardwood. Great sal trees grow in these five ways supported by the Himalayas, the king of mountains. 

In\marginnote{1.9} the same way, a family grows in five ways supported by a family head with faith. What five? Faith, ethics, learning, generosity, and wisdom. A family grows in these five ways supported by a family head with faith. 

\begin{verse}%
Supported\marginnote{2.1} by the mountain crags \\
in the wilds, the formidable forest, \\
the tree grows \\
to become lord of the forest. 

So\marginnote{3.1} too, when the family head \\
is ethical and faithful, \\
supported by them, they grow: \\
children, partners, and kin, \\
colleagues, relatives, \\
and those dependent for their livelihood. 

Seeing\marginnote{4.1} the ethical conduct of the virtuous, \\
the generosity and good deeds, \\
those who see clearly \\
do likewise. 

Having\marginnote{5.1} practiced the teaching here, \\
the path that goes to a good place, \\
they delight in the heavenly realm, \\
enjoying all the pleasures they desire.” 

%
\end{verse}

%
\addtocontents{toc}{\let\protect\contentsline\protect\nopagecontentsline}
\chapter*{The Chapter with King Muṇḍa }
\addcontentsline{toc}{chapter}{\tocchapterline{The Chapter with King Muṇḍa }}
\addtocontents{toc}{\let\protect\contentsline\protect\oldcontentsline}

%
\section*{{\suttatitleacronym AN 5.41}{\suttatitletranslation Getting Rich }{\suttatitleroot Ādiyasutta}}
\addcontentsline{toc}{section}{\tocacronym{AN 5.41} \toctranslation{Getting Rich } \tocroot{Ādiyasutta}}
\markboth{Getting Rich }{Ādiyasutta}
\extramarks{AN 5.41}{AN 5.41}

At\marginnote{1.1} one time the Buddha was staying near \textsanskrit{Sāvatthī} in Jeta’s Grove, \textsanskrit{Anāthapiṇḍika}’s monastery. Then the householder \textsanskrit{Anāthapiṇḍika} went up to the Buddha, bowed, and sat down to one side. The Buddha said to him: 

“Householder,\marginnote{1.3} there are these five reasons to get rich. What five? 

Firstly,\marginnote{1.5} with his legitimate wealth—earned by his efforts and initiative, built up with his own hands, gathered by the sweat of the brow—a noble disciple makes himself happy and pleased, keeping himself properly happy. He makes his mother and father happy … He makes his children, partners, bondservants, workers, and staff happy … This is the first reason to get rich. 

Furthermore,\marginnote{2.1} with his legitimate wealth he makes his friends and colleagues happy … This is the second reason to get rich. 

Furthermore,\marginnote{3.1} with his legitimate wealth he protects himself against losses from such things as fire, water, kings, bandits, or unloved heirs. He keeps himself safe. This is the third reason to get rich. 

Furthermore,\marginnote{4.1} with his legitimate wealth he makes five spirit-offerings: to relatives, guests, ancestors, king, and deities. This is the fourth reason to get rich. 

Furthermore,\marginnote{5.1} with his legitimate wealth he establishes an uplifting religious donation for ascetics and brahmins—those who avoid intoxication and negligence, are settled in patience and gentleness, and who tame, calm, and extinguish themselves—that’s conducive to heaven, ripens in happiness, and leads to heaven. This is the fifth reason to get rich. 

These\marginnote{5.3} are the five reasons to get rich. 

Now\marginnote{6.1} if the riches a noble disciple gets for these five reasons run out, he thinks: ‘So, the riches I have obtained for these reasons are running out.’ And so he has no regrets. 

But\marginnote{6.4} if the riches a noble disciple gets for these five reasons increase, he thinks: ‘So, the riches I have obtained for these reasons are increasing.’ And so he has no regrets in both cases. 

\begin{verse}%
‘I’ve\marginnote{7.1} enjoyed my wealth, supporting those who depend on me; \\
I’ve overcome losses; \\
I’ve given uplifting religious donations; \\
and made the five spirit-offerings. \\
I have looked after the ethical and \\
disciplined spiritual practitioners. 

I’ve\marginnote{8.1} achieved the purpose \\
for which an astute lay person \\
wishes to gain wealth. \\
I don’t regret what I’ve done.’ 

A\marginnote{9.1} mortal person who recollects this \\
stands firm in the teaching of the noble ones. \\
They’re praised in this life by the astute, \\
and they depart to rejoice in heaven.” 

%
\end{verse}

%
\section*{{\suttatitleacronym AN 5.42}{\suttatitletranslation A Good Person }{\suttatitleroot Sappurisasutta}}
\addcontentsline{toc}{section}{\tocacronym{AN 5.42} \toctranslation{A Good Person } \tocroot{Sappurisasutta}}
\markboth{A Good Person }{Sappurisasutta}
\extramarks{AN 5.42}{AN 5.42}

“Mendicants,\marginnote{1.1} a good person is born in a family for the benefit, welfare, and happiness of the people. For the benefit, welfare, and happiness of mother and father; children and partners; bondservants, workers, and staff; friends and colleagues; and ascetics and brahmins. 

It’s\marginnote{2.1} like a great rain cloud, which nourishes all the crops for the benefit, welfare, and happiness of the people. In the same way, a good person is born in a family for the benefit, welfare, and happiness of the people. … 

\begin{verse}%
The\marginnote{3.1} gods protect one who is guarded by principle, \\
who uses their wealth for the welfare of the many. \\
One who is learned, with precepts and observances intact, \\
and steady in principle, doesn’t lose their reputation. 

Firm\marginnote{4.1} in principle, accomplished in ethical conduct, \\
truthful, conscientious; \\
like a pendant of river gold, \\
who is worthy to criticize them? \\
Even the gods praise them, \\
and by \textsanskrit{Brahmā}, too, they’re praised.” 

%
\end{verse}

%
\section*{{\suttatitleacronym AN 5.43}{\suttatitletranslation Likable }{\suttatitleroot Iṭṭhasutta}}
\addcontentsline{toc}{section}{\tocacronym{AN 5.43} \toctranslation{Likable } \tocroot{Iṭṭhasutta}}
\markboth{Likable }{Iṭṭhasutta}
\extramarks{AN 5.43}{AN 5.43}

Then\marginnote{1.1} the householder \textsanskrit{Anāthapiṇḍika} went up to the Buddha, bowed, and sat down to one side. The Buddha said to him: 

“Householder,\marginnote{2.1} these five things that are likable, desirable, and agreeable are hard to get in the world. What five? Long life, beauty, happiness, fame, and heaven. These are the five things that are likable, desirable, and agreeable, but hard to get in the world. 

And\marginnote{3.1} I say that these five things are not got by praying or wishing for them. If they were, who would lack them? 

A\marginnote{4.1} noble disciple who wants to live long ought not pray for it, or hope for it, or pine for it. Instead, they should practice the way that leads to long life. For by practicing that way they gain long life as a god or a human being. 

A\marginnote{5.1} noble disciple who wants to be beautiful ought not pray for it, or hope for it, or pine for it. Instead, they should practice the way that leads to beauty. For by practicing that way they gain beauty as a god or a human being. 

A\marginnote{6.1} noble disciple who wants to be happy ought not pray for it, or hope for it, or pine for it. Instead, they should practice the way that leads to happiness. For by practicing that way they gain happiness as a god or a human being. 

A\marginnote{7.1} noble disciple who wants to be famous ought not pray for it, or hope for it, or pine for it. Instead, they should practice the way that leads to fame. For by practicing that way they gain fame as a god or a human being. 

A\marginnote{8.1} noble disciple who wants to go to heaven ought not pray for it, or hope for it, or pine for it. Instead, they should practice the way that leads to heaven. For by practicing that way they gain heaven, they are one who gains the heavens. 

\begin{verse}%
For\marginnote{9.1} one who desires a continuous flow \\
of exceptional delights—\\
long life, beauty, fame and reputation, \\
heaven, and birth in an eminent family—

the\marginnote{10.1} astute praise diligence \\
in making merit. \\
Being diligent, an astute person \\
secures both benefits: 

the\marginnote{11.1} benefit in this life, \\
and in lives to come. \\
A wise one, comprehending the meaning, \\
is called ‘astute’.” 

%
\end{verse}

%
\section*{{\suttatitleacronym AN 5.44}{\suttatitletranslation Agreeable }{\suttatitleroot Manāpadāyīsutta}}
\addcontentsline{toc}{section}{\tocacronym{AN 5.44} \toctranslation{Agreeable } \tocroot{Manāpadāyīsutta}}
\markboth{Agreeable }{Manāpadāyīsutta}
\extramarks{AN 5.44}{AN 5.44}

At\marginnote{1.1} one time the Buddha was staying near \textsanskrit{Vesālī}, at the Great Wood, in the hall with the peaked roof. Then the Buddha robed up in the morning and, taking his bowl and robe, went to the home of the householder Ugga of \textsanskrit{Vesālī}, where he sat on the seat spread out. 

Then\marginnote{1.3} Ugga went up to the Buddha, bowed, sat down to one side, and said to him, “Sir, I have heard and learned this in the presence of the Buddha: ‘The giver of the agreeable gets the agreeable.’ My sal flower porridge is agreeable: may the Buddha please accept it from me out of compassion.” So the Buddha accepted it out of compassion. 

“Sir,\marginnote{3.1} I have heard and learned this in the presence of the Buddha: ‘The giver of the agreeable gets the agreeable.’ My pork with jujube is agreeable: may the Buddha please accept it from me out of compassion.” So the Buddha accepted it out of compassion. 

“…\marginnote{4.1} My fried vegetable stalks are agreeable: may the Buddha please accept them from me out of compassion.” So the Buddha accepted them out of compassion. 

“…\marginnote{5.1} My boiled fine rice with the dark grains picked out, served with many soups and sauces is agreeable: may the Buddha please accept it from me out of compassion.” So the Buddha accepted it out of compassion. 

“…\marginnote{6.1} My cloths imported from \textsanskrit{Kāsī} are agreeable: may the Buddha please accept them from me out of compassion.” So the Buddha accepted them out of compassion. 

“…\marginnote{7.1} My couch spread with woolen covers—shag-piled or embroidered with flowers—and spread with a fine deer hide, with a canopy above and red pillows at both ends is agreeable. But, sir, I know that this is not proper for the Buddha. However, this plank of sandalwood is worth over a thousand dollars. May the Buddha please accept it from me out of compassion.” So the Buddha accepted it out of compassion. 

And\marginnote{7.9} then the Buddha rejoiced with Ugga with these verses of appreciation: 

\begin{verse}%
“The\marginnote{8.1} giver of the agreeable gets the agreeable, \\
enthusiastically giving clothing, bedding, \\
food and drink, and various requisites \\
to those of upright conduct. 

Knowing\marginnote{9.1} the perfected ones to be like a field \\
for what’s given, offered and not held back, \\
a good person gives what’s hard to give: \\
the giver of the agreeable gets the agreeable.” 

%
\end{verse}

And\marginnote{10.1} then the Buddha, having rejoiced with Ugga with these verses of appreciation, got up from his seat and left. 

Then\marginnote{11.1} after some time Ugga passed away, and was reborn in a host of mind-made gods. At that time the Buddha was staying near \textsanskrit{Sāvatthī} in Jeta’s Grove, \textsanskrit{Anāthapiṇḍika}’s monastery. 

Then,\marginnote{11.4} late at night, the glorious god Ugga, lighting up the entire Jeta’s Grove, went up to the Buddha, bowed, and stood to one side. The Buddha said to him, “Ugga, I trust it is all you wished?” 

“Sir,\marginnote{11.6} it is indeed just as I wished.” Then the Buddha addressed Ugga in verse: 

\begin{verse}%
“The\marginnote{12.1} giver of the agreeable gets the agreeable, \\
the giver of the foremost gets the foremost, \\
the giver of the excellent gets the excellent, \\
the giver of the best gets the best. 

A\marginnote{13.1} person who gives the foremost, \\
the excellent, the best: \\
they’re long-lived and famous \\
wherever they’re reborn.” 

%
\end{verse}

%
\section*{{\suttatitleacronym AN 5.45}{\suttatitletranslation Overflowing Merit }{\suttatitleroot Puññābhisandasutta}}
\addcontentsline{toc}{section}{\tocacronym{AN 5.45} \toctranslation{Overflowing Merit } \tocroot{Puññābhisandasutta}}
\markboth{Overflowing Merit }{Puññābhisandasutta}
\extramarks{AN 5.45}{AN 5.45}

“Mendicants,\marginnote{1.1} there are these five kinds of overflowing merit, overflowing goodness. They nurture happiness and are conducive to heaven, ripening in happiness and leading to heaven. They lead to what is likable, desirable, agreeable, to welfare and happiness. 

What\marginnote{2.1} five? When a mendicant enters and remains in a limitless immersion of heart while using a robe … almsfood … lodging … bed and chair … medicines and supplies for the sick, the overflowing of merit for the donor is limitless … 

These\marginnote{4.2} are the five kinds of overflowing merit, overflowing goodness. They nurture happiness, and are conducive to heaven, ripening in happiness, and leading to heaven. They lead to what is likable, desirable, agreeable, to welfare and happiness. 

When\marginnote{5.1} a noble disciple has these five kinds of overflowing merit and goodness, it’s not easy to grasp how much merit they have by saying that this is the extent of their overflowing merit … that leads to happiness. It’s simply reckoned as an incalculable, immeasurable, great mass of merit. 

It’s\marginnote{6.1} like trying to grasp how much water is in the ocean. It’s not easy to say: ‘This is how many gallons, how many hundreds, thousands, hundreds of thousands of gallons there are.’ It’s simply reckoned as an incalculable, immeasurable, great mass of water. 

In\marginnote{6.4} the same way, when a noble disciple has these five kinds of overflowing merit and goodness, it’s not easy to grasp how much merit they have: ‘This is how much this overflowing merit … leads to happiness.’ It’s simply reckoned as an incalculable, immeasurable, great mass of merit. 

\begin{verse}%
Hosts\marginnote{7.1} of people use the rivers, \\
and though the rivers are many, \\
all reach the great deep, the boundless ocean, \\
the cruel sea that’s home to precious gems. 

So\marginnote{8.1} too, when a person gives food, drink, and clothes; \\
and they’re a giver of beds, seats, and mats—\\
the streams of merit reach that astute person, \\
as the rivers bring their waters to the sea.” 

%
\end{verse}

%
\section*{{\suttatitleacronym AN 5.46}{\suttatitletranslation Success }{\suttatitleroot Sampadāsutta}}
\addcontentsline{toc}{section}{\tocacronym{AN 5.46} \toctranslation{Success } \tocroot{Sampadāsutta}}
\markboth{Success }{Sampadāsutta}
\extramarks{AN 5.46}{AN 5.46}

“Mendicants,\marginnote{1.1} there are five accomplishments. What five? Accomplishment in faith, ethics, learning, generosity, and wisdom. These are the five accomplishments.” 

%
\section*{{\suttatitleacronym AN 5.47}{\suttatitletranslation Wealth }{\suttatitleroot Dhanasutta}}
\addcontentsline{toc}{section}{\tocacronym{AN 5.47} \toctranslation{Wealth } \tocroot{Dhanasutta}}
\markboth{Wealth }{Dhanasutta}
\extramarks{AN 5.47}{AN 5.47}

“Mendicants,\marginnote{1.1} there are these five kinds of wealth. What five? The wealth of faith, ethics, learning, generosity, and wisdom. 

And\marginnote{2.1} what is the wealth of faith? It’s when a noble disciple has faith in the Realized One’s awakening: ‘That Blessed One is perfected, a fully awakened Buddha, accomplished in knowledge and conduct, holy, knower of the world, supreme guide for those who wish to train, teacher of gods and humans, awakened, blessed.’ This is called the wealth of faith. 

And\marginnote{3.1} what is the wealth of ethics? It’s when a noble disciple doesn’t kill living creatures, steal, commit sexual misconduct, lie, or take alcoholic drinks that cause negligence. This is called the wealth of ethics. 

And\marginnote{4.1} what is the wealth of learning? It’s when a noble disciple is very learned, remembering and keeping what they’ve learned. These teachings are good in the beginning, good in the middle, and good in the end, meaningful and well-phrased, describing a spiritual practice that’s totally full and pure. They are very learned in such teachings, remembering them, reciting them, mentally scrutinizing them, and comprehending them theoretically. This is called the wealth of learning. 

And\marginnote{5.1} what is the wealth of generosity? It’s when a noble disciple lives at home rid of the stain of stinginess, freely generous, open-handed, loving to let go, committed to charity, loving to give and to share. This is called the wealth of generosity. 

And\marginnote{6.1} what is the wealth of wisdom? It’s when a noble disciple is wise. They have the wisdom of arising and passing away which is noble, penetrative, and leads to the complete ending of suffering. This is called the wealth of wisdom. 

These\marginnote{6.4} are the five kinds of wealth. 

\begin{verse}%
Whoever\marginnote{7.1} has faith in the Realized One, \\
unwavering and well grounded; \\
whose ethical conduct is good, \\
praised and loved by the noble ones; 

who\marginnote{8.1} has confidence in the \textsanskrit{Saṅgha}, \\
and correct view: \\
they’re said to be prosperous, \\
their life is not in vain. 

So\marginnote{9.1} let the wise devote themselves \\
to faith, ethical behavior, \\
confidence, and insight into the teaching, \\
remembering the instructions of the Buddhas.” 

%
\end{verse}

%
\section*{{\suttatitleacronym AN 5.48}{\suttatitletranslation Things That Cannot Be Had }{\suttatitleroot Alabbhanīyaṭhānasutta}}
\addcontentsline{toc}{section}{\tocacronym{AN 5.48} \toctranslation{Things That Cannot Be Had } \tocroot{Alabbhanīyaṭhānasutta}}
\markboth{Things That Cannot Be Had }{Alabbhanīyaṭhānasutta}
\extramarks{AN 5.48}{AN 5.48}

“Mendicants,\marginnote{1.1} there are five things that cannot be had by any ascetic or brahmin or god or \textsanskrit{Māra} or \textsanskrit{Brahmā} or by anyone in the world. What five? That someone liable to old age should not grow old. That someone liable to sickness should not get sick. … That someone liable to death should not die. … That someone liable to ending should not end. … That someone liable to perishing should not perish. … 

An\marginnote{2.1} unlearned ordinary person has someone liable to old age who grows old. But they don’t reflect on old age: ‘It’s not just me who has someone liable to old age who grows old. For as long as sentient beings come and go, pass away and are reborn, they all have someone liable to old age who grows old. If I were to sorrow and wail and lament, beating my breast and falling into confusion, just because someone liable to old age grows old, I’d lose my appetite and my physical appearance would deteriorate. My work wouldn’t get done, my enemies would be encouraged, and my friends would be dispirited.’ And so, when someone liable to old age grows old, they sorrow and wail and lament, beating their breast and falling into confusion. This is called an unlearned ordinary person struck by sorrow’s poisoned arrow, who only mortifies themselves. 

Furthermore,\marginnote{3.1} an unlearned ordinary person has someone liable to sickness … death … ending … perishing. But they don’t reflect on perishing: ‘It’s not just me who has someone liable to perishing who perishes. For as long as sentient beings come and go, pass away and are reborn, they all have someone liable to perishing who perishes. If I were to sorrow and wail and lament, beating my breast and falling into confusion, just because someone liable to perishing perishes, I’d lose my appetite and my physical appearance would deteriorate. My work wouldn’t get done, my enemies would be encouraged, and my friends would be dispirited.’ And so, when someone liable to perishing perishes, they sorrow and wail and lament, beating their breast and falling into confusion. This is called an unlearned ordinary person struck by sorrow’s poisoned arrow, who only mortifies themselves. 

A\marginnote{4.1} learned noble disciple has someone liable to old age who grows old. So they reflect on old age: ‘It’s not just me who has someone liable to old age who grows old. For as long as sentient beings come and go, pass away and are reborn, they all have someone liable to old age who grows old. If I were to sorrow and wail and lament, beating my breast and falling into confusion, just because someone liable to old age grows old, I’d lose my appetite and my physical appearance would deteriorate. My work wouldn’t get done, my enemies would be encouraged, and my friends would be dispirited.’ And so, when someone liable to old age grows old, they don’t sorrow and wail and lament, beating their breast and falling into confusion. This is called a learned noble disciple who has drawn out sorrow’s poisoned arrow, struck by which unlearned ordinary people only mortify themselves. Sorrowless, free of thorns, that noble disciple only extinguishes themselves. 

Furthermore,\marginnote{5.1} a learned noble disciple has someone liable to sickness … death … ending … perishing. So they reflect on perishing: ‘It’s not just me who has someone liable to perishing who perishes. For as long as sentient beings come and go, pass away and are reborn, they all have someone liable to perishing who perishes. If I were to sorrow and wail and lament, beating my breast and falling into confusion, just because someone liable to perishing perishes, I’d lose my appetite and my physical appearance would deteriorate. My work wouldn’t get done, my enemies would be encouraged, and my friends would be dispirited.’ And so, when someone liable to perishing perishes, they don’t sorrow and wail and lament, beating their breast and falling into confusion. This is called a learned noble disciple who has drawn out sorrow’s poisoned arrow, struck by which unlearned ordinary people only mortify themselves. Sorrowless, free of thorns, that noble disciple only extinguishes themselves. 

These\marginnote{6.1} are the five things that cannot be had by any ascetic or brahmin or god or \textsanskrit{Māra} or \textsanskrit{Brahmā} or by anyone in the world. 

\begin{verse}%
Sorrowing\marginnote{7.1} and lamenting \\
doesn’t do even a little bit of good. \\
When they know that you’re sad, \\
your enemies are encouraged. 

When\marginnote{8.1} an astute person doesn’t waver in the face of adversity, \\
as they’re able to assess what’s beneficial, \\
their enemies suffer, \\
seeing that their normal expression doesn’t change. 

Chants,\marginnote{9.1} recitations, fine sayings, \\
charity or traditions: \\
if by means of any such things you benefit, \\
then by all means keep doing them. 

But\marginnote{10.1} if you understand that ‘this good thing \\
can’t be had by me or by anyone else’, \\
you should accept it without sorrowing, thinking: \\
‘The karma is strong. What can I do now?’” 

%
\end{verse}

%
\section*{{\suttatitleacronym AN 5.49}{\suttatitletranslation The King of Kosala }{\suttatitleroot Kosalasutta}}
\addcontentsline{toc}{section}{\tocacronym{AN 5.49} \toctranslation{The King of Kosala } \tocroot{Kosalasutta}}
\markboth{The King of Kosala }{Kosalasutta}
\extramarks{AN 5.49}{AN 5.49}

At\marginnote{1.1} one time the Buddha was staying near \textsanskrit{Sāvatthī} in Jeta’s Grove, \textsanskrit{Anāthapiṇḍika}’s monastery. Then King Pasenadi of Kosala went up to the Buddha, bowed, and sat down to one side. 

Then\marginnote{2.1} a man went up to the king and whispered in his ear, “Your Majesty, Queen \textsanskrit{Mallikā} has passed away.” 

When\marginnote{2.4} this was said, King Pasenadi was miserable and sad. He sat with his shoulders drooping, downcast, depressed, with nothing to say. 

Knowing\marginnote{3.1} this, the Buddha said to him, “Great king, there are five things that cannot be had by any ascetic or brahmin or god or \textsanskrit{Māra} or \textsanskrit{Brahmā} or by anyone in the world. What five? That someone liable to old age should not grow old. … Sorrowing and lamenting doesn’t do even a little bit of good … ‘The karma is strong. What can I do now?’” 

%
\section*{{\suttatitleacronym AN 5.50}{\suttatitletranslation With Nārada }{\suttatitleroot Nāradasutta}}
\addcontentsline{toc}{section}{\tocacronym{AN 5.50} \toctranslation{With Nārada } \tocroot{Nāradasutta}}
\markboth{With Nārada }{Nāradasutta}
\extramarks{AN 5.50}{AN 5.50}

At\marginnote{1.1} one time Venerable \textsanskrit{Nārada} was staying near \textsanskrit{Pāṭaliputta}, in the Chicken Monastery. 

Now\marginnote{1.2} at that time King \textsanskrit{Muṇḍa}’s dear and beloved Queen \textsanskrit{Bhaddā} had just passed away. And since that time, the king did not bathe, anoint himself, eat his meals, or apply himself to his work. Day and night he brooded over Queen \textsanskrit{Bhaddā}’s corpse. 

Then\marginnote{1.5} King \textsanskrit{Muṇḍa} addressed his treasurer, Piyaka, 

“So,\marginnote{1.6} my good Piyaka, please place Queen \textsanskrit{Bhaddā}’s corpse in an iron case filled with oil. Then close it up with another case, so that we can view Queen \textsanskrit{Bhaddā}’s body even longer.” 

“Yes,\marginnote{1.7} Your Majesty,” replied Piyaka the treasurer, and he did as the king instructed. 

Then\marginnote{2.1} it occurred to Piyaka, “King \textsanskrit{Muṇḍa}’s dear and beloved Queen \textsanskrit{Bhaddā} has passed away. Since then the king does not bathe, anoint himself, eat his meals, or apply himself to his work. Day and night he broods over Queen \textsanskrit{Bhaddā}’s corpse. Now, what ascetic or brahmin might the king pay homage to, whose teaching could help the king give up sorrow’s arrow?” 

Then\marginnote{3.1} it occurred to Piyaka, “This Venerable \textsanskrit{Nārada} is staying in the Chicken Monastery at \textsanskrit{Pāṭaliputta}. He has this good reputation: ‘He is astute, competent, intelligent, learned, a brilliant speaker, eloquent, mature, a perfected one.’ What if King \textsanskrit{Muṇḍa} was to pay homage to Venerable \textsanskrit{Nārada}? Hopefully when he hears \textsanskrit{Nārada}’s teaching, the king could give up sorrow’s arrow.” 

Then\marginnote{4.1} Piyaka went to the king and said to him, “Sire, this Venerable \textsanskrit{Nārada} is staying in the Chicken Monastery at \textsanskrit{Pāṭaliputta}. He has this good reputation: ‘He is astute, competent, intelligent, learned, a brilliant speaker, eloquent, mature, a perfected one.’ What if Your Majesty was to pay homage to Venerable \textsanskrit{Nārada}? Hopefully when you hear \textsanskrit{Nārada}’s teaching, you could give up sorrow’s arrow.” 

“Well\marginnote{4.6} then, my good Piyaka, let \textsanskrit{Nārada} know. For how could one such as I presume to visit an ascetic or brahmin in my realm without first letting them know?” 

“Yes,\marginnote{4.8} Your Majesty,” replied Piyaka the treasurer. He went to \textsanskrit{Nārada}, bowed, sat down to one side, and said to him, “Sir, King \textsanskrit{Muṇḍa}’s dear and beloved Queen \textsanskrit{Bhaddā} has passed away. And since she passed away, the king has not bathed, anointed himself, eaten his meals, or got his business done. Day and night he broods over Queen \textsanskrit{Bhaddā}’s corpse. Sir, please teach the king so that, when he hears your teaching, he can give up sorrow’s arrow.” 

“Please,\marginnote{5.5} Piyaka, let the king come when he likes.” 

Then\marginnote{6.1} Piyaka got up from his seat, bowed, and respectfully circled Venerable \textsanskrit{Nārada}, keeping him on his right, before going to the king and saying, “Sire, the request for an audience with Venerable \textsanskrit{Nārada} has been granted. Please, Your Majesty, go at your convenience.” 

“Well\marginnote{6.4} then, my good Piyaka, harness the finest chariots.” 

“Yes,\marginnote{6.5} Your Majesty,” replied Piyaka the treasurer. He did so, then told the king: 

“Sire,\marginnote{6.6} the finest chariots are harnessed. Please, Your Majesty, go at your convenience.” 

Then\marginnote{7.1} King \textsanskrit{Muṇḍa} mounted a fine carriage and, along with other fine carriages, set out in full royal pomp to see Venerable \textsanskrit{Nārada} at the Chicken Monastery. He went by carriage as far as the terrain allowed, then descended and entered the monastery on foot. Then the king went up to \textsanskrit{Nārada}, bowed, and sat down to one side. Then \textsanskrit{Nārada} said to him: 

“Great\marginnote{8.1} king, there are five things that cannot be had by any ascetic or brahmin or god or \textsanskrit{Māra} or \textsanskrit{Brahmā} or by anyone in the world. What five? That someone liable to old age should not grow old. … That someone liable to sickness should not get sick. … That someone liable to death should not die. … That someone liable to ending should not end. … That someone liable to perishing should not perish. … 

An\marginnote{9.1} unlearned ordinary person has someone liable to old age who grows old. But they don’t reflect on old age: ‘It’s not just me who has someone liable to old age who grows old. For all sentient beings have someone liable to old age who grows old, as long as sentient beings come and go, pass away and are reborn. If I were to sorrow and wail and lament, beating my breast and falling into confusion, just because someone liable to old age grows old, I’d lose my appetite and my physical appearance would deteriorate. My work wouldn’t get done, my enemies would be encouraged, and my friends would be dispirited.’ And so, when someone liable to old age grows old, they sorrow and wail and lament, beating their breast and falling into confusion. This is called an unlearned ordinary person struck by sorrow’s poisoned arrow, who only mortifies themselves. 

Furthermore,\marginnote{10.1} an unlearned ordinary person has someone liable to sickness … death … ending … perishing. But they don’t reflect on perishing: ‘It’s not just me who has someone liable to perishing who perishes. For all sentient beings have someone liable to perishing who perishes, as long as sentient beings come and go, pass away and are reborn. If I were to sorrow and wail and lament, beating my breast and falling into confusion, just because someone liable to perishing perishes, I’d lose my appetite and my physical appearance would deteriorate. My work wouldn’t get done, my enemies would be encouraged, and my friends would be dispirited.’ And so, when someone liable to perishing perishes, they sorrow and wail and lament, beating their breast and falling into confusion. This is called an unlearned ordinary person struck by sorrow’s poisoned arrow, who only mortifies themselves. 

A\marginnote{11.1} learned noble disciple has someone liable to old age who grows old. So they reflect on old age: ‘It’s not just me who has someone liable to old age who grows old. For all sentient beings have someone liable to old age who grows old, as long as sentient beings come and go, pass away and are reborn. If I were to sorrow and wail and lament, beating my breast and falling into confusion, just because someone liable to old age grows old, I’d lose my appetite and my physical appearance would deteriorate. My work wouldn’t get done, my enemies would be encouraged, and my friends would be dispirited.’ And so, when someone liable to old age grows old, they don’t sorrow and wail and lament, beating their breast and falling into confusion. This is called a learned noble disciple who has drawn out sorrow’s poisoned arrow, struck by which unlearned ordinary people only mortify themselves. Sorrowless, free of thorns, that noble disciple only extinguishes themselves. 

Furthermore,\marginnote{12.1} a learned noble disciple has someone liable to sickness … death … ending … perishing. So they reflect on perishing: ‘It’s not just me who has someone liable to perishing who perishes. For all sentient beings have someone liable to perishing who perishes, as long as sentient beings come and go, pass away and are reborn. If I were to sorrow and wail and lament, beating my breast and falling into confusion, just because someone liable to perishing perishes, I’d lose my appetite and my physical appearance would deteriorate. My work wouldn’t get done, my enemies would be encouraged, and my friends would be dispirited.’ And so, when someone liable to perishing perishes, they don’t sorrow and wail and lament, beating their breast and falling into confusion. This is called a learned noble disciple who has drawn out sorrow’s poisoned arrow, struck by which unlearned ordinary people only mortify themselves. Sorrowless, free of thorns, that noble disciple only extinguishes themselves. 

These\marginnote{13.1} are the five things that cannot be had by any ascetic or brahmin or god or \textsanskrit{Māra} or \textsanskrit{Brahmā} or by anyone in the world. 

\begin{verse}%
Sorrowing\marginnote{14.1} and lamenting \\
doesn’t do even a little bit of good. \\
When they know that you’re sad, \\
your enemies are encouraged. 

When\marginnote{15.1} an astute person doesn’t waver in the face of adversity, \\
as they’re able to assess what’s beneficial, \\
their enemies suffer, \\
seeing that their normal expression doesn’t change. 

Chants,\marginnote{16.1} recitations, fine sayings, \\
charity or traditions: \\
if by means of any such things you benefit, \\
then by all means keep doing them. 

But\marginnote{17.1} if you understand that ‘this good thing \\
can’t be had by me or by anyone else’, \\
you should accept it without sorrowing, thinking: \\
‘The karma is strong. What can I do now?’” 

%
\end{verse}

When\marginnote{18.1} he said this, King \textsanskrit{Muṇḍa} said to Venerable \textsanskrit{Nārada}, “Sir, what is the name of this exposition of the teaching?” 

“Great\marginnote{18.3} king, this exposition of the teaching is called ‘Pulling Out Sorrow’s Arrow’.” 

“Indeed,\marginnote{18.4} sir, this is the pulling out of sorrow’s arrow! Hearing this exposition of the teaching, I’ve given up sorrow’s arrow.” 

Then\marginnote{19.1} King \textsanskrit{Muṇḍa} addressed his treasurer, Piyaka, “Well then, my good Piyaka, cremate Queen \textsanskrit{Bhaddā}’s corpse and build a monument. From this day forth, I will bathe, anoint myself, eat my meals, and apply myself to my work.” 

%
\addtocontents{toc}{\let\protect\contentsline\protect\nopagecontentsline}
\pannasa{The Second Fifty }
\addcontentsline{toc}{pannasa}{The Second Fifty }
\markboth{}{}
\addtocontents{toc}{\let\protect\contentsline\protect\oldcontentsline}

%
\addtocontents{toc}{\let\protect\contentsline\protect\nopagecontentsline}
\chapter*{The Chapter on Hindrances }
\addcontentsline{toc}{chapter}{\tocchapterline{The Chapter on Hindrances }}
\addtocontents{toc}{\let\protect\contentsline\protect\oldcontentsline}

%
\section*{{\suttatitleacronym AN 5.51}{\suttatitletranslation Obstacles }{\suttatitleroot Āvaraṇasutta}}
\addcontentsline{toc}{section}{\tocacronym{AN 5.51} \toctranslation{Obstacles } \tocroot{Āvaraṇasutta}}
\markboth{Obstacles }{Āvaraṇasutta}
\extramarks{AN 5.51}{AN 5.51}

\scevam{So\marginnote{1.1} I have heard. }At one time the Buddha was staying near \textsanskrit{Sāvatthī} in Jeta’s Grove, \textsanskrit{Anāthapiṇḍika}’s monastery. There the Buddha addressed the mendicants, “Mendicants!” 

“Venerable\marginnote{1.5} sir,” they replied. The Buddha said this: 

“Mendicants,\marginnote{2.1} there are these five obstacles and hindrances, parasites of the mind that weaken wisdom. What five? Sensual desire … Ill will … Dullness and drowsiness … Restlessness and remorse … Doubt … These are the five obstacles and hindrances, parasites of the mind that weaken wisdom. 

Take\marginnote{3.1} a mendicant who has feeble and weak wisdom, not having given up these five obstacles and hindrances, parasites of the mind that weaken wisdom. It’s simply impossible that they would know what’s for their own good, the good of another, or the good of both; or that they would realize any superhuman distinction in knowledge and vision worthy of the noble ones. 

Suppose\marginnote{3.2} there was a mountain river that flowed swiftly, going far, carrying all before it. But then a man would open channels on both sides, so the mid-river current would be dispersed, spread out, and separated. The river would no longer flow swiftly, going far, carrying all before it. 

In\marginnote{3.5} the same way, take a mendicant who has feeble and weak wisdom, not having given up these five obstacles and hindrances, parasites of the mind that weaken wisdom. It’s simply impossible that they would know what’s for their own good, the good of another, or the good of both; or that they would realize any superhuman distinction in knowledge and vision worthy of the noble ones. 

Take\marginnote{4.1} a mendicant who has powerful wisdom, having given up these five obstacles and hindrances, parasites of the mind that weaken wisdom. It’s quite possible that they would know what’s for their own good, the good of another, or the good of both; or that they would realize any superhuman distinction in knowledge and vision worthy of the noble ones. 

Suppose\marginnote{4.2} there was a mountain river that flowed swiftly, going far, carrying all before it. But then a man would close up the channels on both sides, so the mid-river current would not be dispersed, spread out, and separated. The river would keep flowing swiftly for a long way, carrying all before it. 

In\marginnote{4.5} the same way, take a mendicant who has powerful wisdom, having given up these five obstacles and hindrances, parasites of the mind that weaken wisdom. It’s quite possible that they would know what’s for their own good, the good of another, or the good of both; or that they would realize any superhuman distinction in knowledge and vision worthy of the noble ones.” 

%
\section*{{\suttatitleacronym AN 5.52}{\suttatitletranslation A Heap of the Unskillful }{\suttatitleroot Akusalarāsisutta}}
\addcontentsline{toc}{section}{\tocacronym{AN 5.52} \toctranslation{A Heap of the Unskillful } \tocroot{Akusalarāsisutta}}
\markboth{A Heap of the Unskillful }{Akusalarāsisutta}
\extramarks{AN 5.52}{AN 5.52}

“Mendicants,\marginnote{1.1} rightly speaking, you’d call the five hindrances a ‘heap of the unskillful’. For these five hindrances are entirely a heap of the unskillful. What five? The hindrances of sensual desire, ill will, dullness and drowsiness, restlessness and remorse, and doubt. Rightly speaking, you’d call these five hindrances a ‘heap of the unskillful’. For these five hindrances are entirely a heap of the unskillful.” 

%
\section*{{\suttatitleacronym AN 5.53}{\suttatitletranslation Factors That Support Meditation }{\suttatitleroot Padhāniyaṅgasutta}}
\addcontentsline{toc}{section}{\tocacronym{AN 5.53} \toctranslation{Factors That Support Meditation } \tocroot{Padhāniyaṅgasutta}}
\markboth{Factors That Support Meditation }{Padhāniyaṅgasutta}
\extramarks{AN 5.53}{AN 5.53}

“Mendicants,\marginnote{1.1} there are these five factors that support meditation. What five? 

It’s\marginnote{1.3} when a mendicant has faith in the Realized One’s awakening: ‘That Blessed One is perfected, a fully awakened Buddha, accomplished in knowledge and conduct, holy, knower of the world, supreme guide for those who wish to train, teacher of gods and humans, awakened, blessed.’ 

They\marginnote{1.5} are rarely ill or unwell. Their stomach digests well, being neither too hot nor too cold, but just right, and fit for meditation. 

They’re\marginnote{1.7} not devious or deceitful. They reveal themselves honestly to the Teacher or sensible spiritual companions. 

They\marginnote{1.9} live with energy roused up for giving up unskillful qualities and embracing skillful qualities. They’re strong, staunchly vigorous, not slacking off when it comes to developing skillful qualities. 

They’re\marginnote{1.10} wise. They have the wisdom of arising and passing away which is noble, penetrative, and leads to the complete ending of suffering. 

These\marginnote{1.11} are the five factors that support meditation.” 

%
\section*{{\suttatitleacronym AN 5.54}{\suttatitletranslation Times Good for Meditation }{\suttatitleroot Samayasutta}}
\addcontentsline{toc}{section}{\tocacronym{AN 5.54} \toctranslation{Times Good for Meditation } \tocroot{Samayasutta}}
\markboth{Times Good for Meditation }{Samayasutta}
\extramarks{AN 5.54}{AN 5.54}

“Mendicants,\marginnote{1.1} there are five times that are not good for meditation. What five? 

Firstly,\marginnote{1.3} a mendicant is old, overcome with old age. This is the first time that’s not good for meditation. 

Furthermore,\marginnote{2.1} a mendicant is sick, overcome by sickness. This is the second time that’s not good for meditation. 

Furthermore,\marginnote{3.1} there’s a famine, a bad harvest, so it’s hard to get almsfood, and not easy to keep going by collecting alms. This is the third time that’s not good for meditation. 

Furthermore,\marginnote{4.1} there’s peril from wild savages, and the countryfolk mount their vehicles and flee everywhere. This is the fourth time that’s not good for meditation. 

Furthermore,\marginnote{5.1} there’s a schism in the \textsanskrit{Saṅgha}. When the \textsanskrit{Saṅgha} is split, they abuse, insult, block, and reject each other. This doesn’t inspire confidence in those without it, and it causes some with confidence to change their minds. This is the fifth time that’s not good for meditation. 

These\marginnote{5.5} are the five times that are not good for meditation. 

There\marginnote{6.1} are five times that are good for meditation. What five? 

Firstly,\marginnote{6.3} a mendicant is a youth, young, black-haired, blessed with youth, in the prime of life. This is the first time that’s good for meditation. 

Furthermore,\marginnote{7.1} they are rarely ill or unwell. Their stomach digests well, being neither too hot nor too cold, but just right, and fit for meditation. This is the second time that’s good for meditation. 

Furthermore,\marginnote{8.1} there’s plenty of food, a good harvest, so it’s easy to get almsfood, and easy to keep going by collecting alms. This is the third time that’s good for meditation. 

Furthermore,\marginnote{9.1} people live in harmony, appreciating each other, without quarreling, blending like milk and water, and regarding each other with kindly eyes. This is the fourth time that’s good for meditation. 

Furthermore,\marginnote{10.1} the \textsanskrit{Saṅgha} lives comfortably, in harmony, appreciating each other, without quarreling, with one recitation. When the \textsanskrit{Saṅgha} is in harmony, they don’t abuse, insult, block, or reject each other. This inspires confidence in those without it, and increases confidence in those who have it. This is the fifth time that’s good for meditation. 

These\marginnote{10.5} are the five times that are good for meditation.” 

%
\section*{{\suttatitleacronym AN 5.55}{\suttatitletranslation Mother and Son }{\suttatitleroot Mātāputtasutta}}
\addcontentsline{toc}{section}{\tocacronym{AN 5.55} \toctranslation{Mother and Son } \tocroot{Mātāputtasutta}}
\markboth{Mother and Son }{Mātāputtasutta}
\extramarks{AN 5.55}{AN 5.55}

At\marginnote{1.1} one time the Buddha was staying near \textsanskrit{Sāvatthī} in Jeta’s Grove, \textsanskrit{Anāthapiṇḍika}’s monastery. 

Now,\marginnote{1.2} at that time a mother and son had both entered the rainy season residence at \textsanskrit{Sāvatthī}, as a monk and a nun. They wanted to see each other often. The mother wanted to see her son often, and the son his mother. Seeing each other often, they became close. Being so close, they became intimate. And being intimate, lust overcame them. With their minds swamped by lust, without resigning the training and declaring their inability to continue, they had sex. 

Then\marginnote{2.1} several mendicants went up to the Buddha, bowed, sat down to one side, and told him what had happened. 

“Mendicants,\marginnote{3.1} how could that silly man imagine that a mother cannot lust for her son, or that a son cannot lust for his mother? Compared to the sight of a woman, I do not see a single sight that is so arousing, sensuous, intoxicating, captivating, and infatuating, and such an obstacle to reaching the supreme sanctuary. Sentient beings are lustful, greedy, tied, infatuated, and attached to the sight of a woman. They sorrow for a long time under the sway of a woman’s sight. 

Compared\marginnote{4.1} to the sound … smell … taste … touch of a woman, I do not see a single touch that is so arousing, sensuous, intoxicating, captivating, and infatuating, and such an obstacle to reaching the supreme sanctuary. Sentient beings are lustful, greedy, tied, infatuated, and attached to the touch of a woman. They sorrow for a long time under the sway of a woman’s touch. 

When\marginnote{5.1} a woman walks, she occupies a man’s mind. When a woman stands … sits … lies down … laughs … speaks … sings … cries … is injured, she occupies a man’s mind. Even when a woman is dead, she occupies a man’s mind. For if anyone should be rightly called ‘an all-round snare of \textsanskrit{Māra}’, it’s females. 

\begin{verse}%
You\marginnote{6.1} might chat with someone who has knife in hand. \\
You might even chat with a goblin. \\
You might sit close by a viper, \\
whose bite would take your life. \\
But never should you chat \\
one on one with a female. 

They\marginnote{7.1} captivate the unmindful \\
with a glance and a smile. \\
Or scantily clad, \\
they speak charming words. \\
It’s not good to sit with such a person, \\
even if she’s injured or dead. 

These\marginnote{8.1} five kinds of sensual stimulation \\
are apparent in a woman’s body: \\
sights, sounds, tastes, smells, \\
and touches so delightful. 

Those\marginnote{9.1} swept away by the flood of sensual pleasures, \\
not comprehending them, \\
are governed by transmigration—\\
time and destination, and life after life. 

But\marginnote{10.1} those who completely understand sensual pleasures \\
live fearing nothing from any quarter. \\
They are those in the world who’ve crossed over, \\
having reached the ending of defilements.” 

%
\end{verse}

%
\section*{{\suttatitleacronym AN 5.56}{\suttatitletranslation Mentor }{\suttatitleroot Upajjhāyasutta}}
\addcontentsline{toc}{section}{\tocacronym{AN 5.56} \toctranslation{Mentor } \tocroot{Upajjhāyasutta}}
\markboth{Mentor }{Upajjhāyasutta}
\extramarks{AN 5.56}{AN 5.56}

Then\marginnote{1.1} a mendicant went up to his own mentor, and said, “Now, sir, my body feels like it’s drugged. I’m disorientated, the teachings don’t spring to mind, and dullness and drowsiness fill my mind. I lead the spiritual life dissatisfied, and have doubts about the teachings.” 

Then\marginnote{2.1} that mendicant took his pupil to the Buddha, bowed, sat down to one side, and said to him, “Sir, this mendicant says this: ‘Now, sir, my body feels like it’s drugged. I’m disorientated, the teachings don’t spring to mind, and dullness and drowsiness fill my mind. I lead the spiritual life dissatisfied, and have doubts about the teachings.’” 

“That’s\marginnote{3.1} how it is, mendicant, when your sense doors are unguarded, you eat too much, you’re not dedicated to wakefulness, you’re unable to discern skillful qualities, and you don’t pursue the development of the qualities that lead to awakening in the evening and toward dawn. Your body feels like it’s drugged. You’re disorientated, the teachings don’t spring to mind, and dullness and drowsiness fill your mind. You lead the spiritual life dissatisfied, and have doubts about the teachings. 

So\marginnote{3.2} you should train like this: ‘I will guard my sense doors, eat in moderation, be dedicated to wakefulness, discern skillful qualities, and pursue the development of the qualities that lead to awakening in the evening and toward dawn.’ That’s how you should train.” 

When\marginnote{4.1} that mendicant had been given this advice by the Buddha, he got up from his seat, bowed, and respectfully circled the Buddha, keeping him on his right, before leaving. 

Then\marginnote{4.2} that mendicant, living alone, withdrawn, diligent, keen, and resolute, soon realized the supreme culmination of the spiritual path in this very life. He lived having achieved with his own insight the goal for which gentlemen rightly go forth from the lay life to homelessness. 

He\marginnote{4.3} understood: “Rebirth is ended; the spiritual journey has been completed; what had to be done has been done; there is no return to any state of existence.” And that mendicant became one of the perfected. 

When\marginnote{5.1} that mendicant had attained perfection, he went up to his own mentor, and said, “Now, sir, my body doesn’t feel like it’s drugged. I’m not disorientated, the teachings spring to mind, and dullness and drowsiness don’t fill my mind. I lead the spiritual life satisfied, and have no doubts about the teachings.” 

Then\marginnote{5.3} that mendicant took his pupil to the Buddha, bowed, sat down to one side, and said to him, “Sir, this mendicant says this: ‘Now, sir, my body doesn’t feel like it’s drugged. I’m not disorientated, the teachings spring to mind, and dullness and drowsiness don’t fill my mind. I lead the spiritual life satisfied, and have no doubts about the teachings.’” 

“That’s\marginnote{6.1} how it is, mendicant, when your sense doors are guarded, you’re moderate in eating, you’re dedicated to wakefulness, you’re able to discern skillful qualities, and you pursue the development of the qualities that lead to awakening in the evening and toward dawn. Your body doesn’t feel like it’s drugged. You’re not disorientated, the teachings spring to mind, and dullness and drowsiness don’t fill your mind. You lead the spiritual life satisfied, and have no doubts about the teachings. 

So\marginnote{6.2} you should train like this: ‘We will guard our sense doors, eat in moderation, be dedicated to wakefulness, discern skillful qualities, and pursue the development of the qualities that lead to awakening in the evening and toward dawn.’ That’s how you should train.” 

%
\section*{{\suttatitleacronym AN 5.57}{\suttatitletranslation Subjects for Regular Reviewing }{\suttatitleroot Abhiṇhapaccavekkhitabbaṭhānasutta}}
\addcontentsline{toc}{section}{\tocacronym{AN 5.57} \toctranslation{Subjects for Regular Reviewing } \tocroot{Abhiṇhapaccavekkhitabbaṭhānasutta}}
\markboth{Subjects for Regular Reviewing }{Abhiṇhapaccavekkhitabbaṭhānasutta}
\extramarks{AN 5.57}{AN 5.57}

“Mendicants,\marginnote{1.1} a woman or a man, a layperson or a renunciate should often review these five subjects. What five? 

‘I\marginnote{1.3} am liable to grow old, I am not exempt from old age.’ A woman or a man, a layperson or a renunciate should often review this. 

‘I\marginnote{1.4} am liable to get sick, I am not exempt from sickness.’ … 

‘I\marginnote{1.5} am liable to die, I am not exempt from death.’ … 

‘I\marginnote{1.6} must be parted and separated from all I hold dear and beloved.’ … 

‘I\marginnote{1.7} am the owner of my deeds and heir to my deeds. Deeds are my womb, my relative, and my refuge. 

I\marginnote{1.8} shall be the heir of whatever deeds I do, whether good or bad.’ A woman or a man, a layperson or a renunciate should often review this. 

What\marginnote{2.1} is the advantage for a woman or a man, a layperson or a renunciate of often reviewing this: ‘I am liable to grow old, I am not exempt from old age’? There are sentient beings who, intoxicated with the vanity of youth, do bad things by way of body, speech, and mind. Reviewing this subject often, they entirely give up the vanity of youth, or at least reduce it. This is the advantage for a woman or a man, a layperson or a renunciate of often reviewing this: ‘I am liable to grow old, I am not exempt from old age’. 

What\marginnote{3.1} is the advantage of often reviewing this: ‘I am liable to get sick, I am not exempt from sickness’? There are sentient beings who, drunk on the vanity of health, do bad things by way of body, speech, and mind. Reviewing this subject often, they entirely give up the vanity of health, or at least reduce it. This is the advantage of often reviewing this: ‘I am liable to get sick, I am not exempt from sickness’. 

What\marginnote{4.1} is the advantage of often reviewing this: ‘I am liable to die, I am not exempt from death’? There are sentient beings who, drunk on the vanity of life, do bad things by way of body, speech, and mind. Reviewing this subject often, they entirely give up the vanity of life, or at least reduce it. This is the advantage of often reviewing this: ‘I am liable to die, I am not exempt from death’. 

What\marginnote{5.1} is the advantage of often reviewing this: ‘I must be parted and separated from all I hold dear and beloved’? There are sentient beings who, aroused by desire and lust for their dear and beloved, do bad things by way of body, speech, and mind. Reviewing this subject often, they entirely give up desire and lust for their dear and beloved, or at least reduce it. This is the advantage of often reviewing this: ‘I must be parted and separated from all I hold dear and beloved’. 

What\marginnote{6.1} is the advantage of often reviewing like this: ‘I am the owner of my deeds and heir to my deeds. Deeds are my womb, my relative, and my refuge. I shall be the heir of whatever deeds I do, whether good or bad’? There are sentient beings who do bad things by way of body, speech, and mind. Reviewing this subject often, they entirely give up bad conduct, or at least reduce it. This is the advantage for a woman or a man, a layperson or a renunciate of often reviewing like this: ‘I am the owner of my deeds and heir to my deeds. Deeds are my womb, my relative, and my refuge. I shall be the heir of whatever deeds I do, whether good or bad.’ 

Then\marginnote{7.1} that noble disciple reflects: ‘It’s not just me who is liable to grow old, not being exempt from old age. For all sentient beings grow old according to their nature, as long as they come and go, pass away and are reborn.’ When they review this subject often, the path is born in them. They cultivate, develop, and make much of it. By doing so, they give up the fetters and eliminate the underlying tendencies. 

‘It’s\marginnote{8.1} not just me who is liable to get sick, not being exempt from sickness. For all sentient beings get sick according to their nature, as long as they come and go, pass away and are reborn.’ When they review this subject often, the path is born in them. They cultivate, develop, and make much of it. By doing so, they give up the fetters and eliminate the underlying tendencies. 

‘It’s\marginnote{9.1} not just me who is liable to die, not being exempt from death. For all sentient beings die according to their nature, as long as they come and go, pass away and are reborn.’ When they review this subject often, the path is born in them. They cultivate, develop, and make much of it. By doing so, they give up the fetters and eliminate the underlying tendencies. 

‘It’s\marginnote{10.1} not just me who must be parted and separated from all I hold dear and beloved. For all sentient beings must be parted and separated from all they hold dear and beloved, as long as they come and go, pass away and are reborn.’ When they review this subject often, the path is born in them. They cultivate, develop, and make much of it. By doing so, they give up the fetters and eliminate the underlying tendencies. 

‘It’s\marginnote{11.1} not just me who shall be the owner of my deeds and heir to my deeds. For all sentient beings shall be the owners of their deeds and heirs to their deeds, as long as they come and go, pass away and are reborn.’ When they review this subject often, the path is born in them. They cultivate, develop, and make much of it. By doing so, they give up the fetters and eliminate the underlying tendencies. 

\begin{verse}%
For\marginnote{12.1} others, sickness is natural, \\
and so are old age and death. \\
Though this is how their nature is, \\
ordinary people feel disgusted. 

If\marginnote{13.1} I were to be disgusted \\
with creatures whose nature is such, \\
it would not be appropriate for me, \\
since my life is just the same. 

Living\marginnote{14.1} in such a way, \\
I understood the reality without attachments. \\
I mastered all vanities—\\
of health, of youth, 

and\marginnote{15.1} even of life—\\
seeing renunciation as sanctuary. \\
Zeal sprang up in me \\
as I looked to extinguishment. 

Now\marginnote{16.1} I’m unable \\
to indulge in sensual pleasures; \\
there’s no turning back, \\
I’m committed to the spiritual life.” 

%
\end{verse}

%
\section*{{\suttatitleacronym AN 5.58}{\suttatitletranslation The Licchavi Youths }{\suttatitleroot Licchavikumārakasutta}}
\addcontentsline{toc}{section}{\tocacronym{AN 5.58} \toctranslation{The Licchavi Youths } \tocroot{Licchavikumārakasutta}}
\markboth{The Licchavi Youths }{Licchavikumārakasutta}
\extramarks{AN 5.58}{AN 5.58}

At\marginnote{1.1} one time the Buddha was staying near \textsanskrit{Vesālī}, at the Great Wood, in the hall with the peaked roof. 

Then\marginnote{1.2} the Buddha robed up in the morning and, taking his bowl and robe, entered \textsanskrit{Vesālī} for alms. Then after the meal, on his return from almsround, he plunged deep into the Great Wood and sat at the root of a tree for the day’s meditation. 

Now\marginnote{2.1} at that time several Licchavi youths took strung bows and, escorted by a pack of hounds, were going for a walk in the Great Wood when they saw the Buddha seated at the root of a tree. When they saw him, they put down their strung bows, tied their hounds up to one side, and went up to him. They bowed and silently paid homage to the Buddha with joined palms. 

Now\marginnote{3.1} at that time \textsanskrit{Mahānāma} the Licchavi was going for a walk in the Great Wood when he saw those Licchavi youths silently paying homage to the Buddha with joined palms. Seeing this, he went up to the Buddha, bowed, sat down to one side, and expressed this heartfelt sentiment, “The Vajjis will grow up! The Vajjis will grow up!” 

“But\marginnote{4.1} \textsanskrit{Mahānāma}, why do you say that the Vajjis will grow up?” 

“Sir,\marginnote{4.3} these Licchavi youths are violent, harsh, and brash. Whenever sweets are left out for families—sugar-cane, jujube fruits, pancakes, pies, or fritters—they filch them and eat them up. And they hit women and girls of good families on their backs. But now they’re silently paying homage to the Buddha with joined palms.” 

“\textsanskrit{Mahānāma},\marginnote{5.1} you can expect only growth, not decline, when you find five qualities in any gentleman—whether he’s an anointed aristocratic king, an appointed or hereditary official, an army general, a village chief, a guild chief, or a ruler of his own clan. 

What\marginnote{6.1} five? 

Firstly,\marginnote{6.2} a gentleman uses his legitimate wealth—earned by his efforts and initiative, built up with his own hands, gathered by the sweat of the brow—to honor, respect, esteem, and venerate his mother and father. Honored in this way, his mother and father love him with a good heart, wishing: ‘Live long! Stay alive for a long time!’ When a gentleman is loved by his mother and father, you can expect only growth, not decline. 

Furthermore,\marginnote{7.1} a gentleman uses his legitimate wealth to honor, respect, esteem, and venerate his wives and children, bondservants, workers, and staff. Honored in this way, his wives and children, bondservants, workers, and staff love him with a good heart, wishing: ‘Live long! Stay alive for a long time!’ When a gentleman is loved by his wives and children, bondservants, workers, and staff, you can expect only growth, not decline. 

Furthermore,\marginnote{8.1} a gentleman uses his legitimate wealth to honor, respect, esteem, and venerate those who work the neighboring fields, and those he does business with. Honored in this way, those who work the neighboring fields, and those he does business with love him with a good heart, wishing: ‘Live long! Stay alive for a long time!’ When a gentleman is loved by those who work the neighboring fields, and those he does business with, you can expect only growth, not decline. 

Furthermore,\marginnote{9.1} a gentleman uses his legitimate wealth to honor, respect, esteem, and venerate the deities who receive spirit-offerings. Honored in this way, the deities who receive spirit-offerings love him with a good heart, wishing: ‘Live long! Stay alive for a long time!’ When a gentleman is loved by the deities, you can expect only growth, not decline. 

Furthermore,\marginnote{10.1} a gentleman uses his legitimate wealth to honor, respect, esteem, and venerate ascetics and brahmins. Honored in this way, ascetics and brahmins love him with a good heart, wishing: ‘Live long! Stay alive for a long time!’ When a gentleman is loved by ascetics and brahmins, you can expect only growth, not decline. 

You\marginnote{11.1} can expect only growth, not decline, when you find these five qualities in any gentleman—whether he’s an anointed aristocratic king, an appointed or hereditary official, an army general, a village chief, a guild chief, or a ruler of his own clan. 

\begin{verse}%
He’s\marginnote{12.1} always dutiful to his mother and father, \\
and for the good of his wives and children. \\
He looks after those in his household, \\
and those dependent on him for their livelihood. 

A\marginnote{13.1} kind and ethical person \\
looks after the welfare of relatives—\\
both those who have passed away, \\
and those alive at present. 

While\marginnote{14.1} living at home, an astute person \\
uses legitimate means to give rise to joy \\
for ascetics, brahmins, \\
and also the gods. 

Having\marginnote{15.1} done good, \\
he’s venerable and praiseworthy. \\
They praise him in this life, \\
and he departs to rejoice in heaven.” 

%
\end{verse}

%
\section*{{\suttatitleacronym AN 5.59}{\suttatitletranslation Gone Forth When Old (1st) }{\suttatitleroot Paṭhamavuḍḍhapabbajitasutta}}
\addcontentsline{toc}{section}{\tocacronym{AN 5.59} \toctranslation{Gone Forth When Old (1st) } \tocroot{Paṭhamavuḍḍhapabbajitasutta}}
\markboth{Gone Forth When Old (1st) }{Paṭhamavuḍḍhapabbajitasutta}
\extramarks{AN 5.59}{AN 5.59}

“Mendicants,\marginnote{1.1} it’s hard to find someone gone forth when old who has five qualities. What five? It’s hard to find someone gone forth when old who is sophisticated, well-presented, and learned, who can teach Dhamma, and has memorized the monastic law. It’s hard to find someone gone forth when old who has these five qualities.” 

%
\section*{{\suttatitleacronym AN 5.60}{\suttatitletranslation Gone Forth When Old (2nd) }{\suttatitleroot Dutiyavuḍḍhapabbajitasutta}}
\addcontentsline{toc}{section}{\tocacronym{AN 5.60} \toctranslation{Gone Forth When Old (2nd) } \tocroot{Dutiyavuḍḍhapabbajitasutta}}
\markboth{Gone Forth When Old (2nd) }{Dutiyavuḍḍhapabbajitasutta}
\extramarks{AN 5.60}{AN 5.60}

“Mendicants,\marginnote{1.1} it’s hard to find someone gone forth when old who has five qualities. What five? It’s hard to find someone gone forth when old who is easy to admonish, retains what they learn, and learns respectfully, who can teach the Dhamma, and has memorized the monastic law. It’s hard to find someone gone forth when old who has these five qualities.” 

%
\addtocontents{toc}{\let\protect\contentsline\protect\nopagecontentsline}
\chapter*{The Chapter on Perceptions }
\addcontentsline{toc}{chapter}{\tocchapterline{The Chapter on Perceptions }}
\addtocontents{toc}{\let\protect\contentsline\protect\oldcontentsline}

%
\section*{{\suttatitleacronym AN 5.61}{\suttatitletranslation Perceptions (1st) }{\suttatitleroot Paṭhamasaññāsutta}}
\addcontentsline{toc}{section}{\tocacronym{AN 5.61} \toctranslation{Perceptions (1st) } \tocroot{Paṭhamasaññāsutta}}
\markboth{Perceptions (1st) }{Paṭhamasaññāsutta}
\extramarks{AN 5.61}{AN 5.61}

“Mendicants,\marginnote{1.1} these five perceptions, when developed and cultivated, are very fruitful and beneficial. They culminate in the deathless and end with the deathless. What five? The perceptions of ugliness, death, drawbacks, repulsiveness of food, and dissatisfaction with the whole world. These five perceptions, when developed and cultivated, are very fruitful and beneficial. They culminate in the deathless and end with the deathless.” 

%
\section*{{\suttatitleacronym AN 5.62}{\suttatitletranslation Perceptions (2nd) }{\suttatitleroot Dutiyasaññāsutta}}
\addcontentsline{toc}{section}{\tocacronym{AN 5.62} \toctranslation{Perceptions (2nd) } \tocroot{Dutiyasaññāsutta}}
\markboth{Perceptions (2nd) }{Dutiyasaññāsutta}
\extramarks{AN 5.62}{AN 5.62}

“Mendicants,\marginnote{1.1} these five perceptions, when developed and cultivated, are very fruitful and beneficial. They culminate in the deathless and end with the deathless. What five? The perceptions of impermanence, not-self, death, repulsiveness of food, and dissatisfaction with the whole world. These five perceptions, when developed and cultivated, are very fruitful and beneficial. They culminate in the deathless and end with the deathless.” 

%
\section*{{\suttatitleacronym AN 5.63}{\suttatitletranslation Growth (1st) }{\suttatitleroot Paṭhamavaḍḍhisutta}}
\addcontentsline{toc}{section}{\tocacronym{AN 5.63} \toctranslation{Growth (1st) } \tocroot{Paṭhamavaḍḍhisutta}}
\markboth{Growth (1st) }{Paṭhamavaḍḍhisutta}
\extramarks{AN 5.63}{AN 5.63}

“Mendicants,\marginnote{1.1} a male noble disciple who grows in five ways grows nobly, taking on what is essential and excellent in this life. What five? He grows in faith, ethics, learning, generosity, and wisdom. A male noble disciple who grows in these five ways grows nobly, taking on what is essential and excellent in this life. 

\begin{verse}%
He\marginnote{2.1} who grows in faith and ethics, \\
wisdom, and both generosity and learning—\\
a good man such as he sees clearly, \\
and takes on what is essential for himself in this life.” 

%
\end{verse}

%
\section*{{\suttatitleacronym AN 5.64}{\suttatitletranslation Growth (2nd) }{\suttatitleroot Dutiyavaḍḍhisutta}}
\addcontentsline{toc}{section}{\tocacronym{AN 5.64} \toctranslation{Growth (2nd) } \tocroot{Dutiyavaḍḍhisutta}}
\markboth{Growth (2nd) }{Dutiyavaḍḍhisutta}
\extramarks{AN 5.64}{AN 5.64}

“Mendicants,\marginnote{1.1} a female noble disciple who grows in five ways grows nobly, taking on what is essential and excellent in this life. What five? She grows in faith, ethics, learning, generosity, and wisdom. A female noble disciple who grows in these five ways grows nobly, taking on what is essential and excellent in this life. 

\begin{verse}%
She\marginnote{2.1} who grows in faith and ethics, \\
wisdom, and both generosity and learning—\\
a virtuous laywoman such as she \\
takes on what is essential for herself in this life.” 

%
\end{verse}

%
\section*{{\suttatitleacronym AN 5.65}{\suttatitletranslation Discussion }{\suttatitleroot Sākacchasutta}}
\addcontentsline{toc}{section}{\tocacronym{AN 5.65} \toctranslation{Discussion } \tocroot{Sākacchasutta}}
\markboth{Discussion }{Sākacchasutta}
\extramarks{AN 5.65}{AN 5.65}

“Mendicants,\marginnote{1.1} a mendicant with five qualities is fit to hold a discussion with their spiritual companions. What five? 

A\marginnote{1.3} mendicant is personally accomplished in ethics, and answers questions that come up when discussing accomplishment in ethics. 

They’re\marginnote{1.4} personally accomplished in immersion, and they answer questions that come up when discussing accomplishment in immersion. 

They’re\marginnote{1.5} personally accomplished in wisdom, and they answer questions that come up when discussing accomplishment in wisdom. 

They’re\marginnote{1.6} personally accomplished in freedom, and they answer questions that come up when discussing accomplishment in freedom. 

They’re\marginnote{1.7} personally accomplished in the knowledge and vision of freedom, and they answer questions that come up when discussing accomplishment in the knowledge and vision of freedom. 

A\marginnote{1.8} mendicant with these five qualities is fit to hold a discussion with their spiritual companions.” 

%
\section*{{\suttatitleacronym AN 5.66}{\suttatitletranslation Sharing Life }{\suttatitleroot Sājīvasutta}}
\addcontentsline{toc}{section}{\tocacronym{AN 5.66} \toctranslation{Sharing Life } \tocroot{Sājīvasutta}}
\markboth{Sharing Life }{Sājīvasutta}
\extramarks{AN 5.66}{AN 5.66}

“Mendicants,\marginnote{1.1} a mendicant with five qualities is fit to share their life with their spiritual companions. What five? 

A\marginnote{1.3} mendicant is personally accomplished in ethics, and answers questions posed when discussing accomplishment in ethics. 

They’re\marginnote{1.4} personally accomplished in immersion, and they answer questions posed when discussing accomplishment in immersion. 

They’re\marginnote{1.5} personally accomplished in wisdom, and they answer questions posed when discussing accomplishment in wisdom. 

They’re\marginnote{1.6} personally accomplished in freedom, and they answer questions posed when discussing accomplishment in freedom. 

They’re\marginnote{1.7} personally accomplished in the knowledge and vision of freedom, and they answer questions posed when discussing accomplishment in the knowledge and vision of freedom. 

A\marginnote{1.8} mendicant with these five qualities is fit to share their life with their spiritual companions.” 

%
\section*{{\suttatitleacronym AN 5.67}{\suttatitletranslation Bases of Psychic Power (1st) }{\suttatitleroot Paṭhamaiddhipādasutta}}
\addcontentsline{toc}{section}{\tocacronym{AN 5.67} \toctranslation{Bases of Psychic Power (1st) } \tocroot{Paṭhamaiddhipādasutta}}
\markboth{Bases of Psychic Power (1st) }{Paṭhamaiddhipādasutta}
\extramarks{AN 5.67}{AN 5.67}

“Mendicants,\marginnote{1.1} any monk or nun who develops and cultivates five qualities can expect one of two results: enlightenment in the present life, or if there’s something left over, non-return. 

What\marginnote{2.1} five? 

A\marginnote{2.2} mendicant develops the basis of psychic power that has immersion due to enthusiasm, and active effort … 

A\marginnote{2.3} mendicant develops the basis of psychic power that has immersion due to energy, and active effort … 

A\marginnote{2.4} mendicant develops the basis of psychic power that has immersion due to mental development, and active effort … 

A\marginnote{2.5} mendicant develops the basis of psychic power that has immersion due to inquiry, and active effort. 

And\marginnote{2.6} the fifth is sheer vigor. 

Any\marginnote{2.7} monk or nun who develops and cultivates these five qualities can expect one of two results: enlightenment in the present life, or if there’s something left over, non-return.” 

%
\section*{{\suttatitleacronym AN 5.68}{\suttatitletranslation Bases of Psychic Power (2nd) }{\suttatitleroot Dutiyaiddhipādasutta}}
\addcontentsline{toc}{section}{\tocacronym{AN 5.68} \toctranslation{Bases of Psychic Power (2nd) } \tocroot{Dutiyaiddhipādasutta}}
\markboth{Bases of Psychic Power (2nd) }{Dutiyaiddhipādasutta}
\extramarks{AN 5.68}{AN 5.68}

“Mendicants,\marginnote{1.1} before my awakening—when I was still not awake but intent on awakening—I developed and cultivated five things. What five? 

The\marginnote{1.3} basis of psychic power that has immersion due to enthusiasm, and active effort … the basis of psychic power that has immersion due to energy, and active effort … the basis of psychic power that has immersion due to mental development, and active effort … the basis of psychic power that has immersion due to inquiry, and active effort. And the fifth is sheer vigor. 

When\marginnote{1.8} I had developed and cultivated these five things, with vigor as fifth, I became capable of realizing anything that can be realized by insight to which I extended the mind, in each and every case. 

If\marginnote{2.1} I wished: ‘May I multiply myself and become one again … controlling the body as far as the \textsanskrit{Brahmā} realm.’ I was capable of realizing it, in each and every case. 

If\marginnote{3.1} I wished: … ‘May I realize the undefiled freedom of heart and freedom by wisdom in this very life, and live having realized it with my own insight due to the ending of defilements.’ I was capable of realizing it, in each and every case.” 

%
\section*{{\suttatitleacronym AN 5.69}{\suttatitletranslation Disillusionment }{\suttatitleroot Nibbidāsutta}}
\addcontentsline{toc}{section}{\tocacronym{AN 5.69} \toctranslation{Disillusionment } \tocroot{Nibbidāsutta}}
\markboth{Disillusionment }{Nibbidāsutta}
\extramarks{AN 5.69}{AN 5.69}

“Mendicants,\marginnote{1.1} these five things, when developed and cultivated, lead solely to disillusionment, dispassion, cessation, peace, insight, awakening, and extinguishment. 

What\marginnote{2.1} five? A mendicant meditates observing the ugliness of the body, perceives the repulsiveness of food, perceives dissatisfaction with the whole world, observes the impermanence of all conditions, and has well established the perception of their own death. 

These\marginnote{2.3} five things, when developed and cultivated, lead solely to disillusionment, dispassion, cessation, peace, insight, awakening, and extinguishment.” 

%
\section*{{\suttatitleacronym AN 5.70}{\suttatitletranslation The Ending of Defilements }{\suttatitleroot Āsavakkhayasutta}}
\addcontentsline{toc}{section}{\tocacronym{AN 5.70} \toctranslation{The Ending of Defilements } \tocroot{Āsavakkhayasutta}}
\markboth{The Ending of Defilements }{Āsavakkhayasutta}
\extramarks{AN 5.70}{AN 5.70}

“Mendicants,\marginnote{1.1} these five things, when developed and cultivated, lead to the ending of defilements. What five? A mendicant meditates observing the ugliness of the body, perceives the repulsiveness of food, perceives dissatisfaction with the whole world, observes the impermanence of all conditions, and has well established the perception of their own death. These five things, when developed and cultivated, lead to the ending of defilements.” 

%
\addtocontents{toc}{\let\protect\contentsline\protect\nopagecontentsline}
\chapter*{The Chapter on a Warrior }
\addcontentsline{toc}{chapter}{\tocchapterline{The Chapter on a Warrior }}
\addtocontents{toc}{\let\protect\contentsline\protect\oldcontentsline}

%
\section*{{\suttatitleacronym AN 5.71}{\suttatitletranslation Freedom of Heart is the Fruit (1st) }{\suttatitleroot Paṭhamacetovimuttiphalasutta}}
\addcontentsline{toc}{section}{\tocacronym{AN 5.71} \toctranslation{Freedom of Heart is the Fruit (1st) } \tocroot{Paṭhamacetovimuttiphalasutta}}
\markboth{Freedom of Heart is the Fruit (1st) }{Paṭhamacetovimuttiphalasutta}
\extramarks{AN 5.71}{AN 5.71}

“Mendicants,\marginnote{1.1} these five things, when developed and cultivated, have freedom of heart and freedom by wisdom as their fruit and benefit. 

What\marginnote{2.1} five? A mendicant meditates observing the ugliness of the body, perceives the repulsiveness of food, perceives dissatisfaction with the whole world, observes the impermanence of all conditions, and has well established the perception of their own death. These five things, when developed and cultivated, have freedom of heart and freedom by wisdom as their fruit and benefit. When a mendicant has freedom of heart and freedom by wisdom, they’re called a mendicant who has lifted up the cross-bar, filled in the trench, and pulled up the pillar; they’re unbarred, a noble one with banner and burden put down, detached. 

And\marginnote{3.1} how has a mendicant lifted the cross-bar? It’s when a mendicant has given up ignorance, cut it off at the root, made it like a palm stump, obliterated it, so it’s unable to arise in the future. That’s how a mendicant has lifted the cross-bar. 

And\marginnote{4.1} how has a mendicant filled in the trench? It’s when a mendicant has given up transmigrating through births in future lives, cut it off at the root, made it like a palm stump, obliterated it, so it’s unable to arise in the future. That’s how a mendicant has filled in the trench. 

And\marginnote{5.1} how has a mendicant pulled up the pillar? It’s when a mendicant has given up craving, cut it off at the root, made it like a palm stump, obliterated it, so it’s unable to arise in the future. That’s how a mendicant has pulled up the pillar. 

And\marginnote{6.1} how is a mendicant unbarred? It’s when a mendicant has given up the five lower fetters, cut them off at the root, made them like a palm stump, obliterated them, so they’re unable to arise in the future. That’s how a mendicant is unbarred. 

And\marginnote{7.1} how is a mendicant a noble one with banner and burden put down, detached? It’s when a mendicant has given up the conceit ‘I am’, cut it off at the root, made it like a palm stump, obliterated it, so it’s unable to arise in the future. That’s how a mendicant is a noble one with banner and burden put down, detached.” 

%
\section*{{\suttatitleacronym AN 5.72}{\suttatitletranslation Freedom of Heart is the Fruit (2nd) }{\suttatitleroot Dutiyacetovimuttiphalasutta}}
\addcontentsline{toc}{section}{\tocacronym{AN 5.72} \toctranslation{Freedom of Heart is the Fruit (2nd) } \tocroot{Dutiyacetovimuttiphalasutta}}
\markboth{Freedom of Heart is the Fruit (2nd) }{Dutiyacetovimuttiphalasutta}
\extramarks{AN 5.72}{AN 5.72}

“Mendicants,\marginnote{1.1} these five things, when developed and cultivated, have freedom of heart and freedom by wisdom as their fruit and benefit. What five? 

The\marginnote{1.3} perception of impermanence, the perception of suffering in impermanence, the perception of not-self in suffering, the perception of giving up, and the perception of fading away. 

These\marginnote{1.4} five things, when developed and cultivated, have freedom of heart and freedom by wisdom as their fruit and benefit. 

When\marginnote{1.5} a mendicant has freedom of heart and freedom by wisdom, they’re called a mendicant who has lifted up the cross-bar, filled in the trench, and pulled up the pillar; they’re unbarred, a noble one with banner and burden put down, detached. …” 

%
\section*{{\suttatitleacronym AN 5.73}{\suttatitletranslation One Who Lives by the Teaching (1st) }{\suttatitleroot Paṭhamadhammavihārīsutta}}
\addcontentsline{toc}{section}{\tocacronym{AN 5.73} \toctranslation{One Who Lives by the Teaching (1st) } \tocroot{Paṭhamadhammavihārīsutta}}
\markboth{One Who Lives by the Teaching (1st) }{Paṭhamadhammavihārīsutta}
\extramarks{AN 5.73}{AN 5.73}

Then\marginnote{1.1} a mendicant went up to the Buddha, bowed, sat down to one side, and said to him: 

“Sir,\marginnote{1.2} they speak of ‘one who lives by the teaching’. How is one who lives by the teaching defined?” 

“Mendicant,\marginnote{2.1} take a mendicant who memorizes the teaching—statements, songs, discussions, verses, inspired exclamations, legends, stories of past lives, amazing stories, and classifications. They spend their days studying that teaching. But they neglect retreat, and are not committed to internal serenity of heart. That mendicant is called one who studies a lot, not one who lives by the teaching. 

Furthermore,\marginnote{3.1} a mendicant teaches Dhamma in detail to others as they learned and memorized it. They spend their days advocating that teaching. But they neglect retreat, and are not committed to internal serenity of heart. That mendicant is called one who advocates a lot, not one who lives by the teaching. 

Furthermore,\marginnote{4.1} a mendicant recites the teaching in detail as they learned and memorized it. They spend their days reciting that teaching. But they neglect retreat, and are not committed to internal serenity of heart. That mendicant is called one who recites a lot, not one who lives by the teaching. 

Furthermore,\marginnote{5.1} a mendicant thinks about and considers the teaching in their heart, examining it with the mind as they learned and memorized it. They spend their days thinking about that teaching. But they neglect retreat, and are not committed to internal serenity of heart. That mendicant is called one who thinks a lot, not one who lives by the teaching. 

Take\marginnote{6.1} a mendicant who memorizes the teaching—statements, songs, discussions, verses, inspired exclamations, legends, stories of past lives, amazing stories, and classifications. They don’t spend their days studying that teaching. They don’t neglect retreat, and they’re committed to internal serenity of heart. That’s how a mendicant is one who lives by the teaching. 

So,\marginnote{7.1} mendicant, I’ve taught you the one who studies a lot, the one who advocates a lot, the one who recites a lot, the one who thinks a lot, and the one who lives by the teaching. Out of compassion, I’ve done what a teacher should do who wants what’s best for their disciples. Here are these roots of trees, and here are these empty huts. Practice absorption, mendicant! Don’t be negligent! Don’t regret it later! This is my instruction to you.” 

%
\section*{{\suttatitleacronym AN 5.74}{\suttatitletranslation One Who Lives by the Teaching (2nd) }{\suttatitleroot Dutiyadhammavihārīsutta}}
\addcontentsline{toc}{section}{\tocacronym{AN 5.74} \toctranslation{One Who Lives by the Teaching (2nd) } \tocroot{Dutiyadhammavihārīsutta}}
\markboth{One Who Lives by the Teaching (2nd) }{Dutiyadhammavihārīsutta}
\extramarks{AN 5.74}{AN 5.74}

Then\marginnote{1.1} a mendicant went up to the Buddha, bowed, sat down to one side, and said to him: 

“Sir,\marginnote{1.2} they speak of ‘one who lives by the teaching’. How is one who lives by the teaching defined?” 

“Mendicant,\marginnote{2.1} take a mendicant who memorizes the teaching—statements, songs, discussions, verses, inspired exclamations, legends, stories of past lives, amazing stories, and classifications. But they don’t understand the higher meaning. That mendicant is called one who studies a lot, not one who lives by the teaching. 

Furthermore,\marginnote{3.1} a mendicant teaches Dhamma in detail to others as they learned and memorized it. But they don’t understand the higher meaning. That mendicant is called one who advocates a lot, not one who lives by the teaching. 

Furthermore,\marginnote{4.1} a mendicant recites the teaching in detail as they learned and memorized it. But they don’t understand the higher meaning. That mendicant is called one who recites a lot, not one who lives by the teaching. 

Furthermore,\marginnote{5.1} a mendicant thinks about and considers the teaching in their heart, examining it with the mind as they learned and memorized it. But they don’t understand the higher meaning. That mendicant is called one who thinks a lot, not one who lives by the teaching. 

Take\marginnote{6.1} a mendicant who memorizes the teaching—statements, songs, discussions, verses, inspired exclamations, legends, stories of past lives, amazing stories, and classifications. And they do understand the higher meaning. That’s how a mendicant is one who lives by the teaching. 

So,\marginnote{7.1} mendicant, I’ve taught you the one who studies a lot, the one who advocates a lot, the one who recites a lot, the one who thinks a lot, and the one who lives by the teaching. Out of compassion, I’ve done what a teacher should do who wants what’s best for their disciples. Here are these roots of trees, and here are these empty huts. Practice absorption, mendicant! Don’t be negligent! Don’t regret it later! This is my instruction to you.” 

%
\section*{{\suttatitleacronym AN 5.75}{\suttatitletranslation Warriors (1st) }{\suttatitleroot Paṭhamayodhājīvasutta}}
\addcontentsline{toc}{section}{\tocacronym{AN 5.75} \toctranslation{Warriors (1st) } \tocroot{Paṭhamayodhājīvasutta}}
\markboth{Warriors (1st) }{Paṭhamayodhājīvasutta}
\extramarks{AN 5.75}{AN 5.75}

“Mendicants,\marginnote{1.1} these five warriors are found in the world. What five? 

Firstly,\marginnote{1.3} one warrior falters and founders at the mere sight of a cloud of dust. He doesn’t stay firm, and fails to plunge into battle. Some warriors are like that. This is the first warrior found in the world. 

Furthermore,\marginnote{2.1} one warrior can prevail over a cloud of dust, but he falters and founders at the mere sight of a banner’s crest. He doesn’t stay firm, and fails to plunge into battle. Some warriors are like that. This is the second warrior found in the world. 

Furthermore,\marginnote{3.1} one warrior can prevail over a cloud of dust and a banner’s crest, but he falters and founders at the mere sound of turmoil. He doesn’t stay firm, and fails to plunge into battle. Some warriors are like that. This is the third warrior found in the world. 

Furthermore,\marginnote{4.1} one warrior can prevail over a cloud of dust and a banner’s crest and turmoil, but he’s killed or injured when blows are struck. Some warriors are like that. This is the fourth warrior found in the world. 

Furthermore,\marginnote{5.1} one warrior can prevail over a cloud of dust and a banner’s crest and turmoil and being struck. He wins victory in battle, establishing himself as foremost in battle. Some warriors are like that. This is the fifth warrior found in the world. 

These\marginnote{5.5} are the five warriors found in the world. 

In\marginnote{6.1} the same way, these five people similar to warriors are found among the monks. What five? 

Firstly,\marginnote{6.3} one monk falters and founders at the mere sight of a cloud of dust. He doesn’t stay firm, and fails to keep up the spiritual life. Declaring his inability to continue training, he rejects it and returns to a lesser life. What is his ‘cloud of dust’? It’s when a monk hears: ‘In such and such a village or town there’s a women or a girl who is attractive, good-looking, lovely, of surpassing beauty.’ Hearing this, he falters and founders. He doesn’t stay firm, and fails to keep up the spiritual life. Declaring his inability to continue training, he rejects it and returns to a lesser life. This is his ‘cloud of dust’. 

I\marginnote{7.1} say that this person is like the warrior who falters and founders at the mere sight of a cloud of dust. Some people are like that. This is the first person similar to a warrior found among the monks. 

Furthermore,\marginnote{8.1} one monk can prevail over a cloud of dust, but at the mere sight of a banner’s crest he falters and founders. He doesn’t stay firm, and fails to keep up the spiritual life. Declaring his inability to continue training, he rejects it and returns to a lesser life. What is his ‘banner’s crest’? It’s when a monk doesn’t hear: ‘In such and such a village or town there’s a women or a girl who is attractive, good-looking, lovely, of surpassing beauty.’ But he sees for himself a women or a girl who is attractive, good-looking, lovely, of surpassing beauty. Seeing her, he falters and founders. He doesn’t stay firm, and fails to keep up the spiritual life. Declaring his inability to continue training, he rejects it and returns to a lesser life. This is his ‘banner’s crest’. 

I\marginnote{9.1} say that this person is like the warrior who can prevail over a cloud of dust, but he falters and founders at the mere sight of a banner’s crest. Some people are like that. This is the second person similar to a warrior found among the monks. 

Furthermore,\marginnote{10.1} one monk can prevail over a cloud of dust and a banner’s crest, but he falters and founders at the mere sound of turmoil. He doesn’t stay firm, and fails to keep up the spiritual life. Declaring his inability to continue training, he rejects it and returns to a lesser life. What is his ‘turmoil’? It’s when a mendicant has gone to a wilderness, or to the root of a tree, or to an empty hut, when a female comes up to him. She smiles, chats, laughs, and teases him. He falters and founders. He doesn’t stay firm, and fails to keep up the spiritual life. Declaring his inability to continue training, he rejects it and returns to a lesser life. This is his ‘turmoil’. 

I\marginnote{11.1} say that this person is like the warrior who can prevail over a cloud of dust and a banner’s crest, but he falters and founders at the mere sound of turmoil. Some people are like that. This is the third person similar to a warrior found among the monks. 

Furthermore,\marginnote{12.1} one monk can prevail over a cloud of dust and a banner’s crest and turmoil, but he’s killed or injured when blows are struck. What is his ‘blows are struck’? It’s when a mendicant has gone to a wilderness, or to the root of a tree, or to an empty hut, when a female comes up to him. She sits right by him, lies down, or embraces him. Without resigning the training and declaring his inability to continue, he has sex. This is his ‘blows are struck’. 

I\marginnote{13.1} say that this person is like the warrior who can prevail over a cloud of dust and a banner’s crest and turmoil, but is killed or injured when blows are struck. Some people are like that. This is the fourth person similar to a warrior found among the monks. 

Furthermore,\marginnote{14.1} one monk can prevail over a cloud of dust and a banner’s crest and turmoil, and being struck. He wins victory in battle, establishing himself as foremost in battle. What is his ‘victory in battle’? It’s when a mendicant has gone to a wilderness, or to the root of a tree, or to an empty hut, when a female comes up to him. She sits right by him, lies down, or embraces him. But he disentangles and frees himself, and goes wherever he wants. He frequents a secluded lodging—a wilderness, the root of a tree, a hill, a ravine, a mountain cave, a charnel ground, a forest, the open air, a heap of straw. 

Gone\marginnote{15.1} to a wilderness, or to the root of a tree, or to an empty hut, he sits down cross-legged, with his body straight, and establishes his mindfulness right there. Giving up desire for the world, he meditates with a heart rid of desire, cleansing the mind of desire. Giving up ill will and malevolence, he meditates with a mind rid of ill will, full of compassion for all living beings, cleansing the mind of ill will. Giving up dullness and drowsiness, he meditates with a mind rid of dullness and drowsiness, perceiving light, mindful and aware, cleansing the mind of dullness and drowsiness. Giving up restlessness and remorse, he meditates without restlessness, his mind peaceful inside, cleansing the mind of restlessness and remorse. Giving up doubt, he meditates having gone beyond doubt, not undecided about skillful qualities, cleansing the mind of doubt. He gives up these five hindrances, corruptions of the heart that weaken wisdom. Then, quite secluded from sensual pleasures, secluded from unskillful qualities, he enters and remains in the first absorption … second absorption … third absorption … fourth absorption. 

When\marginnote{16.1} his mind has become immersed in \textsanskrit{samādhi} like this—purified, bright, flawless, rid of corruptions, pliable, workable, steady, and imperturbable—he extends it toward knowledge of the ending of defilements. He truly understands: ‘This is suffering’ … ‘This is the origin of suffering’ … ‘This is the cessation of suffering’ … ‘This is the practice that leads to the cessation of suffering’. He truly understands: ‘These are defilements’ … ‘This is the origin of defilements’ … ‘This is the cessation of defilements’ … ‘This is the practice that leads to the cessation of defilements’. Knowing and seeing like this, his mind is freed from the defilements of sensuality, desire to be reborn, and ignorance. When it is freed, he knows it is freed. 

He\marginnote{16.6} understands: ‘Rebirth is ended, the spiritual journey has been completed, what had to be done has been done, there is no return to any state of existence.’ This is his ‘victory in battle’. 

I\marginnote{17.1} say that this person is like the warrior who can prevail over a cloud of dust and a banner’s crest and turmoil and being struck. He wins victory in battle, establishing himself as foremost in battle. Some people are like that. This is the fifth person similar to a warrior found among the monks. 

These\marginnote{17.5} five people similar to warriors are found among the monks.” 

%
\section*{{\suttatitleacronym AN 5.76}{\suttatitletranslation Warriors (2nd) }{\suttatitleroot Dutiyayodhājīvasutta}}
\addcontentsline{toc}{section}{\tocacronym{AN 5.76} \toctranslation{Warriors (2nd) } \tocroot{Dutiyayodhājīvasutta}}
\markboth{Warriors (2nd) }{Dutiyayodhājīvasutta}
\extramarks{AN 5.76}{AN 5.76}

“Mendicants,\marginnote{1.1} these five warriors are found in the world. What five? 

Firstly,\marginnote{1.3} one warrior dons his sword and shield, fastens his bow and arrows, and plunges into the thick of battle. He strives and struggles in the battle, but his foes kill him and finish him off. Some warriors are like that. This is the first warrior found in the world. 

Furthermore,\marginnote{2.1} one warrior dons his sword and shield, fastens his bow and arrows, and plunges into the thick of battle. He strives and struggles in the battle, but his foes wound him. He’s carried off and taken to his relatives, but he dies on the road before he reaches them. Some warriors are like that. This is the second warrior found in the world. 

Furthermore,\marginnote{3.1} one warrior dons his sword and shield, fastens his bow and arrows, and plunges into the thick of battle. He strives and struggles in the battle, but his foes wound him. He’s carried off and taken to his relatives, who nurse him and care for him. But he dies of his injuries while in their care. Some warriors are like that. This is the third warrior found in the world. 

Furthermore,\marginnote{4.1} one warrior dons his sword and shield, fastens his bow and arrows, and plunges into the thick of battle. He strives and struggles in the battle, but his foes wound him. He’s carried off and taken to his relatives, who nurse him and care for him. And while in their care, he recovers from his injuries. Some warriors are like that. This is the fourth warrior found in the world. 

Furthermore,\marginnote{5.1} one warrior dons his sword and shield, fastens his bow and arrows, and plunges into the thick of battle. He wins victory in battle, establishing himself as foremost in battle. Some warriors are like that. This is the fifth warrior found in the world. These are the five warriors found in the world. 

In\marginnote{6.1} the same way, these five people similar to warriors are found among the monks. What five? Firstly, a mendicant lives supported by a town or village. He robes up in the morning and, taking his bowl and robe, enters a village or town for alms without guarding body, speech, and mind, without establishing mindfulness, and without restraining the sense faculties. There he sees a female scantily clad, with revealing clothes. Lust infects his mind, and, without resigning the training and declaring his inability to continue, he has sex. 

I\marginnote{7.1} say that this person is like the warrior who is killed and finished off by his foes. Some people are like that. This is the first person similar to a warrior found among the monks. 

Furthermore,\marginnote{8.1} a mendicant lives supported by a town or village. He robes up in the morning and, taking his bowl and robe, enters a village or town for alms without guarding body, speech, and mind, without establishing mindfulness, and without restraining the sense faculties. There he sees a female scantily clad, with revealing clothes. Lust infects his mind, and his body and mind burn with it. He thinks: ‘Why don’t I go to the monastery and tell the monks: 

“Reverends,\marginnote{8.8} I am overcome with lust, mired in lust. I am unable to keep up the spiritual life. I declare my inability to continue training. I reject it and will return to a lesser life.”’ But while traveling on the road, before he reaches the monastery he declares his inability to continue training. He rejects it and returns to a lesser life. 

I\marginnote{9.1} say that this person is like the warrior who is taken to his relatives for care, but he dies on the road before he reaches them. Some people are like that. This is the second person similar to a warrior found among the monks. 

Furthermore,\marginnote{10.1} a mendicant lives supported by a town or village. He robes up in the morning and, taking his bowl and robe, enters a village or town for alms without guarding body, speech, and mind, without establishing mindfulness, and without restraining the sense faculties. There he sees a female scantily clad, with revealing clothes. Lust infects his mind, and his body and mind burn with it. He thinks: ‘Why don’t I go to the monastery and tell the monks: 

“Reverends,\marginnote{10.8} I am overcome with lust, mired in lust. I am unable to keep up the spiritual life. I declare my inability to continue training. I reject it and will return to a lesser life.”’ He goes to the monastery and tells the monks: ‘Reverends, I am overcome with lust, mired in lust. I am unable to keep up the spiritual life. I declare my inability to continue training. I reject it and will return to a lesser life.’ 

His\marginnote{11.1} spiritual companions advise and instruct him: ‘Reverend, the Buddha says that sensual pleasures give little gratification and much suffering and distress, and they are all the more full of drawbacks. With the similes of a skeleton … a lump of meat … a grass torch … a pit of glowing coals … a dream … borrowed goods … fruit on a tree … a butcher’s knife and chopping block … a staking sword … a snake’s head, the Buddha says that sensual pleasures give little gratification and much suffering and distress, and they are all the more full of drawbacks. Be happy with the spiritual life. Venerable, please don’t declare your inability to continue training, reject it and return to a lesser life.’ 

When\marginnote{12.1} thus advised and instructed by his spiritual companions, he says: ‘Reverends, even though the Buddha says that sensual pleasures give little gratification and much suffering and distress, and they are all the more full of drawbacks, I am unable to keep up the spiritual life. I declare my inability to continue training. I reject it and will return to a lesser life.’ Declaring his inability to continue training, he rejects it and returns to a lesser life. 

I\marginnote{13.1} say that this person is like the warrior who dies of his injuries while in the care of his relatives. Some people are like that. This is the third person similar to a warrior found among the monks. 

Furthermore,\marginnote{14.1} a mendicant lives supported by a town or village. He robes up in the morning and, taking his bowl and robe, enters a village or town for alms without guarding body, speech, and mind, without establishing mindfulness, and without restraining the sense faculties. There he sees a female scantily clad, with revealing clothes. Lust infects his mind, and his body and mind burn with it. He thinks: ‘Why don’t I go to the monastery and tell the monks: 

“Reverends,\marginnote{14.8} I am overcome with lust, mired in lust. I am unable to keep up the spiritual life. I declare my inability to continue training. I reject it and will return to a lesser life.”’ He goes to the monastery and tells the monks: ‘Reverends, I am overcome with lust, mired in lust. I am unable to keep up the spiritual life. I declare my inability to continue training. I reject it and will return to a lesser life.’ 

His\marginnote{15.1} spiritual companions advise and instruct him: ‘Reverend, the Buddha says that sensual pleasures give little gratification and much suffering and distress, and they are all the more full of drawbacks. With the simile of a skeleton … a lump of meat … a grass torch … a pit of glowing coals … a dream … borrowed goods … fruit on a tree … a butcher’s knife and chopping block … a staking sword … a snake’s head, the Buddha says that sensual pleasures give little gratification and much suffering and distress, and they are all the more full of drawbacks. Be happy with the spiritual life. Venerable, please don’t declare your inability to continue training, reject it and return to a lesser life.’ 

When\marginnote{16.1} thus advised and instructed by his spiritual companions, he says: ‘I’ll try, reverends, I’ll struggle, I’ll be happy. I won’t now declare my inability to continue training, reject it and return to a lesser life.’ 

I\marginnote{17.1} say that this person is like the warrior who recovers from his injuries while in the care of his relatives. Some people are like that. This is the fourth person similar to a warrior found among the monks. 

Furthermore,\marginnote{18.1} a mendicant lives supported by a town or village. He robes up in the morning and, taking his bowl and robe, enters a village or town, guarding body, speech, and mind, establishing mindfulness, and restraining the sense faculties. Seeing a sight with his eyes, he doesn’t get caught up in the features and details. If the faculty of sight were left unrestrained, bad unskillful qualities of desire and aversion would become overwhelming. For this reason, he practices restraint, protecting the faculty of sight, and achieving restraint over it. Hearing a sound with his ears … Smelling an odor with his nose … Tasting a flavor with his tongue … Feeling a touch with his body … Knowing a thought with his mind, he doesn’t get caught up in the features and details. If the faculty of mind were left unrestrained, bad unskillful qualities of desire and aversion would become overwhelming. For this reason, he practices restraint, protecting the faculty of mind, and achieving restraint over it. Then after the meal, on his return from almsround, he frequents a secluded lodging—a wilderness, the root of a tree, a hill, a ravine, a mountain cave, a charnel ground, a forest, the open air, a heap of straw. Gone to a wilderness, or to the root of a tree, or to an empty hut, he sits down cross-legged, with his body straight, and establishes mindfulness right there. He gives up these five hindrances, corruptions of the heart that weaken wisdom. Then, quite secluded from sensual pleasures, secluded from unskillful qualities, he enters and remains in the first absorption … second absorption … third absorption … fourth absorption. 

When\marginnote{19.1} his mind has become immersed in \textsanskrit{samādhi} like this—purified, bright, flawless, rid of corruptions, pliable, workable, steady, and imperturbable—he extends it toward knowledge of the ending of defilements. He truly understands: ‘This is suffering’ … ‘This is the origin of suffering’ … ‘This is the cessation of suffering’ … ‘This is the practice that leads to the cessation of suffering’. He truly understands: ‘These are defilements’ … ‘This is the origin of defilements’ … ‘This is the cessation of defilements’ … ‘This is the practice that leads to the cessation of defilements’. Knowing and seeing like this, his mind is freed from the defilements of sensuality, desire to be reborn, and ignorance. When freed, he knows ‘it is freed’. He understands: ‘Rebirth is ended, the spiritual journey has been completed, what had to be done has been done, there is no return to any state of existence.’ 

I\marginnote{20.1} say that this person is like the warrior who dons his sword and shield, fastens his bow and arrows, and plunges into the thick of battle. He wins victory in battle, establishing himself as foremost in battle. Some people are like that. This is the fifth person similar to a warrior found among the monks. 

These\marginnote{20.4} five people similar to warriors are found among the monks.” 

%
\section*{{\suttatitleacronym AN 5.77}{\suttatitletranslation Future Perils (1st) }{\suttatitleroot Paṭhamaanāgatabhayasutta}}
\addcontentsline{toc}{section}{\tocacronym{AN 5.77} \toctranslation{Future Perils (1st) } \tocroot{Paṭhamaanāgatabhayasutta}}
\markboth{Future Perils (1st) }{Paṭhamaanāgatabhayasutta}
\extramarks{AN 5.77}{AN 5.77}

“Mendicants,\marginnote{1.1} seeing these five future perils is quite enough for a wilderness mendicant to meditate diligently, keenly, and resolutely for attaining the unattained, achieving the unachieved, and realizing the unrealized. 

What\marginnote{2.1} five? Firstly, a wilderness mendicant reflects: ‘Currently I’m living alone in a wilderness. While living here alone I might get bitten by a snake, a scorpion, or a centipede. And if I died from that it would stop my practice. I’d better rouse up energy for attaining the unattained, achieving the unachieved, and realizing the unrealized.’ This is the first future peril … 

Furthermore,\marginnote{3.1} a wilderness mendicant reflects: ‘Currently I’m living alone in a wilderness. While living here alone I might stumble and fall, or get food poisoning, or my bile or phlegm or stabbing wind might get upset. And if I died from that it would stop my practice. I’d better rouse up energy for attaining the unattained, achieving the unachieved, and realizing the unrealized.’ This is the second future peril … 

Furthermore,\marginnote{4.1} a wilderness mendicant reflects: ‘Currently I’m living alone in a wilderness. While living here alone I might encounter wild beasts—a lion, a tiger, a leopard, a bear, or a hyena—which might take my life. And if I died from that it would stop my practice. I’d better rouse up energy for attaining the unattained, achieving the unachieved, and realizing the unrealized.’ This is the third future peril … 

Furthermore,\marginnote{5.1} a wilderness mendicant reflects: ‘Currently I’m living alone in a wilderness. While living here alone I might encounter youths escaping a crime or on their way to commit one, and they might take my life. And if I died from that it would stop my practice. I’d better rouse up energy for attaining the unattained, achieving the unachieved, and realizing the unrealized.’ This is the fourth future peril … 

Furthermore,\marginnote{6.1} a wilderness mendicant reflects: ‘Currently I’m living alone in a wilderness. But in a wilderness there are savage monsters who might take my life. And if I died from that it would stop my practice. I’d better rouse up energy for attaining the unattained, achieving the unachieved, and realizing the unrealized.’ This is the fifth future peril … 

These\marginnote{7.1} are the five future perils, seeing which is quite enough for a wilderness mendicant to meditate diligently, keenly, and resolutely for attaining the unattained, achieving the unachieved, and realizing the unrealized.” 

%
\section*{{\suttatitleacronym AN 5.78}{\suttatitletranslation Future Perils (2nd) }{\suttatitleroot Dutiyaanāgatabhayasutta}}
\addcontentsline{toc}{section}{\tocacronym{AN 5.78} \toctranslation{Future Perils (2nd) } \tocroot{Dutiyaanāgatabhayasutta}}
\markboth{Future Perils (2nd) }{Dutiyaanāgatabhayasutta}
\extramarks{AN 5.78}{AN 5.78}

“Mendicants,\marginnote{1.1} seeing these five future perils is quite enough for a mendicant to meditate diligently, keenly, and resolutely for attaining the unattained, achieving the unachieved, and realizing the unrealized. What five? 

A\marginnote{1.3} mendicant reflects: ‘Currently I’m a youth, young, black-haired, blessed with youth, in the prime of life. But there will come a time when this body is struck with old age. When you’re old, overcome by old age, it’s not easy to focus on the instructions of the Buddhas, and it’s not easy to frequent remote lodgings in the wilderness and the forest. Before that unlikable, undesirable, and disagreeable thing happens, I’d better preempt it by rousing up energy for attaining the unattained, achieving the unachieved, and realizing the unrealized. That way, when it happens, I’ll live comfortably even though I’m old.’ This is the first future peril … 

Furthermore,\marginnote{2.1} a mendicant reflects: ‘Currently, I’m rarely ill or unwell. My stomach digests well, being neither too hot nor too cold, but just right, and fit for meditation. But there will come a time when this body is struck with sickness. When you’re sick, overcome by sickness, it’s not easy to focus on the instructions of the Buddhas, and it’s not easy to frequent remote lodgings in the wilderness and the forest. Before that unlikable, undesirable, and disagreeable thing happens, I’d better preempt it by rousing up energy for attaining the unattained, achieving the unachieved, and realizing the unrealized. That way, when it happens, I’ll live comfortably even though I’m sick.’ This is the second future peril … 

Furthermore,\marginnote{3.1} a mendicant reflects: ‘Currently, there’s plenty of food, a good harvest, so it’s easy to get almsfood, and easy to keep going by collecting alms. But there will come a time of famine, a bad harvest, when it’s hard to get almsfood, and not easy to keep going by collecting alms. In a time of famine, people move to where there’s plenty of food, where they live crowded and cramped together. When you live crowded and cramped together, it’s not easy to focus on the instructions of the Buddhas, and it’s not easy to frequent remote lodgings in the wilderness and the forest. Before that unlikable, undesirable, and disagreeable thing happens, I’d better preempt it by rousing up energy for attaining the unattained, achieving the unachieved, and realizing the unrealized. That way, when it happens, I’ll live comfortably even though there’s a famine.’ This is the third future peril … 

Furthermore,\marginnote{4.1} a mendicant reflects: ‘Currently, people live in harmony, appreciating each other, without quarreling, blending like milk and water, and regarding each other with kindly eyes. But there will come a time of peril from wild savages, when the countryfolk mount their vehicles and flee everywhere. In a time of peril, people move to where there’s sanctuary, where they live crowded and cramped together. When you live crowded and cramped together, it’s not easy to focus on the instructions of the Buddhas, and it’s not easy to frequent remote lodgings in the wilderness and the forest. Before that unlikable, undesirable, and disagreeable thing happens, I’d better preempt it by rousing up energy for attaining the unattained, achieving the unachieved, and realizing the unrealized. That way, when it happens, I’ll live comfortably even in a time of peril.’ This is the fourth future peril … 

Furthermore,\marginnote{5.1} a mendicant reflects: ‘Currently, the \textsanskrit{Saṅgha} lives comfortably, in harmony, appreciating each other, without quarreling, with one recitation. But there will come a time of schism in the \textsanskrit{Saṅgha}. When there is schism in the \textsanskrit{Saṅgha}, it’s not easy to focus on the instructions of the Buddhas, and it’s not easy to frequent remote lodgings in the wilderness and the forest. Before that unlikable, undesirable, and disagreeable thing happens, I’d better preempt it by rousing up energy for attaining the unattained, achieving the unachieved, and realizing the unrealized. That way, when it happens, I’ll live comfortably even though there’s schism in the \textsanskrit{Saṅgha}.’ This is the fifth future peril … 

These\marginnote{6.1} are the five future perils, seeing which is quite enough for a mendicant to meditate diligently, keenly, and resolutely for attaining the unattained, achieving the unachieved, and realizing the unrealized.” 

%
\section*{{\suttatitleacronym AN 5.79}{\suttatitletranslation Future Perils (3rd) }{\suttatitleroot Tatiyaanāgatabhayasutta}}
\addcontentsline{toc}{section}{\tocacronym{AN 5.79} \toctranslation{Future Perils (3rd) } \tocroot{Tatiyaanāgatabhayasutta}}
\markboth{Future Perils (3rd) }{Tatiyaanāgatabhayasutta}
\extramarks{AN 5.79}{AN 5.79}

“Mendicants,\marginnote{1.1} these five future perils have not currently arisen, but they will arise in the future. You should look out for them and try to give them up. 

What\marginnote{2.1} five? In a future time there will be mendicants who have not developed their physical endurance, ethics, mind, and wisdom. They will ordain others, but be unable to guide them in the higher ethics, mind, and wisdom. They too will not develop their physical endurance, ethics, mind, and wisdom. They too will ordain others, but be unable to guide them in the higher ethics, mind, and wisdom. They too will not develop their physical endurance, ethics, mind, and wisdom. And that is how corrupt training comes from corrupt teachings, and corrupt teachings come from corrupt training. This is the first future peril that has not currently arisen, but will arise in the future … 

Furthermore,\marginnote{3.1} in a future time there will be mendicants who have not developed their physical endurance, ethics, mind, and wisdom. They will give dependence to others, but be unable to guide them in the higher ethics, mind, and wisdom. They too will not develop their physical endurance, ethics, mind, and wisdom. They too will give dependence to others, but be unable to guide them in the higher ethics, mind, and wisdom. They too will not develop their physical endurance, ethics, mind, and wisdom. And that is how corrupt training comes from corrupt teachings, and corrupt teachings come from corrupt training. This is the second future peril that has not currently arisen, but will arise in the future … 

Furthermore,\marginnote{4.1} in a future time there will be mendicants who have not developed their physical endurance, ethics, mind, and wisdom. In discussion about the teachings and classifications they’ll fall into dark ideas without realizing it. And that is how corrupt training comes from corrupt teachings, and corrupt teachings come from corrupt training. This is the third future peril that has not currently arisen, but will arise in the future … 

Furthermore,\marginnote{5.1} in a future time there will be mendicants who have not developed their physical endurance, ethics, mind, and wisdom. When discourses spoken by the Realized One—deep, profound, transcendent, dealing with emptiness—are being recited they won’t want to listen. They won’t pay attention or apply their minds to understand them, nor will they think those teachings are worth learning and memorizing. But when discourses composed by poets—poetry, with fancy words and phrases, composed by outsiders or spoken by disciples—are being recited they will want to listen. They’ll pay attention and apply their minds to understand them, and they’ll think those teachings are worth learning and memorizing. And that is how corrupt training comes from corrupt teachings, and corrupt teachings come from corrupt training. This is the fourth future peril that has not currently arisen, but will arise in the future … 

Furthermore,\marginnote{6.1} in a future time there will be mendicants who have not developed their physical endurance, ethics, mind, and wisdom. The senior mendicants will be indulgent and slack, leaders in backsliding, neglecting seclusion, not rousing energy for attaining the unattained, achieving the unachieved, and realizing the unrealized. Those who come after them will follow their example. They too will become indulgent and slack, leaders in backsliding, neglecting seclusion, not rousing energy for attaining the unattained, achieving the unachieved, and realizing the unrealized. And that is how corrupt training comes from corrupt teachings, and corrupt teachings come from corrupt training. This is the fifth future peril that has not currently arisen, but will arise in the future … 

These\marginnote{7.1} are the five future perils that have not currently arisen, but will arise in the future. You should look out for them, and try to give them up.” 

%
\section*{{\suttatitleacronym AN 5.80}{\suttatitletranslation Future Perils (4th) }{\suttatitleroot Catutthaanāgatabhayasutta}}
\addcontentsline{toc}{section}{\tocacronym{AN 5.80} \toctranslation{Future Perils (4th) } \tocroot{Catutthaanāgatabhayasutta}}
\markboth{Future Perils (4th) }{Catutthaanāgatabhayasutta}
\extramarks{AN 5.80}{AN 5.80}

“Mendicants,\marginnote{1.1} these five future perils have not currently arisen, but they will arise in the future. You should look out for them and try to give them up. 

What\marginnote{2.1} five? In a future time there will be mendicants who like nice robes. They will neglect the practice of wearing rag robes and the practice of frequenting remote lodgings in the wilderness and the forest. They will come down to the villages, towns, and capital cities and make their homes there. And they will try to get robes in many kinds of wrong and inappropriate ways. This is the first future peril that has not currently arisen, but will arise in the future … 

Furthermore,\marginnote{3.1} in a future time there will be mendicants who like nice almsfood. They will neglect the practice of walking for almsfood and the practice of frequenting remote lodgings in the wilderness and the forest. They will come down to the villages, towns, and capital cities and make their homes there. And they will try to get almsfood in many kinds of wrong and inappropriate ways. This is the second future peril that has not currently arisen, but will arise in the future … 

Furthermore,\marginnote{4.1} in a future time there will be mendicants who like nice lodgings. They will neglect the practice of staying at the root of a tree and the practice of frequenting remote lodgings in the wilderness and the forest. They will come down to the villages, towns, and capital cities and make their homes there. And they will try to get lodgings in many kinds of wrong and inappropriate ways. This is the third future peril that has not currently arisen, but will arise in the future … 

Furthermore,\marginnote{5.1} in a future time there will be mendicants who mix closely with nuns, trainee nuns, and novice nuns. In such conditions, it can be expected that they will lead the spiritual life dissatisfied, or commit one of the corrupt offenses, or resign the training and return to a lesser life. This is the fourth future peril that has not currently arisen, but will arise in the future … 

Furthermore,\marginnote{6.1} in a future time there will be mendicants who mix closely with monastery attendants and novices. In such conditions it can be expected that they will engage in storing up goods for their own use, and making obvious hints about digging the earth and cutting plants. This is the fifth future peril that has not currently arisen, but will arise in the future … 

These\marginnote{7.1} are the five future perils that have not currently arisen, but will arise in the future. You should look out for them and try to give them up.” 

%
\addtocontents{toc}{\let\protect\contentsline\protect\nopagecontentsline}
\chapter*{The Chapter on Senior Mendicants }
\addcontentsline{toc}{chapter}{\tocchapterline{The Chapter on Senior Mendicants }}
\addtocontents{toc}{\let\protect\contentsline\protect\oldcontentsline}

%
\section*{{\suttatitleacronym AN 5.81}{\suttatitletranslation Desirable }{\suttatitleroot Rajanīyasutta}}
\addcontentsline{toc}{section}{\tocacronym{AN 5.81} \toctranslation{Desirable } \tocroot{Rajanīyasutta}}
\markboth{Desirable }{Rajanīyasutta}
\extramarks{AN 5.81}{AN 5.81}

“Mendicants,\marginnote{1.1} a senior mendicant with five qualities is unlikable and unlovable to their spiritual companions, not respected or admired. What five? They desire the desirable, they hate the hateful, they’re deluded by the delusory, they’re annoyed by the annoying, and they’re intoxicated by the intoxicating. A senior mendicant with these five qualities is unlikable and unlovable by their spiritual companions, not respected or admired. 

A\marginnote{2.1} senior mendicant with five qualities is dear and beloved to their spiritual companions, respected and admired. What five? They don’t desire the desirable, they don’t hate the hateful, they’re not deluded by the delusory, they’re not annoyed by the annoying, and they’re not intoxicated by the intoxicating. A senior mendicant with these five qualities is dear and beloved to their spiritual companions, respected and admired.” 

%
\section*{{\suttatitleacronym AN 5.82}{\suttatitletranslation Free of Greed }{\suttatitleroot Vītarāgasutta}}
\addcontentsline{toc}{section}{\tocacronym{AN 5.82} \toctranslation{Free of Greed } \tocroot{Vītarāgasutta}}
\markboth{Free of Greed }{Vītarāgasutta}
\extramarks{AN 5.82}{AN 5.82}

“Mendicants,\marginnote{1.1} a senior mendicant with five qualities is unlikable and unlovable to their spiritual companions, not respected or admired. What five? They’re not free of greed, hate, and delusion; they are offensive and contemptuous. A senior mendicant with these five qualities is unlikable and unlovable to their spiritual companions, not respected or admired. 

A\marginnote{2.1} senior mendicant with five qualities is dear and beloved to their spiritual companions, respected and admired. What five? They’re free of greed, hate, and delusion; they’re not offensive and contemptuous. A senior mendicant with these five qualities is dear and beloved to their spiritual companions, respected and admired.” 

%
\section*{{\suttatitleacronym AN 5.83}{\suttatitletranslation Deceiver }{\suttatitleroot Kuhakasutta}}
\addcontentsline{toc}{section}{\tocacronym{AN 5.83} \toctranslation{Deceiver } \tocroot{Kuhakasutta}}
\markboth{Deceiver }{Kuhakasutta}
\extramarks{AN 5.83}{AN 5.83}

“Mendicants,\marginnote{1.1} a senior mendicant with five qualities is unlikable and unlovable to their spiritual companions, not respected or admired. What five? They use deceit, flattery, hinting, and belittling, and they use material possessions to chase after other material possessions. A senior mendicant with these five qualities is unlikable and unlovable to their spiritual companions, not respected or admired. 

A\marginnote{2.1} senior mendicant with five qualities is dear and beloved to their spiritual companions, respected and admired. What five? They don’t use deceit, flattery, hinting, or belittling, and they don’t use material possessions to chase after other material possessions. A senior mendicant with these five qualities is dear and beloved to their spiritual companions, respected and admired.” 

%
\section*{{\suttatitleacronym AN 5.84}{\suttatitletranslation Faithless }{\suttatitleroot Assaddhasutta}}
\addcontentsline{toc}{section}{\tocacronym{AN 5.84} \toctranslation{Faithless } \tocroot{Assaddhasutta}}
\markboth{Faithless }{Assaddhasutta}
\extramarks{AN 5.84}{AN 5.84}

“Mendicants,\marginnote{1.1} a senior mendicant with five qualities is unlikable and unlovable to their spiritual companions, not respected or admired. What five? They’re faithless, shameless, imprudent, lazy, and witless. A senior mendicant with these five qualities is unlikable and unlovable to their spiritual companions, not respected or admired. 

A\marginnote{2.1} senior mendicant with five qualities is dear and beloved to their spiritual companions, respected and admired. What five? They’re faithful, conscientious, prudent, energetic, and wise. A senior mendicant with these five qualities is dear and beloved to their spiritual companions, respected and admired.” 

%
\section*{{\suttatitleacronym AN 5.85}{\suttatitletranslation Cannot Endure }{\suttatitleroot Akkhamasutta}}
\addcontentsline{toc}{section}{\tocacronym{AN 5.85} \toctranslation{Cannot Endure } \tocroot{Akkhamasutta}}
\markboth{Cannot Endure }{Akkhamasutta}
\extramarks{AN 5.85}{AN 5.85}

“Mendicants,\marginnote{1.1} a senior mendicant with five qualities is unlikable and unlovable to their spiritual companions, not respected or admired. What five? They can’t endure sights, sounds, smells, tastes, and touches. A senior mendicant with these five qualities is unlikable and unlovable to their spiritual companions, not respected or admired. 

A\marginnote{2.1} senior mendicant with five qualities is dear and beloved to their spiritual companions, respected and admired. What five? They can endure sights, sounds, smells, tastes, and touches. A senior mendicant with these five qualities is dear and beloved to their spiritual companions, respected and admired.” 

%
\section*{{\suttatitleacronym AN 5.86}{\suttatitletranslation Attaining the Methods of Textual Analysis }{\suttatitleroot Paṭisambhidāpattasutta}}
\addcontentsline{toc}{section}{\tocacronym{AN 5.86} \toctranslation{Attaining the Methods of Textual Analysis } \tocroot{Paṭisambhidāpattasutta}}
\markboth{Attaining the Methods of Textual Analysis }{Paṭisambhidāpattasutta}
\extramarks{AN 5.86}{AN 5.86}

“A\marginnote{1.1} senior mendicant with five qualities is dear and beloved to their spiritual companions, respected and admired. What five? They have attained the textual analysis of meaning, text, terminology, and eloquence. And they are skilled and tireless in a diverse spectrum of duties for their spiritual companions, understanding how to go about things in order to complete and organize the work. A senior mendicant with these five qualities is dear and beloved to their spiritual companions, respected and admired.” 

%
\section*{{\suttatitleacronym AN 5.87}{\suttatitletranslation Ethical }{\suttatitleroot Sīlavantasutta}}
\addcontentsline{toc}{section}{\tocacronym{AN 5.87} \toctranslation{Ethical } \tocroot{Sīlavantasutta}}
\markboth{Ethical }{Sīlavantasutta}
\extramarks{AN 5.87}{AN 5.87}

“A\marginnote{1.1} senior mendicant with five qualities is dear and beloved to their spiritual companions, respected and admired. What five? 

They’re\marginnote{1.3} ethical, restrained in the monastic code, conducting themselves well and seeking alms in suitable places. Seeing danger in the slightest fault, they keep the rules they’ve undertaken. 

They’re\marginnote{1.4} very learned, remembering and keeping what they’ve learned. These teachings are good in the beginning, good in the middle, and good in the end, meaningful and well-phrased, describing a spiritual practice that’s entirely full and pure. They are very learned in such teachings, remembering them, reinforcing them by recitation, mentally scrutinizing them, and comprehending them theoretically. 

They’re\marginnote{1.5} a good speaker, with a polished, clear, and articulate voice that expresses the meaning. 

They\marginnote{1.6} get the four absorptions—blissful meditations in the present life that belong to the higher mind—when they want, without trouble or difficulty. 

They\marginnote{1.7} realize the undefiled freedom of heart and freedom by wisdom in this very life. And they live having realized it with their own insight due to the ending of defilements. 

A\marginnote{1.8} senior mendicant with these five qualities is dear and beloved to their spiritual companions, respected and admired.” 

%
\section*{{\suttatitleacronym AN 5.88}{\suttatitletranslation Senior Mendicants }{\suttatitleroot Therasutta}}
\addcontentsline{toc}{section}{\tocacronym{AN 5.88} \toctranslation{Senior Mendicants } \tocroot{Therasutta}}
\markboth{Senior Mendicants }{Therasutta}
\extramarks{AN 5.88}{AN 5.88}

“Mendicants,\marginnote{1.1} a senior mendicant who has five qualities is acting for the hurt and unhappiness of the people, for the harm, hurt, and suffering of gods and humans. 

What\marginnote{2.1} five? 

They\marginnote{2.2} are senior and have long gone forth. 

They’re\marginnote{2.3} well-known, famous, with a large following that includes both laypeople and renunciates. 

They\marginnote{2.4} receive robes, almsfood, lodgings, and medicines and supplies for the sick. 

They’re\marginnote{2.5} very learned, remembering and keeping what they’ve learned. These teachings are good in the beginning, good in the middle, and good in the end, meaningful and well-phrased, describing a spiritual practice that’s entirely full and pure. They are very learned in such teachings, remembering them, reinforcing them by recitation, mentally scrutinizing them, and comprehending them theoretically. 

But\marginnote{2.6} they have wrong view and distorted perspective. They draw many people away from the true teaching and establish them in false teachings. 

People\marginnote{2.7} follow their example, thinking that the senior mendicant is senior and has long gone forth. Or that they’re well-known, famous, with a large following that includes both laypeople and renunciates. Or that they receive robes, almsfood, lodgings, and medicines and supplies for the sick. Or that they’re very learned, remembering and keeping what they’ve learned. A senior mendicant who has these five qualities is acting for the hurt and unhappiness of the people, for the harm, hurt, and suffering of gods and humans. 

A\marginnote{3.1} senior mendicant who has five qualities is acting for the welfare and happiness of the people, for the benefit, welfare, and happiness of gods and humans. 

What\marginnote{4.1} five? 

They\marginnote{4.2} are senior and have long gone forth. 

They’re\marginnote{4.3} well-known, famous, with a large following, including both laypeople and renunciates. 

They\marginnote{4.4} receive robes, almsfood, lodgings, and medicines and supplies for the sick. 

They’re\marginnote{4.5} very learned, remembering and keeping what they’ve learned. These teachings are good in the beginning, good in the middle, and good in the end, meaningful and well-phrased, describing a spiritual practice that’s entirely full and pure. They are very learned in such teachings, remembering them, reinforcing them by recitation, mentally scrutinizing them, and comprehending them theoretically. 

And\marginnote{4.6} they have right view and an undistorted perspective. They draw many people away from false teachings and establish them in the true teaching. 

People\marginnote{4.7} follow their example, thinking that the senior mendicant is senior and has long gone forth. Or that they’re well-known, famous, with a large following that includes both laypeople and renunciates. Or that they receive robes, almsfood, lodgings, and medicines and supplies for the sick. Or that they’re very learned, remembering and keeping what they’ve learned. A senior mendicant who has these five qualities is acting for the welfare and happiness of the people, for the benefit, welfare, and happiness of gods and humans.” 

%
\section*{{\suttatitleacronym AN 5.89}{\suttatitletranslation A Trainee (1st) }{\suttatitleroot Paṭhamasekhasutta}}
\addcontentsline{toc}{section}{\tocacronym{AN 5.89} \toctranslation{A Trainee (1st) } \tocroot{Paṭhamasekhasutta}}
\markboth{A Trainee (1st) }{Paṭhamasekhasutta}
\extramarks{AN 5.89}{AN 5.89}

“These\marginnote{1.1} five things lead to the decline of a mendicant trainee. What five? They relish work, talk, sleep, and company. And they don’t review the extent of their mind’s freedom. These five things lead to the decline of a mendicant trainee. 

These\marginnote{2.1} five things don’t lead to the decline of a mendicant trainee. What five? They don’t relish work, talk, sleep, and company. And they review the extent of their mind’s freedom. These five things don’t lead to the decline of a mendicant trainee.” 

%
\section*{{\suttatitleacronym AN 5.90}{\suttatitletranslation A Trainee (2nd) }{\suttatitleroot Dutiyasekhasutta}}
\addcontentsline{toc}{section}{\tocacronym{AN 5.90} \toctranslation{A Trainee (2nd) } \tocroot{Dutiyasekhasutta}}
\markboth{A Trainee (2nd) }{Dutiyasekhasutta}
\extramarks{AN 5.90}{AN 5.90}

“These\marginnote{1.1} five things lead to the decline of a mendicant trainee. What five? 

Firstly,\marginnote{1.3} a mendicant trainee has many duties and responsibilities, and is competent in many tasks. They neglect retreat, and are not committed to internal serenity of heart. This is the first thing that leads to the decline of a mendicant trainee. 

Furthermore,\marginnote{2.1} a mendicant trainee spends their day doing trivial work. They neglect retreat, and are not committed to internal serenity of heart. This is the second thing that leads to the decline of a mendicant trainee. 

Furthermore,\marginnote{3.1} a mendicant trainee mixes closely with laypeople and renunciates, socializing inappropriately like a layperson. They neglect retreat, and are not committed to internal serenity of heart. This is the third thing that leads to the decline of a mendicant trainee. 

Furthermore,\marginnote{4.1} a mendicant trainee enters the town at the wrong time, and returns too late in the day. They neglect retreat, and are not committed to internal serenity of heart. This is the fourth thing that leads to the decline of a mendicant trainee. 

Furthermore,\marginnote{5.1} a mendicant trainee doesn’t get to take part in talk about self-effacement that helps open the heart, when they want, without trouble or difficulty. That is, talk about fewness of wishes, contentment, seclusion, aloofness, arousing energy, ethics, immersion, wisdom, freedom, and the knowledge and vision of freedom. They neglect retreat, and are not committed to internal serenity of heart. This is the fifth thing that leads to the decline of a mendicant trainee. 

These\marginnote{5.5} five things lead to the decline of a mendicant trainee. 

These\marginnote{6.1} five things don’t lead to the decline of a mendicant trainee. What five? 

Firstly,\marginnote{6.3} a mendicant trainee doesn’t have many duties and responsibilities, even though they are competent in many tasks. They don’t neglect retreat, and are committed to internal serenity of heart. This is the first thing that doesn’t lead to the decline of a mendicant trainee. 

Furthermore,\marginnote{7.1} a mendicant trainee doesn’t spend their day doing trivial work. They don’t neglect retreat, and are committed to internal serenity of heart. This is the second thing that doesn’t lead to the decline of a mendicant trainee. 

Furthermore,\marginnote{8.1} a mendicant trainee doesn’t mix closely with laypeople and renunciates, socializing inappropriately like a layperson. They don’t neglect retreat, and are committed to internal serenity of heart. This is the third thing that doesn’t lead to the decline of a mendicant trainee. 

Furthermore,\marginnote{9.1} a mendicant trainee doesn’t enter the village too early or return too late in the day. They don’t neglect retreat, and are committed to internal serenity of heart. This is the fourth thing that doesn’t lead to the decline of a mendicant trainee. 

Furthermore,\marginnote{10.1} a mendicant trainee gets to take part in talk about self-effacement that helps open the heart, when they want, without trouble or difficulty. That is, talk about fewness of wishes, contentment, seclusion, aloofness, arousing energy, ethics, immersion, wisdom, freedom, and the knowledge and vision of freedom. They don’t neglect retreat, and are committed to internal serenity of heart. This is the fifth thing that doesn't lead to the decline of a mendicant trainee. 

These\marginnote{10.5} five things don’t lead to the decline of a mendicant trainee.” 

%
\addtocontents{toc}{\let\protect\contentsline\protect\nopagecontentsline}
\chapter*{The Chapter with Kakudha }
\addcontentsline{toc}{chapter}{\tocchapterline{The Chapter with Kakudha }}
\addtocontents{toc}{\let\protect\contentsline\protect\oldcontentsline}

%
\section*{{\suttatitleacronym AN 5.91}{\suttatitletranslation Accomplishments (1st) }{\suttatitleroot Paṭhamasampadāsutta}}
\addcontentsline{toc}{section}{\tocacronym{AN 5.91} \toctranslation{Accomplishments (1st) } \tocroot{Paṭhamasampadāsutta}}
\markboth{Accomplishments (1st) }{Paṭhamasampadāsutta}
\extramarks{AN 5.91}{AN 5.91}

“Mendicants,\marginnote{1.1} there are five accomplishments. What five? Accomplishment in faith, ethics, learning, generosity, and wisdom. These are the five accomplishments.” 

%
\section*{{\suttatitleacronym AN 5.92}{\suttatitletranslation Accomplishment (2nd) }{\suttatitleroot Dutiyasampadāsutta}}
\addcontentsline{toc}{section}{\tocacronym{AN 5.92} \toctranslation{Accomplishment (2nd) } \tocroot{Dutiyasampadāsutta}}
\markboth{Accomplishment (2nd) }{Dutiyasampadāsutta}
\extramarks{AN 5.92}{AN 5.92}

“Mendicants,\marginnote{1.1} there are five accomplishments. What five? Accomplishment in ethics, immersion, wisdom, freedom, and the knowledge and vision of freedom. These are the five accomplishments.” 

%
\section*{{\suttatitleacronym AN 5.93}{\suttatitletranslation Declarations }{\suttatitleroot Byākaraṇasutta}}
\addcontentsline{toc}{section}{\tocacronym{AN 5.93} \toctranslation{Declarations } \tocroot{Byākaraṇasutta}}
\markboth{Declarations }{Byākaraṇasutta}
\extramarks{AN 5.93}{AN 5.93}

“Mendicants,\marginnote{1.1} there are five ways of declaring enlightenment. What five? One declares enlightenment out of stupidity and folly. Or because of wicked desires, being naturally full of desires. Or because of madness and mental disorder. Or out of overestimation. Or one declares enlightenment rightly. These are the five ways of declaring enlightenment.” 

%
\section*{{\suttatitleacronym AN 5.94}{\suttatitletranslation Living Comfortably }{\suttatitleroot Phāsuvihārasutta}}
\addcontentsline{toc}{section}{\tocacronym{AN 5.94} \toctranslation{Living Comfortably } \tocroot{Phāsuvihārasutta}}
\markboth{Living Comfortably }{Phāsuvihārasutta}
\extramarks{AN 5.94}{AN 5.94}

“Mendicants,\marginnote{1.1} there are these five ways of living comfortably. What five? It’s when a mendicant, quite secluded from sensual pleasures, secluded from unskillful qualities, enters and remains in the first absorption, which has the rapture and bliss born of seclusion, while placing the mind and keeping it connected. As the placing of the mind and keeping it connected are stilled, they enter and remain in the second absorption … third absorption … fourth absorption … They realize the undefiled freedom of heart and freedom by wisdom in this very life. And they live having realized it with their own insight due to the ending of defilements. These are the five ways of living comfortably.” 

%
\section*{{\suttatitleacronym AN 5.95}{\suttatitletranslation Unshakable }{\suttatitleroot Akuppasutta}}
\addcontentsline{toc}{section}{\tocacronym{AN 5.95} \toctranslation{Unshakable } \tocroot{Akuppasutta}}
\markboth{Unshakable }{Akuppasutta}
\extramarks{AN 5.95}{AN 5.95}

“Mendicants,\marginnote{1.1} a mendicant who has five things will soon penetrate the unshakable. What five? It’s when a mendicant has attained the textual analysis of meaning, text, terminology, and eloquence, and they review the extent of their mind’s freedom. A mendicant who has these five things will soon penetrate the unshakable.” 

%
\section*{{\suttatitleacronym AN 5.96}{\suttatitletranslation Remembering What You’ve Learned }{\suttatitleroot Sutadharasutta}}
\addcontentsline{toc}{section}{\tocacronym{AN 5.96} \toctranslation{Remembering What You’ve Learned } \tocroot{Sutadharasutta}}
\markboth{Remembering What You’ve Learned }{Sutadharasutta}
\extramarks{AN 5.96}{AN 5.96}

“Mendicants,\marginnote{1.1} a mendicant cultivating mindfulness of breathing who has five things will soon penetrate the unshakable. What five? 

It’s\marginnote{1.3} when a mendicant has few requirements and duties, and is unburdensome and contented with life’s necessities. 

They\marginnote{1.4} eat little, not devoted to filling their stomach. 

They\marginnote{1.5} are rarely drowsy, and are dedicated to wakefulness. 

They’re\marginnote{1.6} very learned, remembering and keeping what they’ve learned. These teachings are good in the beginning, good in the middle, and good in the end, meaningful and well-phrased, describing a spiritual practice that’s entirely full and pure. They are very learned in such teachings, remembering them, reinforcing them by recitation, mentally scrutinizing them, and comprehending them theoretically. 

They\marginnote{1.7} review the extent of their mind’s freedom. 

A\marginnote{1.8} mendicant cultivating mindfulness of breathing who has these five things will soon penetrate the unshakable.” 

%
\section*{{\suttatitleacronym AN 5.97}{\suttatitletranslation Talk }{\suttatitleroot Kathāsutta}}
\addcontentsline{toc}{section}{\tocacronym{AN 5.97} \toctranslation{Talk } \tocroot{Kathāsutta}}
\markboth{Talk }{Kathāsutta}
\extramarks{AN 5.97}{AN 5.97}

“Mendicants,\marginnote{1.1} a mendicant developing mindfulness of breathing who has five things will soon penetrate the unshakable. What five? 

It’s\marginnote{1.3} when a mendicant has few requirements and duties, and is unburdensome and contented with life’s necessities. 

They\marginnote{1.4} eat little, not devoted to filling their stomach. 

They\marginnote{1.5} are rarely drowsy, and are dedicated to wakefulness. 

They\marginnote{1.6} get to take part in talk about self-effacement that helps open the heart, when they want, without trouble or difficulty. That is, talk about fewness of wishes, contentment, seclusion, keeping your distance, arousing energy, ethics, immersion, wisdom, freedom, and the knowledge and vision of freedom. 

They\marginnote{1.8} review the extent of their mind’s freedom. 

A\marginnote{1.9} mendicant developing mindfulness of breathing who has these five things will soon penetrate the unshakable.” 

%
\section*{{\suttatitleacronym AN 5.98}{\suttatitletranslation In the Wilderness }{\suttatitleroot Āraññakasutta}}
\addcontentsline{toc}{section}{\tocacronym{AN 5.98} \toctranslation{In the Wilderness } \tocroot{Āraññakasutta}}
\markboth{In the Wilderness }{Āraññakasutta}
\extramarks{AN 5.98}{AN 5.98}

“Mendicants,\marginnote{1.1} a mendicant practicing mindfulness of breathing who has five things will soon penetrate the unshakable. What five? It’s when a mendicant has few requirements and duties, and is unburdensome and contented with life’s necessities. They eat little, not devoted to filling their stomach. They are rarely drowsy, and are dedicated to wakefulness. They live in the wilderness, in remote lodgings. They review the extent of their mind’s freedom. A mendicant practicing mindfulness of breathing who has these five things will soon penetrate the unshakable.” 

%
\section*{{\suttatitleacronym AN 5.99}{\suttatitletranslation The Lion }{\suttatitleroot Sīhasutta}}
\addcontentsline{toc}{section}{\tocacronym{AN 5.99} \toctranslation{The Lion } \tocroot{Sīhasutta}}
\markboth{The Lion }{Sīhasutta}
\extramarks{AN 5.99}{AN 5.99}

“Mendicants,\marginnote{1.1} towards evening the lion, king of beasts, emerges from his den, yawns, looks all around the four quarters, and roars his lion’s roar three times. Then he sets out on the hunt. If he strikes an elephant, he does it carefully, not carelessly. If he strikes a buffalo … a cow … a leopard … or any smaller creatures—even a hare or a cat—he does it carefully, not carelessly. Why is that? Thinking: ‘May I not lose my way.’ 

‘Lion’\marginnote{2.1} is a term for the Realized One, the perfected one, the fully awakened Buddha. When the Realized One teaches Dhamma to an assembly, this is his lion’s roar. When the Realized One teaches the monks … nuns … laymen … laywomen … or ordinary people—even food-carriers and hunters—he teaches them carefully, not carelessly. Why is that? Because the Realized One has respect and reverence for the teaching.” 

%
\section*{{\suttatitleacronym AN 5.100}{\suttatitletranslation With Kakudha }{\suttatitleroot Kakudhatherasutta}}
\addcontentsline{toc}{section}{\tocacronym{AN 5.100} \toctranslation{With Kakudha } \tocroot{Kakudhatherasutta}}
\markboth{With Kakudha }{Kakudhatherasutta}
\extramarks{AN 5.100}{AN 5.100}

\scevam{So\marginnote{1.1} I have heard. }At one time the Buddha was staying near Kosambi, in Ghosita’s Monastery. 

At\marginnote{1.3} that time the Koliyan named Kakudha—Venerable \textsanskrit{Mahāmoggallāna}’s supporter—had recently passed away and been reborn in a certain host of mind-made gods. He was reincarnated in a life-form that was two or three times the size of a Magadhan village with its fields. But with that life-form he didn’t obstruct himself or others. 

Then\marginnote{2.1} the god Kakudha went up to Venerable \textsanskrit{Mahāmoggallāna}, bowed, stood to one side, and said to him, “Sir, this fixed desire arose in Devadatta: ‘I will lead the mendicant \textsanskrit{Saṅgha}.’ And as that thought arose, Devadatta lost that psychic power.” 

That’s\marginnote{2.5} what the god Kakudha said. Then he bowed and respectfully circled \textsanskrit{Mahāmoggallāna}, keeping him on his right side, before vanishing right there. 

Then\marginnote{3.1} \textsanskrit{Mahāmoggallāna} went up to the Buddha, bowed, sat down to one side, and told him what had happened. 

“But\marginnote{5.1} \textsanskrit{Moggallāna}, did you comprehend the god Kakudha’s mind, and know that everything he says is correct and not otherwise?” 

“Indeed\marginnote{5.3} I did, sir.” 

“Mark\marginnote{5.5} these words, \textsanskrit{Moggallāna}! Mark these words! Now that silly man Devadatta will expose himself by his own deeds. 

\textsanskrit{Moggallāna},\marginnote{6.1} there are these five teachers found in the world. What five? 

Firstly,\marginnote{6.3} some teacher with impure conduct claims: ‘I am pure in ethics. My ethical conduct is pure, bright, uncorrupted.’ But their disciples know: ‘This teacher has impure ethical conduct, but claims to be ethically pure. They wouldn’t like it if we were to tell the laypeople. And how could we treat them in a way that they don’t like? But they consent to robes, almsfood, lodgings, and medicines and supplies for the sick. A person will be recognized by their own deeds.’ The disciples of such a teacher cover up their teacher’s conduct, and the teacher expects them to do so. 

Furthermore,\marginnote{7.1} some teacher with impure livelihood claims: ‘I am pure in livelihood. My livelihood is pure, bright, uncorrupted.’ But their disciples know: ‘This teacher has impure livelihood, but claims to have pure livelihood. They wouldn’t like it if we were to tell the laypeople. And how could we treat them in a way that they don’t like? But they consent to robes, almsfood, lodgings, and medicines and supplies for the sick. A person will be recognized by their own deeds.’ The disciples of such a teacher cover up their teacher’s livelihood, and the teacher expects them to do so. 

Furthermore,\marginnote{8.1} some teacher with impure teaching claims: ‘I am pure in teaching. My teaching is pure, bright, uncorrupted.’ But their disciples know: ‘This teacher has impure teaching, but claims to have pure teaching. They wouldn’t like it if we were to tell the laypeople. And how could we treat them in a way that they don’t like? But they consent to robes, almsfood, lodgings, and medicines and supplies for the sick. A person will be recognized by their own deeds.’ The disciples of such a teacher cover up their teacher’s teaching, and the teacher expects them to do so. 

Furthermore,\marginnote{9.1} some teacher with impure answers claims: ‘I am pure in how I answer. My answers are pure, bright, uncorrupted.’ But their disciples know: ‘This teacher has impure answers, but claims to have pure answers. They wouldn’t like it if we were to tell the laypeople. And how could we treat them in a way that they don’t like? But they consent to robes, almsfood, lodgings, and medicines and supplies for the sick. A person will be recognized by their own deeds.’ The disciples of such a teacher cover up their teacher’s answers, and the teacher expects them to do so. 

Furthermore,\marginnote{10.1} some teacher with impure knowledge and vision claims: ‘I am pure in knowledge and vision. My knowledge and vision are pure, bright, uncorrupted.’ But their disciples know: ‘This teacher has impure knowledge and vision, but claims to have pure knowledge and vision. They wouldn’t like it if we were to tell the laypeople. And how could we treat them in a way that they don’t like? But they consent to robes, almsfood, lodgings, and medicines and supplies for the sick. A person will be recognized by their own deeds.’ The disciples of such a teacher cover up their teacher’s knowledge and vision, and the teacher expects them to do so. These are the five teachers found in the world. 

But\marginnote{11.1} \textsanskrit{Moggallāna}, I have pure ethical conduct, and I claim: ‘I am pure in ethical conduct. My ethical conduct is pure, bright, uncorrupted.’ My disciples don’t cover up my conduct, and I don’t expect them to. I have pure livelihood, and I claim: ‘I am pure in livelihood. My livelihood is pure, bright, uncorrupted.’ My disciples don’t cover up my livelihood, and I don’t expect them to. I have pure teaching, and I claim: ‘I am pure in teaching. My teaching is pure, bright, uncorrupted.’ My disciples don’t cover up my teaching, and I don’t expect them to. I have pure answers, and I claim: ‘I am pure in how I answer. My answers are pure, bright, uncorrupted.’ My disciples don’t cover up my answers, and I don’t expect them to. I have pure knowledge and vision, and I claim: ‘I am pure in knowledge and vision. My knowledge and vision are pure, bright, uncorrupted.’ My disciples don’t cover up my knowledge and vision, and I don’t expect them to.” 

%
\addtocontents{toc}{\let\protect\contentsline\protect\nopagecontentsline}
\pannasa{The Third Fifty }
\addcontentsline{toc}{pannasa}{The Third Fifty }
\markboth{}{}
\addtocontents{toc}{\let\protect\contentsline\protect\oldcontentsline}

%
\addtocontents{toc}{\let\protect\contentsline\protect\nopagecontentsline}
\chapter*{The Chapter on Living Comfortably }
\addcontentsline{toc}{chapter}{\tocchapterline{The Chapter on Living Comfortably }}
\addtocontents{toc}{\let\protect\contentsline\protect\oldcontentsline}

%
\section*{{\suttatitleacronym AN 5.101}{\suttatitletranslation Assurance }{\suttatitleroot Sārajjasutta}}
\addcontentsline{toc}{section}{\tocacronym{AN 5.101} \toctranslation{Assurance } \tocroot{Sārajjasutta}}
\markboth{Assurance }{Sārajjasutta}
\extramarks{AN 5.101}{AN 5.101}

“Mendicants,\marginnote{1.1} these five qualities make a trainee assured. What five? It’s when a mendicant is faithful, ethical, learned, energetic, and wise. 

A\marginnote{2.1} person of faith doesn’t have the insecurities of someone who lacks faith. So this quality makes a trainee assured. 

An\marginnote{3.1} ethical person doesn’t have the insecurities of someone who is unethical. So this quality makes a trainee assured. 

A\marginnote{4.1} learned person doesn’t have the insecurities of a person of little learning. So this quality makes a trainee assured. 

An\marginnote{5.1} energetic person doesn’t have the insecurities of a lazy person. So this quality makes a trainee assured. 

A\marginnote{6.1} wise person doesn’t have the insecurities of someone who is witless. So this quality makes a trainee assured. 

These\marginnote{6.3} are the five qualities that make a trainee assured.” 

%
\section*{{\suttatitleacronym AN 5.102}{\suttatitletranslation Suspected }{\suttatitleroot Ussaṅkitasutta}}
\addcontentsline{toc}{section}{\tocacronym{AN 5.102} \toctranslation{Suspected } \tocroot{Ussaṅkitasutta}}
\markboth{Suspected }{Ussaṅkitasutta}
\extramarks{AN 5.102}{AN 5.102}

“Mendicants,\marginnote{1.1} even if a monk is of impeccable character, he might be suspected and distrusted as a ‘bad monk’ for five reasons. 

What\marginnote{2.1} five? It’s when a monk frequently collects alms from prostitutes, widows, voluptuous girls, eunuchs, or nuns. 

Even\marginnote{3.1} if a monk is of impeccable character, he might be suspected and distrusted as a ‘bad monk’ for these five reasons.” 

%
\section*{{\suttatitleacronym AN 5.103}{\suttatitletranslation A Master Thief }{\suttatitleroot Mahācorasutta}}
\addcontentsline{toc}{section}{\tocacronym{AN 5.103} \toctranslation{A Master Thief } \tocroot{Mahācorasutta}}
\markboth{A Master Thief }{Mahācorasutta}
\extramarks{AN 5.103}{AN 5.103}

“Mendicants,\marginnote{1.1} a master thief with five factors breaks into houses, plunders wealth, steals from isolated buildings, and commits highway robbery. What five? A master thief relies on rough ground, on thick cover, and on powerful individuals; they pay bribes, and they act alone. 

And\marginnote{2.1} how does a master thief rely on rough ground? It’s when a master thief relies on inaccessible riverlands or rugged mountains. That’s how a master thief relies on rough ground. 

And\marginnote{3.1} how does a master thief rely on thick cover? It’s when a master thief relies on thick grass, thick trees, a blind spot, or a large dense wood. That’s how a master thief relies on thick cover. 

And\marginnote{4.1} how does a master thief rely on powerful individuals? It’s when a master thief relies on rulers or their ministers. They think: ‘If anyone accuses me of anything, these rulers or their ministers will speak in my defense in the case.’ And that’s exactly what happens. That’s how a master thief relies on powerful individuals. 

And\marginnote{5.1} how does a master thief pay bribes? It’s when a master thief is rich, affluent, and wealthy. They think: ‘If anyone accuses me of anything, I’ll settle it with a bribe.’ And that’s exactly what happens. That’s how a master thief pays bribes. 

And\marginnote{6.1} how does a master thief act alone? It’s when a master thief carries out robbery all alone. Why is that? So that their secret plans are not leaked to others. That’s how a master thief acts alone. 

A\marginnote{7.1} master thief with these five factors breaks into houses, plunders wealth, steals from isolated buildings, and commits highway robbery. 

In\marginnote{8.1} the same way, when a bad mendicant has five qualities, they keep themselves broken and damaged. They deserve to be blamed and criticized by sensible people, and they make much bad karma. What five? A bad mendicant relies on rough ground, on thick cover, and on powerful individuals; they pay bribes, and they act alone. 

And\marginnote{9.1} how does a bad mendicant rely on rough ground? It’s when a bad mendicant has unethical conduct by way of body, speech, and mind. That’s how a bad mendicant relies on rough ground. 

And\marginnote{10.1} how does a bad mendicant rely on thick cover? It’s when a bad mendicant has wrong view, he’s attached to an extremist view. That’s how a bad mendicant relies on thick cover. 

And\marginnote{11.1} how does a bad mendicant rely on powerful individuals? It’s when a bad mendicant relies on rulers or their ministers. They think: ‘If anyone accuses me of anything, these rulers or their ministers will speak in my defense in the case.’ And that’s exactly what happens. That’s how a bad mendicant relies on powerful individuals. 

And\marginnote{12.1} how does a bad mendicant pay bribes? It’s when a bad mendicant receives robes, almsfood, lodgings, and medicines and supplies for the sick. They think: ‘If anyone accuses me of anything, I’ll settle it with a bribe.’ And that’s exactly what happens. That’s how a bad mendicant pays bribes. 

And\marginnote{13.1} how does a bad mendicant act alone? It’s when a bad mendicant dwells alone in the borderlands. They visit families there to get material possessions. That’s how a bad mendicant acts alone. 

When\marginnote{14.1} a bad mendicant has these five qualities, they keep themselves broken and damaged. They deserve to be blamed and criticized by sensible people, and they make much bad karma.” 

%
\section*{{\suttatitleacronym AN 5.104}{\suttatitletranslation An Exquisite Ascetic of Ascetics }{\suttatitleroot Samaṇasukhumālasutta}}
\addcontentsline{toc}{section}{\tocacronym{AN 5.104} \toctranslation{An Exquisite Ascetic of Ascetics } \tocroot{Samaṇasukhumālasutta}}
\markboth{An Exquisite Ascetic of Ascetics }{Samaṇasukhumālasutta}
\extramarks{AN 5.104}{AN 5.104}

“Mendicants,\marginnote{1.1} a mendicant with five qualities is an exquisite ascetic of ascetics. 

What\marginnote{2.1} five? 

It’s\marginnote{2.2} when a mendicant usually uses only what they’ve been invited to accept—robes, almsfood, lodgings, and medicines and supplies for the sick—rarely using them without invitation. 

When\marginnote{2.3} living with other spiritual practitioners, they usually treat them agreeably by way of body, speech, and mind, and rarely disagreeably. And they usually present them with agreeable things, rarely with disagreeable ones. 

They’re\marginnote{2.5} healthy, so the various unpleasant feelings—stemming from disorders of bile, phlegm, wind, or their conjunction; or caused by change in weather, by not taking care of themselves, by overexertion, or as the result of past deeds—usually don’t come up. 

They\marginnote{2.6} get the four absorptions—blissful meditations in the present life that belong to the higher mind—when they want, without trouble or difficulty. 

And\marginnote{2.7} they realize the undefiled freedom of heart and freedom by wisdom in this very life. And they live having realized it with their own insight due to the ending of defilements. 

A\marginnote{2.8} mendicant with these five qualities is an exquisite ascetic of ascetics. 

And\marginnote{3.1} if anyone should be rightly called an exquisite ascetic of ascetics, it’s me. For I usually use only what I’ve been invited to accept. When living with other spiritual practitioners, I usually treat them agreeably. And I usually present them with agreeable things. I’m healthy. I get the four absorptions when I want, without trouble or difficulty. And I’ve realized the undefiled freedom of heart and freedom by wisdom in this very life. So if anyone should be rightly called an exquisite ascetic of ascetics, it’s me.” 

%
\section*{{\suttatitleacronym AN 5.105}{\suttatitletranslation Living Comfortably }{\suttatitleroot Phāsuvihārasutta}}
\addcontentsline{toc}{section}{\tocacronym{AN 5.105} \toctranslation{Living Comfortably } \tocroot{Phāsuvihārasutta}}
\markboth{Living Comfortably }{Phāsuvihārasutta}
\extramarks{AN 5.105}{AN 5.105}

“Mendicants,\marginnote{1.1} there are these five ways of living comfortably. What five? 

It’s\marginnote{1.3} when a mendicant consistently treats their spiritual companions with kindness by way of body, speech, and mind, both in public and in private. 

They\marginnote{1.4} live according to the precepts shared with their spiritual companions, both in public and in private. Those precepts are unbroken, impeccable, spotless, and unmarred, liberating, praised by sensible people, not mistaken, and leading to immersion. 

They\marginnote{1.5} live according to the view shared with their spiritual companions, both in public and in private. That view is noble and emancipating, and brings one who practices it to the complete ending of suffering. 

These\marginnote{1.6} are the five ways of living comfortably.” 

%
\section*{{\suttatitleacronym AN 5.106}{\suttatitletranslation With Ānanda }{\suttatitleroot Ānandasutta}}
\addcontentsline{toc}{section}{\tocacronym{AN 5.106} \toctranslation{With Ānanda } \tocroot{Ānandasutta}}
\markboth{With Ānanda }{Ānandasutta}
\extramarks{AN 5.106}{AN 5.106}

At\marginnote{1.1} one time the Buddha was staying near Kosambi, in Ghosita’s Monastery. 

Then\marginnote{1.2} Venerable Ānanda went up to the Buddha, bowed, sat down to one side, and said to him, “Sir, how could a mendicant live comfortably while staying in a monastic community?” 

“It’s\marginnote{2.2} when a mendicant is accomplished in their own ethical conduct, but they don’t urge others to be ethical. That’s how a mendicant could live comfortably while staying in a monastic community.” 

“But\marginnote{3.1} sir, could there be another way for a mendicant to live comfortably while staying in a monastic community?” 

“There\marginnote{3.2} could, Ānanda. It’s when a mendicant is accomplished in their own ethical conduct, but they don’t urge others to be ethical. And they watch themselves, but don’t watch others. That’s how a mendicant could live comfortably while staying in a monastic community.” 

“But\marginnote{4.1} sir, could there be another way for a mendicant to live comfortably while staying in a monastic community?” 

“There\marginnote{4.2} could, Ānanda. It’s when a mendicant is accomplished in their own ethical conduct, but they don’t urge others to be ethical. And they watch themselves, but don’t watch others. And they’re not well-known, but aren’t bothered by that. That’s how a mendicant could live comfortably while staying in a monastic community.” 

“But\marginnote{5.1} sir, could there be another way for a mendicant to live comfortably while staying in a monastic community?” 

“There\marginnote{5.2} could, Ānanda. It’s when a mendicant is accomplished in their own ethical conduct, but they don’t urge others to be ethical. And they watch themselves, but don’t watch others. And they’re not well-known, but aren’t bothered by that. And they get the four absorptions—blissful meditations in the present life that belong to the higher mind—when they want, without trouble or difficulty. That’s how a mendicant could live comfortably while staying in a monastic community.” 

“But\marginnote{6.1} sir, might there be another way for a mendicant to live comfortably while staying in a monastic community?” 

“There\marginnote{6.2} could, Ānanda. It’s when a mendicant is accomplished in their own ethical conduct, but they don’t urge others to be ethical. And they watch themselves, but don’t watch others. And they’re not well-known, but aren’t bothered by that. And they get the four absorptions—blissful meditations in the present life that belong to the higher mind—when they want, without trouble or difficulty. And they realize the undefiled freedom of heart and freedom by wisdom in this very life. And they live having realized it with their own insight due to the ending of defilements. That’s how a mendicant could live comfortably while staying in a monastic community. 

And\marginnote{7.1} I say that there is no better or finer way of living comfortably than this.” 

%
\section*{{\suttatitleacronym AN 5.107}{\suttatitletranslation Ethics }{\suttatitleroot Sīlasutta}}
\addcontentsline{toc}{section}{\tocacronym{AN 5.107} \toctranslation{Ethics } \tocroot{Sīlasutta}}
\markboth{Ethics }{Sīlasutta}
\extramarks{AN 5.107}{AN 5.107}

“Mendicants,\marginnote{1.1} a mendicant with five qualities is worthy of offerings dedicated to the gods, worthy of hospitality, worthy of a religious donation, worthy of veneration with joined palms, and is the supreme field of merit for the world. 

What\marginnote{2.1} five? It’s when a mendicant is accomplished in ethics, immersion, wisdom, freedom, and the knowledge and vision of freedom. 

A\marginnote{3.1} mendicant with these five qualities is worthy of offerings dedicated to the gods, worthy of hospitality, worthy of a religious donation, worthy of veneration with joined palms, and is the supreme field of merit for the world.” 

%
\section*{{\suttatitleacronym AN 5.108}{\suttatitletranslation An adept }{\suttatitleroot Asekhasutta}}
\addcontentsline{toc}{section}{\tocacronym{AN 5.108} \toctranslation{An adept } \tocroot{Asekhasutta}}
\markboth{An adept }{Asekhasutta}
\extramarks{AN 5.108}{AN 5.108}

“Mendicants,\marginnote{1.1} a mendicant with five qualities is worthy of offerings dedicated to the gods, worthy of hospitality, worthy of a religious donation, worthy of veneration with joined palms, and is the supreme field of merit for the world. 

What\marginnote{2.1} five? It’s when they have the entire spectrum of the master’s ethics, immersion, wisdom, freedom, and knowledge and vision of freedom. A mendicant with these five qualities … is the supreme field of merit for the world.” 

%
\section*{{\suttatitleacronym AN 5.109}{\suttatitletranslation All Four Quarters }{\suttatitleroot Cātuddisasutta}}
\addcontentsline{toc}{section}{\tocacronym{AN 5.109} \toctranslation{All Four Quarters } \tocroot{Cātuddisasutta}}
\markboth{All Four Quarters }{Cātuddisasutta}
\extramarks{AN 5.109}{AN 5.109}

“Mendicants,\marginnote{1.1} a mendicant with five qualities is at ease in any quarter. What five? 

It’s\marginnote{1.3} when mendicant is ethical, restrained in the monastic code, conducting themselves well and seeking alms in suitable places. Seeing danger in the slightest fault, they keep the rules they’ve undertaken. 

They’re\marginnote{1.4} very learned, remembering and keeping what they’ve learned. These teachings are good in the beginning, good in the middle, and good in the end, meaningful and well-phrased, describing a spiritual practice that’s entirely full and pure. They are very learned in such teachings, remembering them, reinforcing them by recitation, mentally scrutinizing them, and comprehending them theoretically. 

They’re\marginnote{1.5} content with any kind of robes, almsfood, lodgings, and medicines and supplies for the sick. 

They\marginnote{1.6} get the four absorptions—blissful meditations in the present life that belong to the higher mind—when they want, without trouble or difficulty. 

They\marginnote{1.7} realize the undefiled freedom of heart and freedom by wisdom in this very life. And they live having realized it with their own insight due to the ending of defilements. 

A\marginnote{1.8} mendicant with these five qualities is at ease in any quarter.” 

%
\section*{{\suttatitleacronym AN 5.110}{\suttatitletranslation Wilderness }{\suttatitleroot Araññasutta}}
\addcontentsline{toc}{section}{\tocacronym{AN 5.110} \toctranslation{Wilderness } \tocroot{Araññasutta}}
\markboth{Wilderness }{Araññasutta}
\extramarks{AN 5.110}{AN 5.110}

“Mendicants,\marginnote{1.1} when a mendicant has five qualities they’re ready to frequent remote lodgings in the wilderness and the forest. What five? 

It’s\marginnote{1.3} when a mendicant is ethical, restrained in the code of conduct, conducting themselves well and seeking alms in suitable places. Seeing danger in the slightest fault, they keep the rules they’ve undertaken. 

They’re\marginnote{1.4} very learned, remembering and keeping what they’ve learned. These teachings are good in the beginning, good in the middle, and good in the end, meaningful and well-phrased, describing a spiritual practice that’s totally full and pure. They are very learned in such teachings, remembering them, reciting them, mentally scrutinizing them, and comprehending them theoretically. 

They\marginnote{1.5} live with energy roused up. They’re strong, staunchly vigorous, not slacking off when it comes to developing skillful qualities. 

They\marginnote{1.6} get the four absorptions—blissful meditations in the present life that belong to the higher mind—when they want, without trouble or difficulty. 

They\marginnote{1.7} realize the undefiled freedom of heart and freedom by wisdom in this very life. And they live having realized it with their own insight due to the ending of defilements. 

When\marginnote{1.8} a mendicant has these five qualities they’re ready to frequent remote lodgings in the wilderness and the forest.” 

%
\addtocontents{toc}{\let\protect\contentsline\protect\nopagecontentsline}
\chapter*{The Chapter at Andhakavinda }
\addcontentsline{toc}{chapter}{\tocchapterline{The Chapter at Andhakavinda }}
\addtocontents{toc}{\let\protect\contentsline\protect\oldcontentsline}

%
\section*{{\suttatitleacronym AN 5.111}{\suttatitletranslation Visiting Families }{\suttatitleroot Kulūpakasutta}}
\addcontentsline{toc}{section}{\tocacronym{AN 5.111} \toctranslation{Visiting Families } \tocroot{Kulūpakasutta}}
\markboth{Visiting Families }{Kulūpakasutta}
\extramarks{AN 5.111}{AN 5.111}

“Mendicants,\marginnote{1.1} a mendicant with five qualities who visits families is unlikable and unlovable, not respected or admired. What five? They act as though they're close to people they hardly know. They give away things they don’t own. They over-associate with close friends. They whisper in the ear. And they ask for too much. A mendicant with these five qualities who visits families is unlikable and unlovable, not respected or admired. 

A\marginnote{2.1} mendicant with five qualities who visits families is dear and beloved, respected and admired. What five? They don’t act as though they're close to people they hardly know. They don’t give away things they don’t own. They don’t over-associate with close friends. They don’t whisper in the ear. And they don’t ask for too much. A mendicant with these five qualities who visits families is dear and beloved, respected and admired.” 

%
\section*{{\suttatitleacronym AN 5.112}{\suttatitletranslation An Ascetic to Follow Behind on Almsround }{\suttatitleroot Pacchāsamaṇasutta}}
\addcontentsline{toc}{section}{\tocacronym{AN 5.112} \toctranslation{An Ascetic to Follow Behind on Almsround } \tocroot{Pacchāsamaṇasutta}}
\markboth{An Ascetic to Follow Behind on Almsround }{Pacchāsamaṇasutta}
\extramarks{AN 5.112}{AN 5.112}

“Mendicants,\marginnote{1.1} you shouldn’t take an ascetic with five qualities to follow behind on almsround. What five? They walk too far away or too close behind. They don’t take your bowl when it’s full. They don’t warn you when your speech is bordering on an offense. They keep on interrupting while you’re speaking. And they’re witless, dull, and stupid. You shouldn’t take an ascetic with these five qualities to follow behind on almsround. 

You\marginnote{2.1} should take an ascetic with five qualities to follow behind on almsround. What five? They don’t walk too far away or too close behind. They take your bowl when it is full. They warn you when your speech is bordering on an offense. They don’t interrupt while you’re speaking. And they’re wise, bright, and clever. You should take an ascetic with these five qualities to follow behind on almsround.” 

%
\section*{{\suttatitleacronym AN 5.113}{\suttatitletranslation Right Immersion }{\suttatitleroot Sammāsamādhisutta}}
\addcontentsline{toc}{section}{\tocacronym{AN 5.113} \toctranslation{Right Immersion } \tocroot{Sammāsamādhisutta}}
\markboth{Right Immersion }{Sammāsamādhisutta}
\extramarks{AN 5.113}{AN 5.113}

“Mendicants,\marginnote{1.1} a mendicant who has five qualities can’t enter and remain in right immersion. What five? It’s when a mendicant can’t endure sights, sounds, smells, tastes, and touches. A mendicant who has these five qualities can’t enter and remain in right immersion. 

A\marginnote{2.1} mendicant who has five qualities can enter and remain in right immersion. What five? It’s when a mendicant can endure sights, sounds, smells, tastes, and touches. A mendicant who has these five qualities can enter and remain in right immersion.” 

%
\section*{{\suttatitleacronym AN 5.114}{\suttatitletranslation At Andhakavinda }{\suttatitleroot Andhakavindasutta}}
\addcontentsline{toc}{section}{\tocacronym{AN 5.114} \toctranslation{At Andhakavinda } \tocroot{Andhakavindasutta}}
\markboth{At Andhakavinda }{Andhakavindasutta}
\extramarks{AN 5.114}{AN 5.114}

At\marginnote{1.1} one time the Buddha was staying in the land of the Magadhans at Andhakavinda. Then Venerable Ānanda went up to the Buddha, bowed, and sat down to one side. The Buddha said to him: 

“Ānanda,\marginnote{2.1} those mendicants who are junior, recently gone forth, newly come to this teaching and training should be encouraged, supported, and established in five things. What five? 

They\marginnote{2.3} should be encouraged, supported, and established in restraint in the monastic code: ‘Reverends, please be ethical. Live restrained in the code of conduct, conducting yourselves well and seeking alms in suitable places. Seeing danger in the slightest fault, keep the rules you’ve undertaken.’ 

They\marginnote{3.1} should be encouraged, supported, and established in sense restraint: ‘Reverends, please live with sense doors guarded, mindfully alert and on guard, with protected mind, having a heart protected by mindfulness.’ 

They\marginnote{4.1} should be encouraged, supported, and established in limiting their speech: ‘Reverends, please speak little. Put a limit on your speech.’ 

They\marginnote{5.1} should be encouraged, supported, and established in retreat: ‘Reverends, please live in the wilderness. Frequent remote lodgings in the wilderness and the forest.’ 

They\marginnote{6.1} should be encouraged, supported, and established in right perspective: ‘Reverends, please hold right view and have right perspective.’ 

Those\marginnote{6.2} mendicants who are junior, recently gone forth, newly come to this teaching and training should be encouraged, supported, and established in these five things.” 

%
\section*{{\suttatitleacronym AN 5.115}{\suttatitletranslation Stingy }{\suttatitleroot Maccharinīsutta}}
\addcontentsline{toc}{section}{\tocacronym{AN 5.115} \toctranslation{Stingy } \tocroot{Maccharinīsutta}}
\markboth{Stingy }{Maccharinīsutta}
\extramarks{AN 5.115}{AN 5.115}

“Mendicants,\marginnote{1.1} a nun with five qualities is cast down to hell. What five? She is stingy with dwellings, families, material possessions, praise, and the teaching. A nun with these five qualities is cast down to hell. 

A\marginnote{2.1} nun with five qualities is raised up to heaven. What five? She is not stingy with dwellings, families, material possessions, praise, or the teaching. A nun with these five qualities is raised up to heaven.” 

%
\section*{{\suttatitleacronym AN 5.116}{\suttatitletranslation Praise }{\suttatitleroot Vaṇṇanāsutta}}
\addcontentsline{toc}{section}{\tocacronym{AN 5.116} \toctranslation{Praise } \tocroot{Vaṇṇanāsutta}}
\markboth{Praise }{Vaṇṇanāsutta}
\extramarks{AN 5.116}{AN 5.116}

“Mendicants,\marginnote{1.1} a nun with five qualities is cast down to hell. What five? Without examining or scrutinizing, she praises those deserving of criticism, and criticizes those deserving of praise. She arouses faith in things that are dubious, and doesn’t arouse faith in things that are inspiring. And she wastes gifts given in faith. A nun with these five qualities is cast down to hell. 

A\marginnote{2.1} nun with five qualities is raised up to heaven. What five? After examining and scrutinizing, she criticizes those deserving of criticism, and praises those deserving of praise. She doesn’t arouse faith in things that are dubious, and does arouse faith in things that are inspiring. And she doesn’t waste gifts given in faith. A nun with these five qualities is raised up to heaven.” 

%
\section*{{\suttatitleacronym AN 5.117}{\suttatitletranslation Jealous }{\suttatitleroot Issukinīsutta}}
\addcontentsline{toc}{section}{\tocacronym{AN 5.117} \toctranslation{Jealous } \tocroot{Issukinīsutta}}
\markboth{Jealous }{Issukinīsutta}
\extramarks{AN 5.117}{AN 5.117}

“Mendicants,\marginnote{1.1} a nun with five qualities is cast down to hell. What five? Without examining or scrutinizing, she praises those deserving of criticism, and criticizes those deserving of praise. She is jealous, stingy, and wastes gifts given in faith. A nun with these five qualities is cast down to hell. 

A\marginnote{2.1} nun with five qualities is raised up to heaven. What five? After examining and scrutinizing, she criticizes those deserving of criticism, and praises those deserving of praise. She is not jealous, or stingy, and doesn’t waste gifts given in faith. A nun with these five qualities is raised up to heaven.” 

%
\section*{{\suttatitleacronym AN 5.118}{\suttatitletranslation Having Wrong View }{\suttatitleroot Micchādiṭṭhikasutta}}
\addcontentsline{toc}{section}{\tocacronym{AN 5.118} \toctranslation{Having Wrong View } \tocroot{Micchādiṭṭhikasutta}}
\markboth{Having Wrong View }{Micchādiṭṭhikasutta}
\extramarks{AN 5.118}{AN 5.118}

“Mendicants,\marginnote{1.1} a nun with five qualities is cast down to hell. What five? Without examining or scrutinizing, she praises those deserving of criticism, and criticizes those deserving of praise. She has wrong view and wrong thought, and wastes gifts given in faith. A nun with these five qualities is cast down to hell. 

A\marginnote{2.1} nun with five qualities is raised up to heaven. What five? After examining and scrutinizing, she criticizes those deserving of criticism, and praises those deserving of praise. She has right view and right thought, and doesn’t waste gifts given in faith. A nun with these five qualities is raised up to heaven.” 

%
\section*{{\suttatitleacronym AN 5.119}{\suttatitletranslation Wrong Speech }{\suttatitleroot Micchāvācāsutta}}
\addcontentsline{toc}{section}{\tocacronym{AN 5.119} \toctranslation{Wrong Speech } \tocroot{Micchāvācāsutta}}
\markboth{Wrong Speech }{Micchāvācāsutta}
\extramarks{AN 5.119}{AN 5.119}

“Mendicants,\marginnote{1.1} a nun with five qualities is cast down to hell. What five? Without examining or scrutinizing, she praises those deserving of criticism, and criticizes those deserving of praise. She has wrong speech and wrong action, and wastes gifts given in faith. A nun with these five qualities is cast down to hell. 

A\marginnote{2.1} nun with five qualities is raised up to heaven. What five? After examining and scrutinizing, she criticizes those deserving of criticism, and praises those deserving of praise. She has right speech and right action, and doesn’t waste gifts given in faith. A nun with these five qualities is raised up to heaven.” 

%
\section*{{\suttatitleacronym AN 5.120}{\suttatitletranslation Wrong Effort }{\suttatitleroot Micchāvāyāmasutta}}
\addcontentsline{toc}{section}{\tocacronym{AN 5.120} \toctranslation{Wrong Effort } \tocroot{Micchāvāyāmasutta}}
\markboth{Wrong Effort }{Micchāvāyāmasutta}
\extramarks{AN 5.120}{AN 5.120}

“Mendicants,\marginnote{1.1} a nun with five qualities is cast down to hell. What five? Without examining or scrutinizing, she praises those deserving of criticism, and criticizes those deserving of praise. She has wrong effort and wrong mindfulness, and wastes gifts given in faith. A nun with these five qualities is cast down to hell. 

A\marginnote{2.1} nun with five qualities is raised up to heaven. What five? After examining and scrutinizing, she criticizes those deserving of criticism, and praises those deserving of praise. She has right effort and right mindfulness, and doesn’t waste gifts given in faith. A nun with these five qualities is raised up to heaven.” 

%
\addtocontents{toc}{\let\protect\contentsline\protect\nopagecontentsline}
\chapter*{The Chapter on Sick }
\addcontentsline{toc}{chapter}{\tocchapterline{The Chapter on Sick }}
\addtocontents{toc}{\let\protect\contentsline\protect\oldcontentsline}

%
\section*{{\suttatitleacronym AN 5.121}{\suttatitletranslation Sick }{\suttatitleroot Gilānasutta}}
\addcontentsline{toc}{section}{\tocacronym{AN 5.121} \toctranslation{Sick } \tocroot{Gilānasutta}}
\markboth{Sick }{Gilānasutta}
\extramarks{AN 5.121}{AN 5.121}

At\marginnote{1.1} one time the Buddha was staying near \textsanskrit{Vesālī}, at the Great Wood, in the hall with the peaked roof. Then in the late afternoon, the Buddha came out of retreat and went to the infirmary, where he saw a certain mendicant who was weak and sick. He sat down on the seat spread out, and addressed the mendicants: 

“Mendicants,\marginnote{2.1} if a weak and sick mendicant does not neglect five things, it can be expected that they will soon realize the undefiled freedom of heart and freedom by wisdom in this very life, and live having realized it with their own insight due to the ending of defilements. 

What\marginnote{3.1} five? It’s when a mendicant meditates observing the ugliness of the body, perceives the repulsiveness of food, perceives dissatisfaction with the whole world, observes the impermanence of all conditions, and has well established the perception of their own death. If a weak and sick mendicant does not neglect these five things, it can be expected that they will soon realize the undefiled freedom of heart and freedom by wisdom in this very life, and live having realized it with their own insight due to the ending of defilements.” 

%
\section*{{\suttatitleacronym AN 5.122}{\suttatitletranslation Mindfulness Well Established }{\suttatitleroot Satisūpaṭṭhitasutta}}
\addcontentsline{toc}{section}{\tocacronym{AN 5.122} \toctranslation{Mindfulness Well Established } \tocroot{Satisūpaṭṭhitasutta}}
\markboth{Mindfulness Well Established }{Satisūpaṭṭhitasutta}
\extramarks{AN 5.122}{AN 5.122}

“Mendicants,\marginnote{1.1} any monk or nun who develops and cultivates five qualities can expect one of two results: enlightenment in the present life, or if there’s something left over, non-return. 

What\marginnote{2.1} five? It’s when a mendicant has well established mindfulness inside themselves in order to understand the arising and passing away of phenomena, meditates observing the ugliness of the body, perceives the repulsiveness of food, perceives dissatisfaction with the whole world, and observes the impermanence of all conditions. Any monk or nun who develops and cultivates these five qualities can expect one of two results: enlightenment in the present life, or if there’s something left over, non-return.” 

%
\section*{{\suttatitleacronym AN 5.123}{\suttatitletranslation A Carer (1st) }{\suttatitleroot Paṭhamaupaṭṭhākasutta}}
\addcontentsline{toc}{section}{\tocacronym{AN 5.123} \toctranslation{A Carer (1st) } \tocroot{Paṭhamaupaṭṭhākasutta}}
\markboth{A Carer (1st) }{Paṭhamaupaṭṭhākasutta}
\extramarks{AN 5.123}{AN 5.123}

“Mendicants,\marginnote{1.1} a patient with five qualities is hard to care for. What five? They do what is unsuitable. They don’t know moderation in what is suitable. They don’t take their medicine. Though their carer wants what’s best for them, they don’t accurately report their symptoms by saying when they’re getting worse, getting better, or staying the same. And they cannot endure physical pain—sharp, severe, acute, unpleasant, disagreeable, and life-threatening. A patient with these five qualities is hard to care for. 

A\marginnote{2.1} patient with five qualities is easy to care for. What five? They do what is suitable. They know moderation in what is suitable. They take their medicine. Because their carer wants what’s best for them, they accurately report their symptoms by saying when they’re getting worse, getting better, or staying the same. And they can endure physical pain—sharp, severe, acute, unpleasant, disagreeable, and life-threatening. A patient with these five qualities is easy to care for.” 

%
\section*{{\suttatitleacronym AN 5.124}{\suttatitletranslation A Carer (2nd) }{\suttatitleroot Dutiyaupaṭṭhākasutta}}
\addcontentsline{toc}{section}{\tocacronym{AN 5.124} \toctranslation{A Carer (2nd) } \tocroot{Dutiyaupaṭṭhākasutta}}
\markboth{A Carer (2nd) }{Dutiyaupaṭṭhākasutta}
\extramarks{AN 5.124}{AN 5.124}

“Mendicants,\marginnote{1.1} a carer with five qualities is not competent to care for a patient. What five? They’re unable to prepare medicine. They don’t know what is suitable and unsuitable, so they supply what is unsuitable and remove what is suitable. They care for the sick for the sake of material benefits, not out of love. They’re disgusted to remove feces, urine, vomit, or spit. They’re unable to educate, encourage, fire up, and inspire the patient with a Dhamma talk from time to time. A carer with these five qualities is not competent to care for a patient. 

A\marginnote{2.1} carer with five qualities is competent to care for a patient. What five? They’re able to prepare medicine. They know what is suitable and unsuitable, so they remove what is unsuitable and supply what is suitable. They care for the sick out of love, not for the sake of material benefits. They’re not disgusted to remove feces, urine, vomit, or spit. They’re able to educate, encourage, fire up, and inspire the patient with a Dhamma talk from time to time. A carer with these five qualities is competent to care for a patient.” 

%
\section*{{\suttatitleacronym AN 5.125}{\suttatitletranslation Longevity (1st) }{\suttatitleroot Paṭhamaanāyussāsutta}}
\addcontentsline{toc}{section}{\tocacronym{AN 5.125} \toctranslation{Longevity (1st) } \tocroot{Paṭhamaanāyussāsutta}}
\markboth{Longevity (1st) }{Paṭhamaanāyussāsutta}
\extramarks{AN 5.125}{AN 5.125}

“Mendicants,\marginnote{1.1} these five things impede longevity. What five? Doing what is unsuitable, not knowing moderation in what is suitable, eating food unfit for consumption, activity at unsuitable times, and unchastity. These are the five things that impede longevity. 

These\marginnote{2.1} five things promote longevity. What five? Doing what is suitable, knowing moderation in what is suitable, eating food fit for consumption, activity at suitable times, and celibacy. These are the five things that promote longevity.” 

%
\section*{{\suttatitleacronym AN 5.126}{\suttatitletranslation Longevity (2nd) }{\suttatitleroot Dutiyaanāyussāsutta}}
\addcontentsline{toc}{section}{\tocacronym{AN 5.126} \toctranslation{Longevity (2nd) } \tocroot{Dutiyaanāyussāsutta}}
\markboth{Longevity (2nd) }{Dutiyaanāyussāsutta}
\extramarks{AN 5.126}{AN 5.126}

“Mendicants,\marginnote{1.1} these five things impede longevity. What five? Doing what is unsuitable, not knowing moderation in what is suitable, eating food unfit for consumption, unethical behavior, and bad friends. These are the five things that impede longevity. 

These\marginnote{2.1} five things promote longevity. What five? Doing what is suitable, knowing moderation in what is suitable, eating food fit for consumption, ethical conduct, and good friends. These are the five things that promote longevity.” 

%
\section*{{\suttatitleacronym AN 5.127}{\suttatitletranslation Living Apart }{\suttatitleroot Vapakāsasutta}}
\addcontentsline{toc}{section}{\tocacronym{AN 5.127} \toctranslation{Living Apart } \tocroot{Vapakāsasutta}}
\markboth{Living Apart }{Vapakāsasutta}
\extramarks{AN 5.127}{AN 5.127}

“Mendicants,\marginnote{1.1} a mendicant with five qualities is not fit to live apart from a \textsanskrit{Saṅgha} community. What five? It’s when a mendicant is not content with any kind of robe, almsfood, lodging, and medicines and supplies for the sick. And they have a lot of sensual thoughts. A mendicant with these five qualities is not fit to live apart from a \textsanskrit{Saṅgha} community. 

A\marginnote{2.1} mendicant with five qualities is fit to live apart from a \textsanskrit{Saṅgha} community. What five? It’s when a mendicant is content with any kind of robe, almsfood, lodging, and medicines and supplies for the sick. And they think a lot about renunciation. A mendicant with these five qualities is fit to live apart from a \textsanskrit{Saṅgha} community.” 

%
\section*{{\suttatitleacronym AN 5.128}{\suttatitletranslation An Ascetic’s Happiness }{\suttatitleroot Samaṇasukhasutta}}
\addcontentsline{toc}{section}{\tocacronym{AN 5.128} \toctranslation{An Ascetic’s Happiness } \tocroot{Samaṇasukhasutta}}
\markboth{An Ascetic’s Happiness }{Samaṇasukhasutta}
\extramarks{AN 5.128}{AN 5.128}

“Mendicants,\marginnote{1.1} there are these five kinds of suffering for an ascetic. What five? It’s when a mendicant is not content with any kind of robe, almsfood, lodging, and medicines and supplies for the sick. And they lead the spiritual life dissatisfied. These are five kinds of suffering for an ascetic. 

There\marginnote{2.1} are these five kinds of happiness for an ascetic. What five? It’s when a mendicant is content with any kind of robe, almsfood, lodging, and medicines and supplies for the sick. And they lead the spiritual life satisfied. These are five kinds of happiness for an ascetic.” 

%
\section*{{\suttatitleacronym AN 5.129}{\suttatitletranslation Fatal Wounds }{\suttatitleroot Parikuppasutta}}
\addcontentsline{toc}{section}{\tocacronym{AN 5.129} \toctranslation{Fatal Wounds } \tocroot{Parikuppasutta}}
\markboth{Fatal Wounds }{Parikuppasutta}
\extramarks{AN 5.129}{AN 5.129}

“Mendicants,\marginnote{1.1} these five fatal wounds lead to a place of loss, to hell. What five? Murdering your mother or father or a perfected one; maliciously shedding the blood of a Realized One; and causing a schism in the \textsanskrit{Saṅgha}. These five fatal wounds lead to a place of loss, to hell.” 

%
\section*{{\suttatitleacronym AN 5.130}{\suttatitletranslation Loss }{\suttatitleroot Byasanasutta}}
\addcontentsline{toc}{section}{\tocacronym{AN 5.130} \toctranslation{Loss } \tocroot{Byasanasutta}}
\markboth{Loss }{Byasanasutta}
\extramarks{AN 5.130}{AN 5.130}

“Mendicants,\marginnote{1.1} there are these five losses. What five? Loss of relatives, wealth, health, ethics, and view. It is not because of loss of relatives, wealth, or health that sentient beings, when their body breaks up, after death, are reborn in a place of loss, a bad place, the underworld, hell. It is because of loss of ethics or view that sentient beings, when their body breaks up, after death, are reborn in a place of loss, a bad place, the underworld, hell. These are the five losses. 

There\marginnote{2.1} are these five endowments. What five? Endowment with relatives, wealth, health, ethics, and view. It is not because of endowment with relatives, wealth, or health that sentient beings, when their body breaks up, after death, are reborn in a good place, a heavenly realm. It is because of endowment with ethics or view that sentient beings, when their body breaks up, after death, are reborn in a good place, a heavenly realm. These are the five endowments.” 

%
\addtocontents{toc}{\let\protect\contentsline\protect\nopagecontentsline}
\chapter*{The Chapter on Kings }
\addcontentsline{toc}{chapter}{\tocchapterline{The Chapter on Kings }}
\addtocontents{toc}{\let\protect\contentsline\protect\oldcontentsline}

%
\section*{{\suttatitleacronym AN 5.131}{\suttatitletranslation Wielding Power (1st) }{\suttatitleroot Paṭhamacakkānuvattanasutta}}
\addcontentsline{toc}{section}{\tocacronym{AN 5.131} \toctranslation{Wielding Power (1st) } \tocroot{Paṭhamacakkānuvattanasutta}}
\markboth{Wielding Power (1st) }{Paṭhamacakkānuvattanasutta}
\extramarks{AN 5.131}{AN 5.131}

“Mendicants,\marginnote{1.1} possessing five factors a wheel-turning monarch wields power only in a principled manner. And this power cannot be undermined by any human enemy. 

What\marginnote{2.1} five? A wheel-turning monarch knows what is right, knows principle, knows moderation, knows the right time, and knows the assembly. A wheel-turning monarch who possesses these five factors wields power only in a principled manner. And this power cannot be undermined by any human enemy. 

In\marginnote{3.1} the same way, possessing five qualities a Realized One, a perfected one, a fully awakened Buddha rolls forth the supreme Wheel of Dhamma only in a principled manner. And that wheel cannot be rolled back by any ascetic or brahmin or god or \textsanskrit{Māra} or \textsanskrit{Brahmā} or by anyone in the world. 

What\marginnote{4.1} five? A Realized One knows what is right, knows principle, knows moderation, knows the right time, and knows the assembly. Possessing these five qualities a Realized One, a perfected one, a fully awakened Buddha rolls forth the supreme Wheel of Dhamma only in a principled manner. And that wheel cannot be rolled back by any ascetic or brahmin or god or \textsanskrit{Māra} or \textsanskrit{Brahmā} or by anyone in the world.” 

%
\section*{{\suttatitleacronym AN 5.132}{\suttatitletranslation Wielding Power (2nd) }{\suttatitleroot Dutiyacakkānuvattanasutta}}
\addcontentsline{toc}{section}{\tocacronym{AN 5.132} \toctranslation{Wielding Power (2nd) } \tocroot{Dutiyacakkānuvattanasutta}}
\markboth{Wielding Power (2nd) }{Dutiyacakkānuvattanasutta}
\extramarks{AN 5.132}{AN 5.132}

“Mendicants,\marginnote{1.1} possessing five factors a wheel-turning monarch’s eldest son continues to wield the power set in motion by his father only in a principled manner. And this power cannot be undermined by any human enemy. 

What\marginnote{2.1} five? A wheel-turning monarch’s oldest son knows what is right, knows principle, knows moderation, knows the right time, and knows the assembly. A wheel-turning monarch’s oldest son who possesses these five factors continues to wield the power set in motion by his father only in a principled manner. And this power cannot be undermined by any human enemy. 

In\marginnote{3.1} the same way, possessing five qualities \textsanskrit{Sāriputta} rightly keeps rolling the supreme Wheel of Dhamma that was rolled forth by the Realized One. And that wheel cannot be turned back by any ascetic or brahmin or god or \textsanskrit{Māra} or \textsanskrit{Brahmā} or by anyone in the world. 

What\marginnote{4.1} five? \textsanskrit{Sāriputta} knows what is right, knows principle, knows moderation, knows the right time, and knows the assembly. Possessing these five qualities \textsanskrit{Sāriputta} rightly keeps rolling the supreme Wheel of Dhamma that was rolled forth by the Realized One. And that wheel cannot be turned back by any ascetic or brahmin or god or \textsanskrit{Māra} or \textsanskrit{Brahmā} or by anyone in the world.” 

%
\section*{{\suttatitleacronym AN 5.133}{\suttatitletranslation A Principled King }{\suttatitleroot Dhammarājāsutta}}
\addcontentsline{toc}{section}{\tocacronym{AN 5.133} \toctranslation{A Principled King } \tocroot{Dhammarājāsutta}}
\markboth{A Principled King }{Dhammarājāsutta}
\extramarks{AN 5.133}{AN 5.133}

“Mendicants,\marginnote{1.1} even a wheel-turning monarch, a just and principled king, does not wield power without having their own king.” 

When\marginnote{1.2} he said this, one of the mendicants asked the Buddha, “But who is the king of the wheel-turning monarch, the just and principled king?” 

“It\marginnote{1.4} is principle, monk,” said the Buddha. 

“Monk,\marginnote{2.1} a wheel-turning monarch provides just protection and security for his court, relying only on principle—honoring, respecting, and venerating principle, having principle as his flag, banner, and authority. 

He\marginnote{3.1} provides just protection and security for his aristocrats, vassals, troops, brahmins and householders, people of town and country, ascetics and brahmins, beasts and birds. When he has done this, he wields power only in a principled manner. And this power cannot be undermined by any human enemy. 

In\marginnote{4.1} the same way, monk, a Realized One, a perfected one, a fully awakened Buddha, a just and principled king, provides just protection and security for the monks, relying only on principle—honoring, respecting, and venerating principle, having principle as his flag, banner, and authority. ‘This kind of bodily action should be cultivated. This kind of bodily action should not be cultivated. This kind of verbal action should be cultivated. This kind of verbal action should not be cultivated. This kind of mental action should be cultivated. This kind of mental action should not be cultivated. This kind of livelihood should be cultivated. This kind of livelihood should not be cultivated. This kind of market town should be cultivated. This kind of market town should not be cultivated.’ 

In\marginnote{5.1} the same way, monk, a Realized One, a perfected one, a fully awakened Buddha, a just and principled king, provides just protection and security for the nuns … laymen … laywomen, relying only on principle—honoring, respecting, and venerating principle, having principle as his flag, banner, and authority. ‘This kind of bodily action should be cultivated. This kind of bodily action should not be cultivated. This kind of verbal action should be cultivated. This kind of verbal action should not be cultivated. This kind of mental action should be cultivated. This kind of mental action should not be cultivated. This kind of livelihood should be cultivated. This kind of livelihood should not be cultivated. This kind of market town should be cultivated. This kind of market town should not be cultivated.’ 

When\marginnote{6.1} a Realized One, a perfected one, a fully awakened Buddha has provided just protection and security for the monks, nuns, laymen, and laywomen, he rolls forth the supreme Wheel of Dhamma only in a principled manner. And that wheel cannot be rolled back by any ascetic or brahmin or god or \textsanskrit{Māra} or \textsanskrit{Brahmā} or by anyone in the world.” 

%
\section*{{\suttatitleacronym AN 5.134}{\suttatitletranslation In Whatever Region }{\suttatitleroot Yassaṁdisaṁsutta}}
\addcontentsline{toc}{section}{\tocacronym{AN 5.134} \toctranslation{In Whatever Region } \tocroot{Yassaṁdisaṁsutta}}
\markboth{In Whatever Region }{Yassaṁdisaṁsutta}
\extramarks{AN 5.134}{AN 5.134}

“Mendicants,\marginnote{1.1} with five factors an anointed aristocratic king lives in his own realm, no matter what region he lives in. 

What\marginnote{2.1} five? 

An\marginnote{2.2} anointed aristocratic king is well born on both his mother’s and father’s side, of pure descent, irrefutable and impeccable in questions of ancestry back to the seventh paternal generation. 

He\marginnote{2.3} is rich, affluent, and wealthy, with a full treasury and storehouses. 

He\marginnote{2.4} is powerful, having an army of four divisions that is obedient and carries out instructions. 

He\marginnote{2.5} has a counselor who is astute, competent, and intelligent, able to think issues through as they bear upon the past, future, and present. 

These\marginnote{2.6} four things bring his fame to fruition. 

With\marginnote{2.7} these five qualities, including fame, an anointed aristocratic king lives in his own realm, no matter what direction he lives in. Why is that? Because that is how it is for victors. 

In\marginnote{3.1} the same way, a mendicant with five qualities lives with mind freed, no matter what region they live in. What five? 

It’s\marginnote{3.3} when mendicant is ethical, restrained in the monastic code, conducting themselves well and seeking alms in suitable places. Seeing danger in the slightest fault, they keep the rules they’ve undertaken. This is like the anointed aristocratic king’s impeccable lineage. 

They’re\marginnote{3.5} very learned, remembering and keeping what they’ve learned. These teachings are good in the beginning, good in the middle, and good in the end, meaningful and well-phrased, describing a spiritual practice that’s entirely full and pure. They are very learned in such teachings, remembering them, reinforcing them by recitation, mentally scrutinizing them, and comprehending them theoretically. This is like the anointed aristocratic king being rich, affluent, and wealthy, with full treasury and storehouses. 

They\marginnote{3.7} live with energy roused up for giving up unskillful qualities and embracing skillful qualities. They’re strong, staunchly vigorous, not slacking off when it comes to developing skillful qualities. This is like the anointed aristocratic king having power. 

They’re\marginnote{3.9} wise. They have the wisdom of arising and passing away which is noble, penetrative, and leads to the complete ending of suffering. This is like the anointed aristocratic king having a counselor. 

These\marginnote{3.11} four qualities bring their freedom to fruition. 

With\marginnote{3.12} these five qualities, including freedom, they live in their own realm, no matter what region they live in. Why is that? Because that is how it is for those whose mind is free.” 

%
\section*{{\suttatitleacronym AN 5.135}{\suttatitletranslation Aspiration (1st) }{\suttatitleroot Paṭhamapatthanāsutta}}
\addcontentsline{toc}{section}{\tocacronym{AN 5.135} \toctranslation{Aspiration (1st) } \tocroot{Paṭhamapatthanāsutta}}
\markboth{Aspiration (1st) }{Paṭhamapatthanāsutta}
\extramarks{AN 5.135}{AN 5.135}

“Mendicants,\marginnote{1.1} an anointed aristocratic king’s eldest son with five factors aspires to kingship. What five? 

It’s\marginnote{1.3} when an anointed aristocratic king’s eldest son is well born on both his mother’s and father’s side, of pure descent, irrefutable and impeccable in questions of ancestry back to the seventh paternal generation. 

He\marginnote{1.4} is attractive, good-looking, lovely, of surpassing beauty. 

He\marginnote{1.5} is dear and beloved to his parents. 

He\marginnote{1.6} is dear and beloved to the people of town and country. 

He\marginnote{1.7} is trained and skilled in the arts of anointed aristocratic kings, such as elephant riding, horse riding, driving a chariot, archery, and swordsmanship. 

He\marginnote{2.1} thinks: ‘I’m well born on both my mother’s and father’s side, of pure descent, irrefutable and impeccable in questions of ancestry back to the seventh paternal generation. Why shouldn’t I aspire to kingship? I’m attractive, good-looking, lovely, of surpassing beauty. Why shouldn’t I aspire to kingship? I’m dear and beloved to my parents. Why shouldn’t I aspire to kingship? I’m dear and beloved to the people of town and country. Why shouldn’t I aspire to kingship? I’m trained and skilled in the arts of anointed aristocratic kings, such as elephant riding, horse riding, driving a chariot, archery, and swordsmanship. Why shouldn’t I aspire to kingship?’ An anointed aristocratic king’s eldest son with these five factors aspires to kingship. 

In\marginnote{3.1} the same way, a mendicant with five qualities aspires to end the defilements. What five? 

It’s\marginnote{3.3} when a mendicant has faith in the Realized One’s awakening: ‘That Blessed One is perfected, a fully awakened Buddha, accomplished in knowledge and conduct, holy, knower of the world, supreme guide for those who wish to train, teacher of gods and humans, awakened, blessed.’ 

They\marginnote{3.5} are rarely ill or unwell. Their stomach digests well, being neither too hot nor too cold, but just right, and fit for meditation. 

They’re\marginnote{3.6} not devious or deceitful. They reveal themselves honestly to the Teacher or sensible spiritual companions. 

They\marginnote{3.7} live with energy roused up for giving up unskillful qualities and embracing skillful qualities. They’re strong, staunchly vigorous, not slacking off when it comes to developing skillful qualities. 

They’re\marginnote{3.8} wise. They have the wisdom of arising and passing away which is noble, penetrative, and leads to the complete ending of suffering. 

They\marginnote{4.1} think: ‘I am a person of faith; I have faith in the Realized One’s awakening … Why shouldn’t I aspire to end the defilements? I’m rarely ill or unwell. My stomach digests well, being neither too hot nor too cold, but just right, and fit for meditation. Why shouldn’t I aspire to end the defilements? I’m not devious or deceitful. I reveal myself honestly to the Teacher or sensible spiritual companions. Why shouldn’t I aspire to end the defilements? I live with energy roused up for giving up unskillful qualities and embracing skillful qualities. I’m strong, staunchly vigorous, not slacking off when it comes to developing skillful qualities. Why shouldn’t I aspire to end the defilements? I’m wise. I have the wisdom of arising and passing away which is noble, penetrative, and leads to the complete ending of suffering. Why shouldn’t I aspire to end the defilements?’ 

A\marginnote{4.13} mendicant with these five qualities aspires to end the defilements.” 

%
\section*{{\suttatitleacronym AN 5.136}{\suttatitletranslation Aspiration (2nd) }{\suttatitleroot Dutiyapatthanāsutta}}
\addcontentsline{toc}{section}{\tocacronym{AN 5.136} \toctranslation{Aspiration (2nd) } \tocroot{Dutiyapatthanāsutta}}
\markboth{Aspiration (2nd) }{Dutiyapatthanāsutta}
\extramarks{AN 5.136}{AN 5.136}

“Mendicants,\marginnote{1.1} an anointed aristocratic king’s eldest son with five factors aspires to become a viceroy. What five? 

It’s\marginnote{1.3} when an anointed aristocratic king’s eldest son is well born on both his mother’s and father’s side, of pure descent, irrefutable and impeccable in questions of ancestry back to the seventh paternal generation. 

He\marginnote{1.4} is attractive, good-looking, lovely, of surpassing beauty. 

He\marginnote{1.5} is dear and beloved to his parents. 

He\marginnote{1.6} is dear and beloved to the armed forces. 

He\marginnote{1.7} is astute, competent, and intelligent, able to think issues through as they bear upon the past, future, and present. 

He\marginnote{2.1} thinks: ‘I’m well born … attractive … dear and beloved to my parents … dear and beloved to the armed forces … I’m astute, competent, and intelligent, able to think issues through as they bear upon the past, future, and present. Why shouldn’t I aspire to become a viceroy?’ An anointed aristocratic king’s eldest son with these five factors aspires to become a viceroy. 

In\marginnote{3.1} the same way, a mendicant with five qualities aspires to end the defilements. What five? 

It’s\marginnote{3.3} when a mendicant is ethical, restrained in the code of conduct, conducting themselves well and seeking alms in suitable places. Seeing danger in the slightest fault, they keep the rules they’ve undertaken. 

They’re\marginnote{3.4} very learned, remembering and keeping what they’ve learned. These teachings are good in the beginning, good in the middle, and good in the end, meaningful and well-phrased, describing a spiritual practice that’s totally full and pure. They are very learned in such teachings, remembering them, reciting them, mentally scrutinizing them, and comprehending them theoretically. 

Their\marginnote{3.5} mind is firmly established in the four kinds of mindfulness meditation. 

They\marginnote{3.6} live with energy roused up for giving up unskillful qualities and embracing skillful qualities. They’re strong, staunchly vigorous, not slacking off when it comes to developing skillful qualities. 

They’re\marginnote{3.7} wise. They have the wisdom of arising and passing away which is noble, penetrative, and leads to the complete ending of suffering. 

They\marginnote{4.1} think: ‘I’m ethical … learned … mindful … energetic … wise. I have the wisdom of arising and passing away which is noble, penetrative, and leads to the complete ending of suffering. Why shouldn’t I aspire to end the defilements?’ A mendicant with these five qualities aspires to end the defilements.” 

%
\section*{{\suttatitleacronym AN 5.137}{\suttatitletranslation Little Sleep }{\suttatitleroot Appaṁsupatisutta}}
\addcontentsline{toc}{section}{\tocacronym{AN 5.137} \toctranslation{Little Sleep } \tocroot{Appaṁsupatisutta}}
\markboth{Little Sleep }{Appaṁsupatisutta}
\extramarks{AN 5.137}{AN 5.137}

“Mendicants,\marginnote{1.1} these five sleep little at night, staying mostly awake. What five? A woman longing for a man. A man longing for a woman. A thief longing for their loot. A king busy with his duties. A mendicant longing for freedom from attachment. These five sleep little at night, staying mostly awake.” 

%
\section*{{\suttatitleacronym AN 5.138}{\suttatitletranslation Eating Food }{\suttatitleroot Bhattādakasutta}}
\addcontentsline{toc}{section}{\tocacronym{AN 5.138} \toctranslation{Eating Food } \tocroot{Bhattādakasutta}}
\markboth{Eating Food }{Bhattādakasutta}
\extramarks{AN 5.138}{AN 5.138}

“Mendicants,\marginnote{1.1} a royal bull elephant with five factors eats food, takes up space, drops dung, and takes a ticket, yet is still considered to be a royal bull elephant. What five? It’s when a royal bull elephant can’t endure sights, sounds, smells, tastes, and touches. A royal bull elephant with these five factors eats food, takes up space, drops dung, and takes a ticket, yet is still considered to be a royal bull elephant. 

In\marginnote{2.1} the same way, a mendicant with five qualities eats food, takes up space, tramples beds and chairs, and takes a ticket, yet is still considered to be a mendicant. What five? It’s when a mendicant can’t endure sights, sounds, smells, tastes, and touches. A mendicant with these five qualities eats food, takes up space, tramples beds and chairs, and takes a ticket, yet is still considered to be a mendicant.” 

%
\section*{{\suttatitleacronym AN 5.139}{\suttatitletranslation Cannot Endure }{\suttatitleroot Akkhamasutta}}
\addcontentsline{toc}{section}{\tocacronym{AN 5.139} \toctranslation{Cannot Endure } \tocroot{Akkhamasutta}}
\markboth{Cannot Endure }{Akkhamasutta}
\extramarks{AN 5.139}{AN 5.139}

“Mendicants,\marginnote{1.1} a royal bull elephant with five factors is not worthy of a king, not fit to serve a king, and is not considered a factor of kingship. What five? It’s when a royal bull elephant can’t endure sights, sounds, smells, tastes, and touches. 

And\marginnote{2.1} how is it that a royal bull elephant can’t endure sights? It’s when a royal bull elephant gone to battle falters and founders at the sight of a division of elephants, of cavalry, of chariots, or of infantry. It doesn’t stay firm, and fails to plunge into battle. That’s how a royal bull elephant can’t endure sights. 

And\marginnote{3.1} how is it that a royal bull elephant can’t endure sounds? It’s when a royal bull elephant gone to battle falters and founders at the sound of a division of elephants, of cavalry, of chariots, or of infantry, or the thunder of the drums, kettledrums, horns, and cymbals. It doesn’t stay firm, and fails to plunge into battle. That’s how a royal bull elephant can’t endure sounds. 

And\marginnote{4.1} how is it that a royal bull elephant can’t endure smells? It’s when a royal bull elephant gone to battle falters and founders when it smells the odor of the feces and urine of battle-hardened, pedigree royal bull elephants. It doesn’t stay firm, and fails to plunge into battle. That’s how a royal bull elephant can’t endure smells. 

And\marginnote{5.1} how is it that a royal bull elephant can’t endure tastes? It’s when a royal bull elephant gone to battle falters and founders when it misses a meal of grass and water, or it misses two, three, four, or five meals. It doesn’t stay firm, and fails to plunge into battle. That’s how a royal bull elephant can’t endure tastes. 

And\marginnote{6.1} how is it that a royal bull elephant can’t endure touches? It’s when a royal bull elephant gone to battle falters and founders when struck by a swift arrow, or by two, three, four, or five swift arrows. It doesn’t stay firm, and fails to plunge into battle. That’s how a royal bull elephant can’t endure touches. 

A\marginnote{7.1} royal bull elephant with these five factors is not worthy of a king, not fit to serve a king, and is not considered a factor of kingship. 

In\marginnote{8.1} the same way, a mendicant with five qualities is not worthy of offerings dedicated to the gods, not worthy of hospitality, not worthy of a religious donation, not worthy of veneration with joined palms, and is not the supreme field of merit for the world. What five? It’s when a mendicant can’t endure sights, sounds, smells, tastes, and touches. 

And\marginnote{9.1} how is it that a mendicant can’t endure sights? It’s when a mendicant, seeing a sight with their eyes, is aroused by a desirable sight, so is not able to still the mind. That’s how a mendicant can’t endure sights. 

And\marginnote{10.1} how is it that a mendicant can’t endure sounds? It’s when a mendicant, hearing a sound with their ears, is aroused by a desirable sound, so is not able to still the mind. That’s how a mendicant can’t endure sounds. 

And\marginnote{11.1} how is it that a mendicant can’t endure smells? It’s when a mendicant, smelling an odor with their nose, is aroused by a desirable smell, so is not able to still the mind. That’s how a mendicant can’t endure smells. 

And\marginnote{12.1} how is it that a mendicant can’t endure tastes? It’s when a mendicant, tasting a flavor with their tongue, is aroused by desirable tastes, so is not able to still the mind. That’s how a mendicant can’t endure tastes. 

And\marginnote{13.1} how is it that a mendicant can’t endure touches? It’s when a mendicant, feeling a touch with their body, is aroused by a desirable touch, so is not able to still the mind. That’s how a mendicant can’t endure touches. 

A\marginnote{14.1} mendicant with these five qualities is not worthy of offerings dedicated to the gods, not worthy of hospitality, not worthy of a religious donation, not worthy of veneration with joined palms, and is not the supreme field of merit for the world. 

A\marginnote{15.1} royal bull elephant with five factors is worthy of a king, fit to serve a king, and is considered a factor of kingship. What five? It’s when a royal bull elephant can endure sights, sounds, smells, tastes, and touches. 

And\marginnote{16.1} how is it that a royal bull elephant can endure sights? It’s when a royal bull elephant gone to battle does not falter or founder at the sight of a division of elephants, of cavalry, of chariots, or of infantry. It stays firm, and plunges into battle. That’s how a royal bull elephant can endure sights. 

And\marginnote{17.1} how is it that a royal bull elephant can endure sounds? It’s when a royal bull elephant does not falter or founder at the sound of a division of elephants, of cavalry, of chariots, or of infantry, or the thunder of the drums, kettledrums, horns, and cymbals. It stays firm, and plunges into battle. That’s how a royal bull elephant can endure sounds. 

And\marginnote{18.1} how is it that a royal bull elephant can endure smells? It’s when a royal bull elephant gone to battle does not falter or founder when it smells the odor of the feces and urine of battle-hardened, pedigree royal bull elephants. It stays firm, and plunges into battle. That’s how a royal bull elephant can endure smells. 

And\marginnote{19.1} how is it that a royal bull elephant can endure tastes? It’s when a royal bull elephant gone to battle does not falter or founder when it misses a meal of grass and water, or it misses two, three, four, or five meals. It stays firm, and plunges into battle. That’s how a royal bull elephant can endure tastes. 

And\marginnote{20.1} how is it that a royal bull elephant can endure touches? It’s when a royal bull elephant gone to battle does not falter or founder when struck by a swift arrow, or by two, three, four, or five swift arrows. It stays firm, and plunges into battle. That’s how a royal bull elephant can endure touches. 

A\marginnote{21.1} royal bull elephant with these five factors is worthy of a king, fit to serve a king, and is considered a factor of kingship. 

In\marginnote{22.1} the same way, a mendicant with five qualities is worthy of offerings dedicated to the gods, worthy of hospitality, worthy of a religious donation, worthy of veneration with joined palms, and is the supreme field of merit for the world. What five? It’s when a mendicant can endure sights, sounds, smells, tastes, and touches. 

And\marginnote{23.1} how is it that a mendicant can endure sights? It’s when a mendicant, seeing a sight with their eyes, is not aroused by a desirable sight, so is able to still the mind. That’s how a mendicant can endure sights. 

And\marginnote{24.1} how is it that a mendicant can endure sounds? It’s when a mendicant, hearing a sound with their ears, is not aroused by desirable sounds, so is able to still the mind. That’s how a mendicant can endure sounds. 

And\marginnote{25.1} how is it that a mendicant can endure smells? It’s when a mendicant, smelling an odor with their nose, is not aroused by a desirable smell, so is able to still the mind. That’s how a mendicant can endure smells. 

And\marginnote{26.1} how is it that a mendicant can endure tastes? It’s when a mendicant, tasting a flavor with their tongue, is not aroused by desirable tastes, so is able to still the mind. That’s how a mendicant can endure tastes. 

And\marginnote{27.1} how is it that a mendicant can endure touches? It’s when a mendicant, feeling a touch with their body, is not aroused by a desirable touch, so is able to still the mind. That’s how a mendicant can endure touches. 

A\marginnote{28.1} mendicant with these five qualities is worthy of offerings dedicated to the gods, worthy of hospitality, worthy of a religious donation, worthy of veneration with joined palms, and is the supreme field of merit for the world.” 

%
\section*{{\suttatitleacronym AN 5.140}{\suttatitletranslation A Listener }{\suttatitleroot Sotasutta}}
\addcontentsline{toc}{section}{\tocacronym{AN 5.140} \toctranslation{A Listener } \tocroot{Sotasutta}}
\markboth{A Listener }{Sotasutta}
\extramarks{AN 5.140}{AN 5.140}

“Mendicants,\marginnote{1.1} a royal bull elephant with five factors is worthy of a king, fit to serve a king, and is considered a factor of kingship. What five? A royal bull elephant listens, destroys, protects, endures, and goes fast. 

And\marginnote{2.1} how does a royal bull elephant listen? It’s when a royal bull elephant pays heed, pays attention, engages wholeheartedly, and lends an ear to whatever task the elephant trainer has it do, whether or not it has done it before. That’s how a royal bull elephant listens. 

And\marginnote{3.1} how does a royal bull elephant destroy? It’s when a royal bull elephant in battle destroys elephants with their riders, horses with their riders, chariots and charioteers, and foot soldiers. That’s how a royal bull elephant destroys. 

And\marginnote{4.1} how does a royal bull elephant protect? It’s when a royal bull elephant in battle protects its fore-quarters and hind-quarters, its fore-feet and hind-feet, and its head, ears, tusks, trunk, tail, and rider. That’s how a royal bull elephant protects. 

And\marginnote{5.1} how does a royal bull elephant endure? It’s when a royal bull elephant in battle endures being struck by spears, swords, arrows, and axes; it endures the thunder of the drums, kettledrums, horns, and cymbals. That’s how a royal bull elephant endures. 

And\marginnote{6.1} how does a royal bull elephant go fast? It’s when a royal bull elephant swiftly goes in whatever direction the elephant trainer sends it, whether or not it has been there before. That’s how a royal bull elephant goes fast. 

A\marginnote{7.1} royal bull elephant with these five factors is worthy of a king, fit to serve a king, and is considered a factor of kingship. 

In\marginnote{8.1} the same way, a mendicant with five qualities is worthy of offerings dedicated to the gods, worthy of hospitality, worthy of a religious donation, worthy of veneration with joined palms, and is the supreme field of merit for the world. What five? A mendicant listens, destroys, protects, endures, and goes fast. 

And\marginnote{9.1} how does a mendicant listen? It’s when a mendicant pays heed, pays attention, engages wholeheartedly, and lends an ear when the teaching and training proclaimed by a Realized One is being taught. That’s how a mendicant listens. 

And\marginnote{10.1} how does a mendicant destroy? It’s when a mendicant doesn’t tolerate a sensual, malicious, or cruel thought. They don’t tolerate any bad, unskillful qualities that have arisen, but give them up, get rid of them, calm them, eliminate them, and obliterate them. That’s how a mendicant destroys. 

And\marginnote{11.1} how does a mendicant protect? When a mendicant sees a sight with the eyes, they don’t get caught up in the features and details. If the faculty of sight were left unrestrained, bad unskillful qualities of desire and aversion would become overwhelming. For this reason, they practice restraint, protecting the faculty of sight, and achieving restraint over it. Hearing a sound with the ears … Smelling an odor with the nose … Tasting a flavor with the tongue … Feeling a touch with the body … Knowing a thought with the mind, they don’t get caught up in the features and details. If the faculty of mind were left unrestrained, bad unskillful qualities of desire and aversion would become overwhelming. For this reason, they practice restraint, protecting the faculty of mind, and achieving restraint over it. That’s how a mendicant protects. 

And\marginnote{12.1} how does a mendicant endure? It’s when a mendicant endures cold, heat, hunger, and thirst; the touch of flies, mosquitoes, wind, sun, and reptiles; rude and unwelcome criticism; and puts up with physical pain—sharp, severe, acute, unpleasant, disagreeable, and life-threatening. That’s how a mendicant endures. 

And\marginnote{13.1} how does a mendicant go fast? It’s when a mendicant swiftly goes in the direction they’ve never gone before in all this long time; that is, the stilling of all activities, the letting go of all attachments, the ending of craving, fading away, cessation, extinguishment. That’s how a mendicant goes fast. 

A\marginnote{14.1} mendicant with these five qualities … is the supreme field of merit for the world.” 

%
\addtocontents{toc}{\let\protect\contentsline\protect\nopagecontentsline}
\chapter*{The Chapter at Tikaṇḍakī }
\addcontentsline{toc}{chapter}{\tocchapterline{The Chapter at Tikaṇḍakī }}
\addtocontents{toc}{\let\protect\contentsline\protect\oldcontentsline}

%
\section*{{\suttatitleacronym AN 5.141}{\suttatitletranslation Scorn }{\suttatitleroot Avajānātisutta}}
\addcontentsline{toc}{section}{\tocacronym{AN 5.141} \toctranslation{Scorn } \tocroot{Avajānātisutta}}
\markboth{Scorn }{Avajānātisutta}
\extramarks{AN 5.141}{AN 5.141}

“Mendicants,\marginnote{1.1} these five people are found in the world. What five? One gives then scorns, one lives together then scorns, one is gullible for gossip, one is impulsive, and one is dull and stupid. 

And\marginnote{2.1} how does a person give then scorn? It’s when a person gives someone robes, almsfood, lodgings, and medicines and supplies for the sick. They think: ‘I give; this one receives.’ They give to that person, then they scorn them. That’s how a person gives then scorns. 

And\marginnote{3.1} how does a person live together then scorn? It’s when a person lives with someone else for two or three years. They live together with that person, then they scorn them. That’s how a person live together then scorns. 

And\marginnote{4.1} how is a person gullible for gossip? It’s when they’re very quick to believe what a certain person says in praise or criticism of another. That’s how a person is gullible for gossip. 

And\marginnote{5.1} how is a person impulsive? It’s when a certain person is fickle in faith, devotion, fondness, and confidence. That’s how a person is impulsive. 

And\marginnote{6.1} how is a person dull and stupid? It’s when they don’t know the difference between qualities that are skillful and unskillful, blameworthy and blameless, inferior and superior, and those on the side of dark and bright. That’s how a person is dull and stupid. 

These\marginnote{6.4} are the five people found in the world.” 

%
\section*{{\suttatitleacronym AN 5.142}{\suttatitletranslation Transgression }{\suttatitleroot Ārabhatisutta}}
\addcontentsline{toc}{section}{\tocacronym{AN 5.142} \toctranslation{Transgression } \tocroot{Ārabhatisutta}}
\markboth{Transgression }{Ārabhatisutta}
\extramarks{AN 5.142}{AN 5.142}

“Mendicants,\marginnote{1.1} these five people are found in the world. What five? 

One\marginnote{1.3} person transgresses and regrets it. And they don’t truly understand the freedom of heart and freedom by wisdom where those arisen bad, unskillful qualities cease without anything left over. 

One\marginnote{2.1} person transgresses and doesn’t regret it. And they don’t understand … 

One\marginnote{3.1} person doesn’t transgress yet still feels regret. And they don’t understand … 

One\marginnote{4.1} person neither transgresses nor regrets. But they don’t understand … 

One\marginnote{5.1} person neither transgresses nor regrets. And they do understand where those arisen bad, unskillful qualities cease without anything left over. 

Take\marginnote{6.1} the case of the person who transgresses and regrets it. And they don’t truly understand the freedom of heart and freedom by wisdom where those arisen bad, unskillful qualities cease without anything left over. They should be told: ‘Venerable, the defilements born of transgression are found in you, and the defilements born of regret grow. You would do well to give up the defilements born of transgression and get rid of the defilements born of regret, and then develop the mind and wisdom. In this way you’ll become just like the fifth person.’ 

Take\marginnote{7.1} the case of the person who transgresses and doesn’t regret it. And they don’t understand… They should be told: ‘Venerable, the defilements born of transgression are found in you, but the defilements born of regret don’t grow. You would do well to give up the defilements born of transgression, and then develop the mind and wisdom. In this way you’ll become just like the fifth person.’ 

Take\marginnote{8.1} the case of the person who doesn’t transgress yet feels regret. And they don’t understand… They should be told: ‘Venerable, the defilements born of transgression are not found in you, yet the defilements born of regret grow. You would do well to get rid of the defilements born of regret, and then develop the mind and wisdom. In this way you’ll become just like the fifth person.’ 

Take\marginnote{9.1} the case of the person who neither transgresses nor regrets. But they don’t understand… They should be told: ‘Venerable, the defilements born of transgression are not found in you, and the defilements born of regret don’t grow. You would do well to develop the mind and wisdom. In this way you’ll become just like the fifth person.’ 

And\marginnote{10.1} so, mendicants, when these four people are advised and instructed by comparison with the fifth person, they gradually attain the ending of defilements.” 

%
\section*{{\suttatitleacronym AN 5.143}{\suttatitletranslation At Sārandada }{\suttatitleroot Sārandadasutta}}
\addcontentsline{toc}{section}{\tocacronym{AN 5.143} \toctranslation{At Sārandada } \tocroot{Sārandadasutta}}
\markboth{At Sārandada }{Sārandadasutta}
\extramarks{AN 5.143}{AN 5.143}

At\marginnote{1.1} one time the Buddha was staying near \textsanskrit{Vesālī}, at the Great Wood, in the hall with the peaked roof. 

Then\marginnote{1.2} the Buddha robed up in the morning and, taking his bowl and robe, entered \textsanskrit{Vesālī} for alms. 

Now\marginnote{1.3} at that time around five hundred Licchavis were sitting together at the \textsanskrit{Sārandada} shrine, and this discussion came up among them, “The appearance of five treasures is rare in the world. What five? The elephant-treasure, the horse-treasure, the jewel-treasure, the woman-treasure, and the householder-treasure. The appearance of these five treasures is rare in the world.” 

Then\marginnote{2.1} those Licchavis sent a man out on to the road, saying, “Mister, please tell us when you see the Buddha.” 

That\marginnote{2.3} man saw the Buddha coming off in the distance. He went to the Licchavis and said, “Sirs, the Blessed One, the perfected one, the fully awakened Buddha is coming. Please go at your convenience.” 

Then\marginnote{3.1} those Licchavis went up to the Buddha, bowed, stood to one side, and said to him, “Please go to the \textsanskrit{Sārandada} shrine, out of compassion.” The Buddha consented in silence. 

Then\marginnote{4.3} the Buddha went up to the \textsanskrit{Sārandada} shrine, where he sat on the seat spread out, and said to the Licchavis, “Licchavis, what were you sitting talking about just now? What conversation was left unfinished?” 

“Well,\marginnote{4.5} Master Gotama, this discussion came up among us while we were sitting together: ‘The appearance of five treasures is rare in the world. …’” 

“You\marginnote{5.1} Licchavis are so fixated on sensual pleasures, that’s the only discussion that came up! Licchavis, the appearance of five treasures is rare in the world. What five? 

The\marginnote{5.4} appearance of a Realized One, a perfected one, a fully awakened Buddha. 

A\marginnote{5.5} person who explains the teaching and training proclaimed by a Realized One. 

A\marginnote{5.6} person who understands the teaching and training proclaimed by a Realized One. 

A\marginnote{5.7} person who practices in line with the teaching. 

A\marginnote{5.8} person who is grateful and thankful. 

The\marginnote{5.9} appearance of these five treasures is rare in the world.” 

%
\section*{{\suttatitleacronym AN 5.144}{\suttatitletranslation At Tikaṇḍakī }{\suttatitleroot Tikaṇḍakīsutta}}
\addcontentsline{toc}{section}{\tocacronym{AN 5.144} \toctranslation{At Tikaṇḍakī } \tocroot{Tikaṇḍakīsutta}}
\markboth{At Tikaṇḍakī }{Tikaṇḍakīsutta}
\extramarks{AN 5.144}{AN 5.144}

At\marginnote{1.1} one time the Buddha was staying near \textsanskrit{Sāketa}, in \textsanskrit{Tikaṇḍakī} Wood. There the Buddha addressed the mendicants, “Mendicants!” 

“Venerable\marginnote{1.4} sir,” they replied. The Buddha said this: 

“Mendicants,\marginnote{2.1} a mendicant would do well to meditate from time to time perceiving the following: the repulsive in the unrepulsive, 

\begin{enumerate}%
\item the unrepulsive in the repulsive, %
\item the repulsive in both the unrepulsive and the repulsive, and %
\item the unrepulsive in both the repulsive and the unrepulsive. %
\item A mendicant would do well to meditate from time to time staying equanimous, mindful and aware, rejecting both the repulsive and the unrepulsive. %
\end{enumerate}

For\marginnote{3.1} what reason should a mendicant meditate perceiving the repulsive in the unrepulsive? ‘May greed not arise in me for things that arouse greed.’ A mendicant should meditate perceiving the repulsive in the unrepulsive for this reason. 

For\marginnote{4.1} what reason should a mendicant meditate perceiving the unrepulsive in the repulsive? ‘May hate not arise in me for things that provoke hate.’ … 

For\marginnote{5.1} what reason should a mendicant meditate perceiving the repulsive in both the unrepulsive and the repulsive? ‘May greed not arise in me for things that arouse greed. May hate not arise in me for things that provoke hate.’ … 

For\marginnote{6.1} what reason should a mendicant meditate perceiving the unrepulsive in both the repulsive and the unrepulsive? ‘May hate not arise in me for things that provoke hate. May greed not arise in me for things that arouse greed.’ … 

For\marginnote{7.1} what reason should a mendicant meditate staying equanimous, mindful and aware, rejecting both the repulsive and the unrepulsive? ‘May no greed for things that arouse greed, hate for things that provoke hate, or delusion for things that promote delusion arise in me in any way at all.’ For this reason a mendicant should meditate staying equanimous, mindful and aware, rejecting both the repulsive and the unrepulsive.” 

%
\section*{{\suttatitleacronym AN 5.145}{\suttatitletranslation Hell }{\suttatitleroot Nirayasutta}}
\addcontentsline{toc}{section}{\tocacronym{AN 5.145} \toctranslation{Hell } \tocroot{Nirayasutta}}
\markboth{Hell }{Nirayasutta}
\extramarks{AN 5.145}{AN 5.145}

“Mendicants,\marginnote{1.1} someone with five qualities is cast down to hell. What five? They kill living creatures, steal, commit sexual misconduct, lie, and use alcoholic drinks that cause negligence. Someone with these five qualities is cast down to hell. 

Someone\marginnote{2.1} with five qualities is raised up to heaven What five? They don’t kill living creatures, steal, commit sexual misconduct, lie, or use alcoholic drinks that cause negligence. Someone with these five qualities is raised up to heaven.” 

%
\section*{{\suttatitleacronym AN 5.146}{\suttatitletranslation A Friend }{\suttatitleroot Mittasutta}}
\addcontentsline{toc}{section}{\tocacronym{AN 5.146} \toctranslation{A Friend } \tocroot{Mittasutta}}
\markboth{A Friend }{Mittasutta}
\extramarks{AN 5.146}{AN 5.146}

“Mendicants,\marginnote{1.1} you shouldn’t associate with a mendicant friend who has five qualities. What five? They start up work projects. They take up disciplinary issues. They conflict with leading mendicants. They like long and aimless wandering. They’re unable to educate, encourage, fire up, and inspire you from time to time with a Dhamma talk. Mendicants, you shouldn’t associate with a mendicant friend who has these five qualities. 

You\marginnote{2.1} should associate with a mendicant friend who has five qualities. What five? They don’t start up work projects. They don’t take up disciplinary issues. They don’t conflict with leading mendicants. They don’t like long and aimless wandering. They’re able to educate, encourage, fire up, and inspire you from time to time with a Dhamma talk. You should associate with a mendicant friend who has these five qualities.” 

%
\section*{{\suttatitleacronym AN 5.147}{\suttatitletranslation Gifts of a Bad Person }{\suttatitleroot Asappurisadānasutta}}
\addcontentsline{toc}{section}{\tocacronym{AN 5.147} \toctranslation{Gifts of a Bad Person } \tocroot{Asappurisadānasutta}}
\markboth{Gifts of a Bad Person }{Asappurisadānasutta}
\extramarks{AN 5.147}{AN 5.147}

“Mendicants,\marginnote{1.1} there are these five gifts of a bad person. What five? They give carelessly. They give thoughtlessly. They don’t give with their own hand. They give the dregs. They give without consideration for consequences. These are the five gifts of a bad person. 

There\marginnote{2.1} are these five gifts of a good person. What five? They give carefully. They give thoughtfully. They give with their own hand. They don’t give the dregs. They give with consideration for consequences. These are the five gifts of a good person.” 

%
\section*{{\suttatitleacronym AN 5.148}{\suttatitletranslation Gifts of a Good Person }{\suttatitleroot Sappurisadānasutta}}
\addcontentsline{toc}{section}{\tocacronym{AN 5.148} \toctranslation{Gifts of a Good Person } \tocroot{Sappurisadānasutta}}
\markboth{Gifts of a Good Person }{Sappurisadānasutta}
\extramarks{AN 5.148}{AN 5.148}

“There\marginnote{1.1} are these five gifts of a good person. What five? They give a gift out of faith. They give a gift carefully. They give a gift at the right time. They give a gift with no strings attached. They give a gift without hurting themselves or others. 

Having\marginnote{2.1} given a gift out of faith, in whatever place the result of that gift manifests they become rich, affluent, and wealthy. And they’re attractive, good-looking, lovely, of surpassing beauty. 

Having\marginnote{3.1} given a gift carefully, in whatever place the result of that gift manifests they become rich, affluent, and wealthy. And their children, wives, bondservants, workers, and staff want to listen. They pay attention and try to understand. 

Having\marginnote{4.1} given a gift at the right time, in whatever place the result of that gift manifests they become rich, affluent, and wealthy. And when the time is right, they get all that they need. 

Having\marginnote{5.1} given a gift with no strings attached, in whatever place the result of that gift manifests they become rich, affluent, and wealthy. And their mind tends to enjoy the five refined kinds of sensual stimulation. 

Having\marginnote{6.1} given a gift without hurting themselves or others, in whatever place the result of that gift manifests they become rich, affluent, and wealthy. And no damage comes to their property from anywhere, whether fire, flood, rulers, bandits, or unloved heirs. 

These\marginnote{6.3} are the five gifts of a good person.” 

%
\section*{{\suttatitleacronym AN 5.149}{\suttatitletranslation Temporarily Free (1st) }{\suttatitleroot Paṭhamasamayavimuttasutta}}
\addcontentsline{toc}{section}{\tocacronym{AN 5.149} \toctranslation{Temporarily Free (1st) } \tocroot{Paṭhamasamayavimuttasutta}}
\markboth{Temporarily Free (1st) }{Paṭhamasamayavimuttasutta}
\extramarks{AN 5.149}{AN 5.149}

“Mendicants,\marginnote{1.1} these five things lead to the decline of a mendicant who is temporarily free. What five? They relish work, talk, sleep, and company. And they don’t review the extent of their mind’s freedom. These five things lead to the decline of a mendicant who is temporarily free. 

These\marginnote{2.1} five things don’t lead to the decline of a mendicant who is temporarily free. What five? They don’t relish work, talk, sleep, and company. And they review the extent of their mind’s freedom. These five things don’t lead to the decline of a mendicant who is temporarily free.” 

%
\section*{{\suttatitleacronym AN 5.150}{\suttatitletranslation Temporarily Free (2nd) }{\suttatitleroot Dutiyasamayavimuttasutta}}
\addcontentsline{toc}{section}{\tocacronym{AN 5.150} \toctranslation{Temporarily Free (2nd) } \tocroot{Dutiyasamayavimuttasutta}}
\markboth{Temporarily Free (2nd) }{Dutiyasamayavimuttasutta}
\extramarks{AN 5.150}{AN 5.150}

“Mendicants,\marginnote{1.1} these five things lead to the decline of a mendicant who is temporarily free. What five? They relish work, talk, and sleep. They don’t guard the sense doors and they eat too much. These five things lead to the decline of a mendicant who is temporarily free. 

These\marginnote{2.1} five things don’t lead to the decline of a mendicant who is temporarily free. What five? They don’t relish work, talk, sleep, and company. They guard the sense doors and they have moderation in eating. These five things don’t lead to the decline of a mendicant who is temporarily free.” 

%
\addtocontents{toc}{\let\protect\contentsline\protect\nopagecontentsline}
\pannasa{The Fourth Fifty }
\addcontentsline{toc}{pannasa}{The Fourth Fifty }
\markboth{}{}
\addtocontents{toc}{\let\protect\contentsline\protect\oldcontentsline}

%
\addtocontents{toc}{\let\protect\contentsline\protect\nopagecontentsline}
\chapter*{The Chapter on the True Teaching }
\addcontentsline{toc}{chapter}{\tocchapterline{The Chapter on the True Teaching }}
\addtocontents{toc}{\let\protect\contentsline\protect\oldcontentsline}

%
\section*{{\suttatitleacronym AN 5.151}{\suttatitletranslation Inevitability Regarding the Right Path (1st) }{\suttatitleroot Paṭhamasammattaniyāmasutta}}
\addcontentsline{toc}{section}{\tocacronym{AN 5.151} \toctranslation{Inevitability Regarding the Right Path (1st) } \tocroot{Paṭhamasammattaniyāmasutta}}
\markboth{Inevitability Regarding the Right Path (1st) }{Paṭhamasammattaniyāmasutta}
\extramarks{AN 5.151}{AN 5.151}

“Mendicants,\marginnote{1.1} someone with five qualities is unable to enter the sure path with regards to skillful qualities even when listening to the true teaching. What five? They disparage the talk, the speaker, or themselves. They listen with distracted and scattered mind. They attend improperly. Someone with these five qualities is unable to enter the sure path with regards to skillful qualities, even when listening to the true teaching. 

Someone\marginnote{2.1} with five qualities is able to enter the sure path with regards to skillful qualities when listening to the true teaching. What five? They don’t disparage the talk, the speaker, or themselves. They listen with undistracted and unified mind. They attend properly. Someone with these five qualities is able to enter the sure path with regards to skillful qualities when listening to the true teaching.” 

%
\section*{{\suttatitleacronym AN 5.152}{\suttatitletranslation Inevitability Regarding the Right Path (2nd) }{\suttatitleroot Dutiyasammattaniyāmasutta}}
\addcontentsline{toc}{section}{\tocacronym{AN 5.152} \toctranslation{Inevitability Regarding the Right Path (2nd) } \tocroot{Dutiyasammattaniyāmasutta}}
\markboth{Inevitability Regarding the Right Path (2nd) }{Dutiyasammattaniyāmasutta}
\extramarks{AN 5.152}{AN 5.152}

“Mendicants,\marginnote{1.1} someone with five qualities is unable to enter the sure path with regards to skillful qualities even when listening to the true teaching. What five? They disparage the talk, the speaker, or themselves. They’re witless, dull, and stupid. They think they know what they don’t know. Someone with these five qualities is unable to enter the sure path with regards to skillful qualities, even when listening to the true teaching. 

Someone\marginnote{2.1} with five qualities is able to enter the sure path with regards to skillful qualities when listening to the true teaching. What five? They don’t disparage the talk, the speaker, or themselves. They’re wise, bright, and clever. They don’t think they know what they don’t know. Someone with these five qualities is able to enter the sure path with regards to skillful qualities when listening to the true teaching.” 

%
\section*{{\suttatitleacronym AN 5.153}{\suttatitletranslation Inevitability Regarding the Right Path (3rd) }{\suttatitleroot Tatiyasammattaniyāmasutta}}
\addcontentsline{toc}{section}{\tocacronym{AN 5.153} \toctranslation{Inevitability Regarding the Right Path (3rd) } \tocroot{Tatiyasammattaniyāmasutta}}
\markboth{Inevitability Regarding the Right Path (3rd) }{Tatiyasammattaniyāmasutta}
\extramarks{AN 5.153}{AN 5.153}

“Mendicants,\marginnote{1.1} someone with five qualities is unable to enter the sure path with regards to skillful qualities even when listening to the true teaching. What five? They listen to the teaching bent only on putting it down. They listen to the teaching with a hostile, fault-finding mind. They’re antagonistic to the teacher, planning to attack them. They’re witless, dull, and stupid. And they think they know what they don’t know. Someone with these five qualities is unable to enter the sure path with regards to skillful qualities even when listening to the true teaching. 

Someone\marginnote{2.1} with five qualities is able to enter the sure path with regards to skillful qualities when listening to the true teaching. What five? They don’t listen to the teaching bent only on putting it down. They don’t listen to the teaching with a hostile, fault-finding mind. They’re not antagonistic to the teacher, and not planning to attack them. They’re wise, bright, and clever. And they don’t think they know what they don’t know. Someone with these five qualities is able to enter the sure path with regards to skillful qualities when listening to the true teaching.” 

%
\section*{{\suttatitleacronym AN 5.154}{\suttatitletranslation The Decline of the True Teaching (1st) }{\suttatitleroot Paṭhamasaddhammasammosasutta}}
\addcontentsline{toc}{section}{\tocacronym{AN 5.154} \toctranslation{The Decline of the True Teaching (1st) } \tocroot{Paṭhamasaddhammasammosasutta}}
\markboth{The Decline of the True Teaching (1st) }{Paṭhamasaddhammasammosasutta}
\extramarks{AN 5.154}{AN 5.154}

“Mendicants,\marginnote{1.1} these five things lead to the decline and disappearance of the true teaching. What five? It’s when mendicants don’t carefully listen to the teachings, memorize them, and remember them. They don’t carefully examine the meaning of teachings that they remember. And they don’t carefully practice in line with the meaning and the teaching they’ve understood. These five things lead to the decline and disappearance of the true teaching. 

These\marginnote{2.1} five things lead to the continuation, persistence, and enduring of the true teaching. What five? It’s when mendicants carefully listen to the teachings, memorize them, and remember them. They carefully examine the meaning of teachings that they remember. And they carefully practice in line with the meaning and the teaching they’ve understood. These five things lead to the continuation, persistence, and enduring of the true teaching.” 

%
\section*{{\suttatitleacronym AN 5.155}{\suttatitletranslation The Decline of the True Teaching (2nd) }{\suttatitleroot Dutiyasaddhammasammosasutta}}
\addcontentsline{toc}{section}{\tocacronym{AN 5.155} \toctranslation{The Decline of the True Teaching (2nd) } \tocroot{Dutiyasaddhammasammosasutta}}
\markboth{The Decline of the True Teaching (2nd) }{Dutiyasaddhammasammosasutta}
\extramarks{AN 5.155}{AN 5.155}

“Mendicants,\marginnote{1.1} these five things lead to the decline and disappearance of the true teaching. What five? 

It’s\marginnote{1.3} when the mendicants don’t memorize the teaching—statements, songs, discussions, verses, inspired exclamations, legends, stories of past lives, amazing stories, and classifications. This is the first thing that leads to the decline and disappearance of the true teaching. 

Furthermore,\marginnote{2.1} the mendicants don’t explain the teaching in detail to others as they learned and memorized it. This is the second thing … 

Furthermore,\marginnote{3.1} the mendicants don’t make others recite the teaching in detail as they learned and memorized it. This is the third thing … 

Furthermore,\marginnote{4.1} the mendicants don’t recite the teaching in detail as they learned and memorized it. This is the fourth thing … 

Furthermore,\marginnote{5.1} the mendicants don’t think about and consider the teaching in their hearts, examining it with their minds as they learned and memorized it. This is the fifth thing that leads to the decline and disappearance of the true teaching. 

These\marginnote{5.3} five things lead to the decline and disappearance of the true teaching. 

These\marginnote{6.1} five things lead to the continuation, persistence, and enduring of the true teaching. What five? 

It’s\marginnote{6.3} when the mendicants memorize the teaching—statements, songs, discussions, verses, inspired exclamations, legends, stories of past lives, amazing stories, and classifications. This is the first thing that leads to the continuation, persistence, and enduring of the true teaching. 

Furthermore,\marginnote{7.1} the mendicants explain the teaching in detail to others as they learned and memorized it. This is the second thing … 

Furthermore,\marginnote{8.1} the mendicants make others recite the teaching in detail as they learned and memorized it. This is the third thing … 

Furthermore,\marginnote{9.1} the mendicants recite the teaching in detail as they learned and memorized it. This is the fourth thing … 

Furthermore,\marginnote{10.1} the mendicants think about and consider the teaching in their hearts, examining it with their minds as they learned and memorized it. This is the fifth thing that leads to the continuation, persistence, and enduring of the true teaching. 

These\marginnote{10.3} five things lead to the continuation, persistence, and enduring of the true teaching.” 

%
\section*{{\suttatitleacronym AN 5.156}{\suttatitletranslation The Decline of the True Teaching (3rd) }{\suttatitleroot Tatiyasaddhammasammosasutta}}
\addcontentsline{toc}{section}{\tocacronym{AN 5.156} \toctranslation{The Decline of the True Teaching (3rd) } \tocroot{Tatiyasaddhammasammosasutta}}
\markboth{The Decline of the True Teaching (3rd) }{Tatiyasaddhammasammosasutta}
\extramarks{AN 5.156}{AN 5.156}

“Mendicants,\marginnote{1.1} these five things lead to the decline and disappearance of the true teaching. What five? 

It’s\marginnote{1.3} when the mendicants memorize discourses that they learned incorrectly, with misplaced words and phrases. When the words and phrases are misplaced, the meaning is misinterpreted. This is the first thing that leads to the decline and disappearance of the true teaching. 

Furthermore,\marginnote{2.1} the mendicants are hard to admonish, having qualities that make them hard to admonish. They’re impatient, and don’t take instruction respectfully. This is the second thing … 

Furthermore,\marginnote{3.1} the mendicants who are very learned—knowledgeable in the scriptures, who have memorized the teachings, the monastic law, and the outlines—don’t carefully make others recite the discourses. When they pass away, the discourses are cut off at the root, with no-one to preserve them. This is the third thing … 

Furthermore,\marginnote{4.1} the senior mendicants are indulgent and slack, leaders in backsliding, neglecting seclusion, not rousing energy for attaining the unattained, achieving the unachieved, and realizing the unrealized. Those who come after them follow their example. They too are indulgent and slack … This is the fourth thing … 

Furthermore,\marginnote{5.1} there’s a schism in the \textsanskrit{Saṅgha}. When the \textsanskrit{Saṅgha} is split, they abuse, insult, block, and reject each other. This doesn’t inspire confidence in those without it, and it causes some with confidence to change their minds. This is the fifth thing that leads to the decline and disappearance of the true teaching. 

These\marginnote{5.5} five things lead to the decline and disappearance of the true teaching. 

These\marginnote{6.1} five things lead to the continuation, persistence, and enduring of the true teaching. What five? It’s when the mendicants memorize discourses that have been learned correctly, with well placed words and phrases. When the words and phrases are well organized, the meaning is correctly interpreted. This is the first thing that leads to the continuation, persistence, and enduring of the true teaching. 

Furthermore,\marginnote{7.1} the mendicants are easy to admonish, having qualities that make them easy to admonish. They’re patient, and take instruction respectfully. This is the second thing … 

Furthermore,\marginnote{8.1} the mendicants who are very learned—knowledgeable in the scriptures, who have memorized the teachings, the monastic law, and the outlines—carefully make others recite the discourses. When they pass away, the discourses aren’t cut off at the root, and they have someone to preserve them. This is the third thing … 

Furthermore,\marginnote{9.1} the senior mendicants are not indulgent and slack, leaders in backsliding, neglecting seclusion. They rouse energy for attaining the unattained, achieving the unachieved, and realizing the unrealized. Those who come after them follow their example. They too are not indulgent or slack … This is the fourth thing … 

Furthermore,\marginnote{10.1} the \textsanskrit{Saṅgha} lives comfortably, in harmony, appreciating each other, without quarreling, with one recitation. When the \textsanskrit{Saṅgha} is in harmony, they don’t abuse, insult, block, or reject each other. This inspires confidence in those without it, and increases confidence in those who have it. This is the fifth thing that leads to the continuation, persistence, and enduring of the true teaching. 

These\marginnote{10.5} five things lead to the continuation, persistence, and enduring of the true teaching.” 

%
\section*{{\suttatitleacronym AN 5.157}{\suttatitletranslation Inappropriate Talk }{\suttatitleroot Dukkathāsutta}}
\addcontentsline{toc}{section}{\tocacronym{AN 5.157} \toctranslation{Inappropriate Talk } \tocroot{Dukkathāsutta}}
\markboth{Inappropriate Talk }{Dukkathāsutta}
\extramarks{AN 5.157}{AN 5.157}

“Mendicants,\marginnote{1.1} it is inappropriate to speak to five kinds of person by comparing that person with someone else. What five? 

It’s\marginnote{1.3} inappropriate to talk to an unfaithful person about faith. It’s inappropriate to talk to an unethical person about ethics. It’s inappropriate to talk to an unlearned person about learning. It’s inappropriate to talk to a stingy person about generosity. It’s inappropriate to talk to a witless person about wisdom. 

And\marginnote{2.1} why is it inappropriate to talk to an unfaithful person about faith? When an unfaithful person is spoken to about faith they lose their temper, becoming annoyed, hostile, and hard-hearted, and displaying annoyance, hate, and bitterness. Why is that? Not seeing that faith in themselves, they don’t get the rapture and joy that faith brings. That’s why it’s inappropriate to talk to an unfaithful person about faith. 

And\marginnote{3.1} why is it inappropriate to talk to an unethical person about ethics? When an unethical person is spoken to about ethics they lose their temper … Why is that? Not seeing that ethical conduct in themselves, they don’t get the rapture and joy that ethical conduct brings. That’s why it’s inappropriate to talk to an unethical person about ethics. 

And\marginnote{4.1} why is it inappropriate to talk to an unlearned person about learning? When an unlearned person is spoken to about learning they lose their temper … Why is that? Not seeing that learning in themselves, they don’t get the rapture and joy that learning brings. That’s why it’s inappropriate to talk to an unlearned person about learning. 

And\marginnote{5.1} why is it inappropriate to talk to a stingy person about generosity? When an stingy person is spoken to about generosity they lose their temper … Why is that? Not seeing that generosity in themselves, they don’t get the rapture and joy that generosity brings. That’s why it’s inappropriate to talk to a stingy person about generosity. 

And\marginnote{6.1} why is it inappropriate to talk to a witless person about wisdom? When a witless person is spoken to about wisdom they lose their temper, becoming annoyed, hostile, and hard-hearted, and displaying annoyance, hate, and bitterness. Why is that? Not seeing that wisdom in themselves, they don’t get the rapture and joy that wisdom brings. That’s why it’s inappropriate to talk to a witless person about wisdom. 

It\marginnote{6.6} is inappropriate to speak to these five kinds of person by comparing that person with someone else. 

It\marginnote{7.1} is appropriate to speak to five kinds of person by comparing that person with someone else. What five? 

It’s\marginnote{7.3} appropriate to talk to a faithful person about faith. It’s appropriate to talk to an ethical person about ethical conduct. It’s appropriate to talk to a learned person about learning. It’s appropriate to talk to a generous person about generosity. It’s appropriate to talk to a wise person about wisdom. 

And\marginnote{8.1} why is it appropriate to talk to a faithful person about faith? When a faithful person is spoken to about faith they don’t lose their temper, they don’t get annoyed, hostile, and hard-hearted, or display annoyance, hate, and bitterness. Why is that? Seeing that faith in themselves, they get the rapture and joy that faith brings. That’s why it’s appropriate to talk to a faithful person about faith. 

And\marginnote{9.1} why is it appropriate to talk to an ethical person about ethical conduct? When an ethical person is spoken to about ethical conduct they don’t lose their temper … Why is that? Seeing that ethical conduct in themselves, they get the rapture and joy that ethical conduct brings. That’s why it’s appropriate to talk to an ethical person about ethical conduct. 

And\marginnote{10.1} why is it appropriate to talk to a learned person about learning? When a learned person is spoken to about learning they don’t lose their temper … Why is that? Seeing that learning in themselves, they get the rapture and joy that learning brings. That’s why it’s appropriate to talk to a learned person about learning. 

And\marginnote{11.1} why is it appropriate to talk to a generous person about generosity? When a generous person is spoken to about generosity they don’t lose their temper … Why is that? Seeing that generosity in themselves, they get the rapture and joy that generosity brings. That’s why it’s appropriate to talk to a generous person about generosity. 

And\marginnote{12.1} why is it appropriate to talk to a wise person about wisdom? When a wise person is spoken to about wisdom they don’t lose their temper, they don’t get annoyed, hostile, and hard-hearted, or display annoyance, hate, and bitterness. Why is that? Seeing that wisdom in themselves, they get the rapture and joy that wisdom brings. That’s why it’s appropriate to talk to a wise person about wisdom. 

It\marginnote{12.6} is appropriate to speak to these five kinds of person by comparing that person with someone else.” 

%
\section*{{\suttatitleacronym AN 5.158}{\suttatitletranslation Timidity }{\suttatitleroot Sārajjasutta}}
\addcontentsline{toc}{section}{\tocacronym{AN 5.158} \toctranslation{Timidity } \tocroot{Sārajjasutta}}
\markboth{Timidity }{Sārajjasutta}
\extramarks{AN 5.158}{AN 5.158}

“Mendicants,\marginnote{1.1} a mendicant with five qualities is overcome by timidity. What five? It’s when a mendicant is faithless, unethical, with little learning, lazy, and witless. A mendicant with these five qualities is overcome by timidity. 

A\marginnote{2.1} mendicant with five qualities is self-assured. What five? It’s when a mendicant is faithful, ethical, learned, energetic, and wise. A mendicant with these five qualities is self-assured.” 

%
\section*{{\suttatitleacronym AN 5.159}{\suttatitletranslation With Udāyī }{\suttatitleroot Udāyīsutta}}
\addcontentsline{toc}{section}{\tocacronym{AN 5.159} \toctranslation{With Udāyī } \tocroot{Udāyīsutta}}
\markboth{With Udāyī }{Udāyīsutta}
\extramarks{AN 5.159}{AN 5.159}

\scevam{So\marginnote{1.1} I have heard. }At one time the Buddha was staying near Kosambi, in Ghosita’s Monastery. Now, at that time Venerable \textsanskrit{Udāyī} was sitting teaching Dhamma, surrounded by a large assembly of laypeople. Seeing this, Venerable Ānanda went up to the Buddha, bowed, sat down to one side, and said to him: 

“Sir,\marginnote{1.6} Venerable \textsanskrit{Udāyī} is teaching Dhamma, surrounded by a large assembly of laypeople.” 

“Ānanda,\marginnote{2.1} it’s not easy to teach Dhamma to others. You should establish five things in yourself before teaching Dhamma to others. What five? 

You\marginnote{2.4} should teach Dhamma to others thinking: ‘I will teach step by step.’ … 

‘I\marginnote{2.5} will teach explaining my methods.’ … 

‘I\marginnote{2.6} will teach out of kindness.’ … 

‘I\marginnote{2.7} will not teach while secretly hoping to profit.’ … 

‘I\marginnote{2.8} will teach without hurting myself or others.’ 

It’s\marginnote{2.9} not easy to teach Dhamma to others. You should establish these five things in yourself before teaching Dhamma to others.” 

%
\section*{{\suttatitleacronym AN 5.160}{\suttatitletranslation Hard to Get Rid Of }{\suttatitleroot Duppaṭivinodayasutta}}
\addcontentsline{toc}{section}{\tocacronym{AN 5.160} \toctranslation{Hard to Get Rid Of } \tocroot{Duppaṭivinodayasutta}}
\markboth{Hard to Get Rid Of }{Duppaṭivinodayasutta}
\extramarks{AN 5.160}{AN 5.160}

“Mendicants,\marginnote{1.1} these five things are hard to get rid of once they’ve arisen. What five? Greed, hate, delusion, the feeling of being inspired to speak out, and thoughts of traveling. These five things are hard to get rid of once they’ve arisen.” 

%
\addtocontents{toc}{\let\protect\contentsline\protect\nopagecontentsline}
\chapter*{The Chapter on Resentment }
\addcontentsline{toc}{chapter}{\tocchapterline{The Chapter on Resentment }}
\addtocontents{toc}{\let\protect\contentsline\protect\oldcontentsline}

%
\section*{{\suttatitleacronym AN 5.161}{\suttatitletranslation Getting Rid of Resentment (1st) }{\suttatitleroot Paṭhamaāghātapaṭivinayasutta}}
\addcontentsline{toc}{section}{\tocacronym{AN 5.161} \toctranslation{Getting Rid of Resentment (1st) } \tocroot{Paṭhamaāghātapaṭivinayasutta}}
\markboth{Getting Rid of Resentment (1st) }{Paṭhamaāghātapaṭivinayasutta}
\extramarks{AN 5.161}{AN 5.161}

“Mendicants,\marginnote{1.1} a mendicant should use these five methods to completely get rid of resentment when it has arisen toward anyone. What five? 

You\marginnote{1.3} should develop love for a person you resent. That’s how to get rid of resentment for that person. 

You\marginnote{1.5} should develop compassion for a person you resent. … 

You\marginnote{1.7} should develop equanimity for a person you resent. … 

You\marginnote{1.9} should disregard a person you resent, paying no attention to them. … 

You\marginnote{1.11} should apply the concept that we are the owners of our deeds to that person: ‘This venerable is the owner of their deeds and heir to their deeds. Deeds are their womb, their relative, and their refuge. They shall be the heir of whatever deeds they do, whether good or bad.’ That’s how to get rid of resentment for that person. 

A\marginnote{1.15} mendicant should use these five methods to completely get rid of resentment when it has arisen toward anyone.” 

%
\section*{{\suttatitleacronym AN 5.162}{\suttatitletranslation Getting Rid of Resentment (2nd) }{\suttatitleroot Dutiyaāghātapaṭivinayasutta}}
\addcontentsline{toc}{section}{\tocacronym{AN 5.162} \toctranslation{Getting Rid of Resentment (2nd) } \tocroot{Dutiyaāghātapaṭivinayasutta}}
\markboth{Getting Rid of Resentment (2nd) }{Dutiyaāghātapaṭivinayasutta}
\extramarks{AN 5.162}{AN 5.162}

There\marginnote{1.1} Venerable \textsanskrit{Sāriputta} addressed the mendicants: “Reverends, mendicants!” 

“Reverend,”\marginnote{1.3} they replied. \textsanskrit{Sāriputta} said this: 

“Reverends,\marginnote{2.1} a mendicant should use these five methods to completely get rid of resentment when it has arisen toward anyone. What five? 

In\marginnote{2.3} the case of a person whose behavior by way of body is impure, but whose behavior by way of speech is pure, you should get rid of resentment for that kind of person. 

In\marginnote{2.5} the case of a person whose behavior by way of speech is impure, but whose behavior by way of body is pure, … 

In\marginnote{2.7} the case of a person whose behavior by way of body and speech is impure, but who gets an openness and clarity of heart from time to time, … 

In\marginnote{2.9} the case of a person whose behavior by way of body and speech is impure, and who doesn’t get an openness and clarity of heart from time to time, … 

In\marginnote{2.11} the case of a person whose behavior by way of body and speech is pure, and who gets an openness and clarity of heart from time to time, you should get rid of resentment for that kind of person. 

How\marginnote{3.1} should you get rid of resentment for a person whose behavior by way of body is impure, but whose behavior by way of speech is pure? Suppose a mendicant wearing rag robes sees a rag by the side of the road. They’d hold it down with the left foot, spread it out with the right foot, tear out what was intact, and take it away with them. In the same way, at that time you should ignore that person’s impure behavior by way of body and focus on their pure behavior by way of speech. That’s how to get rid of resentment for that person. 

How\marginnote{4.1} should you get rid of resentment for a person whose behavior by way of speech is impure, but whose behavior by way of body is pure? Suppose there was a lotus pond covered with moss and aquatic plants. Then along comes a person struggling in the oppressive heat, weary, thirsty, and parched. They’d plunge into the lotus pond, sweep apart the moss and aquatic plants, drink from their cupped hands, and be on their way. In the same way, at that time you should ignore that person’s impure behavior by way of speech and focus on their pure behavior by way of body. That’s how to get rid of resentment for that person. 

How\marginnote{5.1} should you get rid of resentment for a person whose behavior by way of body and speech is impure, but who gets an openness and clarity of heart from time to time? Suppose there was a little water in a cow’s hoofprint. Then along comes a person struggling in the oppressive heat, weary, thirsty, and parched. They might think: ‘This little bit of water is in a cow’s hoofprint. If I drink it with my cupped hands or a bowl, I’ll stir it and disturb it, making it undrinkable. Why don’t I get down on all fours and drink it up like a cow, then be on my way?’ So that’s what they do. In the same way, at that time you should ignore that person’s impure behavior by way of speech and body, and focus on the fact that they get an openness and clarity of heart from time to time. That’s how to get rid of resentment for that person. 

How\marginnote{6.1} should you get rid of resentment for a person whose behavior by way of body and speech is impure, and who doesn’t get an openness and clarity of heart from time to time? Suppose a person was traveling along a road, and they were sick, suffering, gravely ill. And it was a long way to a village, whether ahead or behind. And they didn’t have any suitable food or medicine, or a competent carer, or someone to bring them within a village. Then another person traveling along the road sees them, and thinks of them with nothing but compassion, kindness, and sympathy: ‘Oh, may this person get suitable food or medicine, or a competent carer, or someone to bring them within a village. Why is that? So that they don’t come to ruin right here.’ In the same way, at that time you should ignore that person’s impure behavior by way of speech and body, and the fact that they don’t get an openness and clarity of heart from time to time, and think of them with nothing but compassion, kindness, and sympathy: ‘Oh, may this person give up bad conduct by way of body, speech, and mind, and develop good conduct by way of body, speech, and mind. Why is that? So that, when their body breaks up, after death, they’re not reborn in a place of loss, a bad place, the underworld, hell.’ That’s how to get rid of resentment for that person. 

How\marginnote{7.1} should you get rid of resentment for a person whose behavior by way of body and speech is pure, and who gets an openness and clarity of heart from time to time? Suppose there was a lotus pond with clear, sweet, cool water, clean, with smooth banks, delightful, and shaded by many trees. Then along comes a person struggling in the oppressive heat, weary, thirsty, and parched. They’d plunge into the lotus pond to bathe and drink. And after emerging they’d sit or lie down right there in the shade of the trees. 

In\marginnote{8.1} the same way, at that time you should focus on that person’s pure behavior by way of body and speech, and on the fact that they get an openness and clarity of heart from time to time. That’s how to get rid of resentment for that person. Relying on a person who is impressive all around, the mind becomes confident. 

A\marginnote{9.1} mendicant should use these five methods to completely get rid of resentment when it has arisen toward anyone.” 

%
\section*{{\suttatitleacronym AN 5.163}{\suttatitletranslation Discussions }{\suttatitleroot Sākacchasutta}}
\addcontentsline{toc}{section}{\tocacronym{AN 5.163} \toctranslation{Discussions } \tocroot{Sākacchasutta}}
\markboth{Discussions }{Sākacchasutta}
\extramarks{AN 5.163}{AN 5.163}

There\marginnote{1.1} Venerable \textsanskrit{Sāriputta} addressed the mendicants: “Reverends, mendicants!” 

“Reverend,”\marginnote{1.3} they replied. \textsanskrit{Sāriputta} said this: 

“A\marginnote{2.1} mendicant with five qualities is fit to hold a discussion with their spiritual companions. What five? 

A\marginnote{2.3} mendicant is personally accomplished in ethics, and answers questions that come up when discussing accomplishment in ethics. 

They’re\marginnote{2.4} personally accomplished in immersion, … 

They’re\marginnote{2.5} personally accomplished in wisdom, … 

They’re\marginnote{2.6} personally accomplished in freedom, … 

They’re\marginnote{2.7} personally accomplished in the knowledge and vision of freedom, and they answer questions that come up when discussing accomplishment in the knowledge and vision of freedom. 

A\marginnote{2.8} mendicant with these five qualities is fit to hold a discussion with their spiritual companions.” 

%
\section*{{\suttatitleacronym AN 5.164}{\suttatitletranslation Sharing a Way of Life }{\suttatitleroot Sājīvasutta}}
\addcontentsline{toc}{section}{\tocacronym{AN 5.164} \toctranslation{Sharing a Way of Life } \tocroot{Sājīvasutta}}
\markboth{Sharing a Way of Life }{Sājīvasutta}
\extramarks{AN 5.164}{AN 5.164}

There\marginnote{1.1} Venerable \textsanskrit{Sāriputta} addressed the mendicants: 

“A\marginnote{1.2} mendicant with five qualities is fit to share their life with their spiritual companions. What five? 

A\marginnote{1.4} mendicant is personally accomplished in ethics, and answers questions that come up when discussing accomplishment in ethics. 

They’re\marginnote{1.5} personally accomplished in immersion, … 

They’re\marginnote{1.6} personally accomplished in wisdom, … 

They’re\marginnote{1.7} personally accomplished in freedom, … 

They’re\marginnote{1.8} personally accomplished in the knowledge and vision of freedom, and they answer questions that come up when discussing accomplishment in the knowledge and vision of freedom. 

A\marginnote{1.9} mendicant with these five qualities is fit to share their life with their spiritual companions.” 

%
\section*{{\suttatitleacronym AN 5.165}{\suttatitletranslation Asking Questions }{\suttatitleroot Pañhapucchāsutta}}
\addcontentsline{toc}{section}{\tocacronym{AN 5.165} \toctranslation{Asking Questions } \tocroot{Pañhapucchāsutta}}
\markboth{Asking Questions }{Pañhapucchāsutta}
\extramarks{AN 5.165}{AN 5.165}

There\marginnote{1.1} Venerable \textsanskrit{Sāriputta} addressed the mendicants: … “Whoever asks a question of another, does so for one or other of these five reasons. What five? Someone asks a question of another from stupidity and folly. Or they ask from wicked desires, being naturally full of desires. Or they ask in order to disparage. Or they ask wanting to understand. Or they ask with the thought, ‘If they correctly answer the question I ask it’s good. If not, I’ll correctly answer it for them.’ Whoever asks a question of another, does so for one or other of these five reasons. As for myself, I ask with the thought, ‘If they correctly answer the question I ask it’s good. If not, I’ll correctly answer it for them.’” 

%
\section*{{\suttatitleacronym AN 5.166}{\suttatitletranslation Cessation }{\suttatitleroot Nirodhasutta}}
\addcontentsline{toc}{section}{\tocacronym{AN 5.166} \toctranslation{Cessation } \tocroot{Nirodhasutta}}
\markboth{Cessation }{Nirodhasutta}
\extramarks{AN 5.166}{AN 5.166}

There\marginnote{1.1} Venerable \textsanskrit{Sāriputta} addressed the mendicants: 

“Reverends,\marginnote{1.2} take a mendicant who is accomplished in ethics, immersion, and wisdom. They might enter into and emerge from the cessation of perception and feeling. That is possible. If they don’t reach enlightenment in this very life, then, surpassing the company of gods that consume solid food, they’re reborn in a certain host of mind-made gods. There they might enter into and emerge from the cessation of perception and feeling. That is possible.” 

When\marginnote{2.1} he said this, Venerable \textsanskrit{Udāyī} said to him, “This is not possible, Reverend \textsanskrit{Sāriputta}, it cannot happen!” 

But\marginnote{3.1} for a second … and a third time \textsanskrit{Sāriputta} repeated his statement. 

And\marginnote{4.1} for a third time, \textsanskrit{Udāyī} said to him, “This is not possible, Reverend \textsanskrit{Sāriputta}, it cannot happen!” 

Then\marginnote{5.1} Venerable \textsanskrit{Sāriputta} thought, “Venerable \textsanskrit{Udāyī} disagrees with me three times, and not one mendicant agrees with me. Why don’t I go to see the Buddha?” 

Then\marginnote{5.4} \textsanskrit{Sāriputta} went up to the Buddha, bowed, sat down to one side, and said to the mendicants: 

“Reverends,\marginnote{5.6} take a mendicant who is accomplished in ethics, immersion, and wisdom. They might enter into and emerge from the cessation of perception and feeling. There is such a possibility. If they don’t reach enlightenment in this very life, they’re reborn in the company of a certain host of mind-made gods, who surpass the gods that consume solid food. There they might enter into and emerge from the cessation of perception and feeling. That is possible.” 

When\marginnote{6.1} he said this, \textsanskrit{Udāyī} said to him, “This is not possible, Reverend \textsanskrit{Sāriputta}, it cannot happen!” 

But\marginnote{7.1} for a second … and a third time \textsanskrit{Sāriputta} repeated his statement. 

And\marginnote{8.1} for a third time, \textsanskrit{Udāyī} said to him, “This is not possible, Reverend \textsanskrit{Sāriputta}, it cannot happen!” 

Then\marginnote{9.1} Venerable \textsanskrit{Sāriputta} thought, “Even in front of the Buddha Venerable \textsanskrit{Udāyī} disagrees with me three times, and not one mendicant agrees with me. I’d better stay silent.” Then \textsanskrit{Sāriputta} fell silent. 

Then\marginnote{10.1} the Buddha said to Venerable \textsanskrit{Udāyī}, “But \textsanskrit{Udāyī}, do you believe in a mind-made body?” 

“For\marginnote{10.3} those gods, sir, who are formless, made of perception.” 

“\textsanskrit{Udāyī},\marginnote{10.4} what has an incompetent fool like you got to say? How on earth could you imagine you’ve got something worth saying!” 

Then\marginnote{11.1} the Buddha said to Venerable Ānanda, “Ānanda! There’s a senior mendicant being harassed, and you just watch it happening. Don’t you have any compassion for a senior mendicant who is being harassed?” 

Then\marginnote{12.1} the Buddha addressed the mendicants: 

“Mendicants,\marginnote{12.2} take a mendicant who is accomplished in ethics, immersion, and wisdom. They might enter into and emerge from the cessation of perception and feeling. That is possible. If they don’t reach enlightenment in this very life, they’re reborn in the company of a certain host of mind-made gods, who surpass the gods that consume solid food. There they might enter into and emerge from the cessation of perception and feeling. That is possible.” 

That\marginnote{12.6} is what the Buddha said. When he had spoken, the Holy One got up from his seat and entered his dwelling. 

Then,\marginnote{13.1} not long after the Buddha had left, Venerable Ānanda went to Venerable \textsanskrit{Upavāṇa} and said to him, “Reverend \textsanskrit{Upavāṇa}, they’ve been harassing other senior mendicants, but I didn’t question them. I wouldn’t be surprised if the Buddha makes a statement about this when he comes out of retreat later this afternoon. He might even call upon Venerable \textsanskrit{Upavāṇa} himself. And right now I feel timid.” 

Then\marginnote{13.6} in the late afternoon, the Buddha came out of retreat and went to the assembly hall, where he sat on the seat spread out, and said to \textsanskrit{Upavāṇa}, “\textsanskrit{Upavāṇa}, how many qualities should a senior mendicant have to be dear and beloved to their spiritual companions, respected and admired?” 

“Sir,\marginnote{14.2} a senior mendicant with five qualities is dear and beloved to their spiritual companions, respected and admired. What five? 

It’s\marginnote{14.4} when a mendicant is ethical, restrained in the code of conduct, conducting themselves well and seeking alms in suitable places. Seeing danger in the slightest fault, they keep the rules they’ve undertaken. 

They’re\marginnote{14.5} very learned, remembering and keeping what they’ve learned. These teachings are good in the beginning, good in the middle, and good in the end, meaningful and well-phrased, describing a spiritual practice that’s totally full and pure. They are very learned in such teachings, remembering them, reciting them, mentally scrutinizing them, and comprehending them theoretically. 

They’re\marginnote{14.6} a good speaker, with a polished, clear, and articulate voice that expresses the meaning. 

They\marginnote{14.7} get the four absorptions—blissful meditations in the present life that belong to the higher mind—when they want, without trouble or difficulty. 

They\marginnote{14.8} realize the undefiled freedom of heart and freedom by wisdom in this very life. And they live having realized it with their own insight due to the ending of defilements. 

A\marginnote{14.9} senior mendicant with these five qualities is dear and beloved to their spiritual companions, respected and admired.” 

“Good,\marginnote{15.1} good, \textsanskrit{Upavāṇa}! A senior mendicant with these five qualities is dear and beloved to their spiritual companions, respected and admired. If these five qualities are not found in a senior mendicant, why would their spiritual companions honor, respect, revere, or venerate them? Because of their broken teeth, gray hair, and wrinkled skin? But since these five qualities are found in a senior mendicant, their spiritual companions honor, respect, revere, or venerate them.” 

%
\section*{{\suttatitleacronym AN 5.167}{\suttatitletranslation Accusation }{\suttatitleroot Codanāsutta}}
\addcontentsline{toc}{section}{\tocacronym{AN 5.167} \toctranslation{Accusation } \tocroot{Codanāsutta}}
\markboth{Accusation }{Codanāsutta}
\extramarks{AN 5.167}{AN 5.167}

There\marginnote{1.1} \textsanskrit{Sāriputta} addressed the mendicants: “Reverends, a mendicant who wants to accuse another should first establish five things in themselves. 

What\marginnote{2.1} five? I will speak at the right time, not at the wrong time. I will speak truthfully, not falsely. I will speak gently, not harshly. I will speak beneficially, not harmfully. I will speak lovingly, not from secret hate. A mendicant who wants to accuse another should first establish these five things in themselves. 

Take\marginnote{3.1} a case where I see a certain person being accused at the wrong time, not being disturbed at the right time. They’re accused falsely, not disturbed truthfully. They’re accused harshly, not disturbed gently. They’re accused harmfully, not disturbed beneficially. They’re accused with secret hate, not disturbed lovingly. 

The\marginnote{4.1} mendicant who is accused improperly should be reassured in five ways. ‘Venerable, you were accused at the wrong time, not at the right time. There’s no need for you to feel remorse. You were accused falsely, not truthfully. … You were accused harshly, not gently. … You were accused harmfully, not beneficially. … You were accused with secret hate, not lovingly. There’s no need for you to feel remorse.’ A mendicant who is accused improperly should be reassured in these five ways. 

The\marginnote{5.1} mendicant who makes improper accusations should be chastened in five ways. ‘Reverend, you made an accusation at the wrong time, not at the right time. There’s a reason for you to feel remorse. You made an accusation falsely, not truthfully. … You made an accusation harshly, not gently. … You made an accusation harmfully, not beneficially. … You made an accusation with secret hate, not lovingly. There’s a reason for you to feel remorse.’ The mendicant who makes improper accusations should be chastened in these five ways. Why is that? So that another mendicant wouldn’t think to make a false accusation. 

Take\marginnote{6.1} a case where I see a certain person being accused at the right time, not being disturbed at the wrong time. They’re accused truthfully, not disturbed falsely. They’re accused gently, not disturbed harshly. They’re accused beneficially, not disturbed harmfully. They’re accused lovingly, not disturbed with secret hate. 

The\marginnote{7.1} mendicant who is accused properly should be chastened in five ways. ‘Venerable, you were accused at the right time, not at the wrong time. There’s a reason for you to feel remorse. You were accused truthfully, not falsely. … You were accused gently, not harshly. … You were accused beneficially, not harmfully. … You were accused lovingly, not with secret hate. There’s a reason for you to feel remorse.’ The mendicant who is accused properly should be chastened in these five ways. 

The\marginnote{8.1} mendicant who makes proper accusations should be reassured in five ways. ‘Reverend, you made an accusation at the right time, not at the wrong time. There’s no need for you to feel remorse. You made an accusation truthfully, not falsely. … You made an accusation gently, not harshly. … You made an accusation beneficially, not harmfully. … You made an accusation lovingly, not with secret hate. There’s no need for you to feel remorse.’ The mendicant who makes proper accusations should be reassured in these five ways. Why is that? So that another mendicant would think to make a true accusation. 

A\marginnote{9.1} person who is accused should ground themselves in two things: truth and an even temper. Even if others accuse me—at the right time or the wrong time, truthfully or falsely, gently or harshly, beneficially or harmfully, lovingly or with secret hate—I will still ground myself in two things: truth and an even temper. If I know that that quality is found in me, I will tell them that it is. If I know that that quality is not found in me, I will tell them that it is not.” 

“Even\marginnote{10.1} when you speak like this, \textsanskrit{Sāriputta}, there are still some foolish people here who do not respectfully take it up.” 

“Sir,\marginnote{11.1} there are those faithless people who went forth from the lay life to homelessness not out of faith but to earn a livelihood. They’re devious, deceitful, and sneaky. They’re restless, insolent, fickle, scurrilous, and loose-tongued. They do not guard their sense doors or eat in moderation, and they are not dedicated to wakefulness. They don’t care about the ascetic life, and don’t keenly respect the training. They’re indulgent and slack, leaders in backsliding, neglecting seclusion, lazy, and lacking energy. They’re unmindful, lacking situational awareness and immersion, with straying minds, witless and stupid. When I speak to them like this, they don’t respectfully take it up. 

Sir,\marginnote{12.1} there are those gentlemen who went forth from the lay life to homelessness out of faith. They’re not devious, deceitful, and sneaky. They’re not restless, insolent, fickle, scurrilous, and loose-tongued. They guard their sense doors and eat in moderation, and they are dedicated to wakefulness. They care about the ascetic life, and keenly respect the training. They’re not indulgent or slack, nor are they leaders in backsliding, neglecting seclusion. They’re energetic and determined. They’re mindful, with situational awareness, immersion, and unified minds; wise, not stupid. When I speak to them like this, they do respectfully take it up.” 

“\textsanskrit{Sāriputta},\marginnote{13.1} those faithless people who went forth from the lay life to homelessness not out of faith but to earn a livelihood … Leave them be. 

But\marginnote{14.1} those gentlemen who went forth from the lay life to homelessness out of faith … You should speak to them. \textsanskrit{Sāriputta}, you should advise your spiritual companions! You should instruct your spiritual companions! Thinking: ‘I will draw my spiritual companions away from false teachings and ground them in true teachings.’ That’s how you should train.” 

%
\section*{{\suttatitleacronym AN 5.168}{\suttatitletranslation Ethics }{\suttatitleroot Sīlasutta}}
\addcontentsline{toc}{section}{\tocacronym{AN 5.168} \toctranslation{Ethics } \tocroot{Sīlasutta}}
\markboth{Ethics }{Sīlasutta}
\extramarks{AN 5.168}{AN 5.168}

There\marginnote{1.1} Venerable \textsanskrit{Sāriputta} addressed the mendicants: 

“Reverends,\marginnote{1.2} an unethical person, who lacks ethics, has destroyed a vital condition for right immersion. When there is no right immersion, one who lacks right immersion has destroyed a vital condition for true knowledge and vision. When there is no true knowledge and vision, one who lacks true knowledge and vision has destroyed a vital condition for disillusionment and dispassion. When there is no disillusionment and dispassion, one who lacks disillusionment and dispassion has destroyed a vital condition for knowledge and vision of freedom. 

Suppose\marginnote{1.6} there was a tree that lacked branches and foliage. Its shoots, bark, softwood, and heartwood would not grow to fullness. 

In\marginnote{1.8} the same way, an unethical person, who lacks ethics, has destroyed a vital condition for right immersion. When there is no right immersion, one who lacks right immersion has destroyed a vital condition for true knowledge and vision. When there is no true knowledge and vision, one who lacks true knowledge and vision has destroyed a vital condition for disillusionment and dispassion. When there is no disillusionment and dispassion, one who lacks disillusionment and dispassion has destroyed a vital condition for knowledge and vision of freedom. 

An\marginnote{2.1} ethical person, who has fulfilled ethics, has fulfilled a vital condition for right immersion. When there is right immersion, one who has fulfilled right immersion has fulfilled a vital condition for true knowledge and vision. When there is true knowledge and vision, one who has fulfilled true knowledge and vision has fulfilled a vital condition for disillusionment and dispassion. When there is disillusionment and dispassion, one who has fulfilled disillusionment and dispassion has fulfilled a vital condition for knowledge and vision of freedom. 

Suppose\marginnote{2.5} there was a tree that was complete with branches and foliage. Its shoots, bark, softwood, and heartwood would grow to fullness. In the same way, an ethical person, who has fulfilled ethics, has fulfilled a vital condition for right immersion. 

When\marginnote{2.7} there is right immersion, one who has fulfilled right immersion has fulfilled a vital condition for true knowledge and vision. When there is true knowledge and vision, one who has fulfilled true knowledge and vision has fulfilled a vital condition for disillusionment and dispassion. When there is disillusionment and dispassion, one who has fulfilled disillusionment and dispassion has fulfilled a vital condition for knowledge and vision of freedom.” 

%
\section*{{\suttatitleacronym AN 5.169}{\suttatitletranslation Quick-witted }{\suttatitleroot Khippanisantisutta}}
\addcontentsline{toc}{section}{\tocacronym{AN 5.169} \toctranslation{Quick-witted } \tocroot{Khippanisantisutta}}
\markboth{Quick-witted }{Khippanisantisutta}
\extramarks{AN 5.169}{AN 5.169}

Then\marginnote{1.1} Venerable Ānanda went up to Venerable \textsanskrit{Sāriputta}, and exchanged greetings with him. When the greetings and polite conversation were over, he sat down to one side and said to him: 

“Reverend\marginnote{2.1} \textsanskrit{Sāriputta}, how are we to define a mendicant who is quick-witted when it comes to skillful teachings, who learns well, learns much, and does not forget what they’ve learned?” 

“Well,\marginnote{2.2} Venerable Ānanda, you’re very learned. Why don’t you clarify this yourself?” 

“Well\marginnote{2.4} then, Reverend \textsanskrit{Sāriputta}, listen and pay close attention, I will speak.” 

“Yes,\marginnote{2.5} reverend,” \textsanskrit{Sāriputta} replied. Venerable Ānanda said this: 

“It’s\marginnote{3.1} when a mendicant is skilled in the meaning, skilled in the teaching, skilled in terminology, skilled in phrasing, and skilled in sequence. That is how to define a mendicant who is quick-witted when it comes to skillful teachings, who learns well, learns much, and does not forget what they’ve learned.” 

“It’s\marginnote{3.3} incredible, it’s amazing! How well this was said by Venerable Ānanda! And we will remember Venerable Ānanda as someone who has these five qualities: ‘Reverend Ānanda is skilled in the meaning, skilled in the teaching, skilled in terminology, skilled in phrasing, and skilled in sequence.’” 

%
\section*{{\suttatitleacronym AN 5.170}{\suttatitletranslation With Bhaddaji }{\suttatitleroot Bhaddajisutta}}
\addcontentsline{toc}{section}{\tocacronym{AN 5.170} \toctranslation{With Bhaddaji } \tocroot{Bhaddajisutta}}
\markboth{With Bhaddaji }{Bhaddajisutta}
\extramarks{AN 5.170}{AN 5.170}

At\marginnote{1.1} one time Venerable Ānanda was staying near Kosambi, in Ghosita’s Monastery. Then Venerable Bhaddaji went up to Venerable Ānanda, and exchanged greetings with him. When the greetings and polite conversation were over, he sat down to one side, and Venerable Ānanda said to him: 

“Reverend\marginnote{1.4} Bhaddaji, what is the best sight, the best sound, the best happiness, the best perception, and the best state of existence?” 

“Reverend,\marginnote{2.1} there is this \textsanskrit{Brahmā}, the undefeated, the champion, the universal seer, the wielder of power. When you see \textsanskrit{Brahmā}, that’s the best sight. There are the gods called ‘of streaming radiance’, who are drenched and steeped in pleasure. Every so often they feel inspired to exclaim: ‘Oh, what bliss! Oh, what bliss!’ When you hear that, it’s the best sound. There are the gods called ‘replete with glory’. Since they’re truly content, they experience pleasure. This is the best happiness. There are the gods reborn in the dimension of nothingness. This is the best perception. There are the gods reborn in the dimension of neither perception nor non-perception. This is the best state of existence.” 

“So,\marginnote{2.10} Venerable Bhaddaji, do you agree with what most people say about this?” 

“Well,\marginnote{3.1} Venerable Ānanda, you’re very learned. Why don’t you clarify this yourself?” 

“Well\marginnote{3.3} then, Reverend Bhaddaji, listen and pay close attention, I will speak.” 

“Yes,\marginnote{3.4} reverend,” Bhaddaji replied. Ānanda said this: 

“What\marginnote{4.1} you see when the defilements end in the present life is the best sight. What you hear when the defilements end in the present life is the best sound. The happiness you feel when the defilements end in the present life is the best happiness. What you perceive when the defilements end in the present life is the best perception. The state of existence in which the defilements end in the present life is the best state of existence.” 

%
\addtocontents{toc}{\let\protect\contentsline\protect\nopagecontentsline}
\chapter*{The Chapter on a Lay Follower }
\addcontentsline{toc}{chapter}{\tocchapterline{The Chapter on a Lay Follower }}
\addtocontents{toc}{\let\protect\contentsline\protect\oldcontentsline}

%
\section*{{\suttatitleacronym AN 5.171}{\suttatitletranslation Timidity }{\suttatitleroot Sārajjasutta}}
\addcontentsline{toc}{section}{\tocacronym{AN 5.171} \toctranslation{Timidity } \tocroot{Sārajjasutta}}
\markboth{Timidity }{Sārajjasutta}
\extramarks{AN 5.171}{AN 5.171}

\scevam{So\marginnote{1.1} I have heard. }At one time the Buddha was staying near \textsanskrit{Sāvatthī} in Jeta’s Grove, \textsanskrit{Anāthapiṇḍika}’s monastery. There the Buddha addressed the mendicants, “Mendicants!” 

“Venerable\marginnote{1.5} sir,” they replied. The Buddha said this: 

“A\marginnote{2.1} lay follower with five qualities is overcome by timidity. What five? They kill living creatures, steal, commit sexual misconduct, lie, and use alcoholic drinks that cause negligence. A lay follower with these five qualities is overcome by timidity. 

A\marginnote{3.1} lay follower with five qualities is self-assured. What five? They don’t kill living creatures, steal, commit sexual misconduct, lie, or use alcoholic drinks that cause negligence. A lay follower with these five qualities is self-assured.” 

%
\section*{{\suttatitleacronym AN 5.172}{\suttatitletranslation Assured }{\suttatitleroot Visāradasutta}}
\addcontentsline{toc}{section}{\tocacronym{AN 5.172} \toctranslation{Assured } \tocroot{Visāradasutta}}
\markboth{Assured }{Visāradasutta}
\extramarks{AN 5.172}{AN 5.172}

“A\marginnote{1.1} lay follower living at home with five qualities is not self-assured. What five? They kill living creatures, steal, commit sexual misconduct, lie, and use alcoholic drinks that cause negligence. A lay follower living at home with these five qualities is not self-assured. 

A\marginnote{2.1} lay follower living at home with these five qualities is self-assured. What five? They don’t kill living creatures, steal, commit sexual misconduct, lie, or use alcoholic drinks that cause negligence. A lay follower living at home with these five qualities is self-assured.” 

%
\section*{{\suttatitleacronym AN 5.173}{\suttatitletranslation Hell }{\suttatitleroot Nirayasutta}}
\addcontentsline{toc}{section}{\tocacronym{AN 5.173} \toctranslation{Hell } \tocroot{Nirayasutta}}
\markboth{Hell }{Nirayasutta}
\extramarks{AN 5.173}{AN 5.173}

“Mendicants,\marginnote{1.1} a lay follower with five qualities is cast down to hell. What five? They kill living creatures, steal, commit sexual misconduct, lie, and use alcoholic drinks that cause negligence. A lay follower with these five qualities is cast down to hell. 

A\marginnote{2.1} lay follower with five qualities is raised up to heaven. What five? They don’t kill living creatures, steal, commit sexual misconduct, lie, or use alcoholic drinks that cause negligence. A lay follower with these five qualities is raised up to heaven.” 

%
\section*{{\suttatitleacronym AN 5.174}{\suttatitletranslation Threats }{\suttatitleroot Verasutta}}
\addcontentsline{toc}{section}{\tocacronym{AN 5.174} \toctranslation{Threats } \tocroot{Verasutta}}
\markboth{Threats }{Verasutta}
\extramarks{AN 5.174}{AN 5.174}

Then\marginnote{1.1} the householder \textsanskrit{Anāthapiṇḍika} went up to the Buddha, bowed, and sat down to one side. The Buddha said to him: 

“Householder,\marginnote{2.1} unless these five dangers and threats are given up, one is said to be unethical, and is reborn in hell. What five? Killing living creatures, stealing, committing sexual misconduct, lying, and using alcoholic drinks that cause negligence. Unless these five dangers and threats are given up, one is said to be unethical, and is reborn in hell. 

Once\marginnote{3.1} these five dangers and threats are given up, one is said to be ethical, and is reborn in heaven. What five? Killing living creatures, stealing, committing sexual misconduct, lying, and using alcoholic drinks that cause negligence. Once these five dangers and threats are given up, one is said to be ethical, and is reborn in heaven. 

Anyone\marginnote{4.1} who kills living creatures creates dangers and threats both in the present life and in lives to come, and experiences mental pain and sadness. Anyone who refrains from killing living creatures creates no dangers and threats either in the present life or in lives to come, and doesn’t experience mental pain and sadness. So that danger and threat is quelled for anyone who refrains from killing living creatures. 

Anyone\marginnote{5.1} who steals … 

Anyone\marginnote{6.1} who commits sexual misconduct … 

Anyone\marginnote{7.1} who lies … 

Anyone\marginnote{8.1} who uses alcoholic drinks that cause negligence creates dangers and threats both in the present life and in lives to come, and experiences mental pain and sadness. Anyone who refrains from using alcoholic drinks that cause negligence creates no dangers and threats either in the present life or in lives to come, and doesn’t experience mental pain and sadness. So that danger and threat is quelled for anyone who refrains from using alcoholic drinks that cause negligence. 

\begin{verse}%
Take\marginnote{9.1} anyone in this world \\
who kills living creatures, \\
speaks falsely, steals, \\
commits adultery, \\
and indulges in drinking \\
alcohol and liquor. 

Unless\marginnote{10.1} they give up these five threats, \\
they’re said to be unethical. \\
When their body breaks up, that witless person \\
is reborn in hell. 

A\marginnote{11.1} person in the world doesn’t kill living creatures, \\
speak falsely, \\
steal, \\
commit adultery, \\
or indulge in drinking \\
alcohol and liquor. 

Giving\marginnote{12.1} up these five threats, \\
they’re said to be ethical. \\
When their body breaks up, that wise person \\
is reborn in a good place.” 

%
\end{verse}

%
\section*{{\suttatitleacronym AN 5.175}{\suttatitletranslation Outcaste }{\suttatitleroot Caṇḍālasutta}}
\addcontentsline{toc}{section}{\tocacronym{AN 5.175} \toctranslation{Outcaste } \tocroot{Caṇḍālasutta}}
\markboth{Outcaste }{Caṇḍālasutta}
\extramarks{AN 5.175}{AN 5.175}

“Mendicants,\marginnote{1.1} a lay follower with five qualities is an outcaste, a stain, and a reject among lay followers. What five? They’re faithless. They’re unethical. They practice noisy, superstitious rites, believing in omens rather than deeds. They seek outside of the Buddhist community for those worthy of religious donations. And they make offerings there first. A lay follower with these five qualities is an outcaste, a stain, and a reject among lay followers. 

A\marginnote{2.1} lay follower with five qualities is a gem, a pink lotus, and a white lotus among lay followers. What five? They’re faithful. They’re ethical. They don’t practice noisy, superstitious rites, and believe in deeds rather than omens. They don’t seek outside of the Buddhist community for those worthy of religious donations. And they don’t make offerings there first. A lay follower with these five qualities is a gem, a pink lotus, and a white lotus among lay followers.” 

%
\section*{{\suttatitleacronym AN 5.176}{\suttatitletranslation Rapture }{\suttatitleroot Pītisutta}}
\addcontentsline{toc}{section}{\tocacronym{AN 5.176} \toctranslation{Rapture } \tocroot{Pītisutta}}
\markboth{Rapture }{Pītisutta}
\extramarks{AN 5.176}{AN 5.176}

Then\marginnote{1.1} the householder \textsanskrit{Anāthapiṇḍika}, escorted by around five hundred lay followers, went up to the Buddha, bowed, and sat down to one side. The Buddha said to him: 

“Householders,\marginnote{2.1} you have supplied the mendicant \textsanskrit{Saṅgha} with robes, almsfood, lodgings, and medicines and supplies for the sick. But you should not be content with just this much. So you should train like this: ‘How can we, from time to time, enter and dwell in the rapture of seclusion?’ That’s how you should train.” 

When\marginnote{3.1} he said this, Venerable \textsanskrit{Sāriputta} said to the Buddha, “It’s incredible, sir, it’s amazing! How well said this was by the Buddha: ‘Householders, you have supplied the mendicant \textsanskrit{Saṅgha} with robes, almsfood, lodgings, and medicines and supplies for the sick. But you should not be content with just this much. So you should train like this: “How can we, from time to time, enter and dwell in the rapture of seclusion?” That’s how you should train.’ 

At\marginnote{3.10} a time when a noble disciple enters and dwells in the rapture of seclusion, five things aren’t present in him. The pain and sadness connected with sensual pleasures. The pleasure and happiness connected with sensual pleasures. The pain and sadness connected with the unskillful. The pleasure and happiness connected with the unskillful. The pain and sadness connected with the skillful. At a time when a noble disciple enters and dwells in the rapture of seclusion, these five things aren’t present in him.” 

“Good,\marginnote{4.1} good, \textsanskrit{Sāriputta}! At a time when a noble disciple enters and dwells in the rapture of seclusion, five things aren’t present in him. The pain and sadness connected with sensual pleasures. The pleasure and happiness connected with sensual pleasures. The pain and sadness connected with the unskillful. The pleasure and happiness connected with the unskillful. The pain and sadness connected with the skillful. At a time when a noble disciple enters and dwells in the rapture of seclusion, these five things aren’t present in him.” 

%
\section*{{\suttatitleacronym AN 5.177}{\suttatitletranslation Trades }{\suttatitleroot Vaṇijjāsutta}}
\addcontentsline{toc}{section}{\tocacronym{AN 5.177} \toctranslation{Trades } \tocroot{Vaṇijjāsutta}}
\markboth{Trades }{Vaṇijjāsutta}
\extramarks{AN 5.177}{AN 5.177}

“Mendicants,\marginnote{1.1} a lay follower should not engage in these five trades. What five? Trade in weapons, living creatures, meat, intoxicants, and poisons. A lay follower should not engage in these five trades.” 

%
\section*{{\suttatitleacronym AN 5.178}{\suttatitletranslation Kings }{\suttatitleroot Rājāsutta}}
\addcontentsline{toc}{section}{\tocacronym{AN 5.178} \toctranslation{Kings } \tocroot{Rājāsutta}}
\markboth{Kings }{Rājāsutta}
\extramarks{AN 5.178}{AN 5.178}

“What\marginnote{1.1} do you think, mendicants? Have you ever seen or heard of a person who has given up killing living creatures, and then the kings have them arrested for that, and execute, imprison, or banish them, or do what the case requires?” 

“No,\marginnote{1.5} sir.” 

“Good,\marginnote{1.6} mendicants! I too have never seen or heard of such a thing. Rather, the kings are informed of someone’s bad deed: ‘This person has murdered a man or a woman.’ Then the kings have them arrested for killing, and execute, imprison, or banish them, or do what the case requires. Have you ever seen or heard of such a case?” 

“Sir,\marginnote{1.14} we have seen it and heard of it, and we will hear of it again.” 

“What\marginnote{2.1} do you think, mendicants? Have you ever seen or heard of a person who has given up stealing, and then the kings have them arrested for that …?” 

“No,\marginnote{2.5} sir.” 

“Good,\marginnote{2.6} mendicants! I too have never seen or heard of such a thing. Rather, the kings are informed of someone’s bad deed: ‘This person took something from a village or wilderness, with the intention to commit theft.’ Then the kings have them arrested for stealing … Have you ever seen or heard of such a case?” 

“Sir,\marginnote{2.14} we have seen it and heard of it, and we will hear of it again.” 

“What\marginnote{3.1} do you think, mendicants? Have you ever seen or heard of a person who has given up sexual misconduct, and then the kings have them arrested for that …?” 

“No,\marginnote{3.5} sir.” 

“Good,\marginnote{3.6} mendicants! I too have never seen or heard of such a thing. Rather, the kings are informed of someone’s bad deed: ‘This person had sexual relations with women or maidens under someone else’s protection.’ Then the kings have them arrested for that … Have you ever seen or heard of such a case?” 

“Sir,\marginnote{3.14} we have seen it and heard of it, and we will hear of it again.” 

“What\marginnote{4.1} do you think, mendicants? Have you ever seen or heard of a person who has given up lying, and then the kings have them arrested for that …?” 

“No,\marginnote{4.5} sir.” 

“Good,\marginnote{4.6} mendicants! I too have never seen or heard of such a thing. Rather, the kings are informed of someone’s bad deed: ‘This person has ruined a householder or householder’s child by lying.’ Then the kings have them arrested for that … Have you ever seen or heard of such a case?” 

“Sir,\marginnote{4.14} we have seen it and heard of it, and we will hear of it again.” 

“What\marginnote{5.1} do you think, mendicants? Have you ever seen or heard of a person who has given up alcoholic drinks that cause negligence, and then the kings have them arrested for that, and execute, imprison, or banish them, or do what the case requires?” 

“No,\marginnote{5.5} sir.” 

“Good,\marginnote{5.6} mendicants! I too have never seen or heard of such a thing. Rather, the kings are informed of someone’s bad deed: ‘While under the influence of alcoholic drinks that cause negligence, this person murdered a woman or a man. Or they stole something from a village or wilderness. Or they had sexual relations with women or maidens under someone else’s protection. Or they ruined a householder or householder’s child by lying.’ Then the kings have them arrested for being under the influence of alcoholic drinks that cause negligence, and execute, imprison, or banish them, or do what the case requires. Have you ever seen or heard of such a case?” 

“Sir,\marginnote{5.17} we have seen it and heard of it, and we will hear of it again.” 

%
\section*{{\suttatitleacronym AN 5.179}{\suttatitletranslation A Layperson }{\suttatitleroot Gihisutta}}
\addcontentsline{toc}{section}{\tocacronym{AN 5.179} \toctranslation{A Layperson } \tocroot{Gihisutta}}
\markboth{A Layperson }{Gihisutta}
\extramarks{AN 5.179}{AN 5.179}

Then\marginnote{1.1} the householder \textsanskrit{Anāthapiṇḍika}, escorted by around five hundred lay followers, went up to the Buddha, bowed, and sat down to one side. Then the Buddha said to Venerable \textsanskrit{Sāriputta}: 

“You\marginnote{1.3} should know this, \textsanskrit{Sāriputta}, about those white-clothed laypeople whose actions are restrained in the five precepts, and who get four blissful meditations in the present life belonging to the higher mind when they want, without trouble or difficulty. They may, if they wish, declare of themselves: ‘I’ve finished with rebirth in hell, the animal realm, and the ghost realm. I’ve finished with all places of loss, bad places, the underworld. I am a stream-enterer! I’m not liable to be reborn in the underworld, and am bound for awakening.’ 

And\marginnote{2.1} what are the five precepts in which their actions are restrained? It’s when a noble disciple doesn’t kill living creatures, steal, commit sexual misconduct, lie, or use alcoholic drinks that cause negligence. These are the five precepts in which their actions are restrained. 

And\marginnote{3.1} what are the four blissful meditations in the present life belonging to the higher mind that they get when they want, without trouble or difficulty? 

It’s\marginnote{3.2} when a noble disciple has experiential confidence in the Buddha: ‘That Blessed One is perfected, a fully awakened Buddha, accomplished in knowledge and conduct, holy, knower of the world, supreme guide for those who wish to train, teacher of gods and humans, awakened, blessed.’ This is the first blissful meditation in the present life belonging to the higher mind, which they achieve in order to purify the unpurified mind and cleanse the unclean mind. 

Furthermore,\marginnote{4.1} a noble disciple has experiential confidence in the teaching: ‘The teaching is well explained by the Buddha—visible in this very life, immediately effective, inviting inspection, relevant, so that sensible people can know it for themselves.’ This is the second blissful meditation … 

Furthermore,\marginnote{5.1} a noble disciple has experiential confidence in the \textsanskrit{Saṅgha}: ‘The \textsanskrit{Saṅgha} of the Buddha’s disciples is practicing the way that’s good, direct, methodical, and proper. It consists of the four pairs, the eight individuals. This is the \textsanskrit{Saṅgha} of the Buddha’s disciples that is worthy of offerings dedicated to the gods, worthy of hospitality, worthy of a religious donation, worthy of greeting with joined palms, and is the supreme field of merit for the world.’ This is the third blissful meditation … 

Furthermore,\marginnote{6.1} a noble disciple’s ethical conduct is loved by the noble ones, unbroken, impeccable, spotless, and unmarred, liberating, praised by sensible people, not mistaken, and leading to immersion. This is the fourth blissful meditation in the present life belonging to the higher mind, which they achieve in order to purify the unpurified mind and cleanse the unclean mind. These are the four blissful meditations in the present life belonging to the higher mind that they get when they want, without trouble or difficulty. 

You\marginnote{7.1} should know this, \textsanskrit{Sāriputta}, about those white-clothed laypeople whose actions are restrained in the five precepts, and who get four blissful meditations in the present life belonging to the higher mind when they want, without trouble or difficulty. They may, if they wish, declare of themselves: ‘I’ve finished with rebirth in hell, the animal realm, and the ghost realm. I’ve finished with all places of loss, bad places, the underworld. I am a stream-enterer! I’m not liable to be reborn in the underworld, and am bound for awakening.’ 

\begin{verse}%
Seeing\marginnote{8.1} the peril in the hells, \\
you should shun bad deeds. \\
Taking up the teaching of the noble ones, \\
an astute person should shun them. 

You\marginnote{9.1} shouldn’t harm living beings, \\
so long as strength is found. \\
Nor should you knowingly speak falsehood, \\
or take what is not given. 

Content\marginnote{10.1} with your own partners, \\
you should stay away from the partners of others. \\
A man shouldn’t drink liquor or wine, \\
as they confuse the mind. 

You\marginnote{11.1} should recollect the Buddha, \\
and reflect on the teaching. \\
You should develop a harmless mind of welfare, \\
which leads to the realms of gods. 

When\marginnote{12.1} suitable gifts to give are available \\
to someone who seeks and needs merit, \\
a religious donation is abundant \\
if given first to the peaceful ones. 

I\marginnote{13.1} will tell of the peaceful ones, \\
\textsanskrit{Sāriputta}, listen to me. \\
Cows may be black or white, \\
red or tawny, 

mottled\marginnote{14.1} or uniform, \\
or pigeon-colored. \\
But when one is born among them, \\
the bull that’s tamed, 

—a\marginnote{15.1} behemoth, powerful, \\
well-paced in pulling forward—\\
they yoke the load just to him, \\
regardless of his color. 

So\marginnote{16.1} it is for humans, \\
wherever they may be born, \\
—among aristocrats, brahmins, merchants, \\
workers, or outcastes and scavengers—

but\marginnote{17.1} when one is born among them, \\
tamed, true to their vows. \\
Firm in principle, accomplished in ethical conduct, \\
truthful, conscientious, 

they’ve\marginnote{18.1} given up birth and death, \\
and have completed the spiritual journey. \\
With burden put down, detached, \\
they’ve completed the task and are free of defilements. 

Gone\marginnote{19.1} beyond all things, \\
they’re extinguished by not grasping. \\
In that flawless field, \\
a religious donation is abundant. 

Fools\marginnote{20.1} who don’t understand \\
—stupid, unlearned—\\
give their gifts to those outside, \\
and don’t attend the peaceful ones. 

But\marginnote{21.1} those who do attend the peaceful ones \\
—wise, esteemed as sages—\\
and whose faith in the Holy One \\
has roots planted deep, 

they\marginnote{22.1} go to the realm of the gods, \\
or are born here in a good family. \\
Gradually those astute ones \\
reach extinguishment.” 

%
\end{verse}

%
\section*{{\suttatitleacronym AN 5.180}{\suttatitletranslation About Gavesī }{\suttatitleroot Gavesīsutta}}
\addcontentsline{toc}{section}{\tocacronym{AN 5.180} \toctranslation{About Gavesī } \tocroot{Gavesīsutta}}
\markboth{About Gavesī }{Gavesīsutta}
\extramarks{AN 5.180}{AN 5.180}

At\marginnote{1.1} one time the Buddha was wandering in the land of the Kosalans together with a large \textsanskrit{Saṅgha} of mendicants. While traveling along a road the Buddha saw a large sal grove in a certain spot. He left the road, went to the sal grove, and plunged deep into it. And at a certain spot he smiled. 

Then\marginnote{2.1} Venerable Ānanda thought, “What is the cause, what is the reason why the Buddha smiled? Realized Ones do not smile for no reason.” 

So\marginnote{2.4} Venerable Ānanda said to the Buddha, “What is the cause, what is the reason why the Buddha smiled? Realized Ones do not smile for no reason.” 

“Once\marginnote{3.1} upon a time, Ānanda, there was a city in this spot that was successful and prosperous and full of people. And Kassapa, a blessed one, a perfected one, a fully awakened Buddha, lived supported by that city. 

He\marginnote{3.3} had a lay follower called \textsanskrit{Gavesī} who had not fulfilled all the precepts. And the five hundred lay followers who were taught and advised by \textsanskrit{Gavesī} also had not fulfilled all the precepts. Then \textsanskrit{Gavesī} thought: ‘I’m the helper, leader, and adviser of these five hundred lay followers, yet neither I nor they have fulfilled the precepts. We’re the same, I’m in no way better. So let me do better.’ 

Then\marginnote{4.1} \textsanskrit{Gavesī} went to those five hundred lay followers and said to them: ‘From this day forth may the venerables remember me as one who has fulfilled the precepts.’ Then those five hundred lay followers thought: ‘The venerable \textsanskrit{Gavesī} is our helper, leader, and adviser, and now he will fulfill the precepts. Why don’t we do the same?’ Then those five hundred lay followers went to \textsanskrit{Gavesī} and said to him: ‘From this day forth may Venerable \textsanskrit{Gavesī} remember these five hundred lay followers as having fulfilled the precepts.’ 

Then\marginnote{4.9} \textsanskrit{Gavesī} thought: ‘I’m the helper, leader, and adviser of these five hundred lay followers, and both I and they have fulfilled the precepts. We’re the same, I’m in no way better. So let me do better.’ 

Then\marginnote{5.1} \textsanskrit{Gavesī} went to those five hundred lay followers and said to them: ‘From this day forth may the venerables remember me as one who is celibate, set apart, avoiding the common practice of sex.’ Then those five hundred lay followers did the same. … 

Then\marginnote{5.9} \textsanskrit{Gavesī} thought: ‘These five hundred lay followers … are celibate, set apart, avoiding the common practice of sex. We’re the same, I’m in no way better. So let me do better.’ 

Then\marginnote{6.1} \textsanskrit{Gavesī} went to those five hundred lay followers and said to them: ‘From this day forth may the venerables remember me as one who eats in one part of the day, abstaining from eating at night, and from food at the wrong time.’ Then those five hundred lay followers did the same. … 

Then\marginnote{6.9} \textsanskrit{Gavesī} thought: ‘These five hundred lay followers … eat in one part of the day, abstaining from eating at night, and food at the wrong time. We’re the same, I’m in no way better. So let me do better.’ 

Then\marginnote{7.1} the lay follower \textsanskrit{Gavesī} went up to the blessed one Kassapa, the perfected one, the fully awakened Buddha and said to him: ‘Sir, may I receive the going forth, the ordination in the Buddha’s presence?’ And he received the going forth, the ordination in the Buddha’s presence. Not long after his ordination, the mendicant \textsanskrit{Gavesī}, living alone, withdrawn, diligent, keen, and resolute, realized the supreme culmination of the spiritual path in this very life. He lived having achieved with his own insight the goal for which gentlemen rightly go forth from the lay life to homelessness. 

He\marginnote{7.5} understood: ‘Rebirth is ended; the spiritual journey has been completed; what had to be done has been done; there is no return to any state of existence.’ And the mendicant \textsanskrit{Gavesī} became one of the perfected. 

Then\marginnote{8.1} those five hundred lay followers thought: ‘Venerable \textsanskrit{Gavesī} is our helper, leader, and adviser, He has shaved off his hair and beard, dressed in ocher robes, and gone forth from the lay life to homelessness. Why don’t we do the same?’ Then those five hundred lay followers went up to the blessed one Kassapa, the perfected one, the fully awakened Buddha and said to him: ‘Sir, may we receive the going forth and ordination in the Buddha’s presence?’ And they did receive the going forth and ordination in the Buddha’s presence. 

Then\marginnote{9.1} the mendicant \textsanskrit{Gavesī} thought: ‘I get the supreme bliss of freedom whenever I want, without trouble or difficulty. Oh, may these five hundred mendicants do the same!’ Then those five hundred mendicants, living alone, withdrawn, diligent, keen, and resolute, soon realized the supreme culmination of the spiritual path in this very life. They lived having achieved with their own insight the goal for which gentlemen rightly go forth from the lay life to homelessness. 

They\marginnote{9.6} understood: ‘Rebirth is ended, the spiritual journey has been completed, what had to be done has been done, there is no return to any state of existence.’ 

And\marginnote{10.1} so, Ānanda, those five hundred mendicants headed by \textsanskrit{Gavesī}, trying to go higher and higher, better and better, realized the supreme bliss of freedom. So you should train like this: ‘Trying to go higher and higher, better and better, we will realize the supreme bliss of freedom.’ That’s how you should train.” 

%
\addtocontents{toc}{\let\protect\contentsline\protect\nopagecontentsline}
\chapter*{The Chapter on Wilderness Dwellers }
\addcontentsline{toc}{chapter}{\tocchapterline{The Chapter on Wilderness Dwellers }}
\addtocontents{toc}{\let\protect\contentsline\protect\oldcontentsline}

%
\section*{{\suttatitleacronym AN 5.181}{\suttatitletranslation Wilderness Dwellers }{\suttatitleroot Āraññikasutta}}
\addcontentsline{toc}{section}{\tocacronym{AN 5.181} \toctranslation{Wilderness Dwellers } \tocroot{Āraññikasutta}}
\markboth{Wilderness Dwellers }{Āraññikasutta}
\extramarks{AN 5.181}{AN 5.181}

“Mendicants,\marginnote{1.1} there are these five kinds of wilderness dwellers. What five? A person may be wilderness dweller because of stupidity and folly. Or because of wicked desires, being naturally full of desires. Or because of madness and mental disorder. Or because it is praised by the Buddhas and their disciples. Or for the sake of having few wishes, for the sake of contentment, self-effacement, seclusion, and simplicity. These are the five kinds of wilderness dwellers. But the person who dwells in the wilderness for the sake of having few wishes is the foremost, best, chief, highest, and finest of the five. 

From\marginnote{2.1} a cow comes milk, from milk comes curds, from curds come butter, from butter comes ghee, and from ghee comes cream of ghee. And the cream of ghee is said to be the best of these. In the same way, the person who dwells in the wilderness for the sake of having few wishes is the foremost, best, chief, highest, and finest of the five.” 

%
\section*{{\suttatitleacronym AN 5.182}{\suttatitletranslation Robes }{\suttatitleroot Cīvarasutta}}
\addcontentsline{toc}{section}{\tocacronym{AN 5.182} \toctranslation{Robes } \tocroot{Cīvarasutta}}
\markboth{Robes }{Cīvarasutta}
\extramarks{AN 5.182}{AN 5.182}

“Mendicants\marginnote{1.1} there are these five kinds of people who wear rag robes. What five? A person may wear rag robes because of stupidity … bad desires … madness … because it is praised by the Buddhas … or for the sake of having few wishes … These are the five kinds of people who wear rag robes.” 

%
\section*{{\suttatitleacronym AN 5.183}{\suttatitletranslation Dwelling at the Root of a Tree }{\suttatitleroot Rukkhamūlikasutta}}
\addcontentsline{toc}{section}{\tocacronym{AN 5.183} \toctranslation{Dwelling at the Root of a Tree } \tocroot{Rukkhamūlikasutta}}
\markboth{Dwelling at the Root of a Tree }{Rukkhamūlikasutta}
\extramarks{AN 5.183}{AN 5.183}

“Mendicants,\marginnote{1.1} there are these five kinds of people who dwell at the root of a tree. What five? A person may dwell at the root of a tree because of stupidity … bad desires … madness … because it is praised by the Buddhas … or for the sake of having few wishes … These are the five kinds of people who dwell at the root of a tree.” 

%
\section*{{\suttatitleacronym AN 5.184}{\suttatitletranslation Charnel Ground Dwellers }{\suttatitleroot Sosānikasutta}}
\addcontentsline{toc}{section}{\tocacronym{AN 5.184} \toctranslation{Charnel Ground Dwellers } \tocroot{Sosānikasutta}}
\markboth{Charnel Ground Dwellers }{Sosānikasutta}
\extramarks{AN 5.184}{AN 5.184}

“Mendicants,\marginnote{1.1} there are these five kinds of people who dwell in a charnel ground. What five? A person may dwell in a charnel ground because of stupidity … bad desires … madness … because it is praised by the Buddhas … or for the sake of having few wishes … These are the five kinds of people who dwell in a charnel ground.” 

%
\section*{{\suttatitleacronym AN 5.185}{\suttatitletranslation Open Air Dwellers }{\suttatitleroot Abbhokāsikasutta}}
\addcontentsline{toc}{section}{\tocacronym{AN 5.185} \toctranslation{Open Air Dwellers } \tocroot{Abbhokāsikasutta}}
\markboth{Open Air Dwellers }{Abbhokāsikasutta}
\extramarks{AN 5.185}{AN 5.185}

“There\marginnote{1.1} are these five kinds of people who dwell in the open air. …” 

%
\section*{{\suttatitleacronym AN 5.186}{\suttatitletranslation Those Who Never Lie Down }{\suttatitleroot Nesajjikasutta}}
\addcontentsline{toc}{section}{\tocacronym{AN 5.186} \toctranslation{Those Who Never Lie Down } \tocroot{Nesajjikasutta}}
\markboth{Those Who Never Lie Down }{Nesajjikasutta}
\extramarks{AN 5.186}{AN 5.186}

“There\marginnote{1.1} are these five kinds of people who never lie down. …” 

%
\section*{{\suttatitleacronym AN 5.187}{\suttatitletranslation Those Who Sleep Wherever a Mat is Laid }{\suttatitleroot Yathāsanthatikasutta}}
\addcontentsline{toc}{section}{\tocacronym{AN 5.187} \toctranslation{Those Who Sleep Wherever a Mat is Laid } \tocroot{Yathāsanthatikasutta}}
\markboth{Those Who Sleep Wherever a Mat is Laid }{Yathāsanthatikasutta}
\extramarks{AN 5.187}{AN 5.187}

“There\marginnote{1.1} are these five kinds of people who sleep wherever they lay their mat. …” 

%
\section*{{\suttatitleacronym AN 5.188}{\suttatitletranslation Those Who Eat in One Sitting }{\suttatitleroot Ekāsanikasutta}}
\addcontentsline{toc}{section}{\tocacronym{AN 5.188} \toctranslation{Those Who Eat in One Sitting } \tocroot{Ekāsanikasutta}}
\markboth{Those Who Eat in One Sitting }{Ekāsanikasutta}
\extramarks{AN 5.188}{AN 5.188}

“There\marginnote{1.1} are these five kinds of people who eat in one sitting per day. …” 

%
\section*{{\suttatitleacronym AN 5.189}{\suttatitletranslation Refusers of Late Food }{\suttatitleroot Khalupacchābhattikasutta}}
\addcontentsline{toc}{section}{\tocacronym{AN 5.189} \toctranslation{Refusers of Late Food } \tocroot{Khalupacchābhattikasutta}}
\markboth{Refusers of Late Food }{Khalupacchābhattikasutta}
\extramarks{AN 5.189}{AN 5.189}

“There\marginnote{1.1} are these five kinds of people who refuse to accept food offered after the meal has begun. …” 

%
\section*{{\suttatitleacronym AN 5.190}{\suttatitletranslation Those Who Eat Only From the Almsbowl }{\suttatitleroot Pattapiṇḍikasutta}}
\addcontentsline{toc}{section}{\tocacronym{AN 5.190} \toctranslation{Those Who Eat Only From the Almsbowl } \tocroot{Pattapiṇḍikasutta}}
\markboth{Those Who Eat Only From the Almsbowl }{Pattapiṇḍikasutta}
\extramarks{AN 5.190}{AN 5.190}

“Mendicants,\marginnote{1.1} there are these five kinds of people who eat only from the almsbowl. What five? A person may eat only from the almsbowl because of stupidity and folly. Or because of wicked desires, being naturally full of desires. Or because of madness and mental disorder. Or because it is praised by the Buddhas and their disciples. Or for the sake of having few wishes, for the sake of contentment, self-effacement, seclusion, and simplicity. These are the five kinds of people who eat only from the almsbowl. But the person who eats only from the almsbowl for the sake of having few wishes is the foremost, best, chief, highest, and finest of the five. 

From\marginnote{2.1} a cow comes milk, from milk comes curds, from curds come butter, from butter comes ghee, and from ghee comes cream of ghee. And the cream of ghee is said to be the best of these. In the same way, the person who eats only from the almsbowl for the sake of having few wishes is the foremost, best, chief, highest, and finest of the five.” 

%
\addtocontents{toc}{\let\protect\contentsline\protect\nopagecontentsline}
\chapter*{The Chapter on Brahmins }
\addcontentsline{toc}{chapter}{\tocchapterline{The Chapter on Brahmins }}
\addtocontents{toc}{\let\protect\contentsline\protect\oldcontentsline}

%
\section*{{\suttatitleacronym AN 5.191}{\suttatitletranslation Dogs }{\suttatitleroot Soṇasutta}}
\addcontentsline{toc}{section}{\tocacronym{AN 5.191} \toctranslation{Dogs } \tocroot{Soṇasutta}}
\markboth{Dogs }{Soṇasutta}
\extramarks{AN 5.191}{AN 5.191}

“Mendicants,\marginnote{1.1} these five traditions of the brahmins are seen these days among dogs, but not among brahmins. What five? 

In\marginnote{1.3} the past brahmins had sex only with brahmin women, not with others. These days brahmins have sex with both brahmin women and others. But these days dogs have sex only with female dogs, not with other species. This is the first tradition of the brahmins seen these days among dogs, but not among brahmins. 

In\marginnote{2.1} the past brahmins had sex only with brahmin women in the fertile half of the month that starts with menstruation, not at other times. These days brahmins have sex with brahmin women both in the fertile half of the month and at other times. But these days dogs have sex only with female dogs when they are in heat, not at other times. This is the second tradition of the brahmins seen these days among dogs, but not among brahmins. 

In\marginnote{3.1} the past brahmins neither bought nor sold brahmin women. They lived together because they loved each other and wanted their family line to continue. These days brahmins both buy and sell brahmin women. They live together whether they love each other or not and they want their family line to continue. But these days dogs neither buy nor sell female dogs. They live together because they’re attracted to each other and want their family line to continue. This is the third tradition of the brahmins seen these days among dogs, but not among brahmins. 

In\marginnote{4.1} the past brahmins did not store up money, grain, silver, or gold. These days brahmins do store up money, grain, silver, and gold. But these days dogs don’t store up money, grain, silver, or gold. This is the fourth tradition of the brahmins seen these days among dogs, but not among brahmins. 

In\marginnote{5.1} the past brahmins went looking for almsfood for dinner in the evening, and for breakfast in the morning. These days brahmins eat as much as they like until their bellies are full, then take away the leftovers. But these days dogs go looking for dinner in the evening, and for breakfast in the morning. This is the fifth tradition of the brahmins seen these days among dogs, but not among brahmins. 

These\marginnote{5.5} five traditions of the brahmins are seen these days among dogs, but not among brahmins.” 

%
\section*{{\suttatitleacronym AN 5.192}{\suttatitletranslation With the Brahmin Doṇa }{\suttatitleroot Doṇabrāhmaṇasutta}}
\addcontentsline{toc}{section}{\tocacronym{AN 5.192} \toctranslation{With the Brahmin Doṇa } \tocroot{Doṇabrāhmaṇasutta}}
\markboth{With the Brahmin Doṇa }{Doṇabrāhmaṇasutta}
\extramarks{AN 5.192}{AN 5.192}

Then\marginnote{1.1} \textsanskrit{Doṇa} the brahmin went up to the Buddha, and exchanged greetings with him. When the greetings and polite conversation were over, \textsanskrit{Doṇa} sat down to one side, and said to the Buddha: 

“Master\marginnote{2.1} Gotama, I have heard that the ascetic Gotama doesn’t bow to old brahmins, the elderly and senior, who are advanced in years and have reached the final stage of life; nor does he rise in their presence or offer them a seat. And this is indeed the case, for Master Gotama does not bow to old brahmins, elderly and senior, who are advanced in years and have reached the final stage of life; nor does he rise in their presence or offer them a seat. This is not appropriate, Master Gotama.” 

“\textsanskrit{Doṇa},\marginnote{2.6} do you too claim to be a brahmin?” 

“Master\marginnote{2.7} Gotama, if anyone should be rightly called a brahmin, it’s me. For I am well born on both my mother’s and father’s side, of pure descent, irrefutable and impeccable in questions of ancestry back to the seventh paternal generation. I recite and remember the hymns, and have mastered the three Vedas, together with their vocabularies, ritual, phonology and etymology, and the testament as fifth. I know philology and grammar, and am well versed in cosmology and the marks of a great man.” 

“\textsanskrit{Doṇa},\marginnote{3.1} the ancient brahmin hermits were \textsanskrit{Aṭṭhaka}, \textsanskrit{Vāmaka}, \textsanskrit{Vāmadeva}, \textsanskrit{Vessāmitta}, Yamadaggi, \textsanskrit{Aṅgīrasa}, \textsanskrit{Bhāradvāja}, \textsanskrit{Vāseṭṭha}, Kassapa, and Bhagu. They were the authors and propagators of the hymns, whose hymnal was sung and propagated and compiled in ancient times. These days, brahmins continue to sing and chant it. They continue chanting what was chanted, reciting what was recited, and teaching what was taught. Those seers described five kinds of brahmins. A brahmin who is equal to \textsanskrit{Brahmā}, one who is equal to a god, one who toes the line, one who crosses the line, and the fifth is a brahmin outcaste. Which one of these are you, \textsanskrit{Doṇa}?” 

“Master\marginnote{4.1} Gotama, we don’t know about these five kinds of brahmins. We just know the word ‘brahmin’. Master Gotama, please teach me this matter so I can learn about these five brahmins.” 

“Well\marginnote{4.3} then, brahmin, listen and pay close attention, I will speak.” 

“Yes\marginnote{4.4} sir,” \textsanskrit{Doṇa} replied. The Buddha said this: 

“\textsanskrit{Doṇa},\marginnote{5.1} how is a brahmin equal to \textsanskrit{Brahmā}? 

It’s\marginnote{5.2} when a brahmin is well born on both the mother’s and the father’s sides, coming from a clean womb back to the seventh paternal generation, incontestable and irreproachable in questions of ancestry. For forty-eight years he leads the boy’s spiritual life studying the hymns. Then he seeks a fee for his teacher, but only by legitimate means, not illegitimate. 

In\marginnote{6.1} this context, \textsanskrit{Doṇa}, what is legitimate? Not by farming, trade, raising cattle, archery, government service, or one of the professions, but solely by living on alms, not scorning the alms bowl. Having offered the fee to his teacher, he shaves off his hair and beard, dresses in ocher robes, and goes forth from the lay life to homelessness. 

Then\marginnote{6.4} they meditate spreading a heart full of love to one direction, and to the second, and to the third, and to the fourth. In the same way above, below, across, everywhere, all around, they spread a heart full of love to the whole world—abundant, expansive, limitless, free of enmity and ill will. They meditate spreading a heart full of compassion … rejoicing … equanimity to one direction, and to the second, and to the third, and to the fourth. In the same way above, below, across, everywhere, all around, they spread a heart full of equanimity to the whole world—abundant, expansive, limitless, free of enmity and ill will. Having developed these four \textsanskrit{Brahmā} meditations, when the body breaks up, after death, they’re reborn in a good place, a \textsanskrit{Brahmā} realm. 

That’s\marginnote{6.9} how a brahmin is equal to \textsanskrit{Brahmā}. 

And\marginnote{7.1} how is a brahmin equal to a god? 

It’s\marginnote{7.2} when a brahmin is well born on both the mother’s and the father’s sides … Having offered the fee to his teacher, he seeks a wife, but only by legitimate means, not illegitimate. 

In\marginnote{8.1} this context, \textsanskrit{Doṇa}, what is legitimate? Not by buying or selling, he only accepts a brahmin woman by the pouring of water. He has sex only with a brahmin woman. He does not have sex with a woman from a caste of aristocrats, merchants, workers, outcastes, hunters, bamboo workers, chariot-makers, or waste-collectors. Nor does he have sex with women who are pregnant, breastfeeding, or outside the fertile half of the month that starts with menstruation. 

And\marginnote{8.4} why does the brahmin not have sex with a pregnant woman? If a brahmin had sex with a pregnant woman, the boy or girl would be born in too much filth. That’s why the brahmin doesn’t have sex with a pregnant woman. 

And\marginnote{8.7} why does the brahmin not have sex with a breastfeeding woman? If a brahmin had sex with a breastfeeding woman, the boy or girl would drink back the semen. That’s why the brahmin doesn’t have sex with a breastfeeding woman. 

And\marginnote{8.10} why does the brahmin not have sex outside the fertile half of the month that starts with menstruation? Because his brahmin wife is not there for sensual pleasure, fun, and enjoyment, but only for procreation. Having ensured his progeny through sex, he shaves off his hair and beard, dresses in ocher robes, and goes forth from the lay life to homelessness. 

When\marginnote{8.12} he has gone forth, quite secluded from sensual pleasures, secluded from unskillful qualities, he enters and remains in the first absorption … second absorption … third absorption … fourth absorption. Having developed these four absorptions, when the body breaks up, after death, they’re reborn in a good place, a heavenly realm. 

That’s\marginnote{8.14} how a brahmin is equal to god. 

And\marginnote{9.1} how does a brahmin toe the line? 

It’s\marginnote{9.2} when a brahmin is well born on both the mother’s and the father’s sides … 

Not\marginnote{10.1} by buying or selling, he only accepts a brahmin woman by the pouring of water. Having ensured his progeny through sex, his child makes him happy. Because of this attachment he stays in his family property, and does not go forth from the lay life to homelessness. 

As\marginnote{10.12} far as the line of the ancient brahmins extends, he doesn’t cross over it. That’s why he’s called a brahmin who toes the line. 

That’s\marginnote{10.14} how a brahmin toes the line. 

And\marginnote{11.1} how does a brahmin cross the line? 

It’s\marginnote{11.2} when a brahmin is well born on both the mother’s and the father’s sides … 

Having\marginnote{12.1} offered a fee for his teacher, he seeks a wife by both legitimate and illegitimate means. That is, by buying or selling, as well as accepting a brahmin woman by the pouring of water. He has sex with a brahmin woman, as well as with a woman from a caste of aristocrats, merchants, workers, outcastes, hunters, bamboo workers, chariot-makers, or waste-collectors. And he has sex with women who are pregnant, breastfeeding, or outside the fertile half of the month that starts with menstruation. His brahmin wife is there for sensual pleasure, fun, and enjoyment, as well as for procreation. 

As\marginnote{12.6} far as the line of the ancient brahmins extends, he crosses over it. That’s why he’s called a brahmin who crosses the line. 

That’s\marginnote{12.8} how a brahmin crosses the line. 

And\marginnote{13.1} how is a brahmin a brahmin outcaste? 

It’s\marginnote{13.2} when a brahmin is well born on both the mother’s and the father’s sides, coming from a clean womb back to the seventh paternal generation, incontestable and irreproachable in questions of ancestry. For forty-eight years he leads the virginal spiritual life studying the hymns. Then he seeks a fee for his teacher by legitimate means and illegitimate means. By farming, trade, raising cattle, archery, government service, or one of the professions, not solely by living on alms, not scorning the alms bowl. 

Having\marginnote{14.1} offered a fee for his teacher, he seeks a wife by both legitimate and illegitimate means. That is, by buying or selling, as well as accepting a brahmin woman by the pouring of water. He has sex with a brahmin woman, as well as with a woman from a caste of aristocrats, merchants, workers, outcastes, hunters, bamboo workers, chariot-makers, or waste-collectors. And he has sex with women who are pregnant, breastfeeding, or outside the fertile half of the month that starts with menstruation. His brahmin wife is there for sensual pleasure, fun, and enjoyment, as well as for procreation. 

He\marginnote{14.4} earns a living by any kind of work. The brahmins say to him, ‘My good man, why is it that you claim to be a brahmin, but you earn a living by any kind of work?’ 

He\marginnote{14.7} says, ‘It’s like a fire that burns both pure and filthy substances, but doesn’t become corrupted by them. In the same way, my good man, if a brahmin earns a living by any kind of work, he is not corrupted by that.’ 

A\marginnote{14.10} brahmin is called a brahmin outcaste because he earns a living by any kind of work. 

That’s\marginnote{14.11} how a brahmin is a brahmin outcaste. 

\textsanskrit{Doṇa},\marginnote{15.1} the ancient brahmin hermits were \textsanskrit{Aṭṭhaka}, \textsanskrit{Vāmaka}, \textsanskrit{Vāmadeva}, \textsanskrit{Vessāmitta}, Yamadaggi, \textsanskrit{Aṅgīrasa}, \textsanskrit{Bhāradvāja}, \textsanskrit{Vāseṭṭha}, Kassapa, and Bhagu. They were the authors and propagators of the hymns, whose hymnal was sung and propagated and compiled in ancient times. These days, brahmins continue to sing and chant it. They continue chanting what was chanted, reciting what was recited, and teaching what was taught. 

Those\marginnote{15.2} hermits described five kinds of brahmins. A brahmin who is equal to \textsanskrit{Brahmā}, one who is equal to a god, one who toes the line, one who crosses the line, and the fifth is a brahmin outcaste. Which one of these are you, \textsanskrit{Doṇa}?” 

“This\marginnote{16.1} being so, Master Gotama, I don’t even qualify as a brahmin outcaste. 

Excellent,\marginnote{16.2} Master Gotama! … From this day forth, may Master Gotama remember me as a lay follower who has gone for refuge for life.” 

%
\section*{{\suttatitleacronym AN 5.193}{\suttatitletranslation With Saṅgārava }{\suttatitleroot Saṅgāravasutta}}
\addcontentsline{toc}{section}{\tocacronym{AN 5.193} \toctranslation{With Saṅgārava } \tocroot{Saṅgāravasutta}}
\markboth{With Saṅgārava }{Saṅgāravasutta}
\extramarks{AN 5.193}{AN 5.193}

Then\marginnote{1.1} \textsanskrit{Saṅgārava} the brahmin went up to the Buddha, and exchanged greetings with him. When the greetings and polite conversation were over, \textsanskrit{Saṅgārava} sat down to one side, and said to the Buddha: 

“What\marginnote{1.3} is the cause, Master Gotama, what is the reason why sometimes even hymns that are long-practiced don’t spring to mind, let alone those that are not practiced? And why is it that sometimes even hymns that are long-unpracticed spring to mind, let alone those that are practiced?” 

“Brahmin,\marginnote{2.1} there’s a time when your heart is overcome and mired in sensual desire and you don’t truly understand the escape from sensual desire that has arisen. At that time you don’t truly know or see what is good for yourself, good for another, or good for both. Even hymns that are long-practiced don’t spring to mind, let alone those that are not practiced. Suppose there was a bowl of water that was mixed with dye such as red lac, turmeric, indigo, or rose madder. Even a person with good eyesight checking their own reflection wouldn’t truly know it or see it. In the same way, when your heart is overcome and mired in sensual desire … Even hymns that are long-practiced don’t spring to mind, let alone those that are not practiced. 

Furthermore,\marginnote{3.1} when your heart is overcome and mired in ill will … Even hymns that are long-practiced don’t spring to mind, let alone those that are not practiced. Suppose there was a bowl of water that was heated by fire, boiling and bubbling. Even a person with good eyesight checking their own reflection wouldn’t truly know it or see it. In the same way, when your heart is overcome and mired in ill will … Even hymns that are long-practiced don’t spring to mind, let alone those that are not practiced. 

Furthermore,\marginnote{4.1} when your heart is overcome and mired in dullness and drowsiness … Even hymns that are long-practiced don’t spring to mind, let alone those that are not practiced. Suppose there was a bowl of water overgrown with moss and aquatic plants. Even a person with good eyesight checking their own reflection wouldn’t truly know it or see it. In the same way, when your heart is overcome and mired in dullness and drowsiness … Even hymns that are long-practiced don’t spring to mind, let alone those that are not practiced. 

Furthermore,\marginnote{5.1} when your heart is overcome and mired in restlessness and remorse … Even hymns that are long-practiced don’t spring to mind, let alone those that are not practiced. Suppose there was a bowl of water stirred by the wind, churning, swirling, and rippling. Even a person with good eyesight checking their own reflection wouldn’t truly know it or see it. In the same way, when your heart is overcome and mired in restlessness and remorse … Even hymns that are long-practiced don’t spring to mind, let alone those that are not practiced. 

Furthermore,\marginnote{6.1} when your heart is overcome and mired in doubt … Even hymns that are long-practiced don’t spring to mind, let alone those that are not practiced. Suppose there was a bowl of water that was cloudy, murky, and muddy, hidden in the darkness. Even a person with good eyesight checking their own reflection wouldn’t truly know it or see it. In the same way, there’s a time when your heart is overcome and mired in doubt and you don’t truly understand the escape from doubt that has arisen. At that time you don’t truly know or see what is good for yourself, good for another, or good for both. Even hymns that are long-practiced don’t spring to mind, let alone those that are not practiced. 

There’s\marginnote{7.1} a time when your heart is not overcome and mired in sensual desire and you truly understand the escape from sensual desire that has arisen. At that time you truly know and see what is good for yourself, good for another, and good for both. Even hymns that are long-unpracticed spring to mind, let alone those that are practiced. Suppose there was a bowl of water that was not mixed with dye such as red lac, turmeric, indigo, or rose madder. A person with good eyesight checking their own reflection would truly know it and see it. In the same way, when your heart is not overcome and mired in sensual desire … Even hymns that are long-unpracticed spring to mind, let alone those that are practiced. 

Furthermore,\marginnote{8.1} when your heart is not overcome and mired in ill will … Even hymns that are long-unpracticed spring to mind, let alone those that are practiced. Suppose there was a bowl of water that’s not heated by a fire, boiling and bubbling. A person with good eyesight checking their own reflection would truly know it and see it. In the same way, when your heart is not overcome and mired in ill will … Even hymns that are long-unpracticed spring to mind, let alone those that are practiced. 

Furthermore,\marginnote{9.1} when your heart is not overcome and mired in dullness and drowsiness … Even hymns that are long-unpracticed spring to mind, let alone those that are practiced. Suppose there was a bowl of water that’s not overgrown with moss and aquatic plants. A person with good eyesight checking their own reflection would truly know it and see it. In the same way, when your heart is not overcome and mired in dullness and drowsiness … Even hymns that are long-unpracticed spring to mind, let alone those that are practiced. 

Furthermore,\marginnote{10.1} when your heart is not overcome and mired in restlessness and remorse … Even hymns that are long-unpracticed spring to mind, let alone those that are practiced. Suppose there was a bowl of water that’s not stirred by the wind, churning, swirling, and rippling. A person with good eyesight checking their own reflection would truly know it and see it. In the same way, when your heart is not overcome and mired in restlessness and remorse … Even hymns that are long-unpracticed spring to mind, let alone those that are practiced. 

Furthermore,\marginnote{11.1} when your heart is not overcome and mired in doubt … Even hymns that are long-unpracticed spring to mind, let alone those that are practiced. Suppose there was a bowl of water that’s transparent, clear, and unclouded, brought into the light. A person with good eyesight checking their own reflection would truly know it and see it. In the same way, there’s a time when your heart is not overcome and mired in doubt and you truly understand the escape from doubt that has arisen. At that time you truly know and see what is good for yourself, good for another, and good for both. Even hymns that are long-unpracticed spring to mind, let alone those that are practiced. 

This\marginnote{12.1} is the cause, brahmin, this is the reason why sometimes even hymns that are long-practiced don’t spring to mind, let alone those that are not practiced. And this is why sometimes even hymns that are long-unpracticed spring to mind, let alone those that are practiced.” 

“Excellent,\marginnote{13.1} Master Gotama! … From this day forth, may Master Gotama remember me as a lay follower who has gone for refuge for life.” 

%
\section*{{\suttatitleacronym AN 5.194}{\suttatitletranslation With Kāraṇapālī }{\suttatitleroot Kāraṇapālīsutta}}
\addcontentsline{toc}{section}{\tocacronym{AN 5.194} \toctranslation{With Kāraṇapālī } \tocroot{Kāraṇapālīsutta}}
\markboth{With Kāraṇapālī }{Kāraṇapālīsutta}
\extramarks{AN 5.194}{AN 5.194}

At\marginnote{1.1} one time the Buddha was staying near \textsanskrit{Vesālī}, at the Great Wood, in the hall with the peaked roof. 

Now\marginnote{1.2} at that time the brahmin \textsanskrit{Kāraṇapālī} was working for the Licchavis. He saw the brahmin \textsanskrit{Piṅgiyānī} coming off in the distance and said to him, “So, \textsanskrit{Piṅgiyānī}, where are you coming from in the middle of the day?” 

“I’m\marginnote{2.2} coming, my good man, from the presence of the ascetic Gotama.” 

“What\marginnote{2.3} do you think of the ascetic Gotama’s lucidity of wisdom? Do you think he’s astute?” 

“My\marginnote{2.4} good man, who am I to judge the ascetic Gotama’s lucidity of wisdom? You’d really have to be on the same level to judge his lucidity of wisdom.” 

“Master\marginnote{2.6} \textsanskrit{Piṅgiyānī} praises the ascetic Gotama with magnificent praise indeed.” 

“Who\marginnote{2.7} am I to praise the ascetic Gotama? He is praised by the praised as the first among gods and humans.” 

“But\marginnote{2.9} for what reason are you so devoted to the ascetic Gotama?” 

“Suppose\marginnote{3.1} a person was completely satisfied by the best tasting food. They wouldn’t be attracted to anything that tasted inferior. In the same way, when you hear the ascetic Gotama’s teaching—whatever it may be, whether statements, songs, discussions, or amazing stories—then you’re not attracted to the doctrines of the various ascetics and brahmins. 

Suppose\marginnote{4.1} a person who was weak with hunger was to obtain a honey-cake. Wherever they taste it, they would enjoy a sweet, delicious flavor. In the same way, when you hear the ascetic Gotama’s teaching—whatever it may be, whether statements, songs, discussions, or amazing stories—then you get a sense of uplift, a confidence of the heart. 

Suppose\marginnote{5.1} a person were to obtain a piece of sandalwood, whether yellow or red. Wherever they smelled it—whether at the root, the middle, or the top—they’d enjoy a delicious fragrance. In the same way, when you hear the ascetic Gotama’s teaching—whatever it may be, whether statements, songs, discussions, or amazing stories—then you become filled with joy and happiness. 

Suppose\marginnote{6.1} there was a person who was sick, suffering, gravely ill. A good doctor would cure them on the spot. In the same way, when you hear the ascetic Gotama’s teaching—whatever it may be, whether statements, songs, discussions, or amazing stories—then you make an end of sorrow, lamentation, pain, sadness, and distress. 

Suppose\marginnote{7.1} there was a lotus pond with clear, sweet, cool water, clean, with smooth banks, delightful. Then along comes a person struggling in the oppressive heat, weary, thirsty, and parched. They’d plunge into the lotus pond to bathe and drink. And all their stress, weariness, and heat exhaustion would die down. In the same way, when you hear the ascetic Gotama’s teaching—whatever it may be, whether statements, songs, discussions, or amazing stories—then all your stress, weariness, and exhaustion die down.” 

When\marginnote{8.1} this was said, the brahmin \textsanskrit{Kāraṇapālī} got up from his seat, arranged his robe over one shoulder, knelt on his right knee, raised his joined palms toward the Buddha, and expressed this heartfelt sentiment three times: 

“Homage\marginnote{9.1} to that Blessed One, the perfected one, the fully awakened Buddha! 

Homage\marginnote{10.1} to that Blessed One, the perfected one, the fully awakened Buddha! 

Homage\marginnote{11.1} to that Blessed One, the perfected one, the fully awakened Buddha! 

Excellent,\marginnote{12.1} Master \textsanskrit{Piṅgiyānī}! Excellent! As if he were righting the overturned, or revealing the hidden, or pointing out the path to the lost, or lighting a lamp in the dark so people with good eyes can see what’s there, Master \textsanskrit{Piṅgiyānī} has made the teaching clear in many ways. I go for refuge to Master Gotama, to the teaching, and to the mendicant \textsanskrit{Saṅgha}. From this day forth, may Master \textsanskrit{Piṅgiyānī} remember me as a lay follower who has gone for refuge for life.” 

%
\section*{{\suttatitleacronym AN 5.195}{\suttatitletranslation Piṅgiyānī }{\suttatitleroot Piṅgiyānīsutta}}
\addcontentsline{toc}{section}{\tocacronym{AN 5.195} \toctranslation{Piṅgiyānī } \tocroot{Piṅgiyānīsutta}}
\markboth{Piṅgiyānī }{Piṅgiyānīsutta}
\extramarks{AN 5.195}{AN 5.195}

At\marginnote{1.1} one time the Buddha was staying near \textsanskrit{Vesālī}, at the Great Wood, in the hall with the peaked roof. 

Now\marginnote{1.2} at that time around five hundred Licchavis were visiting the Buddha. Some of the Licchavis were in blue, of blue color, clad in blue, adorned with blue. And some were similarly colored in yellow, red, or white. But the Buddha outshone them all in beauty and glory. 

Then\marginnote{2.1} the brahmin \textsanskrit{Piṅgīyānī} got up from his seat, arranged his robe over one shoulder, raised his joined palms toward the Buddha, and said, “I feel inspired to speak, Blessed One! I feel inspired to speak, Holy One!” 

“Then\marginnote{2.3} speak as you feel inspired,” said the Buddha. So the brahmin \textsanskrit{Piṅgīyānī} extolled the Buddha in his presence with a fitting verse. 

\begin{verse}%
“Like\marginnote{3.1} a fragrant pink lotus \\
that blooms in the morning, its fragrance unfaded—\\
see \textsanskrit{Aṅgīrasa} shine, \\
bright as the sun in the sky!” 

%
\end{verse}

Then\marginnote{4.1} those Licchavis clothed \textsanskrit{Piṅgiyānī} with five hundred upper robes. And \textsanskrit{Piṅgiyānī} clothed the Buddha with them. 

Then\marginnote{5.1} the Buddha said to the Licchavis: 

“Licchavis,\marginnote{5.2} the appearance of five treasures is rare in the world. What five? A Realized One, a perfected one, a fully awakened Buddha. A person who explains the teaching and training proclaimed by a Realized One. A person who understands the teaching and training proclaimed by a Realized One. A person who practices in line with the teaching. A person who is grateful and thankful. The appearance of these five treasures is rare in the world.” 

%
\section*{{\suttatitleacronym AN 5.196}{\suttatitletranslation The Great Dreams }{\suttatitleroot Mahāsupinasutta}}
\addcontentsline{toc}{section}{\tocacronym{AN 5.196} \toctranslation{The Great Dreams } \tocroot{Mahāsupinasutta}}
\markboth{The Great Dreams }{Mahāsupinasutta}
\extramarks{AN 5.196}{AN 5.196}

“Mendicants,\marginnote{1.1} before his awakening five great dreams appeared to the Realized One, the perfected one, the fully awakened Buddha, when he was still not awake but intent on awakening. What five? 

This\marginnote{1.3} great earth was his bed. Himalaya, king of mountains, was his pillow. His left hand was laid down in the eastern sea. His right hand was laid down in the western sea. And both his feet were laid down in the southern sea. This is the first great dream that appeared to the Realized One before his awakening. 

Next,\marginnote{2.1} a kind of grass called ‘the crosser’ grew up from his navel and stood pressing against the cloudy sky. This is the second great dream that appeared to the Realized One before his awakening. 

Next,\marginnote{3.1} white caterpillars with black heads crawled up from his feet and covered his knees. This is the third great dream that appeared to the Realized One before his awakening. 

Next,\marginnote{4.1} four birds of different colors came from the four quarters. They fell at his feet, turning pure white. This is the fourth great dream that appeared to the Realized One before his awakening. 

Next,\marginnote{5.1} he walked back and forth on top of a huge mountain of filth while remaining unsoiled. This is the fifth great dream that appeared to the Realized One before his awakening. 

Now,\marginnote{6.1} as to when, before his awakening, the Realized One, the perfected one, the fully awakened Buddha was still not awake but intent on awakening. This great earth was his bed. Himalaya, king of mountains, was his pillow. His left hand was laid down in the eastern sea. His right hand was laid down in the western sea. And both his feet were laid down in the southern sea. This was fulfilled when the Buddha awakened to the perfect awakening. This was the first great dream that appeared to him while he was still not awakened. 

As\marginnote{7.1} to when a kind of grass called ‘the crosser’ grew up from his navel and stood pressing against the cloudy sky. This was fulfilled when, after the Buddha had awakened to the noble eightfold path, it was well proclaimed wherever there are gods and humans. This was the second great dream that appeared to him while he was still not awakened. 

As\marginnote{8.1} to when white caterpillars with black heads crawled up from his feet and covered his knees. This was fulfilled when many white-clothed laypeople went for refuge to him for life. This was the third great dream that appeared to him while he was still not awakened. 

As\marginnote{9.1} to when four birds of different colors came from the four quarters. They fell at his feet, turning pure white. This was fulfilled when members of the four castes—aristocrats, brahmins, merchants, and workers—went forth from the lay life to homelessness in the teaching and training proclaimed by the Realized One and realized supreme freedom. This was the fourth great dream that appeared to him while he was still not awakened. 

As\marginnote{10.1} to when he walked back and forth on top of a huge mountain of filth while remaining unsoiled. This was fulfilled when the Realized One received robes, almsfood, lodgings, and medicines and supplies for the sick. And he used them untied, uninfatuated, unattached, seeing the drawbacks, and understanding the escape. This was the fifth great dream that appeared to him while he was still not awakened. 

Before\marginnote{11.1} his awakening these five great dreams appeared to the Realized One, the perfected one, the fully awakened Buddha, when he was still not awake but intent on awakening.” 

%
\section*{{\suttatitleacronym AN 5.197}{\suttatitletranslation Obstacles to Rain }{\suttatitleroot Vassasutta}}
\addcontentsline{toc}{section}{\tocacronym{AN 5.197} \toctranslation{Obstacles to Rain } \tocroot{Vassasutta}}
\markboth{Obstacles to Rain }{Vassasutta}
\extramarks{AN 5.197}{AN 5.197}

“Mendicants,\marginnote{1.1} there are these five obstacles to rain, which the forecasters don’t know, and which their vision does not traverse. What five? 

In\marginnote{1.3} the upper atmosphere the fire element flares up, which disperses the clouds. This is the first obstacle to rain, which the forecasters don’t know, and which their vision does not traverse. 

Furthermore,\marginnote{2.1} in the upper atmosphere the air element flares up, which disperses the clouds. This is the second obstacle to rain … 

Furthermore,\marginnote{3.1} \textsanskrit{Rāhu}, lord of demons, receives water in his hand and tosses it in the ocean. This is the third obstacle to rain … 

Furthermore,\marginnote{4.1} the gods of the rain clouds become negligent. This is the fourth obstacle to rain … 

Furthermore,\marginnote{5.1} humans become unprincipled. This is the fifth obstacle to rain, which the forecasters don’t know, and which their vision does not traverse. 

These\marginnote{5.3} are the five obstacles to rain, which the forecasters don’t know, and which their vision does not traverse.” 

%
\section*{{\suttatitleacronym AN 5.198}{\suttatitletranslation Well-Spoken Words }{\suttatitleroot Vācāsutta}}
\addcontentsline{toc}{section}{\tocacronym{AN 5.198} \toctranslation{Well-Spoken Words } \tocroot{Vācāsutta}}
\markboth{Well-Spoken Words }{Vācāsutta}
\extramarks{AN 5.198}{AN 5.198}

“Mendicants,\marginnote{1.1} speech that has five factors is well spoken, not poorly spoken. It’s blameless and is not criticized by sensible people. What five? It is speech that is timely, true, gentle, beneficial, and loving. Speech with these five factors is well spoken, not poorly spoken. It’s blameless and is not criticized by sensible people.” 

%
\section*{{\suttatitleacronym AN 5.199}{\suttatitletranslation Families }{\suttatitleroot Kulasutta}}
\addcontentsline{toc}{section}{\tocacronym{AN 5.199} \toctranslation{Families } \tocroot{Kulasutta}}
\markboth{Families }{Kulasutta}
\extramarks{AN 5.199}{AN 5.199}

“When\marginnote{1.1} ethical renunciates come to a family, the people make much merit for five reasons. What five? 

When\marginnote{1.3} they see ethical renunciates coming to their family, the people bring up confidence in their hearts. At that time the family is practicing a path leading to heaven. 

When\marginnote{2.1} ethical renunciates come to their family, the people rise from their seats, bow down, and offer them a seat. At that time the family is practicing a path leading to a birth in an eminent family. 

When\marginnote{3.1} ethical renunciates come to their family, the people get rid of the stain of stinginess. At that time the family is practicing a path leading to being illustrious. 

When\marginnote{4.1} ethical renunciates come to their family, the people share what they have as best they can. At that time the family is practicing a path leading to great wealth. 

When\marginnote{5.1} ethical renunciates come to their family, the people ask questions and listen to the teachings. At that time the family is practicing a path leading to great wisdom. 

When\marginnote{5.2} ethical renunciates come to a family, the people make much merit for these five reasons.” 

%
\section*{{\suttatitleacronym AN 5.200}{\suttatitletranslation Elements of Escape }{\suttatitleroot Nissāraṇīyasutta}}
\addcontentsline{toc}{section}{\tocacronym{AN 5.200} \toctranslation{Elements of Escape } \tocroot{Nissāraṇīyasutta}}
\markboth{Elements of Escape }{Nissāraṇīyasutta}
\extramarks{AN 5.200}{AN 5.200}

“Mendicants,\marginnote{1.1} there are these five elements of escape. What five? 

Take\marginnote{1.3} a case where a mendicant focuses on sensual pleasures, but their mind isn’t eager, confident, settled, and decided about them. But when they focus on renunciation, their mind is eager, confident, settled, and decided about it. Their mind is in a good state, well developed, well risen, well freed, and well detached from sensual pleasures. They’re freed from the distressing and feverish defilements that arise because of sensual pleasures, so they don’t experience that kind of feeling. This is how the escape from sensual pleasures is explained. 

Take\marginnote{2.1} another case where a mendicant focuses on ill will, but their mind isn’t eager … But when they focus on good will, their mind is eager … Their mind is in a good state … well detached from ill will. They’re freed from the distressing and feverish defilements that arise because of ill will, so they don’t experience that kind of feeling. This is how the escape from ill will is explained. 

Take\marginnote{3.1} another case where a mendicant focuses on harming, but their mind isn’t eager … But when they focus on compassion, their mind is eager … Their mind is in a good state … well detached from harming. They’re freed from the distressing and feverish defilements that arise because of harming, so they don’t experience that kind of feeling. This is how the escape from harming is explained. 

Take\marginnote{4.1} another case where a mendicant focuses on form, but their mind isn’t eager … But when they focus on the formless, their mind is eager … Their mind is in a good state … well detached from forms. They’re freed from the distressing and feverish defilements that arise because of form, so they don’t experience that kind of feeling. This is how the escape from forms is explained. 

Take\marginnote{5.1} a case where a mendicant focuses on identity, but their mind isn’t eager, confident, settled, and decided about it. But when they focus on the ending of identity, their mind is eager, confident, settled, and decided about it. Their mind is in a good state, well developed, well risen, well freed, and well detached from identity. They’re freed from the distressing and feverish defilements that arise because of identity, so they don’t experience that kind of feeling. This is how the escape from identity is explained. 

Delight\marginnote{6.1} in sensual pleasures, ill will, harming, form, and identity don’t linger within them. That’s why they’re called a mendicant who is without underlying tendencies, who has cut off craving, untied the fetters, and by rightly comprehending conceit has made an end of suffering. These are the five elements of escape.” 

%
\addtocontents{toc}{\let\protect\contentsline\protect\nopagecontentsline}
\pannasa{The Fifth Fifty }
\addcontentsline{toc}{pannasa}{The Fifth Fifty }
\markboth{}{}
\addtocontents{toc}{\let\protect\contentsline\protect\oldcontentsline}

%
\addtocontents{toc}{\let\protect\contentsline\protect\nopagecontentsline}
\chapter*{The Chapter with Kimbila }
\addcontentsline{toc}{chapter}{\tocchapterline{The Chapter with Kimbila }}
\addtocontents{toc}{\let\protect\contentsline\protect\oldcontentsline}

%
\section*{{\suttatitleacronym AN 5.201}{\suttatitletranslation With Kimbila }{\suttatitleroot Kimilasutta}}
\addcontentsline{toc}{section}{\tocacronym{AN 5.201} \toctranslation{With Kimbila } \tocroot{Kimilasutta}}
\markboth{With Kimbila }{Kimilasutta}
\extramarks{AN 5.201}{AN 5.201}

At\marginnote{1.1} one time the Buddha was staying near \textsanskrit{Kimbilā} in the Freshwater Mangrove Wood. Then Venerable Kimbila went up to the Buddha, bowed, sat down to one side, and said to him: 

“What\marginnote{1.3} is the cause, sir, what is the reason why the true teaching does not last long after the final extinguishment of the Realized One?” 

“Kimbila,\marginnote{1.4} it’s when the monks, nuns, laymen, and laywomen lack respect and reverence for the Teacher, the teaching, the \textsanskrit{Saṅgha}, the training, and each other after the final extinguishment of the Realized One. This is the cause, this is the reason why the true teaching does not last long after the final extinguishment of the Realized One.” 

“What\marginnote{2.1} is the cause, sir, what is the reason why the true teaching does last long after the final extinguishment of the Realized One?” 

“Kimbila,\marginnote{2.2} it’s when the monks, nuns, laymen, and laywomen maintain respect and reverence for the Teacher, the teaching, the \textsanskrit{Saṅgha}, the training, and each other after the final extinguishment of the Realized One. This is the cause, this is the reason why the true teaching does last long after the final extinguishment of the Realized One.” 

%
\section*{{\suttatitleacronym AN 5.202}{\suttatitletranslation Listening to the Teaching }{\suttatitleroot Dhammassavanasutta}}
\addcontentsline{toc}{section}{\tocacronym{AN 5.202} \toctranslation{Listening to the Teaching } \tocroot{Dhammassavanasutta}}
\markboth{Listening to the Teaching }{Dhammassavanasutta}
\extramarks{AN 5.202}{AN 5.202}

“Mendicants,\marginnote{1.1} there are these five benefits of listening to the teaching. What five? You learn new things, clarify what you’ve learned, get over uncertainty, correct your views, and inspire confidence in your mind. These are the five benefits of listening to the teaching.” 

%
\section*{{\suttatitleacronym AN 5.203}{\suttatitletranslation A Thoroughbred }{\suttatitleroot Assājānīyasutta}}
\addcontentsline{toc}{section}{\tocacronym{AN 5.203} \toctranslation{A Thoroughbred } \tocroot{Assājānīyasutta}}
\markboth{A Thoroughbred }{Assājānīyasutta}
\extramarks{AN 5.203}{AN 5.203}

“Mendicants,\marginnote{1.1} a fine royal thoroughbred with five factors is worthy of a king, fit to serve a king, and is considered a factor of kingship. 

What\marginnote{2.1} five? Integrity, speed, gentleness, patience, and sweetness. A fine royal thoroughbred with these five factors is worthy of a king. … In the same way, a mendicant with five qualities is worthy of offerings dedicated to the gods, worthy of hospitality, worthy of a religious donation, worthy of veneration with joined palms, and is the supreme field of merit for the world. 

What\marginnote{3.1} five? Integrity, speed, gentleness, patience, and sweetness. A mendicant with these five qualities is worthy of offerings dedicated to the gods, worthy of hospitality, worthy of a religious donation, worthy of veneration with joined palms, and is the supreme field of merit for the world.” 

%
\section*{{\suttatitleacronym AN 5.204}{\suttatitletranslation Powers }{\suttatitleroot Balasutta}}
\addcontentsline{toc}{section}{\tocacronym{AN 5.204} \toctranslation{Powers } \tocroot{Balasutta}}
\markboth{Powers }{Balasutta}
\extramarks{AN 5.204}{AN 5.204}

“Mendicants,\marginnote{1.1} there are these five powers. What five? Faith, conscience, prudence, energy, and wisdom. These are the five powers.” 

%
\section*{{\suttatitleacronym AN 5.205}{\suttatitletranslation Emotional Barrenness }{\suttatitleroot Cetokhilasutta}}
\addcontentsline{toc}{section}{\tocacronym{AN 5.205} \toctranslation{Emotional Barrenness } \tocroot{Cetokhilasutta}}
\markboth{Emotional Barrenness }{Cetokhilasutta}
\extramarks{AN 5.205}{AN 5.205}

“Mendicants,\marginnote{1.1} there are these five kinds of emotional barrenness. What five? Firstly, a mendicant has doubts about the Teacher. They’re uncertain, undecided, and lacking confidence. This being so, their mind doesn’t incline toward keenness, commitment, persistence, and striving. This is the first kind of emotional barrenness. 

Furthermore,\marginnote{2.1} a mendicant has doubts about the teaching … the \textsanskrit{Saṅgha} … the training … A mendicant is angry and upset with their spiritual companions, resentful and closed off. This being so, their mind doesn’t incline toward keenness, commitment, persistence, and striving. This is the fifth kind of emotional barrenness. These are the five kinds of emotional barrenness.” 

%
\section*{{\suttatitleacronym AN 5.206}{\suttatitletranslation Shackles }{\suttatitleroot Vinibandhasutta}}
\addcontentsline{toc}{section}{\tocacronym{AN 5.206} \toctranslation{Shackles } \tocroot{Vinibandhasutta}}
\markboth{Shackles }{Vinibandhasutta}
\extramarks{AN 5.206}{AN 5.206}

“Mendicants,\marginnote{1.1} there are these five emotional shackles. What five? Firstly, a mendicant isn’t free of greed, desire, fondness, thirst, passion, and craving for sensual pleasures. This being so, their mind doesn’t incline toward keenness, commitment, persistence, and striving. This is the first emotional shackle. 

Furthermore,\marginnote{2.1} a mendicant isn’t free of greed for the body … They’re not free of greed for form … They eat as much as they like until their belly is full, then indulge in the pleasures of sleeping, lying down, and drowsing … They lead the spiritual life hoping to be reborn in one of the orders of gods, thinking: ‘By this precept or observance or mortification or spiritual life, may I become one of the gods!’ This being so, their mind doesn’t incline toward keenness, commitment, persistence, and striving. This is the fifth emotional shackle. These are the five emotional shackles.” 

%
\section*{{\suttatitleacronym AN 5.207}{\suttatitletranslation Porridge }{\suttatitleroot Yāgusutta}}
\addcontentsline{toc}{section}{\tocacronym{AN 5.207} \toctranslation{Porridge } \tocroot{Yāgusutta}}
\markboth{Porridge }{Yāgusutta}
\extramarks{AN 5.207}{AN 5.207}

“Mendicants,\marginnote{1.1} there are these five benefits of porridge. What five? It wards off hunger, quenches thirst, settles the wind, cleans the bladder, and helps digestion. These are the five benefits of porridge.” 

%
\section*{{\suttatitleacronym AN 5.208}{\suttatitletranslation Chew Sticks }{\suttatitleroot Dantakaṭṭhasutta}}
\addcontentsline{toc}{section}{\tocacronym{AN 5.208} \toctranslation{Chew Sticks } \tocroot{Dantakaṭṭhasutta}}
\markboth{Chew Sticks }{Dantakaṭṭhasutta}
\extramarks{AN 5.208}{AN 5.208}

“Mendicants,\marginnote{1.1} there are these five drawbacks of not using chew sticks. What five? It’s not good for your eyes, you get bad breath, your taste-buds aren’t cleaned, bile and phlegm cover your food, and you lose your appetite. These are the five drawbacks of not using chew sticks. 

There\marginnote{2.1} are these five benefits of using chew sticks. What five? It’s good for your eyes, you don’t get bad breath, your taste-buds are cleaned, bile and phlegm don’t cover your food, and food agrees with you. These are the five benefits of using chew sticks.” 

%
\section*{{\suttatitleacronym AN 5.209}{\suttatitletranslation The Sound of Singing }{\suttatitleroot Gītassarasutta}}
\addcontentsline{toc}{section}{\tocacronym{AN 5.209} \toctranslation{The Sound of Singing } \tocroot{Gītassarasutta}}
\markboth{The Sound of Singing }{Gītassarasutta}
\extramarks{AN 5.209}{AN 5.209}

“Mendicants,\marginnote{1.1} there are these five drawbacks in reciting with a drawn-out singing sound. What five? You relish the sound of your own voice. Others relish the sound of your voice. Householders complain: ‘These ascetics, followers of the Sakyan, sing just like us!’ When you’re enjoying the melody, your immersion breaks up. Those who come after follow your example. These are the five drawbacks in reciting with a drawn-out singing sound.” 

%
\section*{{\suttatitleacronym AN 5.210}{\suttatitletranslation Unmindful }{\suttatitleroot Muṭṭhassatisutta}}
\addcontentsline{toc}{section}{\tocacronym{AN 5.210} \toctranslation{Unmindful } \tocroot{Muṭṭhassatisutta}}
\markboth{Unmindful }{Muṭṭhassatisutta}
\extramarks{AN 5.210}{AN 5.210}

“Mendicants,\marginnote{1.1} there are these five drawbacks of falling asleep unmindful and unaware. What five? You sleep badly and wake miserably. You have bad dreams. The deities don’t protect you. And you emit semen. These are the five drawbacks of falling asleep unmindful and unaware. 

There\marginnote{2.1} are these five benefits of falling asleep mindful and aware. What five? You sleep at ease and wake happily. You don’t have bad dreams. The deities protect you. And you don’t emit semen. These are the five benefits of falling asleep mindful and aware.” 

%
\addtocontents{toc}{\let\protect\contentsline\protect\nopagecontentsline}
\chapter*{The Chapter on Abuse }
\addcontentsline{toc}{chapter}{\tocchapterline{The Chapter on Abuse }}
\addtocontents{toc}{\let\protect\contentsline\protect\oldcontentsline}

%
\section*{{\suttatitleacronym AN 5.211}{\suttatitletranslation An Abuser }{\suttatitleroot Akkosakasutta}}
\addcontentsline{toc}{section}{\tocacronym{AN 5.211} \toctranslation{An Abuser } \tocroot{Akkosakasutta}}
\markboth{An Abuser }{Akkosakasutta}
\extramarks{AN 5.211}{AN 5.211}

“Mendicants,\marginnote{1.1} a mendicant who abuses and insults their spiritual companions, speaking ill of the noble ones, can expect these five drawbacks. What five? They’re expelled, cut off, shut out; or they commit a corrupt offense; or they contract a severe illness. They feel lost when they die. And when their body breaks up, after death, they’re reborn in a place of loss, a bad place, the underworld, hell. A mendicant who abuses and insults their spiritual companions, speaking ill of the noble ones, can expect these five drawbacks.” 

%
\section*{{\suttatitleacronym AN 5.212}{\suttatitletranslation Starting Arguments }{\suttatitleroot Bhaṇḍanakārakasutta}}
\addcontentsline{toc}{section}{\tocacronym{AN 5.212} \toctranslation{Starting Arguments } \tocroot{Bhaṇḍanakārakasutta}}
\markboth{Starting Arguments }{Bhaṇḍanakārakasutta}
\extramarks{AN 5.212}{AN 5.212}

“Mendicants,\marginnote{1.1} a mendicant who starts arguments, quarrels, disputes, debates, and disciplinary issues in the \textsanskrit{Saṅgha} can expect five drawbacks. What five? They don’t achieve the unachieved. What they have achieved falls away. They get a bad reputation. They feel lost when they die. And when their body breaks up, after death, they are reborn in a place of loss, a bad place, the underworld, hell. A mendicant who starts arguments, quarrels, disputes, debates, and disciplinary issues in the \textsanskrit{Saṅgha} can expect these five drawbacks.” 

%
\section*{{\suttatitleacronym AN 5.213}{\suttatitletranslation Ethics }{\suttatitleroot Sīlasutta}}
\addcontentsline{toc}{section}{\tocacronym{AN 5.213} \toctranslation{Ethics } \tocroot{Sīlasutta}}
\markboth{Ethics }{Sīlasutta}
\extramarks{AN 5.213}{AN 5.213}

“Mendicants,\marginnote{1.1} there are these five drawbacks for an unethical person because of their failure in ethics. What five? 

Firstly,\marginnote{1.3} an unethical person loses substantial wealth on account of negligence. This is the first drawback. 

Furthermore,\marginnote{2.1} an unethical person gets a bad reputation. This is the second drawback. 

Furthermore,\marginnote{3.1} an unethical person enters any kind of assembly timid and embarrassed, whether it’s an assembly of aristocrats, brahmins, householders, or ascetics. This is the third drawback. 

Furthermore,\marginnote{4.1} an unethical person feels lost when they die. This is the fourth drawback. 

Furthermore,\marginnote{5.1} an unethical person, when their body breaks up, after death, is reborn in a place of loss, a bad place, the underworld, hell. This is the fifth drawback. These are the five drawbacks for an unethical person because of their failure in ethics. 

There\marginnote{6.1} are these five benefits for an ethical person because of their accomplishment in ethics. What five? Firstly, an ethical person gains substantial wealth on account of diligence. This is the first benefit. 

Furthermore,\marginnote{7.1} an ethical person gets a good reputation. This is the second benefit. 

Furthermore,\marginnote{8.1} an ethical person enters any kind of assembly bold and self-assured, whether it’s an assembly of aristocrats, brahmins, householders, or ascetics. This is the third benefit. 

Furthermore,\marginnote{9.1} an ethical person dies not feeling lost. This is the fourth benefit. 

Furthermore,\marginnote{10.1} when an ethical person’s body breaks up, after death, they’re reborn in a good place, a heavenly realm. This is the fifth benefit. 

These\marginnote{10.3} are the five benefits for an ethical person because of their accomplishment in ethics.” 

%
\section*{{\suttatitleacronym AN 5.214}{\suttatitletranslation Someone Who Talks a Lot }{\suttatitleroot Bahubhāṇisutta}}
\addcontentsline{toc}{section}{\tocacronym{AN 5.214} \toctranslation{Someone Who Talks a Lot } \tocroot{Bahubhāṇisutta}}
\markboth{Someone Who Talks a Lot }{Bahubhāṇisutta}
\extramarks{AN 5.214}{AN 5.214}

“Mendicants,\marginnote{1.1} there are these five drawbacks for a person who talks a lot. What five? They use speech that’s false, divisive, harsh, and nonsensical. When their body breaks up, after death, they’re reborn in a place of loss, a bad place, the underworld, hell. These are the five drawbacks for a person who talks a lot. 

There\marginnote{2.1} are these five benefits for a person who talks thoughtfully. What five? They don’t use speech that’s false, divisive, harsh, and nonsensical. When their body breaks up, after death, they’re reborn in a good place, a heavenly realm. These are the five benefits for a person who talks thoughtfully.” 

%
\section*{{\suttatitleacronym AN 5.215}{\suttatitletranslation Intolerance (1st) }{\suttatitleroot Paṭhamaakkhantisutta}}
\addcontentsline{toc}{section}{\tocacronym{AN 5.215} \toctranslation{Intolerance (1st) } \tocroot{Paṭhamaakkhantisutta}}
\markboth{Intolerance (1st) }{Paṭhamaakkhantisutta}
\extramarks{AN 5.215}{AN 5.215}

“Mendicants,\marginnote{1.1} there are these five drawbacks of intolerance. What five? Most people find you unlikable and unloveable. You have lots of enmity and many faults. You feel lost when you die. And when your body breaks up, after death, you’re reborn in a place of loss, a bad place, the underworld, hell. These are the five drawbacks to intolerance. 

There\marginnote{2.1} are these five benefits of tolerance. What five? Most people find you dear and lovable. You have little enmity and few faults. You don’t feel lost when you die. And when your body breaks up, after death, you’re reborn in a good place, a heavenly realm. These are the five benefits of tolerance.” 

%
\section*{{\suttatitleacronym AN 5.216}{\suttatitletranslation Intolerance (2nd) }{\suttatitleroot Dutiyaakkhantisutta}}
\addcontentsline{toc}{section}{\tocacronym{AN 5.216} \toctranslation{Intolerance (2nd) } \tocroot{Dutiyaakkhantisutta}}
\markboth{Intolerance (2nd) }{Dutiyaakkhantisutta}
\extramarks{AN 5.216}{AN 5.216}

“Mendicants,\marginnote{1.1} there are these five drawbacks of intolerance. What five? Most people find you unlikable and unlovable. You’re cruel and remorseful. You feel lost when you die. And when your body breaks up, after death, you’re reborn in a place of loss, a bad place, the underworld, hell. These are the five drawbacks to intolerance. 

There\marginnote{2.1} are these five benefits of tolerance. What five? Most people find you likable and lovable. You’re neither cruel nor remorseful. You don’t feel lost when you die. And when your body breaks up, after death, you’re reborn in a good place, a heavenly realm. These are the five benefits of tolerance.” 

%
\section*{{\suttatitleacronym AN 5.217}{\suttatitletranslation Uninspiring Conduct (1st) }{\suttatitleroot Paṭhamaapāsādikasutta}}
\addcontentsline{toc}{section}{\tocacronym{AN 5.217} \toctranslation{Uninspiring Conduct (1st) } \tocroot{Paṭhamaapāsādikasutta}}
\markboth{Uninspiring Conduct (1st) }{Paṭhamaapāsādikasutta}
\extramarks{AN 5.217}{AN 5.217}

“Mendicants,\marginnote{1.1} there are these five drawbacks of uninspiring conduct. What five? You blame yourself. After examination, sensible people criticize you. You get a bad reputation. You feel lost when you die. And when your body breaks up, after death, you’re reborn in a place of loss, a bad place, the underworld, hell. These are the five drawbacks of uninspiring conduct. 

There\marginnote{2.1} are these five benefits of inspiring conduct. What five? You don’t blame yourself. After examination, sensible people praise you. You get a good reputation. You don’t feel lost when you die. And when the body breaks up, after death, you’re reborn in a good place, a heavenly realm. These are the five benefits of inspiring conduct.” 

%
\section*{{\suttatitleacronym AN 5.218}{\suttatitletranslation Uninspiring Conduct (2nd) }{\suttatitleroot Dutiyaapāsādikasutta}}
\addcontentsline{toc}{section}{\tocacronym{AN 5.218} \toctranslation{Uninspiring Conduct (2nd) } \tocroot{Dutiyaapāsādikasutta}}
\markboth{Uninspiring Conduct (2nd) }{Dutiyaapāsādikasutta}
\extramarks{AN 5.218}{AN 5.218}

“Mendicants,\marginnote{1.1} there are these five drawbacks of uninspiring conduct. What five? You don’t inspire confidence in those without it. You cause some with confidence to change their minds. You don’t follow the Teacher’s instructions. Those who come after you follow your example. And your mind doesn’t become clear. These are the five drawbacks of uninspiring conduct. 

There\marginnote{2.1} are these five benefits of inspiring conduct. What five? You inspire confidence in those without it. You increase confidence in those who have it. You follow the Teacher’s instructions. Those who come after you follow your example. And your mind becomes clear. These are the five benefits of inspiring conduct.” 

%
\section*{{\suttatitleacronym AN 5.219}{\suttatitletranslation Fire }{\suttatitleroot Aggisutta}}
\addcontentsline{toc}{section}{\tocacronym{AN 5.219} \toctranslation{Fire } \tocroot{Aggisutta}}
\markboth{Fire }{Aggisutta}
\extramarks{AN 5.219}{AN 5.219}

“Mendicants,\marginnote{1.1} there are these five drawbacks of a fire. What five? It’s bad for your eyes. It’s bad for your complexion. It makes you weak. It draws in groups. And it encourages unworthy talk. These are the five drawbacks of a fire.” 

%
\section*{{\suttatitleacronym AN 5.220}{\suttatitletranslation About Madhurā }{\suttatitleroot Madhurāsutta}}
\addcontentsline{toc}{section}{\tocacronym{AN 5.220} \toctranslation{About Madhurā } \tocroot{Madhurāsutta}}
\markboth{About Madhurā }{Madhurāsutta}
\extramarks{AN 5.220}{AN 5.220}

“Mendicants,\marginnote{1.1} there are these five drawbacks of \textsanskrit{Madhurā}. What five? The ground is uneven and dusty, the dogs are fierce, the native spirits are vicious, and it’s hard to get almsfood. These are the five drawbacks of \textsanskrit{Madhurā}.” 

%
\addtocontents{toc}{\let\protect\contentsline\protect\nopagecontentsline}
\chapter*{The Chapter on Long Wandering }
\addcontentsline{toc}{chapter}{\tocchapterline{The Chapter on Long Wandering }}
\addtocontents{toc}{\let\protect\contentsline\protect\oldcontentsline}

%
\section*{{\suttatitleacronym AN 5.221}{\suttatitletranslation Long Wandering (1st) }{\suttatitleroot Paṭhamadīghacārikasutta}}
\addcontentsline{toc}{section}{\tocacronym{AN 5.221} \toctranslation{Long Wandering (1st) } \tocroot{Paṭhamadīghacārikasutta}}
\markboth{Long Wandering (1st) }{Paṭhamadīghacārikasutta}
\extramarks{AN 5.221}{AN 5.221}

“Mendicants,\marginnote{1.1} there are these five drawbacks for someone who likes long and aimless wandering. What five? You don’t learn new things. You don’t clarify what you’ve learned. You lack confidence in some things you have learned. You contract a severe illness. You don’t have any friends. These are the five drawbacks for someone who likes long and aimless wandering. 

There\marginnote{2.1} are these five benefits of a reasonable amount of wandering. What five? You learn new things. You clarify what you’ve learned. You have confidence in some things you have learned. You don’t contract severe illness. You have friends. These are the five benefits of a reasonable amount of wandering.” 

%
\section*{{\suttatitleacronym AN 5.222}{\suttatitletranslation Long Wandering (2nd) }{\suttatitleroot Dutiyadīghacārikasutta}}
\addcontentsline{toc}{section}{\tocacronym{AN 5.222} \toctranslation{Long Wandering (2nd) } \tocroot{Dutiyadīghacārikasutta}}
\markboth{Long Wandering (2nd) }{Dutiyadīghacārikasutta}
\extramarks{AN 5.222}{AN 5.222}

“Mendicants,\marginnote{1.1} there are these five drawbacks for someone who likes long and aimless wandering. What five? You don’t achieve the unachieved. What you have achieved falls away. You lose confidence in some things you’ve achieved. You contract a severe illness. You don’t have any friends. These are the five drawbacks for someone who likes long and aimless wandering. 

There\marginnote{2.1} are these five benefits of a reasonable amount of wandering. What five? You achieve the unachieved. What you have achieved doesn’t fall away. You’re confident in some things you’ve achieved. You don’t contract severe illness. You have friends. These are the five benefits of a reasonable amount of wandering.” 

%
\section*{{\suttatitleacronym AN 5.223}{\suttatitletranslation Overstaying }{\suttatitleroot Atinivāsasutta}}
\addcontentsline{toc}{section}{\tocacronym{AN 5.223} \toctranslation{Overstaying } \tocroot{Atinivāsasutta}}
\markboth{Overstaying }{Atinivāsasutta}
\extramarks{AN 5.223}{AN 5.223}

“Mendicants,\marginnote{1.1} there are these five drawbacks of overstaying. What five? You have a lot of stuff and store it up. You have a lot of medicine and store it up. You have a lot of duties and responsibilities, and become capable in whatever needs to be done. You mix closely with laypeople and renunciates, socializing inappropriately like a layperson. And when you leave that monastery, you miss it. These are the five drawbacks of overstaying. 

There\marginnote{2.1} are these five benefits of staying for a reasonable length of time. What five? You don’t have a lot of stuff and store it up. You don’t have a lot of medicine and store it up. You don’t have a lot of duties and responsibilities, or become capable in whatever needs to be done. You don’t mix closely with laypeople and renunciates, socializing inappropriately like a layperson. And when you leave that monastery, you don’t miss it. These are the five benefits of staying for a reasonable length of time.” 

%
\section*{{\suttatitleacronym AN 5.224}{\suttatitletranslation Stingy }{\suttatitleroot Maccharīsutta}}
\addcontentsline{toc}{section}{\tocacronym{AN 5.224} \toctranslation{Stingy } \tocroot{Maccharīsutta}}
\markboth{Stingy }{Maccharīsutta}
\extramarks{AN 5.224}{AN 5.224}

“Mendicants,\marginnote{1.1} there are these five drawbacks of overstaying. What five? You become stingy with dwellings, families, material possessions, praise, and the teaching. These are the five drawbacks of overstaying. 

There\marginnote{2.1} are these five benefits of staying for a reasonable length of time. What five? You’re not stingy with dwellings, families, material possessions, praise, and the teaching. These are the five benefits of staying for a reasonable length of time.” 

%
\section*{{\suttatitleacronym AN 5.225}{\suttatitletranslation Visiting Families (1st) }{\suttatitleroot Paṭhamakulūpakasutta}}
\addcontentsline{toc}{section}{\tocacronym{AN 5.225} \toctranslation{Visiting Families (1st) } \tocroot{Paṭhamakulūpakasutta}}
\markboth{Visiting Families (1st) }{Paṭhamakulūpakasutta}
\extramarks{AN 5.225}{AN 5.225}

“Mendicants,\marginnote{1.1} there are these five drawbacks of visiting families. What five? You fall into an offense for wandering without leave. You fall into an offense for sitting in a private place with someone of the opposite sex. You fall into an offense for sitting in a hidden place with someone of the opposite sex. You fall into an offense for teaching more than five or six sentences to someone of the opposite sex. You have a lot of sensual thoughts. These are the five drawbacks of visiting families.” 

%
\section*{{\suttatitleacronym AN 5.226}{\suttatitletranslation Visiting Families (2nd) }{\suttatitleroot Dutiyakulūpakasutta}}
\addcontentsline{toc}{section}{\tocacronym{AN 5.226} \toctranslation{Visiting Families (2nd) } \tocroot{Dutiyakulūpakasutta}}
\markboth{Visiting Families (2nd) }{Dutiyakulūpakasutta}
\extramarks{AN 5.226}{AN 5.226}

“Mendicants,\marginnote{1.1} there are these five drawbacks for a mendicant who visits families for too long, mixing closely with them. What five? You often see members of the opposite sex. Seeing them, you become close. Being so close, you become intimate. Being intimate, lust overcomes you. When your mind is swamped by lust, you can expect that you will lead the spiritual life dissatisfied, or commit one of the corrupt offenses, or resign the training and return to a lesser life. These are the five drawbacks for a mendicant who visits families for too long, mixing closely with them.” 

%
\section*{{\suttatitleacronym AN 5.227}{\suttatitletranslation Riches }{\suttatitleroot Bhogasutta}}
\addcontentsline{toc}{section}{\tocacronym{AN 5.227} \toctranslation{Riches } \tocroot{Bhogasutta}}
\markboth{Riches }{Bhogasutta}
\extramarks{AN 5.227}{AN 5.227}

“Mendicants,\marginnote{1.1} there are these five drawbacks of riches. What five? Fire, water, kings, thieves, and unloved heirs all take a share. These are the five drawbacks of riches. 

There\marginnote{2.1} are these five benefits of riches. What five? Riches enable you to bring pleasure and joy to yourself; your mother and father; your children, partners, bondservants, workers, and staff; and your friends and colleagues; and to keep them all happy. And they enable you to establish an uplifting religious donation for ascetics and brahmins that’s conducive to heaven, ripens in happiness, and leads to heaven. These are the five benefits of riches.” 

%
\section*{{\suttatitleacronym AN 5.228}{\suttatitletranslation Eating Late }{\suttatitleroot Ussūrabhattasutta}}
\addcontentsline{toc}{section}{\tocacronym{AN 5.228} \toctranslation{Eating Late } \tocroot{Ussūrabhattasutta}}
\markboth{Eating Late }{Ussūrabhattasutta}
\extramarks{AN 5.228}{AN 5.228}

“Mendicants,\marginnote{1.1} there are these five drawbacks for a family who takes their meals late in the day. What five? When guests visit, they are not served on time. The deities who accept spirit-offerings are not served on time. Ascetics and brahmins who eat in one part of the day, abstaining from eating at night, and from food at the wrong time are not served on time. Bondservants, workers, and staff do their duties neglectfully. A meal eaten during the wrong period is not nutritious. These are the five drawbacks for a family who takes their meals late in the day. 

There\marginnote{2.1} are these five benefits for a family who takes their meals at a proper time. What five? When guests visit, they are served on time. The deities who accept spirit-offerings are served on time. Ascetics and brahmins who eat in one part of the day, abstaining from eating at night, and from food at the wrong time are served on time. Bondservants, workers, and staff do their duties attentively. A meal eaten during the proper period is nutritious. These are the five benefits for a family who takes their meals at a proper time.” 

%
\section*{{\suttatitleacronym AN 5.229}{\suttatitletranslation Black Snakes (1st) }{\suttatitleroot Paṭhamakaṇhasappasutta}}
\addcontentsline{toc}{section}{\tocacronym{AN 5.229} \toctranslation{Black Snakes (1st) } \tocroot{Paṭhamakaṇhasappasutta}}
\markboth{Black Snakes (1st) }{Paṭhamakaṇhasappasutta}
\extramarks{AN 5.229}{AN 5.229}

“Mendicants,\marginnote{1.1} there are these five drawbacks of a black snake. What five? It’s filthy, stinking, cowardly, frightening, and treacherous. These are the five dangers of a black snake. 

In\marginnote{2.1} the same way there are five drawbacks of a female. What five? She’s filthy, stinking, cowardly, frightening, and treacherous. These are the five drawbacks of a female.” 

%
\section*{{\suttatitleacronym AN 5.230}{\suttatitletranslation Black Snakes (2nd) }{\suttatitleroot Dutiyakaṇhasappasutta}}
\addcontentsline{toc}{section}{\tocacronym{AN 5.230} \toctranslation{Black Snakes (2nd) } \tocroot{Dutiyakaṇhasappasutta}}
\markboth{Black Snakes (2nd) }{Dutiyakaṇhasappasutta}
\extramarks{AN 5.230}{AN 5.230}

“Mendicants,\marginnote{1.1} there are these five drawbacks of a black snake. What five? It’s irritable, hostile, venomous, fork-tongued, and treacherous. These are the five dangers of a black snake. 

In\marginnote{2.1} the same way there are five drawbacks of a female. What five? She’s irritable, hostile, venomous, fork-tongued, and treacherous. This is a female’s venom: usually she’s very lustful. This is a female’s forked tongue: usually she speaks divisively. This is a female’s treachery: usually she’s an adulteress. These are the five drawbacks of a female.” 

%
\addtocontents{toc}{\let\protect\contentsline\protect\nopagecontentsline}
\chapter*{The Chapter on a Resident Mendicant }
\addcontentsline{toc}{chapter}{\tocchapterline{The Chapter on a Resident Mendicant }}
\addtocontents{toc}{\let\protect\contentsline\protect\oldcontentsline}

%
\section*{{\suttatitleacronym AN 5.231}{\suttatitletranslation A Resident Mendicant }{\suttatitleroot Āvāsikasutta}}
\addcontentsline{toc}{section}{\tocacronym{AN 5.231} \toctranslation{A Resident Mendicant } \tocroot{Āvāsikasutta}}
\markboth{A Resident Mendicant }{Āvāsikasutta}
\extramarks{AN 5.231}{AN 5.231}

“Mendicants,\marginnote{1.1} a resident mendicant with five qualities is not admirable. What five? They’re not accomplished in being well-presented and doing their duties. They’re not very learned and don’t remember what they’ve learned. They’re not self-effacing and don’t enjoy self-effacement. They’re not a good speaker and don’t speak well. They’re witless, dull, and stupid. A resident mendicant with these five qualities is not admirable. 

A\marginnote{2.1} resident mendicant with these five qualities is admirable. What five? They’re accomplished in being well-presented and doing their duties. They’re very learned and remember what they’ve learned. They’re self-effacing and enjoy self-effacement. They’re a good speaker and speak well. They’re wise, bright, and clever. A resident mendicant with these five qualities is admirable.” 

%
\section*{{\suttatitleacronym AN 5.232}{\suttatitletranslation Liked }{\suttatitleroot Piyasutta}}
\addcontentsline{toc}{section}{\tocacronym{AN 5.232} \toctranslation{Liked } \tocroot{Piyasutta}}
\markboth{Liked }{Piyasutta}
\extramarks{AN 5.232}{AN 5.232}

“Mendicants,\marginnote{1.1} a resident mendicant with five qualities is dear and beloved to their spiritual companions, respected and admired. What five? 

They’re\marginnote{1.3} ethical, restrained in the monastic code, conducting themselves well and seeking alms in suitable places. Seeing danger in the slightest fault, they keep the rules they’ve undertaken. 

They’re\marginnote{1.4} very learned, remembering and keeping what they’ve learned. These teachings are good in the beginning, good in the middle, and good in the end, meaningful and well-phrased, describing a spiritual practice that’s entirely full and pure. They are very learned in such teachings, remembering them, reinforcing them by recitation, mentally scrutinizing them, and comprehending them theoretically. 

They’re\marginnote{1.5} a good speaker, with a polished, clear, and articulate voice that expresses the meaning. 

They\marginnote{1.6} get the four absorptions—blissful meditations in the present life that belong to the higher mind—when they want, without trouble or difficulty. 

They\marginnote{1.7} realize the undefiled freedom of heart and freedom by wisdom in this very life. And they live having realized it with their own insight due to the ending of defilements. 

A\marginnote{1.8} resident mendicant with these five qualities is dear and beloved to their spiritual companions, respected and admired.” 

%
\section*{{\suttatitleacronym AN 5.233}{\suttatitletranslation Beautification }{\suttatitleroot Sobhanasutta}}
\addcontentsline{toc}{section}{\tocacronym{AN 5.233} \toctranslation{Beautification } \tocroot{Sobhanasutta}}
\markboth{Beautification }{Sobhanasutta}
\extramarks{AN 5.233}{AN 5.233}

“Mendicants,\marginnote{1.1} a resident mendicant with five qualities beautifies the monastery. What five? 

They’re\marginnote{1.3} ethical, restrained in the code of conduct, conducting themselves well and seeking alms in suitable places. Seeing danger in the slightest fault, they keep the rules they’ve undertaken. 

They’re\marginnote{1.4} very learned, remembering and keeping what they’ve learned. These teachings are good in the beginning, good in the middle, and good in the end, meaningful and well-phrased, describing a spiritual practice that’s totally full and pure. They are very learned in such teachings, remembering them, reciting them, mentally scrutinizing them, and comprehending them theoretically. 

They’re\marginnote{1.5} a good speaker, with a polished, clear, and articulate voice that expresses the meaning. 

They’re\marginnote{1.6} able to educate, encourage, fire up, and inspire those who approach them with a Dhamma talk. 

They\marginnote{1.7} get the four absorptions—blissful meditations in the present life that belong to the higher mind—when they want, without trouble or difficulty. 

A\marginnote{1.8} resident mendicant with these five qualities beautifies the monastery.” 

%
\section*{{\suttatitleacronym AN 5.234}{\suttatitletranslation Very Helpful }{\suttatitleroot Bahūpakārasutta}}
\addcontentsline{toc}{section}{\tocacronym{AN 5.234} \toctranslation{Very Helpful } \tocroot{Bahūpakārasutta}}
\markboth{Very Helpful }{Bahūpakārasutta}
\extramarks{AN 5.234}{AN 5.234}

“Mendicants,\marginnote{1.1} a resident mendicant with five qualities is very helpful to the monastery. What five? 

They’re\marginnote{1.3} ethical, restrained in the code of conduct, conducting themselves well and seeking alms in suitable places. Seeing danger in the slightest fault, they keep the rules they’ve undertaken. 

They’re\marginnote{1.4} very learned, remembering and keeping what they’ve learned. These teachings are good in the beginning, good in the middle, and good in the end, meaningful and well-phrased, describing a spiritual practice that’s totally full and pure. They are very learned in such teachings, remembering them, reciting them, mentally scrutinizing them, and comprehending them theoretically. 

They\marginnote{1.5} repair what is decayed and damaged. 

When\marginnote{1.6} a large mendicant \textsanskrit{Saṅgha} is arriving with mendicants from abroad, they go to the lay people and announce: 

‘A\marginnote{1.7} large mendicant \textsanskrit{Saṅgha} is arriving with mendicants from abroad. Make merit! Now is the time to make merit!’ 

They\marginnote{1.8} get the four absorptions—blissful meditations in the present life that belong to the higher mind—when they want, without trouble or difficulty. 

A\marginnote{1.9} resident mendicant with these five qualities is very helpful to the monastery.” 

%
\section*{{\suttatitleacronym AN 5.235}{\suttatitletranslation A Compassionate Mendicant }{\suttatitleroot Anukampasutta}}
\addcontentsline{toc}{section}{\tocacronym{AN 5.235} \toctranslation{A Compassionate Mendicant } \tocroot{Anukampasutta}}
\markboth{A Compassionate Mendicant }{Anukampasutta}
\extramarks{AN 5.235}{AN 5.235}

“Mendicants,\marginnote{1.1} a resident mendicant with five qualities shows compassion to the lay people. What five? They encourage them in higher ethics. They equip them to see the truth of the teachings. When they are sick, they go to them and prompt their mindfulness, saying: ‘Establish your mindfulness, good sirs, in what is worthy.’ When a large mendicant \textsanskrit{Saṅgha} is arriving with mendicants from abroad, they go to the lay people and announce: ‘A large mendicant \textsanskrit{Saṅgha} is arriving with mendicants from abroad. Make merit! Now is the time to make merit!’ And they eat whatever food they give them, coarse or fine, not wasting a gift given in faith. A resident mendicant with these five qualities shows compassion to the lay people.” 

%
\section*{{\suttatitleacronym AN 5.236}{\suttatitletranslation Deserving Criticism (1st) }{\suttatitleroot Paṭhamaavaṇṇārahasutta}}
\addcontentsline{toc}{section}{\tocacronym{AN 5.236} \toctranslation{Deserving Criticism (1st) } \tocroot{Paṭhamaavaṇṇārahasutta}}
\markboth{Deserving Criticism (1st) }{Paṭhamaavaṇṇārahasutta}
\extramarks{AN 5.236}{AN 5.236}

“Mendicants,\marginnote{1.1} a resident mendicant with five qualities is cast down to hell. What five? Without examining or scrutinizing, they praise those deserving of criticism, and they criticize those deserving of praise. Without examining or scrutinizing, they arouse faith in things that are dubious, and they don’t arouse faith in things that are inspiring. And they waste a gift given in faith. A resident mendicant with these five qualities is cast down to hell. 

A\marginnote{2.1} resident mendicant with five qualities is raised up to heaven. What five? After examining and scrutinizing, they criticize those deserving of criticism, and they praise those deserving of praise. They don’t arouse faith in things that are dubious, and they do arouse faith in things that are inspiring. And they don’t waste a gift given in faith. A resident mendicant with these five qualities is raised up to heaven.” 

%
\section*{{\suttatitleacronym AN 5.237}{\suttatitletranslation Deserving Criticism (2nd) }{\suttatitleroot Dutiyaavaṇṇārahasutta}}
\addcontentsline{toc}{section}{\tocacronym{AN 5.237} \toctranslation{Deserving Criticism (2nd) } \tocroot{Dutiyaavaṇṇārahasutta}}
\markboth{Deserving Criticism (2nd) }{Dutiyaavaṇṇārahasutta}
\extramarks{AN 5.237}{AN 5.237}

“Mendicants,\marginnote{1.1} a resident mendicant with five qualities is cast down to hell. What five? Without examining or scrutinizing, they praise those deserving of criticism, and they criticize those deserving of praise. They’re stingy and avaricious regarding monasteries. They’re stingy and avaricious regarding families. And they waste a gift given in faith. A resident mendicant with these five qualities is cast down to hell. 

A\marginnote{2.1} resident mendicant with five qualities is raised up to heaven. What five? After examining and scrutinizing, they criticize those deserving of criticism, and they praise those deserving of praise. They’re not stingy and avaricious regarding monasteries. They’re not stingy and avaricious regarding families. And they don’t waste a gift given in faith. A resident mendicant with these five qualities is raised up to heaven.” 

%
\section*{{\suttatitleacronym AN 5.238}{\suttatitletranslation Deserving Criticism (3rd) }{\suttatitleroot Tatiyaavaṇṇārahasutta}}
\addcontentsline{toc}{section}{\tocacronym{AN 5.238} \toctranslation{Deserving Criticism (3rd) } \tocroot{Tatiyaavaṇṇārahasutta}}
\markboth{Deserving Criticism (3rd) }{Tatiyaavaṇṇārahasutta}
\extramarks{AN 5.238}{AN 5.238}

“Mendicants,\marginnote{1.1} a resident mendicant with five qualities is cast down to hell. What five? Without examining or scrutinizing, they praise those deserving of criticism, and they criticize those deserving of praise. They’re stingy regarding monasteries, families, and material possessions. A resident mendicant with these five qualities is cast down to hell. 

A\marginnote{2.1} resident mendicant with five qualities is raised up to heaven. What five? After examining and scrutinizing, they criticize those deserving of criticism, and they praise those deserving of praise. They’re not stingy regarding monasteries, families, and material possessions. A resident mendicant with these five qualities is raised up to heaven.” 

%
\section*{{\suttatitleacronym AN 5.239}{\suttatitletranslation Stinginess (1st) }{\suttatitleroot Paṭhamamacchariyasutta}}
\addcontentsline{toc}{section}{\tocacronym{AN 5.239} \toctranslation{Stinginess (1st) } \tocroot{Paṭhamamacchariyasutta}}
\markboth{Stinginess (1st) }{Paṭhamamacchariyasutta}
\extramarks{AN 5.239}{AN 5.239}

“Mendicants,\marginnote{1.1} a resident mendicant with five qualities is cast down to hell. What five? They’re stingy regarding monasteries, families, material possessions, and praise. And they waste a gift given in faith. A resident mendicant with these five qualities is cast down to hell. 

A\marginnote{2.1} resident mendicant with five qualities is raised up to heaven. What five? They’re not stingy regarding monasteries, families, material possessions, and praise. And they don’t waste a gift given in faith. A resident mendicant with these five qualities is raised up to heaven.” 

%
\section*{{\suttatitleacronym AN 5.240}{\suttatitletranslation Stinginess (2nd) }{\suttatitleroot Dutiyamacchariyasutta}}
\addcontentsline{toc}{section}{\tocacronym{AN 5.240} \toctranslation{Stinginess (2nd) } \tocroot{Dutiyamacchariyasutta}}
\markboth{Stinginess (2nd) }{Dutiyamacchariyasutta}
\extramarks{AN 5.240}{AN 5.240}

“Mendicants,\marginnote{1.1} a resident mendicant with five qualities is cast down to hell. What five? They’re stingy regarding monasteries, families, material possessions, praise, and the teachings. A resident mendicant with these five qualities is cast down to hell. 

A\marginnote{2.1} resident mendicant with five qualities is raised up to heaven. What five? They’re not stingy regarding monasteries, families, material possessions, praise, and the teachings. A resident mendicant with these five qualities is raised up to heaven.” 

%
\addtocontents{toc}{\let\protect\contentsline\protect\nopagecontentsline}
\chapter*{The Chapter on Bad Conduct }
\addcontentsline{toc}{chapter}{\tocchapterline{The Chapter on Bad Conduct }}
\addtocontents{toc}{\let\protect\contentsline\protect\oldcontentsline}

%
\section*{{\suttatitleacronym AN 5.241}{\suttatitletranslation Bad Conduct (1st) }{\suttatitleroot Paṭhamaduccaritasutta}}
\addcontentsline{toc}{section}{\tocacronym{AN 5.241} \toctranslation{Bad Conduct (1st) } \tocroot{Paṭhamaduccaritasutta}}
\markboth{Bad Conduct (1st) }{Paṭhamaduccaritasutta}
\extramarks{AN 5.241}{AN 5.241}

“Mendicants,\marginnote{1.1} there are these five drawbacks of bad conduct. What five? You blame yourself. After examination, sensible people criticize you. You get a bad reputation. You feel lost when you die. And when your body breaks up, after death, you’re reborn in a place of loss, a bad place, the underworld, hell. These are the five drawbacks of bad conduct. 

There\marginnote{2.1} are these five benefits of good conduct. What five? You don’t blame yourself. After examination, sensible people praise you. You get a good reputation. You don’t feel lost when you die. When your body breaks up, after death, you’re reborn in a good place, a heavenly realm. These are the five benefits of good conduct.” 

%
\section*{{\suttatitleacronym AN 5.242}{\suttatitletranslation Bad Bodily Conduct (1st) }{\suttatitleroot Paṭhamakāyaduccaritasutta}}
\addcontentsline{toc}{section}{\tocacronym{AN 5.242} \toctranslation{Bad Bodily Conduct (1st) } \tocroot{Paṭhamakāyaduccaritasutta}}
\markboth{Bad Bodily Conduct (1st) }{Paṭhamakāyaduccaritasutta}
\extramarks{AN 5.242}{AN 5.242}

“Mendicants,\marginnote{1.1} there are these five drawbacks in bad bodily conduct … benefits in good bodily conduct …” 

%
\section*{{\suttatitleacronym AN 5.243}{\suttatitletranslation Bad Verbal Conduct (1st) }{\suttatitleroot Paṭhamavacīduccaritasutta}}
\addcontentsline{toc}{section}{\tocacronym{AN 5.243} \toctranslation{Bad Verbal Conduct (1st) } \tocroot{Paṭhamavacīduccaritasutta}}
\markboth{Bad Verbal Conduct (1st) }{Paṭhamavacīduccaritasutta}
\extramarks{AN 5.243}{AN 5.243}

“Mendicants,\marginnote{1.1} there are these five drawbacks in bad verbal conduct … benefits in good verbal conduct …” 

%
\section*{{\suttatitleacronym AN 5.244}{\suttatitletranslation Bad Mental Conduct (1st) }{\suttatitleroot Paṭhamamanoduccaritasutta}}
\addcontentsline{toc}{section}{\tocacronym{AN 5.244} \toctranslation{Bad Mental Conduct (1st) } \tocroot{Paṭhamamanoduccaritasutta}}
\markboth{Bad Mental Conduct (1st) }{Paṭhamamanoduccaritasutta}
\extramarks{AN 5.244}{AN 5.244}

“Mendicants,\marginnote{1.1} there are these five drawbacks in bad mental conduct … benefits in good mental conduct …” 

%
\section*{{\suttatitleacronym AN 5.245}{\suttatitletranslation Bad Conduct (2nd) }{\suttatitleroot Dutiyaduccaritasutta}}
\addcontentsline{toc}{section}{\tocacronym{AN 5.245} \toctranslation{Bad Conduct (2nd) } \tocroot{Dutiyaduccaritasutta}}
\markboth{Bad Conduct (2nd) }{Dutiyaduccaritasutta}
\extramarks{AN 5.245}{AN 5.245}

“Mendicants,\marginnote{1.1} there are these five drawbacks of bad conduct. What five? You blame yourself. After examination, sensible people criticize you. You get a bad reputation. You drift away from true teachings. You settle on untrue teachings. These are the five drawbacks of bad conduct. 

There\marginnote{2.1} are these five benefits of good conduct. What five? You don’t blame yourself. After examination, sensible people praise you. You get a good reputation. You drift away from untrue teachings. You settle on true teachings. These are the five benefits of good conduct.” 

%
\section*{{\suttatitleacronym AN 5.246}{\suttatitletranslation Bad Bodily Conduct (2nd) }{\suttatitleroot Dutiyakāyaduccaritasutta}}
\addcontentsline{toc}{section}{\tocacronym{AN 5.246} \toctranslation{Bad Bodily Conduct (2nd) } \tocroot{Dutiyakāyaduccaritasutta}}
\markboth{Bad Bodily Conduct (2nd) }{Dutiyakāyaduccaritasutta}
\extramarks{AN 5.246}{AN 5.246}

“Mendicants,\marginnote{1.1} there are these five drawbacks in bad bodily conduct … benefits in good bodily conduct …” 

%
\section*{{\suttatitleacronym AN 5.247}{\suttatitletranslation Bad Verbal Conduct (2nd) }{\suttatitleroot Dutiyavacīduccaritasutta}}
\addcontentsline{toc}{section}{\tocacronym{AN 5.247} \toctranslation{Bad Verbal Conduct (2nd) } \tocroot{Dutiyavacīduccaritasutta}}
\markboth{Bad Verbal Conduct (2nd) }{Dutiyavacīduccaritasutta}
\extramarks{AN 5.247}{AN 5.247}

“Mendicants,\marginnote{1.1} there are these five drawbacks in bad verbal conduct … benefits in good verbal conduct …” 

%
\section*{{\suttatitleacronym AN 5.248}{\suttatitletranslation Bad Mental Conduct (2nd) }{\suttatitleroot Dutiyamanoduccaritasutta}}
\addcontentsline{toc}{section}{\tocacronym{AN 5.248} \toctranslation{Bad Mental Conduct (2nd) } \tocroot{Dutiyamanoduccaritasutta}}
\markboth{Bad Mental Conduct (2nd) }{Dutiyamanoduccaritasutta}
\extramarks{AN 5.248}{AN 5.248}

“Mendicants,\marginnote{1.1} there are these five drawbacks in bad mental conduct … benefits in good mental conduct …” 

%
\section*{{\suttatitleacronym AN 5.249}{\suttatitletranslation A Charnel Ground }{\suttatitleroot Sivathikasutta}}
\addcontentsline{toc}{section}{\tocacronym{AN 5.249} \toctranslation{A Charnel Ground } \tocroot{Sivathikasutta}}
\markboth{A Charnel Ground }{Sivathikasutta}
\extramarks{AN 5.249}{AN 5.249}

“Mendicants,\marginnote{1.1} there are these five drawbacks to a charnel ground. What five? 

It’s\marginnote{1.3} filthy, stinking, frightening, a gathering place for savage monsters, and a weeping place for many people. These are the five drawbacks of a charnel ground. 

In\marginnote{2.1} the same way there are five drawbacks of a person like a charnel ground. What five? To start with, some person has filthy conduct by way of body, speech, and mind. This is how they’re filthy, I say. That person is just as filthy as a charnel ground. 

Because\marginnote{3.1} of their filthy conduct, they get a bad reputation. This is how they’re stinky, I say. That person is just as stinky as a charnel ground. 

Because\marginnote{4.1} of their filthy conduct, good-hearted spiritual companions avoid them from afar. That’s how they’re frightening, I say. That person is just as frightening as a charnel ground. 

Because\marginnote{5.1} of their filthy conduct, they live together with people of a similar character. This is how they gather with savage monsters, I say. That person is just as much a gathering place of savage monsters as a charnel ground. 

Because\marginnote{6.1} of their filthy conduct, when good-hearted spiritual companions see them they complain: ‘Oh, it’s so painful for us to have to live together with such as these.’ This is how there’s weeping, I say. This person is just as much a weeping place for many people as a charnel ground. 

These\marginnote{6.6} are the five drawbacks of a person like a charnel ground.” 

%
\section*{{\suttatitleacronym AN 5.250}{\suttatitletranslation Faith in Individuals }{\suttatitleroot Puggalappasādasutta}}
\addcontentsline{toc}{section}{\tocacronym{AN 5.250} \toctranslation{Faith in Individuals } \tocroot{Puggalappasādasutta}}
\markboth{Faith in Individuals }{Puggalappasādasutta}
\extramarks{AN 5.250}{AN 5.250}

“Mendicants,\marginnote{1.1} there are these five drawbacks of placing faith in an individual. What five? 

The\marginnote{1.3} individual to whom a person is devoted falls into an offense such that the \textsanskrit{Saṅgha} suspends them. It occurs to them: ‘This person dear and beloved to me has been suspended by the \textsanskrit{Saṅgha}.’ They lose much of their faith in mendicants. So they don’t frequent other mendicants, they don’t hear the true teaching, and they fall away from the true teaching. This is the first drawback in placing faith in an individual. 

Furthermore,\marginnote{2.1} the individual to whom a person is devoted falls into an offense such that the \textsanskrit{Saṅgha} makes them sit at the end of the line. … This is the second drawback in placing faith in an individual. 

Furthermore,\marginnote{3.1} the individual to whom a person is devoted departs for another region … disrobes … passes away. It occurs to them: ‘This person dear and beloved to me has passed away.’ So they don’t frequent other mendicants, they don’t hear the true teaching, and they fall away from the true teaching. This is the fifth drawback in placing faith in an individual. 

These\marginnote{3.8} are the five drawbacks of placing faith in an individual.” 

%
\addtocontents{toc}{\let\protect\contentsline\protect\nopagecontentsline}
\pannasa{The Sixth Fifty }
\addcontentsline{toc}{pannasa}{The Sixth Fifty }
\markboth{}{}
\addtocontents{toc}{\let\protect\contentsline\protect\oldcontentsline}

%
\addtocontents{toc}{\let\protect\contentsline\protect\nopagecontentsline}
\chapter*{The Chapter on Ordination }
\addcontentsline{toc}{chapter}{\tocchapterline{The Chapter on Ordination }}
\addtocontents{toc}{\let\protect\contentsline\protect\oldcontentsline}

%
\section*{{\suttatitleacronym AN 5.251}{\suttatitletranslation Who Should Give Ordination }{\suttatitleroot Upasampādetabbasutta}}
\addcontentsline{toc}{section}{\tocacronym{AN 5.251} \toctranslation{Who Should Give Ordination } \tocroot{Upasampādetabbasutta}}
\markboth{Who Should Give Ordination }{Upasampādetabbasutta}
\extramarks{AN 5.251}{AN 5.251}

“Mendicants,\marginnote{1.1} ordination should be given by a mendicant with five qualities. What five? It’s a mendicant who has the entire spectrum of an adept’s ethics, immersion, wisdom, freedom, and the knowledge and vision of freedom. Ordination should be given by a mendicant with these five qualities.” 

%
\section*{{\suttatitleacronym AN 5.252}{\suttatitletranslation Who Should Give Dependence }{\suttatitleroot Nissayasutta}}
\addcontentsline{toc}{section}{\tocacronym{AN 5.252} \toctranslation{Who Should Give Dependence } \tocroot{Nissayasutta}}
\markboth{Who Should Give Dependence }{Nissayasutta}
\extramarks{AN 5.252}{AN 5.252}

“Mendicants,\marginnote{1.1} dependence should be given by a mendicant with five qualities. What five? It’s a mendicant who has the entire spectrum of an adept’s ethics, immersion, wisdom, freedom, and the knowledge and vision of freedom. Dependence should be given by a mendicant with these five qualities.” 

%
\section*{{\suttatitleacronym AN 5.253}{\suttatitletranslation Who Should Have a Novice as Attendant }{\suttatitleroot Sāmaṇerasutta}}
\addcontentsline{toc}{section}{\tocacronym{AN 5.253} \toctranslation{Who Should Have a Novice as Attendant } \tocroot{Sāmaṇerasutta}}
\markboth{Who Should Have a Novice as Attendant }{Sāmaṇerasutta}
\extramarks{AN 5.253}{AN 5.253}

“Mendicants,\marginnote{1.1} a novice should attend on a mendicant with five qualities. What five? It’s a mendicant who has the entire spectrum of an adept’s ethics, immersion, wisdom, freedom, and the knowledge and vision of freedom. A novice should attend on a mendicant with these five qualities.” 

%
\section*{{\suttatitleacronym AN 5.254}{\suttatitletranslation Five Kinds of Stinginess }{\suttatitleroot Pañcamacchariyasutta}}
\addcontentsline{toc}{section}{\tocacronym{AN 5.254} \toctranslation{Five Kinds of Stinginess } \tocroot{Pañcamacchariyasutta}}
\markboth{Five Kinds of Stinginess }{Pañcamacchariyasutta}
\extramarks{AN 5.254}{AN 5.254}

“Mendicants,\marginnote{1.1} there are these five kinds of stinginess. What five? Stinginess with dwellings, families, material possessions, praise, and the teachings. These are the five kinds of stinginess. The most contemptible of these five kinds of stinginess is stinginess with the teachings.” 

%
\section*{{\suttatitleacronym AN 5.255}{\suttatitletranslation Giving Up Stinginess }{\suttatitleroot Macchariyappahānasutta}}
\addcontentsline{toc}{section}{\tocacronym{AN 5.255} \toctranslation{Giving Up Stinginess } \tocroot{Macchariyappahānasutta}}
\markboth{Giving Up Stinginess }{Macchariyappahānasutta}
\extramarks{AN 5.255}{AN 5.255}

“Mendicants,\marginnote{1.1} the spiritual life is lived to give up and cut out these five kinds of stinginess. What five? Stinginess with dwellings, families, material possessions, praise, and the teachings. The spiritual life is lived to give up and cut out these five kinds of stinginess.” 

%
\section*{{\suttatitleacronym AN 5.256}{\suttatitletranslation The First Absorption }{\suttatitleroot Paṭhamajhānasutta}}
\addcontentsline{toc}{section}{\tocacronym{AN 5.256} \toctranslation{The First Absorption } \tocroot{Paṭhamajhānasutta}}
\markboth{The First Absorption }{Paṭhamajhānasutta}
\extramarks{AN 5.256}{AN 5.256}

“Mendicants,\marginnote{1.1} without giving up these five qualities you can’t enter and remain in the first absorption. What five? Stinginess with dwellings, families, material possessions, praise, and the teachings. Without giving up these five qualities you can’t enter and remain in the first absorption. 

But\marginnote{2.1} after giving up these five qualities you can enter and remain in the first absorption. What five? Stinginess with dwellings, families, material possessions, praise, and the teachings. After giving up these five qualities you can enter and remain in the first absorption.” 

%
\section*{{\suttatitleacronym AN 5.257–263}{\suttatitletranslation The Second Absorption, Etc. }{\suttatitleroot Dutiyajhānasuttādisattaka}}
\addcontentsline{toc}{section}{\tocacronym{AN 5.257–263} \toctranslation{The Second Absorption, Etc. } \tocroot{Dutiyajhānasuttādisattaka}}
\markboth{The Second Absorption, Etc. }{Dutiyajhānasuttādisattaka}
\extramarks{AN 5.257–263}{AN 5.257–263}

“Mendicants,\marginnote{1.1} without giving up these five qualities you can’t enter and remain in the second absorption … third absorption … fourth absorption … or realize the fruit of stream-entry … the fruit of once-return … the fruit of non-return … perfection. What five? Stinginess with dwellings, families, material possessions, praise, and the teachings. Without giving up these five qualities you can’t realize perfection. 

But\marginnote{2.1} after giving up these five qualities you can enter and remain in the second absorption … third absorption … fourth absorption … and realize the fruit of stream-entry … the fruit of once-return … the fruit of non-return … perfection. What five? Stinginess with dwellings, families, material possessions, praise, and the teachings. After giving up these five qualities you can realize perfection.” 

%
\section*{{\suttatitleacronym AN 5.264}{\suttatitletranslation Another Discourse on the First Absorption }{\suttatitleroot Aparapaṭhamajhānasutta}}
\addcontentsline{toc}{section}{\tocacronym{AN 5.264} \toctranslation{Another Discourse on the First Absorption } \tocroot{Aparapaṭhamajhānasutta}}
\markboth{Another Discourse on the First Absorption }{Aparapaṭhamajhānasutta}
\extramarks{AN 5.264}{AN 5.264}

“Mendicants,\marginnote{1.1} without giving up these five qualities you can’t enter and remain in the first absorption. What five? Stinginess with dwellings, families, material possessions, praise, and lack of gratitude and thankfulness. Without giving up these five qualities you can’t enter and remain in the first absorption. 

But\marginnote{2.1} after giving up these five qualities you can enter and remain in the first absorption. What five? Stinginess with dwellings, families, material possessions, praise, and lack of gratitude and thankfulness. After giving up these five qualities you can enter and remain in the first absorption.” 

%
\section*{{\suttatitleacronym AN 5.265–271}{\suttatitletranslation Another Discourse on the Second Absorption, Etc. }{\suttatitleroot Aparadutiyajhānasuttādi}}
\addcontentsline{toc}{section}{\tocacronym{AN 5.265–271} \toctranslation{Another Discourse on the Second Absorption, Etc. } \tocroot{Aparadutiyajhānasuttādi}}
\markboth{Another Discourse on the Second Absorption, Etc. }{Aparadutiyajhānasuttādi}
\extramarks{AN 5.265–271}{AN 5.265–271}

“Mendicants,\marginnote{1.1} without giving up these five qualities you can’t enter and remain in the second absorption … third absorption … fourth absorption … or realize the fruit of stream-entry … the fruit of once-return … the fruit of non-return … perfection. What five? Stinginess with dwellings, families, material possessions, praise, and lack of gratitude and thankfulness. Without giving up these five qualities you can’t realize perfection. 

But\marginnote{2.1} after giving up these five qualities you can enter and remain in the second absorption … third absorption … fourth absorption … and realize the fruit of stream-entry … the fruit of once-return … the fruit of non-return … perfection. What five? Stinginess with dwellings, families, material possessions, praise, and lack of gratitude and thankfulness. After giving up these five qualities you can realize perfection.” 

%
\addtocontents{toc}{\let\protect\contentsline\protect\nopagecontentsline}
\chapter*{Abbreviated Texts on Appointments }
\addcontentsline{toc}{chapter}{\tocchapterline{Abbreviated Texts on Appointments }}
\addtocontents{toc}{\let\protect\contentsline\protect\oldcontentsline}

%
\section*{{\suttatitleacronym AN 5.272}{\suttatitletranslation A Meal Designator }{\suttatitleroot Bhattuddesakasutta}}
\addcontentsline{toc}{section}{\tocacronym{AN 5.272} \toctranslation{A Meal Designator } \tocroot{Bhattuddesakasutta}}
\markboth{A Meal Designator }{Bhattuddesakasutta}
\extramarks{AN 5.272}{AN 5.272}

“Mendicants,\marginnote{1.1} a person with five qualities should not be appointed as meal designator. What five? They make decisions prejudiced by favoritism, hostility, stupidity, and cowardice. And they don’t know if a meal has been assigned or not. A person with these five qualities should not be appointed as meal designator. 

A\marginnote{2.1} person with five qualities should be appointed as meal designator. What five? They don’t make decisions prejudiced by favoritism, hostility, stupidity, and cowardice. And they know if a meal has been assigned or not. A person with these five qualities should be appointed as meal allocator. 

A\marginnote{3.1} person with five qualities who has been appointed as meal designator should not be called upon … should be called upon … should be known as a fool … should be known as astute … they keep themselves broken and damaged … they keep themselves unbroken and undamaged … is cast down to hell … is raised up to heaven. What five? They don’t make decisions prejudiced by favoritism, hostility, stupidity, and cowardice. And they know if a meal has been assigned or not. A meal designator with these five qualities is raised up to heaven.” 

%
\section*{{\suttatitleacronym AN 5.273–285}{\suttatitletranslation A Lodgings Assigner }{\suttatitleroot Senāsanapaññāpakasuttādi}}
\addcontentsline{toc}{section}{\tocacronym{AN 5.273–285} \toctranslation{A Lodgings Assigner } \tocroot{Senāsanapaññāpakasuttādi}}
\markboth{A Lodgings Assigner }{Senāsanapaññāpakasuttādi}
\extramarks{AN 5.273–285}{AN 5.273–285}

“Mendicants,\marginnote{1.1} a person with five qualities should not be appointed as lodgings assigner … they don’t know if a lodging has been assigned or not … A person with five qualities should be appointed as lodgings assigner … they know if a lodging has been assigned or not … 

A\marginnote{2.1} person should not be appointed as lodgings allocator … they don’t know if a lodging has been allocated or not … A person should be appointed as lodgings allocator … they know if a lodging has been allocated or not … 

A\marginnote{3.1} person should not be appointed as storeperson … they don’t know if stores are protected or not … A person should be appointed as storeperson … they know if stores are protected or not … 

…\marginnote{4.1} robe receiver … 

…\marginnote{5.1} robe distributor … 

…\marginnote{6.1} porridge distributor … 

…\marginnote{7.1} fruit distributor … 

…\marginnote{8.1} cake distributor … 

…\marginnote{9.1} dispenser of minor accessories … 

…\marginnote{10.1} allocator of bathing cloths … 

…\marginnote{11.1} bowl allocator … 

…\marginnote{12.1} supervisor of monastery staff … 

…\marginnote{13.1} supervisor of novices … 

What\marginnote{14.1} five? They don’t make decisions prejudiced by favoritism, hostility, stupidity, and cowardice. And they know if a novice has been supervised or not. A supervisor of novices with these five qualities is raised up to heaven.” 

%
\addtocontents{toc}{\let\protect\contentsline\protect\nopagecontentsline}
\chapter*{Abbreviated Texts on Training Rules }
\addcontentsline{toc}{chapter}{\tocchapterline{Abbreviated Texts on Training Rules }}
\addtocontents{toc}{\let\protect\contentsline\protect\oldcontentsline}

%
\section*{{\suttatitleacronym AN 5.286}{\suttatitletranslation A Monk }{\suttatitleroot Bhikkhusutta}}
\addcontentsline{toc}{section}{\tocacronym{AN 5.286} \toctranslation{A Monk } \tocroot{Bhikkhusutta}}
\markboth{A Monk }{Bhikkhusutta}
\extramarks{AN 5.286}{AN 5.286}

“Mendicants,\marginnote{1.1} a monk with five qualities is cast down to hell. What five? He kills living creatures, steals, has sex, lies, and uses alcoholic drinks that cause negligence. A monk with these five qualities is cast down to hell. 

A\marginnote{2.1} monk with five qualities is raised up to heaven. What five? He doesn’t kill living creatures, steal, have sex, lie, or use alcoholic drinks that cause negligence. A monk with these five qualities is raised up to heaven.” 

%
\section*{{\suttatitleacronym AN 5.287–292}{\suttatitletranslation A Nun }{\suttatitleroot Bhikkhunīsuttādi}}
\addcontentsline{toc}{section}{\tocacronym{AN 5.287–292} \toctranslation{A Nun } \tocroot{Bhikkhunīsuttādi}}
\markboth{A Nun }{Bhikkhunīsuttādi}
\extramarks{AN 5.287–292}{AN 5.287–292}

“A\marginnote{1.1} nun … trainee nun … novice monk … novice nun … layman … laywoman … with five qualities is cast down to hell. What five? They kill living creatures, steal, commit sexual misconduct, lie, and use alcoholic drinks that cause negligence. With these five qualities they’re cast down to hell. 

A\marginnote{2.1} nun … trainee nun … novice monk … novice nun … layman … laywoman … with five qualities is raised up to heaven. What five? They don’t kill living creatures, steal, commit sexual misconduct, lie, or use alcoholic drinks that cause negligence. With these five qualities they’re raised up to heaven.” 

%
\section*{{\suttatitleacronym AN 5.293}{\suttatitletranslation An Ājīvaka }{\suttatitleroot Ājīvakasutta}}
\addcontentsline{toc}{section}{\tocacronym{AN 5.293} \toctranslation{An Ājīvaka } \tocroot{Ājīvakasutta}}
\markboth{An Ājīvaka }{Ājīvakasutta}
\extramarks{AN 5.293}{AN 5.293}

“Mendicants,\marginnote{1.1} an \textsanskrit{Ājīvaka} ascetic with five qualities is cast down to hell. What five? They kill living creatures, steal, have sex, lie, and use alcoholic drinks that cause negligence. An \textsanskrit{Ājīvaka} ascetic with these five qualities is cast down to hell.” 

%
\section*{{\suttatitleacronym AN 5.294–302}{\suttatitletranslation A Nigaṇṭha, Etc. }{\suttatitleroot Nigaṇṭhasuttādi}}
\addcontentsline{toc}{section}{\tocacronym{AN 5.294–302} \toctranslation{A Nigaṇṭha, Etc. } \tocroot{Nigaṇṭhasuttādi}}
\markboth{A Nigaṇṭha, Etc. }{Nigaṇṭhasuttādi}
\extramarks{AN 5.294–302}{AN 5.294–302}

A\marginnote{1.1} Jain … disciple of the shavelings … a matted-hair ascetic … a wanderer … a follower of \textsanskrit{Māgaṇḍiya} … a trident-bearing ascetic … a follower of the unobstructed … a follower of Gotamaka … one who performs rituals for the gods … with five qualities is cast down to hell. What five? They kill living creatures, steal, commit sexual misconduct, lie, and use alcoholic drinks that cause negligence. With these five qualities they’re cast down to hell.” 

%
\addtocontents{toc}{\let\protect\contentsline\protect\nopagecontentsline}
\chapter*{Abbreviated Texts Beginning With Greed }
\addcontentsline{toc}{chapter}{\tocchapterline{Abbreviated Texts Beginning With Greed }}
\addtocontents{toc}{\let\protect\contentsline\protect\oldcontentsline}

%
\section*{{\suttatitleacronym AN 5.303}{\suttatitletranslation Untitled Discourse on Greed (1st) }{\suttatitleroot \textasciitilde }}
\addcontentsline{toc}{section}{\tocacronym{AN 5.303} \toctranslation{Untitled Discourse on Greed (1st) } \tocroot{\textasciitilde }}
\markboth{Untitled Discourse on Greed (1st) }{\textasciitilde }
\extramarks{AN 5.303}{AN 5.303}

“For\marginnote{1.1} insight into greed, five things should be developed. What five? The perceptions of ugliness, death, drawbacks, repulsiveness of food, and dissatisfaction with the whole world. For insight into greed, these five things should be developed.” 

%
\section*{{\suttatitleacronym AN 5.304}{\suttatitletranslation Untitled Discourse on Greed (2nd) }{\suttatitleroot \textasciitilde }}
\addcontentsline{toc}{section}{\tocacronym{AN 5.304} \toctranslation{Untitled Discourse on Greed (2nd) } \tocroot{\textasciitilde }}
\markboth{Untitled Discourse on Greed (2nd) }{\textasciitilde }
\extramarks{AN 5.304}{AN 5.304}

“For\marginnote{1.1} insight into greed, five things should be developed. What five? The perceptions of impermanence, not-self, death, repulsiveness of food, and dissatisfaction with the whole world. For insight into greed, these five things should be developed.” 

%
\section*{{\suttatitleacronym AN 5.305}{\suttatitletranslation Untitled Discourse on Greed (3rd) }{\suttatitleroot \textasciitilde }}
\addcontentsline{toc}{section}{\tocacronym{AN 5.305} \toctranslation{Untitled Discourse on Greed (3rd) } \tocroot{\textasciitilde }}
\markboth{Untitled Discourse on Greed (3rd) }{\textasciitilde }
\extramarks{AN 5.305}{AN 5.305}

“For\marginnote{1.1} insight into greed, five things should be developed. What five? The perception of impermanence, the perception of suffering in impermanence, the perception of not-self in suffering, the perception of giving up, and the perception of fading away. For insight into greed, these five things should be developed.” 

%
\section*{{\suttatitleacronym AN 5.306}{\suttatitletranslation Untitled Discourse on Greed (4th) }{\suttatitleroot \textasciitilde }}
\addcontentsline{toc}{section}{\tocacronym{AN 5.306} \toctranslation{Untitled Discourse on Greed (4th) } \tocroot{\textasciitilde }}
\markboth{Untitled Discourse on Greed (4th) }{\textasciitilde }
\extramarks{AN 5.306}{AN 5.306}

“For\marginnote{1.1} insight into greed, five things should be developed. What five? The faculties of faith, energy, mindfulness, immersion, and wisdom. For insight into greed, these five things should be developed.” 

%
\section*{{\suttatitleacronym AN 5.307}{\suttatitletranslation Untitled Discourse on Greed (5th) }{\suttatitleroot \textasciitilde }}
\addcontentsline{toc}{section}{\tocacronym{AN 5.307} \toctranslation{Untitled Discourse on Greed (5th) } \tocroot{\textasciitilde }}
\markboth{Untitled Discourse on Greed (5th) }{\textasciitilde }
\extramarks{AN 5.307}{AN 5.307}

“For\marginnote{1.1} insight into greed, five things should be developed. What five? The powers of faith, energy, mindfulness, immersion, and wisdom. For insight into greed, these five things should be developed.” 

%
\section*{{\suttatitleacronym AN 5.308–1152}{\suttatitletranslation Untitled Discourses on Greed, Etc. }{\suttatitleroot \textasciitilde }}
\addcontentsline{toc}{section}{\tocacronym{AN 5.308–1152} \toctranslation{Untitled Discourses on Greed, Etc. } \tocroot{\textasciitilde }}
\markboth{Untitled Discourses on Greed, Etc. }{\textasciitilde }
\extramarks{AN 5.308–1152}{AN 5.308–1152}

“For\marginnote{1.1} the complete understanding … finishing … giving up … ending … vanishing … fading away … cessation … giving away … letting go of greed, five things should be developed.” 

“Of\marginnote{1.2} hate … delusion … anger … hostility … disdain … contempt … jealousy … stinginess … deceit … deviousness … obstinacy … aggression … conceit … arrogance … vanity … negligence … for insight … complete understanding … finishing … giving up … ending … vanishing … fading away … cessation … giving away … letting go … five things should be developed. 

What\marginnote{2.1} five? The powers of faith, energy, mindfulness, immersion, and wisdom. For the letting go of negligence, these five things should be developed.” 

\scendsutta{The Book of the Fives is finished. }

\scendbook{The Book of the Fives is finished. }

%
\addtocontents{toc}{\let\protect\contentsline\protect\nopagecontentsline}
\part*{The Book of the Sixes }
\addcontentsline{toc}{part}{The Book of the Sixes }
\markboth{}{}
\addtocontents{toc}{\let\protect\contentsline\protect\oldcontentsline}

%
%
\addtocontents{toc}{\let\protect\contentsline\protect\nopagecontentsline}
\pannasa{The First Fifty }
\addcontentsline{toc}{pannasa}{The First Fifty }
\markboth{}{}
\addtocontents{toc}{\let\protect\contentsline\protect\oldcontentsline}

%
\addtocontents{toc}{\let\protect\contentsline\protect\nopagecontentsline}
\chapter*{The Chapter on Worthy of Offerings }
\addcontentsline{toc}{chapter}{\tocchapterline{The Chapter on Worthy of Offerings }}
\addtocontents{toc}{\let\protect\contentsline\protect\oldcontentsline}

%
\section*{{\suttatitleacronym AN 6.1}{\suttatitletranslation Worthy of Offerings (1st) }{\suttatitleroot Paṭhamaāhuneyyasutta}}
\addcontentsline{toc}{section}{\tocacronym{AN 6.1} \toctranslation{Worthy of Offerings (1st) } \tocroot{Paṭhamaāhuneyyasutta}}
\markboth{Worthy of Offerings (1st) }{Paṭhamaāhuneyyasutta}
\extramarks{AN 6.1}{AN 6.1}

\scevam{So\marginnote{1.1} I have heard. }At one time the Buddha was staying near \textsanskrit{Sāvatthī} in Jeta’s Grove, \textsanskrit{Anāthapiṇḍika}’s monastery. There the Buddha addressed the mendicants, “Mendicants!” 

“Venerable\marginnote{1.5} sir,” they replied. The Buddha said this: 

“Mendicants,\marginnote{2.1} a mendicant with six qualities is worthy of offerings dedicated to the gods, worthy of hospitality, worthy of a religious donation, worthy of veneration with joined palms, and is the supreme field of merit for the world. What six? 

It’s\marginnote{2.3} a mendicant who, when they see a sight with their eyes, is neither happy nor sad. They remain equanimous, mindful and aware. 

When\marginnote{2.4} they hear a sound with their ears … 

When\marginnote{2.5} they smell an odor with their nose … 

When\marginnote{2.6} they taste a flavor with their tongue … 

When\marginnote{2.7} they feel a touch with their body … 

When\marginnote{2.8} they know a thought with their mind, they’re neither happy nor sad. They remain equanimous, mindful and aware. 

A\marginnote{2.9} mendicant with these six qualities is worthy of offerings dedicated to the gods, worthy of hospitality, worthy of a religious donation, worthy of veneration with joined palms, and is the supreme field of merit for the world.” 

That\marginnote{3.1} is what the Buddha said. Satisfied, the mendicants were happy with what the Buddha said. 

%
\section*{{\suttatitleacronym AN 6.2}{\suttatitletranslation Worthy of Offerings (2nd) }{\suttatitleroot Dutiyaāhuneyyasutta}}
\addcontentsline{toc}{section}{\tocacronym{AN 6.2} \toctranslation{Worthy of Offerings (2nd) } \tocroot{Dutiyaāhuneyyasutta}}
\markboth{Worthy of Offerings (2nd) }{Dutiyaāhuneyyasutta}
\extramarks{AN 6.2}{AN 6.2}

“Mendicants,\marginnote{1.1} a mendicant with six qualities is worthy of offerings dedicated to the gods, worthy of hospitality, worthy of a religious donation, worthy of veneration with joined palms, and is the supreme field of merit for the world. What six? 

It’s\marginnote{1.3} a mendicant who wields the many kinds of psychic power: multiplying themselves and becoming one again; appearing and disappearing; going unimpeded through a wall, a rampart, or a mountain as if through space; diving in and out of the earth as if it were water; walking on water as if it were earth; flying cross-legged through the sky like a bird; touching and stroking with the hand the sun and moon, so mighty and powerful. They control the body as far as the \textsanskrit{Brahmā} realm. 

With\marginnote{2.1} clairaudience that is purified and superhuman, they hear both kinds of sounds, human and divine, whether near or far. 

They\marginnote{3.1} understand the minds of other beings and individuals, having comprehended them with their own mind. They understand mind with greed as ‘mind with greed’, and mind without greed as ‘mind without greed’. They understand mind with hate … mind without hate … mind with delusion … mind without delusion … constricted mind … scattered mind … expansive mind … unexpansive mind … mind that is not supreme … mind that is supreme … mind immersed in \textsanskrit{samādhi} … mind not immersed in \textsanskrit{samādhi} … freed mind … They understand unfreed mind as ‘unfreed mind’. 

They\marginnote{4.1} recollect many kinds of past lives. That is: one, two, three, four, five, ten, twenty, thirty, forty, fifty, a hundred, a thousand, a hundred thousand rebirths; many eons of the world contracting, many eons of the world expanding, many eons of the world contracting and expanding. They remember: ‘There, I was named this, my clan was that, I looked like this, and that was my food. This was how I felt pleasure and pain, and that was how my life ended. When I passed away from that place I was reborn somewhere else. There, too, I was named this, my clan was that, I looked like this, and that was my food. This was how I felt pleasure and pain, and that was how my life ended. When I passed away from that place I was reborn here.’ And so they recollect their many kinds of past lives, with features and details. 

With\marginnote{5.1} clairvoyance that is purified and superhuman, they see sentient beings passing away and being reborn—inferior and superior, beautiful and ugly, in a good place or a bad place. They understand how sentient beings are reborn according to their deeds: ‘These dear beings did bad things by way of body, speech, and mind. They spoke ill of the noble ones; they had wrong view; and they acted out of that wrong view. When their body breaks up, after death, they’re reborn in a place of loss, a bad place, the underworld, hell. These dear beings, however, did good things by way of body, speech, and mind. They never spoke ill of the noble ones; they had right view; and they acted out of that right view. When their body breaks up, after death, they’re reborn in a good place, a heavenly realm.’ And so, with clairvoyance that is purified and superhuman, they see sentient beings passing away and being reborn—inferior and superior, beautiful and ugly, in a good place or a bad place. They understand how sentient beings are reborn according to their deeds. 

They\marginnote{6.1} realize the undefiled freedom of heart and freedom by wisdom in this very life. And they live having realized it with their own insight due to the ending of defilements. 

A\marginnote{7.1} mendicant with these six qualities is worthy of offerings dedicated to the gods, worthy of hospitality, worthy of a religious donation, worthy of veneration with joined palms, and is the supreme field of merit for the world.” 

%
\section*{{\suttatitleacronym AN 6.3}{\suttatitletranslation Faculties }{\suttatitleroot Indriyasutta}}
\addcontentsline{toc}{section}{\tocacronym{AN 6.3} \toctranslation{Faculties } \tocroot{Indriyasutta}}
\markboth{Faculties }{Indriyasutta}
\extramarks{AN 6.3}{AN 6.3}

“Mendicants,\marginnote{1.1} a mendicant with six qualities is worthy of offerings dedicated to the gods, worthy of hospitality, worthy of a religious donation, worthy of veneration with joined palms, and is the supreme field of merit for the world. What six? The faculties of faith, energy, mindfulness, immersion, and wisdom. And they realize the undefiled freedom of heart and freedom by wisdom in this very life, and live having realized it with their own insight due to the ending of defilements. A mendicant with these six qualities is worthy of offerings dedicated to the gods, worthy of hospitality, worthy of a religious donation, worthy of veneration with joined palms, and is the supreme field of merit for the world.” 

%
\section*{{\suttatitleacronym AN 6.4}{\suttatitletranslation Powers }{\suttatitleroot Balasutta}}
\addcontentsline{toc}{section}{\tocacronym{AN 6.4} \toctranslation{Powers } \tocroot{Balasutta}}
\markboth{Powers }{Balasutta}
\extramarks{AN 6.4}{AN 6.4}

“Mendicants,\marginnote{1.1} a mendicant with six qualities is worthy of offerings dedicated to the gods, worthy of hospitality, worthy of a religious donation, worthy of veneration with joined palms, and is the supreme field of merit for the world. What six? The powers of faith, energy, mindfulness, immersion, and wisdom. And they realize the undefiled freedom of heart and freedom by wisdom in this very life, and live having realized it with their own insight due to the ending of defilements. A mendicant with these six qualities is worthy of offerings dedicated to the gods, worthy of hospitality, worthy of a religious donation, worthy of veneration with joined palms, and is the supreme field of merit for the world.” 

%
\section*{{\suttatitleacronym AN 6.5}{\suttatitletranslation The Thoroughbred (1st) }{\suttatitleroot Paṭhamaājānīyasutta}}
\addcontentsline{toc}{section}{\tocacronym{AN 6.5} \toctranslation{The Thoroughbred (1st) } \tocroot{Paṭhamaājānīyasutta}}
\markboth{The Thoroughbred (1st) }{Paṭhamaājānīyasutta}
\extramarks{AN 6.5}{AN 6.5}

“Mendicants,\marginnote{1.1} a fine royal thoroughbred with six factors is worthy of a king, fit to serve a king, and is considered a factor of kingship. 

What\marginnote{2.1} six? It’s when a fine royal thoroughbred can endure sights, sounds, smells, tastes, and touches. And it’s beautiful. A fine royal thoroughbred with these six factors is worthy of a king, fit to serve a king, and is considered a factor of kingship. 

In\marginnote{3.1} the same way, a mendicant with six qualities is worthy of offerings dedicated to the gods, worthy of hospitality, worthy of a religious donation, worthy of veneration with joined palms, and is the supreme field of merit for the world. What six? It’s when a mendicant can endure sights, sounds, smells, tastes, touches, and thoughts. A mendicant with these six qualities is worthy of offerings dedicated to the gods, worthy of hospitality, worthy of a religious donation, worthy of veneration with joined palms, and is the supreme field of merit for the world.” 

%
\section*{{\suttatitleacronym AN 6.6}{\suttatitletranslation The Thoroughbred (2nd) }{\suttatitleroot Dutiyaājānīyasutta}}
\addcontentsline{toc}{section}{\tocacronym{AN 6.6} \toctranslation{The Thoroughbred (2nd) } \tocroot{Dutiyaājānīyasutta}}
\markboth{The Thoroughbred (2nd) }{Dutiyaājānīyasutta}
\extramarks{AN 6.6}{AN 6.6}

“Mendicants,\marginnote{1.1} a fine royal thoroughbred with six factors is worthy of a king, fit to serve a king, and is considered a factor of kingship. What six? It’s when a fine royal thoroughbred can endure sights, sounds, smells, tastes, and touches. And it’s strong. A fine royal thoroughbred with these six factors is worthy of a king, fit to serve a king, and is considered a factor of kingship. 

In\marginnote{2.1} the same way, a mendicant with six qualities is worthy of offerings dedicated to the gods, worthy of hospitality, worthy of a religious donation, worthy of veneration with joined palms, and is the supreme field of merit for the world. What six? It’s when a mendicant can endure sights, sounds, smells, tastes, touches, and thoughts. A mendicant with these six qualities is worthy of offerings dedicated to the gods, worthy of hospitality, worthy of a religious donation, worthy of veneration with joined palms, and is the supreme field of merit for the world.” 

%
\section*{{\suttatitleacronym AN 6.7}{\suttatitletranslation The Thoroughbred (3rd) }{\suttatitleroot Tatiyaājānīyasutta}}
\addcontentsline{toc}{section}{\tocacronym{AN 6.7} \toctranslation{The Thoroughbred (3rd) } \tocroot{Tatiyaājānīyasutta}}
\markboth{The Thoroughbred (3rd) }{Tatiyaājānīyasutta}
\extramarks{AN 6.7}{AN 6.7}

“Mendicants,\marginnote{1.1} a fine royal thoroughbred with six factors is worthy of a king, fit to serve a king, and is considered a factor of kingship. What six? It’s when a fine royal thoroughbred can endure sights, sounds, smells, tastes, and touches. And it’s fast. A fine royal thoroughbred with these six factors is worthy of a king, fit to serve a king, and is considered a factor of kingship. 

In\marginnote{2.1} the same way, a mendicant with six qualities is worthy of offerings dedicated to the gods, worthy of hospitality, worthy of a religious donation, worthy of veneration with joined palms, and is the supreme field of merit for the world. What six? It’s when a mendicant can endure sights, sounds, smells, tastes, touches, and thoughts. A mendicant with these six qualities is worthy of offerings dedicated to the gods, worthy of hospitality, worthy of a religious donation, worthy of veneration with joined palms, and is the supreme field of merit for the world.” 

%
\section*{{\suttatitleacronym AN 6.8}{\suttatitletranslation Unsurpassable }{\suttatitleroot Anuttariyasutta}}
\addcontentsline{toc}{section}{\tocacronym{AN 6.8} \toctranslation{Unsurpassable } \tocroot{Anuttariyasutta}}
\markboth{Unsurpassable }{Anuttariyasutta}
\extramarks{AN 6.8}{AN 6.8}

“Mendicants,\marginnote{1.1} these six things are unsurpassable. What six? The unsurpassable seeing, listening, acquisition, training, service, and recollection. These are the six unsurpassable things.” 

%
\section*{{\suttatitleacronym AN 6.9}{\suttatitletranslation Topics for Recollection }{\suttatitleroot Anussatiṭṭhānasutta}}
\addcontentsline{toc}{section}{\tocacronym{AN 6.9} \toctranslation{Topics for Recollection } \tocroot{Anussatiṭṭhānasutta}}
\markboth{Topics for Recollection }{Anussatiṭṭhānasutta}
\extramarks{AN 6.9}{AN 6.9}

“Mendicants,\marginnote{1.1} there are these six topics for recollection. What six? The recollection of the Buddha, the teaching, the \textsanskrit{Saṅgha}, ethics, generosity, and the deities. These are the six topics for recollection.” 

%
\section*{{\suttatitleacronym AN 6.10}{\suttatitletranslation With Mahānāma }{\suttatitleroot Mahānāmasutta}}
\addcontentsline{toc}{section}{\tocacronym{AN 6.10} \toctranslation{With Mahānāma } \tocroot{Mahānāmasutta}}
\markboth{With Mahānāma }{Mahānāmasutta}
\extramarks{AN 6.10}{AN 6.10}

At\marginnote{1.1} one time the Buddha was staying in the land of the Sakyans, near Kapilavatthu in the Banyan Tree Monastery. Then \textsanskrit{Mahānāma} the Sakyan went up to the Buddha, bowed, sat down to one side, and said to him: 

“Sir,\marginnote{1.3} when a noble disciple has reached the fruit and understood the instructions, what kind of meditation do they frequently practice?” 

“\textsanskrit{Mahānāma},\marginnote{2.1} when a noble disciple has reached the fruit and understood the instructions they frequently practice this kind of meditation. 

Firstly,\marginnote{2.2} a noble disciple recollects the Realized One: ‘That Blessed One is perfected, a fully awakened Buddha, accomplished in knowledge and conduct, holy, knower of the world, supreme guide for those who wish to train, teacher of gods and humans, awakened, blessed.’ When a noble disciple recollects the Realized One their mind is not full of greed, hate, and delusion. At that time their mind is unswerving, based on the Realized One. A noble disciple whose mind is unswerving finds inspiration in the meaning and the teaching, and finds joy connected with the teaching. When they’re joyful, rapture springs up. When the mind is full of rapture, the body becomes tranquil. When the body is tranquil, they feel bliss. And when they’re blissful, the mind becomes immersed in \textsanskrit{samādhi}. This is called a noble disciple who lives in balance among people who are unbalanced, and lives untroubled among people who are troubled. They’ve entered the stream of the teaching and develop the recollection of the Buddha. 

Furthermore,\marginnote{3.1} a noble disciple recollects the teaching: ‘The teaching is well explained by the Buddha—visible in this very life, immediately effective, inviting inspection, relevant, so that sensible people can know it for themselves.’ When a noble disciple recollects the teaching their mind is not full of greed, hate, and delusion. … This is called a noble disciple who lives in balance among people who are unbalanced, and lives untroubled among people who are troubled. They’ve entered the stream of the teaching and develop the recollection of the teaching. 

Furthermore,\marginnote{4.1} a noble disciple recollects the \textsanskrit{Saṅgha}: ‘The \textsanskrit{Saṅgha} of the Buddha’s disciples is practicing the way that’s good, direct, methodical, and proper. It consists of the four pairs, the eight individuals. This is the \textsanskrit{Saṅgha} of the Buddha’s disciples that is worthy of offerings dedicated to the gods, worthy of hospitality, worthy of a religious donation, worthy of greeting with joined palms, and is the supreme field of merit for the world.’ When a noble disciple recollects the \textsanskrit{Saṅgha} their mind is not full of greed, hate, and delusion. … This is called a noble disciple who lives in balance among people who are unbalanced, and lives untroubled among people who are troubled. They’ve entered the stream of the teaching and develop the recollection of the \textsanskrit{Saṅgha}. 

Furthermore,\marginnote{5.1} a noble disciple recollects their own ethical conduct, which is unbroken, impeccable, spotless, and unmarred, liberating, praised by sensible people, not mistaken, and leading to immersion. When a noble disciple recollects their ethical conduct their mind is not full of greed, hate, and delusion. … This is called a noble disciple who lives in balance among people who are unbalanced, and lives untroubled among people who are troubled. They’ve entered the stream of the teaching and develop the recollection of ethics. 

Furthermore,\marginnote{6.1} a noble disciple recollects their own generosity: ‘I’m so fortunate, so very fortunate! Among people full of the stain of stinginess I live at home rid of stinginess, freely generous, open-handed, loving to let go, committed to charity, loving to give and to share.’ When a noble disciple recollects their own generosity their mind is not full of greed, hate, and delusion. … This is called a noble disciple who lives in balance among people who are unbalanced, and lives untroubled among people who are troubled. They’ve entered the stream of the teaching and develop the recollection of generosity. 

Furthermore,\marginnote{7.1} a noble disciple recollects the deities: ‘There are the Gods of the Four Great Kings, the Gods of the Thirty-Three, the Gods of Yama, the Joyful Gods, the Gods Who Love to Create, the Gods Who Control the Creations of Others, the Gods of \textsanskrit{Brahmā}’s Host, and gods even higher than these. When those deities passed away from here, they were reborn there because of their faith, ethics, learning, generosity, and wisdom. I, too, have the same kind of faith, ethics, learning, generosity, and wisdom.’ When a noble disciple recollects the faith, ethics, learning, generosity, and wisdom of both themselves and the deities their mind is not full of greed, hate, and delusion. At that time their mind is unswerving, based on the deities. A noble disciple whose mind is unswerving finds inspiration in the meaning and the teaching, and finds joy connected with the teaching. When you’re joyful, rapture springs up. When the mind is full of rapture, the body becomes tranquil. When the body is tranquil, you feel bliss. And when you’re blissful, the mind becomes immersed in \textsanskrit{samādhi}. This is called a noble disciple who lives in balance among people who are unbalanced, and lives untroubled among people who are troubled. They’ve entered the stream of the teaching and develop the recollection of the deities. 

When\marginnote{8.1} a noble disciple has reached the fruit and understood the instructions this is the kind of meditation they frequently practice.” 

%
\addtocontents{toc}{\let\protect\contentsline\protect\nopagecontentsline}
\chapter*{The Chapter on Warm-hearted }
\addcontentsline{toc}{chapter}{\tocchapterline{The Chapter on Warm-hearted }}
\addtocontents{toc}{\let\protect\contentsline\protect\oldcontentsline}

%
\section*{{\suttatitleacronym AN 6.11}{\suttatitletranslation Warm-hearted (1st) }{\suttatitleroot Paṭhamasāraṇīyasutta}}
\addcontentsline{toc}{section}{\tocacronym{AN 6.11} \toctranslation{Warm-hearted (1st) } \tocroot{Paṭhamasāraṇīyasutta}}
\markboth{Warm-hearted (1st) }{Paṭhamasāraṇīyasutta}
\extramarks{AN 6.11}{AN 6.11}

“Mendicants,\marginnote{1.1} there are these six warm-hearted qualities. What six? 

Firstly,\marginnote{1.3} a mendicant consistently treats their spiritual companions with bodily kindness, both in public and in private. This is a warm-hearted quality. 

Furthermore,\marginnote{2.1} a mendicant consistently treats their spiritual companions with verbal kindness, both in public and in private. This too is a warm-hearted quality. 

Furthermore,\marginnote{3.1} a mendicant consistently treats their spiritual companions with mental kindness … 

Furthermore,\marginnote{4.1} a mendicant shares without reservation any material possessions they have gained by legitimate means, even the food placed in the alms-bowl, using them in common with their ethical spiritual companions. This too is a warm-hearted quality. 

Furthermore,\marginnote{5.1} a mendicant lives according to the precepts shared with their spiritual companions, both in public and in private. Those precepts are unbroken, impeccable, spotless, and unmarred, liberating, praised by sensible people, not mistaken, and leading to immersion. This too is a warm-hearted quality. 

Furthermore,\marginnote{6.1} a mendicant lives according to the view shared with their spiritual companions, both in public and in private. That view is noble and emancipating, and leads one who practices it to the complete ending of suffering. This too is a warm-hearted quality. 

These\marginnote{7.1} are the six warm-hearted qualities.” 

%
\section*{{\suttatitleacronym AN 6.12}{\suttatitletranslation Warm-hearted (2nd) }{\suttatitleroot Dutiyasāraṇīyasutta}}
\addcontentsline{toc}{section}{\tocacronym{AN 6.12} \toctranslation{Warm-hearted (2nd) } \tocroot{Dutiyasāraṇīyasutta}}
\markboth{Warm-hearted (2nd) }{Dutiyasāraṇīyasutta}
\extramarks{AN 6.12}{AN 6.12}

“Mendicants,\marginnote{1.1} these six warm-hearted qualities make for fondness and respect, conducing to inclusion, harmony, and unity, without quarreling. What six? 

Firstly,\marginnote{1.3} a mendicant consistently treats their spiritual companions with bodily kindness, both in public and in private. This warm-hearted quality makes for fondness and respect, conducing to inclusion, harmony, and unity, without quarreling. 

Furthermore,\marginnote{2.1} a mendicant consistently treats their spiritual companions with verbal kindness … 

Furthermore,\marginnote{3.1} a mendicant consistently treats their spiritual companions with mental kindness … 

Furthermore,\marginnote{4.1} a mendicant shares without reservation any material possessions they have gained by legitimate means, even the food placed in the alms-bowl, using them in common with their ethical spiritual companions. This too is a warm-hearted quality. 

Furthermore,\marginnote{5.1} a mendicant lives according to the precepts shared with their spiritual companions, both in public and in private. Those precepts are unbroken, impeccable, spotless, and unmarred, liberating, praised by sensible people, not mistaken, and leading to immersion. This too is a warm-hearted quality. 

Furthermore,\marginnote{6.1} a mendicant lives according to the view shared with their spiritual companions, both in public and in private. That view is noble and emancipating, and leads one who practices it to the complete ending of suffering. This warm-hearted quality makes for fondness and respect, conducing to inclusion, harmony, and unity, without quarreling. 

These\marginnote{7.1} six warm-hearted qualities make for fondness and respect, conducing to inclusion, harmony, and unity, without quarreling.” 

%
\section*{{\suttatitleacronym AN 6.13}{\suttatitletranslation Elements of Escape }{\suttatitleroot Nissāraṇīyasutta}}
\addcontentsline{toc}{section}{\tocacronym{AN 6.13} \toctranslation{Elements of Escape } \tocroot{Nissāraṇīyasutta}}
\markboth{Elements of Escape }{Nissāraṇīyasutta}
\extramarks{AN 6.13}{AN 6.13}

“Mendicants,\marginnote{1.1} there are these six elements of escape. What six? 

Take\marginnote{1.3} a mendicant who says: ‘I’ve developed the heart’s release by love. I’ve cultivated it, made it my vehicle and my basis, kept it up, consolidated it, and properly implemented it. Yet somehow ill will still occupies my mind.’ They should be told, ‘Not so, venerable! Don’t say that. Don’t misrepresent the Buddha, for misrepresentation of the Buddha is not good. And the Buddha would not say that. It’s impossible, reverend, it cannot happen that the heart’s release by love has been developed and properly implemented, yet somehow ill will still occupies the mind. For it is the heart’s release by love that is the escape from ill will.’ 

Take\marginnote{2.1} another mendicant who says: ‘I’ve developed the heart’s release by compassion. I’ve cultivated it, made it my vehicle and my basis, kept it up, consolidated it, and properly implemented it. Yet somehow the thought of harming still occupies my mind.’ They should be told, ‘Not so, venerable! … For it is the heart’s release by compassion that is the escape from thoughts of harming.’ 

Take\marginnote{3.1} another mendicant who says: ‘I’ve developed the heart’s release by rejoicing. I’ve cultivated it, made it my vehicle and my basis, kept it up, consolidated it, and properly implemented it. Yet somehow discontent still occupies my mind.’ They should be told, ‘Not so, venerable! … For it is the heart’s release by rejoicing that is the escape from discontent.’ 

Take\marginnote{4.1} another mendicant who says: ‘I’ve developed the heart’s release by equanimity. I’ve cultivated it, made it my vehicle and my basis, kept it up, consolidated it, and properly implemented it. Yet somehow desire still occupies my mind.’ They should be told, ‘Not so, venerable! … For it is the heart’s release by equanimity that is the escape from desire.’ 

Take\marginnote{5.1} another mendicant who says: ‘I’ve developed the signless release of the heart. I’ve cultivated it, made it my vehicle and my basis, kept it up, consolidated it, and properly implemented it. Yet somehow my consciousness still follows after signs.’ They should be told, ‘Not so, venerable! … For it is the signless release of the heart that is the escape from all signs.’ 

Take\marginnote{6.1} another mendicant who says: ‘I’m rid of the conceit “I am”. And I don’t regard anything as “I am this”. Yet somehow the dart of doubt and indecision still occupies my mind.’ They should be told, ‘Not so, venerable! Don’t say that. Don’t misrepresent the Buddha, for misrepresentation of the Buddha is not good. And the Buddha would not say that. It’s impossible, reverend, it cannot happen that the conceit “I am” has been done away with, and nothing is regarded as “I am this”, yet somehow the dart of doubt and indecision still occupies the mind. For it is the uprooting of the conceit “I am” that is the escape from the dart of doubt and indecision.’ 

These\marginnote{7.1} are the six elements of escape.” 

%
\section*{{\suttatitleacronym AN 6.14}{\suttatitletranslation A Good Death }{\suttatitleroot Bhaddakasutta}}
\addcontentsline{toc}{section}{\tocacronym{AN 6.14} \toctranslation{A Good Death } \tocroot{Bhaddakasutta}}
\markboth{A Good Death }{Bhaddakasutta}
\extramarks{AN 6.14}{AN 6.14}

There\marginnote{1.1} \textsanskrit{Sāriputta} addressed the mendicants: “Reverends, mendicants!” 

“Reverend,”\marginnote{1.3} they replied. \textsanskrit{Sāriputta} said this: 

“A\marginnote{2.1} mendicant lives life so as to not have a good death. And how do they live life so as to not have a good death? 

Take\marginnote{3.1} a mendicant who relishes work, talk, sleep, company, closeness, and proliferation. They love these things and like to relish them. A mendicant who lives life like this does not have a good death. This is called a mendicant who enjoys identity, who hasn’t given up identity to rightly make an end of suffering. 

A\marginnote{4.1} mendicant lives life so as to have a good death. And how do they live life so as to have a good death? 

Take\marginnote{5.1} a mendicant who doesn’t relish work, talk, sleep, company, closeness, and proliferation. They don’t love these things or like to relish them. A mendicant who lives life like this has a good death. This is called a mendicant who delights in extinguishment, who has given up identity to rightly make an end of suffering. 

\begin{verse}%
A\marginnote{6.1} beast who likes to proliferate, \\
enjoying proliferation, \\
fails to win extinguishment, \\
the supreme sanctuary. 

But\marginnote{7.1} one who gives up proliferation, \\
enjoying the state of non-proliferation, \\
wins extinguishment, \\
the supreme sanctuary.” 

%
\end{verse}

%
\section*{{\suttatitleacronym AN 6.15}{\suttatitletranslation Regret }{\suttatitleroot Anutappiyasutta}}
\addcontentsline{toc}{section}{\tocacronym{AN 6.15} \toctranslation{Regret } \tocroot{Anutappiyasutta}}
\markboth{Regret }{Anutappiyasutta}
\extramarks{AN 6.15}{AN 6.15}

There\marginnote{1.1} \textsanskrit{Sāriputta} addressed the mendicants: 

“As\marginnote{1.2} a mendicant makes their bed, so they must lie in it, and die tormented by regrets. And how do they die tormented by regrets? 

Take\marginnote{2.1} a mendicant who relishes work, talk, sleep, company, closeness, and proliferation. They love these things and like to relish them. A mendicant who makes their bed like this must lie in it, and die tormented by regrets. This is called a mendicant who enjoys identity, who hasn’t given up identity to rightly make an end of suffering. 

As\marginnote{3.1} a mendicant makes their bed, so they must lie in it, and die free of regrets. And how do they die free of regrets? 

Take\marginnote{4.1} a mendicant who doesn’t relish work, talk, sleep, company, closeness, and proliferation. They don’t love these things or like to relish them. A mendicant who makes their bed like this must lie in it, and die free of regrets. This is called a mendicant who delights in extinguishment, who has given up identity to rightly make an end of suffering. 

\begin{verse}%
A\marginnote{5.1} beast who likes to proliferate, \\
enjoying proliferation, \\
fails to win extinguishment, \\
the supreme sanctuary. 

But\marginnote{6.1} one who gives up proliferation, \\
enjoying the state of non-proliferation, \\
wins extinguishment, \\
the supreme sanctuary.” 

%
\end{verse}

%
\section*{{\suttatitleacronym AN 6.16}{\suttatitletranslation Nakula’s Father }{\suttatitleroot Nakulapitusutta}}
\addcontentsline{toc}{section}{\tocacronym{AN 6.16} \toctranslation{Nakula’s Father } \tocroot{Nakulapitusutta}}
\markboth{Nakula’s Father }{Nakulapitusutta}
\extramarks{AN 6.16}{AN 6.16}

At\marginnote{1.1} one time the Buddha was staying in the land of the Bhaggas on Crocodile Hill, in the deer park at \textsanskrit{Bhesakaḷā}’s Wood. Now at that time the householder Nakula’s father was sick, suffering, gravely ill. Then the housewife Nakula’s mother said to him: 

“Householder,\marginnote{2.1} don’t pass away with concerns. Such concern is suffering, and it’s criticized by the Buddha. Householder, you might think: ‘When I’ve gone, the housewife Nakula’s mother won’t be able to provide for the children and keep up the household carpets.’ But you should not see it like this. I’m skilled at spinning cotton and carding wool. I’m able to provide for the children and keep up the household carpets. So householder, don’t pass away with concerns … 

Householder,\marginnote{3.1} you might think: ‘When I’ve gone, the housewife Nakula’s mother will take another husband.’ But you should not see it like this. Both you and I know that we have remained celibate while at home for the past sixteen years. So householder, don’t pass away with concerns … 

Householder,\marginnote{4.1} you might think: ‘When I’ve gone, the housewife Nakula’s mother won’t want to see the Buddha and his \textsanskrit{Saṅgha} of mendicants.’ But you should not see it like this. When you’ve gone, I’ll want to see the Buddha and his mendicant \textsanskrit{Saṅgha} even more. So householder, don’t pass away with concerns … 

Householder,\marginnote{5.1} you might think: ‘The housewife Nakula’s mother won’t fulfill ethics.’ But you should not see it like this. I am one of those white-robed disciples of the Buddha who fulfills their ethics. Whoever doubts this can go and ask the Buddha. He is staying in the land of the Bhaggas on Crocodile Hill, in the deer park at \textsanskrit{Bhesakaḷā}’s Wood. So householder, don’t pass away with concerns … 

Householder,\marginnote{6.1} you might think: ‘The housewife Nakula’s mother doesn’t have internal serenity of heart.’ But you should not see it like this. I am one of those white-robed disciples of the Buddha who has internal serenity of heart. Whoever doubts this can go and ask the Buddha. He is staying in the land of the Bhaggas on Crocodile Hill, in the deer park at \textsanskrit{Bhesakaḷā}’s Wood. So householder, don’t pass away with concerns … 

Householder,\marginnote{7.1} you might think: ‘The housewife Nakula’s mother has not gained a basis, a firm basis, and solace in this teaching and training. She has not gone beyond doubt, got rid of indecision, and gained assurance. And she’s not independent of others in the Teacher’s instructions.’ But you should not see it like this. I am one of those white-robed disciples of the Buddha who has gained a basis, a firm basis, and solace in this teaching and training. I have gone beyond doubt, got rid of indecision, and gained assurance. And I am independent of others in the Teacher’s instructions. Whoever doubts this can go and ask the Buddha. He is staying in the land of the Bhaggas on Crocodile Hill, in the deer park at \textsanskrit{Bhesakaḷā}’s Wood. So householder, don’t pass away with concerns. Such concern is suffering, and it’s criticized by the Buddha.” 

And\marginnote{8.1} then, as Nakula’s mother was giving this advice to Nakula’s father, his illness died down on the spot. And that’s how Nakula’s father recovered from that illness. Soon after recovering, leaning on a staff he went to the Buddha, bowed, and sat down to one side. The Buddha said to him: 

“You’re\marginnote{9.1} fortunate, householder, so very fortunate, to have the housewife Nakula’s mother advise and instruct you out of kindness and compassion. 

She\marginnote{9.3} is one of those white-robed disciples of the Buddha who fulfills their ethics. 

She\marginnote{9.4} is one of those white-robed disciples of the Buddha who has internal serenity of heart. 

She\marginnote{9.5} is one of those white-robed disciples of the Buddha who has gained a basis, a firm basis, and solace in this teaching and training. She has gone beyond doubt, got rid of indecision, and gained assurance. And she is independent of others in the Teacher’s instructions. 

You’re\marginnote{9.6} fortunate, householder, so very fortunate, to have the housewife Nakula’s mother advise and instruct you out of kindness and compassion.” 

%
\section*{{\suttatitleacronym AN 6.17}{\suttatitletranslation Sleep }{\suttatitleroot Soppasutta}}
\addcontentsline{toc}{section}{\tocacronym{AN 6.17} \toctranslation{Sleep } \tocroot{Soppasutta}}
\markboth{Sleep }{Soppasutta}
\extramarks{AN 6.17}{AN 6.17}

At\marginnote{1.1} one time the Buddha was staying near \textsanskrit{Sāvatthī} in Jeta’s Grove, \textsanskrit{Anāthapiṇḍika}’s monastery. 

Then\marginnote{1.2} in the late afternoon, the Buddha came out of retreat, went to the assembly hall, and sat down on the seat spread out. Venerable \textsanskrit{Sāriputta} also came out of retreat, went to the assembly hall, bowed to the Buddha and sat down to one side. Venerables \textsanskrit{Mahāmoggallāna}, \textsanskrit{Mahākassapa}, \textsanskrit{Mahākaccāna}, \textsanskrit{Mahākoṭṭhita}, \textsanskrit{Mahācunda}, \textsanskrit{Mahākappina}, Anuruddha, Revata, and Ānanda did the same. The Buddha spent most of the night sitting in meditation, then got up from his seat and entered his dwelling. And soon after the Buddha left those venerables each went to their own dwelling. 

But\marginnote{1.15} those mendicants who were junior, recently gone forth, newly come to this teaching and training slept until the sun came up, snoring. The Buddha saw them doing this, with his clairvoyance that is purified and superhuman. He went to the assembly hall, sat down on the seat spread out, and addressed the mendicants: 

“Mendicants,\marginnote{2.1} where is \textsanskrit{Sāriputta}? Where are \textsanskrit{Mahāmoggallāna}, \textsanskrit{Mahākassapa}, \textsanskrit{Mahākaccāna}, \textsanskrit{Mahākoṭṭhita}, \textsanskrit{Mahācunda}, \textsanskrit{Mahākappina}, Anuruddha, Revata, and Ānanda? Where have these senior disciples gone?” 

“Soon\marginnote{2.12} after the Buddha left those venerables each went to their own dwelling.” 

“So,\marginnote{2.13} mendicants, when the senior mendicants left, why did you sleep until the sun came up, snoring? 

What\marginnote{2.14} do you think, mendicants? Have you ever seen or heard of an anointed aristocratic king who rules his whole life, dear and beloved to the country, while indulging in the pleasures of sleeping, lying down, and drowsing as much as he likes?” 

“No,\marginnote{2.17} sir.” 

“Good,\marginnote{2.18} mendicants! I too have never seen or heard of such a thing. 

What\marginnote{3.1} do you think, mendicants? Have you ever seen or heard of an appointed official … a hereditary official … a general … a village chief … or a guild head who runs the guild his whole life, dear and beloved to the guild, while indulging in the pleasures of sleeping, lying down, and drowsing as much as he likes?” 

“No,\marginnote{3.8} sir.” 

“Good,\marginnote{3.9} mendicants! I too have never seen or heard of such a thing. 

What\marginnote{4.1} do you think, mendicants? Have you ever seen or heard of an ascetic or brahmin who indulges in the pleasures of sleeping, lying down, and drowsing as much as they like? Their sense doors are unguarded, they eat too much, they’re not dedicated to wakefulness, they’re unable to discern skillful qualities, and they don’t pursue the development of the qualities that lead to awakening in the evening and toward dawn. Yet they realize the undefiled freedom of heart and freedom by wisdom in this very life. And they live having realized it with their own insight due to the ending of defilements.” 

“No,\marginnote{4.4} sir.” 

“Good,\marginnote{4.5} mendicants! I too have never seen or heard of such a thing. 

So\marginnote{5.1} you should train like this: ‘We will guard our sense doors, eat in moderation, be dedicated to wakefulness, discern skillful qualities, and pursue the development of the qualities that lead to awakening in the evening and toward dawn.’ That’s how you should train.” 

%
\section*{{\suttatitleacronym AN 6.18}{\suttatitletranslation A Fish Dealer }{\suttatitleroot Macchabandhasutta}}
\addcontentsline{toc}{section}{\tocacronym{AN 6.18} \toctranslation{A Fish Dealer } \tocroot{Macchabandhasutta}}
\markboth{A Fish Dealer }{Macchabandhasutta}
\extramarks{AN 6.18}{AN 6.18}

At\marginnote{1.1} one time the Buddha was wandering in the land of the Kosalans together with a large \textsanskrit{Saṅgha} of mendicants. 

While\marginnote{1.2} walking along the road he saw a fish dealer in a certain spot selling fish that he had killed himself. Seeing this he left the road, sat at the root of a tree on the seat spread out, and addressed the mendicants, “Mendicants, do you see that fish dealer selling fish that he killed himself?” 

“Yes,\marginnote{1.6} sir.” 

“What\marginnote{2.1} do you think, mendicants? Have you ever seen or heard of a fish dealer selling fish that he killed himself who, by means of that work and livelihood, got to travel by elephant, horse, chariot, or vehicle, or to enjoy wealth, or to live off a large fortune?” 

“No,\marginnote{2.4} sir.” 

“Good,\marginnote{2.5} mendicants! I too have never seen or heard of such a thing. Why is that? Because when the fish are led to the slaughter he regards them with bad intentions. 

What\marginnote{3.1} do you think, mendicants? Have you ever seen or heard of a butcher of cattle selling cattle that he killed himself who, by means of that work and livelihood, got to travel by elephant, horse, chariot, or vehicle, or to enjoy wealth, or to live off a large fortune?” 

“No,\marginnote{3.4} sir.” 

“Good,\marginnote{3.5} mendicants! I too have never seen or heard of such a thing. Why is that? Because when the cattle are led to the slaughter he regards them with bad intentions. 

What\marginnote{4.1} do you think, mendicants? Have you ever seen or heard of a butcher of sheep … a butcher of pigs … a butcher of poultry … or a deer-hunter selling deer which he killed himself who, by means of that work and livelihood, got to travel by elephant, horse, chariot, or vehicle, or to enjoy wealth, or to live off a large fortune?” 

“No,\marginnote{4.7} sir.” 

“Good,\marginnote{4.8} mendicants! I too have never seen or heard of such a thing. Why is that? Because when the deer are led to the slaughter he regards them with bad intentions. 

By\marginnote{4.13} regarding even animals led to the slaughter with bad intentions you don’t get to travel by elephant, horse, chariot, or vehicle, or to enjoy wealth, or to live off a large fortune. How much worse is someone who regards human beings brought to the slaughter with bad intentions! This will be for their lasting harm and suffering. When their body breaks up, after death, they’re reborn in a place of loss, a bad place, the underworld, hell.” 

%
\section*{{\suttatitleacronym AN 6.19}{\suttatitletranslation Mindfulness of Death (1st) }{\suttatitleroot Paṭhamamaraṇassatisutta}}
\addcontentsline{toc}{section}{\tocacronym{AN 6.19} \toctranslation{Mindfulness of Death (1st) } \tocroot{Paṭhamamaraṇassatisutta}}
\markboth{Mindfulness of Death (1st) }{Paṭhamamaraṇassatisutta}
\extramarks{AN 6.19}{AN 6.19}

At\marginnote{1.1} one time the Buddha was staying at \textsanskrit{Nādika} in the brick house. There the Buddha addressed the mendicants, “Mendicants!” 

“Venerable\marginnote{1.4} sir,” they replied. The Buddha said this: 

“Mendicants,\marginnote{1.6} when mindfulness of death is developed and cultivated it’s very fruitful and beneficial. It culminates in the deathless and ends with the deathless. But do you develop mindfulness of death?” 

When\marginnote{2.1} he said this, one of the mendicants said to the Buddha, “Sir, I develop mindfulness of death.” 

“But\marginnote{2.3} mendicant, how do you develop it?” 

“In\marginnote{2.4} this case, sir, I think: ‘Oh, if I’d only live for another day and night, I’d focus on the Buddha’s instructions and I could really achieve a lot.’ That’s how I develop mindfulness of death.” 

Another\marginnote{3.1} mendicant said to the Buddha, “Sir, I too develop mindfulness of death.” 

“But\marginnote{3.3} mendicant, how do you develop it?” 

“In\marginnote{3.4} this case, sir, I think: ‘Oh, if I’d only live for another day, I’d focus on the Buddha’s instructions and I could really achieve a lot.’ That’s how I develop mindfulness of death.” 

Another\marginnote{4.1} mendicant said to the Buddha, “Sir, I too develop mindfulness of death.” 

“But\marginnote{4.3} mendicant, how do you develop it?” 

“In\marginnote{4.4} this case, sir, I think: ‘Oh, if I’d only live as long as it takes to eat a meal of almsfood, I’d focus on the Buddha’s instructions and I could really achieve a lot.’ That’s how I develop mindfulness of death.” 

Another\marginnote{5.1} mendicant said to the Buddha, “Sir, I too develop mindfulness of death.” 

“But\marginnote{5.3} mendicant, how do you develop it?” 

“In\marginnote{5.4} this case, sir, I think: ‘Oh, if I’d only live as long as it takes to chew and swallow four or five mouthfuls, I’d focus on the Buddha’s instructions and I could really achieve a lot.’ That’s how I develop mindfulness of death.” 

Another\marginnote{6.1} mendicant said to the Buddha, “Sir, I too develop mindfulness of death.” 

“But\marginnote{6.3} mendicant, how do you develop it?” 

“In\marginnote{6.4} this case, sir, I think: ‘Oh, if I’d only live as long as it takes to chew and swallow a single mouthful, I’d focus on the Buddha’s instructions and I could really achieve a lot.’ That’s how I develop mindfulness of death.” 

Another\marginnote{7.1} mendicant said to the Buddha, “Sir, I too develop mindfulness of death.” 

“But\marginnote{7.3} mendicant, how do you develop it?” 

“In\marginnote{7.4} this case, sir, I think: ‘Oh, if I’d only live as long as it takes to breathe out after breathing in, or to breathe in after breathing out, I’d focus on the Buddha’s instructions and I could really achieve a lot.’ That’s how I develop mindfulness of death.” 

When\marginnote{8.1} this was said, the Buddha said to those mendicants: 

“As\marginnote{8.2} to the mendicants who develop mindfulness of death by wishing to live for a day and night … or to live for a day … or to live as long as it takes to eat a meal of almsfood … or to live as long as it takes to chew and swallow four or five mouthfuls—these are called mendicants who live negligently. They slackly develop mindfulness of death for the ending of defilements. 

But\marginnote{13.1} as to the mendicants who develop mindfulness of death by wishing to live as long as it takes to chew and swallow a single mouthful … or to live as long as it takes to breathe out after breathing in, or to breathe in after breathing out—these are called mendicants who live diligently. They keenly develop mindfulness of death for the ending of defilements. 

So\marginnote{16.1} you should train like this: ‘We will live diligently. We will keenly develop mindfulness of death for the ending of defilements.’ That’s how you should train.” 

%
\section*{{\suttatitleacronym AN 6.20}{\suttatitletranslation Mindfulness of Death (2nd) }{\suttatitleroot Dutiyamaraṇassatisutta}}
\addcontentsline{toc}{section}{\tocacronym{AN 6.20} \toctranslation{Mindfulness of Death (2nd) } \tocroot{Dutiyamaraṇassatisutta}}
\markboth{Mindfulness of Death (2nd) }{Dutiyamaraṇassatisutta}
\extramarks{AN 6.20}{AN 6.20}

At\marginnote{1.1} one time the Buddha was staying at \textsanskrit{Nādika} in the brick house. There the Buddha addressed the mendicants: 

“Mendicants,\marginnote{1.3} when mindfulness of death is developed and cultivated it’s very fruitful and beneficial. It culminates in the deathless and ends with the deathless. And how is mindfulness of death developed and cultivated to be very fruitful and beneficial, to culminate in the deathless and end with the deathless? 

As\marginnote{2.1} day passes by and night draws close, a mendicant reflects: ‘I might die of many causes. A snake might bite me, or a scorpion or centipede might sting me. And if I died from that it would be an obstacle to me. Or I might stumble off a cliff, or get food poisoning, or suffer a disturbance of bile, phlegm, or piercing winds. And if I died from that it would stop my practice. ’ That mendicant should reflect: ‘Are there any bad, unskillful qualities that I haven’t given up, which might be an obstacle to me if I die tonight?’ 

Suppose\marginnote{3.1} that, upon checking, a mendicant knows that there are such bad, unskillful qualities. Then in order to give them up they should apply intense enthusiasm, effort, zeal, vigor, perseverance, mindfulness, and situational awareness. Suppose your clothes or head were on fire. In order to extinguish it, you’d apply intense enthusiasm, effort, zeal, vigor, perseverance, mindfulness, and situational awareness. In the same way, in order to give up those bad, unskillful qualities, that mendicant should apply intense enthusiasm … 

But\marginnote{4.1} suppose that, upon checking, a mendicant knows that there are no such bad, unskillful qualities. Then that mendicant should meditate with rapture and joy, training day and night in skillful qualities. 

Or\marginnote{5.1} else, as night passes by and day draws close, a mendicant reflects: ‘I might die of many causes. A snake might bite me, or a scorpion or centipede might sting me. And if I died from that it would stop my practice. Or I might stumble off a cliff, or get food poisoning, or suffer a disturbance of bile, phlegm, or piercing winds. And if I died from that it would stop my practice. ’ That mendicant should reflect: ‘Are there any bad, unskillful qualities that I haven’t given up, which might be an obstacle to me if I die today?’ 

Suppose\marginnote{6.1} that, upon checking, a mendicant knows that there are such bad, unskillful qualities. Then in order to give them up they should apply intense enthusiasm, effort, zeal, vigor, perseverance, mindfulness, and situational awareness. Suppose your clothes or head were on fire. In order to extinguish it, you’d apply intense enthusiasm, effort, zeal, vigor, perseverance, mindfulness, and situational awareness. In the same way, in order to give up those bad, unskillful qualities, that mendicant should apply intense enthusiasm … 

But\marginnote{7.1} suppose that, upon checking, a mendicant knows that there are no such bad, unskillful qualities. Then that mendicant should meditate with rapture and joy, training day and night in skillful qualities. 

Mindfulness\marginnote{8.1} of death, when developed and cultivated in this way, is very fruitful and beneficial. It culminates in the deathless and ends with the deathless.” 

%
\addtocontents{toc}{\let\protect\contentsline\protect\nopagecontentsline}
\chapter*{The Chapter on Unsurpassable }
\addcontentsline{toc}{chapter}{\tocchapterline{The Chapter on Unsurpassable }}
\addtocontents{toc}{\let\protect\contentsline\protect\oldcontentsline}

%
\section*{{\suttatitleacronym AN 6.21}{\suttatitletranslation At Sāma Village }{\suttatitleroot Sāmakasutta}}
\addcontentsline{toc}{section}{\tocacronym{AN 6.21} \toctranslation{At Sāma Village } \tocroot{Sāmakasutta}}
\markboth{At Sāma Village }{Sāmakasutta}
\extramarks{AN 6.21}{AN 6.21}

At\marginnote{1.1} one time the Buddha was staying among the Sakyans near the little village of \textsanskrit{Sāma}, by a lotus pond. 

Then,\marginnote{1.2} late at night, a glorious deity, lighting up the entire lotus pond, went up to the Buddha, bowed, stood to one side, and said to him, “Sir, three qualities lead to the decline of a mendicant. What three? Relishing work, talk, and sleep. These three qualities lead to the decline of a mendicant.” 

That’s\marginnote{2.5} what that deity said, and the teacher approved. Then that deity, knowing that the teacher approved, bowed, and respectfully circled the Buddha, keeping him on his right, before vanishing right there. 

Then,\marginnote{3.1} when the night had passed, the Buddha told the mendicants all that had happened, adding: 

“It’s\marginnote{3.7} unfortunate for those of you who even the deities know are declining in skillful qualities. I will teach you three more qualities that lead to decline. Listen and pay close attention, I will speak.” 

“Yes,\marginnote{4.3} sir,” they replied. The Buddha said this: 

“And\marginnote{4.5} what, mendicants, are three qualities that lead to decline? Enjoyment of company, being hard to admonish, and having bad friends. These three qualities lead to decline. 

Whether\marginnote{5.1} in the past, future, or present, all those who decline in skillful qualities do so because of these six qualities.” 

%
\section*{{\suttatitleacronym AN 6.22}{\suttatitletranslation Non-decline }{\suttatitleroot Aparihāniyasutta}}
\addcontentsline{toc}{section}{\tocacronym{AN 6.22} \toctranslation{Non-decline } \tocroot{Aparihāniyasutta}}
\markboth{Non-decline }{Aparihāniyasutta}
\extramarks{AN 6.22}{AN 6.22}

“Mendicants,\marginnote{1.1} I will teach you these six qualities that prevent decline. … And what, mendicants, are the six qualities that prevent decline? Not relishing work, talk, sleep, and company, being easy to admonish, and having good friends. These six qualities prevent decline. 

Whether\marginnote{2.1} in the past, future, or present, all those who have not declined in skillful qualities do so because of these six qualities.” 

%
\section*{{\suttatitleacronym AN 6.23}{\suttatitletranslation Dangers }{\suttatitleroot Bhayasutta}}
\addcontentsline{toc}{section}{\tocacronym{AN 6.23} \toctranslation{Dangers } \tocroot{Bhayasutta}}
\markboth{Dangers }{Bhayasutta}
\extramarks{AN 6.23}{AN 6.23}

“‘Danger’,\marginnote{1.1} mendicants, is a term for sensual pleasures. ‘Suffering’, ‘disease’, ‘boil’, ‘snare’, and ‘bog’ are terms for sensual pleasures. 

And\marginnote{2.1} why is ‘danger’ a term for sensual pleasures? Someone who is besotted by sensual greed and shackled by lustful desire is not freed from dangers in the present life or in lives to come. That is why ‘danger’ is a term for sensual pleasures. 

And\marginnote{2.3} why are ‘suffering’, ‘disease’, ‘boil’, ‘snare’, and ‘bog’ terms for sensual pleasures? Someone who is besotted by sensual greed and shackled by lustful desire is not freed from suffering, disease, boils, chains, or bogs in the present life or in lives to come. That is why these are terms for sensual pleasures. 

\begin{verse}%
Danger,\marginnote{3.1} suffering, disease, boils, \\
and snares and bogs both. \\
These describe the sensual pleasures \\
to which ordinary people are attached. 

Seeing\marginnote{4.1} the danger in grasping, \\
the origin of birth and death, \\
the unattached are freed \\
with the ending of birth and death. 

Happy,\marginnote{5.1} they’ve come to a safe place, \\
extinguished in this very life. \\
They’ve gone beyond all threats and dangers, \\
and risen above all suffering.” 

%
\end{verse}

%
\section*{{\suttatitleacronym AN 6.24}{\suttatitletranslation The Himalaya }{\suttatitleroot Himavantasutta}}
\addcontentsline{toc}{section}{\tocacronym{AN 6.24} \toctranslation{The Himalaya } \tocroot{Himavantasutta}}
\markboth{The Himalaya }{Himavantasutta}
\extramarks{AN 6.24}{AN 6.24}

“Mendicants,\marginnote{1.1} a mendicant who has six qualities could shatter Himalaya, the king of mountains, let alone this wretched ignorance! What six? It’s when a mendicant is skilled in entering immersion, skilled in remaining in immersion, skilled in emerging from immersion, skilled in gladdening the mind for immersion, skilled in the meditation subjects for immersion, and skilled in projecting the mind purified by immersion. A mendicant who has these six qualities could shatter Himalaya, the king of mountains, let alone this wretched ignorance!” 

%
\section*{{\suttatitleacronym AN 6.25}{\suttatitletranslation Topics for Recollection }{\suttatitleroot Anussatiṭṭhānasutta}}
\addcontentsline{toc}{section}{\tocacronym{AN 6.25} \toctranslation{Topics for Recollection } \tocroot{Anussatiṭṭhānasutta}}
\markboth{Topics for Recollection }{Anussatiṭṭhānasutta}
\extramarks{AN 6.25}{AN 6.25}

“Mendicants,\marginnote{1.1} there are these six topics for recollection. What six? 

Firstly,\marginnote{1.3} a noble disciple recollects the Realized One: ‘That Blessed One is perfected, a fully awakened Buddha, accomplished in knowledge and conduct, holy, knower of the world, supreme guide for those who wish to train, teacher of gods and humans, awakened, blessed.’ When a noble disciple recollects the Realized One their mind is not full of greed, hate, and delusion. At that time their mind is unswerving. They’ve left behind greed; they’re free of it and have risen above it. ‘Greed’ is a term for the five kinds of sensual stimulation. Relying on this, some sentient beings are purified in this way. 

Furthermore,\marginnote{2.1} a noble disciple recollects the teaching: ‘The teaching is well explained by the Buddha—visible in this very life, immediately effective, inviting inspection, relevant, so that sensible people can know it for themselves.’ When a noble disciple recollects the teaching their mind is not full of greed, hate, and delusion. … 

Furthermore,\marginnote{3.1} a noble disciple recollects the \textsanskrit{Saṅgha}: ‘The \textsanskrit{Saṅgha} of the Buddha’s disciples is practicing the way that’s good, direct, methodical, and proper. It consists of the four pairs, the eight individuals. This is the \textsanskrit{Saṅgha} of the Buddha’s disciples that is worthy of offerings dedicated to the gods, worthy of hospitality, worthy of a religious donation, worthy of greeting with joined palms, and is the supreme field of merit for the world.’ When a noble disciple recollects the \textsanskrit{Saṅgha} their mind is not full of greed, hate, and delusion. … 

Furthermore,\marginnote{4.1} a noble disciple recollects their own ethical precepts, which are unbroken, impeccable, spotless, and unmarred, liberating, praised by sensible people, not mistaken, and leading to immersion. When a noble disciple recollects their ethical precepts their mind is not full of greed, hate, and delusion. … 

Furthermore,\marginnote{5.1} a noble disciple recollects their own generosity: ‘I’m so fortunate, so very fortunate! Among people full of the stain of stinginess I live at home rid of the stain of stinginess, freely generous, open-handed, loving to let go, committed to charity, loving to give and to share.’ When a noble disciple recollects their generosity their mind is not full of greed, hate, and delusion. … 

Furthermore,\marginnote{6.1} a noble disciple recollects the deities: ‘There are the Gods of the Four Great Kings, the Gods of the Thirty-Three, the Gods of Yama, the Joyful Gods, the Gods Who Love to Create, the Gods Who Control the Creations of Others, the Gods of \textsanskrit{Brahmā}’s Host, and gods even higher than these. When those deities passed away from here, they were reborn there because of their faith, ethics, learning, generosity, and wisdom. I, too, have the same kind of faith, ethics, learning, generosity, and wisdom.’ 

When\marginnote{7.1} a noble disciple recollects the faith, ethics, learning, generosity, and wisdom of both themselves and the deities their mind is not full of greed, hate, and delusion. At that time their mind is unswerving. They’ve left behind greed; they’re free of it and have risen above it. ‘Greed’ is a term for the five kinds of sensual stimulation. Relying on this, some sentient beings are purified in this way. 

These\marginnote{8.1} are the six topics for recollection.” 

%
\section*{{\suttatitleacronym AN 6.26}{\suttatitletranslation With Mahākaccāna }{\suttatitleroot Mahākaccānasutta}}
\addcontentsline{toc}{section}{\tocacronym{AN 6.26} \toctranslation{With Mahākaccāna } \tocroot{Mahākaccānasutta}}
\markboth{With Mahākaccāna }{Mahākaccānasutta}
\extramarks{AN 6.26}{AN 6.26}

There\marginnote{1.1} \textsanskrit{Mahākaccāna} addressed the mendicants: “Reverends, mendicants!” 

“Reverend,”\marginnote{1.3} they replied. Venerable \textsanskrit{Mahākaccāna} said this: 

“It’s\marginnote{1.5} incredible, reverends, it’s amazing! How this Blessed One who knows and sees, the perfected one, the fully awakened Buddha, has found an opening in a confined space; that is, the six topics for recollection. They are in order to purify sentient beings, to get past sorrow and crying, to make an end of pain and sadness, to end the cycle of suffering, and to realize extinguishment. What six? 

Firstly,\marginnote{2.2} a noble disciple recollects the Realized One: ‘That Blessed One is perfected, a fully awakened Buddha, accomplished in knowledge and conduct, holy, knower of the world, supreme guide for those who wish to train, teacher of gods and humans, awakened, blessed.’ When a noble disciple recollects the Realized One their mind is not full of greed, hate, and delusion. At that time their mind is unswerving. They’ve left behind greed; they’re free of it and have risen above it. ‘Greed’ is a term for the five kinds of sensual stimulation. That noble disciple meditates with a heart just like space, abundant, expansive, limitless, free of enmity and ill will. Relying on this, some sentient beings become pure in this way. 

Furthermore,\marginnote{3.1} a noble disciple recollects the teaching: ‘The teaching is well explained by the Buddha—visible in this very life, immediately effective, inviting inspection, relevant, so that sensible people can know it for themselves.’ When a noble disciple recollects the teaching their mind is not full of greed, hate, and delusion. … 

Furthermore,\marginnote{4.1} a noble disciple recollects the \textsanskrit{Saṅgha}: ‘The \textsanskrit{Saṅgha} of the Buddha’s disciples is practicing the way that’s good, direct, methodical, and proper. It consists of the four pairs, the eight individuals. This is the \textsanskrit{Saṅgha} of the Buddha’s disciples that is worthy of offerings dedicated to the gods, worthy of hospitality, worthy of a religious donation, worthy of greeting with joined palms, and is the supreme field of merit for the world.’ When a noble disciple recollects the \textsanskrit{Saṅgha} their mind is not full of greed, hate, and delusion. … 

Furthermore,\marginnote{5.1} a noble disciple recollects their own ethical precepts, which are unbroken, impeccable, spotless, and unmarred, liberating, praised by sensible people, not mistaken, and leading to immersion. When a noble disciple recollects their ethical precepts their mind is not full of greed, hate, and delusion. … 

Furthermore,\marginnote{6.1} a noble disciple recollects their own generosity: ‘I’m so fortunate, so very fortunate! Among people full of the stain of stinginess I live at home rid of the stain of stinginess, freely generous, open-handed, loving to let go, committed to charity, loving to give and to share.’ When a noble disciple recollects their own generosity their mind is not full of greed, hate, and delusion. … 

Furthermore,\marginnote{7.1} a noble disciple recollects the deities: ‘There are the Gods of the Four Great Kings, the Gods of the Thirty-Three, the Gods of Yama, the Joyful Gods, the Gods Who Love to Create, the Gods Who Control the Creations of Others, the Gods of \textsanskrit{Brahmā}’s Host, and gods even higher than these. When those deities passed away from here, they were reborn there because of their faith, ethics, learning, generosity, and wisdom. I, too, have the same kind of faith, ethics, learning, generosity, and wisdom.’ When a noble disciple recollects the faith, ethics, learning, generosity, and wisdom of both themselves and the deities their mind is not full of greed, hate, and delusion. At that time their mind is unswerving. They’ve left behind greed; they’re free of it and have risen above it. ‘Greed’ is a term for the five kinds of sensual stimulation. That noble disciple meditates with a heart just like space, abundant, expansive, limitless, free of enmity and ill will. Relying on this, some sentient beings become pure in this way. 

It’s\marginnote{8.1} incredible, reverends, it’s amazing! How this Blessed One who knows and sees, the perfected one, the fully awakened Buddha, has found an opening in a confined space; that is, the six topics for recollection. They are in order to purify sentient beings, to get past sorrow and crying, to make an end of pain and sadness, to end the cycle of suffering, and to realize extinguishment.” 

%
\section*{{\suttatitleacronym AN 6.27}{\suttatitletranslation Proper Occasions (1st) }{\suttatitleroot Paṭhamasamayasutta}}
\addcontentsline{toc}{section}{\tocacronym{AN 6.27} \toctranslation{Proper Occasions (1st) } \tocroot{Paṭhamasamayasutta}}
\markboth{Proper Occasions (1st) }{Paṭhamasamayasutta}
\extramarks{AN 6.27}{AN 6.27}

Then\marginnote{1.1} a mendicant went up to the Buddha, bowed, sat down to one side, and said to him: 

“Sir,\marginnote{1.2} how many occasions are there for going to see an esteemed mendicant?” 

“Mendicant,\marginnote{1.3} there are six occasions for going to see an esteemed mendicant. What six? 

Firstly,\marginnote{2.2} there’s a time when a mendicant’s heart is overcome and mired in sensual desire, and they don’t truly understand the escape from sensual desire that has arisen. On that occasion they should go to an esteemed mendicant and say: ‘My heart is overcome and mired in sensual desire, and I don’t truly understand the escape from sensual desire that has arisen. Venerable, please teach me how to give up sensual desire.’ Then that esteemed mendicant teaches them how to give up sensual desire. This is the first occasion for going to see an esteemed mendicant. 

Furthermore,\marginnote{3.1} there’s a time when a mendicant’s heart is overcome and mired in ill will … This is the second occasion for going to see an esteemed mendicant. 

Furthermore,\marginnote{4.1} there’s a time when a mendicant’s heart is overcome and mired in dullness and drowsiness … This is the third occasion for going to see an esteemed mendicant. 

Furthermore,\marginnote{5.1} there’s a time when a mendicant’s heart is overcome and mired in restlessness and remorse … This is the fourth occasion for going to see an esteemed mendicant. 

Furthermore,\marginnote{6.1} there’s a time when a mendicant’s heart is overcome and mired in doubt … This is the fifth occasion for going to see an esteemed mendicant. 

Furthermore,\marginnote{7.1} there’s a time when a mendicant doesn’t understand what kind of meditation they need to focus on in order to end the defilements in the present life. On that occasion they should go to an esteemed mendicant and say: ‘I don’t understand what kind of meditation to focus on in order to end the defilements in the present life. Venerable, please teach me how to end the defilements.’ Then that esteemed mendicant teaches them how to end the defilements. This is the sixth occasion for going to see an esteemed mendicant. 

These\marginnote{8.1} are the six occasions for going to see an esteemed mendicant.” 

%
\section*{{\suttatitleacronym AN 6.28}{\suttatitletranslation Proper Occasions (2nd) }{\suttatitleroot Dutiyasamayasutta}}
\addcontentsline{toc}{section}{\tocacronym{AN 6.28} \toctranslation{Proper Occasions (2nd) } \tocroot{Dutiyasamayasutta}}
\markboth{Proper Occasions (2nd) }{Dutiyasamayasutta}
\extramarks{AN 6.28}{AN 6.28}

At\marginnote{1.1} one time several senior mendicants were staying near Benares, in the deer park at Isipatana. Then after the meal, on their return from almsround, this discussion came up among them while sitting together in the pavilion. 

“Reverends,\marginnote{1.3} how many occasions are there for going to see an esteemed mendicant?” 

When\marginnote{2.1} this was said, one of the mendicants said to the senior mendicants: 

“Reverends,\marginnote{2.2} there’s a time after an esteemed mendicant’s meal when they return from almsround. Having washed their feet they sit down cross-legged, with their body straight, and establish mindfulness right there. That is the proper occasion for going to see an esteemed mendicant.” 

When\marginnote{3.1} this was said, one of the mendicants said to that mendicant: 

“Reverend,\marginnote{3.2} that’s not the proper occasion for going to see an esteemed mendicant. For at that time the fatigue from walking and from eating has not faded away. There’s a time late in the afternoon when an esteemed mendicant comes out of retreat. They sit in the shade of their porch cross-legged, with their body straight, and establish mindfulness right there. That is the proper occasion for going to see an esteemed mendicant.” 

When\marginnote{4.1} this was said, one of the mendicants said to that mendicant: 

“Reverend,\marginnote{4.2} that’s not the proper occasion for going to see an esteemed mendicant. For at that time they are still practicing the same meditation subject as a foundation of immersion that they focused on during the day. There’s a time when an esteemed mendicant has risen at the crack of dawn. They sit down cross-legged, with their body straight, and establish mindfulness right there. That is the proper occasion for going to see an esteemed mendicant.” 

When\marginnote{5.1} this was said, one of the mendicants said to that mendicant: 

“Reverend,\marginnote{5.2} that’s not the proper occasion for going to see an esteemed mendicant. For at that time their body is full of vitality and they find it easy to focus on the instructions of the Buddhas.” 

When\marginnote{6.1} this was said, Venerable \textsanskrit{Mahākaccāna} said to those senior mendicants: 

“Reverends,\marginnote{6.2} I have heard and learned this in the presence of the Buddha: ‘Mendicants, there are six occasions for going to see an esteemed mendicant. 

What\marginnote{7.1} six? Firstly, there’s a time when a mendicant’s heart is overcome and mired in sensual desire, and they don’t truly understand the escape from sensual desire that has arisen. On that occasion they should go to an esteemed mendicant and say: 

“My\marginnote{7.3} heart is overcome and mired in sensual desire, and I don’t truly understand the escape from sensual desire that has arisen. Venerable, please teach me how to give up sensual desire.” Then that esteemed mendicant teaches them how to give up sensual desire. This is the first occasion for going to see an esteemed mendicant. 

Furthermore,\marginnote{8.1} there’s a time when a mendicant’s heart is overcome and mired in ill will … dullness and drowsiness … restlessness and remorse … doubt … 

Furthermore,\marginnote{12.1} there’s a time when a mendicant doesn’t understand what kind of meditation they need to focus on in order to end the defilements in the present life. On that occasion they should go to an esteemed mendicant and say, “I don’t understand what kind of meditation to focus on in order to end the defilements in the present life. Venerable, please teach me how to end the defilements.” Then that esteemed mendicant teaches them how to end the defilements. This is the sixth occasion for going to see an esteemed mendicant.’ 

Reverends,\marginnote{13.1} I have heard and learned this in the presence of the Buddha: ‘These are the six occasions for going to see an esteemed mendicant.’” 

%
\section*{{\suttatitleacronym AN 6.29}{\suttatitletranslation With Udāyī }{\suttatitleroot Udāyīsutta}}
\addcontentsline{toc}{section}{\tocacronym{AN 6.29} \toctranslation{With Udāyī } \tocroot{Udāyīsutta}}
\markboth{With Udāyī }{Udāyīsutta}
\extramarks{AN 6.29}{AN 6.29}

Then\marginnote{1.1} the Buddha said to \textsanskrit{Udāyī}, “\textsanskrit{Udāyī}, how many topics for recollection are there?” 

When\marginnote{1.3} he said this, \textsanskrit{Udāyī} kept silent. 

And\marginnote{1.4} a second time … and a third time, the Buddha said to him, “\textsanskrit{Udāyī}, how many topics for recollection are there?” 

And\marginnote{1.9} a second time and a third time \textsanskrit{Udāyī} kept silent. 

Then\marginnote{2.1} Venerable Ānanda said to Venerable \textsanskrit{Udāyī}, “Reverend \textsanskrit{Udāyī}, the teacher is addressing you.” 

“Reverend\marginnote{2.3} Ānanda, I hear the Buddha. 

It’s\marginnote{2.4} when a mendicant recollects many kinds of past lives. That is: one, two, three, four, five, ten, twenty, thirty, forty, fifty, a hundred, a thousand, a hundred thousand rebirths; many eons of the world contracting, many eons of the world expanding, many eons of the world contracting and expanding. They remember: ‘There, I was named this, my clan was that, I looked like this, and that was my food. This was how I felt pleasure and pain, and that was how my life ended. When I passed away from that place I was reborn somewhere else. There, too, I was named this, my clan was that, I looked like this, and that was my food. This was how I felt pleasure and pain, and that was how my life ended. When I passed away from that place I was reborn here.’ And so they recollect their many kinds of past lives, with features and details. This is a topic for recollection.” 

Then\marginnote{3.1} the Buddha said to Venerable Ānanda: “Ānanda, I know that this silly man \textsanskrit{Udāyī} is not committed to the higher mind. Ānanda, how many topics for recollection are there?” 

“Sir,\marginnote{4.1} there are five topics for recollection. What five? 

Firstly,\marginnote{4.3} a mendicant, quite secluded from sensual pleasures, secluded from unskillful qualities, enters and remains in the first absorption … second absorption … third absorption. When this topic of recollection is developed and cultivated in this way it leads to blissful meditation in this very life. 

Furthermore,\marginnote{5.1} a mendicant focuses on the perception of light, concentrating on the perception of day regardless of whether it’s night or day. And so, with an open and unenveloped heart, they develop a mind that’s full of radiance. When this topic of recollection is developed and cultivated in this way it leads to knowledge and vision. 

Furthermore,\marginnote{6.1} a mendicant examines their own body up from the soles of the feet and down from the tips of the hairs, wrapped in skin and full of many kinds of filth. ‘In this body there is head hair, body hair, nails, teeth, skin, flesh, sinews, bones, bone marrow, kidneys, heart, liver, diaphragm, spleen, lungs, intestines, mesentery, undigested food, feces, bile, phlegm, pus, blood, sweat, fat, tears, grease, saliva, snot, synovial fluid, urine.’ When this topic of recollection is developed and cultivated in this way it leads to giving up sensual desire. 

Furthermore,\marginnote{7.1} suppose a mendicant were to see a corpse thrown in a charnel ground. And it had been dead for one, two, or three days, bloated, livid, and festering. They’d compare it with their own body: ‘This body is also of that same nature, that same kind, and cannot go beyond that.’ 

Or\marginnote{8.1} suppose they were to see a corpse thrown in a charnel ground being devoured by crows, hawks, vultures, herons, dogs, tigers, leopards, jackals, and many kinds of little creatures. They’d compare it with their own body: ‘This body is also of that same nature, that same kind, and cannot go beyond that.’ 

Furthermore,\marginnote{9.1} suppose they were to see a corpse thrown in a charnel ground, a skeleton with flesh and blood, held together by sinews … A skeleton without flesh but smeared with blood, and held together by sinews … A skeleton rid of flesh and blood, held together by sinews … Bones rid of sinews scattered in every direction. Here a hand-bone, there a foot-bone, here a shin-bone, there a thigh-bone, here a hip-bone, there a rib-bone, here a back-bone, there an arm-bone, here a neck-bone, there a jaw-bone, here a tooth, there the skull … White bones, the color of shells … Decrepit bones, heaped in a pile … Bones rotted and crumbled to powder. They’d compare it with their own body: ‘This body is also of that same nature, that same kind, and cannot go beyond that.’ When this topic of recollection is developed and cultivated in this way it leads to uprooting the conceit ‘I am’. 

Furthermore,\marginnote{10.1} a mendicant, giving up pleasure and pain, and ending former happiness and sadness, enters and remains in the fourth absorption, without pleasure or pain, with pure equanimity and mindfulness. When this topic of recollection is developed and cultivated in this way it leads to the penetration of many elements. These are the five topics for recollection.” 

“Good,\marginnote{11.1} good, Ānanda. Well then, Ānanda, you should also remember this sixth topic for recollection. In this case, a mendicant goes out mindfully, returns mindfully, stands mindfully, sits mindfully, lies down mindfully, and applies themselves to work mindfully. When this topic of recollection is developed and cultivated in this way it leads to mindfulness and situational awareness.” 

%
\section*{{\suttatitleacronym AN 6.30}{\suttatitletranslation Unsurpassable }{\suttatitleroot Anuttariyasutta}}
\addcontentsline{toc}{section}{\tocacronym{AN 6.30} \toctranslation{Unsurpassable } \tocroot{Anuttariyasutta}}
\markboth{Unsurpassable }{Anuttariyasutta}
\extramarks{AN 6.30}{AN 6.30}

“Mendicants,\marginnote{1.1} these six things are unsurpassable. What six? The unsurpassable seeing, listening, acquisition, training, service, and recollection. 

And\marginnote{2.1} what is the unsurpassable seeing? Some people go to see an elephant-treasure, a horse-treasure, a jewel-treasure, or a diverse spectrum of sights; or ascetics and brahmins of wrong view and wrong practice. There is such a seeing, I don’t deny it. That seeing is low, crude, ordinary, ignoble, and pointless. It doesn’t lead to disillusionment, dispassion, cessation, peace, insight, awakening, and extinguishment. The unsurpassable seeing is when someone with settled faith and love, sure and devoted, goes to see a Realized One or their disciple. This is in order to purify sentient beings, to get past sorrow and crying, to make an end of pain and sadness, to end the cycle of suffering, and to realize extinguishment. This is called the unsurpassable seeing. Such is the unsurpassable seeing. 

But\marginnote{3.1} what of the unsurpassable hearing? Some people go to hear the sound of drums, arched harps, singing, or a diverse spectrum of sounds; or ascetics and brahmins of wrong view and wrong practice. There is such a hearing, I don’t deny it. That hearing … doesn’t lead to extinguishment. The unsurpassable hearing is when someone with settled faith and love, sure and devoted, goes to hear the teaching of a Realized One or one of his disciples. … This is called the unsurpassable hearing. Such is the unsurpassable seeing and hearing. 

But\marginnote{4.1} what of the unsurpassable acquisition? Some people acquire a child, a wife, wealth, or a diverse spectrum of things; or they acquire faith in an ascetic or brahmin of wrong view and wrong practice. There is such an acquisition, I don’t deny it. That acquisition … doesn’t lead to extinguishment. The unsurpassable acquisition is when someone with settled faith and love, sure and devoted, acquires faith in a Realized One or their disciple. … This is called the unsurpassable acquisition. Such is the unsurpassable seeing, hearing, and acquisition. 

But\marginnote{5.1} what of the unsurpassable training? Some people train in elephant riding, horse riding, chariot driving, archery, swordsmanship, or a diverse spectrum of things; or they train under an ascetic or brahmin of wrong view and wrong practice. There is such a training, I don’t deny it. That training … doesn’t lead to extinguishment. The unsurpassable training is when someone with settled faith and love, sure and devoted, trains in the higher ethics, the higher mind, and the higher wisdom in the teaching and training proclaimed by a Realized One. … This is called the unsurpassable training. Such is the unsurpassable seeing, hearing, acquisition, and training. 

But\marginnote{6.1} what of the unsurpassable service? Some people serve an aristocrat, a brahmin, a householder, or a diverse spectrum of people; or they serve ascetics and brahmins of wrong view and wrong practice. There is such service, I don’t deny it. That service … doesn’t lead to extinguishment. The unsurpassable service is when someone with settled faith and love, sure and devoted, serves a Realized One or their disciple. … This is called the unsurpassable service. Such is the unsurpassable seeing, listening, acquisition, training, and service. 

But\marginnote{7.1} what of the unsurpassable recollection? Some people recollect a child, a wife, wealth, or a diverse spectrum of things; or they recollect an ascetic or brahmin of wrong view and wrong practice. There is such recollection, I don’t deny it. That recollection is low, crude, ordinary, ignoble, and pointless. It doesn’t lead to disillusionment, dispassion, cessation, peace, insight, awakening, and extinguishment. The unsurpassable recollection is when someone with settled faith and love, sure and devoted, recollects a Realized One or their disciple. … This is called the unsurpassable recollection. 

These\marginnote{8.1} are the six unsurpassable things. 

\begin{verse}%
They’ve\marginnote{9.1} gained the unsurpassed seeing, \\
the unsurpassed hearing, \\
and the unsurpassable acquisition. \\
They enjoy the unsurpassable training 

and\marginnote{10.1} serve with care. \\
Then they develop recollection \\
connected with seclusion, \\
which is safe, and leads to the deathless. 

They\marginnote{11.1} rejoice in diligence, \\
alert and ethically restrained. \\
And in time they arrive \\
at the place where suffering ceases.” 

%
\end{verse}

%
\addtocontents{toc}{\let\protect\contentsline\protect\nopagecontentsline}
\pannasa{The Chapter on Deities }
\addcontentsline{toc}{pannasa}{The Chapter on Deities }
\markboth{}{}
\addtocontents{toc}{\let\protect\contentsline\protect\oldcontentsline}

%
\section*{{\suttatitleacronym AN 6.31}{\suttatitletranslation A Trainee }{\suttatitleroot Sekhasutta}}
\addcontentsline{toc}{section}{\tocacronym{AN 6.31} \toctranslation{A Trainee } \tocroot{Sekhasutta}}
\markboth{A Trainee }{Sekhasutta}
\extramarks{AN 6.31}{AN 6.31}

“These\marginnote{1.1} six things lead to the decline of a mendicant trainee. What six? They relish work, talk, sleep, and company. They don’t guard the sense doors, and they eat too much. These six things lead to the decline of a mendicant trainee. 

These\marginnote{2.1} six things don’t lead to the decline of a mendicant trainee. What six? They don’t relish work, talk, sleep, and company. They guard the sense doors, and they don’t eat too much. These six things don’t lead to the decline of a mendicant trainee.” 

%
\section*{{\suttatitleacronym AN 6.32}{\suttatitletranslation Non-decline (1st) }{\suttatitleroot Paṭhamaaparihānasutta}}
\addcontentsline{toc}{section}{\tocacronym{AN 6.32} \toctranslation{Non-decline (1st) } \tocroot{Paṭhamaaparihānasutta}}
\markboth{Non-decline (1st) }{Paṭhamaaparihānasutta}
\extramarks{AN 6.32}{AN 6.32}

Then,\marginnote{1.1} late at night, a glorious deity, lighting up the entire Jeta’s Grove, went up to the Buddha, bowed, stood to one side, and said to him: 

“Sir,\marginnote{2.1} these six things don’t lead to the decline of a mendicant. What six? Respect for the Teacher, for the teaching, for the \textsanskrit{Saṅgha}, for the training, for diligence, and for hospitality. These six things don’t lead to the decline of a mendicant.” 

That’s\marginnote{2.5} what that deity said, and the teacher approved. Then that deity, knowing that the teacher approved, bowed, and respectfully circled the Buddha, keeping him on his right, before vanishing right there. 

Then,\marginnote{3.1} when the night had passed, the Buddha told the mendicants all that had happened, adding: 

\begin{verse}%
“Respect\marginnote{4.1} for the Teacher and the teaching, \\
and keen respect for the \textsanskrit{Saṅgha}; \\
a mendicant who respects diligence \\
and hospitality \\
can’t decline, \\
and has drawn near to extinguishment.” 

%
\end{verse}

%
\section*{{\suttatitleacronym AN 6.33}{\suttatitletranslation Non-decline (2nd) }{\suttatitleroot Dutiyaaparihānasutta}}
\addcontentsline{toc}{section}{\tocacronym{AN 6.33} \toctranslation{Non-decline (2nd) } \tocroot{Dutiyaaparihānasutta}}
\markboth{Non-decline (2nd) }{Dutiyaaparihānasutta}
\extramarks{AN 6.33}{AN 6.33}

“Tonight,\marginnote{1.1} a glorious deity, lighting up the entire Jeta’s Grove, came to me, bowed, stood to one side, and said to me: ‘Sir, these six things don’t lead to the decline of a mendicant. What six? Respect for the Teacher, for the teaching, for the \textsanskrit{Saṅgha}, for the training, for conscience, and for prudence. These six things don’t lead to the decline of a mendicant.’ 

That\marginnote{1.6} is what that deity said. Then he bowed and respectfully circled me, keeping me on his right side, before vanishing right there. 

\begin{verse}%
Respect\marginnote{2.1} for the Teacher and the teaching, \\
and keen respect for the \textsanskrit{Saṅgha}; \\
having both conscience and prudence, \\
reverential and respectful, \\
such a one can’t decline, \\
and has drawn near to extinguishment.” 

%
\end{verse}

%
\section*{{\suttatitleacronym AN 6.34}{\suttatitletranslation With Mahāmoggallāna }{\suttatitleroot Mahāmoggallānasutta}}
\addcontentsline{toc}{section}{\tocacronym{AN 6.34} \toctranslation{With Mahāmoggallāna } \tocroot{Mahāmoggallānasutta}}
\markboth{With Mahāmoggallāna }{Mahāmoggallānasutta}
\extramarks{AN 6.34}{AN 6.34}

At\marginnote{1.1} one time the Buddha was staying near \textsanskrit{Sāvatthī} in Jeta’s Grove, \textsanskrit{Anāthapiṇḍika}’s monastery. 

Then\marginnote{1.2} as Venerable \textsanskrit{Mahāmoggallāna} was in private retreat this thought came to his mind, “Which gods know that they are stream-enterers, not liable to be reborn in the underworld, bound for awakening?” 

Now,\marginnote{1.5} at that time a monk called Tissa had recently passed away and been reborn in a \textsanskrit{Brahmā} realm. There they knew that Tissa the \textsanskrit{Brahmā} was very mighty and powerful. 

And\marginnote{2.1} then Venerable \textsanskrit{Mahāmoggallāna}, as easily as a strong person would extend or contract their arm, vanished from Jeta’s Grove and reappeared in that \textsanskrit{Brahmā} realm. 

Tissa\marginnote{2.2} saw \textsanskrit{Moggallāna} coming off in the distance, and said to him, “Come, my good \textsanskrit{Moggallāna}! Welcome, my good \textsanskrit{Moggallāna}! It’s been a long time since you took the opportunity to come here. Sit, my good \textsanskrit{Moggallāna}, this seat is for you.” \textsanskrit{Moggallāna} sat down on the seat spread out. Then Tissa bowed to \textsanskrit{Moggallāna} and sat to one side. 

\textsanskrit{Moggallāna}\marginnote{2.9} said to him, “Tissa, which gods know that they are stream-enterers, not liable to be reborn in the underworld, bound for awakening?” 

“The\marginnote{3.2} gods of the Four Great Kings know this.” 

“But\marginnote{4.1} do all of them know this?” 

“No,\marginnote{4.2} my good \textsanskrit{Moggallāna}, not all of them. Those who lack experiential confidence in the Buddha, the teaching, and the \textsanskrit{Saṅgha}, and lack the ethics loved by the noble ones, do not know that they are stream-enterers. But those who have experiential confidence in the Buddha, the teaching, and the \textsanskrit{Saṅgha}, and have the ethics loved by the noble ones, do know that they are stream-enterers.” 

“But\marginnote{5.1} Tissa, is it only the gods of the Four Great Kings who know that they are stream-enterers, or do the gods of the Thirty-Three … the Gods of Yama … the Joyful Gods … the Gods Who Love to Create … and the Gods Who Control the Creations of Others know that they are stream-enterers, not liable to be reborn in the underworld, bound for awakening?” 

“The\marginnote{5.6} gods of these various classes know this.” 

“But\marginnote{6.1} do all of them know this?” 

“No,\marginnote{6.2} my good \textsanskrit{Moggallāna}, not all of them. Those who lack experiential confidence in the Buddha, the teaching, and the \textsanskrit{Saṅgha}, and lack the ethics loved by the noble ones, do not know that they are stream-enterers. But those who have experiential confidence in the Buddha, the teaching, and the \textsanskrit{Saṅgha}, and have the ethics loved by the noble ones, do know that they are stream-enterers.” 

\textsanskrit{Moggallāna}\marginnote{7.1} approved and agreed with what Tissa the \textsanskrit{Brahmā} said. Then, as easily as a strong person would extend or contract their arm, he vanished from that \textsanskrit{Brahmā} realm and reappeared in Jeta’s Grove. 

%
\section*{{\suttatitleacronym AN 6.35}{\suttatitletranslation Things That Play a Part in Realization }{\suttatitleroot Vijjābhāgiyasutta}}
\addcontentsline{toc}{section}{\tocacronym{AN 6.35} \toctranslation{Things That Play a Part in Realization } \tocroot{Vijjābhāgiyasutta}}
\markboth{Things That Play a Part in Realization }{Vijjābhāgiyasutta}
\extramarks{AN 6.35}{AN 6.35}

“These\marginnote{1.1} six things play a part in realization. What six? The perception of impermanence, the perception of suffering in impermanence, the perception of not-self in suffering, the perception of giving up, the perception of fading away, and the perception of cessation. These are the six things that play a part in realization.” 

%
\section*{{\suttatitleacronym AN 6.36}{\suttatitletranslation Roots of Arguments }{\suttatitleroot Vivādamūlasutta}}
\addcontentsline{toc}{section}{\tocacronym{AN 6.36} \toctranslation{Roots of Arguments } \tocroot{Vivādamūlasutta}}
\markboth{Roots of Arguments }{Vivādamūlasutta}
\extramarks{AN 6.36}{AN 6.36}

“Mendicants,\marginnote{1.1} there are these six roots of arguments. What six? Firstly, a mendicant is irritable and hostile. Such a mendicant lacks respect and reverence for the Teacher, the teaching, and the \textsanskrit{Saṅgha}, and they don’t fulfill the training. They create a dispute in the \textsanskrit{Saṅgha}, which is for the hurt and unhappiness of the people, for the harm, hurt, and suffering of gods and humans. If you see such a root of arguments in yourselves or others, you should try to give up this bad thing. If you don’t see it, you should practice so that it doesn’t come up in the future. That’s how to give up this bad root of arguments, so it doesn’t come up in the future. 

Furthermore,\marginnote{2.1} a mendicant is offensive and contemptuous … They’re jealous and stingy … devious and deceitful … with wicked desires and wrong view … They’re attached to their own views, holding them tight, and refusing to let go. Such a mendicant lacks respect and reverence for the Teacher, the teaching, and the \textsanskrit{Saṅgha}, and they don’t fulfill the training. They create a dispute in the \textsanskrit{Saṅgha}, which is for the hurt and unhappiness of the people, for the harm, hurt, and suffering of gods and humans. If you see such a root of arguments in yourselves or others, you should try to give up this bad thing. If you don’t see it, you should practice so that it doesn’t come up in the future. That’s how to give up this bad root of arguments, so it doesn’t come up in the future. These are the six roots of arguments.” 

%
\section*{{\suttatitleacronym AN 6.37}{\suttatitletranslation A Gift With Six Factors }{\suttatitleroot Chaḷaṅgadānasutta}}
\addcontentsline{toc}{section}{\tocacronym{AN 6.37} \toctranslation{A Gift With Six Factors } \tocroot{Chaḷaṅgadānasutta}}
\markboth{A Gift With Six Factors }{Chaḷaṅgadānasutta}
\extramarks{AN 6.37}{AN 6.37}

At\marginnote{1.1} one time the Buddha was staying near \textsanskrit{Sāvatthī} in Jeta’s Grove, \textsanskrit{Anāthapiṇḍika}’s monastery. 

Now\marginnote{1.2} at that time \textsanskrit{Veḷukaṇṭakī}, Nanda’s mother, was preparing a religious donation with six factors for the mendicant \textsanskrit{Saṅgha} headed by \textsanskrit{Sāriputta} and \textsanskrit{Moggallāna}. The Buddha saw her doing this, with his clairvoyance that is purified and superhuman, and he addressed the mendicants: 

“This\marginnote{1.5} \textsanskrit{Veḷukaṇṭakī}, Nanda’s mother, is preparing a religious donation with six factors for the mendicant \textsanskrit{Saṅgha} headed by \textsanskrit{Sāriputta} and \textsanskrit{Moggallāna}. 

And\marginnote{2.1} how does a religious donation have six factors? Three factors apply to the donor and three to the recipients. 

What\marginnote{2.3} three factors apply to the donor? It’s when a donor is in a good mood before giving, while giving they feel confident, and after giving they’re uplifted. These three factors apply to the donor. 

What\marginnote{3.1} three factors apply to the recipients? It’s when the recipients are free of greed, hate, and delusion, or practicing to be free of them. These three factors apply to the recipients. 

Thus\marginnote{3.4} three factors apply to the donor and three to the recipients. That’s how a religious donation has six factors. 

It’s\marginnote{4.1} not easy to grasp the merit of such a six-factored donation by saying that this is the extent of their overflowing merit, overflowing goodness that nurtures happiness and is conducive to heaven, ripening in happiness and leading to heaven. And it leads to what is likable, desirable, agreeable, to welfare and happiness. It’s simply reckoned as an incalculable, immeasurable, great mass of merit. 

It’s\marginnote{5.1} like trying to grasp how much water is in the ocean. It’s not easy to say how many gallons, how many hundreds, thousands, hundreds of thousands of gallons there are. It’s simply reckoned as an incalculable, immeasurable, great mass of water. In the same way, it’s not easy to grasp the merit of such a six-factored donation … 

\begin{verse}%
A\marginnote{6.1} good mood before giving, \\
confidence while giving, \\
feeling uplifted after giving: \\
this is the perfect sacrifice. 

Free\marginnote{7.1} of greed, free of hate, \\
free of delusion, undefiled; \\
this is the field for the perfect sacrifice, \\
the disciplined spiritual practitioners. 

After\marginnote{8.1} rinsing, \\
you give with your own hands. \\
This sacrifice is very fruitful \\
for both yourself and others. 

When\marginnote{9.1} an intelligent, faithful person, \\
sacrifices like this, with a mind of letting go, \\
that astute one is reborn \\
in a happy, pleasing world.” 

%
\end{verse}

%
\section*{{\suttatitleacronym AN 6.38}{\suttatitletranslation One’s Own Volition }{\suttatitleroot Attakārīsutta}}
\addcontentsline{toc}{section}{\tocacronym{AN 6.38} \toctranslation{One’s Own Volition } \tocroot{Attakārīsutta}}
\markboth{One’s Own Volition }{Attakārīsutta}
\extramarks{AN 6.38}{AN 6.38}

Then\marginnote{1.1} a certain brahmin went up to the Buddha, and exchanged greetings with him. When the greetings and polite conversation were over, he sat down to one side and said to the Buddha: 

“Master\marginnote{1.3} Gotama, this is my doctrine and view: One does not act of one’s own volition, nor does one act of another’s volition.” 

“Well,\marginnote{1.5} brahmin, I’ve never seen or heard of anyone holding such a doctrine or view. How on earth can someone who comes and goes on his own say that one does not act of one’s own volition, nor does one act of another’s volition? 

What\marginnote{2.1} do you think, brahmin, is there an element of initiative?” 

“Yes,\marginnote{2.2} sir.” 

“Since\marginnote{2.3} this is so, do we find sentient beings who initiate activity?” 

“Yes,\marginnote{2.4} sir.” 

“Since\marginnote{2.5} there is an element of initiative, and sentient beings who initiate activity are found, sentient beings act of their own volition or that of another. 

What\marginnote{3.1} do you think, brahmin, is there an element of persistence … exertion … strength … endurance … energy?” 

“Yes,\marginnote{3.6} sir.” 

“Since\marginnote{3.7} this is so, do we find sentient beings who have energy?” 

“Yes,\marginnote{3.8} sir.” 

“Since\marginnote{3.9} there is an element of energy, and sentient beings who have energy are found, sentient beings act of their own volition or that of another. 

Well,\marginnote{4.1} brahmin, I’ve never seen or heard of anyone holding such a doctrine or view. How on earth can someone who comes and goes on his own say that one does not act of one’s own volition, nor does one act of another’s volition?” 

“Excellent,\marginnote{5.1} Master Gotama! Excellent! … From this day forth, may Master Gotama remember me as a lay follower who has gone for refuge for life.” 

%
\section*{{\suttatitleacronym AN 6.39}{\suttatitletranslation Sources }{\suttatitleroot Nidānasutta}}
\addcontentsline{toc}{section}{\tocacronym{AN 6.39} \toctranslation{Sources } \tocroot{Nidānasutta}}
\markboth{Sources }{Nidānasutta}
\extramarks{AN 6.39}{AN 6.39}

“Mendicants,\marginnote{1.1} there are these three sources that give rise to deeds. What three? Greed, hate, and delusion are sources that give rise to deeds. Greed doesn’t give rise to contentment. Rather, greed just gives rise to greed. Hate doesn’t give rise to love. Rather, hate just gives rise to hate. Delusion doesn’t give rise to understanding. Rather, delusion just gives rise to delusion. It’s not because of deeds born of greed, hate, and delusion that gods, humans, or those in any other good places are found. Rather, it’s because of deeds born of greed, hate, and delusion that hell, the animal realm, the ghost realm, or any other bad places are found. These are three sources that give rise to deeds. 

Mendicants,\marginnote{2.1} there are these three sources that give rise to deeds. What three? Contentment, love, and understanding are sources that give rise to deeds. Contentment doesn’t give rise to greed. Rather, contentment just gives rise to contentment. Love doesn’t give rise to hate. Rather, love just gives rise to love. Understanding doesn’t give rise to delusion. Rather, understanding just gives rise to understanding. It’s not because of deeds born of contentment, love, and understanding that hell, the animal realm, the ghost realm, or any other bad places are found. Rather, it’s because of deeds born of contentment, love, and understanding that gods, humans, or those in any other good places are found. These are three sources that give rise to deeds.” 

%
\section*{{\suttatitleacronym AN 6.40}{\suttatitletranslation With Kimbila }{\suttatitleroot Kimilasutta}}
\addcontentsline{toc}{section}{\tocacronym{AN 6.40} \toctranslation{With Kimbila } \tocroot{Kimilasutta}}
\markboth{With Kimbila }{Kimilasutta}
\extramarks{AN 6.40}{AN 6.40}

\scevam{So\marginnote{1.1} I have heard. }At one time the Buddha was staying near \textsanskrit{Kimbilā} in the Freshwater Mangrove Wood. Then Venerable Kimbila went up to the Buddha, bowed, sat down to one side, and said to him: 

“What\marginnote{1.4} is the cause, sir, what is the reason why the true teaching does not last long after the final extinguishment of the Realized One?” 

“Kimbila,\marginnote{1.5} it’s when the monks, nuns, laymen, and laywomen lack respect and reverence for the Teacher, the teaching, the \textsanskrit{Saṅgha}, the training, diligence, and hospitality after the final extinguishment of the Realized One. This is the cause, this is the reason why the true teaching does not last long after the final extinguishment of the Realized One.” 

“What\marginnote{2.1} is the cause, sir, what is the reason why the true teaching does last long after the final extinguishment of the Realized One?” 

“Kimbila,\marginnote{2.2} it’s when the monks, nuns, laymen, and laywomen maintain respect and reverence for the Teacher, the teaching, the \textsanskrit{Saṅgha}, the training, diligence, and hospitality after the final extinguishment of the Realized One. This is the cause, this is the reason why the true teaching does last long after the final extinguishment of the Realized One.” 

%
\section*{{\suttatitleacronym AN 6.41}{\suttatitletranslation A Tree Trunk }{\suttatitleroot Dārukkhandhasutta}}
\addcontentsline{toc}{section}{\tocacronym{AN 6.41} \toctranslation{A Tree Trunk } \tocroot{Dārukkhandhasutta}}
\markboth{A Tree Trunk }{Dārukkhandhasutta}
\extramarks{AN 6.41}{AN 6.41}

\scevam{So\marginnote{1.1} I have heard. }At one time Venerable \textsanskrit{Sāriputta} was staying near \textsanskrit{Rājagaha}, on the Vulture’s Peak Mountain. 

Then\marginnote{1.3} Venerable \textsanskrit{Sāriputta} robed up in the morning and, taking his bowl and robe, descended the Vulture’s Peak together with several mendicants. At a certain spot he saw a large tree trunk, and he addressed the mendicants, “Reverends, do you see this large tree trunk?” 

“Yes,\marginnote{1.6} reverend.” 

“If\marginnote{2.1} they wanted to, a mendicant with psychic powers who has mastered their mind could determine this tree trunk to be nothing but earth. Why is that? Because the earth element exists in the tree trunk. Relying on that a mendicant with psychic powers could determine it to be nothing but earth. If they wanted to, a mendicant with psychic powers who has mastered their mind could determine this tree trunk to be nothing but water. … Or they could determine it to be nothing but fire … Or they could determine it to be nothing but air … Or they could determine it to be nothing but beautiful … Or they could determine it to be nothing but ugly. Why is that? Because the element of ugliness exists in the tree trunk. Relying on that a mendicant with psychic powers could determine it to be nothing but ugly.” 

%
\section*{{\suttatitleacronym AN 6.42}{\suttatitletranslation With Nāgita }{\suttatitleroot Nāgitasutta}}
\addcontentsline{toc}{section}{\tocacronym{AN 6.42} \toctranslation{With Nāgita } \tocroot{Nāgitasutta}}
\markboth{With Nāgita }{Nāgitasutta}
\extramarks{AN 6.42}{AN 6.42}

\scevam{So\marginnote{1.1} I have heard. }At one time the Buddha was wandering in the land of the Kosalans together with a large \textsanskrit{Saṅgha} of mendicants when he arrived at a village of the Kosalan brahmins named \textsanskrit{Icchānaṅgala}. He stayed in a forest near \textsanskrit{Icchānaṅgala}. The brahmins and householders of \textsanskrit{Icchānaṅgala} heard: 

“It\marginnote{1.5} seems the ascetic Gotama—a Sakyan, gone forth from a Sakyan family—has arrived at \textsanskrit{Icchānaṅgala}. He is staying in a forest near \textsanskrit{Icchānaṅgala}. He has this good reputation: ‘That Blessed One is perfected, a fully awakened Buddha, accomplished in knowledge and conduct, holy, knower of the world, supreme guide for those who wish to train, teacher of gods and humans, awakened, blessed.’ He has realized with his own insight this world—with its gods, \textsanskrit{Māras} and \textsanskrit{Brahmās}, this population with its ascetics and brahmins, gods and humans—and he makes it known to others. He teaches Dhamma that’s good in the beginning, good in the middle, and good in the end, meaningful and well-phrased; and he explains a spiritual practice that’s entirely full and pure. It’s good to see such perfected ones.” Then, when the night had passed, they took many different foods and went to the forest near \textsanskrit{Icchānaṅgala}, where they stood outside the gates making a dreadful racket. 

Now,\marginnote{2.1} at that time Venerable \textsanskrit{Nāgita} was the Buddha’s attendant. Then the Buddha said to \textsanskrit{Nāgita}, “\textsanskrit{Nāgita}, who’s making that dreadful racket? You’d think it was fishermen hauling in a catch!” 

“Sir,\marginnote{2.4} it’s these brahmins and householders of \textsanskrit{Icchānaṅgala}. They’ve brought many different foods, and they’re standing outside the gates wanting to offer it specially to the Buddha and the mendicant \textsanskrit{Saṅgha}.” 

“\textsanskrit{Nāgita},\marginnote{2.5} may I never become famous. May fame not come to me. There are those who can’t get the bliss of renunciation, the bliss of seclusion, the bliss of peace, the bliss of awakening when they want, without trouble or difficulty like I can. Let them enjoy the filthy, lazy pleasure of possessions, honor, and popularity.” 

“Sir,\marginnote{3.1} may the Blessed One please relent now! May the Holy One relent! Now is the time for the Buddha to relent. Wherever the Buddha now goes, the brahmins and householders will incline the same way, as will the people of town and country. It’s like when it rains heavily and the water flows downhill. In the same way, wherever the Buddha now goes, the brahmins and householders will incline the same way, as will the people of town and country. Why is that? Because of the Buddha’s ethics and wisdom.” 

“\textsanskrit{Nāgita},\marginnote{4.1} may I never become famous. May fame not come to me. There are those who can’t get the bliss of renunciation, the bliss of seclusion, the bliss of peace, the bliss of awakening when they want, without trouble or difficulty like I can. Let them enjoy the filthy, lazy pleasure of possessions, honor, and popularity. 

Take\marginnote{5.1} a mendicant living within a village who I see sitting immersed in \textsanskrit{samādhi}. I think to myself: ‘Now a monastery worker, a novice, or a fellow practitioner will make this venerable fall from immersion.’ So I’m not pleased that that mendicant is living within a village. 

Take\marginnote{6.1} a mendicant in the wilderness who I see sitting nodding in meditation. I think to myself: ‘Now this venerable, having dispelled that sleepiness and weariness, will focus just on the unified perception of wilderness.’ So I’m pleased that that mendicant is living in the wilderness. 

Take\marginnote{7.1} a mendicant in the wilderness who I see sitting without being immersed in \textsanskrit{samādhi}. I think to myself: ‘Now if this venerable’s mind is not immersed in \textsanskrit{samādhi} they will immerse it, or if it is immersed in \textsanskrit{samādhi}, they will preserve it.’ So I’m pleased that that mendicant is living in the wilderness. 

Take\marginnote{8.1} a mendicant in the wilderness who I see sitting immersed in \textsanskrit{samādhi}. I think to myself: ‘Now this venerable will free the unfreed mind or preserve the freed mind.’ So I’m pleased that that mendicant is living in the wilderness. 

Take\marginnote{9.1} a mendicant who I see living within a village receiving robes, almsfood, lodgings, and medicines and supplies for the sick. Enjoying possessions, honor, and popularity they neglect retreat, and they neglect remote lodgings in the wilderness and the forest. They come down to villages, towns, and capital cities and make their home there. So I’m not pleased that that mendicant is living within a village. 

Take\marginnote{10.1} a mendicant who I see in the wilderness receiving robes, almsfood, lodgings, and medicines and supplies for the sick. Fending off possessions, honor, and popularity they don’t neglect retreat, and they don’t neglect remote lodgings in the wilderness and the forest. So I’m pleased that that mendicant is living in the wilderness. 

\textsanskrit{Nāgita},\marginnote{11.1} when I’m walking along a road and I don’t see anyone ahead or behind I feel relaxed, even if I need to urinate or defecate.” 

%
\addtocontents{toc}{\let\protect\contentsline\protect\nopagecontentsline}
\chapter*{The Chapter with Dhammika }
\addcontentsline{toc}{chapter}{\tocchapterline{The Chapter with Dhammika }}
\addtocontents{toc}{\let\protect\contentsline\protect\oldcontentsline}

%
\section*{{\suttatitleacronym AN 6.43}{\suttatitletranslation The Giant }{\suttatitleroot Nāgasutta}}
\addcontentsline{toc}{section}{\tocacronym{AN 6.43} \toctranslation{The Giant } \tocroot{Nāgasutta}}
\markboth{The Giant }{Nāgasutta}
\extramarks{AN 6.43}{AN 6.43}

At\marginnote{1.1} one time the Buddha was staying near \textsanskrit{Sāvatthī} in Jeta’s Grove, \textsanskrit{Anāthapiṇḍika}’s monastery. 

Then\marginnote{1.2} the Buddha robed up in the morning and, taking his bowl and robe, entered \textsanskrit{Sāvatthī} for alms. Then, after the meal, on his return from almsround, he addressed Venerable Ānanda, “Come, Ānanda, let’s go to the Eastern Monastery, the stilt longhouse of \textsanskrit{Migāra}’s mother for the day’s meditation.” 

“Yes,\marginnote{1.5} sir,” Ānanda replied. 

So\marginnote{2.1} the Buddha went with Ānanda to the Eastern Monastery. In the late afternoon the Buddha came out of retreat and addressed Ānanda, “Come, Ānanda, let’s go to the eastern gate to bathe.” 

“Yes,\marginnote{2.4} sir,” Ānanda replied. So the Buddha went with Ānanda to the eastern gate to bathe. When he had bathed and emerged from the water he stood in one robe drying himself. 

Now,\marginnote{3.1} at that time King Pasenadi had a giant bull elephant called “White”. It emerged from the eastern gate to the beating and playing of musical instruments. 

When\marginnote{3.2} people saw it they said, “The royal giant is so handsome! The royal giant is so good-looking! The royal giant is so lovely! The royal giant has such a huge body!” 

When\marginnote{3.4} they said this, Venerable \textsanskrit{Udāyī} said to the Buddha, “Sir, is it only when they see elephants with such a huge, formidable body that people say: ‘A giant, such a giant’? Or do they say it when they see any other creatures with huge, formidable bodies?” 

“\textsanskrit{Udāyī},\marginnote{3.7} when they see elephants with such a huge, formidable body people say: ‘A giant, such a giant!’ 

And\marginnote{3.9} also when they see a horse with a huge, formidable body … 

When\marginnote{3.10} they see a bull with a huge, formidable body … 

When\marginnote{3.11} they see a serpent with a huge, formidable body … 

When\marginnote{3.12} they see a tree with a huge, formidable body … 

And\marginnote{3.13} when they see a human being with such a huge, formidable body people say: ‘A giant, such a giant!’ 

But\marginnote{3.15} \textsanskrit{Udāyī}, one who does nothing monstrous by way of body, speech, and mind is who I call a ‘giant’ in this world with its gods, \textsanskrit{Māras}, and \textsanskrit{Brahmās}, this population with its ascetics and brahmins, its gods and humans.” 

“It’s\marginnote{4.1} incredible, sir, it’s amazing! How well said this was by the Buddha: ‘But \textsanskrit{Udāyī}, one who does nothing monstrous by way of body, speech, and mind is who I call a “giant” in this world with its gods, \textsanskrit{Māras}, and \textsanskrit{Brahmās}, this population with its ascetics and brahmins, its gods and humans.’ And I celebrate the well-spoken words of the Buddha with these verses: 

\begin{verse}%
Awakened\marginnote{5.1} as a human being, \\
self-tamed and immersed in \textsanskrit{samādhi}, \\
following the spiritual path, \\
he loves peace of mind. 

Revered\marginnote{6.1} by people, \\
gone beyond all things, \\
even the gods revere him; \\
so I’ve heard from the perfected one. 

He\marginnote{7.1} has transcended all fetters \\
and escaped from entanglements. \\
Delighting to renounce sensual pleasures, \\
he’s freed like gold from stone. 

That\marginnote{8.1} giant outshines all, \\
like the Himalaya beside other mountains. \\
Of all those named ‘giant’, \\
he is truly named, supreme. 

I\marginnote{9.1} shall extol the giant for you, \\
for he does nothing monstrous. \\
Gentleness and harmlessness \\
are two feet of the giant. 

Austerity\marginnote{10.1} and celibacy \\
are his two other feet. \\
Faith is the giant’s trunk, \\
and equanimity his white tusks. 

Mindfulness\marginnote{11.1} is his neck, his head is wisdom—\\
inquiry and thinking about principles. \\
His belly is the sacred hearth of the Dhamma, \\
and his tail is seclusion. 

Practicing\marginnote{12.1} absorption, enjoying the breath, \\
he is serene within. \\
The giant is serene when walking, \\
the giant is serene when standing, 

the\marginnote{13.1} giant is serene when lying down, \\
and when sitting, the giant is serene. \\
The giant is restrained everywhere: \\
this is the accomplishment of the giant. 

He\marginnote{14.1} eats blameless things, \\
he doesn’t eat blameworthy things. \\
When he gets food and clothes, \\
he avoids storing them up. 

Having\marginnote{15.1} severed all bonds, \\
fetters large and small, \\
wherever he goes, \\
he goes without concern. 

A\marginnote{16.1} white lotus, \\
fragrant and delightful, \\
sprouts in water and grows there, \\
but water does not stick to it. 

Just\marginnote{17.1} so the Buddha is born in the world, \\
and lives in the world, \\
but the world does not stick to him, \\
as water does not stick to the lotus. 

A\marginnote{18.1} great blazing fire \\
dies down when the fuel runs out. \\
When the coals have gone out \\
it’s said to be ‘extinguished’. 

This\marginnote{19.1} simile is taught by the discerning \\
to express the meaning clearly. \\
Great giants will understand \\
what the giant taught the giant. 

Free\marginnote{20.1} of greed, free of hate, \\
free of delusion, undefiled; \\
the giant, giving up his body, \\
being undefiled, will be fully extinguished.” 

%
\end{verse}

%
\section*{{\suttatitleacronym AN 6.44}{\suttatitletranslation With Migasālā }{\suttatitleroot Migasālāsutta}}
\addcontentsline{toc}{section}{\tocacronym{AN 6.44} \toctranslation{With Migasālā } \tocroot{Migasālāsutta}}
\markboth{With Migasālā }{Migasālāsutta}
\extramarks{AN 6.44}{AN 6.44}

Then\marginnote{1.1} Venerable Ānanda robed up in the morning and, taking his bowl and robe, went to the home of the laywoman \textsanskrit{Migasālā}, where he sat on the seat spread out. 

Then\marginnote{1.2} the laywoman \textsanskrit{Migasālā} went up to Ānanda, bowed, sat down to one side, and said to him, “Sir, Ānanda, how on earth are we supposed to understand the teaching taught by the Buddha, when the chaste and the unchaste are both reborn in exactly the same place in the next life? 

My\marginnote{2.2} father \textsanskrit{Purāṇa} was celibate, set apart, avoiding the common practice of sex. When he passed away the Buddha declared that he was a once-returner, who was reborn in the host of Joyful Gods. 

But\marginnote{2.4} my uncle Isidatta was not celibate; he lived content with his wife. When he passed away the Buddha declared that he was also a once-returner, who was reborn in the host of Joyful Gods. 

How\marginnote{2.6} on earth are we supposed to understand the teaching taught by the Buddha, when the chaste and the unchaste are both reborn in exactly the same place in the next life?” 

“You’re\marginnote{2.7} right, sister, but that’s how the Buddha declared it.” 

Then\marginnote{3.1} Ānanda, after receiving almsfood at \textsanskrit{Migasālā}’s home, rose from his seat and left. Then after the meal, on his return from almsround, Ānanda went to the Buddha, bowed, sat down to one side, and told him what had happened. 

“Ānanda,\marginnote{5.1} who is this laywoman \textsanskrit{Migasālā}, a foolish incompetent aunty, with an aunty’s wit? And who is it that knows how to assess individuals? These six people are found in the world. What six? 

Take\marginnote{6.2} a certain person who is sweet-natured and pleasant to be with. And spiritual companions enjoy living together with them. And they’ve not listened or learned or comprehended theoretically or found even temporary freedom. When their body breaks up, after death, they’re headed for a lower place, not a higher. They’re going to a lower place, not a higher. 

Take\marginnote{7.1} another person who is sweet-natured and pleasant to be with. And spiritual companions enjoy living together with them. And they’ve listened and learned and comprehended theoretically and found temporary freedom. When their body breaks up, after death, they’re headed for a higher place, not a lower. They’re going to a higher place, not a lower. 

Judgmental\marginnote{8.1} people compare them, saying: ‘This one has just the same qualities as the other, so why is one worse and one better?’ This will be for their lasting harm and suffering. 

In\marginnote{9.1} this case, the person who is sweet-natured … and has listened, learned, comprehended theoretically, and found temporary freedom is better and finer than the other person. Why is that? Because the stream of the teaching carries them along. But who knows the difference between them except a Realized One? 

So,\marginnote{9.6} Ānanda, don’t be judgmental about people. Don’t pass judgment on people. Those who pass judgment on people harm themselves. I, or someone like me, may pass judgment on people. 

Take\marginnote{10.1} another person who is angry and conceited, and from time to time has greedy thoughts. And they’ve not listened or learned or comprehended theoretically or found even temporary freedom. When their body breaks up, after death, they’re headed for a lower place, not a higher. They’re going to a lower place, not a higher. 

Take\marginnote{11.1} another person who is angry and conceited, and from time to time has greedy thoughts. … Because the stream of the teaching carries them along. … When their body breaks up, after death, they’re headed for a higher place, not a lower. They’re going to a higher place, not a lower. 

Judgmental\marginnote{12.1} people compare them … 

I,\marginnote{12.2} or someone like me, may pass judgment on people. 

Take\marginnote{13.1} another person who is angry and conceited, and from time to time has the impulse to speak inappropriately. And they’ve not listened or learned or comprehended theoretically or found even temporary freedom. When their body breaks up, after death, they’re headed for a lower place, not a higher. They’re going to a lower place, not a higher. 

Take\marginnote{14.1} another person who is angry and conceited, and from time to time has the impulse to speak inappropriately. But they’ve listened and learned and comprehended theoretically and found temporary freedom. When their body breaks up, after death, they’re headed for a higher place, not a lower. They’re going to a higher place, not a lower. 

Judgmental\marginnote{15.1} people compare them, saying: ‘This one has just the same qualities as the other, so why is one worse and one better?’ This will be for their lasting harm and suffering. 

In\marginnote{16.1} this case, the person who is angry and conceited, but has listened, learned, comprehended theoretically, and found temporary freedom is better and finer than the other person. Why is that? Because the stream of the teaching carries them along. But who knows the difference between them except a Realized One? 

So,\marginnote{16.6} Ānanda, don’t be judgmental about people. Don’t pass judgment on people. Those who pass judgment on people harm themselves. I, or someone like me, may pass judgment on people. 

Who\marginnote{17.1} is this laywoman \textsanskrit{Migasālā}, a foolish incompetent aunty, with an aunty’s wit? And who is it that knows how to assess individuals? These six people are found in the world. 

If\marginnote{18.1} Isidatta had achieved \textsanskrit{Purāṇa}’s level of ethical conduct, \textsanskrit{Purāṇa} could not have even known Isidatta’s destination. And if \textsanskrit{Purāṇa} had achieved Isidatta’s level of wisdom, Isidatta could not have even known \textsanskrit{Purāṇa}’s destination. So both individuals were lacking in one respect.” 

%
\section*{{\suttatitleacronym AN 6.45}{\suttatitletranslation Debt }{\suttatitleroot Iṇasutta}}
\addcontentsline{toc}{section}{\tocacronym{AN 6.45} \toctranslation{Debt } \tocroot{Iṇasutta}}
\markboth{Debt }{Iṇasutta}
\extramarks{AN 6.45}{AN 6.45}

“Mendicants,\marginnote{1.1} isn’t poverty suffering in the world for a person who enjoys sensual pleasures?” 

“Yes,\marginnote{1.2} sir.” 

“When\marginnote{2.1} a poor, penniless person falls into debt, isn’t being in debt also suffering in the world for a person who enjoys sensual pleasures?” 

“Yes,\marginnote{2.2} sir.” 

“When\marginnote{3.1} a poor person who has fallen into debt agrees to pay interest, isn’t the interest also suffering in the world for a person who enjoys sensual pleasures?” 

“Yes,\marginnote{3.2} sir.” 

“When\marginnote{4.1} a poor person who has fallen into debt and agreed to pay interest fails to pay it when it falls due, they get a warning. Isn’t being warned suffering in the world for a person who enjoys sensual pleasures?” 

“Yes,\marginnote{4.3} sir.” 

“When\marginnote{5.1} a poor person fails to pay after getting a warning, they’re prosecuted. Isn’t being prosecuted suffering in the world for a person who enjoys sensual pleasures?” 

“Yes,\marginnote{5.3} sir.” 

“When\marginnote{6.1} a poor person fails to pay after being prosecuted, they’re imprisoned. Isn’t being imprisoned suffering in the world for a person who enjoys sensual pleasures?” 

“Yes,\marginnote{6.3} sir.” 

“So\marginnote{7.1} mendicants, poverty, debt, interest, warnings, prosecution, and imprisonment are suffering in the world for those who enjoy sensual pleasures. In the same way, whoever has no faith, conscience, prudence, energy, and wisdom when it comes to skillful qualities is called poor and penniless in the training of the Noble One. 

Since\marginnote{8.1} they have no faith, conscience, prudence, energy, or wisdom when it comes to skillful qualities, they do bad things by way of body, speech, and mind. This is how they’re in debt, I say. 

In\marginnote{9.1} order to conceal the bad things they do by way of body, speech, and mind they harbour corrupt wishes. They wish, plan, speak, and act with the thought: ‘May no-one find me out!’ This is how they pay interest, I say. 

Good-hearted\marginnote{10.1} spiritual companions say this about them: ‘This venerable acts like this, and behaves like that.’ This is how they’re warned, I say. 

When\marginnote{11.1} they go to a wilderness, the root of a tree, or an empty hut, they’re beset by remorseful, unskillful thoughts. This is how they’re prosecuted, I say. 

That\marginnote{12.1} poor, penniless person has done bad things by way of body, speech, and mind. When their body breaks up, after death, they’re trapped in the prison of hell or the animal realm. I don’t see a single prison that’s as brutal, as vicious, and such an obstacle to reaching the supreme sanctuary as the prison of hell or the animal realm. 

\begin{verse}%
Poverty\marginnote{13.1} is said to be suffering in the world, \\
and so is being in debt. \\
A poor person who has fallen into debt \\
frets even when spending the loan. 

And\marginnote{14.1} then they’re prosecuted, \\
or even thrown in jail. \\
Such imprisonment is true suffering \\
for someone who prays for pleasure and possessions. 

In\marginnote{15.1} the same way, in the noble one’s training \\
whoever has no faith, \\
no conscience or prudence, \\
contemplates bad deeds. 

After\marginnote{16.1} doing bad things \\
by way of body, \\
speech, and mind, \\
they wish, ‘May no-one find me out!’ 

Their\marginnote{17.1} behavior is creepy \\
by body, speech, and mind. \\
They pile up bad deeds \\
on and on, life after life. 

That\marginnote{18.1} stupid evildoer, \\
knowing their own misdeeds, \\
is a poor person who has fallen into debt, \\
and frets even when spending the loan. 

And\marginnote{19.1} when in village or wilderness \\
they’re prosecuted \\
by painful mental plans, \\
which are born of remorse. 

That\marginnote{20.1} stupid evildoer, \\
knowing their own misdeeds, \\
goes to one of the animal realms, \\
or is trapped in hell. 

Such\marginnote{21.1} imprisonment is true suffering, \\
from which a wise one is released. \\
With confident heart, they give \\
with wealth that is properly earned. 

That\marginnote{22.1} faithful householder \\
wins both ways: \\
welfare and benefit in this life, \\
and happiness in the next. \\
This is how, for a householder, \\
merit grows by generosity. 

In\marginnote{23.1} the same way, in the noble one’s training, \\
whoever is grounded in faith, \\
with conscience and prudence, \\
wise, and ethically restrained, 

is\marginnote{24.1} said to live happily \\
in the noble one’s training. \\
After gaining spiritual bliss, \\
they concentrate on equanimity. 

They\marginnote{25.1} give up the five hindrances, \\
constantly energetic, \\
and enter the absorptions, \\
unified, alert, and mindful. 

Truly\marginnote{26.1} knowing in this way \\
the end of all fetters, \\
by not grasping in any way, \\
their mind is rightly freed. 

To\marginnote{27.1} that poised one, rightly freed \\
with the end of the fetters of rebirth, \\
the knowledge comes: \\
‘My freedom is unshakable.’ 

This\marginnote{28.1} is the ultimate knowledge. \\
This is the supreme happiness. \\
Sorrowless, stainless, secure: \\
this is the highest freedom from debt.” 

%
\end{verse}

%
\section*{{\suttatitleacronym AN 6.46}{\suttatitletranslation By Mahācunda }{\suttatitleroot Mahācundasutta}}
\addcontentsline{toc}{section}{\tocacronym{AN 6.46} \toctranslation{By Mahācunda } \tocroot{Mahācundasutta}}
\markboth{By Mahācunda }{Mahācundasutta}
\extramarks{AN 6.46}{AN 6.46}

\scevam{So\marginnote{1.1} I have heard. }At one time Venerable \textsanskrit{Mahācunda} was staying in the land of the Cetis at \textsanskrit{Sahajāti}. There he addressed the mendicants: “Reverends, mendicants!” 

“Reverend,”\marginnote{1.5} they replied. Venerable \textsanskrit{Mahācunda} said this: 

“Take\marginnote{2.1} a case where mendicants who practice discernment of principles rebuke mendicants who practice absorption meditation: ‘They say, “We practice absorption meditation! We practice absorption meditation!” And they meditate and concentrate and contemplate and ruminate. Why do they practice absorption meditation? In what way do they practice absorption meditation? How do they practice absorption meditation?’ In this case the mendicants who practice discernment of principles are not inspired, and the mendicants who practice absorption meditation are not inspired. And they’re not acting for the welfare and happiness of the people, for the benefit, welfare, and happiness of gods and humans. 

Now,\marginnote{3.1} take a case where mendicants who practice absorption meditation rebuke mendicants who practice discernment of principles: ‘They say, “We practice discernment of principles! We practice discernment of principles!” But they’re restless, insolent, fickle, scurrilous, loose-tongued, unmindful, lacking situational awareness and immersion, with straying minds and undisciplined faculties. Why do they practice discernment of principles? In what way do they practice discernment of principles? How do they practice discernment of principles?’ In this case the mendicants who practice absorption meditation are not inspired, and the mendicants who practice discernment of principles are not inspired. And they’re not acting for the welfare and happiness of the people, for the benefit, welfare, and happiness of gods and humans. 

Now,\marginnote{4.1} take a case where mendicants who practice discernment of principles praise only others like them, not mendicants who practice absorption meditation. In this case the mendicants who practice discernment of principles are not inspired, and the mendicants who practice absorption meditation are not inspired. And they’re not acting for the welfare and happiness of the people, for the benefit, welfare, and happiness of gods and humans. 

And\marginnote{5.1} take a case where mendicants who practice absorption meditation praise only others like them, not mendicants who practice discernment of principles. In this case the mendicants who practice absorption meditation are not inspired, and the mendicants who practice discernment of principles are not inspired. And they’re not acting for the welfare and happiness of the people, for the benefit, welfare, and happiness of gods and humans. 

So\marginnote{6.1} you should train like this: ‘As mendicants who practice discernment of principles, we will praise mendicants who practice absorption meditation.’ That’s how you should train. Why is that? Because it’s incredibly rare to find individuals in the world who have direct meditative experience of the deathless element. 

So\marginnote{7.1} you should train like this: ‘As mendicants who practice absorption meditation, we will praise mendicants who practice discernment of principles.’ That’s how you should train. Why is that? Because it’s incredibly rare to find individuals in the world who see the meaning of a deep saying with penetrating wisdom.” 

%
\section*{{\suttatitleacronym AN 6.47}{\suttatitletranslation Visible in This Very Life (1st) }{\suttatitleroot Paṭhamasandiṭṭhikasutta}}
\addcontentsline{toc}{section}{\tocacronym{AN 6.47} \toctranslation{Visible in This Very Life (1st) } \tocroot{Paṭhamasandiṭṭhikasutta}}
\markboth{Visible in This Very Life (1st) }{Paṭhamasandiṭṭhikasutta}
\extramarks{AN 6.47}{AN 6.47}

And\marginnote{1.1} then the wanderer \textsanskrit{Moliyasīvaka} went up to the Buddha, and exchanged greetings with him. When the greetings and polite conversation were over, he sat down to one side and said to the Buddha: 

“Sir,\marginnote{1.3} they speak of ‘a teaching visible in this very life’. In what way is the teaching visible in this very life, immediately effective, inviting inspection, relevant, so that sensible people can know it for themselves?” 

“Well\marginnote{2.1} then, \textsanskrit{Sīvaka}, I’ll ask you about this in return, and you can answer as you like. What do you think, \textsanskrit{Sīvaka}? When there’s greed in you, do you understand ‘I have greed in me’? And when there’s no greed in you, do you understand ‘I have no greed in me’?” 

“Yes,\marginnote{2.4} sir.” 

“Since\marginnote{2.5} you know this, this is how the teaching is visible in this very life, immediately effective, inviting inspection, relevant, so that sensible people can know it for themselves. 

What\marginnote{3.1} do you think, \textsanskrit{Sīvaka}? When there’s hate … delusion … greedy thoughts … hateful thoughts … When there are delusional thoughts in you, do you understand ‘I have delusional thoughts in me’? And when there are no delusional thoughts in you, do you understand ‘I have no delusional thoughts in me’?” 

“Yes,\marginnote{3.7} sir.” 

“Since\marginnote{3.8} you know this, this is how the teaching is visible in this very life, immediately effective, inviting inspection, relevant, so that sensible people can know it for themselves.” 

“Excellent,\marginnote{4.1} sir! Excellent! From this day forth, may the Buddha remember me as a lay follower who has gone for refuge for life.” 

%
\section*{{\suttatitleacronym AN 6.48}{\suttatitletranslation Visible in This Very Life (2nd) }{\suttatitleroot Dutiyasandiṭṭhikasutta}}
\addcontentsline{toc}{section}{\tocacronym{AN 6.48} \toctranslation{Visible in This Very Life (2nd) } \tocroot{Dutiyasandiṭṭhikasutta}}
\markboth{Visible in This Very Life (2nd) }{Dutiyasandiṭṭhikasutta}
\extramarks{AN 6.48}{AN 6.48}

Then\marginnote{1.1} a certain brahmin went up to the Buddha, and exchanged greetings with him. When the greetings and polite conversation were over, he sat down to one side and said to the Buddha: 

“Master\marginnote{1.3} Gotama, they speak of ‘a teaching visible in this very life’. In what way is the teaching visible in this very life, immediately effective, inviting inspection, relevant, so that sensible people can know it for themselves?” 

“Well\marginnote{2.1} then, brahmin, I’ll ask you about this in return, and you can answer as you like. What do you think, brahmin? When there’s greed in you, do you understand ‘I have greed in me’? And when there’s no greed in you, do you understand ‘I have no greed in me’?” 

“Yes,\marginnote{2.4} sir.” 

“Since\marginnote{2.5} you know this, this is how the teaching is visible in this very life, immediately effective, inviting inspection, relevant, so that sensible people can know it for themselves. 

What\marginnote{3.1} do you think, brahmin? When there’s hate … delusion … corruption that leads to physical deeds … corruption that leads to speech … When there’s corruption that leads to mental deeds in you, do you understand ‘I have corruption that leads to mental deeds in me’? And when there’s no corruption that leads to mental deeds in you, do you understand ‘I have no corruption that leads to mental deeds in me’?” 

“Yes,\marginnote{3.7} sir.” 

“Since\marginnote{3.8} you know this, this is how the teaching is visible in this very life, immediately effective, inviting inspection, relevant, so that sensible people can know it for themselves.” 

“Excellent,\marginnote{4.1} Master Gotama! Excellent! … From this day forth, may Master Gotama remember me as a lay follower who has gone for refuge for life.” 

%
\section*{{\suttatitleacronym AN 6.49}{\suttatitletranslation With Khema }{\suttatitleroot Khemasutta}}
\addcontentsline{toc}{section}{\tocacronym{AN 6.49} \toctranslation{With Khema } \tocroot{Khemasutta}}
\markboth{With Khema }{Khemasutta}
\extramarks{AN 6.49}{AN 6.49}

At\marginnote{1.1} one time the Buddha was staying near \textsanskrit{Sāvatthī} in Jeta’s Grove, \textsanskrit{Anāthapiṇḍika}’s monastery. 

Now\marginnote{1.2} at that time Venerable Khema and Venerable Sumana were staying near \textsanskrit{Sāvatthī} in the Dark Forest. Then they went up to the Buddha, bowed, and sat down to one side. Venerable Khema said to the Buddha: 

“Sir,\marginnote{2.1} a mendicant who is perfected—with defilements ended, who has completed the spiritual journey, done what had to be done, laid down the burden, achieved their own true goal, utterly ended the fetters of rebirth, and is rightly freed through enlightenment—does not think: ‘There is someone better than me, or equal to me, or worse than me.’” 

That\marginnote{2.3} is what Khema said, and the teacher approved. Then Khema, knowing that the teacher approved, got up from his seat, bowed, and respectfully circled the Buddha, keeping him on his right, before leaving. 

And\marginnote{3.1} then, not long after Khema had left, Sumana said to the Buddha: 

“Sir,\marginnote{3.2} a mendicant who is perfected—with defilements ended, who has completed the spiritual journey, done what had to be done, laid down the burden, achieved their own true goal, utterly ended the fetters of rebirth, and is rightly freed through enlightenment—does not think: ‘There is no-one better than me, or equal to me, or worse than me.’” 

That\marginnote{3.4} is what Sumana said, and the teacher approved. Then Sumana, knowing that the teacher approved, got up from his seat, bowed, and respectfully circled the Buddha, keeping him on his right, before leaving. 

And\marginnote{4.1} then, soon after Khema and Sumana had left, the Buddha addressed the mendicants: “Mendicants, this is how gentlemen declare enlightenment. The goal is spoken of, but the self is not involved. But it seems that there are some foolish people here who declare enlightenment as a joke. Later they will fall into anguish. 

\begin{verse}%
They\marginnote{5.1} don’t rank themselves \\
as being among superiors, inferiors, or equals. \\
Rebirth is ended, the spiritual journey has been completed. \\
They live freed from fetters.” 

%
\end{verse}

%
\section*{{\suttatitleacronym AN 6.50}{\suttatitletranslation Sense Restraint }{\suttatitleroot Indriyasaṁvarasutta}}
\addcontentsline{toc}{section}{\tocacronym{AN 6.50} \toctranslation{Sense Restraint } \tocroot{Indriyasaṁvarasutta}}
\markboth{Sense Restraint }{Indriyasaṁvarasutta}
\extramarks{AN 6.50}{AN 6.50}

“Mendicants,\marginnote{1.1} when there is no sense restraint, one who lacks sense restraint has destroyed a vital condition for ethical conduct. When there is no ethical conduct, one who lacks ethics has destroyed a vital condition for right immersion. When there is no right immersion, one who lacks right immersion has destroyed a vital condition for true knowledge and vision. When there is no true knowledge and vision, one who lacks true knowledge and vision has destroyed a vital condition for disillusionment and dispassion. When there is no disillusionment and dispassion, one who lacks disillusionment and dispassion has destroyed a vital condition for knowledge and vision of freedom. 

Suppose\marginnote{1.6} there was a tree that lacked branches and foliage. Its shoots, bark, softwood, and heartwood would not grow to fullness. 

In\marginnote{1.8} the same way, when there is no sense restraint, one who lacks sense restraint has destroyed a vital condition for ethical conduct. … One who lacks disillusionment and dispassion has destroyed a vital condition for knowledge and vision of freedom. 

When\marginnote{2.1} there is sense restraint, one who has sense restraint has fulfilled a vital condition for ethical conduct. When there is ethical conduct, one who has fulfilled ethical conduct has fulfilled a vital condition for right immersion. When there is right immersion, one who has fulfilled right immersion has fulfilled a vital condition for true knowledge and vision. When there is true knowledge and vision, one who has fulfilled true knowledge and vision has fulfilled a vital condition for disillusionment and dispassion. When there is disillusionment and dispassion, one who has fulfilled disillusionment and dispassion has fulfilled a vital condition for knowledge and vision of freedom. 

Suppose\marginnote{2.6} there was a tree that was complete with branches and foliage. Its shoots, bark, softwood, and heartwood would all grow to fullness. 

In\marginnote{2.7} the same way, when there is sense restraint, one who has fulfilled sense restraint has fulfilled a vital condition for ethical conduct. … One who has fulfilled disillusionment and dispassion has fulfilled a vital condition for knowledge and vision of freedom.” 

%
\section*{{\suttatitleacronym AN 6.51}{\suttatitletranslation With Ānanda }{\suttatitleroot Ānandasutta}}
\addcontentsline{toc}{section}{\tocacronym{AN 6.51} \toctranslation{With Ānanda } \tocroot{Ānandasutta}}
\markboth{With Ānanda }{Ānandasutta}
\extramarks{AN 6.51}{AN 6.51}

Then\marginnote{1.1} Venerable Ānanda went up to Venerable \textsanskrit{Sāriputta}, and exchanged greetings with him. When the greetings and polite conversation were over, Ānanda sat down to one side, and said to \textsanskrit{Sāriputta}: 

“Reverend\marginnote{2.1} \textsanskrit{Sāriputta}, how does a mendicant get to hear a teaching they haven’t heard before? How do they remember those teachings they have heard? How do they keep rehearsing the teachings they’ve already got to know? And how do they come to understand what they haven’t understood before?” 

“Well,\marginnote{2.2} Venerable Ānanda, you’re very learned. Why don’t you clarify this yourself?” 

“Well\marginnote{2.4} then, Reverend \textsanskrit{Sāriputta}, listen and pay close attention, I will speak.” 

“Yes,\marginnote{2.5} reverend,” \textsanskrit{Sāriputta} replied. Ānanda said this: 

“Reverend\marginnote{3.1} \textsanskrit{Sāriputta}, take a mendicant who memorizes the teaching—statements, songs, discussions, verses, inspired exclamations, legends, stories of past lives, amazing stories, and classifications. 

Then,\marginnote{3.3} just as they learned and memorized it, they teach others in detail, make them recite in detail, practice reciting in detail, and think about and consider the teaching in their heart, examining it with the mind. 

They\marginnote{3.4} enter the rains retreat in a monastery with senior mendicants who are very learned, knowledgeable in the scriptures, who have memorized the teachings, the monastic law, and the outlines. From time to time they go up to those mendicants and ask them questions: ‘Why, sir, does it say this? What does that mean?’ Those venerables clarify what is unclear, reveal what is obscure, and dispel doubt regarding the many doubtful matters. 

This\marginnote{3.8} is how a mendicant gets to hear a teaching they haven’t heard before. It’s how they remember those teachings they have heard. It’s how they keep rehearsing the teachings they’ve already got to know. And it’s how they come to understand what they haven’t understood before.” 

“It’s\marginnote{4.1} incredible, reverend, it’s amazing! How well said this was by Venerable Ānanda! And we will remember Venerable Ānanda as someone who has these six qualities. 

For\marginnote{4.3} Ānanda memorizes the teaching … statements, songs, discussions, verses, inspired exclamations, legends, stories of past lives, amazing stories, and classifications. Those venerables clarify to Ānanda what is unclear, reveal what is obscure, and dispel doubt regarding the many doubtful matters.” 

%
\section*{{\suttatitleacronym AN 6.52}{\suttatitletranslation Aristocrats }{\suttatitleroot Khattiyasutta}}
\addcontentsline{toc}{section}{\tocacronym{AN 6.52} \toctranslation{Aristocrats } \tocroot{Khattiyasutta}}
\markboth{Aristocrats }{Khattiyasutta}
\extramarks{AN 6.52}{AN 6.52}

And\marginnote{1.1} then the brahmin \textsanskrit{Jāṇussoṇi} went up to the Buddha, and exchanged greetings with him. When the greetings and polite conversation were over, he sat down to one side and said to the Buddha: 

“Aristocrats,\marginnote{2.1} Master Gotama, have what as their ambition? What is their preoccupation? What are they dedicated to? What do they insist on? What is their ultimate goal?” 

“Aristocrats,\marginnote{2.2} brahmin, have wealth as their ambition. They’re preoccupied with wisdom. They’re dedicated to power. They insist on territory. Their ultimate goal is authority.” 

“Brahmins,\marginnote{3.1} Master Gotama, have what as their ambition? What is their preoccupation? What are they dedicated to? What do they insist on? What is their ultimate goal?” 

“Brahmins\marginnote{3.2} have wealth as their ambition. They’re preoccupied with wisdom. They’re dedicated to the hymns. They insist on sacrifice. Their ultimate goal is the \textsanskrit{Brahmā} realm.” 

“Householders,\marginnote{4.1} Master Gotama, have what as their ambition? What is their preoccupation? What are they dedicated to? What do they insist on? What is their ultimate goal?” 

“Householders\marginnote{4.2} have wealth as their ambition. They’re preoccupied with wisdom. They’re dedicated to their profession. They insist on work. Their ultimate goal is to complete their work.” 

“Women,\marginnote{5.1} Master Gotama, have what as their ambition? What is their preoccupation? What are they dedicated to? What do they insist on? What is their ultimate goal?” 

“Women\marginnote{5.2} have a man as their ambition. They’re preoccupied with adornments. They’re dedicated to their children. They insist on being without a co-wife. Their ultimate goal is authority.” 

“Bandits,\marginnote{6.1} Master Gotama, have what as their ambition? What is their preoccupation? What are they dedicated to? What do they insist on? What is their ultimate goal?” 

“Bandits\marginnote{6.2} have theft as their ambition. They’re preoccupied with a hiding place. They’re dedicated to their sword. They insist on darkness. Their ultimate goal is invisibility.” 

“Ascetics,\marginnote{7.1} Master Gotama, have what as their ambition? What is their preoccupation? What are they dedicated to? What do they insist on? What is their ultimate goal?” 

“Ascetics\marginnote{7.2} have patience and gentleness as their ambition. They’re preoccupied with wisdom. They’re dedicated to ethical conduct. They insist on owning nothing. Their ultimate goal is extinguishment.” 

“It’s\marginnote{8.1} incredible, Master Gotama, it’s amazing! Master Gotama knows the ambition, preoccupation, dedication, insistence, and ultimate goal of aristocrats, brahmins, householders, women, bandits, and ascetics. Excellent, Master Gotama! Excellent! … From this day forth, may Master Gotama remember me as a lay follower who has gone for refuge for life.” 

%
\section*{{\suttatitleacronym AN 6.53}{\suttatitletranslation Diligence }{\suttatitleroot Appamādasutta}}
\addcontentsline{toc}{section}{\tocacronym{AN 6.53} \toctranslation{Diligence } \tocroot{Appamādasutta}}
\markboth{Diligence }{Appamādasutta}
\extramarks{AN 6.53}{AN 6.53}

Then\marginnote{1.1} a certain brahmin went up to the Buddha, and exchanged greetings with him. When the greetings and polite conversation were over, he sat down to one side and said to the Buddha: 

“Master\marginnote{2.1} Gotama, is there one thing that, when developed and cultivated, secures benefits for both the present life and lives to come?” 

“There\marginnote{2.2} is, brahmin.” 

“So\marginnote{3.1} what is it?” 

“Diligence,\marginnote{3.2} brahmin, is one thing that, when developed and cultivated, secures benefits for both the present life and lives to come. 

The\marginnote{4.1} footprints of all creatures that walk can fit inside an elephant’s footprint. So an elephant’s footprint is said to be the biggest of them all. In the same way, diligence is one thing that, when developed and cultivated, secures benefits for both the present life and lives to come. 

The\marginnote{5.1} rafters of a bungalow all lean to the peak, slope to the peak, and meet at the peak, so the peak is said to be the topmost of them all. In the same way, diligence is one thing … 

A\marginnote{6.1} reed-cutter, having cut the reeds, grabs them at the top and shakes them down, shakes them about, and shakes them off. In the same way, diligence is one thing … 

When\marginnote{7.1} the stalk of a bunch of mangoes is cut, all the mangoes attached to the stalk will follow along. In the same way, diligence is one thing … 

All\marginnote{8.1} lesser rulers are vassals of a wheel-turning monarch, so the wheel-turning monarch is said to be the foremost of them all. In the same way, diligence is one thing … 

The\marginnote{9.1} radiance of all the stars is not worth a sixteenth part of the moon’s radiance, so the moon’s radiance is said to be the best of them all. In the same way, diligence is one thing that, when developed and cultivated, secures benefits for both the present life and lives to come. 

This\marginnote{10.1} is the one thing that, when developed and cultivated, secures benefits for both the present life and lives to come.” 

“Excellent,\marginnote{11.1} Master Gotama! Excellent! … From this day forth, may Master Gotama remember me as a lay follower who has gone for refuge for life.” 

%
\section*{{\suttatitleacronym AN 6.54}{\suttatitletranslation About Dhammika }{\suttatitleroot Dhammikasutta}}
\addcontentsline{toc}{section}{\tocacronym{AN 6.54} \toctranslation{About Dhammika } \tocroot{Dhammikasutta}}
\markboth{About Dhammika }{Dhammikasutta}
\extramarks{AN 6.54}{AN 6.54}

At\marginnote{1.1} one time the Buddha was staying near \textsanskrit{Rājagaha}, on the Vulture’s Peak Mountain. 

Now\marginnote{1.2} at that time Venerable Dhammika was a resident in all seven monasteries of his native land. There he abused visiting mendicants; he insulted, harmed, attacked, and harassed them. The visiting mendicants who were treated in this way did not stay. They left, abandoning the monastery. 

Then\marginnote{2.1} the local lay followers thought to themselves, “We have supplied the mendicant \textsanskrit{Saṅgha} with robes, almsfood, lodgings, and medicines and supplies for the sick. But the visiting mendicants don’t stay. They leave, abandoning the monastery. What is the cause, what is the reason for this?” 

Then\marginnote{2.5} the local lay followers thought to themselves, “This Venerable Dhammika abuses visiting mendicants; he insults, harms, attacks, and harasses them. The visiting mendicants who were treated in this way do not stay. They leave, abandoning the monastery. Why don’t we banish Venerable Dhammika?” 

Then\marginnote{3.1} the local lay followers went up to Venerable Dhammika and said to him, “Sir, please leave this monastery. You’ve stayed here long enough.” 

Then\marginnote{3.4} Venerable Dhammika left and went to another monastery. There he abused visiting mendicants; he insulted, harmed, attacked, and harassed them. The visiting mendicants who were treated in this way did not stay. They left, abandoning the monastery. 

Then\marginnote{4.1} the local lay followers thought to themselves: … 

They\marginnote{5.1} said to Venerable Dhammika, “Sir, please leave this monastery. You’ve stayed here long enough.” 

Then\marginnote{5.4} Venerable Dhammika left and went to another monastery. There he abused visiting mendicants; he insulted, harmed, attacked, and harassed them. The visiting mendicants who were treated in this way did not stay. They left, abandoning the monastery. 

Then\marginnote{6.1} the local lay followers thought to themselves, “Why don’t we banish Venerable Dhammika from all seven monasteries in our native land?” 

Then\marginnote{6.7} the local lay followers went up to Venerable Dhammika and said to him, “Sir, please leave all seven monasteries in our native land.” 

Then\marginnote{6.9} Venerable Dhammika thought, “I’ve been banished by the local lay followers from all seven monasteries in my native land. Where am I to go now?” He thought, “Why don’t I go to see the Buddha?” 

Then\marginnote{7.1} Venerable Dhammika took his bowl and robe and set out for \textsanskrit{Rājagaha}. Eventually he came to \textsanskrit{Rājagaha} and the Vulture’s Peak. He went up to the Buddha, bowed, and sat down to one side. The Buddha said to him, “So, Brahmin Dhammika, where have you come from?” 

“Sir,\marginnote{7.4} I’ve been banished by the local lay followers from all seven monasteries in my native land.” 

“Enough,\marginnote{7.5} Brahmin Dhammika, what’s that to you? Now that you’ve been banished from all of those places, you have come to me. 

Once\marginnote{8.1} upon a time, some sea-merchants set sail for the ocean deeps, taking with them a land-spotting bird. When their ship was out of sight of land, they released the bird. It flew right away to the east, the west, the north, the south, upwards, and in-between. If it saw land on any side, it went there and stayed. But if it saw no land on any side it returned to the ship. In the same way, now that you’ve been banished from all of those places, you have come to me. 

Once\marginnote{9.1} upon a time, King Koravya had a royal banyan tree with five trunks called ‘Well Planted’. It was shady and lovely. Its canopy spread over twelve leagues, while the network of roots spread for five leagues. Its fruits were as large as a rice pot. And they were as sweet as pure wild honey. The king and harem made use of one trunk, the troops another, the people of town and country another, ascetics and brahmins another, and beasts and birds another. No-one guarded the fruit, yet no-one damaged another’s fruits. 

Then\marginnote{10.1} a certain person ate as much as he liked of the fruit, then broke off a branch and left. Then the deity haunting the royal banyan tree thought, ‘It’s incredible, it’s amazing! How wicked this person is, to eat as much as they like, then break off a branch and leave! Why don’t I make sure that the royal banyan tree gives no fruit in future?’ Then the royal banyan tree gave no more fruit. 

Then\marginnote{11.1} King Koravya went up to Sakka, lord of gods, and said to him, ‘Please sir, you should know that the royal banyan tree called Well Planted gives no fruit.’ Then Sakka used his psychic powers to will that a violent storm come. And it felled and uprooted the royal banyan tree. Then the deity haunting the tree stood to one side, miserable and sad, weeping, with a tearful face. 

Then\marginnote{12.1} Sakka went up to that deity, and said, ‘Why, god, are you standing to one side, miserable and sad, weeping, with a tearful face?’ 

‘Because,\marginnote{12.3} my good sir, a violent storm came and felled and uprooted my home.’ 

‘Well,\marginnote{12.4} did you stand by your tree’s duty when the storm came?’ 

‘But\marginnote{12.5} my good sir, how does a tree stand by its duty?’ 

‘It’s\marginnote{12.6} when those who need the tree’s roots, bark, leaves, flowers, or fruit take what they need. Yet the deity is not displeased or upset because of this. This is how a tree stands by its duty.’ 

‘I\marginnote{12.9} was not standing by a tree’s duty when the storm came and felled and uprooted my home.’ 

‘God,\marginnote{12.10} if you were to stand by a tree’s duty, your home may be as it was before.’ 

‘I\marginnote{12.11} will stand by a tree’s duty! May my home be as it was before!’ 

Then\marginnote{13.1} Sakka used his psychic power to will that a violent storm come. And it raised up that mighty banyan tree and the bark of the roots was healed. 

In\marginnote{13.2} the same way, Brahmin Dhammika, were you standing by an ascetic’s duty when the local lay followers banished you from all seven of the monasteries in your native land?” 

“But\marginnote{13.3} sir, how do I stand by an ascetic’s duty?” 

“When\marginnote{13.4} someone abuses, annoys, or argues with an ascetic, the ascetic doesn’t abuse, annoy, or argue back at them. That’s how an ascetic stands by an ascetic’s duty.” 

“I\marginnote{13.6} was not standing by an ascetic’s duty when the local lay followers banished me from all seven of the monasteries in my native land.” 

“Once\marginnote{13.7} upon a time, there was a Teacher called Sunetta. He was a religious founder and was free of sensual desire. He had many hundreds of disciples. He taught them the path to rebirth in the company of \textsanskrit{Brahmā}. Those lacking confidence in Sunetta were—when their body broke up, after death—reborn in a place of loss, a bad place, the underworld, hell. Those full of confidence in Sunetta were—when their body broke up, after death—reborn in a good place, a heavenly realm. 

Once\marginnote{14.1} upon a time there was a teacher called \textsanskrit{Mūgapakkha} … Aranemi … \textsanskrit{Kuddālaka} … \textsanskrit{Hatthipāla} … \textsanskrit{Jotipāla}. He was a religious founder and was free of sensual desire. He had many hundreds of disciples. He taught them the way to rebirth in the company of \textsanskrit{Brahmā}. Those lacking confidence in \textsanskrit{Jotipāla} were—when their body broke up, after death—reborn in a place of loss, a bad place, the underworld, hell. Those full of confidence in \textsanskrit{Jotipāla} were—when their body broke up, after death—reborn in a good place, a heavenly realm. 

What\marginnote{19.1} do you think, Brahmin Dhammika? If someone with malicious intent were to abuse and insult these six teachers with their hundreds of followers, would they not make much bad karma?” 

“Yes,\marginnote{19.3} sir.” 

“They\marginnote{19.4} would indeed. But someone who abuses and insults a single person accomplished in view with malicious intent makes even more bad karma. Why is that? Brahmin Dhammika, I say that any injury done by those outside of the Buddhist community does not compare with what is done to one’s own spiritual companions. So you should train like this: ‘We will have no malicious intent for those who we want to have as our spiritual companions.’ That is how you should train. 

\begin{verse}%
Sunetta\marginnote{20.1} and \textsanskrit{Mūgapakkha}, \\
and Aranemi the brahmin, \\
\textsanskrit{Hatthipāla} the student, \\
and \textsanskrit{Kuddālaka} were Teachers. 

And\marginnote{21.1} \textsanskrit{Jotipāla} Govinda \\
was priest for seven kings. \\
These six famous teachers, \\
harmless ones of the past, 

were\marginnote{22.1} free of putrefaction, compassionate, \\
gone beyond the fetter of sensuality. \\
Detached from sensual desire, \\
they were reborn in the \textsanskrit{Brahmā} realm. 

Many\marginnote{23.1} hundreds of \\
their disciples were also \\
free of putrefaction-stench, compassionate, \\
gone beyond the fetter of sensuality. \\
Detached from sensual desire, \\
they were reborn in the \textsanskrit{Brahmā} realm. 

One\marginnote{24.1} who insults \\
with malicious intent \\
these non-Buddhist hermits, \\
free of desire, immersed in \textsanskrit{samādhi}; \\
such a man \\
makes much bad karma. 

But\marginnote{25.1} one who insults \\
with malicious intent \\
a single person accomplished in view, \\
a mendicant disciple of the Buddha; \\
that man \\
makes even more bad karma. 

You\marginnote{26.1} shouldn’t attack a holy person, \\
who has given up the grounds for views. \\
This person is called \\
the seventh of the noble \textsanskrit{Saṅgha}. 

They’re\marginnote{27.1} not free of desire for sensual pleasures, \\
and their faculties are still immature: \\
faith, mindfulness, and energy, \\
serenity and discernment. 

If\marginnote{28.1} you attack such a mendicant, \\
you first hurt yourself. \\
Having hurt yourself, \\
you harm the other. 

But\marginnote{29.1} if you protect yourself, \\
the other is also protected. \\
So you should protect yourself. \\
An astute person is always uninjured.” 

%
\end{verse}

%
\addtocontents{toc}{\let\protect\contentsline\protect\nopagecontentsline}
\pannasa{The Second Fifty }
\addcontentsline{toc}{pannasa}{The Second Fifty }
\markboth{}{}
\addtocontents{toc}{\let\protect\contentsline\protect\oldcontentsline}

%
\addtocontents{toc}{\let\protect\contentsline\protect\nopagecontentsline}
\chapter*{The Great Chapter }
\addcontentsline{toc}{chapter}{\tocchapterline{The Great Chapter }}
\addtocontents{toc}{\let\protect\contentsline\protect\oldcontentsline}

%
\section*{{\suttatitleacronym AN 6.55}{\suttatitletranslation With Soṇa }{\suttatitleroot Soṇasutta}}
\addcontentsline{toc}{section}{\tocacronym{AN 6.55} \toctranslation{With Soṇa } \tocroot{Soṇasutta}}
\markboth{With Soṇa }{Soṇasutta}
\extramarks{AN 6.55}{AN 6.55}

\scevam{So\marginnote{1.1} I have heard. }At one time the Buddha was staying near \textsanskrit{Rājagaha}, on the Vulture’s Peak Mountain. 

Now\marginnote{1.3} at that time Venerable \textsanskrit{Soṇa} was staying near \textsanskrit{Rājagaha} in the Cool Grove. Then as he was in private retreat this thought came to his mind, “I am one of the Buddha’s most energetic disciples. Yet my mind is not freed from defilements by not grasping. But my family has wealth. I could enjoy that wealth and make merit. Why don’t I resign the training and return to a lesser life, so I can enjoy my wealth and make merit?” 

Then\marginnote{2.1} the Buddha knew what Venerable \textsanskrit{Soṇa} was thinking. As easily as a strong person would extend or contract their arm, he vanished from the Vulture’s Peak and reappeared in the Cool Grove in front of \textsanskrit{Soṇa}, and sat on the seat spread out. \textsanskrit{Soṇa} bowed to the Buddha and sat down to one side. 

The\marginnote{2.4} Buddha said to him, “\textsanskrit{Soṇa}, as you were in private retreat didn’t this thought come to your mind: ‘I am one of the Buddha’s most energetic disciples. Yet my mind is not freed from defilements by not grasping. But my family has wealth. I could enjoy that wealth and make merit. Why don’t I resign the training and return to a lesser life, so I can enjoy my wealth and make merit?’” 

“Yes,\marginnote{3.5} sir.” 

“What\marginnote{4.1} do you think, \textsanskrit{Soṇa}? When you were still a layman, weren’t you a good player of the arched harp?” 

“Yes,\marginnote{4.3} sir.” 

“When\marginnote{4.4} your harp’s strings were tuned too tight, was it resonant and playable?” 

“No,\marginnote{4.5} sir.” 

“When\marginnote{5.1} your harp’s strings were tuned too slack, was it resonant and playable?” 

“No,\marginnote{5.2} sir.” 

“But\marginnote{6.1} when your harp’s strings were tuned neither too tight nor too slack, but fixed at an even tension, was it resonant and playable?” 

“Yes,\marginnote{6.2} sir.” 

“In\marginnote{7.1} the same way, \textsanskrit{Soṇa}, when energy is too forceful it leads to restlessness. When energy is too slack it leads to laziness. So, \textsanskrit{Soṇa}, you should apply yourself to energy and serenity, find a balance of the faculties, and learn the pattern of this situation.” 

“Yes,\marginnote{7.3} sir,” \textsanskrit{Soṇa} replied. 

After\marginnote{7.4} advising \textsanskrit{Soṇa} like this, the Buddha, as easily as a strong person would extend or contract their arm, vanished from the Cool Grove and reappeared on the Vulture’s Peak. 

After\marginnote{8.1} some time \textsanskrit{Soṇa} applied himself to energy and serenity, found a balance of the faculties, and learned the pattern of this situation. Then \textsanskrit{Soṇa}, living alone, withdrawn, diligent, keen, and resolute, soon realized the supreme culmination of the spiritual path in this very life. He lived having achieved with his own insight the goal for which gentlemen rightly go forth from the lay life to homelessness. 

He\marginnote{8.3} understood: “Rebirth is ended; the spiritual journey has been completed; what had to be done has been done; there is no return to any state of existence.” And Venerable \textsanskrit{Soṇa} became one of the perfected. 

Then,\marginnote{9.1} when \textsanskrit{Soṇa} had attained perfection, he thought, “Why don’t I go to the Buddha and declare my enlightenment in his presence?” Then \textsanskrit{Soṇa} went up to the Buddha, bowed, sat down to one side, and said to him: 

“Sir,\marginnote{10.1} a mendicant who is perfected—with defilements ended, who has completed the spiritual journey, done what had to be done, laid down the burden, achieved their own true goal, utterly ended the fetters of rebirth, and is rightly freed through enlightenment—is dedicated to six things. They are dedicated to renunciation, seclusion, kindness, the ending of craving, the ending of grasping, and mental clarity. 

It\marginnote{11.1} may be, sir, that one of the venerables here thinks: ‘Maybe this venerable is dedicated to renunciation solely out of mere faith.’ But it should not be seen like this. A mendicant with defilements ended does not see in themselves anything more to do, or anything that needs improvement. They’re dedicated to renunciation because they’re free of greed, hate, and delusion with the end of greed, hate, and delusion. 

It\marginnote{12.1} may be, sir, that one of the venerables here thinks: ‘Maybe this venerable is dedicated to seclusion because they enjoy possessions, honor, and popularity.’ But it should not be seen like this. … 

It\marginnote{13.1} may be, sir, that one of the venerables here thinks: ‘Maybe this venerable is dedicated to kindness because they believe that adhering to precepts and observances is the most important thing.’ But it should not be seen like this. … 

They’re\marginnote{14.1} dedicated to the ending of craving because they’re free of greed, hate, and delusion with the end of greed, hate, and delusion. 

They’re\marginnote{15.1} dedicated to the ending of grasping because they’re free of greed, hate, and delusion with the end of greed, hate, and delusion. 

They’re\marginnote{16.1} dedicated to clarity of mind because they’re free of greed, hate, and delusion with the end of greed, hate, and delusion. 

When\marginnote{17.1} a mendicant’s mind is rightly freed like this, even if compelling sights come into the range of vision they don’t overcome their mind. The mind remains unaffected. It is steady, imperturbable, observing disappearance. Even if compelling sounds … smells … tastes … touches … and thoughts come into the range of the mind they don’t overcome the mind. The mind remains unaffected. It is steady, imperturbable, observing disappearance. 

Suppose\marginnote{17.9} there was a mountain that was one solid mass of rock, without cracks or holes. Even if violent storms were to blow up out of the east, the west, the north, and the south, they couldn’t make it shake or rock or tremble. 

In\marginnote{17.13} the same way, when a mendicant’s mind is rightly freed like this, even if compelling sights come into the range of vision they don’t overcome their mind. … The mind remains unaffected. It is steady, imperturbable, observing disappearance. 

\begin{verse}%
When\marginnote{18.1} you’re dedicated to renunciation \\
and seclusion of the heart; \\
when you’re dedicated to kindness \\
and the end of grasping; 

when\marginnote{19.1} you’re dedicated to the ending of craving \\
and clarity of heart; \\
and you’ve seen the arising of the senses, \\
your mind is rightly freed. 

For\marginnote{20.1} that one, rightly freed, \\
a mendicant with peaceful mind, \\
there’s nothing to be improved, \\
and nothing more to do. 

As\marginnote{21.1} the wind cannot stir \\
a solid mass of rock, \\
so too sights, tastes, sounds, \\
smells, and touches—the lot—

and\marginnote{22.1} thoughts, whether liked or disliked, \\
don’t disturb the poised one. \\
Their mind is steady and free \\
as they observe disappearance.” 

%
\end{verse}

%
\section*{{\suttatitleacronym AN 6.56}{\suttatitletranslation With Phagguna }{\suttatitleroot Phaggunasutta}}
\addcontentsline{toc}{section}{\tocacronym{AN 6.56} \toctranslation{With Phagguna } \tocroot{Phaggunasutta}}
\markboth{With Phagguna }{Phaggunasutta}
\extramarks{AN 6.56}{AN 6.56}

Now\marginnote{1.1} at that time Venerable Phagguna was sick, suffering, gravely ill. Then Venerable Ānanda went up to the Buddha, bowed, sat down to one side, and said to him: 

“Sir,\marginnote{1.3} Venerable Phagguna is sick. Sir, please go to Venerable Phagguna out of compassion.” The Buddha consented in silence. 

Then\marginnote{1.6} in the late afternoon, the Buddha came out of retreat and went to Venerable Phagguna. Venerable Phagguna saw the Buddha coming off in the distance and tried to rise on his cot. 

The\marginnote{1.9} Buddha said to him, “It’s all right, Phagguna, don’t get up. There are some seats spread out by others, I will sit there.” 

He\marginnote{1.12} sat on the seat spread out and said to Venerable Phagguna: “I hope you’re keeping well, Phagguna; I hope you’re alright. And I hope the pain is fading, not growing, that its fading is evident, not its growing.” 

“Sir,\marginnote{2.2} I’m not keeping well, I’m not alright. The pain is terrible and growing, not fading; its growing is evident, not its fading. 

The\marginnote{3.1} winds piercing my head are so severe, it feels like a strong man drilling into my head with a sharp point. I’m not keeping well. 

The\marginnote{4.1} pain in my head is so severe, it feels like a strong man tightening a tough leather strap around my head. I’m not keeping well. 

The\marginnote{5.1} winds slicing my belly are so severe, like a deft butcher or their apprentice were slicing open a cows’s belly open with a meat cleaver. I’m not keeping well. 

The\marginnote{6.1} burning in my body is so severe, it feels like two strong men grabbing a weaker man by the arms to burn and scorch him on a pit of glowing coals. I’m not keeping well, I’m not alright. The pain is terrible and growing, not fading; its growing is evident, not its fading.” 

Then\marginnote{6.3} the Buddha educated, encouraged, fired up, and inspired Venerable Phagguna with a Dhamma talk, after which he got up from his seat and left. 

Not\marginnote{7.1} long after the Buddha left, Venerable Phagguna passed away. At the time of his death, his faculties were bright and clear. Then Venerable Ānanda went up to the Buddha, bowed, sat down to one side, and said to him, “Sir, soon after the Buddha left, Venerable Phagguna died. At the time of his death, his faculties were bright and clear.” 

“And\marginnote{8.1} why shouldn’t his faculties be bright and clear? The mendicant Phagguna’s mind was not freed from the five lower fetters. But when he heard that teaching his mind was freed from them. 

Ānanda,\marginnote{9.1} there are these six benefits to hearing the teaching at the right time and examining the meaning at the right time. What six? 

Firstly,\marginnote{9.3} take the case of a mendicant whose mind is not freed from the five lower fetters. At the time of death they get to see the Realized One. The Realized One teaches them Dhamma that’s good in the beginning, good in the middle, and good in the end, meaningful and well-phrased. And he reveals a spiritual practice that’s entirely full and pure. When they hear that teaching their mind is freed from the five lower fetters. This is the first benefit of listening to the teaching. 

Next,\marginnote{10.1} take the case of another mendicant whose mind is not freed from the five lower fetters. At the time of death they don’t get to see the Realized One, but they get to see a Realized One’s disciple. The Realized One’s disciple teaches them Dhamma … When they hear that teaching their mind is freed from the five lower fetters. This is the second benefit of listening to the teaching. 

Next,\marginnote{11.1} take the case of another mendicant whose mind is not freed from the five lower fetters. At the time of death they don’t get to see the Realized One, or to see a Realized One’s disciple. But they think about and consider the teaching in their heart, examining it with the mind as they learned and memorized it. As they do so their mind is freed from the five lower fetters. This is the third benefit of listening to the teaching. 

Next,\marginnote{12.1} take the case of a mendicant whose mind is freed from the five lower fetters, but not with the supreme ending of attachments. At the time of death they get to see the Realized One. The Realized One teaches them Dhamma … When they hear that teaching their mind is freed with the supreme ending of attachments. This is the fourth benefit of listening to the teaching. 

Next,\marginnote{13.1} take the case of another mendicant whose mind is freed from the five lower fetters, but not with the supreme ending of attachments. At the time of death they don’t get to see the Realized One, but they get to see a Realized One’s disciple. The Realized One’s disciple teaches them Dhamma … When they hear that teaching their mind is freed with the supreme ending of attachments. This is the fifth benefit of listening to the teaching. 

Next,\marginnote{14.1} take the case of another mendicant whose mind is freed from the five lower fetters, but not with the supreme ending of attachments. At the time of death they don’t get to see the Realized One, or to see a Realized One’s disciple. But they think about and consider the teaching in their heart, examining it with the mind as they learned and memorized it. As they do so their mind is freed with the supreme ending of attachments. This is the sixth benefit of listening to the teaching. 

These\marginnote{15.1} are the six benefits to hearing the teaching at the right time and examining the meaning at the right time.” 

%
\section*{{\suttatitleacronym AN 6.57}{\suttatitletranslation The Six Classes of Rebirth }{\suttatitleroot Chaḷabhijātisutta}}
\addcontentsline{toc}{section}{\tocacronym{AN 6.57} \toctranslation{The Six Classes of Rebirth } \tocroot{Chaḷabhijātisutta}}
\markboth{The Six Classes of Rebirth }{Chaḷabhijātisutta}
\extramarks{AN 6.57}{AN 6.57}

At\marginnote{1.1} one time the Buddha was staying near \textsanskrit{Rājagaha}, on the Vulture’s Peak Mountain. Then Venerable Ānanda went up to the Buddha, bowed, sat down to one side, and said to him: 

“Sir,\marginnote{1.3} \textsanskrit{Pūraṇa} Kassapa describes six classes of rebirth: black, blue, red, yellow, white, and ultimate white. 

The\marginnote{2.1} black class of rebirth consists of slaughterers of sheep, pigs, poultry, or deer, hunters or fishers, bandits, executioners, butchers of cattle, jailers, and any others with a cruel livelihood. 

The\marginnote{3.1} blue class of rebirth consists of mendicants who live on thorns, and any others who teach the efficacy of deeds and action. 

The\marginnote{4.1} red class of rebirth consists of the Jain ascetics who wear one cloth. 

The\marginnote{5.1} yellow class of rebirth consists of the lay people dressed in white who are disciples of the naked ascetics. 

The\marginnote{6.1} white class of rebirth consists of male and female \textsanskrit{Ājīvaka} ascetics. 

And\marginnote{7.1} the ultimate white class of rebirth consists of Nanda Vaccha, Kisa \textsanskrit{Saṅkicca}, and Makkhali \textsanskrit{Gosāla}. 

These\marginnote{8.1} are the six classes of rebirth that \textsanskrit{Pūraṇa} Kassapa describes.” 

“But\marginnote{9.1} Ānanda, did the whole world authorize \textsanskrit{Pūraṇa} Kassapa to describe these six classes of rebirth?” 

“No,\marginnote{9.2} sir.” 

“It’s\marginnote{9.3} as if they were to force a steak on a poor, penniless person, telling them they must eat it and then pay for it. In the same way, \textsanskrit{Pūraṇa} Kassapa has described these six classes of rebirth without the consent of those ascetics and brahmins. And he has done so in a foolish, incompetent, unskilled way, lacking common sense. 

I,\marginnote{10.1} however, also describe six classes of rebirth. Listen and pay close attention, I will speak.” 

“Yes,\marginnote{10.3} sir,” Ānanda replied. The Buddha said this: 

“And\marginnote{10.5} what, Ānanda, are the six classes of rebirth? Someone born into a dark class gives rise to a dark result. Someone born into a dark class gives rise to a bright result. Someone born into a dark class gives rise to extinguishment, which is neither dark nor bright. Someone born into a bright class gives rise to a dark result. Someone born into a bright class gives rise to a bright result. Someone born into a bright class gives rise to extinguishment, which is neither dark nor bright. 

And\marginnote{11.1} how does someone born into a dark class give rise to a dark result? It’s when someone is reborn in a low family—a family of outcastes, bamboo-workers, hunters, chariot-makers, or waste-collectors—poor, with little to eat or drink, where life is tough, and food and shelter are hard to find. And they’re ugly, unsightly, deformed, chronically ill—one-eyed, crippled, lame, or half-paralyzed. They don’t get to have food, drink, clothes, and vehicles; garlands, fragrance, and makeup; or bed, house, and lighting. And they do bad things by way of body, speech, and mind. When their body breaks up, after death, they’re reborn in a place of loss, a bad place, the underworld, hell. That’s how someone born into a dark class gives rise to a dark result. 

And\marginnote{12.1} how does someone born into a dark class give rise to a bright result? It’s when some person is reborn in a low family … But they do good things by way of body, speech, and mind. When their body breaks up, after death, they’re reborn in a good place, a heavenly realm. That’s how someone born into a dark class gives rise to a bright result. 

And\marginnote{13.1} how does someone born into a dark class give rise to extinguishment, which is neither dark nor bright? It’s when some person is reborn in a low family … They shave off their hair and beard, dress in ocher robes, and go forth from the lay life to homelessness. They give up the five hindrances, corruptions of the heart that weaken wisdom. They firmly establish their mind in the four kinds of mindfulness meditation. They truly develop the seven awakening factors. And then they give rise to extinguishment, which is neither dark nor bright. That’s how someone born in a dark class gives rise to extinguishment, which is neither dark nor bright. 

And\marginnote{14.1} how does someone born into a bright class give rise to a dark result? It’s when some person is reborn in an eminent family—a well-to-do family of aristocrats, brahmins, or householders—rich, affluent, and wealthy, with lots of gold and silver, lots of property and assets, and lots of money and grain. And they’re attractive, good-looking, lovely, of surpassing beauty. They get to have food, drink, clothes, and vehicles; garlands, fragrance, and makeup; and bed, house, and lighting. But they do bad things by way of body, speech, and mind. When their body breaks up, after death, they’re reborn in a place of loss, a bad place, the underworld, hell. That’s how someone born into a bright class gives rise to a dark result. 

And\marginnote{15.1} how does someone born into a bright class give rise to a bright result? It’s when some person is reborn in an eminent family … And they do good things by way of body, speech, and mind. When their body breaks up, after death, they’re reborn in a good place, a heavenly realm. That’s how someone born into a bright class give rise to a bright result. 

And\marginnote{16.1} how does someone born into a bright class give rise to extinguishment, which is neither dark nor bright? It’s when some person is reborn in an eminent family … They shave off their hair and beard, dress in ocher robes, and go forth from the lay life to homelessness. They give up the five hindrances, corruptions of the heart that weaken wisdom. They firmly establish their mind in the four kinds of mindfulness meditation. They truly develop the seven awakening factors. And then they give rise to extinguishment, which is neither dark nor bright. That’s how someone born into a bright class gives rise to extinguishment, which is neither dark nor bright. 

These\marginnote{17.1} are the six classes of rebirth.” 

%
\section*{{\suttatitleacronym AN 6.58}{\suttatitletranslation Defilements }{\suttatitleroot Āsavasutta}}
\addcontentsline{toc}{section}{\tocacronym{AN 6.58} \toctranslation{Defilements } \tocroot{Āsavasutta}}
\markboth{Defilements }{Āsavasutta}
\extramarks{AN 6.58}{AN 6.58}

“Mendicants,\marginnote{1.1} a mendicant with six qualities is worthy of offerings dedicated to the gods, worthy of hospitality, worthy of a religious donation, worthy of veneration with joined palms, and is the supreme field of merit for the world. What six? 

It’s\marginnote{2.2} a mendicant who, by restraint, has given up the defilements that should be given up by restraint. By using, they’ve given up the defilements that should be given up by using. By enduring, they’ve given up the defilements that should be given up by enduring. By avoiding, they’ve given up the defilements that should be given up by avoiding. By getting rid, they’ve given up the defilements that should be given up by getting rid. By developing, they’ve given up the defilements that should be given up by developing. 

And\marginnote{3.1} what are the defilements that should be given up by restraint? Take a mendicant who, reflecting properly, lives restraining the eye faculty. For the distressing and feverish defilements that might arise in someone who lives without restraint of the eye faculty do not arise when there is such restraint. Reflecting properly, they live restraining the ear faculty … the nose faculty … the tongue faculty … the body faculty … the mind faculty. For the distressing and feverish defilements that might arise in someone who lives without restraint of the mind faculty do not arise when there is such restraint. These are called the defilements that should be given up by restraint. 

And\marginnote{4.1} what are the defilements that should be given up by using? Take a mendicant who, reflecting properly, makes use of robes: ‘Only for the sake of warding off cold and heat; for warding off the touch of flies, mosquitoes, wind, sun, and reptiles; and for covering up the private parts.’ Reflecting properly, they make use of almsfood: ‘Not for fun, indulgence, adornment, or decoration, but only to sustain this body, to avoid harm, and to support spiritual practice. In this way, I shall put an end to old discomfort and not give rise to new discomfort, and I will live blamelessly and at ease.’ Reflecting properly, they make use of lodgings: ‘Only for the sake of warding off cold and heat; for warding off the touch of flies, mosquitoes, wind, sun, and reptiles; to shelter from harsh weather and to enjoy retreat.’ Reflecting properly, they make use of medicines and supplies for the sick: ‘Only for the sake of warding off the pains of illness and to promote good health.’ For the distressing and feverish defilements that might arise in someone who lives without using these things do not arise when they are used. These are called the defilements that should be given up by using. 

And\marginnote{5.1} what are the defilements that should be given up by enduring? Take a mendicant who, reflecting properly, endures cold, heat, hunger, and thirst. They endure the touch of flies, mosquitoes, wind, sun, and reptiles. They endure rude and unwelcome criticism. And they put up with physical pain—sharp, severe, acute, unpleasant, disagreeable, and life-threatening. For the distressing and feverish defilements that might arise in someone who lives without enduring these things do not arise when they are endured. These are called the defilements that should be given up by enduring. 

And\marginnote{6.1} what are the defilements that should be given up by avoiding? Take a mendicant who, reflecting properly, avoids a wild elephant, a wild horse, a wild ox, a wild dog, a snake, a stump, thorny ground, a pit, a cliff, a swamp, and a sewer. Reflecting properly, they avoid sitting on inappropriate seats, walking in inappropriate neighborhoods, and mixing with bad friends—whatever sensible spiritual companions would believe to be a bad setting. For the distressing and feverish defilements that might arise in someone who lives without avoiding these things do not arise when they are avoided. These are called the defilements that should be given up by avoiding. 

And\marginnote{7.1} what are the defilements that should be given up by getting rid? Take a mendicant who, reflecting properly, doesn’t tolerate a sensual, malicious, or cruel thought that has arisen. They don’t tolerate any bad, unskillful qualities that have arisen, but give them up, get rid of them, eliminate them, and obliterate them. For the distressing and feverish defilements that might arise in someone who lives without getting rid of these things do not arise when they are gotten rid of. These are called the defilements that should be given up by getting rid. 

And\marginnote{8.1} what are the defilements that should be given up by developing? Take a mendicant who, reflecting properly, develops the awakening factors of mindfulness, investigation of principles, energy, rapture, tranquility, immersion, and equanimity, which rely on seclusion, fading away, and cessation, and ripen as letting go. For the distressing and feverish defilements that might arise in someone who lives without developing these things do not arise when they are developed. These are called the defilements that should be given up by developing. 

A\marginnote{9.1} mendicant with these six qualities is worthy of offerings dedicated to the gods, worthy of hospitality, worthy of a religious donation, worthy of veneration with joined palms, and is the supreme field of merit for the world.” 

%
\section*{{\suttatitleacronym AN 6.59}{\suttatitletranslation With Dārukammika }{\suttatitleroot Dārukammikasutta}}
\addcontentsline{toc}{section}{\tocacronym{AN 6.59} \toctranslation{With Dārukammika } \tocroot{Dārukammikasutta}}
\markboth{With Dārukammika }{Dārukammikasutta}
\extramarks{AN 6.59}{AN 6.59}

\scevam{So\marginnote{1.1} I have heard. }At one time the Buddha was staying at \textsanskrit{Nādika} in the brick house. 

Then\marginnote{1.3} the householder \textsanskrit{Dārukammika} went up to the Buddha, bowed, and sat down to one side. The Buddha said to him, “Householder, I wonder whether your family gives gifts?” 

“It\marginnote{1.5} does, sir. Gifts are given to those mendicants who are perfected or on the path to perfection; they live in the wilderness, eat only almsfood, and wear rag robes.” 

“Householder,\marginnote{2.1} as a layman enjoying sensual pleasures, living at home with your children, using sandalwood imported from \textsanskrit{Kāsi}, wearing garlands, fragrance, and makeup, and accepting gold and money, it’s hard for you to know who is perfected or on the path to perfection. 

If\marginnote{3.1} a mendicant living in the wilderness is restless, insolent, fickle, scurrilous, loose-tongued, unmindful, lacking situational awareness and immersion, with straying mind and undisciplined faculties, then in this respect they’re reprehensible. If a mendicant living in the wilderness is not restless, insolent, fickle, scurrilous, or loose-tongued, but has established mindfulness, situational awareness and immersion, with unified mind and restrained faculties, then in this respect they’re praiseworthy. 

If\marginnote{4.1} a mendicant who lives within a village is restless … then in this respect they’re reprehensible. If a mendicant who lives within a village is not restless … then in this respect they’re praiseworthy. 

If\marginnote{5.1} a mendicant who eats only almsfood is restless … then in this respect they’re reprehensible. If a mendicant who eats only almsfood is not restless … then in this respect they’re praiseworthy. 

If\marginnote{6.1} a mendicant who accepts invitations is restless … then in this respect they’re reprehensible. If a mendicant who accepts invitations is not restless … then in this respect they’re praiseworthy. 

If\marginnote{7.1} a mendicant who wears rag robes is restless … then in this respect they’re reprehensible. If a mendicant who wears rag robes is not restless … then in this respect they’re praiseworthy. 

If\marginnote{8.1} a mendicant who wears robes offered by householders is restless, insolent, fickle, scurrilous, loose-tongued, unmindful, lacking situational awareness and immersion, with straying mind and undisciplined faculties, then in this respect they’re reprehensible. If a mendicant who wears robes offered by householders is not restless, insolent, fickle, scurrilous, or loose-tongued, but has established mindfulness, situational awareness and immersion, with unified mind and restrained faculties, then in this respect they’re praiseworthy. 

Go\marginnote{9.1} ahead, householder, give gifts to the \textsanskrit{Saṅgha}. Your mind will become bright and clear, and when your body breaks up, after death, you’ll be reborn in a good place, a heavenly realm.” 

“Sir,\marginnote{9.4} from this day forth I will give gifts to the \textsanskrit{Saṅgha}.” 

%
\section*{{\suttatitleacronym AN 6.60}{\suttatitletranslation With Hatthisāriputta }{\suttatitleroot Hatthisāriputtasutta}}
\addcontentsline{toc}{section}{\tocacronym{AN 6.60} \toctranslation{With Hatthisāriputta } \tocroot{Hatthisāriputtasutta}}
\markboth{With Hatthisāriputta }{Hatthisāriputtasutta}
\extramarks{AN 6.60}{AN 6.60}

\scevam{So\marginnote{1.1} I have heard. }At one time the Buddha was staying near Benares, in the deer park at Isipatana. 

Now\marginnote{1.3} at that time several senior mendicants, after the meal, on their return from almsround, sat together in the pavilion talking about the teachings. Venerable Citta \textsanskrit{Hatthisāriputta} interrupted them while they were talking. 

Then\marginnote{1.5} Venerable \textsanskrit{Mahākoṭṭhita} said to Venerable Citta \textsanskrit{Hatthisāriputta}, “Venerable, please don’t interrupt the senior mendicants while they’re talking about the teachings. Wait until the end of the discussion.” 

When\marginnote{1.7} he said this, Citta \textsanskrit{Hatthisāriputta}’s companions said to \textsanskrit{Mahākoṭṭhita}, “Venerable, please don’t rebuke Citta \textsanskrit{Hatthisāriputta}. He is astute, and quite capable of talking about the teachings with the senior mendicants.” 

“It’s\marginnote{2.1} not easy to know this, reverends, for those who don’t comprehend another’s mind. 

Take\marginnote{2.2} a person who is the sweetest of the sweet, the most even-tempered of the even-tempered, the calmest of the calm, so long as they live relying on the Teacher or a spiritual companion in a teacher’s role. But when they’re separated from the Teacher or a spiritual companion in a teacher’s role, they mix closely with monks, nuns, laymen, and laywomen; with rulers and their ministers, and with teachers of other paths and their followers. As they mix closely, they become intimate and loose, spending time chatting, and so lust infects their mind. They resign the training and return to a lesser life. 

Suppose\marginnote{3.1} an ox fond of crops was tied up or shut in a pen. Would it be right to say that that ox will never again invade the crops?” 

“No\marginnote{3.3} it would not, reverend. For it’s quite possible that that ox will snap the ropes or break out of the pen, and then invade the crops.” 

“In\marginnote{3.5} the same way, take a person who is the sweetest of the sweet … As they mix closely, they become intimate and loose, spending time chatting, and so lust infects their mind. They resign the training and return to a lesser life. 

Take\marginnote{4.1} the case of a person who, quite secluded from sensual pleasures … enters and remains in the first absorption. Thinking, ‘I get the first absorption!’ they mix closely with monks … They resign the training and return to a lesser life. 

Suppose\marginnote{4.4} it was raining heavily at the crossroads so that the dust vanished and mud appeared. Would it be right to say that now dust will never appear at this crossroad again?” 

“No\marginnote{4.6} it would not, reverend. For it is quite possible that people or cattle and so on will cross over the crossroad, or that the wind and sun will evaporate the moisture so that the dust appears again.” 

“In\marginnote{4.8} the same way, take the case of a person who, quite secluded from sensual pleasures … enters and remains in the first absorption. Thinking, ‘I get the first absorption!’ they mix closely with monks … They resign the training and return to a lesser life. 

Take\marginnote{5.1} another case of a person who, as the placing of the mind and keeping it connected are stilled … enters and remains in the second absorption. Thinking, ‘I get the second absorption!’ they mix closely with monks … They resign the training and return to a lesser life. 

Suppose\marginnote{5.4} there was a large pond not far from a town or village. After it rained heavily there the clams and mussels, and pebbles and gravel would vanish. Would it be right to say that now the clams and mussels, and pebbles and gravel will never appear here again?” 

“No\marginnote{5.7} it would not, reverend. For it’s quite possible that people or cattle and so on will drink from the pond, or that the wind and sun will evaporate it so that the clams and mussels, and pebbles and gravel appear again.” 

“In\marginnote{5.9} the same way, take another case of a person who, as the placing of the mind and keeping it connected are stilled … enters and remains in the second absorption. Thinking, ‘I get the second absorption!’ they mix closely with monks … They resign the training and return to a lesser life. 

Take\marginnote{6.1} the case of another person who, with the fading away of rapture … enters and remains in the third absorption. Thinking, ‘I get the third absorption!’ they mix closely with monks … They resign the training and return to a lesser life. 

Suppose\marginnote{6.4} a person had finished a delicious meal. They’d have no appetite for leftovers. Would it be right to say that now food will never appeal to this person again?” 

“No\marginnote{6.6} it would not, reverend. For it’s quite possible that other food won’t appeal to that person as long as the nourishment is still present. But when the nourishment vanishes food will appeal again.” 

“In\marginnote{6.9} the same way, take the case of a person who, with the fading away of rapture … enters and remains in the third absorption. Thinking, ‘I get the third absorption!’ they mix closely with monks … They resign the training and return to a lesser life. 

Take\marginnote{7.1} the case of another person who, giving up pleasure and pain … enters and remains in the fourth absorption. Thinking, ‘I get the fourth absorption!’ they mix closely with monks … They resign the training and return to a lesser life. 

Suppose\marginnote{7.4} that in a mountain glen there was a lake, unruffled and free of waves. Would it be right to say that now waves will never appear in this lake again?” 

“No\marginnote{7.6} it would not, reverend. For it is quite possible that a violent storm could blow up out of the east, west, north, or south, and stir up waves in that lake.” 

“In\marginnote{7.8} the same way, take the case of a person who, giving up pleasure and pain … enters and remains in the fourth absorption. Thinking, ‘I get the fourth absorption!’ they mix closely with monks … They resign the training and return to a lesser life. 

Take\marginnote{8.1} the case of another person who, not focusing on any signs, enters and remains in the signless immersion of the heart. Thinking, ‘I get the signless immersion of the heart!’ they mix closely with monks, nuns, laymen, and laywomen; with rulers and their ministers, and with teachers of other paths and their followers. As they mix closely, they become intimate and loose, spending time chatting, and so lust infects their mind. They resign the training and return to a lesser life. 

Suppose\marginnote{8.5} a ruler or their minister, while walking along the road with an army of four divisions, was to arrive at a forest grove where they set up camp for the night. There, because of the noise of the elephants, horses, chariots, soldiers, and the drums, kettledrums, horns, and cymbals, the chirping of crickets would vanish. Would it be right to say that now the chirping of crickets will never be heard in this woodland grove again?” 

“No\marginnote{8.8} it would not, reverend. For it is quite possible that the ruler or their minister will depart from that woodland grove so that the chirping of crickets will be heard there again.” 

“In\marginnote{8.10} the same way, take the case of a person who, not focusing on any signs, enters and remains in the signless immersion of the heart … They resign the training and return to a lesser life.” 

Then\marginnote{9.1} after some time Venerable Citta \textsanskrit{Hatthisāriputta} resigned the training and returned to a lesser life. Then the mendicants who were his companions went up to Venerable \textsanskrit{Mahākoṭṭhita} and said, “Did Venerable \textsanskrit{Mahākoṭṭhita} comprehend Citta \textsanskrit{Hatthisāriputta}’s mind and know that he had gained such and such meditative attainments, yet he would still resign the training and return to a lesser life? Or did deities tell you about it?” 

“Reverends,\marginnote{9.7} I comprehended his mind and knew this. And deities also told me.” 

Then\marginnote{10.1} the mendicants who were Citta \textsanskrit{Hatthisāriputta}’s companions went up to the Buddha, bowed, sat down to one side, and said to him, “Sir, Citta \textsanskrit{Hatthisāriputta}, who had gained such and such meditative attainments, has still resigned the training and returned to a lesser life.” 

“Mendicants,\marginnote{10.3} soon Citta will remember renunciation.” 

And\marginnote{11.1} not long after Citta \textsanskrit{Hatthisāriputta} shaved off his hair and beard, dressed in ocher robes, and went forth from the lay life to homelessness. Then Citta \textsanskrit{Hatthisāriputta}, living alone, withdrawn, diligent, keen, and resolute, soon realized the supreme culmination of the spiritual path in this very life. He lived having achieved with his own insight the goal for which gentlemen rightly go forth from the lay life to homelessness. 

He\marginnote{11.3} understood: “Rebirth is ended; the spiritual journey has been completed; what had to be done has been done; there is no return to any state of existence.” And Venerable Citta \textsanskrit{Hatthisāriputta} became one of the perfected. 

%
\section*{{\suttatitleacronym AN 6.61}{\suttatitletranslation In the Middle }{\suttatitleroot Majjhesutta}}
\addcontentsline{toc}{section}{\tocacronym{AN 6.61} \toctranslation{In the Middle } \tocroot{Majjhesutta}}
\markboth{In the Middle }{Majjhesutta}
\extramarks{AN 6.61}{AN 6.61}

\scevam{So\marginnote{1.1} I have heard. }At one time the Buddha was staying near Benares, in the deer park at Isipatana. 

Now\marginnote{1.3} at that time, after the meal, on return from almsround, several senior mendicants sat together in the pavilion and this discussion came up among them, “Reverends, this was said by the Buddha in ‘The Way to the Far Shore’, in ‘The Questions of Metteyya’: 

\begin{verse}%
‘The\marginnote{2.1} thoughtful one who has known both ends, \\
and is not stuck in the middle: \\
he is a great man, I declare, \\
he has escaped the seamstress here.’ 

%
\end{verse}

But\marginnote{3.1} what is one end? What’s the second end? What’s the middle? And who is the seamstress?” When this was said, one of the mendicants said to the senior mendicants: 

“Contact,\marginnote{3.3} reverends, is one end. The origin of contact is the second end. The cessation of contact is the middle. And craving is the seamstress, for craving weaves one to being reborn in one state of existence or another. That’s how a mendicant directly knows what should be directly known and completely understands what should be completely understood. Knowing and understanding thus they make an end of suffering in this very life.” 

When\marginnote{4.1} this was said, one of the mendicants said to the senior mendicants: 

“The\marginnote{4.2} past, reverends, is one end. The future is the second end. The present is the middle. And craving is the seamstress … That’s how a mendicant directly knows … an end of suffering in this very life.” 

When\marginnote{5.1} this was said, one of the mendicants said to the senior mendicants: 

“Pleasant\marginnote{5.2} feeling, reverends, is one end. Painful feeling is the second end. Neutral feeling is the middle. And craving is the seamstress … That’s how a mendicant directly knows … an end of suffering in this very life.” 

When\marginnote{6.1} this was said, one of the mendicants said to the senior mendicants: 

“Name,\marginnote{6.2} reverends, is one end. Form is the second end. Consciousness is the middle. And craving is the seamstress … That’s how a mendicant directly knows … an end of suffering in this very life.” 

When\marginnote{7.1} this was said, one of the mendicants said to the senior mendicants: 

“The\marginnote{7.2} six interior sense fields, reverends, are one end. The six exterior sense fields are the second end. Consciousness is the middle. And craving is the seamstress … That’s how a mendicant directly knows … an end of suffering in this very life.” 

When\marginnote{8.1} this was said, one of the mendicants said to the senior mendicants: 

“Identity,\marginnote{8.2} reverends, is one end. The origin of identity is the second end. The cessation of identity is the middle. And craving is the seamstress, for craving weaves one to being reborn in one state of existence or another. That’s how a mendicant directly knows what should be directly known and completely understands what should be completely understood. Knowing and understanding thus they make an end of suffering in this very life.” 

When\marginnote{9.1} this was said, one of the mendicants said to the senior mendicants: 

“Each\marginnote{9.2} of us has spoken from the heart. Come, reverends, let’s go to the Buddha, and inform him about this. As he answers, so we’ll remember it.” 

“Yes,\marginnote{10.1} reverend,” those senior mendicants replied. Then those senior mendicants went up to the Buddha, bowed, sat down to one side, and informed the Buddha of all they had discussed. They asked, “Sir, who has spoken well?” 

“Mendicants,\marginnote{10.5} you’ve all spoken well in a way. However, this is what I was referring to in ‘The Way to the Far Shore’, in ‘The Questions of Metteyya’ when I said: 

\begin{verse}%
‘The\marginnote{11.1} sage has known both ends, \\
and is not stuck in the middle. \\
He is a great man, I declare, \\
he has escaped the seamstress here.’ 

%
\end{verse}

Listen\marginnote{12.1} and pay close attention, I will speak.” 

“Yes,\marginnote{12.2} sir,” they replied. The Buddha said this: 

“Contact,\marginnote{12.4} mendicants, is one end. The origin of contact is the second end. The cessation of contact is the middle. And craving is the seamstress, for craving weaves one to being reborn in one state of existence or another. That’s how a mendicant directly knows what should be directly known and completely understands what should be completely understood. Knowing and understanding thus they make an end of suffering in this very life.” 

%
\section*{{\suttatitleacronym AN 6.62}{\suttatitletranslation Knowledge of the Faculties of Persons }{\suttatitleroot Purisindriyañāṇasutta}}
\addcontentsline{toc}{section}{\tocacronym{AN 6.62} \toctranslation{Knowledge of the Faculties of Persons } \tocroot{Purisindriyañāṇasutta}}
\markboth{Knowledge of the Faculties of Persons }{Purisindriyañāṇasutta}
\extramarks{AN 6.62}{AN 6.62}

\scevam{So\marginnote{1.1} I have heard. }At one time the Buddha was wandering in the land of the Kosalans together with a large \textsanskrit{Saṅgha} of mendicants when he arrived at a town of the Kosalans named \textsanskrit{Daṇḍakappaka}. The Buddha left the road and sat at the root of a tree on the seat spread out. The mendicants entered \textsanskrit{Daṇḍakappaka} to look for a guest house. 

Then\marginnote{2.1} Venerable Ānanda together with several mendicants went to the Aciravati River to bathe. When he had bathed and emerged from the water he stood in one robe drying himself. 

Then\marginnote{2.3} a certain mendicant went up to Venerable Ānanda, and said to him, “Reverend Ānanda, when the Buddha declared that Devadatta was going to a place of loss, to hell, there to remain for an eon, irredeemable, did he do so after wholeheartedly deliberating, or was this just a way of speaking?” 

“You’re\marginnote{2.6} right, reverend, that’s how the Buddha declared it.” 

Then\marginnote{3.1} Venerable Ānanda went up to the Buddha, bowed, sat down to one side, and told him what had happened. 

“Ānanda,\marginnote{4.1} that mendicant must be junior, recently gone forth, or else a foolish, incompetent senior mendicant. How on earth can he take something that I have declared definitively to be ambiguous? I do not see a single other person about whom I have given such whole-hearted deliberation before making a declaration as I did in the case of Devadatta. 

As\marginnote{4.4} long as I saw even a fraction of a hair’s tip of goodness in Devadatta I did not declare that he was going to a place of loss, to hell, there to remain for an eon, irredeemable. But when I saw that there was not even a fraction of a hair’s tip of goodness in Devadatta I declared that he was going to a place of loss, to hell, there to remain for an eon, irredeemable. 

Suppose\marginnote{5.1} there was a sewer deeper than a man’s height, full to the brim with feces, and someone was sunk into it over their head. Then along comes a person who wants to help make them safe, who wants to lift them out of that sewer. But circling all around the sewer they couldn’t see even a fraction of a hair’s tip of that person that was not smeared with feces. 

In\marginnote{5.5} the same way, when I saw that there was not even a fraction of a hair’s tip of goodness in Devadatta I declared that he was going to a place of loss, to hell, there to remain for an eon, irredeemable. Ānanda, if only you would all listen to the Realized One’s analysis of the knowledges of the faculties of individuals.” 

“Now\marginnote{6.1} is the time, Blessed One! Now is the time, Holy One! Let the Buddha analyze the faculties of persons. The mendicants will listen and remember it.” 

“Well\marginnote{6.3} then, Ānanda, listen and pay close attention, I will speak.” 

“Yes,\marginnote{6.4} sir,” Ānanda replied. The Buddha said this: 

“Ānanda,\marginnote{7.1} when I’ve comprehended the mind of a person, I understand: ‘Both skillful and unskillful qualities are found in this person.’ After some time I comprehend their mind and understand: ‘The skillful qualities of this person have vanished, but the unskillful qualities are still present. Nevertheless, their skillful root is unbroken, and from that the skillful will appear. So this person is not liable to decline in the future.’ Suppose some seeds were intact, unspoiled, not weather-damaged, fertile, and well-kept. They’re sown in a well-prepared, productive field. Wouldn’t you know that those seeds would grow, increase, and mature?” 

“Yes,\marginnote{7.9} sir.” 

“In\marginnote{7.10} the same way, when I’ve comprehended the mind of a person, I understand … This person is not liable to decline in the future … This is how another individual is known to the Realized One by comprehending their mind. And this is how the Realized One knows a person’s faculties by comprehending their mind. And this is how the Realized One knows the future origination of a person’s qualities by comprehending their mind. 

When\marginnote{8.1} I’ve comprehended the mind of a person, I understand: ‘Both skillful and unskillful qualities are found in this person.’ After some time I comprehend their mind and understand: ‘The unskillful qualities of this person have vanished, but the skillful qualities are still present. Nevertheless, their unskillful root is unbroken, and from that the unskillful will appear. So this person is still liable to decline in the future.’ Suppose some seeds were intact, unspoiled, not weather-damaged, fertile, and well-kept. And they were sown on a broad rock. Wouldn’t you know that those seeds would not grow, increase, and mature?” 

“Yes,\marginnote{8.9} sir.” 

“In\marginnote{8.10} the same way, when I’ve comprehended the mind of a person, I understand … This person is still liable to decline in the future … This is how another individual is known to the Realized One … 

When\marginnote{9.1} I’ve comprehended the mind of a person, I understand: ‘Both skillful and unskillful qualities are found in this person.’ After some time I comprehend their mind and understand: ‘This person has not even a fraction of a hair’s tip of goodness. They have exclusively dark, unskillful qualities. When their body breaks up, after death, they will be reborn in a place of loss, a bad place, the underworld, hell.’ Suppose some seeds were broken, spoiled, weather-damaged. They’re sown in a well-prepared, productive field. Wouldn’t you know that those seeds would not grow, increase, and mature?” 

“Yes,\marginnote{9.7} sir.” 

“In\marginnote{9.8} the same way, when I’ve comprehended the mind of a person, I understand … ‘This person has not even a fraction of a hair’s tip of goodness. They have exclusively dark, unskillful qualities. When their body breaks up, after death, they will be reborn in a place of loss, a bad place, the underworld, hell.’ …” 

When\marginnote{10.1} he said this, Venerable Ānanda said to the Buddha, “Sir, can you describe three other persons who are counterparts of these three?” 

“I\marginnote{10.3} can, Ānanda,” said the Buddha. “Ānanda, when I’ve comprehended the mind of a person, I understand: ‘Both skillful and unskillful qualities are found in this person.’ After some time I comprehend their mind and understand: ‘The skillful qualities of this person have vanished, but the unskillful qualities are still present. Nevertheless, their skillful root is unbroken, but it’s about to be totally destroyed. So this person is still liable to decline in the future.’ Suppose that there were some burning coals, blazing and glowing. And they were placed on a broad rock. Wouldn’t you know that those coals would not grow, increase, and spread?” 

“Yes,\marginnote{10.12} sir.” 

“Or\marginnote{10.13} suppose it was the late afternoon and the sun was going down. Wouldn’t you know that the light was about to vanish and darkness appear?” 

“Yes,\marginnote{10.14} sir.” 

“Or\marginnote{10.15} suppose that it’s nearly time for the midnight meal. Wouldn’t you know that the light had vanished and the darkness appeared?” 

“Yes,\marginnote{10.16} sir.” 

“In\marginnote{10.17} the same way, when I’ve comprehended the mind of a person, I understand … This person is still liable to decline in the future … 

When\marginnote{11.1} I’ve comprehended the mind of a person, I understand: ‘Both skillful and unskillful qualities are found in this person.’ After some time I comprehend their mind and understand: ‘The unskillful qualities of this person have vanished, but the skillful qualities are still present. Nevertheless, their unskillful root is unbroken, but it’s about to be totally destroyed. So this person is not liable to decline in the future.’ Suppose that there were some burning coals, blazing and glowing. They were placed on a pile of grass or timber. Wouldn’t you know that those coals would grow, increase, and spread?” 

“Yes,\marginnote{11.9} sir.” 

“Suppose\marginnote{11.10} it’s the crack of dawn and the sun is rising. Wouldn’t you know that the dark will vanish and the light appear?” 

“Yes,\marginnote{11.11} sir.” 

“Or\marginnote{11.12} suppose that it’s nearly time for the midday meal. Wouldn’t you know that the dark had vanished and the light appeared?” 

“Yes,\marginnote{11.13} sir.” 

“In\marginnote{11.14} the same way, when I’ve comprehended the mind of a person, I understand … This person is not liable to decline in the future … 

When\marginnote{12.1} I’ve comprehended the mind of a person, I understand: ‘Both skillful and unskillful qualities are found in this person.’ After some time I comprehend their mind and understand: ‘This person has not even a fraction of a hair’s tip of unskillful qualities. They have exclusively bright, blameless qualities. They will become extinguished in this very life.’ Suppose that there were some cool, extinguished coals. They were placed on a pile of grass or timber. Wouldn’t you know that those coals would not grow, increase, and spread?” 

“Yes,\marginnote{12.7} sir.” 

“In\marginnote{12.8} the same way, when I’ve comprehended the mind of a person, I understand … ‘This person has not even a fraction of a hair’s tip of unskillful qualities. They have exclusively bright, blameless qualities. They will become extinguished in this very life.’ This is how another individual is known to the Realized One by comprehending their mind. And this is how the Realized One knows a person’s faculties by comprehending their mind. And this is how the Realized One knows the future origination of a person’s qualities by comprehending their mind. 

And\marginnote{13.1} so, Ānanda, of the first three people one is not liable to decline, one is liable to decline, and one is bound for a place of loss, hell. And of the second three people, one is liable to decline, one is not liable to decline, and one is bound to become extinguished.” 

%
\section*{{\suttatitleacronym AN 6.63}{\suttatitletranslation Penetrative }{\suttatitleroot Nibbedhikasutta}}
\addcontentsline{toc}{section}{\tocacronym{AN 6.63} \toctranslation{Penetrative } \tocroot{Nibbedhikasutta}}
\markboth{Penetrative }{Nibbedhikasutta}
\extramarks{AN 6.63}{AN 6.63}

“Mendicants,\marginnote{1.1} I will teach you a penetrative exposition of the teaching. Listen and pay close attention, I will speak.” 

“Yes,\marginnote{1.3} sir,” they replied. The Buddha said this: 

“Mendicants,\marginnote{2.1} what is the penetrative exposition of the teaching? Sensual pleasures should be known. And their source, diversity, result, cessation, and the practice that leads to their cessation should be known. 

Feelings\marginnote{3.1} should be known. And their source, diversity, result, cessation, and the practice that leads to their cessation should be known. 

Perceptions\marginnote{4.1} should be known. And their source, diversity, result, cessation, and the practice that leads to their cessation should be known. 

Defilements\marginnote{5.1} should be known. And their source, diversity, result, cessation, and the practice that leads to their cessation should be known. 

Deeds\marginnote{6.1} should be known. And their source, diversity, result, cessation, and the practice that leads to their cessation should be known. 

Suffering\marginnote{7.1} should be known. And its source, diversity, result, cessation, and the practice that leads to its cessation should be known. 

‘Sensual\marginnote{8.1} pleasures should be known. And their source, diversity, result, cessation, and the practice that leads to their cessation should be known.’ That’s what I said, but why did I say it? There are these five kinds of sensual stimulation. Sights known by the eye that are likable, desirable, agreeable, pleasant, sensual, and arousing. Sounds known by the ear … Smells known by the nose … Tastes known by the tongue … Touches known by the body that are likable, desirable, agreeable, pleasant, sensual, and arousing. However, these are not sensual pleasures. In the training of the Noble One they’re called ‘kinds of sensual stimulation’. 

\begin{verse}%
Greedy\marginnote{8.10} intention is a person’s sensual pleasure. \\
The world’s pretty things aren’t sensual pleasures. \\
Greedy intention is a person’s sensual pleasure. \\
The world’s pretty things stay just as they are, \\
but a wise one removes desire for them. 

%
\end{verse}

And\marginnote{9.1} what is the source of sensual pleasures? Contact is their source. 

And\marginnote{11.1} what is the diversity of sensual pleasures? The sensual desire for sights, sounds, smells, tastes, and touches are all different. This is called the diversity of sensual pleasures. 

And\marginnote{12.1} what is the result of sensual pleasures? When one who desires sensual pleasures creates a corresponding life-form, with the attributes of either good or bad deeds—this is called the result of sensual pleasures. 

And\marginnote{13.1} what is the cessation of sensual pleasures? When contact ceases, sensual pleasures cease. The practice that leads to the cessation of sensual pleasures is simply this noble eightfold path, that is: right view, right thought, right speech, right action, right livelihood, right effort, right mindfulness, and right immersion. 

When\marginnote{14.1} a noble disciple understands sensual pleasures in this way—and understands their source, diversity, result, cessation, and the practice that leads to their cessation—they understand that this penetrative spiritual life is the cessation of sensual pleasures. ‘Sensual pleasures should be known. And their source, diversity, result, cessation, and the practice that leads to their cessation should be known.’ That’s what I said, and this is why I said it. 

‘Feelings\marginnote{15.1} should be known. And their source, diversity, result, cessation, and the practice that leads to their cessation should be known.’ That’s what I said, but why did I say it? There are these three feelings: pleasant, painful, and neutral. 

And\marginnote{16.1} what is the source of feelings? Contact is their source. 

And\marginnote{17.1} what is the diversity of feelings? There are material pleasant feelings, spiritual pleasant feelings, material painful feelings, spiritual painful feelings, material neutral feelings, and spiritual neutral feelings. This is called the diversity of feelings. 

And\marginnote{18.1} what is the result of feelings? When one who feels creates a corresponding life-form, with the attributes of either good or bad deeds—this is called the result of feelings. 

And\marginnote{19.1} what is the cessation of feelings? When contact ceases, feelings cease. The practice that leads to the cessation of feelings is simply this noble eightfold path, that is: right view, right thought, right speech, right action, right livelihood, right effort, right mindfulness, and right immersion. 

When\marginnote{20.1} a noble disciple understands feelings in this way … they understand that this penetrative spiritual life is the cessation of feelings. ‘Feelings should be known. And their source, diversity, result, cessation, and the practice that leads to their cessation should be known.’ That’s what I said, and this is why I said it. 

‘Perceptions\marginnote{21.1} should be known. And their source, diversity, result, cessation, and the practice that leads to their cessation should be known.’ That’s what I said, but why did I say it? There are these six perceptions: perceptions of sights, sounds, smells, tastes, touches, and thoughts. 

And\marginnote{22.1} what is the source of perceptions? Contact is their source. 

And\marginnote{23.1} what is the diversity of perceptions? The perceptions of sights, sounds, smells, tastes, touches, and thoughts are all different. This is called the diversity of perceptions. 

And\marginnote{24.1} what is the result of perceptions? Communication is the result of perception, I say. You communicate something in whatever manner you perceive it, saying ‘That’s what I perceived.’ This is called the result of perceptions. 

And\marginnote{25.1} what is the cessation of perception? When contact ceases, perception ceases. The practice that leads to the cessation of perceptions is simply this noble eightfold path, that is: right view, right thought, right speech, right action, right livelihood, right effort, right mindfulness, and right immersion. 

When\marginnote{26.1} a noble disciple understands perception in this way … they understand that this penetrative spiritual life is the cessation of perception. ‘Perceptions should be known. And their source, diversity, result, cessation, and the practice that leads to their cessation should be known.’ That’s what I said, and this is why I said it. 

‘Defilements\marginnote{27.1} should be known. And their source, diversity, result, cessation, and the practice that leads to their cessation should be known.’ That’s what I said, but why did I say it? There are these three defilements: the defilements of sensuality, desire to be reborn, and ignorance. 

And\marginnote{28.1} what is the source of defilements? Ignorance is the source of defilements. 

And\marginnote{29.1} what is the diversity of defilements? There are defilements that lead to rebirth in hell, the animal realm, the ghost realm, the human world, and the world of the gods. This is called the diversity of defilements. 

And\marginnote{30.1} what is the result of defilements? When one who is ignorant creates a corresponding life-form, with the attributes of either good or bad deeds—this is called the result of defilements. 

And\marginnote{31.1} what is the cessation of defilements? When ignorance ceases, defilements cease. The practice that leads to the cessation of defilements is simply this noble eightfold path, that is: right view, right thought, right speech, right action, right livelihood, right effort, right mindfulness, and right immersion. 

When\marginnote{32.1} a noble disciple understands defilements in this way … they understand that this penetrative spiritual life is the cessation of defilements. ‘Defilements should be known. And their source, diversity, result, cessation, and the practice that leads to their cessation should be known.’ That’s what I said, and this is why I said it. 

‘Deeds\marginnote{33.1} should be known. And their source, diversity, result, cessation, and the practice that leads to their cessation should be known.’ That’s what I said, but why did I say it? It is intention that I call deeds. For after making a choice one acts by way of body, speech, and mind. 

And\marginnote{34.1} what is the source of deeds? Contact is their source. 

And\marginnote{35.1} what is the diversity of deeds? There are deeds that lead to rebirth in hell, the animal realm, the ghost realm, the human world, and the world of the gods. This is called the diversity of deeds. 

And\marginnote{36.1} what is the result of deeds? The result of deeds is threefold, I say: in this very life, on rebirth in the next life, or at some later time. This is called the result of deeds. 

And\marginnote{37.1} what is the cessation of deeds? When contact ceases, deeds cease. The practice that leads to the cessation of deeds is simply this noble eightfold path, that is: right view, right thought, right speech, right action, right livelihood, right effort, right mindfulness, and right immersion. 

When\marginnote{38.1} a noble disciple understands deeds in this way … they understand that this penetrative spiritual life is the cessation of deeds. ‘Deeds should be known. And their source, diversity, result, cessation, and the practice that leads to their cessation should be known.’ That’s what I said, and this is why I said it. 

‘Suffering\marginnote{39.1} should be known. And its source, diversity, result, cessation, and the practice that leads to its cessation should be known.’ That’s what I said, but why did I say it? Rebirth is suffering; old age is suffering; illness is suffering; death is suffering; sorrow, lamentation, pain, sadness, and distress are suffering; not getting what you wish for is suffering. In brief, the five grasping aggregates are suffering. 

And\marginnote{40.1} what is the source of suffering? Craving is the source of suffering. 

And\marginnote{41.1} what is the diversity of suffering? There is suffering that is severe, mild, slow to fade, and quick to fade. This is called the diversity of suffering. 

And\marginnote{42.1} what is the result of suffering? It’s when someone who is overcome and overwhelmed by suffering sorrows and wails and laments, beating their breast and falling into confusion. Or else, overcome by that suffering, they begin an external search, wondering: ‘Who knows one or two phrases to stop this suffering?’ The result of suffering is either confusion or a search, I say. This is called the result of suffering. 

And\marginnote{43.1} what is the cessation of suffering? When craving ceases, suffering ceases. The practice that leads to the cessation of suffering is simply this noble eightfold path, that is: right view, right thought, right speech, right action, right livelihood, right effort, right mindfulness, and right immersion. 

When\marginnote{44.1} a noble disciple understands suffering in this way … they understand that this penetrative spiritual life is the cessation of suffering. ‘Suffering should be known. And its source, diversity, result, cessation, and the practice that leads to its cessation should be known.’ That’s what I said, and this is why I said it. 

This\marginnote{45.1} is the penetrative exposition of the teaching.” 

%
\section*{{\suttatitleacronym AN 6.64}{\suttatitletranslation The Lion’s Roar }{\suttatitleroot Sīhanādasutta}}
\addcontentsline{toc}{section}{\tocacronym{AN 6.64} \toctranslation{The Lion’s Roar } \tocroot{Sīhanādasutta}}
\markboth{The Lion’s Roar }{Sīhanādasutta}
\extramarks{AN 6.64}{AN 6.64}

“Mendicants,\marginnote{1.1} the Realized One possesses six powers of a Realized One. With these he claims the bull’s place, roars his lion’s roar in the assemblies, and turns the holy wheel. What six? 

Firstly,\marginnote{1.3} the Realized One truly understands the possible as possible and the impossible as impossible. Since he truly understands this, this is a power of the Realized One. Relying on this he claims the bull’s place, roars his lion’s roar in the assemblies, and turns the holy wheel. 

Furthermore,\marginnote{2.1} the Realized One truly understands the result of deeds undertaken in the past, future, and present in terms of causes and reasons. Since he truly understands this, this is a power of the Realized One. … 

Furthermore,\marginnote{3.1} the Realized One truly understands corruption, cleansing, and emergence regarding the absorptions, liberations, immersions, and attainments. Since he truly understands this, this is a power of the Realized One. … 

Furthermore,\marginnote{4.1} the Realized One recollects many kinds of past lives. That is: one, two, three, four, five, ten, twenty, thirty, forty, fifty, a hundred, a thousand, a hundred thousand rebirths; many eons of the world contracting, many eons of the world expanding, many eons of the world contracting and expanding. He remembers: ‘There, I was named this, my clan was that, I looked like this, and that was my food. This was how I felt pleasure and pain, and that was how my life ended. When I passed away from that place I was reborn somewhere else. There, too, I was named this, my clan was that, I looked like this, and that was my food. This was how I felt pleasure and pain, and that was how my life ended. When I passed away from that place I was reborn here.’ And so he recollects his many kinds of past lives, with features and details. Since he truly understands this, this is a power of the Realized One. … 

Furthermore,\marginnote{5.1} with clairvoyance that is purified and superhuman, the Realized One sees sentient beings passing away and being reborn—inferior and superior, beautiful and ugly, in a good place or a bad place. He understands how sentient beings are reborn according to their deeds. Since he truly understands this, this is a power of the Realized One. … 

Furthermore,\marginnote{6.1} the Realized One has realized the undefiled freedom of heart and freedom by wisdom in this very life. And he lives having realized it with his own insight due to the ending of defilements. Since he truly understands this, this is a power of the Realized One. Relying on this he claims the bull’s place, roars his lion’s roar in the assemblies, and turns the holy wheel. These are the six powers of a Realized One that the Realized One possesses. With these he claims the bull’s place, roars his lion’s roar in the assemblies, and turns the holy wheel. 

If\marginnote{7.1} others come to the Realized One and ask questions about his true knowledge of the possible as possible and the impossible as impossible, the Realized One answers them in whatever manner he has truly known it. 

If\marginnote{8.1} others come to the Realized One and ask questions about his true knowledge of the result of deeds undertaken in the past, future, and present in terms of causes and reasons, the Realized One answers them in whatever manner he has truly known it. 

If\marginnote{9.1} others come to the Realized One and ask questions about his true knowledge of corruption, cleansing, and emergence regarding the absorptions, liberations, immersions, and attainments, the Realized One answers them in whatever manner he has truly known it. 

If\marginnote{10.1} others come to the Realized One and ask questions about his true knowledge of recollection of past lives, the Realized One answers them in whatever manner he has truly known it. 

If\marginnote{11.1} others come to the Realized One and ask questions about his true knowledge of passing away and rebirth of sentient beings, the Realized One answers them in whatever manner he has truly known it. 

If\marginnote{12.1} others come to the Realized One and ask questions about his true knowledge of the ending of defilements, the Realized One answers them in whatever manner he has truly known it. 

And\marginnote{13.1} I say that true knowledge of the possible as possible and the impossible as impossible is for those with immersion, not for those without immersion. And true knowledge of the result of deeds undertaken in the past, future, and present in terms of causes and reasons is for those with immersion, not for those without immersion. And true knowledge of corruption, cleansing, and emergence regarding the absorptions, liberations, immersions, and attainments is for those with immersion, not for those without immersion. And true knowledge of the recollection of past lives is for those with immersion, not for those without immersion. And true knowledge of the passing away and rebirth of sentient beings is for those with immersion, not for those without immersion. And true knowledge of the ending of defilements is for those with immersion, not for those without immersion. 

And\marginnote{13.7} so, mendicants, immersion is the path. No immersion is the wrong path.” 

%
\addtocontents{toc}{\let\protect\contentsline\protect\nopagecontentsline}
\pannasa{The Chapter on Deities }
\addcontentsline{toc}{pannasa}{The Chapter on Deities }
\markboth{}{}
\addtocontents{toc}{\let\protect\contentsline\protect\oldcontentsline}

%
\section*{{\suttatitleacronym AN 6.65}{\suttatitletranslation The Fruit of Non-Return }{\suttatitleroot Anāgāmiphalasutta}}
\addcontentsline{toc}{section}{\tocacronym{AN 6.65} \toctranslation{The Fruit of Non-Return } \tocroot{Anāgāmiphalasutta}}
\markboth{The Fruit of Non-Return }{Anāgāmiphalasutta}
\extramarks{AN 6.65}{AN 6.65}

“Mendicants,\marginnote{1.1} without giving up six things you can’t realize the fruit of non-return. What six? Lack of faith, conscience, and prudence; laziness, unmindfulness, and witlessness. Without giving up these six things you can’t realize the fruit of non-return. 

After\marginnote{2.1} giving up six things you can realize the fruit of non-return. What six? Lack of faith, conscience, and prudence; laziness, unmindfulness, and witlessness. After giving up these six things you can realize the fruit of non-return.” 

%
\section*{{\suttatitleacronym AN 6.66}{\suttatitletranslation Perfection }{\suttatitleroot Arahattasutta}}
\addcontentsline{toc}{section}{\tocacronym{AN 6.66} \toctranslation{Perfection } \tocroot{Arahattasutta}}
\markboth{Perfection }{Arahattasutta}
\extramarks{AN 6.66}{AN 6.66}

“Mendicants,\marginnote{1.1} without giving up six things you can’t realize perfection. What six? Dullness, drowsiness, restlessness, remorse, lack of faith, and negligence. Without giving up these six things you can’t realize perfection. 

After\marginnote{2.1} giving up six things you can realize perfection. What six? Dullness, drowsiness, restlessness, remorse, lack of faith, and negligence. After giving up these six things you can realize perfection.” 

%
\section*{{\suttatitleacronym AN 6.67}{\suttatitletranslation Friends }{\suttatitleroot Mittasutta}}
\addcontentsline{toc}{section}{\tocacronym{AN 6.67} \toctranslation{Friends } \tocroot{Mittasutta}}
\markboth{Friends }{Mittasutta}
\extramarks{AN 6.67}{AN 6.67}

“Mendicants,\marginnote{1.1} it’s totally impossible that a mendicant with bad friends, companions, and associates, while frequenting, accompanying, and attending, and following their example, will fulfill the practice dealing with the supplementary regulations. Without fulfilling the practice dealing with supplementary regulations, it’s impossible to fulfill the practice of a trainee. Without fulfilling the practice of a trainee, it’s impossible to fulfill ethics. Without fulfilling ethics, it’s impossible give up sensual desire, or desire to be reborn in the realm of luminous form or in the formless realm. 

It’s\marginnote{2.1} possible that a mendicant with good friends, companions, and associates, while frequenting, accompanying, and attending, and following their example, will fulfill the practice dealing with the supplementary regulations. Having fulfilled the practice dealing with supplementary regulations, it’s possible to fulfill the practice of a trainee. Having fulfilled the practice of a trainee, it’s possible to fulfill ethics. Having fulfilled ethics, it’s possible give up sensual desire, and desire to be reborn in the realm of luminous form and in the formless realm.” 

%
\section*{{\suttatitleacronym AN 6.68}{\suttatitletranslation Enjoying Company }{\suttatitleroot Saṅgaṇikārāmasutta}}
\addcontentsline{toc}{section}{\tocacronym{AN 6.68} \toctranslation{Enjoying Company } \tocroot{Saṅgaṇikārāmasutta}}
\markboth{Enjoying Company }{Saṅgaṇikārāmasutta}
\extramarks{AN 6.68}{AN 6.68}

“Mendicants,\marginnote{1.1} it’s totally impossible that a mendicant who enjoys company and groups, who loves them and likes to enjoy them, should take pleasure in being alone in seclusion. Without taking pleasure in being alone in seclusion, it’s impossible to learn the patterns of the mind. Without learning the patterns of the mind, it’s impossible to fulfill right view. Without fulfilling right view, it’s impossible to fulfill right immersion. Without fulfilling right immersion, it’s impossible to give up the fetters. Without giving up the fetters, it’s impossible to realize extinguishment. 

It’s\marginnote{2.1} totally possible that a mendicant who doesn’t enjoy company and groups, who doesn’t love them and like to enjoy them, should take pleasure in being alone in seclusion. For someone who takes pleasure in being alone in seclusion, it’s possible to learn the patterns of the mind. For someone who learns the patterns of the mind, it’s possible to fulfill right view. Having fulfilled right view, it’s possible to fulfill right immersion. Having fulfilled right immersion, it’s possible to give up the fetters. Having given up the fetters, it’s possible to realize extinguishment.” 

%
\section*{{\suttatitleacronym AN 6.69}{\suttatitletranslation A God }{\suttatitleroot Devatāsutta}}
\addcontentsline{toc}{section}{\tocacronym{AN 6.69} \toctranslation{A God } \tocroot{Devatāsutta}}
\markboth{A God }{Devatāsutta}
\extramarks{AN 6.69}{AN 6.69}

Then,\marginnote{1.1} late at night, a glorious deity, lighting up the entire Jeta’s Grove, went up to the Buddha, bowed, stood to one side, and said to him: 

“Sir,\marginnote{1.2} these six things don’t lead to the decline of a mendicant. What six? Respect for the Teacher, for the teaching, for the \textsanskrit{Saṅgha}, for the training; being easy to admonish, and good friendship. These six things don’t lead to the decline of a mendicant.” 

That’s\marginnote{1.6} what that deity said, and the teacher approved. Then that deity, knowing that the teacher approved, bowed, and respectfully circled the Buddha, keeping him on his right, before vanishing right there. 

Then,\marginnote{2.1} when the night had passed, the Buddha addressed the mendicants: 

“Tonight,\marginnote{2.2} a glorious deity, lighting up the entire Jeta’s Grove, came to me, bowed, stood to one side, and said to me: ‘Sir, these six things don’t lead to the decline of a mendicant. What six? Respect for the Teacher, for the teaching, for the \textsanskrit{Saṅgha}, for the training; being easy to admonish, and good friendship. These six things don’t lead to the decline of a mendicant.’ 

That\marginnote{2.7} is what that deity said. Then he bowed and respectfully circled me, keeping me on his right side, before vanishing right there.” 

When\marginnote{3.1} he said this, Venerable \textsanskrit{Sāriputta} said to the Buddha: 

“Sir,\marginnote{3.2} this is how I understand the detailed meaning of the Buddha’s brief statement. It’s when a mendicant personally respects the Teacher and praises such respect. And they encourage other mendicants who lack such respect to respect the Teacher. And they praise other mendicants who respect the Teacher at the right time, truthfully and substantively. They personally respect the teaching … They personally respect the \textsanskrit{Saṅgha} … They personally respect the training … They are personally easy to admonish … They personally have good friends, and praise such friendship. And they encourage other mendicants who lack good friends to develop good friendship. And they praise other mendicants who have good friends at the right time, truthfully and substantively. That’s how I understand the detailed meaning of the Buddha’s brief statement.” 

“Good,\marginnote{4.1} good, \textsanskrit{Sāriputta}! It’s good that you understand the detailed meaning of what I’ve said in brief like this. 

It’s\marginnote{4.3} when a mendicant personally respects the Teacher … They personally respect the teaching … They personally respect the \textsanskrit{Saṅgha} … They personally respect the training … They are personally easy to admonish … They personally have good friends, and praise such friendship. And they encourage other mendicants who lack good friends to develop good friendship. And they praise other mendicants who have good friends at the right time, truthfully and substantively. This is how to understand the detailed meaning of what I said in brief.” 

%
\section*{{\suttatitleacronym AN 6.70}{\suttatitletranslation Immersion }{\suttatitleroot Samādhisutta}}
\addcontentsline{toc}{section}{\tocacronym{AN 6.70} \toctranslation{Immersion } \tocroot{Samādhisutta}}
\markboth{Immersion }{Samādhisutta}
\extramarks{AN 6.70}{AN 6.70}

“Mendicants,\marginnote{1.1} it’s totally impossible that a mendicant without immersion that is peaceful, refined, tranquil, and unified will wield the many kinds of psychic power: multiplying themselves and becoming one again; appearing and disappearing; going unimpeded through a wall, a rampart, or a mountain as if through space; diving in and out of the earth as if it were water; walking on water as if it were earth; flying cross-legged through the sky like a bird; touching and stroking with the hand the sun and moon, so mighty and powerful. They control the body as far as the \textsanskrit{Brahmā} realm. 

It’s\marginnote{1.2} impossible that with clairaudience that is purified and superhuman, they’ll hear both kinds of sounds, human and divine, whether near or far. 

It’s\marginnote{1.3} impossible that they’ll understand the minds of other beings and individuals, having comprehended them with their own mind, understanding mind with greed as ‘mind with greed’ … and freed mind as ‘freed mind’. 

It’s\marginnote{1.6} impossible that they’ll recollect many kinds of past lives, with features and details. 

It’s\marginnote{1.7} impossible that with clairvoyance that is purified and surpasses the human, they’ll understand how sentient beings are reborn according to their deeds. 

It’s\marginnote{1.8} impossible that they’ll realize the undefiled freedom of heart and freedom by wisdom in this very life, and live having realized it with their own insight due to the ending of defilements. 

But\marginnote{2.1} it’s totally possible that a mendicant who has immersion that is peaceful, refined, tranquil, and unified will wield the many kinds of psychic power … 

It’s\marginnote{2.2} possible that with clairaudience that is purified and superhuman, they’ll hear both kinds of sounds … 

It’s\marginnote{2.3} possible that they’ll understand the minds of other beings … 

It’s\marginnote{2.6} possible that they’ll recollect many kinds of past lives, with features and details. 

It’s\marginnote{2.7} possible that with clairvoyance that is purified and superhuman, they’ll understand how sentient beings are reborn according to their deeds. 

It’s\marginnote{2.8} possible that they’ll realize the undefiled freedom of heart and freedom by wisdom in this very life, and live having realized it with their own insight due to the ending of defilements.” 

%
\section*{{\suttatitleacronym AN 6.71}{\suttatitletranslation Capable of Realizing }{\suttatitleroot Sakkhibhabbasutta}}
\addcontentsline{toc}{section}{\tocacronym{AN 6.71} \toctranslation{Capable of Realizing } \tocroot{Sakkhibhabbasutta}}
\markboth{Capable of Realizing }{Sakkhibhabbasutta}
\extramarks{AN 6.71}{AN 6.71}

“Mendicants,\marginnote{1.1} a mendicant with six qualities is incapable of realizing anything that can be realized, in each and every case. What six? It’s when a mendicant doesn’t truly understand which qualities make things worse, which keep things steady, which lead to distinction, and which lead to penetration. And they don’t practice carefully or do what’s suitable. A mendicant with these six qualities is incapable of realizing anything that can be realized, in each and every case. 

A\marginnote{2.1} mendicant with six qualities is capable of realizing anything that can be realized, in each and every case. What six? It’s when a mendicant truly understands which qualities make things worse, which keep things steady, which lead to distinction, and which lead to penetration. And they practice carefully and do what’s suitable. A mendicant with these six qualities is capable of realizing anything that can be realized, in each and every case.” 

%
\section*{{\suttatitleacronym AN 6.72}{\suttatitletranslation Strength }{\suttatitleroot Balasutta}}
\addcontentsline{toc}{section}{\tocacronym{AN 6.72} \toctranslation{Strength } \tocroot{Balasutta}}
\markboth{Strength }{Balasutta}
\extramarks{AN 6.72}{AN 6.72}

“Mendicants,\marginnote{1.1} a mendicant who has six qualities can’t attain strength in immersion. What six? It’s when a mendicant is not skilled in entering immersion, skilled in remaining in immersion, or skilled in emerging from immersion. And they don’t practice carefully and persistently, and they don’t do what’s suitable. A mendicant who has these six qualities can’t attain strength in immersion. 

A\marginnote{2.1} mendicant who has six qualities can attain strength in immersion. What six? It’s when a mendicant is skilled in entering immersion, skilled in remaining in immersion, and skilled in emerging from immersion. And they practice carefully and persistently, and do what’s suitable. A mendicant who has these six qualities can attain strength in immersion.” 

%
\section*{{\suttatitleacronym AN 6.73}{\suttatitletranslation First Absorption (1st) }{\suttatitleroot Paṭhamatajjhānasutta}}
\addcontentsline{toc}{section}{\tocacronym{AN 6.73} \toctranslation{First Absorption (1st) } \tocroot{Paṭhamatajjhānasutta}}
\markboth{First Absorption (1st) }{Paṭhamatajjhānasutta}
\extramarks{AN 6.73}{AN 6.73}

“Mendicants,\marginnote{1.1} without giving up these six qualities you can’t enter and remain in the first absorption. What six? Desire for sensual pleasures, ill will, dullness and drowsiness, restlessness and remorse, and doubt. And the drawbacks of sensual pleasures haven’t been truly seen clearly with right wisdom. Without giving up these six qualities you can’t enter and remain in the first absorption. 

But\marginnote{2.1} after giving up these six qualities you can enter and remain in the first absorption. What six? Desire for sensual pleasures, ill will, dullness and drowsiness, restlessness and remorse, and doubt. And the drawbacks of sensual pleasures have been truly seen clearly with right wisdom. After giving up these six qualities you can enter and remain in the first absorption.” 

%
\section*{{\suttatitleacronym AN 6.74}{\suttatitletranslation First Absorption (2nd) }{\suttatitleroot Dutiyatajjhānasutta}}
\addcontentsline{toc}{section}{\tocacronym{AN 6.74} \toctranslation{First Absorption (2nd) } \tocroot{Dutiyatajjhānasutta}}
\markboth{First Absorption (2nd) }{Dutiyatajjhānasutta}
\extramarks{AN 6.74}{AN 6.74}

“Mendicants,\marginnote{1.1} without giving up these six qualities you can’t enter and remain in the first absorption. What six? Sensual, malicious, and cruel thoughts; and sensual, malicious, and cruel perceptions. Without giving up these six qualities you can’t enter and remain in the first absorption. 

But\marginnote{2.1} after giving up these six qualities you can enter and remain in the first absorption. What six? Sensual, malicious, and cruel thoughts; and sensual, malicious, and cruel perceptions. After giving up these six qualities you can enter and remain in the first absorption.” 

%
\addtocontents{toc}{\let\protect\contentsline\protect\nopagecontentsline}
\chapter*{The Chapter on Perfection }
\addcontentsline{toc}{chapter}{\tocchapterline{The Chapter on Perfection }}
\addtocontents{toc}{\let\protect\contentsline\protect\oldcontentsline}

%
\section*{{\suttatitleacronym AN 6.75}{\suttatitletranslation Suffering }{\suttatitleroot Dukkhasutta}}
\addcontentsline{toc}{section}{\tocacronym{AN 6.75} \toctranslation{Suffering } \tocroot{Dukkhasutta}}
\markboth{Suffering }{Dukkhasutta}
\extramarks{AN 6.75}{AN 6.75}

“Mendicants,\marginnote{1.1} when a mendicant has six qualities they live unhappily in the present life—with distress, anguish, and fever—and when the body breaks up, after death, they can expect a bad rebirth. What six? Sensual, malicious, and cruel thoughts; and sensual, malicious, and cruel perceptions. When a mendicant has these six qualities they live unhappily in the present life—with distress, anguish, and fever—and when the body breaks up, after death, they can expect a bad rebirth. 

When\marginnote{2.1} a mendicant has six qualities they live happily in the present life—without distress, anguish, or fever—and when the body breaks up, after death, they can expect a good rebirth. What six? Thoughts of renunciation, good will, and harmlessness. And perceptions of renunciation, good will, and harmlessness. When a mendicant has these six qualities they live happily in the present life—without distress, anguish, or fever—and when the body breaks up, after death, they can expect a good rebirth.” 

%
\section*{{\suttatitleacronym AN 6.76}{\suttatitletranslation Perfection }{\suttatitleroot Arahattasutta}}
\addcontentsline{toc}{section}{\tocacronym{AN 6.76} \toctranslation{Perfection } \tocroot{Arahattasutta}}
\markboth{Perfection }{Arahattasutta}
\extramarks{AN 6.76}{AN 6.76}

“Mendicants,\marginnote{1.1} without giving up six things you can’t realize perfection. What six? Conceit, inferiority complex, superiority complex, overestimation, obstinacy, and groveling. Without giving up these six qualities you can’t realize perfection. 

After\marginnote{2.1} giving up six things you can realize perfection. What six? Conceit, inferiority complex, superiority complex, overestimation, obstinacy, and groveling. After giving up these six things you can realize perfection.” 

%
\section*{{\suttatitleacronym AN 6.77}{\suttatitletranslation Superhuman States }{\suttatitleroot Uttarimanussadhammasutta}}
\addcontentsline{toc}{section}{\tocacronym{AN 6.77} \toctranslation{Superhuman States } \tocroot{Uttarimanussadhammasutta}}
\markboth{Superhuman States }{Uttarimanussadhammasutta}
\extramarks{AN 6.77}{AN 6.77}

“Mendicants,\marginnote{1.1} without giving up six qualities you can’t realize a superhuman distinction in knowledge and vision worthy of the noble ones. What six? Lack of mindfulness and situational awareness, not guarding the sense doors, eating too much, deceit, and flattery. Without giving up these six qualities you can’t realize a superhuman distinction in knowledge and vision worthy of the noble ones. 

But\marginnote{2.1} after giving up six qualities you can realize a superhuman distinction in knowledge and vision worthy of the noble ones. What six? Lack of mindfulness and situational awareness, not guarding the sense doors, eating too much, deceit, and flattery. After giving up these six qualities you can realize a superhuman distinction in knowledge and vision worthy of the noble ones.” 

%
\section*{{\suttatitleacronym AN 6.78}{\suttatitletranslation Joy and Happiness }{\suttatitleroot Sukhasomanassasutta}}
\addcontentsline{toc}{section}{\tocacronym{AN 6.78} \toctranslation{Joy and Happiness } \tocroot{Sukhasomanassasutta}}
\markboth{Joy and Happiness }{Sukhasomanassasutta}
\extramarks{AN 6.78}{AN 6.78}

“Mendicants,\marginnote{1.1} when a mendicant has six things they’re full of joy and happiness in the present life, and they have laid the groundwork for ending the defilements. What six? It’s when a mendicant enjoys the teaching, meditation, giving up, seclusion, kindness, and non-proliferation. When a mendicant has these six things they’re full of joy and happiness in the present life, and they have laid the groundwork for ending the defilements.” 

%
\section*{{\suttatitleacronym AN 6.79}{\suttatitletranslation Achievement }{\suttatitleroot Adhigamasutta}}
\addcontentsline{toc}{section}{\tocacronym{AN 6.79} \toctranslation{Achievement } \tocroot{Adhigamasutta}}
\markboth{Achievement }{Adhigamasutta}
\extramarks{AN 6.79}{AN 6.79}

“Mendicants,\marginnote{1.1} a mendicant who has six qualities is unable to acquire more skillful qualities or to increase the skillful qualities they’ve already acquired. What six? It’s when a mendicant is not skilled in profit, skilled in loss, and skilled in means. They don’t generate enthusiasm to achieve skillful qualities not yet achieved. They don’t protect skillful qualities they have achieved. And they don’t try to persevere in the task. A mendicant who has these six qualities is unable to acquire more skillful qualities or to increase the skillful qualities they’ve already acquired. 

A\marginnote{2.1} mendicant who has six qualities is able to acquire more skillful qualities or to increase the skillful qualities they’ve already acquired. What six? It’s when a mendicant is skilled in profit, skilled in loss, and skilled in means. They generate enthusiasm to achieve skillful qualities not yet achieved. They protect skillful qualities they have achieved. And they try to persevere in the task. A mendicant who has these six qualities is able to acquire more skillful qualities or to increase the skillful qualities they’ve already acquired.” 

%
\section*{{\suttatitleacronym AN 6.80}{\suttatitletranslation Greatness }{\suttatitleroot Mahantattasutta}}
\addcontentsline{toc}{section}{\tocacronym{AN 6.80} \toctranslation{Greatness } \tocroot{Mahantattasutta}}
\markboth{Greatness }{Mahantattasutta}
\extramarks{AN 6.80}{AN 6.80}

“Mendicants,\marginnote{1.1} a mendicant with six qualities soon acquires great and abundant good qualities. What six? It’s when a mendicant is full of light, full of practice, full of inspiration, and full of eagerness. They don’t slack off when it comes to developing skillful qualities. They reach further. A mendicant who has these six qualities soon acquires great and abundant good qualities.” 

%
\section*{{\suttatitleacronym AN 6.81}{\suttatitletranslation Hell (1st) }{\suttatitleroot Paṭhamanirayasutta}}
\addcontentsline{toc}{section}{\tocacronym{AN 6.81} \toctranslation{Hell (1st) } \tocroot{Paṭhamanirayasutta}}
\markboth{Hell (1st) }{Paṭhamanirayasutta}
\extramarks{AN 6.81}{AN 6.81}

“Mendicants,\marginnote{1.1} someone with six qualities is cast down to hell. What six? They kill living creatures, steal, commit sexual misconduct, and lie. And they have wicked desires and wrong view. Someone with these six qualities is cast down to hell. 

Someone\marginnote{2.1} with six qualities is raised up to heaven. What six? They don’t kill living creatures, steal, commit sexual misconduct, or lie. And they have few desires and right view. Someone with these six qualities is raised up to heaven.” 

%
\section*{{\suttatitleacronym AN 6.82}{\suttatitletranslation Hell (2nd) }{\suttatitleroot Dutiyanirayasutta}}
\addcontentsline{toc}{section}{\tocacronym{AN 6.82} \toctranslation{Hell (2nd) } \tocroot{Dutiyanirayasutta}}
\markboth{Hell (2nd) }{Dutiyanirayasutta}
\extramarks{AN 6.82}{AN 6.82}

“Mendicants,\marginnote{1.1} someone with six qualities is cast down to hell. What six? They kill living creatures, steal, commit sexual misconduct, and lie. And they’re greedy and rude. Someone with these six qualities is cast down to hell. 

Someone\marginnote{2.1} with six qualities is raised up to heaven. What six? They don’t kill living creatures, steal, commit sexual misconduct, or lie. And they’re not greedy or rude. Someone with these six qualities is raised up to heaven.” 

%
\section*{{\suttatitleacronym AN 6.83}{\suttatitletranslation The Best Thing }{\suttatitleroot Aggadhammasutta}}
\addcontentsline{toc}{section}{\tocacronym{AN 6.83} \toctranslation{The Best Thing } \tocroot{Aggadhammasutta}}
\markboth{The Best Thing }{Aggadhammasutta}
\extramarks{AN 6.83}{AN 6.83}

“Mendicants,\marginnote{1.1} a mendicant with six qualities can’t realize the best thing, perfection. What six? It’s when a mendicant is faithless, shameless, imprudent, lazy, and witless. And they’re concerned with their body and their life. A mendicant with these six qualities can’t realize the best thing, perfection. 

A\marginnote{2.1} mendicant with six qualities can realize the best thing, perfection. What six? It’s when a mendicant is faithful, conscientious, prudent, energetic, and wise. And they have no concern for their body and their life. A mendicant with these six qualities can realize the best thing, perfection.” 

%
\section*{{\suttatitleacronym AN 6.84}{\suttatitletranslation Day and Night }{\suttatitleroot Rattidivasasutta}}
\addcontentsline{toc}{section}{\tocacronym{AN 6.84} \toctranslation{Day and Night } \tocroot{Rattidivasasutta}}
\markboth{Day and Night }{Rattidivasasutta}
\extramarks{AN 6.84}{AN 6.84}

“Mendicants,\marginnote{1.1} a mendicant with six qualities can expect decline, not growth, in skillful qualities, whether by day or by night. What six? It’s when a mendicant has many desires—they’re frustrated and not content with any kind of robes, almsfood, lodgings, and medicines and supplies for the sick. And they’re faithless, unethical, unmindful, and witless. A mendicant with these six qualities can expect decline, not growth, in skillful qualities, whether by day or by night. 

A\marginnote{2.1} mendicant with six qualities can expect growth, not decline, in skillful qualities, whether by day or by night. What six? It’s when a mendicant doesn’t have many desires—they’re not frustrated but content with any kind of robes, almsfood, lodgings, and medicines and supplies for the sick. And they’re faithful, ethical, mindful, and wise. A mendicant with these six qualities can expect growth, not decline, in skillful qualities, whether by day or by night.” 

%
\addtocontents{toc}{\let\protect\contentsline\protect\nopagecontentsline}
\chapter*{The Chapter on Coolness }
\addcontentsline{toc}{chapter}{\tocchapterline{The Chapter on Coolness }}
\addtocontents{toc}{\let\protect\contentsline\protect\oldcontentsline}

%
\section*{{\suttatitleacronym AN 6.85}{\suttatitletranslation Coolness }{\suttatitleroot Sītibhāvasutta}}
\addcontentsline{toc}{section}{\tocacronym{AN 6.85} \toctranslation{Coolness } \tocroot{Sītibhāvasutta}}
\markboth{Coolness }{Sītibhāvasutta}
\extramarks{AN 6.85}{AN 6.85}

“Mendicants,\marginnote{1.1} a mendicant with six qualities can’t realize supreme coolness. What six? It’s when a mendicant doesn’t keep their mind in check when they should. They don’t exert their mind when they should. They don’t encourage the mind when they should. They don’t watch over the mind with equanimity when they should. They have bad convictions. They delight in identity. A mendicant with these six qualities can’t realize supreme coolness. 

A\marginnote{2.1} mendicant with six qualities can realize supreme coolness. What six? It’s when a mendicant keeps their mind in check when they should. They exert their mind when they should. They encourage the mind when they should. They watch over the mind with equanimity when they should. They have excellent convictions. They delight in extinguishment. A mendicant with these six qualities can realize supreme coolness.” 

%
\section*{{\suttatitleacronym AN 6.86}{\suttatitletranslation Obstacles }{\suttatitleroot Āvaraṇasutta}}
\addcontentsline{toc}{section}{\tocacronym{AN 6.86} \toctranslation{Obstacles } \tocroot{Āvaraṇasutta}}
\markboth{Obstacles }{Āvaraṇasutta}
\extramarks{AN 6.86}{AN 6.86}

“Mendicants,\marginnote{1.1} someone with six qualities is unable to enter the sure path with regards to skillful qualities even when listening to the true teaching. What six? They’re obstructed by deeds, defilements, or results. And they’re faithless, unenthusiastic, and witless. Someone with these six qualities is unable to enter the sure path with regards to skillful qualities, even when listening to the true teaching. 

Someone\marginnote{2.1} with six qualities is able to enter the sure path with regards to skillful qualities when listening to the true teaching. What six? They’re not obstructed by deeds, defilements, or results. And they’re faithful, enthusiastic, and wise. Someone with these six qualities is able to enter the sure path with regards to skillful qualities when listening to the true teaching.” 

%
\section*{{\suttatitleacronym AN 6.87}{\suttatitletranslation A Murderer }{\suttatitleroot Voropitasutta}}
\addcontentsline{toc}{section}{\tocacronym{AN 6.87} \toctranslation{A Murderer } \tocroot{Voropitasutta}}
\markboth{A Murderer }{Voropitasutta}
\extramarks{AN 6.87}{AN 6.87}

“Mendicants,\marginnote{1.1} someone with six qualities is unable to enter the sure path with regards to skillful qualities even when listening to the true teaching. What six? They murder their mother or father or a perfected one. They maliciously shed the blood of a Realized One. They cause a schism in the \textsanskrit{Saṅgha}. They’re witless, dull, and stupid. Someone with these six qualities is unable to enter the sure path with regards to skillful qualities, even when listening to the true teaching. 

Someone\marginnote{2.1} with six qualities is able to enter the sure path with regards to skillful qualities when listening to the true teaching. What six? They don’t murder their mother or father or a perfected one. They don’t maliciously shed the blood of a Realized One. They don’t cause a schism in the \textsanskrit{Saṅgha}. They’re wise, bright, and clever. Someone with these six qualities is able to enter the sure path with regards to skillful qualities when listening to the true teaching.” 

%
\section*{{\suttatitleacronym AN 6.88}{\suttatitletranslation Wanting to Listen }{\suttatitleroot Sussūsatisutta}}
\addcontentsline{toc}{section}{\tocacronym{AN 6.88} \toctranslation{Wanting to Listen } \tocroot{Sussūsatisutta}}
\markboth{Wanting to Listen }{Sussūsatisutta}
\extramarks{AN 6.88}{AN 6.88}

“Mendicants,\marginnote{1.1} someone with six qualities is unable to enter the sure path with regards to skillful qualities even when listening to the true teaching. What six? When the teaching and practice proclaimed by the Realized One is being taught they don’t want to listen. They don’t lend an ear or apply their mind to understand them. They learn the incorrect meaning and reject the correct meaning. They accept views that contradict the teaching. Someone with these six qualities is unable to enter the sure path with regards to skillful qualities, even when listening to the true teaching. 

Someone\marginnote{2.1} with six qualities is able to enter the sure path with regards to skillful qualities when listening to the true teaching. What six? When the teaching and practice proclaimed by the Realized One is being taught they want to listen. They lend an ear and apply their mind to understand them. They learn the correct meaning and reject the incorrect meaning. They accept views that agree with the teaching. Someone with these six qualities is able to enter the sure path with regards to skillful qualities when listening to the true teaching.” 

%
\section*{{\suttatitleacronym AN 6.89}{\suttatitletranslation Not Giving Up }{\suttatitleroot Appahāyasutta}}
\addcontentsline{toc}{section}{\tocacronym{AN 6.89} \toctranslation{Not Giving Up } \tocroot{Appahāyasutta}}
\markboth{Not Giving Up }{Appahāyasutta}
\extramarks{AN 6.89}{AN 6.89}

“Mendicants,\marginnote{1.1} without giving up six things you can’t become accomplished in view. What six? Identity view, doubt, misapprehension of precepts and observances, and forms of greed, hate, and delusion that lead to rebirth in places of loss. Without giving up these six things you can’t become accomplished in view. 

After\marginnote{2.1} giving up six things you can become accomplished in view. What six? Identity view, doubt, misapprehension of precepts and observances, and forms of greed, hate, and delusion that lead to rebirth in places of loss. After giving up these six things you can become accomplished in view.” 

%
\section*{{\suttatitleacronym AN 6.90}{\suttatitletranslation Given Up }{\suttatitleroot Pahīnasutta}}
\addcontentsline{toc}{section}{\tocacronym{AN 6.90} \toctranslation{Given Up } \tocroot{Pahīnasutta}}
\markboth{Given Up }{Pahīnasutta}
\extramarks{AN 6.90}{AN 6.90}

“Mendicants,\marginnote{1.1} a person accomplished in view has given up six things. What six? Identity view, doubt, misapprehension of precepts and observances, and forms of greed, hate, and delusion that lead to rebirth in places of loss. A person accomplished in view has given up these six things.” 

%
\section*{{\suttatitleacronym AN 6.91}{\suttatitletranslation Can’t Give Rise }{\suttatitleroot Abhabbasutta}}
\addcontentsline{toc}{section}{\tocacronym{AN 6.91} \toctranslation{Can’t Give Rise } \tocroot{Abhabbasutta}}
\markboth{Can’t Give Rise }{Abhabbasutta}
\extramarks{AN 6.91}{AN 6.91}

“Mendicants,\marginnote{1.1} a person accomplished in view can’t give rise to six things. What six? Identity view, doubt, misapprehension of precepts and observances, and forms of greed, hate, and delusion that lead to rebirth in places of loss. A person accomplished in view can’t give rise to these six things.” 

%
\section*{{\suttatitleacronym AN 6.92}{\suttatitletranslation Things That Can’t Be Done (1st) }{\suttatitleroot Paṭhamaabhabbaṭṭhānasutta}}
\addcontentsline{toc}{section}{\tocacronym{AN 6.92} \toctranslation{Things That Can’t Be Done (1st) } \tocroot{Paṭhamaabhabbaṭṭhānasutta}}
\markboth{Things That Can’t Be Done (1st) }{Paṭhamaabhabbaṭṭhānasutta}
\extramarks{AN 6.92}{AN 6.92}

“Mendicants,\marginnote{1.1} these six things can’t be done. What six? A person accomplished in view can’t live disrespectful and irreverent toward the Teacher, the teaching, the \textsanskrit{Saṅgha}, or the training. They can’t establish their belief on unreliable grounds. And they can’t generate an eighth rebirth. These are the six things that can’t be done.” 

%
\section*{{\suttatitleacronym AN 6.93}{\suttatitletranslation Things That Can’t Be Done (2nd) }{\suttatitleroot Dutiyaabhabbaṭṭhānasutta}}
\addcontentsline{toc}{section}{\tocacronym{AN 6.93} \toctranslation{Things That Can’t Be Done (2nd) } \tocroot{Dutiyaabhabbaṭṭhānasutta}}
\markboth{Things That Can’t Be Done (2nd) }{Dutiyaabhabbaṭṭhānasutta}
\extramarks{AN 6.93}{AN 6.93}

“Mendicants,\marginnote{1.1} these six things can’t be done. What six? A person accomplished in view can’t take conditions to be permanent, happiness, or self. They can’t do deeds with fixed result in the next life. They can’t fall back on purification through noisy, superstitious rites. They can’t seek outside of the Buddhist community for those worthy of religious donations. These are the six things that can’t be done.” 

%
\section*{{\suttatitleacronym AN 6.94}{\suttatitletranslation Things That Can’t Be Done (3rd) }{\suttatitleroot Tatiyaabhabbaṭṭhānasutta}}
\addcontentsline{toc}{section}{\tocacronym{AN 6.94} \toctranslation{Things That Can’t Be Done (3rd) } \tocroot{Tatiyaabhabbaṭṭhānasutta}}
\markboth{Things That Can’t Be Done (3rd) }{Tatiyaabhabbaṭṭhānasutta}
\extramarks{AN 6.94}{AN 6.94}

“Mendicants,\marginnote{1.1} these six things can’t be done. What six? A person accomplished in view can’t murder their mother or father or a perfected one. They can’t maliciously shed the blood of the Realized One. They can’t cause a schism in the \textsanskrit{Saṅgha}. They can’t acknowledge another teacher. These are the six things that can’t be done.” 

%
\section*{{\suttatitleacronym AN 6.95}{\suttatitletranslation Things That Can’t Be Done (4th) }{\suttatitleroot Catutthaabhabbaṭṭhānasutta}}
\addcontentsline{toc}{section}{\tocacronym{AN 6.95} \toctranslation{Things That Can’t Be Done (4th) } \tocroot{Catutthaabhabbaṭṭhānasutta}}
\markboth{Things That Can’t Be Done (4th) }{Catutthaabhabbaṭṭhānasutta}
\extramarks{AN 6.95}{AN 6.95}

“Mendicants,\marginnote{1.1} these six things can’t be done. What six? A person accomplished in view can’t fall back on the idea that pleasure and pain are made by oneself, or that they’re made by another, or that they’re made by both. Nor can they fall back on the idea that pleasure and pain arise by chance, not made by oneself, by another, or by both. Why is that? It is because a person accomplished in view has clearly seen causes and the phenomena that arise from causes. These are the six things that can’t be done.” 

%
\addtocontents{toc}{\let\protect\contentsline\protect\nopagecontentsline}
\chapter*{The Chapter on Benefit }
\addcontentsline{toc}{chapter}{\tocchapterline{The Chapter on Benefit }}
\addtocontents{toc}{\let\protect\contentsline\protect\oldcontentsline}

%
\section*{{\suttatitleacronym AN 6.96}{\suttatitletranslation Appearance }{\suttatitleroot Pātubhāvasutta}}
\addcontentsline{toc}{section}{\tocacronym{AN 6.96} \toctranslation{Appearance } \tocroot{Pātubhāvasutta}}
\markboth{Appearance }{Pātubhāvasutta}
\extramarks{AN 6.96}{AN 6.96}

“Mendicants,\marginnote{1.1} the appearance of six things is rare in the world. What six? A Realized One, a perfected one, a fully awakened Buddha. A person who teaches the teaching and training proclaimed by a Realized One. Rebirth in a civilized region. Unimpaired sense faculties. Being bright and clever. Enthusiasm for skillful qualities. The appearance of these six things is rare in the world.” 

%
\section*{{\suttatitleacronym AN 6.97}{\suttatitletranslation Benefit }{\suttatitleroot Ānisaṁsasutta}}
\addcontentsline{toc}{section}{\tocacronym{AN 6.97} \toctranslation{Benefit } \tocroot{Ānisaṁsasutta}}
\markboth{Benefit }{Ānisaṁsasutta}
\extramarks{AN 6.97}{AN 6.97}

“Mendicants,\marginnote{1.1} these are the six benefits of realizing the fruit of stream-entry. What six? You’re bound for the true teaching. You’re not liable to decline. You suffer only for a limited period. You have unshared knowledge. You’ve clearly seen causes and the phenomena that arise from causes. These are the six benefits of realizing the fruit of stream-entry.” 

%
\section*{{\suttatitleacronym AN 6.98}{\suttatitletranslation Impermanence }{\suttatitleroot Aniccasutta}}
\addcontentsline{toc}{section}{\tocacronym{AN 6.98} \toctranslation{Impermanence } \tocroot{Aniccasutta}}
\markboth{Impermanence }{Aniccasutta}
\extramarks{AN 6.98}{AN 6.98}

“Mendicants,\marginnote{1.1} it’s totally impossible for a mendicant who regards any condition as permanent to accept views that agree with the teaching. Without accepting views that agree with the teaching, it’s impossible to enter the sure path with regards to skillful qualities. Without entering the sure path, it’s impossible to realize the fruit of stream-entry, once-return, non-return, or perfection. 

It’s\marginnote{2.1} totally possible for a mendicant who regards all conditions as impermanent to accept views that agree with the teaching. Having accepted views that agree with the teaching, it’s possible to enter the sure path. Having entered the sure path, it’s possible to realize the fruit of stream-entry, once-return, non-return, or perfection.” 

%
\section*{{\suttatitleacronym AN 6.99}{\suttatitletranslation Suffering }{\suttatitleroot Dukkhasutta}}
\addcontentsline{toc}{section}{\tocacronym{AN 6.99} \toctranslation{Suffering } \tocroot{Dukkhasutta}}
\markboth{Suffering }{Dukkhasutta}
\extramarks{AN 6.99}{AN 6.99}

“Mendicants,\marginnote{1.1} it’s totally impossible for a mendicant who regards any condition as pleasurable to accept views that agree with the teaching. … It’s totally possible for a mendicant who regards all conditions as suffering to accept views that agree with the teaching. …” 

%
\section*{{\suttatitleacronym AN 6.100}{\suttatitletranslation Not-Self }{\suttatitleroot Anattasutta}}
\addcontentsline{toc}{section}{\tocacronym{AN 6.100} \toctranslation{Not-Self } \tocroot{Anattasutta}}
\markboth{Not-Self }{Anattasutta}
\extramarks{AN 6.100}{AN 6.100}

“Mendicants,\marginnote{1.1} it’s totally impossible for a mendicant who regards any condition as self to accept views that agree with the teaching. … It’s totally possible for a mendicant who regards all things as not-self to accept views that agree with the teaching. …” 

%
\section*{{\suttatitleacronym AN 6.101}{\suttatitletranslation Extinguished }{\suttatitleroot Nibbānasutta}}
\addcontentsline{toc}{section}{\tocacronym{AN 6.101} \toctranslation{Extinguished } \tocroot{Nibbānasutta}}
\markboth{Extinguished }{Nibbānasutta}
\extramarks{AN 6.101}{AN 6.101}

“Mendicants,\marginnote{1.1} it’s totally impossible for a mendicant who regards extinguishment as suffering to accept views that agree with the teaching. … 

It’s\marginnote{2.1} totally possible for a mendicant who regards extinguishment as pleasurable to accept views that agree with the teaching. …” 

%
\section*{{\suttatitleacronym AN 6.102}{\suttatitletranslation Transience }{\suttatitleroot Anavatthitasutta}}
\addcontentsline{toc}{section}{\tocacronym{AN 6.102} \toctranslation{Transience } \tocroot{Anavatthitasutta}}
\markboth{Transience }{Anavatthitasutta}
\extramarks{AN 6.102}{AN 6.102}

“Mendicants,\marginnote{1.1} seeing six benefits is quite enough to establish the perception of impermanence in all conditions without qualification. What six? ‘All conditions will appear to me as transient.’ ‘My mind will not delight anywhere in the world.’ ‘My mind will rise above the whole world.’ ‘My mind will incline to extinguishment.’ ‘My fetters will be given up.’ ‘I will achieve the ultimate goal of the ascetic life.’ Seeing these six benefits is quite enough to establish the perception of impermanence in all conditions without qualification.” 

%
\section*{{\suttatitleacronym AN 6.103}{\suttatitletranslation With a Drawn Sword }{\suttatitleroot Ukkhittāsikasutta}}
\addcontentsline{toc}{section}{\tocacronym{AN 6.103} \toctranslation{With a Drawn Sword } \tocroot{Ukkhittāsikasutta}}
\markboth{With a Drawn Sword }{Ukkhittāsikasutta}
\extramarks{AN 6.103}{AN 6.103}

“Mendicants,\marginnote{1.1} seeing six benefits is quite enough to establish the perception of suffering in all conditions without qualification. What six? ‘Perception of disillusionment will be established in me for all conditions, like a killer with a drawn sword.’ ‘My mind will rise above the whole world.’ ‘I will see extinguishment as peaceful.’ ‘My underlying tendencies will be uprooted.’ ‘I will fulfill my duty.’ ‘I will have served my Teacher with love.’ Seeing these six benefits is quite enough to establish the perception of suffering in all conditions without qualification.” 

%
\section*{{\suttatitleacronym AN 6.104}{\suttatitletranslation Non-identification }{\suttatitleroot Atammayasutta}}
\addcontentsline{toc}{section}{\tocacronym{AN 6.104} \toctranslation{Non-identification } \tocroot{Atammayasutta}}
\markboth{Non-identification }{Atammayasutta}
\extramarks{AN 6.104}{AN 6.104}

“Mendicants,\marginnote{1.1} seeing six benefits is quite enough to establish the perception of not-self in all things without qualification. What six? ‘I will be without identification in the whole world.’ ‘My egoism will stop.’ ‘My possessiveness will stop.’ ‘I will have unshared knowledge.’ ‘I will clearly see causes and the phenomena that arise from causes.’ Seeing these six benefits is quite enough to establish the perception of not-self in all things without qualification.” 

%
\section*{{\suttatitleacronym AN 6.105}{\suttatitletranslation States of Existence }{\suttatitleroot Bhavasutta}}
\addcontentsline{toc}{section}{\tocacronym{AN 6.105} \toctranslation{States of Existence } \tocroot{Bhavasutta}}
\markboth{States of Existence }{Bhavasutta}
\extramarks{AN 6.105}{AN 6.105}

“Mendicants,\marginnote{1.1} you should give up these three states of existence. And you should train in three trainings. What are the three states of existence you should give up? Existence in the sensual realm, the realm of luminous form, and the formless realm. These are the three states of existence you should give up. What are the three trainings you should train in? The training in the higher ethics, the higher mind, and the higher wisdom. These are the three trainings you should train in. When a mendicant has given up these three states of existence and has trained in these three trainings they’re called a mendicant who has cut off craving, untied the fetters, and by rightly comprehending conceit has made an end of suffering.” 

%
\section*{{\suttatitleacronym AN 6.106}{\suttatitletranslation Craving }{\suttatitleroot Taṇhāsutta}}
\addcontentsline{toc}{section}{\tocacronym{AN 6.106} \toctranslation{Craving } \tocroot{Taṇhāsutta}}
\markboth{Craving }{Taṇhāsutta}
\extramarks{AN 6.106}{AN 6.106}

“Mendicants,\marginnote{1.1} you should give up these three cravings and three conceits. What three cravings should you give up? Craving for sensual pleasures, craving for continued existence, and craving to end existence. These are the three cravings you should give up. What three conceits should you give up? Conceit, inferiority complex, and superiority complex. These are the three conceits you should give up. When a mendicant has given up these three cravings and these three conceits they’re called a mendicant who has cut off craving, untied the fetters, and by rightly comprehending conceit has made an end of suffering.” 

%
\addtocontents{toc}{\let\protect\contentsline\protect\nopagecontentsline}
\chapter*{The Chapter on Triads }
\addcontentsline{toc}{chapter}{\tocchapterline{The Chapter on Triads }}
\addtocontents{toc}{\let\protect\contentsline\protect\oldcontentsline}

%
\section*{{\suttatitleacronym AN 6.107}{\suttatitletranslation Greed }{\suttatitleroot Rāgasutta}}
\addcontentsline{toc}{section}{\tocacronym{AN 6.107} \toctranslation{Greed } \tocroot{Rāgasutta}}
\markboth{Greed }{Rāgasutta}
\extramarks{AN 6.107}{AN 6.107}

“Mendicants,\marginnote{1.1} there are these three things. What three? Greed, hate, and delusion. These are the three things. To give up these three things you should develop three things. What three? You should develop the perception of ugliness to give up greed, love to give up hate, and wisdom to give up delusion. These are the three things you should develop to give up those three things.” 

%
\section*{{\suttatitleacronym AN 6.108}{\suttatitletranslation Bad Conduct }{\suttatitleroot Duccaritasutta}}
\addcontentsline{toc}{section}{\tocacronym{AN 6.108} \toctranslation{Bad Conduct } \tocroot{Duccaritasutta}}
\markboth{Bad Conduct }{Duccaritasutta}
\extramarks{AN 6.108}{AN 6.108}

“Mendicants,\marginnote{1.1} there are these three things. What three? Bad conduct by way of body, speech, and mind. These are the three things. To give up these three things you should develop three things. What three? You should develop good bodily conduct to give up bad bodily conduct, good verbal conduct to give up bad verbal conduct, and good mental conduct to give up bad mental conduct. These are the three things you should develop to give up those three things.” 

%
\section*{{\suttatitleacronym AN 6.109}{\suttatitletranslation Thoughts }{\suttatitleroot Vitakkasutta}}
\addcontentsline{toc}{section}{\tocacronym{AN 6.109} \toctranslation{Thoughts } \tocroot{Vitakkasutta}}
\markboth{Thoughts }{Vitakkasutta}
\extramarks{AN 6.109}{AN 6.109}

“Mendicants,\marginnote{1.1} there are these three things. What three? Sensual, malicious, and cruel thoughts. These are the three things. To give up these three things you should develop three things. What three? You should develop thoughts of renunciation to give up sensual thoughts, thoughts of good will to give up malicious thoughts, and thoughts of harmlessness to give up cruel thoughts. These are the three things you should develop to give up those three things.” 

%
\section*{{\suttatitleacronym AN 6.110}{\suttatitletranslation Perceptions }{\suttatitleroot Saññāsutta}}
\addcontentsline{toc}{section}{\tocacronym{AN 6.110} \toctranslation{Perceptions } \tocroot{Saññāsutta}}
\markboth{Perceptions }{Saññāsutta}
\extramarks{AN 6.110}{AN 6.110}

“Mendicants,\marginnote{1.1} there are these three things. What three? Sensual, malicious, and cruel perceptions. These are the three things. To give up these three things you should develop three things. What three? You should develop perceptions of renunciation to give up sensual perceptions, perceptions of good will to give up malicious perceptions, and perceptions of harmlessness to give up cruel perceptions. These are the three things you should develop to give up those three things.” 

%
\section*{{\suttatitleacronym AN 6.111}{\suttatitletranslation Elements }{\suttatitleroot Dhātusutta}}
\addcontentsline{toc}{section}{\tocacronym{AN 6.111} \toctranslation{Elements } \tocroot{Dhātusutta}}
\markboth{Elements }{Dhātusutta}
\extramarks{AN 6.111}{AN 6.111}

“Mendicants,\marginnote{1.1} there are these three things. What three? The elements of sensuality, malice, and cruelty. These are the three things. To give up these three things you should develop three things. What three? You should develop the element of renunciation to give up the element of sensuality, the element of good will to give up the element of malice, and the element of harmlessness to give up the element of cruelty. These are the three things you should develop to give up those three things.” 

%
\section*{{\suttatitleacronym AN 6.112}{\suttatitletranslation Gratification }{\suttatitleroot Assādasutta}}
\addcontentsline{toc}{section}{\tocacronym{AN 6.112} \toctranslation{Gratification } \tocroot{Assādasutta}}
\markboth{Gratification }{Assādasutta}
\extramarks{AN 6.112}{AN 6.112}

“Mendicants,\marginnote{1.1} there are these three things. What three? The view that things are gratifying, the view of self, and wrong view. These are the three things. To give up these three things you should develop three things. What three? You should develop the perception of impermanence to give up the view that things are gratifying; the perception of not-self to give up the view of self; and right view to give up wrong view. These are the three things you should develop to give up those three things.” 

%
\section*{{\suttatitleacronym AN 6.113}{\suttatitletranslation Discontent }{\suttatitleroot Aratisutta}}
\addcontentsline{toc}{section}{\tocacronym{AN 6.113} \toctranslation{Discontent } \tocroot{Aratisutta}}
\markboth{Discontent }{Aratisutta}
\extramarks{AN 6.113}{AN 6.113}

“Mendicants,\marginnote{1.1} there are these three things. What three? Discontent, cruelty, and unprincipled conduct. These are the three things. To give up these three things you should develop three things. What three? You should develop rejoicing to give up discontent, harmlessness to give up cruelty, and principled conduct to give up unprincipled conduct. These are the three things you should develop to give up those three things.” 

%
\section*{{\suttatitleacronym AN 6.114}{\suttatitletranslation Contentment }{\suttatitleroot Santuṭṭhitāsutta}}
\addcontentsline{toc}{section}{\tocacronym{AN 6.114} \toctranslation{Contentment } \tocroot{Santuṭṭhitāsutta}}
\markboth{Contentment }{Santuṭṭhitāsutta}
\extramarks{AN 6.114}{AN 6.114}

“Mendicants,\marginnote{1.1} there are these three things. What three? Lack of contentment, lack of situational awareness, and having many wishes. These are the three things. To give up these three things you should develop three things. What three? You should develop contentment to give up lack of contentment, situational awareness to give up lack of situational awareness, and having few wishes to give up having many wishes. These are the three things you should develop to give up those three things.” 

%
\section*{{\suttatitleacronym AN 6.115}{\suttatitletranslation Hard to Admonish }{\suttatitleroot Dovacassatāsutta}}
\addcontentsline{toc}{section}{\tocacronym{AN 6.115} \toctranslation{Hard to Admonish } \tocroot{Dovacassatāsutta}}
\markboth{Hard to Admonish }{Dovacassatāsutta}
\extramarks{AN 6.115}{AN 6.115}

“Mendicants,\marginnote{1.1} there are these three things. What three? Being hard to admonish, bad friendship, and a distracted mind. These are the three things. To give up these three things you should develop three things. What three? You should develop being easy to admonish to give up being hard to admonish, good friendship to give up bad friendship, and mindfulness of breathing to give up a distracted mind. These are the three things you should develop to give up those three things.” 

%
\section*{{\suttatitleacronym AN 6.116}{\suttatitletranslation Restlessness }{\suttatitleroot Uddhaccasutta}}
\addcontentsline{toc}{section}{\tocacronym{AN 6.116} \toctranslation{Restlessness } \tocroot{Uddhaccasutta}}
\markboth{Restlessness }{Uddhaccasutta}
\extramarks{AN 6.116}{AN 6.116}

“Mendicants,\marginnote{1.1} there are these three things. What three? Restlessness, lack of restraint, and negligence. These are the three things. To give up these three things you should develop three things. What three? You should develop serenity to give up restlessness, restraint to give up lack of restraint, and diligence to give up negligence. These are the three things you should develop to give up those three things.” 

%
\addtocontents{toc}{\let\protect\contentsline\protect\nopagecontentsline}
\chapter*{The Chapter on the Ascetic Life }
\addcontentsline{toc}{chapter}{\tocchapterline{The Chapter on the Ascetic Life }}
\addtocontents{toc}{\let\protect\contentsline\protect\oldcontentsline}

%
\section*{{\suttatitleacronym AN 6.117}{\suttatitletranslation Observing the Body }{\suttatitleroot Kāyānupassīsutta}}
\addcontentsline{toc}{section}{\tocacronym{AN 6.117} \toctranslation{Observing the Body } \tocroot{Kāyānupassīsutta}}
\markboth{Observing the Body }{Kāyānupassīsutta}
\extramarks{AN 6.117}{AN 6.117}

“Mendicants,\marginnote{1.1} without giving up these six qualities you can’t meditate observing an aspect of the body. What six? Relishing work, talk, sleep, and company, not guarding the sense doors, and eating too much. Without giving up these six qualities you can’t meditate observing an aspect of the body. 

But\marginnote{2.1} after giving up these six qualities you can meditate observing an aspect of the body. What six? Relishing work, talk, sleep, and company, not guarding the sense doors, and eating too much. After giving up these six qualities you can meditate observing an aspect of the body.” 

%
\section*{{\suttatitleacronym AN 6.118}{\suttatitletranslation Observing Principles, Etc. }{\suttatitleroot Dhammānupassīsutta}}
\addcontentsline{toc}{section}{\tocacronym{AN 6.118} \toctranslation{Observing Principles, Etc. } \tocroot{Dhammānupassīsutta}}
\markboth{Observing Principles, Etc. }{Dhammānupassīsutta}
\extramarks{AN 6.118}{AN 6.118}

“Mendicants,\marginnote{1.1} without giving up six things you can’t meditate observing an aspect of the body internally … body externally … body internally and externally … feelings internally … feelings externally … feelings internally and externally … mind internally … mind externally … mind internally and externally … principles internally … principles externally … principles internally and externally. What six? Relishing work, talk, sleep, and company, not guarding the sense doors, and eating too much. After giving up these six qualities you can meditate observing an aspect of principles internally and externally.” 

%
\section*{{\suttatitleacronym AN 6.119}{\suttatitletranslation About Tapussa }{\suttatitleroot Tapussasutta}}
\addcontentsline{toc}{section}{\tocacronym{AN 6.119} \toctranslation{About Tapussa } \tocroot{Tapussasutta}}
\markboth{About Tapussa }{Tapussasutta}
\extramarks{AN 6.119}{AN 6.119}

“Mendicants,\marginnote{1.1} having six qualities the householder Tapussa is certain about the Realized One, sees the deathless, and lives having realized the deathless. What six? Experiential confidence in the Buddha, the teaching, and the \textsanskrit{Saṅgha}, and noble ethics, knowledge, and freedom. Having these six qualities the householder Tapussa is certain about the Realized One, sees the deathless, and lives having realized the deathless.” 

%
\section*{{\suttatitleacronym AN 6.120–139}{\suttatitletranslation About Bhallika, Etc. }{\suttatitleroot Bhallikādisutta}}
\addcontentsline{toc}{section}{\tocacronym{AN 6.120–139} \toctranslation{About Bhallika, Etc. } \tocroot{Bhallikādisutta}}
\markboth{About Bhallika, Etc. }{Bhallikādisutta}
\extramarks{AN 6.120–139}{AN 6.120–139}

“Mendicants,\marginnote{1.1} having six qualities the householders Bhallika … Sudatta \textsanskrit{Anāthapiṇḍika} … Citta of \textsanskrit{Macchikāsaṇḍa} … Hatthaka of \textsanskrit{Āḷavī} … \textsanskrit{Mahānāma} the Sakyan … Ugga of \textsanskrit{Vesālī} … Uggata … \textsanskrit{Sūra} of \textsanskrit{Ambaṭṭha} … \textsanskrit{Jīvaka} \textsanskrit{Komārabhacca} … Nakula’s father … \textsanskrit{Tavakaṇṇika} … \textsanskrit{Pūraṇa} … Isidatta … \textsanskrit{Sandhāna} … Vijaya … \textsanskrit{Vijayamāhita} … \textsanskrit{Meṇḍaka} … the lay followers \textsanskrit{Vāseṭṭha} … \textsanskrit{Ariṭṭha} … and \textsanskrit{Sāragga} are certain about the Realized One, see the deathless, and live having realized the deathless. What six? Experiential confidence in the Buddha, the teaching, and the \textsanskrit{Saṅgha}, and noble ethics, knowledge, and freedom. Having these six qualities the lay follower \textsanskrit{Sāragga} is certain about the Realized One, sees the deathless, and lives having realized the deathless.” 

%
\addtocontents{toc}{\let\protect\contentsline\protect\nopagecontentsline}
\chapter*{Abbreviated Texts Beginning With Greed }
\addcontentsline{toc}{chapter}{\tocchapterline{Abbreviated Texts Beginning With Greed }}
\addtocontents{toc}{\let\protect\contentsline\protect\oldcontentsline}

%
\section*{{\suttatitleacronym AN 6.140}{\suttatitletranslation Untitled Discourse on Greed (1st) }{\suttatitleroot \textasciitilde }}
\addcontentsline{toc}{section}{\tocacronym{AN 6.140} \toctranslation{Untitled Discourse on Greed (1st) } \tocroot{\textasciitilde }}
\markboth{Untitled Discourse on Greed (1st) }{\textasciitilde }
\extramarks{AN 6.140}{AN 6.140}

“For\marginnote{1.1} insight into greed, six things should be developed. What six? The unsurpassable seeing, listening, acquisition, training, service, and recollection. For insight into greed, these six things should be developed.” 

%
\section*{{\suttatitleacronym AN 6.141}{\suttatitletranslation Untitled Discourse on Greed (2nd) }{\suttatitleroot \textasciitilde }}
\addcontentsline{toc}{section}{\tocacronym{AN 6.141} \toctranslation{Untitled Discourse on Greed (2nd) } \tocroot{\textasciitilde }}
\markboth{Untitled Discourse on Greed (2nd) }{\textasciitilde }
\extramarks{AN 6.141}{AN 6.141}

“For\marginnote{1.1} insight into greed, six things should be developed. What six? The recollection of the Buddha, the teaching, the \textsanskrit{Saṅgha}, ethics, generosity, and the deities. For insight into greed, these six things should be developed.” 

%
\section*{{\suttatitleacronym AN 6.142}{\suttatitletranslation Untitled Discourse on Greed (3rd) }{\suttatitleroot \textasciitilde }}
\addcontentsline{toc}{section}{\tocacronym{AN 6.142} \toctranslation{Untitled Discourse on Greed (3rd) } \tocroot{\textasciitilde }}
\markboth{Untitled Discourse on Greed (3rd) }{\textasciitilde }
\extramarks{AN 6.142}{AN 6.142}

“For\marginnote{1.1} insight into greed, six things should be developed. What six? The perception of impermanence, the perception of suffering in impermanence, the perception of not-self in suffering, the perception of giving up, the perception of fading away, and the perception of cessation. For insight into greed, these six things should be developed.” 

%
\section*{{\suttatitleacronym AN 6.143–169}{\suttatitletranslation Untitled Discourses on Greed, Etc. }{\suttatitleroot \textasciitilde }}
\addcontentsline{toc}{section}{\tocacronym{AN 6.143–169} \toctranslation{Untitled Discourses on Greed, Etc. } \tocroot{\textasciitilde }}
\markboth{Untitled Discourses on Greed, Etc. }{\textasciitilde }
\extramarks{AN 6.143–169}{AN 6.143–169}

“For\marginnote{1.1} the complete understanding of greed … complete ending … giving up … ending … vanishing … fading away … cessation … giving away … letting go of greed these six things should be developed.” 

%
\section*{{\suttatitleacronym AN 6.170–649}{\suttatitletranslation Untitled Discourses on Hate, Etc. }{\suttatitleroot \textasciitilde }}
\addcontentsline{toc}{section}{\tocacronym{AN 6.170–649} \toctranslation{Untitled Discourses on Hate, Etc. } \tocroot{\textasciitilde }}
\markboth{Untitled Discourses on Hate, Etc. }{\textasciitilde }
\extramarks{AN 6.170–649}{AN 6.170–649}

“Of\marginnote{1.1} hate … delusion … anger … hostility … disdain … contempt … jealousy … stinginess … deceitfulness … deviousness … obstinacy … aggression … conceit … arrogance … vanity … for insight into negligence … complete understanding … complete ending … giving up … ending … vanishing … fading away … cessation … giving away … letting go of negligence these six things should be developed.” 

That\marginnote{1.27} is what the Buddha said. Satisfied, the mendicants were happy with what the Buddha said. 

\scendbook{The Book of the Sixes is finished. }

%
\backmatter%
\chapter*{Colophon}
\addcontentsline{toc}{chapter}{Colophon}
\markboth{Colophon}{Colophon}

\section*{The Translator}

Bhikkhu Sujato was born as Anthony Aidan Best on 4/11/1966 in Perth, Western Australia. He grew up in the pleasant suburbs of Mt Lawley and Attadale alongside his sister Nicola, who was the good child. His mother, Margaret Lorraine Huntsman née Pinder, said “he’ll either be a priest or a poet”, while his father, Anthony Thomas Best, advised him to “never do anything for money”. He attended Aquinas College, a Catholic school, where he decided to become an atheist. At the University of WA he studied philosophy, aiming to learn what he wanted to do with his life. Finding that what he wanted to do was play guitar, he dropped out. His main band was named Martha’s Vineyard, which achieved modest success in the indie circuit. 

A seemingly random encounter with a roadside joey took him to Thailand, where he entered his first meditation retreat at Wat Ram Poeng, Chieng Mai in 1992. Feeling the call to the Buddha’s path, he took full ordination in Wat Pa Nanachat in 1994, where his teachers were Ajahn Pasanno and Ajahn Jayasaro. In 1997 he returned to Perth to study with Ajahn Brahm at Bodhinyana Monastery. 

He spent several years practicing in seclusion in Malaysia and Thailand before establishing Santi Forest Monastery in Bundanoon, NSW, in 2003. There he was instrumental in supporting the establishment of the Theravada bhikkhuni order in Australia and advocating for women’s rights. He continues to teach in Australia and globally, with a special concern for the moral implications of climate change and other forms of environmental destruction. He has published a series of books of original and groundbreaking research on early Buddhism. 

In 2005 he founded SuttaCentral together with Rod Bucknell and John Kelly. In 2015, seeing the need for a complete, accurate, plain English translation of the Pali texts, he undertook the task, spending nearly three years in isolation on the isle of Qi Mei off the coast of the nation of Taiwan. He completed the four main \textsanskrit{Nikāyas} in 2018, and the early books of the Khuddaka \textsanskrit{Nikāya} were complete by 2021. All this work is dedicated to the public domain and is entirely free of copyright encumbrance. 

In 2019 he returned to Sydney where he established Lokanta Vihara (The Monastery at the End of the World). 

\section*{Creation Process}

Primary source was the digital \textsanskrit{Mahāsaṅgīti} edition of the Pali \textsanskrit{Tipiṭaka}. Translated from the Pali, with reference to several English translations, especially those of Bhikkhu Bodhi.

\section*{The Translation}

This translation was part of a project to translate the four Pali \textsanskrit{Nikāyas} with the following aims: plain, approachable English; consistent terminology; accurate rendition of the Pali; free of copyright. It was made during 2016–2018 while Bhikkhu Sujato was staying in Qimei, Taiwan.

\section*{About SuttaCentral}

SuttaCentral publishes early Buddhist texts. Since 2005 we have provided root texts in Pali, Chinese, Sanskrit, Tibetan, and other languages, parallels between these texts, and translations in many modern languages. We build on the work of generations of scholars, and offer our contribution freely.

SuttaCentral is driven by volunteer contributions, and in addition we employ professional developers. We offer a sponsorship program for high quality translations from the original languages. Financial support for SuttaCentral is handled by the SuttaCentral Development Trust, a charitable trust registered in Australia.

\section*{About Bilara}

“Bilara” means “cat” in Pali, and it is the name of our Computer Assisted Translation (CAT) software. Bilara is a web app that enables translators to translate early Buddhist texts into their own language. These translations are published on SuttaCentral with the root text and translation side by side.

\section*{About SuttaCentral Editions}

The SuttaCentral Editions project makes high quality books from selected Bilara translations. These are published in formats including HTML, EPUB, PDF, and print.

If you want to print any of our Editions, please let us know and we will help prepare a file to your specifications.

%
\end{document}