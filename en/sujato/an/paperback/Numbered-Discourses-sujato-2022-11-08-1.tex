\documentclass[12pt,openany]{book}%
\usepackage{lastpage}%
%
\usepackage[inner=1in, outer=1in, top=.7in, bottom=1in, papersize={6in,9in}, headheight=13pt]{geometry}
\usepackage{polyglossia}
\usepackage[12pt]{moresize}
\usepackage{soul}%
\usepackage{microtype}
\usepackage{tocbasic}
\usepackage{realscripts}
\usepackage{epigraph}%
\usepackage{setspace}%
\usepackage{sectsty}
\usepackage{fontspec}
\usepackage{marginnote}
\usepackage[bottom]{footmisc}
\usepackage{enumitem}
\usepackage{fancyhdr}
\usepackage{extramarks}
\usepackage{graphicx}
\usepackage{verse}
\usepackage{relsize}
\usepackage{etoolbox}
\usepackage[a-3u]{pdfx}

\hypersetup{
colorlinks=true,
urlcolor=black,
linkcolor=black,
citecolor=black
}

% use a small amount of tracking on small caps
\SetTracking[ spacing = {25*,166, } ]{ encoding = *, shape = sc }{ 25 }

% add a blank page
\newcommand{\blankpage}{
\newpage
\thispagestyle{empty}
\mbox{}
\newpage
}

% define languages
\setdefaultlanguage[]{english}
\setotherlanguage[script=Latin]{sanskrit}

%\usepackage{pagegrid}
%\pagegridsetup{top-left, step=.25in}

% define fonts
% use if arno sanskrit is unavailable
%\setmainfont{Gentium Plus}
%\newfontfamily\Semiboldsubheadfont[]{Gentium Plus}
%\newfontfamily\Semiboldnormalfont[]{Gentium Plus}
%\newfontfamily\Lightfont[]{Gentium Plus}
%\newfontfamily\Marginalfont[]{Gentium Plus}
%\newfontfamily\Allsmallcapsfont[RawFeature=+c2sc]{Gentium Plus}
%\newfontfamily\Noligaturefont[Renderer=Basic]{Gentium Plus}
%\newfontfamily\Noligaturecaptionfont[Renderer=Basic]{Gentium Plus}
%\newfontfamily\Fleuronfont[Ornament=1]{Gentium Plus}

% use if arno sanskrit is available. display is applied to \chapter and \part, subhead to \section and \subsection. When specifying semibold, the italic must be defined.
\setmainfont[Numbers=OldStyle]{Arno Pro}
\newfontfamily\Semibolddisplayfont[BoldItalicFont = Arno Pro Semibold Italic Display]{Arno Pro Semibold Display} %
\newfontfamily\Semiboldsubheadfont[BoldItalicFont = Arno Pro Semibold Italic Subhead]{Arno Pro Semibold Subhead}
\newfontfamily\Semiboldnormalfont[BoldItalicFont = Arno Pro Semibold Italic]{Arno Pro Semibold}
\newfontfamily\Marginalfont[RawFeature=+subs]{Arno Pro Regular}
\newfontfamily\Allsmallcapsfont[RawFeature=+c2sc]{Arno Pro}
\newfontfamily\Noligaturefont[Renderer=Basic]{Arno Pro}
\newfontfamily\Noligaturecaptionfont[Renderer=Basic]{Arno Pro Caption}

% chinese fonts
\newfontfamily\cjk{Noto Serif TC}
\newcommand*{\langlzh}[1]{\cjk{#1}\normalfont}%

% logo
\newfontfamily\Logofont{sclogo.ttf}
\newcommand*{\sclogo}[1]{\large\Logofont{#1}}

% use subscript numerals for margin notes
\renewcommand*{\marginfont}{\Marginalfont}

% ensure margin notes have consistent vertical alignment
\renewcommand*{\marginnotevadjust}{-.17em}

% use compact lists
\setitemize{noitemsep,leftmargin=1em}
\setenumerate{noitemsep,leftmargin=1em}
\setdescription{noitemsep, style=unboxed, leftmargin=0em}

% style ToC
\DeclareTOCStyleEntries[
  raggedentrytext,
  linefill=\hfill,
  pagenumberwidth=.5in,
  pagenumberformat=\normalfont,
  entryformat=\normalfont
]{tocline}{chapter,section}


  \setlength\topsep{0pt}%
  \setlength\parskip{0pt}%

% define new \centerpars command for use in ToC. This ensures centering, proper wrapping, and no page break after
\def\startcenter{%
  \par
  \begingroup
  \leftskip=0pt plus 1fil
  \rightskip=\leftskip
  \parindent=0pt
  \parfillskip=0pt
}
\def\stopcenter{%
  \par
  \endgroup
}
\long\def\centerpars#1{\startcenter#1\stopcenter}

% redefine part, so that it adds a toc entry without page number
\let\oldcontentsline\contentsline
\newcommand{\nopagecontentsline}[3]{\oldcontentsline{#1}{#2}{}}

    \makeatletter
\renewcommand*\l@part[2]{%
  \ifnum \c@tocdepth >-2\relax
    \addpenalty{-\@highpenalty}%
    \addvspace{0em \@plus\p@}%
    \setlength\@tempdima{3em}%
    \begingroup
      \parindent \z@ \rightskip \@pnumwidth
      \parfillskip -\@pnumwidth
      {\leavevmode
       \setstretch{.85}\large\scshape\centerpars{#1}\vspace*{-1em}\llap{#2}}\par
       \nobreak
         \global\@nobreaktrue
         \everypar{\global\@nobreakfalse\everypar{}}%
    \endgroup
  \fi}
\makeatother

\makeatletter
\def\@pnumwidth{2em}
\makeatother

% define new sectioning command, which is only used in volumes where the pannasa is found in some parts but not others, especially in an and sn

\newcommand*{\pannasa}[1]{\clearpage\thispagestyle{empty}\begin{center}\vspace*{14em}\setstretch{.85}\huge\itshape\scshape\MakeLowercase{#1}\end{center}}

    \makeatletter
\newcommand*\l@pannasa[2]{%
  \ifnum \c@tocdepth >-2\relax
    \addpenalty{-\@highpenalty}%
    \addvspace{.5em \@plus\p@}%
    \setlength\@tempdima{3em}%
    \begingroup
      \parindent \z@ \rightskip \@pnumwidth
      \parfillskip -\@pnumwidth
      {\leavevmode
       \setstretch{.85}\large\itshape\scshape\lowercase{\centerpars{#1}}\vspace*{-1em}\llap{#2}}\par
       \nobreak
         \global\@nobreaktrue
         \everypar{\global\@nobreakfalse\everypar{}}%
    \endgroup
  \fi}
\makeatother

% don't put page number on first page of toc (relies on etoolbox)
\patchcmd{\chapter}{plain}{empty}{}{}

% global line height
\setstretch{1.05}

% allow linebreak after em-dash
\catcode`\—=13
\protected\def—{\unskip\textemdash\allowbreak}

% style headings with secsty. chapter and section are defined per-edition
\partfont{\setstretch{.85}\normalfont\centering\textsc}
\subsectionfont{\setstretch{.85}\Semiboldsubheadfont}%
\subsubsectionfont{\setstretch{.85}\Semiboldnormalfont}

% style elements of suttatitle
\newcommand*{\suttatitleacronym}[1]{\smaller[2]{#1}\vspace*{.3em}}
\newcommand*{\suttatitletranslation}[1]{\linebreak{#1}}
\newcommand*{\suttatitleroot}[1]{\linebreak\smaller[2]\itshape{#1}}

\DeclareTOCStyleEntries[
  indent=3.3em,
  dynindent,
  beforeskip=.2em plus -2pt minus -1pt,
]{tocline}{section}

\DeclareTOCStyleEntries[
  indent=0em,
  dynindent,
  beforeskip=.4em plus -2pt minus -1pt,
]{tocline}{chapter}

\newcommand*{\tocacronym}[1]{\hspace*{-3.3em}{#1}\quad}
\newcommand*{\toctranslation}[1]{#1}
\newcommand*{\tocroot}[1]{(\textit{#1})}
\newcommand*{\tocchapterline}[1]{\bfseries\itshape{#1}}


% redefine paragraph and subparagraph headings to not be inline
\makeatletter
% Change the style of paragraph headings %
\renewcommand\paragraph{\@startsection{paragraph}{4}{\z@}%
            {-2.5ex\@plus -1ex \@minus -.25ex}%
            {1.25ex \@plus .25ex}%
            {\noindent\Semiboldnormalfont\normalsize}}

% Change the style of subparagraph headings %
\renewcommand\subparagraph{\@startsection{subparagraph}{5}{\z@}%
            {-2.5ex\@plus -1ex \@minus -.25ex}%
            {1.25ex \@plus .25ex}%
            {\noindent\Semiboldnormalfont\small}}
\makeatother

% use etoolbox to suppress page numbers on \part
\patchcmd{\part}{\thispagestyle{plain}}{\thispagestyle{empty}}
  {}{\errmessage{Cannot patch \string\part}}

% and to reduce margins on quotation
\patchcmd{\quotation}{\rightmargin}{\leftmargin 1.2em \rightmargin}{}{}
\AtBeginEnvironment{quotation}{\small}

% titlepage
\newcommand*{\titlepageTranslationTitle}[1]{{\begin{center}\begin{large}{#1}\end{large}\end{center}}}
\newcommand*{\titlepageCreatorName}[1]{{\begin{center}\begin{normalsize}{#1}\end{normalsize}\end{center}}}

% halftitlepage
\newcommand*{\halftitlepageTranslationTitle}[1]{\setstretch{2.5}{\begin{Huge}\uppercase{\so{#1}}\end{Huge}}}
\newcommand*{\halftitlepageTranslationSubtitle}[1]{\setstretch{1.2}{\begin{large}{#1}\end{large}}}
\newcommand*{\halftitlepageFleuron}[1]{{\begin{large}\Fleuronfont{{#1}}\end{large}}}
\newcommand*{\halftitlepageByline}[1]{{\begin{normalsize}\textit{{#1}}\end{normalsize}}}
\newcommand*{\halftitlepageCreatorName}[1]{{\begin{LARGE}{\textsc{#1}}\end{LARGE}}}
\newcommand*{\halftitlepageVolumeNumber}[1]{{\begin{normalsize}{\Allsmallcapsfont{\textsc{#1}}}\end{normalsize}}}
\newcommand*{\halftitlepageVolumeAcronym}[1]{{\begin{normalsize}{#1}\end{normalsize}}}
\newcommand*{\halftitlepageVolumeTranslationTitle}[1]{{\begin{Large}{\textsc{#1}}\end{Large}}}
\newcommand*{\halftitlepageVolumeRootTitle}[1]{{\begin{normalsize}{\Allsmallcapsfont{\textsc{\itshape #1}}}\end{normalsize}}}
\newcommand*{\halftitlepagePublisher}[1]{{\begin{large}{\Noligaturecaptionfont\textsc{#1}}\end{large}}}

% epigraph
\renewcommand{\epigraphflush}{center}
\renewcommand*{\epigraphwidth}{.85\textwidth}
\newcommand*{\epigraphTranslatedTitle}[1]{\vspace*{.5em}\footnotesize\textsc{#1}\\}%
\newcommand*{\epigraphRootTitle}[1]{\footnotesize\textit{#1}\\}%
\newcommand*{\epigraphReference}[1]{\footnotesize{#1}}%

% custom commands for html styling classes
\newcommand*{\scnamo}[1]{\begin{center}\textit{#1}\end{center}}
\newcommand*{\scendsection}[1]{\begin{center}\textit{#1}\end{center}}
\newcommand*{\scendsutta}[1]{\begin{center}\textit{#1}\end{center}}
\newcommand*{\scendbook}[1]{\begin{center}\uppercase{#1}\end{center}}
\newcommand*{\scendkanda}[1]{\begin{center}\textbf{#1}\end{center}}
\newcommand*{\scend}[1]{\begin{center}\textit{#1}\end{center}}
\newcommand*{\scuddanaintro}[1]{\textit{#1}}
\newcommand*{\scendvagga}[1]{\begin{center}\textbf{#1}\end{center}}
\newcommand*{\scrule}[1]{\textbf{#1}}
\newcommand*{\scadd}[1]{\textit{#1}}
\newcommand*{\scevam}[1]{\textsc{#1}}
\newcommand*{\scspeaker}[1]{\hspace{2em}\textit{#1}}
\newcommand*{\scbyline}[1]{\begin{flushright}\textit{#1}\end{flushright}\bigskip}

% custom command for thematic break = hr
\newcommand*{\thematicbreak}{\begin{center}\rule[.5ex]{6em}{.4pt}\begin{normalsize}\quad\Fleuronfont{•}\quad\end{normalsize}\rule[.5ex]{6em}{.4pt}\end{center}}

% manage and style page header and footer. "fancy" has header and footer, "plain" has footer only

\pagestyle{fancy}
\fancyhf{}
\fancyfoot[RE,LO]{\thepage}
\fancyfoot[LE,RO]{\footnotesize\lastleftxmark}
\fancyhead[CE]{\setstretch{.85}\Noligaturefont\MakeLowercase{\textsc{\firstrightmark}}}
\fancyhead[CO]{\setstretch{.85}\Noligaturefont\MakeLowercase{\textsc{\firstleftmark}}}
\renewcommand{\headrulewidth}{0pt}
\fancypagestyle{plain}{ %
\fancyhf{} % remove everything
\fancyfoot[RE,LO]{\thepage}
\fancyfoot[LE,RO]{\footnotesize\lastleftxmark}
\renewcommand{\headrulewidth}{0pt}
\renewcommand{\footrulewidth}{0pt}}

% style footnotes
\setlength{\skip\footins}{1em}

\makeatletter
\newcommand{\@makefntextcustom}[1]{%
    \parindent 0em%
    \thefootnote.\enskip #1%
}
\renewcommand{\@makefntext}[1]{\@makefntextcustom{#1}}
\makeatother

% hang quotes (requires microtype)
\microtypesetup{
  protrusion = true,
  expansion  = true,
  tracking   = true,
  factor     = 1000,
  patch      = all,
  final
}

% Custom protrusion rules to allow hanging punctuation
\SetProtrusion
{ encoding = *}
{
% char   right left
  {-} = {    , 500 },
  % Double Quotes
  \textquotedblleft
      = {1000,     },
  \textquotedblright
      = {    , 1000},
  \quotedblbase
      = {1000,     },
  % Single Quotes
  \textquoteleft
      = {1000,     },
  \textquoteright
      = {    , 1000},
  \quotesinglbase
      = {1000,     }
}

% make latex use actual font em for parindent, not Computer Modern Roman
\AtBeginDocument{\setlength{\parindent}{1em}}%
%

% Default values; a bit sloppier than normal
\tolerance 1414
\hbadness 1414
\emergencystretch 1.5em
\hfuzz 0.3pt
\clubpenalty = 10000
\widowpenalty = 10000
\displaywidowpenalty = 10000
\hfuzz \vfuzz
 \raggedbottom%

\title{Numbered Discourses}
\author{Bhikkhu Sujato}
\date{}%
% define a different fleuron for each edition
\newfontfamily\Fleuronfont[Ornament=18]{Arno Pro}

% Define heading styles per edition for chapter and section. Suttatitle can be either of these, depending on the volume. 

\let\oldfrontmatter\frontmatter
\renewcommand{\frontmatter}{%
\chapterfont{\setstretch{.85}\normalfont\centering}%
\sectionfont{\setstretch{.85}\Semiboldsubheadfont}%
\oldfrontmatter}

\let\oldmainmatter\mainmatter
\renewcommand{\mainmatter}{%
\chapterfont{\setstretch{.85}\normalfont\centering}%
\sectionfont{\setstretch{.85}\normalfont\centering}%
\oldmainmatter}

\let\oldbackmatter\backmatter
\renewcommand{\backmatter}{%
\chapterfont{\setstretch{.85}\normalfont\centering}%
\sectionfont{\setstretch{.85}\Semiboldsubheadfont}%
\oldbackmatter}
%
%
\begin{document}%
\normalsize%
\frontmatter%
\setlength{\parindent}{0cm}

\pagestyle{empty}

\maketitle

\blankpage%
\begin{center}

\vspace*{2.2em}

\halftitlepageTranslationTitle{Numbered Discourses}

\vspace*{1em}

\halftitlepageTranslationSubtitle{A sensible translation of the Aṅguttara Nikāya}

\vspace*{2em}

\halftitlepageFleuron{•}

\vspace*{2em}

\halftitlepageByline{translated and introduced by}

\vspace*{.5em}

\halftitlepageCreatorName{Bhikkhu Sujato}

\vspace*{4em}

\halftitlepageVolumeNumber{Volume 1}

\smallskip

\halftitlepageVolumeAcronym{AN 1–3}

\smallskip

\halftitlepageVolumeTranslationTitle{}

\smallskip

\halftitlepageVolumeRootTitle{}

\vspace*{\fill}

\sclogo{0}
 \halftitlepagePublisher{SuttaCentral}

\end{center}

\newpage
%
\setstretch{1.05}

\begin{footnotesize}

\textit{Numbered Discourses} is a translation of the Aṅguttaranikāya by Bhikkhu Sujato.

\medskip

Creative Commons Zero (CC0)

To the extent possible under law, Bhikkhu Sujato has waived all copyright and related or neighboring rights to \textit{Numbered Discourses}.

\medskip

This work is published from Australia.

\begin{center}
\textit{This translation is an expression of an ancient spiritual text that has been passed down by the Buddhist tradition for the benefit of all sentient beings. It is dedicated to the public domain via Creative Commons Zero (CC0). You are encouraged to copy, reproduce, adapt, alter, or otherwise make use of this translation. The translator respectfully requests that any use be in accordance with the values and principles of the Buddhist community.}
\end{center}

\medskip

\begin{description}
    \item[Web publication date] 2018
    \item[This edition] 2022-11-08 07:11:28
    \item[Publication type] paperback
    \item[Edition] ed5
    \item[Number of volumes] 5
    \item[Publication ISBN] 978-1-76132-037-8
    \item[Publication URL] https://suttacentral.net/editions/an/en/sujato
    \item[Source URL] https://github.com/suttacentral/bilara-data/tree/published/translation/en/sujato/sutta/an
    \item[Publication number] scpub5
\end{description}

\medskip

Published by SuttaCentral

\medskip

\textit{SuttaCentral,\\
c/o Alwis \& Alwis Pty Ltd\\
Kaurna Country,\\
Suite 12,\\
198 Greenhill Road,\\
Eastwood,\\
SA 5063,\\
Australia}

\end{footnotesize}

\newpage

\setlength{\parindent}{1.5em}%%
\newpage

\vspace*{\fill}

\begin{center}
\epigraph{These two things lead to the continuation, persistence, and enduring of the true teaching. What two? The words and phrases are well organized, and the meaning is correctly interpreted. When the words and phrases are well organized, the meaning is correctly interpreted. These two things lead to the continuation, persistence, and enduring of the true teaching.}
{
\epigraphTranslatedTitle{}
\epigraphRootTitle{}
\epigraphReference{\textsanskrit{Aṅguttara} \textsanskrit{Nikāya} 2.20}
}
\end{center}

\vspace*{2in}

\vspace*{\fill}

\blankpage%

\setlength{\parindent}{1em}
%
\tableofcontents
\newpage
\pagestyle{fancy}
%
\chapter*{The SuttaCentral Editions Series}
\addcontentsline{toc}{chapter}{The SuttaCentral Editions Series}
\markboth{The SuttaCentral Editions Series}{The SuttaCentral Editions Series}

Since 2005 SuttaCentral has provided access to the texts, translations, and parallels of early Buddhist texts. In 2018 we started creating and publishing our own translations of these seminal spiritual classics. The “Editions” series now makes selected translations available as books in various forms, including print, PDF, and EPUB.

Editions are selected from our most complete, well-crafted, and reliable translations. They aim to bring these texts to a wider audience in forms that reward mindful reading. Care is taken with every detail of the production, and we aim to meet or exceed professional best standards in every way. These are the core scriptures underlying the entire Buddhist tradition, and we believe that they deserve to be preserved and made available in highest quality without compromise.

SuttaCentral is a charitable organization. Our work is accomplished by volunteers and through the generosity of our donors. Everything we create is offered to all of humanity free of any copyright or licensing restrictions. 

%
\chapter*{Preface}
\addcontentsline{toc}{chapter}{Preface}
\markboth{Preface}{Preface}

The topic of the future weighs heavily on the mind of the \textsanskrit{Aṅguttara}. We live in that future or beyond it. The “future perils” of the \textsanskrit{Aṅguttara} (AN 5.77–80) are upon us, and have been for quite some time. King Ashoka referred to these suttas, which include the peril of corruption in the \textsanskrit{Saṅgha}; perhaps he worried that his own generosity, though intended to support Buddhism, would ultimately lead to its decay. 

Today the future perils are greater than ever. It is as if the seven suns are appearing in the sky (AN 7.66). We want to face the future with hope for a better life, but the horizon dims and draws closer, while we clutch our loved ones and prepare for the worst.

It was the novelist William Gibson who said, “the future is already here—it’s just not very evenly distributed”. His saying conceals layers of meaning. The future is here because we see scientific, technological, and human marvels every day that are far beyond our comprehension. Yet such advances are for the few, and only trickle gradually, if at all, to people in need. The future is also here in a negative sense, in that the catastrophic effects of climate collapse are readily apparent; and yet there too, their effects are unevenly distributed, massively impacting those who have least capacity to deal with them. 

The way of the Dhamma is to neither deny such things nor to be paralyzed by them, but to live, urgently and vitally, in the present. We live as if the future was uncertain, because it is. Our only certainty is that all this will disappear. 

The Buddha stayed calm in the face of terrors. He acted as if the present were all that mattered, because it is. Live well now, and let the future take care of itself. The Buddha did not pretend to be able to control the future, so why should we? 

As a Buddhist, I might rephrase Gibson’s saying: “the present is already here—it’s just not very evenly distributed”. The art of meditation is the art of presence, and in that presence we can be grateful for all that we have, mindful of all our blessings, and fearless when facing the future. It is when we lose our presence that our fears can overcome us. So let us not lose our presence. 

%
\chapter*{The Numbered Discourses: things that are useful every day}
\addcontentsline{toc}{chapter}{The Numbered Discourses: things that are useful every day}
\markboth{The Numbered Discourses: things that are useful every day}{The Numbered Discourses: things that are useful every day}

\scbyline{Bhikkhu Sujato, 2019}

The \textsanskrit{Aṅguttara} \textsanskrit{Nikāya} is the last and longest of the four primary divisions of the Sutta \textsanskrit{Piṭaka}. The word \textit{\textsanskrit{aṅguttara}} literally means “up by one factor”, i.e. “incremental”. It refers to the fact that the discourses are arranged by numbered sets, with the numbers increasing by one. I have translated it as \textit{Numbered Discourses}, while previously it has been translated as the \textit{Numerical Discourses} or the \textit{Gradual Sayings}.

SuttaCentral follows Bhikkhu Bodhi’s translation in counting 8122 discourses in total. The summary verse at the end of the collection, however, says there are 9,557 suttas. This scribal remark does not say how this count was arrived at; it must have been quite a process to count so many discourses when dealing only with palm-leaf manuscripts. In any case, as with the \textsanskrit{Saṁyutta} \textsanskrit{Nikāya}, this count is largely a product of discourses repeated according to templates. Many of these consist only of a single word; indeed, the process of abbreviation is carried to such extremes that hundreds of suttas do not, in fact, exist at all in the text; they are merely numbers to be filled out. Also, in the case of the Ones and Twos, most of the suttas are longer texts that have been divided to make the numbers. On SuttaCentral, these are treated as if one \textit{vagga} was a sutta, and the abbreviated texts likewise are combined. If we count the files of the texts combined in this way, we arrive at more reasonable, but still very large, 1407 texts of substance.

The focus of the \textit{Numbered Discourses} is on practical matters of everyday relevance. Guidelines of ethics and character predominate. If the \textsanskrit{Saṁyutta} \textsanskrit{Nikāya} gathers the chief teachings on \emph{doctrines}, the \textsanskrit{Aṅguttara} gathers the teachings on \emph{persons}. The concerns of the lay community are a major focus, and many teachings deal with how to teach.

The use of numbered sets is found throughout the Buddhist texts, but here it becomes the main organizing principle. The typical \textsanskrit{Aṅguttara} discourse consists of a statement that there is certain number of something; then an explanation of each item; then a conclusion that echoes the introduction. Sometimes a verse is added that summarizes the content. This formal pattern is highly optimized to reinforce learning and memorization. It is, in essence, exactly the same format that is used in the nightly news: begin by listing the news items for today; give the stories of each of the items; and then summarize the highlights once more. The use of numbered sets remains popular today, with the “listicle” being a favorite format for internet articles.

Unlike these modern examples, however, the sets of teachings in the \textsanskrit{Aṅguttara} are strongly structured. They are not merely collections of items on a theme, but make up an integrated sequence. The first item is the most fundamental; the subsequent items evolve from or build upon that; and the final item caps off the sequence.

For this reason the \textsanskrit{Aṅguttara} provides an excellent entry point to the canon, especially for those with a limited amount of time. It only takes a few minutes to read a sutta, and it will contain within itself a complete and useful teaching.

The \textsanskrit{Aṅguttara} \textsanskrit{Nikāya} has a counterpart in the \textsanskrit{Ekottarikāgama} preserved in the Chinese canon (EA). The \textsanskrit{Ekottarikāgama} is a peculiar text of uncertain (possibly \textsanskrit{Mahāsaṅghika}) affiliation, and it differs from the Pali text to a much greater extent than the parallels for DN, MN, and SN. In addition, there are two partial Ekottarikas in Chinese, as well as a number of independent Ekottarika-style suttas. Moreover, a substantial portion of a Sanskrit \textsanskrit{Ekottarāgama} was discovered at Gilgit and has been edited and partially reconstructed by \textsanskrit{Tripāṭhi}. While it is difficult to generalize, it seems as if most of these materials lie closer to the Pali text than does the main EA in Chinese.

\section*{How the \textsanskrit{Aṅguttara} is Organized}

The \textsanskrit{Aṅguttara} \textsanskrit{Nikāya} consists of major “books” (\textit{\textsanskrit{nipāta}}) numbered one through eleven. Each of these contains discourses consisting of the corresponding number of items. As usual, the discourses are gathered into \textit{vaggas}, which sometimes have a loose theme. Each \textit{\textsanskrit{nipāta}}, except the first, organizes its \textit{vaggas} into \textit{\textsanskrit{paṇṇāsas}}.

I don’t know why the \textsanskrit{Aṅguttara} counts to eleven; I would expect a round number. Eleven is shared in common with the Chinese \textsanskrit{Ekottarikāgama}, which suggests it was an early feature, yet it does not appear to be driven by the texts themselves, as most of the items in the Book of the Elevens consist of teachings familiar elsewhere, with the addition of an item or two.

It sometimes feels as if the \textsanskrit{Aṅguttara} was assembled from leftovers. After the long suttas were gathered in the Majjhima and \textsanskrit{Dīgha}, and the shorter suttas on central themes into the \textsanskrit{Saṁyutta}, a large mass of texts remained that resisted easy categorization. This included many fascinating and profound teachings, as well as a large mass of stock repetitions, and it trailed off into odds and ends of increasingly obscure value. It’s as if the redactors, faced with a warehouse of leftovers and bric-à-brac, tried their best to shelve and stack the items in a logical way, but were often left with just plonking things on shelves as best they could. Since the texts usually had a distinct number in the teaching, this was taken as the organizational principle, in lieu of anything more meaningful. Even texts that don’t explicitly mention a number can often be analyzed into a set of items, so they could be included too. (See for example AN 3.31 and AN 3.32.)

To be clear, it should not be thought that the \textsanskrit{Aṅguttara} lacks the standard teachings familiar from the rest of the \textit{\textsanskrit{nikāyas}}. On the contrary, we find the four absorptions (AN 3.58), the four noble truths and dependent origination (AN 3.61), the faculties and powers (AN 4.163; the latter in some detail at AN 5.12–16), the threefold training (AN 3.81), the divine meditations (AN 3.63), and many more. But such teachings are scattered throughout a large mass of suttas on a diverse range of topics.

In the introduction to his translation, Bhikkhu Bodhi practically abandons any attempt to make sense of the structure. He gives an example of a chapter with several seemingly unrelated discourses, remarking: “With such apparently arbitrary organization, one cannot but wonder what the compilers had in mind.” (\textit{The Numerical Discourses of the Buddha}, introduction, p. 22). As a result, rather than analyze the content as it occurs in the \textsanskrit{Aṅguttara}, he developed an extensive and carefully-considered thematic analysis. This essay is available on SuttaCentral, and I encourage anyone interested in a serious study of the \textsanskrit{Aṅguttara} to read it.

I would like to approach the material from a different perspective, however, one that lies closer to the experience of reading the text. I find Bhikkhu Bodhi’s question an intriguing one: what were the redactors thinking?

While there is no doubt that the sequence of suttas and ideas in the \textsanskrit{Aṅguttara} is to some extent chaotic, is it really plausible that the same body of people who displayed such rigorous dedication to classification in the \textsanskrit{Saṁyutta} should simply abandon their efforts in the \textsanskrit{Aṅguttara}? Perhaps to understand the redactors better, and through them the teachings that they worked with, we must approach the problem in a new way.

Here are three organizational principles that I have noticed while reading the \textsanskrit{Aṅguttara}:

\begin{enumerate}%
\item Numerological meaning.%
\item Thematic clusters, segues, and arcs.%
\item Spaced repetition.%
\end{enumerate}

Below I will show how these things work out over the first three \textit{\textsanskrit{nipātas}}. In this way I hope to guide a reader through the wilds until they feel comfortable proceeding on their own.

It is surely not the case that these are the only organizational principles at work. Nor is it the case that they fully explain all, or even most, of the randomness. But they do, I believe, hint at a guiding understanding that shaped the collection in the form we have it today.

One general thought first. Much of how we organize and relate to the world is not through reason, but through association. If we think of it in terms of the five aggregates, a collection such as the \textsanskrit{Saṁyutta} \textsanskrit{Nikāya} has an overall structure that is deliberately thought through and constructed, i.e. it is based on rational choices or \textit{\textsanskrit{saṅkhārās}}. Perhaps what we need to look for in the \textsanskrit{Aṅguttara} is a different way of thinking, one based on perception, memory, and association (\textit{\textsanskrit{saññā}}).

But why would such a large mass of texts be organized in such a way? The answer is not hard to find. Like so many of the principles that organize the texts, it is for memorization. For a reciter who has to keep hundreds of texts in order, \emph{any} kind of connection works. It doesn’t really matter if it’s the topic, a shared word, a syntax, a rhyme, or anything else. Perception recognizes patterns. It associates one thing with the next, regardless of how significant the connecting feature is.

Imagine, if you will, that you’re organizing your personal library. You could use a rational system: alphabetical order, subject matter, or size to fit your shelves. But it’s your library, you can do what you want. Maybe on one shelf you put books with blue covers; on another, books you haven’t read; and on another, books whose smell reminds you of old friends. To anyone else it seems chaotic, but to you it makes perfect sense. You can find the book you need when you want to. Perception does the heavy lifting for you, without the cognitive strain of having to work through the rational system every time.

\subsection*{The meaning of numbers}

For the most part the use of numbers in Buddhist texts is entirely pragmatic. Once you know that a set has a certain number of items, you can tell if you’ve forgotten something.

Yet numbers have always been imbued with a significance and meaning that transcends mere accounting. They allows us to make sense of a complex cosmos through a simple set of conventions. Numbers are used in Buddhism to provoke awe, even fear, at the “astronomical” scope of transmigration. Is it possible that the symbolic meaning of numbers lends a sense of unity to the various \textit{\textsanskrit{nipātas}}?

Symbolic meaning is, by its very nature, impossible to pin down with precision. Unlike rational definition, it does not serve to limit the scope of meaning, but to amplify it through suggestion, hints, and connotations. The symbolic meaning of numbers has, so far as I know, mostly been ignored in Buddhist studies. A number of numerological observations were made in the number entries in the Rhys Davids’ and Stede’s \textit{Pali-English Dictionary}, but I am aware of little since then. However, we can make a few general observations.

\begin{description}%
\item[One] The number of harmony, simplicity, and supremacy. It is specially emphasized in the context of deep meditation (\textit{\textsanskrit{jhāna}} or \textit{\textsanskrit{samādhi}}). However, unlike many spiritual contexts, in Buddhism it never has a metaphysical sense: \textsanskrit{Nibbāna} is zero, not one; it is emptiness, not unity.%
\item[Two] Used for pairs, which may be partners—hands, eyes, man and wife—opponents—good vs. evil, light vs. dark, pain vs. pleasure—or successors—skill in entering meditation and skill in leaving it. It represents the dualities of the world.%
\item[Three] Made up of 2 + 1. It adds an extra, often spiritual, dimension to the worldly dualism of two. This is quite explicit in such sets as “gratification, drawback, and escape”, and more subtly in, say, pleasant, painful, and neutral feeling. Three represents the \emph{other}; and it is the other which contains the seed of transcendence.%
\item[Four] The most characteristic number of Buddhism, and the Book of the Fours is the largest of the \textit{\textsanskrit{nipātas}}. Its primary metaphor is the four quarters, and thus connotes totality and balance, most obviously in the four noble truths. Multiples of four have the same meaning at a higher order. Ten is similar, in that it is derived from the four quarters, the four intermediate quarters, and above and below.%
\item[Five] Stems from the hand, which is what we use to count; hence it divides into 4 + 1 (fingers and thumb) rather than 2 + 3. The most obvious example of this is the five “grasping” aggregates, where consciousness stands against the other four. Likewise in, say, the five faculties and powers, wisdom is the “thumb”.%
\item[Six] Takes as its root metaphor the body: four limbs, torso, and head. The general sense is a “large whole”, and the most prominent set is the six sense fields.%
\item[Seven] An astronomical number, derived from the lunar cycles and the heavenly bodies (sun + moon + five visible planets). It is used especially commonly in myth, and has the general sense of “the entire cycle of life and death”. It appears in this sense in the story of the Bodhisatta’s birth.%
\end{description}

I believe that we can indeed discern traces of these meanings in the \textsanskrit{Aṅguttara}, and offer some examples below. In some cases the texts of a certain number would have been simply imported into the collection, while in other cases the text would have been edited specifically to create the necessary numbered set. Either way, having some sense of these meanings gives us a perspective through which to see the \textit{\textsanskrit{nipātas}} as meaningful wholes.

With such general meanings, and doubtless many exceptions and contradictions, it is not really possible to establish beyond doubt that the numbers of the \textsanskrit{Aṅguttara} have a symbolic meaning. If you dislike any attempt at reading symbolic meaning, I cannot prove you wrong. But it does, I believe, give us an approach through which to appreciate the efforts of the redactors and the manner in which they dealt with their diverse and complex material.

\subsection*{Thematic clusters, segues, and arcs}

Despite its chaotic impression, suttas in the \textsanskrit{Aṅguttara} are rarely isolated. Most of the time they appear in \emph{thematic clusters} that deal with the same topic. This might be just a pair of suttas, though it’s not uncommon to find an entire \textit{vagga} on a specific theme. These are often closely related suttas, simply varying a few details from one to the next. Or they may have a loose thematic thread, featuring, for example, the same person, or group of persons. In several cases, \textit{vaggas} of the same name and theme appear multiple times in the collection.

Such thematic clusters are easy to recognize; but still, it often seems as if there is nothing that connects one cluster with the next. However, this is not always the case. Often the shift from one cluster to another happens by means of what might be called a \emph{thematic segue}. When moving from one thematic cluster to the next, some common element is maintained. This might be a topic, or simply some formal feature—a question format, a word, a syntax, etc. Such hooks help smooth the transition from cluster to cluster.

In such cases we find that there is some element in the first cluster [A], which is combined with a second element [AB] to form a new cluster or extend the old one. Then the second element is combined with something else to make yet another cluster [BC]. And perhaps later the second element is dropped altogether leaving just the third [C], or it is recombined with something new. If you compare the first element [A] with the third [C] there’s nothing in common. Yet the progress from one to the other is clearly gradual. And the frequency with which this occurs shows that it is by no means accidental.

Similar techniques are a stock in trade of musical composition. After introducing a motif, the composer gradually transforms and develops it. Eventually they might arrive at a new motif, which shares nothing in common with the original, but from which it clearly evolved.

A thematic segue is often a purely formal technique that says little about the subject matter. However, with careful attention we can see that thematic clusters, chained together with segues, sometimes evolve over larger spans of text to create a loosely organized \emph{thematic arc}. Such arcs echo teaching frameworks that are familiar from elsewhere, such as the Gradual Training. This is used, for example, to inform the shape of the first 75 suttas of the Book of the Ones. Such arcs are by no means as clear and formally structured as the teachings on which they are based. Yet the progress from one topic to the next is undeniable.

Indeed, each \textit{\textsanskrit{nipāta}} can be seen as forming its own arc, as they typically begin with basic practices, and end with realizing \textsanskrit{Nibbāna}. The repetition series that round out each \textit{\textsanskrit{nipāta}} also have their own internal arc, leading towards the highest qualities.

\subsection*{Spaced repetition}

In the \textsanskrit{Dīgha} and Majjhima \textsanskrit{Nikāyas}, a student would spend a fair period of time memorizing one specific text, rehearsing it, and—if they are a good student—inquiring and questioning about the meaning. Only when it was mastered would they move on to the next. In the \textsanskrit{Saṁyutta}, a student would learn dozens, even hundreds of suttas on the same topic, sharing similar passages and ideas, and often varying little one to the other. Such suttas may be memorized quickly, and interpretive problems often arise at the level of the topic rather than the individual text.

But memorizing long texts, or many texts on the same subject, can get boring, for the mind is stimulated by variety and surprise. In the \textsanskrit{Aṅguttara}, a student would learn one or two suttas on a topic, or maybe a few more, then something else, then back to the original topic, then a third. Now, as we have seen, there are various features that help them keep the sequence of texts straight. But perhaps there is something more to it: perhaps the very randomness and repetition helps them to learn.

This is a lot like the modern technique known as “spaced repetition”, commonly used for language learning. A vocabulary of words is introduced one at a time in a random sequence. After learning one word, one moves on to another. But the first word is then re-introduced a little later to reinforce learning. And so it goes, with the same words coming back again and again. In terms of the sequence from one word to the next, everything is random. But the overall pattern is carefully optimized to reinforce and speed up memorization.

Perhaps we could think of it like a school. The \textsanskrit{Saṁyutta} is like a school \emph{curriculum}: everything you need to know on a topic, all in one place. But the \textsanskrit{Aṅguttara} is like a school \emph{day}. One class follows the other, and there is no real rhyme and reason to it. Some things happen fairly regularly and predictably, while others seem to just pop up at random. Despite its more chaotic nature, it works: that’s how we learn. No-one would suggest that school subjects are best mastered by first learning the science curriculum, then the maths, then the history. Not only does the spaced repetition reinforce learning, but it provokes us to see new and unexpected connections between things.

\section*{The Book of the Ones}

I have suggested that the number one carries with it a specific set of connotations, notably harmony, simplicity, and supremacy. If this is framed as an overall theme, it might be something like this: keep your spiritual practice simple and focused to help your mind attain deep immersion, and in that way you can realize the supreme Dhamma. Let us see how the Book of the Ones exemplifies these attributes.

The Book of the Ones is a rather special case in that virtually the entire \textit{\textsanskrit{nipāta}} is constructed from fragments and templates. The collection begins with the striking assertion that no sight occupies a man’s mind like that of a woman, or a woman’s mind like that of a man. The remainder of the exterior senses are listed for each gender binary, making ten suttas in all for the first \textit{vagga}. This very much has the appearance of a single sutta divided into ten. In the Fives (AN 5.55) we find a similar set of statements given in a particular context, dealing with the masculine perspective only. And in the Chinese we find that the parallels at EA 9.7 and EA 9.8 fit the two halves of this \textit{vagga}. This supports the idea that these texts were originally combined to form a single sutta, or perhaps a pair of suttas.

An even clearer example of this is provided by the three pairs of suttas at AN 1.76 to AN 1.81. Each of the pairs follows the same pattern, exemplified by the first pair. AN 1.76 says that loss of relatives is a small thing, while wisdom is the worst thing to lose. AN 1.77 presents the inverse: growth of relatives is a small thing, for wisdom is the best thing to grow. But it continues to round off the sutta by urging the mendicants to train to grow in wisdom. This conclusion is lacking in the first of the pair, and is a clear sign that the text has been divided.

One should not conclude from this that the text has been assembled haphazardly. On the contrary, we can identify a series of arcs that bind long series of suttas together. The opening chapters are designed to show the development of meditation, echoing the meditator’s progress in the Gradual Training.

\begin{itemize}%
\item The first chapter, as we have seen already, deals with the restraint of sexuality, one of the foundations of meditation.%
\item The second chapter deals with the hindrances which must be abandoned before entering deep meditation. This is linked via thematic segue from the previous chapter, the link being the phrase “I do not see a single thing”.%
\item The third and fourth chapters deal with the advantages of the developed mind, which has been purified through the process of meditation: nothing brings greater happiness and benefit. They continue using the phrase “I do not see a single thing”.%
\item The fifth chapter abandons the phrase “I do not see a single thing”. Here the thematic segue is the topic of “mind” (\textit{citta}) and its development.%
\item The fifth chapter ends with two discourses that mention the famous “radiant mind”. These are fragments, and a more complete statement is found in the following suttas that start the next chapter. It is somewhat unusual to find such closely connected suttas broken over a \textit{vagga} boundary like this. Note that the “radiant mind” is not a metaphysical term, and neither here nor anywhere else in the early Buddhist texts is the mind said to be “intrinsically” or “naturally” or “originally” radiant or luminous. On the contrary, the mind is conditioned and hence is not “intrinsically” anything at all. The radiant mind is simply a way of talking about meditative absorption or \textit{\textsanskrit{jhāna}}.%
\item The sixth chapter continues on the theme of absorption. However, it changes theme at AN 1.56—though maintaining focus on “mind”—and continues by addressing the causality of good and bad qualities. In context, these can be understood as pertaining to the wisdom portion of meditation, as treated in the fourth of the \textit{\textsanskrit{satipaṭṭhānas}}. This series culminates at AN 1.75 with the perfection of the awakening factors, thus signifying the completion of the path.%
\end{itemize}

A structure such as this is particularly telling as it reveals the intent of the redactors. This thematic arc spanning 75 fragmentary suttas does not exist in the sources at all: it is purely implied by the choices of the redactors. Their method was to reduce the statements of the Dhamma to their simplest meaningful elements, then reassemble them according to the principles of the Dhamma as they understood them.

And the redactors were even more subtle than that. For not only are these fragments assembled to form a coherent whole, but the choice of theme was quite deliberate. Of all the doctrinal contexts in Buddhism, it is \textit{\textsanskrit{samādhi}} or “unification of mind” where the number one is most prominent. In starting with the Ones, the redactors were sensitive to the use of numbers in the canon, and arranged their texts to bring the most important “one” to the fore.

From here, the text shifts focus. As noted above, AN 1.76–81 deals with pairs of gain and loss. Then from AN 1.82 we have a series of texts on those things that are harmful and beneficial, starting with the pair of negligence and diligence. While these teachings are of course common throughout the canon, it is fitting that they appear here to represent the \textsanskrit{Aṅguttara}’s special focus on the fundamentals of a good life. Here they exemplify the aspect of \emph{simplicity}, helping a student to focus on just one aspect of Dhamma at a time.

The same set of factors is treated a few times with slightly varying templates, the final of which says that each of these harmful things leads to the disappearance of Buddhism, while the good things lead to its continuation. This leads us up to AN 1.129.

From AN 1.130–169, the topic of the preservation of the Dhamma is continued, but applied to a new theme, one that is quite distinctive of the \textsanskrit{Aṅguttara}: teaching the Dhamma. Specifically, that those who present the Dhamma accurately make much merit and preserve Buddhism, while those who distort or misrepresent the Dhamma make bad karma and destroy Buddhism. This series of texts displays its own inner structure, as it begins with simply the “teaching” and then continues to differentiate the Dhamma more and more finely, especially with the introduction of the Vinaya and a rather extensive list of technical terms for monastic discipline. It does not take much to see that an originally simple statement could have been drawn out by adding multiple aspects of the teaching, conveniently giving the students of the \textsanskrit{Aṅguttara} some Vinaya material to learn.

This series of suttas clearly grows out of the former, with the theme of preserving Buddhism as the thematic segue. Thus we have, from AN 1.82 through AN 1.169, a second thematic arc consisting of 88 suttas.

From AN 1.170 a new theme is introduced, one that also represents a key aspect of the \textsanskrit{Aṅguttara}: persons. Buddhism is, of course, most famous for its teachings on not-self, and its impersonal analysis of psychological processes. But there is plenty of material throughout the suttas that deals with persons, or character types, and much of that is in the \textsanskrit{Aṅguttara}. These texts were later assembled to form the Abhidhamma text the \textsanskrit{Puggalapaññatti}, the “Description of Persons”.

Of all the persons in Buddhism, the incomparable one is the Buddha himself. While there is a series of Buddhas over the ages, in our age he is unique. Hence these suttas speak of the “one person” who arises in the world who is uniquely beneficial and transcendent. At AN 1.187 the Buddha’s chief disciple, \textsanskrit{Sāriputta}, is praised as the one who continues to roll the wheel of Dhamma after the Buddha. This segues into the next series of suttas, which single out individual followers of the Buddha for particular praise. This is a rather fascinating list, in which appear many characters from all over the canon; not only the four \textit{\textsanskrit{nikāyas}}, but the Vinaya and the Khuddaka as well. Prominent monks appear in AN 1.188–234; nuns from AN 1.235–247; laymen from AN 1.248–257; and laywomen, from AN 1.258–267. All of these people are “number one” in their field, exemplifying the sense of “one” as \emph{supremacy}.

The next chapter continues the theme of “persons”, enumerating various things that are possible or impossible for various people. For example, it is impossible for one “attained to view”—that is, a stream-enterer—to take any condition as permanent. But from AN 1.284–295, once again we find a thematic segue; the “person” vanishes and the theme of possible and impossible is applied rather on an impersonal level: good things cannot come from bad deeds.

This makes up the third great thematic arc in the Ones, 126 suttas from AN 1.170 to AN 1.295. The remainder of the Ones continues in a similar way, with fragmented suttas assembled along loose thematic lines. The themes remain similar, with one difference. As the Book draws closer to its end, the subject of \textsanskrit{Nibbāna}, the final goal of Buddhism, becomes ever more prominent. The final \textit{vagga} is called the “Chapter on the Deathless”, and it deals directly with the path to full awakening. Thus the sense of thematic unity that has been evident to multiple sections of the Book of the Ones is also evident in its overall structure, assembled by the redactors to culminate in awakening.

\section*{The Book of the Twos}

The second \textit{\textsanskrit{nipāta}} is a kind of bridge between the “arcs of fragments” that characterize the first \textit{\textsanskrit{nipāta}} and the more complete suttas that become prominent in the remainder of the \textsanskrit{Aṅguttara}. It echoes and amplifies the themes of the first \textit{\textsanskrit{nipāta}}, while also introducing new ideas.

It begins with a series of suttas that speak of the fundamental principles of the good life: doing good and avoiding bad (AN 2.3–4) and the results of deeds in this life and the next (AN 2.1). This, I think, announces what the redactors aimed to be the chief theme of this \textit{\textsanskrit{nipāta}}: the idea that there is a moral order in the world, there is good and evil, and if we comprehend this we can live our life well. The collection starts by emphasizing this fact, and the dire consequences of ignoring it.

The second chapter builds on this, speaking on the “power of reflection” to look back and understand this moral order, and the “power of development” to move on from the bad and develop the good (AN 2.11–13). A specific example of this is given in the case of a disciplinary measure within the \textsanskrit{Saṅgha} (AN 2.15; cp. AN 2.21). When one mendicant accuses another of wrongdoing, both should “reflect” on what really happened and their own role in the affair, and only then can the issue be properly healed and everyone move on. This chapter also details in various respects the way that good and bad deeds lead to various results (AN 2.16, AN 2.17), spelling out a series of results that pertain both to this life and the next (AN 2.18). The Buddha then introduces the idea of a deliberate practice: one should not only recognize these things and reflect on them within oneself, but develop the good and give up the bad, for it is possible to do so (AN 2.19).

These suttas (and others) build on the teachings found in the first \textit{\textsanskrit{nipāta}} that deal with basic principles. They conclude the opening arc that establishes the theme of this \textit{\textsanskrit{nipāta}}: the worldly duality of good and bad, which creates both a responsibility and an opportunity to respond.

But it should not be thought that these chapters are fully coherent and systematic. One finds the occasional sutta that appears quite random, for example AN 2.10 on entering the rainy season retreat; or AN 2.60 on why fauns (\textit{kimpurisa}) do not use human speech.

Meanwhile, distinct themes from the first \textit{\textsanskrit{nipāta}} are also introduced, mixed up without clear order. AN 2.14 mentions two ways of teaching Dhamma—in brief and in detail—while AN 2.20 says that the survival of Buddhism depends on getting both the meaning and the phrasing of the texts correct (also see AN 2.41). AN 2.23 reprises the theme that one who distorts the teaching misrepresents the Buddha and contributes to the ending of Buddhism. This applies to those who claim that things were spoken by the Buddha when they were not.

AN 2.24 introduces the contrast between the discourses that require interpretation (\textit{neyyattha}) and those whose meaning is explicit (\textit{\textsanskrit{nītattha}}). In some suttas (eg. MN 133), we find that the Buddha gives a brief statement which the mendicants do not understand, so they seek advice on how to interpret it. In other cases a verse or doctrinal statement is unclear and the mendicants discuss it. These examples show that the process of discussion and analysis of the Buddha’s teaching was underway from the very beginning. This process was eventually to be formalized as the various sets of Abhidhamma texts, and spelled out in the commentaries. But these later texts did not yet exist, and should not be read back into the suttas.

AN 2.31 reintroduces another of the themes of the first \textit{\textsanskrit{nipāta}}: meditation. The pair of serenity (\textit{samatha}) and discernment (or “insight”, \textit{\textsanskrit{vipassanā}}) are said to play a part in realization: serenity develops the mind, while \textit{\textsanskrit{vipassanā}} develops wisdom. Together they lead to the two aspects of awakening: the freedom of heart and the freedom by wisdom.

But this is, for the moment, an isolated text, for the next series of texts returns to the theme of persons. In fact this theme was briefly introduced earlier, when AN 2.2 contrasted the efforts of lay folk and renunciants. AN 2.32 says that a good person knows gratitude, while the bad one does not. AN 2.33 speaks of the strongest and most emotional ties of gratitude, those of a child to their parents. The Buddha says that even by carrying your parents around for the rest of their lives, feeding and cleaning them, you cannot repay them the gift of life. Only by establishing them in the principles of the Dhamma can you repay them. The theme of respect for parents is further developed in AN 3.31.

In AN 2.35 the Buddha tells a brahmin that the traditional religious donation (\textit{\textsanskrit{dakkhiṇa}}) is owed to those who are purified, that is, the trainee and master on the path. These replace the sacrifice of the brahmins. Next follow some teachings by disciples, in which both \textsanskrit{Sāriputta} (AN 2.36) and \textsanskrit{Mahākaccāna} (AN 2.38) make a distinction between inner and outer practice, while \textsanskrit{Mahākaccāna} makes a shrewd observation: householders argue about sensual pleasures, but renunciants argue over views (AN 2.37). In AN 2.39 the Buddha makes a rather biting comparison between a kingdom overrun with bandits and a \textsanskrit{Saṅgha} where the good mendicants are weak, cowed into silence. The following chapter (AN 2.42–51) expands on this by contrasting good and bad assemblies.

From assemblies as groups of people, the text revisits yet another theme of the first \textit{\textsanskrit{nipāta}}: the Buddha as the supreme person. Here he is paired with his worldly counterpart, the wheel-turning emperor (AN 2.52–55). Continuing with the theme of kinds of people, AN 2.62 and AN 2.63 describe procedures in the \textsanskrit{Saṅgha} for settling disputes and living harmoniously.

The next chapter is tightly constructed on the subject of happiness (\textit{sukha}). It introduces the topic by contrasting the happiness of lay people and that of renunciants, of which the latter is better (AN 2.64). This continues the theme of persons, specifically the contrast between lay folk and renunciants, which was already stated in the second text of the \textit{\textsanskrit{nipāta}}. Here it is combined with the topic of happiness, which is new. A series of suttas expands on this theme; but it uses a thematic segue to move away from the focus on “persons” and speak of happiness in purely psychological terms more reminiscent of the \textsanskrit{Saṁyutta}.

This chapter itself forms another segue—a nested segue if you will—to the next series of chapters, the unifying characteristic being the tight integration of short suttas on a single pattern in a \textit{vagga}, returning to the kind of “vagga as sutta” that we saw in the Book of the Ones. Chapter 8 deals with the causes for good and bad qualities; Chapter 9 deals with various miscellaneous pairs of “things” (\textit{\textsanskrit{dhammā}}); Chapter 10 deals with the contrast between the fool and the astute; and Chapter 11, while a little more diverse, caps off this series.

AN 2.130–AN 2.133 eulogize great disciples, reminding us of the lists of the foremost disciples in the Book of the Ones. A few discourses then revisit the theme of good and bad people inheriting the results of their deeds (AN 2.134–AN 2.137). The remainder of the \textit{\textsanskrit{nipāta}} lists a long series of pairs of contrasted qualities. Particularly interesting is the series at AN 2.280–309 where the Buddha gives the reasons for laying down rules for monastics. Normally this is a list of ten reasons, but here they are arranged as pairs. The final series speaks of developing deep understanding and letting go by means of the pair of serenity and discernment. Thus, despite its main focus on worldly ethics and results, the second \textit{\textsanskrit{nipāta}}, like the first, ends with the practices leading to awakening.

\section*{The Book of the Threes}

With this book we move on from the fragmentary assemblages of the Book of the Ones and partly the Book of the Twos, and find more conventionally unified suttas. Of course, this is never an absolute, as all the books of the \textsanskrit{Aṅguttara} retain extensive repetition series. Still, the focus from now is clearly on the whole sutta; and as a consequence, the hand of the redactors is harder to discern.

That does not mean, however, that there is a dramatic break from the first books. On the contrary, the Book of the Threes begins with a thematic \textit{vagga} that focuses on the familiar contrasting pair of the fool and the astute. The number three is represented in the qualities that are said to characterize them. The \textsanskrit{Bālavagga} (Chapter on the Fool) corresponds to the similarly-named third and tenth chapters of the Twos, as well as the second chapter of the Fives.

Like the Books of the Ones and Twos, this \textit{\textsanskrit{nipāta}} begins by emphasizing the problematic situation that we are in, the tensions and struggles of our worldly situation. The second \textit{vagga} shifts focus to what we can do about it.

Thus in AN 3.13 we first see a clear example of the number three as a worldly binary and a transcendent dimension that resolves the contradiction. This sutta speaks of one without hope—someone afflicted by poverty and misery of station, as well as illness of body—and a hopeful person, who looks to a bright future. But then there is the one who has done away with hope: since they have achieved their goal, there is nothing to look forward to.

AN 3.14 makes an important political point: even the greatest of rulers are subject to the rule of law (\textit{dhamma}). Here the relevant group of three is action by body, speech, and mind. This triad—which is pre-Buddhist—expresses one of the fundamental principles of the Dhamma, the focus on ethical choices, on doing good deeds. It is found constantly throughout the Threes. Whereas a worldly philosophy might take into consideration only a person’s external acts of body and speech, for Buddhism the mind is always the most important. It is the mind that is ultimately responsible for what we do and say, and it is through the mind that freedom is found. Thus the mind here points to the transcendent dimension of escape.

The next discourse continues with the threefold division of body, speech, and mind, giving some practical advice as to how to work for their proper development. For the first time the \textsanskrit{Aṅguttara} ventures into narrative. It tells the story of a chariot-maker of the past, who was commissioned by the king to make a new war chariot for a battle in six months. This is a rather striking setup: the scale of society is so small that it is unremarkable for a king to personally speak to a chariot-maker. And apparently a single chariot is an adequate military build-up for war; a war that is, politely enough, scheduled precisely six months in advance. The small scale and low stakes of this charming story are a strong contrast with the elaborate and fanciful legends of the \textsanskrit{Dīgha}.

The chariot-maker completes the first wheel only six days before the battle, and is urged to rush the second one. But time matters: the first wheel is well-formed and stable, while the second wobbles and crashes. The Buddha then goes on to identify himself as the chariot-maker. This too is remarkable, as it is one of the very few \textsanskrit{Jātaka} stories in the four \textit{\textsanskrit{nikāyas}}. The Buddha-to-be’s humble station as a lowly chariot-maker is unusual, as usually he is said to be a great king of the past. It’s also noteworthy that his commission is morally dubious: he is an arms manufacturer. Later the \textsanskrit{Aṅguttara} will say that trade in arms is one of the forms of wrong livelihood (AN 5.177). One might argue that a chariot is not a weapon; but it is explicitly required as a war vehicle, and today we would not hesitate to count, say, tanks or a fighter jets as weapons platforms. In the later \textsanskrit{Jātaka} collection, it is not unusual for the Bodhisatta to break precepts or commit various acts of dubious morality; after all, the whole point is that he is not yet perfect. Still, this adds to the striking impression of this little tale, so much more down to earth and realistic than the other \textsanskrit{Jātakas} in the \textit{\textsanskrit{nikāyas}}.

AN 3.16 introduces the idea of the “guaranteed practice”, which consists of three of the elements of the Gradual Training: sense restraint, eating in moderation, and wakefulness. These implicitly call back to the very first chapter of the \textsanskrit{Aṅguttara}, here presented in a more standard form.

One of the \textsanskrit{Aṅguttara}’s characteristic rhetorical devices is to contrast the worldly with the sacred; remember how AN 2.2 spoke of the efforts of the lay and the renunciant. AN 3.19 expands this theme, comparing a shopkeeper who must work morning, noon, and night with a mendicant who applies themselves to their meditation morning, noon, and night. AN 3.20 applies the same metaphor in a different way. These suttas cap off the first arc, which deals with understanding the dangers of the world, and working to escape from it.

The third chapter revisits the theme of “persons”. It begins with a discussion among some senior mendicants regarding who is best out of three kinds of spiritually attained person; or in other words, who has best implemented the practice encouraged in the first two chapters (AN 3.21). An interesting comparison is made between treatment of illness and providing spiritual assistance: you can’t always help, but you should at least try (AN 3.22).

A number of suttas in the Threes share a pattern where the first item is a foundation, the second is the realization of the four noble truths, and the third is full awakening. This first appears at AN 3.12, and is applied in different ways in AN 3.24 and AN 3.25.

Returning to the theme of meditative immersion with which the entire collection started, we are introduced to an intriguing teaching that reappears multiple times in the \textsanskrit{Aṅguttara}, but nowhere else in the canon (AN 3.32, AN 10.6, AN 10.7, AN 11.7, AN 11.18–AN 11.21). It begins with a question: could it be that a mendicant might attain a state of immersion that is free of all ego and conceit? Normally it is understood that the meditative absorptions are shared between ordinary people and enlightened beings on the path. The perfected ones are distinguished by having let go the cause of suffering, not because they have attained some special state of meditation. But these suttas, with their striking note of wonder, imply that there is a special meditative state attained only by the perfected ones.

This discourse is also distinguished by the fact that it finishes by quoting a verse currently included in the Sutta \textsanskrit{Nipāta}, and even correctly names the chapter, the “The Way to the Beyond”. It is not at all obvious that the verse was originally intended to refer to a state of meditation. This shows that free and imaginative readings of suttas were found even in the earliest times.

The next discourse (AN 3.33) continues with the theme of going beyond ego and conceit, and it too quotes from “The Way to the Beyond”. But it starts with the Buddha in what appears to be an uncharacteristically despondent mood, saying that whether he teaches in brief or in detail—harking back to AN 2.14—it’s hard to find anyone who understands.

AN 3.35 narrates a personal encounter with a fellow by the name of Hatthaka, who came across the Buddha meditating near his home town of \textsanskrit{Āḷavī}. He asks if the Buddha slept well, considering the harshness of his outdoors living. The Buddha replies that he is one of those who sleep well in the world, as he is rid of the greed, hate, and delusion that disturb people in their sleep.

This is the second time this classic triad appears in the Threes. They first appeared in the previous sutta, AN 3.34, as the source of deeds, and will recur in this sense in multiple suttas in this \textit{\textsanskrit{nipāta}}. Like the triad of body, speech, and mind, they can be seen to exemplify the 2 + 1 pattern. Greed and hate are a codependent pair, the ugly opposites. Delusion underlies them both; but at the same time, the counterpart of delusion is wisdom, and it is through wisdom that transcendence is possible.

This narrative mood ventures into mythology in the next discourse, which gives the Buddhist account of the god Yama, lord of the dead. While one might expect a death god to be fearsome, here he takes a decidedly Buddhist approach to the afterlife. When the departed are brought to Yama, he neither judges nor punishes. Rather, he asks the departed whether he took heed of the messengers sent by the gods: an old person, a sick person, and a corpse. These, of course, are three of the four divine messengers seen by the Bodhisattva before he went forth (canonically found in the life of \textsanskrit{Vipassī}; see DN 14). When the departed one replies that he took no heed, Yama castigates him for his negligence and then falls silent. The departed is dragged off to endure the sufferings of hell, here recounted in a briefer form than MN 129 and MN 130. Yama goes on to lament the pitiful state of mortals, including himself, and wishes he could be reborn as a human and practice Buddhism.

The mythological mood continues in the next couple of suttas (AN 3.37, AN 3.38), which introduce a new topic that will be very important for the \textsanskrit{Aṅguttara}; namely, the \textit{uposatha} or “sabbath”. This was a special “holy day” for religious observance observed weekly or on certain special days. Apparently the ministers of the Four Great Kings survey the earth on such days to see if people are honoring their betters and doing good. If they are, they rejoice, for they know that such people will be reborn in heaven to swell the hosts of the gods, whereas if they are not they fear the hosts of the demons will increase. This will of course have serious military implications in the ongoing war between the two.

The next couple of suttas deal with renunciation, first as the Buddha’s recollection of his own delicate upbringing (AN 3.39), then as a reflection for how a renunciate is to reflect with integrity on their choices (AN 3.40). AN 3.41 and AN 3.42 look at the other side of the coin, the qualities that make merit for a lay donor, especially faith and generosity as well as desire to learn. Learning is taken up as the theme of the next two suttas (AN 3.43 and AN 3.44). Then the themes of generosity, faith, and the worthy spiritual life that make merit fruitful are revisited (AN 3.45, AN 3.46, AN 3.48). Mendicants are then urged to be diligent (AN 3.49) and the nature of bad mendicants is disclosed (AN 3.50). Taken together, this series of 12 suttas can be read as a small thematic arc on the relation between lay folk and renunciants, the need for both to have integrity and the proper sense of values in their own sphere, and the mutual support of each other through generosity of material things and of teachings.

Next begins a new \textit{vagga}, “On Brahmins”, which as one might expect, depicts the Buddha in conversation with brahmins. In AN 3.51 and AN 3.52, the Buddha is approached by two brahmins, who confess that they have not lived a good life, and now, in their dotage, seek for help. The Buddha acknowledges the brevity of life and urges restraint of body, speech, and mind. AN 3.53 has the Buddha speaking to another brahmin on how the Dhamma is to be realized in this very life. He gives a similar teaching to a wanderer (AN 3.54) and to the brahmin \textsanskrit{Jāṇussoṇi} (AN 3.55). That the Buddha’s teaching may be realized in this life is a stock characteristic of the Dhamma (\textit{\textsanskrit{sandiṭṭhiko} \textsanskrit{akāliko}}), but it is easy to overlook how directly this was a rebuke of pre-existing religious traditions. They looked forward to rewards in the future—whether a heavenly rebirth or the eventual annihilation of suffering—but the Buddha, while not denying the reality and importance of future fruits, refocused spiritual life on the present.

AN 3.56 gives a different kind of teaching to a brahmin. In a message with a special poignancy in our troubled times, the Buddha explains why civilizations collapse, namely, unbridled greed. In AN 3.57 the Buddha refutes the wanderer Vacchagotta’s accusation that he only encourages giving to his own followers. When the brahmin \textsanskrit{Tikaṇṇa} (“Three-ear”) praises true brahmins, the Buddha responds with his own redefinition of a brahmin, rejecting the value of birth and Vedic learning, and giving the second part of the Gradual Training, starting with the absorptions (AN 3.58; AN 3.59 is similar). In AN 3.60, the Buddha not only rejects the value of the Vedic sacrifice, he shows that by teaching Dhamma one can benefit many more people.

The seventh \textit{vagga} is titled the “Great Chapter”, and it introduces a series of discourses on a larger scale. It begins with a thematic segue; AN 3.61 continues the theme of the relation between Buddhist and non-Buddhist theories, but it does so as a straight doctrinal discourse to the mendicants, rather than as an interfaith dialogue. This magnificent discourse offers an important framing of dependent origination and it deserves detailed study. This Great Chapter is unified by length of sutta rather than by subject; however a number of other suttas deal with non-Buddhist philosophy and relations (AN 3.64, AN 3.68, AN 3.70), including the famous \textsanskrit{Kālāma} Sutta (AN 3.65; also AN 3.66).

Here I will end my analysis, as I think enough examples have been given to illustrate both the connectedness and the chaos of this collection. Hopefully the reader can find their own way from here, and not feel so bewildered by the sudden shifts and changes they encounter.

\section*{A Brief Textual History}

The \textsanskrit{Aṅguttara} \textsanskrit{Nikāya} was edited by R. Morris (vols. 1 and 2) and E. Hardy (vols 3–5) on the basis of manuscripts in Sinhalese and Burmese scripts; Hardy also made use of the then recently-published royal Thai edition. It was published in Latin script by the Pali Text Society from 1885 to 1900. Indexes by M. Hunt and Mrs C.A.F. Rhys Davids were added in 1910. The first translation followed in 1932–36 by F.L. Woodward (vols. 1, 2, and 5) and E.M. Hare (vols. 3 and 4) under the title \textit{The Book of the Gradual Sayings}.

As was the case with the Majjhima and \textsanskrit{Saṁyutta}, a number of disparate individual suttas from the \textsanskrit{Aṅguttara} were published in book form or the web. However a complete new translation had to wait for Bhikkhu Bodhi to complete his work on the \textsanskrit{Saṁyutta} \textsanskrit{Nikāya}. As described in the introduction to his translation, in the late 1990s Bhikkhu Bodhi collected Nyanaponika Thera’s four-part series of Wheel booklets into a single volume for the International Sacred Literature Trust as \textit{An \textsanskrit{Aṅguttara} \textsanskrit{Nikāya} Anthology}. He then added sixty more suttas and published the total of 208 suttas as \textit{The Numerical Discourses of the Buddha: An Anthology of Suttas from the \textsanskrit{Aṅguttara} \textsanskrit{Nikāya}} with AltaMira Press in 1999. In 2012 he completed the full translation, which was published as \textit{The Numerical Discourses of the Buddha} through Wisdom Publications. His Introduction was even more extensive than his previous works; less technical than the \textsanskrit{Saṁyutta} introduction, the bulk of it focused on an overview of the teachings found in the \textsanskrit{Aṅguttara}. As with his previous translations, this work constituted a major leap forward in accuracy and readability and introduced the \textsanskrit{Aṅguttara} to a new generation.

Where the Pali was unclear I frequently referred to the earlier work of Bodhi, and rarely to Woodward/Hare and various translations of specific texts. An article by Tse-fu Kuan (關則富)—\textit{Some Reflections on Translating the Pali Texts: Literary Conventions, Buddhist Thought, Cultural Background and Textual History}, 2019, Acta Orientalia Academiae Scientiarum Hungaricae, vol. 72 (1), pp. 1–23—provided helpful corrections in several passages.

%
\chapter*{Acknowledgements}
\addcontentsline{toc}{chapter}{Acknowledgements}
\markboth{Acknowledgements}{Acknowledgements}

I remember with gratitude all those from whom I have learned the Dhamma, especially Ajahn Brahm and Bhikkhu Bodhi, the two monks who more than anyone else showed me the depth, meaning, and practical value of the Suttas.

Special thanks to Dustin and Keiko Cheah and family, who sponsored my stay in Qi Mei while I made this translation.

Thanks also for Blake Walshe, who provided essential software support for my translation work.

Throughout the process of translation, I have frequently sought feedback and suggestions from the community on the SuttaCentral community on our forum, “Discuss and Discover”. I want to thank all those who have made suggestions and contributed to my understanding, as well as to the moderators who have made the forum possible. A special thanks is due to \textsanskrit{Sabbamittā}, a true friend of all, who has tirelessly and precisely checked my work.

Finally my everlasting thanks to all those people, far too many to mention, who have supported SuttaCentral, and those who have supported my life as a monastic. None of this would be possible without you.

%
\mainmatter%
\pagestyle{fancy}%
\addtocontents{toc}{\let\protect\contentsline\protect\nopagecontentsline}
\part*{The Book of the Ones }
\addcontentsline{toc}{part}{The Book of the Ones }
\markboth{}{}
\addtocontents{toc}{\let\protect\contentsline\protect\oldcontentsline}

%
%
\addtocontents{toc}{\let\protect\contentsline\protect\nopagecontentsline}
\chapter*{The Chapter on What Occupies the Mind }
\addcontentsline{toc}{chapter}{\tocchapterline{The Chapter on What Occupies the Mind }}
\addtocontents{toc}{\let\protect\contentsline\protect\oldcontentsline}

%
\section*{{\suttatitleacronym AN 1.1–10}{\suttatitleroot Cittapariyādānavagga}}
\addcontentsline{toc}{section}{\tocacronym{AN 1.1–10} \tocroot{Cittapariyādānavagga}}
\markboth{1. Sights, Etc. }{Cittapariyādānavagga}
\extramarks{AN 1.1–10}{AN 1.1–10}

\subsection*{1 }

\scevam{So\marginnote{1.1} I have heard. }At one time the Buddha was staying near \textsanskrit{Sāvatthī} in Jeta’s Grove, \textsanskrit{Anāthapiṇḍika}’s monastery. There the Buddha addressed the mendicants, “Mendicants!” 

“Venerable\marginnote{1.5} sir,” they replied. The Buddha said this: 

“Mendicants,\marginnote{2.1} I do not see a single sight that occupies a man’s mind like the sight of a woman. The sight of a woman occupies a man’s mind.” 

\subsection*{2 }

“Mendicants,\marginnote{1.1} I do not see a single sound that occupies a man’s mind like the sound of a woman. The sound of a woman occupies a man’s mind.” 

\subsection*{3 }

“Mendicants,\marginnote{1.1} I do not see a single smell that occupies a man’s mind like the smell of a woman. The smell of a woman occupies a man’s mind.” 

\subsection*{4 }

“Mendicants,\marginnote{1.1} I do not see a single taste that occupies a man’s mind like the taste of a woman. The taste of a woman occupies a man’s mind.” 

\subsection*{5 }

“Mendicants,\marginnote{1.1} I do not see a single touch that occupies a man’s mind like the touch of a woman. The touch of a woman occupies a man’s mind.” 

\subsection*{6 }

“Mendicants,\marginnote{1.1} I do not see a single sight that occupies a woman’s mind like the sight of a man. The sight of a man occupies a woman’s mind.” 

\subsection*{7 }

“Mendicants,\marginnote{1.1} I do not see a single sound that occupies a woman’s mind like the sound of a man. The sound of a man occupies a woman’s mind.” 

\subsection*{8 }

“Mendicants,\marginnote{1.1} I do not see a single smell that occupies a woman’s mind like the smell of a man. The smell of a man occupies a woman’s mind.” 

\subsection*{9 }

“Mendicants,\marginnote{1.1} I do not see a single taste that occupies a woman’s mind like the taste of a man. The taste of a man occupies a woman’s mind.” 

\subsection*{10 }

“Mendicants,\marginnote{1.1} I do not see a single touch that occupies a woman’s mind like the touch of a man. The touch of a man occupies a woman’s mind.” 

%
\addtocontents{toc}{\let\protect\contentsline\protect\nopagecontentsline}
\chapter*{The Chapter on Giving Up the Hindrances }
\addcontentsline{toc}{chapter}{\tocchapterline{The Chapter on Giving Up the Hindrances }}
\addtocontents{toc}{\let\protect\contentsline\protect\oldcontentsline}

%
\section*{{\suttatitleacronym AN 1.11–20}{\suttatitleroot Nīvaraṇappahānavagga}}
\addcontentsline{toc}{section}{\tocacronym{AN 1.11–20} \tocroot{Nīvaraṇappahānavagga}}
\markboth{2. Giving Up the Hindrances }{Nīvaraṇappahānavagga}
\extramarks{AN 1.11–20}{AN 1.11–20}

\subsection*{11 }

“Mendicants,\marginnote{1.1} I do not see a single thing that gives rise to sensual desire, or, when it has arisen, makes it increase and grow like the feature of beauty. When you attend improperly to the feature of beauty, sensual desire arises, and once arisen it increases and grows.” 

\subsection*{12 }

“Mendicants,\marginnote{1.1} I do not see a single thing that gives rise to ill will, or, when it has arisen, makes it increase and grow like the feature of harshness. When you attend improperly to the feature of harshness, ill will arises, and once arisen it increases and grows.” 

\subsection*{13 }

“Mendicants,\marginnote{1.1} I do not see a single thing that gives rise to dullness and drowsiness, or, when they have arisen, makes them increase and grow like discontent, sloth, yawning, sleepiness after eating, and mental sluggishness. When you have a sluggish mind, dullness and drowsiness arise, and once arisen they increase and grow.” 

\subsection*{14 }

“Mendicants,\marginnote{1.1} I do not see a single thing that gives rise to restlessness and remorse, or, when they have arisen, makes them increase and grow like an unsettled mind. When you have no peace of mind, restlessness and remorse arise, and once arisen they increase and grow.” 

\subsection*{15 }

“Mendicants,\marginnote{1.1} I do not see a single thing that gives rise to doubt, or, when it has arisen, makes it increase and grow like improper attention. When you attend improperly, doubt arises, and once arisen it increases and grows.” 

\subsection*{16 }

“Mendicants,\marginnote{1.1} I do not see a single thing that prevents sensual desire from arising, or, when it has arisen, abandons it like the feature of ugliness. When you attend properly to the feature of ugliness, sensual desire does not arise, or, if it has already arisen, it’s given up.” 

\subsection*{17 }

“Mendicants,\marginnote{1.1} I do not see a single thing that prevents ill will from arising, or, when it has arisen, abandons it like the heart’s release by love. When you attend properly on the heart’s release by love, ill will does not arise, or, if it has already arisen, it’s given up.” 

\subsection*{18 }

“Mendicants,\marginnote{1.1} I do not see a single thing that prevents dullness and drowsiness from arising, or, when they have arisen, gives them up like the elements of initiative, persistence, and vigor. When you’re energetic, dullness and drowsiness do not arise, or, if they’ve already arisen, they’re given up.” 

\subsection*{19 }

“Mendicants,\marginnote{1.1} I do not see a single thing that prevents restlessness and remorse from arising, or, when they have arisen, gives them up like peace of mind. When your mind is peaceful, restlessness and remorse do not arise, or, if they’ve already arisen, they’re given up.” 

\subsection*{20 }

“Mendicants,\marginnote{1.1} I do not see a single thing that prevents doubt from arising, or, when it has arisen, gives it up like proper attention. When you attend properly, doubt does not arise, or, if it’s already arisen, it’s given up.” 

%
\addtocontents{toc}{\let\protect\contentsline\protect\nopagecontentsline}
\chapter*{The Chapter on the Useless Mind }
\addcontentsline{toc}{chapter}{\tocchapterline{The Chapter on the Useless Mind }}
\addtocontents{toc}{\let\protect\contentsline\protect\oldcontentsline}

%
\section*{{\suttatitleacronym AN 1.21–30}{\suttatitleroot Akammaniyavagga}}
\addcontentsline{toc}{section}{\tocacronym{AN 1.21–30} \tocroot{Akammaniyavagga}}
\markboth{3. Useless }{Akammaniyavagga}
\extramarks{AN 1.21–30}{AN 1.21–30}

\subsection*{21 }

“Mendicants,\marginnote{1.1} I do not see a single thing that, when it’s not developed like this, is as useless as the mind. An undeveloped mind is useless.” 

\subsection*{22 }

“Mendicants,\marginnote{1.1} I do not see a single thing that, when it is developed like this, is as workable as the mind. A developed mind is workable.” 

\subsection*{23 }

“Mendicants,\marginnote{1.1} I do not see a single thing that, when it’s not developed like this, is so very harmful as the mind. An undeveloped mind is very harmful.” 

\subsection*{24 }

“Mendicants,\marginnote{1.1} I do not see a single thing that, when it is developed like this, is so very beneficial as the mind. A developed mind is very beneficial.” 

\subsection*{25 }

“Mendicants,\marginnote{1.1} I do not see a single thing that, when it’s not developed, with such untapped potential, is so very harmful as the mind. An undeveloped mind, with untapped potential, is very harmful.” 

\subsection*{26 }

“Mendicants,\marginnote{1.1} I do not see a single thing that, when it is developed, with its potential realized, is so very beneficial as the mind. A developed mind, with its potential realized, is very beneficial.” 

\subsection*{27 }

“Mendicants,\marginnote{1.1} I do not see a single thing that, when it’s not developed and cultivated, is so very harmful as the mind. An undeveloped and uncultivated mind is very harmful.” 

\subsection*{28 }

“Mendicants,\marginnote{1.1} I do not see a single thing that, when it is developed and cultivated, is so very beneficial as the mind. A developed and cultivated mind is very beneficial.” 

\subsection*{29 }

“Mendicants,\marginnote{1.1} I do not see a single thing that, when it’s not developed and cultivated, brings such suffering as the mind. An undeveloped and uncultivated mind brings suffering.” 

\subsection*{30 }

“Mendicants,\marginnote{1.1} I do not see a single thing that, when it is developed and cultivated, brings such happiness as the mind. A developed and cultivated mind brings happiness.” 

%
\addtocontents{toc}{\let\protect\contentsline\protect\nopagecontentsline}
\chapter*{The Chapter on the Wild Mind }
\addcontentsline{toc}{chapter}{\tocchapterline{The Chapter on the Wild Mind }}
\addtocontents{toc}{\let\protect\contentsline\protect\oldcontentsline}

%
\section*{{\suttatitleacronym AN 1.31–40}{\suttatitleroot Adantavagga}}
\addcontentsline{toc}{section}{\tocacronym{AN 1.31–40} \tocroot{Adantavagga}}
\markboth{4. Wild }{Adantavagga}
\extramarks{AN 1.31–40}{AN 1.31–40}

\subsection*{31 }

“Mendicants,\marginnote{1.1} I do not see a single thing that, when it’s not tamed, is so very harmful as the mind. A wild mind is very harmful.” 

\subsection*{32 }

“Mendicants,\marginnote{1.1} I do not see a single thing that, when it is tamed, is so very beneficial as the mind. A tamed mind is very beneficial.” 

\subsection*{33 }

“Mendicants,\marginnote{1.1} I do not see a single thing that, when it’s not guarded, is so very harmful as the mind. An unguarded mind is very harmful.” 

\subsection*{34 }

“Mendicants,\marginnote{1.1} I do not see a single thing that, when it is guarded, is so very beneficial as the mind. A guarded mind is very beneficial.” 

\subsection*{35 }

“Mendicants,\marginnote{1.1} I do not see a single thing that, when it’s not protected, is so very harmful as the mind. An unprotected mind is very harmful.” 

\subsection*{36 }

“Mendicants,\marginnote{1.1} I do not see a single thing that, when it is protected, is so very beneficial as the mind. A protected mind is very beneficial.” 

\subsection*{37 }

“Mendicants,\marginnote{1.1} I do not see a single thing that, when it’s not restrained, is so very harmful as the mind. An unrestrained mind is very harmful.” 

\subsection*{38 }

“Mendicants,\marginnote{1.1} I do not see a single thing that, when it is restrained, is so very beneficial as the mind. A restrained mind is very beneficial.” 

\subsection*{39 }

“Mendicants,\marginnote{1.1} I do not see a single thing that, when it is not tamed, guarded, protected, and restrained, is so very harmful as the mind. An untamed, unguarded, unprotected, and unrestrained mind is very harmful.” 

\subsection*{40 }

“Mendicants,\marginnote{1.1} I do not see a single thing that, when it is tamed, guarded, protected, and restrained, is so very beneficial as the mind. A tamed, guarded, protected, and restrained mind is very beneficial.” 

%
\addtocontents{toc}{\let\protect\contentsline\protect\nopagecontentsline}
\chapter*{The Chapter on a Spike }
\addcontentsline{toc}{chapter}{\tocchapterline{The Chapter on a Spike }}
\addtocontents{toc}{\let\protect\contentsline\protect\oldcontentsline}

%
\section*{{\suttatitleacronym AN 1.41–50}{\suttatitleroot Paṇihitaacchavagga}}
\addcontentsline{toc}{section}{\tocacronym{AN 1.41–50} \tocroot{Paṇihitaacchavagga}}
\markboth{5. A Spike }{Paṇihitaacchavagga}
\extramarks{AN 1.41–50}{AN 1.41–50}

\subsection*{41 }

“Mendicants,\marginnote{1.1} suppose a spike of rice or barley was pointing the wrong way. If you trod on it with hand or foot, there’s no way it could break the skin and produce blood. Why is that? Because the spike is pointing the wrong way. In the same way, a mendicant whose mind is pointing the wrong way cannot break ignorance, produce knowledge, and realize extinguishment. Why is that? Because their mind is pointing the wrong way.” 

\subsection*{42 }

“Mendicants,\marginnote{1.1} suppose a spike of rice or barley was pointing the right way. If you trod on it with hand or foot, it may well break the skin and produce blood. Why is that? Because the spike is pointing the right way. In the same way, a mendicant whose mind is pointing the right way can break ignorance, produce knowledge, and realize extinguishment. Why is that? Because the mind is pointing the right way.” 

\subsection*{43 }

“Mendicants,\marginnote{1.1} when I’ve comprehended the mind of a person whose mind is corrupted, I understand: ‘If this person were to die right now, they would be cast down to hell.’ Why is that? Because their mind is corrupted. Corruption of mind is the reason why some sentient beings, when their body breaks up, after death, are reborn in a place of loss, a bad place, the underworld, hell.” 

\subsection*{44 }

“Mendicants,\marginnote{1.1} when I’ve comprehended the mind of a person whose mind is pure, I understand: ‘If this person were to die right now, they would be raised up to heaven.’ Why is that? Because their mind is pure. Purity of mind is the reason why some sentient beings, when their body breaks up, after death, are reborn in a good place, a heavenly realm.” 

\subsection*{45 }

“Suppose,\marginnote{1.1} mendicants, there was a lake that was cloudy, murky, and muddy. A person with good eyesight standing on the bank would not see the clams and mussels, and pebbles and gravel, and schools of fish swimming about or staying still. Why is that? Because the water is clouded. In the same way, that a mendicant whose mind is clouded would know what’s for their own good, the good of another, or the good of both; or that they would realize any superhuman distinction in knowledge and vision worthy of the noble ones: this is not possible. Why is that? Because their mind is clouded.” 

\subsection*{46 }

“Suppose,\marginnote{1.1} mendicants, there was a lake that was transparent, clear, and unclouded. A person with good eyesight standing on the bank would see the clams and mussels, and pebbles and gravel, and schools of fish swimming about or staying still. Why is that? Because the water is unclouded. In the same way, that a mendicant whose mind is not clouded would know what’s for their own good, the good of another, or the good of both; or that they would realize any superhuman distinction in knowledge and vision worthy of the noble ones: this is possible. Why is that? Because their mind is unclouded.” 

\subsection*{47 }

“Just\marginnote{1.1} as, mendicants, the boxwood is said to be the best kind of tree in terms of its pliability and workability, so too, I do not see a single thing that’s as pliable and workable as the mind, when it is developed and cultivated. A mind that is developed and cultivated is pliable and workable.” 

\subsection*{48 }

“Mendicants,\marginnote{1.1} I do not see a single thing that’s as quick to change as the mind. So much so that it’s not easy to give a simile for how quickly the mind changes.” 

\subsection*{49 }

“This\marginnote{1.1} mind, mendicants, is radiant. But it’s corrupted by passing corruptions.” 

\subsection*{50 }

“This\marginnote{1.1} mind, mendicants, is radiant. And it is freed from passing corruptions.” 

%
\addtocontents{toc}{\let\protect\contentsline\protect\nopagecontentsline}
\chapter*{The Chapter on a Finger Snap }
\addcontentsline{toc}{chapter}{\tocchapterline{The Chapter on a Finger Snap }}
\addtocontents{toc}{\let\protect\contentsline\protect\oldcontentsline}

%
\section*{{\suttatitleacronym AN 1.51–60}{\suttatitleroot Accharāsaṅghātavagga}}
\addcontentsline{toc}{section}{\tocacronym{AN 1.51–60} \tocroot{Accharāsaṅghātavagga}}
\markboth{6. Finger-Snap }{Accharāsaṅghātavagga}
\extramarks{AN 1.51–60}{AN 1.51–60}

\subsection*{51 }

“This\marginnote{1.1} mind, mendicants, is radiant. But it is corrupted by passing corruptions. An uneducated ordinary person does not truly understand this. So I say that the uneducated ordinary person has no development of the mind.” 

\subsection*{52 }

“This\marginnote{1.1} mind, mendicants, is radiant. And it is freed from passing corruptions. An educated noble disciple truly understands this. So I say that the educated noble disciple has development of the mind.” 

\subsection*{53 }

“If,\marginnote{1.1} mendicants, a mendicant cultivates a mind of love even as long as a finger-snap, they’re called a mendicant who does not lack absorption, who follows the Teacher’s instructions, who responds to advice, and who does not eat the country’s alms in vain. How much more so those who make much of it!” 

\subsection*{54 }

“If,\marginnote{1.1} mendicants, a mendicant develops a mind of love even as long as a finger-snap, they’re called a mendicant who does not lack absorption, who follows the Teacher’s instructions, who responds to advice, and who does not eat the country’s alms in vain. How much more so those who make much of it!” 

\subsection*{55 }

“If,\marginnote{1.1} mendicants, a mendicant focuses on a mind of love even as long as a finger-snap, they’re called a mendicant who does not lack absorption, who follows the Teacher’s instructions, who responds to advice, and who does not eat the country’s alms in vain. How much more so those who make much of it!” 

\subsection*{56 }

“Mendicants,\marginnote{1.1} intention shapes all phenomena whatsoever that are unskillful, part of the unskillful, on the side of the unskillful. Intention is the first of those phenomena to arise, and unskillful phenomena follow right behind.” 

\subsection*{57 }

“Mendicants,\marginnote{1.1} intention shapes all phenomena whatsoever that are skillful, part of the skillful, on the side of the skillful. Intention is the first of those phenomena to arise, and skillful phenomena follow right behind.” 

\subsection*{58 }

“Mendicants,\marginnote{1.1} I do not see a single thing that gives rise to unskillful qualities, or makes skillful qualities decline like negligence. When you’re negligent, unskillful qualities arise and skillful qualities decline.” 

\subsection*{59 }

“Mendicants,\marginnote{1.1} I do not see a single thing that gives rise to skillful qualities, or makes unskillful qualities decline like diligence. When you’re diligent, skillful qualities arise and unskillful qualities decline.” 

\subsection*{60 }

“Mendicants,\marginnote{1.1} I do not see a single thing that gives rise to unskillful qualities, or makes skillful qualities decline like laziness. When you’re lazy, unskillful qualities arise and skillful qualities decline.” 

%
\addtocontents{toc}{\let\protect\contentsline\protect\nopagecontentsline}
\chapter*{The Chapter on Arousing Energy }
\addcontentsline{toc}{chapter}{\tocchapterline{The Chapter on Arousing Energy }}
\addtocontents{toc}{\let\protect\contentsline\protect\oldcontentsline}

%
\section*{{\suttatitleacronym AN 1.61–70}{\suttatitleroot Vīriyārambhādivagga}}
\addcontentsline{toc}{section}{\tocacronym{AN 1.61–70} \tocroot{Vīriyārambhādivagga}}
\markboth{7. Arousing Energy }{Vīriyārambhādivagga}
\extramarks{AN 1.61–70}{AN 1.61–70}

\subsection*{61 }

“Mendicants,\marginnote{1.1} I do not see a single thing that gives rise to skillful qualities, or makes unskillful qualities decline like arousing energy. When you’re energetic, skillful qualities arise and unskillful qualities decline.” 

\subsection*{62 }

“Mendicants,\marginnote{1.1} I do not see a single thing that gives rise to unskillful qualities, or makes skillful qualities decline like having many wishes. When you have many wishes, unskillful qualities arise and skillful qualities decline.” 

\subsection*{63 }

“Mendicants,\marginnote{1.1} I do not see a single thing that gives rise to skillful qualities, or makes unskillful qualities decline like having few wishes. When you have few wishes, skillful qualities arise and unskillful qualities decline.” 

\subsection*{64 }

“Mendicants,\marginnote{1.1} I do not see a single thing that gives rise to unskillful qualities, or makes skillful qualities decline like lack of contentment. When you lack contentment, unskillful qualities arise and skillful qualities decline.” 

\subsection*{65 }

“Mendicants,\marginnote{1.1} I do not see a single thing that gives rise to skillful qualities, or makes unskillful qualities decline like contentment. When you’re contented, skillful qualities arise and unskillful qualities decline.” 

\subsection*{66 }

“Mendicants,\marginnote{1.1} I do not see a single thing that gives rise to unskillful qualities, or makes skillful qualities decline like improper attention. When you attend improperly, unskillful qualities arise and skillful qualities decline.” 

\subsection*{67 }

“Mendicants,\marginnote{1.1} I do not see a single thing that gives rise to skillful qualities, or makes unskillful qualities decline like proper attention. When you attend properly, skillful qualities arise and unskillful qualities decline.” 

\subsection*{68 }

“Mendicants,\marginnote{1.1} I do not see a single thing that gives rise to unskillful qualities, or makes skillful qualities decline like lack of situational awareness. When you lack situational awareness, unskillful qualities arise and skillful qualities decline.” 

\subsection*{69 }

“Mendicants,\marginnote{1.1} I do not see a single thing that gives rise to skillful qualities, or makes unskillful qualities decline like situational awareness. When you have situational awareness, skillful qualities arise and unskillful qualities decline.” 

\subsection*{70 }

“Mendicants,\marginnote{1.1} I do not see a single thing that gives rise to unskillful qualities, or makes skillful qualities decline like bad friends. When you have bad friends, unskillful qualities arise and skillful qualities decline.” 

%
\addtocontents{toc}{\let\protect\contentsline\protect\nopagecontentsline}
\chapter*{The Chapter on Good Friends }
\addcontentsline{toc}{chapter}{\tocchapterline{The Chapter on Good Friends }}
\addtocontents{toc}{\let\protect\contentsline\protect\oldcontentsline}

%
\section*{{\suttatitleacronym AN 1.71–81}{\suttatitleroot Kalyāṇamittādivagga}}
\addcontentsline{toc}{section}{\tocacronym{AN 1.71–81} \tocroot{Kalyāṇamittādivagga}}
\markboth{8. Good Friends }{Kalyāṇamittādivagga}
\extramarks{AN 1.71–81}{AN 1.71–81}

\subsection*{71 }

“Mendicants,\marginnote{1.1} I do not see a single thing that gives rise to skillful qualities, or makes unskillful qualities decline like good friends. When you have good friends, skillful qualities arise and unskillful qualities decline.” 

\subsection*{72 }

“Mendicants,\marginnote{1.1} I do not see a single thing that gives rise to unskillful qualities, or makes skillful qualities decline like pursuing bad habits and not good habits. When you pursue bad habits and not good habits, unskillful qualities arise and skillful qualities decline.” 

\subsection*{73 }

“Mendicants,\marginnote{1.1} I do not see a single thing that gives rise to skillful qualities, or makes unskillful qualities decline like pursuing good habits and not bad habits. When you pursue good habits and not bad habits, skillful qualities arise and unskillful qualities decline.” 

\subsection*{74 }

“Mendicants,\marginnote{1.1} I do not see a single thing that prevents the awakening factors from arising, or, if they’ve already arisen, prevents them from being fully developed like improper attention. When you attend improperly, the awakening factors don’t arise, or, if they’ve already arisen, they’re not fully developed.” 

\subsection*{75 }

“Mendicants,\marginnote{1.1} I do not see a single thing that gives rise to the awakening factors, or, if they’ve already arisen, fully develops them like proper attention. When you attend properly, the awakening factors arise, or, if they’ve already arisen, they’re fully developed.” 

\subsection*{76 }

“Loss\marginnote{1.1} of relatives, mendicants, is a small thing. Wisdom is the worst thing to lose.” 

\subsection*{77 }

“Growth\marginnote{1.1} of relatives, mendicants, is a small thing. Wisdom is the best thing to grow. 

So\marginnote{1.3} you should train like this: ‘We will grow in wisdom.’ That’s how you should train.” 

\subsection*{78 }

“Loss\marginnote{1.1} of wealth, mendicants, is a small thing. Wisdom is the worst thing to lose.” 

\subsection*{79 }

“Growth\marginnote{1.1} of wealth, mendicants, is a small thing. Wisdom is the best thing to grow. 

So\marginnote{1.3} you should train like this: ‘We will grow in wisdom.’ That’s how you should train.” 

\subsection*{80 }

“Loss\marginnote{1.1} of fame, mendicants, is a small thing. Wisdom is the worst thing to lose.” 

\subsection*{81 }

“Growth\marginnote{1.1} of fame, mendicants, is a small thing. Wisdom is the best thing to grow. 

So\marginnote{1.3} you should train like this: ‘We will grow in wisdom.’ That’s how you should train.” 

%
\addtocontents{toc}{\let\protect\contentsline\protect\nopagecontentsline}
\chapter*{The Chapter on Negligence }
\addcontentsline{toc}{chapter}{\tocchapterline{The Chapter on Negligence }}
\addtocontents{toc}{\let\protect\contentsline\protect\oldcontentsline}

%
\section*{{\suttatitleacronym AN 1.82–97}{\suttatitleroot Pamādādivagga}}
\addcontentsline{toc}{section}{\tocacronym{AN 1.82–97} \tocroot{Pamādādivagga}}
\markboth{9. Negligence }{Pamādādivagga}
\extramarks{AN 1.82–97}{AN 1.82–97}

\subsection*{82 }

“Mendicants,\marginnote{1.1} I do not see a single thing that is so very harmful as negligence. Negligence is very harmful.” 

\subsection*{83 }

“Mendicants,\marginnote{1.1} I do not see a single thing that is so very beneficial as diligence. Diligence is very beneficial.” 

\subsection*{84 }

“Mendicants,\marginnote{1.1} I do not see a single thing that is so very harmful as laziness. Laziness is very harmful.” 

\subsection*{85 }

“Mendicants,\marginnote{1.1} I do not see a single thing that is so very beneficial as arousing energy. Arousing energy is very beneficial.” 

\subsection*{86 }

“Mendicants,\marginnote{1.1} I do not see a single thing that is so very harmful as having many wishes. Having many wishes is very harmful.” 

\subsection*{87 }

“Mendicants,\marginnote{1.1} I do not see a single thing that is so very beneficial as having few wishes. Having few wishes is very beneficial.” 

\subsection*{88 }

“Mendicants,\marginnote{1.1} I do not see a single thing that is so very harmful as lack of contentment. Lack of contentment is very harmful.” 

\subsection*{89 }

“Mendicants,\marginnote{1.1} I do not see a single thing that is so very beneficial as contentment. Contentment is very beneficial.” 

\subsection*{90 }

“Mendicants,\marginnote{1.1} I do not see a single thing that is so very harmful as improper attention. Improper attention is very harmful.” 

\subsection*{91 }

“Mendicants,\marginnote{1.1} I do not see a single thing that is so very beneficial as proper attention. Proper attention is very beneficial.” 

\subsection*{92 }

“Mendicants,\marginnote{1.1} I do not see a single thing that is so very harmful as lack of situational awareness. Lack of situational awareness is very harmful.” 

\subsection*{93 }

“Mendicants,\marginnote{1.1} I do not see a single thing that is so very beneficial as situational awareness. Situational awareness is very beneficial.” 

\subsection*{94 }

“Mendicants,\marginnote{1.1} I do not see a single thing that is so very harmful as bad friends. Bad friends are very harmful.” 

\subsection*{95 }

“Mendicants,\marginnote{1.1} I do not see a single thing that is so very beneficial as good friends. Good friends are very beneficial.” 

\subsection*{96 }

“Mendicants,\marginnote{1.1} I do not see a single thing that is so very harmful as pursuing bad habits and not good habits. Pursuing bad habits and not good habits is very harmful.” 

\subsection*{97 }

“Mendicants,\marginnote{1.1} I do not see a single thing that is so very beneficial as pursuing good habits and not bad habits. Pursuing good habits and not bad habits is very beneficial.” 

%
\addtocontents{toc}{\let\protect\contentsline\protect\nopagecontentsline}
\chapter*{The Chapter on Negligence (2nd) }
\addcontentsline{toc}{chapter}{\tocchapterline{The Chapter on Negligence (2nd) }}
\addtocontents{toc}{\let\protect\contentsline\protect\oldcontentsline}

%
\section*{{\suttatitleacronym AN 1.98–139}{\suttatitleroot Dutiyapamādādivagga}}
\addcontentsline{toc}{section}{\tocacronym{AN 1.98–139} \tocroot{Dutiyapamādādivagga}}
\markboth{10. Negligence (2nd) }{Dutiyapamādādivagga}
\extramarks{AN 1.98–139}{AN 1.98–139}

\subsection*{98 }

“Taking\marginnote{1.1} into account interior factors, mendicants, I do not see a single one that is so very harmful as negligence. Negligence is very harmful.” 

\subsection*{99 }

“Taking\marginnote{1.1} into account interior factors, mendicants, I do not see a single one that is so very beneficial as diligence. Diligence is very beneficial.” 

\subsection*{100 }

“Taking\marginnote{1.1} into account interior factors, mendicants, I do not see a single one that is so very harmful as laziness. Laziness is very harmful.” 

\subsection*{101 }

“Taking\marginnote{1.1} into account interior factors, mendicants, I do not see a single one that is so very beneficial as arousing energy. Arousing energy is very beneficial.” 

\subsection*{102–109 }

“Taking\marginnote{1.1} into account interior factors, mendicants, I do not see a single one that is so very harmful as having many wishes … having few wishes … lack of contentment … contentment … improper attention … proper attention … lack of situational awareness … situational awareness …” 

\subsection*{110 }

“Taking\marginnote{1.1} into account exterior factors, mendicants, I do not see a single one that is so very harmful as bad friends. Bad friends are very harmful.” 

\subsection*{111 }

“Taking\marginnote{1.1} into account exterior factors, mendicants, I do not see a single one that is so very beneficial as good friends. Good friends are very beneficial.” 

\subsection*{112 }

“Taking\marginnote{1.1} into account interior factors, mendicants, I do not see a single one that is so very harmful as pursuing bad habits and not good habits. Pursuing bad habits and not good habits is very harmful.” 

\subsection*{113 }

“Taking\marginnote{1.1} into account interior factors, mendicants, I do not see a single one that is so very beneficial as pursuing good habits and not bad habits. Pursuing good habits and not bad habits is very beneficial.” 

\subsection*{114 }

“Mendicants,\marginnote{1.1} I do not see a single thing that leads to the decline and disappearance of the true teaching like negligence. Negligence leads to the decline and disappearance of the true teaching.” 

\subsection*{115 }

“Mendicants,\marginnote{1.1} I do not see a single thing that leads to the continuation, persistence, and enduring of the true teaching like diligence. Diligence leads to the continuation, persistence, and enduring of the true teaching.” 

\subsection*{116 }

“Mendicants,\marginnote{1.1} I do not see a single thing that leads to the decline and disappearance of the true teaching like laziness. Laziness leads to the decline and disappearance of the true teaching.” 

\subsection*{117 }

“Mendicants,\marginnote{1.1} I do not see a single thing that leads to the continuation, persistence, and enduring of the true teaching like arousing energy. Arousing energy leads to the continuation, persistence, and enduring of the true teaching.” 

\subsection*{118–128 }

“Mendicants,\marginnote{1.1} I do not see a single thing that leads to the decline and disappearance of the true teaching like having many wishes … having few wishes … lack of contentment … contentment … improper attention … proper attention … lack of situational awareness … situational awareness … bad friends … good friends … pursuing bad habits and not good habits. Pursuing bad habits and not good habits leads to the decline and disappearance of the true teaching.” 

\subsection*{129 }

“Mendicants,\marginnote{1.1} I do not see a single thing that leads to the continuation, persistence, and enduring of the true teaching like pursuing good habits and not bad habits. Pursuing good habits and not bad habits leads to the continuation, persistence, and enduring of the true teaching” 

\subsection*{130 }

“Mendicants,\marginnote{1.1} those mendicants who explain what is not the teaching as the teaching are acting for the hurt and unhappiness of the people, for the harm, hurt, and suffering of gods and humans. They make much bad karma and make the true teaching disappear.” 

\subsection*{131 }

“Mendicants,\marginnote{1.1} those mendicants who explain what is the teaching as not the teaching are acting for the hurt and unhappiness of the people, for the harm, hurt, and suffering of gods and humans. They make much bad karma and make the true teaching disappear.” 

\subsection*{132–139 }

“Those\marginnote{1.1} mendicants who explain what is not found in the monastic law as found in the monastic law … what is found in the the monastic law as not found in the monastic law … what was not spoken and stated by the Realized One as spoken and stated by the Realized One … what was spoken and stated by the Realized One as not spoken and stated by the Realized One … what was not practiced by the Realized One as practiced by the Realized One … what was practiced by the Realized One as not practiced by the Realized One … what was not prescribed by the Realized One as prescribed by the Realized One … what was prescribed by the Realized One as not prescribed by the Realized One are acting for the hurt and unhappiness of the people, for the harm, hurt, and suffering of gods and humans. They make much bad karma and make the true teaching disappear.” 

%
\addtocontents{toc}{\let\protect\contentsline\protect\nopagecontentsline}
\chapter*{The Chapter on Not the Teaching }
\addcontentsline{toc}{chapter}{\tocchapterline{The Chapter on Not the Teaching }}
\addtocontents{toc}{\let\protect\contentsline\protect\oldcontentsline}

%
\section*{{\suttatitleacronym AN 1.140–149}{\suttatitleroot Adhammavagga}}
\addcontentsline{toc}{section}{\tocacronym{AN 1.140–149} \tocroot{Adhammavagga}}
\markboth{11. Not the Teaching }{Adhammavagga}
\extramarks{AN 1.140–149}{AN 1.140–149}

\subsection*{140 }

“Mendicants,\marginnote{1.1} those mendicants who explain what is not the teaching as not the teaching are acting for the welfare and happiness of the people, for the benefit, welfare, and happiness of gods and humans. They make much merit and make the true teaching continue.” 

\subsection*{141 }

“Mendicants,\marginnote{1.1} those mendicants who explain what is the teaching as the teaching are acting for the welfare and happiness of the people, for the benefit, welfare, and happiness of gods and humans. They make much merit and make the true teaching continue.” 

\subsection*{142–149 }

“Those\marginnote{1.1} mendicants who explain what is not found in the monastic law as not found in the monastic law … what is found in the monastic law as found in the monastic law … what was not spoken and stated by the Realized One as not spoken and stated by the Realized One … what was spoken and stated by the Realized One as spoken and stated by the Realized One … what was not practiced by the Realized One as not practiced by the Realized One … what was practiced by the Realized One as practiced by the Realized One … what was not prescribed by the Realized One as not prescribed by the Realized One … what was prescribed by the Realized One as prescribed by the Realized One … are acting for the welfare and happiness of the people, for the benefit, welfare, and happiness of gods and humans. They make much merit and make the true teaching continue.” 

%
\addtocontents{toc}{\let\protect\contentsline\protect\nopagecontentsline}
\chapter*{The Chapter on Non-offense }
\addcontentsline{toc}{chapter}{\tocchapterline{The Chapter on Non-offense }}
\addtocontents{toc}{\let\protect\contentsline\protect\oldcontentsline}

%
\section*{{\suttatitleacronym AN 1.150–169}{\suttatitleroot Anāpattivagga}}
\addcontentsline{toc}{section}{\tocacronym{AN 1.150–169} \tocroot{Anāpattivagga}}
\markboth{12. Non-offense }{Anāpattivagga}
\extramarks{AN 1.150–169}{AN 1.150–169}

\subsection*{150 }

“Mendicants,\marginnote{1.1} those mendicants who explain non-offense as an offense are acting for the hurt and unhappiness of the people, for the harm, hurt, and suffering of gods and humans. They make much bad karma and make the true teaching disappear.” 

\subsection*{151 }

“Mendicants,\marginnote{1.1} those mendicants who explain an offense as non-offense are acting for the hurt and unhappiness of the people, for the harm, hurt, and suffering of gods and humans. Those mendicants make much bad karma and make the true teaching disappear.” 

\subsection*{152–159 }

“Those\marginnote{1.1} mendicants who explain a light offense as a serious offense … a serious offense as a light offense … an offense committed with corrupt intention as an offense not committed with corrupt intention … an offense not committed with corrupt intention as an offense committed with corrupt intention … an offense requiring rehabilitation as an offense not requiring rehabilitation … an offense not requiring rehabilitation as an offense requiring rehabilitation … an offense with redress as an offense without redress … an offense without redress as an offense with redress are acting for the hurt and unhappiness of the people, for the harm, hurt, and suffering of gods and humans. Those mendicants make much bad karma and make the true teaching disappear.” 

\subsection*{160 }

“Mendicants,\marginnote{1.1} those mendicants who explain non-offense as non-offense are acting for the welfare and happiness of the people, for the benefit, welfare, and happiness of gods and humans. They make much merit and make the true teaching continue.” 

\subsection*{161 }

“Mendicants,\marginnote{1.1} those mendicants who explain an offense as an offense are acting for the welfare and happiness of the people, for the benefit, welfare, and happiness of gods and humans. They make much merit and make the true teaching continue.” 

\subsection*{162–169 }

“Those\marginnote{1.1} mendicants who explain a light offense as a light offense … a serious offense as a serious offense … an offense committed with corrupt intention as an offense committed with corrupt intention … an offense not committed with corrupt intention as an offense not committed with corrupt intention … an offense requiring rehabilitation as an offense requiring rehabilitation … an offense not requiring rehabilitation as an offense not requiring rehabilitation … an offense with redress as an offense with redress … an offense without redress as an offense without redress are acting for the welfare and happiness of the people, for the benefit, welfare, and happiness of the people, of gods and humans. They make much merit and make the true teaching continue.” 

%
\addtocontents{toc}{\let\protect\contentsline\protect\nopagecontentsline}
\chapter*{The Chapter on One Person }
\addcontentsline{toc}{chapter}{\tocchapterline{The Chapter on One Person }}
\addtocontents{toc}{\let\protect\contentsline\protect\oldcontentsline}

%
\section*{{\suttatitleacronym AN 1.170–187}{\suttatitleroot Ekapuggalavagga}}
\addcontentsline{toc}{section}{\tocacronym{AN 1.170–187} \tocroot{Ekapuggalavagga}}
\markboth{13. One Person }{Ekapuggalavagga}
\extramarks{AN 1.170–187}{AN 1.170–187}

\subsection*{170 }

“One\marginnote{1.1} person, mendicants, arises in the world for the welfare and happiness of the people, out of compassion for the world, for the benefit, welfare, and happiness of gods and humans. What one person? The Realized One, the perfected one, the fully awakened Buddha. This is the one person, mendicants, who arises in the world for the welfare and happiness of the people, out of compassion for the world, for the benefit, welfare, and happiness of gods and humans.” 

\subsection*{171 }

“The\marginnote{1.1} appearance of one person, mendicants, is rare in the world. What one person? The Realized One, the perfected one, the fully awakened Buddha. This is the one person, mendicants, whose appearance is rare in the world.” 

\subsection*{172 }

“One\marginnote{1.1} person, mendicants, arises in the world who is an incredible human being. What one person? The Realized One, the perfected one, the fully awakened Buddha. This is the one person, mendicants, who arises in the world who is an incredible human being.” 

\subsection*{173 }

“The\marginnote{1.1} death of one person, mendicants, is regretted by many people. What one person? The Realized One, the perfected one, the fully awakened Buddha. This is the one person, mendicants, whose death is regretted by many people.” 

\subsection*{174 }

“One\marginnote{1.1} person, mendicants, arises in the world unique, without peer or counterpart, incomparable, matchless, unrivaled, unequaled, without equal, the best of men. What one person? The Realized One, the perfected one, the fully awakened Buddha. This is the one person, mendicants, who arises in the world unique, without peer or counterpart, incomparable, matchless, unrivaled, unequaled, without equal, the best of men.” 

\subsection*{175–186 }

“With\marginnote{1.1} the appearance of one person, mendicants, there is the appearance of a great eye, a great light, a great radiance, and the six unsurpassable things; the realization of the four kinds of textual analysis; the penetration of many and diverse elements; the realization of the fruit of knowledge and freedom; the realization of the fruits of stream-entry, once-return, non-return, and perfection. What one person? The Realized One, the perfected one, the fully awakened Buddha. This is the one person whose appearance brings the appearance of a great eye, a great light, a great radiance, and the six unsurpassable things; the realization of the four kinds of textual analysis; the penetration of many and diverse elements; the realization of the fruit of knowledge and release; the realization of the fruits of stream-entry, once-return, non-return, and perfection.” 

\subsection*{187 }

“Mendicants,\marginnote{1.1} I do not see a single other person who rightly keeps rolling the supreme Wheel of Dhamma that was rolled forth by the Realized One like \textsanskrit{Sāriputta}. \textsanskrit{Sāriputta} rightly keeps rolling the supreme Wheel of Dhamma that was rolled forth by the Realized One.” 

%
\addtocontents{toc}{\let\protect\contentsline\protect\nopagecontentsline}
\chapter*{Seven Chapters on the Foremost Persons }
\addcontentsline{toc}{chapter}{\tocchapterline{Seven Chapters on the Foremost Persons }}
\addtocontents{toc}{\let\protect\contentsline\protect\oldcontentsline}

%
\section*{{\suttatitleacronym AN 1.188–197}{\suttatitleroot Paṭhamavagga}}
\addcontentsline{toc}{section}{\tocacronym{AN 1.188–197} \tocroot{Paṭhamavagga}}
\markboth{14. First }{Paṭhamavagga}
\extramarks{AN 1.188–197}{AN 1.188–197}

“The\marginnote{1.1} foremost of my monk disciples in seniority is \textsanskrit{Koṇḍañña} Who Understood. 

…\marginnote{1.1} with great wisdom is \textsanskrit{Sāriputta}. 

…\marginnote{1.1} with psychic power is \textsanskrit{Mahāmoggallāna}. 

…\marginnote{1.1} who advocate austerities is \textsanskrit{Mahākassapa}. 

…\marginnote{1.1} with clairvoyance is Anuruddha. 

…\marginnote{1.1} from eminent families is Bhaddiya son of \textsanskrit{Kāḷīgodhā}. 

…\marginnote{1.1} with a charming voice is Bhaddiya the Dwarf. 

…\marginnote{1.1} with a lion’s roar is \textsanskrit{Bhāradvāja} the Alms-gatherer. 

…\marginnote{1.1} who speak on the teaching is \textsanskrit{Puṇṇa} son of \textsanskrit{Mantāṇī}. 

…\marginnote{1.1} who explain in detail the meaning of a brief statement is \textsanskrit{Mahākaccāna}.” 

%
\section*{{\suttatitleacronym AN 1.198–208}{\suttatitleroot Dutiyavagga}}
\addcontentsline{toc}{section}{\tocacronym{AN 1.198–208} \tocroot{Dutiyavagga}}
\markboth{15. Second }{Dutiyavagga}
\extramarks{AN 1.198–208}{AN 1.198–208}

“The\marginnote{1.1} foremost of my monk disciples in creating a mind-made body is \textsanskrit{Cūḷapanthaka}. 

…\marginnote{1.1} who are skilled in the evolution of consciousness is \textsanskrit{Cūḷapanthaka}. 

…\marginnote{1.1} who are skilled in the evolution of perception is \textsanskrit{Mahāpanthaka}. 

…\marginnote{1.1} who live without conflict is \textsanskrit{Subhūti}. 

…\marginnote{1.1} who are worthy of a religious donation is \textsanskrit{Subhūti}. 

…\marginnote{1.1} who stay in the wilderness is Revata of the Acacia Wood. 

…\marginnote{1.1} who practice absorption is Revata the Doubter. 

…\marginnote{1.1} who are energetic is \textsanskrit{Soṇa} \textsanskrit{Koḷivisa}. 

…\marginnote{1.1} who are good speakers is \textsanskrit{Soṇa} of the Sharp Ears. 

…\marginnote{1.1} who receive many possessions is \textsanskrit{Sīvali}. 

…\marginnote{1.1} who are committed to faith is \textsanskrit{Vakkalī}.” 

%
\section*{{\suttatitleacronym AN 1.209–218}{\suttatitleroot Tatiyavagga}}
\addcontentsline{toc}{section}{\tocacronym{AN 1.209–218} \tocroot{Tatiyavagga}}
\markboth{16. Third }{Tatiyavagga}
\extramarks{AN 1.209–218}{AN 1.209–218}

“The\marginnote{1.1} foremost of my monk disciples who want to train is \textsanskrit{Rāhula}. 

…\marginnote{1.1} who went forth out of faith is \textsanskrit{Raṭṭhapāla}. 

…\marginnote{1.1} who are the first to pick up a ballot slip is \textsanskrit{Kuṇḍadhāna}. 

…\marginnote{1.1} who are eloquent poets is \textsanskrit{Vaṅgīsa}. 

…\marginnote{1.1} who are impressive all around is Upasena son of \textsanskrit{Vaṅgantā}. 

…\marginnote{1.1} who assign lodgings is Dabba Mallaputta. 

…\marginnote{1.1} who are beloved of the deities is Pilindavaccha. 

…\marginnote{1.1} with swift insight is \textsanskrit{Bāhiya} of the Bark Cloth. 

…\marginnote{1.1} with brilliant speech is Kassapa the Prince. 

…\marginnote{1.1} who have attained the methods of textual analysis is \textsanskrit{Mahākoṭṭhita}.” 

%
\section*{{\suttatitleacronym AN 1.219–234}{\suttatitleroot Catutthavagga}}
\addcontentsline{toc}{section}{\tocacronym{AN 1.219–234} \tocroot{Catutthavagga}}
\markboth{17. Fourth }{Catutthavagga}
\extramarks{AN 1.219–234}{AN 1.219–234}

“The\marginnote{1.1} foremost of my monk disciples who are very learned is Ānanda. 

…\marginnote{1.1} with a good memory is Ānanda. 

…\marginnote{1.1} with an extensive range is Ānanda. 

…\marginnote{1.1} in retention is Ānanda. 

…\marginnote{1.1} as a personal attendant is Ānanda. 

…\marginnote{1.1} with a large congregation is Kassapa of \textsanskrit{Uruvelā}. 

…\marginnote{1.1} who inspire lay families is \textsanskrit{Kāḷudāyī}. 

…\marginnote{1.1} with good health is Bakkula. 

…\marginnote{1.1} who recollect past lives is Sobhita. 

…\marginnote{1.1} who have memorized the monastic law is \textsanskrit{Upāli}. 

…\marginnote{1.1} who advise the nuns is Nandaka. 

…\marginnote{1.1} who guard the sense doors is Nanda. 

…\marginnote{1.1} who advise the monks is \textsanskrit{Mahākappina}. 

…\marginnote{1.1} who are skilled in the fire element is \textsanskrit{Sāgata}. 

…\marginnote{1.1} who inspire eloquent teachings is \textsanskrit{Rādha}. 

…\marginnote{1.1} who wear coarse robes is \textsanskrit{Mogharājā}.” 

%
\section*{{\suttatitleacronym AN 1.235–247}{\suttatitleroot Pañcamavagga}}
\addcontentsline{toc}{section}{\tocacronym{AN 1.235–247} \tocroot{Pañcamavagga}}
\markboth{18. Fifth }{Pañcamavagga}
\extramarks{AN 1.235–247}{AN 1.235–247}

“The\marginnote{1.1} foremost of my nun disciples in seniority is \textsanskrit{Mahāpajāpatī} \textsanskrit{Gotamī}. 

…\marginnote{1.1} with great wisdom is \textsanskrit{Khemā}. 

…\marginnote{1.1} with psychic power is \textsanskrit{Uppalavaṇṇā}. 

…\marginnote{1.1} who have memorized the monastic law is \textsanskrit{Paṭācārā}. 

…\marginnote{1.1} who speak on the teaching is \textsanskrit{Dhammadinnā}. 

…\marginnote{1.1} who practice absorption is \textsanskrit{Nandā}. 

…\marginnote{1.1} who are energetic is \textsanskrit{Soṇā}. 

…\marginnote{1.1} with clairvoyance is \textsanskrit{Sakulā}. 

…\marginnote{1.1} with swift insight is \textsanskrit{Bhaddā} of the Curly Hair. 

…\marginnote{1.1} who recollect past lives is \textsanskrit{Bhaddā} daughter of Kapila. 

…\marginnote{1.1} who have attained great insight is \textsanskrit{Bhaddakaccānā}. 

…\marginnote{1.1} who wear coarse robes is \textsanskrit{Kisāgotamī}. 

…\marginnote{1.1} who are committed to faith is \textsanskrit{Siṅgāla}’s Mother.” 

%
\section*{{\suttatitleacronym AN 1.248–257}{\suttatitleroot Chaṭṭhavagga}}
\addcontentsline{toc}{section}{\tocacronym{AN 1.248–257} \tocroot{Chaṭṭhavagga}}
\markboth{19. Sixth }{Chaṭṭhavagga}
\extramarks{AN 1.248–257}{AN 1.248–257}

“The\marginnote{1.1} foremost of my laymen in first going for refuge are the merchants Tapussa and Bhallika. 

…\marginnote{1.1} as a donor is the householder Sudatta \textsanskrit{Anāthapiṇḍika}. 

…\marginnote{1.1} who speak on the teaching is the householder Citta \textsanskrit{Macchikāsaṇḍika}. 

…\marginnote{1.1} who attract a congregation by the four ways of being inclusive is Hatthaka of \textsanskrit{Āḷavī}. 

…\marginnote{1.1} who donate fine things is \textsanskrit{Mahānāma} Sakka. 

…\marginnote{1.1} who donate nice things is the householder Ugga of \textsanskrit{Vesālī}. 

…\marginnote{1.1} who attend on the \textsanskrit{Saṅgha} is the householder Uggata of Elephant Village. 

…\marginnote{1.1} who have experiential confidence is \textsanskrit{Sūrambaṭṭha}. 

…\marginnote{1.1} who have confidence in a person is \textsanskrit{Jīvaka} \textsanskrit{Komārabhacca}. 

…\marginnote{1.1} who are intimate is the householder Nakula’s father.” 

%
\section*{{\suttatitleacronym AN 1.258–267}{\suttatitleroot Sattamavagga}}
\addcontentsline{toc}{section}{\tocacronym{AN 1.258–267} \tocroot{Sattamavagga}}
\markboth{20. Seventh }{Sattamavagga}
\extramarks{AN 1.258–267}{AN 1.258–267}

“The\marginnote{1.1} foremost of my laywomen in first going for refuge is \textsanskrit{Sujātā} the general’s daughter. 

…\marginnote{1.1} as a donor is \textsanskrit{Visākhā}, \textsanskrit{Migāra}’s mother. 

…\marginnote{1.1} who are very learned is \textsanskrit{Khujjuttarā}. 

…\marginnote{1.1} who dwell in love is \textsanskrit{Sāmāvatī}. 

…\marginnote{1.1} who practice absorption is \textsanskrit{Uttarānanda}’s mother. 

…\marginnote{1.1} who give fine things is \textsanskrit{Suppavāsā} the Koliyan. 

…\marginnote{1.1} who care for the sick is the laywoman \textsanskrit{Suppiyā}. 

…\marginnote{1.1} who have experiential confidence is \textsanskrit{Kātiyānī}. 

…\marginnote{1.1} who are intimate is the householder Nakula’s mother. 

…\marginnote{1.1} whose confidence is based on oral transmission is the laywoman \textsanskrit{Kāḷī} of Kuraraghara.” 

%
\addtocontents{toc}{\let\protect\contentsline\protect\nopagecontentsline}
\chapter*{Three Chapters on the Impossible }
\addcontentsline{toc}{chapter}{\tocchapterline{Three Chapters on the Impossible }}
\addtocontents{toc}{\let\protect\contentsline\protect\oldcontentsline}

%
\section*{{\suttatitleacronym AN 1.268–277}{\suttatitleroot Paṭhamavagga}}
\addcontentsline{toc}{section}{\tocacronym{AN 1.268–277} \tocroot{Paṭhamavagga}}
\markboth{21. First }{Paṭhamavagga}
\extramarks{AN 1.268–277}{AN 1.268–277}

\subsection*{268 }

“It\marginnote{1.1} is impossible, mendicants, it cannot happen for a person accomplished in view to take any condition as permanent. That is not possible. But it is possible for an ordinary person to take some condition as permanent. That is possible.” 

\subsection*{269 }

“It\marginnote{1.1} is impossible, mendicants, it cannot happen for a person accomplished in view to take any condition as pleasant. But it is possible for an ordinary person to take some condition as pleasant.” 

\subsection*{270 }

“It\marginnote{1.1} is impossible, mendicants, it cannot happen for a person accomplished in view to take anything as self. But it is possible for an ordinary person to take something as self.” 

\subsection*{271 }

“It\marginnote{1.1} is impossible, mendicants, it cannot happen for a person accomplished in view to murder their mother. But it is possible for an ordinary person to murder their mother.” 

\subsection*{272 }

“It\marginnote{1.1} is impossible, mendicants, it cannot happen for a person accomplished in view to murder their father. But it is possible for an ordinary person to murder their father.” 

\subsection*{273 }

“It\marginnote{1.1} is impossible, mendicants, it cannot happen for a person accomplished in view to murder a perfected one. But it is possible for an ordinary person to murder a perfected one.” 

\subsection*{274 }

“It\marginnote{1.1} is impossible, mendicants, it cannot happen for a person accomplished in view to injure a Realized One with malicious intent. But it is possible for an ordinary person to injure a Realized One with malicious intent.” 

\subsection*{275 }

“It\marginnote{1.1} is impossible, mendicants, it cannot happen for a person accomplished in view to cause a schism in the \textsanskrit{Saṅgha}. But it is possible for an ordinary person to cause a schism in the \textsanskrit{Saṅgha}.” 

\subsection*{276 }

“It\marginnote{1.1} is impossible, mendicants, it cannot happen for a person accomplished in view to acknowledge another teacher. But it is possible for an ordinary person to acknowledge another teacher.” 

\subsection*{277 }

“It\marginnote{1.1} is impossible, mendicants, it cannot happen for two perfected ones, fully awakened Buddhas to arise in the same solar system at the same time. But it is possible for just one perfected one, a fully awakened Buddha, to arise in one solar system.” 

%
\section*{{\suttatitleacronym AN 1.278–286}{\suttatitleroot Dutiyavagga}}
\addcontentsline{toc}{section}{\tocacronym{AN 1.278–286} \tocroot{Dutiyavagga}}
\markboth{22. Second }{Dutiyavagga}
\extramarks{AN 1.278–286}{AN 1.278–286}

\subsection*{278 }

“It\marginnote{1.1} is impossible, mendicants, it cannot happen for two wheel-turning monarchs to arise in the same solar system at the same time. But it is possible for just one wheel-turning monarch to arise in one solar system.” 

\subsection*{279 }

“It\marginnote{1.1} is impossible, mendicants, it cannot happen for a woman to be a perfected one, a fully awakened Buddha. But it is possible for a man to be a perfected one, a fully awakened Buddha.” 

\subsection*{280 }

“It\marginnote{1.1} is impossible, mendicants, it cannot happen for a woman to be a wheel-turning monarch. But it is possible for a man to be a wheel-turning monarch.” 

\subsection*{281–283 }

“It\marginnote{1.1} is impossible, mendicants, it cannot happen for a woman to perform the role of Sakka, \textsanskrit{Māra}, or \textsanskrit{Brahmā}. But it is possible for a man to perform the role of Sakka, \textsanskrit{Māra}, or \textsanskrit{Brahmā}.” 

\subsection*{284 }

“It\marginnote{1.1} is impossible, mendicants, it cannot happen for a likable, desirable, agreeable result to come from bad bodily conduct. But it is possible for an unlikable, undesirable, disagreeable result to come from bad bodily conduct.” 

\subsection*{285–286 }

“It\marginnote{1.1} is impossible, mendicants, it cannot happen for a likable, desirable, agreeable result to come from bad verbal … bad mental conduct. But it is possible for an unlikable, undesirable, disagreeable result to come from bad verbal … bad mental conduct.” 

%
\section*{{\suttatitleacronym AN 1.287–295}{\suttatitleroot Tatiyavagga}}
\addcontentsline{toc}{section}{\tocacronym{AN 1.287–295} \tocroot{Tatiyavagga}}
\markboth{23. Third }{Tatiyavagga}
\extramarks{AN 1.287–295}{AN 1.287–295}

\subsection*{287 }

“It\marginnote{1.1} is impossible, mendicants, it cannot happen for an unlikable, undesirable, disagreeable result to come from good bodily conduct. But it is possible for a likable, desirable, agreeable result to come from good bodily conduct.” 

\subsection*{288–289 }

“It\marginnote{1.1} is impossible, mendicants, it cannot happen for an unlikable, undesirable, disagreeable result to come from good verbal … good mental conduct. But it is possible for a likable, desirable, agreeable result to come from good verbal … good mental conduct.” 

\subsection*{290 }

“It\marginnote{1.1} is impossible, mendicants, it cannot happen that someone who has engaged in bad bodily conduct, could for that reason alone, when their body breaks up, after death, be reborn in a good place, a heavenly realm. But it is possible that someone who has engaged in bad bodily conduct could, for that reason alone, when their body breaks up, after death, be reborn in a place of loss, a bad place, the underworld, hell.” 

\subsection*{291–292 }

“It\marginnote{1.1} is impossible, mendicants, it cannot happen that someone who has engaged in bad verbal … bad mental conduct could, for that reason alone, when their body breaks up, after death, be reborn in a good place, a heavenly realm. But it is possible that someone who has engaged in bad verbal … bad mental conduct could, for that reason alone, when their body breaks up, after death, be reborn in a place of loss, a bad place, the underworld, hell.” 

\subsection*{293 }

“It\marginnote{1.1} is impossible, mendicants, it cannot happen that someone who has engaged in good bodily conduct could, for that reason alone, when their body breaks up, after death, be reborn in a place of loss, the underworld, a lower realm, hell. But it is possible that someone who has engaged in good bodily conduct could, for that reason alone, when their body breaks up, after death, be reborn in a good place, a heavenly realm.” 

\subsection*{294–295 }

“It\marginnote{1.1} is impossible, mendicants, it cannot happen that someone who has engaged in good verbal … good mental conduct could, for that reason alone, when their body breaks up, after death, be reborn in a place of loss, a bad place, the underworld, hell. But it is possible that someone who has engaged in good verbal … good mental conduct could, for that reason alone, when their body breaks up, after death, be reborn in a good place, heavenly realm.” 

%
\addtocontents{toc}{\let\protect\contentsline\protect\nopagecontentsline}
\chapter*{Four Chapters on One Thing }
\addcontentsline{toc}{chapter}{\tocchapterline{Four Chapters on One Thing }}
\addtocontents{toc}{\let\protect\contentsline\protect\oldcontentsline}

%
\section*{{\suttatitleacronym AN 1.296–305}{\suttatitleroot Paṭhamavagga}}
\addcontentsline{toc}{section}{\tocacronym{AN 1.296–305} \tocroot{Paṭhamavagga}}
\markboth{24. First }{Paṭhamavagga}
\extramarks{AN 1.296–305}{AN 1.296–305}

\subsection*{296 }

“One\marginnote{1.1} thing, mendicants, when developed and cultivated, leads solely to disillusionment, dispassion, cessation, peace, insight, awakening, and extinguishment. What one thing? Recollection of the Buddha. This one thing, when developed and cultivated, leads solely to disillusionment, dispassion, cessation, peace, insight, awakening, and extinguishment.” 

\subsection*{297–305 }

“One\marginnote{1.1} thing, mendicants, when developed and cultivated, leads solely to disillusionment, dispassion, cessation, peace, insight, awakening, and extinguishment. What one thing? Recollection of the teaching … Recollection of the \textsanskrit{Saṅgha} … Recollection of ethical conduct … Recollection of generosity … Recollection of the deities … Mindfulness of breathing … Mindfulness of death … Mindfulness of the body … Recollection of peace. This one thing, when developed and cultivated, leads solely to disillusionment, dispassion, cessation, peace, insight, awakening, and extinguishment.” 

%
\section*{{\suttatitleacronym AN 1.306–315}{\suttatitleroot Dutiyavagga}}
\addcontentsline{toc}{section}{\tocacronym{AN 1.306–315} \tocroot{Dutiyavagga}}
\markboth{25. Second }{Dutiyavagga}
\extramarks{AN 1.306–315}{AN 1.306–315}

\subsection*{306 }

“Mendicants,\marginnote{1.1} I do not see a single thing that gives rise to unskillful qualities, or, when they have arisen, makes them increase and grow like wrong view. When you have wrong view, unskillful qualities arise or, when they have arisen, they increase and grow.” 

\subsection*{307 }

“Mendicants,\marginnote{1.1} I do not see a single thing that gives rise to skillful qualities, or, when they have arisen, makes them increase and grow like right view. When you have right view, unarisen skillful qualities arise or, when they have arisen, they increase and grow.” 

\subsection*{308 }

“Mendicants,\marginnote{1.1} I do not see a single thing that gives rise to unskillful qualities, or makes skillful qualities decline like wrong view. When you have wrong view, unskillful qualities arise and skillful qualities decline.” 

\subsection*{309 }

“Mendicants,\marginnote{1.1} I do not see a single thing that gives rise to skillful qualities, or makes unskillful qualities decline like right view. When you have right view, skillful qualities arise and unskillful qualities decline.” 

\subsection*{310 }

“Mendicants,\marginnote{1.1} I do not see a single thing that gives rise to wrong view, and once arisen, makes it grow like improper attention. When you attend improperly, wrong view arises, and once arisen it grows.” 

\subsection*{311 }

“Mendicants,\marginnote{1.1} I do not see a single thing that gives rise to right view, or, once it has already arisen, makes it grow like proper attention. When you attend properly, right view arises, and once arisen it grows.” 

\subsection*{312 }

“Mendicants,\marginnote{1.1} I do not see a single thing that causes sentient beings to be reborn, when their body breaks up, after death, in a place of loss, a bad place, the underworld, hell like wrong view. It is because they have wrong view that sentient beings, when their body breaks up, after death, are reborn in a place of loss, a bad place, the underworld, hell.” 

\subsection*{313 }

“Mendicants,\marginnote{1.1} I do not see a single thing that causes sentient beings to be reborn, when their body breaks up, after death, in a good place, a heavenly realm like right view. It is because they have right view that sentient beings, when their body breaks up, after death, are reborn in a good place, a heavenly realm.” 

\subsection*{314 }

“Mendicants,\marginnote{1.1} when an individual has wrong view, whatever bodily, verbal, or mental deeds they undertake in line with that view, their intentions, aims, wishes, and choices all lead to what is unlikable, undesirable, disagreeable, harmful, and suffering. Why is that? Because their view is bad. Suppose a seed of neem, angled gourd, or bitter gourd was planted in moist earth. Whatever nutrients it takes up from the earth and water would lead to its bitter, acerbic, and unpleasant taste. Why is that? Because the seed is bad. In the same way, when an individual has wrong view, whatever bodily, verbal, or mental deeds they undertake in line with that view, their intentions, aims, wishes, and choices all lead to what is unlikable, undesirable, disagreeable, harmful, and suffering. Why is that? Because their view is bad.” 

\subsection*{315 }

“Mendicants,\marginnote{1.1} when an individual has right view, whatever bodily, verbal, or mental deeds they undertake in line with that view, their intentions, aims, wishes, and choices all lead to what is likable, desirable, agreeable, beneficial, and pleasant. Why is that? Because their view is good. Suppose a seed of sugar cane, fine rice, or grape was planted in moist earth. Whatever nutrients it takes up from the earth and water would lead to its sweet, pleasant, and delicious taste. Why is that? Because the seed is good. In the same way, when an individual has right view, whatever bodily, verbal, or mental deeds they undertake in line with that view, their intentions, aims, wishes, and choices all lead to what is likable, desirable, agreeable, beneficial, and pleasant. Why is that? Because their view is good.” 

%
\section*{{\suttatitleacronym AN 1.316–332}{\suttatitleroot Tatiyavagga}}
\addcontentsline{toc}{section}{\tocacronym{AN 1.316–332} \tocroot{Tatiyavagga}}
\markboth{26. Third }{Tatiyavagga}
\extramarks{AN 1.316–332}{AN 1.316–332}

\subsection*{316 }

“One\marginnote{1.1} person, mendicants, arises in the world for the hurt and unhappiness of the people, for the harm, hurt, and suffering of gods and humans. What one person? Someone with wrong view, whose perspective is distorted. They draw many people away from the true teaching and establish them in false teachings. This is one person who arises in the world for the hurt and unhappiness of the people, for the harm, hurt, and suffering of gods and humans.” 

\subsection*{317 }

“One\marginnote{1.1} person, mendicants, arises in the world for the welfare and happiness of the people, for the benefit, welfare, and happiness of gods and humans. What one person? Someone with right view, whose perspective is undistorted. They draw many people away from false teachings and establish them in the true teaching. This is one person who arises in the world for the welfare and happiness of the people, for the benefit, welfare, and happiness of gods and humans.” 

\subsection*{318 }

“Mendicants,\marginnote{1.1} I do not see a single thing that is so very blameworthy as wrong view. Wrong view is the most blameworthy thing of all.” 

\subsection*{319 }

“Mendicants,\marginnote{1.1} I do not see a single other person who acts for the hurt and unhappiness of the people, for the harm, hurt, and suffering of many people, of gods and humans like that silly man, Makkhali. Just as a trap set at the mouth of a river would bring harm, suffering, calamity, and disaster for many fish, so too that silly man, Makkhali, is a trap for humans, it seems to me. He has arisen in the world for the harm, suffering, calamity, and disaster of many beings.” 

\subsection*{320 }

“Mendicants,\marginnote{1.1} the one who encourages someone in a poorly explained teaching and training, the one who they encourage, and the one who practices accordingly all make much bad karma. Why is that? Because the teaching is poorly explained.” 

\subsection*{321 }

“Mendicants,\marginnote{1.1} the one who encourages someone in a well explained teaching and training, the one who they encourage, and the one who practices accordingly all make much merit. Why is that? Because the teaching is well explained.” 

\subsection*{322 }

“Mendicants,\marginnote{1.1} in a poorly explained teaching and training, the donor should know moderation, not the recipient. Why is that? Because the teaching is poorly explained.” 

\subsection*{323 }

“Mendicants,\marginnote{1.1} in a well explained teaching and training, the recipient should know moderation, not the donor. Why is that? Because the teaching is well explained.” 

\subsection*{324 }

“Mendicants,\marginnote{1.1} in a poorly explained teaching and training an energetic person lives in suffering. Why is that? Because the teaching is poorly explained.” 

\subsection*{325 }

“Mendicants,\marginnote{1.1} in a well explained teaching and training a lazy person lives in suffering. Why is that? Because the teaching is well explained.” 

\subsection*{326 }

“Mendicants,\marginnote{1.1} in a poorly explained teaching and training a lazy person lives happily. Why is that? Because the teaching is poorly explained.” 

\subsection*{327 }

“Mendicants,\marginnote{1.1} in a well explained teaching and training an energetic person lives happily. Why is that? Because the teaching is well explained.” 

\subsection*{328 }

“Just\marginnote{1.1} as, mendicants, even a tiny bit of fecal matter still stinks, so too I don’t approve of even a tiny bit of continued existence, not even as long as a finger-snap.” 

\subsection*{329–332 }

“Just\marginnote{1.1} as even a tiny bit of urine, or spit, or pus, or blood still stinks, so too I don’t approve of even a tiny bit of continued existence, not even as long as a finger-snap.” 

%
\section*{{\suttatitleacronym AN 1.333–377}{\suttatitleroot Catutthavagga}}
\addcontentsline{toc}{section}{\tocacronym{AN 1.333–377} \tocroot{Catutthavagga}}
\markboth{27. Fourth }{Catutthavagga}
\extramarks{AN 1.333–377}{AN 1.333–377}

\subsection*{333 }

“Just\marginnote{1.1} as, mendicants, in India the delightful parks, woods, meadows, and lotus ponds are few, while the hilly terrain, inaccessible riverlands, stumps and thorns, and rugged mountains are many; so too the sentient beings born on land are few, while those born in water are many. 

\subsection*{334 }

…\marginnote{1.1} so too the sentient beings reborn as humans are few, while those not reborn as humans are many. 

\subsection*{335 }

…\marginnote{1.1} so too the sentient beings reborn in civilized countries are few, while those reborn in the borderlands, among strange barbarian tribes, are many. 

\subsection*{336 }

…\marginnote{1.1} so too the sentient beings who are wise, bright, clever, and able to distinguish what is well said from what is poorly said are few, while the sentient beings who are witless, dull, stupid, and unable to distinguish what is well said from what is poorly said are many. 

\subsection*{337 }

…\marginnote{1.1} so too the sentient beings who have the noble eye of wisdom are few, while those who are ignorant and confused are many. 

\subsection*{338 }

…\marginnote{1.1} so too the sentient beings who get to see a Realized One are few, while those who don’t get to see a Realized One are many. 

\subsection*{339 }

…\marginnote{1.1} so too the sentient beings who get to hear the teaching and training proclaimed by a Realized One are few, while those sentient beings who don’t get to hear the teaching and training proclaimed by a Realized One are many. 

\subsection*{340 }

…\marginnote{1.1} so too the sentient beings who remember the teachings they hear are few, while those who don’t remember the teachings are many. 

\subsection*{341 }

…\marginnote{1.1} so too the sentient beings who examine the meaning of the teachings they have memorized are few, while those who don’t examine the meaning of the teachings are many. 

\subsection*{342 }

…\marginnote{1.1} so too the sentient beings who understand the meaning and the teaching and practice accordingly are few, while those who understand the meaning and the teaching but don’t practice accordingly are many. 

\subsection*{343 }

…\marginnote{1.1} so too the sentient beings inspired by inspiring places are few, while those who are uninspired are many. 

\subsection*{344 }

…\marginnote{1.1} so too the sentient beings who, being inspired, strive effectively are few, while those who, even though inspired, don’t strive effectively are many. 

\subsection*{345 }

…\marginnote{1.1} so too the sentient beings who, relying on letting go, gain immersion, gain unification of mind are few, while those who don’t gain immersion, don’t gain unification of mind relying on letting go are many. 

\subsection*{346 }

…\marginnote{1.1} so too the sentient beings who get the best food and flavors are few, while those who don’t get the best food and flavors, but get by with scraps in an alms bowl are many. 

\subsection*{347 }

…\marginnote{1.1} so too the sentient beings who get the essence of the meaning, the essence of the teaching, and the essence of freedom are few, while the sentient beings who don’t get the essence of the meaning, the essence of the teaching, and the essence of freedom are many. 

So\marginnote{1.3} you should train like this: ‘We will get the essence of the meaning, the essence of the teaching, the essence of freedom.’ That’s how you should train.” 

\subsection*{348–350 }

“Just\marginnote{1.1} as, mendicants, in India the delightful parks, woods, meadows, and lotus ponds are few, while the hilly terrain, inaccessible riverlands, stumps and thorns, and rugged mountains are many; so too, those who die as humans and are reborn as humans are few, while those who die as humans and are reborn in hell, or the animal realm, or the ghost realm are many.” 

\subsection*{351–353 }

“…\marginnote{1.1} the sentient beings who die as humans and are reborn as gods are few, while those who die as humans and are reborn in hell, or the animal realm, or the ghost realm are many.” 

\subsection*{354–356 }

“…\marginnote{1.1} the sentient beings who die as gods and are reborn as gods are few, while those who die as gods and are reborn in hell, or the animal realm, or the ghost realm are many.” 

\subsection*{357–359 }

“…\marginnote{1.1} the sentient beings who die as gods and are reborn as humans are few, while those who die as gods and are reborn in hell, or the animal realm, or the ghost realm are many.” 

\subsection*{360–362 }

“…\marginnote{1.1} the sentient beings who die in hell and are reborn as humans are few, while those who die in hell and are reborn in hell, or the animal realm, or the ghost realm are many.” 

\subsection*{363–365 }

“…\marginnote{1.1} the sentient beings who die in hell and are reborn as gods are few, while those who die in hell and are reborn in hell, or the animal realm, or the ghost realm are many.” 

\subsection*{366–368 }

“…\marginnote{1.1} the sentient beings who die as animals and are reborn as humans are few, while those who die as animals and are reborn in hell, or the animal realm, or the ghost realm are many.” 

\subsection*{369–371 }

“…\marginnote{1.1} the sentient beings who die as animals and are reborn as gods are few, while those who die as animals and are reborn in hell, or the animal realm, or the ghost realm are many.” 

\subsection*{372–374 }

“…\marginnote{1.1} the sentient beings who die as ghosts and are reborn as humans are few, while those who die as ghosts and are reborn in hell, or the animal realm, or the ghost realm are many.” 

\subsection*{375–377 }

“…\marginnote{1.1} the sentient beings who die as ghosts and are reborn as gods are few, while those who die as ghosts and are reborn in hell, or the animal realm, or the ghost realm are many.” 

%
\addtocontents{toc}{\let\protect\contentsline\protect\nopagecontentsline}
\chapter*{The Chapter on Inspiring Qualities }
\addcontentsline{toc}{chapter}{\tocchapterline{The Chapter on Inspiring Qualities }}
\addtocontents{toc}{\let\protect\contentsline\protect\oldcontentsline}

%
\section*{{\suttatitleacronym AN 1.378–393}{\suttatitleroot Pasādakaradhammavagga}}
\addcontentsline{toc}{section}{\tocacronym{AN 1.378–393} \tocroot{Pasādakaradhammavagga}}
\markboth{28. Inspirational }{Pasādakaradhammavagga}
\extramarks{AN 1.378–393}{AN 1.378–393}

“Mendicants,\marginnote{1.1} this is definitely something worth having, that is, living in the wilderness … 

eating\marginnote{1.1} only almsfood … 

wearing\marginnote{1.1} rag robes … 

having\marginnote{1.1} just three robes … 

teaching\marginnote{1.1} Dhamma … 

memorizing\marginnote{1.1} the monastic law … 

being\marginnote{1.1} very learned … 

being\marginnote{1.1} respected … 

being\marginnote{1.1} well-presented … 

having\marginnote{1.1} a following … 

having\marginnote{1.1} a large following … 

coming\marginnote{1.1} from a good family … 

being\marginnote{1.1} handsome … 

being\marginnote{1.1} a good speaker … 

having\marginnote{1.1} few wishes … 

having\marginnote{1.1} good health.” 

%
\addtocontents{toc}{\let\protect\contentsline\protect\nopagecontentsline}
\chapter*{Another Chapter on a Finger Snap }
\addcontentsline{toc}{chapter}{\tocchapterline{Another Chapter on a Finger Snap }}
\addtocontents{toc}{\let\protect\contentsline\protect\oldcontentsline}

%
\section*{{\suttatitleacronym AN 1.394–574}{\suttatitleroot Aparaaccharāsaṅghātavagga}}
\addcontentsline{toc}{section}{\tocacronym{AN 1.394–574} \tocroot{Aparaaccharāsaṅghātavagga}}
\markboth{29. Another Chapter on a Finger-Snap }{Aparaaccharāsaṅghātavagga}
\extramarks{AN 1.394–574}{AN 1.394–574}

\subsection*{394 }

“If,\marginnote{1.1} mendicants, a mendicant develops the first absorption, even as long as a finger-snap, they are called a mendicant who does not lack absorption, who follows the Teacher’s instructions, who responds to advice, and who does not eat the country’s alms in vain. How much more so those who make much of it!” 

\subsection*{395–401 }

“If,\marginnote{1.1} mendicants, a mendicant develops the second … third … or fourth absorption … or the heart’s release by love … or the heart’s release by compassion … or the heart’s release by rejoicing … or the heart’s release by equanimity, even as long as a finger-snap … 

\subsection*{402–405 }

If\marginnote{1.1} a mendicant meditates by observing an aspect of the body … feelings … mind … principles—keen, aware, and mindful, rid of desire and aversion for the world, even for the time of a finger-snap … 

\subsection*{406–409 }

If\marginnote{1.1} they generate enthusiasm, try, make an effort, exert the mind, and strive so that bad, unskillful qualities don’t arise, even for the time of a finger-snap … If they generate enthusiasm, try, make an effort, exert the mind, and strive so that bad, unskillful qualities that have arisen are given up, even for the time of a finger-snap … If they generate enthusiasm, try, make an effort, exert the mind, and strive so that skillful qualities that have not arisen do arise, even for the time of a finger-snap … If they generate enthusiasm, try, make an effort, exert the mind, and strive so that skillful qualities that have arisen remain, are not lost, but increase, mature, and are fulfilled by development, even for the time of a finger-snap … 

\subsection*{410–413 }

If\marginnote{1.1} they develop the basis of psychic power that has immersion due to enthusiasm, and active effort … the basis of psychic power that has immersion due to energy, and active effort … the basis of psychic power that has immersion due to mental development, and active effort … the basis of psychic power that has immersion due to inquiry, and active effort, even for the time of a finger-snap … 

\subsection*{414–418 }

If\marginnote{1.1} they develop the faculty of faith … the faculty of energy … the faculty of mindfulness … the faculty of immersion … the faculty of wisdom, even for the time of a finger-snap … 

\subsection*{419–423 }

If\marginnote{1.1} they develop the power of faith … the power of energy … the power of mindfulness … the power of immersion … the power of wisdom, even for the time of a finger-snap … 

\subsection*{424–430 }

If\marginnote{1.1} they develop the awakening factor of mindfulness … the awakening factor of investigation of principles … the awakening factor of energy … the awakening factor of rapture … the awakening factor of tranquility … the awakening factor of immersion … the awakening factor of equanimity, even for the time of a finger-snap … 

\subsection*{431–438 }

If\marginnote{1.1} they develop right view … right thought … right speech … right action … right livelihood … right effort … right mindfulness … right immersion, even for the time of a finger-snap … 

\subsection*{439–446 }

Perceiving\marginnote{1.1} form internally, they see visions externally, limited, both pretty and ugly. Having mastered this, they are aware that: ‘I know and see.’ … Perceiving form internally, they see visions externally, limitless, both pretty and ugly. Having mastered this, they are aware that: ‘I know and see.’ … Not perceiving form internally, they see visions externally, limited, both pretty and ugly. Having mastered this, they are aware that: ‘I know and see.’ … Not perceiving form internally, they see visions externally, limitless, both pretty and ugly. Having mastered this, they are aware that: ‘I know and see.’ … Not perceiving form internally, they see visions externally that are blue, with blue color, blue hue, and blue tint. Having mastered this, they are aware that: ‘I know and see.’ … Not perceiving form internally, they see visions externally that are yellow, with yellow color, yellow hue, and yellow tint. Having mastered this, they are aware that: ‘I know and see.’ … Not perceiving form internally, they see visions externally that are red, with red color, red hue, and red tint. Having mastered this, they are aware that: ‘I know and see.’ … Not perceiving form internally, they see visions externally that are white, with white color, white hue, and white tint. Having mastered this, they are aware that: ‘I know and see.’ … 

\subsection*{447–454 }

Having\marginnote{1.1} physical form, they see visions … not perceiving form internally, they see visions externally … they’re focused only on beauty … going totally beyond perceptions of form, with the ending of perceptions of impingement, not focusing on perceptions of diversity, aware that ‘space is infinite’, they enter and remain in the dimension of infinite space … going totally beyond the dimension of infinite space, aware that ‘consciousness is infinite’, they enter and remain in the dimension of infinite consciousness … going totally beyond the dimension of infinite consciousness, aware that ‘there is nothing at all’, they enter and remain in the dimension of nothingness … going totally beyond the dimension of nothingness, they enter and remain in the dimension of neither perception nor non-perception … 

\subsection*{455–464 }

They\marginnote{1.1} develop the meditation on universal earth … the meditation on universal water … the meditation on universal fire … the meditation on universal air … the meditation on universal blue … the meditation on universal yellow … the meditation on universal red … the meditation on universal white … the meditation on universal space … the meditation on universal consciousness … 

\subsection*{465–474 }

They\marginnote{1.1} develop the perception of ugliness … the perception of death … the perception of the repulsiveness of food … the perception of dissatisfaction with the whole world … the perception of impermanence … the perception of suffering in impermanence … the perception of not-self in suffering … the perception of giving up … the perception of fading away … the perception of cessation … 

\subsection*{475–484 }

They\marginnote{1.1} develop the perception of impermanence … the perception of not-self … the perception of death … the perception of the repulsiveness of food … the perception of dissatisfaction with the whole world … the perception of a skeleton … the perception of the worm-infested corpse … the perception of the livid corpse … the perception of the split open corpse … the perception of the bloated corpse … 

\subsection*{485–494 }

They\marginnote{1.1} develop the recollection of the Buddha … the recollection of the teaching … the recollection of the \textsanskrit{Saṅgha} … the recollection of ethical conduct … the recollection of generosity … the recollection of the deities … mindfulness of breathing … the recollection of death … mindfulness of the body … the recollection of peace … 

\subsection*{495–574 }

They\marginnote{1.1} develop the faculty of faith together with the first absorption … the faculty of energy … the faculty of mindfulness … the faculty of immersion … the faculty of wisdom … the power of faith … the power of energy … the power of mindfulness … the power of immersion … the power of wisdom together with the first absorption … 

Together\marginnote{1.1} with the second absorption … 

the\marginnote{1.1} third absorption … 

the\marginnote{1.1} fourth absorption … 

love\marginnote{1.1} … 

compassion\marginnote{1.1} … 

rejoicing\marginnote{1.1} … 

They\marginnote{1.1} develop the faculty of faith together with equanimity … They develop the faculty of energy … the faculty of mindfulness … the faculty of immersion … the faculty of wisdom … the power of faith … the power of energy … the power of mindfulness … the power of immersion … the power of wisdom. 

That\marginnote{2.1} mendicant is called a mendicant who does not lack absorption, who follows the Teacher’s instructions, who responds to advice, and who does not eat the country’s alms in vain. How much more so those who make much of it!” 

%
\addtocontents{toc}{\let\protect\contentsline\protect\nopagecontentsline}
\chapter*{The Chapter on Mindfulness of the Body }
\addcontentsline{toc}{chapter}{\tocchapterline{The Chapter on Mindfulness of the Body }}
\addtocontents{toc}{\let\protect\contentsline\protect\oldcontentsline}

%
\section*{{\suttatitleacronym AN 1.575–615}{\suttatitleroot Kāyagatāsativagga}}
\addcontentsline{toc}{section}{\tocacronym{AN 1.575–615} \tocroot{Kāyagatāsativagga}}
\markboth{30. Mindfulness of the Body }{Kāyagatāsativagga}
\extramarks{AN 1.575–615}{AN 1.575–615}

\subsection*{575 }

“Mendicants,\marginnote{1.1} anyone who brings into their mind the great ocean includes all of the streams that run into it. In the same way, anyone who has developed and cultivated mindfulness of the body includes all of the skillful qualities that play a part in realization.” 

\subsection*{576–582 }

“One\marginnote{1.1} thing, mendicants, when developed and cultivated leads to great urgency … great benefit … great sanctuary … mindfulness and awareness … gaining knowledge and vision … blissful meditation in the present life … the realization of the fruit of knowledge and freedom. What one thing? Mindfulness of the body. This one thing, when developed and cultivated, leads to great urgency … great benefit … great sanctuary … mindfulness and awareness … gaining knowledge and vision … a happy abiding in the present life … the realization of the fruit of knowledge and freedom.” 

\subsection*{583 }

“When\marginnote{1.1} one thing, mendicants, is developed and cultivated the body and mind become tranquil, thinking and considering settle down, and all of the qualities that play a part in realization are fully developed. What one thing? Mindfulness of the body. When this one thing is developed and cultivated, the body and mind become tranquil, thinking and considering settle down, and all of the qualities that play a part in realization are fully developed.” 

\subsection*{584 }

“When\marginnote{1.1} one thing, mendicants, is developed and cultivated, unskillful qualities do not arise, and, if they’ve already arisen, they are given up. What one thing? Mindfulness of the body. When this one thing is developed and cultivated, unskillful qualities do not arise, and, if they’ve already arisen, they are given up.” 

\subsection*{585 }

“When\marginnote{1.1} one thing, mendicants, is developed and cultivated, skillful qualities arise, and, once they’ve arisen, they increase and grow. What one thing? Mindfulness of the body. When this one thing is developed and cultivated, skillful qualities arise, and, once they’ve arisen, they increase and grow.” 

\subsection*{586–590 }

“When\marginnote{1.1} one thing, mendicants, is developed and cultivated, ignorance is given up … knowledge arises … the conceit ‘I am’ is given up … the underlying tendencies are uprooted … the fetters are given up. What one thing? Mindfulness of the body. When this one thing is developed and cultivated, ignorance is given up … knowledge arises … the conceit ‘I am’ is given up … the underlying tendencies are uprooted … the fetters are given up.” 

\subsection*{591–592 }

“One\marginnote{1.1} thing, mendicants, when developed and cultivated leads to demolition by wisdom … to extinguishment by not grasping. What one thing? Mindfulness of the body. This one thing, mendicants, when developed and cultivated leads to demolition by wisdom … to extinguishment by not grasping.” 

\subsection*{593–595 }

“When\marginnote{1.1} one thing is developed and cultivated there is the penetration of many elements … the penetration of diverse elements … the analysis of many elements. What one thing? Mindfulness of the body. When this one thing is developed and cultivated there is the penetration of many elements … the penetration of diverse elements … the analysis of many elements.” 

\subsection*{596–599 }

“One\marginnote{1.1} thing, mendicants, when developed and cultivated leads to the realization of the fruit of stream-entry … once-return … non-return … perfection. What one thing? Mindfulness of the body. This one thing, when developed and cultivated, leads to the realization of the fruit of stream-entry … once-return … non-return … perfection.” 

\subsection*{600–615 }

“One\marginnote{1.1} thing, mendicants, when developed and cultivated, leads to the getting of wisdom … the growth of wisdom … the increase of wisdom … to great wisdom … to widespread wisdom … to abundant wisdom … to deep wisdom … to extraordinary wisdom … to vast wisdom … to much wisdom … to fast wisdom … to light wisdom … to laughing wisdom … to swift wisdom … to sharp wisdom … to penetrating wisdom. What one thing? Mindfulness of the body. This one thing, when developed and cultivated, leads to the getting of wisdom … the growth of wisdom … the increase of wisdom … to great wisdom … to widespread wisdom … to abundant wisdom … to deep wisdom … to extraordinary wisdom … to vast wisdom … to much wisdom … to fast wisdom … to light wisdom … to laughing wisdom … to swift wisdom … to sharp wisdom … to penetrating wisdom.” 

%
\addtocontents{toc}{\let\protect\contentsline\protect\nopagecontentsline}
\chapter*{The Chapter on the Deathless }
\addcontentsline{toc}{chapter}{\tocchapterline{The Chapter on the Deathless }}
\addtocontents{toc}{\let\protect\contentsline\protect\oldcontentsline}

%
\section*{{\suttatitleacronym AN 1.616–627}{\suttatitleroot Amatavagga}}
\addcontentsline{toc}{section}{\tocacronym{AN 1.616–627} \tocroot{Amatavagga}}
\markboth{31. The Deathless }{Amatavagga}
\extramarks{AN 1.616–627}{AN 1.616–627}

\subsection*{616 }

“Mendicants,\marginnote{1.1} those who don’t enjoy mindfulness of the body don’t enjoy the deathless. Those who enjoy mindfulness of the body enjoy the deathless.” 

\subsection*{617 }

“Mendicants,\marginnote{1.1} those who haven’t enjoyed mindfulness of the body haven’t enjoyed the deathless. Those who have enjoyed mindfulness of the body have enjoyed the deathless.” 

\subsection*{618 }

“Mendicants,\marginnote{1.1} those who have lost mindfulness of the body have lost the deathless. Those who haven’t lost mindfulness of the body haven’t lost the deathless.” 

\subsection*{619 }

“Mendicants,\marginnote{1.1} those who have missed out on mindfulness of the body have missed out on the deathless. Those who have undertaken mindfulness of the body have not missed out on the deathless.” 

\subsection*{620 }

“Mendicants,\marginnote{1.1} those who have neglected mindfulness of the body have neglected the deathless. Those who have not neglected mindfulness of the body have not neglected the deathless.” 

\subsection*{621 }

“Mendicants,\marginnote{1.1} those who have forgotten mindfulness of the body have forgotten the deathless. Those who haven’t forgotten mindfulness of the body haven’t forgotten the deathless.” 

\subsection*{622 }

“Mendicants,\marginnote{1.1} those who haven’t cultivated mindfulness of the body haven’t cultivated the deathless. Those who have cultivated mindfulness of the body have cultivated the deathless.” 

\subsection*{623 }

“Mendicants,\marginnote{1.1} those who haven’t developed mindfulness of the body haven’t developed the deathless. Those who have developed mindfulness of the body have developed the deathless.” 

\subsection*{624 }

“Mendicants,\marginnote{1.1} those who haven’t practiced mindfulness of the body haven’t practiced the deathless. Those who have practiced mindfulness of the body have practiced the deathless.” 

\subsection*{625 }

“Mendicants,\marginnote{1.1} those who haven’t had insight into mindfulness of the body haven’t had insight into the deathless. Those who have had insight into mindfulness of the body have had insight into the deathless.” 

\subsection*{626 }

“Mendicants,\marginnote{1.1} those who haven’t completely understood mindfulness of the body haven’t completely understood the deathless. Those who have completely understood mindfulness of the body have completely understood the deathless.” 

\subsection*{627 }

“Mendicants,\marginnote{1.1} those who haven’t realized mindfulness of the body haven’t realized the deathless. Those who have realized mindfulness of the body have realized the deathless.” 

The\marginnote{1.3} thousand discourses of the Ones are completed. 

That\marginnote{2.1} is what the Buddha said. Satisfied, the mendicants were happy with what the Buddha said. 

\scendbook{The Book of the Ones is finished. }

%
\addtocontents{toc}{\let\protect\contentsline\protect\nopagecontentsline}
\part*{The Book of the Twos }
\addcontentsline{toc}{part}{The Book of the Twos }
\markboth{}{}
\addtocontents{toc}{\let\protect\contentsline\protect\oldcontentsline}

%
%
\addtocontents{toc}{\let\protect\contentsline\protect\nopagecontentsline}
\pannasa{The First Fifty }
\addcontentsline{toc}{pannasa}{The First Fifty }
\markboth{}{}
\addtocontents{toc}{\let\protect\contentsline\protect\oldcontentsline}

%
\addtocontents{toc}{\let\protect\contentsline\protect\nopagecontentsline}
\chapter*{The Chapter on Punishments }
\addcontentsline{toc}{chapter}{\tocchapterline{The Chapter on Punishments }}
\addtocontents{toc}{\let\protect\contentsline\protect\oldcontentsline}

%
\section*{{\suttatitleacronym AN 2.1–10}{\suttatitleroot Kammakaraṇavagga}}
\addcontentsline{toc}{section}{\tocacronym{AN 2.1–10} \tocroot{Kammakaraṇavagga}}
\markboth{1. Punishments }{Kammakaraṇavagga}
\extramarks{AN 2.1–10}{AN 2.1–10}

\subsection*{1. Faults }

\scevam{So\marginnote{1.1} I have heard. }At one time the Buddha was staying near \textsanskrit{Sāvatthī} in Jeta’s Grove, \textsanskrit{Anāthapiṇḍika}’s monastery. There the Buddha addressed the mendicants, “Mendicants!” 

“Venerable\marginnote{1.5} sir,” they replied. The Buddha said this: 

“There\marginnote{2.1} are, mendicants, these two faults. What two? The fault apparent in the present life, and the fault to do with lives to come. 

What\marginnote{2.4} is the fault apparent in the present life? It’s when someone sees that kings have arrested a bandit, a criminal, and subjected them to various punishments—whipping, caning, and clubbing; cutting off hands or feet, or both; cutting off ears or nose, or both; the ‘porridge pot’, the ‘shell-shave’, the ‘demon’s mouth’, the ‘garland of fire’, the ‘burning hand’, the ‘grass blades’, the ‘bark dress’, the ‘antelope’, the ‘meat hook’, the ‘coins’, the ‘caustic pickle’, the ‘twisting bar’, the ‘straw mat’; being splashed with hot oil, being fed to the dogs, being impaled alive, and being beheaded. 

It\marginnote{3.1} occurs to them: ‘If I were to commit the kinds of bad deeds for which the kings arrested that bandit, that criminal, the rulers would arrest me and subject me to the same punishments. Afraid of the fault apparent in the present life, they do not steal the belongings of others. This is called the fault apparent in the present life. 

What\marginnote{4.1} is the fault to do with lives to come? It’s when someone reflects: ‘Bad conduct of body, speech, or mind has a bad, painful result in the next life. If I conduct myself badly, then, when my body breaks up, after death, won’t I be reborn in a place of loss, a bad place, the underworld, hell?’ Afraid of the fault to do with lives to come, they give up bad conduct by way of body, speech, and mind, and develop good conduct by way of body, speech, and mind, keeping themselves pure. This is called the fault to do with lives to come. 

These\marginnote{4.8} are the two faults. 

So\marginnote{4.9} you should train like this: ‘We will fear the fault apparent in the present life, and we will fear the fault to do with lives to come. We will fear faults, seeing the danger in faults.’ That’s how you should train. If you fear faults, seeing the danger in faults, you can expect to be freed from all faults.” 

\subsection*{2. Endeavor }

“These\marginnote{1.1} two endeavors are challenging in the world. What two? The endeavor of laypeople staying in a home to provide robes, almsfood, lodgings, and medicines and supplies for the sick. And the endeavor of those gone forth from the lay life to homelessness to let go of all attachments. These are the two endeavors that are challenging in the world. 

The\marginnote{2.1} better of these two endeavors is the effort to let go of all attachments. 

So\marginnote{2.2} you should train like this: ‘We shall endeavor to let go of all attachments.’ That’s how you should train.” 

\subsection*{3. Mortifying }

“These\marginnote{1.1} two things, mendicants, are mortifying. What two? It’s when someone has done bad things and not done good things, by way of body, speech, and mind. Thinking, ‘I’ve done bad things by way of body, speech, and mind’, they’re mortified. Thinking, ‘I haven’t done good things by way of body, speech, and mind’, they’re mortified. These are the two things that are mortifying.” 

\subsection*{4. Not Mortifying }

“These\marginnote{1.1} two things, mendicants, are not mortifying. What two? It’s when someone has done good things and not done bad things, by way of body, speech, and mind. Thinking, ‘I’ve done good things by way of body, speech, and mind’, they’re not mortified. Thinking, ‘I haven’t done bad things by way of body, speech, and mind’, they’re not mortified. These are the two things that are not mortifying.” 

\subsection*{5. Learned for Myself }

“Mendicants,\marginnote{1.1} I have learned these two things for myself—to never be content with skillful qualities, and to never stop trying. 

I\marginnote{1.3} never stopped trying, thinking: ‘Gladly, let only skin, sinews, and bones remain! Let the flesh and blood waste away in my body! I will not stop trying until I have achieved what is possible by human strength, energy, and vigor.’ 

It\marginnote{1.5} was by diligence that I achieved awakening, and by diligence that I achieved the supreme sanctuary. 

If\marginnote{1.6} you too never stop trying, thinking: ‘Gladly, let only skin, sinews, and bones remain! Let the flesh and blood waste away in my body! I will not stop trying until I have achieved what is possible by human strength, energy, and vigor.’ You will soon realize the supreme culmination of the spiritual path in this very life. You will live having achieved with your own insight the goal for which gentlemen rightly go forth from the lay life to homelessness. 

So\marginnote{1.9} you should train like this: ‘We will never stop trying, thinking: “Gladly, let only skin, sinews, and bones remain! Let the flesh and blood waste away in my body! I will not stop trying until I have achieved what is possible by human strength, energy, and vigor.”’ That’s how you should train.” 

\subsection*{6. Fetters }

“There\marginnote{1.1} are, mendicants, these two things. What two? Seeing things that are prone to being fettered as gratifying, and seeing things that are prone to being fettered as boring. When you keep seeing things that are prone to being fettered as gratifying, you don’t give up greed, hate, and delusion. When these are not given up, you’re not freed from rebirth, old age, and death, from sorrow, lamentation, pain, sadness, and distress. You’re not freed from suffering, I say. 

When\marginnote{2.1} you keep seeing things that are prone to being fettered as boring, you give up greed, hate, and delusion. When these are given up, you’re freed from rebirth, old age, and death, from sorrow, lamentation, pain, sadness, and distress. You’re freed from suffering, I say. These are the two things.” 

\subsection*{7. Dark }

“These\marginnote{1.1} two things, mendicants, are dark. What two? Lack of conscience and prudence. These are the two things that are dark.” 

\subsection*{8. Bright }

“These\marginnote{1.1} two things, mendicants, are bright. What two? Conscience and prudence. These are the two things that are bright.” 

\subsection*{9. Conduct }

“These\marginnote{1.1} two bright things, mendicants, protect the world. What two? Conscience and prudence. If these two bright things did not protect the world, there would be no recognition of the status of mother, aunts, or wives and partners of teachers and respected people. The world would become promiscuous, like goats and sheep, chickens and pigs, and dogs and jackals. But because the two bright things protect the world, there is recognition of the status of mother, aunts, and wives and partners of teachers and respected people.” 

\subsection*{10. Entering the Rainy Season }

“There\marginnote{1.1} are, mendicants, these two entries to the rainy season. What two? Earlier and later. These are the two entries to the rainy season.” 

%
\addtocontents{toc}{\let\protect\contentsline\protect\nopagecontentsline}
\chapter*{The Chapter on Disciplinary Issues }
\addcontentsline{toc}{chapter}{\tocchapterline{The Chapter on Disciplinary Issues }}
\addtocontents{toc}{\let\protect\contentsline\protect\oldcontentsline}

%
\section*{{\suttatitleacronym AN 2.11–20}{\suttatitleroot Adhikaraṇavagga}}
\addcontentsline{toc}{section}{\tocacronym{AN 2.11–20} \tocroot{Adhikaraṇavagga}}
\markboth{2. Issues }{Adhikaraṇavagga}
\extramarks{AN 2.11–20}{AN 2.11–20}

\subsection*{11 }

“There\marginnote{1.1} are, mendicants, these two powers. What two? The power of reflection and the power of development. And what, mendicants, is the power of reflection? It’s when someone reflects: ‘Bad conduct of body, speech, or mind has a bad, painful result in both this life and the next.’ Reflecting like this, they give up bad conduct by way of body, speech, and mind, and develop good conduct by way of body, speech, and mind, keeping themselves pure. This is called the power of reflection. 

And\marginnote{2.1} what, mendicants, is the power of development? In this context, the power of development is the power of the trainees. For when you rely on the power of a trainee, you give up greed, hate, and delusion. Then you don’t do anything unskillful, or practice anything bad. This is called the power of development. These are the two powers.” 

\subsection*{12 }

“There\marginnote{1.1} are, mendicants, these two powers. What two? The power of reflection and the power of development. And what, mendicants, is the power of reflection? It’s when someone reflects: ‘Bad conduct of body, speech, or mind has a bad, painful result in both this life and the next.’ Reflecting like this, they give up bad conduct by way of body, speech, and mind, and develop good conduct by way of body, speech, and mind, keeping themselves pure. This is called the power of reflection. 

And\marginnote{2.1} what, mendicants, is the power of development? It’s when a mendicant develops the awakening factors of mindfulness, investigation of principles, energy, rapture, tranquility, immersion, and equanimity, which rely on seclusion, fading away, and cessation, and ripen as letting go. This is called the power of development. These are the two powers.” 

\subsection*{13 }

“There\marginnote{1.1} are, mendicants, these two powers. What two? The power of reflection and the power of development. And what, mendicants, is the power of reflection? It’s when someone reflects: ‘Bad conduct of body, speech, or mind has a bad, painful result in both this life and the next.’ Reflecting like this, they give up bad conduct by way of body, speech, and mind, and develop good conduct by way of body, speech, and mind, keeping themselves pure. This is called the power of reflection. 

And\marginnote{2.1} what, mendicants, is the power of development? It’s when a mendicant, quite secluded from sensual pleasures, secluded from unskillful qualities, enters and remains in the first absorption, which has the rapture and bliss born of seclusion, while placing the mind and keeping it connected. As the placing of the mind and keeping it connected are stilled, they enter and remain in the second absorption, which has the rapture and bliss born of immersion, with internal clarity and confidence, and unified mind, without placing the mind and keeping it connected. And with the fading away of rapture, they enter and remain in the third absorption, where they meditate with equanimity, mindful and aware, personally experiencing the bliss of which the noble ones declare, ‘Equanimous and mindful, one meditates in bliss.’ Giving up pleasure and pain, and ending former happiness and sadness, they enter and remain in the fourth absorption, without pleasure or pain, with pure equanimity and mindfulness. This is called the power of development. These are the two powers.” 

\subsection*{14 }

“There\marginnote{1.1} are, mendicants, these two ways of teaching the Dhamma. What two? In brief and in detail. These are two ways of teaching the Dhamma.” 

\subsection*{15 }

“Mendicants,\marginnote{1.1} in a disciplinary issue, if neither the offending mendicant nor the accusing mendicant carefully checks themselves, you can expect that issue will lead to lasting acrimony and enmity, and the mendicants won’t live comfortably. But in a disciplinary issue, if both the offending mendicant and the accusing mendicant carefully check themselves, you can expect that issue won’t lead to lasting acrimony and enmity, and the mendicants will live comfortably. 

And\marginnote{2.1} how, mendicants, does an offending mendicant carefully check themselves? An offending mendicant reflects: ‘I have committed a certain unskillful offense with the body. That mendicant saw me do this. If I hadn’t committed that offense, they wouldn’t have seen me. But since I did commit that offense, they did see me. When they saw me, they were upset, and they voiced their unhappiness to me. Then I also got upset, so I told others. So the mistake is mine alone, like someone who owes customs duty on their goods.’ That’s how, mendicants, an offending mendicant carefully checks themselves. 

And\marginnote{3.1} how, mendicants, does an accusing mendicant carefully check themselves? An accusing mendicant reflects: ‘This mendicant has committed a certain unskillful offense with the body. I saw them do that. If they hadn’t committed that offense, I wouldn’t have seen them. But since they did commit that offense, I did see them. When I saw them, I was upset, and I voiced my unhappiness to them. Then they also got upset, so they told others. So the mistake is mine alone, like someone who owes customs duty on their goods.’ That’s how, mendicants, an accusing mendicant carefully checks themselves. 

In\marginnote{4.1} a disciplinary issue, if neither the offending mendicant nor the accusing mendicant carefully checks themselves, you can expect that issue will lead to lasting acrimony and enmity, and the mendicants won’t live comfortably. But in a disciplinary issue, if both the offending mendicant and the accusing mendicant carefully checks themselves, you can expect that issue won’t lead to lasting acrimony and enmity, and the mendicants will live comfortably.” 

\subsection*{16 }

Then\marginnote{1.1} a certain brahmin went up to the Buddha and exchanged greetings with him. When the greetings and polite conversation were over, he sat down to one side and said to the Buddha: 

“What\marginnote{1.3} is the cause, Master Gotama, what is the reason why some sentient beings, when their body breaks up, after death, are reborn in a place of loss, a bad place, the underworld, hell?” 

“Unprincipled\marginnote{1.4} and immoral conduct is the reason why some sentient beings, when their body breaks up, after death, are reborn in a place of loss, a bad place, the underworld, hell.” 

“What\marginnote{2.1} is the cause, Master Gotama, what is the reason why some sentient beings, when their body breaks up, after death, are reborn in a good place, a heavenly realm?” 

“Principled\marginnote{2.2} and moral conduct is the reason why some sentient beings, when their body breaks up, after death, are reborn in a good place, a heavenly realm.” 

“Excellent,\marginnote{3.1} Master Gotama! Excellent! As if he was righting the overturned, or revealing the hidden, or pointing out the path to the lost, or lighting a lamp in the dark so people with good eyes can see what’s there, Master Gotama has made the teaching clear in many ways. I go for refuge to Master Gotama, to the teaching, and to the mendicant \textsanskrit{Saṅgha}. From this day forth, may Master Gotama remember me as a lay follower who has gone for refuge for life.” 

\subsection*{17 }

Then\marginnote{1.1} the brahmin \textsanskrit{Jāṇussoṇi} went up to the Buddha, and exchanged greetings with him. When the greetings and polite conversation were over, he sat down to one side and said to the Buddha: 

“What\marginnote{1.3} is the cause, Master Gotama, what is the reason why some sentient beings, when their body breaks up, after death, are reborn in a place of loss, a bad place, the underworld, hell?” 

“What\marginnote{1.4} they’ve done and what they’ve not done. That’s why some sentient beings, when their body breaks up, after death, are reborn in a place of loss, a bad place, the underworld, hell.” 

“But\marginnote{1.6} what is the cause, Master Gotama, what is the reason why some sentient beings, when their body breaks up, after death, are reborn in a good place, a heavenly realm?” 

“What\marginnote{1.7} they’ve done and what they’ve not done. That’s why some sentient beings, when their body breaks up, after death, are reborn in a good place, a heavenly realm.” 

“I\marginnote{1.9} don’t understand the meaning of what Master Gotama has said in brief, without explaining the details. Master Gotama, please teach me this matter in detail so I can understand the meaning.” 

“Well\marginnote{1.11} then, brahmin, listen and pay close attention, I will speak.” 

“Yes\marginnote{1.12} sir,” \textsanskrit{Jāṇussoṇi} replied. The Buddha said this: 

“Firstly,\marginnote{2.1} brahmin, someone has done bad things and not done good things by way of body, speech, and mind. So what they’ve done and what they’ve not done is why some sentient beings, when their body breaks up, after death, are reborn in a place of loss, a bad place, the underworld, hell. Furthermore, brahmin, someone has done good things and not done bad things by way of body, speech, and mind. So what they’ve done and what they’ve not done is why some sentient beings, when their body breaks up, after death, are reborn in a good place, a heavenly realm.” 

“Excellent,\marginnote{3.1} Master Gotama! … From this day forth, may Master Gotama remember me as a lay follower who has gone for refuge for life.” 

\subsection*{18 }

Then\marginnote{1.1} Venerable Ānanda went up to the Buddha, bowed, and sat down to one side. The Buddha said to him, “Ānanda, I absolutely say that you should not do bad things by way of body, speech, and mind.” 

“But,\marginnote{1.3} sir, if someone does these things that should not be done, what drawbacks should they expect?” 

“They\marginnote{1.4} should expect these drawbacks. They blame themselves. After examination, sensible people criticize them. They get a bad reputation. They feel lost when they die. And when their body breaks up, after death, they are reborn in a place of loss, a bad place, the underworld, hell. These are the drawbacks they should expect.” 

“Ānanda,\marginnote{2.1} I absolutely say that you should do good things by way of body, speech, and mind.” 

“But,\marginnote{2.2} sir, if someone does these things that should be done, what benefits should they expect?” 

“They\marginnote{2.3} should expect these benefits. They don’t blame themselves. After examination, sensible people praise them. They get a good reputation. They don’t feel lost when they die. And when the body breaks up, after death, they are reborn in a good place, a heavenly realm. These are the benefits they should expect.” 

\subsection*{19 }

“Mendicants,\marginnote{1.1} give up the unskillful. It is possible to give up the unskillful. If it wasn’t possible, I wouldn’t say: ‘Give up the unskillful.’ But it is possible, and so I say: ‘Give up the unskillful.’ And if giving up the unskillful led to harm and suffering, I would not say: ‘Give up the unskillful.’ But giving up the unskillful leads to welfare and happiness, so I say: ‘Give up the unskillful.’ 

Mendicants,\marginnote{2.1} develop the skillful. It is possible to develop the skillful. If it wasn’t possible, I wouldn’t say: ‘Develop the skillful.’ But it is possible, and so I say: ‘Develop the skillful.’ If developing the skillful led to harm and suffering I wouldn’t say: ‘Develop the skillful.’ But developing the skillful leads to welfare and happiness, so I say: ‘Develop the skillful.’” 

\subsection*{20 }

“These\marginnote{1.1} two things, mendicants, lead to the decline and disappearance of the true teaching. What two? The words and phrases are misplaced, and the meaning is misinterpreted. When the words and phrases are misplaced, the meaning is misinterpreted. These two things lead to the decline and disappearance of the true teaching. 

These\marginnote{2.1} two things lead to the continuation, persistence, and enduring of the true teaching. What two? The words and phrases are well organized, and the meaning is correctly interpreted. When the words and phrases are well organized, the meaning is correctly interpreted. These two things lead to the continuation, persistence, and enduring of the true teaching.” 

%
\addtocontents{toc}{\let\protect\contentsline\protect\nopagecontentsline}
\chapter*{The Chapter on Fools }
\addcontentsline{toc}{chapter}{\tocchapterline{The Chapter on Fools }}
\addtocontents{toc}{\let\protect\contentsline\protect\oldcontentsline}

%
\section*{{\suttatitleacronym AN 2.21–31}{\suttatitleroot Bālavagga}}
\addcontentsline{toc}{section}{\tocacronym{AN 2.21–31} \tocroot{Bālavagga}}
\markboth{3. Fools }{Bālavagga}
\extramarks{AN 2.21–31}{AN 2.21–31}

\subsection*{21 }

“Mendicants,\marginnote{1.1} there are two fools. What two? One who doesn’t recognize when they’ve made a mistake. And one who doesn’t properly accept the confession of someone who’s made a mistake. These are the two fools. 

There\marginnote{2.1} are two who are astute. What two? One who recognizes when they’ve made a mistake. And one who properly accepts the confession of someone who’s made a mistake. These are the two who are astute.” 

\subsection*{22 }

“Mendicants,\marginnote{1.1} these two misrepresent the Realized One. What two? One who is hateful and hides it, and one whose faith is mistaken. These two misrepresent the Realized One.” 

\subsection*{23 }

“Mendicants,\marginnote{1.1} these two misrepresent the Realized One. What two? One who explains what was not spoken by the Realized One as spoken by him. And one who explains what was spoken by the Realized One as not spoken by him. These two misrepresent the Realized One. 

These\marginnote{2.1} two don’t misrepresent the Realized One. What two? One who explains what was not spoken by the Realized One as not spoken by him. And one who explains what was spoken by the Realized One as spoken by him. These two don’t misrepresent the Realized One.” 

\subsection*{24 }

“Mendicants,\marginnote{1.1} these two misrepresent the Realized One. What two? One who explains a discourse in need of interpretation as a discourse whose meaning is explicit. And one who explains a discourse whose meaning is explicit as a discourse in need of interpretation. These two misrepresent the Realized One.” 

\subsection*{25 }

“These\marginnote{1.1} two don’t misrepresent the Realized One. What two? One who explains a discourse in need of interpretation as a discourse in need of interpretation. And one who explains a discourse whose meaning is explicit as a discourse whose meaning is explicit. These two don’t misrepresent the Realized One.” 

\subsection*{26 }

“Mendicants,\marginnote{1.1} when you hide your misdeeds, you can expect one of two destinies: hell or the animal realm. 

When\marginnote{2.1} you don’t hide your misdeeds, you can expect one of two destinies: as a god or a human.” 

\subsection*{27 }

“Mendicants,\marginnote{1.1} when you have wrong view, you can expect one of two destinies: hell or the animal realm.” 

\subsection*{28 }

“Mendicants,\marginnote{1.1} when you have right view, you can expect one of two destinies: as a god or a human.” 

\subsection*{29 }

“There\marginnote{1.1} are two places waiting to receive an unethical person: hell and the animal realm. 

There\marginnote{2.1} are two places waiting to receive an ethical person: the realms of gods and humans.” 

\subsection*{30 }

“Mendicants,\marginnote{1.1} I see two reasons to frequent remote lodgings in the wilderness and the forest. What two? Seeing a happy life for oneself in the present, and having compassion for future generations. I see two reasons to frequent remote lodgings in the wilderness and the forest.” 

\subsection*{31 }

“These\marginnote{1.1} two things play a part in realization. What two? Serenity and discernment. 

What\marginnote{1.4} is the benefit of developing serenity? The mind is developed. What is the benefit of developing the mind? Greed is given up. 

What\marginnote{1.8} is the benefit of developing discernment? Wisdom is developed. What is the benefit of developing wisdom? Ignorance is given up. 

The\marginnote{1.12} mind contaminated by greed is not free; and wisdom contaminated by ignorance does not grow. In this way, freedom of heart comes from the fading away of greed, while freedom by wisdom comes from the fading away of ignorance.” 

%
\addtocontents{toc}{\let\protect\contentsline\protect\nopagecontentsline}
\chapter*{The Chapter on the Peaceful Mind }
\addcontentsline{toc}{chapter}{\tocchapterline{The Chapter on the Peaceful Mind }}
\addtocontents{toc}{\let\protect\contentsline\protect\oldcontentsline}

%
\section*{{\suttatitleacronym AN 2.32–41}{\suttatitleroot Samacittavagga}}
\addcontentsline{toc}{section}{\tocacronym{AN 2.32–41} \tocroot{Samacittavagga}}
\markboth{4. Peaceful Mind }{Samacittavagga}
\extramarks{AN 2.32–41}{AN 2.32–41}

\subsection*{32 }

“Mendicants,\marginnote{1.1} I will teach you the level of the bad person and the level of the good person. Listen and pay close attention, I will speak.” 

“Yes,\marginnote{1.3} sir,” they replied. The Buddha said this: 

“What\marginnote{2.1} is the level of the bad person? The bad person is ungrateful and thankless, for the wicked only know how to be ungrateful and thankless. It is totally the level of a bad person to be ungrateful and thankless. The good person is grateful and thankful, for the virtuous only know how to be grateful and thankful. It is totally the level of a good person to be grateful and thankful.” 

\subsection*{33 }

“Mendicants,\marginnote{1.1} I say that these two people cannot easily be repaid. What two? Mother and father. 

You\marginnote{1.4} would not have done enough to repay your mother and father even if you were to carry your mother around on one shoulder and your father on the other, and if you lived like this for a hundred years, and if you were to anoint, massage, bathe, and rub them; and even if they were to defecate and urinate right there. 

Even\marginnote{1.6} if you were to establish your mother and father as supreme monarchs of this great earth, abounding in the seven treasures, you would still not have done enough to repay them. Why is that? Parents are very helpful to their children: they raise them, nurture them, and show them the world. 

But\marginnote{1.9} you have done enough, more than enough, to repay them if you encourage, settle, and ground unfaithful parents in faith, unethical parents in ethical conduct, stingy parents in generosity, or ignorant parents in wisdom.” 

\subsection*{34 }

Then\marginnote{1.1} a certain brahmin went up to the Buddha, and exchanged greetings with him. When the greetings and polite conversation were over, he sat down to one side and said to the Buddha, “What does Master Gotama teach? What does he explain?” 

“Brahmin,\marginnote{1.3} I teach action and inaction.” 

“But\marginnote{1.4} in what way does Master Gotama teach action and inaction?” 

“I\marginnote{2.1} teach inaction regarding bad bodily, verbal, and mental conduct, and the many kinds of unskillful things. I teach action regarding good bodily, verbal, and mental conduct, and the many kinds of skillful things. This is the kind of action and inaction that I teach.” 

“Excellent,\marginnote{3.1} Master Gotama! … From this day forth, may Master Gotama remember me as a lay follower who has gone for refuge for life.” 

\subsection*{35 }

Then\marginnote{1.1} the householder \textsanskrit{Anāthapiṇḍika} went up to the Buddha, bowed, sat down to one side, and said to the Buddha, “How many kinds of people in the world are worthy of a religious donation? And where should a gift be given?” 

“Householder,\marginnote{1.3} there are two kinds of people in the world who are worthy of a religious donation: the trainee and the master. These are two kinds of people in the world who are worthy of a religious donation, and that’s where you should give a gift.” 

That\marginnote{2.1} is what the Buddha said. Then the Holy One, the Teacher, went on to say: 

\begin{verse}%
“In\marginnote{3.1} this world, the trainee and the master, \\
are worthy of offerings dedicated to the gods. \\
They are upright in body, \\
speech, and mind. \\
This is the field for sponsors of sacrifice—\\
what’s given here is very fruitful.” 

%
\end{verse}

\subsection*{36 }

\scevam{So\marginnote{1.1} I have heard. }At one time the Buddha was staying near \textsanskrit{Sāvatthī} in Jeta’s Grove, \textsanskrit{Anāthapiṇḍika}’s monastery. 

Now\marginnote{1.3} at that time Venerable \textsanskrit{Sāriputta} was staying near \textsanskrit{Sāvatthī} in the Eastern Monastery, the stilt longhouse of \textsanskrit{Migāra}’s mother. There \textsanskrit{Sāriputta} addressed the mendicants: “Reverends, mendicants!” 

“Reverend,”\marginnote{1.6} they replied. \textsanskrit{Sāriputta} said this: 

“I\marginnote{1.8} will teach you about a person fettered internally and one fettered externally. Listen and pay close attention, I will speak.” 

“Yes,\marginnote{1.10} reverend,” they replied. \textsanskrit{Sāriputta} said this: 

“Who\marginnote{2.1} is a person fettered internally? It’s a mendicant who is ethical, restrained in the monastic code, conducting themselves well and seeking alms in suitable places. Seeing danger in the slightest fault, they keep the rules they’ve undertaken. When their body breaks up, after death, they’re reborn in one of the orders of gods. When they pass away from there, they’re a returner, who comes back to this state of existence. This is called a person who is fettered internally, a returner, who comes back to this state of existence. 

Who\marginnote{3.1} is a person fettered externally? It’s a mendicant who is ethical, restrained in the monastic code, conducting themselves well and seeking alms in suitable places. Seeing danger in the slightest fault, they keep the rules they’ve undertaken. They enter and remain in a certain peaceful state of freed mind. When their body breaks up, after death, they’re reborn in one of the orders of gods. When they pass away from there, they’re a non-returner, not coming back to this state of existence. This is called a person who is fettered externally, a non-returner, who does not come back to this state of existence. 

Furthermore,\marginnote{4.1} a mendicant is ethical … they keep the rules they’ve undertaken. They simply practice for disillusionment, dispassion, and cessation regarding sensual pleasures. They simply practice for disillusionment, dispassion, and cessation regarding future lives. They practice for the ending of craving. They practice for the ending of greed. When their body breaks up, after death, they are reborn in one of the orders of gods. When they pass away from there, they are non-returners, not coming back to this state of existence. This is called a person who is fettered externally, a non-returner, who does not come back to this state of existence.” 

Then\marginnote{5.1} several peaceful-minded deities went up to the Buddha, bowed, stood to one side, and said to the Buddha, “Sir, Venerable \textsanskrit{Sāriputta} is in the Eastern Monastery, the stilt longhouse of \textsanskrit{Migāra}’s mother, where he is teaching the mendicants about a person with internal fetters and one with external fetters. The assembly is overjoyed! Sir, please go to Venerable \textsanskrit{Sāriputta} out of compassion.” The Buddha consented in silence. 

Then\marginnote{5.6} the Buddha, as easily as a strong person would extend or contract their arm, vanished from Jeta’s Grove and reappeared in the Eastern Monastery, the stilt longhouse of \textsanskrit{Migāra}’s mother, in front of \textsanskrit{Sāriputta}. He sat on the seat spread out. \textsanskrit{Sāriputta} bowed to the Buddha and sat down to one side. The Buddha said to him: 

“Just\marginnote{6.1} now, \textsanskrit{Sāriputta}, several peaceful-minded deities came up to me, bowed, and stood to one side. Those deities said to me: ‘Sir, Venerable \textsanskrit{Sāriputta} is in the Eastern Monastery, the stilt longhouse of \textsanskrit{Migāra}’s mother, where he is teaching the mendicants about a person with internal fetters and one with external fetters. The assembly is overjoyed! Sir, please go to Venerable \textsanskrit{Sāriputta} out of compassion.’ 

Those\marginnote{6.5} deities, though they number ten, twenty, thirty, forty, fifty, or sixty, can stand on the point of a needle without bumping up against each other. 

\textsanskrit{Sāriputta},\marginnote{6.6} you might think: ‘Surely those deities, since so many of them can stand on the point of a needle without bumping up against each other, must have developed their minds in that place.’ But you should not see it like this. It was right here that those deities developed their minds. 

So\marginnote{6.10} you should train like this: ‘We shall have peaceful faculties and peaceful minds.’ That’s how you should train. When your faculties and mind are peaceful, your acts of body, speech, and mind will be peaceful, thinking: ‘We shall present the gift of peace to our spiritual companions.’ That’s how you should train. 

Those\marginnote{6.16} wanderers who follow other paths, \textsanskrit{Sāriputta}, who have not heard this exposition of the teaching are lost.” 

\subsection*{37 }

\scevam{So\marginnote{1.1} I have heard. }At one time Venerable \textsanskrit{Mahākaccāna} was staying at \textsanskrit{Varaṇā}, on the bank of the Kaddama Lake. 

Then\marginnote{1.3} the brahmin \textsanskrit{Ārāmadaṇḍa} went up to \textsanskrit{Mahākaccāna}, and exchanged greetings with him. When the greetings and polite conversation were over, he sat down to one side and said to \textsanskrit{Mahākaccāna}: 

“What\marginnote{1.5} is the cause, Master \textsanskrit{Kaccāna}, what is the reason why aristocrats fight with aristocrats, brahmins fight with brahmins, and householders fight with householders?” 

“It\marginnote{1.6} is because of their insistence on sensual desire, their shackles, avarice, and attachment, that aristocrats fight with aristocrats, brahmins fight with brahmins, and householders fight with householders.” 

“What\marginnote{2.1} is the cause, Master \textsanskrit{Kaccāna}, what is the reason why ascetics fight with ascetics?” 

“It\marginnote{2.2} is because of their insistence on views, their shackles, avarice, and attachment, that ascetics fight with ascetics.” 

“Master\marginnote{3.1} \textsanskrit{Kaccāna}, is there anyone in the world who has gone beyond the insistence on sensual desire and the insistence on views?” 

“There\marginnote{3.2} is, brahmin.” 

“Who\marginnote{4.1} in the world has gone beyond the insistence on sensual desire and the insistence on views?” 

“In\marginnote{4.2} the eastern lands there is a city called \textsanskrit{Sāvatthī}. There the Blessed One is now staying, the perfected one, the fully awakened Buddha. He, brahmin, has gone beyond the insistence on sensual desire and the insistence on views.” 

When\marginnote{5.1} this was said, the brahmin \textsanskrit{Ārāmadaṇḍa} got up from his seat, arranged his robe over one shoulder, knelt on his right knee, raised his joined palms toward the Buddha, and expressed this heartfelt sentiment three times: 

“Homage\marginnote{6.1} to that Blessed One, the perfected one, the fully awakened Buddha! 

Homage\marginnote{7.1} to that Blessed One, the perfected one, the fully awakened Buddha! 

Homage\marginnote{8.1} to that Blessed One, the perfected one, the fully awakened Buddha! 

He\marginnote{9.1} who has gone beyond the insistence on sensual desire and the insistence on views. 

Excellent,\marginnote{10.1} Master \textsanskrit{Kaccāna}! Excellent! As if he were righting the overturned, or revealing the hidden, or pointing out the path to the lost, or lighting a lamp in the dark so people with good eyes can see what’s there, Master \textsanskrit{Kaccāna} has made the teaching clear in many ways. I go for refuge to Master Gotama, to the teaching, and to the mendicant \textsanskrit{Saṅgha}. From this day forth, may Master \textsanskrit{Kaccāna} remember me as a lay follower who has gone for refuge for life.” 

\subsection*{38 }

At\marginnote{1.1} one time Venerable \textsanskrit{Mahākaccāna} was staying near \textsanskrit{Madhurā}, in Gunda’s Grove. 

Then\marginnote{1.2} the brahmin \textsanskrit{Kandarāyana} went up to \textsanskrit{Mahākaccāna}, and exchanged greetings with him … He sat down to one side and said to \textsanskrit{Mahākaccāna}: 

“I\marginnote{1.3} have heard, Master \textsanskrit{Kaccāna}, that the ascetic \textsanskrit{Kaccāna} doesn’t bow to old brahmins, the elderly and senior, who are advanced in years and have reached the final stage of life; nor does he rise in their presence or offer them a seat. And this is indeed the case, for the ascetic \textsanskrit{Kaccāna} does not bow to old brahmins, elderly and senior, who are advanced in years and have reached the final stage of life; nor does he rise in their presence or offer them a seat. This is not appropriate, Master \textsanskrit{Kaccāna}.” 

“There\marginnote{2.1} is the stage of an elder and the stage of youth as explained by the Blessed One, who knows and sees, the perfected one, the fully awakened Buddha. If an elder, though eighty, ninety, or a hundred years old, still dwells in the midst of sensual pleasures, enjoying them, consumed by thoughts of them, burning with fever for them, and eagerly seeking more, they are reckoned as a child, not a senior. If a youth, young, black-haired, blessed with youth, in the prime of life, does not dwell in the midst of sensual pleasures, enjoying them, consumed by thoughts of them, burning with fever for them, and eagerly seeking more, they are reckoned as astute, a senior.” 

When\marginnote{3.1} this was said, the brahmin \textsanskrit{Kandarāyana} got up from his seat, placed his robe over one shoulder, and bowed with his head at the feet of the young mendicants, saying, “The masters are elders, at the stage of the elder; we are youths, at the stage of youth. 

Excellent,\marginnote{4.1} Master \textsanskrit{Kaccāna}! … From this day forth, may Master \textsanskrit{Kaccāna} remember me as a lay follower who has gone for refuge for life.” 

\subsection*{39 }

“At\marginnote{1.1} a time when bandits are strong, kings are weak. Then the king is not at ease when going out or coming back or when touring the provinces. The brahmins and householders, likewise, are not at ease when going out or coming back, or when inspecting their business activities. In the same way, at a time when bad mendicants are strong, good-hearted mendicants are weak. Then the good-hearted mendicants continually adhere to silence in the midst of the \textsanskrit{Saṅgha}, or they stay in the borderlands. This is for the hurt and unhappiness of the people, for the harm, hurt, and suffering of many people, of gods and humans. 

At\marginnote{2.1} a time when kings are strong, bandits are weak. Then the king is at ease when going out or coming back or when inspecting the provinces. The brahmins and householders, likewise, are at ease when going out or coming back, or when inspecting their business activities. In the same way, at a time when good-hearted mendicants are strong, bad mendicants are weak. Then the bad mendicants continually adhere to silence in the midst of the \textsanskrit{Saṅgha}, or they leave for some place or other. This is for the welfare and happiness of the people, for the benefit, welfare, and happiness of gods and humans.” 

\subsection*{40 }

“Mendicants,\marginnote{1.1} I don’t praise wrong practice for these two, for laypeople and renunciates. Because of wrong practice, neither laypeople nor renunciates succeed in completing the procedure of the skillful teaching. 

I\marginnote{2.1} praise right practice for these two, for laypeople and renunciates. Because of right practice, both laypeople and renunciates succeed in completing the procedure of the skillful teaching.” 

\subsection*{41 }

“Mendicants,\marginnote{1.1} by memorizing the discourses incorrectly, taking only a semblance of the phrasing, some mendicants shut out the meaning and the teaching. They act for the hurt and unhappiness of the people, for the harm, hurt, and suffering of many people, of gods and humans. They make much bad karma and make the true teaching disappear. 

But\marginnote{2.1} by memorizing the discourses well, not taking only a semblance of the phrasing, some mendicants reinforce the meaning and the teaching. They act for the welfare and happiness of the people, for the benefit, welfare, and happiness of the people, of gods and humans. They make much merit and make the true teaching continue.” 

%
\addtocontents{toc}{\let\protect\contentsline\protect\nopagecontentsline}
\chapter*{The Chapter on Assemblies }
\addcontentsline{toc}{chapter}{\tocchapterline{The Chapter on Assemblies }}
\addtocontents{toc}{\let\protect\contentsline\protect\oldcontentsline}

%
\section*{{\suttatitleacronym AN 2.42–51}{\suttatitleroot Parisavagga}}
\addcontentsline{toc}{section}{\tocacronym{AN 2.42–51} \tocroot{Parisavagga}}
\markboth{5. Assemblies }{Parisavagga}
\extramarks{AN 2.42–51}{AN 2.42–51}

\subsection*{42 }

“There\marginnote{1.1} are, mendicants, these two assemblies. What two? A shallow assembly and a deep assembly. And what is a shallow assembly? An assembly where the mendicants are restless, insolent, fickle, scurrilous, loose-tongued, unmindful, lacking situational awareness and immersion, with straying minds and undisciplined faculties. This is called a shallow assembly. 

And\marginnote{2.1} what is a deep assembly? An assembly where the mendicants are not restless, insolent, fickle, scurrilous, or loose-tongued, but have established mindfulness, situational awareness, immersion, unified minds, and restrained faculties. This is called a deep assembly. These are the two assemblies. The better of these two assemblies is the deep assembly.” 

\subsection*{43 }

“There\marginnote{1.1} are, mendicants, these two assemblies. What two? A divided assembly and a harmonious assembly. And what is a divided assembly? An assembly where the mendicants argue, quarrel, and dispute, continually wounding each other with barbed words. This is called a divided assembly. 

And\marginnote{2.1} what is a harmonious assembly? An assembly where the mendicants live in harmony, appreciating each other, without quarreling, blending like milk and water, and regarding each other with kindly eyes. This is called a harmonious assembly. These are the two assemblies. The better of these two assemblies is the harmonious assembly.” 

\subsection*{44 }

“There\marginnote{1.1} are, mendicants, these two assemblies. What two? An assembly of the worst and an assembly of the best. And what is an assembly of the worst? An assembly where the senior mendicants are indulgent and slack, leaders in backsliding, neglecting seclusion, not rousing energy for attaining the unattained, achieving the unachieved, and realizing the unrealized. Those who come after follow their example. They too become indulgent and slack, leaders in backsliding, neglecting seclusion, not rousing energy for attaining the unattained, achieving the unachieved, and realizing the unrealized. This is called an assembly of the worst. 

And\marginnote{2.1} what is an assembly of the best? An assembly where the senior mendicants are not indulgent or slack, nor are they backsliders; instead, they take the lead in seclusion, rousing energy for attaining the unattained, achieving the unachieved, and realizing the unrealized. Those who come after follow their example. They too are not indulgent or slack, nor are they backsliders; instead, they take the lead in seclusion, rousing energy for attaining the unattained, achieving the unachieved, and realizing the unrealized. This is called an assembly of the best. These are the two assemblies. The better of these two assemblies is the assembly of the best.” 

\subsection*{45 }

“There\marginnote{1.1} are, mendicants, these two assemblies. What two? An ignoble assembly and a noble assembly. And what is an ignoble assembly? An assembly where the mendicants don’t truly understand: ‘This is suffering’ … ‘This is the origin of suffering’ … ‘This is the cessation of suffering’ … ‘This is the practice that leads to the cessation of suffering’. This is called an ignoble assembly. 

And\marginnote{2.1} what is a noble assembly? An assembly where the mendicants truly understand: ‘This is suffering’ … ‘This is the origin of suffering’ … ‘This is the cessation of suffering’ … ‘This is the practice that leads to the cessation of suffering’. This is called a noble assembly. These are the two assemblies. The better of these two assemblies is the noble assembly.” 

\subsection*{46 }

“There\marginnote{1.1} are, mendicants, these two assemblies. What two? An assembly of the dregs and an assembly of the cream. And what is an assembly of the dregs? An assembly where the mendicants make decisions prejudiced by favoritism, hostility, stupidity, and cowardice. This is called an assembly of the dregs. 

And\marginnote{2.1} what is an assembly of the cream? An assembly where the mendicants make decisions unprejudiced by favoritism, hostility, stupidity, and cowardice. This is called an assembly of the cream. These are the two assemblies. The better of these two assemblies is the assembly of the cream.” 

\subsection*{47 }

“There\marginnote{1.1} are, mendicants, these two assemblies. What two? An assembly educated in fancy talk, not in questioning, and an assembly educated in questioning, not in fancy talk. And what is an assembly educated in fancy talk, not in questioning? It is an assembly where, when discourses spoken by the Realized One—deep, profound, transcendent, dealing with emptiness—are being recited the mendicants do not want to listen. They don’t pay attention or apply their minds to understand them, nor do they think those teachings are worth learning and memorizing. But when discourses composed by poets—poetry, with fancy words and phrases, composed by outsiders or spoken by disciples—are being recited the mendicants do want to listen. They pay attention and apply their minds to understand them, and they think those teachings are worth learning and memorizing. But when they’ve learned those teachings they don’t question or examine each other, saying: ‘Why does it say this? What does that mean?’ So they don’t clarify what is unclear, or reveal what is obscure, or dispel doubt regarding the many doubtful matters. This is called an assembly educated in fancy talk, not in questioning. 

And\marginnote{2.1} what is an assembly educated in questioning, not in fancy talk? It is an assembly where, when discourses composed by poets—poetry, with fancy words and phrases, composed by outsiders or spoken by disciples—are being recited the mendicants do not want to listen. They don’t pay attention or apply their minds to understand them, nor do they think those teachings are worth learning and memorizing. But when discourses spoken by the Realized One—deep, profound, transcendent, dealing with emptiness—are being recited the mendicants do want to listen. They pay attention and apply their minds to understand them, and they think those teachings are worth learning and memorizing. And when they’ve learned those teachings they question and examine each other, saying: ‘Why does it say this? What does that mean?’ So they clarify what is unclear, reveal what is obscure, and dispel doubt regarding the many doubtful matters. This is called an assembly educated in questioning, not in fancy talk. These are the two assemblies. The better of these two assemblies is the assembly educated in questioning, not in fancy talk.” 

\subsection*{48 }

“There\marginnote{1.1} are, mendicants, these two assemblies. What two? An assembly that values material things, not the true teaching, and an assembly that values the true teaching, not material things. And what is an assembly that values material things, not the true teaching? It is an assembly where the mendicants praise each other in front of the white-clothed laypeople, saying: ‘The mendicant so-and-so is freed both ways; so-and-so is freed by wisdom; so-and-so is a personal witness; so-and-so is attained to view; so-and-so is freed by faith; so-and-so is a follower of the teachings; so-and-so is a follower by faith; so-and-so is ethical, of good character; so-and-so is unethical, of bad character.’ In this way they get material things. And when they get these things, they use them tied, infatuated, attached, blind to the drawbacks, and not understanding the escape. This is called an assembly that values material things, not the true teaching. 

And\marginnote{2.1} what is an assembly that values the true teaching, not material things? It is an assembly where the mendicants don’t praise each other in front of the white-clothed laypeople, saying: ‘The mendicant so-and-so is freed both ways; so-and-so is freed by wisdom; so-and-so is a personal witness; so-and-so is attained to view; so-and-so is freed by faith; so-and-so is a follower of the teachings; so-and-so is a follower by faith; so-and-so is ethical, of good character; so-and-so is unethical, of bad character.’ In this way they get material things. And when they get these things, they use them untied, uninfatuated, unattached, seeing the drawbacks, and understanding the escape. This is called an assembly that values the true teaching, not material things. These are the two assemblies. The better of these two assemblies is the assembly that values the true teaching, not material things.” 

\subsection*{49 }

“There\marginnote{1.1} are, mendicants, these two assemblies. What two? An unjust assembly and a just assembly. And what is an unjust assembly? An assembly where legal acts against the teaching proceed, while legal acts in line with the teaching don’t proceed. Legal acts against the training proceed, while legal acts in line with the training don’t proceed. Legal acts against the teaching are explained, while legal acts in line with the teaching aren’t explained. Legal acts against the training are explained, while legal acts in line with the training aren’t explained. This is called an unjust assembly. 

And\marginnote{2.1} what is a just assembly? An assembly where legal acts in line with the teaching proceed, while legal acts against the teaching don’t proceed. Legal acts in line with the training proceed, while legal acts against the training don’t proceed. Legal acts in line with the teaching are explained, while legal acts against the teaching aren’t explained. Legal acts in line with the training are explained, while legal acts against the training aren’t explained. This is called a just assembly. These are the two assemblies. The better of these two assemblies is the just assembly.” 

\subsection*{50 }

“There\marginnote{1.1} are, mendicants, these two assemblies. What two? An unprincipled assembly and a principled assembly … The better of these two assemblies is the principled assembly.” 

\subsection*{51 }

“There\marginnote{1.1} are, mendicants, these two assemblies. What two? An assembly with unprincipled speech, and an assembly with principled speech. And what is an assembly with unprincipled speech? It is an assembly where the mendicants take up disciplinary issues, whether legitimate or not. But they don’t persuade each other or allow themselves to be persuaded, nor do they convince each other or allow themselves to be convinced. Unable to persuade or convince each other, they can’t let go of their opinions. They obstinately stick to that disciplinary issue, insisting that: ‘This is the only truth, other ideas are silly.’ This is called an assembly with unprincipled speech. 

And\marginnote{2.1} what is an assembly with principled speech? It is an assembly where the mendicants take up disciplinary issues, whether legitimate or not. Then they persuade each other or allow themselves to be persuaded, and they convince each other or allow themselves to be convinced. Since they are able to persuade and convince each other, they let go of their opinions. They don’t obstinately stick to that disciplinary issue or insist that: ‘This is the only truth, other ideas are silly.’ This is called an assembly with principled speech. These are the two assemblies. The better of these two assemblies is the assembly with principled speech.” 

%
\addtocontents{toc}{\let\protect\contentsline\protect\nopagecontentsline}
\pannasa{The Second Fifty }
\addcontentsline{toc}{pannasa}{The Second Fifty }
\markboth{}{}
\addtocontents{toc}{\let\protect\contentsline\protect\oldcontentsline}

%
\addtocontents{toc}{\let\protect\contentsline\protect\nopagecontentsline}
\chapter*{The Chapter on Persons }
\addcontentsline{toc}{chapter}{\tocchapterline{The Chapter on Persons }}
\addtocontents{toc}{\let\protect\contentsline\protect\oldcontentsline}

%
\section*{{\suttatitleacronym AN 2.52–63}{\suttatitleroot Puggalavagga}}
\addcontentsline{toc}{section}{\tocacronym{AN 2.52–63} \tocroot{Puggalavagga}}
\markboth{6. Persons }{Puggalavagga}
\extramarks{AN 2.52–63}{AN 2.52–63}

\subsection*{52 }

“Two\marginnote{1.1} people, mendicants, arise in the world for the welfare and happiness of the people, for the benefit, welfare, and happiness of gods and humans. What two? The Realized One, the perfected one, the fully awakened Buddha; and the wheel-turning monarch. These two people arise in the world for the welfare and happiness of the people, for the benefit, welfare, and happiness of gods and humans.” 

\subsection*{53 }

“Two\marginnote{1.1} people, mendicants, arise in the world who are incredible human beings. What two? The Realized One, the perfected one, the fully awakened Buddha; and the wheel-turning monarch. These are the two people who arise in the world who are incredible human beings.” 

\subsection*{54 }

“There\marginnote{1.1} are two people, mendicants, whose death is regretted by many people. What two? The Realized One, the perfected one, the fully awakened Buddha; and the wheel-turning monarch. These are the two people, mendicants, whose death is regretted by many people.” 

\subsection*{55 }

“Mendicants,\marginnote{1.1} these two are worthy of a monument. What two? The Realized One, the perfected one, the fully awakened Buddha; and the wheel-turning monarch. These are the two who are worthy of a monument.” 

\subsection*{56 }

“Mendicants,\marginnote{1.1} there are two kinds of Buddhas. What two? The Realized One, the perfected one, the fully awakened Buddha; and the Buddha awakened for themselves. These are the two kinds of Buddhas.” 

\subsection*{57 }

“These\marginnote{1.1} two, mendicants, are not startled by a crack of thunder. What two? A mendicant who has ended defilements; and a thoroughbred elephant. These are the two who are not startled by a crack of thunder.” 

\subsection*{58 }

“These\marginnote{1.1} two, mendicants, are not startled by a crack of thunder. What two? A mendicant who has ended defilements; and a thoroughbred horse. These are the two who are not startled by a crack of thunder.” 

\subsection*{59 }

“These\marginnote{1.1} two, mendicants, are not startled by a crack of thunder. What two? A mendicant who has ended defilements; and a lion, king of beasts. These are the two who are not startled by a crack of thunder.” 

\subsection*{60 }

“Fauns\marginnote{1.1} see two reasons not to use human speech. What two? Thinking: ‘May we not lie, and may we not misrepresent others with falsehoods.’ Fauns see two reasons not to use human speech.” 

\subsection*{61 }

“Mendicants,\marginnote{1.1} females die without getting enough of two things. What two? Sexual intercourse and giving birth. Females die without getting enough of these two things.” 

\subsection*{62 }

“Mendicants,\marginnote{1.1} I will teach you about living with bad people and living with good people. Listen and pay close attention, I will speak.” 

“Yes,\marginnote{1.3} sir,” they replied. The Buddha said this: 

“What\marginnote{2.1} is it like living with bad people? How do bad people live together? 

It’s\marginnote{2.2} when a senior mendicant thinks: ‘No mendicant, whether senior, middle, or junior, should admonish me; and I shouldn’t admonish any mendicant, whether senior, middle, or junior. If a mendicant—whether senior, middle, or junior—were to admonish me, they wouldn’t be sympathetic, and I’d bother them by saying “No!” And anyway I wouldn’t deal with it even if I saw what I did wrong.’ 

And\marginnote{2.6} a middle or a junior mendicant also thinks: ‘No mendicant, whether senior, middle, or junior, should admonish me; and I shouldn’t admonish any mendicant, whether senior, middle, or junior. If a mendicant—whether senior, middle, or junior—were to admonish me, they wouldn’t be sympathetic, and I’d bother them by saying “No!” And anyway I wouldn’t deal with it even if I saw what I did wrong.’ 

That’s\marginnote{2.12} what it’s like living with bad people; that’s how bad people live together. 

What\marginnote{3.1} is it like living with good people? How do good people live together? It’s when a senior mendicant thinks: 

‘Any\marginnote{3.3} mendicant, whether senior, middle, or junior, should admonish me; and I should admonish any mendicant, whether senior, middle, or junior. If a mendicant—whether senior, middle, or junior—were to admonish me, they’d be sympathetic, so I wouldn’t bother them, but say “Thank you!” And I’d deal with it when I saw what I did wrong.’ 

And\marginnote{3.7} a middle or a junior mendicant also thinks: ‘Any mendicant, whether senior, middle, or junior, may admonish me; and I’ll admonish any mendicant, whether senior, middle, or junior. If a mendicant—whether senior, middle, or junior—were to admonish me, they’d be sympathetic, so I wouldn’t bother them, but say “Thank you!” And I’d deal with it when I saw what I did wrong.’ 

That’s\marginnote{3.12} what it’s like living with good people; that’s how good people live together.” 

\subsection*{63 }

“In\marginnote{1.1} a disciplinary issue, when the tale-bearing on both sides—with contempt for each other’s views, resentful, bitter, and exasperated—is not settled internally, you can expect that this disciplinary issue will be long, fractious, and troublesome, and the mendicants won’t live comfortably. 

In\marginnote{2.1} a disciplinary issue, when the tale-bearing on both sides—with contempt for each other’s views, resentful, bitter, and exasperated—is well settled internally, you can expect that this disciplinary issue won’t lead to lasting acrimony and enmity, and the mendicants will live comfortably.” 

%
\addtocontents{toc}{\let\protect\contentsline\protect\nopagecontentsline}
\chapter*{The Chapter on Happiness }
\addcontentsline{toc}{chapter}{\tocchapterline{The Chapter on Happiness }}
\addtocontents{toc}{\let\protect\contentsline\protect\oldcontentsline}

%
\section*{{\suttatitleacronym AN 2.64–76}{\suttatitleroot Sukhavagga}}
\addcontentsline{toc}{section}{\tocacronym{AN 2.64–76} \tocroot{Sukhavagga}}
\markboth{7. Happiness }{Sukhavagga}
\extramarks{AN 2.64–76}{AN 2.64–76}

\subsection*{64 }

“There\marginnote{1.1} are, mendicants, these two kinds of happiness. What two? The happiness of laypeople, and the happiness of renunciates. These are the two kinds of happiness. The better of these two kinds of happiness is the happiness of renunciates.” 

\subsection*{65 }

“There\marginnote{1.1} are, mendicants, these two kinds of happiness. What two? Sensual happiness and the happiness of renunciation. These are the two kinds of happiness. The better of these two kinds of happiness is the happiness of renunciation.” 

\subsection*{66 }

“There\marginnote{1.1} are, mendicants, these two kinds of happiness. What two? The happiness of attachments, and the happiness of no attachments. These are the two kinds of happiness. The better of these two kinds of happiness is the happiness of no attachments.” 

\subsection*{67 }

“There\marginnote{1.1} are, mendicants, these two kinds of happiness. What two? Defiled happiness and undefiled happiness. These are the two kinds of happiness. The better of these two kinds of happiness is the happiness of no defilements.” 

\subsection*{68 }

“There\marginnote{1.1} are, mendicants, these two kinds of happiness. What two? Material happiness and spiritual happiness. These are the two kinds of happiness. The better of these two kinds of happiness is spiritual happiness.” 

\subsection*{69 }

“There\marginnote{1.1} are, mendicants, these two kinds of happiness. What two? Noble happiness and ignoble happiness. These are the two kinds of happiness. The better of these two kinds of happiness is noble happiness.” 

\subsection*{70 }

“There\marginnote{1.1} are, mendicants, these two kinds of happiness. What two? Physical happiness and mental happiness. These are the two kinds of happiness. The better of these two kinds of happiness is mental happiness.” 

\subsection*{71 }

“There\marginnote{1.1} are, mendicants, these two kinds of happiness. What two? Happiness with rapture and happiness free of rapture. These are the two kinds of happiness. The better of these two kinds of happiness is happiness free of rapture.” 

\subsection*{72 }

“There\marginnote{1.1} are, mendicants, these two kinds of happiness. What two? The happiness of pleasure and the happiness of equanimity. These are the two kinds of happiness. The better of these two kinds of happiness is the happiness of equanimity.” 

\subsection*{73 }

“There\marginnote{1.1} are, mendicants, these two kinds of happiness. What two? The happiness of immersion and the happiness without immersion. These are the two kinds of happiness. The better of these two kinds of happiness is the happiness of immersion.” 

\subsection*{74 }

“There\marginnote{1.1} are, mendicants, these two kinds of happiness. What two? Happiness that relies on rapture and happiness that relies on freedom from rapture. These are the two kinds of happiness. The better of these two kinds of happiness is happiness that relies on freedom from rapture.” 

\subsection*{75 }

“There\marginnote{1.1} are, mendicants, these two kinds of happiness. What two? Happiness that relies on pleasure and happiness that relies on equanimity. These are the two kinds of happiness. The better of these two kinds of happiness is happiness that relies on equanimity.” 

\subsection*{76 }

“There\marginnote{1.1} are, mendicants, these two kinds of happiness. What two? Happiness that relies on form and happiness that relies on the formless. These are the two kinds of happiness. The better of these two kinds of happiness is happiness that relies on the formless.” 

%
\addtocontents{toc}{\let\protect\contentsline\protect\nopagecontentsline}
\chapter*{The Chapter with a Foundation }
\addcontentsline{toc}{chapter}{\tocchapterline{The Chapter with a Foundation }}
\addtocontents{toc}{\let\protect\contentsline\protect\oldcontentsline}

%
\section*{{\suttatitleacronym AN 2.77–86}{\suttatitleroot Sanimittavagga}}
\addcontentsline{toc}{section}{\tocacronym{AN 2.77–86} \tocroot{Sanimittavagga}}
\markboth{8. With a Foundation }{Sanimittavagga}
\extramarks{AN 2.77–86}{AN 2.77–86}

\subsection*{77 }

“Bad,\marginnote{1.1} unskillful qualities, mendicants, arise with a foundation, not without a foundation. By giving up that foundation, those bad, unskillful qualities do not occur.” 

\subsection*{78 }

“Bad,\marginnote{1.1} unskillful qualities, mendicants, arise with a source, not without a source. By giving up that source, those bad, unskillful qualities do not occur.” 

\subsection*{79 }

“Bad,\marginnote{1.1} unskillful qualities, mendicants, arise with a cause, not without a cause. By giving up that cause, those bad, unskillful qualities do not occur.” 

\subsection*{80 }

“Bad,\marginnote{1.1} unskillful qualities, mendicants, arise with conditions, not without conditions. By giving up those conditions, those bad, unskillful qualities do not occur.” 

\subsection*{81 }

“Bad,\marginnote{1.1} unskillful qualities, mendicants, arise with a reason, not without a reason. By giving up that reason, those bad, unskillful qualities do not occur.” 

\subsection*{82 }

“Bad,\marginnote{1.1} unskillful qualities, mendicants, arise with form, not without form. By giving up that form, those bad, unskillful qualities do not occur.” 

\subsection*{83 }

“Bad,\marginnote{1.1} unskillful qualities, mendicants, arise with feeling, not without feeling. By giving up that feeling, those bad, unskillful qualities do not occur.” 

\subsection*{84 }

“Bad,\marginnote{1.1} unskillful qualities, mendicants, arise with perception, not without perception. By giving up that perception, those bad, unskillful qualities do not occur.” 

\subsection*{85 }

“Bad,\marginnote{1.1} unskillful qualities, mendicants, arise with consciousness, not without consciousness. By giving up that consciousness, those bad, unskillful qualities do not occur.” 

\subsection*{86 }

“Bad,\marginnote{1.1} unskillful qualities, mendicants, arise with a conditioned basis, not without a conditioned basis. By giving up that conditioned basis, those bad, unskillful qualities do not occur.” 

%
\addtocontents{toc}{\let\protect\contentsline\protect\nopagecontentsline}
\chapter*{The Chapter on Two Things }
\addcontentsline{toc}{chapter}{\tocchapterline{The Chapter on Two Things }}
\addtocontents{toc}{\let\protect\contentsline\protect\oldcontentsline}

%
\section*{{\suttatitleacronym AN 2.87–97}{\suttatitleroot Dhammavagga}}
\addcontentsline{toc}{section}{\tocacronym{AN 2.87–97} \tocroot{Dhammavagga}}
\markboth{9. Two Things }{Dhammavagga}
\extramarks{AN 2.87–97}{AN 2.87–97}

\subsection*{87 }

“There\marginnote{1.1} are, mendicants, these two things. What two? Freedom of heart and freedom by wisdom. These are the two things.” 

\subsection*{88 }

“There\marginnote{1.1} are, mendicants, these two things. What two? Exertion, and not being distracted. These are the two things.” 

\subsection*{89 }

“There\marginnote{1.1} are, mendicants, these two things. What two? Name and form. These are the two things.” 

\subsection*{90 }

“There\marginnote{1.1} are, mendicants, these two things. What two? Knowledge and freedom. These are the two things.” 

\subsection*{91 }

“There\marginnote{1.1} are, mendicants, these two things. What two? Views favoring continued existence and views favoring ending existence. These are the two things.” 

\subsection*{92 }

“There\marginnote{1.1} are, mendicants, these two things. What two? Lack of conscience and prudence. These are the two things.” 

\subsection*{93 }

“There\marginnote{1.1} are, mendicants, these two things. What two? Conscience and prudence. These are the two things.” 

\subsection*{94 }

“There\marginnote{1.1} are, mendicants, these two things. What two? Being hard to admonish and having bad friends. These are the two things.” 

\subsection*{95 }

“There\marginnote{1.1} are, mendicants, these two things. What two? Being easy to admonish and having good friends. These are the two things.” 

\subsection*{96 }

“There\marginnote{1.1} are, mendicants, these two things. What two? Skill in the elements and skill in attention. These are the two things.” 

\subsection*{97 }

“There\marginnote{1.1} are, mendicants, these two things. What two? Skill in offenses and skill in rehabilitation from offenses. These are the two things.” 

%
\addtocontents{toc}{\let\protect\contentsline\protect\nopagecontentsline}
\chapter*{The Chapter on Fools }
\addcontentsline{toc}{chapter}{\tocchapterline{The Chapter on Fools }}
\addtocontents{toc}{\let\protect\contentsline\protect\oldcontentsline}

%
\section*{{\suttatitleacronym AN 2.98–117}{\suttatitleroot Bālavagga}}
\addcontentsline{toc}{section}{\tocacronym{AN 2.98–117} \tocroot{Bālavagga}}
\markboth{10. Fools }{Bālavagga}
\extramarks{AN 2.98–117}{AN 2.98–117}

\subsection*{98 }

“Mendicants,\marginnote{1.1} there are two fools. What two? One who takes responsibility for what has not come to pass, and one who doesn’t take responsibility for what has come to pass. These are the two fools.” 

\subsection*{99 }

“There\marginnote{1.1} are two who are astute. What two? One who doesn’t take responsibility for what has not come to pass, and one who does take responsibility for what has come to pass. These are the two who are astute.” 

\subsection*{100 }

“Mendicants,\marginnote{1.1} there are two fools. What two? One who perceives what is unallowable as allowable, and one who perceives what is allowable as unallowable. These are the two fools.” 

\subsection*{101 }

“There\marginnote{1.1} are two who are astute. What two? One who perceives what is unallowable as unallowable, and one who perceives what is allowable as allowable. These are the two who are astute.” 

\subsection*{102 }

“Mendicants,\marginnote{1.1} there are two fools. What two? One who perceives a non-offense as an offense, and one who perceives an offense as a non-offense. These are the two fools.” 

\subsection*{103 }

“There\marginnote{1.1} are two who are astute. What two? One who perceives a non-offense as a non-offense, and one who perceives an offense as an offense. These are the two who are astute.” 

\subsection*{104 }

“Mendicants,\marginnote{1.1} there are two fools. What two? One who perceives what is not the teaching as the teaching, and one who perceives the teaching as not the teaching. These are the two fools.” 

\subsection*{105 }

“There\marginnote{1.1} are two who are astute. What two? One who perceives the teaching as the teaching, and one who perceives what is not the teaching as not the teaching. These are the two who are astute.” 

\subsection*{106 }

“Mendicants,\marginnote{1.1} there are two fools. What two? One who perceives what is not the training as the training, and one who perceives what is the training as not the training. These are the two fools.” 

\subsection*{107 }

“There\marginnote{1.1} are two who are astute. What two? One who perceives what is not the training as not the training, and one who perceives what is the training as the training. These are the two who are astute.” 

\subsection*{108 }

“For\marginnote{1.1} these two, defilements grow. What two? One who is remorseful over something they shouldn’t be, and one who isn’t remorseful over something they should be. These are the two whose defilements grow.” 

\subsection*{109 }

“For\marginnote{1.1} these two, defilements don’t grow. What two? One who isn’t remorseful over something they shouldn’t be, and one who is remorseful over something they should be. These are the two whose defilements don’t grow.” 

\subsection*{110 }

“For\marginnote{1.1} these two, defilements grow. What two? One who perceives what is unallowable as allowable, and one who perceives what is allowable as unallowable. These are the two whose defilements grow.” 

\subsection*{111 }

“For\marginnote{1.1} these two, defilements don’t grow. What two? One who perceives what is unallowable as unallowable, and one who perceives what is allowable as allowable. These are the two whose defilements don’t grow.” 

\subsection*{112 }

“For\marginnote{1.1} these two, defilements grow. What two? One who perceives an offense as a non-offense, and one who perceives a non-offense as an offense. These are the two whose defilements grow.” 

\subsection*{113 }

“For\marginnote{1.1} these two, defilements don’t grow. What two? One who perceives an offense as an offense, and one who perceives a non-offense as a non-offense. These are the two whose defilements don’t grow.” 

\subsection*{114 }

“For\marginnote{1.1} these two, defilements grow. What two? One who perceives what is not the teaching as the teaching, and one who perceives the teaching as not the teaching. These are the two whose defilements grow.” 

\subsection*{115 }

“For\marginnote{1.1} these two, defilements don’t grow. What two? One who perceives the teaching as the teaching, and one who perceives what is not the teaching as not the teaching. These are the two whose defilements don’t grow.” 

\subsection*{116 }

“For\marginnote{1.1} these two, defilements grow. What two? One who perceives what is not the training as the training, and one who perceives what is the training as not the training. These are the two whose defilements grow.” 

\subsection*{117 }

“For\marginnote{1.1} these two, defilements don’t grow. What two? One who perceives what is not the training as not the training, and one who perceives what is the training as the training. These are the two whose defilements don’t grow.” 

%
\addtocontents{toc}{\let\protect\contentsline\protect\nopagecontentsline}
\pannasa{The Third Fifty }
\addcontentsline{toc}{pannasa}{The Third Fifty }
\markboth{}{}
\addtocontents{toc}{\let\protect\contentsline\protect\oldcontentsline}

%
\addtocontents{toc}{\let\protect\contentsline\protect\nopagecontentsline}
\chapter*{The Chapter on Hopes That Are Hard to Give Up }
\addcontentsline{toc}{chapter}{\tocchapterline{The Chapter on Hopes That Are Hard to Give Up }}
\addtocontents{toc}{\let\protect\contentsline\protect\oldcontentsline}

%
\section*{{\suttatitleacronym AN 2.118–129}{\suttatitleroot Āsāduppajahavagga}}
\addcontentsline{toc}{section}{\tocacronym{AN 2.118–129} \tocroot{Āsāduppajahavagga}}
\markboth{11. Hopes That Are Hard to Give Up }{Āsāduppajahavagga}
\extramarks{AN 2.118–129}{AN 2.118–129}

\subsection*{118 }

“These\marginnote{1.1} two hopes are hard to give up. What two? The hope for wealth and the hope for long life. These are two hopes that are hard to give up.” 

\subsection*{119 }

“These\marginnote{1.1} two people are rare in the world. What two? One who takes the initiative, and one who is grateful and thankful. These are the two people who are rare in the world.” 

\subsection*{120 }

“These\marginnote{1.1} two people are rare in the world. What two? One who is satisfied, and one who satisfies others. These are the two people who are rare in the world.” 

\subsection*{121 }

“These\marginnote{1.1} two people are hard to satisfy in the world. What two? One who continually hoards wealth, and one who continually wastes wealth. These are the two people who are hard to satisfy in the world.” 

\subsection*{122 }

“These\marginnote{1.1} two people are easy to satisfy in the world. What two? One who does not continually hoard wealth, and one who does not continually waste wealth. These are the two people who are easy to satisfy in the world.” 

\subsection*{123 }

“There\marginnote{1.1} are two conditions for the arising of greed. What two? The feature of beauty and improper attention. These are the two conditions for the arising of greed.” 

\subsection*{124 }

“There\marginnote{1.1} are two conditions for the arising of hate. What two? The feature of harshness and improper attention. These are the two conditions for the arising of hate.” 

\subsection*{125 }

“There\marginnote{1.1} are two conditions for the arising of wrong view. What two? The words of another and improper attention. These are the two conditions for the arising of wrong view.” 

\subsection*{126 }

“There\marginnote{1.1} are two conditions for the arising of right view. What two? The words of another and proper attention. These are the two conditions for the arising of right view.” 

\subsection*{127 }

“There\marginnote{1.1} are these two offenses. What two? A light offense and a serious offense. These are the two offenses.” 

\subsection*{128 }

“There\marginnote{1.1} are these two offenses. What two? An offense with corrupt intention and an offense without corrupt intention. These are the two offenses.” 

\subsection*{129 }

“There\marginnote{1.1} are these two offenses. What two? An offense requiring rehabilitation and an offense not requiring rehabilitation. These are the two offenses.” 

%
\addtocontents{toc}{\let\protect\contentsline\protect\nopagecontentsline}
\chapter*{The Chapter on Aspiration }
\addcontentsline{toc}{chapter}{\tocchapterline{The Chapter on Aspiration }}
\addtocontents{toc}{\let\protect\contentsline\protect\oldcontentsline}

%
\section*{{\suttatitleacronym AN 2.130–140}{\suttatitleroot Āyācanavagga}}
\addcontentsline{toc}{section}{\tocacronym{AN 2.130–140} \tocroot{Āyācanavagga}}
\markboth{12. Aspiration }{Āyācanavagga}
\extramarks{AN 2.130–140}{AN 2.130–140}

\subsection*{130 }

“A\marginnote{1.1} faithful monk would rightly aspire: ‘May I be like \textsanskrit{Sāriputta} and \textsanskrit{Moggallāna}!’ These are a standard and a measure for my monk disciples, that is, \textsanskrit{Sāriputta} and \textsanskrit{Moggallāna}.” 

\subsection*{131 }

“A\marginnote{1.1} faithful nun would rightly aspire: ‘May I be like the nuns \textsanskrit{Khemā} and \textsanskrit{Uppalavaṇṇā}!’ These are a standard and a measure for my nun disciples, that is, the nuns \textsanskrit{Khemā} and \textsanskrit{Uppalavaṇṇā}.” 

\subsection*{132 }

“A\marginnote{1.1} faithful layman would rightly aspire: ‘May I be like the householder Citta and Hatthaka of \textsanskrit{Ãḷavī}!’ These are a standard and a measure for my male lay followers, that is, the householder Citta and Hatthaka of \textsanskrit{Ãḷavī}.” 

\subsection*{133 }

“A\marginnote{1.1} faithful laywoman would rightly aspire: ‘May I be like the laywomen \textsanskrit{Khujjuttarā} and \textsanskrit{Veḷukaṇṭakī}, Nanda’s mother!’ These are a standard and a measure for my female lay disciples, that is, the laywomen \textsanskrit{Khujjuttarā} and \textsanskrit{Veḷukaṇṭakī}, Nanda’s mother.” 

\subsection*{134 }

“When\marginnote{1.1} a foolish, incompetent bad person has two qualities they keep themselves broken and damaged. They deserve to be blamed and criticized by sensible people, and they make much bad karma. What two? Without examining or scrutinizing, they praise those deserving of criticism and they criticize those deserving of praise. When a foolish, incompetent bad person has these two qualities they keep themselves broken and damaged. They deserve to be blamed and criticized by sensible people, and they make much bad karma. 

When\marginnote{2.1} an astute, competent good person has two qualities they keep themselves healthy and whole. They don’t deserve to be blamed and criticized by sensible people, and they make much merit. What two? After examining and scrutinizing, they criticize those deserving of criticism and they praise those deserving of praise. When an astute, competent good person has these two qualities they keep themselves healthy and whole. They don’t deserve to be blamed and criticized by sensible people, and they make much merit.” 

\subsection*{135 }

“When\marginnote{1.1} a foolish, incompetent bad person has two qualities they keep themselves broken and damaged. They deserve to be blamed and criticized by sensible people, and they make much bad karma. What two? Without examining or scrutinizing, they arouse faith in things that are dubious, and they don’t arouse faith in things that are inspiring. When a foolish, incompetent bad person has these two qualities they keep themselves broken and damaged. They deserve to be blamed and criticized by sensible people, and they make much bad karma. 

When\marginnote{2.1} an astute, competent good person has two qualities they keep themselves healthy and whole. They don’t deserve to be blamed and criticized by sensible people, and they make much merit. What two? After examining or scrutinizing, they don’t arouse faith in things that are dubious, and they do arouse faith in things that are inspiring. When an astute, competent good person has these two qualities they keep themselves healthy and whole. They don’t deserve to be blamed and criticized by sensible people, and they make much merit.” 

\subsection*{136 }

“When\marginnote{1.1} a foolish, incompetent bad person acts wrongly toward two people they keep themselves broken and damaged. They deserve to be blamed and criticized by sensible people, and they make much bad karma. What two? Mother and father. When a foolish, incompetent bad person acts wrongly toward these two people they keep themselves broken and damaged. They deserve to be blamed and criticized by sensible people, and they make much bad karma. 

When\marginnote{2.1} an astute, competent good person acts rightly toward two people they keep themselves healthy and whole. They don’t deserve to be blamed and criticized by sensible people, and they make much merit. What two? Mother and father. When an astute, competent good person acts rightly toward these two people they keep themselves healthy and whole. They don’t deserve to be blamed and criticized by sensible people, and they make much merit.” 

\subsection*{137 }

“When\marginnote{1.1} a foolish, incompetent bad person acts wrongly toward two people they keep themselves broken and damaged. They deserve to be blamed and criticized by sensible people, and they make much bad karma. What two? The Realized One and a disciple of the Realized One. When a foolish, incompetent bad person acts wrongly toward these people they keep themselves broken and damaged. They deserve to be blamed and criticized by sensible people, and they make much bad karma. 

When\marginnote{2.1} an astute, competent good person acts rightly toward two people they keep themselves healthy and whole. They don’t deserve to be blamed and criticized by sensible people, and they make much merit. What two? The Realized One and a disciple of the Realized One. When an astute, competent good person acts rightly toward these two people they keep themselves healthy and whole. They don’t deserve to be blamed and criticized by sensible people, and they make much merit.” 

\subsection*{138 }

“There\marginnote{1.1} are these two things. What two? Cleaning your own mind, and not grasping at anything in the world. These are the two things.” 

\subsection*{139 }

“There\marginnote{1.1} are these two things. What two? Anger and hostility. These are the two things.” 

\subsection*{140 }

“There\marginnote{1.1} are these two things. What two? Dispelling anger and dispelling hostility. These are the two things.” 

%
\addtocontents{toc}{\let\protect\contentsline\protect\nopagecontentsline}
\chapter*{The Chapter on Giving }
\addcontentsline{toc}{chapter}{\tocchapterline{The Chapter on Giving }}
\addtocontents{toc}{\let\protect\contentsline\protect\oldcontentsline}

%
\section*{{\suttatitleacronym AN 2.141–150}{\suttatitleroot Dānavagga}}
\addcontentsline{toc}{section}{\tocacronym{AN 2.141–150} \tocroot{Dānavagga}}
\markboth{13. Giving }{Dānavagga}
\extramarks{AN 2.141–150}{AN 2.141–150}

\subsection*{141 }

“There\marginnote{1.1} are these two gifts. What two? A gift of material things and a gift of the teaching. These are the two gifts. The better of these two gifts is the gift of the teaching.” 

\subsection*{142 }

“There\marginnote{1.1} are these two offerings. What two? An offering of material things and an offering of the teaching. These are the two offerings. The better of these two offerings is an offering of the teaching.” 

\subsection*{143 }

“There\marginnote{1.1} are these two acts of generosity. What two? Generosity with material things and generosity with the teaching. These are the two acts of generosity. The better of these two acts of generosity is generosity with the teaching.” 

\subsection*{144 }

“There\marginnote{1.1} are these two kinds of charity. What two? Charity in material things and charity in the teaching. These are the two kinds of charity. The better of these two kinds of charity is a charity in the teaching.” 

\subsection*{145 }

“There\marginnote{1.1} are these two riches. What two? Riches in material things and riches in the teaching. These are the two riches. The better of these two riches is riches in the teaching.” 

\subsection*{146 }

“There\marginnote{1.1} are these two kinds of enjoyment. What two? Enjoyment of material things and enjoyment of the teaching. These are the two kinds of enjoyment. The better of these two kinds of enjoyment is the enjoyment of the teaching.” 

\subsection*{147 }

“There\marginnote{1.1} are these two kinds of sharing. What two? Sharing material things and sharing the teaching. These are the two kinds of sharing. The better of these two kinds of sharing is sharing the teaching.” 

\subsection*{148 }

“There\marginnote{1.1} are these two kinds of inclusion. What two? Inclusion in material things and inclusion in the teaching. These are the two kinds of inclusion. The better of these two kinds of inclusion is inclusion in the teaching.” 

\subsection*{149 }

“There\marginnote{1.1} are these two kinds of support. What two? Support in material things and support in the teaching. These are the two kinds of support. The better of these two kinds of support is support in the teaching.” 

\subsection*{150 }

“There\marginnote{1.1} are these two kinds of sympathy. What two? Sympathy in material things and sympathy in the teaching. These are the two kinds of sympathy. The better of these two kinds of sympathy is sympathy in the teaching.” 

%
\addtocontents{toc}{\let\protect\contentsline\protect\nopagecontentsline}
\chapter*{The Chapter on Welcome }
\addcontentsline{toc}{chapter}{\tocchapterline{The Chapter on Welcome }}
\addtocontents{toc}{\let\protect\contentsline\protect\oldcontentsline}

%
\section*{{\suttatitleacronym AN 2.151–162}{\suttatitleroot Santhāravagga}}
\addcontentsline{toc}{section}{\tocacronym{AN 2.151–162} \tocroot{Santhāravagga}}
\markboth{14. Welcome }{Santhāravagga}
\extramarks{AN 2.151–162}{AN 2.151–162}

\subsection*{151 }

“There\marginnote{1.1} are these two kinds of welcome. What two? Welcome in material things and welcome in the teaching. These are the two kinds of welcome. The better of these two kinds of welcome is the welcome in the teaching.” 

\subsection*{152 }

“There\marginnote{1.1} are these two kinds of hospitality. What two? Hospitality in material things and hospitality in the teaching. These are the two kinds of hospitality. The better of these two kinds of hospitality is hospitality in the teaching.” 

\subsection*{153 }

“There\marginnote{1.1} are these two quests. What two? The quest for material things and the quest for the teaching. These are the two quests. The better of these two quests is the quest for the teaching.” 

\subsection*{154 }

“There\marginnote{1.1} are these two searches. What two? The search for material things and the search for the teaching. These are the two searches. The better of these two searches is the search for the teaching.” 

\subsection*{155 }

“There\marginnote{1.1} are these two kinds of seeking. What two? Seeking for material things and seeking for the teaching. These are the two kinds of seeking. The better of these two kinds of seeking is seeking for the teaching.” 

\subsection*{156 }

“There\marginnote{1.1} are these two kinds of worship. What two? Worship of material things and worship of the teaching. These are the two kinds of worship. The better of these two kinds of worship is worship of the teaching.” 

\subsection*{157 }

“There\marginnote{1.1} are these two ways of serving guests. What two? Serving guests with material things and serving guests with the teaching. These are the two ways of serving guests. The better of these two ways of serving guests is to serve them with the teaching.” 

\subsection*{158 }

“There\marginnote{1.1} are, mendicants, these two successes. What two? Success in material things and success in the teaching. These are the two successes. The better of these two successes is success in the teaching.” 

\subsection*{159 }

“There\marginnote{1.1} are, mendicants, these two kinds of growth. What two? Growth in material things and growth in the teaching. These are the two kinds of growth. The better of these two kinds of growth is growth in the teaching.” 

\subsection*{160 }

“There\marginnote{1.1} are these two treasures. What two? The treasure of material things and the treasure of the teaching. These are the two treasures. The better of these two treasures is the treasure of the teaching.” 

\subsection*{161 }

“There\marginnote{1.1} are these two kinds of accumulation. What two? Accumulation of material things and accumulation of the teaching. These are the two kinds of accumulation. The better of these two kinds of accumulation is the accumulation of the teaching.” 

\subsection*{162 }

“There\marginnote{1.1} are, mendicants, these two kinds of increase. What two? Increase in material things and increase in the teaching. These are the two kinds of increase. The better of these two kinds of increase is increase in the teaching.” 

%
\addtocontents{toc}{\let\protect\contentsline\protect\nopagecontentsline}
\chapter*{The Chapter on Attainment }
\addcontentsline{toc}{chapter}{\tocchapterline{The Chapter on Attainment }}
\addtocontents{toc}{\let\protect\contentsline\protect\oldcontentsline}

%
\section*{{\suttatitleacronym AN 2.163–179}{\suttatitleroot Samāpattivagga}}
\addcontentsline{toc}{section}{\tocacronym{AN 2.163–179} \tocroot{Samāpattivagga}}
\markboth{15. Attainment }{Samāpattivagga}
\extramarks{AN 2.163–179}{AN 2.163–179}

\subsection*{163 }

“There\marginnote{1.1} are these two things. What two? Skill in meditative attainments and skill in emerging from those attainments. These are the two things.” 

\subsection*{164 }

“There\marginnote{1.1} are these two things. What two? Integrity and gentleness. These are the two things.” 

\subsection*{165 }

“There\marginnote{1.1} are these two things. What two? Patience and gentleness. These are the two things.” 

\subsection*{166 }

“There\marginnote{1.1} are these two things. What two? Friendliness and hospitality. These are the two things.” 

\subsection*{167 }

“There\marginnote{1.1} are these two things. What two? Harmlessness and purity. These are the two things.” 

\subsection*{168 }

“There\marginnote{1.1} are these two things. What two? Not guarding the sense doors and eating too much. These are the two things.” 

\subsection*{169 }

“There\marginnote{1.1} are these two things. What two? Guarding the sense doors and moderation in eating. These are the two things.” 

\subsection*{170 }

“There\marginnote{1.1} are these two things. What two? The power of reflection and the power of development. These are the two things.” 

\subsection*{171 }

“There\marginnote{1.1} are these two things. What two? The power of mindfulness and the power of immersion. These are the two things.” 

\subsection*{172 }

“There\marginnote{1.1} are these two things. What two? Serenity and discernment. These are the two things.” 

\subsection*{173 }

“There\marginnote{1.1} are these two things. What two? Failure in ethical conduct and failure in view. These are the two things.” 

\subsection*{174 }

“There\marginnote{1.1} are these two things. What two? Accomplishment in ethical conduct and accomplishment in view. These are the two things.” 

\subsection*{175 }

“There\marginnote{1.1} are these two things. What two? Purification of ethics and purification of view. These are the two things.” 

\subsection*{176 }

“There\marginnote{1.1} are these two things. What two? Purification of view and making an effort in line with that view. These are the two things.” 

\subsection*{177 }

“There\marginnote{1.1} are these two things. What two? To never be content with skillful qualities, and to never stop trying. These are the two things.” 

\subsection*{178 }

“There\marginnote{1.1} are these two things. What two? Lack of mindfulness and lack of situational awareness. These are the two things.” 

\subsection*{179 }

“There\marginnote{1.1} are these two things. What two? Mindfulness and situational awareness. These are the two things.” 

%
\addtocontents{toc}{\let\protect\contentsline\protect\nopagecontentsline}
\pannasa{Abbreviated Texts }
\addcontentsline{toc}{pannasa}{Abbreviated Texts }
\markboth{}{}
\addtocontents{toc}{\let\protect\contentsline\protect\oldcontentsline}

%
\addtocontents{toc}{\let\protect\contentsline\protect\nopagecontentsline}
\chapter*{The Chapter on Abbreviated Texts Beginning with Anger }
\addcontentsline{toc}{chapter}{\tocchapterline{The Chapter on Abbreviated Texts Beginning with Anger }}
\addtocontents{toc}{\let\protect\contentsline\protect\oldcontentsline}

%
\section*{{\suttatitleacronym AN 2.180–229}{\suttatitleroot Kodhapeyyālavagga}}
\addcontentsline{toc}{section}{\tocacronym{AN 2.180–229} \tocroot{Kodhapeyyālavagga}}
\markboth{16. Abbreviated Texts Beginning with Anger }{Kodhapeyyālavagga}
\extramarks{AN 2.180–229}{AN 2.180–229}

\subsection*{180–184 }

“There\marginnote{1.1} are these two things. What two? Anger and hostility … disdain and contempt … jealousy and stinginess … deceit and deviousness … lack of conscience and prudence. These are the two things.” 

\subsection*{185–189 }

“There\marginnote{1.1} are these two things. What two? Freedom from anger and hostility … freedom from disdain and contempt … freedom from jealousy and stinginess … freedom from deceit and deviousness … conscience and prudence. These are the two things.” 

\subsection*{190–194 }

“Anyone\marginnote{1.1} who has two things lives in suffering. What two? Anger and hostility … disdain and contempt … jealousy and stinginess … deceit and deviousness … lack of conscience and prudence. Anyone who has these two things lives in suffering.” 

\subsection*{195–199 }

“Anyone\marginnote{1.1} who has these two things lives happily. What two? Freedom from anger and hostility … freedom from disdain and contempt … freedom from jealousy and stinginess … freedom from deceit and deviousness … conscience and prudence. Anyone who has these two things lives happily.” 

\subsection*{200–204 }

“These\marginnote{1.1} two things lead to the decline of a mendicant trainee. What two? Anger and hostility … disdain and contempt … jealousy and stinginess … deceit and deviousness … lack of conscience and prudence. These two things lead to the decline of a mendicant trainee.” 

\subsection*{205–209 }

“These\marginnote{1.1} two things don’t lead to the decline of a mendicant trainee. What two? Freedom from anger and hostility … freedom from disdain and contempt … freedom from jealousy and stinginess … freedom from deceit and deviousness … conscience and prudence. These two things don’t lead to the decline of a mendicant trainee.” 

\subsection*{210–214 }

“Anyone\marginnote{1.1} who has two things is cast down to hell. What two? Anger and hostility … disdain and contempt … jealousy and stinginess … deceit and deviousness … lack of conscience and prudence. Anyone who has these two things is cast down to hell.” 

\subsection*{215–219 }

“Anyone\marginnote{1.1} who has two things is raised up to heaven. What two? Freedom from anger and hostility … freedom from disdain and contempt … freedom from jealousy and stinginess … freedom from deceit and deviousness … conscience and prudence. Anyone who has these two things is raised up to heaven.” 

\subsection*{220–224 }

“When\marginnote{1.1} they have two things, some people, when their body breaks up, after death, are reborn in a place of loss, a bad place, the underworld, hell. What two? Anger and hostility … disdain and contempt … jealousy and stinginess … deceit and deviousness … lack of conscience and prudence. When they have two things, some people, when their body breaks up, after death, are reborn in a place of loss, a bad place, the underworld, hell. 

\subsection*{225–229 }

“When\marginnote{1.1} they have two things, some people—when their body breaks up, after death—are reborn in a good place, a heavenly realm. What two? Freedom from anger and hostility … freedom from disdain and contempt … freedom from jealousy and stinginess … freedom from deceit and deviousness … conscience and prudence. When they have these two things, some people—when their body breaks up, after death—are reborn in a good place, a heavenly realm. 

%
\addtocontents{toc}{\let\protect\contentsline\protect\nopagecontentsline}
\chapter*{The Chapter on Abbreviated Texts Beginning with the Unskillful }
\addcontentsline{toc}{chapter}{\tocchapterline{The Chapter on Abbreviated Texts Beginning with the Unskillful }}
\addtocontents{toc}{\let\protect\contentsline\protect\oldcontentsline}

%
\section*{{\suttatitleacronym AN 2.230–279}{\suttatitleroot Akusalapeyyālavagga}}
\addcontentsline{toc}{section}{\tocacronym{AN 2.230–279} \tocroot{Akusalapeyyālavagga}}
\markboth{17. Abbreviated Texts Beginning with the Unskillful }{Akusalapeyyālavagga}
\extramarks{AN 2.230–279}{AN 2.230–279}

“These\marginnote{1.1} two things are unskillful … are skillful … are blameworthy … are blameless … have suffering as outcome … have happiness as outcome … result in suffering … result in happiness … are hurtful … are not hurtful. What two? Freedom from anger and hostility … freedom from disdain and contempt … freedom from jealousy and stinginess … freedom from deceit and deviousness … conscience and prudence. These are the two things that are not hurtful.” 

%
\addtocontents{toc}{\let\protect\contentsline\protect\nopagecontentsline}
\chapter*{The Chapter on Abbreviated Texts on Monastic Training }
\addcontentsline{toc}{chapter}{\tocchapterline{The Chapter on Abbreviated Texts on Monastic Training }}
\addtocontents{toc}{\let\protect\contentsline\protect\oldcontentsline}

%
\section*{{\suttatitleacronym AN 2.280–309}{\suttatitleroot Vinayapeyyālavagga}}
\addcontentsline{toc}{section}{\tocacronym{AN 2.280–309} \tocroot{Vinayapeyyālavagga}}
\markboth{18. Abbreviated Texts on Monastic Law }{Vinayapeyyālavagga}
\extramarks{AN 2.280–309}{AN 2.280–309}

\subsection*{280 }

“For\marginnote{1.1} two reasons the Realized One laid down training rules for his disciples. 

What\marginnote{1.2} two? For the well-being and comfort of the \textsanskrit{Saṅgha} … For keeping difficult persons in check and for the comfort of good-hearted mendicants … For restraining defilements that affect the present life and protecting against defilements that affect lives to come … For restraining threats to the present life and protecting against threats to lives to come … For restraining faults that affect the present life and protecting against faults that affect lives to come … For restraining hazards that affect the present life and protecting against hazards that affect lives to come … For restraining unskillful qualities that affect the present life and protecting against unskillful qualities that affect lives to come … Out of sympathy for laypeople and for breaking up factions of mendicants with wicked desires … For inspiring confidence in those without it, and increasing confidence in those who have it … For the continuation of the true teaching and the support of the training. These are the two reasons why the Realized One laid down training rules for his disciples.” 

\subsection*{281–309 }

“For\marginnote{1.1} two reasons the Realized One laid down for his disciples the monastic code … the recitation of the monastic code … the suspension of the recitation of the monastic code … the invitation to admonish … the setting aside of the invitation to admonish … the disciplinary act of censure … placing under dependence … banishment … reconciliation … debarment … probation … being sent back to the beginning … penance … reinstatement … restoration … removal … ordination … an act with a motion … an act with a motion and one announcement … an act with a motion and three announcements … laying down what was not previously laid down … amending what was laid down … the settling of a disciplinary matter in the presence of those concerned … the settling of a disciplinary matter by accurate recollection … the settling of a disciplinary matter due to recovery from madness … the settling of a disciplinary matter due to the acknowledgement of the offense … the settling of a disciplinary matter by the decision of a majority … the settling of a disciplinary matter by a verdict of aggravated misconduct … the settling of a disciplinary matter by covering over with grass. 

What\marginnote{1.2} two? For the well-being and comfort of the \textsanskrit{Saṅgha} … For keeping difficult persons in check and for the comfort of good-hearted mendicants … For restraining defilements that affect the present life and protecting against defilements that affect lives to come … For restraining threats to the present life and protecting against threats to lives to come … For restraining faults that affect the present life and protecting against faults that affect lives to come … For restraining hazards that affect the present life and protecting against hazards that affect lives to come … For restraining unskillful qualities that affect the present life and protecting against unskillful qualities that affect lives to come … Out of sympathy for laypeople and for breaking up factions of mendicants with wicked desires … For inspiring confidence in those without it, and increasing confidence in those who have it … For the continuation of the true teaching and the support of the training. 

These\marginnote{1.4} are the two reasons why the Realized One laid down the settlement of a disciplinary matter by covering over with grass for his disciples.” 

%
\addtocontents{toc}{\let\protect\contentsline\protect\nopagecontentsline}
\chapter*{The Chapter on Abbreviated Texts Beginning with Greed }
\addcontentsline{toc}{chapter}{\tocchapterline{The Chapter on Abbreviated Texts Beginning with Greed }}
\addtocontents{toc}{\let\protect\contentsline\protect\oldcontentsline}

%
\section*{{\suttatitleacronym AN 2.310–479}{\suttatitleroot Rāgapeyyālavagga}}
\addcontentsline{toc}{section}{\tocacronym{AN 2.310–479} \tocroot{Rāgapeyyālavagga}}
\markboth{19. Abbreviated Texts Beginning with Greed }{Rāgapeyyālavagga}
\extramarks{AN 2.310–479}{AN 2.310–479}

\subsection*{310–319 }

“For\marginnote{1.1} insight into greed, two things should be developed. What two? Serenity and discernment. For insight into greed, these two things should be developed.” 

“For\marginnote{2.1} the complete understanding … finishing … giving up … ending … vanishing … fading away … cessation … giving away … letting go of greed, two things should be developed.” 

\subsection*{320–479 }

“Of\marginnote{1.1} hate … delusion … anger … hostility … disdain … contempt … jealousy … stinginess … deceit … deviousness … obstinacy … aggression … conceit … arrogance … vanity … negligence … for insight … complete understanding … finishing … giving up … ending … vanishing … fading away … cessation … giving away … letting go … two things should be developed. What two? Serenity and discernment. For the letting go of negligence, these two things should be developed.” 

That\marginnote{2.1} is what the Buddha said. Satisfied, the mendicants were happy with what the Buddha said. 

\scendbook{The Book of the Twos is finished. }

%
\addtocontents{toc}{\let\protect\contentsline\protect\nopagecontentsline}
\part*{The Book of the Threes }
\addcontentsline{toc}{part}{The Book of the Threes }
\markboth{}{}
\addtocontents{toc}{\let\protect\contentsline\protect\oldcontentsline}

%
%
\addtocontents{toc}{\let\protect\contentsline\protect\nopagecontentsline}
\pannasa{The First Fifty }
\addcontentsline{toc}{pannasa}{The First Fifty }
\markboth{}{}
\addtocontents{toc}{\let\protect\contentsline\protect\oldcontentsline}

%
\addtocontents{toc}{\let\protect\contentsline\protect\nopagecontentsline}
\chapter*{The Chapter on Fools }
\addcontentsline{toc}{chapter}{\tocchapterline{The Chapter on Fools }}
\addtocontents{toc}{\let\protect\contentsline\protect\oldcontentsline}

%
\section*{{\suttatitleacronym AN 3.1}{\suttatitletranslation Perils }{\suttatitleroot Bhayasutta}}
\addcontentsline{toc}{section}{\tocacronym{AN 3.1} \toctranslation{Perils } \tocroot{Bhayasutta}}
\markboth{Perils }{Bhayasutta}
\extramarks{AN 3.1}{AN 3.1}

\scevam{So\marginnote{1.1} I have heard. }At one time the Buddha was staying near \textsanskrit{Sāvatthī} in Jeta’s Grove, \textsanskrit{Anāthapiṇḍika}’s monastery. There the Buddha addressed the mendicants, “Mendicants!” 

“Venerable\marginnote{1.5} sir,” they replied. The Buddha said this: 

“Whatever\marginnote{2.1} dangers there are, all come from the foolish, not from the astute. Whatever perils there are, all come from the foolish, not from the astute. Whatever hazards there are, all come from the foolish, not from the astute. It’s like a fire that spreads from a hut made of reeds or grass, and burns down even a bungalow, plastered inside and out, draft-free, with latches fastened and windows shuttered. In the same way, whatever dangers there are, all come from the foolish, not from the astute. Whatever perils there are, all come from the foolish, not from the astute. Whatever hazards there are, all come from the foolish, not from the astute. 

So,\marginnote{3.1} the fool is dangerous, but the astute person is safe. The fool is perilous, but the astute person is not. The fool is hazardous, but the astute person is not. There’s no danger, peril, or hazard that comes from the astute. 

So\marginnote{4.1} you should train like this: ‘We will reject the three things by which a fool is known, and we will undertake and follow the three things by which an astute person is known.’ That’s how you should train.” 

%
\section*{{\suttatitleacronym AN 3.2}{\suttatitletranslation Characteristics }{\suttatitleroot Lakkhaṇasutta}}
\addcontentsline{toc}{section}{\tocacronym{AN 3.2} \toctranslation{Characteristics } \tocroot{Lakkhaṇasutta}}
\markboth{Characteristics }{Lakkhaṇasutta}
\extramarks{AN 3.2}{AN 3.2}

“A\marginnote{1.1} fool is characterized by their deeds; an astute person is characterized by their deeds. And wisdom makes behavior beautiful. A fool is known by three things. What three? Bad conduct by way of body, speech, and mind. 

These\marginnote{1.5} are the three things by which a fool is known. 

An\marginnote{2.1} astute person is known by three things. What three? Good conduct by way of body, speech, and mind. 

These\marginnote{2.4} are the three things by which an astute person is known. 

So\marginnote{3.1} you should train like this: ‘We will reject the three things by which a fool is known, and we will undertake and follow the three things by which an astute person is known.’ That’s how you should train.” 

%
\section*{{\suttatitleacronym AN 3.3}{\suttatitletranslation Thinking }{\suttatitleroot Cintīsutta}}
\addcontentsline{toc}{section}{\tocacronym{AN 3.3} \toctranslation{Thinking } \tocroot{Cintīsutta}}
\markboth{Thinking }{Cintīsutta}
\extramarks{AN 3.3}{AN 3.3}

“There\marginnote{1.1} are these three characteristics, signs, and manifestations of a fool. What three? A fool thinks poorly, speaks poorly, and acts poorly. If a fool didn’t think poorly, speak poorly, and act poorly, then how would the astute know of them: ‘This fellow is a fool, a bad person’? But since a fool does think poorly, speak poorly, and act poorly, then the astute do know of them: ‘This fellow is a fool, a bad person’. These are the three characteristics, signs, and manifestations of a fool. 

There\marginnote{2.1} are these three characteristics, signs, and manifestations of an astute person. What three? An astute person thinks well, speaks well, and acts well. If an astute person didn’t think well, speak well, and act well, then how would the astute know of them: ‘This fellow is astute, a good person’? But since an astute person does think well, speak well, and act well, then the astute do know of them: ‘This fellow is astute, a good person’. These are the three characteristics, signs, and manifestations of an astute person. So you should train …” 

%
\section*{{\suttatitleacronym AN 3.4}{\suttatitletranslation Mistakes }{\suttatitleroot Accayasutta}}
\addcontentsline{toc}{section}{\tocacronym{AN 3.4} \toctranslation{Mistakes } \tocroot{Accayasutta}}
\markboth{Mistakes }{Accayasutta}
\extramarks{AN 3.4}{AN 3.4}

“A\marginnote{1.1} fool is known by three things. What three? They don’t recognize when they’ve made a mistake. When they do recognize it they don’t deal with it properly. And when someone else confesses a mistake to them, they don’t accept it properly. 

These\marginnote{1.4} are the three things by which a fool is known. 

An\marginnote{2.1} astute person is known by three things. What three? They recognize when they’ve made a mistake. When they recognize it they deal with it properly. And when someone else confesses a mistake to them, they accept it properly. 

These\marginnote{2.4} are the three things by which an astute person is known. So you should train …” 

%
\section*{{\suttatitleacronym AN 3.5}{\suttatitletranslation Improper }{\suttatitleroot Ayonisosutta}}
\addcontentsline{toc}{section}{\tocacronym{AN 3.5} \toctranslation{Improper } \tocroot{Ayonisosutta}}
\markboth{Improper }{Ayonisosutta}
\extramarks{AN 3.5}{AN 3.5}

“A\marginnote{1.1} fool is known by three things. What three? They ask a question improperly. They answer a question improperly. And when someone else answers a question properly—with well-rounded, coherent, and relevant words and phrases—they disagree with it. 

These\marginnote{1.4} are the three things by which a fool is known. 

An\marginnote{2.1} astute person is known by three things. What three? They ask a question properly. They answer a question properly. And when someone else answers a question properly—with well-rounded, coherent, and relevant words and phrases—they agree with it. 

These\marginnote{2.4} are the three things by which an astute person is known. So you should train …” 

%
\section*{{\suttatitleacronym AN 3.6}{\suttatitletranslation Unskillful }{\suttatitleroot Akusalasutta}}
\addcontentsline{toc}{section}{\tocacronym{AN 3.6} \toctranslation{Unskillful } \tocroot{Akusalasutta}}
\markboth{Unskillful }{Akusalasutta}
\extramarks{AN 3.6}{AN 3.6}

“A\marginnote{1.1} fool is known by three things. What three? Unskillful deeds by way of body, speech, and mind. 

These\marginnote{1.4} are the three things by which a fool is known. 

An\marginnote{2.1} astute person is known by three things. What three? Skillful deeds by way of body, speech, and mind. 

These\marginnote{2.4} are the three things by which an astute person is known. So you should train …” 

%
\section*{{\suttatitleacronym AN 3.7}{\suttatitletranslation Blameworthy }{\suttatitleroot Sāvajjasutta}}
\addcontentsline{toc}{section}{\tocacronym{AN 3.7} \toctranslation{Blameworthy } \tocroot{Sāvajjasutta}}
\markboth{Blameworthy }{Sāvajjasutta}
\extramarks{AN 3.7}{AN 3.7}

“A\marginnote{1.1} fool is known by three things. What three? Blameworthy deeds by way of body, speech, and mind. … An astute person is known by blameless deeds by way of body, speech, and mind. …” 

%
\section*{{\suttatitleacronym AN 3.8}{\suttatitletranslation Hurtful }{\suttatitleroot Sabyābajjhasutta}}
\addcontentsline{toc}{section}{\tocacronym{AN 3.8} \toctranslation{Hurtful } \tocroot{Sabyābajjhasutta}}
\markboth{Hurtful }{Sabyābajjhasutta}
\extramarks{AN 3.8}{AN 3.8}

“A\marginnote{1.1} fool is known by three things. What three? Hurtful deeds by way of body, speech, and mind. … An astute person is known by kind deeds by way of body, speech, and mind. These are the three things by which an astute person is known. 

So\marginnote{2.1} you should train like this: ‘We will reject the three qualities by which a fool is known, and we will undertake and follow the three qualities by which an astute person is known.’ That’s how you should train.” 

%
\section*{{\suttatitleacronym AN 3.9}{\suttatitletranslation Broken }{\suttatitleroot Khatasutta}}
\addcontentsline{toc}{section}{\tocacronym{AN 3.9} \toctranslation{Broken } \tocroot{Khatasutta}}
\markboth{Broken }{Khatasutta}
\extramarks{AN 3.9}{AN 3.9}

“When\marginnote{1.1} a foolish, incompetent, bad person has three qualities they keep themselves broken and damaged. They deserve to be blamed and criticized by sensible people, and they make much bad karma. What three? Bad conduct by way of body, speech, and mind. 

When\marginnote{1.4} a foolish, incompetent bad person has these three qualities they keep themselves broken and damaged. They deserve to be blamed and criticized by sensible people, and they make much bad karma. 

When\marginnote{2.1} an astute, competent good person has three qualities they keep themselves healthy and whole. They don’t deserve to be blamed and criticized by sensible people, and they make much merit. What three? Good conduct by way of body, speech, and mind. 

When\marginnote{2.4} an astute, competent good person has these three qualities they keep themselves healthy and whole. They don’t deserve to be blamed and criticized by sensible people, and they make much merit.” 

%
\section*{{\suttatitleacronym AN 3.10}{\suttatitletranslation Stains }{\suttatitleroot Malasutta}}
\addcontentsline{toc}{section}{\tocacronym{AN 3.10} \toctranslation{Stains } \tocroot{Malasutta}}
\markboth{Stains }{Malasutta}
\extramarks{AN 3.10}{AN 3.10}

“Anyone\marginnote{1.1} who has three qualities, and has not given up three stains, is cast down to hell. What three? They’re unethical, and haven’t given up the stain of immorality. They’re jealous, and haven’t given up the stain of jealousy. They’re stingy, and haven’t given up the stain of stinginess. 

Anyone\marginnote{1.6} who has these three qualities, and has not given up these three stains, is cast down to hell. 

Anyone\marginnote{2.1} who has three qualities, and has given up three stains, is raised up to heaven. What three? They’re ethical, and have given up the stain of immorality. They’re not jealous, and have given up the stain of jealousy. They’re not stingy, and have given up the stain of stinginess. 

Anyone\marginnote{2.6} who has these three qualities, and has given up these three stains, is raised up to heaven.” 

%
\addtocontents{toc}{\let\protect\contentsline\protect\nopagecontentsline}
\chapter*{The Chapter on the Chariot-maker }
\addcontentsline{toc}{chapter}{\tocchapterline{The Chapter on the Chariot-maker }}
\addtocontents{toc}{\let\protect\contentsline\protect\oldcontentsline}

%
\section*{{\suttatitleacronym AN 3.11}{\suttatitletranslation Well-known }{\suttatitleroot Ñātasutta}}
\addcontentsline{toc}{section}{\tocacronym{AN 3.11} \toctranslation{Well-known } \tocroot{Ñātasutta}}
\markboth{Well-known }{Ñātasutta}
\extramarks{AN 3.11}{AN 3.11}

“Mendicants,\marginnote{1.1} a well-known mendicant who has three qualities is acting for the hurt and unhappiness of the people, for the harm, hurt, and suffering of gods and humans. What three? They encourage deeds of body and speech, as well as principles, that don’t reinforce good qualities. 

A\marginnote{1.4} well-known mendicant who has these three qualities is acting for the hurt and unhappiness of the people, for the harm, hurt, and suffering of gods and humans. 

A\marginnote{2.1} well-known mendicant who has three qualities is acting for the welfare and happiness of the people, for the benefit, welfare, and happiness of gods and humans. What three? They encourage deeds of body and speech, as well as principles, that reinforce good qualities. 

A\marginnote{2.4} well-known mendicant who has these three qualities is acting for the welfare and happiness of the people, for the benefit, welfare, and happiness of gods and humans.” 

%
\section*{{\suttatitleacronym AN 3.12}{\suttatitletranslation Commemoration }{\suttatitleroot Sāraṇīyasutta}}
\addcontentsline{toc}{section}{\tocacronym{AN 3.12} \toctranslation{Commemoration } \tocroot{Sāraṇīyasutta}}
\markboth{Commemoration }{Sāraṇīyasutta}
\extramarks{AN 3.12}{AN 3.12}

“An\marginnote{1.1} anointed aristocratic king should commemorate three places as long as he lives. What three? The place he was born. This is the first place. 

The\marginnote{2.1} place he was anointed as king. This is the second place. 

The\marginnote{3.1} place where he won victory in battle, establishing himself as foremost in battle. This is the third place. These are the three places an anointed king should commemorate as long as he lives. 

In\marginnote{4.1} the same way, a mendicant should commemorate three places as long as they live. What three? The place where the mendicant shaved off their hair and beard, dressed in ocher robes, and went forth from the lay life to homelessness. This is the first place. 

The\marginnote{5.1} place where the mendicant truly understands: ‘This is suffering’ … ‘This is the origin of suffering’ … ‘This is the cessation of suffering’ … ‘This is the practice that leads to the cessation of suffering’. This is the second place. 

The\marginnote{6.1} place where the mendicant realizes the undefiled freedom of heart and freedom by wisdom in this very life. And they live having realized it with their own insight due to the ending of defilements. This is the third place. These are the three places a mendicant should commemorate as long as they live.” 

%
\section*{{\suttatitleacronym AN 3.13}{\suttatitletranslation Hopes }{\suttatitleroot Āsaṁsasutta}}
\addcontentsline{toc}{section}{\tocacronym{AN 3.13} \toctranslation{Hopes } \tocroot{Āsaṁsasutta}}
\markboth{Hopes }{Āsaṁsasutta}
\extramarks{AN 3.13}{AN 3.13}

“These\marginnote{1.1} three kinds of people are found in the world. What three? The hopeless, the hopeful, and the one who has done away with hope. And what, mendicants, is a hopeless person? It’s when some person is reborn in a low family—a family of outcastes, bamboo-workers, hunters, chariot-makers, or waste-collectors—poor, with little to eat or drink, where life is tough, and food and shelter are hard to find. And they’re ugly, unsightly, deformed, chronically ill—one-eyed, crippled, lame, or half-paralyzed. They don’t get to have food, drink, clothes, and vehicles; garlands, fragrances, and makeup; or bed, house, and lighting. They hear this: ‘They say the aristocrats have anointed the aristocrat named so-and-so as king.’ It never occurs to them: ‘Oh, when will the aristocrats anoint me too as king?’ This is called a hopeless person. 

And\marginnote{2.1} what, mendicants, is a hopeful person? It’s when some person is the eldest son of an anointed aristocratic king. He has not yet been anointed, but is eligible, and has been confirmed in the succession. He hears this: ‘They say that the aristocrats have anointed the aristocrat named so-and-so as king.’ It occurs to him: ‘Oh, when will the aristocrats anoint me too as king?’ This is called a hopeful person. 

And\marginnote{3.1} what, mendicants, is a person who has done away with hope? It’s when a king has been anointed. He hears this: ‘They say that the aristocrats have anointed the aristocrat named so-and-so as king.’ It never occurs to him: ‘Oh, when will the aristocrats anoint me too as king?’ Why is that? Because the former hope he had to be anointed has now died down. This is called a person who has done away with hope. 

These\marginnote{3.10} are the three kinds of people found in the world. 

In\marginnote{4.1} the same way, these three kinds of people are found among the mendicants. What three? The hopeless, the hopeful, and the one who has done away with hope. And what, mendicants, is a hopeless person? It’s when some person is unethical, of bad qualities, filthy, with suspicious behavior, underhand, no true ascetic or spiritual practitioner—though claiming to be one—rotten inside, corrupt, and depraved. They hear this: ‘They say that the mendicant named so-and-so has realized the undefiled freedom of heart and freedom by wisdom in this very life. And they live having realized it with their own insight due to the ending of defilements.’ It never occurs to them: ‘Oh, when will I too realize the undefiled freedom of heart and freedom by wisdom in this very life, and live having realized it with my own insight due to the ending of defilements.’ This is called a hopeless person. 

And\marginnote{5.1} what, mendicants, is a hopeful person? It’s when a mendicant is ethical, of good character. They hear this: ‘They say that the mendicant named so-and-so has realized the undefiled freedom of heart and freedom by wisdom in this very life. And they live having realized it with their own insight due to the ending of defilements.’ It occurs to them: ‘Oh, when will I too realize the undefiled freedom of heart and freedom by wisdom in this very life, and live having realized it with my own insight due to the ending of defilements.’ This is called a hopeful person. 

And\marginnote{6.1} what, mendicants, is a person who has done away with hope? It’s when a mendicant is a perfected one, who has ended all defilements. They hear this: ‘They say that the mendicant named so-and-so has realized the undefiled freedom of heart and freedom by wisdom in this very life. And they live having realized it with their own insight due to the ending of defilements.’ It never occurs to them: ‘Oh, when will I too realize the undefiled freedom of heart and freedom by wisdom in this very life, and live having realized it with my own insight due to the ending of defilements.’ Why is that? Because the former hope they had to be freed has now died down. This is called a person who has done away with hope. 

These\marginnote{6.10} are the three people found among the mendicants.” 

%
\section*{{\suttatitleacronym AN 3.14}{\suttatitletranslation The Wheel-Turning Monarch }{\suttatitleroot Cakkavattisutta}}
\addcontentsline{toc}{section}{\tocacronym{AN 3.14} \toctranslation{The Wheel-Turning Monarch } \tocroot{Cakkavattisutta}}
\markboth{The Wheel-Turning Monarch }{Cakkavattisutta}
\extramarks{AN 3.14}{AN 3.14}

“Mendicants,\marginnote{1.1} even a wheel-turning monarch, a just and principled king, does not wield power without having their own king.” When he said this, one of the mendicants asked the Buddha: 

“But\marginnote{1.3} who is the king of the wheel-turning monarch, the just and principled king?” 

“It\marginnote{1.4} is principle, monk,” said the Buddha. 

“Monk,\marginnote{1.5} a wheel-turning monarch provides just protection and security for his court, relying only on principle—honoring, respecting, and venerating principle, having principle as his flag, banner, and authority. 

He\marginnote{2.1} provides just protection and security for his aristocrats, vassals, troops, brahmins and householders, people of town and country, ascetics and brahmins, beasts and birds. When he has done this, he wields power only in a principled manner. And this power cannot be undermined by any human enemy. 

In\marginnote{3.1} the same way, monk, a Realized One, a perfected one, a fully awakened Buddha, a just and principled king, provides just protection and security regarding bodily actions, relying only on principle—honoring, respecting, and venerating principle, having principle as his flag, banner, and authority. ‘This kind of bodily action should be cultivated. This kind of bodily action should not be cultivated.’ 

Furthermore,\marginnote{4.1} a Realized One … provides just protection and security regarding verbal actions, saying: ‘This kind of verbal action should be cultivated. This kind of verbal action should not be cultivated.’ … And regarding mental actions: ‘This kind of mental action should be cultivated. This kind of mental action should not be cultivated.’ 

And\marginnote{5.1} when a Realized One, a perfected one, a fully awakened Buddha has provided just protection and security regarding actions of body, speech, and mind, he rolls forth the supreme Wheel of Dhamma. And that wheel cannot be rolled back by any ascetic or brahmin or god or \textsanskrit{Māra} or \textsanskrit{Brahmā} or by anyone in the world.” 

%
\section*{{\suttatitleacronym AN 3.15}{\suttatitletranslation About Pacetana }{\suttatitleroot Sacetanasutta}}
\addcontentsline{toc}{section}{\tocacronym{AN 3.15} \toctranslation{About Pacetana } \tocroot{Sacetanasutta}}
\markboth{About Pacetana }{Sacetanasutta}
\extramarks{AN 3.15}{AN 3.15}

At\marginnote{1.1} one time the Buddha was staying near Benares, in the deer park at Isipatana. There the Buddha addressed the mendicants, “Mendicants!” 

“Venerable\marginnote{1.4} sir,” they replied. The Buddha said this: 

“Once\marginnote{2.1} upon a time there was a king named Pacetana. Then King Pacetana addressed his chariot-maker, ‘In six months’ time, my good chariot-maker, there will be a battle. Are you able to make me a new pair of wheels?’ 

‘I\marginnote{2.5} can, Your Majesty,’ replied the chariot-maker. Then, when it was six days less than six months later, the chariot-maker had finished one wheel. 

Then\marginnote{2.7} King Pacetana addressed his chariot-maker, ‘In six days’ time there will be a battle. Is my new pair of wheels finished?’ 

‘Now\marginnote{2.9} that it is six days less than six months, Your Majesty, I have finished one wheel.’ 

‘Are\marginnote{2.10} you able to finish the second wheel in these six days?’ 

Saying,\marginnote{2.11} ‘I can, Your Majesty,’ the chariot-maker finished the second wheel in six days. Taking the pair of wheels he went up to King Pacetana, and said this to the king, ‘Your Majesty, these are your two new wheels, finished.’ 

‘But,\marginnote{2.13} my good chariot-maker, what is the difference between the wheel that was finished in six days less than six months, and the wheel finished in just six days? Because I can’t see any difference between them.’ 

‘But,\marginnote{2.15} Your Majesty, there is a difference. See now what it is.’ 

Then\marginnote{3.1} the chariot-maker rolled forth the wheel that had been finished in six days. It rolled as far as the original impetus took it, then wobbled and fell down. Then he rolled forth the wheel that had been finished in six days less than six months. It rolled as far as the original impetus took it, then stood still as if fixed to an axle. 

‘But\marginnote{4.1} what is the cause, my good chariot-maker, what is the reason why the wheel that was finished in six days wobbled and fell, while the one that was finished in six days less than six months stood still as if fixed to an axle?’ 

‘The\marginnote{4.3} wheel that was finished in six days, Your Majesty, is crooked, flawed, and defective in rim, spoke, and hub. That’s why it wobbled and fell. The wheel that was finished in six days less than six months, Your Majesty, is not crooked, flawed, and defective in rim, spoke, and hub. That’s why it stood still as if fixed to an axle.’ 

Now,\marginnote{5.1} mendicants, you might think: ‘Surely that chariot-maker must have been someone else at that time?’ But you should not see it like that. I myself was the chariot-maker at that time. Then I was skilled in the crooks, flaws, and defects of wood. 

Now\marginnote{5.6} that I am a perfected one, a fully awakened Buddha, I am skilled in the crooks, flaws, and defects of actions by body, speech, and mind. Whatever monk or nun has not given up the crooks, flaws, and defects of body, speech, and mind has fallen from the teaching and training, just like the wheel that was finished in six days. 

Whatever\marginnote{6.1} monk or nun has given up the crooks, flaws, and defects of body, speech, and mind is established in the teaching and training, just like the wheel that was finished in six days less than six months. 

So\marginnote{7.1} you should train like this: ‘We will give up the crooks, flaws, and defects of body, speech, and mind.’ That’s how you should train.” 

%
\section*{{\suttatitleacronym AN 3.16}{\suttatitletranslation Guaranteed }{\suttatitleroot Apaṇṇakasutta}}
\addcontentsline{toc}{section}{\tocacronym{AN 3.16} \toctranslation{Guaranteed } \tocroot{Apaṇṇakasutta}}
\markboth{Guaranteed }{Apaṇṇakasutta}
\extramarks{AN 3.16}{AN 3.16}

“Mendicants,\marginnote{1.1} when a mendicant has three things their practice is guaranteed, and they have laid the groundwork for ending the defilements. What three? It’s when a mendicant guards the sense doors, eats in moderation, and is dedicated to wakefulness. 

And\marginnote{2.1} how does a mendicant guard the sense doors? When a mendicant sees a sight with their eyes, they don’t get caught up in the features and details. If the faculty of sight were left unrestrained, bad unskillful qualities of desire and aversion would become overwhelming. For this reason, they practice restraint, protecting the faculty of sight, and achieving its restraint. When they hear a sound with their ears … When they smell an odor with their nose … When they taste a flavor with their tongue … When they feel a touch with their body … When they know a thought with their mind, they don’t get caught up in the features and details. If the faculty of mind were left unrestrained, bad unskillful qualities of desire and aversion would become overwhelming. For this reason, they practice restraint, protecting the faculty of mind, and achieving its restraint. That’s how a mendicant guards the sense doors. 

And\marginnote{3.1} how does a mendicant eat in moderation? It’s when a mendicant reflects properly on the food that they eat: ‘Not for fun, indulgence, adornment, or decoration, but only to sustain this body, to avoid harm, and to support spiritual practice. In this way, I shall put an end to old discomfort and not give rise to new discomfort, and I will live blamelessly and at ease.’ That’s how a mendicant eats in moderation. 

And\marginnote{4.1} how is a mendicant dedicated to wakefulness? It’s when a mendicant practices walking and sitting meditation by day, purifying their mind from obstacles. In the evening, they continue to practice walking and sitting meditation. In the middle of the night, they lie down in the lion’s posture—on the right side, placing one foot on top of the other—mindful and aware, and focused on the time of getting up. In the last part of the night, they get up and continue to practice walking and sitting meditation, purifying their mind from obstacles. This is how a mendicant is dedicated to wakefulness. 

When\marginnote{4.4} a mendicant has these three things their practice is guaranteed, and they have laid the groundwork for ending the defilements.” 

%
\section*{{\suttatitleacronym AN 3.17}{\suttatitletranslation Hurting Yourself }{\suttatitleroot Attabyābādhasutta}}
\addcontentsline{toc}{section}{\tocacronym{AN 3.17} \toctranslation{Hurting Yourself } \tocroot{Attabyābādhasutta}}
\markboth{Hurting Yourself }{Attabyābādhasutta}
\extramarks{AN 3.17}{AN 3.17}

“These\marginnote{1.1} three things, mendicants, lead to hurting yourself, hurting others, and hurting both. What three? Bad conduct by way of body, speech, and mind. 

These\marginnote{1.4} are three things that lead to hurting yourself, hurting others, and hurting both. 

These\marginnote{2.1} three things, mendicants, don’t lead to hurting yourself, hurting others, or hurting both. What three? Good conduct by way of body, speech, and mind. 

These\marginnote{2.4} are three things that don’t lead to hurting yourself, hurting others, or hurting both.” 

%
\section*{{\suttatitleacronym AN 3.18}{\suttatitletranslation The Realm of the Gods }{\suttatitleroot Devalokasutta}}
\addcontentsline{toc}{section}{\tocacronym{AN 3.18} \toctranslation{The Realm of the Gods } \tocroot{Devalokasutta}}
\markboth{The Realm of the Gods }{Devalokasutta}
\extramarks{AN 3.18}{AN 3.18}

“Mendicants,\marginnote{1.1} if wanderers who follow another path were to ask you: ‘Reverend, do you lead the spiritual life with the ascetic Gotama so that you can be reborn in the realm of the gods?’ Being questioned like this, wouldn’t you be horrified, repelled, and disgusted?” 

“Yes,\marginnote{1.4} sir.” 

“So\marginnote{1.5} it seems that you are horrified, repelled, and disgusted by divine lifespan, beauty, happiness, fame, and sovereignty. How much more then should you be horrified, embarrassed, and disgusted by bad conduct by way of body, speech, and mind.” 

%
\section*{{\suttatitleacronym AN 3.19}{\suttatitletranslation A Shopkeeper (1st) }{\suttatitleroot Paṭhamapāpaṇikasutta}}
\addcontentsline{toc}{section}{\tocacronym{AN 3.19} \toctranslation{A Shopkeeper (1st) } \tocroot{Paṭhamapāpaṇikasutta}}
\markboth{A Shopkeeper (1st) }{Paṭhamapāpaṇikasutta}
\extramarks{AN 3.19}{AN 3.19}

“Mendicants,\marginnote{1.1} a shopkeeper who has three factors is unable to acquire more wealth or to increase the wealth they’ve already acquired. What three? It’s when a shopkeeper doesn’t carefully apply themselves to their work in the morning, at midday, and in the afternoon. A shopkeeper who has these three factors is unable to acquire more wealth or to increase the wealth they’ve already acquired. 

In\marginnote{2.1} the same way, a mendicant who has three qualities is unable to acquire more skillful qualities or to increase the skillful qualities they’ve already acquired. What three? It’s when a mendicant doesn’t carefully apply themselves to a meditation subject as a foundation of immersion in the morning, at midday, and in the afternoon. 

A\marginnote{2.4} mendicant who has these three qualities is unable to acquire more skillful qualities or to increase the skillful qualities they’ve already acquired. 

A\marginnote{3.1} shopkeeper who has three factors is able to acquire more wealth or to increase the wealth they’ve already acquired. What three? It’s when a shopkeeper carefully applies themselves to their work in the morning, at midday, and in the afternoon. A shopkeeper who has these three factors is able to acquire more wealth or to increase the wealth they’ve already acquired. 

In\marginnote{4.1} the same way, a mendicant who has three qualities is able to acquire more skillful qualities or to increase the skillful qualities they’ve already acquired. What three? It’s when a mendicant carefully applies themselves to a meditation subject as a foundation of immersion in the morning, at midday, and in the afternoon. 

A\marginnote{4.4} mendicant who has these three qualities is able to acquire more skillful qualities or to increase the skillful qualities they’ve already acquired.” 

%
\section*{{\suttatitleacronym AN 3.20}{\suttatitletranslation A Shopkeeper (2nd) }{\suttatitleroot Dutiyapāpaṇikasutta}}
\addcontentsline{toc}{section}{\tocacronym{AN 3.20} \toctranslation{A Shopkeeper (2nd) } \tocroot{Dutiyapāpaṇikasutta}}
\markboth{A Shopkeeper (2nd) }{Dutiyapāpaṇikasutta}
\extramarks{AN 3.20}{AN 3.20}

“Mendicants,\marginnote{1.1} a shopkeeper who has three factors soon acquires great and abundant wealth. What three? It’s when a shopkeeper sees clearly, is indefatigable, and has supporters. 

And\marginnote{1.4} how does a shopkeeper see clearly? It’s when a shopkeeper knows of a product: ‘This product is bought at this price and is selling at this price. With this much investment, it’ll bring this much profit.’ That’s how a shopkeeper sees clearly. 

And\marginnote{2.1} how is a shopkeeper indefatigable? It’s when a shopkeeper is skilled in buying and selling products. That’s how a shopkeeper is indefatigable. 

And\marginnote{3.1} how does a shopkeeper have supporters? It’s when rich, affluent, and wealthy householders or householders’ children know of him: ‘This good shopkeeper sees clearly and is indefatigable. They are capable of providing for their wives and children, and paying us back from time to time.’ They deposit money with the shopkeeper, saying: ‘With this, friend shopkeeper, earn money to raise your wives and children, and pay us back from time to time.’ That’s how a shopkeeper has supporters. 

A\marginnote{3.7} shopkeeper who has these three factors soon acquires great and abundant wealth. 

In\marginnote{4.1} the same way, a mendicant who has three qualities soon acquires great and abundant skillful qualities. What three? It’s when a mendicant sees clearly, is indefatigable, and has supporters. 

And\marginnote{4.4} how does a mendicant see clearly? It’s when a mendicant truly understands: ‘This is suffering’ … ‘This is the origin of suffering’ … ‘This is the cessation of suffering’ … ‘This is the practice that leads to the cessation of suffering’. That’s how a mendicant sees clearly. 

And\marginnote{5.1} how is a mendicant indefatigable? It’s when a mendicant lives with energy roused up for giving up unskillful qualities and embracing skillful qualities. They are strong, staunchly vigorous, not slacking off when it comes to developing skillful qualities. That’s how a mendicant is indefatigable. 

And\marginnote{6.1} how does a mendicant have supporters? It’s when from time to time a mendicant goes up to those mendicants who are very learned—knowledgeable in the scriptures, who have memorized the teachings, the monastic law, and the outlines—and asks them questions: ‘Why, sir, does it say this? What does that mean?’ Those venerables clarify what is unclear, reveal what is obscure, and dispel doubt regarding the many doubtful matters. That’s how a mendicant has supporters. 

A\marginnote{6.6} mendicant who has these three qualities soon acquires great and abundant skillful qualities.” 

\scend{The first recitation section is finished. }

%
\addtocontents{toc}{\let\protect\contentsline\protect\nopagecontentsline}
\chapter*{The Chapter on Persons }
\addcontentsline{toc}{chapter}{\tocchapterline{The Chapter on Persons }}
\addtocontents{toc}{\let\protect\contentsline\protect\oldcontentsline}

%
\section*{{\suttatitleacronym AN 3.21}{\suttatitletranslation With Saviṭṭha }{\suttatitleroot Samiddhasutta}}
\addcontentsline{toc}{section}{\tocacronym{AN 3.21} \toctranslation{With Saviṭṭha } \tocroot{Samiddhasutta}}
\markboth{With Saviṭṭha }{Samiddhasutta}
\extramarks{AN 3.21}{AN 3.21}

\scevam{So\marginnote{1.1} I have heard. }At one time the Buddha was staying near \textsanskrit{Sāvatthī} in Jeta’s Grove, \textsanskrit{Anāthapiṇḍika}’s monastery. 

Then\marginnote{1.3} Venerable \textsanskrit{Saviṭṭha} and Venerable \textsanskrit{Mahākoṭṭhita} went up to Venerable \textsanskrit{Sāriputta}, and exchanged greetings with him. 

When\marginnote{1.4} the greetings and polite conversation were over, they sat down to one side and Venerable \textsanskrit{Sāriputta} said to Venerable \textsanskrit{Saviṭṭha}, “Reverend \textsanskrit{Saviṭṭha}, these three people are found in the world. What three? The personal witness, the one attained to view, and the one freed by faith. These are the three people found in the world. Of these three people, who do you believe to be the finest?” 

“Reverend\marginnote{3.1} \textsanskrit{Sāriputta}, these three people are found in the world. What three? The personal witness, the one attained to view, and the one freed by faith. These are the three people found in the world. Of these three people, I believe the person freed by faith to be finest. Why is that? Because this person’s faculty of faith is outstanding.” 

Then\marginnote{4.1} \textsanskrit{Sāriputta} said to \textsanskrit{Mahākoṭṭhita}, “Reverend \textsanskrit{Koṭṭhika}, these three people are found in the world. What three? The personal witness, the one attained to view, and the one freed by faith. These are the three people found in the world. Of these three people, who do you believe to be the finest?” 

“Reverend\marginnote{5.1} \textsanskrit{Sāriputta}, these three people are found in the world. What three? The personal witness, the one attained to view, and the one freed by faith. These are the three people found in the world. Of these three people, I believe the personal witness to be finest. Why is that? Because this person’s faculty of immersion is outstanding.” 

Then\marginnote{6.1} \textsanskrit{Mahākoṭṭhita} said to \textsanskrit{Sāriputta}, “Reverend \textsanskrit{Sāriputta}, these three people are found in the world. What three? The personal witness, the one attained to view, and the one freed by faith. These are the three people found in the world. Of these three people, who do you believe to be the finest?” 

“Reverend\marginnote{7.1} \textsanskrit{Koṭṭhika}, these three people are found in the world. What three? The personal witness, the one attained to view, and the one freed by faith. These are the three people found in the world. Of these three people, I believe the person attained to view to be finest. Why is that? Because this person’s faculty of wisdom is outstanding.” 

Then\marginnote{8.1} \textsanskrit{Sāriputta} said to \textsanskrit{Saviṭṭha} and \textsanskrit{Mahākoṭṭhita}, “Each of us has spoken from the heart. Come, reverends, let’s go to the Buddha, and tell him about this. As he answers, so we’ll remember it.” 

“Yes,\marginnote{8.5} reverend,” they replied. Then \textsanskrit{Sāriputta}, \textsanskrit{Saviṭṭha}, and \textsanskrit{Mahākoṭṭhita} went up to the Buddha, bowed, and sat down to one side. Then \textsanskrit{Sāriputta} told the Buddha of all they had discussed. 

“In\marginnote{9.1} this matter, \textsanskrit{Sāriputta}, it’s not easy to definitively declare that one of these three people is finest. In some cases, a person who is freed by faith is practicing for perfection, while the personal witness and the one attained to view are once-returners or non-returners. 

In\marginnote{10.1} this matter, it’s not easy to definitively declare that one of these three people is finest. In some cases, a personal witness is practicing for perfection, while the one freed by faith and the one attained to view are once-returners or non-returners. 

In\marginnote{11.1} this matter, it’s not easy to definitively declare that one of these three people is finest. In some cases, one attained to view is practicing for perfection, while the one freed by faith and the personal witness are once-returners or non-returners. 

In\marginnote{12.1} this matter, it’s not easy to definitively declare that one of these three people is finest.” 

%
\section*{{\suttatitleacronym AN 3.22}{\suttatitletranslation Patients }{\suttatitleroot Gilānasutta}}
\addcontentsline{toc}{section}{\tocacronym{AN 3.22} \toctranslation{Patients } \tocroot{Gilānasutta}}
\markboth{Patients }{Gilānasutta}
\extramarks{AN 3.22}{AN 3.22}

“These\marginnote{1.1} three patients are found in the world. What three? 

In\marginnote{1.3} some cases a patient won’t recover from an illness, regardless of whether or not they get suitable food and medicines, and a capable carer. 

In\marginnote{2.1} some cases a patient will recover from an illness, regardless of whether or not they get suitable food and medicines, and a capable carer. 

In\marginnote{3.1} some cases a patient can recover from an illness, but only if they get suitable food and medicines, and a capable carer, and not if they don’t get these things. 

Now,\marginnote{4.1} it’s for the sake of the last patient—who will recover only if they get suitable food and medicines, and a capable carer—that food, medicines, and a carer are prescribed. But also, for the sake of this patient, the other patients should be looked after. 

These\marginnote{4.3} are the three kinds of patients found in the world. 

In\marginnote{5.1} the same way, these three people similar to patients are found in the world. What three? Some people don’t enter the sure path with regards to skillful qualities, regardless of whether or not they get to see a Realized One, and to hear the teaching and training that he proclaims. 

Some\marginnote{6.1} people do enter the sure path with regards to skillful qualities, regardless of whether or not they get to see a Realized One, and to hear the teaching and training that he proclaims. 

Some\marginnote{7.1} people can enter the sure path with regards to skillful qualities, but only if they get to see a Realized One, and to hear the teaching and training that he proclaims, and not when they don’t get those things. 

Now,\marginnote{8.1} it’s for the sake of this last person that teaching the Dhamma is prescribed, that is, the one who can enter the sure path with regards to skillful qualities, but only if they get to see a Realized One, and to hear the teaching and training that he proclaims. But also, for the sake of this person, the other people should be taught Dhamma. 

These\marginnote{8.3} are the three people similar to patients found in the world.” 

%
\section*{{\suttatitleacronym AN 3.23}{\suttatitletranslation Choices }{\suttatitleroot Saṅkhārasutta}}
\addcontentsline{toc}{section}{\tocacronym{AN 3.23} \toctranslation{Choices } \tocroot{Saṅkhārasutta}}
\markboth{Choices }{Saṅkhārasutta}
\extramarks{AN 3.23}{AN 3.23}

“These\marginnote{1.1} three people are found in the world. What three? 

Firstly,\marginnote{1.3} a certain person makes hurtful choices by way of body, speech, and mind. Having made these choices, they’re reborn in a hurtful world, where hurtful contacts strike them. Touched by hurtful contacts, they experience hurtful feelings that are exclusively painful—like the beings in hell. 

Furthermore,\marginnote{2.1} another person makes pleasing choices by way of body, speech, and mind. Having made these choices, they are reborn in a pleasing world, where pleasing contacts strike them. Touched by pleasing contacts, they experience pleasing feelings that are exclusively happy—like the gods replete with glory. 

Furthermore,\marginnote{3.1} another person makes both hurtful and pleasing choices by way of body, speech, and mind. Having made these choices, they are reborn in a world that is both hurtful and pleasing, where hurtful and pleasing contacts strike them. Touched by both hurtful and pleasing contacts, they experience both hurtful and pleasing feelings that are a mixture of pleasure and pain—like humans, some gods, and some beings in the underworld. 

These\marginnote{3.5} are the three people found in the world.” 

%
\section*{{\suttatitleacronym AN 3.24}{\suttatitletranslation Very Helpful }{\suttatitleroot Bahukārasutta}}
\addcontentsline{toc}{section}{\tocacronym{AN 3.24} \toctranslation{Very Helpful } \tocroot{Bahukārasutta}}
\markboth{Very Helpful }{Bahukārasutta}
\extramarks{AN 3.24}{AN 3.24}

“Mendicants,\marginnote{1.1} these three people are very helpful to another. What three? 

The\marginnote{1.3} person who has enabled you to go for refuge to the Buddha, the teaching, and the \textsanskrit{Saṅgha}. This person is very helpful to another. 

Furthermore,\marginnote{2.1} the person who has enabled you to truly understand: ‘This is suffering’ … ‘This is the origin of suffering’ … ‘This is the cessation of suffering’ … ‘This is the practice that leads to the cessation of suffering’. This person is very helpful to another. 

Furthermore,\marginnote{3.1} the person who has enabled you to realize the undefiled freedom of heart and freedom by wisdom in this very life, and live having realized it with your own insight due to the ending of defilements. This person is very helpful to another. 

These\marginnote{3.3} are the three people who are very helpful to another person. 

And\marginnote{4.1} I say that there is no-one more helpful to another than these three people. And I also say that it’s not easy to repay these three people by bowing down to them, rising up for them, greeting them with joined palms, and observing proper etiquette for them; or by providing them with robes, almsfood, lodgings, and medicines and supplies for the sick.” 

%
\section*{{\suttatitleacronym AN 3.25}{\suttatitletranslation Like Diamond }{\suttatitleroot Vajirūpamasutta}}
\addcontentsline{toc}{section}{\tocacronym{AN 3.25} \toctranslation{Like Diamond } \tocroot{Vajirūpamasutta}}
\markboth{Like Diamond }{Vajirūpamasutta}
\extramarks{AN 3.25}{AN 3.25}

“These\marginnote{1.1} three people are found in the world. What three? A person with a mind like an open sore, a person with a mind like lightning, and a person with a mind like diamond. 

And\marginnote{1.4} who has a mind like an open sore? It’s someone who is irritable and bad-tempered. Even when lightly criticized they lose their temper, becoming annoyed, hostile, and hard-hearted, and they display annoyance, hate, and bitterness. They’re like a festering sore, which, when you hit it with a stick or a stone, discharges even more. In the same way, someone is irritable and bad-tempered. Even when lightly criticized they lose their temper, becoming annoyed, hostile, and hard-hearted, and they display annoyance, hate, and bitterness. This is called a person with a mind like an open sore. 

And\marginnote{2.1} who has a mind like lightning? It’s someone who truly understands: ‘This is suffering’ … ‘This is the origin of suffering’ … ‘This is the cessation of suffering’ … ‘This is the practice that leads to the cessation of suffering’. They’re like a person with keen eyes in the dark of the night, who sees by a flash of lightning. In the same way, someone truly understands: ‘This is suffering’ … ‘This is the origin of suffering’ … ‘This is the cessation of suffering’ … ‘This is the practice that leads to the cessation of suffering’. This is called a person with a mind like lightning. 

And\marginnote{3.1} who has a mind like diamond? It’s someone who realizes the undefiled freedom of heart and freedom by wisdom in this very life. And they live having realized it with their own insight due to the ending of defilements. It’s like a diamond, which can’t be cut by anything at all, not even a gem or a stone. In the same way, someone realizes the undefiled freedom of heart and freedom by wisdom in this very life. … This is called a person with a mind like diamond. 

These\marginnote{3.6} are the three people found in the world.” 

%
\section*{{\suttatitleacronym AN 3.26}{\suttatitletranslation Associates }{\suttatitleroot Sevitabbasutta}}
\addcontentsline{toc}{section}{\tocacronym{AN 3.26} \toctranslation{Associates } \tocroot{Sevitabbasutta}}
\markboth{Associates }{Sevitabbasutta}
\extramarks{AN 3.26}{AN 3.26}

“These\marginnote{1.1} three people are found in the world. What three? There is a person you shouldn’t associate with, accompany, or attend. There is a person you should associate with, accompany, and attend. There is a person you should associate with, accompany, and attend with honor and respect. 

Who\marginnote{1.6} is the person you shouldn’t associate with, accompany, or attend? It’s someone who is inferior in terms of ethics, immersion, and wisdom. You shouldn’t associate with, accompany, or attend such a person, except out of kindness and compassion. 

Who\marginnote{2.1} is the person you should associate with, accompany, and attend? It’s someone who is similar in terms of ethics, immersion, and wisdom. You should associate with, accompany, and attend such a person. Why is that? Thinking, ‘Since our ethical conduct is similar, we can discuss ethics, the conversation will flow, and we’ll both be at ease. Since our immersion is similar, we can discuss immersion, the conversation will flow, and we’ll both be at ease. Since our wisdom is similar, we can discuss wisdom, the conversation will flow, and we’ll both be at ease.’ That’s why you should associate with, accompany, and attend such a person. 

Who\marginnote{3.1} is the person you should associate with, accompany, and attend with honor and respect? It’s someone who is superior in terms of ethics, immersion, and wisdom. You should associate with, accompany, and attend such a person with honor and respect. Why is that? Thinking, ‘I’ll fulfill the entire spectrum of ethical conduct I haven’t yet fulfilled, or support with wisdom in every situation the ethical conduct I’ve already fulfilled. I’ll fulfill the entire spectrum of immersion I haven’t yet fulfilled, or support with wisdom in every situation the immersion I’ve already fulfilled. I’ll fulfill the entire spectrum of wisdom I haven’t yet fulfilled, or support with wisdom in every situation the wisdom I’ve already fulfilled.’ That’s why you should associate with, accompany, and attend this person with honor and respect. 

These\marginnote{3.9} are the three people found in the world. 

\begin{verse}%
A\marginnote{4.1} man who associates with an inferior goes downhill, \\
but associating with an equal, you’ll never decline; \\
following the best, you’ll quickly rise up, \\
so you should keep company with people better than you.” 

%
\end{verse}

%
\section*{{\suttatitleacronym AN 3.27}{\suttatitletranslation You Should be Disgusted }{\suttatitleroot Jigucchitabbasutta}}
\addcontentsline{toc}{section}{\tocacronym{AN 3.27} \toctranslation{You Should be Disgusted } \tocroot{Jigucchitabbasutta}}
\markboth{You Should be Disgusted }{Jigucchitabbasutta}
\extramarks{AN 3.27}{AN 3.27}

“These\marginnote{1.1} three people are found in the world. What three? There is a person who you should be disgusted by, and you shouldn’t associate with, accompany, or attend them. There is a person you should regard with equanimity, and you shouldn’t associate with, accompany, or attend them. There is a person you should associate with, accompany, and attend. 

Who\marginnote{1.6} is the person you should be disgusted by, and not associate with, accompany, or attend? It’s a person who is unethical, of bad qualities, filthy, with suspicious behavior, underhand, no true ascetic or spiritual practitioner—though claiming to be one—rotten inside, corrupt, and depraved. You should be disgusted by such a person, and you shouldn’t associate with, accompany, or attend them. Why is that? Even if you don’t follow the example of such a person, you still get a bad reputation: ‘That individual has bad friends, companions, and associates.’ They’re like a snake that’s been living in a pile of dung. Even if it doesn’t bite, it’ll still rub off on you. In the same way, even if you don’t follow the example of such a person, you still get a bad reputation: ‘That individual has bad friends, companions, and associates.’ That’s why you should be disgusted by such a person, and you shouldn’t associate with, accompany, or attend them. 

Who\marginnote{2.1} is the person you should regard with equanimity, and not associate with, accompany, or attend? It’s a person who is irritable and bad-tempered. Even when lightly criticized they lose their temper, becoming annoyed, hostile, and hard-hearted, and they display annoyance, hate, and bitterness. They’re like a festering sore, which, when you hit it with a stick or a stone, discharges even more. In the same way, someone is irritable and bad-tempered. They’re like a firebrand of pale-moon ebony, which, when you hit it with a stick or a stone, sizzles and crackles even more. In the same way, someone is irritable and bad-tempered. They’re like a sewer, which, when you stir it with a stick or a stone, stinks even more. In the same way, someone is irritable and bad-tempered. Even when lightly criticized they lose their temper, becoming annoyed, hostile, and hard-hearted, and they display annoyance, hate, and bitterness. You should regard such a person with equanimity, and you shouldn’t associate with, accompany, or attend them. Why is that? Thinking, ‘They might abuse or insult me, or do me harm.’ That’s why you should regard such a person with equanimity, and you shouldn’t associate with, accompany, or attend them. 

Who\marginnote{3.1} is the person you should associate with, accompany, and attend? It’s someone who is ethical, of good character. You should associate with, accompany, and attend such a person. Why is that? Even if you don’t follow the example of such a person, you still get a good reputation: ‘That individual has good friends, companions, and associates.’ That’s why you should associate with, accompany, and attend such a person. 

These\marginnote{3.8} are the three people found in the world. 

\begin{verse}%
A\marginnote{4.1} man who associates with an inferior goes downhill, \\
but associating with an equal, you’ll never decline; \\
following the best, you’ll quickly rise up, \\
so you should keep company with people better than you.” 

%
\end{verse}

%
\section*{{\suttatitleacronym AN 3.28}{\suttatitletranslation Speech like Dung }{\suttatitleroot Gūthabhāṇīsutta}}
\addcontentsline{toc}{section}{\tocacronym{AN 3.28} \toctranslation{Speech like Dung } \tocroot{Gūthabhāṇīsutta}}
\markboth{Speech like Dung }{Gūthabhāṇīsutta}
\extramarks{AN 3.28}{AN 3.28}

“These\marginnote{1.1} three kinds of people are found in the world. What three? One with speech like dung, one with speech like flowers, and one with speech like honey. 

And\marginnote{1.4} who has speech like dung? It’s someone who is summoned to a council, an assembly, a family meeting, a guild, or to the royal court, and asked to bear witness: ‘Please, mister, say what you know.’ Not knowing, they say ‘I know.’ Knowing, they say ‘I don’t know.’ Not seeing, they say ‘I see.’ And seeing, they say ‘I don’t see.’ So they deliberately lie for the sake of themselves or another, or for some trivial worldly reason. This is called a person with speech like dung. 

And\marginnote{2.1} who has speech like flowers? It’s someone who is summoned to a council, an assembly, a family meeting, a guild, or to the royal court, and asked to bear witness: ‘Please, mister, say what you know.’ Not knowing, they say ‘I don’t know.’ Knowing, they say ‘I know.’ Not seeing, they say ‘I don’t see.’ And seeing, they say ‘I see.’ So they don’t deliberately lie for the sake of themselves or another, or for some trivial worldly reason. This is called a person with speech like flowers. 

And\marginnote{3.1} who has speech like honey? It’s someone who gives up harsh speech. They speak in a way that’s mellow, pleasing to the ear, lovely, going to the heart, polite, likable and agreeable to the people. This is called a person with speech like honey. 

These\marginnote{3.5} are the three people found in the world.” 

%
\section*{{\suttatitleacronym AN 3.29}{\suttatitletranslation Blind }{\suttatitleroot Andhasutta}}
\addcontentsline{toc}{section}{\tocacronym{AN 3.29} \toctranslation{Blind } \tocroot{Andhasutta}}
\markboth{Blind }{Andhasutta}
\extramarks{AN 3.29}{AN 3.29}

“These\marginnote{1.1} three kinds of people are found in the world. What three? The blind, the one-eyed, and the two-eyed. 

Who\marginnote{1.4} is the blind person? It’s someone who doesn’t have the kind of vision that’s needed to acquire more wealth or to increase the wealth they’ve already acquired. Nor do they have the kind of vision that’s needed to know the difference between qualities that are skillful and unskillful, blameworthy and blameless, inferior and superior, and those on the side of dark and bright. This is called a blind person. 

Who\marginnote{2.1} is the person with one eye? It’s someone who has the kind of vision that’s needed to acquire more wealth and to increase the wealth they’ve already acquired. But they don’t have the kind of vision that’s needed to know the difference between qualities that are skillful and unskillful, blameworthy and blameless, inferior and superior, and those on the side of dark and bright. This is called a one-eyed person. 

Who\marginnote{3.1} is the person with two eyes? It’s someone who has the kind of vision that’s needed to acquire more wealth and to increase the wealth they’ve already acquired. And they have the kind of vision that’s needed to know the difference between skillful and unskillful, blameworthy and blameless, inferior and superior, or qualities on the side of dark and bright. This is called a two-eyed person. 

These\marginnote{3.6} are the three people found in the world. 

\begin{verse}%
Neither\marginnote{4.1} suitable wealth, \\
nor merit do they make. \\
They lose on both counts, \\
those who are blind, with ruined eyes. 

And\marginnote{5.1} now the one-eyed \\
person is explained. \\
By methods good and bad, \\
that devious person seeks wealth. 

Both\marginnote{6.1} by fraudulent, thieving deeds, \\
and also by lies, \\
the young man’s skilled at piling up money, \\
and enjoying sensual pleasures. \\
From here they go to hell—\\
the one-eyed person is ruined. 

And\marginnote{7.1} now the two-eyed is explained, \\
the best individual. \\
Their wealth is earned legitimately, \\
money acquired by their own hard work. 

They\marginnote{8.1} give with best of intentions, \\
that peaceful-hearted person. \\
They go to a good place, \\
where there is no sorrow. 

The\marginnote{9.1} blind and the one-eyed, \\
you should avoid from afar. \\
But you should keep the two-eyed close, \\
the best individual.” 

%
\end{verse}

%
\section*{{\suttatitleacronym AN 3.30}{\suttatitletranslation Upside-down }{\suttatitleroot Avakujjasutta}}
\addcontentsline{toc}{section}{\tocacronym{AN 3.30} \toctranslation{Upside-down } \tocroot{Avakujjasutta}}
\markboth{Upside-down }{Avakujjasutta}
\extramarks{AN 3.30}{AN 3.30}

“These\marginnote{1.1} three kinds of people are found in the world. What three? One with upside-down wisdom, one with wisdom on their lap, and one with wide wisdom. 

And\marginnote{1.4} who is the person with upside-down wisdom? It’s someone who often goes to the monastery to hear the teaching in the presence of the mendicants. The mendicants teach them Dhamma that’s good in the beginning, good in the middle, and good in the end, meaningful and well-phrased. And they reveal a spiritual practice that’s entirely full and pure. But even while sitting there, that person doesn’t pay attention to the beginning, middle, or end of the discussion. And when they get up from their seat, they don’t pay attention to the beginning, middle, or end of the discussion. It’s like when a pot full of water is tipped over, so the water drains out and doesn’t stay. In the same way, someone often goes to the monastery to hear the teaching in the presence of the mendicants. The mendicants teach them Dhamma that’s good in the beginning, good in the middle, and good in the end, meaningful and well-phrased. And they reveal a spiritual practice that’s entirely full and pure. But even while sitting there, that person doesn’t pay attention to the discussion in the beginning, middle, or end. And when they get up from their seat, they don’t pay attention to the beginning, middle, or end of the discussion. This is called a person with upside-down wisdom. 

And\marginnote{2.1} who is the person with wisdom on their lap? It’s someone who often goes to the monastery to hear the teaching in the presence of the mendicants. The mendicants teach them Dhamma that’s good in the beginning, good in the middle, and good in the end, meaningful and well-phrased. And they reveal a spiritual practice that’s entirely full and pure. While sitting there, that person pays attention to the discussion in the beginning, middle, and end. But when they get up from their seat, they don’t pay attention to the beginning, middle, or end of the discussion. It’s like a person who has different kinds of food crammed on their lap—such as sesame, rice, sweets, or jujube—so that if they get up from the seat without mindfulness, everything gets scattered. In the same way, someone often goes to the monastery to hear the teaching in the presence of the mendicants. The mendicants teach them Dhamma that’s good in the beginning, good in the middle, and good in the end, meaningful and well-phrased. And they reveal a spiritual practice that’s entirely full and pure. While sitting there, that person pays attention to the discussion in the beginning, middle, and end. But when they get up from their seat, they don’t pay attention to the beginning, middle, or end of the discussion. This is called a person with wisdom on their lap. 

And\marginnote{3.1} who is the person with wide wisdom? It’s someone who often goes to the monastery to hear the teaching in the presence of the mendicants. The mendicants teach them Dhamma that’s good in the beginning, good in the middle, and good in the end, meaningful and well-phrased. And they reveal a spiritual practice that’s entirely full and pure. While sitting there, that person pays attention to the discussion in the beginning, middle, and end. And when they get up from their seat, they continue to pay attention to the beginning, middle, or end of the discussion. It’s like when a pot full of water is set straight, so the water stays and doesn’t drain out. In the same way, someone often goes to the monastery to hear the teaching in the presence of the mendicants. The mendicants teach them Dhamma that’s good in the beginning, good in the middle, and good in the end, meaningful and well-phrased. And they reveal a spiritual practice that’s entirely full and pure. While sitting there, that person pays attention to the discussion in the beginning, middle, and end. And when they get up from their seat, they continue to pay attention to the beginning, middle, or end of the discussion. This is called a person with wide wisdom. 

These\marginnote{3.12} are the three kinds of people found in the world. 

\begin{verse}%
A\marginnote{4.1} person with upside-down wisdom, \\
is stupid and cannot see, \\
and even if they frequently \\
go into the mendicants’ presence, 

such\marginnote{5.1} a person can’t learn \\
the beginning, middle, or end \\
of the discussion, \\
for their wisdom is lacking. 

The\marginnote{6.1} person with wisdom on their lap \\
is better than that, it’s said; \\
but even if they frequently \\
go into the mendicants’ presence, 

such\marginnote{7.1} a person can only learn \\
the beginning, middle, and end \\
while sitting in that seat; \\
but they’ve only grasped the phrasing, \\
for when they get up their understanding fails, \\
and what they’ve learned is lost. 

The\marginnote{8.1} person with wide wisdom \\
is better than that, it’s said; \\
and if they, too, frequently \\
go into the mendicants’ presence, 

such\marginnote{9.1} a person can learn \\
the beginning, middle, and end \\
while sitting in that seat; \\
and when they’ve grasped the phrasing, 

they\marginnote{10.1} remember it with the best of intentions. \\
That peaceful-hearted person, \\
practicing in line with the teaching, \\
would make an end of suffering.” 

%
\end{verse}

%
\addtocontents{toc}{\let\protect\contentsline\protect\nopagecontentsline}
\chapter*{The Chapter on Messengers of the Gods }
\addcontentsline{toc}{chapter}{\tocchapterline{The Chapter on Messengers of the Gods }}
\addtocontents{toc}{\let\protect\contentsline\protect\oldcontentsline}

%
\section*{{\suttatitleacronym AN 3.31}{\suttatitletranslation With Brahmā }{\suttatitleroot Sabrahmakasutta}}
\addcontentsline{toc}{section}{\tocacronym{AN 3.31} \toctranslation{With Brahmā } \tocroot{Sabrahmakasutta}}
\markboth{With Brahmā }{Sabrahmakasutta}
\extramarks{AN 3.31}{AN 3.31}

“Mendicants,\marginnote{1.1} a family where the children honor their parents in their home is said to live with \textsanskrit{Brahmā}. A family where the children honor their parents in their home is said to live with the first teachers. A family where the children honor their parents in their home is said to live with those worthy of offerings dedicated to the gods. 

‘\textsanskrit{Brahmā}’\marginnote{1.4} is a term for your parents. 

‘First\marginnote{1.5} teachers’ is a term for your parents. 

‘Worthy\marginnote{1.6} of offerings dedicated to the gods’ is a term for your parents. 

Why\marginnote{1.7} is that? Parents are very helpful to their children, they raise them, nurture them, and show them the world. 

\begin{verse}%
Parents\marginnote{2.1} are said to be ‘\textsanskrit{Brahmā}’ \\
and ‘first teachers’, it’s said. \\
They’re worthy of offerings dedicated to the gods from their children, \\
for they love their offspring. 

Therefore\marginnote{3.1} an astute person \\
would revere them and honor them \\
with food and drink, \\
clothes and bedding, \\
anointing and bathing, \\
and by washing their feet. 

Because\marginnote{4.1} they look after \\
their parents like this, \\
in this life they’re praised by the astute, \\
and they depart to rejoice in heaven.” 

%
\end{verse}

%
\section*{{\suttatitleacronym AN 3.32}{\suttatitletranslation With Ānanda }{\suttatitleroot Ānandasutta}}
\addcontentsline{toc}{section}{\tocacronym{AN 3.32} \toctranslation{With Ānanda } \tocroot{Ānandasutta}}
\markboth{With Ānanda }{Ānandasutta}
\extramarks{AN 3.32}{AN 3.32}

Then\marginnote{1.1} Venerable Ānanda went up to the Buddha, bowed, sat down to one side, and said to the Buddha: 

“Could\marginnote{2.1} it be, sir, that a mendicant might gain a state of immersion such that there’s no ego, possessiveness, or underlying tendency to conceit for this conscious body; and no ego, possessiveness, or underlying tendency to conceit for all external stimuli; and that they’d live having attained the freedom of heart and freedom by wisdom where ego, possessiveness, and underlying tendency to conceit are no more?” 

“It\marginnote{2.3} could be, Ānanda, that a mendicant gains a state of immersion such that they have no ego, possessiveness, or underlying tendency to conceit for this conscious body; and no ego, possessiveness, or underlying tendency to conceit for all external stimuli; and that they’d live having attained the freedom of heart and freedom by wisdom where ego, possessiveness, and underlying tendency to conceit are no more.” 

“But\marginnote{3.1} how could this be, sir?” 

“Ānanda,\marginnote{4.1} it’s when a mendicant thinks: ‘This is peaceful; this is sublime—that is, the stilling of all activities, the letting go of all attachments, the ending of craving, fading away, cessation, extinguishment.’ 

That’s\marginnote{4.3} how, Ānanda, a mendicant might gain a state of immersion such that there’s no ego, possessiveness, or underlying tendency to conceit for this conscious body; and no ego, possessiveness, or underlying tendency to conceit for all external stimuli; and that they’d live having achieved the freedom of heart and freedom by wisdom where ego, possessiveness, and underlying tendency to conceit are no more. 

And\marginnote{5.1} Ānanda, this is what I was referring to in ‘The Way to the Beyond’, in ‘The Questions of \textsanskrit{Puṇṇaka}’ when I said: 

\begin{verse}%
‘Having\marginnote{6.1} assessed the world high and low, \\
there is nothing in the world that disturbs them. \\
Peaceful, unclouded, untroubled, with no need for hope—\\
they’ve crossed over rebirth and old age, I declare.’” 

%
\end{verse}

%
\section*{{\suttatitleacronym AN 3.33}{\suttatitletranslation With Sāriputta }{\suttatitleroot Sāriputtasutta}}
\addcontentsline{toc}{section}{\tocacronym{AN 3.33} \toctranslation{With Sāriputta } \tocroot{Sāriputtasutta}}
\markboth{With Sāriputta }{Sāriputtasutta}
\extramarks{AN 3.33}{AN 3.33}

Then\marginnote{1.1} Venerable \textsanskrit{Sāriputta} went up to the Buddha, bowed, and sat down to one side. The Buddha said to him, “Maybe I’ll teach Dhamma in brief, maybe in detail, maybe both in brief and in detail. But it’s hard to find anyone who understands.” 

“Now\marginnote{1.6} is the time, Blessed One! Now is the time, Holy One! Let the Buddha teach Dhamma in brief, in detail, and both in brief and in detail. There will be those who understand the teaching!” 

“So,\marginnote{2.1} \textsanskrit{Sāriputta}, you should train like this: ‘There’ll be no ego, possessiveness, or underlying tendency to conceit for this conscious body; and no ego, possessiveness, or underlying tendency to conceit for all external stimuli; and we’ll live having achieved the freedom of heart and freedom by wisdom where ego, possessiveness, and underlying tendency to conceit are no more.’ That’s how you should train. 

When\marginnote{3.1} a mendicant has no ego, possessiveness, or underlying tendency to conceit for this conscious body; and no ego, possessiveness, or underlying tendency to conceit for all external stimuli; and they live having attained the freedom of heart and freedom by wisdom where ego, possessiveness, and underlying tendency to conceit are no more—they’re called a mendicant who has cut off craving, untied the fetters, and by rightly comprehending conceit has made an end of suffering. 

And\marginnote{3.4} \textsanskrit{Sāriputta}, this is what I was referring to in ‘The Way to the Beyond’, in ‘The Questions of Udaya’ when I said: 

\begin{verse}%
‘The\marginnote{4.1} giving up of sensual desires \\
and aversions, both; \\
the dispelling of dullness, \\
and the cessation of remorse. 

Pure\marginnote{5.1} equanimity and mindfulness, \\
preceded by investigation of principles—\\
this, I declare, is liberation by enlightenment, \\
the smashing of ignorance.’” 

%
\end{verse}

%
\section*{{\suttatitleacronym AN 3.34}{\suttatitletranslation Sources }{\suttatitleroot Nidānasutta}}
\addcontentsline{toc}{section}{\tocacronym{AN 3.34} \toctranslation{Sources } \tocroot{Nidānasutta}}
\markboth{Sources }{Nidānasutta}
\extramarks{AN 3.34}{AN 3.34}

“Mendicants,\marginnote{1.1} there are these three sources that give rise to deeds. What three? Greed, hate, and delusion are sources that give rise to deeds. 

Any\marginnote{2.1} deed that emerges from greed—born, sourced, and originated from greed—ripens where that new life-form is born. And wherever that deed ripens, its result is experienced—either in the present life, or in the next life, or in some subsequent period. 

Any\marginnote{3.1} deed that emerges from hate—born, sourced, and originated from hate—ripens where that new life-form is born. And wherever that deed ripens, its result is experienced—either in the present life, or in the next life, or in some subsequent period. 

Any\marginnote{4.1} deed that emerges from delusion—born, sourced, and originated from delusion—ripens where that new life-form is born. And wherever that deed ripens, its result is experienced—either in the present life, or in the next life, or in some subsequent period. 

Suppose\marginnote{5.1} some seeds were intact, unspoiled, not weather-damaged, fertile, and well-kept. They’re sown in a well-prepared, productive field, and the heavens provide plenty of rain. Then those seeds would grow, increase, and mature. 

In\marginnote{5.4} the same way, any deed that emerges from greed—born, sourced, and originated from greed—ripens where that new life-form is born. And wherever that deed ripens, its result is experienced—either in the present life, or in the next life, or in some subsequent period. 

Any\marginnote{6.1} deed that emerges from hate … 

Any\marginnote{6.2} deed that emerges from delusion—born, sourced, and originated from delusion—ripens where that new life-form is born. And wherever that deed ripens, its result is experienced—either in the present life, or in the next life, or in some subsequent period. These are three sources that give rise to deeds. 

Mendicants,\marginnote{7.1} there are these three sources that give rise to deeds. What three? Contentment, love, and understanding are sources that give rise to deeds. 

Any\marginnote{8.1} deed that emerges from contentment—born, sourced, and originated from contentment—is given up when greed is done away with. It’s cut off at the root, made like a palm stump, obliterated, and unable to arise in the future. 

Any\marginnote{9.1} deed that emerges from love—born, sourced, and originated from love—is abandoned when hate is done away with. It’s cut off at the root, made like a palm stump, obliterated, and unable to arise in the future. 

Any\marginnote{10.1} deed that emerges from understanding—born, sourced, and originated from understanding—is abandoned when delusion is done away with. It’s cut off at the root, made like a palm stump, obliterated, and unable to arise in the future. 

Suppose\marginnote{11.1} some seeds were intact, unspoiled, not damaged by wind and sun, fertile, and well-kept. But someone would burn them with fire, reduce them to ashes, and sweep away the ashes in a strong wind, or float them away down a swift stream. Then those seeds would be cut off at the root, made like a palm stump, obliterated, and unable to arise in the future. 

In\marginnote{11.6} the same way, any deed that emerges from contentment—born, sourced, and originated from contentment—is abandoned when greed is done away with. It’s cut off at the root, made like a palm stump, obliterated, and unable to arise in the future. 

Any\marginnote{12.1} deed that emerges from love … Any deed that emerges from understanding—born, sourced, and originated from understanding—is abandoned when delusion is done away with. It’s cut off at the root, made like a palm stump, obliterated, and unable to arise in the future. 

These\marginnote{12.4} are three sources that give rise to deeds. 

\begin{verse}%
When\marginnote{13.1} an ignorant person acts \\
out of greed, hate, or delusion, \\
any deeds they have performed \\
—whether a little or a lot—\\
are to be experienced right here, \\
not in any other place. 

So\marginnote{14.1} a wise person, \\
a mendicant arousing knowledge \\
of the outcome of greed, hate, and delusion, \\
would cast off all bad destinies.” 

%
\end{verse}

%
\section*{{\suttatitleacronym AN 3.35}{\suttatitletranslation With Hatthaka }{\suttatitleroot Hatthakasutta}}
\addcontentsline{toc}{section}{\tocacronym{AN 3.35} \toctranslation{With Hatthaka } \tocroot{Hatthakasutta}}
\markboth{With Hatthaka }{Hatthakasutta}
\extramarks{AN 3.35}{AN 3.35}

\scevam{So\marginnote{1.1} I have heard. }At one time the Buddha was staying near \textsanskrit{Āḷavī}, on a mat of leaves by a cow-path in a grove of Indian Rosewood. 

Then\marginnote{1.3} as Hatthaka of \textsanskrit{Āḷavī} was going for a walk he saw the Buddha sitting on that mat of leaves. He went up to the Buddha, bowed, sat down to one side, and said, “Sir, I trust the Buddha slept well?” 

“Yes,\marginnote{1.6} prince, I slept well. I am one of those who sleep at ease in the world.” 

“The\marginnote{2.1} winter nights are cold, sir, and it’s the week of mid-winter, when the snow falls. Rough is the ground trampled under the cows’ hooves, and thin is the mat of leaves. The leaves are sparse on the trees, the ocher robes are cold, and cold blows the north wind. And yet the Buddha says, ‘Yes, prince, I slept well. I am one of those who sleep at ease in the world.’” 

“Well\marginnote{3.1} then, prince, I’ll ask you about this in return, and you can answer as you like. What do you think? Take the case of a householder or his son, who lives in a bungalow, plastered inside and out, draft-free, with latches fastened and windows shuttered. His couch is spread with woolen covers—shag-piled, pure white, or embroidered with flowers—and spread with a fine deer hide. It has a canopy above and red pillows at both ends. An oil lamp is burning there, while his four wives attend to him in all manner of agreeable ways. What do you think, prince, would he sleep at ease, or not? Or how do you see this?” 

“He\marginnote{3.8} would sleep at ease, sir. Of those who sleep at ease in the world, he would be one.” 

“What\marginnote{4.1} do you think, prince? Is it not possible that a fever born of greed—physical or mental—might arise in that householder or householder’s son, burning him so he sleeps badly?” 

“Yes,\marginnote{4.3} sir.” 

“The\marginnote{5.1} greed that burns that householder or householder’s son, making them sleep badly, has been cut off at the root by the Realized One, made like a palm stump, obliterated, and unable to arise in the future. That’s why I sleep at ease. 

What\marginnote{6.1} do you think, prince? Is it not possible that a fever born of hate … or a fever born of delusion—physical or mental—might arise in that householder or householder’s son, burning him so he sleeps badly?” 

“Yes,\marginnote{6.4} sir.” 

“The\marginnote{7.1} delusion that burns that householder or householder’s son, making them sleep badly, has been cut off at the root by the Realized One, made like a palm stump, obliterated, and unable to arise in the future. That’s why I sleep at ease. 

\begin{verse}%
A\marginnote{8.1} brahmin who is fully extinguished \\
always sleeps at ease. \\
Sensual pleasures slip off them, \\
they’re cooled, free of attachments. 

Since\marginnote{9.1} they’ve cut off all clinging, \\
and removed the stress from the heart, \\
the peaceful sleep at ease, \\
having found peace of mind.” 

%
\end{verse}

%
\section*{{\suttatitleacronym AN 3.36}{\suttatitletranslation Messengers of the Gods }{\suttatitleroot Devadūtasutta}}
\addcontentsline{toc}{section}{\tocacronym{AN 3.36} \toctranslation{Messengers of the Gods } \tocroot{Devadūtasutta}}
\markboth{Messengers of the Gods }{Devadūtasutta}
\extramarks{AN 3.36}{AN 3.36}

“There\marginnote{1.1} are, mendicants, these three messengers of the gods. What three? 

Firstly,\marginnote{1.3} someone does bad things by way of body, speech, and mind. When their body breaks up, after death, they’re reborn in a place of loss, a bad place, the underworld, hell. Then the wardens of hell take them by the arms and present them to King Yama, saying: ‘Your Majesty, this person did not pay due respect to their mother and father, ascetics and brahmins, or honor the elders in the family. May Your Majesty punish them!’ 

Then\marginnote{2.1} King Yama pursues, presses, and grills them about the first messenger of the gods: ‘Mister, did you not see the first messenger of the gods that appeared among human beings?’ 

They\marginnote{2.3} say, ‘I saw nothing, sir.’ 

Then\marginnote{3.1} King Yama says, ‘Mister, did you not see among human beings an elderly woman or a man—eighty, ninety, or a hundred years old—bent double, crooked, leaning on a staff, trembling as they walk, ailing, past their prime, with teeth broken, hair grey and scanty or bald, skin wrinkled, and limbs blotchy?’ 

They\marginnote{3.3} say, ‘I saw that, sir.’ 

Then\marginnote{4.1} King Yama says, ‘Mister, did it not occur to you—being sensible and mature—“I, too, am liable to grow old. I’m not exempt from old age. I’d better do good by way of body, speech, and mind”?’ 

They\marginnote{4.4} say, ‘I couldn’t, sir. I was negligent.’ 

Then\marginnote{5.1} King Yama says, ‘Mister, because you were negligent, you didn’t do good by way of body, speech, and mind. Indeed, they’ll definitely punish you to fit your negligence. That bad deed wasn’t done by your mother, father, brother, or sister. It wasn’t done by friends and colleagues, by relatives and kin, by the deities, or by ascetics and brahmins. That bad deed was done by you alone, and you alone will experience the result.’ 

Then\marginnote{6.1} King Yama grills them about the second messenger of the gods: ‘Mister, did you not see the second messenger of the gods that appeared among human beings?’ 

They\marginnote{6.3} say, ‘I saw nothing, sir.’ Then King Yama says, ‘Mister, did you not see among human beings a woman or a man, sick, suffering, gravely ill, collapsed in their own urine and feces, being picked up by some and put down by others?’ 

They\marginnote{6.7} say, ‘I saw that, sir.’ 

Then\marginnote{7.1} King Yama says, ‘Mister, did it not occur to you—being sensible and mature—“I, too, am liable to become sick. I’m not exempt from sickness. I’d better do good by way of body, speech, and mind”?’ 

They\marginnote{7.4} say, ‘I couldn’t, sir. I was negligent.’ 

Then\marginnote{8.1} King Yama says, ‘Mister, because you were negligent, you didn’t do good by way of body, speech, and mind. Well, they’ll definitely punish you to fit your negligence. That bad deed wasn’t done by your mother, father, brother, or sister. It wasn’t done by friends and colleagues, by relatives and kin, by the deities, or by ascetics and brahmins. That bad deed was done by you alone, and you alone will experience the result.’ 

Then\marginnote{9.1} King Yama grills them about the third messenger of the gods: ‘Mister, did you not see the third messenger of the gods that appeared among human beings?’ 

They\marginnote{9.3} say, ‘I saw nothing, sir.’ 

Then\marginnote{10.1} King Yama says, ‘Mister, did you not see among human beings a woman or a man, dead for one, two, or three days, bloated, livid, and festering?’ 

They\marginnote{10.3} say, ‘I saw that, sir.’ 

Then\marginnote{11.1} King Yama says, ‘Mister, did it not occur to you—being sensible and mature—“I, too, am liable to die. I’m not exempt from death. I’d better do good by way of body, speech, and mind”?’ 

They\marginnote{11.4} say, ‘I couldn’t, sir. I was negligent.’ 

Then\marginnote{12.1} King Yama says, ‘Mister, because you were negligent, you didn’t do good by way of body, speech, and mind. Well, they’ll definitely punish you to fit your negligence. That bad deed wasn’t done by your mother, father, brother, or sister. It wasn’t done by friends and colleagues, by relatives and kin, by the deities, or by ascetics and brahmins. That bad deed was done by you alone, and you alone will experience the result.’ 

Then,\marginnote{13.1} after grilling them about the third messenger of the gods, King Yama falls silent. Then the wardens of hell punish them with the five-fold crucifixion. They drive red-hot stakes through the hands and feet, and another in the middle of the chest. And there they suffer painful, sharp, severe, acute feelings—but they don’t die until that bad deed is eliminated. 

Then\marginnote{14.1} the wardens of hell throw them down and hack them with axes. … 

They\marginnote{15.1} hang them upside-down and hack them with hatchets. … 

They\marginnote{15.2} harness them to a chariot, and drive them back and forth across burning ground, blazing and glowing. … 

They\marginnote{15.3} make them climb up and down a huge mountain of burning coals, blazing and glowing. … 

Then\marginnote{15.4} the wardens of hell turn them upside down and throw them in a red-hot copper pot, burning, blazing, and glowing. There they’re seared in boiling scum, and they’re swept up and down and round and round. And there they suffer painful, sharp, severe, acute feelings—but they don’t die until that bad deed is eliminated. Then the wardens of hell toss them into the Great Hell. 

Now,\marginnote{15.8} about that Great Hell: 

\begin{verse}%
‘Four\marginnote{16.1} are its corners, four its doors, \\
neatly divided in equal parts. \\
Surrounded by an iron wall, \\
of iron is its roof. 

The\marginnote{17.1} ground is even made of iron, \\
it burns with fierce fire. \\
The heat forever radiates \\
a hundred leagues around.’ 

%
\end{verse}

Once\marginnote{18.1} upon a time, King Yama thought, ‘Those who do such bad deeds in the world receive these many different punishments. Oh, I hope I may be reborn as a human being! And that a Realized One—a perfected one, a fully awakened Buddha—arises in the world! And that I may pay homage to the Buddha! Then the Buddha can teach me Dhamma, so that I may understand his teaching.’ 

Now,\marginnote{18.5} I don’t say this because I’ve heard it from some other ascetic or brahmin. I only say it because I’ve known, seen, and realized it for myself. 

\begin{verse}%
Those\marginnote{19.1} people who are negligent, \\
when warned by the gods’ messengers: \\
a long time they sorrow, \\
when they go to that wretched place. 

But\marginnote{20.1} those good and peaceful people, \\
when warned by the god’s messengers, \\
never neglect \\
the teaching of the noble ones. 

Seeing\marginnote{21.1} the peril in grasping, \\
the origin of birth and death, \\
the unattached are freed \\
with the ending of birth and death. 

Happy,\marginnote{22.1} they’ve come to a safe place, \\
extinguished in this very life. \\
They’ve gone beyond all threats and perils, \\
and risen above all suffering.” 

%
\end{verse}

%
\section*{{\suttatitleacronym AN 3.37}{\suttatitletranslation The Four Great Kings (1st) }{\suttatitleroot Catumahārājasutta}}
\addcontentsline{toc}{section}{\tocacronym{AN 3.37} \toctranslation{The Four Great Kings (1st) } \tocroot{Catumahārājasutta}}
\markboth{The Four Great Kings (1st) }{Catumahārājasutta}
\extramarks{AN 3.37}{AN 3.37}

“On\marginnote{1.1} the eighth day of the fortnight, mendicants, the ministers and counselors of the Four Great Kings wander about the world, thinking: ‘Hopefully most humans are paying due respect to their parents, ascetics and brahmins, honoring the elders in their families, observing and keeping vigil on the sabbath, and making merit.’ 

And\marginnote{1.3} on the fourteenth day of the fortnight, the sons of the Four Great Kings wander about the world, thinking: ‘Hopefully most humans are paying due respect to their parents … and making merit.’ 

And\marginnote{1.5} on the fifteenth day sabbath, the Four Great Kings themselves wander about the world, thinking: ‘Hopefully most humans are paying due respect to their parents … and making merit.’ 

If\marginnote{2.1} only a few humans are paying due respect to their parents … and making merit, then the Four Great Kings address the gods of the Thirty-Three, seated together in the Hall of Justice: ‘Only a few humans are paying due respect to their parents … and making merit.’ Then the gods of the Thirty-Three are disappointed, thinking, ‘The heavenly hosts will dwindle, while the demon hosts will swell!’ 

But\marginnote{3.1} if many humans are paying due respect to their parents … and making merit, then the Four Great Kings address the gods of the Thirty-Three, seated together in the Hall of Justice: ‘Many humans are paying due respect to their parents … and making merit.’ Then the gods of the Thirty-Three are pleased, thinking, ‘The heavenly hosts will swell, while the demon hosts will dwindle!’ 

Once\marginnote{4.1} upon a time, Sakka, lord of gods, guiding the gods of the Thirty-Three, recited this verse: 

\begin{verse}%
‘Whoever\marginnote{5.1} wants to be like me \\
would observe the sabbath \\
complete in all eight factors, \\
on the fourteenth and the fifteenth days, \\
and the eighth day of the fortnight, \\
as well as on the fortnightly special displays.’ 

%
\end{verse}

But\marginnote{6.1} that verse was poorly sung by Sakka, lord of gods, not well sung; poorly spoken, not well spoken. Why is that? Sakka, lord of gods, is not free of greed, hate, and delusion. 

But\marginnote{7.1} for a mendicant who is perfected—with defilements ended, who has completed the spiritual journey, done what had to be done, laid down the burden, achieved their own true goal, utterly ended the fetters of rebirth, and is rightly freed through enlightenment—it is appropriate to say: 

\begin{verse}%
‘Whoever\marginnote{8.1} wants to be like me \\
would observe the sabbath, \\
complete in all eight factors, \\
on the fourteenth and the fifteenth days, \\
and the eighth day of the fortnight, \\
as well as on the fortnightly special displays.’ 

%
\end{verse}

Why\marginnote{9.1} is that? Because that mendicant is free of greed, hate, and delusion.” 

%
\section*{{\suttatitleacronym AN 3.38}{\suttatitletranslation The Four Great Kings (2nd) }{\suttatitleroot Dutiyacatumahārājasutta}}
\addcontentsline{toc}{section}{\tocacronym{AN 3.38} \toctranslation{The Four Great Kings (2nd) } \tocroot{Dutiyacatumahārājasutta}}
\markboth{The Four Great Kings (2nd) }{Dutiyacatumahārājasutta}
\extramarks{AN 3.38}{AN 3.38}

“Once\marginnote{1.1} upon a time, mendicants, Sakka, lord of gods, guiding the gods of the Thirty-Three, recited this verse: 

\begin{verse}%
‘Whoever\marginnote{2.1} wants to be like me \\
would observe the sabbath \\
complete in all eight factors, \\
on the fourteenth and the fifteenth days, \\
and the eighth day of the fortnight, \\
as well as on the fortnightly special displays.’ 

%
\end{verse}

But\marginnote{3.1} that verse was poorly sung by Sakka, lord of gods, not well sung; poorly spoken, not well spoken. Why is that? Because Sakka, lord of gods, is not exempt from rebirth, old age, and death, from sorrow, lamentation, pain, sadness, and distress. He is not exempt from suffering, I say. 

But\marginnote{4.1} for a mendicant who is perfected—with defilements ended, who has completed the spiritual journey, done what had to be done, laid down the burden, achieved their own true goal, utterly ended the fetters of rebirth, and is rightly freed through enlightenment—it is appropriate to say: 

\begin{verse}%
‘Whoever\marginnote{5.1} wants to be like me \\
would observe the sabbath, \\
complete in all eight factors, \\
on the fourteenth and the fifteenth days, \\
and the eighth day of the fortnight, \\
as well as on the fortnightly special displays.’ 

%
\end{verse}

Why\marginnote{6.1} is that? Because that mendicant is exempt from rebirth, old age, and death, from sorrow, lamentation, pain, sadness, and distress. He is exempt from suffering, I say.” 

%
\section*{{\suttatitleacronym AN 3.39}{\suttatitletranslation A Delicate Lifestyle }{\suttatitleroot Sukhumālasutta}}
\addcontentsline{toc}{section}{\tocacronym{AN 3.39} \toctranslation{A Delicate Lifestyle } \tocroot{Sukhumālasutta}}
\markboth{A Delicate Lifestyle }{Sukhumālasutta}
\extramarks{AN 3.39}{AN 3.39}

“My\marginnote{1.1} lifestyle was delicate, mendicants, most delicate, extremely delicate. 

In\marginnote{1.2} my father’s home, lotus ponds were made just for me. In some, blue water lilies blossomed, while in others, there were pink or white lotuses, just for my benefit. I only used sandalwood from \textsanskrit{Kāsī}, and my turbans, jackets, sarongs, and upper robes also came from \textsanskrit{Kāsī}. And a white parasol was held over me night and day, with the thought: ‘Don’t let cold, heat, grass, dust, or damp bother him.’ 

I\marginnote{2.1} had three stilt longhouses—one for the winter, one for the summer, and one for the rainy season. I stayed in a stilt longhouse without coming downstairs for the four months of the rainy season, where I was entertained by musicians—none of them men. 

While\marginnote{2.3} the bondservants, workers, and staff in other houses are given rough gruel with pickles to eat, in my father’s home they eat fine rice with meat. 

Amid\marginnote{3.1} such prosperity and such a delicate lifestyle, I thought: ‘When an uneducated ordinary person—who is liable to grow old, not being exempt from old age—sees someone else who is old, they’re horrified, repelled, and disgusted, overlooking the fact that they themselves are in the same situation. But since I, too, am liable to grow old, it would not be appropriate for me to be horrified, embarrassed, and disgusted, when I see someone else who is old.’ Reflecting like this, I entirely gave up the vanity of youth. 

‘When\marginnote{4.1} an uneducated ordinary person—who is liable to get sick, not being exempt from sickness—sees someone else who is sick, they’re horrified, repelled, and disgusted, overlooking the fact that they themselves are in the same situation. But since I, too, am liable to get sick, it would not be appropriate for me to be horrified, embarrassed, and disgusted, when I see someone else who is sick.’ Reflecting like this, I entirely gave up the vanity of health. 

‘When\marginnote{5.1} an uneducated ordinary person—who is liable to die, not being exempt from death—sees someone else who is dead, they’re horrified, repelled, and disgusted, overlooking the fact that they themselves are in the same situation. But since I, too, am liable to die, it would not be appropriate for me to be horrified, embarrassed, and disgusted, when I see someone else who is dead.’ Reflecting like this, I entirely gave up the vanity of life. 

There\marginnote{6.1} are these three vanities. What three? The vanity of youth, of health, and of life. 

Intoxicated\marginnote{6.4} with the vanity of youth, an uneducated ordinary person does bad things by way of body, speech, and mind. When their body breaks up, after death, they’re reborn in a place of loss, a bad place, the underworld, hell. 

Intoxicated\marginnote{6.6} with the vanity of health … 

Intoxicated\marginnote{6.7} with the vanity of life, an uneducated ordinary person does bad things by way of body, speech, and mind. When their body breaks up, after death, they’re reborn in a place of loss, a bad place, the underworld, hell. 

Intoxicated\marginnote{7.1} with the vanity of youth, health, or life, a mendicant resigns the training and returns to a lesser life. 

\begin{verse}%
For\marginnote{8.1} others, sickness is natural, \\
and so are old age and death. \\
Though this is how their nature is, \\
ordinary people feel disgusted. 

If\marginnote{9.1} I were to be disgusted \\
with creatures whose nature is such, \\
it would not be appropriate for me, \\
since my life is just the same. 

Living\marginnote{10.1} in such a way, \\
I understood the reality without attachments. \\
I mastered all vanities—\\
of health, of youth, 

and\marginnote{11.1} even of life—\\
seeing renunciation as sanctuary. \\
Zeal sprang up in me \\
as I looked to extinguishment. 

Now\marginnote{12.1} I’m unable \\
to indulge in sensual pleasures; \\
there’s no turning back, \\
I’m committed to the spiritual life.” 

%
\end{verse}

%
\section*{{\suttatitleacronym AN 3.40}{\suttatitletranslation In Charge }{\suttatitleroot Ādhipateyyasutta}}
\addcontentsline{toc}{section}{\tocacronym{AN 3.40} \toctranslation{In Charge } \tocroot{Ādhipateyyasutta}}
\markboth{In Charge }{Ādhipateyyasutta}
\extramarks{AN 3.40}{AN 3.40}

“There\marginnote{1.1} are, mendicants, these three things to put in charge. What three? Putting oneself, the world, or the teaching in charge. 

And\marginnote{1.4} what, mendicants, is putting oneself in charge? It’s when a mendicant has gone to a wilderness, or to the root of a tree, or to an empty hut, and reflects like this: ‘I didn’t go forth from the lay life to homelessness for the sake of a robe, almsfood, lodgings, or rebirth in this or that state. But I was swamped by rebirth, old age, and death; by sorrow, lamentation, pain, sadness, and distress. I was swamped by suffering, mired in suffering. And I thought, “Hopefully I can find an end to this entire mass of suffering.” But it would not be appropriate for me to seek sensual pleasures like those I abandoned when I went forth, or even worse.’ Then they reflect: ‘My energy shall be roused up and unflagging, mindfulness shall be established and lucid, my body shall be tranquil and undisturbed, and my mind shall be immersed in \textsanskrit{samādhi}.’ Putting themselves in charge, they give up the unskillful and develop the skillful, they give up the blameworthy and develop the blameless, and they keep themselves pure. This is called putting oneself in charge. 

And\marginnote{2.1} what, mendicants, is putting the world in charge? It’s when a mendicant has gone to a wilderness, or to the root of a tree, or to an empty hut, and reflects like this: ‘I didn’t go forth from the lay life to homelessness for the sake of a robe, almsfood, lodgings, or rebirth in this or that state. But I was swamped by rebirth, old age, and death, by sorrow, lamentation, pain, sadness, and distress. I was swamped by suffering, mired in suffering. And I thought, “Hopefully I can find an end to this entire mass of suffering.” And now, since I’ve now gone forth, I might have sensual, malicious, or cruel thoughts. But the population of the world is large, and there are ascetics and brahmins who have psychic power—they’re clairvoyant, and can read the minds of others. They see far without being seen, even by those close; and they understand the minds of others. They would know me: 

“Look\marginnote{2.11} at this gentleman; he’s gone forth out of faith from the lay life to homelessness, but he’s living mixed up with bad, unskillful qualities.” And there are deities, too, who have psychic power—they’re clairvoyant, and can read the minds of others. They see far without being seen, even by those close; and they understand the minds of others. They would know me: 

“Look\marginnote{2.15} at this gentleman; he’s gone forth out of faith from the lay life to homelessness, but he’s living mixed up with bad, unskillful qualities.”’ Then they reflect: ‘My energy shall be roused up and unflagging, mindfulness shall be established and lucid, my body shall be tranquil and undisturbed, and my mind shall be immersed in \textsanskrit{samādhi}.’ Putting the world in charge, they give up the unskillful and develop the skillful, they give up the blameworthy and develop the blameless, and they keep themselves pure. This is called putting the world in charge. 

And\marginnote{3.1} what, mendicants, is putting the teaching in charge? It’s when a mendicant has gone to a wilderness, or to the root of a tree, or to an empty hut, and reflects like this: ‘I didn’t go forth from the lay life to homelessness for the sake of a robe, almsfood, lodgings, or rebirth in this or that state. But I was swamped by rebirth, old age, and death, by sorrow, lamentation, pain, sadness, and distress. I was swamped by suffering, mired in suffering. And I thought, “Hopefully I can find an end to this entire mass of suffering.” The teaching is well explained by the Buddha—visible in this very life, immediately effective, inviting inspection, relevant, so that sensible people can know it for themselves. I have spiritual companions who live knowing and seeing. Now that I’ve gone forth in this well explained teaching and training, it would not be appropriate for me to live lazy and heedless.’ Then they reflect: ‘My energy shall be roused up and unflagging, mindfulness shall be established and lucid, my body shall be tranquil and undisturbed, and my mind shall be immersed in \textsanskrit{samādhi}.’ Putting the teaching in charge, they give up the unskillful and develop the skillful, they give up the blameworthy and develop the blameless, and they keep themselves pure. This is called putting the teaching in charge. 

These\marginnote{3.14} are the three things to put in charge. 

\begin{verse}%
There’s\marginnote{4.1} no privacy in the world, \\
for someone who does bad deeds. \\
You’ll know for yourself, \\
whether you’ve lied or told the truth. 

When\marginnote{5.1} you witness your good self, \\
you despise it; \\
while you disguise \\
your bad self inside yourself. 

The\marginnote{6.1} gods and the Realized One see \\
the fool who lives unjustly in the world. \\
So with yourself in charge, live mindfully; \\
with the world in charge, be alert and practice absorption; \\
with the teaching in charge, live in line with that teaching: \\
a sage who tries for the truth doesn’t deteriorate. 

\textsanskrit{Māra}’s\marginnote{7.1} destroyed; the terminator’s overcome: \\
one who strives reaches the end of rebirth. \\
Poised, clever, knowing the world—\\
that sage identifies with nothing at all.” 

%
\end{verse}

%
\addtocontents{toc}{\let\protect\contentsline\protect\nopagecontentsline}
\chapter*{A Short Chapter }
\addcontentsline{toc}{chapter}{\tocchapterline{A Short Chapter }}
\addtocontents{toc}{\let\protect\contentsline\protect\oldcontentsline}

%
\section*{{\suttatitleacronym AN 3.41}{\suttatitletranslation Present }{\suttatitleroot Sammukhībhāvasutta}}
\addcontentsline{toc}{section}{\tocacronym{AN 3.41} \toctranslation{Present } \tocroot{Sammukhībhāvasutta}}
\markboth{Present }{Sammukhībhāvasutta}
\extramarks{AN 3.41}{AN 3.41}

“Mendicants,\marginnote{1.1} when three things are present, a faithful gentleman makes much merit. What three? When faith is present, when a gift to give is present, and when those worthy of a religious donation are present. When these three things are present, a faithful gentleman makes much merit.” 

%
\section*{{\suttatitleacronym AN 3.42}{\suttatitletranslation Three Grounds }{\suttatitleroot Tiṭhānasutta}}
\addcontentsline{toc}{section}{\tocacronym{AN 3.42} \toctranslation{Three Grounds } \tocroot{Tiṭhānasutta}}
\markboth{Three Grounds }{Tiṭhānasutta}
\extramarks{AN 3.42}{AN 3.42}

“There\marginnote{1.1} are three grounds, mendicants, by which a person with faith and confidence can be known. What three? They like to see ethical people. They like to hear the true teaching. And they live at home rid of the stain of stinginess, freely generous, open-handed, loving to let go, committed to charity, loving to give and to share. These are the three grounds by which a person with faith and confidence can be known. 

\begin{verse}%
They\marginnote{2.1} like to see ethical people; \\
they want to hear the true teaching; \\
they’ve driven out the stain of stinginess: \\
that’s who’s called a person of faith.” 

%
\end{verse}

%
\section*{{\suttatitleacronym AN 3.43}{\suttatitletranslation Good Reasons }{\suttatitleroot Atthavasasutta}}
\addcontentsline{toc}{section}{\tocacronym{AN 3.43} \toctranslation{Good Reasons } \tocroot{Atthavasasutta}}
\markboth{Good Reasons }{Atthavasasutta}
\extramarks{AN 3.43}{AN 3.43}

“Mendicants,\marginnote{1.1} taking three reasons into consideration provides quite enough motivation to teach Dhamma to another. What three? When the teacher understands the meaning and the teaching. When the audience understands the meaning and the teaching. When both the teacher and the audience understand the meaning and the teaching. 

Taking\marginnote{1.6} these three reasons into consideration provides quite enough motivation to teach Dhamma to another.” 

%
\section*{{\suttatitleacronym AN 3.44}{\suttatitletranslation When Conversation Flows }{\suttatitleroot Kathāpavattisutta}}
\addcontentsline{toc}{section}{\tocacronym{AN 3.44} \toctranslation{When Conversation Flows } \tocroot{Kathāpavattisutta}}
\markboth{When Conversation Flows }{Kathāpavattisutta}
\extramarks{AN 3.44}{AN 3.44}

“In\marginnote{1.1} three situations, mendicants, conversation flows. What three? When the teacher understands the meaning and the teaching. When the audience understands the meaning and the teaching. When both the teacher and the audience understand the meaning and the teaching. These are the three situations in which conversation flows.” 

%
\section*{{\suttatitleacronym AN 3.45}{\suttatitletranslation Recommended by the Astute }{\suttatitleroot Paṇḍitasutta}}
\addcontentsline{toc}{section}{\tocacronym{AN 3.45} \toctranslation{Recommended by the Astute } \tocroot{Paṇḍitasutta}}
\markboth{Recommended by the Astute }{Paṇḍitasutta}
\extramarks{AN 3.45}{AN 3.45}

“Mendicants,\marginnote{1.1} these three things are recommended by astute and good people. What three? Giving, going forth, and taking care of your mother and father. These are the three things recommended by astute and good people. 

\begin{verse}%
The\marginnote{2.1} virtuous recommend giving, \\
harmlessness, restraint, and self-control; \\
caring for mother and father, \\
and peaceful spiritual practitioners. 

These\marginnote{3.1} are the things recommended by the good, \\
which the astute should cultivate. \\
A noble one, having vision, \\
will enjoy a world of grace.” 

%
\end{verse}

%
\section*{{\suttatitleacronym AN 3.46}{\suttatitletranslation Ethical }{\suttatitleroot Sīlavantasutta}}
\addcontentsline{toc}{section}{\tocacronym{AN 3.46} \toctranslation{Ethical } \tocroot{Sīlavantasutta}}
\markboth{Ethical }{Sīlavantasutta}
\extramarks{AN 3.46}{AN 3.46}

“Mendicants,\marginnote{1.1} when ethical renunciates are supported by a town or village, the people there make much merit in three ways. What three? By way of body, speech, and mind. When ethical renunciates are supported by a town or village, the people there make much merit in these three ways.” 

%
\section*{{\suttatitleacronym AN 3.47}{\suttatitletranslation Characteristics of the Conditioned }{\suttatitleroot Saṅkhatalakkhaṇasutta}}
\addcontentsline{toc}{section}{\tocacronym{AN 3.47} \toctranslation{Characteristics of the Conditioned } \tocroot{Saṅkhatalakkhaṇasutta}}
\markboth{Characteristics of the Conditioned }{Saṅkhatalakkhaṇasutta}
\extramarks{AN 3.47}{AN 3.47}

“Mendicants,\marginnote{1.1} conditioned phenomena have these three characteristics. What three? Arising is evident, vanishing is evident, and change while persisting is evident. These are the three characteristics of conditioned phenomena.” 

\subsection*{Characteristics of the Unconditioned }

“Unconditioned\marginnote{2.1} phenomena have these three characteristics. What three? No arising is evident, no vanishing is evident, and no change while persisting is evident. These are the three characteristics of unconditioned phenomena.” 

%
\section*{{\suttatitleacronym AN 3.48}{\suttatitletranslation The King of Mountains }{\suttatitleroot Pabbatarājasutta}}
\addcontentsline{toc}{section}{\tocacronym{AN 3.48} \toctranslation{The King of Mountains } \tocroot{Pabbatarājasutta}}
\markboth{The King of Mountains }{Pabbatarājasutta}
\extramarks{AN 3.48}{AN 3.48}

“Mendicants,\marginnote{1.1} great sal trees grow in three ways supported by the Himalayas, the king of mountains. What three? The branches, leaves, and foliage; the bark and shoots; and the softwood and heartwood. Great sal trees grow in these three ways supported by the Himalayas, the king of mountains. 

In\marginnote{2.1} the same way, a family grows in three ways supported by a family head with faith. What three? Faith, ethics, and wisdom. A family grows in these three ways supported by a family head with faith. 

\begin{verse}%
Supported\marginnote{3.1} by the rocky mountain \\
in the wilds, the formidable forest, \\
the tree grows \\
to become lord of the forest. 

So\marginnote{4.1} too, when the family head \\
is ethical and faithful, \\
supported by them, they grow: \\
children, partners, and kin, \\
colleagues, relatives, \\
and those dependent for their livelihood. 

Seeing\marginnote{5.1} the ethical conduct of the virtuous, \\
the generosity and good deeds, \\
those who see clearly \\
do likewise. 

Having\marginnote{6.1} practiced the teaching here, \\
the path that goes to a good place, \\
they delight in the heavenly realm, \\
enjoying all the pleasures they desire.” 

%
\end{verse}

%
\section*{{\suttatitleacronym AN 3.49}{\suttatitletranslation Keen }{\suttatitleroot Ātappakaraṇīyasutta}}
\addcontentsline{toc}{section}{\tocacronym{AN 3.49} \toctranslation{Keen } \tocroot{Ātappakaraṇīyasutta}}
\markboth{Keen }{Ātappakaraṇīyasutta}
\extramarks{AN 3.49}{AN 3.49}

“In\marginnote{1.1} three situations, mendicants, you should be keen. What three? You should be keen to prevent bad, unskillful qualities from arising. You should be keen to give rise to skillful qualities. And you should be keen to endure physical pain—sharp, severe, acute, unpleasant, disagreeable, life-threatening. In these three situations, you should be keen. 

It’s\marginnote{2.1} a mendicant who is keen to prevent bad, unskillful qualities from arising. They’re keen to give rise to skillful qualities. And they’re keen to endure physical pain—sharp, severe, acute, unpleasant, disagreeable, life-threatening. This is called a mendicant who is keen, alert, and mindful so as to rightly make an end of suffering.” 

%
\section*{{\suttatitleacronym AN 3.50}{\suttatitletranslation A Master Thief }{\suttatitleroot Mahācorasutta}}
\addcontentsline{toc}{section}{\tocacronym{AN 3.50} \toctranslation{A Master Thief } \tocroot{Mahācorasutta}}
\markboth{A Master Thief }{Mahācorasutta}
\extramarks{AN 3.50}{AN 3.50}

“Mendicants,\marginnote{1.1} a master thief with three factors breaks into houses, plunders wealth, steals from isolated buildings, and commits highway robbery. What three? 

A\marginnote{1.3} master thief relies on uneven ground, on thick cover, and on powerful individuals. And how does a master thief rely on uneven ground? It’s when a master thief relies on inaccessible riverlands or rugged mountains. That’s how a master thief relies on uneven ground. 

And\marginnote{2.1} how does a master thief rely on thick cover? It’s when a master thief relies on thick grass, thick trees, a ridge, or a large dense wood. That’s how a master thief relies on thick cover. 

And\marginnote{3.1} how does a master thief rely on powerful individuals? It’s when a master thief relies on rulers or their ministers. They think: ‘If anyone accuses me of anything, these rulers or their ministers will speak in my defense in the case.’ And that’s exactly what happens. That’s how a master thief relies on powerful individuals. 

A\marginnote{3.7} master thief with these three factors breaks into houses, plunders wealth, steals from isolated buildings, and commits highway robbery. 

In\marginnote{4.1} the same way, when a bad mendicant has three factors, they keep themselves broken and damaged. They deserve to be blamed and criticized by sensible people, and they make much bad karma. What three? 

A\marginnote{4.3} bad mendicant relies on uneven ground, on thick cover, and on powerful individuals. 

And\marginnote{5.1} how does a bad mendicant rely on uneven ground? It’s when a bad mendicant has unethical conduct by way of body, speech, and mind. That’s how a bad mendicant relies on uneven ground. 

And\marginnote{6.1} how does a bad mendicant rely on thick cover? It’s when a bad mendicant has wrong view, he’s attached to an extremist view. That’s how a bad mendicant relies on thick cover. 

And\marginnote{7.1} how does a bad mendicant rely on powerful individuals? It’s when a bad mendicant relies on rulers or their ministers. They think: ‘If anyone accuses me of anything, these rulers or their ministers will speak in my defense in the case.’ And that’s exactly what happens. That’s how a bad mendicant relies on powerful individuals. 

When\marginnote{7.7} a bad mendicant has these three qualities, they keep themselves broken and damaged. They deserve to be blamed and criticized by sensible people, and they make much bad karma.” 

%
\addtocontents{toc}{\let\protect\contentsline\protect\nopagecontentsline}
\pannasa{The Second Fifty }
\addcontentsline{toc}{pannasa}{The Second Fifty }
\markboth{}{}
\addtocontents{toc}{\let\protect\contentsline\protect\oldcontentsline}

%
\addtocontents{toc}{\let\protect\contentsline\protect\nopagecontentsline}
\chapter*{The Chapter on Brahmins }
\addcontentsline{toc}{chapter}{\tocchapterline{The Chapter on Brahmins }}
\addtocontents{toc}{\let\protect\contentsline\protect\oldcontentsline}

%
\section*{{\suttatitleacronym AN 3.51}{\suttatitletranslation Two Brahmins (1st) }{\suttatitleroot Paṭhamadvebrāhmaṇasutta}}
\addcontentsline{toc}{section}{\tocacronym{AN 3.51} \toctranslation{Two Brahmins (1st) } \tocroot{Paṭhamadvebrāhmaṇasutta}}
\markboth{Two Brahmins (1st) }{Paṭhamadvebrāhmaṇasutta}
\extramarks{AN 3.51}{AN 3.51}

Then\marginnote{1.1} two old brahmins—elderly and senior, who were advanced in years and had reached the final stage of life, a hundred and twenty years old—went up to the Buddha, and exchanged greetings with him. When the greetings and polite conversation were over, they sat down to one side, and said to the Buddha: 

“We\marginnote{1.3} brahmins, Master Gotama, are old, elderly and senior, we’re advanced in years and have reached the final stage of life; we’re a hundred and twenty years old. And we haven’t done what is good and skillful, nor have we made a shelter from fear. Advise us, Master Gotama, instruct us! It will be for our lasting welfare and happiness.” 

“Indeed,\marginnote{2.1} brahmins, you’re old, elderly and senior. And you haven’t done what is good and skillful, nor have you made a shelter from fear. This world is led on by old age, sickness, and death. But restraint here by way of body, speech, and mind is the shelter, protection, island, refuge, and haven for the departed. 

\begin{verse}%
This\marginnote{3.1} life, so very short, is led onward. \\
There’s no shelter for someone who’s been led on by old age. \\
Seeing this peril in death, \\
you should do good deeds that bring happiness. 

The\marginnote{4.1} restraint practiced here—\\
of body, speech, and mind—\\
leads the departed to happiness, \\
as the good deeds done while living.” 

%
\end{verse}

%
\section*{{\suttatitleacronym AN 3.52}{\suttatitletranslation Two Brahmins (2nd) }{\suttatitleroot Dutiyadvebrāhmaṇasutta}}
\addcontentsline{toc}{section}{\tocacronym{AN 3.52} \toctranslation{Two Brahmins (2nd) } \tocroot{Dutiyadvebrāhmaṇasutta}}
\markboth{Two Brahmins (2nd) }{Dutiyadvebrāhmaṇasutta}
\extramarks{AN 3.52}{AN 3.52}

Then\marginnote{1.1} two old brahmins—elderly and senior, who were advanced in years and had reached the final stage of life, being a hundred and twenty years old—went up to the Buddha, bowed, sat down to one side, and said to the Buddha: 

“We\marginnote{1.2} brahmins, Master Gotama, are old, elderly and senior, we’re advanced in years and have reached the final stage of life; we’re a hundred and twenty years old. And we haven’t done what is good and skillful, nor have we made a shelter from fear. Advise us, Master Gotama, instruct us! It will be for our lasting welfare and happiness.” 

“Indeed,\marginnote{2.1} brahmins, you’re old, elderly and senior. And you haven’t done what is good and skillful, nor have you made a shelter from fear. This world is burning with old age, sickness, and death. But restraint here by way of body, speech, and mind is the shelter, protection, island, refuge, and haven for the departed. 

\begin{verse}%
When\marginnote{3.1} your house is on fire, \\
you rescue the pot \\
that’s useful, \\
not the one that’s burnt. 

And\marginnote{4.1} as the world is on fire \\
with old age and death, \\
you should rescue by giving, \\
for what’s given is rescued. 

The\marginnote{5.1} restraint practiced here—\\
of body, speech, and mind—\\
leads the departed to happiness, \\
as the good deeds done while living.” 

%
\end{verse}

%
\section*{{\suttatitleacronym AN 3.53}{\suttatitletranslation A Certain Brahmin }{\suttatitleroot Aññatarabrāhmaṇasutta}}
\addcontentsline{toc}{section}{\tocacronym{AN 3.53} \toctranslation{A Certain Brahmin } \tocroot{Aññatarabrāhmaṇasutta}}
\markboth{A Certain Brahmin }{Aññatarabrāhmaṇasutta}
\extramarks{AN 3.53}{AN 3.53}

Then\marginnote{1.1} a brahmin went up to the Buddha, and exchanged greetings with him. Seated to one side he said to the Buddha: 

“Master\marginnote{1.2} Gotama, they speak of ‘a teaching visible in this very life’. In what way is the teaching visible in this very life, immediately effective, inviting inspection, relevant, so that sensible people can know it for themselves?” 

“A\marginnote{2.1} greedy person, overcome and overwhelmed by greed, intends to hurt themselves, hurt others, and hurt both. They experience mental pain and sadness. When greed has been given up, they don’t intend to hurt themselves, hurt others, and hurt both. They don’t experience mental pain and sadness. This is how the teaching is visible in this very life, immediately effective, inviting inspection, relevant, so that sensible people can know it for themselves. 

A\marginnote{3.1} hateful person, overcome by hate, intends to hurt themselves, hurt others, and hurt both. They experience mental pain and sadness. When hate has been given up, they don’t intend to hurt themselves, hurt others, and hurt both. They don’t experience mental pain and sadness. This, too, is how the teaching is visible in this very life, immediately effective, inviting inspection, relevant, so that sensible people can know it for themselves. 

A\marginnote{4.1} deluded person, overcome by delusion, intends to hurt themselves, hurt others, and hurt both. They experience mental pain and sadness. When delusion has been given up, they don’t intend to hurt themselves, hurt others, and hurt both. They don’t experience mental pain and sadness. This, too, is how the teaching is visible in this very life, immediately effective, inviting inspection, relevant, so that sensible people can know it for themselves.” 

“Excellent,\marginnote{5.1} Master Gotama! Excellent! As if he were righting the overturned, or revealing the hidden, or pointing out the path to the lost, or lighting a lamp in the dark so people with good eyes can see what’s there, Master Gotama has made the teaching clear in many ways. I go for refuge to Master Gotama, to the teaching, and to the mendicant \textsanskrit{Saṅgha}. From this day forth, may Master Gotama remember me as a lay follower who has gone for refuge for life.” 

%
\section*{{\suttatitleacronym AN 3.54}{\suttatitletranslation A Wanderer }{\suttatitleroot Paribbājakasutta}}
\addcontentsline{toc}{section}{\tocacronym{AN 3.54} \toctranslation{A Wanderer } \tocroot{Paribbājakasutta}}
\markboth{A Wanderer }{Paribbājakasutta}
\extramarks{AN 3.54}{AN 3.54}

Then\marginnote{1.1} a brahmin wanderer went up to the Buddha … Seated to one side he said to the Buddha: 

“Master\marginnote{1.2} Gotama, they speak of ‘a teaching visible in this very life’. In what way is the teaching visible in this very life, immediately effective, inviting inspection, relevant, so that sensible people can know it for themselves?” 

“A\marginnote{2.1} greedy person, overcome by greed, intends to hurt themselves, hurt others, and hurt both. They experience mental pain and sadness. When greed has been given up, they don’t intend to hurt themselves, hurt others, and hurt both. They don’t experience mental pain and sadness. 

A\marginnote{3.1} greedy person does bad things by way of body, speech, and mind. When greed has been given up, they don’t do bad things by way of body, speech, and mind. 

A\marginnote{4.1} greedy person doesn’t truly understand what’s for their own good, the good of another, or the good of both. When greed has been given up, they truly understand what’s for their own good, the good of another, or the good of both. This is how the teaching is visible in this very life, immediately effective, inviting inspection, relevant, so that sensible people can know it for themselves. 

A\marginnote{5.1} hateful person … A deluded person, overcome by delusion, intends to hurt themselves, hurt others, and hurt both. They experience mental pain and sadness. When delusion has been given up, they don’t intend to hurt themselves, hurt others, and hurt both. They don’t experience mental pain and sadness. 

A\marginnote{6.1} deluded person does bad things by way of body, speech, and mind. When delusion has been given up, they don’t do bad things by way of body, speech, and mind. 

A\marginnote{7.1} deluded person doesn’t truly understand what’s for their own good, the good of another, or the good of both. When delusion has been given up, they truly understand what’s for their own good, the good of another, or the good of both. This, too, is how the teaching is visible in this very life, immediately effective, inviting inspection, relevant, so that sensible people can know it for themselves.” 

“Excellent,\marginnote{8.1} Master Gotama! Excellent! … From this day forth, may Master Gotama remember me as a lay follower who has gone for refuge for life.” 

%
\section*{{\suttatitleacronym AN 3.55}{\suttatitletranslation Extinguished }{\suttatitleroot Nibbutasutta}}
\addcontentsline{toc}{section}{\tocacronym{AN 3.55} \toctranslation{Extinguished } \tocroot{Nibbutasutta}}
\markboth{Extinguished }{Nibbutasutta}
\extramarks{AN 3.55}{AN 3.55}

Then\marginnote{1.1} the brahmin \textsanskrit{Jāṇussoṇi} went up to the Buddha, bowed, sat down to one side, and said to the Buddha: 

“Master\marginnote{1.2} Gotama, they say that ‘extinguishment is visible in this very life’. In what way is extinguishment visible in this very life, immediately effective, inviting inspection, relevant, so that sensible people can know it for themselves?” 

“A\marginnote{2.1} greedy person, overcome by greed, intends to hurt themselves, hurt others, and hurt both. They experience mental pain and sadness. When greed has been given up, they don’t intend to hurt themselves, hurt others, and hurt both. They don’t experience mental pain and sadness. This is how extinguishment is visible in this very life. 

A\marginnote{3.1} hateful person … 

A\marginnote{3.2} deluded person, overcome by delusion, intends to hurt themselves, hurt others, and hurt both. They experience mental pain and sadness. When delusion has been given up, they don’t intend to hurt themselves, hurt others, and hurt both. They don’t experience mental pain and sadness. This, too, is how extinguishment is visible in this very life. 

When\marginnote{4.1} you experience the ending of greed, hate, and delusion without anything left over, that’s how extinguishment is visible in this very life, immediately effective, inviting inspection, relevant, so that sensible people can know it for themselves.” 

“Excellent,\marginnote{5.1} Master Gotama! Excellent! … From this day forth, may Master Gotama remember me as a lay follower who has gone for refuge for life.” 

%
\section*{{\suttatitleacronym AN 3.56}{\suttatitletranslation Falling Apart }{\suttatitleroot Palokasutta}}
\addcontentsline{toc}{section}{\tocacronym{AN 3.56} \toctranslation{Falling Apart } \tocroot{Palokasutta}}
\markboth{Falling Apart }{Palokasutta}
\extramarks{AN 3.56}{AN 3.56}

Then\marginnote{1.1} a well-to-do Brahmin went up to the Buddha, and seated to one side he said to him: 

“Master\marginnote{1.2} Gotama, I have heard that brahmins of the past who were elderly and senior, the teachers of teachers, said: ‘In the old days this world was as crowded as hell, just full of people. The villages, towns and capital cities were no more than a chicken’s flight apart.’ What is the cause, sir, what is the reason why these days human numbers have dwindled, a decline in population is evident, and whole villages, towns, cities, and countries have disappeared?” 

“These\marginnote{2.1} days, brahmin, humans just love illicit desire. They’re overcome by immoral greed, and mired in wrong thoughts. Taking up sharp knives, they murder each other. And so many people perish. This is the cause, this is the reason why these days human numbers have dwindled. 

Furthermore,\marginnote{3.1} because these days humans just love illicit desire … the heavens don’t provide enough rain, so there’s famine, a bad harvest, with blighted crops that turn to straw. And so many people perish. This is the cause, this is the reason why these days human numbers have dwindled. 

Furthermore,\marginnote{4.1} because these days humans just love illicit desire … native spirits let vicious monsters loose. And so many people perish. This is the cause, this is the reason why these days human numbers have dwindled.” 

“Excellent,\marginnote{5.1} Master Gotama! Excellent! … From this day forth, may Master Gotama remember me as a lay follower who has gone for refuge for life.” 

%
\section*{{\suttatitleacronym AN 3.57}{\suttatitletranslation With Vacchagotta }{\suttatitleroot Vacchagottasutta}}
\addcontentsline{toc}{section}{\tocacronym{AN 3.57} \toctranslation{With Vacchagotta } \tocroot{Vacchagottasutta}}
\markboth{With Vacchagotta }{Vacchagottasutta}
\extramarks{AN 3.57}{AN 3.57}

Then\marginnote{1.1} the wanderer Vacchagotta went up to the Buddha, and exchanged greetings with him. When the greetings and polite conversation were over, he sat down to one side and said to the Buddha: 

“I\marginnote{1.3} have heard, Master Gotama, that the ascetic Gotama says this: ‘Gifts should only be given to me, not to others. Gifts should only be given to my disciples, not to the disciples of others. Only what is given to me is very fruitful, not what is given to others. Only what is given to my disciples is very fruitful, not what is given to the disciples of others.’ 

I\marginnote{1.13} trust that those who say this repeat what the Buddha has said, and do not misrepresent him with an untruth? Is their explanation in line with the teaching? Are there any legitimate grounds for rebuke and criticism? For we don’t want to misrepresent Master Gotama.” 

“Vaccha,\marginnote{2.1} those who say this do not repeat what I have said. They misrepresent me with what is false and untrue. 

Anyone\marginnote{2.6} who prevents another from giving makes an obstacle and a barrier for three people. What three? The giver is obstructed from making merit. The receiver is obstructed from getting what is offered. And they’ve already broken and damaged themselves. Anyone who prevents another from giving makes an obstacle and a barrier for these three people. 

Vaccha,\marginnote{3.1} this is what I say: ‘You even make merit by tipping out dish-washing water in a cesspool or a sump with living creatures in it, thinking, “May any creatures here be nourished!”’ How much more then for human beings! 

However,\marginnote{3.5} I also say that a gift to an ethical person is more fruitful than one to an unethical person. They’ve given up five factors, and possess five factors. 

What\marginnote{4.1} are the five factors they’ve given up? Sensual desire, ill will, dullness and drowsiness, restlessness and remorse, and doubt. These are the five factors they’ve given up. 

What\marginnote{5.1} are the five factors they possess? The entire spectrum of an adept’s ethics, immersion, wisdom, freedom, and knowledge and vision of freedom. These are the five factors they possess. 

I\marginnote{5.4} say that a gift to anyone who has given up these five factors and possesses these five factors is very fruitful. 

\begin{verse}%
Cows\marginnote{6.1} may be black or white, \\
red or tawny, \\
mottled or uniform, \\
or pigeon-colored. 

But\marginnote{7.1} when one is born among them, \\
the bull that’s tamed \\
—a behemoth, powerful, \\
well-paced in pulling forward—\\
they yoke the load just to him, \\
regardless of his color. 

So\marginnote{8.1} it is for humans, \\
wherever they may be born \\
—among aristocrats, brahmins, merchants, \\
workers, or outcastes and scavengers—

one\marginnote{9.1} is born among them, \\
tamed, true to their vows. \\
Firm in principle, accomplished in ethical conduct, \\
truthful, conscientious, 

they’ve\marginnote{10.1} given up birth and death. \\
Complete in the spiritual journey, \\
with burden put down, detached, \\
they’ve completed the task and are free of defilements. 

Gone\marginnote{11.1} beyond all things, \\
they’re extinguished by not grasping. \\
In that flawless field, \\
a religious donation is abundant. 

Fools\marginnote{12.1} who don’t understand \\
—stupid, uneducated—\\
give their gifts to those outside, \\
and don’t attend the peaceful ones. 

But\marginnote{13.1} those who do attend the peaceful ones \\
—wise, esteemed as sages—\\
and whose faith in the Holy One \\
has roots planted deep, 

they\marginnote{14.1} go to the realm of the gods, \\
or are born here in a good family. \\
Gradually those astute ones \\
reach extinguishment.” 

%
\end{verse}

%
\section*{{\suttatitleacronym AN 3.58}{\suttatitletranslation With Tikaṇṇa }{\suttatitleroot Tikaṇṇasutta}}
\addcontentsline{toc}{section}{\tocacronym{AN 3.58} \toctranslation{With Tikaṇṇa } \tocroot{Tikaṇṇasutta}}
\markboth{With Tikaṇṇa }{Tikaṇṇasutta}
\extramarks{AN 3.58}{AN 3.58}

Then\marginnote{1.1} \textsanskrit{Tikaṇṇa} the brahmin went up to the Buddha, and exchanged greetings with him. Seated to one side, in front of the Buddha, \textsanskrit{Tikaṇṇa} praised the brahmins who were proficient in the three Vedas, “Such are the brahmins, masters of the three Vedic knowledges! Thus are the brahmins, masters of the three Vedic knowledges!” 

“But\marginnote{2.1} brahmin, how do the brahmins describe a brahmin who is master of the three Vedic knowledges?” 

“Master\marginnote{2.2} Gotama, it’s when a brahmin is well born on both his mother’s and father’s side, of pure descent, irrefutable and impeccable in questions of ancestry back to the seventh paternal generation. He recites and remembers the hymns, and has mastered the three Vedas, together with their vocabularies, ritual, phonology and etymology, and the testament as fifth. He knows philology and grammar, and is well versed in cosmology and the marks of a great man. That’s how the brahmins describe a brahmin who is master of the three Vedic knowledges.” 

“Brahmin,\marginnote{3.1} a master of three knowledges according to the brahmins is quite different from a master of the three knowledges in the training of the Noble One.” 

“But\marginnote{3.2} Master Gotama, how is one a master of the three knowledges in the training of the Noble One? Master Gotama, please teach me this.” 

“Well\marginnote{3.4} then, brahmin, listen and pay close attention, I will speak.” 

“Yes\marginnote{3.5} sir,” \textsanskrit{Tikaṇṇa} replied. The Buddha said this: 

“Brahmin,\marginnote{4.1} it’s when a mendicant, quite secluded from sensual pleasures, secluded from unskillful qualities, enters and remains in the first absorption, which has the rapture and bliss born of seclusion, while placing the mind and keeping it connected. As the placing of the mind and keeping it connected are stilled, they enter and remain in the second absorption, which has the rapture and bliss born of immersion, with internal clarity and confidence, and unified mind, without placing the mind and keeping it connected. And with the fading away of rapture, they enter and remain in the third absorption, where they meditate with equanimity, mindful and aware, personally experiencing the bliss of which the noble ones declare, ‘Equanimous and mindful, one meditates in bliss.’ With the giving up of pleasure and pain, and the ending of former happiness and sadness, they enter and remain in the fourth absorption, without pleasure or pain, with pure equanimity and mindfulness. 

When\marginnote{5.1} their mind has become immersed in \textsanskrit{samādhi} like this—purified, bright, flawless, rid of corruptions, pliable, workable, steady, and imperturbable—they extend it toward recollection of past lives. They recollect many kinds of past lives. That is: one, two, three, four, five, ten, twenty, thirty, forty, fifty, a hundred, a thousand, a hundred thousand rebirths; many eons of the world contracting, many eons of the world expanding, many eons of the world contracting and expanding. They remember: ‘There, I was named this, my clan was that, I looked like this, and that was my food. This was how I felt pleasure and pain, and that was how my life ended. When I passed away from that place I was reborn somewhere else. There, too, I was named this, my clan was that, I looked like this, and that was my food. This was how I felt pleasure and pain, and that was how my life ended. When I passed away from that place I was reborn here.’ And so they recollect their many kinds of past lives, with features and details. This is the first knowledge that they attain. Ignorance is destroyed and knowledge has arisen; darkness is destroyed and light has arisen, as happens for a meditator who is diligent, keen, and resolute. 

When\marginnote{6.1} their mind has become immersed in \textsanskrit{samādhi} like this—purified, bright, flawless, rid of corruptions, pliable, workable, steady, and imperturbable—they extend it toward knowledge of the death and rebirth of sentient beings. With clairvoyance that is purified and superhuman, they see sentient beings passing away and being reborn—inferior and superior, beautiful and ugly, in a good place or a bad place. They understand how sentient beings are reborn according to their deeds: ‘These dear beings did bad things by way of body, speech, and mind. They spoke ill of the noble ones; they had wrong view; and they acted out of that wrong view. When their body breaks up, after death, they’re reborn in a place of loss, a bad place, the underworld, hell. These dear beings, however, did good things by way of body, speech, and mind. They never spoke ill of the noble ones; they had right view; and they acted out of that right view. When their body breaks up, after death, they’re reborn in a good place, a heavenly realm.’ And so, with clairvoyance that is purified and superhuman, they see sentient beings passing away and being reborn—inferior and superior, beautiful and ugly, in a good place or a bad place. They understand how sentient beings are reborn according to their deeds. This is the second knowledge that they attain. Ignorance is destroyed and knowledge has arisen; darkness is destroyed and light has arisen, as happens for a meditator who is diligent, keen, and resolute. 

When\marginnote{7.1} their mind has become immersed in \textsanskrit{samādhi} like this—purified, bright, flawless, rid of corruptions, pliable, workable, steady, and imperturbable—they extend it toward knowledge of the ending of defilements. They truly understand: ‘This is suffering’ … ‘This is the origin of suffering’ … ‘This is the cessation of suffering’ … ‘This is the practice that leads to the cessation of suffering’. They truly understand: ‘These are defilements’ … ‘This is the origin of defilements’ … ‘This is the cessation of defilements’ … ‘This is the practice that leads to the cessation of defilements’. Knowing and seeing like this, their mind is freed from the defilements of sensuality, desire to be reborn, and ignorance. When they’re freed, they know they’re freed. 

They\marginnote{7.6} understand: ‘Rebirth is ended, the spiritual journey has been completed, what had to be done has been done, there is no return to any state of existence.’ This is the third knowledge that they attain. Ignorance is destroyed and knowledge has arisen; darkness is destroyed, and light has arisen, as happens for a meditator who is diligent, keen, and resolute. 

\begin{verse}%
For\marginnote{8.1} someone whose ethical conduct doesn’t waver, \\
who is alert, practicing absorption; \\
whose mind is mastered, \\
unified, serene. 

That\marginnote{9.1} wise one dispels the darkness, \\
master of the three knowledges, conqueror of death. \\
For the welfare of gods and humans, \\
he has given up everything, they say. 

Accomplished\marginnote{10.1} in the three knowledges, \\
living without confusion, \\
bearing the final body, \\
they revere the awakened Gotama. 

Who\marginnote{11.1} knows their past lives, \\
and sees heaven and places of loss, \\
and has attained the ending of rebirth, \\
that sage has perfect insight. 

It’s\marginnote{12.1} because of these three knowledges \\
that a brahmin is a master of the three knowledges. \\
That’s who I call a three-knowledge master, \\
and not some mere reciter. 

%
\end{verse}

This,\marginnote{13.1} brahmin, is a master of the three knowledges in the training of the Noble One.” 

“Master\marginnote{13.2} Gotama, a master of three knowledges according to the brahmins is quite different from a master of the three knowledges in the training of the Noble One. And, Master Gotama, a master of three knowledges according to the brahmins is not worth a sixteenth part of a master of the three knowledges in the training of the Noble One. 

Excellent,\marginnote{14.1} Master Gotama! Excellent! … From this day forth, may Master Gotama remember me as a lay follower who has gone for refuge for life.” 

%
\section*{{\suttatitleacronym AN 3.59}{\suttatitletranslation With Jāṇussoṇi }{\suttatitleroot Jāṇussoṇisutta}}
\addcontentsline{toc}{section}{\tocacronym{AN 3.59} \toctranslation{With Jāṇussoṇi } \tocroot{Jāṇussoṇisutta}}
\markboth{With Jāṇussoṇi }{Jāṇussoṇisutta}
\extramarks{AN 3.59}{AN 3.59}

Then\marginnote{1.1} the brahmin \textsanskrit{Jāṇussoṇi} went up to the Buddha, and exchanged greetings with him. Seated to one side he said to the Buddha: 

“Master\marginnote{1.2} Gotama, whoever has a sacrifice, an offering of food for ancestors, a dish of milk-rice prepared for an auspicious ceremony, or a gift to give, should give it to the brahmins who have mastered the three Vedic knowledges.” 

“But\marginnote{1.3} brahmin, how do the brahmins describe a brahmin who is proficient in the three Vedic knowledges?” 

“Master\marginnote{1.4} Gotama, it’s when a brahmin is well born on both his mother’s and father’s side, of pure descent, irrefutable and impeccable in questions of ancestry back to the seventh paternal generation. He recites and remembers the hymns, and has mastered the three Vedas, together with their vocabularies, ritual, phonology and etymology, and the testament as fifth. He knows philology and grammar, and is well versed in cosmology and the marks of a great man. That’s how the brahmins describe a brahmin who is proficient in the three Vedic knowledges.” 

“Brahmin,\marginnote{2.1} a master of three knowledges according to the brahmins is quite different from a master of the three knowledges in the training of the Noble One.” 

“But\marginnote{2.2} Master Gotama, how is one a master of the three knowledges in the training of the Noble One? Master Gotama, please teach me this.” 

“Well\marginnote{2.4} then, brahmin, listen and pay close attention, I will speak.” 

“Yes\marginnote{2.5} sir,” \textsanskrit{Jāṇussoṇi} replied. The Buddha said this: 

“Brahmin,\marginnote{3.1} it’s when a mendicant, quite secluded from sensual pleasures … enters and remains in the fourth absorption. 

When\marginnote{4.1} their mind has become immersed in \textsanskrit{samādhi} like this—purified, bright, flawless, rid of corruptions, pliable, workable, steady, and imperturbable—they extend it toward recollection of past lives. They recollect many kinds of past lives, with features and details. This is the first knowledge that they attain. Ignorance is destroyed and knowledge has arisen; darkness is destroyed and light has arisen, as happens for a meditator who is diligent, keen, and resolute. 

When\marginnote{5.1} their mind has become immersed in \textsanskrit{samādhi} like this—purified, bright, flawless, rid of corruptions, pliable, workable, steady, and imperturbable—they extend it toward knowledge of the death and rebirth of sentient beings. With clairvoyance that is purified and surpasses the human, they understand how sentient beings are reborn according to their deeds. This is the second knowledge that they attain. Ignorance is destroyed and knowledge has arisen; darkness is destroyed and light has arisen, as happens for a meditator who is diligent, keen, and resolute. 

When\marginnote{6.1} their mind has become immersed in \textsanskrit{samādhi} like this—purified, bright, flawless, rid of corruptions, pliable, workable, steady, and imperturbable—they extend it toward knowledge of the ending of defilements. They truly understand: ‘This is suffering’ … ‘This is the origin of suffering’ … ‘This is the cessation of suffering’ … ‘This is the practice that leads to the cessation of suffering’. They truly understand: ‘These are defilements’ … ‘This is the origin of defilements’ … ‘This is the cessation of defilements’ … ‘This is the practice that leads to the cessation of defilements’. Knowing and seeing like this, their mind is freed from the defilements of sensuality, desire to be reborn, and ignorance. When they’re freed, they know they’re freed. 

They\marginnote{6.6} understand: ‘Rebirth is ended, the spiritual journey has been completed, what had to be done has been done, there is no return to any state of existence.’ This is the third knowledge that they attain. Ignorance is destroyed and knowledge has arisen; darkness is destroyed, and light has arisen, as happens for a meditator who is diligent, keen, and resolute. 

\begin{verse}%
One\marginnote{7.1} who is perfect in precepts and observances, \\
resolute and serene, \\
whose mind is mastered, \\
unified, serene; 

who\marginnote{8.1} knows their past lives, \\
and sees heaven and places of loss, \\
and has attained the end of rebirth, \\
that sage has perfect insight. 

Because\marginnote{9.1} of these three knowledges \\
a brahmin is a master of the three knowledges. \\
That’s who I call a three-knowledge master, \\
and not some mere reciter. 

%
\end{verse}

This,\marginnote{10.1} brahmin, is a master of the three knowledges in the training of the Noble One.” 

“Master\marginnote{10.2} Gotama, the master of three knowledges according to the brahmins is quite different from a master of the three knowledges in the training of the Noble One. And, Master Gotama, a master of three knowledges according to the brahmins is not worth a sixteenth part of a master of the three knowledges in the training of the Noble One. 

Excellent,\marginnote{11.1} Master Gotama! Excellent! … From this day forth, may Master Gotama remember me as a lay follower who has gone for refuge for life.” 

%
\section*{{\suttatitleacronym AN 3.60}{\suttatitletranslation With Saṅgārava }{\suttatitleroot Saṅgāravasutta}}
\addcontentsline{toc}{section}{\tocacronym{AN 3.60} \toctranslation{With Saṅgārava } \tocroot{Saṅgāravasutta}}
\markboth{With Saṅgārava }{Saṅgāravasutta}
\extramarks{AN 3.60}{AN 3.60}

Then\marginnote{1.1} \textsanskrit{Saṅgārava} the brahmin went up to the Buddha, and exchanged greetings with him. When the greetings and polite conversation were over, he sat down to one side and said to the Buddha: 

“Master\marginnote{1.3} Gotama, we who are called brahmins make sacrifices and encourage others to make sacrifices. Now, Master Gotama, both of these people—the one who sacrifices and the one who encourages others to sacrifice—are doing good for many people on account of that sacrifice. But, Master Gotama, when someone has gone forth from the lay life to homelessness, they tame, calm, and extinguish themselves alone. That being so, they are doing good for just one person on account of that going forth.” 

“Well\marginnote{2.1} then, brahmin, I’ll ask you about this in return, and you can answer as you like. What do you think, brahmin? A Realized One arises in the world, perfected, a fully awakened Buddha, accomplished in knowledge and conduct, holy, knower of the world, supreme guide for those who wish to train, teacher of gods and humans, awakened, blessed. He says, ‘Come, this is the path, this is the practice. Practicing like this, I realized the supreme culmination of the spiritual life with my own insight, and I make it known. Please, all of you, practice like this, and you too will realize the supreme culmination of the spiritual life, and will live having realized it with your own insight.’ So the teacher teaches Dhamma, and others practice accordingly, in their hundreds and thousands, and hundreds of thousands. 

What\marginnote{3.1} do you think, brahmin? This being so, are they doing good for just one person or for many people on account of going forth?” 

“This\marginnote{3.3} being so, Master Gotama, they are doing good for many people on account of going forth.” 

When\marginnote{4.1} he said this, Venerable Ānanda said to \textsanskrit{Saṅgārava}, “Brahmin, which of these two practices do you believe has fewer requirements and undertakings, yet is more fruitful and beneficial?” 

\textsanskrit{Saṅgārava}\marginnote{4.3} said to Ānanda, “Those such as the masters Gotama and Ānanda are honored and praised by me!” 

For\marginnote{5.1} a second time, Ānanda said to \textsanskrit{Saṅgārava}, “Brahmin, I didn’t ask you who you honor and praise. I asked you which of these two practices do you believe has fewer requirements and undertakings, yet is more fruitful and beneficial?” 

For\marginnote{5.6} a second time \textsanskrit{Saṅgārava} said to Ānanda, “Those such as the masters Gotama and Ānanda are honored and praised by me!” 

For\marginnote{6.1} a third time, Ānanda said to \textsanskrit{Saṅgārava}, “Brahmin, I didn’t ask you who you honor and praise. I asked you which of these two practices do you believe has fewer requirements and undertakings, yet is more fruitful and beneficial?” 

For\marginnote{6.6} a third time \textsanskrit{Saṅgārava} said to Ānanda, “Those such as the masters Gotama and Ānanda are honored and praised by me!” 

Then\marginnote{7.1} it occurred to the Buddha, “Though Ānanda asked him a sensible question three times, \textsanskrit{Saṅgārava} falters without answering. Why don’t I give him a way out?” 

So\marginnote{7.4} the Buddha said to \textsanskrit{Saṅgārava}, “Brahmin, what came up in the conversation among the king’s retinue today, sitting together in the royal compound?” 

“Master\marginnote{7.6} Gotama, this came up: ‘Formerly, it seems, there were fewer mendicants, but more of them displayed superhuman demonstrations of psychic power; while these days, there are more mendicants, but fewer display superhuman demonstrations of psychic power.’ This is what came up in the conversation among the king’s retinue today, while sitting together in the royal compound.” 

“Brahmin,\marginnote{8.1} there are three kinds of demonstration. What three? A demonstration of psychic power, a demonstration of revealing, and a demonstration of instruction. 

And\marginnote{8.4} what is the demonstration of psychic power? It’s when someone wields the many kinds of psychic power: multiplying themselves and becoming one again; appearing and disappearing; going unimpeded through a wall, a rampart, or a mountain as if through space; diving in and out of the earth as if it were water; walking on water as if it were earth; flying cross-legged through the sky like a bird; touching and stroking with the hand the sun and moon, so mighty and powerful. They control the body as far as the \textsanskrit{Brahmā} realm. This is called the demonstration of psychic power. 

And\marginnote{9.1} what is the demonstration of revealing? In one case, someone reveals by means of a sign: ‘This is what you’re thinking, such is your thought, and thus is your state of mind.’ And even if they reveal this many times, it turns out exactly so, not otherwise. 

In\marginnote{10.1} another case, someone reveals after hearing it from humans or non-humans or deities: ‘This is what you’re thinking, such is your thought, and thus is your state of mind.’ And even if they reveal this many times, it turns out exactly so, not otherwise. 

In\marginnote{11.1} another case, someone reveals by hearing the sound of thought spreading as someone thinks and considers: ‘This is what you’re thinking, such is your thought, and thus is your state of mind.’ And even if they reveal this many times, it turns out exactly so, not otherwise. 

In\marginnote{12.1} another case, someone comprehends the mind of a person who has attained the immersion that’s free of placing the mind and keeping it connected. They understand: ‘Judging by the way this person’s intentions are directed, immediately after this mind state, they’ll think this thought.’ And even if they reveal this many times, it turns out exactly so, not otherwise. This is called the demonstration of revealing. 

And\marginnote{13.1} what is a demonstration of instruction? It’s when someone instructs others like this: ‘Think like this, not like that. Focus your mind like this, not like that. Give up this, and live having achieved that.’ This is called a demonstration of instruction. 

These\marginnote{13.5} are the three kinds of demonstration. Of these three kinds of demonstration, which do you consider to be the finest?” 

“Regarding\marginnote{14.1} this, Master Gotama, a demonstration of psychic power is experienced only by the one who performs it, occurring only to them. This seems to me like a magic trick. 

And\marginnote{15.1} the demonstration where someone reveals something by means of a sign, or after hearing it from humans, non-humans, or deities, or by hearing the sound of thought spreading as someone thinks and considers, or by comprehending the mind of another person, is also experienced only by the one who performs it, occurring only to them. This also seems to me like a magic trick. 

But\marginnote{16.1} as to the demonstration where someone instructs others: ‘Think like this, not like that. Focus your mind like this, not like that. Give up this, and live having achieved that.’ I prefer this demonstration, Master Gotama. It’s the finest of the three kinds of demonstration. 

It’s\marginnote{17.1} incredible, Master Gotama, it’s amazing, how well this was said by Master Gotama. We regard Master Gotama as someone who possesses these three kinds of demonstration. For Master Gotama wields the many kinds of psychic power … controlling the body as far as the \textsanskrit{Brahmā} realm. And Master Gotama comprehends the mind of another person who has attained the immersion that is free of placing the mind and keeping it connected. He understands: ‘Judging by the way this person’s intentions are directed, immediately after this mind state they’ll think this thought.’ And Master Gotama instructs others like this: ‘Think like this, not like that. Focus your mind like this, not like that. Give up this, and live having achieved that.’” 

“Your\marginnote{18.1} words are clearly invasive and intrusive, brahmin. Nevertheless, I will answer you. For I do wield the many kinds of psychic power … controlling the body as far as the \textsanskrit{Brahmā} realm. And I do comprehend the mind of another person who has attained the immersion that is free of placing the mind and keeping it connected. I understand: ‘Judging by the way this person’s intentions are directed, immediately after this mind state they’ll think this thought.’ And I do instruct others like this: ‘Think like this, not like that. Focus your mind like this, not like that. Give up this, and live having achieved that.’” 

“But\marginnote{19.1} Master Gotama, is there even one other mendicant who possesses these three kinds of demonstration, apart from Master Gotama?” 

“There’s\marginnote{19.2} not just one hundred mendicants, brahmin, who possess these three kinds of demonstration, nor two, three, four, or five hundred, but many more than that.” 

“But\marginnote{19.3} where are these mendicants now staying?” 

“Right\marginnote{19.4} here, brahmin, in this \textsanskrit{Saṅgha} of mendicants.” 

“Excellent,\marginnote{20.1} Master Gotama! Excellent! As if he were righting the overturned, or revealing the hidden, or pointing out the path to the lost, or lighting a lamp in the dark so people with good eyes can see what’s there, Master Gotama has made the teaching clear in many ways. I go for refuge to Master Gotama, to the teaching, and to the mendicant \textsanskrit{Saṅgha}. From this day forth, may Master Gotama remember me as a lay follower who has gone for refuge for life.” 

%
\addtocontents{toc}{\let\protect\contentsline\protect\nopagecontentsline}
\chapter*{The Great Chapter }
\addcontentsline{toc}{chapter}{\tocchapterline{The Great Chapter }}
\addtocontents{toc}{\let\protect\contentsline\protect\oldcontentsline}

%
\section*{{\suttatitleacronym AN 3.61}{\suttatitletranslation Sectarian Tenets }{\suttatitleroot Titthāyatanasutta}}
\addcontentsline{toc}{section}{\tocacronym{AN 3.61} \toctranslation{Sectarian Tenets } \tocroot{Titthāyatanasutta}}
\markboth{Sectarian Tenets }{Titthāyatanasutta}
\extramarks{AN 3.61}{AN 3.61}

“Mendicants,\marginnote{1.1} these three sectarian tenets—as pursued, pressed, and grilled by the astute—when taken to their conclusion, end with inaction. What three? 

There\marginnote{1.3} are some ascetics and brahmins who have this doctrine and view: ‘Everything this individual experiences—pleasurable, painful, or neutral—is because of past deeds.’ 

There\marginnote{1.5} are some ascetics and brahmins who have this doctrine and view: ‘Everything this individual experiences—pleasurable, painful, or neutral—is because of the Lord God’s creation.’ 

There\marginnote{1.7} are some ascetics and brahmins who have this doctrine and view: ‘Everything this individual experiences—pleasurable, painful, or neutral—has no cause or reason.’ 

Regarding\marginnote{2.1} this, I went up to the ascetics and brahmins whose view is that everything that is experienced is because of past deeds, and I said to them: ‘Is it really true that this is the venerables’ view?’ And they answered, ‘Yes’. I said to them: ‘In that case, you might kill living creatures, steal, be unchaste; use speech that’s false, divisive, harsh, or nonsensical; be covetous, malicious, or have wrong view, all because of past deeds.’ 

Those\marginnote{3.1} who believe that past deeds are the most important thing have no enthusiasm or effort, no idea that there are things that should and should not be done. Since they don’t acknowledge as a genuine fact that there are things that should and should not be done, they’re unmindful and careless, and can’t rightly be called ascetics. This is my first legitimate refutation of the ascetics and brahmins who have this doctrine and view. 

Regarding\marginnote{4.1} this, I went up to the ascetics and brahmins whose view is that everything that is experienced is because of the Lord God’s creation, and I said to them: ‘Is it really true that this is the venerables’ view?’ And they answered, ‘Yes’. I said to them: ‘In that case, you might kill living creatures, steal, be unchaste; use speech that’s false, divisive, harsh, or nonsensical; be covetous, malicious, or have wrong view, all because of the Lord God’s creation.’ 

Those\marginnote{5.1} who believe that the Lord God’s creative power is the most important thing have no enthusiasm, no effort, no idea that there are things that should and should not be done. Since they don’t acknowledge as a genuine fact that there are things that should and should not be done, they’re unmindful and careless, and can’t rightly be called ascetics. This is my second legitimate refutation of the ascetics and brahmins who have this doctrine and view. 

Regarding\marginnote{6.1} this, I went up to the ascetics and brahmins whose view is that everything that is experienced has no cause or reason, and I said to them: ‘Is it really true that this is the venerables’ view?’ And they answered, ‘Yes’. I said to them: ‘In that case, you might kill living creatures, steal, be unchaste; use speech that’s false, divisive, harsh, or nonsensical; be covetous, malicious, or have wrong view, all without cause or reason.’ 

Those\marginnote{7.1} who believe that the absence of cause or reason is the most important thing have no enthusiasm, no effort, no idea that there are things that should and should not be done. Since they don’t acknowledge as a genuine fact that there are things that should and should not be done, they’re unmindful and careless, and can’t rightly be called ascetics. This is my third legitimate refutation of the ascetics and brahmins who have this doctrine and view. 

These\marginnote{8.1} are the three sectarian tenets—as pursued, pressed, and grilled by the astute—which, when taken to their conclusion, end with inaction. 

But\marginnote{9.1} the Dhamma that I’ve taught is irrefutable, uncorrupted, beyond reproach, and not scorned by sensible ascetics and brahmins. What is the Dhamma that I’ve taught? 

‘These\marginnote{9.3} are the six elements’: this is the Dhamma I’ve taught … 

‘These\marginnote{9.4} are the six fields of contact’: this is the Dhamma I’ve taught … 

‘These\marginnote{9.5} are the eighteen mental preoccupations’: this is the Dhamma I’ve taught … 

‘These\marginnote{9.6} are the four noble truths’: this is the Dhamma I’ve taught that is irrefutable, uncorrupted, beyond reproach, and is not scorned by sensible ascetics and brahmins. 

‘“These\marginnote{10.1} are the six elements”: this is the Dhamma I’ve taught …’ That’s what I said, but why did I say it? 

There\marginnote{10.4} are these six elements: the elements of earth, water, fire, air, space, and consciousness. 

‘“These\marginnote{10.6} are the six elements”: this is the Dhamma I’ve taught …’ That’s what I said, and this is why I said it. 

‘“These\marginnote{11.1} are the six fields of contact”: this is the Dhamma I’ve taught …’ That’s what I said, but why did I say it? 

There\marginnote{11.4} are these six fields of contact: eye, ear, nose, tongue, body, and mind contact. 

‘“These\marginnote{11.6} are the six fields of contact”: this is the Dhamma I’ve taught …’ That’s what I said, and this is why I said it. 

‘“These\marginnote{12.1} are the eighteen mental preoccupations”: this is the Dhamma I’ve taught …’ This is what I said, but why did I say it? 

Seeing\marginnote{12.4} a sight with the eye, one is preoccupied with a sight that’s a basis for happiness or sadness or equanimity. 

Hearing\marginnote{12.5} a sound with the ear … 

Smelling\marginnote{12.6} an odor with the nose … 

Tasting\marginnote{12.7} a flavor with the tongue … 

Feeling\marginnote{12.8} a touch with the body … 

Becoming\marginnote{12.9} conscious of a thought with the mind, one is preoccupied with a thought that’s a basis for happiness or sadness or equanimity. 

‘“These\marginnote{12.10} are the eighteen mental preoccupations”: this is the Dhamma I’ve taught …’ That’s what I said, and this is why I said it. 

‘“These\marginnote{13.1} are the four noble truths”: this is the Dhamma I’ve taught …’ That’s what I said, but why did I say it? 

Supported\marginnote{13.4} by the six elements, an embryo is conceived. When it is conceived, there are name and form. Name and form are conditions for the six sense fields. The six sense fields are conditions for contact. Contact is a condition for feeling. It’s for one who feels that I declare: ‘This is suffering’ … ‘This is the origin of suffering’ … ‘This is the cessation of suffering’ … ‘This is the practice that leads to the cessation of suffering’. 

And\marginnote{14.1} what is the noble truth of suffering? Rebirth is suffering; old age is suffering; death is suffering; sorrow, lamentation, pain, sadness, and distress are suffering; association with the disliked is suffering; separation from the liked is suffering; not getting what you wish for is suffering. In brief, the five grasping aggregates are suffering. This is called the noble truth of suffering. 

And\marginnote{15.1} what is the noble truth of the origin of suffering? Ignorance is a condition for choices. Choices are a condition for consciousness. Consciousness is a condition for name and form. Name and form are conditions for the six sense fields. The six sense fields are conditions for contact. Contact is a condition for feeling. Feeling is a condition for craving. Craving is a condition for grasping. Grasping is a condition for continued existence. Continued existence is a condition for rebirth. Rebirth is a condition for old age and death, sorrow, lamentation, pain, sadness, and distress to come to be. That is how this entire mass of suffering originates. This is called the noble truth of the origin of suffering. 

And\marginnote{16.1} what is the noble truth of the cessation of suffering? When ignorance fades away and ceases with nothing left over, choices cease. When choices cease, consciousness ceases. When consciousness ceases, name and form cease. When name and form cease, the six sense fields cease. When the six sense fields cease, contact ceases. When contact ceases, feeling ceases. When feeling ceases, craving ceases. When craving ceases, grasping ceases. When grasping ceases, continued existence ceases. When continued existence ceases, rebirth ceases. When rebirth ceases, old age and death, sorrow, lamentation, pain, sadness, and distress cease. That is how this entire mass of suffering ceases. This is called the noble truth of the cessation of suffering. 

And\marginnote{17.1} what is the noble truth of the practice that leads to the cessation of suffering? It is simply this noble eightfold path, that is: right view, right thought, right speech, right action, right livelihood, right effort, right mindfulness, and right immersion. This is called the noble truth of the practice that leads to the cessation of suffering. 

‘“These\marginnote{17.5} are the four noble truths”: this is the Dhamma I’ve taught that is irrefutable, uncorrupted, beyond reproach, and is not scorned by sensible ascetics and brahmins.’ That’s what I said, and this is why I said it.” 

%
\section*{{\suttatitleacronym AN 3.62}{\suttatitletranslation Perils }{\suttatitleroot Bhayasutta}}
\addcontentsline{toc}{section}{\tocacronym{AN 3.62} \toctranslation{Perils } \tocroot{Bhayasutta}}
\markboth{Perils }{Bhayasutta}
\extramarks{AN 3.62}{AN 3.62}

“Mendicants,\marginnote{1.1} an uneducated ordinary person speaks of three perils that tear mothers and children apart. What three? 

There\marginnote{1.3} comes a time when a great fire flares up, and it burns villages, towns, and cities. When this happens, a mother can’t find her child, and a child can’t find their mother. This is the first peril that tears mothers and children apart. 

Furthermore,\marginnote{2.1} there comes a time when a great storm gathers, and it unleashes a mighty flood that sweeps away villages, towns, and cities. When this happens, a mother can’t find her child, and a child can’t find their mother. This is the second peril that tears mothers and children apart. 

Furthermore,\marginnote{3.1} there comes a time of peril from wild savages, and the countryfolk mount their vehicles and flee everywhere. When this happens, a mother can’t find her child, and a child can’t find their mother. This is the third peril that tears mothers and children apart. 

These\marginnote{3.4} are the three perils an uneducated ordinary person speaks of that tear mothers and children apart. 

Mendicants,\marginnote{4.1} an uneducated ordinary person speaks of three perils that don’t tear mothers and children apart. What three? 

There\marginnote{4.3} comes a time when a great fire flares up, and it burns villages, towns, and cities. When this happens, sometimes a mother can find her child, and a child can find their mother. This is the first peril that doesn’t tear mothers and children apart. 

Furthermore,\marginnote{5.1} there comes a time when a great storm gathers, and it unleashes a mighty flood that sweeps away villages, towns, and cities. When this happens, sometimes a mother can find her child, and a child can find their mother. This is the second peril that doesn’t tear mothers and children apart. 

Furthermore,\marginnote{6.1} there comes a time of peril from wild savages, and the countryfolk mount their vehicles and flee everywhere. When this happens, sometimes a mother can find her child, and a child can find their mother. This is the third peril that doesn’t tear mothers and children apart. 

These\marginnote{6.4} are the three perils an uneducated ordinary person speaks of that don’t tear mothers and children apart. 

There\marginnote{7.1} are three perils that tear mothers and children apart. What three? 

The\marginnote{7.3} perils of old age, sickness, and death. When a child is growing old, a mother doesn’t get her wish: ‘Let me grow old, may my child not grow old!’ When a mother is growing old, a child doesn’t get their wish: ‘Let me grow old, may my mother not grow old!’ 

When\marginnote{8.1} a child is sick, a mother doesn’t get her wish: ‘Let me be sick, may my child not be sick!’ When a mother is sick, a child doesn’t get their wish: ‘Let me be sick, may my mother not be sick!’ 

When\marginnote{9.1} a child is dying, a mother doesn’t get her wish: ‘Let me die, may my child not die!’ When a mother is dying, a child doesn’t get their wish: ‘Let me die, may my mother not die!’ These are the three perils that tear mothers and children apart. 

There\marginnote{10.1} is a path and a practice that leads to giving up and going beyond the three perils that don’t tear mothers and children apart, and the three perils that do tear mothers and children apart. What is that path and practice? It is simply this noble eightfold path, that is: right view, right thought, right speech, right action, right livelihood, right effort, right mindfulness, and right immersion. This is the path, this is the practice that leads to giving up and going beyond the three perils that don’t tear mothers and children apart, and the three perils that do tear mothers and children apart.” 

%
\section*{{\suttatitleacronym AN 3.63}{\suttatitletranslation At Venāgapura }{\suttatitleroot Venāgapurasutta}}
\addcontentsline{toc}{section}{\tocacronym{AN 3.63} \toctranslation{At Venāgapura } \tocroot{Venāgapurasutta}}
\markboth{At Venāgapura }{Venāgapurasutta}
\extramarks{AN 3.63}{AN 3.63}

At\marginnote{1.1} one time the Buddha was wandering in the land of the Kosalans together with a large \textsanskrit{Saṅgha} of mendicants when he arrived at a village of the Kosalan brahmins named \textsanskrit{Venāgapura}. The brahmins and householders of \textsanskrit{Venāgapura} heard: 

“It\marginnote{1.3} seems the ascetic Gotama—a Sakyan, gone forth from a Sakyan family—has arrived at \textsanskrit{Venāgapura}. He has this good reputation: ‘That Blessed One is perfected, a fully awakened Buddha, accomplished in knowledge and conduct, holy, knower of the world, supreme guide for those who wish to train, teacher of gods and humans, awakened, blessed.’ He has realized with his own insight this world—with its gods, \textsanskrit{Māras} and \textsanskrit{Brahmās}, this population with its ascetics and brahmins, gods and humans—and he makes it known to others. He teaches Dhamma that’s good in the beginning, good in the middle, and good in the end, meaningful and well-phrased. And he reveals a spiritual practice that’s entirely full and pure. It’s good to see such perfected ones.” 

Then\marginnote{2.1} the brahmins and householders of \textsanskrit{Venāgapura} went up to the Buddha. Before sitting down to one side, some bowed, some exchanged greetings and polite conversation, some held up their joined palms toward the Buddha, some announced their name and clan, while some kept silent. Then the brahmin Vacchagotta of \textsanskrit{Venāgapura} said to the Buddha: 

“It’s\marginnote{3.1} incredible, Master Gotama, it’s amazing, how your faculties are so very clear, and the complexion of your skin is pure and bright. It’s like a golden brown jujube in the autumn, or a palm fruit freshly plucked from the stalk, or a pendant of river gold, fashioned by a deft smith, well-wrought in the forge, and placed on a cream rug where it shines and glows and radiates. In the same way, your faculties are so very clear, and the complexion of your skin is pure and bright. 

Surely\marginnote{3.9} Master Gotama gets when he wants, without trouble or difficulty, various kinds of high and luxurious bedding, such as: sofas, couches, woolen covers—shag-piled, colorful, white, embroidered with flowers, quilted, embroidered with animals, double-or single-fringed—and silk covers studded with gems, as well as silken sheets, woven carpets, rugs for elephants, horses, or chariots, antelope hide rugs, and spreads of fine deer hide, with a canopy above and red cushions at both ends.” 

“Brahmin,\marginnote{4.1} these various kinds of high and luxurious bedding are hard for renunciates to find. And even if they do get them, they’re not allowed. 

There\marginnote{5.1} are, brahmin, these three high and luxurious beds that I get these days when I want, without trouble or difficulty. What three? The high and luxurious beds of the gods, of \textsanskrit{Brahmā}, and of the noble ones. These are the three high and luxurious beds that I get these days when I want, without trouble or difficulty.” 

“But\marginnote{6.1} what, Master Gotama, is the high and luxurious bed of the gods?” 

“Brahmin,\marginnote{6.2} when I am living supported by a village or town, I robe up in the morning and, taking my bowl and robe, enter the town or village for alms. After the meal, on my return from almsround, I enter within a forest. I gather up some grass or leaves into a pile, and sit down cross-legged, with my body straight, and establish mindfulness right there. Quite secluded from sensual pleasures, secluded from unskillful qualities, I enter and remain in the first absorption, which has the rapture and bliss born of seclusion, while placing the mind and keeping it connected. As the placing of the mind and keeping it connected are stilled, I enter and remain in the second absorption, which has the rapture and bliss born of immersion, with internal clarity and confidence, and unified mind, without placing the mind and keeping it connected. And with the fading away of rapture, I enter and remain in the third absorption, where I meditate with equanimity, mindful and aware, personally experiencing the bliss of which the noble ones declare, ‘Equanimous and mindful, one meditates in bliss.’ With the giving up of pleasure and pain, and the ending of former happiness and sadness, I enter and remain in the fourth absorption, without pleasure or pain, with pure equanimity and mindfulness. When I’m practicing like this, if I walk, at that time I walk like the gods. When I’m practicing like this, if I stand, at that time I stand like the gods. When I’m practicing like this, if I sit, at that time I sit like the gods. When I’m practicing like this, if I lie down, at that time I lie down like the gods. This is the high and luxurious bed of the gods that I get these days when I want, without trouble or difficulty.” 

“It’s\marginnote{7.1} incredible, Master Gotama, it’s amazing! Who but Master Gotama could get such a high and luxurious bed of the gods when he wants, without trouble or difficulty? 

But\marginnote{8.1} what, Master Gotama, is the high and luxurious bed of \textsanskrit{Brahmā}?” 

“Brahmin,\marginnote{8.2} when I am living supported by a village or town, I robe up in the morning and, taking my bowl and robe, enter the town or village for alms. After the meal, on my return from almsround, I enter within a forest. I gather up some grass or leaves into a pile, and sit down cross-legged, with my body straight, and establish mindfulness right there. I meditate spreading a heart full of love to one direction, and to the second, and to the third, and to the fourth. In the same way above, below, across, everywhere, all around, I spread a heart full of love to the whole world—abundant, expansive, limitless, free of enmity and ill will. I meditate spreading a heart full of compassion to one direction, and to the second, and to the third, and to the fourth. In the same way above, below, across, everywhere, all around, I spread a heart full of compassion to the whole world—abundant, expansive, limitless, free of enmity and ill will. I meditate spreading a heart full of rejoicing to one direction, and to the second, and to the third, and to the fourth. In the same way above, below, across, everywhere, all around, I spread a heart full of rejoicing to the whole world—abundant, expansive, limitless, free of enmity and ill will. I meditate spreading a heart full of equanimity to one direction, and to the second, and to the third, and to the fourth. In the same way above, below, across, everywhere, all around, I spread a heart full of equanimity to the whole world—abundant, expansive, limitless, free of enmity and ill will. When I’m practicing like this, if I walk, at that time I walk like \textsanskrit{Brahmā}. … I stand like \textsanskrit{Brahmā}. … I sit like \textsanskrit{Brahmā} … When I’m practicing like this, if I lie down, at that time I lie down like \textsanskrit{Brahmā}. This is the high and luxurious bed of \textsanskrit{Brahmā} that I get these days when I want, without trouble or difficulty.” 

“It’s\marginnote{9.1} incredible, Master Gotama, it’s amazing! Who but Master Gotama could get such a high and luxurious bed of \textsanskrit{Brahmā} when he wants, without trouble or difficulty? 

But\marginnote{10.1} what, Master Gotama, is the high and luxurious bed of the noble ones?” 

“Brahmin,\marginnote{10.2} when I am living supported by a village or town, I robe up in the morning and, taking my bowl and robe, enter the town or village for alms. After the meal, on my return from almsround, I enter within a forest. I gather up some grass or leaves into a pile, and sit down cross-legged, with my body straight, and establish mindfulness right there. I know this: ‘I’ve given up greed, hate, and delusion, cut them off at the root, made them like a palm stump, obliterated them, so they’re unable to arise in the future.’ When I’m practicing like this, if I walk, at that time I walk like the noble ones. … I stand like the noble ones … I sit like the noble ones … When I’m practicing like this, if I lie down, at that time I lie down like the noble ones. This is the high and luxurious bed of the noble ones that I get these days when I want, without trouble or difficulty.” 

“It’s\marginnote{11.1} incredible, Master Gotama, it’s amazing! Who but Master Gotama could get such a high and luxurious bed of the noble ones when he wants, without trouble or difficulty? 

Excellent,\marginnote{12.1} Master Gotama! Excellent! As if he were righting the overturned, or revealing the hidden, or pointing out the path to the lost, or lighting a lamp in the dark so people with good eyes can see what’s there, Master Gotama has made the teaching clear in many ways. We go for refuge to Master Gotama, to the teaching, and to the mendicant \textsanskrit{Saṅgha}. From this day forth, may Master Gotama remember us as lay followers who have gone for refuge for life.” 

%
\section*{{\suttatitleacronym AN 3.64}{\suttatitletranslation With Sarabha }{\suttatitleroot Sarabhasutta}}
\addcontentsline{toc}{section}{\tocacronym{AN 3.64} \toctranslation{With Sarabha } \tocroot{Sarabhasutta}}
\markboth{With Sarabha }{Sarabhasutta}
\extramarks{AN 3.64}{AN 3.64}

\scevam{So\marginnote{1.1} I have heard. }At one time the Buddha was staying near \textsanskrit{Rājagaha}, on the Vulture’s Peak Mountain. 

Now\marginnote{1.3} at that time a wanderer called Sarabha had recently left this teaching and training. He was telling a crowd in \textsanskrit{Rājagaha}, “I learned the teaching of the ascetics who follow the Sakyan, then I left their teaching and training.” 

Then\marginnote{1.7} several mendicants robed up in the morning and, taking their bowls and robes, entered \textsanskrit{Rājagaha} for alms. They heard what Sarabha was saying. 

Then,\marginnote{2.1} after the meal, when they returned from almsround, they went up to the Buddha, bowed, sat down to one side, and said to him, “The wanderer called Sarabha has recently left this teaching and training. He was telling a crowd in \textsanskrit{Rājagaha}: ‘I learned the teaching of the ascetics who follow the Sakyan, then I left their teaching and training.’ Sir, please go to the wanderers’ monastery on the banks of the \textsanskrit{Sappinī} river to see Sarabha the wanderer out of compassion.” The Buddha consented in silence. 

Then\marginnote{3.1} in the late afternoon, the Buddha came out of retreat and went to the wanderers’ monastery on the banks of the \textsanskrit{Sappinī} river to visit Sarabha the wanderer. He sat on the seat spread out, and said to the wanderer Sarabha, “Is it really true, Sarabha, that you’ve been saying: ‘I learned the teaching of the ascetics who follow the Sakyan, then I left their teaching and training.’” When he said this, Sarabha kept silent. 

For\marginnote{4.1} a second time, the Buddha said to Sarabha, “Tell me, Sarabha, what exactly have you learned of the teachings of the ascetics who follow the Sakyan? If you’ve not learned it fully, I’ll fill you in. But if you have learned it fully, I’ll agree.” For a second time, Sarabha kept silent. 

For\marginnote{5.1} a third time, the Buddha said to Sarabha, “Sarabha, the teachings of the ascetics who follow the Sakyan are clear to me. What exactly have you learned of the teachings of the ascetics who follow the Sakyan? If you’ve not learned it fully, I’ll fill you in. But if you have learned it fully, I’ll agree.” For a third time, Sarabha kept silent. 

Then\marginnote{6.1} those wanderers said to Sarabha, “The ascetic Gotama has offered to tell you anything you ask for. Speak, reverend Sarabha, what exactly have you learned of the teachings of the ascetics who follow the Sakyan? If you’ve not learned it fully, he’ll fill you in. But if you have learned it fully, he’ll agree.” When this was said, Sarabha sat silent, embarrassed, shoulders drooping, downcast, depressed, with nothing to say. 

Knowing\marginnote{7.1} this, the Buddha said to the wanderers: 

“Wanderers,\marginnote{8.1} someone might say to me: ‘You claim to be a fully awakened Buddha, but regarding these things you’re not fully awakened.’ Then I’d carefully pursue, press, and grill them on that point. When grilled by me, they would, without a doubt, fall into one of these three categories. They’d dodge the issue, distracting the discussion with irrelevant points. They’d display annoyance, hate, and bitterness. Or they’d sit silent, embarrassed, shoulders drooping, downcast, depressed, with nothing to say, like Sarabha. 

Wanderers,\marginnote{9.1} someone might say to me: ‘You claim to have ended all defilements, but you still have these defilements.’ Then I’d carefully pursue, press, and grill them on that point. When grilled by me, they would, without a doubt, fall into one of these three categories. They’d dodge the issue, distracting the discussion with irrelevant points. They’d display annoyance, hate, and bitterness. Or they’d sit silent, embarrassed, shoulders drooping, downcast, depressed, with nothing to say, like Sarabha. 

Wanderers,\marginnote{10.1} someone might say to me: ‘Your teaching does not lead someone who practices it to the goal of the complete ending of suffering.’ Then I’d carefully pursue, press, and grill them on that point. When grilled by me, they would, without a doubt, fall into one of these three categories. They’d dodge the issue, distracting the discussion with irrelevant points. They’d display annoyance, hate, and bitterness. Or they’d sit silent, embarrassed, shoulders drooping, downcast, depressed, with nothing to say, like Sarabha.” 

Then\marginnote{10.4} the Buddha, having roared his lion’s roar three times in the wanderers’ monastery on the bank of the \textsanskrit{Sappinī} river, rose into the sky and flew away. 

Soon\marginnote{11.1} after the Buddha left, those wanderers gave Sarabha a comprehensive tongue-lashing: “Reverend Sarabha, you’re just like an old jackal in the formidable wilderness who thinks, ‘I’ll roar a lion’s roar!’ but they still only manage to squeal and yelp like a jackal. In the same way, when the ascetic Gotama wasn’t here you said ‘I’ll roar a lion’s roar!’ but you only managed to squeal and yelp like a jackal. 

You’re\marginnote{11.4} just like a golden oriole who thinks, ‘I’ll cry like a cuckoo!’ but they still only manage to cry like a golden oriole. In the same way, when the ascetic Gotama wasn’t here you said ‘I’ll cry like a cuckoo!’ but you still only managed to cry like a golden oriole. 

You’re\marginnote{11.6} just like a bull that thinks to bellow only when the cowstall is empty. In the same way, you only thought to bellow when the ascetic Gotama wasn’t here.” That’s how those wanderers gave Sarabha a comprehensive tongue-lashing. 

%
\section*{{\suttatitleacronym AN 3.65}{\suttatitletranslation With the Kālāmas of Kesamutta }{\suttatitleroot Kesamuttisutta}}
\addcontentsline{toc}{section}{\tocacronym{AN 3.65} \toctranslation{With the Kālāmas of Kesamutta } \tocroot{Kesamuttisutta}}
\markboth{With the Kālāmas of Kesamutta }{Kesamuttisutta}
\extramarks{AN 3.65}{AN 3.65}

\scevam{So\marginnote{1.1} I have heard. }At one time the Buddha was wandering in the land of the Kosalans together with a large \textsanskrit{Saṅgha} of mendicants when he arrived at a town of the \textsanskrit{Kālāmas} named Kesamutta. The \textsanskrit{Kālāmas} of Kesamutta heard: 

“It\marginnote{1.4} seems the ascetic Gotama—a Sakyan, gone forth from a Sakyan family—has arrived at Kesamutta. He has this good reputation: ‘That Blessed One is perfected, a fully awakened Buddha …’ It’s good to see such perfected ones.” 

Then\marginnote{2.1} the \textsanskrit{Kālāmas} went up to the Buddha. Before sitting down to one side, some bowed, some exchanged greetings and polite conversation, some held up their joined palms toward the Buddha, some announced their name and clan, while some kept silent. Seated to one side the \textsanskrit{Kālāmas} said to the Buddha: 

“There\marginnote{3.1} are, sir, some ascetics and brahmins who come to Kesamutta. They explain and promote only their own doctrine, while they attack, badmouth, disparage, and smear the doctrines of others. Then some other ascetics and brahmins come to Kesamutta. They too explain and promote only their own doctrine, while they attack, badmouth, disparage, and smear the doctrines of others. So, sir, we’re doubting and uncertain: ‘I wonder who of these respected ascetics and brahmins speaks the truth, and who speaks falsehood?’” 

“It\marginnote{3.7} is enough, \textsanskrit{Kālāmas}, for you to be doubting and uncertain. Doubt has come up in you about an uncertain matter. 

Please,\marginnote{4.1} \textsanskrit{Kālāmas}, don’t go by oral transmission, don’t go by lineage, don’t go by testament, don’t go by canonical authority, don’t rely on logic, don’t rely on inference, don’t go by reasoned contemplation, don’t go by the acceptance of a view after consideration, don’t go by the appearance of competence, and don’t think ‘The ascetic is our respected teacher.’ But when you know for yourselves: ‘These things are unskillful, blameworthy, criticized by sensible people, and when you undertake them, they lead to harm and suffering’, then you should give them up. 

What\marginnote{5.1} do you think, \textsanskrit{Kālāmas}? Does greed come up in a person for their welfare or harm?” 

“Harm,\marginnote{6.1} sir.” 

“A\marginnote{7.1} greedy individual, overcome by greed, kills living creatures, steals, commits adultery, lies, and encourages others to do the same. Is that for their lasting harm and suffering?” 

“Yes,\marginnote{8.1} sir.” 

“What\marginnote{9.1} do you think, \textsanskrit{Kālāmas}? Does hate come up in a person for their welfare or harm?” 

“Harm,\marginnote{10.1} sir.” 

“A\marginnote{11.1} hateful individual, overcome by hate, kills living creatures, steals, commits adultery, lies, and encourages others to do the same. Is that for their lasting harm and suffering?” 

“Yes,\marginnote{12.1} sir.” 

“What\marginnote{13.1} do you think, \textsanskrit{Kālāmas}? Does delusion come up in a person for their welfare or harm?” 

“Harm,\marginnote{14.1} sir.” 

“A\marginnote{15.1} deluded individual, overcome by delusion, kills living creatures, steals, commits adultery, lies, and encourages others to do the same. Is that for their lasting harm and suffering?” 

“Yes,\marginnote{16.1} sir.” 

“What\marginnote{17.1} do you think, \textsanskrit{Kālāmas}, are these things skillful or unskillful?” 

“Unskillful,\marginnote{18.1} sir.” 

“Blameworthy\marginnote{19.1} or blameless?” 

“Blameworthy,\marginnote{20.1} sir.” 

“Criticized\marginnote{21.1} or praised by sensible people?” 

“Criticized\marginnote{22.1} by sensible people, sir.” 

“When\marginnote{23.1} you undertake them, do they lead to harm and suffering, or not? Or how do you see this?” 

“When\marginnote{24.1} you undertake them, they lead to harm and suffering. That’s how we see it.” 

“So,\marginnote{25.1} \textsanskrit{Kālāmas}, when I said: ‘Please, don’t go by oral transmission, don’t go by lineage, don’t go by testament, don’t go by canonical authority, don’t rely on logic, don’t rely on inference, don’t go by reasoned contemplation, don’t go by the acceptance of a view after consideration, don’t go by the appearance of competence, and don’t think “The ascetic is our respected teacher.” But when you know for yourselves: “These things are unskillful, blameworthy, criticized by sensible people, and when you undertake them, they lead to harm and suffering”, then you should give them up.’ That’s what I said, and this is why I said it. 

Please,\marginnote{26.1} \textsanskrit{Kālāmas}, don’t go by oral transmission, don’t go by lineage, don’t go by testament, don’t go by canonical authority, don’t rely on logic, don’t rely on inference, don’t go by reasoned contemplation, don’t go by the acceptance of a view after consideration, don’t go by the appearance of competence, and don’t think ‘The ascetic is our respected teacher.’ But when you know for yourselves: ‘These things are skillful, blameless, praised by sensible people, and when you undertake them, they lead to welfare and happiness’, then you should acquire them and keep them. 

What\marginnote{27.1} do you think, \textsanskrit{Kālāmas}? Does contentment come up in a person for their welfare or harm?” 

“Welfare,\marginnote{28.1} sir.” 

“An\marginnote{29.1} individual who is content, not overcome by greed, doesn’t kill living creatures, steal, commit adultery, lie, or encourage others to do the same. Is that for their lasting welfare and happiness?” 

“Yes,\marginnote{30.1} sir.” 

“What\marginnote{31.1} do you think, \textsanskrit{Kālāmas}? Does love come up in a person for their welfare or harm? … Does understanding come up in a person for their welfare or harm? … Is that for their lasting welfare and happiness?” 

“Yes,\marginnote{32.1} sir.” 

“What\marginnote{33.1} do you think, \textsanskrit{Kālāmas}, are these things skillful or unskillful?” 

“Skillful,\marginnote{34.1} sir.” 

“Blameworthy\marginnote{35.1} or blameless?” 

“Blameless,\marginnote{36.1} sir.” 

“Criticized\marginnote{37.1} or praised by sensible people?” 

“Praised\marginnote{38.1} by sensible people, sir.” 

“When\marginnote{39.1} you undertake them, do they lead to welfare and happiness, or not? Or how do you see this?” 

“When\marginnote{40.1} you undertake them, they lead to welfare and happiness. That’s how we see it.” 

“So,\marginnote{41.1} \textsanskrit{Kālāmas}, when I said: ‘Please, don’t go by oral transmission, don’t go by lineage, don’t go by testament, don’t go by canonical authority, don’t rely on logic, don’t rely on inference, don’t go by reasoned contemplation, don’t go by the acceptance of a view after consideration, don’t go by the appearance of competence, and don’t think “The ascetic is our respected teacher.” But when you know for yourselves: 

“These\marginnote{41.5} things are skillful, blameless, praised by sensible people, and when you undertake them, they lead to welfare and happiness”, then you should acquire them and keep them.’ That’s what I said, and this is why I said it. 

Then\marginnote{42.1} that noble disciple is rid of desire, rid of ill will, unconfused, aware, and mindful. They meditate spreading a heart full of love to one direction, and to the second, and to the third, and to the fourth. In the same way above, below, across, everywhere, all around, they spread a heart full of love to the whole world—abundant, expansive, limitless, free of enmity and ill will. 

They\marginnote{42.2} meditate spreading a heart full of compassion to one direction, and to the second, and to the third, and to the fourth. In the same way above, below, across, everywhere, all around, they spread a heart full of compassion to the whole world—abundant, expansive, limitless, free of enmity and ill will. 

They\marginnote{42.3} meditate spreading a heart full of rejoicing to one direction, and to the second, and to the third, and to the fourth. In the same way above, below, across, everywhere, all around, they spread a heart full of rejoicing to the whole world—abundant, expansive, limitless, free of enmity and ill will. 

They\marginnote{42.4} meditate spreading a heart full of equanimity to one direction, and to the second, and to the third, and to the fourth. In the same way above, below, across, everywhere, all around, they spread a heart full of equanimity to the whole world—abundant, expansive, limitless, free of enmity and ill will. 

When\marginnote{43.1} that noble disciple has a mind that’s free of enmity and ill will, uncorrupted and purified, they’ve won four consolations in the present life. ‘If it turns out there is another world, and good and bad deeds have a result, then—when the body breaks up, after death—I’ll be reborn in a good place, a heavenly realm.’ This is the first consolation they’ve won. 

‘If\marginnote{44.1} it turns out there is no other world, and good and bad deeds don’t have a result, then in the present life I’ll keep myself free of enmity and ill will, untroubled and happy.’ This is the second consolation they’ve won. 

‘If\marginnote{45.1} it turns out that bad things happen to people who do bad things, then since I have no bad intentions, and since I’m not doing anything bad, how can suffering touch me?’ This is the third consolation they’ve won. 

‘If\marginnote{46.1} it turns out that bad things don’t happen to people who do bad things, then I still see myself pure on both sides.’ This is the fourth consolation they’ve won. 

When\marginnote{47.1} that noble disciple has a mind that’s free of enmity and ill will, undefiled and purified, they’ve won these four consolations in the present life.” 

“That’s\marginnote{48.1} so true, Blessed One! That’s so true, Holy One! When that noble disciple has a mind that’s free of enmity and ill will, undefiled and purified, they’ve won these four consolations in the present life. … 

Excellent,\marginnote{53.1} sir! Excellent! … We go for refuge to Master Gotama, to the teaching, and to the mendicant \textsanskrit{Saṅgha}. From this day forth, may the Buddha remember us as lay followers who have gone for refuge for life.” 

%
\section*{{\suttatitleacronym AN 3.66}{\suttatitletranslation With Sāḷha and His Friend }{\suttatitleroot Sāḷhasutta}}
\addcontentsline{toc}{section}{\tocacronym{AN 3.66} \toctranslation{With Sāḷha and His Friend } \tocroot{Sāḷhasutta}}
\markboth{With Sāḷha and His Friend }{Sāḷhasutta}
\extramarks{AN 3.66}{AN 3.66}

\scevam{So\marginnote{1.1} I have heard. }Now at that time Venerable Nandaka was staying near \textsanskrit{Sāvatthī} in the Eastern Monastery, the stilt longhouse of \textsanskrit{Migāra}’s mother. Then \textsanskrit{Sāḷha}, \textsanskrit{Migāra}’s grandson, and \textsanskrit{Rohaṇa}, \textsanskrit{Pekhuṇiya}’s grandson went up to Venerable Nandaka, bowed, and sat down to one side. Then Venerable Nandaka said to \textsanskrit{Sāḷha}: 

“Please,\marginnote{2.1} \textsanskrit{Sāḷha} and friend, don’t go by oral transmission, don’t go by lineage, don’t go by testament, don’t go by canonical authority, don’t rely on logic, don’t rely on inference, don’t go by reasoned contemplation, don’t go by the acceptance of a view after consideration, don’t go by the appearance of competence, and don’t think ‘The ascetic is our respected teacher.’ But when you know for yourselves: ‘These things are unskillful, blameworthy, criticized by sensible people, and when you undertake them, they lead to harm and suffering’, then you should give them up. 

What\marginnote{3.1} do you think, \textsanskrit{Sāḷha}? Is greed real?” 

“Yes,\marginnote{4.1} sir.” 

“‘Covetousness’\marginnote{5.1} is what I mean by this. A person who is greedy and covetous kills living creatures, steals, commits adultery, lies, and encourages others to do the same. Is that for their lasting harm and suffering?” 

“Yes,\marginnote{6.1} sir.” 

“What\marginnote{7.1} do you think, \textsanskrit{Sāḷha}? Is hate real?” 

“Yes,\marginnote{8.1} sir.” 

“‘Malice’\marginnote{9.1} is what I mean by this. A hateful and malicious person kills living creatures, steals, commits adultery, lies, and encourages others to do the same. Is that for their lasting harm and suffering?” 

“Yes,\marginnote{10.1} sir.” 

“What\marginnote{11.1} do you think, \textsanskrit{Sāḷha}? Is delusion real?” 

“Yes,\marginnote{12.1} sir.” 

“‘Ignorance’\marginnote{13.1} is what I mean by this. A person who is deluded and ignorant kills living creatures, steals, commits adultery, lies, and encourages others to do the same. Is that for their lasting harm and suffering?” 

“Yes,\marginnote{14.1} sir.” 

“What\marginnote{15.1} do you think, \textsanskrit{Sāḷha}, are these things skillful or unskillful?” 

“Unskillful,\marginnote{16.1} sir.” 

“Blameworthy\marginnote{17.1} or blameless?” 

“Blameworthy,\marginnote{18.1} sir.” 

“Criticized\marginnote{19.1} or praised by sensible people?” 

“Criticized\marginnote{20.1} by sensible people, sir.” 

“When\marginnote{21.1} you undertake them, do they lead to harm and suffering, or not? Or how do you see this?” 

“When\marginnote{22.1} you undertake them, they lead to harm and suffering. That’s how we see it.” 

“So,\marginnote{23.1} \textsanskrit{Sāḷha} and friend, when I said: ‘Please, don’t go by oral transmission, don’t go by lineage, don’t go by testament, don’t go by canonical authority, don’t rely on logic, don’t rely on inference, don’t go by reasoned contemplation, don’t go by the acceptance of a view after consideration, don’t go by the appearance of competence, and don’t think “The ascetic is our respected teacher.” But when you know for yourselves: “These things are unskillful, blameworthy, criticized by sensible people, and when you undertake them, they lead to harm and suffering”, then you should give them up.’ That’s what I said, and this is why I said it. 

Please,\marginnote{24.1} \textsanskrit{Sāḷha} and friend, don’t go by oral transmission, don’t go by lineage, don’t go by testament, don’t go by canonical authority, don’t rely on logic, don’t rely on inference, don’t go by reasoned contemplation, don’t go by the acceptance of a view after consideration, don’t go by the appearance of competence, and don’t think ‘The ascetic is our respected teacher.’ But when you know for yourselves: ‘These things are skillful, blameless, praised by sensible people, and when you undertake them, they lead to welfare and happiness’, then you should acquire them and keep them. 

What\marginnote{25.1} do you think? Is contentment real?” 

“Yes,\marginnote{26.1} sir.” 

“‘Satisfaction’\marginnote{27.1} is what I mean by this. A person who is content and satisfied doesn’t kill living creatures, steal, commit adultery, lie, or encourage others to do the same. Is that for their lasting welfare and happiness?” 

“Yes,\marginnote{28.1} sir.” 

What\marginnote{29.1} do you think? Is love real?” 

“Yes,\marginnote{30.1} sir.” 

“‘Kindness’\marginnote{31.1} is what I mean by this. A loving and kind-hearted person doesn’t kill living creatures, steal, commit adultery, lie, or encourage others to do the same. Is that for their lasting welfare and happiness?” 

“Yes,\marginnote{32.1} sir.” 

“What\marginnote{33.1} do you think, \textsanskrit{Sāḷha}? Is understanding real?” 

“Yes,\marginnote{34.1} sir.” 

“‘Knowledge’\marginnote{35.1} is what I mean by this. A person who understands and knows doesn’t kill living creatures, steal, commit adultery, lie, or encourage others to do the same. Is that for their lasting welfare and happiness?” 

“Yes,\marginnote{36.1} sir.” 

“What\marginnote{37.1} do you think, \textsanskrit{Sāḷha}, are these things skillful or unskillful?” 

“Skillful,\marginnote{38.1} sir.” 

“Blameworthy\marginnote{39.1} or blameless?” 

“Blameless,\marginnote{40.1} sir.” 

“Criticized\marginnote{41.1} or praised by sensible people?” 

“Praised\marginnote{42.1} by sensible people, sir.” 

“When\marginnote{43.1} you undertake them, do they lead to welfare and happiness, or not? Or how do you see this?” 

“When\marginnote{44.1} you undertake them, they lead to welfare and happiness. That’s how we see it.” 

“So,\marginnote{45.1} \textsanskrit{Sāḷha} and friend, when I said: ‘Please, don’t go by oral transmission, don’t go by lineage, don’t go by testament, don’t go by canonical authority, don’t rely on logic, don’t rely on inference, don’t go by reasoned contemplation, don’t go by the acceptance of a view after consideration, don’t go by the appearance of competence, and don’t think “The ascetic is our respected teacher.” But when you know for yourselves: 

“These\marginnote{45.4} things are skillful, blameless, praised by sensible people, and when you undertake them, they lead to welfare and happiness”, then you should acquire them and keep them.’ That’s what I said, and this is why I said it. 

Then\marginnote{46.1} that noble disciple is rid of desire, rid of ill will, unconfused, aware, and mindful. They meditate spreading a heart full of love … compassion … rejoicing … equanimity to one direction, and to the second, and to the third, and to the fourth. In the same way above, below, across, everywhere, all around, they spread a heart full of equanimity to the whole world—abundant, expansive, limitless, free of enmity and ill will. 

They\marginnote{46.5} understand: ‘There is this, there is what is worse than this, there is what is better than this, and there is an escape beyond the scope of perception.’ Knowing and seeing like this, their mind is freed from the defilements of sensuality, desire to be reborn, and ignorance. When they’re freed, they know they’re freed. 

They\marginnote{46.9} understand: ‘Rebirth is ended, the spiritual journey has been completed, what had to be done has been done, there is no return to any state of existence.’ 

They\marginnote{47.1} understand: ‘Formerly there was greed, which was unskillful. Now there is none, so that’s skillful. Formerly there was hate, which was unskillful. Now there is none, so that’s skillful. Formerly there was delusion, which was unskillful. Now there is none, so that’s skillful.’ So they live without wishes in the present life, extinguished, cooled, experiencing bliss, having become holy in themselves.” 

%
\section*{{\suttatitleacronym AN 3.67}{\suttatitletranslation Topics of Discussion }{\suttatitleroot Kathāvatthusutta}}
\addcontentsline{toc}{section}{\tocacronym{AN 3.67} \toctranslation{Topics of Discussion } \tocroot{Kathāvatthusutta}}
\markboth{Topics of Discussion }{Kathāvatthusutta}
\extramarks{AN 3.67}{AN 3.67}

“There\marginnote{1.1} are, mendicants, these three topics of discussion. What three? You might discuss the past: ‘That is how it was in the past.’ You might discuss the future: ‘That is how it will be in the future.’ Or you might discuss the present: ‘This is how it is at present.’ 

You\marginnote{2.1} can know whether or not a person is competent to hold a discussion by seeing how they take part in a discussion. When a person is asked a question, if it needs to be answered definitively and they don’t answer it definitively; or if it needs analysis and they answer without analyzing it; or if it needs a counter-question and they answer without a counter-question; or if it should be set aside and they don’t set it aside, then that person is not competent to hold a discussion. When a person is asked a question, if it needs to be answered definitively and they answer it definitively; or if it needs analysis and they answer after analyzing it; or if it needs a counter-question and they answer with a counter-question; or if it should be set aside and they set it aside, then that person is competent to hold a discussion. 

You\marginnote{3.1} can know whether or not a person is competent to hold a discussion by seeing how they take part in a discussion. When a person is asked a question, if they’re not consistent about what their position is and what it isn’t; about what they propose; about speaking from what they know; and about the appropriate procedure, then that person is not competent to hold a discussion. When a person is asked a question, if they are consistent about what their position is and what it isn’t; about what they propose; about speaking from what they know; and about the appropriate procedure, then that person is competent to hold a discussion. 

You\marginnote{4.1} can know whether or not a person is competent to hold a discussion by seeing how they take part in a discussion. When a person is asked a question, if they dodge the issue; distract the discussion with irrelevant points; or display annoyance, hate, and bitterness, then that person is not competent to hold a discussion. When a person is asked a question, if they don’t dodge the issue; distract the discussion with irrelevant points; or display annoyance, hate, and bitterness, then that person is competent to hold a discussion. 

You\marginnote{5.1} can know whether or not a person is competent to hold a discussion by seeing how they take part in a discussion. When a person is asked a question, if they intimidate, crush, mock, or seize on trivial mistakes, then that person is not competent to hold a discussion. When a person is asked a question, if they don’t intimidate, crush, mock, or seize on trivial mistakes, then that person is competent to hold a discussion. 

You\marginnote{6.1} can know whether or not a person has what’s required by seeing how they take part in a discussion. If they lend an ear they have what’s required; if they don’t lend an ear they don’t have what’s required. Someone who has what’s required directly knows one thing, completely understands one thing, gives up one thing, and realizes one thing—and then they experience complete freedom. This is the purpose of discussion, consultation, the requirements, and listening well, that is, the liberation of the mind by not grasping. 

\begin{verse}%
Those\marginnote{7.1} who converse with hostility, \\
too sure of themselves, arrogant, \\
ignoble, attacking virtues, \\
they look for flaws in each other. 

They\marginnote{8.1} rejoice together when their opponent \\
speaks poorly and makes a mistake, \\
becoming confused and defeated—\\
but the noble ones don’t discuss like this. 

If\marginnote{9.1} an astute person wants to hold a discussion \\
connected with the teaching and its meaning—\\
the kind of discussion that noble ones hold—\\
then that wise one should start the discussion, 

knowing\marginnote{10.1} when the time is right, \\
neither hostile nor arrogant. \\
Not over-excited, \\
contemptuous, or aggressive, 

or\marginnote{11.1} with a mind full of jealousy, \\
they’d speak from what they rightly know. \\
They agree with what was well spoken, \\
without criticizing what was poorly said. 

They’d\marginnote{12.1} not persist in finding faults, \\
nor seize on trivial mistakes, \\
neither intimidating nor crushing the other, \\
nor would they speak suggestively. 

Good\marginnote{13.1} people consult \\
for the sake of knowledge and clarity. \\
That’s how the noble ones consult, \\
this is a noble consultation. \\
Knowing this, an intelligent person \\
would consult without arrogance.” 

%
\end{verse}

%
\section*{{\suttatitleacronym AN 3.68}{\suttatitletranslation Followers of Other Paths }{\suttatitleroot Aññatitthiyasutta}}
\addcontentsline{toc}{section}{\tocacronym{AN 3.68} \toctranslation{Followers of Other Paths } \tocroot{Aññatitthiyasutta}}
\markboth{Followers of Other Paths }{Aññatitthiyasutta}
\extramarks{AN 3.68}{AN 3.68}

“Mendicants,\marginnote{1.1} if wanderers who follow other paths were to ask: ‘There are these three things. What three? Greed, hate, and delusion. These are the three things. What’s the difference between them?’ How would you answer them?” 

“Our\marginnote{1.8} teachings are rooted in the Buddha. He is our guide and our refuge. Sir, may the Buddha himself please clarify the meaning of this. The mendicants will listen and remember it.” 

“Well\marginnote{1.9} then, mendicants, listen and pay close attention, I will speak.” 

“Yes,\marginnote{1.10} sir,” they replied. The Buddha said this: 

“Mendicants,\marginnote{2.1} if wanderers who follow other paths were to ask: ‘There are these three things. What three? Greed, hate, and delusion. These are the three things. What’s the difference between them?’ You should answer them: ‘Greed, reverends, is mildly blameworthy, but slow to fade away. Hate is very blameworthy, but quick to fade away. Delusion is very blameworthy, and slow to fade away.’ 

And\marginnote{3.1} if they ask: ‘What is the cause, what is the reason why greed arises, and once arisen it increases and grows?’ You should say: ‘The beautiful feature of things. When you attend improperly to the beautiful feature of things, greed arises, and once arisen it increases and grows. This is the cause, this is the reason why greed arises, and once arisen it increases and grows.’ 

And\marginnote{4.1} if they ask: ‘What is the cause, what is the reason why hate arises, and once arisen it increases and grows?’ You should say: ‘The feature of harshness. When you attend improperly to the feature of harshness, hate arises, and once arisen it increases and grows. This is the cause, this is the reason why hate arises, and once arisen it increases and grows.’ 

And\marginnote{5.1} if they ask: ‘What is the cause, what is the reason why delusion arises, and once arisen it increases and grows?’ You should say: ‘Improper attention. When you attend improperly, delusion arises, and once arisen it increases and grows. This is the cause, this is the reason why delusion arises, and once arisen it increases and grows.’ 

And\marginnote{6.1} if they ask, ‘What is the cause, what is the reason why greed doesn’t arise, or if it’s already arisen it’s given up?’ You should say: ‘The ugly feature of things. When you attend properly on the ugly feature of things, greed doesn’t arise, or if it’s already arisen it’s given up. This is the cause, this is the reason why greed doesn’t arise, or if it’s already arisen it’s given up.’ 

And\marginnote{7.1} if they ask, ‘What is the cause, what is the reason why hate doesn’t arise, or if it’s already arisen it’s given up?’ You should say: ‘The heart’s release by love.’ When you attend properly on the heart’s release by love, hate doesn’t arise, or if it’s already arisen it’s given up. This is the cause, this is the reason why hate doesn’t arise, or if it’s already arisen it’s given up.’ 

And\marginnote{8.1} if they ask, ‘What is the cause, what is the reason why delusion doesn’t arise, or if it’s already arisen it’s given up?’ You should say: ‘Proper attention. When you attend properly, delusion doesn’t arise, or if it’s already arisen it’s given up. This is the cause, this is the reason why delusion doesn’t arise, or if it’s already arisen it’s given up.’” 

%
\section*{{\suttatitleacronym AN 3.69}{\suttatitletranslation Unskillful Roots }{\suttatitleroot Akusalamūlasutta}}
\addcontentsline{toc}{section}{\tocacronym{AN 3.69} \toctranslation{Unskillful Roots } \tocroot{Akusalamūlasutta}}
\markboth{Unskillful Roots }{Akusalamūlasutta}
\extramarks{AN 3.69}{AN 3.69}

“Mendicants,\marginnote{1.1} there are these three unskillful roots. What three? Greed, hate, and delusion. 

Greed\marginnote{2.1} is a root of the unskillful. When a greedy person chooses to act by way of body, speech, or mind, that too is unskillful. When a greedy person, overcome by greed, causes another to suffer under a false pretext—by execution or imprisonment or confiscation or condemnation or banishment—thinking ‘I’m powerful, I want power’, that too is unskillful. And so these many bad, unskillful things are produced in them, born, sourced, originated, and conditioned by greed. 

Hate\marginnote{3.1} is a root of the unskillful. When a hateful person chooses to act by way of body, speech, or mind, that too is unskillful. When a hateful person, overcome by hate, causes another to suffer under a false pretext—by execution or imprisonment or confiscation or condemnation or banishment—thinking ‘I’m powerful, I want power’, that too is unskillful. And so these many bad, unskillful things are produced in them, born, sourced, originated, and conditioned by hate. 

Delusion\marginnote{4.1} is a root of the unskillful. When a deluded person chooses to act by way of body, speech, or mind, that too is unskillful. When a deluded person, overcome by delusion, causes another to suffer under a false pretext—by execution or imprisonment or confiscation or condemnation or banishment—thinking ‘I’m powerful, I want power’, that too is unskillful. And so these many bad, unskillful things are produced in them, born, sourced, originated, and conditioned by delusion. Such a person is said to have speech that’s ill-timed, false, meaningless, not in line with the teaching and training. 

Why\marginnote{5.1} is this? This person causes another to suffer under a false pretext—by execution or imprisonment or confiscation or condemnation or banishment—thinking ‘I’m powerful, I want power’. So when someone makes a valid criticism, they’re scornful and admit nothing. When someone makes a baseless criticism, they make no effort to explain, ‘This is why that’s untrue, this is why that’s false.’ That’s why such a person is said to have speech that’s ill-timed, false, meaningless, not in line with the teaching and training. 

Such\marginnote{6.1} a person—overcome with bad, unskillful qualities born of greed, hate, and delusion—suffers in the present life, with anguish, distress, and fever. And when the body breaks up, after death, they can expect to be reborn in a bad place. 

Suppose\marginnote{7.1} a sal, axlewood, or boxwood tree was choked and engulfed by three camel’s foot creepers. It would fall to ruin and disaster. In the same way, such a person—overcome with bad, unskillful qualities born of greed, hate, and delusion—suffers in the present life, with anguish, distress, and fever. And when the body breaks up, after death, they can expect to be reborn in a bad place. 

These\marginnote{8.1} are the three unskillful roots. 

There\marginnote{9.1} are these three skillful roots. What three? Contentment, love, and understanding. 

Contentment\marginnote{10.1} is a root of the skillful. When a contented person chooses to act by way of body, speech, or mind, that too is skillful. When a contented person, not overcome by greed, doesn’t cause another to suffer under a false pretext—by execution or imprisonment or confiscation or condemnation or banishment—thinking ‘I’m powerful, I want power’, that too is skillful. And so these many skillful things are produced in them, born, sourced, originated, and conditioned by contentment. 

Love\marginnote{11.1} is a root of the skillful. When a loving person chooses to act by way of body, speech, or mind, that too is skillful. When a loving person, not overcome by hate, doesn’t cause another to suffer under a false pretext—by execution or imprisonment or confiscation or condemnation or banishment—thinking ‘I’m powerful, I want power’, that too is skillful. And so these many skillful things are produced in them, born, sourced, originated, and conditioned by love. 

Understanding\marginnote{12.1} is a root of the skillful. When an understanding person chooses to act by way of body, speech, or mind, that too is skillful. When an understanding person, not overcome by delusion, doesn’t cause another to suffer under a false pretext—by execution or imprisonment or confiscation or condemnation or banishment—thinking ‘I’m powerful, I want power’, that too is skillful. And so these many skillful things are produced in them, born, sourced, originated, and conditioned by understanding. Such a person is said to have speech that’s well-timed, true, meaningful, in line with the teaching and training. 

Why\marginnote{13.1} is this? This person doesn’t cause another to suffer under a false pretext—by execution or imprisonment or confiscation or condemnation or banishment—thinking ‘I’m powerful, I want power’. So when someone makes a valid criticism, they admit it and aren’t scornful. When someone makes a baseless criticism, they make an effort to explain, ‘This is why that’s untrue, this is why that’s false.’ That’s why such a person is said to have speech that’s well-timed, true, meaningful, in line with the teaching and training. 

For\marginnote{14.1} such a person, bad unskillful qualities born of greed, hate, and delusion are cut off at the root, made like a palm stump, obliterated, and unable to arise in the future. In the present life they’re happy, free of anguish, distress, and fever, and they’re also extinguished in the present life. 

Suppose\marginnote{15.1} a sal, axlewood, or boxwood tree was choked and engulfed by three camel’s foot creepers. Then along comes a person with a spade and basket. They’d cut the creeper out by the roots, dig them up, and pull them out, down to the fibers and stems. Then they’d split the creeper apart, cut up the parts, and chop it into splinters. They’d dry the splinters in the wind and sun, burn them with fire, and reduce them to ashes. Then they’d sweep away the ashes in a strong wind, or float them away down a swift stream. In the same way, for such a person, bad unskillful qualities born of greed, hate, and delusion are cut off at the root, made like a palm stump, obliterated, and unable to arise in the future. In the present life they’re happy, free of anguish, distress, and fever, and they’re also extinguished in the present life. 

These\marginnote{16.1} are the three skillful roots.” 

%
\section*{{\suttatitleacronym AN 3.70}{\suttatitletranslation Sabbath }{\suttatitleroot Uposathasutta}}
\addcontentsline{toc}{section}{\tocacronym{AN 3.70} \toctranslation{Sabbath } \tocroot{Uposathasutta}}
\markboth{Sabbath }{Uposathasutta}
\extramarks{AN 3.70}{AN 3.70}

\scevam{So\marginnote{1.1} I have heard. }At one time the Buddha was staying near \textsanskrit{Sāvatthī} in the Eastern Monastery, the stilt longhouse of \textsanskrit{Migāra}’s mother. 

Then\marginnote{1.3} \textsanskrit{Visākhā}, \textsanskrit{Migāra}’s mother, went up to the Buddha, bowed, and sat down to one side. The Buddha said to her, “So, \textsanskrit{Visākhā}, where are you coming from in the middle of the day?” 

“Today,\marginnote{1.5} sir, I’m observing the sabbath.” 

“There\marginnote{2.1} are, \textsanskrit{Visākhā}, these three sabbaths. What three? The sabbath of the cowherds, the sabbath of the Jains, and the sabbath of the noble ones. 

And\marginnote{2.4} what is the sabbath of the cowherds? It’s just like a cowherd who, in the late afternoon, takes the cows back to their owners. They reflect: ‘Today the cows grazed in this place and that, and they drank in this place and that. Tomorrow the cows will graze in this place and that, and drink in this place and that.’ In the same way, someone keeping the sabbath reflects: ‘Today I ate this and that, and had a meal of this and that. Tomorrow I’ll eat this and that, and have a meal of this and that.’ And so they spend their day with a mind full of covetousness. That’s the sabbath of the cowherds. When the cowherd’s sabbath is observed like this it’s not very fruitful or beneficial or splendid or bountiful. 

And\marginnote{3.1} what is the sabbath of the Jains? There’s a kind of ascetic belonging to a group called the Jains. They encourage their disciples: ‘Please, good people, don’t hurt any living creatures more than a hundred leagues away to the east. Don’t hurt any living creatures more than a hundred leagues away to the west. Don’t hurt any living creatures more than a hundred leagues away to the north. Don’t hurt any living creatures more than a hundred leagues away to the south.’ So they encourage kindness and compassion for some creatures and not others. On the sabbath, they encourage their disciples: ‘Please, good people, take off all your clothes and say: “I don’t belong to anyone anywhere! And nothing belongs to me anywhere!”’ But their mother and father still know, ‘This is our child.’ And they know, ‘This is my mother and father.’ Partner and child still know, ‘This is our supporter.’ And they know, ‘This is my partner and child.’ Bondservants, workers, and staff still know: ‘This is our master.’ And they know, ‘These are my bondservants, workers, and staff.’ So, at a time when they should be encouraged to speak the truth, the Jains encourage them to lie. This, I say, is lying. When the night has passed they use their possessions once more, though they’ve not been given back to them. This, I say, is stealing. That’s the sabbath of the Jains. When the Jain’s sabbath is observed like this it’s not very fruitful or beneficial or splendid or bountiful. 

And\marginnote{4.1} what is the sabbath of the noble ones? A corrupt mind is cleaned by applying effort. And how is a corrupt mind cleaned by applying effort? It’s when a noble disciple recollects the Realized One: ‘That Blessed One is perfected, a fully awakened Buddha, accomplished in knowledge and conduct, holy, knower of the world, supreme guide for those who wish to train, teacher of gods and humans, awakened, blessed.’ As they recollect the Realized One, their mind becomes clear, joy arises, and mental corruptions are given up. It’s just like cleaning a dirty head by applying effort. 

And\marginnote{5.1} how is a dirty head cleaned by applying effort? With cleansing paste, clay, and water, and by applying the appropriate effort. In the same way, a corrupt mind is cleaned by applying effort. 

And\marginnote{6.1} how is a corrupt mind cleaned by applying effort? It’s when a noble disciple recollects the Realized One: ‘That Blessed One is perfected, a fully awakened Buddha, accomplished in knowledge and conduct, holy, knower of the world, supreme guide for those who wish to train, teacher of gods and humans, awakened, blessed.’ As they recollect the Realized One, their mind becomes clear, joy arises, and mental corruptions are given up. This is called: ‘A noble disciple who observes the sabbath of \textsanskrit{Brahmā}, living together with \textsanskrit{Brahmā}. And because they think of \textsanskrit{Brahmā} their mind becomes clear, joy arises, and mental corruptions are given up.’ That’s how a corrupt mind is cleaned by applying effort. 

A\marginnote{7.1} corrupt mind is cleaned by applying effort. And how is a corrupt mind cleaned by applying effort? It’s when a noble disciple recollects the teaching: ‘The teaching is well explained by the Buddha—visible in this very life, immediately effective, inviting inspection, relevant, so that sensible people can know it for themselves.’ As they recollect the teaching, their mind becomes clear, joy arises, and mental corruptions are given up. It’s just like cleaning a dirty body by applying effort. 

And\marginnote{8.1} how is a dirty body cleaned by applying effort? With pastes of powdered shells and herbs, water, and by applying the appropriate effort. That’s how a dirty body is cleaned by applying effort. In the same way, a corrupt mind is cleaned by applying effort. 

And\marginnote{9.1} how is a corrupt mind cleaned by applying effort? It’s when a noble disciple recollects the teaching: ‘The teaching is well explained by the Buddha—visible in this very life, immediately effective, inviting inspection, relevant, so that sensible people can know it for themselves.’ As they recollect the teaching, their mind becomes clear, joy arises, and mental corruptions are given up. This is called: ‘A noble disciple who observes the sabbath of Dhamma, living together with Dhamma. And because they think of the Dhamma their mind becomes clear, joy arises, and mental corruptions are given up.’ That’s how a corrupt mind is cleaned by applying effort. 

A\marginnote{10.1} corrupt mind is cleaned by applying effort. And how is a corrupt mind cleaned by applying effort? It’s when a noble disciple recollects the \textsanskrit{Saṅgha}: ‘The \textsanskrit{Saṅgha} of the Buddha’s disciples is practicing the way that’s good, direct, methodical, and proper. It consists of the four pairs, the eight individuals. This is the \textsanskrit{Saṅgha} of the Buddha’s disciples that is worthy of offerings dedicated to the gods, worthy of hospitality, worthy of a religious donation, worthy of greeting with joined palms, and is the supreme field of merit for the world.’ As they recollect the \textsanskrit{Saṅgha}, their mind becomes clear, joy arises, and mental corruptions are given up. It’s just like cleaning a dirty cloth by applying effort. 

And\marginnote{11.1} how is a dirty cloth cleaned by applying effort? With salt, lye, cow dung, and water, and by applying the appropriate effort. That’s how a dirty cloth is cleaned by applying effort. In the same way, a corrupt mind is cleaned by applying effort. 

And\marginnote{12.1} how is a corrupt mind cleaned by applying effort? It’s when a noble disciple recollects the \textsanskrit{Saṅgha}: ‘The \textsanskrit{Saṅgha} of the Buddha’s disciples is practicing the way that’s good, direct, methodical, and proper. It consists of the four pairs, the eight individuals. This \textsanskrit{Saṅgha} of the Buddha’s disciples is worthy of offerings dedicated to the gods, worthy of hospitality, worthy of a religious donation, and worthy of veneration with joined palms. It is the supreme field of merit for the world.’ As they recollect the \textsanskrit{Saṅgha}, their mind becomes clear, joy arises, and mental corruptions are given up. This is called: ‘A noble disciple who observes the sabbath of the \textsanskrit{Saṅgha}, living together with the \textsanskrit{Saṅgha}. And because they think of the \textsanskrit{Saṅgha} their mind becomes clear, joy arises, and mental corruptions are given up.’ That’s how a corrupt mind is cleaned by applying effort. 

A\marginnote{13.1} corrupt mind is cleaned by applying effort. And how is a corrupt mind cleaned by applying effort? It’s when a noble disciple recollects their own ethical conduct, which is unbroken, impeccable, spotless, and unmarred, liberating, praised by sensible people, not mistaken, and leading to immersion. As they recollect their ethical conduct, their mind becomes clear, joy arises, and mental corruptions are given up. It’s just like cleaning a dirty mirror by applying effort. 

And\marginnote{14.1} how is a dirty mirror cleaned by applying effort? With oil, ash, a rolled-up cloth, and by applying the appropriate effort. That’s how a dirty mirror is cleaned by applying effort. In the same way, a corrupt mind is cleaned by applying effort. 

And\marginnote{15.1} how is a corrupt mind cleaned by applying effort? It’s when a noble disciple recollects their own ethical conduct, which is unbroken, impeccable, spotless, and unmarred, liberating, praised by sensible people, not mistaken, and leading to immersion. As they recollect their ethical conduct, their mind becomes clear, joy arises, and mental corruptions are given up. This is called: ‘A noble disciple who observes the sabbath of ethical conduct, living together with ethics. And because they think of their ethical conduct their mind becomes clear, joy arises, and mental corruptions are given up.’ That’s how a corrupt mind is cleaned by applying effort. 

A\marginnote{16.1} corrupt mind is cleaned by applying effort. And how is a corrupt mind cleaned by applying effort? It’s when a noble disciple recollects the deities: ‘There are the Gods of the Four Great Kings, the Gods of the Thirty-Three, the Gods of Yama, the Joyful Gods, the Gods Who Love to Create, the Gods Who Control the Creations of Others, the Gods of \textsanskrit{Brahmā}’s Host, and gods even higher than these. When those deities passed away from here, they were reborn there because of their faith, ethics, learning, generosity, and wisdom. I, too, have the same kind of faith, ethics, learning, generosity, and wisdom.’ As they recollect the faith, ethics, learning, generosity, and wisdom of both themselves and those deities, their mind becomes clear, joy arises, and mental corruptions are given up. It’s just like cleaning dirty gold by applying effort. 

And\marginnote{17.1} how is dirty gold cleaned by applying effort? With a furnace, flux, a blowpipe, and tongs, and by applying the appropriate effort. That’s how dirty gold is cleaned by applying effort. In the same way, a corrupt mind is cleaned by applying effort. 

And\marginnote{18.1} how is a corrupt mind cleaned by applying effort? It’s when a noble disciple recollects the deities: ‘There are the Gods of the Four Great Kings, the Gods of the Thirty-Three, the Gods of Yama, the Joyful Gods, the Gods Who Love to Create, the Gods Who Control the Creations of Others, the Gods of \textsanskrit{Brahmā}’s Host, and gods even higher than these. When those deities passed away from here, they were reborn there because of their faith, ethics, learning, generosity, and wisdom. I, too, have the same kind of faith, ethics, learning, generosity, and wisdom.’ As they recollect the faith, ethics, learning, generosity, and wisdom of both themselves and those deities, their mind becomes clear, joy arises, and mental corruptions are given up. This is called: ‘A noble disciple who observes the sabbath of the deities, living together with the deities. And because they think of the deities their mind becomes clear, joy arises, and mental corruptions are given up.’ That’s how a corrupt mind is cleaned by applying effort. 

Then\marginnote{19.1} that noble disciple reflects: ‘As long as they live, the perfected ones give up killing living creatures, renouncing the rod and the sword. They are scrupulous and kind, and live full of compassion for all living beings. I, too, for this day and night will give up killing living creatures, renouncing the rod and the sword. I’ll be scrupulous and kind, and live full of compassion for all living beings. I will observe the sabbath by doing as the perfected ones do in this respect. 

As\marginnote{20.1} long as they live, the perfected ones give up stealing. They take only what’s given, and expect only what’s given. They keep themselves clean by not thieving. I, too, for this day and night will give up stealing. I’ll take only what’s given, and expect only what’s given. I’ll keep myself clean by not thieving. I will observe the sabbath by doing as the perfected ones do in this respect. 

As\marginnote{21.1} long as they live, the perfected ones give up unchastity. They are celibate, set apart, avoiding the common practice of sex. I, too, for this day and night will give up unchastity. I will be celibate, set apart, avoiding the common practice of sex. I will observe the sabbath by doing as the perfected ones do in this respect. 

As\marginnote{22.1} long as they live, the perfected ones give up lying. They speak the truth and stick to the truth. They’re honest and trustworthy, and don’t trick the world with their words. I, too, for this day and night will give up lying. I’ll speak the truth and stick to the truth. I’ll be honest and trustworthy, and won’t trick the world with my words. I will observe the sabbath by doing as the perfected ones do in this respect. 

As\marginnote{23.1} long as they live, the perfected ones give up alcoholic drinks that cause negligence. I, too, for this day and night will give up alcoholic drinks that cause negligence. I will observe the sabbath by doing as the perfected ones do in this respect. 

As\marginnote{24.1} long as they live, the perfected ones eat in one part of the day, abstaining from eating at night and from food at the wrong time. I, too, for this day and night will eat in one part of the day, abstaining from eating at night and food at the wrong time. I will observe the sabbath by doing as the perfected ones do in this respect. 

As\marginnote{25.1} long as they live, the perfected ones avoid dancing, singing, music, and seeing shows; and beautifying and adorning themselves with garlands, fragrance, and makeup. I, too, for this day and night will avoid dancing, singing, music, and seeing shows; and beautifying and adorning myself with garlands, fragrance, and makeup. I will observe the sabbath by doing as the perfected ones do in this respect. 

As\marginnote{26.1} long as they live, the perfected ones give up high and luxurious beds. They sleep in a low place, either a cot or a straw mat. I, too, for this day and night will give up high and luxurious beds. I’ll sleep in a low place, either a cot or a straw mat. I will observe the sabbath by doing as the perfected ones do in this respect.’ 

That’s\marginnote{27.1} the sabbath of the noble ones. When the sabbath of the noble ones is observed like this it’s very fruitful and beneficial and splendid and bountiful. 

How\marginnote{28.1} much so? Suppose you were to rule as sovereign lord over these sixteen great countries—\textsanskrit{Aṅga}, Magadha, \textsanskrit{Kāsī}, Kosala, \textsanskrit{Vajjī}, Malla, Ceti, \textsanskrit{Vaṅga}, Kuru, \textsanskrit{Pañcāla}, Maccha, \textsanskrit{Sūrusena}, Assaka, Avanti, \textsanskrit{Gandhāra}, and Kamboja—full of the seven treasures. This wouldn’t be worth a sixteenth part of the sabbath with its eight factors. Why is that? Because human kingship is a poor thing compared to the happiness of the gods. 

Fifty\marginnote{29.1} years in the human realm is one day and night for the gods of the Four Great Kings. Thirty such days make up a month. Twelve such months make up a year. The life span of the gods of the Four Great Kings is five hundred of these divine years. It’s possible that a woman or man who has observed the eight-factored sabbath will—when their body breaks up, after death—be reborn in the company of the gods of the Four Great Kings. This is what I was referring to when I said: ‘Human kingship is a poor thing compared to the happiness of the gods.’ 

A\marginnote{30.1} hundred years in the human realm is one day and night for the Gods of the Thirty-Three. Thirty such days make up a month. Twelve such months make up a year. The life span of the Gods of the Thirty-Three is a thousand of these divine years. It’s possible that a woman or man who has observed the eight-factored sabbath will—when their body breaks up, after death—be reborn in the company of the Gods of the Thirty-Three. This is what I was referring to when I said: ‘Human kingship is a poor thing compared to the happiness of the gods.’ 

Two\marginnote{31.1} hundred years in the human realm is one day and night for the Gods of Yama. Thirty such days make up a month. Twelve such months make up a year. The life span of the Gods of Yama is two thousand of these divine years. It’s possible that a woman or man who has observed the eight-factored sabbath will—when their body breaks up, after death—be reborn in the company of the Gods of Yama. This is what I was referring to when I said: ‘Human kingship is a poor thing compared to the happiness of the gods.’ 

Four\marginnote{32.1} hundred years in the human realm is one day and night for the Joyful Gods. Thirty such days make up a month. Twelve such months make up a year. The life span of the Joyful Gods is four thousand of these divine years. It’s possible that a woman or man who has observed the eight-factored sabbath will—when their body breaks up, after death—be reborn in the company of the Joyful Gods. This is what I was referring to when I said: ‘Human kingship is a poor thing compared to the happiness of the gods.’ 

Eight\marginnote{33.1} hundred years in the human realm is one day and night for the Gods Who Love to Create. Thirty such days make up a month. Twelve such months make up a year. The life span of the Gods Who Love to Create is eight thousand of these divine years. It’s possible that a woman or man who has observed the eight-factored sabbath will—when their body breaks up, after death—be reborn in the company of the Gods Who Love to Create. This is what I was referring to when I said: ‘Human kingship is a poor thing compared to the happiness of the gods.’ 

Sixteen\marginnote{34.1} hundred years in the human realm is one day and night for the Gods Who Control the Creations of Others. Thirty such days make up a month. Twelve such months make up a year. The life span of the Gods Who Control the Creations of Others is sixteen thousand of these divine years. It’s possible that a woman or man who has observed the eight-factored sabbath will—when their body breaks up, after death—be reborn in the company of the Gods Who Control the Creations of Others. This is what I was referring to when I said: ‘Human kingship is a poor thing compared to the happiness of the gods.’ 

\begin{verse}%
You\marginnote{35.1} shouldn’t kill living creatures, or steal, \\
or lie, or drink alcohol. \\
Be celibate, refraining from sex, \\
and don’t eat at night, the wrong time. 

Not\marginnote{36.1} wearing garlands or applying fragrance, \\
you should sleep on a low bed, or a mat on the ground. \\
This is the eight-factored sabbath, they say, \\
explained by the Buddha, who has gone to suffering’s end. 

The\marginnote{37.1} moon and sun are both fair to see, \\
radiating as far as they revolve. \\
Those shining ones in the sky light up the quarters, \\
dispelling the darkness as they traverse the heavens. 

All\marginnote{38.1} of the wealth that’s found in this realm—\\
pearls, gems, fine beryl too, \\
rose-gold or pure gold, \\
or natural gold dug up by marmots—

they’re\marginnote{39.1} not worth a sixteenth part \\
of the sabbath with its eight factors, \\
as starlight cannot rival the moon. 

So\marginnote{40.1} an ethical woman or man, \\
who has observed the eight-factored sabbath, \\
having made merit whose outcome is happiness, \\
blameless, they go to a heavenly place.” 

%
\end{verse}

%
\addtocontents{toc}{\let\protect\contentsline\protect\nopagecontentsline}
\chapter*{The Chapter with Ānanda }
\addcontentsline{toc}{chapter}{\tocchapterline{The Chapter with Ānanda }}
\addtocontents{toc}{\let\protect\contentsline\protect\oldcontentsline}

%
\section*{{\suttatitleacronym AN 3.71}{\suttatitletranslation With Channa }{\suttatitleroot Channasutta}}
\addcontentsline{toc}{section}{\tocacronym{AN 3.71} \toctranslation{With Channa } \tocroot{Channasutta}}
\markboth{With Channa }{Channasutta}
\extramarks{AN 3.71}{AN 3.71}

At\marginnote{1.1} one time the Buddha was staying near \textsanskrit{Sāvatthī} in Jeta’s Grove, \textsanskrit{Anāthapiṇḍika}’s monastery. Then the wanderer Channa went up to Venerable Ānanda and exchanged greetings with him. When the greetings and polite conversation were over, he sat down to one side and said to Ānanda: 

“Reverend\marginnote{1.4} Ānanda, do you advocate giving up greed, hate, and delusion?” 

“We\marginnote{1.5} do, reverend.” 

“But\marginnote{2.1} what drawbacks have you seen, Reverend Ānanda, that you advocate giving up greed, hate, and delusion?” 

“A\marginnote{3.1} greedy person, overcome by greed, intends to hurt themselves, hurt others, and hurt both. They experience mental pain and sadness. When greed has been given up, they don’t intend to hurt themselves, hurt others, and hurt both. They don’t experience mental pain and sadness. A greedy person does bad things by way of body, speech, and mind. When greed has been given up, they don’t do bad things by way of body, speech, and mind. A greedy person doesn’t truly understand what’s for their own good, the good of another, or the good of both. When greed has been given up, they truly understand what’s for their own good, the good of another, or the good of both. Greed is a destroyer of sight, vision, and knowledge. It blocks wisdom, it’s on the side of anguish, and it doesn’t lead to extinguishment. 

A\marginnote{4.1} hateful person, overcome by hate, intends to hurt themselves, hurt others, and hurt both. … 

A\marginnote{4.2} deluded person, overcome by delusion, intends to hurt themselves, hurt others, and hurt both. They experience mental pain and sadness. When delusion has been given up, they don’t intend to hurt themselves, hurt others, and hurt both. They don’t experience mental pain and sadness. A deluded person does bad things by way of body, speech, and mind. When delusion has been given up, they don’t do bad things by way of body, speech, and mind. A deluded person doesn’t truly understand what’s for their own good, the good of another, or the good of both. When delusion has been given up, they truly understand what’s for their own good, the good of another, or the good of both. Delusion is a destroyer of sight, vision, and knowledge; it blocks wisdom, it’s on the side of anguish, and it doesn’t lead to extinguishment. 

This\marginnote{4.9} is the drawback we’ve seen in greed, hate, and delusion, and this is why we advocate giving them up.” 

“But,\marginnote{5.1} reverend, is there a path and a practice for giving up that greed, hate, and delusion?” 

“There\marginnote{5.2} is, reverend, a path and a practice for giving up that greed, hate, and delusion.” 

“Well,\marginnote{5.3} what is it?” 

“It\marginnote{5.4} is simply this noble eightfold path, that is: right view, right thought, right speech, right action, right livelihood, right effort, right mindfulness, and right immersion. This is the path, this is the practice for giving up that greed, hate, and delusion.” 

“This\marginnote{5.7} is a fine path, a fine practice, for giving up that greed, hate, and delusion. Just this much is enough to be diligent.” 

%
\section*{{\suttatitleacronym AN 3.72}{\suttatitletranslation A Disciple of the Ājīvakas }{\suttatitleroot Ājīvakasutta}}
\addcontentsline{toc}{section}{\tocacronym{AN 3.72} \toctranslation{A Disciple of the Ājīvakas } \tocroot{Ājīvakasutta}}
\markboth{A Disciple of the Ājīvakas }{Ājīvakasutta}
\extramarks{AN 3.72}{AN 3.72}

At\marginnote{1.1} one time Venerable Ānanda was staying near Kosambi, in Ghosita’s Monastery. Then a householder who was a disciple of the \textsanskrit{Ājīvaka} ascetics went up to Venerable Ānanda, bowed, sat down to one side, and said to Ānanda: 

“Sir,\marginnote{2.1} whose teaching is well explained? Who in the world is practicing well? Who in the world has done well?” 

“Well\marginnote{2.4} then, householder, I’ll ask you about this in return, and you can answer as you like. What do you think, householder? Is the teaching of those who teach for giving up greed, hate, and delusion well explained or not? Or how do you see this?” 

“The\marginnote{2.8} teaching of those who teach for giving up greed, hate, and delusion is well explained. That’s how I see it.” 

“What\marginnote{3.1} do you think, householder? Are those who practice for giving up greed, hate, and delusion practicing well or not? Or how do you see this?” 

“Those\marginnote{3.4} who practice for giving up greed, hate, and delusion are practicing well. That’s how I see it.” 

“What\marginnote{4.1} do you think, householder? Have those who’ve given up greed, hate, and delusion—so they’re cut off at the root, made like a palm stump, obliterated, and unable to arise in the future—done well in the world, or not? Or how do you see this?” 

“Those\marginnote{4.4} who’ve given up greed, hate, and delusion have done well in the world. That’s how I see it.” 

“So,\marginnote{5.1} householder, you’ve declared: ‘The teaching of those who teach for giving up greed, hate, and delusion is well explained.’ And you’ve declared: ‘Those who practice for giving up greed, hate, and delusion are practicing well.’ And you’ve declared: ‘Those who’ve given up greed, hate, and delusion have done well in the world.’” 

“It’s\marginnote{6.1} incredible, sir, it’s amazing! There’s no acclaiming your own teaching or disrespecting someone else’s, just teaching what’s relevant in that context. The goal is spoken of, but the self is not involved. You, sir, teach Dhamma for giving up greed, hate, and delusion. Your teaching is well explained. You, sir, practice for giving up greed, hate, and delusion. You in the world are practicing well. You’ve given up greed, hate, and delusion. You in the world have done well. 

Excellent,\marginnote{7.1} sir! Excellent! As if he were righting the overturned, or revealing the hidden, or pointing out the path to the lost, or lighting a lamp in the dark so people with good eyes can see what’s there, Venerable Ānanda has made the teaching clear in many ways. I go for refuge to the Buddha, to the teaching, and to the mendicant \textsanskrit{Saṅgha}. From this day forth, may Venerable Ānanda remember me as a lay follower who has gone for refuge for life.” 

%
\section*{{\suttatitleacronym AN 3.73}{\suttatitletranslation With Mahānāma the Sakyan }{\suttatitleroot Mahānāmasakkasutta}}
\addcontentsline{toc}{section}{\tocacronym{AN 3.73} \toctranslation{With Mahānāma the Sakyan } \tocroot{Mahānāmasakkasutta}}
\markboth{With Mahānāma the Sakyan }{Mahānāmasakkasutta}
\extramarks{AN 3.73}{AN 3.73}

\scevam{So\marginnote{1.1} I have heard. }At one time the Buddha was staying in the land of the Sakyans, near Kapilavatthu in the Banyan Tree Monastery. Now at that time the Buddha had recently recovered from an illness. Then \textsanskrit{Mahānāma} the Sakyan went up to the Buddha, bowed, sat down to one side, and said to him: 

“For\marginnote{1.5} a long time, sir, I have understood your teaching like this: ‘Knowledge is for those with immersion, not those without immersion.’ But, sir, does immersion come first, then knowledge? Or does knowledge come first, then immersion?” 

Then\marginnote{1.9} Venerable Ānanda thought, “The Buddha has recently recovered from an illness, and this \textsanskrit{Mahānāma} asks him a question that’s too deep. Why don’t I take him off to one side and teach him the Dhamma?” 

Then\marginnote{2.1} Ānanda took \textsanskrit{Mahānāma} by the arm, led him off to one side, and said to him, “\textsanskrit{Mahānāma}, the Buddha has spoken of the ethics, immersion, and wisdom of a trainee; and the ethics, immersion, and wisdom of an adept. What is the ethics of a trainee? It’s when a mendicant is ethical, restrained in the monastic code, conducting themselves well and seeking alms in suitable places. Seeing danger in the slightest fault, they keep the rules they’ve undertaken. This is called the ethics of a trainee. 

And\marginnote{3.1} what is the immersion of a trainee? It’s when a mendicant, quite secluded from sensual pleasures, secluded from unskillful qualities, enters and remains in the first absorption … second absorption … third absorption … fourth absorption. This is called the immersion of a trainee. 

And\marginnote{4.1} what is the wisdom of a trainee? They truly understand: ‘This is suffering’ … ‘This is the origin of suffering’ … ‘This is the cessation of suffering’ … ‘This is the practice that leads to the cessation of suffering’. This is called the wisdom of a trainee. 

Then\marginnote{5.1} a noble disciple—accomplished in ethics, immersion, and wisdom—realizes the undefiled freedom of heart and freedom by wisdom in this very life. And they live having realized it with their own insight due to the ending of defilements. 

In\marginnote{5.2} this way the Buddha has spoken of the ethics, immersion, and wisdom of both a trainee and an adept.” 

%
\section*{{\suttatitleacronym AN 3.74}{\suttatitletranslation Jains }{\suttatitleroot Nigaṇṭhasutta}}
\addcontentsline{toc}{section}{\tocacronym{AN 3.74} \toctranslation{Jains } \tocroot{Nigaṇṭhasutta}}
\markboth{Jains }{Nigaṇṭhasutta}
\extramarks{AN 3.74}{AN 3.74}

At\marginnote{1.1} one time Venerable Ānanda was staying near \textsanskrit{Vesālī}, at the Great Wood, in the hall with the peaked roof. Then the Licchavis Abhaya and \textsanskrit{Paṇḍitakumāra} went up to Venerable Ānanda, bowed, sat down to one side, and said to him: 

“Sir,\marginnote{1.3} the Jain leader \textsanskrit{Nāṭaputta} claims to be all-knowing and all-seeing, to know and see everything without exception, thus: ‘Knowledge and vision are constantly and continually present to me, while walking, standing, sleeping, and waking.’ He advocates the elimination of past karma by mortification, and breaking the bridge by not making new karma. So with the ending of karma, suffering ends; with the ending of suffering, feeling ends; and with the ending of feeling, all suffering will have been worn away. This is how to go beyond suffering by means of this purification by wearing away in this very life. What, sir, does the Buddha say about this?” 

“Abhaya,\marginnote{2.1} these three kinds of purification by wearing away have been rightly explained by the Blessed One, who knows and sees, the perfected one, the fully awakened Buddha. They are in order to purify sentient beings, to get past sorrow and crying, to make an end of pain and sadness, to end the cycle of suffering, and to realize extinguishment. What three? 

It’s\marginnote{2.3} when, Abhaya, a mendicant is ethical, restrained in the code of conduct, conducting themselves well and seeking alms in suitable places. Seeing danger in the slightest fault, they keep the rules they’ve undertaken. They don’t perform any new deeds, and old deeds are eliminated by experiencing their results little by little. This wearing away is visible in this very life, immediately effective, inviting inspection, relevant, so that sensible people can know it for themselves. 

Then\marginnote{3.1} a mendicant accomplished in ethics, quite secluded from sensual pleasures, secluded from unskillful qualities, enters and remains in the first absorption … second absorption … third absorption … fourth absorption. They don’t perform any new deeds, and old deeds are eliminated by experiencing their results little by little. This wearing away is visible in this very life, immediately effective, inviting inspection, relevant, so that sensible people can know it for themselves. 

Then\marginnote{4.1} a mendicant accomplished in immersion realizes the undefiled freedom of heart and freedom by wisdom in this very life. And they live having realized it with their own insight due to the ending of defilements. They don’t perform any new deeds, and old deeds are eliminated by experiencing their results little by little. This wearing away is visible in this very life, immediately effective, inviting inspection, relevant, so that sensible people can know it for themselves. 

These\marginnote{4.4} are the three kinds of purification by wearing away that have been rightly explained by the Buddha … in order to realize extinguishment.” 

When\marginnote{5.1} he said this, \textsanskrit{Paṇḍitakumāra} said to Abhaya, “Dear Abhaya, is there anything in what Ānanda has said so well that you would disagree with?” 

“How\marginnote{5.3} could I not agree with what was said so well by Ānanda? If anyone didn’t agree with him, their head would explode!” 

%
\section*{{\suttatitleacronym AN 3.75}{\suttatitletranslation Support }{\suttatitleroot Nivesakasutta}}
\addcontentsline{toc}{section}{\tocacronym{AN 3.75} \toctranslation{Support } \tocroot{Nivesakasutta}}
\markboth{Support }{Nivesakasutta}
\extramarks{AN 3.75}{AN 3.75}

Then\marginnote{1.1} Venerable Ānanda went up to the Buddha, bowed, and sat down to one side. The Buddha said to him: 

“Ānanda,\marginnote{2.1} those who you have sympathy for, and those worth listening to—friends and colleagues, relatives and family—should be encouraged, supported, and established in three things. What three? 

Experiential\marginnote{2.3} confidence in the Buddha: ‘That Blessed One is perfected, a fully awakened Buddha, accomplished in knowledge and conduct, holy, knower of the world, supreme guide for those who wish to train, teacher of gods and humans, awakened, blessed.’ 

Experiential\marginnote{2.5} confidence in the teaching: ‘The teaching is well explained by the Buddha—visible in this very life, immediately effective, inviting inspection, relevant, so that sensible people can know it for themselves.’ 

Experiential\marginnote{2.7} confidence in the \textsanskrit{Saṅgha}: ‘The \textsanskrit{Saṅgha} of the Buddha’s disciples is practicing the way that’s good, direct, methodical, and proper. It consists of the four pairs, the eight individuals. This is the \textsanskrit{Saṅgha} of the Buddha’s disciples that is worthy of offerings dedicated to the gods, worthy of hospitality, worthy of a religious donation, worthy of greeting with joined palms, and is the supreme field of merit for the world.’ 

There\marginnote{3.1} might be change in the four primary elements—earth, water, air, and fire—but a noble disciple with experiential confidence in the Buddha would never change. In this context, ‘change’ means that such a noble disciple will be reborn in hell, the animal realm, or the ghost realm: this is not possible. 

There\marginnote{4.1} might be change in the four primary elements—earth, water, air, and fire—but a noble disciple with experiential confidence in the teaching … or the \textsanskrit{Saṅgha} would never change. In this context, ‘change’ means that such a noble disciple will be reborn in hell, the animal realm, or the ghost realm: this is not possible. 

Those\marginnote{5.1} who you have sympathy for, and those worth listening to—friends and colleagues, relatives and family—should be encouraged, supported, and established in these three things.” 

%
\section*{{\suttatitleacronym AN 3.76}{\suttatitletranslation Continued Existence (1st) }{\suttatitleroot Paṭhamabhavasutta}}
\addcontentsline{toc}{section}{\tocacronym{AN 3.76} \toctranslation{Continued Existence (1st) } \tocroot{Paṭhamabhavasutta}}
\markboth{Continued Existence (1st) }{Paṭhamabhavasutta}
\extramarks{AN 3.76}{AN 3.76}

Then\marginnote{1.1} Venerable Ānanda went up to the Buddha, bowed, sat down to one side, and said to him: 

“Sir,\marginnote{1.2} they speak of this thing called ‘continued existence’. How is continued existence defined?” 

“If,\marginnote{2.1} Ānanda, there were no deeds to result in the sensual realm, would continued existence in the sensual realm still come about?” 

“No,\marginnote{2.2} sir.” 

“So,\marginnote{2.3} Ānanda, deeds are the field, consciousness is the seed, and craving is the moisture. The consciousness of sentient beings—hindered by ignorance and fettered by craving—is established in a lower realm. That’s how there is rebirth into a new state of existence in the future. 

If\marginnote{3.1} there were no deeds to result in the realm of luminous form, would continued existence in the realm of luminous form still come about?” 

“No,\marginnote{3.2} sir.” 

“So,\marginnote{3.3} Ānanda, deeds are the field, consciousness is the seed, and craving is the moisture. The consciousness of sentient beings—hindered by ignorance and fettered by craving—is established in a middle realm. That’s how there is rebirth into a new state of existence in the future. 

If\marginnote{4.1} there were no deeds to result in the formless realm, would continued existence in the formless realm still come about?” 

“No,\marginnote{4.2} sir.” 

“So,\marginnote{4.3} Ānanda, deeds are the field, consciousness is the seed, and craving is the moisture. The consciousness of sentient beings—hindered by ignorance and fettered by craving—is established in a higher realm. That’s how there is rebirth into a new state of existence in the future. That’s how continued existence is defined.” 

%
\section*{{\suttatitleacronym AN 3.77}{\suttatitletranslation Continued Existence (2nd) }{\suttatitleroot Dutiyabhavasutta}}
\addcontentsline{toc}{section}{\tocacronym{AN 3.77} \toctranslation{Continued Existence (2nd) } \tocroot{Dutiyabhavasutta}}
\markboth{Continued Existence (2nd) }{Dutiyabhavasutta}
\extramarks{AN 3.77}{AN 3.77}

Then\marginnote{1.1} Venerable Ānanda went up to the Buddha, bowed, sat down to one side, and said to him: 

“Sir,\marginnote{1.2} they speak of this thing called ‘continued existence’. How is continued existence defined?” 

“If,\marginnote{2.1} Ānanda, there were no deeds to result in the sensual realm, would continued existence in the sensual realm still come about?” 

“No,\marginnote{2.2} sir.” 

“So,\marginnote{2.3} Ānanda, deeds are the field, consciousness is the seed, and craving is the moisture. The intention and aim of sentient beings—hindered by ignorance and fettered by craving—is established in a lower realm. That’s how there is rebirth into a new state of existence in the future. 

If\marginnote{3.1} there were no deeds to result in the realm of luminous form, would continued existence in the realm of luminous form still come about?” 

“No,\marginnote{3.2} sir.” 

“So,\marginnote{3.3} Ānanda, deeds are the field, consciousness is the seed, and craving is the moisture. The intention and aim of sentient beings—hindered by ignorance and fettered by craving—is established in a middle realm. That’s how there is rebirth into a new state of existence in the future. 

If\marginnote{4.1} there were no deeds to result in the formless realm, would continued existence in the formless realm still come about?” 

“No,\marginnote{4.2} sir.” 

“So,\marginnote{4.3} deeds are the field, consciousness is the seed, and craving is the moisture. The intention and aim of sentient beings—hindered by ignorance and fettered by craving—is established in a higher realm. That’s how there is rebirth into a new state of existence in the future. That’s how continued existence is defined.” 

%
\section*{{\suttatitleacronym AN 3.78}{\suttatitletranslation Precepts and Observances }{\suttatitleroot Sīlabbatasutta}}
\addcontentsline{toc}{section}{\tocacronym{AN 3.78} \toctranslation{Precepts and Observances } \tocroot{Sīlabbatasutta}}
\markboth{Precepts and Observances }{Sīlabbatasutta}
\extramarks{AN 3.78}{AN 3.78}

Then\marginnote{1.1} Venerable Ānanda went up to the Buddha, bowed, and sat down to one side. The Buddha said to him: “Ānanda, are all precepts and observances, lifestyles, and spiritual paths fruitful when taken as the essence?” 

“This\marginnote{1.3} is no simple matter, sir.” 

“Well\marginnote{1.4} then, Ānanda, break it down.” 

“Take\marginnote{2.1} the case of someone who cultivates precepts and observances, a lifestyle, and a spiritual path, taking this as the essence. If unskillful qualities grow while skillful qualities decline, that’s not fruitful. However, if unskillful qualities decline while skillful qualities grow, that is fruitful.” 

That’s\marginnote{2.3} what Ānanda said, and the teacher approved. 

Then\marginnote{3.1} Ānanda, knowing that the teacher approved, got up from his seat, bowed, and respectfully circled the Buddha, keeping him on his right, before leaving. Then, not long after Ānanda had left, the Buddha addressed the mendicants: “Mendicants, Ānanda is a trainee, but it’s not easy to find his equal in wisdom.” 

%
\section*{{\suttatitleacronym AN 3.79}{\suttatitletranslation Fragrances }{\suttatitleroot Gandhajātasutta}}
\addcontentsline{toc}{section}{\tocacronym{AN 3.79} \toctranslation{Fragrances } \tocroot{Gandhajātasutta}}
\markboth{Fragrances }{Gandhajātasutta}
\extramarks{AN 3.79}{AN 3.79}

Then\marginnote{1.1} Venerable Ānanda went up to the Buddha, bowed, sat down to one side, and said to him: 

“Sir,\marginnote{2.1} there are these three kinds of fragrance that spread only with the wind, not against it. What three? The fragrance of roots, heartwood, and flowers. These are the three kinds of fragrance that spread only with the wind, not against it. Is there a kind of fragrance that spreads with the wind, and against it, and both ways?” 

“There\marginnote{3.1} is, Ānanda, such a kind of fragrance.” 

“So\marginnote{3.2} what, sir, is that kind of fragrance?” 

“It’s\marginnote{4.1} when, Ānanda, in some village or town, a woman or man has gone for refuge to the Buddha, the teaching, and the \textsanskrit{Saṅgha}. They don’t kill living creatures, steal, commit sexual misconduct, lie, or take alcoholic drinks that cause negligence. They’re ethical, of good character. They live at home with a heart rid of the stain of stinginess, freely generous, open-handed, loving to let go, committed to charity, loving to give and to share. 

Ascetics\marginnote{5.1} and brahmins everywhere praise them for these good qualities; even the deities praise them. This is the kind of fragrance that spreads with the wind, and against it, and both. 

\begin{verse}%
The\marginnote{7.1} fragrance of flowers doesn’t spread against the wind, \\
nor sandalwood, pinwheel flowers, or jasmine; \\
but the fragrance of the good spreads against the wind; \\
a good person’s virtue spreads in every direction.” 

%
\end{verse}

%
\section*{{\suttatitleacronym AN 3.80}{\suttatitletranslation Lesser }{\suttatitleroot Cūḷanikāsutta}}
\addcontentsline{toc}{section}{\tocacronym{AN 3.80} \toctranslation{Lesser } \tocroot{Cūḷanikāsutta}}
\markboth{Lesser }{Cūḷanikāsutta}
\extramarks{AN 3.80}{AN 3.80}

Then\marginnote{1.1} Venerable Ānanda went up to the Buddha, bowed, sat down to one side, and said to him: 

“Sir,\marginnote{1.2} I have heard and learned this in the presence of the Buddha: ‘Ānanda, the Buddha Sikhi had a disciple called \textsanskrit{Abhibhū}. Standing in the \textsanskrit{Brahmā} realm, he could make his voice heard throughout the galaxy.’ I wonder how far a Blessed One, a perfected one, a fully awakened Buddha can make their voice heard?” 

“He\marginnote{1.5} was a disciple, Ānanda. Realized Ones are immeasurable.” 

For\marginnote{2.1} a second time … 

For\marginnote{3.1} a third time, Ānanda said to the Buddha: “… I wonder how far a Blessed One, a perfected one, a fully awakened Buddha can make their voice heard?” 

“Ānanda,\marginnote{3.5} have you heard of a thousandfold lesser world system, a galaxy?” 

“Now\marginnote{3.6} is the time, Blessed One! Now is the time, Holy One! Let the Buddha speak. The mendicants will listen and remember it.” 

“Well\marginnote{3.8} then, Ānanda, listen and pay close attention, I will speak.” 

“Yes,\marginnote{3.9} sir,” Ānanda replied. The Buddha said this: 

“Ānanda,\marginnote{4.1} a galaxy extends a thousand times as far as the moon and sun revolve and the shining ones light up the quarters. In that galaxy there are a thousand moons, a thousand suns, a thousand Sinerus king of mountains, a thousand Indias, a thousand Western Continents, a thousand Northern Continents, a thousand Eastern Continents, four thousand oceans, four thousand Great Kings, a thousand realms of the Gods of the Four Great Kings, a thousand realms of the Gods of the Thirty-Three, of the Gods of Yama, of the Joyful Gods, of the Gods who Love to Create, of the Gods who Control the Creations of Others, and a thousand \textsanskrit{Brahmā} realms. This is called a thousandfold lesser world system, a ‘galaxy’. 

A\marginnote{5.1} world system that extends for a thousand galaxies is called a millionfold middling world system, a ‘galactic cluster’. 

A\marginnote{6.1} world system that extends for a thousand galactic clusters is called a billionfold great world system, a ‘galactic supercluster’. 

If\marginnote{7.1} he wished, Ānanda, a Realized One could make his voice heard throughout a galactic supercluster, or as far as he wants.” 

“But\marginnote{8.1} how would the Buddha make his voice heard so far?” 

“First,\marginnote{8.2} Ānanda, a Realized One would fill the galactic supercluster with light. When sentient beings saw the light, the Realized One would project his call so that they’d hear the sound. That’s how a Realized One could make his voice heard throughout a galactic supercluster, or as far as he wants.” 

When\marginnote{9.1} he said this, Venerable Ānanda said, “I’m so fortunate, so very fortunate, to have a teacher with such power and might!” 

When\marginnote{9.4} he said this, Venerable \textsanskrit{Udāyī} said to Venerable Ānanda, “What is it to you, Reverend Ānanda, if your teacher has such power and might?” 

When\marginnote{9.6} he said this, the Buddha said to Venerable \textsanskrit{Udāyī}, “Not so, \textsanskrit{Udāyī}, not so! If Ānanda were to die while still not free of greed, he would rule as king of the gods for seven lifetimes, and as king of all India for seven lifetimes, because of the confidence of his heart. However, Ānanda will be extinguished in the present life.” 

%
\addtocontents{toc}{\let\protect\contentsline\protect\nopagecontentsline}
\chapter*{The Chapter on Ascetics }
\addcontentsline{toc}{chapter}{\tocchapterline{The Chapter on Ascetics }}
\addtocontents{toc}{\let\protect\contentsline\protect\oldcontentsline}

%
\section*{{\suttatitleacronym AN 3.81}{\suttatitletranslation Ascetics }{\suttatitleroot Samaṇasutta}}
\addcontentsline{toc}{section}{\tocacronym{AN 3.81} \toctranslation{Ascetics } \tocroot{Samaṇasutta}}
\markboth{Ascetics }{Samaṇasutta}
\extramarks{AN 3.81}{AN 3.81}

“Mendicants,\marginnote{1.1} there are three duties of an ascetic. What three? Undertaking the training in the higher ethics, the higher mind, and the higher wisdom. These are the three duties of an ascetic. 

So\marginnote{2.1} you should train like this: ‘We will have keen enthusiasm for undertaking the training in the higher ethics, the higher mind, and the higher wisdom.’ That’s how you should train.” 

%
\section*{{\suttatitleacronym AN 3.82}{\suttatitletranslation The Donkey }{\suttatitleroot Gadrabhasutta}}
\addcontentsline{toc}{section}{\tocacronym{AN 3.82} \toctranslation{The Donkey } \tocroot{Gadrabhasutta}}
\markboth{The Donkey }{Gadrabhasutta}
\extramarks{AN 3.82}{AN 3.82}

“Suppose,\marginnote{1.1} mendicants, a donkey followed behind a herd of cattle, thinking: ‘I can moo too! I can moo too!’ But it doesn’t look like a cow, or sound like a cow, or leave a footprint like a cow. Still it follows behind a herd of cattle, thinking: ‘I can moo too! I can moo too!’ 

In\marginnote{2.1} the same way, some mendicant follows behind the mendicant \textsanskrit{Saṅgha}, thinking: ‘I’m a monk too! I’m a monk too!’ But they don’t have the same enthusiasm for undertaking the training in the higher ethics, the higher mind, and the higher wisdom as the other mendicants. Still they follow behind the mendicant \textsanskrit{Saṇgha}, thinking: ‘I’m a monk too! I’m a monk too!’ 

So\marginnote{3.1} you should train like this: ‘We will have keen enthusiasm for undertaking the training in the higher ethics, the higher mind, and the higher wisdom.’ That’s how you should train.” 

%
\section*{{\suttatitleacronym AN 3.83}{\suttatitletranslation Fields }{\suttatitleroot Khettasutta}}
\addcontentsline{toc}{section}{\tocacronym{AN 3.83} \toctranslation{Fields } \tocroot{Khettasutta}}
\markboth{Fields }{Khettasutta}
\extramarks{AN 3.83}{AN 3.83}

“Mendicants,\marginnote{1.1} a farmer has three primary duties. What three? A farmer first of all makes sure the field is well ploughed and tilled. Next they plant seeds in season. When the time is right, they irrigate the field and then drain it. These are the three primary duties of a farmer. 

In\marginnote{2.1} the same way, a mendicant has three primary duties. What three? Undertaking the training in the higher ethics, the higher mind, and the higher wisdom. These are the three primary duties of a mendicant. 

So\marginnote{3.1} you should train like this: ‘We will have keen enthusiasm for undertaking the training in the higher ethics, the higher mind, and the higher wisdom.’ That’s how you should train.” 

%
\section*{{\suttatitleacronym AN 3.84}{\suttatitletranslation The Vajji }{\suttatitleroot Vajjiputtasutta}}
\addcontentsline{toc}{section}{\tocacronym{AN 3.84} \toctranslation{The Vajji } \tocroot{Vajjiputtasutta}}
\markboth{The Vajji }{Vajjiputtasutta}
\extramarks{AN 3.84}{AN 3.84}

At\marginnote{1.1} one time the Buddha was staying near \textsanskrit{Vesālī}, at the Great Wood, in the hall with the peaked roof. Then a certain Vajji monk went up to the Buddha, bowed, sat down to one side, and said to him: 

“Sir,\marginnote{1.3} each fortnight over a hundred and fifty training rules are recited. I’m not able to train in them.” 

“But\marginnote{1.5} monk, are you able to train in three trainings: the higher ethics, the higher mind, and the higher wisdom?” 

“I\marginnote{1.7} am, sir.” 

“So,\marginnote{1.9} monk, you should train in these three trainings: the higher ethics, the higher mind, and the higher wisdom. 

As\marginnote{2.1} you train in these, you will give up greed, hate, and delusion. Then you won’t do anything unskillful, or practice anything bad.” 

After\marginnote{3.1} some time that monk trained in the higher ethics, the higher mind, and the higher wisdom. He gave up greed, hate, and delusion. Then he didn’t do anything unskillful, or practice anything bad. 

%
\section*{{\suttatitleacronym AN 3.85}{\suttatitletranslation A Trainee }{\suttatitleroot Sekkhasutta}}
\addcontentsline{toc}{section}{\tocacronym{AN 3.85} \toctranslation{A Trainee } \tocroot{Sekkhasutta}}
\markboth{A Trainee }{Sekkhasutta}
\extramarks{AN 3.85}{AN 3.85}

Then\marginnote{1.1} a mendicant went up to the Buddha, bowed, sat down to one side, and said to him: 

“Sir,\marginnote{2.1} they speak of this person called ‘a trainee’. How is a trainee defined?” 

“They\marginnote{2.3} train, mendicant, that’s why they’re called ‘a trainee’. What is their training? They train in the higher ethics, the higher mind, and the higher wisdom. They train, that’s why they’re called ‘a trainee’. 

\begin{verse}%
As\marginnote{3.1} a trainee trains, \\
following the straight road, \\
first they know about ending; \\
enlightenment follows in the same lifetime. 

Then\marginnote{4.1} the knowledge comes \\
such a one, freed through enlightenment, \\
with the end of the fetters of rebirth: \\
‘My freedom is unshakable.’” 

%
\end{verse}

%
\section*{{\suttatitleacronym AN 3.86}{\suttatitletranslation Training (1st) }{\suttatitleroot Paṭhamasikkhāsutta}}
\addcontentsline{toc}{section}{\tocacronym{AN 3.86} \toctranslation{Training (1st) } \tocroot{Paṭhamasikkhāsutta}}
\markboth{Training (1st) }{Paṭhamasikkhāsutta}
\extramarks{AN 3.86}{AN 3.86}

“Mendicants,\marginnote{1.1} each fortnight over a hundred and fifty training rules come up for recitation, in which gentlemen who love themselves train. These are all included in the three trainings. What three? The training in the higher ethics, the higher mind, and the higher wisdom. These are the three trainings that include them all. 

Take\marginnote{2.1} the case of a mendicant who has fulfilled their ethics, but has limited immersion and wisdom. They break some lesser and minor training rules, but are restored. Why is that? Because I don’t say they’re incapable of that. But they’re constant and steady in their precepts regarding the training rules that are fundamental, befitting the spiritual path. They keep the rules they’ve undertaken. With the ending of three fetters they’re a stream-enterer, not liable to be reborn in the underworld, bound for awakening. 

Take\marginnote{3.1} another case of a mendicant who has fulfilled their ethics, but has limited immersion and wisdom. They break some lesser and minor training rules, but are restored. Why is that? Because I don’t say they’re incapable of that. But they’re constant and steady in their precepts regarding the training rules that are fundamental, befitting the spiritual path. They keep the rules they’ve undertaken. With the ending of three fetters, and the weakening of greed, hate, and delusion, they’re a once-returner. They come back to this world once only, then make an end of suffering. 

Take\marginnote{4.1} another case of a mendicant who has fulfilled their ethics and immersion, but has limited wisdom. They break some lesser and minor training rules, but are restored. Why is that? Because I don’t say they’re incapable of that. But they’re constant and steady in their precepts regarding the training rules that are fundamental, befitting the spiritual path. They keep the rules they’ve undertaken. With the ending of the five lower fetters they’re reborn spontaneously. They are extinguished there, and are not liable to return from that world. 

Take\marginnote{5.1} another case of a mendicant who has fulfilled their ethics, immersion, and wisdom. They break some lesser and minor training rules, but are restored. Why is that? Because I don’t say they’re incapable of that. But they’re constant and steady in their precepts regarding the training rules that are fundamental, befitting the spiritual path. They keep the rules they’ve undertaken. They realize the undefiled freedom of heart and freedom by wisdom in this very life. And they live having realized it with their own insight due to the ending of defilements. 

So,\marginnote{6.1} mendicants, if you practice partially you succeed partially. If you practice fully you succeed fully. These training rules are not a waste, I say.” 

%
\section*{{\suttatitleacronym AN 3.87}{\suttatitletranslation Training (2nd) }{\suttatitleroot Dutiyasikkhāsutta}}
\addcontentsline{toc}{section}{\tocacronym{AN 3.87} \toctranslation{Training (2nd) } \tocroot{Dutiyasikkhāsutta}}
\markboth{Training (2nd) }{Dutiyasikkhāsutta}
\extramarks{AN 3.87}{AN 3.87}

“Mendicants,\marginnote{1.1} each fortnight over a hundred and fifty training rules come up for recitation, in which gentlemen who love themselves train. These are all included in the three trainings. What three? The training in the higher ethics, the higher mind, and the higher wisdom. These are the three trainings that include them all. 

Take\marginnote{2.1} the case of a mendicant who has fulfilled their ethics, but has limited immersion and wisdom. They break some lesser and minor training rules, but are restored. Why is that? Because I don’t say they’re incapable of that. But they’re constant and steady in their precepts regarding the training rules that are fundamental, befitting the spiritual life. They keep the rules they’ve undertaken. With the ending of three fetters they have at most seven rebirths. They will transmigrate at most seven times among gods and humans and then make an end of suffering. With the ending of three fetters, they go from family to family. They will transmigrate between two or three families and then make an end of suffering. With the ending of three fetters, they’re a one-seeder. They will be reborn just one time in a human existence, then make an end of suffering. With the ending of three fetters, and the weakening of greed, hate, and delusion, they’re a once-returner. They come back to this world once only, then make an end of suffering. 

Take\marginnote{3.1} another case of a mendicant who has fulfilled their ethics and immersion, but has limited wisdom. They break some lesser and minor training rules, but are restored. Why is that? Because I don’t say they’re incapable of that. But they’re constant and steady in their precepts regarding the training rules that are fundamental, befitting the spiritual path. They keep the rules they’ve undertaken. With the ending of the five lower fetters they head upstream, going to the \textsanskrit{Akaniṭṭha} realm. With the ending of the five lower fetters they’re extinguished with extra effort. With the ending of the five lower fetters they’re extinguished without extra effort. With the ending of the five lower fetters they’re extinguished upon landing. With the ending of the five lower fetters they’re extinguished between one life and the next. 

Take\marginnote{4.1} another case of a mendicant who has fulfilled their ethics, immersion, and wisdom. They break some lesser and minor training rules, but are restored. Why is that? Because I don’t say they’re incapable of that. But they’re constant and steady in their precepts regarding the training rules that are fundamental, befitting the spiritual path. They keep the rules they’ve undertaken. They realize the undefiled freedom of heart and freedom by wisdom in this very life. And they live having realized it with their own insight due to the ending of defilements. 

So,\marginnote{5.1} mendicants, if you practice partially you succeed partially. If you practice fully you succeed fully. These training rules are not a waste, I say.” 

%
\section*{{\suttatitleacronym AN 3.88}{\suttatitletranslation Training (3rd) }{\suttatitleroot Tatiyasikkhāsutta}}
\addcontentsline{toc}{section}{\tocacronym{AN 3.88} \toctranslation{Training (3rd) } \tocroot{Tatiyasikkhāsutta}}
\markboth{Training (3rd) }{Tatiyasikkhāsutta}
\extramarks{AN 3.88}{AN 3.88}

“Mendicants,\marginnote{1.1} each fortnight over a hundred and fifty training rules come up for recitation, in which gentlemen who love themselves train. These are all included in the three trainings. What three? The training in the higher ethics, the higher mind, and the higher wisdom. These are the three trainings that include them all. 

Take\marginnote{2.1} the case of a mendicant who has fulfilled their ethics, immersion, and wisdom. They break some lesser and minor training rules, but are restored. Why is that? Because I don’t say they’re incapable of that. But they’re constant and steady in their precepts regarding the training rules that are fundamental, befitting the spiritual path. They keep the rules they’ve undertaken. 

They\marginnote{2.6} realize the undefiled freedom of heart and freedom by wisdom in this very life. And they live having realized it with their own insight due to the ending of defilements. 

If\marginnote{2.7} they don’t penetrate so far, with the ending of the five lower fetters they’re extinguished between one life and the next. 

If\marginnote{2.8} they don’t penetrate so far, with the ending of the five lower fetters they’re extinguished upon landing. 

If\marginnote{2.9} they don’t penetrate so far, with the ending of the five lower fetters they’re extinguished without extra effort. 

If\marginnote{2.10} they don’t penetrate so far, with the ending of the five lower fetters they’re extinguished with extra effort. 

If\marginnote{2.11} they don’t penetrate so far, with the ending of the five lower fetters they head upstream, going to the \textsanskrit{Akaniṭṭha} realm. 

If\marginnote{2.12} they don’t penetrate so far, with the ending of three fetters, and the weakening of greed, hate, and delusion, they’re a once-returner. They come back to this world once only, then make an end of suffering. 

If\marginnote{2.13} they don’t penetrate so far, with the ending of three fetters, they’re a one-seeder. They will be reborn just one time in a human existence, then make an end of suffering. 

If\marginnote{2.14} they don’t penetrate so far, with the ending of three fetters, they go from family to family. They will transmigrate between two or three families and then make an end of suffering. 

If\marginnote{2.15} they don’t penetrate so far, with the ending of three fetters, they have at most seven rebirths. They will transmigrate at most seven times among gods and humans and then make an end of suffering. 

So,\marginnote{3.1} mendicants, if you practice fully you succeed fully. If you practice partially you succeed partially. These training rules are not a waste, I say.” 

%
\section*{{\suttatitleacronym AN 3.89}{\suttatitletranslation Three Trainings (1st) }{\suttatitleroot Paṭhamasikkhattayasutta}}
\addcontentsline{toc}{section}{\tocacronym{AN 3.89} \toctranslation{Three Trainings (1st) } \tocroot{Paṭhamasikkhattayasutta}}
\markboth{Three Trainings (1st) }{Paṭhamasikkhattayasutta}
\extramarks{AN 3.89}{AN 3.89}

“Mendicants,\marginnote{1.1} these are the three trainings. What three? The training in the higher ethics, the higher mind, and the higher wisdom. 

And\marginnote{2.1} what is the training in the higher ethics? It’s when a mendicant is ethical, restrained in the code of conduct, conducting themselves well and seeking alms in suitable places. Seeing danger in the slightest fault, they keep the rules they’ve undertaken. This is called the training in the higher ethics. 

And\marginnote{3.1} what is the training in the higher mind? It’s when a mendicant, quite secluded from sensual pleasures, secluded from unskillful qualities, enters and remains in the first absorption … second absorption … third absorption … fourth absorption. This is called the training in the higher mind. 

And\marginnote{4.1} what is the training in the higher wisdom? They truly understand: ‘This is suffering’ … ‘This is the origin of suffering’ … ‘This is the cessation of suffering’ … ‘This is the practice that leads to the cessation of suffering’. This is called the training in the higher wisdom. These are the three trainings.” 

%
\section*{{\suttatitleacronym AN 3.90}{\suttatitletranslation Three Trainings (2nd) }{\suttatitleroot Dutiyasikkhattayasutta}}
\addcontentsline{toc}{section}{\tocacronym{AN 3.90} \toctranslation{Three Trainings (2nd) } \tocroot{Dutiyasikkhattayasutta}}
\markboth{Three Trainings (2nd) }{Dutiyasikkhattayasutta}
\extramarks{AN 3.90}{AN 3.90}

“Mendicants,\marginnote{1.1} these are the three trainings. What three? The training in the higher ethics, the higher mind, and the higher wisdom. 

And\marginnote{2.1} what is the training in the higher ethics? It’s when a mendicant is ethical, restrained in the code of conduct, conducting themselves well and seeking alms in suitable places. Seeing danger in the slightest fault, they keep the rules they’ve undertaken. This is called the training in the higher ethics. 

And\marginnote{3.1} what is the training in the higher mind? It’s when a mendicant, quite secluded from sensual pleasures, secluded from unskillful qualities, enters and remains in the first absorption … second absorption … third absorption … fourth absorption. This is called the training in the higher mind. 

And\marginnote{4.1} what is the training in the higher wisdom? It’s when a mendicant realizes the undefiled freedom of heart and freedom by wisdom in this very life. And they live having realized it with their own insight due to the ending of defilements. This is called the training in the higher wisdom. These are the three trainings. 

\begin{verse}%
The\marginnote{5.1} higher ethics, the higher mind, \\
and the higher wisdom should be practiced \\
by those energetic, steadfast, and resolute, \\
practicing absorption, mindful, with guarded senses. 

As\marginnote{6.1} before, so after; \\
as after, so before. \\
As below, so above; \\
as above, so below. 

As\marginnote{7.1} by day, so by night; \\
as by night, so by day. \\
Having mastered every direction \\
with limitless immersion, 

they\marginnote{8.1} call them a ‘trainee on the path’, \\
and ‘one living a pure life’. \\
But a wise one who has gone to the end of the path \\
they call a ‘Buddha’ in the world. 

With\marginnote{9.1} the cessation of consciousness, \\
freed by the ending of craving, \\
the liberation of their heart \\
is like a lamp going out.” 

%
\end{verse}

%
\section*{{\suttatitleacronym AN 3.91}{\suttatitletranslation At Paṅkadhā }{\suttatitleroot Saṅkavāsutta}}
\addcontentsline{toc}{section}{\tocacronym{AN 3.91} \toctranslation{At Paṅkadhā } \tocroot{Saṅkavāsutta}}
\markboth{At Paṅkadhā }{Saṅkavāsutta}
\extramarks{AN 3.91}{AN 3.91}

At\marginnote{1.1} one time the Buddha was wandering in the land of the Kosalans together with a large \textsanskrit{Saṅgha} of mendicants. He arrived at a town of the Kosalans named \textsanskrit{Paṅkadhā}, and stayed there. 

Now,\marginnote{1.3} at that time a monk called Kassapagotta was resident at \textsanskrit{Paṅkadhā}. There the Buddha educated, encouraged, fired up, and inspired the mendicants with a Dhamma talk about the training rules. Kassapagotta became quite impatient and bitter, thinking, “This ascetic is much too strict.” 

When\marginnote{1.7} the Buddha had stayed in \textsanskrit{Paṅkadhā} as long as he wished, he set out for \textsanskrit{Rājagaha}. Traveling stage by stage, he arrived at \textsanskrit{Rājagaha}, and stayed there. 

Soon\marginnote{2.1} after the Buddha left, Kassapagotta became quite remorseful and regretful, thinking, “It’s my loss, my misfortune, that when the Buddha was talking about the training rules I became quite impatient and bitter, thinking he was much too strict. Why don’t I go to the Buddha and confess my mistake to him?” 

Then\marginnote{2.6} Kassapagotta set his lodgings in order and, taking his bowl and robe, set out for \textsanskrit{Rājagaha}. Eventually he came to \textsanskrit{Rājagaha} and the Vulture’s Peak. He went up to the Buddha, bowed, sat down to one side, and told him what had happened, saying: 

“I\marginnote{3.1} have made a mistake, sir. It was foolish, stupid, and unskillful of me to become impatient and bitter when the Buddha was educating, encouraging, firing up, and inspiring the mendicants with a Dhamma talk about the training rules, and to think, ‘This ascetic is much too strict.’ Please, sir, accept my mistake for what it is, so I will restrain myself in future.” 

“Indeed,\marginnote{4.1} Kassapa, you made a mistake. But since you have recognized your mistake for what it is, and have dealt with it properly, I accept it. For it is growth in the training of the Noble One to recognize a mistake for what it is, deal with it properly, and commit to restraint in the future. 

Kassapa,\marginnote{5.1} take the case of a senior mendicant who doesn’t want to train and doesn’t praise taking up the training. They don’t encourage other mendicants who don’t want to train to take up the training. And they don’t truthfully and substantively praise at the right time those mendicants who do want to train. I don’t praise that kind of senior mendicant. Why is that? Because, hearing that I praised that mendicant, other mendicants might want to keep company with them. Then they might follow their example, which would be for their lasting harm and suffering. That’s why I don’t praise that kind of senior mendicant. 

Take\marginnote{6.1} the case of a middle mendicant who doesn’t want to train … 

Take\marginnote{6.2} the case of a junior mendicant who doesn’t want to train … That’s why I don’t praise that kind of junior mendicant. 

Kassapa,\marginnote{7.1} take the case of a senior mendicant who does want to train and praises taking up the training. They encourage other mendicants who don’t want to train to take up the training. And they truthfully and substantively praise at the right time those mendicants who do want to train. I praise that kind of senior mendicant. Why is that? Because, hearing that I praised that mendicant, other mendicants might want to keep company with them. Then they might follow their example, which would be for their lasting welfare and happiness. That’s why I praise that kind of senior mendicant. 

Take\marginnote{8.1} the case of a middle mendicant who wants to train … 

Take\marginnote{8.2} the case of a junior mendicant who wants to train … That’s why I praise that kind of junior mendicant.” 

%
\addtocontents{toc}{\let\protect\contentsline\protect\nopagecontentsline}
\chapter*{The Chapter on a Lump of Salt }
\addcontentsline{toc}{chapter}{\tocchapterline{The Chapter on a Lump of Salt }}
\addtocontents{toc}{\let\protect\contentsline\protect\oldcontentsline}

%
\section*{{\suttatitleacronym AN 3.92}{\suttatitletranslation Urgent }{\suttatitleroot Accāyikasutta}}
\addcontentsline{toc}{section}{\tocacronym{AN 3.92} \toctranslation{Urgent } \tocroot{Accāyikasutta}}
\markboth{Urgent }{Accāyikasutta}
\extramarks{AN 3.92}{AN 3.92}

“Mendicants,\marginnote{1.1} a farmer has three urgent duties. What three? A farmer swiftly makes sure the field is well ploughed and tilled. Next they swiftly plant seeds in season. When the time is right, they swiftly irrigate or drain the field. These are the three urgent duties of a farmer. That farmer has no special power or ability to say: ‘Let the crops germinate today! Let them flower tomorrow! Let them ripen the day after!’ But there comes a time when that farmer’s crops germinate, flower, and ripen as the seasons change. 

In\marginnote{2.1} the same way, a mendicant has three urgent duties. What three? Undertaking the training in the higher ethics, the higher mind, and the higher wisdom. These are the three urgent duties of a mendicant. That mendicant has no special power or ability to say: ‘Let my mind be freed from defilements by not grasping today! Or tomorrow! Or the day after!’ But there comes a time—as that mendicant trains in the higher ethics, the higher mind, and the higher wisdom—that their mind is freed from defilements by not grasping. 

So\marginnote{3.1} you should train like this: ‘We will have keen enthusiasm for undertaking the training in the higher ethics, the higher mind, and the higher wisdom.’ That’s how you should train.” 

%
\section*{{\suttatitleacronym AN 3.93}{\suttatitletranslation Seclusion }{\suttatitleroot Pavivekasutta}}
\addcontentsline{toc}{section}{\tocacronym{AN 3.93} \toctranslation{Seclusion } \tocroot{Pavivekasutta}}
\markboth{Seclusion }{Pavivekasutta}
\extramarks{AN 3.93}{AN 3.93}

“Mendicants,\marginnote{1.1} wanderers who follow other paths advocate three kinds of seclusion. What three? Seclusion in robes, almsfood, and lodgings. 

Wanderers\marginnote{2.1} who follow other paths advocate this kind of seclusion in robes. They wear robes of sunn hemp, mixed hemp, corpse-wrapping cloth, rags, lodh tree bark, antelope hide (whole or in strips), kusa grass, bark, wood-chips, human hair, horse-tail hair, or owls’ wings. This is what wanderers who follow other paths advocate for seclusion in robes. 

Wanderers\marginnote{3.1} who follow other paths advocate this kind of seclusion in almsfood. They eat herbs, millet, wild rice, poor rice, water lettuce, rice bran, scum from boiling rice, sesame flour, grass, or cow dung. They survive on forest roots and fruits, or eating fallen fruit. This is what the wanderers who follow other paths advocate for seclusion in almsfood. 

Wanderers\marginnote{4.1} who follow other paths advocate this kind of seclusion in lodgings. They stay in a wilderness, at the root of a tree, in a charnel ground, a forest, the open air, a heap of straw, or a threshing-hut. This is what wanderers who follow other paths advocate for seclusion in lodgings. These are the three kinds of seclusion that wanderers who follow other paths advocate. 

In\marginnote{5.1} this teaching and training, there are three kinds of seclusion for a mendicant. What three? Firstly, a mendicant is ethical, giving up unethical conduct, being secluded from it. They have right view, giving up wrong view, being secluded from it. They’ve ended defilements, giving up defilements, being secluded from them. When a mendicant has these three kinds of seclusion, they’re called a mendicant who has reached the peak and the pith, being pure and grounded in the essential. 

When\marginnote{6.1} a farmer’s rice field is ripe, they’d have the rice cut swiftly, gathered swiftly, transported swiftly, made into heaps swiftly, threshed swiftly, the straw and chaff removed swiftly, winnowed swiftly, brought over swiftly, pounded swiftly, and have the husks removed swiftly. In this way that farmer’s crop would reach the peak and the pith, being pure and consisting only of what is essential. 

In\marginnote{7.1} the same way, when a mendicant is ethical, giving up unethical conduct, being secluded from it; when they have right view, giving up wrong view, being secluded from it; when they’ve ended defilements, giving up defilements, being secluded from them: they’re called a mendicant who has reached the peak and the pith, being pure and grounded in the essential.” 

%
\section*{{\suttatitleacronym AN 3.94}{\suttatitletranslation Springtime }{\suttatitleroot Saradasutta}}
\addcontentsline{toc}{section}{\tocacronym{AN 3.94} \toctranslation{Springtime } \tocroot{Saradasutta}}
\markboth{Springtime }{Saradasutta}
\extramarks{AN 3.94}{AN 3.94}

“After\marginnote{1.1} the rainy season the sky is clear and cloudless. And when the sun rises, it dispels all the darkness from the sky as it shines and glows and radiates. 

In\marginnote{2.1} the same way, when the stainless, immaculate vision of the teaching arises in a noble disciple, three fetters are given up: identity view, doubt, and misapprehension of precepts and observances. 

Afterwards\marginnote{3.1} they get rid of two things: desire and aversion. Quite secluded from sensual pleasures, secluded from unskillful qualities, they enter and remain in the first absorption, which has the rapture and bliss born of seclusion, while placing the mind and keeping it connected. If that noble disciple passed away at that time, they’re bound by no fetter that might return them to this world.” 

%
\section*{{\suttatitleacronym AN 3.95}{\suttatitletranslation Assemblies }{\suttatitleroot Parisāsutta}}
\addcontentsline{toc}{section}{\tocacronym{AN 3.95} \toctranslation{Assemblies } \tocroot{Parisāsutta}}
\markboth{Assemblies }{Parisāsutta}
\extramarks{AN 3.95}{AN 3.95}

“Mendicants,\marginnote{1.1} these are the three assemblies. What three? An assembly of the best, a divided assembly, and a harmonious assembly. 

And\marginnote{2.1} what is an assembly of the best? An assembly where the senior mendicants are not indulgent or slack, nor are they backsliders. Instead, they take the lead in seclusion, rousing energy for attaining the unattained, achieving the unachieved, and realizing the unrealized. And those who come afterwards follow their example. They too are not indulgent or slack, nor are they backsliders. Instead, they take the lead in seclusion, rousing energy for attaining the unattained, achieving the unachieved, and realizing the unrealized. This is called an assembly of the best. 

And\marginnote{3.1} what is a divided assembly? An assembly where the mendicants argue, quarrel, and dispute, continually wounding each other with barbed words. This is called a divided assembly. 

And\marginnote{4.1} what is a harmonious assembly? An assembly where the mendicants live in harmony, appreciating each other, without quarreling, blending like milk and water, and regarding each other with kindly eyes. This is called a harmonious assembly. 

When\marginnote{5.1} the mendicants live in harmony like this, they make much merit. At that time the mendicants live in a holy dwelling, that is, the heart’s release by rejoicing. When they’re joyful, rapture springs up. When the mind is full of rapture, the body becomes tranquil. When the body is tranquil, they feel bliss. And when they’re blissful, the mind becomes immersed in \textsanskrit{samādhi}. 

It’s\marginnote{6.1} like when it rains heavily on a mountain top, and the water flows downhill to fill the hollows, crevices, and creeks. As they become full, they fill up the pools. The pools fill up the lakes, the lakes fill up the streams, and the streams fill up the rivers. And as the rivers become full, they fill up the ocean. 

In\marginnote{7.1} the same way, when the mendicants are in harmony, appreciating each other, without quarreling, blending like milk and water, and regarding each other with kindly eyes, they make much merit. At that time the mendicants live in a holy dwelling, that is, the heart’s release by rejoicing. When they’re joyful, rapture springs up. When the mind is full of rapture, the body becomes tranquil. When the body is tranquil, they feel bliss. And when they’re blissful, the mind becomes immersed in \textsanskrit{samādhi}. 

These\marginnote{7.4} are the three assemblies.” 

%
\section*{{\suttatitleacronym AN 3.96}{\suttatitletranslation The Thoroughbred (1st) }{\suttatitleroot Paṭhamaājānīyasutta}}
\addcontentsline{toc}{section}{\tocacronym{AN 3.96} \toctranslation{The Thoroughbred (1st) } \tocroot{Paṭhamaājānīyasutta}}
\markboth{The Thoroughbred (1st) }{Paṭhamaājānīyasutta}
\extramarks{AN 3.96}{AN 3.96}

“Mendicants,\marginnote{1.1} a fine royal thoroughbred with three factors is worthy of a king, fit to serve a king, and reckoned as a factor of kingship. What three? It’s when a fine royal thoroughbred is beautiful, strong, and fast. A fine royal thoroughbred with these three factors is worthy of a king. … 

In\marginnote{1.5} the same way, a mendicant with three qualities is worthy of offerings dedicated to the gods, worthy of hospitality, worthy of a religious donation, worthy of veneration with joined palms, and is the supreme field of merit for the world. What three? It’s when a mendicant is beautiful, strong, and fast. 

And\marginnote{2.1} how is a mendicant beautiful? It’s when a mendicant is ethical, restrained in the monastic code, conducting themselves well and seeking alms in suitable places. Seeing danger in the slightest fault, they keep the rules they’ve undertaken. That’s how a mendicant is beautiful. 

And\marginnote{3.1} how is a mendicant strong? It’s when a mendicant lives with energy roused up for giving up unskillful qualities and embracing skillful qualities. They are strong, staunchly vigorous, not slacking off when it comes to developing skillful qualities. That’s how a mendicant is strong. 

And\marginnote{4.1} how is a mendicant fast? It’s when a mendicant truly understands: ‘This is suffering’ … ‘This is the origin of suffering’ … ‘This is the cessation of suffering’ … ‘This is the practice that leads to the cessation of suffering’. That’s how a mendicant is fast. 

A\marginnote{4.7} mendicant with these three qualities is worthy of offerings dedicated to the gods, worthy of hospitality, worthy of a religious donation, worthy of veneration with joined palms, and is the supreme field of merit for the world.” 

%
\section*{{\suttatitleacronym AN 3.97}{\suttatitletranslation The Thoroughbred (2nd) }{\suttatitleroot Dutiyaājānīyasutta}}
\addcontentsline{toc}{section}{\tocacronym{AN 3.97} \toctranslation{The Thoroughbred (2nd) } \tocroot{Dutiyaājānīyasutta}}
\markboth{The Thoroughbred (2nd) }{Dutiyaājānīyasutta}
\extramarks{AN 3.97}{AN 3.97}

“Mendicants,\marginnote{1.1} a fine royal thoroughbred with three factors is worthy of a king, fit to serve a king, and considered a factor of kingship. What three? It’s when a fine royal thoroughbred is beautiful, strong, and fast. A fine royal thoroughbred with these three factors is worthy of a king, … 

In\marginnote{1.5} the same way, a mendicant with three qualities is worthy of offerings dedicated to the gods, worthy of hospitality, worthy of a religious donation, worthy of veneration with joined palms, and is the supreme field of merit for the world. What three? It’s when a mendicant is beautiful, strong, and fast. 

And\marginnote{2.1} how is a mendicant beautiful? It’s when a mendicant is ethical, restrained in the code of conduct, conducting themselves well and seeking alms in suitable places. Seeing danger in the slightest fault, they keep the rules they’ve undertaken. That’s how a mendicant is beautiful. 

And\marginnote{3.1} how is a mendicant strong? It’s when a mendicant lives with energy roused up for giving up unskillful qualities and embracing skillful qualities. They are strong, staunchly vigorous, not slacking off when it comes to developing skillful qualities. That’s how a mendicant is strong. 

And\marginnote{4.1} how is a mendicant fast? It’s when a mendicant, with the ending of the five lower fetters, is reborn spontaneously. They’re extinguished there, and are not liable to return from that world. That’s how a mendicant is fast. 

A\marginnote{4.4} mendicant with these three qualities is worthy of offerings dedicated to the gods, worthy of hospitality, worthy of a religious donation, worthy of veneration with joined palms, and is the supreme field of merit for the world.” 

%
\section*{{\suttatitleacronym AN 3.98}{\suttatitletranslation The Thoroughbred (3rd) }{\suttatitleroot Tatiyaājānīyasutta}}
\addcontentsline{toc}{section}{\tocacronym{AN 3.98} \toctranslation{The Thoroughbred (3rd) } \tocroot{Tatiyaājānīyasutta}}
\markboth{The Thoroughbred (3rd) }{Tatiyaājānīyasutta}
\extramarks{AN 3.98}{AN 3.98}

“Mendicants,\marginnote{1.1} a fine royal thoroughbred with three factors is worthy of a king, fit to serve a king, and considered a factor of kingship. What three? It’s when a fine royal thoroughbred is beautiful, strong, and fast. A fine royal thoroughbred with these three factors is worthy of a king. … 

In\marginnote{1.5} the same way, a mendicant with three qualities is worthy of offerings dedicated to the gods, worthy of hospitality, worthy of a religious donation, worthy of veneration with joined palms, and is the supreme field of merit for the world. What three? It’s when a mendicant is beautiful, strong, and fast. 

And\marginnote{2.1} how is a mendicant beautiful? It’s when a mendicant is ethical, restrained in the monastic code, conducting themselves well and seeking alms in suitable places. Seeing danger in the slightest fault, they keep the rules they’ve undertaken. That’s how a mendicant is beautiful. 

And\marginnote{3.1} how is a mendicant strong? It’s when a mendicant lives with energy roused up for giving up unskillful qualities and embracing skillful qualities. They are strong, staunchly vigorous, not slacking off when it comes to developing skillful qualities. That’s how a mendicant is strong. 

And\marginnote{4.1} how is a mendicant fast? It’s when a mendicant realizes the undefiled freedom of heart and freedom by wisdom in this very life. And they live having realized it with their own insight due to the ending of defilements. That’s how a mendicant is fast. 

A\marginnote{4.4} mendicant with these three qualities is worthy of offerings dedicated to the gods, worthy of hospitality, worthy of a religious donation, worthy of veneration with joined palms, and is the supreme field of merit for the world.” 

%
\section*{{\suttatitleacronym AN 3.99}{\suttatitletranslation Jute }{\suttatitleroot Potthakasutta}}
\addcontentsline{toc}{section}{\tocacronym{AN 3.99} \toctranslation{Jute } \tocroot{Potthakasutta}}
\markboth{Jute }{Potthakasutta}
\extramarks{AN 3.99}{AN 3.99}

“Jute\marginnote{1.1} canvas is ugly, unpleasant to touch, and worthless whether it’s new, worn in, or worn out. They use worn out jute canvas for scrubbing pots, or else they just throw it away on the rubbish heap. 

In\marginnote{2.1} the same way, if a junior mendicant is unethical, of bad character, this is how they’re ugly, I say. That person is just as ugly as jute canvas. If you associate with, accompany, and attend to that person, following their example, it’ll be for your lasting harm and suffering. This is how they’re unpleasant to touch, I say. That person is just as unpleasant to touch as jute canvas. Any robes, almsfood, lodgings, and medicines and supplies for the sick that they receive are not very fruitful or beneficial for the donor. This is how they’re worthless, I say. That person is just as worthless as jute canvas. 

If\marginnote{2.10} a middle mendicant is unethical, of bad character, this is how they’re ugly, I say. … 

If\marginnote{2.11} a senior mendicant is unethical, of bad character, this is how they’re ugly, I say. … If you associate with, accompany, and attend to that person, following their example, it’ll be for your lasting harm and suffering. … 

If\marginnote{3.1} such a senior mendicant speaks among the \textsanskrit{Saṅgha}, the mendicants say: ‘What’s an incompetent fool like you got to say? How on earth could you imagine you’ve got something worth saying!’ That person becomes angry and upset, and blurts out things that make the \textsanskrit{Saṅgha} throw them out, as if they were throwing jute canvas away on the rubbish heap. 

Cloth\marginnote{4.1} from \textsanskrit{Kāsī} is beautiful, pleasant to touch, and valuable whether it’s new, worn in, or worn out. They use worn out cloth from \textsanskrit{Kāsī} for wrapping, or else they place it in a fragrant casket. 

In\marginnote{5.1} the same way, if a junior mendicant is ethical, of good character, this is how they’re beautiful, I say. That person is just as beautiful as cloth from \textsanskrit{Kāsī}. If you associate with, accompany, and attend to such a person, following their example, it will be for your lasting welfare and happiness. This is how they’re pleasant to touch, I say. That person is just as pleasant to touch as cloth from \textsanskrit{Kāsī}. Any robes, almsfood, lodgings, and medicines and supplies for the sick that they receive are very fruitful and beneficial for the donor. This is how they’re valuable, I say. That person is just as valuable as cloth from \textsanskrit{Kāsī}. 

If\marginnote{5.9} a middle mendicant is ethical, of good character, this is how they’re beautiful, I say. … 

If\marginnote{5.10} a senior mendicant is ethical, of good character, this is how they’re beautiful, I say. … 

If\marginnote{6.1} such a senior mendicant speaks in the midst of the \textsanskrit{Saṅgha}, the mendicants say: ‘Venerables, be quiet! The senior mendicant is speaking on the teaching and training.’ 

So\marginnote{6.4} you should train like this: ‘We will be like cloth from \textsanskrit{Kāsī}, not like jute canvas.’ That’s how you should train.” 

%
\section*{{\suttatitleacronym AN 3.100}{\suttatitletranslation A Lump of Salt }{\suttatitleroot Loṇakapallasutta}}
\addcontentsline{toc}{section}{\tocacronym{AN 3.100} \toctranslation{A Lump of Salt } \tocroot{Loṇakapallasutta}}
\markboth{A Lump of Salt }{Loṇakapallasutta}
\extramarks{AN 3.100}{AN 3.100}

“Mendicants,\marginnote{1.1} suppose you say: ‘No matter how this person performs a deed, they experience it the same way.’ This being so, the spiritual life could not be lived, and there’d be no chance of making a complete end of suffering. 

Suppose\marginnote{1.3} you say: ‘No matter how this person performs a deed, they experience the result as it should be experienced.’ This being so, the spiritual life can be lived, and there is a chance of making a complete end of suffering. 

Take\marginnote{1.5} the case of a person who does a trivial bad deed, but it lands them in hell. Meanwhile, another person does the same trivial bad deed, but experiences it in the present life, without even a bit left over, let alone a lot. 

What\marginnote{2.1} kind of person does a trivial bad deed, but it lands them in hell? A person who hasn’t developed their physical endurance, ethics, mind, or wisdom. They’re small-minded and mean-spirited, living in suffering. That kind of person does a trivial bad deed, but it lands them in hell. 

What\marginnote{3.1} kind of person does the same trivial bad deed, but experiences it in the present life, without even a bit left over, let alone a lot? A person who has developed their physical endurance, ethics, mind, and wisdom. They’re not small-minded, but are big-hearted, living without limits. That kind of person does the same trivial bad deed, but experiences it in the present life, without even a bit left over, not to speak of a lot. 

Suppose\marginnote{4.1} a person was to drop a lump of salt into a small bowl of water. What do you think, mendicants? Would that small bowl of water become salty and undrinkable?” 

“Yes,\marginnote{4.4} sir. Why is that? Because there is only a little water in the bowl.” 

“Suppose\marginnote{4.7} a person was to drop a lump of salt into the Ganges river. What do you think, mendicants? Would the Ganges river become salty and undrinkable?” 

“No,\marginnote{4.10} sir. Why is that? Because the Ganges river is a vast mass of water.” 

“This\marginnote{5.1} is how it is in the case of a person who does a trivial bad deed, but it lands them in hell. Meanwhile, another person does the same trivial bad deed, but experiences it in the present life, without even a bit left over, not to speak of a lot. … 

Take\marginnote{8.1} the case of a person who is thrown in jail for stealing half a dollar, a dollar, or a hundred dollars. While another person isn’t thrown in jail for stealing half a dollar, a dollar, or a hundred dollars. 

What\marginnote{9.1} kind of person is thrown in jail for stealing half a dollar, a dollar, or a hundred dollars? A person who is poor, with few possessions and little wealth. That kind of person is thrown in jail for stealing half a dollar, a dollar, or a hundred dollars. 

What\marginnote{10.1} kind of person isn’t thrown in jail for stealing half a dollar, a dollar, or a hundred dollars? A person who is rich, affluent, and wealthy. That kind of person isn’t thrown in jail for stealing half a dollar, a dollar, or a hundred dollars. 

This\marginnote{10.4} is how it is in the case of a person who does a trivial bad deed, but they go to hell. Meanwhile, another person does the same trivial bad deed, but experiences it in the present life, without even a bit left over, not to speak of a lot. … 

It’s\marginnote{13.1} like a sheep dealer or butcher. They can execute, jail, fine, or otherwise punish one person who steals from them, but not another. 

What\marginnote{14.1} kind of person can they punish? A person who is poor, with few possessions and little wealth. That’s the kind of person they can punish. 

What\marginnote{15.1} kind of person can’t they punish? A ruler or their minister who is rich, affluent, and wealthy. That’s the kind of person they can’t punish. In fact, all they can do is raise their joined palms and ask: ‘Please, good sir, give me my sheep or pay me for it.’ 

This\marginnote{15.6} is how it is in the case of a person who does a trivial bad deed, but it lands them in hell. Meanwhile, another person does the same trivial bad deed, but experiences it in the present life, without even a bit left over, not to speak of a lot. … 

Mendicants,\marginnote{18.1} suppose you say: ‘No matter how this person performs a deed, they experience it the same way.’ This being so, the spiritual life could not be lived, and there’d be no chance of making a complete end of suffering. 

Suppose\marginnote{18.3} you say: ‘No matter how this person performs a deed, they experience the result as it should be experienced.’ This being so, the spiritual life can be lived, and there is a chance of making a complete end of suffering.” 

%
\section*{{\suttatitleacronym AN 3.101}{\suttatitletranslation A Panner }{\suttatitleroot Paṁsudhovakasutta}}
\addcontentsline{toc}{section}{\tocacronym{AN 3.101} \toctranslation{A Panner } \tocroot{Paṁsudhovakasutta}}
\markboth{A Panner }{Paṁsudhovakasutta}
\extramarks{AN 3.101}{AN 3.101}

“Gold\marginnote{1.1} has coarse corruptions: sand, soil, and gravel. A panner or their apprentice pours it into a pan, where they wash, rinse, and clean it. When that’s been eliminated, there are medium corruptions in the gold: fine grit and coarse sand. The panner washes it again. When that’s been eliminated, there are fine corruptions in the gold: fine sand and black grime. The panner washes it again. When that’s been eliminated, only gold dust is left. A goldsmith or their apprentice places the gold in a crucible where they blow, melt, and smelt it. Still the gold is not settled and the dross is not totally gone. It’s not pliable, workable, or radiant, but is brittle and not completely ready for working. But the goldsmith keeps on blowing, melting, and smelting it. The gold becomes pliable, workable, and radiant, not brittle, and ready to be worked. Then the goldsmith can successfully create any kind of ornament they want, whether a bracelet, earrings, a necklace, or a golden garland. 

In\marginnote{2.1} the same way, a mendicant who is committed to the higher mind has coarse corruptions: bad bodily, verbal, and mental conduct. A sincere, capable mendicant gives these up, gets rid of, eliminates, and obliterates them. 

When\marginnote{2.2} they’ve been given up and eliminated, there are middling corruptions: sensual, malicious, or cruel thoughts. A sincere, capable mendicant gives these up, gets rid of, eliminates, and obliterates them. 

When\marginnote{2.3} they’ve been given up and eliminated, there are fine corruptions: thoughts of family, country, and being looked up to. A sincere, capable mendicant gives these up, gets rid of, eliminates, and obliterates them. 

When\marginnote{2.4} they’ve been given up and eliminated, only thoughts about the teaching are left. That immersion is not peaceful or sublime or tranquil or unified, but is held in place by forceful suppression. 

But\marginnote{2.6} there comes a time when that mind is stilled internally; it settles, unifies, and becomes immersed in \textsanskrit{samādhi}. That immersion is peaceful and sublime and tranquil and unified, not held in place by forceful suppression. They become capable of realizing anything that can be realized by insight to which they extend the mind, in each and every case. 

If\marginnote{3.1} they wish: ‘May I wield the many kinds of psychic power: multiplying myself and becoming one again; appearing and disappearing; going unimpeded through a wall, a rampart, or a mountain as if through space; diving in and out of the earth as if it were water; walking on water as if it were earth; flying cross-legged through the sky like a bird; touching and stroking with my hand the sun and moon, so mighty and powerful; controlling my body as far as the \textsanskrit{Brahmā} realm.’ They are capable of realizing it, in each and every case. 

If\marginnote{4.1} they wish: ‘With clairaudience that is purified and superhuman, may I hear both kinds of sounds, human and divine, whether near or far.’ They are capable of realizing it, in each and every case. 

If\marginnote{5.1} they wish: ‘May I understand the minds of other beings and individuals, having comprehended them with my mind. May I understand mind with greed as “mind with greed”, and mind without greed as “mind without greed”; mind with hate as “mind with hate”, and mind without hate as “mind without hate”; mind with delusion as “mind with delusion”, and mind without delusion as “mind without delusion”; constricted mind as “constricted mind”, and scattered mind as “scattered mind”; expansive mind as “expansive mind”, and unexpansive mind as “unexpansive mind”; mind that is not supreme as “mind that is not supreme”, and mind that is supreme as “mind that is supreme”; mind immersed in \textsanskrit{samādhi} as “mind immersed in \textsanskrit{samādhi}”, and mind not immersed in \textsanskrit{samādhi} as “mind not immersed in \textsanskrit{samādhi}”; freed mind as “freed mind”, and unfreed mind as “unfreed mind”.’ They are capable of realizing it, in each and every case. 

If\marginnote{6.1} they wish: ‘May I recollect many kinds of past lives. That is: one, two, three, four, five, ten, twenty, thirty, forty, fifty, a hundred, a thousand, a hundred thousand rebirths; many eons of the world contracting, many eons of the world expanding, many eons of the world contracting and expanding. May I remember: “There, I was named this, my clan was that, I looked like this, and that was my food. This was how I felt pleasure and pain, and that was how my life ended. When I passed away from that place I was reborn somewhere else. There, too, I was named this, my clan was that, I looked like this, and that was my food. This was how I felt pleasure and pain, and that was how my life ended. When I passed away from that place I was reborn here.” May I recollect my many past lives, with features and details.’ They are capable of realizing it, in each and every case. 

If\marginnote{7.1} they wish: ‘With clairvoyance that is purified and superhuman, may I see sentient beings passing away and being reborn—inferior and superior, beautiful and ugly, in a good place or a bad place—and understand how sentient beings are reborn according to their deeds: “These dear beings did bad things by way of body, speech, and mind. They spoke ill of the noble ones; they had wrong view; and they acted out of that wrong view. When their body breaks up, after death, they’re reborn in a place of loss, a bad place, the underworld, hell. These dear beings, however, did good things by way of body, speech, and mind. They never spoke ill of the noble ones; they had right view; and they acted out of that right view. When their body breaks up, after death, they’re reborn in a good place, a heavenly realm.” And so, with clairvoyance that is purified and superhuman, may I see sentient beings passing away and being reborn—inferior and superior, beautiful and ugly, in a good place or a bad place. And may I understand how sentient beings are reborn according to their deeds.’ They are capable of realizing it, in each and every case. 

If\marginnote{8.1} they wish: ‘May I realize the undefiled freedom of heart and freedom by wisdom in this very life, and live having realized it with my own insight due to the ending of defilements.’ They are capable of realizing it, in each and every case.” 

%
\section*{{\suttatitleacronym AN 3.102}{\suttatitletranslation Foundations }{\suttatitleroot Nimittasutta}}
\addcontentsline{toc}{section}{\tocacronym{AN 3.102} \toctranslation{Foundations } \tocroot{Nimittasutta}}
\markboth{Foundations }{Nimittasutta}
\extramarks{AN 3.102}{AN 3.102}

“Mendicants,\marginnote{1.1} a mendicant committed to the higher mind should focus on three foundations from time to time: the foundation of immersion, the foundation of exertion, and the foundation of equanimity. 

If\marginnote{1.3} a mendicant dedicated to the higher mind focuses solely on the foundation of immersion, it’s likely their mind will incline to laziness. 

If\marginnote{1.4} they focus solely on the foundation of exertion, it’s likely their mind will incline to restlessness. 

If\marginnote{1.5} they focus solely on the foundation of equanimity, it’s likely their mind won’t properly become immersed in \textsanskrit{samādhi} for the ending of defilements. 

But\marginnote{1.6} when a mendicant dedicated to the higher mind focuses from time to time on the foundation of immersion, the foundation of exertion, and the foundation of equanimity, their mind becomes pliable, workable, and radiant, not brittle, and has properly entered immersion for the ending of defilements. 

It’s\marginnote{2.1} like when a goldsmith or a goldsmith’s apprentice prepares a forge, fires the crucible, picks up some gold with tongs and puts it in the crucible. From time to time they fan it, from time to time they sprinkle water on it, and from time to time they just watch over it. If they solely fanned it, the gold would likely be scorched. If they solely sprinkled water on it, the gold would likely cool down. If they solely watched over it, the gold would likely not be properly processed. But when that goldsmith fans it from time to time, sprinkles water on it from time to time, and watches over it from time to time, that gold becomes pliable, workable, and radiant, not brittle, and is ready to be worked. Then the goldsmith can successfully create any kind of ornament they want, whether a bracelet, earrings, a necklace, or a golden garland. 

In\marginnote{3.1} the same way, a mendicant committed to the higher mind should focus on three foundations from time to time: the foundation of immersion, the foundation of exertion, and the foundation of equanimity. … 

When\marginnote{3.6} a mendicant dedicated to the higher mind focuses from time to time on the foundation of immersion, the foundation of exertion, and the foundation of equanimity, their mind becomes pliable, workable, and radiant, not brittle, and has properly entered immersion for the ending of defilements. They become capable of realizing anything that can be realized by insight to which they extend the mind, in each and every case. 

If\marginnote{4.1} they wish: ‘May I wield the many kinds of psychic power’ … 

‘With\marginnote{4.2} clairaudience that is purified and superhuman, may I hear both kinds of sounds, human and divine, whether near or far.’ … ‘May I recollect many kinds of past lives.’ … ‘With clairvoyance that is purified and superhuman, may I see sentient beings passing away and being reborn.’ … ‘May I realize the undefiled freedom of heart and freedom by wisdom in this very life, and live having realized it with my own insight due to the ending of defilements.’ They are capable of realizing it, in each and every case.” 

%
\addtocontents{toc}{\let\protect\contentsline\protect\nopagecontentsline}
\pannasa{The Third Fifty }
\addcontentsline{toc}{pannasa}{The Third Fifty }
\markboth{}{}
\addtocontents{toc}{\let\protect\contentsline\protect\oldcontentsline}

%
\addtocontents{toc}{\let\protect\contentsline\protect\nopagecontentsline}
\chapter*{The Chapter on Awakening }
\addcontentsline{toc}{chapter}{\tocchapterline{The Chapter on Awakening }}
\addtocontents{toc}{\let\protect\contentsline\protect\oldcontentsline}

%
\section*{{\suttatitleacronym AN 3.103}{\suttatitletranslation Before Awakening }{\suttatitleroot Pubbevasambodhasutta}}
\addcontentsline{toc}{section}{\tocacronym{AN 3.103} \toctranslation{Before Awakening } \tocroot{Pubbevasambodhasutta}}
\markboth{Before Awakening }{Pubbevasambodhasutta}
\extramarks{AN 3.103}{AN 3.103}

“Mendicants,\marginnote{1.1} before my awakening—when I was still unawakened but intent on awakening—I thought: ‘What’s the gratification in the world? What’s the drawback? What’s the escape?’ 

Then\marginnote{1.3} it occurred to me: ‘The pleasure and happiness that arise from the world: this is its gratification. 

That\marginnote{1.5} the world is impermanent, suffering, and perishable: this is its drawback. 

Removing\marginnote{1.6} and giving up desire and greed for the world: this is its escape.’ 

As\marginnote{1.7} long as I didn’t truly understand the world’s gratification, drawback, and escape in this way for what they are, I didn’t announce my supreme perfect awakening in this world with its gods, \textsanskrit{Māras}, and \textsanskrit{Brahmās}, this population with its ascetics and brahmins, its gods and humans. 

But\marginnote{1.8} when I did truly understand the world’s gratification, drawback, and escape in this way for what they are, I announced my supreme perfect awakening in this world with its gods, \textsanskrit{Māras}, and \textsanskrit{Brahmās}, this population with its ascetics and brahmins, its gods and humans. 

Knowledge\marginnote{1.9} and vision arose in me: ‘My freedom is unshakable; this is my last rebirth; now there’ll be no more future lives.’” 

%
\section*{{\suttatitleacronym AN 3.104}{\suttatitletranslation Gratification (1st) }{\suttatitleroot Paṭhamaassādasutta}}
\addcontentsline{toc}{section}{\tocacronym{AN 3.104} \toctranslation{Gratification (1st) } \tocroot{Paṭhamaassādasutta}}
\markboth{Gratification (1st) }{Paṭhamaassādasutta}
\extramarks{AN 3.104}{AN 3.104}

“Mendicants,\marginnote{1.1} I went in search of the world’s gratification, and I found it. I’ve seen clearly with wisdom the full extent of gratification in the world. I went in search of the world’s drawbacks, and I found them. I’ve seen clearly with wisdom the full extent of the drawbacks in the world. I went in search of escape from the world, and I found it. I’ve seen clearly with wisdom the full extent of escape from the world. 

As\marginnote{1.7} long as I didn’t truly understand the world’s gratification, drawback, and escape for what they are, I didn’t announce my supreme perfect awakening in this world with its gods, \textsanskrit{Māras}, and \textsanskrit{Brahmās}, this population with its ascetics and brahmins, its gods and humans. 

But\marginnote{1.8} when I did truly understand the world’s gratification, drawback, and escape for what they are, I announced my supreme perfect awakening in this world with its gods, \textsanskrit{Māras}, and \textsanskrit{Brahmās}, this population with its ascetics and brahmins, its gods and humans. 

Knowledge\marginnote{1.9} and vision arose in me: ‘My freedom is unshakable; this is my last rebirth; now there’ll be no more future lives.’” 

%
\section*{{\suttatitleacronym AN 3.105}{\suttatitletranslation Gratification (2nd) }{\suttatitleroot Dutiyaassādasutta}}
\addcontentsline{toc}{section}{\tocacronym{AN 3.105} \toctranslation{Gratification (2nd) } \tocroot{Dutiyaassādasutta}}
\markboth{Gratification (2nd) }{Dutiyaassādasutta}
\extramarks{AN 3.105}{AN 3.105}

“Mendicants,\marginnote{1.1} if there were no gratification in the world, sentient beings wouldn’t love it. But because there is gratification in the world, sentient beings do love it. 

If\marginnote{1.3} the world had no drawback, sentient beings wouldn’t grow disillusioned with it. But since the world has a drawback, sentient beings do grow disillusioned with it. 

If\marginnote{1.5} there were no escape from the world, sentient beings wouldn’t escape from it. But since there is an escape from the world, sentient beings do escape from it. 

As\marginnote{1.7} long as sentient beings don’t truly understand the world’s gratification, drawback, and escape for what they are, they haven’t escaped from this world—with its gods, \textsanskrit{Māras}, and \textsanskrit{Brahmās}, this population with its ascetics and brahmins, its gods and humans—and they don’t live detached, liberated, with a mind free of limits. 

But\marginnote{1.8} when sentient beings truly understand the world’s gratification, drawback, and escape for what they are, they’ve escaped from this world—with its gods, \textsanskrit{Māras}, and \textsanskrit{Brahmās}, this population with its ascetics and brahmins, its gods and humans—and they live detached, liberated, with a mind free of limits.” 

%
\section*{{\suttatitleacronym AN 3.106}{\suttatitletranslation Ascetics and Brahmins }{\suttatitleroot Samaṇabrāhmaṇasutta}}
\addcontentsline{toc}{section}{\tocacronym{AN 3.106} \toctranslation{Ascetics and Brahmins } \tocroot{Samaṇabrāhmaṇasutta}}
\markboth{Ascetics and Brahmins }{Samaṇabrāhmaṇasutta}
\extramarks{AN 3.106}{AN 3.106}

“Mendicants,\marginnote{1.1} there are ascetics and brahmins who don’t truly understand the world’s gratification, drawback, and escape for what they are. I don’t regard them as true ascetics and brahmins. Those venerables don’t realize the goal of life as an ascetic or brahmin, and don’t live having realized it with their own insight. 

There\marginnote{1.3} are ascetics and brahmins who do truly understand the world’s gratification, drawback, and escape for what they are. I regard them as true ascetics and brahmins. Those venerables realize the goal of life as an ascetic or brahmin, and live having realized it with their own insight.” 

%
\section*{{\suttatitleacronym AN 3.107}{\suttatitletranslation Wailing }{\suttatitleroot Ruṇṇasutta}}
\addcontentsline{toc}{section}{\tocacronym{AN 3.107} \toctranslation{Wailing } \tocroot{Ruṇṇasutta}}
\markboth{Wailing }{Ruṇṇasutta}
\extramarks{AN 3.107}{AN 3.107}

“Singing\marginnote{1.1} is regarded as wailing in the training of the Noble One. Dancing is regarded as madness. Too much laughter, showing the teeth, is regarded as childish. So break off singing and dancing; and when you’re appropriately pleased, it’s enough to simply smile.” 

%
\section*{{\suttatitleacronym AN 3.108}{\suttatitletranslation Satisfaction }{\suttatitleroot Atittisutta}}
\addcontentsline{toc}{section}{\tocacronym{AN 3.108} \toctranslation{Satisfaction } \tocroot{Atittisutta}}
\markboth{Satisfaction }{Atittisutta}
\extramarks{AN 3.108}{AN 3.108}

“Mendicants,\marginnote{1.1} there are three indulgences that never satisfy. What three? Sleep, alcoholic drinks, and sexual intercourse. These are the three indulgences that never satisfy.” 

%
\section*{{\suttatitleacronym AN 3.109}{\suttatitletranslation Unprotected }{\suttatitleroot Arakkhitasutta}}
\addcontentsline{toc}{section}{\tocacronym{AN 3.109} \toctranslation{Unprotected } \tocroot{Arakkhitasutta}}
\markboth{Unprotected }{Arakkhitasutta}
\extramarks{AN 3.109}{AN 3.109}

Then\marginnote{1.1} the householder \textsanskrit{Anāthapiṇḍika} went up to the Buddha, bowed, and sat down to one side. The Buddha said to him: 

“Householder,\marginnote{1.2} when the mind is unprotected, deeds of body, speech, and mind are unprotected. When deeds are unprotected, they become corrupted. When deeds are corrupted, they become rotten. Someone whose deeds of body, speech, and mind are rotten will not have a good death. 

It’s\marginnote{2.1} like a bungalow with a bad roof. The roof peak, rafters, and walls are unprotected. They get soaked, and become rotten. 

In\marginnote{3.1} the same way, when the mind is unprotected, bodily, verbal, and mental deeds are unprotected. … Someone whose deeds of body, speech, and mind are rotten will not have a good death. 

When\marginnote{4.1} the mind is protected, bodily, verbal, and mental deeds are protected. When deeds are protected, they don’t become corrupted. When deeds aren’t corrupted, they don’t become rotten. Someone whose deeds of body, speech, and mind aren’t rotten will have a good death. 

It’s\marginnote{5.1} like a bungalow with a good roof. The roof peak, rafters, and walls are protected. They don’t get soaked, and they don’t become rotten. 

In\marginnote{6.1} the same way, when the mind is protected, bodily, verbal, and mental deeds are protected. … Someone whose deeds of body, speech, and mind aren’t rotten will have a good death.” 

%
\section*{{\suttatitleacronym AN 3.110}{\suttatitletranslation Fallen }{\suttatitleroot Byāpannasutta}}
\addcontentsline{toc}{section}{\tocacronym{AN 3.110} \toctranslation{Fallen } \tocroot{Byāpannasutta}}
\markboth{Fallen }{Byāpannasutta}
\extramarks{AN 3.110}{AN 3.110}

Seated\marginnote{1.1} to one side, the Buddha said to the householder \textsanskrit{Anāthapiṇḍika}: 

“Householder,\marginnote{1.2} when the mind is fallen, bodily, verbal, and mental deeds are fallen. Someone whose deeds of body, speech, and mind are fallen will not have a good death. It’s like a bungalow with a bad roof. The roof peak, rafters, and walls fall in. In the same way, when the mind is fallen, bodily, verbal, and mental deeds are fallen. Someone whose deeds of body, speech, and mind are fallen will not have a good death. 

When\marginnote{2.1} the mind is not fallen, bodily, verbal, and mental deeds are not fallen. Someone whose deeds of body, speech, and mind are not fallen will have a good death. It’s like a bungalow with a good roof. The roof peak, rafters, and walls are not fallen in. In the same way, when the mind is not fallen, bodily, verbal, and mental deeds are not fallen. Someone whose deeds of body, speech, and mind are not fallen will have a good death.” 

%
\section*{{\suttatitleacronym AN 3.111}{\suttatitletranslation Sources (1st) }{\suttatitleroot Paṭhamanidānasutta}}
\addcontentsline{toc}{section}{\tocacronym{AN 3.111} \toctranslation{Sources (1st) } \tocroot{Paṭhamanidānasutta}}
\markboth{Sources (1st) }{Paṭhamanidānasutta}
\extramarks{AN 3.111}{AN 3.111}

“Mendicants,\marginnote{1.1} there are these three sources that give rise to deeds. What three? Greed, hate, and delusion are sources that give rise to deeds. Any deed that emerges from greed, hate, or delusion—born, sourced, and originated from greed, hate, or delusion—is unskillful, blameworthy, results in suffering, and leads to the creation of more deeds, not their cessation. These are three sources that give rise to deeds. 

There\marginnote{2.1} are these three sources that give rise to deeds. What three? Contentment, love, and understanding are sources that give rise to deeds. Any deed that emerges from contentment, love, or understanding—born, sourced, and originated from contentment, love, or understanding—is skillful, blameless, results in happiness, and leads to the cessation of more deeds, not their creation. These are three sources that give rise to deeds.” 

%
\section*{{\suttatitleacronym AN 3.112}{\suttatitletranslation Sources (2nd) }{\suttatitleroot Dutiyanidānasutta}}
\addcontentsline{toc}{section}{\tocacronym{AN 3.112} \toctranslation{Sources (2nd) } \tocroot{Dutiyanidānasutta}}
\markboth{Sources (2nd) }{Dutiyanidānasutta}
\extramarks{AN 3.112}{AN 3.112}

“Mendicants,\marginnote{1.1} there are these three sources that give rise to deeds. What three? 

Desire\marginnote{1.3} comes up for things that stimulate desire and greed in the past, future, or present. And how does desire come up for things that stimulate desire and greed in the past, future, or present? In your heart you think about and consider things that stimulate desire and greed in the past, future, or present. When you do this, desire comes up, and you get attached to those things. This lust in the heart is what I call a fetter. That’s how desire comes up for things that stimulate desire and greed in the past, future, or present. 

These\marginnote{3.1} are three sources that give rise to deeds. 

There\marginnote{4.1} are these three sources that give rise to deeds. What three? Desire doesn’t come up for things that stimulate desire and greed in the past, future, or present. And how does desire not come up for things that stimulate desire and greed in the past, future, or present? You understand the future result of things that stimulate desire and greed in the past, future, or present. When you know this, you grow disillusioned, your heart becomes dispassionate, and you see it with penetrating wisdom. That’s how desire doesn’t come up for things that stimulate desire and greed in the past, future, or present. 

These\marginnote{7.1} are three sources that give rise to deeds.” 

%
\addtocontents{toc}{\let\protect\contentsline\protect\nopagecontentsline}
\chapter*{The Chapter on Bound for Loss }
\addcontentsline{toc}{chapter}{\tocchapterline{The Chapter on Bound for Loss }}
\addtocontents{toc}{\let\protect\contentsline\protect\oldcontentsline}

%
\section*{{\suttatitleacronym AN 3.113}{\suttatitletranslation Bound for Loss }{\suttatitleroot Āpāyikasutta}}
\addcontentsline{toc}{section}{\tocacronym{AN 3.113} \toctranslation{Bound for Loss } \tocroot{Āpāyikasutta}}
\markboth{Bound for Loss }{Āpāyikasutta}
\extramarks{AN 3.113}{AN 3.113}

“Mendicants,\marginnote{1.1} three kinds of people are bound for a place of loss, bound for hell, if they don’t give up this fault. What three? 

Someone\marginnote{1.3} who is unchaste, but claims to be celibate; someone who makes a groundless accusation of unchastity against a person whose celibacy is pure; and someone who has the view, ‘There is nothing wrong with sensual pleasures,’ so they throw themselves into sensual pleasures. 

These\marginnote{1.6} are the three kinds of people bound for a place of loss, bound for hell, if they don’t give up this fault.” 

%
\section*{{\suttatitleacronym AN 3.114}{\suttatitletranslation Rare }{\suttatitleroot Dullabhasutta}}
\addcontentsline{toc}{section}{\tocacronym{AN 3.114} \toctranslation{Rare } \tocroot{Dullabhasutta}}
\markboth{Rare }{Dullabhasutta}
\extramarks{AN 3.114}{AN 3.114}

“Mendicants,\marginnote{1.1} the appearance of three people is rare in the world. What three? A Realized One, a perfected one, a fully awakened Buddha. A person who teaches the teaching and training proclaimed by a Realized One. A person who is grateful and thankful. The appearance of these three people is rare in the world.” 

%
\section*{{\suttatitleacronym AN 3.115}{\suttatitletranslation Immeasurable }{\suttatitleroot Appameyyasutta}}
\addcontentsline{toc}{section}{\tocacronym{AN 3.115} \toctranslation{Immeasurable } \tocroot{Appameyyasutta}}
\markboth{Immeasurable }{Appameyyasutta}
\extramarks{AN 3.115}{AN 3.115}

“Mendicants,\marginnote{1.1} these three people are found in the world. What three? Someone easy to measure, someone hard to measure, and someone who is immeasurable. 

And\marginnote{1.4} who is the person easy to measure? It’s a person who is restless, insolent, fickle, scurrilous, loose-tongued, unmindful, lacking situational awareness and immersion, with straying mind and undisciplined faculties. This is called ‘a person easy to measure’. 

And\marginnote{2.1} who is the person hard to measure? It’s a person who is not restless, insolent, fickle, scurrilous, or loose-tongued. They have established mindfulness, situational awareness and immersion, with unified mind and restrained faculties. This is called ‘a person hard to measure’. 

And\marginnote{3.1} who is the immeasurable person? It’s a mendicant who is perfected, and has ended defilements. This is called ‘an immeasurable person’. 

These\marginnote{3.4} are the three people found in the world.” 

%
\section*{{\suttatitleacronym AN 3.116}{\suttatitletranslation Imperturbable }{\suttatitleroot Āneñjasutta}}
\addcontentsline{toc}{section}{\tocacronym{AN 3.116} \toctranslation{Imperturbable } \tocroot{Āneñjasutta}}
\markboth{Imperturbable }{Āneñjasutta}
\extramarks{AN 3.116}{AN 3.116}

“Mendicants,\marginnote{1.1} these three people are found in the world. What three? 

First,\marginnote{1.3} a person, going totally beyond perceptions of form, with the ending of perceptions of impingement, not focusing on perceptions of diversity, aware that ‘space is infinite’, enters and remains in the dimension of infinite space. They enjoy it and like it and find it satisfying. If they’re set on that, committed to it, and meditate on it often without losing it, when they die they’re reborn in the company of the gods of the dimension of infinite space. The lifespan of the gods of infinite space is twenty thousand eons. An ordinary person stays there until the lifespan of those gods is spent, then they go to hell or the animal realm or the ghost realm. But a disciple of the Buddha stays there until the lifespan of those gods is spent, then they’re extinguished in that very life. This is the difference between an educated noble disciple and an uneducated ordinary person as regards their place of rebirth. 

Furthermore,\marginnote{2.1} a person, going totally beyond the dimension of infinite space, aware that ‘consciousness is infinite’, enters and remains in the dimension of infinite consciousness. They enjoy it and like it and find it satisfying. If they’re set on that, committed to it, and meditate on it often without losing it, when they die they’re reborn in the company of the gods of the dimension of infinite consciousness. The lifespan of the gods of infinite consciousness is forty thousand eons. An ordinary person stays there until the lifespan of those gods is spent, then they go to hell or the animal realm or the ghost realm. But a disciple of the Buddha stays there until the lifespan of those gods is spent, then they’re extinguished in that very life. This is the difference between an educated noble disciple and an uneducated ordinary person as regards their place of rebirth. 

Furthermore,\marginnote{3.1} a person, going totally beyond the dimension of infinite consciousness, aware that ‘there is nothing at all’, enters and remains in the dimension of nothingness. They enjoy it and like it and find it satisfying. If they’re set on that, committed to it, and meditate on it often without losing it, when they die they’re reborn in the company of the gods of the dimension of nothingness. The lifespan of the gods of nothingness is sixty thousand eons. An ordinary person stays there until the lifespan of those gods is spent, then they go to hell or the animal realm or the ghost realm. But a disciple of the Buddha stays there until the lifespan of those gods is spent, then they’re extinguished in that very life. This is the difference between an educated noble disciple and an uneducated ordinary person as regards their place of rebirth. 

These\marginnote{3.7} are the three people found in the world.” 

%
\section*{{\suttatitleacronym AN 3.117}{\suttatitletranslation Failures and Accomplishments }{\suttatitleroot Vipattisampadāsutta}}
\addcontentsline{toc}{section}{\tocacronym{AN 3.117} \toctranslation{Failures and Accomplishments } \tocroot{Vipattisampadāsutta}}
\markboth{Failures and Accomplishments }{Vipattisampadāsutta}
\extramarks{AN 3.117}{AN 3.117}

“Mendicants,\marginnote{1.1} there are three failures. What three? Failure in ethics, mind, and view. And what is failure in ethics? It’s when someone kills living creatures, steals, commits sexual misconduct, and uses speech that’s false, divisive, harsh, or nonsensical. This is called ‘failure in ethics’. 

And\marginnote{2.1} what is failure in mind? It’s when someone is covetous and malicious. This is called ‘failure in mind’. 

And\marginnote{3.1} what is failure in view? It’s when someone has wrong view, a distorted perspective, such as: ‘There’s no meaning in giving, sacrifice, or offerings. There’s no fruit or result of good and bad deeds. There’s no afterlife. There’s no obligation to mother and father. No beings are reborn spontaneously. And there’s no ascetic or brahmin who is well attained and practiced, and who describes the afterlife after realizing it with their own insight.’ This is called ‘failure in view’. Some sentient beings, when their body breaks up, after death, are reborn in a place of loss, a bad place, the underworld, hell because of failure in ethics, mind, or view. 

These\marginnote{3.8} are the three failures. 

There\marginnote{4.1} are three accomplishments. What three? Accomplishment in ethics, mind, and view. 

And\marginnote{5.1} what is accomplishment in ethics? It’s when someone doesn’t kill living creatures, steal, commit sexual misconduct, or use speech that’s false, divisive, harsh, or nonsensical. This is called accomplishment in ethics. 

And\marginnote{6.1} what is accomplishment in mind? It’s when someone is content and kind-hearted. This is called accomplishment in mind. 

And\marginnote{7.1} what is accomplishment in view? It’s when someone has right view, an undistorted perspective, such as: ‘There is meaning in giving, sacrifice, and offerings. There are fruits and results of good and bad deeds. There is an afterlife. There are obligation to mother and father. There are beings reborn spontaneously. And there are ascetics and brahmins who are well attained and practiced, and who describe the afterlife after realizing it with their own insight.’ This is called accomplishment in view. Some sentient beings, when their body breaks up, after death, are reborn in a good place, a heavenly realm because of accomplishment in ethics, mind, or view. 

These\marginnote{7.8} are the three accomplishments.” 

%
\section*{{\suttatitleacronym AN 3.118}{\suttatitletranslation Loaded Dice }{\suttatitleroot Apaṇṇakasutta}}
\addcontentsline{toc}{section}{\tocacronym{AN 3.118} \toctranslation{Loaded Dice } \tocroot{Apaṇṇakasutta}}
\markboth{Loaded Dice }{Apaṇṇakasutta}
\extramarks{AN 3.118}{AN 3.118}

“Mendicants,\marginnote{1.1} there are three failures. What three? Failure in ethics, mind, and view. 

And\marginnote{2.1} what is failure in ethics? It’s when someone kills living creatures, steals, commits sexual misconduct, and uses speech that’s false, divisive, harsh, or nonsensical. This is called ‘failure in ethics’. 

And\marginnote{3.1} what is failure in mind? It’s when someone is covetous and malicious. This is called ‘failure in mind’. 

And\marginnote{4.1} what is failure in view? It’s when someone has wrong view, a distorted perspective, such as: ‘There’s no meaning in giving, sacrifice, or offerings. There’s no fruit or result of good and bad deeds. There’s no afterlife. There’s no obligation to mother and father. No beings are reborn spontaneously. And there’s no ascetic or brahmin who is well attained and practiced, and who describes the afterlife after realizing it with their own insight.’ This is called ‘failure in view’. Some sentient beings, when their body breaks up, after death, are reborn in a place of loss, a bad place, the underworld, hell because of failure in ethics, mind, or view. It’s like throwing loaded dice: they always fall the right side up. In the same way, some sentient beings, when their body breaks up, after death, are reborn in a place of loss, a bad place, the underworld, hell because of failure in ethics, mind, or view. 

These\marginnote{4.8} are the three failures. 

There\marginnote{5.1} are three accomplishments. What three? Accomplishment in ethics, mind, and view. 

And\marginnote{6.1} what is accomplishment in ethics? It’s when someone doesn’t kill living creatures, steal, commit sexual misconduct, or use speech that’s false, divisive, harsh, or nonsensical. This is called accomplishment in ethics. 

And\marginnote{7.1} what is accomplishment in mind? It’s when someone is content and kind-hearted. This is called accomplishment in mind. 

And\marginnote{8.1} what is accomplishment in view? It’s when someone has right view, an undistorted perspective, such as: ‘There is meaning in giving, sacrifice, and offerings. There are fruits and results of good and bad deeds. There is an afterlife. There is obligation to mother and father. There are beings reborn spontaneously. And there are ascetics and brahmins who are well attained and practiced, and who describe the afterlife after realizing it with their own insight.’ This is called accomplishment in view. Some sentient beings, when their body breaks up, after death, are reborn in a good place, a heavenly realm because of accomplishment in ethics, mind, or view. It’s like throwing loaded dice: they always fall the right side up. In the same way, some sentient beings, when their body breaks up, after death, are reborn in a good place, a heavenly realm because of accomplishment in ethics, mind, or view. 

These\marginnote{8.8} are the three accomplishments.” 

%
\section*{{\suttatitleacronym AN 3.119}{\suttatitletranslation Action }{\suttatitleroot Kammantasutta}}
\addcontentsline{toc}{section}{\tocacronym{AN 3.119} \toctranslation{Action } \tocroot{Kammantasutta}}
\markboth{Action }{Kammantasutta}
\extramarks{AN 3.119}{AN 3.119}

“Mendicants,\marginnote{1.1} there are three failures. What three? Failure in action, livelihood, and view. 

And\marginnote{1.4} what is failure in action? It’s when someone kills living creatures, steals, commits sexual misconduct, and uses speech that’s false, divisive, harsh, or nonsensical. This is called ‘failure in action’. 

And\marginnote{2.1} what is failure in livelihood? It’s when a noble disciple has wrong livelihood and earns a living by wrong livelihood. This is called ‘failure in livelihood’. 

And\marginnote{3.1} what is failure in view? It’s when someone has wrong view, a distorted perspective, such as: ‘There’s no meaning in giving, sacrifice, or offerings. There’s no fruit or result of good and bad deeds. There’s no afterlife. There’s no obligation to mother and father. No beings are reborn spontaneously. And there’s no ascetic or brahmin who is well attained and practiced, and who describes the afterlife after realizing it with their own insight.’ This is called ‘failure in view’. These are the three failures. 

There\marginnote{4.1} are three accomplishments. What three? Accomplishment in action, livelihood, and view. 

And\marginnote{4.4} what is accomplishment in action? It’s when someone doesn’t kill living creatures, steal, commit sexual misconduct, or use speech that’s false, divisive, harsh, or nonsensical. This is called ‘accomplishment in action’. 

And\marginnote{5.1} what is accomplishment in livelihood? It’s when a noble disciple has right livelihood and earns a living by right livelihood. This is called ‘accomplishment in livelihood’. 

And\marginnote{6.1} what is accomplishment in view? It’s when someone has right view, an undistorted perspective, such as: ‘There is meaning in giving, sacrifice, and offerings. There are fruits and results of good and bad deeds. There is an afterlife. There is obligation to mother and father. There are beings reborn spontaneously. And there are ascetics and brahmins who are well attained and practiced, and who describe the afterlife after realizing it with their own insight.’ This is called ‘accomplishment in view’. 

These\marginnote{6.5} are the three accomplishments.” 

%
\section*{{\suttatitleacronym AN 3.120}{\suttatitletranslation Purity (1st) }{\suttatitleroot Paṭhamasoceyyasutta}}
\addcontentsline{toc}{section}{\tocacronym{AN 3.120} \toctranslation{Purity (1st) } \tocroot{Paṭhamasoceyyasutta}}
\markboth{Purity (1st) }{Paṭhamasoceyyasutta}
\extramarks{AN 3.120}{AN 3.120}

“Mendicants,\marginnote{1.1} there are these three kinds of purity. What three? Purity of body, speech, and mind. 

And\marginnote{1.4} what is purity of body? It’s when someone doesn’t kill living creatures, steal, or commit sexual misconduct. This is called ‘purity of body’. 

And\marginnote{2.1} what is purity of speech? It’s when someone doesn’t use speech that’s false, divisive, harsh, or nonsensical. This is called ‘purity of speech’. 

And\marginnote{3.1} what is purity of mind? It’s when someone is content, kind-hearted, and has right view. This is called ‘purity of mind’. 

These\marginnote{3.4} are the three kinds of purity.” 

%
\section*{{\suttatitleacronym AN 3.121}{\suttatitletranslation Purity (2nd) }{\suttatitleroot Dutiyasoceyyasutta}}
\addcontentsline{toc}{section}{\tocacronym{AN 3.121} \toctranslation{Purity (2nd) } \tocroot{Dutiyasoceyyasutta}}
\markboth{Purity (2nd) }{Dutiyasoceyyasutta}
\extramarks{AN 3.121}{AN 3.121}

“Mendicants,\marginnote{1.1} there are these three kinds of purity. What three? Purity of body, speech, and mind. 

And\marginnote{1.4} what is purity of body? It’s when a mendicant doesn’t kill living creatures, steal, or have sex. This is called ‘purity of body’. 

And\marginnote{2.1} what is purity of speech? It’s when a mendicant doesn’t use speech that’s false, divisive, harsh, or nonsensical. This is called ‘purity of speech’. 

And\marginnote{3.1} what is purity of mind? 

It’s\marginnote{3.2} when a mendicant who has sensual desire in them understands ‘I have sensual desire in me.’ When they don’t have sensual desire in them, they understand ‘I don’t have sensual desire in me.’ They understand how sensual desire arises; how, when it’s already arisen, it’s given up; and how, once it’s given up, it doesn’t arise again in the future. 

When\marginnote{3.3} they have ill will in them they understand ‘I have ill will in me’; and when they don’t have ill will in them they understand ‘I don’t have ill will in me’. They understand how ill will arises; how, when it’s already arisen, it’s given up; and how, once it’s given up, it doesn’t arise again in the future. 

When\marginnote{3.4} they have dullness and drowsiness in them they understand ‘I have dullness and drowsiness in me’; and when they don’t have dullness and drowsiness in them they understand ‘I don’t have dullness and drowsiness in me’. They understand how dullness and drowsiness arise; how, when they’ve already arisen, they’re given up; and how, once they’re given up, they don’t arise again in the future. 

When\marginnote{3.5} they have restlessness and remorse in them they understand ‘I have restlessness and remorse in me’; and when they don’t have restlessness and remorse in them they understand ‘I don’t have restlessness and remorse in me’. They understand how restlessness and remorse arise; how, when they’ve already arisen, they’re given up; and how, once they’re given up, they don’t arise again in the future. 

When\marginnote{3.6} they have doubt in them they understand ‘I have doubt in me’; and when they don’t have doubt in them they understand ‘I don’t have doubt in me’. They understand how doubt arises; how, when it’s already arisen, it’s given up; and how, once it’s given up, it doesn’t arise again in the future. 

This\marginnote{3.7} is called ‘purity of mind’. 

These\marginnote{3.8} are the three kinds of purity. 

\begin{verse}%
Purity\marginnote{4.1} of body, purity of speech, \\
and undefiled purity of heart. \\
A pure person, blessed with purity, \\
has washed off all bad things, they say.” 

%
\end{verse}

%
\section*{{\suttatitleacronym AN 3.122}{\suttatitletranslation Sagacity }{\suttatitleroot Moneyyasutta}}
\addcontentsline{toc}{section}{\tocacronym{AN 3.122} \toctranslation{Sagacity } \tocroot{Moneyyasutta}}
\markboth{Sagacity }{Moneyyasutta}
\extramarks{AN 3.122}{AN 3.122}

“Mendicants,\marginnote{1.1} there are these three kinds of sagacity. What three? Sagacity of body, speech, and mind. 

And\marginnote{1.4} what is sagacity of body? It’s when a mendicant doesn’t kill living creatures, steal, or have sex. This is called ‘sagacity of body’. 

And\marginnote{2.1} what is sagacity of speech? It’s when a mendicant doesn’t use speech that’s false, divisive, harsh, or nonsensical. This is called ‘sagacity of speech’. 

And\marginnote{3.1} what is sagacity of mind? It’s when a mendicant realizes the undefiled freedom of heart and freedom by wisdom in this very life. And they live having realized it with their own insight due to the ending of defilements. This is called ‘sagacity of mind’. These are the three kinds of sagacity. 

\begin{verse}%
A\marginnote{4.1} sage in body, a sage in speech, \\
a sage undefiled in heart; \\
a sage, blessed with sagacity, \\
has given up everything, they say.” 

%
\end{verse}

%
\addtocontents{toc}{\let\protect\contentsline\protect\nopagecontentsline}
\chapter*{The Chapter at Kusināra }
\addcontentsline{toc}{chapter}{\tocchapterline{The Chapter at Kusināra }}
\addtocontents{toc}{\let\protect\contentsline\protect\oldcontentsline}

%
\section*{{\suttatitleacronym AN 3.123}{\suttatitletranslation At Kusināra }{\suttatitleroot Kusinārasutta}}
\addcontentsline{toc}{section}{\tocacronym{AN 3.123} \toctranslation{At Kusināra } \tocroot{Kusinārasutta}}
\markboth{At Kusināra }{Kusinārasutta}
\extramarks{AN 3.123}{AN 3.123}

At\marginnote{1.1} one time the Buddha was staying near \textsanskrit{Kusināra}, in the Forest of Offerings. There the Buddha addressed the mendicants, “Mendicants!” 

“Venerable\marginnote{1.4} sir,” they replied. The Buddha said this: 

“Mendicants,\marginnote{2.1} take the case of a mendicant living supported by a town or village. A householder or their child approaches and invites them for the next day’s meal. The mendicant accepts if they want. When the night has passed, they robe up in the morning, take their bowl and robe, and approach that householder’s home, where they sit on the seat spread out. The householder or their child serves and satisfies them with their own hands with a variety of delicious foods. 

The\marginnote{3.1} mendicant thinks: ‘It’s so good that this householder serves me with a variety of delicious foods.’ Then they think: ‘I really hope this householder serves me with a variety of delicious foods in the future, too.’ They eat that food tied, infatuated, attached, blind to the drawbacks, and not understanding the escape. They think about it with sensual, malicious, or cruel thoughts. A gift to such a mendicant is not very fruitful, I say. Why is that? Because that mendicant is negligent. 

Take\marginnote{4.1} another case of a mendicant living supported by a town or village. A householder or their child approaches and invites them for the next day’s meal. The mendicant accepts if they want. When the night has passed, they robe up in the morning, take their bowl and robe, and approach that householder’s home, where they sit on the seat spread out. The householder or their child serves and satisfies them with their own hands with a variety of delicious foods. 

It\marginnote{5.1} never occurs to them: ‘It’s so good that the householder or their child serves and satisfies me with their own hands with a variety of delicious foods.’ They don’t think: ‘I really hope this householder serves me with a variety of delicious foods in the future, too.’ They eat that almsfood untied, uninfatuated, unattached, seeing the drawback, and understanding the escape. They think about it with thoughts of renunciation, good will, or harmlessness. A gift to such a mendicant is very fruitful, I say. Why is that? Because that mendicant is diligent.” 

%
\section*{{\suttatitleacronym AN 3.124}{\suttatitletranslation Arguments }{\suttatitleroot Bhaṇḍanasutta}}
\addcontentsline{toc}{section}{\tocacronym{AN 3.124} \toctranslation{Arguments } \tocroot{Bhaṇḍanasutta}}
\markboth{Arguments }{Bhaṇḍanasutta}
\extramarks{AN 3.124}{AN 3.124}

“Mendicants,\marginnote{1.1} I’m not even comfortable thinking about a place where mendicants argue—quarreling and disputing, continually wounding each other with barbed words—let alone going there. I come to a conclusion about them: ‘Clearly those venerables have given up three things and cultivated three things.’ What three things have they given up? Thoughts of renunciation, good will, and harmlessness. What three things have they cultivated? Sensual, malicious, and cruel thoughts. … I come to a conclusion about them: ‘Clearly those venerables have given up three things and cultivated three things.’ 

I\marginnote{2.1} feel comfortable going to a place where the mendicants live in harmony—appreciating each other, without quarreling, blending like milk and water, and regarding each other with kindly eyes—let alone thinking about it. I come to a conclusion about them: ‘Clearly those venerables have given up three things and cultivated three things.’ What three things have they given up? Sensual, malicious, and cruel thoughts. What three things have they cultivated? Thoughts of renunciation, good will, and harmlessness. … I come to a conclusion about them: ‘Clearly those venerables have given up three things and cultivated three things.’” 

%
\section*{{\suttatitleacronym AN 3.125}{\suttatitletranslation The Gotamaka Shrine }{\suttatitleroot Gotamakacetiyasutta}}
\addcontentsline{toc}{section}{\tocacronym{AN 3.125} \toctranslation{The Gotamaka Shrine } \tocroot{Gotamakacetiyasutta}}
\markboth{The Gotamaka Shrine }{Gotamakacetiyasutta}
\extramarks{AN 3.125}{AN 3.125}

At\marginnote{1.1} one time the Buddha was staying near \textsanskrit{Vesālī}, at the Gotamaka Tree-shrine. There the Buddha addressed the mendicants, “Mendicants!” 

“Venerable\marginnote{1.4} sir,” they replied. The Buddha said this: 

“Mendicants,\marginnote{2.1} I teach based on direct knowledge, not without direct knowledge. I teach with reasons, not without them. I teach with a demonstrable basis, not without it. Since this is so, you should follow my advice and instruction. This is enough for you to feel joyful, delighted, and happy: ‘The Blessed One is a fully awakened Buddha! The teaching is well explained! The \textsanskrit{Saṅgha} is practicing well!’” 

That\marginnote{3.1} is what the Buddha said. Satisfied, the mendicants were happy with what the Buddha said. And while this discourse was being spoken, the galaxy shook. 

%
\section*{{\suttatitleacronym AN 3.126}{\suttatitletranslation Bharaṇḍu Kālāma }{\suttatitleroot Bharaṇḍukālāmasutta}}
\addcontentsline{toc}{section}{\tocacronym{AN 3.126} \toctranslation{Bharaṇḍu Kālāma } \tocroot{Bharaṇḍukālāmasutta}}
\markboth{Bharaṇḍu Kālāma }{Bharaṇḍukālāmasutta}
\extramarks{AN 3.126}{AN 3.126}

At\marginnote{1.1} one time the Buddha was wandering in the land of the Kosalans when he arrived at Kapilavatthu. 

\textsanskrit{Mahānāma}\marginnote{1.2} the Sakyan heard that he had arrived. He went up to the Buddha, bowed, and stood to one side. The Buddha said to him, “Go into Kapilavatthu, \textsanskrit{Mahānāma}, and check if there’s a suitable guest house where I can spend the night.” 

“Yes,\marginnote{2.2} sir,” replied \textsanskrit{Mahānāma}. He returned to Kapilavatthu and searched all over the city, but he couldn’t see a suitable guest house for the Buddha to spend the night. 

Then\marginnote{3.1} \textsanskrit{Mahānāma} went up to the Buddha, and said to him, “Sir, there’s no suitable guest house in Kapilavatthu for you to spend the night. But there is this \textsanskrit{Bharaṇḍu} the \textsanskrit{Kālāma}, who used to be the Buddha’s spiritual companion. Why don’t you spend the night at his hermitage?” 

“Go,\marginnote{3.5} \textsanskrit{Mahānāma}, and set out a mat.” 

“Yes,\marginnote{3.6} sir,” replied \textsanskrit{Mahānāma}. He went to \textsanskrit{Bharaṇḍu}’s hermitage, where he set out a mat, and got foot-washing water ready. Then he went back to the Buddha and said to him, “The mat and foot-washing water are set out. Please, sir, go at your convenience.” 

Then\marginnote{4.1} the Buddha went to \textsanskrit{Bharaṇḍu}’s hermitage, sat down on the seat spread out, and washed his feet. 

Then\marginnote{4.3} it occurred to \textsanskrit{Mahānāma}, “It’s too late to pay homage to the Buddha today. He’s tired. Tomorrow I’ll pay homage to the Buddha.” He bowed to the Buddha and respectfully circled him, keeping him on his right, then he left. 

Then,\marginnote{5.1} when the night had passed, \textsanskrit{Mahānāma} the Sakyan went up to the Buddha, and sat down to one side. The Buddha said to him: 

“\textsanskrit{Mahānāma},\marginnote{5.2} there are these three teachers found in the world. What three? One teacher advocates the complete understanding of sensual pleasures, but not of sights or feelings. One teacher advocates the complete understanding of sensual pleasures and sights, but not of feelings. One teacher advocates the complete understanding of sensual pleasures, sights, and feelings. These are the three teachers found in the world. Do these three teachers have the same goal or different goals?” 

When\marginnote{6.1} he said this, \textsanskrit{Bharaṇḍu} said to \textsanskrit{Mahānāma}, “Say they’re the same, \textsanskrit{Mahānāma}!” 

The\marginnote{6.3} Buddha said, “Say they’re different, \textsanskrit{Mahānāma}!” 

For\marginnote{6.5} a second time, \textsanskrit{Bharaṇḍu} said, “Say they’re the same, \textsanskrit{Mahānāma}!” 

The\marginnote{6.7} Buddha said, “Say they’re different, \textsanskrit{Mahānāma}!” 

For\marginnote{6.9} a third time, \textsanskrit{Bharaṇḍu} said, “Say they’re the same, \textsanskrit{Mahānāma}!” 

The\marginnote{6.11} Buddha said, “Say they’re different, \textsanskrit{Mahānāma}!” 

Then\marginnote{7.1} it occurred to \textsanskrit{Bharaṇḍu}, “The Buddha has rebuked me three times in front of this illustrious \textsanskrit{Mahānāma}. Why don’t I leave Kapilavatthu?” Then \textsanskrit{Bharaṇḍu} the \textsanskrit{Kālāma} left Kapilavatthu, never to return. 

%
\section*{{\suttatitleacronym AN 3.127}{\suttatitletranslation With Hatthaka }{\suttatitleroot Hatthakasutta}}
\addcontentsline{toc}{section}{\tocacronym{AN 3.127} \toctranslation{With Hatthaka } \tocroot{Hatthakasutta}}
\markboth{With Hatthaka }{Hatthakasutta}
\extramarks{AN 3.127}{AN 3.127}

At\marginnote{1.1} one time the Buddha was staying near \textsanskrit{Sāvatthī} in Jeta’s Grove, \textsanskrit{Anāthapiṇḍika}’s monastery. 

Then,\marginnote{1.2} late at night, the glorious god Hatthaka, lighting up the entire Jeta’s Grove, went up to the Buddha. Thinking, “I will stand before the Buddha,” he sank and melted down, and wasn’t able to stay still. It’s like when ghee or oil is poured on sand, it sinks and melts down, and can’t remain stable. 

Then\marginnote{2.1} the Buddha said to Hatthaka, “Hatthaka, manifest in a solid life-form.” 

“Yes,\marginnote{2.3} sir,” replied Hatthaka. He manifested in a solid life-form, bowed to the Buddha, and stood to one side. 

The\marginnote{2.4} Buddha said to him, “Hatthaka, I wonder whether you still rehearse now the teachings that you rehearsed when you were a human being?” 

“I\marginnote{3.2} still rehearse now the teachings that I rehearsed as a human being. And I also rehearse teachings that I didn’t rehearse as a human being. 

Just\marginnote{3.4} as the Buddha these days lives crowded by monks, nuns, laymen, and laywomen; by rulers and their ministers, and teachers of other paths and their disciples, so I live crowded by the gods. The gods come from far away, thinking, ‘We’ll hear the teaching in the presence of Hatthaka.’ 

Sir,\marginnote{3.7} I passed away without getting enough of three things. What three? Seeing the Buddha; hearing the true teaching; and serving the \textsanskrit{Saṅgha}. I passed away without getting enough of these three things. 

\begin{verse}%
I\marginnote{4.1} could never get enough \\
of seeing the Buddha, \\
serving the \textsanskrit{Saṅgha}, \\
or hearing the teaching. 

Training\marginnote{5.1} in the higher ethics, \\
loving to hear the true teaching, \\
Hatthaka has gone to the Aviha realm \\
without getting enough of these three things.” 

%
\end{verse}

%
\section*{{\suttatitleacronym AN 3.128}{\suttatitletranslation Bitter }{\suttatitleroot Kaṭuviyasutta}}
\addcontentsline{toc}{section}{\tocacronym{AN 3.128} \toctranslation{Bitter } \tocroot{Kaṭuviyasutta}}
\markboth{Bitter }{Kaṭuviyasutta}
\extramarks{AN 3.128}{AN 3.128}

At\marginnote{1.1} one time the Buddha was staying near Benares, in the deer park at Isipatana. 

Then\marginnote{1.2} the Buddha robed up in the morning and, taking his bowl and robe, entered Benares for alms. While the Buddha was walking for alms near the cow-hitching place at the wavy leaf fig, he saw a disgruntled monk who was looking for pleasure in external things, unmindful, without situational awareness or immersion, with straying mind and undisciplined faculties. 

The\marginnote{1.4} Buddha said to him, “Monk, don’t be bitter. If you’re bitter, corrupted by putrefaction, flies will, without a doubt, plague and infest you.” 

Hearing\marginnote{2.3} this advice of the Buddha, that monk was struck with a sense of urgency. Then, after the meal, on his return from almsround, the Buddha told the mendicants what had happened. … 

When\marginnote{4.1} he said this, one of the mendicants asked the Buddha: 

“Sir,\marginnote{4.5} what is this ‘bitterness’? What is ‘putrefaction’? And what are the ‘flies’?” 

“Desire\marginnote{5.1} is bitterness; ill will is the putrefaction; and bad, unskillful thoughts are the flies. If you’re bitter, corrupted by putrefaction, flies will, without a doubt, plague and infest you. 

\begin{verse}%
When\marginnote{6.1} your eyes and ears are unguarded, \\
and you’re not restrained in your sense faculties, \\
flies—those lustful thoughts—\\
will plague you. 

A\marginnote{7.1} mendicant who’s bitter, \\
corrupted by putrefaction, \\
is far from being extinguished, \\
anguish is their lot. 

Whether\marginnote{8.1} in village or wilderness, \\
if they don’t find serenity in themselves, \\
the fool, void of wisdom, \\
is honored only by flies. 

But\marginnote{9.1} those who have ethics, \\
lovers of wisdom and peace, \\
they, being peaceful, sleep at ease, \\
since they’ve got rid of the flies.” 

%
\end{verse}

%
\section*{{\suttatitleacronym AN 3.129}{\suttatitletranslation With Anuruddha (1st) }{\suttatitleroot Paṭhamaanuruddhasutta}}
\addcontentsline{toc}{section}{\tocacronym{AN 3.129} \toctranslation{With Anuruddha (1st) } \tocroot{Paṭhamaanuruddhasutta}}
\markboth{With Anuruddha (1st) }{Paṭhamaanuruddhasutta}
\extramarks{AN 3.129}{AN 3.129}

Then\marginnote{1.1} Venerable Anuruddha went up to the Buddha, bowed, sat down to one side, and said to him: 

“Sometimes,\marginnote{1.2} sir, with my clairvoyance that’s purified and superhuman, I see that females—when their body breaks up, after death—are mostly reborn in a place of loss, a bad place, the underworld, hell. How many qualities do females have so that they’re reborn in a place of loss, a bad place, the underworld, hell?” 

“When\marginnote{2.1} females have three qualities, when their body breaks up, after death, they are reborn in a place of loss, a bad place, the underworld, hell. What three? A female lives at home with a heart full of the stain of stinginess in the morning, jealousy at midday, and sexual desire in the evening. When females have these three qualities, when their body breaks up, after death, they are reborn in a place of loss, a bad place, the underworld, hell.” 

%
\section*{{\suttatitleacronym AN 3.130}{\suttatitletranslation With Anuruddha (2nd) }{\suttatitleroot Dutiyaanuruddhasutta}}
\addcontentsline{toc}{section}{\tocacronym{AN 3.130} \toctranslation{With Anuruddha (2nd) } \tocroot{Dutiyaanuruddhasutta}}
\markboth{With Anuruddha (2nd) }{Dutiyaanuruddhasutta}
\extramarks{AN 3.130}{AN 3.130}

Then\marginnote{1.1} Venerable Anuruddha went up to Venerable \textsanskrit{Sāriputta}, and exchanged greetings with him. When the greetings and polite conversation were over, he sat down to one side and said to him: 

“Here’s\marginnote{1.3} the thing, Reverend \textsanskrit{Sāriputta}. With clairvoyance that is purified and surpasses the human, I survey the entire galaxy. My energy is roused up and unflagging, my mindfulness is established and lucid, my body is tranquil and undisturbed, and my mind is immersed in \textsanskrit{samādhi}. But my mind is not freed from the defilements by not grasping.” 

“Well,\marginnote{2.1} Reverend Anuruddha, when you say: ‘With clairvoyance that is purified and surpasses the human, I survey the entire galaxy,’ that’s your conceit. And when you say: ‘My energy is roused up and unflagging, my mindfulness is established and lucid, my body is tranquil and undisturbed, and my mind is immersed in \textsanskrit{samādhi},’ that’s your restlessness. And when you say: ‘But my mind is not freed from the defilements by not grasping,’ that’s your remorse. It would be good to give up these three things. Instead of focusing on them, apply your mind to the deathless.” 

After\marginnote{3.1} some time Anuruddha gave up these three things. Instead of focusing on them, he applied his mind to the deathless. Then Anuruddha, living alone, withdrawn, diligent, keen, and resolute, soon realized the supreme culmination of the spiritual path in this very life. He lived having achieved with his own insight the goal for which gentlemen rightly go forth from the lay life to homelessness. 

He\marginnote{3.3} understood: “Rebirth is ended; the spiritual journey has been completed; what had to be done has been done; there is no return to any state of existence.” And Venerable Anuruddha became one of the perfected. 

%
\section*{{\suttatitleacronym AN 3.131}{\suttatitletranslation Under Cover }{\suttatitleroot Paṭicchannasutta}}
\addcontentsline{toc}{section}{\tocacronym{AN 3.131} \toctranslation{Under Cover } \tocroot{Paṭicchannasutta}}
\markboth{Under Cover }{Paṭicchannasutta}
\extramarks{AN 3.131}{AN 3.131}

“Mendicants,\marginnote{1.1} three things are conveyed under cover, not in the open. What three? Females are married with a veil, not unveiled. Brahmin hymns are conveyed under cover, not openly. Wrong view is conveyed under cover, not in the open. These three things are conveyed under cover, not in the open. 

Three\marginnote{2.1} things shine in the open, not under cover. What three? The moon shines in the open, not under cover. The sun shines in the open, not under cover. The teaching and training proclaimed by a Realized One shine in the open, not under cover. These three things shine in the open, not under cover.” 

%
\section*{{\suttatitleacronym AN 3.132}{\suttatitletranslation Etchings }{\suttatitleroot Lekhasutta}}
\addcontentsline{toc}{section}{\tocacronym{AN 3.132} \toctranslation{Etchings } \tocroot{Lekhasutta}}
\markboth{Etchings }{Lekhasutta}
\extramarks{AN 3.132}{AN 3.132}

“Mendicants,\marginnote{1.1} these three people are found in the world. What three? A person like a line drawn in stone, a person like a line drawn in sand, and a person like a line drawn in water. 

And\marginnote{1.4} who is the person like a line drawn in stone? It’s a person who is often angry, and their anger lingers for a long time. It’s like a line drawn in stone, which isn’t quickly worn away by wind and water, but lasts for a long time. In the same way, this person is often angry, and their anger lingers for a long time. This is called a person like a line drawn in stone. 

And\marginnote{2.1} who is the person like a line drawn in sand? It’s a person who is often angry, but their anger doesn’t linger long. It’s like a line drawn in sand, which is quickly worn away by wind and water, and doesn’t last long. In the same way, this person is often angry, but their anger doesn’t linger long. This is called a person like a line drawn in sand. 

And\marginnote{3.1} who is the person like a line drawn in water? It’s a person who, though spoken to by someone in a rough, harsh, and disagreeable manner, still stays in touch, interacts with, and greets them. It’s like a line drawn in water, which vanishes right away, and doesn’t last long. In the same way, this person, though spoken to by someone in a rough, harsh, and disagreeable manner, still stays in touch, interacts with, and greets them. This is called a person like a line drawn in water. 

These\marginnote{3.6} are the three people found in the world.” 

%
\addtocontents{toc}{\let\protect\contentsline\protect\nopagecontentsline}
\chapter*{The Chapter on a Warrior }
\addcontentsline{toc}{chapter}{\tocchapterline{The Chapter on a Warrior }}
\addtocontents{toc}{\let\protect\contentsline\protect\oldcontentsline}

%
\section*{{\suttatitleacronym AN 3.133}{\suttatitletranslation A Warrior }{\suttatitleroot Yodhājīvasutta}}
\addcontentsline{toc}{section}{\tocacronym{AN 3.133} \toctranslation{A Warrior } \tocroot{Yodhājīvasutta}}
\markboth{A Warrior }{Yodhājīvasutta}
\extramarks{AN 3.133}{AN 3.133}

“Mendicants,\marginnote{1.1} a warrior with three factors is worthy of a king, fit to serve a king, and is reckoned as a factor of kingship. What three? He’s a long-distance shooter, a marksman, one who shatters large objects. A warrior with these three factors is worthy of a king, fit to serve a king, and is reckoned as a factor of kingship. 

In\marginnote{1.5} the same way, a mendicant with three factors is worthy of offerings dedicated to the gods, worthy of hospitality, worthy of a religious donation, worthy of veneration with joined palms, and is the supreme field of merit for the world. What three? They’re a long-distance shooter, a marksman, and one who shatters large objects. 

And\marginnote{2.1} how is a mendicant a long-distance shooter? It’s when a mendicant truly sees any kind of form at all—past, future, or present; internal or external; coarse or fine; inferior or superior; far or near: \emph{all} form—with right understanding: ‘This is not mine, I am not this, this is not my self.’ They truly see any kind of feeling at all—past, future, or present; internal or external; coarse or fine; inferior or superior; far or near: \emph{all} feeling—with right understanding: ‘This is not mine, I am not this, this is not my self.’ They truly see any kind of perception at all—past, future, or present; internal or external; coarse or fine; inferior or superior; far or near: \emph{all} perception—with right understanding: ‘This is not mine, I am not this, this is not my self.’ They truly see any kind of choices at all—past, future, or present; internal or external; coarse or fine; inferior or superior; far or near: \emph{all} choices—with right understanding: ‘This is not mine, I am not this, this is not my self.’ They truly see any kind of consciousness at all—past, future, or present; internal or external; coarse or fine; inferior or superior; far or near, \emph{all} consciousness—with right understanding: ‘This is not mine, I am not this, this is not my self.’ That’s how a mendicant is a long-distance shooter. 

And\marginnote{3.1} how is a mendicant a marksman? It’s when a mendicant truly understands: ‘This is suffering’ … ‘This is the origin of suffering’ … ‘This is the cessation of suffering’ … ‘This is the practice that leads to the cessation of suffering’. That’s how a mendicant is a marksman. 

And\marginnote{4.1} how does a mendicant shatter large objects? It’s when a mendicant shatters the great mass of ignorance. That’s how a mendicant shatters large objects. 

A\marginnote{4.4} mendicant with these three qualities is worthy of offerings dedicated to the gods, worthy of hospitality, worthy of a religious donation, worthy of veneration with joined palms, and is the supreme field of merit for the world.” 

%
\section*{{\suttatitleacronym AN 3.134}{\suttatitletranslation Assemblies }{\suttatitleroot Parisāsutta}}
\addcontentsline{toc}{section}{\tocacronym{AN 3.134} \toctranslation{Assemblies } \tocroot{Parisāsutta}}
\markboth{Assemblies }{Parisāsutta}
\extramarks{AN 3.134}{AN 3.134}

“Mendicants,\marginnote{1.1} there are these three assemblies. What three? An assembly educated in fancy talk, an assembly educated in questioning, and an assembly educated to the fullest extent. These are the three assemblies.” 

%
\section*{{\suttatitleacronym AN 3.135}{\suttatitletranslation A Friend }{\suttatitleroot Mittasutta}}
\addcontentsline{toc}{section}{\tocacronym{AN 3.135} \toctranslation{A Friend } \tocroot{Mittasutta}}
\markboth{A Friend }{Mittasutta}
\extramarks{AN 3.135}{AN 3.135}

“Mendicants,\marginnote{1.1} you should associate with a friend who has three factors. What three? They give what is hard to give, they do what is hard to do, and they bear what is hard to bear. You should associate with a friend who has these three factors.” 

%
\section*{{\suttatitleacronym AN 3.136}{\suttatitletranslation Arising }{\suttatitleroot Uppādāsutta}}
\addcontentsline{toc}{section}{\tocacronym{AN 3.136} \toctranslation{Arising } \tocroot{Uppādāsutta}}
\markboth{Arising }{Uppādāsutta}
\extramarks{AN 3.136}{AN 3.136}

“Mendicants,\marginnote{1.1} whether Realized Ones arise or not, this law of nature persists, this regularity of natural principles, this invariance of natural principles: all conditions are impermanent. A Realized One understands this and comprehends it, then he explains, teaches, asserts, establishes, clarifies, analyzes, and reveals it: ‘All conditions are impermanent.’ 

Whether\marginnote{1.6} Realized Ones arise or not, this law of nature persists, this regularity of natural principles, this invariance of natural principles: all conditions are suffering. A Realized One understands this and comprehends it, then he explains, teaches, asserts, establishes, clarifies, analyzes, and reveals it: ‘All conditions are suffering.’ 

Whether\marginnote{1.11} Realized Ones arise or not, this law of nature persists, this regularity of natural principles, this invariance of natural principles: all things are not-self. A Realized One understands this and comprehends it, then he explains, teaches, asserts, establishes, clarifies, analyzes, and reveals it: ‘All things are not-self.’” 

%
\section*{{\suttatitleacronym AN 3.137}{\suttatitletranslation A Hair Blanket }{\suttatitleroot Kesakambalasutta}}
\addcontentsline{toc}{section}{\tocacronym{AN 3.137} \toctranslation{A Hair Blanket } \tocroot{Kesakambalasutta}}
\markboth{A Hair Blanket }{Kesakambalasutta}
\extramarks{AN 3.137}{AN 3.137}

“Mendicants,\marginnote{1.1} a hair blanket is said to be the worst kind of woven cloth. It’s cold in the cold, hot in the heat, ugly, smelly, and unpleasant to touch. In the same way, the teaching of Makkhali is said to be the worst of all the doctrines of the various ascetics and brahmins. 

Makkhali,\marginnote{2.1} that silly man, has this doctrine and view: ‘There is no power in deeds, action, or energy.’ 

Now,\marginnote{2.3} all the perfected ones, the fully awakened Buddhas who lived in the past taught the efficacy of deeds, action, and energy. But Makkhali opposes them by saying: ‘There is no power in deeds, action, or energy.’ 

All\marginnote{2.6} the perfected ones, the fully awakened Buddhas who will live in the future will teach the efficacy of deeds, action, and energy. But Makkhali opposes them by saying: ‘There is no power in deeds, action, or energy.’ 

I\marginnote{2.9} too, the perfected one, the fully awakened Buddha in the present, teach the efficacy of deeds, action, and energy. But Makkhali opposes me by saying: ‘There is no power in deeds, action, or energy.’ 

It’s\marginnote{3.1} like a trap set at the mouth of a river, which would bring harm, suffering, calamity, and disaster for many fish. In the same way that silly man Makkhali is a trap for humans, it seems to me. He has come into the world for the harm, suffering, calamity, and disaster of many beings.” 

%
\section*{{\suttatitleacronym AN 3.138}{\suttatitletranslation Accomplishment }{\suttatitleroot Sampadāsutta}}
\addcontentsline{toc}{section}{\tocacronym{AN 3.138} \toctranslation{Accomplishment } \tocroot{Sampadāsutta}}
\markboth{Accomplishment }{Sampadāsutta}
\extramarks{AN 3.138}{AN 3.138}

“Mendicants,\marginnote{1.1} there are three accomplishments. What three? Accomplishment in faith, ethics, and wisdom. These are the three accomplishments.” 

%
\section*{{\suttatitleacronym AN 3.139}{\suttatitletranslation Growth }{\suttatitleroot Vuddhisutta}}
\addcontentsline{toc}{section}{\tocacronym{AN 3.139} \toctranslation{Growth } \tocroot{Vuddhisutta}}
\markboth{Growth }{Vuddhisutta}
\extramarks{AN 3.139}{AN 3.139}

“Mendicants,\marginnote{1.1} there are three kinds of growth. What three? Growth in faith, ethics, and wisdom. These are the three kinds of growth.” 

%
\section*{{\suttatitleacronym AN 3.140}{\suttatitletranslation A Wild Colt }{\suttatitleroot Assakhaḷuṅkasutta}}
\addcontentsline{toc}{section}{\tocacronym{AN 3.140} \toctranslation{A Wild Colt } \tocroot{Assakhaḷuṅkasutta}}
\markboth{A Wild Colt }{Assakhaḷuṅkasutta}
\extramarks{AN 3.140}{AN 3.140}

“Mendicants,\marginnote{1.1} I will teach you about three wild colts and three wild people. Listen and pay close attention, I will speak.” 

“Yes,\marginnote{1.3} sir,” they replied. The Buddha said this: 

“What\marginnote{2.1} are the three wild colts? One wild colt is fast, but not beautiful or well proportioned. Another wild colt is fast and beautiful, but not well proportioned. While another wild colt is fast, beautiful, and well proportioned. 

These\marginnote{2.5} are the three wild colts. 

And\marginnote{3.1} what are the three wild people? One wild person is fast, but not beautiful or well proportioned. Another wild person is fast and beautiful, but not well proportioned. While another wild person is fast, beautiful, and well proportioned. 

And\marginnote{4.1} how is a wild person fast, but not beautiful or well proportioned? It’s when a mendicant truly understands: ‘This is suffering’ … ‘This is the origin of suffering’ … ‘This is the cessation of suffering’ … ‘This is the practice that leads to the cessation of suffering’. This is how they’re fast, I say. But when asked a question about the teaching or training, they falter without answering. This is how they’re not beautiful, I say. And they don’t receive robes, almsfood, lodgings, and medicines and supplies for the sick. This is how they’re not well proportioned, I say. This is how a wild person is fast, but not beautiful or well proportioned. 

And\marginnote{5.1} how is a wild person fast and beautiful, but not well proportioned? It’s when a mendicant truly understands: ‘This is suffering’ … ‘This is the origin of suffering’ … ‘This is the cessation of suffering’ … ‘This is the practice that leads to the cessation of suffering’. This is how they’re fast, I say. When asked a question about the teaching or training, they answer without faltering. This is how they’re beautiful, I say. But they don’t receive robes, almsfood, lodgings, and medicines and supplies for the sick. This is how they’re not well proportioned, I say. This is how a wild person is fast and beautiful, but not well proportioned. 

And\marginnote{6.1} how is a wild person fast, beautiful, and well proportioned? It’s when a mendicant truly understands: ‘This is suffering’ … ‘This is the origin of suffering’ … ‘This is the cessation of suffering’ … ‘This is the practice that leads to the cessation of suffering’. This is how they’re fast, I say. When asked a question about the teaching or training, they answer without faltering. This is how they’re beautiful, I say. They receive robes, almsfood, lodgings, and medicines and supplies for the sick. This is how they’re well proportioned, I say. This is how a wild person is fast, beautiful, and well proportioned. 

These\marginnote{6.9} are the three wild people.” 

%
\section*{{\suttatitleacronym AN 3.141}{\suttatitletranslation Excellent Horses }{\suttatitleroot Assaparassasutta}}
\addcontentsline{toc}{section}{\tocacronym{AN 3.141} \toctranslation{Excellent Horses } \tocroot{Assaparassasutta}}
\markboth{Excellent Horses }{Assaparassasutta}
\extramarks{AN 3.141}{AN 3.141}

“Mendicants,\marginnote{1.1} I will teach you the three excellent horses and the three excellent people. Listen and pay close attention, I will speak.” 

“Yes,\marginnote{1.3} sir,” they replied. The Buddha said this: 

“What\marginnote{2.1} are the three excellent horses? One excellent horse is fast, but not beautiful or well proportioned. Another excellent horse is fast and beautiful, but not well proportioned. While another excellent horse is fast, beautiful, and well proportioned. 

These\marginnote{2.5} are the three excellent horses. 

“What\marginnote{3.1} are the three excellent people? One excellent person is fast, but not beautiful or well proportioned. Another excellent person is fast and beautiful, but not well proportioned. While another excellent person is fast, beautiful, and well proportioned. 

And\marginnote{4.1} how is an excellent person fast, but not beautiful or well proportioned? It’s when a mendicant, with the ending of the five lower fetters, is reborn spontaneously. They’re extinguished there, and are not liable to return from that world. This is how they’re fast, I say. But when asked a question about the teaching or training, they falter without answering. This is how they’re not beautiful, I say. And they don’t receive robes, almsfood, lodgings, and medicines and supplies for the sick. This is how they’re not well proportioned, I say. This is how an excellent person is fast, but not beautiful or well proportioned. 

And\marginnote{5.1} how is an excellent person fast and beautiful, but not well proportioned? It’s when a mendicant, with the ending of the five lower fetters, is reborn spontaneously. They’re extinguished there, and are not liable to return from that world. This is how they’re fast, I say. When asked a question about the teaching or training, they answer without faltering. This is how they’re beautiful, I say. But they don’t receive robes, almsfood, lodgings, and medicines and supplies for the sick. This is how they’re not well proportioned, I say. This is how an excellent person is fast and beautiful, but not well proportioned. 

And\marginnote{6.1} how is an excellent person fast, beautiful, and well proportioned? It’s when a mendicant, with the ending of the five lower fetters, is reborn spontaneously. They’re extinguished there, and are not liable to return from that world. This is how they’re fast, I say. When asked a question about the teaching or training, they answer without faltering. This is how they’re beautiful, I say. They receive robes, almsfood, lodgings, and medicines and supplies for the sick. This is how they’re well proportioned, I say. This is how an excellent person is fast, beautiful, and well proportioned. 

These\marginnote{6.9} are the three excellent people.” 

%
\section*{{\suttatitleacronym AN 3.142}{\suttatitletranslation The Thoroughbred }{\suttatitleroot Assājānīyasutta}}
\addcontentsline{toc}{section}{\tocacronym{AN 3.142} \toctranslation{The Thoroughbred } \tocroot{Assājānīyasutta}}
\markboth{The Thoroughbred }{Assājānīyasutta}
\extramarks{AN 3.142}{AN 3.142}

“Mendicants,\marginnote{1.1} I will teach you the three fine thoroughbred horses, and the three fine thoroughbred people. Listen and pay close attention, I will speak.” 

“Yes,\marginnote{1.3} sir,” they replied. The Buddha said this: 

“What\marginnote{2.1} are the three fine thoroughbred horses? One fine thoroughbred horse … is fast, beautiful, and well proportioned. 

These\marginnote{2.4} are the three fine thoroughbred horses. 

And\marginnote{3.1} what are the three fine thoroughbred people? One fine thoroughbred person … is fast, beautiful, and well proportioned. 

And\marginnote{4.1} how is a fine thoroughbred person … fast, beautiful, and well proportioned? It’s when a mendicant realizes the undefiled freedom of heart and freedom by wisdom in this very life. And they live having realized it with their own insight due to the ending of defilements. This is how they’re fast, I say. When asked a question about the teaching or training, they answer without faltering. This is how they’re beautiful, I say. They receive robes, almsfood, lodgings, and medicines and supplies for the sick. This is how they’re well proportioned, I say. This is how a fine thoroughbred person is fast, beautiful, and well proportioned. 

These\marginnote{4.10} are the three fine thoroughbred people.” 

%
\section*{{\suttatitleacronym AN 3.143}{\suttatitletranslation At the Peacocks’ Feeding Ground (1st) }{\suttatitleroot Paṭhamamoranivāpasutta}}
\addcontentsline{toc}{section}{\tocacronym{AN 3.143} \toctranslation{At the Peacocks’ Feeding Ground (1st) } \tocroot{Paṭhamamoranivāpasutta}}
\markboth{At the Peacocks’ Feeding Ground (1st) }{Paṭhamamoranivāpasutta}
\extramarks{AN 3.143}{AN 3.143}

At\marginnote{1.1} one time the Buddha was staying near \textsanskrit{Rājagaha}, at the monastery of the wanderers in the peacocks’ feeding ground. There the Buddha addressed the mendicants, “Mendicants!” 

“Venerable\marginnote{1.4} sir,” they replied. The Buddha said this: 

“Mendicants,\marginnote{2.1} a mendicant with three qualities has reached the ultimate end, the ultimate sanctuary, the ultimate spiritual life, the ultimate goal. They are the best among gods and humans. What three? The entire spectrum of an adept’s ethics, immersion, and wisdom. 

A\marginnote{2.4} mendicant with these three qualities has reached the ultimate end, the ultimate sanctuary, the ultimate spiritual life, the ultimate goal. They are the best among gods and humans.” 

%
\section*{{\suttatitleacronym AN 3.144}{\suttatitletranslation At the Peacocks’ Feeding Ground (2nd) }{\suttatitleroot Dutiyamoranivāpasutta}}
\addcontentsline{toc}{section}{\tocacronym{AN 3.144} \toctranslation{At the Peacocks’ Feeding Ground (2nd) } \tocroot{Dutiyamoranivāpasutta}}
\markboth{At the Peacocks’ Feeding Ground (2nd) }{Dutiyamoranivāpasutta}
\extramarks{AN 3.144}{AN 3.144}

“Mendicants,\marginnote{1.1} a mendicant who has three qualities has reached the ultimate end, the ultimate sanctuary, the ultimate spiritual life, the ultimate goal. They are the best among gods and humans. What three? A demonstration of psychic power, a demonstration of revealing, and a demonstration of instruction. 

A\marginnote{1.4} mendicant with these three qualities has reached the ultimate end, the ultimate sanctuary, the ultimate spiritual life, the ultimate goal. They are the best among gods and humans.” 

%
\section*{{\suttatitleacronym AN 3.145}{\suttatitletranslation At the Peacocks’ Feeding Ground (3rd) }{\suttatitleroot Tatiyamoranivāpasutta}}
\addcontentsline{toc}{section}{\tocacronym{AN 3.145} \toctranslation{At the Peacocks’ Feeding Ground (3rd) } \tocroot{Tatiyamoranivāpasutta}}
\markboth{At the Peacocks’ Feeding Ground (3rd) }{Tatiyamoranivāpasutta}
\extramarks{AN 3.145}{AN 3.145}

“Mendicants,\marginnote{1.1} a mendicant who has three qualities has reached the ultimate end, the ultimate sanctuary, the ultimate spiritual life, the ultimate goal. They are the best among gods and humans. What three? Right view, right knowledge, and right freedom. 

A\marginnote{1.4} mendicant with these three qualities has reached the ultimate end, the ultimate sanctuary, the ultimate spiritual life, the ultimate goal. They are the best among gods and humans.” 

%
\addtocontents{toc}{\let\protect\contentsline\protect\nopagecontentsline}
\chapter*{The Chapter on Good Fortune }
\addcontentsline{toc}{chapter}{\tocchapterline{The Chapter on Good Fortune }}
\addtocontents{toc}{\let\protect\contentsline\protect\oldcontentsline}

%
\section*{{\suttatitleacronym AN 3.146}{\suttatitletranslation Unskillful }{\suttatitleroot Akusalasutta}}
\addcontentsline{toc}{section}{\tocacronym{AN 3.146} \toctranslation{Unskillful } \tocroot{Akusalasutta}}
\markboth{Unskillful }{Akusalasutta}
\extramarks{AN 3.146}{AN 3.146}

“Someone\marginnote{1.1} with three qualities is cast down to hell. What three? Unskillful deeds by way of body, speech, and mind. 

Someone\marginnote{1.4} with these three qualities is cast down to hell. 

Someone\marginnote{2.1} with three qualities is raised up to heaven. What three? Skillful deeds by way of body, speech, and mind. 

Someone\marginnote{2.4} with these three qualities is raised up to heaven.” 

%
\section*{{\suttatitleacronym AN 3.147}{\suttatitletranslation Blameworthy }{\suttatitleroot Sāvajjasutta}}
\addcontentsline{toc}{section}{\tocacronym{AN 3.147} \toctranslation{Blameworthy } \tocroot{Sāvajjasutta}}
\markboth{Blameworthy }{Sāvajjasutta}
\extramarks{AN 3.147}{AN 3.147}

“Someone\marginnote{1.1} with three qualities is cast down to hell. What three? Blameworthy deeds by way of body, speech, and mind. 

Someone\marginnote{1.4} with these three qualities is cast down to hell. 

Someone\marginnote{2.1} with three qualities is raised up to heaven. What three? Blameless deeds by way of body, speech, and mind. Someone with these three qualities is raised up to heaven.” 

%
\section*{{\suttatitleacronym AN 3.148}{\suttatitletranslation Unethical }{\suttatitleroot Visamasutta}}
\addcontentsline{toc}{section}{\tocacronym{AN 3.148} \toctranslation{Unethical } \tocroot{Visamasutta}}
\markboth{Unethical }{Visamasutta}
\extramarks{AN 3.148}{AN 3.148}

“Someone\marginnote{1.1} with three qualities is cast down to hell. … Unethical deeds by way of body, speech, and mind. … 

Someone\marginnote{2.1} with three qualities is raised up to heaven. … Ethical deeds by way of body, speech, and mind. …” 

%
\section*{{\suttatitleacronym AN 3.149}{\suttatitletranslation Impure }{\suttatitleroot Asucisutta}}
\addcontentsline{toc}{section}{\tocacronym{AN 3.149} \toctranslation{Impure } \tocroot{Asucisutta}}
\markboth{Impure }{Asucisutta}
\extramarks{AN 3.149}{AN 3.149}

“Someone\marginnote{1.1} with three qualities is cast down to hell. … Impure deeds by way of body, speech, and mind. … 

Someone\marginnote{2.1} with three qualities is raised up to heaven. … Pure deeds by way of body, speech, and mind. …” 

%
\section*{{\suttatitleacronym AN 3.150}{\suttatitletranslation Broken (1st) }{\suttatitleroot Paṭhamakhatasutta}}
\addcontentsline{toc}{section}{\tocacronym{AN 3.150} \toctranslation{Broken (1st) } \tocroot{Paṭhamakhatasutta}}
\markboth{Broken (1st) }{Paṭhamakhatasutta}
\extramarks{AN 3.150}{AN 3.150}

“When\marginnote{1.1} a foolish, incompetent, bad person has three qualities they keep themselves broken and damaged. They deserve to be blamed and reproved by sensible people, and they make much bad karma. What three? Unskillful deeds by way of body, speech, and mind. … 

When\marginnote{2.1} an astute, competent, good person has three qualities they keep themselves healthy and whole. They don’t deserve to be blamed and criticized by sensible people, and they make much merit. What three? Skillful deeds by way of body, speech, and mind. …” 

%
\section*{{\suttatitleacronym AN 3.151}{\suttatitletranslation Broken (2nd) }{\suttatitleroot Dutiyakhatasutta}}
\addcontentsline{toc}{section}{\tocacronym{AN 3.151} \toctranslation{Broken (2nd) } \tocroot{Dutiyakhatasutta}}
\markboth{Broken (2nd) }{Dutiyakhatasutta}
\extramarks{AN 3.151}{AN 3.151}

“When\marginnote{1.1} a foolish, incompetent, bad person has three qualities they keep themselves broken and damaged. … Blameworthy deeds by way of body, speech, and mind. … 

When\marginnote{2.1} an astute, competent, good person has three qualities they keep themselves healthy and whole. … Blameless deeds by way of body, speech, and mind. …” 

%
\section*{{\suttatitleacronym AN 3.152}{\suttatitletranslation Broken (3rd) }{\suttatitleroot Tatiyakhatasutta}}
\addcontentsline{toc}{section}{\tocacronym{AN 3.152} \toctranslation{Broken (3rd) } \tocroot{Tatiyakhatasutta}}
\markboth{Broken (3rd) }{Tatiyakhatasutta}
\extramarks{AN 3.152}{AN 3.152}

“When\marginnote{1.1} a foolish, incompetent, bad person has three qualities they keep themselves broken and damaged. … Unethical deeds by way of body, speech, and mind. … 

When\marginnote{2.1} an astute, competent, good person has three qualities they keep themselves healthy and whole. … Ethical deeds by way of body, speech, and mind. …” 

%
\section*{{\suttatitleacronym AN 3.153}{\suttatitletranslation Broken (4th) }{\suttatitleroot Catutthakhatasutta}}
\addcontentsline{toc}{section}{\tocacronym{AN 3.153} \toctranslation{Broken (4th) } \tocroot{Catutthakhatasutta}}
\markboth{Broken (4th) }{Catutthakhatasutta}
\extramarks{AN 3.153}{AN 3.153}

“When\marginnote{1.1} a foolish, incompetent, bad person has three qualities they keep themselves broken and damaged. … Impure deeds by way of body, speech, and mind. … 

When\marginnote{2.1} an astute, competent, good person has three qualities they keep themselves healthy and whole. … Pure deeds by way of body, speech, and mind. …” 

%
\section*{{\suttatitleacronym AN 3.154}{\suttatitletranslation Homage }{\suttatitleroot Vandanāsutta}}
\addcontentsline{toc}{section}{\tocacronym{AN 3.154} \toctranslation{Homage } \tocroot{Vandanāsutta}}
\markboth{Homage }{Vandanāsutta}
\extramarks{AN 3.154}{AN 3.154}

“Mendicants,\marginnote{1.1} there are three kinds of homage. What three? By way of body, speech, and mind. These are the three kinds of homage.” 

%
\section*{{\suttatitleacronym AN 3.155}{\suttatitletranslation Morning }{\suttatitleroot Pubbaṇhasutta}}
\addcontentsline{toc}{section}{\tocacronym{AN 3.155} \toctranslation{Morning } \tocroot{Pubbaṇhasutta}}
\markboth{Morning }{Pubbaṇhasutta}
\extramarks{AN 3.155}{AN 3.155}

“Mendicants,\marginnote{1.1} those sentient beings who do good things in the morning by way of body, speech, and mind have a good morning. 

Those\marginnote{2.1} sentient beings who do good things at midday by way of body, speech, and mind have a good midday. 

Those\marginnote{3.1} sentient beings who do good things in the evening by way of body, speech, and mind have a good evening. 

\begin{verse}%
A\marginnote{4.1} good star, a good fortune, \\
a good dawn, a good rising, \\
a good moment, a good hour: \\
these come with good gifts to spiritual practitioners. 

Worthy\marginnote{5.1} deeds of body, \\
verbal worthy deeds, \\
worthy deeds of mind, \\
worthy resolutions: \\
when your deeds have been worthy, \\
you get worthy benefits. 

Those\marginnote{6.1} happy with these benefits \\
flourish in the Buddha’s teaching. \\
May you and all your relatives \\
be healthy and happy!” 

%
\end{verse}

%
\addtocontents{toc}{\let\protect\contentsline\protect\nopagecontentsline}
\chapter*{The Chapter on Practice }
\addcontentsline{toc}{chapter}{\tocchapterline{The Chapter on Practice }}
\addtocontents{toc}{\let\protect\contentsline\protect\oldcontentsline}

%
\section*{{\suttatitleacronym AN 3.156–162}{\suttatitletranslation Untitled Discourses on Three Practices }{\suttatitleroot Acelakavagga}}
\addcontentsline{toc}{section}{\tocacronym{AN 3.156–162} \toctranslation{Untitled Discourses on Three Practices } \tocroot{Acelakavagga}}
\markboth{Untitled Discourses on Three Practices }{Acelakavagga}
\extramarks{AN 3.156–162}{AN 3.156–162}

“Mendicants,\marginnote{1.1} there are three practices. What three? The addicted practice, the scorching practice, and the middle practice. 

And\marginnote{1.4} what’s the addicted practice? It’s when someone has this doctrine and view: ‘There’s nothing wrong with sensual pleasures’; so they throw themselves into sensual pleasures. This is called the addicted practice. 

And\marginnote{2.1} what’s the scorching practice? It’s when someone goes naked, ignoring conventions. They lick their hands, and don’t come or wait when called. They don’t consent to food brought to them, or food prepared on purpose for them, or an invitation for a meal. They don’t receive anything from a pot or bowl; or from someone who keeps sheep, or who has a weapon or a shovel in their home; or where a couple is eating; or where there is a woman who is pregnant, breastfeeding, or who has a man in her home; or where there’s a dog waiting or flies buzzing. They accept no fish or meat or liquor or wine, and drink no beer. They go to just one house for alms, taking just one mouthful, or two houses and two mouthfuls, up to seven houses and seven mouthfuls. They feed on one saucer a day, two saucers a day, up to seven saucers a day. They eat once a day, once every second day, up to once a week, and so on, even up to once a fortnight. They live pursuing the practice of eating food at set intervals. 

They\marginnote{3.1} eat herbs, millet, wild rice, poor rice, water lettuce, rice bran, scum from boiling rice, sesame flour, grass, or cow dung. They survive on forest roots and fruits, or eating fallen fruit. 

They\marginnote{4.1} wear robes of sunn hemp, mixed hemp, corpse-wrapping cloth, rags, lodh tree bark, antelope hide (whole or in strips), kusa grass, bark, wood-chips, human hair, horse-tail hair, or owls’ wings. They tear out their hair and beard, committed to this practice. They constantly stand, refusing seats. They squat, committed to persisting in the squatting position. They lie on a mat of thorns, making a mat of thorns their bed. They pursue the practice of immersion in water three times a day, including the evening. And so they live pursuing these various ways of mortifying and tormenting the body. This is called the scorching practice. 

And\marginnote{5.1} what’s the middle practice? It’s when a mendicant meditates by observing an aspect of the body—keen, aware, and mindful, rid of desire and aversion for the world. They meditate observing an aspect of feelings … They meditate observing an aspect of the mind … They meditate observing an aspect of principles—keen, aware, and mindful, rid of desire and aversion for the world. This is called the middle practice. 

These\marginnote{5.7} are the three practices. 

Mendicants,\marginnote{1.1} there are three practices. What three? The addicted practice, the scorching practice, the middle practice. 

And\marginnote{1.4} what’s the addicted practice? … This is called the addicted practice. 

And\marginnote{2.1} what is the scorching practice? … This is called the scorching practice. 

And\marginnote{3.1} what’s the middle practice? It’s when a mendicant generates enthusiasm, tries, makes an effort, exerts the mind, and strives so that bad, unskillful qualities don’t arise. They generate enthusiasm, try, make an effort, exert the mind, and strive so that bad, unskillful qualities that have arisen are given up. They generate enthusiasm, try, make an effort, exert the mind, and strive so that skillful qualities arise. They generate enthusiasm, try, make an effort, exert the mind, and strive so that skillful qualities that have arisen remain, are not lost, but increase, mature, and are completed by development. … 

They\marginnote{1.1} develop the basis of psychic power that has immersion due to enthusiasm, and active effort. They develop the basis of psychic power that has immersion due to energy, and active effort. They develop the basis of psychic power that has immersion due to mental development, and active effort. They develop the basis of psychic power that has immersion due to inquiry, and active effort. … 

They\marginnote{1.1} develop the faculty of faith … energy … mindfulness … immersion … wisdom … 

They\marginnote{1.1} develop the power of faith … energy … mindfulness … immersion … wisdom … 

They\marginnote{1.1} develop the awakening factor of mindfulness … investigation of principles … energy … rapture … tranquility … immersion … equanimity … 

They\marginnote{1.1} develop right view … right thought … right speech … right action … right livelihood … right effort … right mindfulness … right immersion … This is called the middle practice. These are the three practices.” 

%
\addtocontents{toc}{\let\protect\contentsline\protect\nopagecontentsline}
\chapter*{The Chapter on Ways of Performing Deeds }
\addcontentsline{toc}{chapter}{\tocchapterline{The Chapter on Ways of Performing Deeds }}
\addtocontents{toc}{\let\protect\contentsline\protect\oldcontentsline}

%
\section*{{\suttatitleacronym AN 3.163–182}{\suttatitletranslation Untitled Discourses on Three Qualities }{\suttatitleroot Kammapathapeyyāla}}
\addcontentsline{toc}{section}{\tocacronym{AN 3.163–182} \toctranslation{Untitled Discourses on Three Qualities } \tocroot{Kammapathapeyyāla}}
\markboth{Untitled Discourses on Three Qualities }{Kammapathapeyyāla}
\extramarks{AN 3.163–182}{AN 3.163–182}

“Someone\marginnote{1.1} with three qualities is cast down to hell. What three? They themselves kill living creatures. They encourage others to kill living creatures. And they approve of killing living creatures. 

Someone\marginnote{1.4} with these three qualities is cast down to hell. 

Someone\marginnote{1.1} with three qualities is raised up to heaven. What three? They don’t themselves kill living creatures. They encourage others to not kill living creatures. And they approve of not killing living creatures. … 

They\marginnote{1.1} themselves steal. They encourage others to steal. And they approve of stealing. … 

They\marginnote{1.1} don’t themselves steal. They encourage others to not steal. And they approve of not stealing. … 

They\marginnote{1.1} themselves commit sexual misconduct. They encourage others to commit sexual misconduct. And they approve of committing sexual misconduct. … 

They\marginnote{1.1} don’t themselves commit sexual misconduct. They encourage others to not commit sexual misconduct. And they approve of not committing sexual misconduct. … 

They\marginnote{1.1} themselves lie. They encourage others to lie. And they approve of lying. … 

They\marginnote{1.1} don’t themselves lie. They encourage others to not lie. And they approve of not lying. … 

They\marginnote{1.1} themselves speak divisively. They encourage others to speak divisively. And they approve of speaking divisively. … 

They\marginnote{1.1} don’t themselves speak divisively. They encourage others to not speak divisively. And they approve of not speaking divisively. … 

They\marginnote{1.1} themselves speak harshly. They encourage others to speak harshly. And they approve of speaking harshly. … 

They\marginnote{1.1} don’t themselves speak harshly. They encourage others to not speak harshly. And they approve of not speaking harshly. … 

They\marginnote{1.1} themselves talk nonsense. They encourage others to talk nonsense. And they approve of talking nonsense. … 

They\marginnote{1.1} don’t themselves talk nonsense. They encourage others to not talk nonsense. And they approve of not talking nonsense. … 

They\marginnote{1.1} themselves are covetous. They encourage others to be covetous. And they approve of covetousness. … 

They\marginnote{1.1} themselves are content. They encourage others to be contented. And they approve of being contented. … 

They\marginnote{1.1} themselves have ill will. They encourage others to have ill will. And they approve of having ill will. … 

They\marginnote{1.1} themselves are kind-hearted. They encourage others to be kind-hearted. And they approve of kind-heartedness. … 

They\marginnote{1.1} themselves have wrong view. They encourage others to have wrong view. And they approve of wrong view. … 

They\marginnote{1.1} themselves have right view. They encourage others to have right view. And they approve of right view. 

Someone\marginnote{1.2} with these three qualities is raised up to heaven.” 

%
\addtocontents{toc}{\let\protect\contentsline\protect\nopagecontentsline}
\chapter*{The Chapter on Abbreviated Texts Beginning with Greed }
\addcontentsline{toc}{chapter}{\tocchapterline{The Chapter on Abbreviated Texts Beginning with Greed }}
\addtocontents{toc}{\let\protect\contentsline\protect\oldcontentsline}

%
\section*{{\suttatitleacronym AN 3.183–352}{\suttatitletranslation Untitled Discourses on Greed, Etc. }{\suttatitleroot Rāgapeyyāla}}
\addcontentsline{toc}{section}{\tocacronym{AN 3.183–352} \toctranslation{Untitled Discourses on Greed, Etc. } \tocroot{Rāgapeyyāla}}
\markboth{Untitled Discourses on Greed, Etc. }{Rāgapeyyāla}
\extramarks{AN 3.183–352}{AN 3.183–352}

“For\marginnote{1.1} insight into greed, three things should be developed. What three? Emptiness immersion; signless immersion; and undirected immersion. For insight into greed, these three things should be developed. 

For\marginnote{2.1} the complete understanding of greed … complete ending … giving up … ending … vanishing … fading away … cessation … giving away … letting go … 

hate\marginnote{3.1} … delusion … anger … hostility … disdain … contempt … jealousy … stinginess … deceitfulness … deviousness … obstinacy … aggression … conceit … arrogance … vanity … negligence … insight … complete understanding … complete ending … giving up … ending … vanishing … fading away … cessation … giving away … For the letting go of negligence, these three things should be developed.” 

That\marginnote{4.1} is what the Buddha said. Satisfied, the mendicants were happy with what the Buddha said. 

\scendbook{The Book of the Threes is finished. }

%
\backmatter%
\chapter*{Colophon}
\addcontentsline{toc}{chapter}{Colophon}
\markboth{Colophon}{Colophon}

\section*{The Translator}

Bhikkhu Sujato was born as Anthony Aidan Best on 4/11/1966 in Perth, Western Australia. He grew up in the pleasant suburbs of Mt Lawley and Attadale alongside his sister Nicola, who was the good child. His mother, Margaret Lorraine Huntsman née Pinder, said “he’ll either be a priest or a poet”, while his father, Anthony Thomas Best, advised him to “never do anything for money”. He attended Aquinas College, a Catholic school, where he decided to become an atheist. At the University of WA he studied philosophy, aiming to learn what he wanted to do with his life. Finding that what he wanted to do was play guitar, he dropped out. His main band was named Martha’s Vineyard, which achieved modest success in the indie circuit. Then it broke up, because everyone thought they personally were reason for the success, which, oddly enough, turns out not to have been the case. 

A seemingly random encounter with a roadside joey took him to Thailand, where he entered his first meditation retreat at Wat Ram Poeng, Chieng Mai in 1992. He decided to devote himself to the Buddha’s path, and took full ordination in Wat Pa Nanachat in 1994, where his teachers were Ajahn Pasanno and Ajahn Jayasaro. In 1997 he returned to Perth to study with Ajahn Brahm at Bodhinyana Monastery. 

He spent several years practicing in seclusion in Malaysia and Thailand before establishing Santi Forest Monastery in Bundanoon, NSW, in 2003. There he was instrumental in supporting the establishment of the Theravada bhikkhuni order in Australia and advocating for women’s rights. He continues to teach in Australia and globally, with a special concern for the moral implications of climate change and other forms of environmental destruction. He has published a series of books of original and groundbreaking research on early Buddhism. 

In 2005 he founded SuttaCentral together with Rod Bucknell and John Kelly. In 2015, seeing the need for a complete, accurate, plain English translation of the Pali texts, he undertook the task, spending nearly three years in isolation on the isle of Qi Mei off the coast of the nation of Taiwan. He completed the four main \textsanskrit{Nikāyas} in 2018, and the early books of the Khuddaka \textsanskrit{Nikāya} were complete by 2021. All this work is dedicated to the public domain and is entirely free of copyright encumbrance. 

In 2019 he returned to Sydney where, together with Bhikkhu Akaliko, he established Lokanta Vihara (The Monastery at the End of the World). 

\section*{Creation Process}

Primary source was the digital \textsanskrit{Mahāsaṅgīti} edition of the Pali \textsanskrit{Tipiṭaka}. Translated from the Pali, with reference to several English translations, especially those of Bhikkhu Bodhi.

\section*{The Translation}

This translation was part of a project to translate the four Pali \textsanskrit{Nikāyas} with the following aims: plain, approachable English; consistent terminology; accurate rendition of the Pali; free of copyright. It was made during 2016–2018 while Bhikkhu Sujato was staying in Qimei, Taiwan.

\section*{About SuttaCentral}

SuttaCentral publishes early Buddhist texts. Since 2005 we have provided root texts in Pali, Chinese, Sanskrit, Tibetan, and other languages, parallels between these texts, and translations in many modern languages. We build on the work of generations of scholars, and offer our contribution freely.

SuttaCentral is driven by volunteer contributions, and in addition we employ professional developers. We offer a sponsorship program for high quality translations from the original languages. Financial support for SuttaCentral is handled by the SuttaCentral Development Trust, a charitable trust registered in Australia.

\section*{About Bilara}

“Bilara” means “cat” in Pali, and it is the name of our Computer Assisted Translation (CAT) software. Bilara is a web app that enables translators to translate early Buddhist texts into their own language. These translations are published on SuttaCentral with the root text and translation side by side.

\section*{About SuttaCentral Editions}

The SuttaCentral Editions project makes high quality books from selected Bilara translations. These are published in formats including HTML, EPUB, PDF, and print.

If you want to print any of our Editions, please let us know and we will help prepare a file to your specifications.

%
\end{document}