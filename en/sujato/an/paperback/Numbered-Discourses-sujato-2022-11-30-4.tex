\documentclass[12pt,openany]{book}%
\usepackage{lastpage}%
%
\usepackage[inner=1in, outer=1in, top=.7in, bottom=1in, papersize={6in,9in}, headheight=13pt]{geometry}
\usepackage{polyglossia}
\usepackage[12pt]{moresize}
\usepackage{soul}%
\usepackage{microtype}
\usepackage{tocbasic}
\usepackage{realscripts}
\usepackage{epigraph}%
\usepackage{setspace}%
\usepackage{sectsty}
\usepackage{fontspec}
\usepackage{marginnote}
\usepackage[bottom]{footmisc}
\usepackage{enumitem}
\usepackage{fancyhdr}
\usepackage{extramarks}
\usepackage{graphicx}
\usepackage{verse}
\usepackage{relsize}
\usepackage{etoolbox}
\usepackage[a-3u]{pdfx}

\hypersetup{
colorlinks=true,
urlcolor=black,
linkcolor=black,
citecolor=black
}

% use a small amount of tracking on small caps
\SetTracking[ spacing = {25*,166, } ]{ encoding = *, shape = sc }{ 25 }

% add a blank page
\newcommand{\blankpage}{
\newpage
\thispagestyle{empty}
\mbox{}
\newpage
}

% define languages
\setdefaultlanguage[]{english}
\setotherlanguage[script=Latin]{sanskrit}

%\usepackage{pagegrid}
%\pagegridsetup{top-left, step=.25in}

% define fonts
% use if arno sanskrit is unavailable
%\setmainfont{Gentium Plus}
%\newfontfamily\Semiboldsubheadfont[]{Gentium Plus}
%\newfontfamily\Semiboldnormalfont[]{Gentium Plus}
%\newfontfamily\Lightfont[]{Gentium Plus}
%\newfontfamily\Marginalfont[]{Gentium Plus}
%\newfontfamily\Allsmallcapsfont[RawFeature=+c2sc]{Gentium Plus}
%\newfontfamily\Noligaturefont[Renderer=Basic]{Gentium Plus}
%\newfontfamily\Noligaturecaptionfont[Renderer=Basic]{Gentium Plus}
%\newfontfamily\Fleuronfont[Ornament=1]{Gentium Plus}

% use if arno sanskrit is available. display is applied to \chapter and \part, subhead to \section and \subsection. When specifying semibold, the italic must be defined.
\setmainfont[Numbers=OldStyle]{Arno Pro}
\newfontfamily\Semibolddisplayfont[BoldItalicFont = Arno Pro Semibold Italic Display]{Arno Pro Semibold Display} %
\newfontfamily\Semiboldsubheadfont[BoldItalicFont = Arno Pro Semibold Italic Subhead]{Arno Pro Semibold Subhead}
\newfontfamily\Semiboldnormalfont[BoldItalicFont = Arno Pro Semibold Italic]{Arno Pro Semibold}
\newfontfamily\Marginalfont[RawFeature=+subs]{Arno Pro Regular}
\newfontfamily\Allsmallcapsfont[RawFeature=+c2sc]{Arno Pro}
\newfontfamily\Noligaturefont[Renderer=Basic]{Arno Pro}
\newfontfamily\Noligaturecaptionfont[Renderer=Basic]{Arno Pro Caption}

% chinese fonts
\newfontfamily\cjk{Noto Serif TC}
\newcommand*{\langlzh}[1]{\cjk{#1}\normalfont}%

% logo
\newfontfamily\Logofont{sclogo.ttf}
\newcommand*{\sclogo}[1]{\large\Logofont{#1}}

% use subscript numerals for margin notes
\renewcommand*{\marginfont}{\Marginalfont}

% ensure margin notes have consistent vertical alignment
\renewcommand*{\marginnotevadjust}{-.17em}

% use compact lists
\setitemize{noitemsep,leftmargin=1em}
\setenumerate{noitemsep,leftmargin=1em}
\setdescription{noitemsep, style=unboxed, leftmargin=0em}

% style ToC
\DeclareTOCStyleEntries[
  raggedentrytext,
  linefill=\hfill,
  pagenumberwidth=.5in,
  pagenumberformat=\normalfont,
  entryformat=\normalfont
]{tocline}{chapter,section}


  \setlength\topsep{0pt}%
  \setlength\parskip{0pt}%

% define new \centerpars command for use in ToC. This ensures centering, proper wrapping, and no page break after
\def\startcenter{%
  \par
  \begingroup
  \leftskip=0pt plus 1fil
  \rightskip=\leftskip
  \parindent=0pt
  \parfillskip=0pt
}
\def\stopcenter{%
  \par
  \endgroup
}
\long\def\centerpars#1{\startcenter#1\stopcenter}

% redefine part, so that it adds a toc entry without page number
\let\oldcontentsline\contentsline
\newcommand{\nopagecontentsline}[3]{\oldcontentsline{#1}{#2}{}}

    \makeatletter
\renewcommand*\l@part[2]{%
  \ifnum \c@tocdepth >-2\relax
    \addpenalty{-\@highpenalty}%
    \addvspace{0em \@plus\p@}%
    \setlength\@tempdima{3em}%
    \begingroup
      \parindent \z@ \rightskip \@pnumwidth
      \parfillskip -\@pnumwidth
      {\leavevmode
       \setstretch{.85}\large\scshape\centerpars{#1}\vspace*{-1em}\llap{#2}}\par
       \nobreak
         \global\@nobreaktrue
         \everypar{\global\@nobreakfalse\everypar{}}%
    \endgroup
  \fi}
\makeatother

\makeatletter
\def\@pnumwidth{2em}
\makeatother

% define new sectioning command, which is only used in volumes where the pannasa is found in some parts but not others, especially in an and sn

\newcommand*{\pannasa}[1]{\clearpage\thispagestyle{empty}\begin{center}\vspace*{14em}\setstretch{.85}\huge\itshape\scshape\MakeLowercase{#1}\end{center}}

    \makeatletter
\newcommand*\l@pannasa[2]{%
  \ifnum \c@tocdepth >-2\relax
    \addpenalty{-\@highpenalty}%
    \addvspace{.5em \@plus\p@}%
    \setlength\@tempdima{3em}%
    \begingroup
      \parindent \z@ \rightskip \@pnumwidth
      \parfillskip -\@pnumwidth
      {\leavevmode
       \setstretch{.85}\large\itshape\scshape\lowercase{\centerpars{#1}}\vspace*{-1em}\llap{#2}}\par
       \nobreak
         \global\@nobreaktrue
         \everypar{\global\@nobreakfalse\everypar{}}%
    \endgroup
  \fi}
\makeatother

% don't put page number on first page of toc (relies on etoolbox)
\patchcmd{\chapter}{plain}{empty}{}{}

% global line height
\setstretch{1.05}

% allow linebreak after em-dash
\catcode`\—=13
\protected\def—{\unskip\textemdash\allowbreak}

% style headings with secsty. chapter and section are defined per-edition
\partfont{\setstretch{.85}\normalfont\centering\textsc}
\subsectionfont{\setstretch{.85}\Semiboldsubheadfont}%
\subsubsectionfont{\setstretch{.85}\Semiboldnormalfont}

% style elements of suttatitle
\newcommand*{\suttatitleacronym}[1]{\smaller[2]{#1}\vspace*{.3em}}
\newcommand*{\suttatitletranslation}[1]{\linebreak{#1}}
\newcommand*{\suttatitleroot}[1]{\linebreak\smaller[2]\itshape{#1}}

\DeclareTOCStyleEntries[
  indent=3.3em,
  dynindent,
  beforeskip=.2em plus -2pt minus -1pt,
]{tocline}{section}

\DeclareTOCStyleEntries[
  indent=0em,
  dynindent,
  beforeskip=.4em plus -2pt minus -1pt,
]{tocline}{chapter}

\newcommand*{\tocacronym}[1]{\hspace*{-3.3em}{#1}\quad}
\newcommand*{\toctranslation}[1]{#1}
\newcommand*{\tocroot}[1]{(\textit{#1})}
\newcommand*{\tocchapterline}[1]{\bfseries\itshape{#1}}


% redefine paragraph and subparagraph headings to not be inline
\makeatletter
% Change the style of paragraph headings %
\renewcommand\paragraph{\@startsection{paragraph}{4}{\z@}%
            {-2.5ex\@plus -1ex \@minus -.25ex}%
            {1.25ex \@plus .25ex}%
            {\noindent\Semiboldnormalfont\normalsize}}

% Change the style of subparagraph headings %
\renewcommand\subparagraph{\@startsection{subparagraph}{5}{\z@}%
            {-2.5ex\@plus -1ex \@minus -.25ex}%
            {1.25ex \@plus .25ex}%
            {\noindent\Semiboldnormalfont\small}}
\makeatother

% use etoolbox to suppress page numbers on \part
\patchcmd{\part}{\thispagestyle{plain}}{\thispagestyle{empty}}
  {}{\errmessage{Cannot patch \string\part}}

% and to reduce margins on quotation
\patchcmd{\quotation}{\rightmargin}{\leftmargin 1.2em \rightmargin}{}{}
\AtBeginEnvironment{quotation}{\small}

% titlepage
\newcommand*{\titlepageTranslationTitle}[1]{{\begin{center}\begin{large}{#1}\end{large}\end{center}}}
\newcommand*{\titlepageCreatorName}[1]{{\begin{center}\begin{normalsize}{#1}\end{normalsize}\end{center}}}

% halftitlepage
\newcommand*{\halftitlepageTranslationTitle}[1]{\setstretch{2.5}{\begin{Huge}\uppercase{\so{#1}}\end{Huge}}}
\newcommand*{\halftitlepageTranslationSubtitle}[1]{\setstretch{1.2}{\begin{large}{#1}\end{large}}}
\newcommand*{\halftitlepageFleuron}[1]{{\begin{large}\Fleuronfont{{#1}}\end{large}}}
\newcommand*{\halftitlepageByline}[1]{{\begin{normalsize}\textit{{#1}}\end{normalsize}}}
\newcommand*{\halftitlepageCreatorName}[1]{{\begin{LARGE}{\textsc{#1}}\end{LARGE}}}
\newcommand*{\halftitlepageVolumeNumber}[1]{{\begin{normalsize}{\Allsmallcapsfont{\textsc{#1}}}\end{normalsize}}}
\newcommand*{\halftitlepageVolumeAcronym}[1]{{\begin{normalsize}{#1}\end{normalsize}}}
\newcommand*{\halftitlepageVolumeTranslationTitle}[1]{{\begin{Large}{\textsc{#1}}\end{Large}}}
\newcommand*{\halftitlepageVolumeRootTitle}[1]{{\begin{normalsize}{\Allsmallcapsfont{\textsc{\itshape #1}}}\end{normalsize}}}
\newcommand*{\halftitlepagePublisher}[1]{{\begin{large}{\Noligaturecaptionfont\textsc{#1}}\end{large}}}

% epigraph
\renewcommand{\epigraphflush}{center}
\renewcommand*{\epigraphwidth}{.85\textwidth}
\newcommand*{\epigraphTranslatedTitle}[1]{\vspace*{.5em}\footnotesize\textsc{#1}\\}%
\newcommand*{\epigraphRootTitle}[1]{\footnotesize\textit{#1}\\}%
\newcommand*{\epigraphReference}[1]{\footnotesize{#1}}%

% custom commands for html styling classes
\newcommand*{\scnamo}[1]{\begin{center}\textit{#1}\end{center}}
\newcommand*{\scendsection}[1]{\begin{center}\textit{#1}\end{center}}
\newcommand*{\scendsutta}[1]{\begin{center}\textit{#1}\end{center}}
\newcommand*{\scendbook}[1]{\begin{center}\uppercase{#1}\end{center}}
\newcommand*{\scendkanda}[1]{\begin{center}\textbf{#1}\end{center}}
\newcommand*{\scend}[1]{\begin{center}\textit{#1}\end{center}}
\newcommand*{\scuddanaintro}[1]{\textit{#1}}
\newcommand*{\scendvagga}[1]{\begin{center}\textbf{#1}\end{center}}
\newcommand*{\scrule}[1]{\textbf{#1}}
\newcommand*{\scadd}[1]{\textit{#1}}
\newcommand*{\scevam}[1]{\textsc{#1}}
\newcommand*{\scspeaker}[1]{\hspace{2em}\textit{#1}}
\newcommand*{\scbyline}[1]{\begin{flushright}\textit{#1}\end{flushright}\bigskip}

% custom command for thematic break = hr
\newcommand*{\thematicbreak}{\begin{center}\rule[.5ex]{6em}{.4pt}\begin{normalsize}\quad\Fleuronfont{•}\quad\end{normalsize}\rule[.5ex]{6em}{.4pt}\end{center}}

% manage and style page header and footer. "fancy" has header and footer, "plain" has footer only

\pagestyle{fancy}
\fancyhf{}
\fancyfoot[RE,LO]{\thepage}
\fancyfoot[LE,RO]{\footnotesize\lastleftxmark}
\fancyhead[CE]{\setstretch{.85}\Noligaturefont\MakeLowercase{\textsc{\firstrightmark}}}
\fancyhead[CO]{\setstretch{.85}\Noligaturefont\MakeLowercase{\textsc{\firstleftmark}}}
\renewcommand{\headrulewidth}{0pt}
\fancypagestyle{plain}{ %
\fancyhf{} % remove everything
\fancyfoot[RE,LO]{\thepage}
\fancyfoot[LE,RO]{\footnotesize\lastleftxmark}
\renewcommand{\headrulewidth}{0pt}
\renewcommand{\footrulewidth}{0pt}}

% style footnotes
\setlength{\skip\footins}{1em}

\makeatletter
\newcommand{\@makefntextcustom}[1]{%
    \parindent 0em%
    \thefootnote.\enskip #1%
}
\renewcommand{\@makefntext}[1]{\@makefntextcustom{#1}}
\makeatother

% hang quotes (requires microtype)
\microtypesetup{
  protrusion = true,
  expansion  = true,
  tracking   = true,
  factor     = 1000,
  patch      = all,
  final
}

% Custom protrusion rules to allow hanging punctuation
\SetProtrusion
{ encoding = *}
{
% char   right left
  {-} = {    , 500 },
  % Double Quotes
  \textquotedblleft
      = {1000,     },
  \textquotedblright
      = {    , 1000},
  \quotedblbase
      = {1000,     },
  % Single Quotes
  \textquoteleft
      = {1000,     },
  \textquoteright
      = {    , 1000},
  \quotesinglbase
      = {1000,     }
}

% make latex use actual font em for parindent, not Computer Modern Roman
\AtBeginDocument{\setlength{\parindent}{1em}}%
%

% Default values; a bit sloppier than normal
\tolerance 1414
\hbadness 1414
\emergencystretch 1.5em
\hfuzz 0.3pt
\clubpenalty = 10000
\widowpenalty = 10000
\displaywidowpenalty = 10000
\hfuzz \vfuzz
 \raggedbottom%

\title{Numbered Discourses}
\author{Bhikkhu Sujato}
\date{}%
% define a different fleuron for each edition
\newfontfamily\Fleuronfont[Ornament=18]{Arno Pro}

% Define heading styles per edition for chapter and section. Suttatitle can be either of these, depending on the volume. 

\let\oldfrontmatter\frontmatter
\renewcommand{\frontmatter}{%
\chapterfont{\setstretch{.85}\normalfont\centering}%
\sectionfont{\setstretch{.85}\Semiboldsubheadfont}%
\oldfrontmatter}

\let\oldmainmatter\mainmatter
\renewcommand{\mainmatter}{%
\chapterfont{\setstretch{.85}\normalfont\centering}%
\sectionfont{\setstretch{.85}\normalfont\centering}%
\oldmainmatter}

\let\oldbackmatter\backmatter
\renewcommand{\backmatter}{%
\chapterfont{\setstretch{.85}\normalfont\centering}%
\sectionfont{\setstretch{.85}\Semiboldsubheadfont}%
\oldbackmatter}
%
%
\begin{document}%
\normalsize%
\frontmatter%
\setlength{\parindent}{0cm}

\pagestyle{empty}

\maketitle

\blankpage%
\begin{center}

\vspace*{2.2em}

\halftitlepageTranslationTitle{Numbered Discourses}

\vspace*{1em}

\halftitlepageTranslationSubtitle{A sensible translation of the Aṅguttara Nikāya}

\vspace*{2em}

\halftitlepageFleuron{•}

\vspace*{2em}

\halftitlepageByline{translated and introduced by}

\vspace*{.5em}

\halftitlepageCreatorName{Bhikkhu Sujato}

\vspace*{4em}

\halftitlepageVolumeNumber{Volume 4}

\smallskip

\halftitlepageVolumeAcronym{AN 7–9}

\smallskip

\halftitlepageVolumeTranslationTitle{}

\smallskip

\halftitlepageVolumeRootTitle{}

\vspace*{\fill}

\sclogo{0}
 \halftitlepagePublisher{SuttaCentral}

\end{center}

\newpage
%
\setstretch{1.05}

\begin{footnotesize}

\textit{Numbered Discourses} is a translation of the Aṅguttaranikāya by Bhikkhu Sujato.

\medskip

Creative Commons Zero (CC0)

To the extent possible under law, Bhikkhu Sujato has waived all copyright and related or neighboring rights to \textit{Numbered Discourses}.

\medskip

This work is published from Australia.

\begin{center}
\textit{This translation is an expression of an ancient spiritual text that has been passed down by the Buddhist tradition for the benefit of all sentient beings. It is dedicated to the public domain via Creative Commons Zero (CC0). You are encouraged to copy, reproduce, adapt, alter, or otherwise make use of this translation. The translator respectfully requests that any use be in accordance with the values and principles of the Buddhist community.}
\end{center}

\medskip

\begin{description}
    \item[Web publication date] 2018
    \item[This edition] 2022-11-30 08:48:21
    \item[Publication type] paperback
    \item[Edition] ed5
    \item[Number of volumes] 5
    \item[Publication ISBN] 978-1-76132-037-8
    \item[Publication URL] https://suttacentral.net/editions/an/en/sujato
    \item[Source URL] https://github.com/suttacentral/bilara-data/tree/published/translation/en/sujato/sutta/an
    \item[Publication number] scpub5
\end{description}

\medskip

Published by SuttaCentral

\medskip

\textit{SuttaCentral,\\
c/o Alwis \& Alwis Pty Ltd\\
Kaurna Country,\\
Suite 12,\\
198 Greenhill Road,\\
Eastwood,\\
SA 5063,\\
Australia}

\end{footnotesize}

\newpage

\setlength{\parindent}{1.5em}%%
\tableofcontents
\newpage
\pagestyle{fancy}
%
\mainmatter%
\pagestyle{fancy}%
\addtocontents{toc}{\let\protect\contentsline\protect\nopagecontentsline}
\part*{The Book of the Sevens }
\addcontentsline{toc}{part}{The Book of the Sevens }
\markboth{}{}
\addtocontents{toc}{\let\protect\contentsline\protect\oldcontentsline}

%
%
\addtocontents{toc}{\let\protect\contentsline\protect\nopagecontentsline}
\pannasa{The First Fifty }
\addcontentsline{toc}{pannasa}{The First Fifty }
\markboth{}{}
\addtocontents{toc}{\let\protect\contentsline\protect\oldcontentsline}

%
\addtocontents{toc}{\let\protect\contentsline\protect\nopagecontentsline}
\chapter*{The Chapter on Wealth }
\addcontentsline{toc}{chapter}{\tocchapterline{The Chapter on Wealth }}
\addtocontents{toc}{\let\protect\contentsline\protect\oldcontentsline}

%
\section*{{\suttatitleacronym AN 7.1}{\suttatitletranslation Pleasing (1st) }{\suttatitleroot Paṭhamapiyasutta}}
\addcontentsline{toc}{section}{\tocacronym{AN 7.1} \toctranslation{Pleasing (1st) } \tocroot{Paṭhamapiyasutta}}
\markboth{Pleasing (1st) }{Paṭhamapiyasutta}
\extramarks{AN 7.1}{AN 7.1}

\scevam{So\marginnote{1.1} I have heard. }At one time the Buddha was staying near \textsanskrit{Sāvatthī} in Jeta’s Grove, \textsanskrit{Anāthapiṇḍika}’s monastery. There the Buddha addressed the mendicants, “Mendicants!” 

“Venerable\marginnote{1.5} sir,” they replied. The Buddha said this: 

“Mendicants,\marginnote{2.1} a mendicant with seven qualities is disliked and disapproved by their spiritual companions, not respected or admired. What seven? It’s when a mendicant desires material possessions, honor, and to be looked up to. They lack conscience and prudence. They have wicked desires and wrong view. A mendicant with these seven qualities is disliked and disapproved by their spiritual companions, not respected or admired. 

A\marginnote{3.1} mendicant with seven qualities is liked and approved by their spiritual companions, respected and admired. What seven? It’s when a mendicant doesn’t desire material possessions, honor, and to be looked up to. They have conscience and prudence. They have few desires and right view. A mendicant with these seven qualities is liked and approved by their spiritual companions, respected and admired.” 

%
\section*{{\suttatitleacronym AN 7.2}{\suttatitletranslation Pleasing (2nd) }{\suttatitleroot Dutiyapiyasutta}}
\addcontentsline{toc}{section}{\tocacronym{AN 7.2} \toctranslation{Pleasing (2nd) } \tocroot{Dutiyapiyasutta}}
\markboth{Pleasing (2nd) }{Dutiyapiyasutta}
\extramarks{AN 7.2}{AN 7.2}

“Mendicants,\marginnote{1.1} a mendicant with seven qualities is disliked and disapproved by their spiritual companions, not respected or admired. What seven? It’s when a mendicant desires material possessions, honor, and to be looked up to. They lack conscience and prudence. They’re jealous and stingy. A mendicant with these seven qualities is disliked and disapproved by their spiritual companions, not respected or admired. 

A\marginnote{2.1} mendicant with seven qualities is liked and approved by their spiritual companions, respected and admired. What seven? It’s when a mendicant doesn’t desire material possessions, honor, and to be looked up to. They have conscience and prudence. They’re not jealous or stingy. A mendicant with these seven qualities is liked and approved by their spiritual companions, respected and admired.” 

%
\section*{{\suttatitleacronym AN 7.3}{\suttatitletranslation Powers in Brief }{\suttatitleroot Saṁkhittabalasutta}}
\addcontentsline{toc}{section}{\tocacronym{AN 7.3} \toctranslation{Powers in Brief } \tocroot{Saṁkhittabalasutta}}
\markboth{Powers in Brief }{Saṁkhittabalasutta}
\extramarks{AN 7.3}{AN 7.3}

\scevam{So\marginnote{1.1} I have heard. }At one time the Buddha was staying near \textsanskrit{Sāvatthī} in Jeta’s Grove, \textsanskrit{Anāthapiṇḍika}’s monastery. … “Mendicants, there are these seven powers. What seven? The powers of faith, energy, conscience, prudence, mindfulness, immersion, and wisdom. These are the seven powers. 

\begin{verse}%
The\marginnote{2.1} powers are faith and energy, \\
conscience and prudence, \\
mindfulness and immersion, \\
and wisdom as the seventh power. \\
Empowered by these, \\
an astute mendicant lives happily. 

They\marginnote{3.1} should examine the teaching rationally, \\
discerning the meaning with wisdom. \\
The liberation of their heart \\
is like a lamp going out.” 

%
\end{verse}

%
\section*{{\suttatitleacronym AN 7.4}{\suttatitletranslation Powers in Detail }{\suttatitleroot Vitthatabalasutta}}
\addcontentsline{toc}{section}{\tocacronym{AN 7.4} \toctranslation{Powers in Detail } \tocroot{Vitthatabalasutta}}
\markboth{Powers in Detail }{Vitthatabalasutta}
\extramarks{AN 7.4}{AN 7.4}

“Mendicants,\marginnote{1.1} there are these seven powers. What seven? The powers of faith, energy, conscience, prudence, mindfulness, immersion, and wisdom. 

And\marginnote{2.1} what is the power of faith? It’s when a noble disciple has faith in the Realized One’s awakening: ‘That Blessed One is perfected, a fully awakened Buddha, accomplished in knowledge and conduct, holy, knower of the world, supreme guide for those who wish to train, teacher of gods and humans, awakened, blessed.’ This is called the power of faith. 

And\marginnote{3.1} what is the power of energy? It’s when a mendicant lives with energy roused up for giving up unskillful qualities and embracing skillful qualities. They’re strong, staunchly vigorous, not slacking off when it comes to developing skillful qualities. This is called the power of energy. 

And\marginnote{4.1} what is the power of conscience? It’s when a noble disciple has a conscience. They’re conscientious about bad conduct by way of body, speech, and mind, and conscientious about acquiring any bad, unskillful qualities. This is called the power of conscience. 

And\marginnote{5.1} what is the power of prudence? It’s when a noble disciple is prudent. They’re prudent when it comes to bad conduct by way of body, speech, and mind, and prudent when it comes to the acquiring of any bad, unskillful qualities. This is called the power of prudence. 

And\marginnote{6.1} what is the power of mindfulness? It’s when a noble disciple is mindful. They have utmost mindfulness and alertness, and can remember and recall what was said and done long ago. This is called the power of mindfulness. 

And\marginnote{7.1} what is the power of immersion? It’s when a mendicant, quite secluded from sensual pleasures, secluded from unskillful qualities, enters and remains in the first absorption, which has the rapture and bliss born of seclusion, while placing the mind and keeping it connected. … Giving up pleasure and pain, and ending former happiness and sadness, they enter and remain in the fourth absorption, without pleasure or pain, with pure equanimity and mindfulness. This is called the power of immersion. 

And\marginnote{8.1} what is the power of wisdom? It’s when a noble disciple is wise. They have the wisdom of arising and passing away which is noble, penetrative, and leads to the complete ending of suffering. This is called the power of wisdom. 

These\marginnote{9.1} are the seven powers. 

\begin{verse}%
The\marginnote{10.1} powers are faith and energy, \\
conscience and prudence, \\
mindfulness and immersion, \\
and wisdom as the seventh power. \\
Empowered by these, \\
an astute mendicant lives happily. 

They\marginnote{11.1} should examine the teaching rationally, \\
discerning the meaning with wisdom. \\
The liberation of their heart \\
is like a lamp going out.” 

%
\end{verse}

%
\section*{{\suttatitleacronym AN 7.5}{\suttatitletranslation Wealth in Brief }{\suttatitleroot Saṁkhittadhanasutta}}
\addcontentsline{toc}{section}{\tocacronym{AN 7.5} \toctranslation{Wealth in Brief } \tocroot{Saṁkhittadhanasutta}}
\markboth{Wealth in Brief }{Saṁkhittadhanasutta}
\extramarks{AN 7.5}{AN 7.5}

“Mendicants,\marginnote{1.1} there are these seven kinds of wealth. What seven? The wealth of faith, ethics, conscience, prudence, learning, generosity, and wisdom. These are the seven kinds of wealth. 

\begin{verse}%
Faith\marginnote{2.1} and ethical conduct are kinds of wealth, \\
as are conscience and prudence, \\
learning and generosity, \\
and wisdom is the seventh kind of wealth. 

When\marginnote{3.1} a woman or man \\
has these kinds of wealth, \\
they’re said to be prosperous, \\
their life is not in vain. 

So\marginnote{4.1} let the wise devote themselves \\
to faith, ethical behavior, \\
confidence, and insight into the teaching, \\
remembering the instructions of the Buddhas.” 

%
\end{verse}

%
\section*{{\suttatitleacronym AN 7.6}{\suttatitletranslation Wealth in Detail }{\suttatitleroot Vitthatadhanasutta}}
\addcontentsline{toc}{section}{\tocacronym{AN 7.6} \toctranslation{Wealth in Detail } \tocroot{Vitthatadhanasutta}}
\markboth{Wealth in Detail }{Vitthatadhanasutta}
\extramarks{AN 7.6}{AN 7.6}

“Mendicants,\marginnote{1.1} there are these seven kinds of wealth. What seven? The wealth of faith, ethics, conscience, prudence, learning, generosity, and wisdom. 

And\marginnote{2.1} what is the wealth of faith? It’s when a noble disciple has faith in the Realized One’s awakening … This is called the wealth of faith. 

And\marginnote{3.1} what is the wealth of ethical conduct? It’s when a noble disciple doesn’t kill living creatures, steal, commit sexual misconduct, use speech that’s false, divisive, harsh, or nonsensical, or consume alcoholic drinks that cause negligence. This is called the wealth of ethical conduct. 

And\marginnote{4.1} what is the wealth of conscience? It’s when a noble disciple has a conscience. They’re conscientious about bad conduct by way of body, speech, and mind, and conscientious about having any bad, unskillful qualities. This is called the wealth of conscience. 

And\marginnote{5.1} what is the wealth of prudence? It’s when a noble disciple is prudent. They’re prudent when it comes to bad conduct by way of body, speech, and mind, and prudent when it comes to the acquiring of any bad, unskillful qualities. This is called the wealth of prudence. 

And\marginnote{6.1} what is the wealth of learning? It’s when a noble disciple is very learned, remembering and keeping what they’ve learned. These teachings are good in the beginning, good in the middle, and good in the end, meaningful and well-phrased, describing a spiritual practice that’s entirely full and pure. They are very learned in such teachings, remembering them, reciting them, mentally scrutinizing them, and comprehending them theoretically. This is called the wealth of learning. 

And\marginnote{7.1} what is the wealth of generosity? It’s when a noble disciple lives at home rid of the stain of stinginess, freely generous, open-handed, loving to let go, committed to charity, loving to give and to share. This is called the wealth of generosity. 

And\marginnote{8.1} what is the wealth of wisdom? It’s when a noble disciple is wise. They have the wisdom of arising and passing away which is noble, penetrative, and leads to the complete ending of suffering. This is called the wealth of wisdom. 

These\marginnote{9.1} are the seven kinds of wealth. 

\begin{verse}%
Faith\marginnote{10.1} and ethical conduct are kinds of wealth, \\
as are conscience and prudence, \\
learning and generosity, \\
and wisdom is the seventh kind of wealth. 

When\marginnote{11.1} a woman or man \\
has these kinds of wealth, \\
they’re said to be prosperous, \\
their life is not in vain. 

So\marginnote{12.1} let the wise devote themselves \\
to faith, ethical behavior, \\
confidence, and insight into the teaching, \\
remembering the instructions of the Buddhas.” 

%
\end{verse}

%
\section*{{\suttatitleacronym AN 7.7}{\suttatitletranslation With Ugga }{\suttatitleroot Uggasutta}}
\addcontentsline{toc}{section}{\tocacronym{AN 7.7} \toctranslation{With Ugga } \tocroot{Uggasutta}}
\markboth{With Ugga }{Uggasutta}
\extramarks{AN 7.7}{AN 7.7}

Then\marginnote{1.1} Ugga the government minister went up to the Buddha, bowed, sat down to one side, and said to him, “It’s incredible, sir, it’s amazing! \textsanskrit{Migāra} of \textsanskrit{Rohaṇa} is so rich, so very wealthy.” 

“But\marginnote{2.3} Ugga, how rich is he?” 

“He\marginnote{2.4} has a hundred thousand gold coins, not to mention the silver!” 

“Well,\marginnote{2.5} Ugga, that is wealth, I can’t deny it. But fire, water, rulers, thieves, and unloved heirs all take a share of that wealth. There are these seven kinds of wealth that they can’t take a share of. What seven? The wealth of faith, ethics, conscience, prudence, learning, generosity, and wisdom. There are these seven kinds of wealth that fire, water, rulers, thieves, and unloved heirs can’t take a share of. 

\begin{verse}%
Faith\marginnote{3.1} and ethical conduct are kinds of wealth, \\
as are conscience and prudence, \\
learning and generosity, \\
and wisdom is the seventh kind of wealth. 

When\marginnote{4.1} a woman or man \\
has these kinds of wealth, \\
they’re really rich in the world, \\
invincible among gods and humans. 

So\marginnote{5.1} let the wise devote themselves \\
to faith, ethical behavior, \\
confidence, and insight into the teaching, \\
remembering the instructions of the Buddhas.” 

%
\end{verse}

%
\section*{{\suttatitleacronym AN 7.8}{\suttatitletranslation Fetters }{\suttatitleroot Saṁyojanasutta}}
\addcontentsline{toc}{section}{\tocacronym{AN 7.8} \toctranslation{Fetters } \tocroot{Saṁyojanasutta}}
\markboth{Fetters }{Saṁyojanasutta}
\extramarks{AN 7.8}{AN 7.8}

“Mendicants,\marginnote{1.1} there are these seven fetters. What seven? The fetters of attraction, repulsion, views, doubt, conceit, desire to be reborn, and ignorance. These are the seven fetters.” 

%
\section*{{\suttatitleacronym AN 7.9}{\suttatitletranslation Giving Up }{\suttatitleroot Pahānasutta}}
\addcontentsline{toc}{section}{\tocacronym{AN 7.9} \toctranslation{Giving Up } \tocroot{Pahānasutta}}
\markboth{Giving Up }{Pahānasutta}
\extramarks{AN 7.9}{AN 7.9}

“Mendicants,\marginnote{1.1} the spiritual life is lived to give up and cut out these seven fetters. What seven? The fetters of attraction, repulsion, views, doubt, conceit, desire to be reborn, and ignorance. The spiritual life is lived to give up and cut out these seven fetters. When a mendicant has given up the fetters of attraction, repulsion, views, doubt, conceit, desire to be reborn, and ignorance—cut them off at the root, made them like a palm stump, obliterated them, so they are unable to arise in the future—they’re called a mendicant who has cut off craving, untied the fetters, and by rightly comprehending conceit has made an end of suffering.” 

%
\section*{{\suttatitleacronym AN 7.10}{\suttatitletranslation Stinginess }{\suttatitleroot Macchariyasutta}}
\addcontentsline{toc}{section}{\tocacronym{AN 7.10} \toctranslation{Stinginess } \tocroot{Macchariyasutta}}
\markboth{Stinginess }{Macchariyasutta}
\extramarks{AN 7.10}{AN 7.10}

“Mendicants,\marginnote{1.1} there are these seven fetters. What seven? The fetters of attraction, repulsion, views, doubt, conceit, jealousy, and stinginess. These are the seven fetters.” 

%
\addtocontents{toc}{\let\protect\contentsline\protect\nopagecontentsline}
\chapter*{The Chapter on Tendencies }
\addcontentsline{toc}{chapter}{\tocchapterline{The Chapter on Tendencies }}
\addtocontents{toc}{\let\protect\contentsline\protect\oldcontentsline}

%
\section*{{\suttatitleacronym AN 7.11}{\suttatitletranslation Underlying Tendencies (1st) }{\suttatitleroot Paṭhamaanusayasutta}}
\addcontentsline{toc}{section}{\tocacronym{AN 7.11} \toctranslation{Underlying Tendencies (1st) } \tocroot{Paṭhamaanusayasutta}}
\markboth{Underlying Tendencies (1st) }{Paṭhamaanusayasutta}
\extramarks{AN 7.11}{AN 7.11}

“Mendicants,\marginnote{1.1} there are these seven underlying tendencies. What seven? The underlying tendencies of sensual desire, repulsion, views, doubt, conceit, desire to be reborn, and ignorance. These are the seven underlying tendencies.” 

%
\section*{{\suttatitleacronym AN 7.12}{\suttatitletranslation Underlying Tendencies (2nd) }{\suttatitleroot Dutiyaanusayasutta}}
\addcontentsline{toc}{section}{\tocacronym{AN 7.12} \toctranslation{Underlying Tendencies (2nd) } \tocroot{Dutiyaanusayasutta}}
\markboth{Underlying Tendencies (2nd) }{Dutiyaanusayasutta}
\extramarks{AN 7.12}{AN 7.12}

“Mendicants,\marginnote{1.1} the spiritual life is lived to give up and cut out these seven underlying tendencies. What seven? The underlying tendencies of sensual desire, repulsion, views, doubt, conceit, desire to be reborn, and ignorance. The spiritual life is lived to give up and cut out these seven underlying tendencies. 

When\marginnote{2.1} a mendicant has given up the underlying tendencies of sensual desire, repulsion, views, doubt, conceit, desire to be reborn, and ignorance—cut them off at the root, made them like a palm stump, obliterated them, so they are unable to arise in the future—they’re called a mendicant who has cut off craving, untied the fetters, and by rightly comprehending conceit has made an end of suffering.” 

%
\section*{{\suttatitleacronym AN 7.13}{\suttatitletranslation A Family }{\suttatitleroot Kulasutta}}
\addcontentsline{toc}{section}{\tocacronym{AN 7.13} \toctranslation{A Family } \tocroot{Kulasutta}}
\markboth{A Family }{Kulasutta}
\extramarks{AN 7.13}{AN 7.13}

“Mendicants,\marginnote{1.1} visiting a family with seven factors is not worthwhile, or if you’ve already arrived, sitting down is not worthwhile. What seven? They don’t politely rise, bow, or offer a seat. They hide what they have. Even when they have much they give little. Even when they have refined things they give coarse things. They give carelessly, not carefully. Visiting a family with these seven factors is not worthwhile, or if you’ve already arrived, sitting down is not worthwhile. 

Visiting\marginnote{2.1} a family with seven factors is worthwhile, or if you’ve already arrived, sitting down is worthwhile. What seven? They politely rise, bow, and offer a seat. They don’t hide what they have. When they have much they give much. When they have refined things they give refined things. They give carefully, not carelessly. Visiting a family with these seven factors is worthwhile, or if you’ve already arrived, sitting down is worthwhile.” 

%
\section*{{\suttatitleacronym AN 7.14}{\suttatitletranslation Persons }{\suttatitleroot Puggalasutta}}
\addcontentsline{toc}{section}{\tocacronym{AN 7.14} \toctranslation{Persons } \tocroot{Puggalasutta}}
\markboth{Persons }{Puggalasutta}
\extramarks{AN 7.14}{AN 7.14}

“Mendicants,\marginnote{1.1} these seven people are worthy of offerings dedicated to the gods, worthy of hospitality, worthy of a religious donation, worthy of greeting with joined palms, and are the supreme field of merit for the world. What seven? The one freed both ways, the one freed by wisdom, the personal witness, the one attained to view, the one freed by faith, the follower of the teachings, and the follower by faith. These are the seven people who are worthy of offerings dedicated to the gods, worthy of hospitality, worthy of a religious donation, worthy of greeting with joined palms, and are the supreme field of merit for the world.” 

%
\section*{{\suttatitleacronym AN 7.15}{\suttatitletranslation A Simile With Water }{\suttatitleroot Udakūpamāsutta}}
\addcontentsline{toc}{section}{\tocacronym{AN 7.15} \toctranslation{A Simile With Water } \tocroot{Udakūpamāsutta}}
\markboth{A Simile With Water }{Udakūpamāsutta}
\extramarks{AN 7.15}{AN 7.15}

“Mendicants,\marginnote{1.1} these seven people found in the world are like those in water. 

What\marginnote{1.2} seven? One person sinks under once and stays under. One person rises up then sinks under. One person rises up then stays put. One person rises up then sees and discerns. One person rises up then crosses over. One person rises up then finds a footing. One person has risen up, crossed over, and gone beyond, and that brahmin stands on the shore. 

And\marginnote{2.1} what kind of person sinks under once and stays under? It’s the kind of person who has exclusively dark, unskillful qualities. This kind of person sinks under once and stays under. 

And\marginnote{3.1} what kind of person rises up then sinks under? It’s the kind of person who, rising up, thinks: ‘It’s good to have faith, conscience, prudence, energy, and wisdom regarding skillful qualities.’ However their faith, conscience, prudence, energy, and wisdom don’t last or grow, but dwindle away. This kind of person rises up then sinks under. 

And\marginnote{4.1} what kind of person rises up then stays put? It’s the kind of person who, rising up, thinks: ‘It’s good to have faith, conscience, prudence, energy, and wisdom regarding skillful qualities.’ And their faith, conscience, prudence, energy, and wisdom lasts, neither dwindling nor growing. This kind of person rises up then stays put. 

And\marginnote{5.1} what kind of person rises up then sees and discerns? It’s the kind of person who, rising up, thinks: ‘It’s good to have faith, conscience, prudence, energy, and wisdom regarding skillful qualities.’ With the ending of three fetters they’re a stream-enterer, not liable to be reborn in the underworld, bound for awakening. This kind of person rises up then sees and discerns. 

And\marginnote{6.1} what kind of person rises up then crosses over? It’s the kind of person who, rising up, thinks: ‘It’s good to have faith, conscience, prudence, energy, and wisdom regarding skillful qualities.’ With the ending of three fetters, and the weakening of greed, hate, and delusion, they’re a once-returner. They come back to this world once only, then make an end of suffering. This kind of person rises up then crosses over. 

And\marginnote{7.1} what kind of person rises up then finds a footing? It’s the kind of person who, rising up, thinks: ‘It’s good to have faith, conscience, prudence, energy, and wisdom regarding skillful qualities.’ With the ending of the five lower fetters they’re reborn spontaneously. They are extinguished there, and are not liable to return from that world. This kind of person rises up then finds a footing. 

And\marginnote{8.1} what kind of person has risen up, crossed over, and gone beyond, a brahmin who stands on the shore? It’s the kind of person who, rising up, thinks: ‘It’s good to have faith, conscience, prudence, energy, and wisdom regarding skillful qualities.’ They realize the undefiled freedom of heart and freedom by wisdom in this very life. And they live having realized it with their own insight due to the ending of defilements. This kind of person has risen up, crossed over, and gone beyond, a brahmin who stands on the shore. 

These\marginnote{9.1} seven people found in the world are like those in water.” 

%
\section*{{\suttatitleacronym AN 7.16}{\suttatitletranslation Observing Impermanence }{\suttatitleroot Aniccānupassīsutta}}
\addcontentsline{toc}{section}{\tocacronym{AN 7.16} \toctranslation{Observing Impermanence } \tocroot{Aniccānupassīsutta}}
\markboth{Observing Impermanence }{Aniccānupassīsutta}
\extramarks{AN 7.16}{AN 7.16}

“Mendicants,\marginnote{1.1} these seven people are worthy of offerings dedicated to the gods, worthy of hospitality, worthy of a religious donation, worthy of greeting with joined palms, and are the supreme field of merit for the world. What seven? 

First,\marginnote{1.3} take a person who meditates observing impermanence in all conditions. They perceive impermanence and experience impermanence. Constantly, continually, and without interruption, they apply the mind and fathom with wisdom. They’ve realized the undefiled freedom of heart and freedom by wisdom in this very life, and live having realized it with their own insight due to the ending of defilements. This is the first person. 

Next,\marginnote{2.1} take a person who meditates observing impermanence in all conditions. Their defilements and their life come to an end at exactly the same time. This is the second person. 

Next,\marginnote{3.1} take a person who meditates observing impermanence in all conditions. With the ending of the five lower fetters they’re extinguished between one life and the next. … 

With\marginnote{3.3} the ending of the five lower fetters they’re extinguished upon landing. … 

With\marginnote{3.4} the ending of the five lower fetters they’re extinguished without extra effort. … 

With\marginnote{3.5} the ending of the five lower fetters they’re extinguished with extra effort. … 

With\marginnote{3.6} the ending of the five lower fetters they head upstream, going to the \textsanskrit{Akaniṭṭha} realm. This is the seventh person. 

These\marginnote{3.8} are the seven people who are worthy of offerings dedicated to the gods, worthy of hospitality, worthy of a religious donation, worthy of greeting with joined palms, and are the supreme field of merit for the world.” 

%
\section*{{\suttatitleacronym AN 7.17}{\suttatitletranslation Observing Suffering }{\suttatitleroot Dukkhānupassīsutta}}
\addcontentsline{toc}{section}{\tocacronym{AN 7.17} \toctranslation{Observing Suffering } \tocroot{Dukkhānupassīsutta}}
\markboth{Observing Suffering }{Dukkhānupassīsutta}
\extramarks{AN 7.17}{AN 7.17}

“Mendicants,\marginnote{1.1} these seven people are worthy of offerings dedicated to the gods, worthy of hospitality, worthy of a religious donation, worthy of greeting with joined palms, and are the supreme field of merit for the world. What seven? First, take a person who meditates observing suffering in all conditions. They perceive suffering and experience suffering. Constantly, continually, and without interruption, they apply the mind and fathom with wisdom. …” 

%
\section*{{\suttatitleacronym AN 7.18}{\suttatitletranslation Observing Not-self }{\suttatitleroot Anattānupassīsutta}}
\addcontentsline{toc}{section}{\tocacronym{AN 7.18} \toctranslation{Observing Not-self } \tocroot{Anattānupassīsutta}}
\markboth{Observing Not-self }{Anattānupassīsutta}
\extramarks{AN 7.18}{AN 7.18}

“First,\marginnote{1.1} take a person who meditates observing not-self in all things. They perceive not-self and experience not-self. Constantly, continually, and without interruption, they apply the mind and fathom with wisdom. …” 

%
\section*{{\suttatitleacronym AN 7.19}{\suttatitletranslation Extinguishment }{\suttatitleroot Nibbānasutta}}
\addcontentsline{toc}{section}{\tocacronym{AN 7.19} \toctranslation{Extinguishment } \tocroot{Nibbānasutta}}
\markboth{Extinguishment }{Nibbānasutta}
\extramarks{AN 7.19}{AN 7.19}

“First,\marginnote{1.1} take a person who meditates observing the happiness in extinguishment. They perceive happiness and experience happiness. Constantly, continually, and without interruption, they apply the mind and fathom with wisdom. They’ve realized the undefiled freedom of heart and freedom by wisdom in this very life, and live having realized it with their own insight due to the ending of defilements. This is the first person worthy of offerings. 

Next,\marginnote{2.1} take a person who meditates observing the happiness in extinguishment. They perceive happiness and experience happiness. Constantly, continually, and without interruption, they apply the mind and fathom with wisdom. Their defilements and their life come to an end at exactly the same time. This is the second person. 

Next,\marginnote{3.1} take a person who meditates observing the happiness in extinguishment. They perceive happiness and experience happiness. Constantly, continually, and without interruption, they apply the mind and fathom with wisdom. 

With\marginnote{3.2} the ending of the five lower fetters they’re extinguished between one life and the next. … 

With\marginnote{3.3} the ending of the five lower fetters they’re extinguished upon landing. … 

With\marginnote{3.4} the ending of the five lower fetters they’re extinguished without extra effort. … 

With\marginnote{3.5} the ending of the five lower fetters they’re extinguished with extra effort. … 

With\marginnote{3.6} the ending of the five lower fetters they head upstream, going to the \textsanskrit{Akaniṭṭha} realm. This is the seventh person. 

These\marginnote{3.8} are the seven people who are worthy of offerings dedicated to the gods, worthy of hospitality, worthy of a religious donation, worthy of greeting with joined palms, and are the supreme field of merit for the world.” 

%
\section*{{\suttatitleacronym AN 7.20}{\suttatitletranslation Qualifications for Graduation }{\suttatitleroot Niddasavatthusutta}}
\addcontentsline{toc}{section}{\tocacronym{AN 7.20} \toctranslation{Qualifications for Graduation } \tocroot{Niddasavatthusutta}}
\markboth{Qualifications for Graduation }{Niddasavatthusutta}
\extramarks{AN 7.20}{AN 7.20}

“Mendicants,\marginnote{1.1} there are these seven qualifications for graduation. What seven? It’s when a mendicant has a keen enthusiasm to undertake the training … to examine the teachings … to get rid of desires … for retreat … to rouse up energy … for mindfulness and alertness … to penetrate theoretically. And they don’t lose these desires in the future. These are the seven qualifications for graduation.” 

%
\addtocontents{toc}{\let\protect\contentsline\protect\nopagecontentsline}
\chapter*{The Chapter on the Vajji Seven }
\addcontentsline{toc}{chapter}{\tocchapterline{The Chapter on the Vajji Seven }}
\addtocontents{toc}{\let\protect\contentsline\protect\oldcontentsline}

%
\section*{{\suttatitleacronym AN 7.21}{\suttatitletranslation At Sārandada }{\suttatitleroot Sārandadasutta}}
\addcontentsline{toc}{section}{\tocacronym{AN 7.21} \toctranslation{At Sārandada } \tocroot{Sārandadasutta}}
\markboth{At Sārandada }{Sārandadasutta}
\extramarks{AN 7.21}{AN 7.21}

\scevam{So\marginnote{1.1} I have heard. }At one time the Buddha was staying near \textsanskrit{Vesālī}, at the \textsanskrit{Sārandada} Tree-shrine. Then several Licchavis went up to the Buddha, bowed, sat down to one side, and the Buddha said to these Licchavis: 

“Licchavis,\marginnote{1.4} I will teach you these seven principles that prevent decline. Listen and pay close attention, I will speak.” 

“Yes,\marginnote{1.6} sir,” they replied. The Buddha said this: 

“And\marginnote{2.1} what are the seven principles that prevent decline? As long as the Vajjis meet frequently and have many meetings, they can expect growth, not decline. 

As\marginnote{3.1} long as the Vajjis meet in harmony, leave in harmony, and carry on their business in harmony, they can expect growth, not decline. 

As\marginnote{4.1} long as the Vajjis don’t make new decrees or abolish existing decrees, but undertake and follow the traditional Vajjian principles as they have been decreed, they can expect growth, not decline. 

As\marginnote{5.1} long as the Vajjis honor, respect, esteem, and venerate Vajjian elders, and think them worth listening to, they can expect growth, not decline. 

As\marginnote{6.1} long as the Vajjis don’t rape or abduct women or girls from their families and force them to live with them, they can expect growth, not decline. 

As\marginnote{7.1} long as the Vajjis honor, respect, esteem, and venerate the Vajjian shrines, whether inner or outer, not neglecting the proper spirit-offerings that were given and made in the past, they can expect growth, not decline. 

As\marginnote{8.1} long as the Vajjis organize proper protection, shelter, and security for perfected ones, so that more perfected ones might come to the realm and those already here may live in comfort, they can expect growth, not decline. 

As\marginnote{9.1} long as these seven principles that prevent decline last among the Vajjis, and as long as the Vajjis are seen following them, they can expect growth, not decline.” 

%
\section*{{\suttatitleacronym AN 7.22}{\suttatitletranslation With Vassakāra }{\suttatitleroot Vassakārasutta}}
\addcontentsline{toc}{section}{\tocacronym{AN 7.22} \toctranslation{With Vassakāra } \tocroot{Vassakārasutta}}
\markboth{With Vassakāra }{Vassakārasutta}
\extramarks{AN 7.22}{AN 7.22}

\scevam{So\marginnote{1.1} I have heard. }At one time the Buddha was staying near \textsanskrit{Rājagaha}, on the Vulture’s Peak Mountain. 

Now\marginnote{1.3} at that time King \textsanskrit{Ajātasattu} Vedehiputta of \textsanskrit{Māgadha} wanted to invade the Vajjis. He declared: “I shall wipe out these Vajjis, so mighty and powerful! I shall destroy them, and lay ruin and devastation upon them!” 

And\marginnote{2.1} then King \textsanskrit{Ajātasattu} addressed \textsanskrit{Vassakāra} the brahmin minister of \textsanskrit{Māgadha}, “Please, brahmin, go to the Buddha, and in my name bow with your head to his feet. Ask him if he is healthy and well, nimble, strong, and living comfortably. And then say: ‘Sir, King \textsanskrit{Ajātasattu} Vedehiputta of \textsanskrit{Māgadha} wants to invade the Vajjis. He has declared: “I shall wipe out these Vajjis, so mighty and powerful! I shall destroy them, and lay ruin and devastation upon them!”’ Remember well how the Buddha answers and tell it to me. For Realized Ones say nothing that is not so.” 

“Yes,\marginnote{3.1} sir,” \textsanskrit{Vassakāra} replied. He went to the Buddha and exchanged greetings with him. When the greetings and polite conversation were over, he sat down to one side and said to the Buddha: 

“Master\marginnote{3.3} Gotama, King \textsanskrit{Ajātasattu} bows with his head to your feet. He asks if you are healthy and well, nimble, strong, and living comfortably. King \textsanskrit{Ajātasattu} wants to invade the Vajjis. He has declared: ‘I shall wipe out these Vajjis, so mighty and powerful! I shall destroy them, and lay ruin and devastation upon them!’” 

Now\marginnote{4.1} at that time Venerable Ānanda was standing behind the Buddha fanning him. Then the Buddha said to him: “Ānanda, have you heard that the Vajjis meet frequently and have many meetings?” 

“I\marginnote{4.4} have heard that, sir.” 

“As\marginnote{4.5} long as the Vajjis meet frequently and have many meetings, they can expect growth, not decline. 

Ānanda,\marginnote{5.1} have you heard that the Vajjis meet in harmony, leave in harmony, and carry on their business in harmony?” 

“I\marginnote{5.2} have heard that, sir.” 

“As\marginnote{5.3} long as the Vajjis meet in harmony, leave in harmony, and carry on their business in harmony, they can expect growth, not decline. 

Ānanda,\marginnote{6.1} have you heard that the Vajjis don’t make new decrees or abolish existing decrees, but proceed having undertaken the ancient Vajjian principles as they have been decreed?” 

“I\marginnote{6.2} have heard that, sir.” 

“As\marginnote{6.3} long as the Vajjis don’t make new decrees or abolish existing decrees, but proceed having undertaken the traditional Vajjian principles as they have been decreed, they can expect growth, not decline. 

Ānanda,\marginnote{7.1} have you heard that the Vajjis honor, respect, esteem, and venerate Vajjian elders, and think them worth listening to?” 

“I\marginnote{7.2} have heard that, sir.” 

“As\marginnote{7.3} long as the Vajjis honor, respect, esteem, and venerate Vajjian elders, and think them worth listening to, they can expect growth, not decline. 

Ānanda,\marginnote{8.1} have you heard that the Vajjis don’t rape or abduct women or girls from their families and force them to live with them?” 

“I\marginnote{8.2} have heard that, sir.” 

“As\marginnote{8.3} long as the Vajjis don’t rape or abduct women or girls from their families and force them to live with them, they can expect growth, not decline. 

Ānanda,\marginnote{9.1} have you heard that the Vajjis honor, respect, esteem, and venerate the Vajjian shrines, whether inner or outer, not neglecting the proper spirit-offerings that were given and made in the past?” 

“I\marginnote{9.2} have heard that, sir.” 

“As\marginnote{9.3} long as the Vajjis honor, respect, esteem, and venerate the Vajjian shrines, whether inner or outer, not neglecting the proper spirit-offerings that were given and made in the past, they can expect growth, not decline. 

Ānanda,\marginnote{10.1} have you heard that the Vajjis organize proper protection, shelter, and security for perfected ones, so that more perfected ones might come to the realm and those already here may live in comfort?” 

“I\marginnote{10.2} have heard that, sir.” 

“As\marginnote{10.3} long as the Vajjis organize proper protection, shelter, and security for perfected ones, so that more perfected ones might come to the realm and those already here may live in comfort, they can expect growth, not decline.” 

Then\marginnote{11.1} the Buddha said to \textsanskrit{Vassakāra}: 

“Brahmin,\marginnote{11.2} one time I was staying near \textsanskrit{Vesālī} at the \textsanskrit{Sārandada} woodland shrine. There I taught the Vajjis these principles that prevent decline. As long as these seven principles that prevent decline last among the Vajjis, and as long as the Vajjis are seen following them, they can expect growth, not decline.” 

When\marginnote{12.1} the Buddha had spoken, \textsanskrit{Vassakāra} said to him: “Master Gotama, if the Vajjis follow even a single one of these principles they can expect growth, not decline. How much more so all seven! King \textsanskrit{Ajātasattu} cannot defeat the Vajjis in war, unless by diplomacy or by sowing dissension. Well, now, Master Gotama, I must go. I have many duties, and much to do.” 

“Please,\marginnote{12.5} brahmin, go at your convenience.” Then \textsanskrit{Vassakāra} the brahmin, having approved and agreed with what the Buddha said, got up from his seat and left. 

%
\section*{{\suttatitleacronym AN 7.23}{\suttatitletranslation Non-Decline for Mendicants (1st) }{\suttatitleroot Paṭhamasattakasutta}}
\addcontentsline{toc}{section}{\tocacronym{AN 7.23} \toctranslation{Non-Decline for Mendicants (1st) } \tocroot{Paṭhamasattakasutta}}
\markboth{Non-Decline for Mendicants (1st) }{Paṭhamasattakasutta}
\extramarks{AN 7.23}{AN 7.23}

\scevam{So\marginnote{1.1} I have heard. }At one time the Buddha was staying near \textsanskrit{Rājagaha}, on the Vulture’s Peak Mountain. There the Buddha addressed the mendicants: 

“Mendicants,\marginnote{1.4} I will teach you these seven principles that prevent decline. Listen and pay close attention, I will speak.” 

“Yes,\marginnote{1.6} sir,” they replied. The Buddha said this: 

“What\marginnote{2.1} are the seven principles that prevent decline? As long as the mendicants meet frequently and have many meetings, they can expect growth, not decline. 

As\marginnote{3.1} long as the mendicants meet in harmony, leave in harmony, and carry on their business in harmony, they can expect growth, not decline. 

As\marginnote{4.1} long as the mendicants don’t make new decrees or abolish existing decrees, but undertake and follow the training rules as they have been decreed, they can expect growth, not decline. 

As\marginnote{5.1} long as the mendicants honor, respect, esteem, and venerate the senior mendicants—of long standing, long gone forth, fathers and leaders of the \textsanskrit{Saṅgha}—and think them worth listening to, they can expect growth, not decline. 

As\marginnote{6.1} long as the mendicants don’t fall under the sway of arisen craving for future lives, they can expect growth, not decline. 

As\marginnote{7.1} long as the mendicants take care to live in wilderness lodgings, they can expect growth, not decline. 

As\marginnote{8.1} long as the mendicants individually establish mindfulness, so that more good-hearted spiritual companions might come, and those that have already come may live comfortably, they can expect growth, not decline. 

As\marginnote{9.1} long as these seven principles that prevent decline last among the mendicants, and as long as the mendicants are seen following them, they can expect growth, not decline.” 

%
\section*{{\suttatitleacronym AN 7.24}{\suttatitletranslation Non-Decline for Mendicants (2nd) }{\suttatitleroot Dutiyasattakasutta}}
\addcontentsline{toc}{section}{\tocacronym{AN 7.24} \toctranslation{Non-Decline for Mendicants (2nd) } \tocroot{Dutiyasattakasutta}}
\markboth{Non-Decline for Mendicants (2nd) }{Dutiyasattakasutta}
\extramarks{AN 7.24}{AN 7.24}

“Mendicants,\marginnote{1.1} I will teach you seven principles that prevent decline. Listen and pay close attention … And what are the seven principles that prevent decline? 

As\marginnote{2.1} long as the mendicants don’t relish work, loving it and liking to relish it, they can expect growth, not decline. 

As\marginnote{3.1} long as they don’t enjoy talk … sleep … company … they don’t have wicked desires, falling under the sway of wicked desires … they don’t have bad friends, companions, and associates … they don’t stop half-way after achieving some insignificant distinction, they can expect growth, not decline. 

As\marginnote{4.1} long as these seven principles that prevent decline last among the mendicants, and as long as the mendicants are seen following them, they can expect growth, not decline.” 

%
\section*{{\suttatitleacronym AN 7.25}{\suttatitletranslation Non-Decline for Mendicants (3rd) }{\suttatitleroot Tatiyasattakasutta}}
\addcontentsline{toc}{section}{\tocacronym{AN 7.25} \toctranslation{Non-Decline for Mendicants (3rd) } \tocroot{Tatiyasattakasutta}}
\markboth{Non-Decline for Mendicants (3rd) }{Tatiyasattakasutta}
\extramarks{AN 7.25}{AN 7.25}

“Mendicants,\marginnote{1.1} I will teach you seven principles that prevent decline. Listen and pay close attention … And what are the seven principles that prevent decline? As long as the mendicants are faithful … conscientious … prudent … learned … energetic … mindful … wise, they can expect growth, not decline. 

As\marginnote{3.1} long as these seven principles that prevent decline last among the mendicants, and as long as the mendicants are seen following them, they can expect growth, not decline.” 

%
\section*{{\suttatitleacronym AN 7.26}{\suttatitletranslation Awakening Factors }{\suttatitleroot Bojjhaṅgasutta}}
\addcontentsline{toc}{section}{\tocacronym{AN 7.26} \toctranslation{Awakening Factors } \tocroot{Bojjhaṅgasutta}}
\markboth{Awakening Factors }{Bojjhaṅgasutta}
\extramarks{AN 7.26}{AN 7.26}

“Mendicants,\marginnote{1.1} I will teach you seven principles that prevent decline. Listen and pay close attention … And what are the seven principles that prevent decline? As long as the mendicants develop the awakening factor of mindfulness … investigation of principles … energy … rapture … tranquility … immersion … equanimity, they can expect growth, not decline. 

As\marginnote{3.1} long as these seven principles that prevent decline last among the mendicants, and as long as the mendicants are seen following them, they can expect growth, not decline.” 

%
\section*{{\suttatitleacronym AN 7.27}{\suttatitletranslation Perceptions }{\suttatitleroot Saññāsutta}}
\addcontentsline{toc}{section}{\tocacronym{AN 7.27} \toctranslation{Perceptions } \tocroot{Saññāsutta}}
\markboth{Perceptions }{Saññāsutta}
\extramarks{AN 7.27}{AN 7.27}

“Mendicants,\marginnote{1.1} I will teach you seven principles that prevent decline. Listen and pay close attention … And what are the seven principles that prevent decline? As long as the mendicants develop the perception of impermanence … 

not-self\marginnote{2.1} … ugliness … drawbacks … giving up … fading away … cessation, they can expect growth, not decline. As long as these seven principles that prevent decline last among the mendicants, and as long as the mendicants are seen following them, they can expect growth, not decline.” 

%
\section*{{\suttatitleacronym AN 7.28}{\suttatitletranslation Non-decline for a Mendicant Trainee }{\suttatitleroot Paṭhamaparihānisutta}}
\addcontentsline{toc}{section}{\tocacronym{AN 7.28} \toctranslation{Non-decline for a Mendicant Trainee } \tocroot{Paṭhamaparihānisutta}}
\markboth{Non-decline for a Mendicant Trainee }{Paṭhamaparihānisutta}
\extramarks{AN 7.28}{AN 7.28}

\scevam{So\marginnote{1.1} I have heard. }At one time the Buddha was staying near \textsanskrit{Sāvatthī} in Jeta’s Grove, \textsanskrit{Anāthapiṇḍika}’s monastery. There the Buddha addressed the mendicants: 

“These\marginnote{1.4} seven things lead to the decline of a mendicant trainee. What seven? They relish work, talk, sleep, and company. They don’t guard the sense doors and they eat too much. And when there is \textsanskrit{Saṅgha} business to be carried out, they don’t reflect: ‘There are senior mendicants in the \textsanskrit{Saṅgha} of long standing, long gone forth, responsible. They’ll be known for taking care of this.’ So they try to do it themselves. These seven things lead to the decline of a mendicant trainee. 

These\marginnote{2.1} seven things don’t lead to the decline of a mendicant trainee. What seven? They don’t relish work, talk, sleep, and company. They guard the sense doors and don’t they eat too much. And when there is \textsanskrit{Saṅgha} business to be carried out, they reflect: ‘There are senior mendicants in the \textsanskrit{Saṅgha} of long standing, long gone forth, responsible. They’ll be known for taking care of this.’ So they don’t try to do it themselves. These seven things don’t lead to the decline of a mendicant trainee.” 

%
\section*{{\suttatitleacronym AN 7.29}{\suttatitletranslation Non-decline for a Lay Follower }{\suttatitleroot Dutiyaparihānisutta}}
\addcontentsline{toc}{section}{\tocacronym{AN 7.29} \toctranslation{Non-decline for a Lay Follower } \tocroot{Dutiyaparihānisutta}}
\markboth{Non-decline for a Lay Follower }{Dutiyaparihānisutta}
\extramarks{AN 7.29}{AN 7.29}

“These\marginnote{1.1} seven things lead to the decline of a lay follower. What seven? They miss out on seeing the mendicants. They neglect listening to the true teaching. They don’t train in higher ethical conduct. They’re very suspicious about mendicants, whether senior, junior, or middle. They listen to the teaching with a hostile, fault-finding mind. They seek outside of the Buddhist community for those worthy of religious donations. And they serve them first. These seven things lead to the decline of a lay follower. 

These\marginnote{2.1} seven things don’t lead to the decline of a lay follower. What seven? They don’t miss out on seeing the mendicants. They don’t neglect listening to the true teaching. They train in higher ethical conduct. They’re very confident about mendicants, whether senior, junior, or middle. They don’t listen to the teaching with a hostile, fault-finding mind. They don’t seek outside of the Buddhist community for those worthy of religious donations. And they serve the Buddhist community first. These seven things don’t lead to the decline of a lay follower.” 

That\marginnote{2.11} is what the Buddha said. Then the Holy One, the Teacher, went on to say: 

\begin{verse}%
“A\marginnote{3.1} lay follower misses out on seeing \\
those who are evolved \\
and listening to the teachings of the Noble One. \\
They don’t train in higher ethical conduct, 

and\marginnote{4.1} their suspicion about mendicants \\
just grows and grows. \\
They want to listen to the true teaching \\
with a fault-finding mind. 

They\marginnote{5.1} seek outside the Buddhist community \\
for someone else worthy of religious donations, \\
and that lay follower \\
serves them first. 

These\marginnote{6.1} seven principles leading to decline \\
have been well taught. \\
A lay follower who practices them \\
falls away from the true teaching. 

A\marginnote{7.1} lay follower doesn’t miss out on seeing \\
those who are evolved \\
and listening to the teachings of the Noble One. \\
They train in higher ethical conduct, 

and\marginnote{8.1} their confidence in mendicants \\
just grows and grows. \\
They want to listen to the true teaching \\
without a fault-finding mind. 

They\marginnote{9.1} don’t seek outside the Buddhist community \\
for someone else worthy of religious donations, \\
and that lay follower \\
serves the Buddhist community first. 

These\marginnote{10.1} seven principles that prevent decline \\
have been well taught. \\
A lay follower who practices them \\
doesn’t fall away from the true teaching.” 

%
\end{verse}

%
\section*{{\suttatitleacronym AN 7.30}{\suttatitletranslation Failures for a Lay Follower }{\suttatitleroot Vipattisutta}}
\addcontentsline{toc}{section}{\tocacronym{AN 7.30} \toctranslation{Failures for a Lay Follower } \tocroot{Vipattisutta}}
\markboth{Failures for a Lay Follower }{Vipattisutta}
\extramarks{AN 7.30}{AN 7.30}

“Mendicants,\marginnote{1.1} there are these seven failures for a lay follower … 

There\marginnote{1.2} are these seven accomplishments for a lay follower …” 

%
\section*{{\suttatitleacronym AN 7.31}{\suttatitletranslation Downfalls for a Lay Follower }{\suttatitleroot Parābhavasutta}}
\addcontentsline{toc}{section}{\tocacronym{AN 7.31} \toctranslation{Downfalls for a Lay Follower } \tocroot{Parābhavasutta}}
\markboth{Downfalls for a Lay Follower }{Parābhavasutta}
\extramarks{AN 7.31}{AN 7.31}

“Mendicants,\marginnote{1.1} there are these seven downfalls for a lay follower … There are these seven successes for a lay follower. What seven? 

They\marginnote{1.4} don’t miss out on seeing the mendicants. 

They\marginnote{1.5} don’t neglect listening to the true teaching. 

They\marginnote{1.6} train in higher ethical conduct. 

They’re\marginnote{1.7} very confident about mendicants, whether senior, junior, or middle. 

They\marginnote{1.8} don’t listen to the teaching with a hostile, fault-finding mind. 

They\marginnote{1.9} don’t seek outside of the Buddhist community for those worthy of religious donations. 

And\marginnote{1.10} they serve the Buddhist community first. 

These\marginnote{1.11} are the seven successes for a lay follower. 

\begin{verse}%
A\marginnote{2.1} lay follower misses out on seeing \\
those who are evolved … 

A\marginnote{5.1} lay follower who practices these \\
falls away from the true teaching. 

A\marginnote{6.1} lay follower doesn’t miss out on seeing \\
those who are evolved … 

A\marginnote{9.1} lay follower who practices these \\
doesn’t fall away from the true teaching.” 

%
\end{verse}

%
\addtocontents{toc}{\let\protect\contentsline\protect\nopagecontentsline}
\chapter*{The Chapter on Deities }
\addcontentsline{toc}{chapter}{\tocchapterline{The Chapter on Deities }}
\addtocontents{toc}{\let\protect\contentsline\protect\oldcontentsline}

%
\section*{{\suttatitleacronym AN 7.32}{\suttatitletranslation Respect for Diligence }{\suttatitleroot Appamādagāravasutta}}
\addcontentsline{toc}{section}{\tocacronym{AN 7.32} \toctranslation{Respect for Diligence } \tocroot{Appamādagāravasutta}}
\markboth{Respect for Diligence }{Appamādagāravasutta}
\extramarks{AN 7.32}{AN 7.32}

Then,\marginnote{1.1} late at night, a glorious deity, lighting up the entire Jeta’s Grove, went up to the Buddha, bowed, stood to one side, and said to him: 

“Sir,\marginnote{2.1} these seven things don’t lead to the decline of a mendicant trainee. What seven? Respect for the Teacher, for the teaching, for the \textsanskrit{Saṅgha}, for the training, for immersion, for diligence, and for hospitality. These seven things don’t lead to the decline of a mendicant trainee.” 

That’s\marginnote{2.5} what that deity said, and the teacher approved. Then that deity, knowing that the teacher approved, bowed and respectfully circled the Buddha, keeping him on his right, before vanishing right there. 

Then,\marginnote{3.1} when the night had passed, the Buddha told the mendicants all that had happened, adding: 

\begin{verse}%
“Respect\marginnote{4.1} for the Teacher and the teaching, \\
and keen respect for the \textsanskrit{Saṅgha}; \\
respect for immersion, being energetic, \\
and keen respect for the training. 

A\marginnote{5.1} mendicant who respects diligence \\
and hospitality \\
can’t decline, \\
and has drawn near to extinguishment.” 

%
\end{verse}

%
\section*{{\suttatitleacronym AN 7.33}{\suttatitletranslation Respect for Conscience }{\suttatitleroot Hirigāravasutta}}
\addcontentsline{toc}{section}{\tocacronym{AN 7.33} \toctranslation{Respect for Conscience } \tocroot{Hirigāravasutta}}
\markboth{Respect for Conscience }{Hirigāravasutta}
\extramarks{AN 7.33}{AN 7.33}

“Mendicants,\marginnote{1.1} tonight, a glorious deity, lighting up the entire Jeta’s Grove, came to me, bowed, stood to one side, and said to me: ‘Sir, these seven things don’t lead to the decline of a mendicant trainee. What seven? Respect for the Teacher, for the teaching, for the \textsanskrit{Saṅgha}, for the training, for immersion, for conscience, and for prudence. These seven things don’t lead to the decline of a mendicant trainee.’ 

That\marginnote{1.6} is what that deity said. Then he bowed and respectfully circled me, keeping me on his right side, before vanishing right there. 

\begin{verse}%
Respect\marginnote{2.1} for the Teacher and the teaching, \\
and keen respect for the \textsanskrit{Saṅgha}; \\
respect for immersion, being energetic, \\
and keen respect for the training. 

One\marginnote{3.1} with both conscience and prudence, \\
reverential and respectful, \\
can’t decline, \\
and has drawn near to extinguishment.” 

%
\end{verse}

%
\section*{{\suttatitleacronym AN 7.34}{\suttatitletranslation Easy to Admonish (1st) }{\suttatitleroot Paṭhamasovacassatāsutta}}
\addcontentsline{toc}{section}{\tocacronym{AN 7.34} \toctranslation{Easy to Admonish (1st) } \tocroot{Paṭhamasovacassatāsutta}}
\markboth{Easy to Admonish (1st) }{Paṭhamasovacassatāsutta}
\extramarks{AN 7.34}{AN 7.34}

“Mendicants,\marginnote{1.1} tonight a deity … said to me: ‘Sir, these seven things don’t lead to the decline of a mendicant trainee. What seven? Respect for the Teacher, for the teaching, for the \textsanskrit{Saṅgha}, for the training, for immersion; being easy to admonish, and good friendship. These seven things don’t lead to the decline of a mendicant trainee.’ 

That\marginnote{1.6} is what that deity said. Then he bowed and respectfully circled me, keeping me on his right side, before vanishing right there. 

\begin{verse}%
Respect\marginnote{2.1} for the Teacher and the teaching, \\
and keen respect for the \textsanskrit{Saṅgha}; \\
respect for immersion, being energetic, \\
and keen respect for the training. 

One\marginnote{3.1} with good friends, easy to admonish, \\
reverential and respectful, \\
can’t decline, \\
and has drawn near to extinguishment.” 

%
\end{verse}

%
\section*{{\suttatitleacronym AN 7.35}{\suttatitletranslation Easy to Admonish (2nd) }{\suttatitleroot Dutiyasovacassatāsutta}}
\addcontentsline{toc}{section}{\tocacronym{AN 7.35} \toctranslation{Easy to Admonish (2nd) } \tocroot{Dutiyasovacassatāsutta}}
\markboth{Easy to Admonish (2nd) }{Dutiyasovacassatāsutta}
\extramarks{AN 7.35}{AN 7.35}

“Mendicants,\marginnote{1.1} tonight a deity … said to me: ‘Sir, these seven things don’t lead to the decline of a mendicant trainee. What seven? Respect for the Teacher, for the teaching, for the \textsanskrit{Saṅgha}, for the training, for immersion; being easy to admonish, and good friendship. These seven things don’t lead to the decline of a mendicant trainee.’ 

That\marginnote{1.6} is what that deity said. Then he bowed and respectfully circled me, keeping me on his right side, before vanishing right there.” 

When\marginnote{2.1} he said this, Venerable \textsanskrit{Sāriputta} said to the Buddha: 

“Sir,\marginnote{2.2} this is how I understand the detailed meaning of the Buddha’s brief statement. 

It’s\marginnote{2.3} when a mendicant personally respects the Teacher and praises such respect. And they encourage other mendicants who lack such respect to respect the Teacher. And they praise other mendicants who respect the Teacher at the right time, truthfully and substantively. 

They\marginnote{2.6} personally respect the teaching … 

They\marginnote{2.7} personally respect the \textsanskrit{Saṅgha} … 

They\marginnote{2.8} personally respect the training … 

They\marginnote{2.9} personally respect immersion … 

They\marginnote{2.10} are personally easy to admonish … 

They\marginnote{2.11} personally have good friends, and praise such friendship. And they encourage other mendicants who lack good friends to develop good friendship. And they praise other mendicants who have good friends at the right time, truthfully and substantively. 

That’s\marginnote{2.14} how I understand the detailed meaning of the Buddha’s brief statement.” 

“Good,\marginnote{3.1} good, \textsanskrit{Sāriputta}! It’s good that you understand the detailed meaning of what I’ve said in brief like this. 

It’s\marginnote{3.3} when a mendicant personally respects the Teacher … And they encourage other mendicants who lack such respect to respect the Teacher. And they praise other mendicants who respect the Teacher at the right time, truthfully and substantively. 

They\marginnote{3.6} personally respect the teaching … 

They\marginnote{3.7} personally respect the \textsanskrit{Saṅgha} … 

They\marginnote{3.8} personally respect the training … 

They\marginnote{3.9} personally respect immersion … 

They\marginnote{3.10} are personally easy to admonish … 

They\marginnote{3.11} personally have good friends, and praise such friendship. And they encourage other mendicants who lack good friends to develop good friendship. And they praise other mendicants who have good friends at the right time, truthfully and substantively. 

This\marginnote{3.14} is how to understand the detailed meaning of what I said in brief.” 

%
\section*{{\suttatitleacronym AN 7.36}{\suttatitletranslation A Friend (1st) }{\suttatitleroot Paṭhamamittasutta}}
\addcontentsline{toc}{section}{\tocacronym{AN 7.36} \toctranslation{A Friend (1st) } \tocroot{Paṭhamamittasutta}}
\markboth{A Friend (1st) }{Paṭhamamittasutta}
\extramarks{AN 7.36}{AN 7.36}

“Mendicants,\marginnote{1.1} you should associate with a friend who has seven factors. What seven? They give what is hard to give. They do what is hard to do. They endure what is hard to endure. They reveal their secrets to you. They keep your secrets. They don’t abandon you in times of trouble. They don’t look down on you in times of loss. You should associate with a friend who has these seven factors. 

\begin{verse}%
A\marginnote{2.1} friend gives what is hard to give, \\
and does what’s hard to do. \\
They put up with your harsh words, \\
and with things hard to endure. 

They\marginnote{3.1} tell you their secrets, \\
and keep your secrets for you. \\
They don’t abandon you in times of trouble, \\
or look down on you in times of loss. 

The\marginnote{4.1} person in whom \\
these things are found is your friend. \\
If you want to have a friend, \\
you should keep company with such a person.” 

%
\end{verse}

%
\section*{{\suttatitleacronym AN 7.37}{\suttatitletranslation A Friend (2nd) }{\suttatitleroot Dutiyamittasutta}}
\addcontentsline{toc}{section}{\tocacronym{AN 7.37} \toctranslation{A Friend (2nd) } \tocroot{Dutiyamittasutta}}
\markboth{A Friend (2nd) }{Dutiyamittasutta}
\extramarks{AN 7.37}{AN 7.37}

“Mendicants,\marginnote{1.1} when a friend has seven qualities you should associate with, accompany, and attend them, even if they send you away. What seven? They’re likable, agreeable, respected, and admired. They admonish you and they accept admonishment. They speak on deep matters. And they don’t urge you to do bad things. When a friend has these seven qualities you should associate with, accompany, and attend them, even if they send you away. 

\begin{verse}%
They’re\marginnote{2.1} lovable, respected, and admired, \\
an admonisher who accepts admonishment, \\
speaks on deep matters, \\
and doesn’t urge you to do bad. 

The\marginnote{3.1} person in whom \\
these things are found is your friend. \\
If you want to have a friend, \\
benevolent and compassionate, \\
you should keep company with such a person, \\
even if they send you away.” 

%
\end{verse}

%
\section*{{\suttatitleacronym AN 7.38}{\suttatitletranslation Textual Analysis (1st) }{\suttatitleroot Paṭhamapaṭisambhidāsutta}}
\addcontentsline{toc}{section}{\tocacronym{AN 7.38} \toctranslation{Textual Analysis (1st) } \tocroot{Paṭhamapaṭisambhidāsutta}}
\markboth{Textual Analysis (1st) }{Paṭhamapaṭisambhidāsutta}
\extramarks{AN 7.38}{AN 7.38}

“Mendicants,\marginnote{1.1} a mendicant with seven qualities will soon realize the four kinds of textual analysis and live having achieved them with their own insight. What seven? 

It’s\marginnote{1.3} when a mendicant truly understands: ‘This is mental sluggishness’. 

They\marginnote{1.4} truly understand internally constricted mind as ‘internally constricted mind’. 

They\marginnote{1.5} truly understand externally scattered mind as ‘externally scattered mind’. 

They\marginnote{1.6} know feelings as they arise, as they remain, and as they go away. 

They\marginnote{1.7} know perceptions as they arise, as they remain, and as they go away. 

They\marginnote{1.8} know thoughts as they arise, as they remain, and as they go away. 

The\marginnote{1.9} patterns of qualities—suitable or unsuitable, inferior or superior, or those on the side of dark or bright—are properly grasped, attended, borne in mind, and comprehended with wisdom. 

A\marginnote{1.10} mendicant with these seven qualities will soon realize the four kinds of textual analysis and live having achieved them with their own insight.” 

%
\section*{{\suttatitleacronym AN 7.39}{\suttatitletranslation Textual Analysis (2nd) }{\suttatitleroot Dutiyapaṭisambhidāsutta}}
\addcontentsline{toc}{section}{\tocacronym{AN 7.39} \toctranslation{Textual Analysis (2nd) } \tocroot{Dutiyapaṭisambhidāsutta}}
\markboth{Textual Analysis (2nd) }{Dutiyapaṭisambhidāsutta}
\extramarks{AN 7.39}{AN 7.39}

“Mendicants,\marginnote{1.1} having seven qualities, \textsanskrit{Sāriputta} realized the four kinds of textual analysis and lives having achieved them with his own insight. What seven? 

It’s\marginnote{1.3} when \textsanskrit{Sāriputta} truly understood: ‘This is mental sluggishness’. 

He\marginnote{1.4} truly understood internally constricted mind as ‘internally constricted mind’. 

He\marginnote{1.5} truly understood externally scattered mind as ‘externally scattered mind’. 

He\marginnote{1.6} knew feelings, perceptions, and thoughts as they arose, as they remained, and as they went away. 

The\marginnote{1.9} patterns of qualities—suitable or unsuitable, inferior or superior, or those on the side of dark or bright—were properly grasped, attended, borne in mind, and comprehended with wisdom. 

Having\marginnote{1.10} these seven qualities, \textsanskrit{Sāriputta} realized the four kinds of textual analysis and lives having achieved them with his own insight.” 

%
\section*{{\suttatitleacronym AN 7.40}{\suttatitletranslation Mastery of the Mind (1st) }{\suttatitleroot Paṭhamavasasutta}}
\addcontentsline{toc}{section}{\tocacronym{AN 7.40} \toctranslation{Mastery of the Mind (1st) } \tocroot{Paṭhamavasasutta}}
\markboth{Mastery of the Mind (1st) }{Paṭhamavasasutta}
\extramarks{AN 7.40}{AN 7.40}

“Mendicants,\marginnote{1.1} a mendicant with seven qualities masters their mind and is not mastered by it. What seven? It’s when a mendicant is skilled at immersion, skilled in entering immersion, skilled in remaining in immersion, skilled in emerging from immersion, skilled in gladdening the mind for immersion, skilled in the mindfulness meditation subjects for immersion, and skilled in projecting the mind purified by immersion. A mendicant with these seven qualities masters their mind and is not mastered by it.” 

%
\section*{{\suttatitleacronym AN 7.41}{\suttatitletranslation Mastery of the Mind (2nd) }{\suttatitleroot Dutiyavasasutta}}
\addcontentsline{toc}{section}{\tocacronym{AN 7.41} \toctranslation{Mastery of the Mind (2nd) } \tocroot{Dutiyavasasutta}}
\markboth{Mastery of the Mind (2nd) }{Dutiyavasasutta}
\extramarks{AN 7.41}{AN 7.41}

“Mendicants,\marginnote{1.1} having seven qualities \textsanskrit{Sāriputta} has mastered his mind and is not mastered by it. What seven? \textsanskrit{Sāriputta} is skilled at immersion, skilled in entering immersion, skilled in remaining in immersion, skilled in emerging from immersion, skilled in gladdening the mind for immersion, skilled in the mindfulness meditation subjects for immersion, and skilled in projecting the mind purified by immersion. Having these seven qualities \textsanskrit{Sāriputta} has mastered his mind and is not mastered by it.” 

%
\section*{{\suttatitleacronym AN 7.42}{\suttatitletranslation Graduation (1st) }{\suttatitleroot Paṭhamaniddasasutta}}
\addcontentsline{toc}{section}{\tocacronym{AN 7.42} \toctranslation{Graduation (1st) } \tocroot{Paṭhamaniddasasutta}}
\markboth{Graduation (1st) }{Paṭhamaniddasasutta}
\extramarks{AN 7.42}{AN 7.42}

Then\marginnote{1.1} Venerable \textsanskrit{Sāriputta} robed up in the morning and, taking his bowl and robe, entered \textsanskrit{Sāvatthī} for alms. Then it occurred to him, “It’s too early to wander for alms in \textsanskrit{Sāvatthī}. Why don’t I go to the monastery of the wanderers who follow other paths?” 

Then\marginnote{1.5} he went to the monastery of the wanderers who follow other paths, and exchanged greetings with the wanderers there. When the greetings and polite conversation were over, he sat down to one side. 

Now\marginnote{1.7} at that time while those wanderers who follow other paths were sitting together this discussion came up among them, “Reverends, anyone who lives the full and pure spiritual life for twelve years is qualified to be called a ‘graduate mendicant’.” 

\textsanskrit{Sāriputta}\marginnote{2.1} neither approved nor dismissed that statement of the wanderers who follow other paths. He got up from his seat, thinking, “I will learn the meaning of this statement from the Buddha himself.” 

Then\marginnote{2.4} \textsanskrit{Sāriputta} wandered for alms in \textsanskrit{Sāvatthī}. After the meal, on his return from almsround, he went to the Buddha, bowed, sat down to one side, and told him what had happened, adding: 

“Sir,\marginnote{3.1} in this teaching and training can we describe a mendicant as a ‘graduate’ solely because they have completed a certain number of years?” 

“No,\marginnote{4.1} \textsanskrit{Sāriputta}, we cannot. I make known these seven qualifications for graduation after realizing them with my own insight. 

What\marginnote{5.1} seven? It’s when a mendicant has a keen enthusiasm to undertake the training … to examine the teachings … to get rid of desires … for retreat … to rouse up energy … for mindfulness and alertness … to penetrate theoretically. And they don’t lose these desires in the future. These are the seven qualifications for graduation that I make known after realizing them with my own insight. A mendicant who has these seven qualifications for graduation is qualified to be called a ‘graduate mendicant’. This is so whether they have lived the full and pure spiritual life for twelve years, twenty-four years, thirty-six years, or forty-eight years.” 

%
\section*{{\suttatitleacronym AN 7.43}{\suttatitletranslation Graduation (2nd) }{\suttatitleroot Dutiyaniddasasutta}}
\addcontentsline{toc}{section}{\tocacronym{AN 7.43} \toctranslation{Graduation (2nd) } \tocroot{Dutiyaniddasasutta}}
\markboth{Graduation (2nd) }{Dutiyaniddasasutta}
\extramarks{AN 7.43}{AN 7.43}

\scevam{So\marginnote{1.1} I have heard. }At one time the Buddha was staying near Kosambi, in Ghosita’s Monastery. 

Then\marginnote{1.3} Venerable Ānanda robed up in the morning and, taking his bowl and robe, entered Kosambi for alms. Then it occurred to him, “It’s too early to wander for alms in Kosambi. Why don’t I go to the monastery of the wanderers who follow other paths?” 

Then\marginnote{1.7} he went to the monastery of the wanderers who follow other paths, and exchanged greetings with the wanderers there. When the greetings and polite conversation were over, he sat down to one side. 

Now\marginnote{2.1} at that time while those wanderers who follow other paths were sitting together this discussion came up among them, “Reverends, anyone who lives the full and pure spiritual life for twelve years is qualified to be called a ‘graduate mendicant’.” 

Ānanda\marginnote{3.1} neither approved nor dismissed that statement of the wanderers who follow other paths. He got up from his seat, thinking, “I will learn the meaning of this statement from the Buddha himself.” 

Then\marginnote{3.4} Ānanda wandered for alms in Kosambi. After the meal, on his return from almsround, he went to the Buddha, bowed, sat down to one side, and told him what had happened, adding: 

“Sir,\marginnote{5.1} in this teaching and training can we describe a mendicant as a ‘graduate’ solely because they have completed a certain number of years?” 

“No,\marginnote{6.1} Ānanda, we cannot. These are the seven qualifications for graduation that I make known after realizing them with my own insight. 

What\marginnote{7.1} seven? It’s when someone is faithful, conscientious, prudent, learned, energetic, mindful, and wise. These are the seven qualifications for graduation that I make known after realizing them with my own insight. A mendicant who has these seven qualifications for graduation is qualified to be called a ‘graduate mendicant’. This is so whether they have lived the full and pure spiritual life for twelve years, twenty-four years, thirty-six years, or forty-eight years.” 

%
\addtocontents{toc}{\let\protect\contentsline\protect\nopagecontentsline}
\chapter*{The Chapter on a Great Sacrifice }
\addcontentsline{toc}{chapter}{\tocchapterline{The Chapter on a Great Sacrifice }}
\addtocontents{toc}{\let\protect\contentsline\protect\oldcontentsline}

%
\section*{{\suttatitleacronym AN 7.44}{\suttatitletranslation Planes of Consciousness }{\suttatitleroot Sattaviññāṇaṭṭhitisutta}}
\addcontentsline{toc}{section}{\tocacronym{AN 7.44} \toctranslation{Planes of Consciousness } \tocroot{Sattaviññāṇaṭṭhitisutta}}
\markboth{Planes of Consciousness }{Sattaviññāṇaṭṭhitisutta}
\extramarks{AN 7.44}{AN 7.44}

“Mendicants,\marginnote{1.1} there are these seven planes of consciousness. What seven? 

There\marginnote{1.3} are sentient beings that are diverse in body and diverse in perception, such as human beings, some gods, and some beings in the underworld. This is the first plane of consciousness. 

There\marginnote{2.1} are sentient beings that are diverse in body and unified in perception, such as the gods reborn in \textsanskrit{Brahmā}’s Host through the first absorption. This is the second plane of consciousness. 

There\marginnote{3.1} are sentient beings that are unified in body and diverse in perception, such as the gods of streaming radiance. This is the third plane of consciousness. 

There\marginnote{4.1} are sentient beings that are unified in body and unified in perception, such as the gods replete with glory. This is the fourth plane of consciousness. 

There\marginnote{5.1} are sentient beings that have gone totally beyond perceptions of form. With the ending of perceptions of impingement, not focusing on perceptions of diversity, aware that ‘space is infinite’, they have been reborn in the dimension of infinite space. This is the fifth plane of consciousness. 

There\marginnote{6.1} are sentient beings that have gone totally beyond the dimension of infinite space. Aware that ‘consciousness is infinite’, they have been reborn in the dimension of infinite consciousness. This is the sixth plane of consciousness. 

There\marginnote{7.1} are sentient beings that have gone totally beyond the dimension of infinite consciousness. Aware that ‘there is nothing at all’, they have been reborn in the dimension of nothingness. This is the seventh plane of consciousness. 

These\marginnote{8.1} are the seven planes of consciousness.” 

%
\section*{{\suttatitleacronym AN 7.45}{\suttatitletranslation Prerequisites for Immersion }{\suttatitleroot Samādhiparikkhārasutta}}
\addcontentsline{toc}{section}{\tocacronym{AN 7.45} \toctranslation{Prerequisites for Immersion } \tocroot{Samādhiparikkhārasutta}}
\markboth{Prerequisites for Immersion }{Samādhiparikkhārasutta}
\extramarks{AN 7.45}{AN 7.45}

“Mendicants,\marginnote{1.1} there are these seven prerequisites for immersion. What seven? Right view, right thought, right speech, right action, right livelihood, right effort, and right mindfulness. Unification of mind with these seven factors as prerequisites is called noble right immersion ‘with its vital conditions’ and ‘with its prerequisites’.” 

%
\section*{{\suttatitleacronym AN 7.46}{\suttatitletranslation Fires (1st) }{\suttatitleroot Paṭhamaaggisutta}}
\addcontentsline{toc}{section}{\tocacronym{AN 7.46} \toctranslation{Fires (1st) } \tocroot{Paṭhamaaggisutta}}
\markboth{Fires (1st) }{Paṭhamaaggisutta}
\extramarks{AN 7.46}{AN 7.46}

“Mendicants,\marginnote{1.1} there are these seven fires. What seven? The fires of greed, hate, delusion. The fire of those worthy of offerings dedicated to the gods. A householder’s fire. The fire of those worthy of a religious donation. And a wood fire. These are the seven fires.” 

%
\section*{{\suttatitleacronym AN 7.47}{\suttatitletranslation Fires (2nd) }{\suttatitleroot Dutiyaaggisutta}}
\addcontentsline{toc}{section}{\tocacronym{AN 7.47} \toctranslation{Fires (2nd) } \tocroot{Dutiyaaggisutta}}
\markboth{Fires (2nd) }{Dutiyaaggisutta}
\extramarks{AN 7.47}{AN 7.47}

Now\marginnote{1.1} at that time the brahmin \textsanskrit{Uggatasarīra} had prepared a large sacrifice. Bulls, bullocks, heifers, goats and rams—five hundred of each—had been led to the post for the sacrifice. 

Then\marginnote{1.3} the brahmin \textsanskrit{Uggatasarīra} went up to the Buddha, and exchanged greetings with him. When the greetings and polite conversation were over, he sat down to one side and said to the Buddha, “Master Gotama, I have heard that kindling the sacrificial fire and raising the sacrificial post is very fruitful and beneficial.” 

“I’ve\marginnote{2.2} also heard this, brahmin.” 

For\marginnote{2.3} a second time … and third time \textsanskrit{Uggatasarīra} said to the Buddha, “Master Gotama, I have heard that kindling the sacrificial fire and raising the sacrificial post is very fruitful and beneficial.” 

“I’ve\marginnote{2.6} also heard this, brahmin.” 

“Then\marginnote{2.7} Master Gotama and I are in total agreement in this matter.” 

When\marginnote{3.1} he said this, Venerable Ānanda said to \textsanskrit{Uggatasarīra}, “Brahmin, you shouldn’t ask the Buddha in this way. You should ask in this way: ‘Sir, I want to kindle the sacrificial fire and raise the sacrificial post. May the Buddha please advise and instruct me. It will be for my lasting welfare and happiness.’” 

Then\marginnote{4.1} \textsanskrit{Uggatasarīra} said to the Buddha, “Sir, I want to kindle the sacrificial fire and raise the sacrificial post. May Master Gotama please advise and instruct me. It will be for my lasting welfare and happiness.” 

“Even\marginnote{5.1} before kindling the sacrificial fire and raising the sacrificial post, one raises three unskillful knives which ripen and result in suffering. What three? The knives of the body, speech, and mind. Even before kindling the sacrificial fire and raising the sacrificial post one gives rise to the thought: ‘May this many bulls, bullocks, heifers, goats, and rams be slaughtered for the sacrifice!’ Thinking, ‘May I make merit’, one makes bad karma. Thinking, ‘May I do good’, one does bad. Thinking, ‘May I seek the path to a good rebirth’, one seeks the path to a bad rebirth. Even before kindling the sacrificial fire and raising the sacrificial post one raises this first unskillful mental knife which ripens and results in suffering. 

Furthermore,\marginnote{6.1} even before kindling the sacrificial fire and raising the sacrificial post, one says such things as: ‘May this many bulls, bullocks, heifers, goats, and rams be slaughtered for the sacrifice!’ Thinking, ‘May I make merit’, one makes bad karma. Thinking, ‘May I do good’, one does bad. Thinking, ‘May I seek the path to a good rebirth’, one seeks the path to a bad rebirth. Even before kindling the sacrificial fire and raising the sacrificial post one raises this second unskillful verbal knife which ripens and results in suffering. 

Furthermore,\marginnote{7.1} even before kindling the sacrificial fire and raising the sacrificial post one first personally undertakes preparations for the sacrificial slaughter of bulls, bullocks, heifers, goats, and rams. Thinking, ‘May I make merit’, one makes bad karma. Thinking, ‘May I do good’, one does bad. Thinking, ‘May I seek the path to a good rebirth’, one seeks the path to a bad rebirth. Even before kindling the sacrificial fire and raising the sacrificial post, one raises this third unskillful bodily knife which ripens and results in suffering. Even before kindling the sacrificial fire and raising the sacrificial post, one raises these three unskillful knives which ripen and result in suffering. 

Brahmin,\marginnote{8.1} these three fires should be given up and rejected, not cultivated. What three? The fires of greed, hate, and delusion. 

And\marginnote{9.1} why should the fire of greed be given up and rejected, not cultivated? A greedy person does bad things by way of body, speech, and mind. When their body breaks up, after death, they’re reborn in a place of loss, a bad place, the underworld, hell. That’s why the fire of greed should be given up and rejected, not cultivated. 

And\marginnote{10.1} why should the fire of hate be given up and rejected, not cultivated? A hateful person does bad things by way of body, speech, and mind. When their body breaks up, after death, they’re reborn in a place of loss, a bad place, the underworld, hell. That’s why the fire of hate should be given up and rejected, not cultivated. 

And\marginnote{11.1} why should the fire of delusion be given up and rejected, not cultivated? A deluded person does bad things by way of body, speech, and mind. When their body breaks up, after death, they’re reborn in a place of loss, a bad place, the underworld, hell. That’s why the fire of delusion should be given up and rejected, not cultivated. These three fires should be given up and rejected, not cultivated. 

Brahmin,\marginnote{12.1} you should properly and happily take care of three fires, honoring, respecting, esteeming, and venerating them. What three? The fire of those worthy of offerings dedicated to the gods. The fire of a householder. And the fire of those worthy of a religious donation. 

And\marginnote{13.1} what is the fire of those worthy of offerings dedicated to the gods? Your mother and father are called the fire of those worthy of offerings dedicated to the gods. Why is that? Since it is from them that you’ve been incubated and produced. So you should properly and happily take care of this fire, honoring, respecting, esteeming, and venerating it. 

And\marginnote{14.1} what is the fire of a householder? Your children, partners, bondservants, workers, and staff are called a householder’s fire. So you should properly and happily take care of this fire, honoring, respecting, esteeming, and venerating it. 

And\marginnote{15.1} what is the fire of those worthy of a religious donation? The ascetics and brahmins who avoid intoxication and negligence, are settled in patience and gentleness, and who tame, calm, and extinguish themselves are called the fire of those worthy of a religious donation. So you should properly and happily take care of this fire, honoring, respecting, esteeming, and venerating it. You should properly and happily take care of these three fires, honoring, respecting, esteeming, and venerating them. 

But\marginnote{16.1} the wood fire, brahmin, should, from time to time, be fanned, watched over with equanimity, extinguished, or put aside.” 

When\marginnote{17.1} he said this, the brahmin \textsanskrit{Uggatasarīra} said to the Buddha, “Excellent, Master Gotama! Excellent! … From this day forth, may Master Gotama remember me as a lay follower who has gone for refuge for life. Master Gotama, I now set free these five hundred bulls, five hundred bullocks, five hundred heifers, five hundred goats, and five hundred rams. I give them life! May they eat grass and drink cool water and enjoy a cool breeze!” 

%
\section*{{\suttatitleacronym AN 7.48}{\suttatitletranslation Perceptions in Brief }{\suttatitleroot Paṭhamasaññāsutta}}
\addcontentsline{toc}{section}{\tocacronym{AN 7.48} \toctranslation{Perceptions in Brief } \tocroot{Paṭhamasaññāsutta}}
\markboth{Perceptions in Brief }{Paṭhamasaññāsutta}
\extramarks{AN 7.48}{AN 7.48}

“Mendicants,\marginnote{1.1} these seven perceptions, when developed and cultivated, are very fruitful and beneficial. They culminate in the deathless and end with the deathless. 

What\marginnote{2.1} seven? The perceptions of ugliness, death, repulsiveness of food, dissatisfaction with the whole world, impermanence, suffering in impermanence, and not-self in suffering. These seven perceptions, when developed and cultivated, are very fruitful and beneficial. They culminate in the deathless and end with the deathless.” 

%
\section*{{\suttatitleacronym AN 7.49}{\suttatitletranslation Perceptions in Detail }{\suttatitleroot Dutiyasaññāsutta}}
\addcontentsline{toc}{section}{\tocacronym{AN 7.49} \toctranslation{Perceptions in Detail } \tocroot{Dutiyasaññāsutta}}
\markboth{Perceptions in Detail }{Dutiyasaññāsutta}
\extramarks{AN 7.49}{AN 7.49}

“Mendicants,\marginnote{1.1} these seven perceptions, when developed and cultivated, are very fruitful and beneficial. They culminate in the deathless and end with the deathless. What seven? The perceptions of ugliness, death, repulsiveness of food, dissatisfaction with the whole world, impermanence, suffering in impermanence, and not-self in suffering. These seven perceptions, when developed and cultivated, are very fruitful and beneficial. They culminate in the deathless and end with the deathless. 

‘When\marginnote{2.1} the perception of ugliness is developed and cultivated it’s very fruitful and beneficial. It culminates in the deathless and ends with the deathless.’ That’s what I said, but why did I say it? When a mendicant often meditates with a mind reinforced with the perception of ugliness, their mind draws back from sexual intercourse. They shrink away, turn aside, and don’t get drawn into it. And either equanimity or revulsion become stabilized. It’s like a chicken’s feather or a scrap of sinew thrown in a fire. It shrivels up, shrinks, rolls up, and doesn’t stretch out. In the same way, when a mendicant often meditates with a mind reinforced with the perception of ugliness, their mind draws back from sexual intercourse. … 

If\marginnote{3.1} a mendicant often meditates with a mind reinforced with the perception of ugliness, but their mind is drawn to sexual intercourse, and not repulsed, they should know: ‘My perception of ugliness is undeveloped. I don’t have any distinction higher than before. I haven’t attained a fruit of development.’ In this way they are aware of the situation. But if a mendicant often meditates with a mind reinforced with the perception of ugliness, their mind draws back from sexual intercourse … they should know: ‘My perception of ugliness is well developed. I have realized a distinction higher than before. I have attained a fruit of development.’ In this way they are aware of the situation. ‘When the perception of ugliness is developed and cultivated it’s very fruitful and beneficial. It culminates in the deathless and ends with the deathless.’ That’s what I said, and this is why I said it. 

‘When\marginnote{4.1} the perception of death is developed and cultivated it’s very fruitful and beneficial. It culminates in the deathless and ends with the deathless.’ That’s what I said, but why did I say it? When a mendicant often meditates with a mind reinforced with the perception of death, their mind draws back from desire to be reborn. … That’s what I said, and this is why I said it. 

‘When\marginnote{6.1} the perception of the repulsiveness of food is developed and cultivated it’s very fruitful and beneficial. It culminates in the deathless and ends with the deathless.’ That’s what I said, but why did I say it? When a mendicant often meditates with a mind reinforced with the perception of the repulsiveness of food, their mind draws back from craving for tastes. … That’s what I said, and this is why I said it. 

‘When\marginnote{8.1} the perception of dissatisfaction with the whole world is developed and cultivated it’s very fruitful and beneficial. It culminates in the deathless and ends with the deathless.’ That’s what I said, but why did I say it? When a mendicant often meditates with a mind reinforced with the perception of dissatisfaction with the whole world, their mind draws back from the world’s shiny things. … That’s what I said, and this is why I said it. 

‘When\marginnote{10.1} the perception of impermanence is developed and cultivated it’s very fruitful and beneficial. It culminates in the deathless and ends with the deathless.’ That’s what I said, but why did I say it? When a mendicant often meditates with a mind reinforced with the perception of impermanence, their mind draws back from material possessions, honors, and fame. … That’s what I said, and this is why I said it. 

‘When\marginnote{12.1} the perception of suffering in impermanence is developed and cultivated it’s very fruitful and beneficial. It culminates in the deathless and ends with the deathless.’ That’s what I said, but why did I say it? When a mendicant often meditates with a mind reinforced with the perception of suffering in impermanence, they establish a keen perception of the danger of sloth, laziness, slackness, negligence, lack of commitment, and failure to review, like a killer with a drawn sword. … That’s what I said, and this is why I said it. 

‘When\marginnote{14.1} the perception of not-self in suffering is developed and cultivated it’s very fruitful and beneficial. It culminates in the deathless and ends with the deathless.’ That’s what I said, but why did I say it? When a mendicant often meditates with a mind reinforced with the perception of not-self in suffering, their mind is rid of ego, possessiveness, and conceit for this conscious body and all external stimuli. It has gone beyond discrimination, and is peaceful and well freed. 

If\marginnote{15.1} a mendicant often meditates with a mind reinforced with the perception of not-self in suffering, but their mind is not rid of  ego, possessiveness, and conceit for this conscious body and all external stimuli; nor has it gone beyond discrimination, and is not peaceful or well freed, they should know: ‘My perception of not-self in suffering is undeveloped. I don’t have any distinction higher than before. I haven’t attained a fruit of development.’ In this way they are aware of the situation. 

But\marginnote{16.1} if a mendicant often meditates with a mind reinforced with the perception of not-self in suffering, and their mind is rid of  ego, possessiveness, and conceit for this conscious body and all external stimuli; and it has gone beyond discrimination, and is peaceful and well freed, they should know: ‘My perception of not-self in suffering is well developed. I have realized a distinction higher than before. I have attained a fruit of development.’ In this way they are aware of the situation. ‘When the perception of not-self in suffering is developed and cultivated it’s very fruitful and beneficial. It culminates in the deathless and ends with the deathless.’ That’s what I said, and this is why I said it. 

These\marginnote{17.1} seven perceptions, when developed and cultivated, are very fruitful and beneficial. They culminate in the deathless and end with the deathless.” 

%
\section*{{\suttatitleacronym AN 7.50}{\suttatitletranslation Sex }{\suttatitleroot Methunasutta}}
\addcontentsline{toc}{section}{\tocacronym{AN 7.50} \toctranslation{Sex } \tocroot{Methunasutta}}
\markboth{Sex }{Methunasutta}
\extramarks{AN 7.50}{AN 7.50}

Then\marginnote{1.1} the brahmin \textsanskrit{Jāṇussoṇi} went up to the Buddha, and exchanged greetings with him. When the greetings and polite conversation were over, he sat down to one side and said to the Buddha, “Does Master Gotama claim to be celibate?” 

“Brahmin,\marginnote{1.4} if anyone should be rightly said to live the celibate life unbroken, impeccable, spotless, and unmarred, full and pure, it’s me.” 

“But\marginnote{1.7} what, Master Gotama, is a break, taint, stain, or mar in celibacy?” 

“Firstly,\marginnote{2.1} an ascetic or brahmin who claims to be perfectly celibate does not mutually engage in sex with a female. However, they consent to being anointed, massaged, bathed, and rubbed by a female. They enjoy it and like it and find it satisfying. This is a break, taint, stain, or mar in celibacy. This is called one who lives the celibate life impurely, tied to the fetter of sex. They’re not freed from rebirth, old age, death, sorrow, lamentation, pain, sadness, and distress. They’re not freed from suffering, I say. 

Furthermore,\marginnote{3.1} an ascetic or brahmin who claims to be perfectly celibate does not mutually engage in sex with a female. Nor do they consent to massage and bathing. However, they giggle and play and have fun with females. … 

they\marginnote{4.1} gaze into a female’s eyes. … 

they\marginnote{5.1} listen through a wall or rampart to the sound of females laughing or chatting or singing or crying. … 

they\marginnote{6.1} recall when they used to laugh, chat, and have fun with females … 

they\marginnote{7.1} see a householder or their child amusing themselves, supplied and provided with the five kinds of sensual stimulation. … 

They\marginnote{8.1} don’t see a householder or their child amusing themselves, supplied and provided with the five kinds of sensual stimulation. However, they live the celibate life wishing to be reborn in one of the orders of gods. They think: ‘By this precept or observance or mortification or spiritual life, may I become one of the gods!’ They enjoy it and like it and find it satisfying. This is a break, taint, stain, or mar in celibacy. This is called one who lives the celibate life impurely, tied to the fetter of sex. They’re not free from rebirth, old age, death, sorrow, lamentation, pain, sadness, and distress. They’re not free from suffering, I say. 

As\marginnote{9.1} long as I saw that these seven sexual fetters—or even one of them—had not been given up in me, I didn’t announce my supreme perfect awakening in this world with its gods, \textsanskrit{Māras}, and \textsanskrit{Brahmās}, this population with its ascetics and brahmins, its gods and humans. 

But\marginnote{10.1} when I saw that these seven sexual fetters—every one of them—had been given up in me, I announced my supreme perfect awakening in this world with its gods, \textsanskrit{Māras}, and \textsanskrit{Brahmās}, this population with its ascetics and brahmins, its gods and humans. Knowledge and vision arose in me: ‘My freedom is unshakable; this is my last rebirth; now there’ll be no more future lives.’” 

When\marginnote{11.1} he said this, the brahmin \textsanskrit{Jāṇussoṇi} said to the Buddha, “Excellent, Master Gotama! Excellent! … From this day forth, may Master Gotama remember me as a lay follower who has gone for refuge for life.” 

%
\section*{{\suttatitleacronym AN 7.51}{\suttatitletranslation Bound and Unbound }{\suttatitleroot Saṁyogasutta}}
\addcontentsline{toc}{section}{\tocacronym{AN 7.51} \toctranslation{Bound and Unbound } \tocroot{Saṁyogasutta}}
\markboth{Bound and Unbound }{Saṁyogasutta}
\extramarks{AN 7.51}{AN 7.51}

“Mendicants,\marginnote{1.1} I will teach you an exposition of the teaching on the bound and the unbound. Listen and pay close attention, I will speak. … And what is the exposition of the teaching on the bound and the unbound? 

A\marginnote{2.1} woman focuses on her own femininity: her feminine moves, feminine appearance, feminine ways, feminine desires, feminine voice, and feminine adornment. She’s stimulated by this and takes pleasure in it. So she focuses on the masculinity of others: masculine moves, masculine appearance, masculine ways, masculine desires, masculine voice, and masculine adornment. She’s stimulated by this and takes pleasure in it. So she desires to bond with another. And she desires the pleasure and happiness that comes from such a bond. Sentient beings who are attached to their femininity are bound to men. This is how a woman does not transcend her femininity. 

A\marginnote{3.1} man focuses on his own masculinity: his masculine moves, masculine appearance, masculine ways, masculine desires, masculine voice, and masculine adornment. He’s stimulated by this and takes pleasure in it. So he focuses on the femininity of others: feminine moves, feminine appearance, feminine ways, feminine desires, feminine voice, and feminine adornment. He’s stimulated by this and takes pleasure in it. So he desires to bond with another. And he desires the pleasure and happiness that comes from such a bond. Sentient beings who are attached to their masculinity are bound to women. This is how a man does not transcend his masculinity. This is how one is bound. 

And\marginnote{4.1} how does one become unbound? A woman doesn’t focus on her own femininity: her feminine moves, feminine appearance, feminine ways, feminine desires, feminine voice, and feminine adornment. She isn’t stimulated by this and takes no pleasure in it. So she doesn’t focus on the masculinity of others: masculine moves, masculine appearance, masculine ways, masculine desires, masculine voice, and masculine adornment. She isn’t stimulated by this and takes no pleasure in it. So she doesn’t desire to bond with another. Nor does she desire the pleasure and happiness that comes from such a bond. Sentient beings who are not attached to their femininity are not bound to men. This is how a woman transcends her femininity. 

A\marginnote{5.1} man doesn’t focus on his own masculinity: masculine moves, masculine appearance, masculine ways, masculine desires, masculine voice, and masculine adornment. He isn’t stimulated by this and takes no pleasure in it. So he doesn’t focus on the femininity of others: feminine moves, feminine appearance, feminine ways, feminine desires, feminine voice, and feminine adornment. He isn’t stimulated by this and takes no pleasure in it. So he doesn’t desire to bond with another. Nor does he desire the pleasure and happiness that comes from such a bond. Sentient beings who are not attached to their masculinity are not bound to women. This is how a man transcends his masculinity. This is how one is unbound. This is the exposition of the teaching on the bound and the unbound.” 

%
\section*{{\suttatitleacronym AN 7.52}{\suttatitletranslation A Very Fruitful Gift }{\suttatitleroot Dānamahapphalasutta}}
\addcontentsline{toc}{section}{\tocacronym{AN 7.52} \toctranslation{A Very Fruitful Gift } \tocroot{Dānamahapphalasutta}}
\markboth{A Very Fruitful Gift }{Dānamahapphalasutta}
\extramarks{AN 7.52}{AN 7.52}

At\marginnote{1.1} one time the Buddha was staying near \textsanskrit{Campā} on the banks of the \textsanskrit{Gaggarā} Lotus Pond. 

Then\marginnote{1.2} several lay followers of \textsanskrit{Campā} went to Venerable \textsanskrit{Sāriputta}, bowed, sat down to one side, and said to him, “Sir, it’s been a long time since we’ve heard a Dhamma talk from the Buddha. It would be good if we got to hear a Dhamma talk from the Buddha.” 

“Well\marginnote{1.5} then, reverends, come on the next sabbath day. Hopefully you’ll get to hear a Dhamma talk from the Buddha.” 

“Yes,\marginnote{1.7} sir” they replied. Then they rose from their seats, bowed to \textsanskrit{Sāriputta}, and respectfully circled him before leaving. 

Then\marginnote{2.1} on the next sabbath the lay followers of \textsanskrit{Campā} went to Venerable \textsanskrit{Sāriputta}, bowed, and stood to one side. Then they went together with \textsanskrit{Sāriputta} to the Buddha, bowed, and sat down to one side. \textsanskrit{Sāriputta} said to the Buddha: 

“Sir,\marginnote{3.1} could it be that someone gives a gift and it is not very fruitful or beneficial, while someone else gives exactly the same gift and it is very fruitful and beneficial?” 

“Indeed\marginnote{3.3} it could, \textsanskrit{Sāriputta}.” 

“Sir,\marginnote{3.5} what is the cause, what is the reason for this?” 

“\textsanskrit{Sāriputta},\marginnote{4.1} take the case of a someone who gives a gift as an investment, their mind tied to it, expecting to keep it, thinking ‘I’ll enjoy this in my next life’. They give to ascetics or brahmins such things as food, drink, clothing, vehicles; garlands, fragrance, and makeup; and bed, house, and lighting. What do you think, \textsanskrit{Sāriputta}, don’t some people give gifts in this way?” 

“Yes,\marginnote{4.4} sir.” 

“\textsanskrit{Sāriputta},\marginnote{5.1} someone who gives a gift as an investment, when their body breaks up, after death, is reborn in the company of the gods of the Four Great Kings. When that deed, success, fame, and sovereignty is spent they return to this state of existence. 

Next,\marginnote{6.1} take the case of a someone who gives a gift not as an investment, their mind not tied to it, not expecting to keep it, and not thinking, ‘I’ll enjoy this in my next life’. But they give a gift thinking, ‘It’s good to give’ … 

They\marginnote{7.1} give a gift thinking, ‘Giving was practiced by my father and my father’s father. It would not be right for me to abandon this family tradition.’ … 

They\marginnote{8.1} give a gift thinking, ‘I cook, they don’t. It wouldn’t be right for me to not give to them.’ … 

They\marginnote{9.1} give a gift thinking, ‘The ancient brahmin hermits were \textsanskrit{Aṭṭhaka}, \textsanskrit{Vāmaka}, \textsanskrit{Vāmadeva}, \textsanskrit{Vessāmitta}, Yamadaggi, \textsanskrit{Aṅgīrasa}, \textsanskrit{Bhāradvāja}, \textsanskrit{Vāseṭṭha}, Kassapa, and Bhagu. Just as they performed great sacrifices, I will share a gift.’ … 

They\marginnote{10.1} give a gift thinking, ‘When giving this gift my mind becomes clear, and I become happy and joyful.’ … 

They\marginnote{11.1} don’t give a gift thinking, ‘When giving this gift my mind becomes clear, and I become happy and joyful.’ But they give a gift thinking, ‘This is an adornment and requisite for the mind.’ They give to ascetics or brahmins such things as food, drink, clothing, vehicles; garlands, fragrance, and makeup; and bed, house, and lighting. What do you think, \textsanskrit{Sāriputta}, don’t some people give gifts in this way?” 

“Yes,\marginnote{11.5} sir.” 

“\textsanskrit{Sāriputta},\marginnote{12.1} someone who gives gifts, not for any other reason, but thinking, ‘This is an adornment and requisite for the mind’, when their body breaks up, after death, is reborn among the gods of \textsanskrit{Brahmā}’s Host. When that deed, success, fame, and sovereignty is spent they are a non-returner; they do not return to this state of existence. 

This\marginnote{13.1} is the cause, this is the reason why someone gives a gift and it is not very fruitful or beneficial, while someone else gives exactly the same gift and it is very fruitful and beneficial.” 

%
\section*{{\suttatitleacronym AN 7.53}{\suttatitletranslation Nanda’s Mother }{\suttatitleroot Nandamātāsutta}}
\addcontentsline{toc}{section}{\tocacronym{AN 7.53} \toctranslation{Nanda’s Mother } \tocroot{Nandamātāsutta}}
\markboth{Nanda’s Mother }{Nandamātāsutta}
\extramarks{AN 7.53}{AN 7.53}

\scevam{So\marginnote{1.1} I have heard. }At one time the venerables \textsanskrit{Sāriputta} and \textsanskrit{Mahāmoggallāna} were wandering in the Southern Hills together with a large \textsanskrit{Saṅgha} of mendicants. Now at that time the laywoman \textsanskrit{Veḷukaṇṭakī}, Nanda’s mother, rose at the crack of dawn and recited the verses of “The Way to the Far Shore”. 

And\marginnote{2.1} at that time the great king \textsanskrit{Vessavaṇa} was on his way from the north to the south on some business. He heard Nanda’s Mother reciting, and stood waiting for her to finish. 

Then\marginnote{3.1} when her recital was over she fell silent. Then, knowing she had finished, \textsanskrit{Vessavaṇa} applauded, saying, “Good, sister! Good, sister!” 

“But\marginnote{3.4} who might you be, my dear?” 

“Sister,\marginnote{3.5} I am your brother \textsanskrit{Vessavaṇa}, the great king.” 

“Good,\marginnote{3.6} my dear! Then may my recital of the teaching be my offering to you as my guest.” 

‘Good,\marginnote{3.7} sister! And let this also be your offering to me as your guest. Tomorrow, the mendicant \textsanskrit{Saṅgha} headed by \textsanskrit{Sāriputta} and \textsanskrit{Moggallāna} will arrive at \textsanskrit{Veḷukaṇṭa} before breakfast. When you’ve served the \textsanskrit{Saṅgha}, please dedicate the religious donation to me. Then that will also be your offering to me as your guest.” 

And\marginnote{4.1} when the night had passed the lay woman Nanda’s Mother had a variety of delicious foods prepared in her own home. Then the \textsanskrit{Saṅgha} of mendicants headed by \textsanskrit{Sāriputta} and \textsanskrit{Moggallāna} arrived at \textsanskrit{Veḷukaṇṭa}. Then Nanda’s Mother addressed a man, “Please, mister, go to the monastery and announce the time to the \textsanskrit{Saṅgha}, saying: ‘Sirs, it’s time. The meal is ready in the home of the lady Nanda’s Mother.’” 

“Yes,\marginnote{4.6} Ma’am,” that man replied, and he did as she said. 

And\marginnote{4.8} then the \textsanskrit{Saṅgha} of mendicants headed by \textsanskrit{Sāriputta} and \textsanskrit{Moggallāna} robed up in the morning and, taking their bowls and robes, went to the home of Nanda’s Mother, where they sat on the seats spread out. Then Nanda’s Mother served and satisfied them with her own hands with a variety of delicious foods. 

When\marginnote{5.1} \textsanskrit{Sāriputta} had eaten and washed his hand and bowl, Nanda’s Mother sat down to one side. \textsanskrit{Sāriputta} said to her, “Nanda’s Mother, who told you that the \textsanskrit{Saṅgha} of mendicants was about to arrive?” 

“Sir,\marginnote{6.1} last night I rose at the crack of dawn and recited the verses of ‘The Way to the Far Shore’, and then I fell silent. Then the great king \textsanskrit{Vessavaṇa}, knowing I had finished, applauded me, ‘Good, sister! Good, sister!’ 

I\marginnote{6.4} asked: ‘But who might you be, my dear?’ 

‘Sister,\marginnote{6.5} I am your brother \textsanskrit{Vessavaṇa}, the great king.’ 

‘Good,\marginnote{6.6} my dear! Then may my recital of the teaching be my offering to you as my guest.’ 

‘Good,\marginnote{6.7} sister! And let this also be your offering to me as your guest. Tomorrow, the mendicant \textsanskrit{Saṅgha} headed by \textsanskrit{Sāriputta} and \textsanskrit{Moggallāna} will arrive at \textsanskrit{Veḷukaṇṭa} before breakfast. When you’ve served the \textsanskrit{Saṅgha}, please dedicate the religious donation to me. Then that will also be your offering to me as your guest.’ 

And\marginnote{6.10} so, sir, may the merit and the growth of merit in this gift be for the happiness of the great king \textsanskrit{Vessavaṇa}.” 

“It’s\marginnote{7.1} incredible, Nanda’s Mother, it’s amazing that you converse face to face with a mighty and illustrious god like the great king \textsanskrit{Vessavaṇa}.” 

“Sir,\marginnote{8.1} this is not my only incredible and amazing quality; there is another. I had an only son called Nanda who I loved dearly. The rulers forcibly abducted him on some pretext and had him executed. But I can’t recall getting upset when my boy was under arrest or being arrested, imprisoned or being put in prison, killed or being killed.” 

“It’s\marginnote{8.6} incredible, Nanda’s Mother, it’s amazing that you purify even the arising of a thought.” 

“Sir,\marginnote{9.1} this is not my only incredible and amazing quality; there is another. When my husband passed away he was reborn in one of the realms of spirits. He revealed to me his previous life-form. But I can’t recall getting upset on that account.” 

“It’s\marginnote{9.6} incredible, Nanda’s Mother, it’s amazing that you purify even the arising of a thought.” 

“Sir,\marginnote{10.1} this is not my only incredible and amazing quality; there is another. Ever since we were both young, and I was given in marriage to my husband, I can’t recall betraying him even in thought, still less in deed.” 

“It’s\marginnote{10.4} incredible, Nanda’s Mother, it’s amazing that you purify even the arising of a thought.” 

“Sir,\marginnote{11.1} this is not my only incredible and amazing quality; there is another. Ever since I declared myself a lay follower, I can’t recall deliberately breaking any precept.” 

“It’s\marginnote{11.4} incredible, Nanda’s Mother, it’s amazing!” 

“Sir,\marginnote{12.1} this is not my only incredible and amazing quality; there is another. Whenever I want, quite secluded from sensual pleasures, secluded from unskillful qualities, I enter and remain in the first absorption, which has the rapture and bliss born of seclusion, while placing the mind and keeping it connected. As the placing of the mind and keeping it connected are stilled, I enter and remain in the second absorption, which has the rapture and bliss born of immersion, with internal clarity and confidence, and unified mind, without placing the mind and keeping it connected. And with the fading away of rapture, I enter and remain in the third absorption, where I meditate with equanimity, mindful and aware, personally experiencing the bliss of which the noble ones declare, ‘Equanimous and mindful, one meditates in bliss.’ With the giving up of pleasure and pain, and the ending of former happiness and sadness, I enter and remain in the fourth absorption, without pleasure or pain, with pure equanimity and mindfulness.” 

“It’s\marginnote{12.7} incredible, Nanda’s Mother, it’s amazing!” 

“Sir,\marginnote{13.1} this is not my only incredible and amazing quality; there is another. Of the five lower fetters taught by the Buddha, I don’t see any that I haven’t given up.” 

“It’s\marginnote{13.4} incredible, Nanda’s Mother, it’s amazing!” 

Then\marginnote{14.1} Venerable \textsanskrit{Sāriputta} educated, encouraged, fired up, and inspired Nanda’s Mother with a Dhamma talk, after which he got up from his seat and left. 

%
\addtocontents{toc}{\let\protect\contentsline\protect\nopagecontentsline}
\pannasa{The Second Fifty }
\addcontentsline{toc}{pannasa}{The Second Fifty }
\markboth{}{}
\addtocontents{toc}{\let\protect\contentsline\protect\oldcontentsline}

%
\addtocontents{toc}{\let\protect\contentsline\protect\nopagecontentsline}
\chapter*{The Chapter on the Undeclared Points }
\addcontentsline{toc}{chapter}{\tocchapterline{The Chapter on the Undeclared Points }}
\addtocontents{toc}{\let\protect\contentsline\protect\oldcontentsline}

%
\section*{{\suttatitleacronym AN 7.54}{\suttatitletranslation The Undeclared Points }{\suttatitleroot Abyākatasutta}}
\addcontentsline{toc}{section}{\tocacronym{AN 7.54} \toctranslation{The Undeclared Points } \tocroot{Abyākatasutta}}
\markboth{The Undeclared Points }{Abyākatasutta}
\extramarks{AN 7.54}{AN 7.54}

Then\marginnote{1.1} a mendicant went up to the Buddha, bowed, sat down to one side, and said to him: 

“Sir,\marginnote{1.2} what is the cause, what is the reason why a learned noble disciple has no doubts regarding the undeclared points?” 

“Mendicant,\marginnote{2.1} it’s due to the cessation of views that a learned noble disciple has no doubts regarding the undeclared points. ‘A Realized One exists after death’: this is a misconception. ‘A Realized One doesn’t exist after death’: this is a misconception. ‘A Realized One both exists and doesn’t exist after death’: this is a misconception. ‘A Realized One neither exists nor doesn’t exist after death’: this is a misconception. An unlearned ordinary person doesn’t understand views, their origin, their cessation, or the practice that leads to their cessation. And so their views grow. They’re not freed from rebirth, old age, and death, from sorrow, lamentation, pain, sadness, and distress. They’re not freed from suffering, I say. 

A\marginnote{3.1} learned noble disciple does understand views, their origin, their cessation, and the practice that leads to their cessation. And so their views cease. They’re freed from rebirth, old age, and death, from sorrow, lamentation, pain, sadness, and distress. They’re freed from suffering, I say. Knowing and seeing this, a learned noble disciple does not answer: ‘A Realized One exists after death’, ‘a Realized One doesn’t exist after death’, ‘a Realized One both exists and doesn’t exist after death’, ‘a Realized One neither exists nor doesn’t exist after death.’ Knowing and seeing this, a learned noble disciple does not declare the undeclared points. Knowing and seeing this, a learned noble disciple doesn’t shake, tremble, quake, or get nervous regarding the undeclared points. 

‘A\marginnote{4.1} Realized One exists after death’: this is just about craving. … it’s just about perception … it’s an identification … it’s a proliferation … it’s just about grasping … ‘A Realized One exists after death’: this is a regret. ‘A Realized One doesn’t exist after death’: this is a regret. ‘A Realized One both exists and doesn’t exist after death’: this is a regret. ‘A Realized One neither exists nor doesn’t exist after death’: this is a regret. An unlearned ordinary person doesn’t understand regrets, their origin, their cessation, or the practice that leads to their cessation. And so their regrets grow. They’re not freed from rebirth, old age, and death, from sorrow, lamentation, pain, sadness, and distress. They’re not freed from suffering, I say. 

A\marginnote{5.1} learned noble disciple does understand regrets, their origin, their cessation, and the practice that leads to their cessation. And so their regrets cease. They’re freed from rebirth, old age, and death, from sorrow, lamentation, pain, sadness, and distress. They’re freed from suffering, I say. Knowing and seeing this, a learned noble disciple does not answer: ‘A Realized One exists after death’ … ‘a Realized One neither exists nor doesn’t exist after death.’ Knowing and seeing this, a learned noble disciple does not declare the undeclared points. Knowing and seeing this, a learned noble disciple doesn’t shake, tremble, quake, or get nervous regarding the undeclared points. This is the cause, this is the reason why a learned noble disciple has no doubts regarding the undeclared points.” 

%
\section*{{\suttatitleacronym AN 7.55}{\suttatitletranslation Places People Are Reborn }{\suttatitleroot Purisagatisutta}}
\addcontentsline{toc}{section}{\tocacronym{AN 7.55} \toctranslation{Places People Are Reborn } \tocroot{Purisagatisutta}}
\markboth{Places People Are Reborn }{Purisagatisutta}
\extramarks{AN 7.55}{AN 7.55}

“Mendicants,\marginnote{1.1} I will teach you seven places people are reborn, and extinguishment by not grasping. Listen and pay close attention, I will speak.” 

“Yes,\marginnote{1.3} sir,” the mendicants replied. The Buddha said this: 

“And\marginnote{1.5} what are the seven places people are reborn? 

Take\marginnote{2.1} a mendicant who practices like this: ‘It might not be, and it might not be mine. It will not be, and it will not be mine. I am giving up what exists, what has come to be.’ They gain equanimity. They’re not attached to life, or to creating a new life. And they see with right wisdom that there is a peaceful state beyond. But they haven’t completely realized that state. They haven’t totally given up the underlying tendencies of conceit, desire to be reborn, and ignorance. With the ending of the five lower fetters they’re extinguished between one life and the next. Suppose you struck an iron pot that had been heated all day. Any spark that flew off would be extinguished. In the same way, a mendicant who practices like this … With the ending of the five lower fetters they’re extinguished between one life and the next. 

Take\marginnote{3.1} a mendicant who practices like this: ‘It might not be, and it might not be mine. It will not be, and it will not be mine. I am giving up what exists, what has come to be.’ They gain equanimity. They’re not attached to life, or to creating a new life. And they see with right wisdom that there is a peaceful state beyond. But they haven’t totally realized that state. They haven’t completely given up the underlying tendencies of conceit, desire to be reborn, and ignorance. With the ending of the five lower fetters they’re extinguished between one life and the next. Suppose you struck an iron pot that had been heated all day. Any spark that flew off and floated away would be extinguished. In the same way, a mendicant who practices like this … With the ending of the five lower fetters they’re extinguished between one life and the next. 

Take\marginnote{4.1} a mendicant who practices like this: ‘It might not be, and it might not be mine. …’ With the ending of the five lower fetters they’re extinguished between one life and the next. Suppose you struck an iron pot that had been heated all day. Any spark that flew off and floated away would be extinguished just before landing. In the same way, a mendicant who practices like this … With the ending of the five lower fetters they’re extinguished between one life and the next. 

Take\marginnote{5.1} a mendicant who practices like this: ‘It might not be, and it might not be mine. …’ With the ending of the five lower fetters they’re extinguished upon landing. Suppose you struck an iron pot that had been heated all day. Any spark that flew off and floated away would be extinguished on landing. In the same way, a mendicant who practices like this … ‘It might not be, and it might not be mine. …’ With the ending of the five lower fetters they’re extinguished upon landing. 

Take\marginnote{6.1} a mendicant who practices like this: ‘It might not be, and it might not be mine. …’ With the ending of the five lower fetters they’re extinguished without extra effort. Suppose you struck an iron pot that had been heated all day. Any spark that flew off and floated away would fall on a little heap of grass or twigs. There it would ignite a fire and produce smoke. But the fire would consume the grass or twigs and become extinguished for lack of fuel. In the same way, a mendicant who practices like this … ‘It might not be, and it might not be mine. …’ With the ending of the five lower fetters they’re extinguished without extra effort. 

Take\marginnote{7.1} a mendicant who practices like this: ‘It might not be, and it might not be mine. …’ With the ending of the five lower fetters they’re extinguished with extra effort. Suppose you struck an iron pot that had been heated all day. Any spark that flew off and floated away would fall on a large heap of grass or twigs. There it would ignite a fire and produce smoke. But the fire would consume the grass or twigs and become extinguished for lack of fuel. In the same way, a mendicant who practices like this … ‘It might not be, and it might not be mine. …’ With the ending of the five lower fetters they’re extinguished with extra effort. 

Take\marginnote{8.1} a mendicant who practices like this: ‘It might not be, and it might not be mine. It will not be, and it will not be mine. I am giving up what exists, what has come to be.’ They gain equanimity. They’re not attached to life, or to creating a new life. And they see with right wisdom that there is a peaceful state beyond. But they haven’t totally realized that state. They haven’t completely given up the underlying tendencies of conceit, desire to be reborn, and ignorance. With the ending of the five lower fetters they head upstream, going to the \textsanskrit{Akaniṭṭha} realm. Suppose you struck an iron pot that had been heated all day. Any spark that flew off and floated away would fall on a huge heap of grass or twigs. There it would ignite a fire and produce smoke. And after consuming the grass and twigs, the fire would burn up plants and trees until it reached a green field, a roadside, a cliff’s edge, a body of water, or cleared parkland, where it would be extinguished for lack of fuel. In the same way, a mendicant who practices like this … ‘It might not be, and it might not be mine. …’ With the ending of the five lower fetters they head upstream, going to the \textsanskrit{Akaniṭṭha} realm. These are the seven places people are reborn. 

And\marginnote{9.1} what is extinguishment by not grasping? Take a mendicant who practices like this: ‘It might not be, and it might not be mine. It will not be, and it will not be mine. I am giving up what exists, what has come to be.’ They gain equanimity. They’re not attached to life, or to creating a new life. And they see with right wisdom that there is a peaceful state beyond. And they have totally realized that state. They’ve completely given up the underlying tendencies of conceit, desire to be reborn, and ignorance. They’ve realized the undefiled freedom of heart and freedom by wisdom in this very life, and live having realized it with their own insight due to the ending of defilements. This is called extinguishment by not grasping. 

These\marginnote{9.8} are the seven places people are reborn, and extinguishment by not grasping.” 

%
\section*{{\suttatitleacronym AN 7.56}{\suttatitletranslation Tissa the Brahmā }{\suttatitleroot Tissabrahmāsutta}}
\addcontentsline{toc}{section}{\tocacronym{AN 7.56} \toctranslation{Tissa the Brahmā } \tocroot{Tissabrahmāsutta}}
\markboth{Tissa the Brahmā }{Tissabrahmāsutta}
\extramarks{AN 7.56}{AN 7.56}

\scevam{So\marginnote{1.1} I have heard. }At one time the Buddha was staying near \textsanskrit{Rājagaha}, on the Vulture’s Peak Mountain. 

Then,\marginnote{1.3} late at night, two glorious deities, lighting up the entire Vulture’s Peak, went up to the Buddha, bowed, and stood to one side. One deity said to him, “Sir, these nuns are freed!” 

The\marginnote{1.5} other deity said to him, “Sir, these nuns are well freed without anything left over!” 

This\marginnote{1.7} is what those deities said, and the teacher approved. Then those deities, knowing that the teacher approved, bowed and respectfully circled the Buddha, keeping him on his right, before vanishing right there. 

Then,\marginnote{2.1} when the night had passed, the Buddha told the mendicants all that had happened. 

Now\marginnote{3.1} at that time Venerable \textsanskrit{Mahāmoggallāna} was sitting not far from the Buddha. He thought, “Which gods know whether a person has anything left over or not?” 

Now,\marginnote{3.5} at that time a monk called Tissa had recently passed away and been reborn in a \textsanskrit{Brahmā} realm. There they knew that Tissa the \textsanskrit{Brahmā} was very mighty and powerful. 

And\marginnote{4.1} then Venerable \textsanskrit{Mahāmoggallāna}, as easily as a strong person would extend or contract their arm, vanished from the Vulture’s Peak and reappeared in that \textsanskrit{Brahmā} realm. 

Tissa\marginnote{4.2} saw \textsanskrit{Moggallāna} coming off in the distance, and said to him, “Come, my good \textsanskrit{Moggallāna}! Welcome, my good \textsanskrit{Moggallāna}! It’s been a long time since you took the opportunity to come here. Sit, my good \textsanskrit{Moggallāna}, this seat is for you.” 

\textsanskrit{Moggallāna}\marginnote{4.8} sat down on the seat spread out. Then Tissa bowed to \textsanskrit{Moggallāna} and sat to one side. 

\textsanskrit{Moggallāna}\marginnote{4.10} said to him, “Which gods know whether a person has anything left over or not?” 

“The\marginnote{4.13} gods of \textsanskrit{Brahmā}’s Host know this.” 

“But\marginnote{5.1} do all of them know this?” 

“No,\marginnote{5.3} my good \textsanskrit{Moggallāna}, not all of them. 

Those\marginnote{6.1} gods of \textsanskrit{Brahmā}’s Host who are content with the lifespan of \textsanskrit{Brahmā}, with the beauty, happiness, fame, and sovereignty of \textsanskrit{Brahmā}, and who don’t truly understand any higher escape: they don’t know this. But those gods of \textsanskrit{Brahmā}’s Host who are not content with the lifespan of \textsanskrit{Brahmā}, with the beauty, happiness, fame, and sovereignty of \textsanskrit{Brahmā}, and who do truly understand a higher escape: they do know this. 

Take\marginnote{7.1} a mendicant who is freed both ways. The gods know of them: ‘This venerable is freed both ways. As long as their body remains they will be seen by gods and humans. But when their body breaks up gods and humans will see them no more.’ This too is how those gods know whether a person has anything left over or not. 

Take\marginnote{8.1} a mendicant who is freed by wisdom. The gods know of them: ‘This venerable is freed by wisdom. As long as their body remains they will be seen by gods and humans. But when their body breaks up gods and humans will see them no more.’ This too is how those gods know whether a person has anything left over or not. 

Take\marginnote{9.1} a mendicant who is a personal witness. The gods know of them: ‘This venerable is a personal witness. Hopefully this venerable will frequent appropriate lodgings, associate with good friends, and control their faculties. Then they might realize the supreme culmination of the spiritual path in this very life, and live having achieved with their own insight the goal for which gentlemen rightly go forth from the lay life to homelessness.’ This too is how those gods know whether a person has anything left over or not. 

Take\marginnote{10.1} a mendicant who is attained to view. … freed by faith … a follower of the teachings. The gods know of them: ‘This venerable is a follower of the teachings. Hopefully this venerable will frequent appropriate lodgings, associate with good friends, and control their faculties. Then they might realize the supreme culmination of the spiritual path in this very life, and live having achieved with their own insight the goal for which gentlemen rightly go forth from the lay life to homelessness.’ This too is how those gods know whether a person has anything left over or not.” 

\textsanskrit{Moggallāna}\marginnote{11.1} approved and agreed with what Tissa the \textsanskrit{Brahmā} said. Then, as easily as a strong person would extend or contract their arm, he vanished from the \textsanskrit{Brahmā} realm and reappeared on the Vulture’s Peak. Then \textsanskrit{Mahāmoggallāna} went up to the Buddha, bowed, sat down to one side, and told him what had happened. 

“But\marginnote{12.1} \textsanskrit{Moggallāna}, Tissa the \textsanskrit{Brahmā} didn’t teach the seventh person, the signless meditator.” 

“Now\marginnote{12.2} is the time, Blessed One! Now is the time, Holy One! May the Buddha teach the seventh person, the signless meditator. The mendicants will listen and remember it.” 

“Well\marginnote{12.4} then, \textsanskrit{Moggallāna}, listen and pay close attention, I will speak.” 

“Yes,\marginnote{12.5} sir,” \textsanskrit{Mahāmoggallāna} replied. The Buddha said this: 

“\textsanskrit{Moggallāna},\marginnote{13.1} take the case of a mendicant who, not focusing on any signs, enters and remains in the signless immersion of the heart. The gods know of them: ‘This venerable, not focusing on any signs, enters and remains in the signless immersion of the heart. Hopefully this venerable will frequent appropriate lodgings, associate with good friends, and control their faculties. Then they might realize the supreme culmination of the spiritual path in this very life, and live having achieved with their own insight the goal for which gentlemen rightly go forth from the lay life to homelessness.’ This too is how those gods know whether a person has anything left over or not.” 

%
\section*{{\suttatitleacronym AN 7.57}{\suttatitletranslation General Sīha }{\suttatitleroot Sīhasenāpatisutta}}
\addcontentsline{toc}{section}{\tocacronym{AN 7.57} \toctranslation{General Sīha } \tocroot{Sīhasenāpatisutta}}
\markboth{General Sīha }{Sīhasenāpatisutta}
\extramarks{AN 7.57}{AN 7.57}

\scevam{So\marginnote{1.1} I have heard. }At one time the Buddha was staying near \textsanskrit{Vesālī}, at the Great Wood, in the hall with the peaked roof. Then General \textsanskrit{Sīha} went up to the Buddha, bowed, sat down to one side, and said to him: 

“Sir,\marginnote{1.4} can you point out a fruit of giving that’s apparent in the present life?” 

“Well\marginnote{2.1} then, \textsanskrit{Sīha}, I’ll ask you about this in return, and you can answer as you like. What do you think, \textsanskrit{Sīha}? Consider two people. One is faithless, stingy, miserly, and abusive. One is a faithful donor who loves charity. Which do you think the perfected ones will show compassion for first?” 

“Why\marginnote{3.1} would the perfected ones first show compassion for the person who is faithless, stingy, miserly, and abusive? They’d show compassion first for the faithful donor who loves charity.” 

“Which\marginnote{4.1} do you think the perfected ones will first approach?” “They’d first approach the faithful donor who loves charity.” 

“Which\marginnote{5.1} do you think the perfected ones will receive alms from first?” “They’d receive alms first from the faithful donor who loves charity.” 

“Which\marginnote{6.1} do you think the perfected ones will teach the Dhamma to first?” “They’d first teach the Dhamma to the faithful donor who loves charity.” 

“Which\marginnote{7.1} do you think would get a good reputation?” “The faithful donor who loves charity would get a good reputation.” 

“Which\marginnote{8.1} do you think would enter any kind of assembly bold and assured, whether it’s an assembly of aristocrats, brahmins, householders, or ascetics?” 

“How\marginnote{8.3} could the person who is faithless, stingy, miserly, and abusive enter any kind of assembly bold and assured, whether it’s an assembly of aristocrats, brahmins, householders, or ascetics? The faithful donor who loves charity would enter any kind of assembly bold and assured, whether it’s an assembly of aristocrats, brahmins, householders, or ascetics.” 

“When\marginnote{9.1} their body breaks up, after death, which do you think would be reborn in a good place, a heavenly realm?” 

“Why\marginnote{9.3} would the person who is faithless, stingy, miserly, and abusive be reborn in a good place, a heavenly realm? The faithful donor who loves charity would, when their body breaks up, after death, be reborn in a good place, a heavenly realm. 

When\marginnote{10.1} it comes to these fruits of giving that are apparent in the present life, I don’t have to rely on faith in the Buddha, for I know them too. I’m a giver, a donor, and the perfected ones show compassion for me first. I’m a giver, and the perfected ones approach me first. I’m a giver, and the perfected ones receive alms from me first. I’m a giver, and the perfected ones teach me Dhamma first. I’m a giver, and I have this good reputation: ‘General \textsanskrit{Sīha} gives, serves, and attends on the \textsanskrit{Saṅgha}.’ I’m a giver, I enter any kind of assembly bold and assured, whether it’s an assembly of aristocrats, brahmins, householders, or ascetics. When it comes to these fruits of giving that are apparent in the present life, I don’t have to rely on faith in the Buddha, for I know them too. But when the Buddha says: ‘When a giver’s body breaks up, after death, they’re reborn in a good place, a heavenly realm.’ I don’t know this, so I have to rely on faith in the Buddha.” 

“That’s\marginnote{10.15} so true, \textsanskrit{Sīha}! That’s so true! When a giver’s body breaks up, after death, they’re reborn in a good place, a heavenly realm.” 

%
\section*{{\suttatitleacronym AN 7.58}{\suttatitletranslation Nothing to Hide }{\suttatitleroot Arakkheyyasutta}}
\addcontentsline{toc}{section}{\tocacronym{AN 7.58} \toctranslation{Nothing to Hide } \tocroot{Arakkheyyasutta}}
\markboth{Nothing to Hide }{Arakkheyyasutta}
\extramarks{AN 7.58}{AN 7.58}

“Mendicants,\marginnote{1.1} there are four areas where the Realized One has nothing to hide, and three ways he is irreproachable. What are the four areas where the Realized One has nothing to hide? 

His\marginnote{1.3} bodily behavior is pure. So the Realized One has no bodily misconduct to hide, thinking: ‘Don’t let others find this out about me!’ 

His\marginnote{2.1} verbal behavior is pure. So the Realized One has no verbal misconduct to hide, thinking: ‘Don’t let others find this out about me!’ 

His\marginnote{3.1} mental behavior is pure. So the Realized One has no mental misconduct to hide, thinking: ‘Don’t let others find this out about me!’ 

His\marginnote{4.1} livelihood is pure. So the Realized One has no wrong livelihood to hide, thinking: ‘Don’t let others find this out about me!’ 

These\marginnote{5.1} are the four areas where the Realized One has nothing to hide. 

What\marginnote{6.1} are the three ways the Realized One is irreproachable? 

The\marginnote{6.2} Realized One has explained the teaching well. I see no reason for anyone—whether ascetic, brahmin, god, \textsanskrit{Māra}, or \textsanskrit{Brahmā}, or anyone else in the world—to legitimately scold me, saying: ‘For such and such reasons you haven’t explained the teaching well.’ Since I see no such reason, I live secure, fearless, and assured. 

I\marginnote{7.1} have clearly described the practice that leads to extinguishment for my disciples. Practicing in accordance with this, my disciples realize the undefiled freedom of heart and freedom by wisdom in this very life. And they live having realized it with their own insight due to the ending of defilements. I see no reason for anyone—whether ascetic, brahmin, god, \textsanskrit{Māra}, or \textsanskrit{Brahmā}, or anyone else in the world—to legitimately scold me, saying: ‘For such and such reasons you haven’t clearly described the practice that leads to extinguishment for your disciples.’ Since I see no such reason, I live secure, fearless, and assured. 

Many\marginnote{8.1} hundreds in my assembly of disciples have realized the undefiled freedom of heart and freedom by wisdom in this very life. And they live having realized it with their own insight due to the ending of defilements. I see no reason for anyone—whether ascetic, brahmin, god, \textsanskrit{Māra}, or \textsanskrit{Brahmā}, or anyone else in the world—to legitimately scold me, saying: ‘For such and such reasons you don’t have many hundreds of disciples in your following who have realized the undefiled freedom of heart and freedom by wisdom in this very life, and who live having realized it with their own insight due to the ending of defilements.’ Since I see no such reason, I live secure, fearless, and assured. 

These\marginnote{9.1} are the three ways the Realized One is irreproachable. 

These\marginnote{10.1} are the four areas where the Realized One has nothing to hide, and the three ways he is irreproachable.” 

%
\section*{{\suttatitleacronym AN 7.59}{\suttatitletranslation With Kimbila }{\suttatitleroot Kimilasutta}}
\addcontentsline{toc}{section}{\tocacronym{AN 7.59} \toctranslation{With Kimbila } \tocroot{Kimilasutta}}
\markboth{With Kimbila }{Kimilasutta}
\extramarks{AN 7.59}{AN 7.59}

\scevam{So\marginnote{1.1} I have heard. }At one time the Buddha was staying near \textsanskrit{Kimbilā} in the Freshwater Mangrove Wood. Then Venerable Kimbila went up to the Buddha, bowed, sat down to one side, and said to him: 

“What\marginnote{1.4} is the cause, sir, what is the reason why the true teaching does not last long after the final extinguishment of the Realized One?” 

“Kimbila,\marginnote{2.1} it’s when the monks, nuns, laymen, and laywomen lack respect and reverence for the Teacher, the teaching, the \textsanskrit{Saṅgha}, the training, immersion, diligence, and hospitality after the final extinguishment of the Realized One. This is the cause, this is the reason why the true teaching does not last long after the final extinguishment of the Realized One.” 

“What\marginnote{3.1} is the cause, sir, what is the reason why the true teaching does last long after the final extinguishment of the Realized One?” 

“Kimbila,\marginnote{3.2} it’s when the monks, nuns, laymen, and laywomen maintain respect and reverence for the Teacher, the teaching, the \textsanskrit{Saṅgha}, the training, immersion, diligence, and hospitality after the final extinguishment of the Realized One. This is the cause, this is the reason why the true teaching does last long after the final extinguishment of the Realized One.” 

%
\section*{{\suttatitleacronym AN 7.60}{\suttatitletranslation Seven Qualities }{\suttatitleroot Sattadhammasutta}}
\addcontentsline{toc}{section}{\tocacronym{AN 7.60} \toctranslation{Seven Qualities } \tocroot{Sattadhammasutta}}
\markboth{Seven Qualities }{Sattadhammasutta}
\extramarks{AN 7.60}{AN 7.60}

“Mendicants,\marginnote{1.1} a mendicant with seven qualities soon realizes the supreme culmination of the spiritual path in this very life. They live having achieved with their own insight the goal for which gentlemen rightly go forth from the lay life to homelessness. What seven? It’s when a mendicant is faithful, ethical, learned, secluded, energetic, mindful, and wise. A mendicant with these seven qualities soon realizes the supreme culmination of the spiritual path in this very life. They live having achieved with their own insight the goal for which gentlemen rightly go forth from the lay life to homelessness.” 

%
\section*{{\suttatitleacronym AN 7.61}{\suttatitletranslation Nodding Off }{\suttatitleroot Pacalāyamānasutta}}
\addcontentsline{toc}{section}{\tocacronym{AN 7.61} \toctranslation{Nodding Off } \tocroot{Pacalāyamānasutta}}
\markboth{Nodding Off }{Pacalāyamānasutta}
\extramarks{AN 7.61}{AN 7.61}

\scevam{So\marginnote{1.1} I have heard. }At one time the Buddha was staying in the land of the Bhaggas on Crocodile Hill, in the deer park at \textsanskrit{Bhesakaḷā}’s Wood. 

Now\marginnote{1.3} at that time, in the land of the Magadhans near \textsanskrit{Kallavāḷamutta} Village, Venerable \textsanskrit{Mahāmoggallāna} was nodding off while meditating. The Buddha saw him with his clairvoyance that is purified and superhuman. Then, as easily as a strong person would extend or contract their arm, he vanished from the deer park at \textsanskrit{Bhesakaḷā}’s Wood in the land of the Bhaggas and reappeared in front of \textsanskrit{Mahāmoggallāna} near \textsanskrit{Kallavāḷamutta} Village in the land of the Magadhans. 

He\marginnote{1.7} sat on the seat spread out and said to \textsanskrit{Mahāmoggallāna}, “Are you nodding off, \textsanskrit{Moggallāna}? Are you nodding off?” 

“Yes,\marginnote{2.2} sir.” 

“So,\marginnote{2.3} \textsanskrit{Moggallāna}, don’t focus on or cultivate the perception that you were meditating on when you fell drowsy. It’s possible that you’ll give up drowsiness in this way. 

But\marginnote{3.1} what if that doesn’t work? Then think about and consider the teaching as you’ve learned and memorized it, examining it with your mind. It’s possible that you’ll give up drowsiness in this way. 

But\marginnote{4.1} what if that doesn’t work? Then recite in detail the teaching as you’ve learned and memorized it. It’s possible that you’ll give up drowsiness in this way. 

But\marginnote{5.1} what if that doesn’t work? Then pinch your ears and rub your limbs. It’s possible that you’ll give up drowsiness in this way. 

But\marginnote{6.1} what if that doesn’t work? Then get up from your seat, flush your eyes with water, look around in every direction, and look up at the stars and constellations. It’s possible that you’ll give up drowsiness in this way. 

But\marginnote{7.1} what if that doesn’t work? Then focus on the perception of light, concentrating on the perception of day, regardless of whether it’s night or day. And so, with an open and unenveloped heart, develop a mind that’s full of radiance. It’s possible that you’ll give up drowsiness in this way. 

But\marginnote{8.1} what if that doesn’t work? Then walk mindfully, concentrating on the perception of continuity, your faculties directed inwards and your mind not scattered outside. It’s possible that you’ll give up drowsiness in this way. 

But\marginnote{9.1} what if that doesn’t work? Then lie down in the lion’s posture—on the right side, placing one foot on top of the other—mindful and aware, and focused on the time of getting up. When you wake, you should get up quickly, thinking: ‘I will not live indulging in the pleasures of sleeping, lying down, and drowsing.’ That’s how you should train. 

So\marginnote{10.1} you should train like this: ‘I will not approach families with my head swollen with vanity.’ That’s how you should train. What happens if a mendicant approaches families with a head swollen with vanity? Well, families have business to attend to, so people might not notice when a mendicant arrives. In that case the mendicant thinks: ‘Who on earth has turned this family against me? It seems they don’t like me any more.’ And so, because they don’t get anything they feel dismayed. Being dismayed, they become restless. Being restless, they lose restraint. And without restraint the mind is far from immersion. 

So\marginnote{11.1} you should train like this: ‘I won’t get into arguments.’ That’s how you should train. When there’s an argument, you can expect there’ll be lots of talking. When there’s lots of talking, people become restless. Being restless, they lose restraint. And without restraint the mind is far from immersion. \textsanskrit{Moggallāna}, I don’t praise all kinds of closeness. Nor do I criticize all kinds of closeness. I don’t praise closeness with laypeople and renunciates. I do praise closeness with those lodgings that are quiet and still, far from the madding crowd, remote from human settlements, and fit for retreat.” 

When\marginnote{12.1} he said this, Venerable \textsanskrit{Moggallāna} asked the Buddha, “Sir, how do you briefly define a mendicant who is freed through the ending of craving, who has reached the ultimate end, the ultimate sanctuary, the ultimate spiritual life, the ultimate goal, and is best among gods and humans?” 

“It’s\marginnote{13.1} when a mendicant has heard: ‘Nothing is worth insisting on.’ When a mendicant has heard that nothing is worth insisting on, they directly know all things. Directly knowing all things, they completely understand all things. Having completely understood all things, when they experience any kind of feeling—pleasant, unpleasant, or neutral—they meditate observing impermanence, dispassion, cessation, and letting go in those feelings. Meditating in this way, they don’t grasp at anything in the world. Not grasping, they’re not anxious. Not being anxious, they personally become extinguished. 

They\marginnote{13.9} understand: ‘Rebirth is ended, the spiritual journey has been completed, what had to be done has been done, there is no return to any state of existence.’ That’s how I briefly define a mendicant who is freed through the ending of craving, who has reached the ultimate end, the ultimate sanctuary, the ultimate spiritual life, the ultimate goal, and is best among gods and humans.” 

%
\section*{{\suttatitleacronym AN 7.62}{\suttatitletranslation Don’t Fear Good Deeds }{\suttatitleroot Mettasutta}}
\addcontentsline{toc}{section}{\tocacronym{AN 7.62} \toctranslation{Don’t Fear Good Deeds } \tocroot{Mettasutta}}
\markboth{Don’t Fear Good Deeds }{Mettasutta}
\extramarks{AN 7.62}{AN 7.62}

“Mendicants,\marginnote{1.1} don’t fear good deeds. For ‘good deeds’ is a term for happiness. I recall undergoing for a long time the likable, desirable, and agreeable results of good deeds performed over a long time. I developed a mind of love for seven years. As a result, for seven eons of the cosmos contracting and expanding I didn’t return to this world again. As the cosmos contracted I went to the realm of streaming radiance. As it expanded I was reborn in an empty mansion of \textsanskrit{Brahmā}. 

There\marginnote{2.1} I was \textsanskrit{Brahmā}, the Great \textsanskrit{Brahmā}, the undefeated, the champion, the universal seer, the wielder of power. I was Sakka, lord of gods, thirty-six times. Many hundreds of times I was a king, a wheel-turning monarch, a just and principled king. My dominion extended to all four sides, I achieved stability in the country, and I possessed the seven treasures. These were my seven treasures: the wheel, the elephant, the horse, the jewel, the woman, the treasurer, and the counselor as the seventh treasure. I had over a thousand sons who were valiant and heroic, crushing the armies of my enemies. After conquering this land girt by sea, I reigned by principle, without rod or sword. 

\begin{verse}%
See\marginnote{3.1} the result of good deeds, \\
of skillful deeds, for one seeking happiness. \\
I developed a mind of love \\
for seven years, mendicants. \\
For seven eons of expansion and contraction \\
I didn’t return to this world again. 

As\marginnote{4.1} the world contracted \\
I went to the realm of streaming radiance. \\
And when it expanded \\
I went to an empty mansion of \textsanskrit{Brahmā}. 

Seven\marginnote{5.1} times I was a Great \textsanskrit{Brahmā}, \\
and at that time I was the wielder of power. \\
Thirty-six times I was lord of gods, \\
acting as ruler of the gods. 

Then\marginnote{6.1} I was king, a wheel-turning monarch, \\
ruler of all India. \\
An anointed aristocrat, \\
I was sovereign of all humans. 

Without\marginnote{7.1} rod or sword, \\
I conquered this land. \\
Through non-violent action \\
I guided it justly. 

After\marginnote{8.1} ruling this vast territory \\
by means of principle, \\
I was born in a rich family, \\
affluent and wealthy. 

It\marginnote{9.1} was replete with all sense pleasures, \\
and the seven treasures. \\
This was well taught by the Buddhas, \\
who bring the world together. 

This\marginnote{10.1} is the cause of greatness \\
by which one is called a lord of the land. \\
I was a majestic king, \\
with lots of property and assets. 

Successful\marginnote{11.1} and glorious, \\
I was lord of India. \\
Who would not be inspired by this, \\
even someone of dark birth. 

Therefore\marginnote{12.1} someone who cares for their own welfare, \\
and wants to become the very best they can be, \\
should respect the true teaching, \\
remembering the instructions of the Buddhas.” 

%
\end{verse}

%
\section*{{\suttatitleacronym AN 7.63}{\suttatitletranslation Kinds of Wives }{\suttatitleroot Bhariyāsutta}}
\addcontentsline{toc}{section}{\tocacronym{AN 7.63} \toctranslation{Kinds of Wives } \tocroot{Bhariyāsutta}}
\markboth{Kinds of Wives }{Bhariyāsutta}
\extramarks{AN 7.63}{AN 7.63}

Then\marginnote{1.1} the Buddha robed up in the morning and, taking his bowl and robe, went to the home of the householder \textsanskrit{Anāthapiṇḍika}, where he sat on the seat spread out. 

Now\marginnote{1.2} at that time people in \textsanskrit{Anāthapiṇḍika}’s home were making a dreadful racket. Then the householder \textsanskrit{Anāthapiṇḍika} went up to the Buddha, bowed, and sat down to one side. The Buddha said to him, “Householder, what’s with the people making that dreadful racket in your home? You’d think it was fishermen hauling in a catch!” 

“Sir,\marginnote{2.2} that’s my daughter-in-law \textsanskrit{Sujātā}. She’s been brought here from a wealthy family. She doesn’t obey her mother-in-law or father-in-law or her husband. And she does not honor, respect, esteem, and venerate the Buddha.” 

Then\marginnote{3.1} the Buddha addressed \textsanskrit{Sujātā}, saying, “Come, \textsanskrit{Sujātā}.” 

“Yes,\marginnote{3.3} sir,” she replied. She went up to the Buddha, bowed, and sat down to one side. The Buddha said to her: 

“\textsanskrit{Sujātā},\marginnote{4.1} a man can have seven kinds of wife. What seven? A wife like a killer, a wife like a thief, a wife like a lord, a wife like a mother, a wife like a sister, a wife like a friend, and a wife like a bondservant. These are the kinds of wife that a man can have. Which one of these are you?” 

“Sir,\marginnote{4.6} I don’t understand the detailed meaning of what the Buddha has said in brief. Please teach me this matter so I can understand the detailed meaning.” 

“Well\marginnote{4.8} then, \textsanskrit{Sujātā}, listen and pay close attention, I will speak.” 

“Yes,\marginnote{4.9} sir,” she replied. The Buddha said this: 

\begin{verse}%
“With\marginnote{5.1} a mind full of hate and no kindness, \\
lusting for others, looking down on her husband, \\
she longs to murder the one who paid the price for her. \\
A man’s wife of this sort \\
is called a wife and a killer. 

A\marginnote{6.1} woman’s husband earns his wealth \\
by applying oneself to a profession, trade, or farming. \\
And even if it’s only a little, she wants to take it. \\
A man’s wife of this sort \\
is called a wife and a thief. 

She’s\marginnote{7.1} an idle glutton who doesn’t want to work. \\
Her words are harsh, fierce, and rude. \\
She rules over him, though he rises early. \\
A man’s wife of this sort \\
is called a wife and a lord. 

She’s\marginnote{8.1} always caring and kind, \\
looking after her husband like a mother her child. \\
She keeps the wealth that he has earned secure. \\
A man’s wife of this sort \\
is called a wife and a mother. 

She\marginnote{9.1} respects her husband \\
as a younger sister respects her elder. \\
Conscientious, she does what her husband says. \\
A man’s wife of this sort \\
is called a wife and a sister. 

She’s\marginnote{10.1} delighted to see him, \\
like one reunited with a long-lost friend. \\
She’s well-raised, virtuous, and devoted. \\
A man’s wife of this sort \\
is called a wife and a friend. 

She\marginnote{11.1} has no anger when threatened with violence by the rod. \\
Without hate or anger, \\
she endures her husband and does what he says. \\
A man’s wife of this sort \\
is called a wife and a bondservant. 

The\marginnote{12.1} kinds of wives here called \\
killer, thief, and lord; \\
immoral, harsh, and lacking regard for others, \\
when their body breaks up they set course for hell. 

But\marginnote{13.1} the kinds of wives here called \\
mother, sister, friend, and bondservant; \\
steadfast in their own morality, restrained for a long time, \\
when their body breaks up they set course for a good place. 

%
\end{verse}

\textsanskrit{Sujātā},\marginnote{14.1} these are the seven kinds of wife that a man can have. Which one of these are you?” 

“Sir,\marginnote{14.3} from this day forth may the Buddha remember me as a wife like a bondservant.” 

%
\section*{{\suttatitleacronym AN 7.64}{\suttatitletranslation Irritable }{\suttatitleroot Kodhanasutta}}
\addcontentsline{toc}{section}{\tocacronym{AN 7.64} \toctranslation{Irritable } \tocroot{Kodhanasutta}}
\markboth{Irritable }{Kodhanasutta}
\extramarks{AN 7.64}{AN 7.64}

“Mendicants,\marginnote{1.1} these seven things that please and assist an enemy happen to an irritable woman or man. What seven? 

Firstly,\marginnote{1.3} an enemy wishes for an enemy: ‘If only they’d become ugly!’ Why is that? Because an enemy doesn’t like to have a beautiful enemy. An irritable person, overcome and overwhelmed by anger, is ugly, even though they’re nicely bathed and anointed, with hair and beard dressed, and wearing white clothes. This is the first thing that pleases and assists an enemy which happens to an irritable woman or man. 

Furthermore,\marginnote{2.1} an enemy wishes for an enemy: ‘If only they’d sleep badly!’ Why is that? Because an enemy doesn’t like to have an enemy who sleeps at ease. An irritable person, overcome and overwhelmed by anger, sleeps badly, even though they sleep on a couch spread with woolen covers—shag-piled, pure white, or embroidered with flowers—and spread with a fine deer hide, with a canopy above and red pillows at both ends. This is the second thing … 

Furthermore,\marginnote{3.1} an enemy wishes for an enemy: ‘If only they don’t get all they need!’ Why is that? Because an enemy doesn’t like to have an enemy who gets all they need. When an irritable person, overcome and overwhelmed by anger, gets what they don’t need they think, ‘I’ve got what I need.’ When they get what they need they think, ‘I’ve got what I don’t need.’ When an angry person gets these things that are the exact opposite of what they need, it’s for their lasting harm and suffering. This is the third thing … 

Furthermore,\marginnote{4.1} an enemy wishes for an enemy: ‘If only they weren’t wealthy!’ Why is that? Because an enemy doesn’t like to have an enemy who is wealthy. When a person is irritable, overcome and overwhelmed by anger, the rulers seize the legitimate wealth they’ve earned by their efforts, built up with their own hands, gathered by the sweat of their brow. This is the fourth thing … 

Furthermore,\marginnote{5.1} an enemy wishes for an enemy: ‘If only they weren’t famous!’ Why is that? Because an enemy doesn’t like to have a famous enemy. When a person is irritable, overcome and overwhelmed by anger, any fame they have acquired by diligence falls to dust. This is the fifth thing … 

Furthermore,\marginnote{6.1} an enemy wishes for an enemy: ‘If only they had no friends!’ Why is that? Because an enemy doesn’t like to have an enemy with friends. When a person is irritable, overcome and overwhelmed by anger, their friends and colleagues, relatives and kin avoid them from afar. This is the sixth thing … 

Furthermore,\marginnote{7.1} an enemy wishes for an enemy: ‘If only, when their body breaks up, after death, they’re reborn in a place of loss, a bad place, the underworld, hell!’ Why is that? Because an enemy doesn’t like to have an enemy who goes to a good place. When a person is irritable, overcome and overwhelmed by anger, they do bad things by way of body, speech, and mind. When their body breaks up, after death, they’re reborn in a place of loss, a bad place, the underworld, hell. This is the seventh thing that pleases and assists an enemy which happens to an irritable woman or man. 

These\marginnote{8.1} are the seven things that please and assist an enemy which happen to an irritable woman or man. 

\begin{verse}%
An\marginnote{9.1} irritable person is ugly \\
and they sleep badly. \\
When they get what they need, \\
they take it to be what they don’t need. 

An\marginnote{10.1} angry person \\
kills with body or speech; \\
overcome with anger, \\
they lose their wealth. 

Mad\marginnote{11.1} with anger, \\
they fall into disgrace. \\
Family, friends, and loved ones \\
avoid an irritable person. 

Anger\marginnote{12.1} creates harm; \\
anger upsets the mind. \\
That person doesn’t recognize \\
the danger that arises within. 

An\marginnote{13.1} angry person doesn’t know the good. \\
An angry person doesn’t see the truth. \\
When a person is beset by anger, \\
only blind darkness is left. 

An\marginnote{14.1} angry person destroys with ease \\
what was hard to build. \\
Afterwards, when the anger is spent, \\
they’re tormented as if burnt by fire. 

Their\marginnote{15.1} look betrays their sulkiness \\
like a fire’s smoky plume. \\
And when their anger flares up, \\
they make others angry. 

They\marginnote{16.1} have no conscience or prudence, \\
nor any respectful speech. \\
One overcome by anger \\
has no island refuge anywhere. 

The\marginnote{17.1} deeds that torment a man \\
are far from those that are good. \\
I’ll explain them now; \\
listen to this, for it is the truth. 

An\marginnote{18.1} angry person slays their father; \\
their mother, too, they slay. \\
An angry person slays a saint; \\
a normal person, too, they slay. 

A\marginnote{19.1} man is raised by his mother, \\
who shows him the world. \\
But an angry ordinary person slays \\
even that good woman who gave him life. 

Like\marginnote{20.1} oneself, all sentient beings \\
hold themselves most dear. \\
But angry people kill themselves all kinds of ways, \\
distraught for many reasons. 

Some\marginnote{21.1} kill themselves with swords, \\
some, distraught, take poison. \\
Some hang themselves with rope, \\
or fling themselves down a mountain gorge. 

When\marginnote{22.1} they commit deeds of destroying life \\
and killing themselves, \\
they don’t realize what they do, \\
for anger leads to their downfall. 

The\marginnote{23.1} snare of death in the form of anger \\
lies hidden in the heart. \\
You should cut it out by self-control, \\
by wisdom, energy, and right ideas. 

An\marginnote{24.1} astute person should cut out \\
this unskillful thing. \\
And they’d train in the teaching in just the same way, \\
not yielding to sulkiness. 

Free\marginnote{25.1} of anger, free of despair, \\
free of greed, with no more longing, \\
tamed, having given up anger, \\
the undefiled become fully extinguished. 

%
\end{verse}

%
\addtocontents{toc}{\let\protect\contentsline\protect\nopagecontentsline}
\chapter*{The Great Chapter }
\addcontentsline{toc}{chapter}{\tocchapterline{The Great Chapter }}
\addtocontents{toc}{\let\protect\contentsline\protect\oldcontentsline}

%
\section*{{\suttatitleacronym AN 7.65}{\suttatitletranslation Conscience and Prudence }{\suttatitleroot Hirīottappasutta}}
\addcontentsline{toc}{section}{\tocacronym{AN 7.65} \toctranslation{Conscience and Prudence } \tocroot{Hirīottappasutta}}
\markboth{Conscience and Prudence }{Hirīottappasutta}
\extramarks{AN 7.65}{AN 7.65}

“Mendicants,\marginnote{1.1} when there is no conscience and prudence, one who lacks conscience and prudence has destroyed a vital condition for sense restraint. When there is no sense restraint, one who lacks sense restraint has destroyed a vital condition for ethical conduct. When there is no ethical conduct, one who lacks ethics has destroyed a vital condition for right immersion. When there is no right immersion, one who lacks right immersion has destroyed a vital condition for true knowledge and vision. When there is no true knowledge and vision, one who lacks true knowledge and vision has destroyed a vital condition for disillusionment and dispassion. When there is no disillusionment and dispassion, one who lacks disillusionment and dispassion has destroyed a vital condition for knowledge and vision of freedom. 

Suppose\marginnote{1.7} there was a tree that lacked branches and foliage. Its shoots, bark, softwood, and heartwood would not grow to fullness. 

In\marginnote{1.9} the same way, when there is no conscience and prudence, a person who lacks conscience and prudence has destroyed a vital condition for sense restraint. When there is no sense restraint, one who lacks sense restraint has destroyed a vital condition for ethical conduct. When there is no ethical conduct, one who lacks ethics has destroyed a vital condition for right immersion. When there is no right immersion, one who lacks right immersion has destroyed a vital condition for true knowledge and vision. When there is no true knowledge and vision, one who lacks true knowledge and vision has destroyed a vital condition for disillusionment and dispassion. When there is no disillusionment and dispassion, one who lacks disillusionment and dispassion has destroyed a vital condition for knowledge and vision of freedom. 

When\marginnote{2.1} there is conscience and prudence, a person who has fulfilled conscience and prudence has fulfilled a vital condition for sense restraint. When there is sense restraint, one who has fulfilled sense restraint has fulfilled a vital condition for ethical conduct. When there is ethical conduct, one who has fulfilled ethical conduct has fulfilled a vital condition for right immersion. When there is right immersion, one who has fulfilled right immersion has fulfilled a vital condition for true knowledge and vision. When there is true knowledge and vision, one who has fulfilled true knowledge and vision has fulfilled a vital condition for disillusionment and dispassion. When there is disillusionment and dispassion, one who has fulfilled disillusionment and dispassion has fulfilled a vital condition for knowledge and vision of freedom. 

Suppose\marginnote{2.7} there was a tree that was complete with branches and foliage. Its shoots, bark, softwood, and heartwood would grow to fullness. 

In\marginnote{2.8} the same way, when there is conscience and prudence, a person who has fulfilled conscience and prudence has fulfilled a vital condition for sense restraint. … One who has fulfilled disillusionment and dispassion has fulfilled a vital condition for knowledge and vision of freedom.” 

%
\section*{{\suttatitleacronym AN 7.66}{\suttatitletranslation The Seven Suns }{\suttatitleroot Sattasūriyasutta}}
\addcontentsline{toc}{section}{\tocacronym{AN 7.66} \toctranslation{The Seven Suns } \tocroot{Sattasūriyasutta}}
\markboth{The Seven Suns }{Sattasūriyasutta}
\extramarks{AN 7.66}{AN 7.66}

\scevam{So\marginnote{1.1} I have heard. }At one time the Buddha was staying near \textsanskrit{Vesālī}, in \textsanskrit{Ambapālī}’s Wood. There the Buddha addressed the mendicants, “Mendicants!” 

“Venerable\marginnote{1.5} sir,” they replied. The Buddha said this: 

“Mendicants,\marginnote{2.1} conditions are impermanent. Conditions are unstable. Conditions are unreliable. This is quite enough for you to become disillusioned, dispassionate, and freed regarding all conditions. 

Sineru,\marginnote{3.1} the king of mountains, is 84,000 leagues long and 84,000 leagues wide. It sinks 84,000 leagues below the ocean and rises 84,000 leagues above it. There comes a time when, after a very long period has passed, the rain doesn’t fall. For many years, many hundreds, many thousands, many hundreds of thousands of years no rain falls. When this happens, the plants and seeds, the herbs, grass, and big trees wither away and dry up, and are no more. So impermanent are conditions, so unstable, so unreliable. This is quite enough for you to become disillusioned, dispassionate, and freed regarding all conditions. 

There\marginnote{4.1} comes a time when, after a very long period has passed, a second sun appears. When this happens, the streams and pools wither away and dry up, and are no more. So impermanent are conditions … 

There\marginnote{5.1} comes a time when, after a very long period has passed, a third sun appears. When this happens, the great rivers—the Ganges, \textsanskrit{Yamunā}, \textsanskrit{Aciravatī}, \textsanskrit{Sarabhū}, and \textsanskrit{Mahī}—wither away and dry up, and are no more. So impermanent are conditions … 

There\marginnote{6.1} comes a time when, after a very long period has passed, a fourth sun appears. When this happens, the great lakes from which the rivers originate—the \textsanskrit{Anotattā}, \textsanskrit{Sīhapapātā}, \textsanskrit{Rathakārā}, \textsanskrit{Kaṇṇamuṇḍā}, \textsanskrit{Kuṇālā}, \textsanskrit{Chaddantā}, and \textsanskrit{Mandākinī}—wither away and dry up, and are no more. So impermanent are conditions … 

There\marginnote{7.1} comes a time when, after a very long period has passed, a fifth sun appears. When this happens, the water in the ocean sinks by a hundred leagues. It sinks by two, three, four, five, six, or even seven hundred leagues. The water that remains in the ocean is only seven palm trees deep. It’s six, five, four, three, two, or even one palm tree deep. The water that remains in the ocean is only seven fathoms deep. It’s six, five, four, three, two, one or even half a fathom deep. It’s waist high, knee high, or even ankle high. It’s like the time after the rainy season, when the rain falls heavily and water remains here and there in the cows’ hoofprints. In the same way, water in the ocean remains here and there in puddles like cows’ hoofprints. When the fifth sun appears there’s not even enough water in the great ocean to wet a toe-joint. So impermanent are conditions … 

There\marginnote{8.1} comes a time when, after a very long period has passed, a sixth sun appears. When this happens, this great earth and Sineru the king of mountains smoke and smolder and give off fumes. It’s like when a potter’s kiln is first kindled, and it smokes and smolders and gives off fumes. In the same way, this great earth and Sineru the king of mountains smoke and smolder and give off fumes. So impermanent are conditions … 

There\marginnote{9.1} comes a time when, after a very long period has passed, a seventh sun appears. When this happens, this great earth and Sineru the king of mountains erupt in one burning mass of fire. And as they blaze and burn the flames are swept by the wind as far as the \textsanskrit{Brahmā} realm. Sineru the king of mountains blazes and burns, crumbling as it’s overcome by the great fire. And meanwhile, mountain peaks a hundred leagues high, or two, three, four, or five hundred leagues high disintegrate as they burn. And when the great earth and Sineru the king of mountains blaze and burn, no soot or ash is found. It’s like when ghee or oil blaze and burn, and neither ashes nor soot are found. In the same way, when the great earth and Sineru the king of mountains blaze and burn, no soot or ash is found. So impermanent are conditions, so unstable are conditions, so unreliable are conditions. This is quite enough for you to become disillusioned, dispassionate, and freed regarding all conditions. 

Mendicants,\marginnote{10.1} who would ever think or believe that this earth and Sineru, king of mountains, will burn and crumble and be no more, except for one who has seen the truth? 

Once\marginnote{11.1} upon a time, there was a teacher called Sunetta. He was a religious founder and was free of sensual desire. He had many hundreds of disciples. He taught them the path to rebirth in the company of \textsanskrit{Brahmā}. Those who totally understood Sunetta’s teachings were—when their body broke up, after death—reborn in a good place, the company of \textsanskrit{Brahmā}. Of those who didn’t totally understand Sunetta’s teachings, some—when their body broke up, after death—were reborn in the company of the Gods Who Control the Creations of Others. Some were reborn in the company of the Gods Who Love to Create, some with the Joyful Gods, some with the Gods of Yama, some with the Gods of the Thirty-Three, and some with the Gods of the Four Great Kings. Some were reborn in the company of well-to-do aristocrats or brahmins or householders. 

Then\marginnote{12.1} the teacher Sunetta thought: ‘It’s not proper for me to be reborn in the next life in exactly the same place as my disciples. Why don’t I further develop love?’ 

Then\marginnote{13.1} Sunetta developed love for seven years. Having done so he did not return to this world for seven eons of cosmic expansion and contraction. As the cosmos contracted he went to the realm of streaming radiance. As it expanded he was reborn in an empty mansion of \textsanskrit{Brahmā}. There he was \textsanskrit{Brahmā}, the Great \textsanskrit{Brahmā}, the undefeated, the champion, the universal seer, the wielder of power. He was Sakka, lord of gods, thirty-six times. Many hundreds of times he was a king, a wheel-turning monarch, a just and principled king. His dominion extended to all four sides, he achieved stability in the country, and he possessed the seven treasures. He had over a thousand sons who were valiant and heroic, crushing the armies of his enemies. After conquering this land girt by sea, he reigned by principle, without rod or sword. Yet even though Sunetta lived so long, he was not exempt from rebirth, old age, and death. He was not exempt from sorrow, lamentation, pain, sadness, and distress, I say. 

Why\marginnote{14.1} is that? Because of not understanding and not penetrating four things. What four? Noble ethics, immersion, wisdom, and freedom. These noble ethics, immersion, wisdom, and freedom have been understood and comprehended. Craving for continued existence has been cut off; the conduit to rebirth is ended; now there’ll be no more future lives.” 

That\marginnote{14.6} is what the Buddha said. Then the Holy One, the Teacher, went on to say: 

\begin{verse}%
“Ethics,\marginnote{15.1} immersion, and wisdom, \\
and the supreme freedom: \\
these things have been understood \\
by Gotama the renowned. 

And\marginnote{16.1} so the Buddha, having insight, \\
explained this teaching to the mendicants. \\
The Teacher has made an end of suffering; \\
seeing clearly, he is extinguished.” 

%
\end{verse}

%
\section*{{\suttatitleacronym AN 7.67}{\suttatitletranslation The Simile of the Citadel }{\suttatitleroot Nagaropamasutta}}
\addcontentsline{toc}{section}{\tocacronym{AN 7.67} \toctranslation{The Simile of the Citadel } \tocroot{Nagaropamasutta}}
\markboth{The Simile of the Citadel }{Nagaropamasutta}
\extramarks{AN 7.67}{AN 7.67}

“Mendicants,\marginnote{1.1} when a king’s frontier citadel is well provided with seven essentials and gets four kinds of sustenance when needed, without trouble or difficulty, it is then called a king’s frontier citadel that cannot be overrun by external foes and enemies. 

With\marginnote{2.1} what seven essentials is a citadel well provided? 

Firstly,\marginnote{2.2} a citadel has a pillar with deep foundations, firmly embedded, imperturbable and unshakable. This is the first essential with which a king’s frontier citadel is well provided, to defend those within and repel those outside. 

Furthermore,\marginnote{3.1} a citadel has a moat that is deep and wide. This is the second essential … 

Furthermore,\marginnote{4.1} a citadel has a patrol path that is high and wide. This is the third essential … 

Furthermore,\marginnote{5.1} a citadel has stores of many weapons, both projectile and hand-held. This is the fourth essential … 

Furthermore,\marginnote{6.1} many kinds of armed forces reside in a citadel, such as elephant riders, cavalry, charioteers, archers, bannermen, adjutants, food servers, warrior-chiefs, princes, chargers, great warriors, heroes, leather-clad soldiers, and sons of bondservants. This is the fifth essential … 

Furthermore,\marginnote{7.1} a citadel has a gatekeeper who is astute, competent, and intelligent. He keeps strangers out and lets known people in. This is the sixth essential … 

Furthermore,\marginnote{8.1} a citadel has a wall that’s high and wide, covered with plaster. This is the seventh essential with which a king’s frontier citadel is well provided, to defend those within and repel those outside. 

With\marginnote{8.3} these seven essentials a citadel is well provided. 

What\marginnote{9.1} are the four kinds of sustenance it gets when needed, without trouble or difficulty? 

Firstly,\marginnote{9.2} a king’s frontier citadel has much hay, wood, and water stored up for the enjoyment, relief, and comfort of those within and to repel those outside. 

Furthermore,\marginnote{10.1} a king’s frontier citadel has much rice and barley stored up for those within. 

Furthermore,\marginnote{11.1} a king’s frontier citadel has much food such as sesame, green gram, and black gram stored up for those within. 

Furthermore,\marginnote{12.1} a king’s frontier citadel has much medicine—ghee, butter, oil, honey, molasses, and salt—stored up for the enjoyment, relief, and comfort of those within and to repel those outside. 

These\marginnote{12.3} are the four kinds of sustenance it gets when needed, without trouble or difficulty. 

When\marginnote{13.1} a king’s frontier citadel is well provided with seven essentials and gets four kinds of sustenance when needed, without trouble or difficulty, it is then called a king’s frontier citadel that cannot be overrun by external foes and enemies. In the same way, when a noble disciple has seven good qualities, and they get the four absorptions—blissful meditations in the present life that belong to the higher mind—when they want, without trouble or difficulty, they are then called a noble disciple who cannot be overrun by \textsanskrit{Māra}, who cannot be overrun by the Wicked One. What are the seven good qualities that they have? 

Just\marginnote{14.1} as a king’s frontier citadel has a pillar with deep foundations, firmly embedded, imperturbable and unshakable, to defend those within and repel those outside, in the same way a noble disciple has faith in the Realized One’s awakening: ‘That Blessed One is perfected, a fully awakened Buddha, accomplished in knowledge and conduct, holy, knower of the world, supreme guide for those who wish to train, teacher of gods and humans, awakened, blessed.’ A noble disciple with faith as their pillar gives up the unskillful and develops the skillful, they give up the blameworthy and develop the blameless, and they keep themselves pure. This is the first good quality they have. 

Just\marginnote{15.1} as a citadel has a moat that is deep and wide, in the same way a noble disciple has a conscience. They’re conscientious about bad conduct by way of body, speech, and mind, and conscientious about having any bad, unskillful qualities. A noble disciple with a conscience as their moat gives up the unskillful and develops the skillful, they give up the blameworthy and develop the blameless, and they keep themselves pure. This is the second good quality they have. 

Just\marginnote{16.1} as a citadel has a patrol path that is high and wide, in the same way a noble disciple is prudent. They’re prudent when it comes to bad conduct by way of body, speech, and mind, and prudent when it comes to acquiring any bad, unskillful qualities. A noble disciple with prudence as their patrol path gives up the unskillful and develops the skillful, they give up the blameworthy and develop the blameless, and they keep themselves pure. This is the third good quality they have. 

Just\marginnote{17.1} as a citadel has stores of many weapons, both projectile and hand-held, in the same way a noble disciple is very learned. They remember and keep what they’ve learned. These teachings are good in the beginning, good in the middle, and good in the end, meaningful and well-phrased, describing a spiritual practice that’s entirely full and pure. They are very learned in such teachings, remembering them, reciting them, mentally scrutinizing them, and comprehending them theoretically. A noble disciple with learning as their weapon gives up the unskillful and develops the skillful, they give up the blameworthy and develop the blameless, and they keep themselves pure. This is the fourth good quality they have. 

Just\marginnote{18.1} as many kinds of armed forces reside in a citadel … in the same way a noble disciple is energetic. They live with energy roused up for giving up unskillful qualities and embracing skillful qualities. They are strong, staunchly vigorous, not slacking off when it comes to developing skillful qualities. A noble disciple with energy as their armed forces gives up the unskillful and develops the skillful, they give up the blameworthy and develop the blameless, and they keep themselves pure. This is the fifth good quality they have. 

Just\marginnote{19.1} as a citadel has a gatekeeper who is astute, competent, and intelligent, who keeps strangers out and lets known people in, in the same way a noble disciple is mindful. They have utmost mindfulness and alertness, and can remember and recall what was said and done long ago. A noble disciple with mindfulness as their gatekeeper gives up the unskillful and develops the skillful, they give up the blameworthy and develop the blameless, and they keep themselves pure. This is the sixth good quality they have. 

Just\marginnote{20.1} as a citadel has a wall that’s high and wide, covered with plaster, to defend those within and repel those outside, in the same way a noble disciple is wise. They have the wisdom of arising and passing away which is noble, penetrative, and leads to the complete ending of suffering. A noble disciple with wisdom as their wall gives up the unskillful and develops the skillful, they give up the blameworthy and develop the blameless, and they keep themselves pure. This is the seventh good quality they have. These are the seven good qualities that they have. 

And\marginnote{21.1} what are the four absorptions—blissful meditations in the present life that belong to the higher mind—that they get when they want, without trouble or difficulty? Just as a king’s frontier citadel has much hay, wood, and water stored up for the enjoyment, relief, and comfort of those within and to repel those outside, in the same way a noble disciple, quite secluded from sensual pleasures, secluded from unskillful qualities, enters and remains in the first absorption, which has the rapture and bliss born of seclusion, while placing the mind and keeping it connected. This is for their own enjoyment, relief, and comfort, and for alighting upon extinguishment. 

Just\marginnote{22.1} as a king’s frontier citadel has much rice and barley stored up, in the same way, as the placing of the mind and keeping it connected are stilled, a noble disciple enters and remains in the second absorption, which has the rapture and bliss born of immersion, with internal clarity and confidence, and unified mind, without placing the mind and keeping it connected. This is for their own enjoyment, relief, and comfort, and for alighting upon extinguishment. 

Just\marginnote{23.1} as a king’s frontier citadel has much food such as sesame, green gram, and black gram stored up, in the same way with the fading away of rapture, a noble disciple enters and remains in the third absorption, where they meditate with equanimity, mindful and aware, personally experiencing the bliss of which the noble ones declare, ‘Equanimous and mindful, one meditates in bliss.’ This is for their own enjoyment, relief, and comfort, and for alighting upon extinguishment. 

Just\marginnote{24.1} as a king’s frontier citadel has much medicine—ghee, butter, oil, honey, molasses, and salt—stored up for the enjoyment, relief, and comfort of those within and to repel those outside, in the same way, giving up pleasure and pain, and ending former happiness and sadness, a noble disciple enters and remains in the fourth absorption, without pleasure or pain, with pure equanimity and mindfulness. This is for their own enjoyment, relief, and comfort, and for alighting upon extinguishment. These are the four absorptions—blissful meditations in the present life that belong to the higher mind—which they get when they want, without trouble or difficulty. 

When\marginnote{25.1} a noble disciple has seven good qualities, and they get the four absorptions—blissful meditations in the present life that belong to the higher mind—when they want, without trouble or difficulty, they are then called a noble disciple who cannot be overrun by \textsanskrit{Māra}, who cannot be overrun by the Wicked One.” 

%
\section*{{\suttatitleacronym AN 7.68}{\suttatitletranslation One Who Knows the Teachings }{\suttatitleroot Dhammaññūsutta}}
\addcontentsline{toc}{section}{\tocacronym{AN 7.68} \toctranslation{One Who Knows the Teachings } \tocroot{Dhammaññūsutta}}
\markboth{One Who Knows the Teachings }{Dhammaññūsutta}
\extramarks{AN 7.68}{AN 7.68}

“A\marginnote{1.1} mendicant with seven qualities is worthy of offerings dedicated to the gods, worthy of hospitality, worthy of a religious donation, worthy of veneration with joined palms, and is the supreme field of merit for the world. What seven? It’s when a mendicant knows the teachings, knows the meaning, has self-knowledge, knows moderation, knows the right time, knows assemblies, and knows people high and low. 

And\marginnote{2.1} how is a mendicant one who knows the teachings? It’s when a mendicant knows the teachings: statements, songs, discussions, verses, inspired exclamations, legends, stories of past lives, amazing stories, and classifications. If a mendicant did not know these teachings, they would not be called ‘one who knows the teachings’. But because they do know these teachings, they are called ‘one who knows the teachings’. Such is the one who knows the teachings. 

And\marginnote{3.1} how are they one who knows the meaning? It’s when a mendicant knows the meaning of this or that statement: ‘This is what that statement means; that is what this statement means.’ If a mendicant did not know the meaning of this or that statement, they would not be called ‘one who knows the meaning’. But because they do know the meaning of this or that statement, they are called ‘one who knows the meaning’. Such is the one who knows the teachings and the one who knows the meaning. 

And\marginnote{4.1} how are they one who has self-knowledge? It’s when a mendicant has self-knowledge: ‘This is the extent of my faith, ethics, learning, generosity, wisdom, and eloquence.’ If a mendicant did not have self-knowledge, they would not be called ‘one who has self-knowledge’. But because they do have self-knowledge, they are called ‘one who has self-knowledge’. Such is the one who knows the teachings, the one who knows the meaning, and the one who has self-knowledge. 

And\marginnote{5.1} how are they one who knows moderation? It’s when a mendicant knows moderation when receiving robes, almsfood, lodgings, and medicines and supplies for the sick. If a mendicant did not know moderation, they would not be called ‘one who knows moderation’. But because they do know moderation, they are called ‘one who knows moderation’. Such is the one who knows the teachings, the one who knows the meaning, the one who has self-knowledge, and the one who knows moderation. 

And\marginnote{6.1} how are they one who knows the right time? It’s when a mendicant knows the right time: ‘This is the time for recitation; this is the time for questioning; this is the time for meditation; this is the time for retreat.’ If a mendicant did not know the right time, they would not be called ‘one who knows the right time’. But because they do know the right time, they are called ‘one who knows the right time’. Such is the one who knows the teachings, the one who knows the meaning, the one who has self-knowledge, the one who knows moderation, and the one who knows the right time. 

And\marginnote{7.1} how are they one who knows assemblies? It’s when a mendicant knows assemblies: ‘This is an assembly of aristocrats, of brahmins, of householders, or of ascetics. This one should be approached in this way. This is how to stand, to act, to sit, to speak, or to stay silent when there.’ If a mendicant did not know assemblies, they would not be called ‘one who knows assemblies’. But because they do know assemblies, they are called ‘one who knows assemblies’. Such is the one who knows the teachings, the one who knows the meaning, the one who has self-knowledge, the one who knows moderation, the one who knows the right time, and the one who knows assemblies. 

And\marginnote{8.1} how are they one who knows people high and low? It’s when a mendicant understands people in terms of pairs. Two people: one likes to see the noble ones, one does not. The person who doesn’t like to see the noble ones is reprehensible in that respect. The person who does like to see the noble ones is praiseworthy in that respect. 

Two\marginnote{9.1} people like to see the noble ones: one likes to hear the true teaching, one does not. The person who doesn’t like to hear the true teaching is reprehensible in that respect. The person who does like to hear the true teaching is praiseworthy in that respect. 

Two\marginnote{10.1} people like to hear the true teaching: one lends an ear to the teaching, one does not. The person who doesn’t lend an ear to the teaching is reprehensible in that respect. The person who does lend an ear to the teaching is praiseworthy in that respect. 

Two\marginnote{11.1} people lend an ear to the teaching: one remembers the teaching they’ve heard, one does not. The person who doesn’t remember the teaching they’ve heard is reprehensible in that respect. The person who does remember the teaching they’ve heard is praiseworthy in that respect. 

Two\marginnote{12.1} people remember the teaching they’ve heard: one reflects on the meaning of the teachings they have remembered, one does not. The person who does not reflect on the meaning of the teachings they have remembered is reprehensible in that respect. The person who does reflect on the meaning of the teachings they have remembered is praiseworthy in that respect. 

Two\marginnote{13.1} people reflect on the meaning of the teachings they have remembered: one understands the meaning and the teaching and practices accordingly, one understands the meaning and the teaching but does not practice accordingly. The person who understands the meaning and the teaching but does not practice accordingly is reprehensible in that respect. The person who understands the meaning and the teaching and practices accordingly is praiseworthy in that respect. 

Two\marginnote{14.1} people understand the meaning and the teaching and practice accordingly: one practices to benefit themselves but not others, and one practices to benefit both themselves and others. The person who practices to benefit themselves but not others is reprehensible in that respect. The person who practices to benefit both themselves and others is praiseworthy in that respect. 

That’s\marginnote{15.1} how a mendicant understands people in terms of pairs. 

That’s\marginnote{16.1} how a mendicant is one who knows people high and low. A mendicant with these seven qualities is worthy of offerings dedicated to the gods, worthy of hospitality, worthy of a religious donation, worthy of veneration with joined palms, and is the supreme field of merit for the world.” 

%
\section*{{\suttatitleacronym AN 7.69}{\suttatitletranslation The Shady Orchid Tree }{\suttatitleroot Pāricchattakasutta}}
\addcontentsline{toc}{section}{\tocacronym{AN 7.69} \toctranslation{The Shady Orchid Tree } \tocroot{Pāricchattakasutta}}
\markboth{The Shady Orchid Tree }{Pāricchattakasutta}
\extramarks{AN 7.69}{AN 7.69}

“Mendicants,\marginnote{1.1} when the leaves on the Shady Orchid Tree belonging to the Gods of the Thirty-Three turn brown, the gods are elated. They think: ‘Now the leaves on the Shady Orchid Tree have turned brown! It won’t be long until they fall.’ 

When\marginnote{2.1} the leaves have fallen, the gods are elated. They think: ‘Now the leaves on the Shady Orchid Tree have fallen. It won’t be long until its foliage starts to regrow.’ 

When\marginnote{3.1} the foliage starts to regrow, the gods are elated. They think: ‘Now the foliage of the Shady Orchid Tree has started to regrow. It won’t be long until it’s ready to grow flowers and leaves separately.’ 

When\marginnote{4.1} it’s ready to grow flowers and leaves separately, the gods are elated. They think: ‘Now the Shady Orchid Tree is ready to grow flowers and leaves separately. It won’t be long until buds start to form.’ 

When\marginnote{5.1} the buds start to form, the gods are elated. They think: ‘Now the buds of the Shady Orchid Tree have started to form. It won’t be long until the buds burst.’ 

When\marginnote{6.1} the buds have burst, the gods are elated. They think: ‘Now the buds of the Shady Orchid Tree have burst. It won’t be long until it fully blossoms.’ 

When\marginnote{7.1} the Shady Orchid Tree of the Gods of the Thirty-Three has fully blossomed, the gods are elated. For four heavenly months they amused themselves at the root of the tree, supplied and provided with the five kinds of sensual stimulation. When the Shady Orchid Tree has fully blossomed, its radiance spreads for fifty leagues, while its fragrance wafts for a hundred leagues. Such is the majesty of the Shady Orchid Tree. 

In\marginnote{8.1} the same way, when a noble disciple plans to go forth from the lay life to homelessness, they’re like the Shady Orchid Tree when its leaves turn brown. 

When\marginnote{9.1} a noble disciple shaves off their hair and beard, dresses in ocher robes, and goes forth from the lay life to homelessness, they’re like the Shady Orchid Tree when its leaves fall. 

When\marginnote{10.1} a noble disciple, quite secluded from sensual pleasures, secluded from unskillful qualities, enters and remains in the first absorption, which has the rapture and bliss born of seclusion, while placing the mind and keeping it connected, they’re like the Shady Orchid Tree when its foliage starts to regrow. 

When,\marginnote{11.1} as the placing of the mind and keeping it connected are stilled, a noble disciple enters and remains in the second absorption, which has the rapture and bliss born of immersion, with internal clarity and confidence, and unified mind, without placing the mind and keeping it connected, they’re like the Shady Orchid Tree when it’s ready to grow flowers and leaves separately. 

When,\marginnote{12.1} with the fading away of rapture, a noble disciple enters and remains in the third absorption, where they meditate with equanimity, mindful and aware, personally experiencing the bliss of which the noble ones declare, ‘Equanimous and mindful, one meditates in bliss’, they’re like the Shady Orchid Tree when its buds start to form. 

When,\marginnote{13.1} giving up pleasure and pain, and ending former happiness and sadness, a noble disciple enters and remains in the fourth absorption, without pleasure or pain, with pure equanimity and mindfulness, they’re like the Shady Orchid Tree when its buds burst. 

When\marginnote{14.1} a noble disciple realizes the undefiled freedom of heart and freedom by wisdom in this very life, and they live having realized it with their own insight due to the ending of defilements, they’re like the Shady Orchid tree when it fully blossoms. 

At\marginnote{15.1} that time the earth gods raise the cry: ‘This venerable named so-and-so, from such-and-such village or town, the pupil of the venerable named so-and-so, went forth from the lay life to homelessness. They’ve realized the undefiled freedom of heart and freedom by wisdom in this very life. And they live having realized it with their own insight due to the ending of defilements.’ 

Hearing\marginnote{16.1} the cry of the Earth Gods, the Gods of the Four Great Kings … the Gods of the Thirty-Three … the Gods of Yama … the Joyful Gods … the Gods Who Love to Create … the Gods Who Control the Creations of Others … the Gods of \textsanskrit{Brahmā}’s Host raise the cry: ‘This venerable named so-and-so, from such-and-such village or town, the pupil of the venerable named so-and-so, went forth from the lay life to homelessness. They’ve realized the undefiled freedom of heart and freedom by wisdom in this very life. And they live having realized it with their own insight due to the ending of defilements.’ And so at that moment, in that instant, the cry soars up to the \textsanskrit{Brahmā} realm. Such is the majesty of a mendicant who has ended the defilements.” 

%
\section*{{\suttatitleacronym AN 7.70}{\suttatitletranslation Honor }{\suttatitleroot Sakkaccasutta}}
\addcontentsline{toc}{section}{\tocacronym{AN 7.70} \toctranslation{Honor } \tocroot{Sakkaccasutta}}
\markboth{Honor }{Sakkaccasutta}
\extramarks{AN 7.70}{AN 7.70}

Then\marginnote{1.1} as Venerable \textsanskrit{Sāriputta} was in private retreat this thought came to his mind, “What should a mendicant honor and respect and rely on, to give up the unskillful and develop the skillful?” 

Then\marginnote{1.3} he thought, “A mendicant should honor and respect and rely on the Teacher … the teaching … the \textsanskrit{Saṅgha} … the training … immersion … diligence … A mendicant should honor and respect and rely on hospitality, to give up the unskillful and develop the skillful.” 

Then\marginnote{2.1} he thought, “These qualities are pure and bright in me. Why don’t I go and tell them to the Buddha? Then these qualities will not only be purified in me, but will be better known as purified. Suppose a man were to acquire a gold coin, pure and bright. They’d think, ‘My gold coin is pure and bright. Why don’t I take it to show the smiths? Then it will not only be purified, but will be better known as purified.’ In the same way, these qualities are pure and bright in me. Why don’t I go and tell them to the Buddha? Then these qualities will not only be purified in me, but will be better known as purified.” 

Then\marginnote{3.1} in the late afternoon, \textsanskrit{Sāriputta} came out of retreat and went to the Buddha. He bowed, sat down to one side, and told the Buddha of his thoughts while on retreat. 

“Good,\marginnote{5.1} good, \textsanskrit{Sāriputta}! A mendicant should honor and respect and rely on the Teacher, to give up the unskillful and develop the skillful. A mendicant should honor and respect and rely on the teaching … the \textsanskrit{Saṅgha} … the training … immersion … diligence … A mendicant should honor and respect and rely on hospitality, to give up the unskillful and develop the skillful.” 

When\marginnote{6.1} he said this, Venerable \textsanskrit{Sāriputta} said to the Buddha: 

“Sir,\marginnote{6.2} this is how I understand the detailed meaning of the Buddha’s brief statement. It’s quite impossible for a mendicant who doesn’t respect the Teacher to respect the teaching. A mendicant who disrespects the Teacher disrespects the teaching. 

It’s\marginnote{7.1} quite impossible for a mendicant who doesn’t respect the Teacher and the teaching to respect the \textsanskrit{Saṅgha}. A mendicant who disrespects the Teacher and the teaching disrespects the \textsanskrit{Saṅgha}. 

It’s\marginnote{8.1} quite impossible for a mendicant who doesn’t respect the Teacher, the teaching, and the \textsanskrit{Saṅgha} to respect the training. A mendicant who disrespects the Teacher, the teaching, and the \textsanskrit{Saṅgha} disrespects the training. 

It’s\marginnote{9.1} quite impossible for a mendicant who doesn’t respect the Teacher, the teaching, the \textsanskrit{Saṅgha}, and the training to respect immersion. A mendicant who disrespects the Teacher, the teaching, the \textsanskrit{Saṅgha}, and the training disrespects immersion. 

It’s\marginnote{10.1} quite impossible for a mendicant who doesn’t respect the Teacher, the teaching, the \textsanskrit{Saṅgha}, the training, and immersion to respect diligence. A mendicant who disrespects the Teacher, the teaching, the \textsanskrit{Saṅgha}, the training, and immersion disrespects diligence. 

It’s\marginnote{11.1} quite impossible for a mendicant who doesn’t respect the Teacher, the teaching, the \textsanskrit{Saṅgha}, the training, immersion, and diligence to respect hospitality. A mendicant who disrespects the Teacher, the teaching, the \textsanskrit{Saṅgha}, the training, immersion, and diligence disrespects hospitality. 

It’s\marginnote{12.1} quite impossible for a mendicant who does respect the Teacher to disrespect the teaching. … 

A\marginnote{13.1} mendicant who respects the Teacher, the teaching, the \textsanskrit{Saṅgha}, the training, immersion, and diligence respects hospitality. 

It’s\marginnote{14.1} quite possible for a mendicant who respects the Teacher to respect teaching. … 

A\marginnote{15.1} mendicant who respects the Teacher, the teaching, the \textsanskrit{Saṅgha}, the training, immersion, and diligence respects hospitality. 

That’s\marginnote{16.1} how I understand the detailed meaning of the Buddha’s brief statement.” 

“Good,\marginnote{17.1} good, \textsanskrit{Sāriputta}! It’s good that you understand the detailed meaning of what I’ve said in brief like this. 

It’s\marginnote{17.3} quite impossible for a mendicant who doesn’t respect the Teacher to respect the teaching. … 

A\marginnote{18.1} mendicant who disrespects the Teacher, the teaching, the \textsanskrit{Saṅgha}, the training, immersion, and diligence disrespects hospitality. 

It’s\marginnote{19.1} quite impossible for a mendicant who does respect the Teacher to disrespect the teaching. … 

A\marginnote{20.1} mendicant who respects the Teacher, the teaching, the \textsanskrit{Saṅgha}, the training, immersion, and diligence respects hospitality. 

It’s\marginnote{21.1} quite possible for a mendicant who does respect the Teacher to respect the teaching. … 

A\marginnote{22.1} mendicant who respects the Teacher, the teaching, the \textsanskrit{Saṅgha}, the training, immersion, and diligence respects hospitality. 

This\marginnote{23.1} is how to understand the detailed meaning of what I said in brief.” 

%
\section*{{\suttatitleacronym AN 7.71}{\suttatitletranslation Committed to Development }{\suttatitleroot Bhāvanāsutta}}
\addcontentsline{toc}{section}{\tocacronym{AN 7.71} \toctranslation{Committed to Development } \tocroot{Bhāvanāsutta}}
\markboth{Committed to Development }{Bhāvanāsutta}
\extramarks{AN 7.71}{AN 7.71}

“Mendicants,\marginnote{1.1} when a mendicant is not committed to development, they might wish: ‘If only my mind were freed from the defilements by not grasping!’ Even so, their mind is not freed from defilements by not grasping. Why is that? You should say: ‘It’s because they’re undeveloped.’ Undeveloped in what? The four kinds of mindfulness meditation, the four right efforts, the four bases of psychic power, the five faculties, the five powers, the seven awakening factors, and the noble eightfold path. 

Suppose\marginnote{2.1} there was a chicken with eight or ten or twelve eggs. But she had not properly sat on them to keep them warm and incubated. Even if that chicken might wish: ‘If only my chicks could break out of the eggshell with their claws and beak and hatch safely!’ Still they can’t break out and hatch safely. Why is that? Because she has not properly sat on them to keep them warm and incubated. 

In\marginnote{2.8} the same way, when a mendicant is not committed to development, they might wish: ‘If only my mind was freed from the defilements by not grasping!’ Even so, their mind is not freed from defilements by not grasping. Why is that? You should say: ‘It’s because they’re undeveloped.’ Undeveloped in what? The four kinds of mindfulness meditation, the four right efforts, the four bases of psychic power, the five faculties, the five powers, the seven awakening factors, and the noble eightfold path. 

When\marginnote{3.1} a mendicant is committed to development, they might not wish: ‘If only my mind was freed from the defilements by not grasping!’ Even so, their mind is freed from defilements by not grasping. Why is that? You should say: ‘It’s because they are developed.’ Developed in what? The four kinds of mindfulness meditation, the four right efforts, the four bases of psychic power, the five faculties, the five powers, the seven awakening factors, and the noble eightfold path. 

Suppose\marginnote{4.1} there was a chicken with eight or ten or twelve eggs. And she properly sat on them to keep them warm and incubated. Even if that chicken doesn’t wish: ‘If only my chicks could break out of the eggshell with their claws and beak and hatch safely!’ Still they can break out and hatch safely. Why is that? Because she properly sat on them to keep them warm and incubated. 

In\marginnote{4.8} the same way, when a mendicant is committed to development, they might not wish: ‘If only my mind was freed from the defilements by not grasping!’ Even so, their mind is freed from defilements by not grasping. Why is that? You should say: ‘It’s because they are developed.’ Developed in what? The four kinds of mindfulness meditation, the four right efforts, the four bases of psychic power, the five faculties, the five powers, the seven awakening factors, and the noble eightfold path. 

Suppose\marginnote{5.1} a carpenter or their apprentice sees the marks of his fingers and thumb on the handle of his adze. They don’t know how much of the handle was worn away today, how much yesterday, and how much previously. They just know what has been worn away. In the same way, when a mendicant is committed to development, they don’t know how much of the defilements were worn away today, how much yesterday, and how much previously. They just know what has been worn away. 

Suppose\marginnote{6.1} there was a sea-faring ship bound together with ropes. For six months they deteriorated in the water. Then in the cold season it was hauled up on dry land, where the ropes were weathered by wind and sun. When the clouds soaked it with rain, the ropes would readily collapse and rot away. In the same way, when a mendicant is committed to development their fetters readily collapse and rot away.” 

%
\section*{{\suttatitleacronym AN 7.72}{\suttatitletranslation The Simile of the Bonfire }{\suttatitleroot Aggikkhandhopamasutta}}
\addcontentsline{toc}{section}{\tocacronym{AN 7.72} \toctranslation{The Simile of the Bonfire } \tocroot{Aggikkhandhopamasutta}}
\markboth{The Simile of the Bonfire }{Aggikkhandhopamasutta}
\extramarks{AN 7.72}{AN 7.72}

\scevam{So\marginnote{1.1} I have heard. }At one time the Buddha was wandering in the land of the Kosalans together with a large \textsanskrit{Saṅgha} of mendicants. 

While\marginnote{1.3} walking along the road, at a certain spot he saw a bonfire burning, blazing, and glowing. Seeing this he left the road, sat at the root of a tree on a seat spread out, and addressed the mendicants, “Mendicants, do you see that bonfire burning, blazing, and glowing?” 

“Yes,\marginnote{1.7} sir.” 

“What\marginnote{2.1} do you think, mendicants? Which is better—to sit or lie down embracing that bonfire? Or to sit or lie down embracing a girl of the aristocrats or brahmins or householders with soft and tender hands and feet?” 

“Sir,\marginnote{2.3} it would be much better to sit or lie down embracing a girl of the aristocrats or brahmins or householders with soft and tender hands and feet. For it would be painful to sit or lie down embracing that bonfire.” 

“I\marginnote{3.1} declare this to you, mendicants, I announce this to you! It would be better for that unethical man—of bad qualities, filthy, with suspicious behavior, underhand, no true ascetic or spiritual practitioner, though claiming to be one, rotten inside, corrupt, and depraved—to sit or lie down embracing that bonfire. Why is that? Because that might result in death or deadly pain. But when his body breaks up, after death, it would not cause him to be reborn in a place of loss, a bad place, the underworld, hell. 

But\marginnote{4.1} when such an unethical man sits or lies down embracing a girl of the aristocrats or brahmins or householders with soft and tender hands and feet, that brings him lasting harm and suffering. When his body breaks up, after death, he’s reborn in a place of loss, a bad place, the underworld, hell. 

What\marginnote{5.1} do you think, mendicants? Which is better—to have a strong man twist a tough horse-hair rope around both shins and tighten it so that it cuts through your outer skin, your inner skin, your flesh, sinews, and bones, until it reaches your marrow and stays pressing there? Or to consent to well-to-do aristocrats or brahmins or householders bowing down to you?” 

“Sir,\marginnote{5.3} it would be much better to consent to well-to-do aristocrats or brahmins or householders bowing down. For it would be painful to have a strong man twist a tough horse-hair rope around your shins and tighten it so that it cut through the outer skin until it reached the marrow and stayed pressing there.” 

“I\marginnote{6.1} declare this to you, mendicants, I announce this to you! It would be better for that unethical man to have a strong man twist a tough horse-hair rope around both shins and tighten it until it reached the marrow and stayed pressing there. Why is that? Because that might result in death or deadly pain. But when his body breaks up, after death, it would not cause him to be reborn in a place of loss, a bad place, the underworld, hell. But when such an unethical man consents to well-to-do aristocrats or brahmins or householders bowing down, that brings him lasting harm and suffering. When his body breaks up, after death, he’s reborn in a place of loss, a bad place, the underworld, hell. 

What\marginnote{7.1} do you think, mendicants? Which is better—to have a strong man stab you in the chest with a sharp, oiled sword? Or to consent to well-to-do aristocrats or brahmins or householders revering you with joined palms?” 

“Sir,\marginnote{7.3} it would be much better to consent to well-to-do aristocrats or brahmins or householders revering you with joined palms. For it would be painful to have a strong man stab you in the chest with a sharp, oiled sword.” 

“I\marginnote{8.1} declare this to you, mendicants, I announce this to you! It would be better for that unethical man to have a strong man stab him in the chest with a sharp, oiled sword. Why is that? Because that might result in death or deadly pain. But when his body breaks up, after death, it would not cause him to be reborn in a place of loss, a bad place, the underworld, hell. But when such an unethical man consents to well-to-do aristocrats or brahmins or householders revering him with joined palms, that brings him lasting harm and suffering. When his body breaks up, after death, he’s reborn in a place of loss, a bad place, the underworld, hell. 

What\marginnote{9.1} do you think, mendicants? Which is better—to have a strong man wrap you up in a red-hot sheet of iron, burning, blazing, and glowing? Or to enjoy the use of a robe given in faith by well-to-do aristocrats or brahmins or householders?” 

“Sir,\marginnote{9.3} it would be much better to enjoy the use of a robe given in faith by well-to-do aristocrats or brahmins or householders. For it would be painful to have a strong man wrap you up in a red-hot sheet of iron, burning, blazing, and glowing.” 

“I\marginnote{10.1} declare this to you, mendicants, I announce this to you! It would be better for that unethical man to have a strong man wrap him up in a red-hot sheet of iron, burning, blazing, and glowing. Why is that? Because that might result in death or deadly pain. But when his body breaks up, after death, it would not cause him to be reborn in a place of loss, a bad place, the underworld, hell. But when such an unethical man enjoys the use of a robe given in faith by well-to-do aristocrats or brahmins or householders, that brings him lasting harm and suffering. When his body breaks up, after death, he’s reborn in a place of loss, a bad place, the underworld, hell. 

What\marginnote{11.1} do you think, mendicants? Which is better—to have a strong man force your mouth open with a hot iron spike and shove in a red-hot copper ball, burning, blazing, and glowing, that burns your lips, mouth, tongue, throat, and stomach before coming out below dragging your entrails? Or to enjoy almsfood given in faith by well-to-do aristocrats or brahmins or householders?” 

“Sir,\marginnote{11.3} it would be much better to enjoy almsfood given in faith by well-to-do aristocrats or brahmins or householders. For it would be painful to have a strong man force your mouth open with a hot iron spike and shove in a red-hot copper ball, burning, blazing, and glowing, that burns your lips, mouth, tongue, throat, and stomach before coming out below dragging your entrails.” 

“I\marginnote{12.1} declare this to you, mendicants, I announce this to you! It would be better for that unethical man to have a strong man force his mouth open with a hot iron spike and shove in a red-hot copper ball, burning, blazing, and glowing, that burns his lips, mouth, tongue, throat, and stomach before coming out below with his entrails. Why is that? Because that might result in death or deadly pain. But when his body breaks up, after death, it would not cause him to be reborn in a place of loss, a bad place, the underworld, hell. But when such an unethical man enjoy almsfood given in faith by well-to-do aristocrats or brahmins or householders, that brings him lasting harm and suffering. When his body breaks up, after death, he’s reborn in a place of loss, a bad place, the underworld, hell. 

What\marginnote{13.1} do you think, mendicants? Which is better—to have a strong man grab you by the head or shoulders and make you sit or lie down on red-hot iron bed or seat? Or to enjoy the use of beds and chairs given in faith by well-to-do aristocrats or brahmins or householders?” 

“Sir,\marginnote{13.3} it would be much better to enjoy the use of beds and chairs given in faith by well-to-do aristocrats or brahmins or householders. For it would be painful to have a strong man grab you by the head or shoulders and make you sit or lie down on a red-hot iron bed or seat.” 

“I\marginnote{14.1} declare this to you, mendicants, I announce this to you! It would be better for that unethical man to have a strong man grab him by the head or shoulders and make him sit or lie down on a red-hot iron bed or seat. Why is that? Because that might result in death or deadly pain. But when his body breaks up, after death, it would not cause him to be reborn in a place of loss, a bad place, the underworld, hell. But when such an unethical man enjoys the use of beds and seats given in faith by well-to-do aristocrats or brahmins or householders, that brings him lasting harm and suffering. When his body breaks up, after death, he’s reborn in a place of loss, a bad place, the underworld, hell. 

What\marginnote{15.1} do you think, mendicants? Which is better—to have a strong man grab you, turn you upside down, and shove you in a red-hot copper pot, burning, blazing, and glowing, where you’re seared in boiling scum, and swept up and down and round and round. Or to enjoy the use of dwellings given in faith by well-to-do aristocrats or brahmins or householders?” 

“Sir,\marginnote{15.3} it would be much better to enjoy the use of dwellings given in faith by well-to-do aristocrats or brahmins or householders. For it would be painful to have a strong man grab you, turn you upside down, and shove you in a red-hot copper pot, burning, blazing, and glowing, where you’re seared in boiling scum, and swept up and down and round and round.” 

“I\marginnote{16.1} declare this to you, mendicants, I announce this to you! It would be better for that unethical man to have a strong man grab him, turn him upside down, and shove him in a red-hot copper pot, burning, blazing, and glowing, where he’s seared in boiling scum, and swept up and down and round and round. Why is that? Because that might result in death or deadly pain. But when his body breaks up, after death, it would not cause him to be reborn in a place of loss, a bad place, the underworld, hell. But when such an unethical man enjoys the use of dwellings given in faith by well-to-do aristocrats or brahmins or householders, that brings him lasting harm and suffering. When his body breaks up, after death, he’s reborn in a place of loss, a bad place, the underworld, hell. 

So\marginnote{17.1} you should train like this: ‘Our use of robes, almsfood, lodgings, and medicines and supplies for the sick shall be of great fruit and benefit for those who offered them. And our going forth will not be wasted, but will be fruitful and fertile.’ That’s how you should train. Considering what is good for yourself, mendicants, is quite enough for you to persist with diligence. Considering what is good for others is quite enough for you to persist with diligence. Considering what is good for both is quite enough for you to persist with diligence.” 

That\marginnote{18.1} is what the Buddha said. And while this discourse was being spoken, sixty monks spewed hot blood from their mouths. Sixty mendicants resigned the training and returned to a lesser life, saying: 

“It’s\marginnote{18.4} too hard, Blessed One! It’s just too hard!” And sixty monks were freed from defilements by not grasping. 

%
\section*{{\suttatitleacronym AN 7.73}{\suttatitletranslation About Sunetta }{\suttatitleroot Sunettasutta}}
\addcontentsline{toc}{section}{\tocacronym{AN 7.73} \toctranslation{About Sunetta } \tocroot{Sunettasutta}}
\markboth{About Sunetta }{Sunettasutta}
\extramarks{AN 7.73}{AN 7.73}

“Once\marginnote{1.1} upon a time, mendicants, there was a Teacher called Sunetta. He was a religious founder and was free of sensual desire. He had many hundreds of disciples. He taught them the path to rebirth in the company of \textsanskrit{Brahmā}. Those lacking confidence in Sunetta were—when their body broke up, after death—reborn in a place of loss, a bad place, the underworld, hell. Those full of confidence in Sunetta were—when their body broke up, after death—reborn in a good place, a heavenly realm. 

Once\marginnote{2.1} upon a time there was a teacher called \textsanskrit{Mūgapakkha} … Aranemi … \textsanskrit{Kuddālaka} … \textsanskrit{Hatthipāla} … \textsanskrit{Jotipāla} … Araka. He was a religious founder and was free of sensual desire. He had many hundreds of disciples. He taught them the way to rebirth in the company of \textsanskrit{Brahmā}. Those lacking confidence in Araka were—when their body broke up, after death—reborn in a place of loss, a bad place, the underworld, hell. Those full of confidence in Araka were—when their body broke up, after death—reborn in a good place, a heavenly realm. 

What\marginnote{3.1} do you think, mendicants? If someone with malicious intent were to abuse and insult these seven teachers with their hundreds of followers, would they not make much bad karma?” 

“Yes,\marginnote{3.3} sir.” 

“They\marginnote{3.4} would indeed. But someone who abuses and insults a single person accomplished in view with malicious intent makes even more bad karma. Why is that? I say that any injury done by those outside of the Buddhist community does not compare with what is done to one’s own spiritual companions. 

So\marginnote{4.1} you should train like this: ‘We will have no malicious intent for our spiritual companions.’ That’s how you should train.” 

%
\section*{{\suttatitleacronym AN 7.74}{\suttatitletranslation About Araka }{\suttatitleroot Arakasutta}}
\addcontentsline{toc}{section}{\tocacronym{AN 7.74} \toctranslation{About Araka } \tocroot{Arakasutta}}
\markboth{About Araka }{Arakasutta}
\extramarks{AN 7.74}{AN 7.74}

“Once\marginnote{1.1} upon a time, mendicants, there was a Teacher called Araka. He was a religious founder and was free of sensual desire. He had many hundreds of disciples, and he taught them like this: ‘Brahmins, life as a human is short, brief, and fleeting, full of suffering and distress. Be thoughtful and wake up! Do what’s good and lead the spiritual life, for no-one born can escape death. 

It’s\marginnote{2.1} like a drop of dew on a grass tip. When the sun comes up it quickly evaporates and doesn’t last long. In the same way, life as a human is like a dew-drop. It’s brief and fleeting, full of suffering and distress. Be thoughtful and wake up! Do what’s good and lead the spiritual life, for no-one born can escape death. 

It’s\marginnote{3.1} like when the rain falls heavily. The bubbles quickly vanish and don’t last long. In the same way, life as a human is like a bubble. … 

It’s\marginnote{4.1} like a line drawn in water. It vanishes quickly and doesn’t last long. In the same way, life as a human is like a line drawn in water. … 

It’s\marginnote{5.1} like a mountain river traveling far, flowing fast, carrying all before it. It doesn’t turn back—not for a moment, a second, an instant—but runs, rolls, and flows on. In the same way, life as a human is like a mountain river. … 

It’s\marginnote{6.1} like a strong man who has formed a glob of spit on the tip of his tongue. He could easily spit it out. In the same way, life as a human is like a glob of spit. … 

Suppose\marginnote{7.1} there was an iron cauldron that had been heated all day. If you tossed a lump of meat in, it would quickly vanish and not last long. In the same way, life as a human is like a lump of meat. … 

It’s\marginnote{8.1} like a cow being led to the slaughterhouse. With every step she comes closer to the slaughter, closer to death. In the same way, life as a human is like a cow being slaughtered. It’s brief and fleeting, full of suffering and distress. Be thoughtful and wake up! Do what’s good and lead the spiritual life, for no-one born can escape death.’ 

Now,\marginnote{9.1} mendicants, at that time human beings had a life span of 60,000 years. Girls could be married at 500 years of age. And human beings only had six afflictions: cold, heat, hunger, thirst, and the need to defecate and urinate. But even though humans were so long-lived with so few afflictions, Araka still taught in this way: ‘Life as a human is short, brief, and fleeting, full of suffering and distress. Be thoughtful and wake up! Do what’s good and lead the spiritual life, for no-one born can escape death.’ 

These\marginnote{10.1} days it’d be right to say: ‘Life as a human is short, brief, and fleeting, full of suffering and distress. Be thoughtful and wake up! Do what’s good and lead the spiritual life, for no-one born can escape death.’ For these days a long life is a hundred years or a little more. Living for a hundred years, there are just three hundred seasons, a hundred each of the winter, summer, and rains. Living for three hundred seasons, there are just twelve hundred months, four hundred in each of the winter, summer, and rains. Living for twelve hundred months, there are just twenty-four hundred fortnights, eight hundred in each of the winter, summer, and rains. Living for 2,400 fortnights, there are just 36,000 days, 12,000 in each of the summer, winter, and rains. Living for 36,000 days, you just eat 72,000 meals, 24,000 in each of the summer, winter, and rains, including when you’re suckling at the breast, and when you’re prevented from eating. 

Things\marginnote{11.1} that prevent you from eating include anger, pain, sickness, sabbath, or being unable to get food. So mendicants, for a human being with a hundred years life span I have counted the life span, the limit of the life span, the seasons, the years, the months, the fortnights, the nights, the days, the meals, and the things that prevent them from eating. Out of compassion, I’ve done what a teacher should do who wants what’s best for their disciples. Here are these roots of trees, and here are these empty huts. Practice absorption, mendicants! Don’t be negligent! Don’t regret it later! This is my instruction to you.” 

%
\addtocontents{toc}{\let\protect\contentsline\protect\nopagecontentsline}
\chapter*{The Chapter on the Monastic Law }
\addcontentsline{toc}{chapter}{\tocchapterline{The Chapter on the Monastic Law }}
\addtocontents{toc}{\let\protect\contentsline\protect\oldcontentsline}

%
\section*{{\suttatitleacronym AN 7.75}{\suttatitletranslation An Expert in the Monastic Law (1st) }{\suttatitleroot Paṭhamavinayadharasutta}}
\addcontentsline{toc}{section}{\tocacronym{AN 7.75} \toctranslation{An Expert in the Monastic Law (1st) } \tocroot{Paṭhamavinayadharasutta}}
\markboth{An Expert in the Monastic Law (1st) }{Paṭhamavinayadharasutta}
\extramarks{AN 7.75}{AN 7.75}

“Mendicants,\marginnote{1.1} a mendicant with seven qualities is an expert in the monastic law. What seven? They know what is an offense. They know what is not an offense. They know what is a light offense. They know what is a serious offense. They’re ethical, restrained in the monastic code, conducting themselves well and seeking alms in suitable places. Seeing danger in the slightest fault, they keep the rules they’ve undertaken. They get the four absorptions—blissful meditations in the present life that belong to the higher mind—when they want, without trouble or difficulty. They realize the undefiled freedom of heart and freedom by wisdom in this very life, and live having realized it with their own insight due to the ending of defilements. A mendicant with these seven qualities is an expert in the monastic law.” 

%
\section*{{\suttatitleacronym AN 7.76}{\suttatitletranslation An Expert in the Monastic Law (2nd) }{\suttatitleroot Dutiyavinayadharasutta}}
\addcontentsline{toc}{section}{\tocacronym{AN 7.76} \toctranslation{An Expert in the Monastic Law (2nd) } \tocroot{Dutiyavinayadharasutta}}
\markboth{An Expert in the Monastic Law (2nd) }{Dutiyavinayadharasutta}
\extramarks{AN 7.76}{AN 7.76}

“Mendicants,\marginnote{1.1} a mendicant with seven qualities is an expert in the monastic law. What seven? They know what is an offense. They know what is not an offense. They know what is a light offense. They know what is a serious offense. Both monastic codes have been passed down to them in detail, well analyzed, well mastered, well judged in both the rules and accompanying material. They get the four absorptions—blissful meditations in the present life that belong to the higher mind—when they want, without trouble or difficulty. They realize the undefiled freedom of heart and freedom by wisdom in this very life, and live having realized it with their own insight due to the ending of defilements. A mendicant with these seven qualities is an expert in the monastic law.” 

%
\section*{{\suttatitleacronym AN 7.77}{\suttatitletranslation An Expert in the Monastic Law (3rd) }{\suttatitleroot Tatiyavinayadharasutta}}
\addcontentsline{toc}{section}{\tocacronym{AN 7.77} \toctranslation{An Expert in the Monastic Law (3rd) } \tocroot{Tatiyavinayadharasutta}}
\markboth{An Expert in the Monastic Law (3rd) }{Tatiyavinayadharasutta}
\extramarks{AN 7.77}{AN 7.77}

“Mendicants,\marginnote{1.1} a mendicant with seven qualities is an expert in the monastic law. What seven? They know what is an offense. They know what is not an offense. They know what is a light offense. They know what is a serious offense. They’re firm and unfaltering in the training. They get the four absorptions—blissful meditations in the present life that belong to the higher mind—when they want, without trouble or difficulty. They realize the undefiled freedom of heart and freedom by wisdom in this very life, and live having realized it with their own insight due to the ending of defilements. A mendicant with these seven qualities is an expert in the monastic law.” 

%
\section*{{\suttatitleacronym AN 7.78}{\suttatitletranslation An Expert in the Monastic Law (4th) }{\suttatitleroot Catutthavinayadharasutta}}
\addcontentsline{toc}{section}{\tocacronym{AN 7.78} \toctranslation{An Expert in the Monastic Law (4th) } \tocroot{Catutthavinayadharasutta}}
\markboth{An Expert in the Monastic Law (4th) }{Catutthavinayadharasutta}
\extramarks{AN 7.78}{AN 7.78}

“Mendicants,\marginnote{1.1} a mendicant with seven qualities is an expert in the monastic law. What seven? They know what is an offense. They know what is not an offense. They know what is a light offense. They know what is a serious offense. They recollect their many kinds of past lives, with features and details. With clairvoyance that is purified and superhuman, they understand how sentient beings are reborn according to their deeds. They realize the undefiled freedom of heart and freedom by wisdom in this very life, and live having realized it with their own insight due to the ending of defilements. A mendicant with these seven qualities is an expert in the monastic law.” 

%
\section*{{\suttatitleacronym AN 7.79}{\suttatitletranslation Shines as an Expert in the Monastic Law (1st) }{\suttatitleroot Paṭhamavinayadharasobhanasutta}}
\addcontentsline{toc}{section}{\tocacronym{AN 7.79} \toctranslation{Shines as an Expert in the Monastic Law (1st) } \tocroot{Paṭhamavinayadharasobhanasutta}}
\markboth{Shines as an Expert in the Monastic Law (1st) }{Paṭhamavinayadharasobhanasutta}
\extramarks{AN 7.79}{AN 7.79}

“Mendicants,\marginnote{1.1} a mendicant with seven qualities shines as an expert in the monastic law. What seven? They know what is an offense. They know what is not an offense. They know what is a light offense. They know what is a serious offense. They’re ethical, restrained in the code of conduct, conducting themselves well and seeking alms in suitable places; seeing danger in the slightest fault, they keep the rules they’ve undertaken. They get the four absorptions—blissful meditations in the present life that belong to the higher mind—when they want, without trouble or difficulty. They realize the undefiled freedom of heart and freedom by wisdom in this very life, and live having realized it with their own insight due to the ending of defilements. A mendicant with these seven qualities shines as an expert in the monastic law.” 

%
\section*{{\suttatitleacronym AN 7.80}{\suttatitletranslation Shines as an Expert in the Monastic Law (2nd) }{\suttatitleroot Dutiyavinayadharasobhanasutta}}
\addcontentsline{toc}{section}{\tocacronym{AN 7.80} \toctranslation{Shines as an Expert in the Monastic Law (2nd) } \tocroot{Dutiyavinayadharasobhanasutta}}
\markboth{Shines as an Expert in the Monastic Law (2nd) }{Dutiyavinayadharasobhanasutta}
\extramarks{AN 7.80}{AN 7.80}

“Mendicants,\marginnote{1.1} a mendicant with seven qualities shines as an expert in the monastic law. What seven? They know what is an offense. They know what is not an offense. They know what is a light offense. They know what is a serious offense. Both monastic codes have been passed down to them in detail, well analyzed, well mastered, well judged in both the rules and accompanying material. They get the four absorptions—blissful meditations in the present life that belong to the higher mind—when they want, without trouble or difficulty. They realize the undefiled freedom of heart and freedom by wisdom in this very life, and live having realized it with their own insight due to the ending of defilements. A mendicant with these seven qualities shines as an expert in the monastic law.” 

%
\section*{{\suttatitleacronym AN 7.81}{\suttatitletranslation Shines as an Expert in the Monastic Law (3rd) }{\suttatitleroot Tatiyavinayadharasobhanasutta}}
\addcontentsline{toc}{section}{\tocacronym{AN 7.81} \toctranslation{Shines as an Expert in the Monastic Law (3rd) } \tocroot{Tatiyavinayadharasobhanasutta}}
\markboth{Shines as an Expert in the Monastic Law (3rd) }{Tatiyavinayadharasobhanasutta}
\extramarks{AN 7.81}{AN 7.81}

“Mendicants,\marginnote{1.1} a mendicant with seven qualities shines as an expert in the monastic law. What seven? They know what is an offense. They know what is not an offense. They know what is a light offense. They know what is a serious offense. They’re firm and unfaltering in the training. They get the four absorptions—blissful meditations in the present life that belong to the higher mind—when they want, without trouble or difficulty. They realize the undefiled freedom of heart and freedom by wisdom in this very life, and live having realized it with their own insight due to the ending of defilements. A mendicant with these seven qualities shines as an expert in the monastic law.” 

%
\section*{{\suttatitleacronym AN 7.82}{\suttatitletranslation Shines as an Expert in the Monastic Law (4th) }{\suttatitleroot Catutthavinayadharasobhanasutta}}
\addcontentsline{toc}{section}{\tocacronym{AN 7.82} \toctranslation{Shines as an Expert in the Monastic Law (4th) } \tocroot{Catutthavinayadharasobhanasutta}}
\markboth{Shines as an Expert in the Monastic Law (4th) }{Catutthavinayadharasobhanasutta}
\extramarks{AN 7.82}{AN 7.82}

“Mendicants,\marginnote{1.1} a mendicant with seven qualities shines as an expert in the monastic law. What seven? They know what is an offense. They know what is not an offense. They know what is a light offense. They know what is a serious offense. They recollect their many kinds of past lives, with features and details. With clairvoyance that is purified and superhuman, they understand how sentient beings are reborn according to their deeds. They realize the undefiled freedom of heart and freedom by wisdom in this very life, and live having realized it with their own insight due to the ending of defilements. A mendicant with these seven qualities shines as an expert in the monastic law.” 

%
\section*{{\suttatitleacronym AN 7.83}{\suttatitletranslation The Teacher’s Instructions }{\suttatitleroot Satthusāsanasutta}}
\addcontentsline{toc}{section}{\tocacronym{AN 7.83} \toctranslation{The Teacher’s Instructions } \tocroot{Satthusāsanasutta}}
\markboth{The Teacher’s Instructions }{Satthusāsanasutta}
\extramarks{AN 7.83}{AN 7.83}

Then\marginnote{1.1} Venerable \textsanskrit{Upāli} went up to the Buddha, bowed, sat down to one side, and said to him: 

“Sir,\marginnote{2.1} may the Buddha please teach me Dhamma in brief. When I’ve heard it, I’ll live alone, withdrawn, diligent, keen, and resolute.” 

“\textsanskrit{Upāli},\marginnote{2.2} you might know that certain things don’t lead solely to disillusionment, dispassion, cessation, peace, insight, awakening, and extinguishment. You should definitely bear in mind that such things are not the teaching, not the training, and not the Teacher’s instructions. You might know that certain things do lead solely to disillusionment, dispassion, cessation, peace, insight, awakening, and extinguishment. You should definitely bear in mind that such things are the teaching, the training, and the Teacher’s instructions.” 

%
\section*{{\suttatitleacronym AN 7.84}{\suttatitletranslation Settlement of Disciplinary Issues }{\suttatitleroot Adhikaraṇasamathasutta}}
\addcontentsline{toc}{section}{\tocacronym{AN 7.84} \toctranslation{Settlement of Disciplinary Issues } \tocroot{Adhikaraṇasamathasutta}}
\markboth{Settlement of Disciplinary Issues }{Adhikaraṇasamathasutta}
\extramarks{AN 7.84}{AN 7.84}

“Mendicants,\marginnote{1.1} there are these seven principles for the settlement of any disciplinary issues that might arise. What seven? Removal in the presence of those concerned is applicable. Removal by accurate recollection is applicable. Removal due to recovery from madness is applicable. The acknowledgement of the offense is applicable. The decision of a majority is applicable. A verdict of aggravated misconduct is applicable. Covering over with grass is applicable. These are the seven principles for the settlement of any disciplinary issues that might arise.” 

%
\addtocontents{toc}{\let\protect\contentsline\protect\nopagecontentsline}
\chapter*{The Chapter on Ascetics }
\addcontentsline{toc}{chapter}{\tocchapterline{The Chapter on Ascetics }}
\addtocontents{toc}{\let\protect\contentsline\protect\oldcontentsline}

%
\section*{{\suttatitleacronym AN 7.85}{\suttatitletranslation A Mendicant }{\suttatitleroot Bhikkhusutta}}
\addcontentsline{toc}{section}{\tocacronym{AN 7.85} \toctranslation{A Mendicant } \tocroot{Bhikkhusutta}}
\markboth{A Mendicant }{Bhikkhusutta}
\extramarks{AN 7.85}{AN 7.85}

“Mendicants,\marginnote{1.1} it’s because of breaking seven things that you become a mendicant. What seven? Identity view, doubt, misapprehension of precepts and observances, greed, hate, delusion, and conceit. It’s because of breaking these seven things that you become a mendicant.” 

%
\section*{{\suttatitleacronym AN 7.86}{\suttatitletranslation An Ascetic }{\suttatitleroot Samaṇasutta}}
\addcontentsline{toc}{section}{\tocacronym{AN 7.86} \toctranslation{An Ascetic } \tocroot{Samaṇasutta}}
\markboth{An Ascetic }{Samaṇasutta}
\extramarks{AN 7.86}{AN 7.86}

“Mendicants,\marginnote{1.1} it’s because of quelling seven things that you become an ascetic …” 

%
\section*{{\suttatitleacronym AN 7.87}{\suttatitletranslation Brahmin }{\suttatitleroot Brāhmaṇasutta}}
\addcontentsline{toc}{section}{\tocacronym{AN 7.87} \toctranslation{Brahmin } \tocroot{Brāhmaṇasutta}}
\markboth{Brahmin }{Brāhmaṇasutta}
\extramarks{AN 7.87}{AN 7.87}

“Mendicants,\marginnote{1.1} it’s because of barring out seven things that you become a brahmin …” 

%
\section*{{\suttatitleacronym AN 7.88}{\suttatitletranslation Scholar }{\suttatitleroot Sottiyasutta}}
\addcontentsline{toc}{section}{\tocacronym{AN 7.88} \toctranslation{Scholar } \tocroot{Sottiyasutta}}
\markboth{Scholar }{Sottiyasutta}
\extramarks{AN 7.88}{AN 7.88}

“Mendicants,\marginnote{1.1} it’s because of scouring off seven things that you become a scholar …” 

%
\section*{{\suttatitleacronym AN 7.89}{\suttatitletranslation Bathed }{\suttatitleroot Nhātakasutta}}
\addcontentsline{toc}{section}{\tocacronym{AN 7.89} \toctranslation{Bathed } \tocroot{Nhātakasutta}}
\markboth{Bathed }{Nhātakasutta}
\extramarks{AN 7.89}{AN 7.89}

“Mendicants,\marginnote{1.1} it’s because of bathing off seven things that you become a bathed initiate …” 

%
\section*{{\suttatitleacronym AN 7.90}{\suttatitletranslation A Knowledge Master }{\suttatitleroot Vedagūsutta}}
\addcontentsline{toc}{section}{\tocacronym{AN 7.90} \toctranslation{A Knowledge Master } \tocroot{Vedagūsutta}}
\markboth{A Knowledge Master }{Vedagūsutta}
\extramarks{AN 7.90}{AN 7.90}

“Mendicants,\marginnote{1.1} it’s because of knowing seven things that you become a knowledge master …” 

%
\section*{{\suttatitleacronym AN 7.91}{\suttatitletranslation A Noble One }{\suttatitleroot Ariyasutta}}
\addcontentsline{toc}{section}{\tocacronym{AN 7.91} \toctranslation{A Noble One } \tocroot{Ariyasutta}}
\markboth{A Noble One }{Ariyasutta}
\extramarks{AN 7.91}{AN 7.91}

“Mendicants,\marginnote{1.1} it’s because seven foes have been slain that you become a noble one …” 

%
\section*{{\suttatitleacronym AN 7.92}{\suttatitletranslation A Perfected One }{\suttatitleroot Arahāsutta}}
\addcontentsline{toc}{section}{\tocacronym{AN 7.92} \toctranslation{A Perfected One } \tocroot{Arahāsutta}}
\markboth{A Perfected One }{Arahāsutta}
\extramarks{AN 7.92}{AN 7.92}

“Mendicants,\marginnote{1.1} it’s by being far from seven things that you become a perfected one. What seven? Identity view, doubt, misapprehension of precepts and observances, greed, hate, delusion, and conceit. It’s because of being far from these seven things that you become a perfected one.” 

%
\section*{{\suttatitleacronym AN 7.93}{\suttatitletranslation Bad Qualities }{\suttatitleroot Asaddhammasutta}}
\addcontentsline{toc}{section}{\tocacronym{AN 7.93} \toctranslation{Bad Qualities } \tocroot{Asaddhammasutta}}
\markboth{Bad Qualities }{Asaddhammasutta}
\extramarks{AN 7.93}{AN 7.93}

“Mendicants,\marginnote{1.1} there are these seven bad qualities. What seven? Someone is faithless, shameless, imprudent, unlearned, lazy, unmindful, and witless. These are the seven bad qualities.” 

%
\section*{{\suttatitleacronym AN 7.94}{\suttatitletranslation Good Qualities }{\suttatitleroot Saddhammasutta}}
\addcontentsline{toc}{section}{\tocacronym{AN 7.94} \toctranslation{Good Qualities } \tocroot{Saddhammasutta}}
\markboth{Good Qualities }{Saddhammasutta}
\extramarks{AN 7.94}{AN 7.94}

“Mendicants,\marginnote{1.1} there are these seven good qualities. What seven? Someone is faithful, conscientious, prudent, learned, energetic, mindful, and wise. These are the seven good qualities.” 

%
\addtocontents{toc}{\let\protect\contentsline\protect\nopagecontentsline}
\chapter*{The Chapter on Worthy of Offerings }
\addcontentsline{toc}{chapter}{\tocchapterline{The Chapter on Worthy of Offerings }}
\addtocontents{toc}{\let\protect\contentsline\protect\oldcontentsline}

%
\section*{{\suttatitleacronym AN 7.95}{\suttatitletranslation Observing Impermanence in the Eye }{\suttatitleroot \textasciitilde }}
\addcontentsline{toc}{section}{\tocacronym{AN 7.95} \toctranslation{Observing Impermanence in the Eye } \tocroot{\textasciitilde }}
\markboth{Observing Impermanence in the Eye }{\textasciitilde }
\extramarks{AN 7.95}{AN 7.95}

“Mendicants,\marginnote{1.1} these seven people are worthy of offerings dedicated to the gods, worthy of hospitality, worthy of a religious donation, worthy of greeting with joined palms, and are the supreme field of merit for the world. What seven? 

First,\marginnote{1.3} take a person who meditates observing impermanence in the eye. They perceive impermanence and experience impermanence. Constantly, continually, and without interruption, they apply the mind and fathom with wisdom. They realize the undefiled freedom of heart and freedom by wisdom in this very life. And they live having realized it with their own insight due to the ending of defilements. This is the first person who is worthy of offerings dedicated to the gods, worthy of hospitality, worthy of a religious donation, worthy of greeting with joined palms, and is the supreme field of merit for the world. 

Next,\marginnote{2.1} take a person who meditates observing impermanence in the eye. … Their defilements and their life come to an end at exactly the same time. This is the second person who is worthy of offerings … 

Next,\marginnote{3.1} take a person who meditates observing impermanence in the eye. … With the ending of the five lower fetters they’re extinguished between one life and the next. … 

With\marginnote{3.3} the ending of the five lower fetters they’re extinguished upon landing. … 

With\marginnote{3.4} the ending of the five lower fetters they’re extinguished without extra effort. … 

With\marginnote{3.5} the ending of the five lower fetters they’re extinguished with extra effort. … 

With\marginnote{3.6} the ending of the five lower fetters they head upstream, going to the \textsanskrit{Akaniṭṭha} realm. … This is the seventh person. 

These\marginnote{3.8} are the seven people who are worthy of offerings dedicated to the gods, worthy of hospitality, worthy of a religious donation, worthy of greeting with joined palms, and are the supreme field of merit for the world.” 

%
\section*{{\suttatitleacronym AN 7.96–614}{\suttatitletranslation Observing Suffering in the Eye, Etc. }{\suttatitleroot \textasciitilde }}
\addcontentsline{toc}{section}{\tocacronym{AN 7.96–614} \toctranslation{Observing Suffering in the Eye, Etc. } \tocroot{\textasciitilde }}
\markboth{Observing Suffering in the Eye, Etc. }{\textasciitilde }
\extramarks{AN 7.96–614}{AN 7.96–614}

“Mendicants,\marginnote{1.1} these seven people are worthy of offerings … What seven? 

First,\marginnote{1.3} take a person who meditates observing suffering in the eye. … observing not-self in the eye. … observing ending in the eye. … observing vanishing in the eye. … observing fading away in the eye. … observing cessation in the eye. … observing letting go in the eye. … 

ear\marginnote{2.1} … nose … tongue … body … mind … 

sights\marginnote{3.1} … sounds … smells … tastes … touches … thoughts … 

eye\marginnote{4.1} consciousness … ear consciousness … nose consciousness … tongue consciousness … body consciousness … mind consciousness … 

eye\marginnote{5.1} contact … ear contact … nose contact … tongue contact … body contact … mind contact … 

feeling\marginnote{6.1} born of eye contact … feeling born of ear contact … feeling born of nose contact … feeling born of tongue contact … feeling born of body contact … feeling born of mind contact … 

perception\marginnote{7.1} of sights … perception of sounds … perception of smells … perception of tastes … perception of touches … perception of thoughts … 

intention\marginnote{8.1} regarding sights … intention regarding sounds … intention regarding smells … intention regarding tastes … intention regarding touches … intention regarding thoughts … 

craving\marginnote{9.1} for sights … craving for sounds … craving for smells … craving for tastes … craving for touches … craving for thoughts … 

thoughts\marginnote{10.1} about sights … thoughts about sounds … thoughts about smells … thoughts about tastes … thoughts about touches … thoughts about thoughts … 

considerations\marginnote{11.1} regarding sights … considerations regarding sounds … considerations regarding smells … considerations regarding tastes … considerations regarding touches … considerations regarding thoughts … 

meditates\marginnote{12.1} observing impermanence in the five aggregates … the aggregate of form … the aggregate of feeling … the aggregate of perception … the aggregate of choices … the aggregate of consciousness … meditates observing suffering … not-self … ending … vanishing … fading away … cessation … letting go …” 

%
\addtocontents{toc}{\let\protect\contentsline\protect\nopagecontentsline}
\chapter*{Abbreviated Texts Beginning With Greed }
\addcontentsline{toc}{chapter}{\tocchapterline{Abbreviated Texts Beginning With Greed }}
\addtocontents{toc}{\let\protect\contentsline\protect\oldcontentsline}

%
\section*{{\suttatitleacronym AN 7.615}{\suttatitletranslation Untitled Discourse on Greed (1st) }{\suttatitleroot \textasciitilde }}
\addcontentsline{toc}{section}{\tocacronym{AN 7.615} \toctranslation{Untitled Discourse on Greed (1st) } \tocroot{\textasciitilde }}
\markboth{Untitled Discourse on Greed (1st) }{\textasciitilde }
\extramarks{AN 7.615}{AN 7.615}

“Mendicants,\marginnote{1.1} for insight into greed, seven things should be developed. What seven? The awakening factor of mindfulness … the awakening factor of equanimity. These seven things should be developed for insight into greed.” 

%
\section*{{\suttatitleacronym AN 7.616}{\suttatitletranslation Untitled Discourse on Greed (2nd) }{\suttatitleroot \textasciitilde }}
\addcontentsline{toc}{section}{\tocacronym{AN 7.616} \toctranslation{Untitled Discourse on Greed (2nd) } \tocroot{\textasciitilde }}
\markboth{Untitled Discourse on Greed (2nd) }{\textasciitilde }
\extramarks{AN 7.616}{AN 7.616}

“Mendicants,\marginnote{1.1} for insight into greed, seven things should be developed. What seven? The perception of impermanence, the perception of not-self, the perception of ugliness, the perception of drawbacks, the perception of giving up, the perception of fading away, and the perception of cessation. These seven things should be developed for insight into greed.” 

%
\section*{{\suttatitleacronym AN 7.617}{\suttatitletranslation Untitled Discourse on Greed (3rd) }{\suttatitleroot \textasciitilde }}
\addcontentsline{toc}{section}{\tocacronym{AN 7.617} \toctranslation{Untitled Discourse on Greed (3rd) } \tocroot{\textasciitilde }}
\markboth{Untitled Discourse on Greed (3rd) }{\textasciitilde }
\extramarks{AN 7.617}{AN 7.617}

“Mendicants,\marginnote{1.1} for insight into greed, seven things should be developed. What seven? The perceptions of ugliness, death, repulsiveness of food, dissatisfaction with the whole world, impermanence, suffering in impermanence, and not-self in suffering. These seven things should be developed for insight into greed.” 

%
\section*{{\suttatitleacronym AN 7.618–644}{\suttatitletranslation Untitled Discourses on Greed }{\suttatitleroot \textasciitilde }}
\addcontentsline{toc}{section}{\tocacronym{AN 7.618–644} \toctranslation{Untitled Discourses on Greed } \tocroot{\textasciitilde }}
\markboth{Untitled Discourses on Greed }{\textasciitilde }
\extramarks{AN 7.618–644}{AN 7.618–644}

“For\marginnote{1.1} the complete understanding of greed … complete ending … giving up … ending … vanishing … fading away … cessation … giving away … For the letting go of greed, these seven things should be developed.” 

%
\section*{{\suttatitleacronym AN 7.645–1124}{\suttatitletranslation Untitled Discourses on Hate, Etc. }{\suttatitleroot \textasciitilde }}
\addcontentsline{toc}{section}{\tocacronym{AN 7.645–1124} \toctranslation{Untitled Discourses on Hate, Etc. } \tocroot{\textasciitilde }}
\markboth{Untitled Discourses on Hate, Etc. }{\textasciitilde }
\extramarks{AN 7.645–1124}{AN 7.645–1124}

“Of\marginnote{1.1} hate … delusion … anger … hostility … disdain … contempt … jealousy … stinginess … deceitfulness … deviousness … obstinacy … aggression … conceit … arrogance … vanity … for insight into negligence … complete understanding … complete ending … giving up … ending … vanishing … fading away … cessation … giving away … letting go of negligence these seven things should be developed.” 

That\marginnote{2.1} is what the Buddha said. Satisfied, the mendicants were happy with what the Buddha said. 

\scendbook{The Book of the Sevens is finished. }

%
\addtocontents{toc}{\let\protect\contentsline\protect\nopagecontentsline}
\part*{The Book of the Eights }
\addcontentsline{toc}{part}{The Book of the Eights }
\markboth{}{}
\addtocontents{toc}{\let\protect\contentsline\protect\oldcontentsline}

%
%
\addtocontents{toc}{\let\protect\contentsline\protect\nopagecontentsline}
\pannasa{The First Fifty }
\addcontentsline{toc}{pannasa}{The First Fifty }
\markboth{}{}
\addtocontents{toc}{\let\protect\contentsline\protect\oldcontentsline}

%
\addtocontents{toc}{\let\protect\contentsline\protect\nopagecontentsline}
\chapter*{The Chapter on Love }
\addcontentsline{toc}{chapter}{\tocchapterline{The Chapter on Love }}
\addtocontents{toc}{\let\protect\contentsline\protect\oldcontentsline}

%
\section*{{\suttatitleacronym AN 8.1}{\suttatitletranslation The Benefits of Love }{\suttatitleroot Mettāsutta}}
\addcontentsline{toc}{section}{\tocacronym{AN 8.1} \toctranslation{The Benefits of Love } \tocroot{Mettāsutta}}
\markboth{The Benefits of Love }{Mettāsutta}
\extramarks{AN 8.1}{AN 8.1}

\scevam{So\marginnote{1.1} I have heard. }At one time the Buddha was staying near \textsanskrit{Sāvatthī} in Jeta’s Grove, \textsanskrit{Anāthapiṇḍika}’s monastery. There the Buddha addressed the mendicants, “Mendicants!” 

“Venerable\marginnote{1.5} sir,” they replied. The Buddha said this: 

“Mendicants,\marginnote{2.1} you can expect eight benefits when the heart’s release by love has been cultivated, developed, and practiced, made a vehicle and a basis, kept up, consolidated, and properly implemented. What eight? You sleep at ease. You wake happily. You don’t have bad dreams. Humans love you. Non-humans love you. Deities protect you. You can’t be harmed by fire, poison, or blade. If you don’t reach any higher, you’ll be reborn in a \textsanskrit{Brahmā} realm. You can expect these eight benefits when the heart’s release by love has been cultivated, developed, and practiced, made a vehicle and a basis, kept up, consolidated, and properly implemented. 

\begin{verse}%
A\marginnote{3.1} mindful one who develops \\
limitless love \\
weakens the fetters, \\
seeing the ending of attachments. 

Loving\marginnote{4.1} just one creature with a hateless heart \\
makes you a good person. \\
Compassionate for all creatures, \\
a noble one creates abundant merit. 

The\marginnote{5.1} royal potentates conquered this land \\
and traveled around sponsoring sacrifices—\\
horse sacrifice, human sacrifice, \\
the sacrifices of the ‘stick-casting’, the ‘royal soma drinking’, and the ‘unbarred’. 

These\marginnote{6.1} are not worth a sixteenth part \\
of the mind developed with love, \\
as all the constellations of stars \\
aren’t worth a sixteenth part of the moon’s light. 

Don’t\marginnote{7.1} kill or cause others to kill, \\
don’t conquer or encourage others to conquer, \\
with love for all sentient beings, \\
you’ll have no enmity for anyone.” 

%
\end{verse}

%
\section*{{\suttatitleacronym AN 8.2}{\suttatitletranslation Wisdom }{\suttatitleroot Paññāsutta}}
\addcontentsline{toc}{section}{\tocacronym{AN 8.2} \toctranslation{Wisdom } \tocroot{Paññāsutta}}
\markboth{Wisdom }{Paññāsutta}
\extramarks{AN 8.2}{AN 8.2}

“Mendicants,\marginnote{1.1} there are eight causes and reasons that lead to acquiring the wisdom fundamental to the spiritual life, and to its increase, growth, and full development once it has been acquired. What eight? 

It’s\marginnote{1.3} when a mendicant lives relying on the Teacher or a spiritual companion in a teacher’s role. And they set up a keen sense of conscience and prudence for them, with warmth and respect. This is the first cause. 

When\marginnote{2.1} a mendicant lives relying on the Teacher or a spiritual companion in a teacher’s role—with a keen sense of conscience and prudence for them, with warmth and respect—from time to time they go and ask them questions: ‘Why, sir, does it say this? What does that mean?’ Those venerables clarify what is unclear, reveal what is obscure, and dispel doubt regarding the many doubtful matters. This is the second cause. 

After\marginnote{3.1} hearing that teaching they perfect withdrawal of both body and mind. This is the third cause. 

A\marginnote{4.1} mendicant is ethical, restrained in the monastic code, conducting themselves well and seeking alms in suitable places. Seeing danger in the slightest fault, they keep the rules they’ve undertaken. This is the fourth cause. 

They’re\marginnote{5.1} very learned, remembering and keeping what they’ve learned. These teachings are good in the beginning, good in the middle, and good in the end, meaningful and well-phrased, describing a spiritual practice that’s entirely full and pure. They are very learned in such teachings, remembering them, reinforcing them by recitation, mentally scrutinizing them, and comprehending them theoretically. This is the fifth cause. 

They\marginnote{6.1} live with energy roused up for giving up unskillful qualities and embracing skillful qualities. They’re strong, staunchly vigorous, not slacking off when it comes to developing skillful qualities. This is the sixth cause. 

When\marginnote{7.1} in the \textsanskrit{Saṅgha} they don’t engage in motley talk or unworthy talk. Either they talk on Dhamma, or they invite someone else to do so, or they respect noble silence. This is the seventh cause. 

They\marginnote{8.1} meditate observing rise and fall in the five grasping aggregates. ‘Such is form, such is the origin of form, such is the ending of form. Such is feeling, such is the origin of feeling, such is the ending of feeling. Such is perception, such is the origin of perception, such is the ending of perception. Such are choices, such is the origin of choices, such is the ending of choices. Such is consciousness, such is the origin of consciousness, such is the ending of consciousness.’ This is the eighth cause. 

Their\marginnote{9.1} spiritual companions esteem them: ‘This venerable lives relying on the Teacher or a spiritual companion in a teacher’s role. They set up a keen sense of conscience and prudence for them, with warmth and respect. Clearly this venerable knows and sees.’ This quality leads to fondness, respect, esteem, harmony, and unity. 

‘This\marginnote{10.1} venerable lives relying on the Teacher or a spiritual companion in a teacher’s role, and from time to time they go and ask them questions … Clearly this venerable knows and sees.’ This quality also leads to fondness, respect, esteem, harmony, and unity. 

‘After\marginnote{11.1} hearing that teaching they perfect withdrawal of both body and mind. Clearly this venerable knows and sees.’ This quality also leads to fondness, respect, esteem, harmony, and unity. 

‘This\marginnote{12.1} venerable is ethical … Clearly this venerable knows and sees.’ This quality also leads to fondness, respect, esteem, harmony, and unity. 

‘This\marginnote{13.1} venerable is very learned, remembering and keeping what they’ve learned. … Clearly this venerable knows and sees.’ This quality also leads to fondness, respect, esteem, harmony, and unity. 

‘This\marginnote{14.1} venerable lives with energy roused up … Clearly this venerable knows and sees.’ This quality also leads to fondness, respect, esteem, harmony, and unity. 

‘When\marginnote{15.1} in the \textsanskrit{Saṅgha} they don’t engage in motley talk or unworthy talk. Either they talk on Dhamma, or they invite someone else to do so, or they respect noble silence. Clearly this venerable knows and sees.’ This quality also leads to fondness, respect, esteem, harmony, and unity. 

‘They\marginnote{16.1} meditate observing rise and fall in the five grasping aggregates. … Clearly this venerable knows and sees.’ This quality also leads to fondness, respect, esteem, harmony, and unity. 

These\marginnote{17.1} are the eight causes and reasons that lead to acquiring the wisdom fundamental to the spiritual life, and to its increase, growth, and full development once it has been acquired.” 

%
\section*{{\suttatitleacronym AN 8.3}{\suttatitletranslation Disliked (1st) }{\suttatitleroot Paṭhamaappiyasutta}}
\addcontentsline{toc}{section}{\tocacronym{AN 8.3} \toctranslation{Disliked (1st) } \tocroot{Paṭhamaappiyasutta}}
\markboth{Disliked (1st) }{Paṭhamaappiyasutta}
\extramarks{AN 8.3}{AN 8.3}

“Mendicants,\marginnote{1.1} a mendicant with eight qualities is disliked and disapproved by their spiritual companions, not respected or admired. What eight? It’s when a mendicant praises the disliked and criticizes the liked. They desire material possessions and honor. They lack conscience and prudence. They have wicked desires and wrong view. A mendicant with these eight qualities is disliked and disapproved by their spiritual companions, not respected or admired. 

A\marginnote{2.1} mendicant with eight qualities is liked and approved by their spiritual companions, and respected and admired. What eight? It’s when a mendicant doesn’t praise the disliked and criticize the liked. They don’t desire material possessions and honor. They have conscience and prudence. They have few desires and right view. A mendicant with these eight qualities is liked and approved by their spiritual companions, and respected and admired.” 

%
\section*{{\suttatitleacronym AN 8.4}{\suttatitletranslation Disliked (2nd) }{\suttatitleroot Dutiyaappiyasutta}}
\addcontentsline{toc}{section}{\tocacronym{AN 8.4} \toctranslation{Disliked (2nd) } \tocroot{Dutiyaappiyasutta}}
\markboth{Disliked (2nd) }{Dutiyaappiyasutta}
\extramarks{AN 8.4}{AN 8.4}

“Mendicants,\marginnote{1.1} a mendicant with eight qualities is disliked and disapproved by their spiritual companions, not respected or admired. What eight? It’s when a mendicant desires material possessions, honor, and to be looked up to. They know neither moderation nor the proper time. Their conduct is impure, they talk a lot, and they insult and abuse their spiritual companions. A mendicant with these eight qualities is disliked and disapproved by their spiritual companions, not respected or admired. 

A\marginnote{2.1} mendicant with eight qualities is liked and approved by their spiritual companions, and respected and admired. What eight? It’s when a mendicant doesn’t desire material possessions, honor, and to be looked up to. They know moderation and the proper time. Their conduct is pure, they don’t talk a lot, and they don’t insult and abuse their spiritual companions. A mendicant with these eight qualities is liked and approved by their spiritual companions, and respected and admired.” 

%
\section*{{\suttatitleacronym AN 8.5}{\suttatitletranslation Worldly Conditions (1st) }{\suttatitleroot Paṭhamalokadhammasutta}}
\addcontentsline{toc}{section}{\tocacronym{AN 8.5} \toctranslation{Worldly Conditions (1st) } \tocroot{Paṭhamalokadhammasutta}}
\markboth{Worldly Conditions (1st) }{Paṭhamalokadhammasutta}
\extramarks{AN 8.5}{AN 8.5}

“Mendicants,\marginnote{1.1} the eight worldly conditions revolve around the world, and the world revolves around the eight worldly conditions. What eight? Gain and loss, fame and disgrace, blame and praise, pleasure and pain. These eight worldly conditions revolve around the world, and the world revolves around these eight worldly conditions. 

\begin{verse}%
Gain\marginnote{2.1} and loss, fame and disgrace, \\
blame and praise, and pleasure and pain. \\
These qualities among people are impermanent, \\
transient, and perishable. 

A\marginnote{3.1} clever and mindful person knows these things, \\
seeing that they’re perishable. \\
Desirable things don’t disturb their mind, \\
nor are they repelled by the undesirable. 

Both\marginnote{4.1} favoring and opposing \\
are cleared and ended, they are no more. \\
Knowing the stainless, sorrowless state, \\
they who have gone beyond rebirth understand rightly.” 

%
\end{verse}

%
\section*{{\suttatitleacronym AN 8.6}{\suttatitletranslation Worldly Conditions (2nd) }{\suttatitleroot Dutiyalokadhammasutta}}
\addcontentsline{toc}{section}{\tocacronym{AN 8.6} \toctranslation{Worldly Conditions (2nd) } \tocroot{Dutiyalokadhammasutta}}
\markboth{Worldly Conditions (2nd) }{Dutiyalokadhammasutta}
\extramarks{AN 8.6}{AN 8.6}

“Mendicants,\marginnote{1.1} the eight worldly conditions revolve around the world, and the world revolves around the eight worldly conditions. What eight? Gain and loss, fame and disgrace, blame and praise, pleasure and pain. These eight worldly conditions revolve around the world, and the world revolves around these eight worldly conditions. 

An\marginnote{2.1} unlearned ordinary person encounters gain and loss, fame and disgrace, blame and praise, and pleasure and pain. And so does a learned noble disciple. What, then, is the difference between an ordinary unlearned person and a learned noble disciple?” 

“Our\marginnote{2.4} teachings are rooted in the Buddha. He is our guide and our refuge. Sir, may the Buddha himself please clarify the meaning of this. The mendicants will listen and remember it.” 

“Well\marginnote{3.1} then, mendicants, listen and pay close attention, I will speak.” 

“Yes,\marginnote{3.2} sir,” they replied. The Buddha said this: 

“Mendicants,\marginnote{3.4} an unlearned ordinary person encounters gain. They don’t reflect: ‘I’ve encountered this gain. It’s impermanent, suffering, and perishable.’ They don’t truly understand it. They encounter loss … fame … disgrace … blame … praise … pleasure … pain. They don’t reflect: ‘I’ve encountered this pain. It’s impermanent, suffering, and perishable.’ They don’t truly understand it. 

So\marginnote{4.1} gain and loss, fame and disgrace, blame and praise, and pleasure and pain occupy their mind. They favor gain and oppose loss. They favor fame and oppose disgrace. They favor praise and oppose blame. They favor pleasure and oppose pain. Being so full of favoring and opposing, they’re not freed from rebirth, old age, and death, from sorrow, lamentation, pain, sadness, and distress. They’re not freed from suffering, I say. 

A\marginnote{5.1} learned noble disciple encounters gain. They reflect: ‘I’ve encountered this gain. It’s impermanent, suffering, and perishable.’ They truly understand it. They encounter loss … fame … disgrace … blame … praise … pleasure … pain. They reflect: ‘I’ve encountered this pain. It’s impermanent, suffering, and perishable.’ They truly understand it. 

So\marginnote{6.1} gain and loss, fame and disgrace, blame and praise, and pleasure and pain don’t occupy their mind. They don’t favor gain or oppose loss. They don’t favor fame or oppose disgrace. They don’t favor praise or oppose blame. They don’t favor pleasure or oppose pain. Having given up favoring and opposing, they’re freed from rebirth, old age, and death, from sorrow, lamentation, pain, sadness, and distress. They’re freed from suffering, I say. This is the difference between a learned noble disciple and an unlearned ordinary person. 

\begin{verse}%
Gain\marginnote{7.1} and loss, fame and disgrace, \\
blame and praise, and pleasure and pain. \\
These qualities among people are impermanent, \\
transient, and perishable. 

A\marginnote{8.1} clever and mindful person knows these things, \\
seeing that they’re perishable. \\
Desirable things don’t disturb their mind, \\
nor are they repelled by the undesirable. 

Both\marginnote{9.1} favoring and opposing \\
are cleared and ended, they are no more. \\
Knowing the stainless, sorrowless state, \\
they who have gone beyond rebirth understand rightly.” 

%
\end{verse}

%
\section*{{\suttatitleacronym AN 8.7}{\suttatitletranslation Devadatta’s Failure }{\suttatitleroot Devadattavipattisutta}}
\addcontentsline{toc}{section}{\tocacronym{AN 8.7} \toctranslation{Devadatta’s Failure } \tocroot{Devadattavipattisutta}}
\markboth{Devadatta’s Failure }{Devadattavipattisutta}
\extramarks{AN 8.7}{AN 8.7}

At\marginnote{1.1} one time the Buddha was staying near \textsanskrit{Rājagaha}, on the Vulture’s Peak Mountain, not long after Devadatta had left. There the Buddha spoke to the mendicants about Devadatta: 

“Mendicants,\marginnote{1.3} it’s good for a mendicant to check their own failings from time to time. It’s good for a mendicant to check the failings of others from time to time. It’s good for a mendicant to check their own successes from time to time. It’s good for a mendicant to check the successes of others from time to time. Overcome and overwhelmed by eight things that oppose the true teaching, Devadatta is going to a place of loss, to hell, there to remain for an eon, irredeemable. 

What\marginnote{2.1} eight? Overcome and overwhelmed by gain … loss … fame … disgrace … honor … dishonor … wicked desires … bad friendship, Devadatta is going to a place of loss, to hell, there to remain for an eon, irredeemable. Overcome and overwhelmed by these eight things that oppose the true teaching, Devadatta is going to a place of loss, to hell, there to remain for an eon, irredeemable. 

It’s\marginnote{3.1} good for a mendicant, whenever they encounter it, to overcome gain … loss … fame … disgrace … honor … dishonor … wicked desires … bad friendship. 

What\marginnote{4.1} advantage does a mendicant gain by overcoming these eight things? 

The\marginnote{5.1} distressing and feverish defilements that might arise in someone who lives without overcoming these eight things do not arise when they have overcome them. This is the advantage that a mendicant gains by overcoming these eight things. 

So,\marginnote{6.1} mendicants, you should train like this: ‘Whenever we encounter it, we will overcome gain … loss … fame … disgrace … honor … dishonor … wicked desires … bad friendship.’ That’s how you should train.” 

%
\section*{{\suttatitleacronym AN 8.8}{\suttatitletranslation Uttara on Failure }{\suttatitleroot Uttaravipattisutta}}
\addcontentsline{toc}{section}{\tocacronym{AN 8.8} \toctranslation{Uttara on Failure } \tocroot{Uttaravipattisutta}}
\markboth{Uttara on Failure }{Uttaravipattisutta}
\extramarks{AN 8.8}{AN 8.8}

At\marginnote{1.1} one time Venerable Uttara was staying on the \textsanskrit{Saṅkheyyaka} Mountain in the Mahisa region near \textsanskrit{Dhavajālikā}. There Uttara addressed the mendicants: “Mendicants, it’s good for a mendicant to check their own failings from time to time. It’s good for a mendicant to check the failings of others from time to time. It’s good for a mendicant to check their own successes from time to time. It’s good for a mendicant to check the successes of others from time to time.” 

Now\marginnote{2.1} at that time the great king \textsanskrit{Vessavaṇa} was on his way from the north to the south on some business. He heard Venerable Uttara teaching this to the mendicants on \textsanskrit{Saṅkheyyaka} Mountain. 

Then\marginnote{3.1} \textsanskrit{Vessavaṇa} vanished from \textsanskrit{Saṅkheyyaka} Mountain and appeared among the gods of the Thirty-Three, as easily as a strong person would extend or contract their arm. Then he went up to Sakka, lord of gods, and said to him: 

“Please\marginnote{3.3} sir, you should know this. Venerable Uttara is teaching the mendicants on \textsanskrit{Saṅkheyyaka} Mountain in this way: ‘It’s good for a mendicant from time to time to check their own failings. … the failings of others … their own successes … the successes of others.’” 

Then,\marginnote{4.1} as easily as a strong person would extend or contract their arm, Sakka vanished from the gods of the Thirty-Three and reappeared on \textsanskrit{Saṅkheyyaka} Mountain in front of Venerable Uttara. Then Sakka went up to Venerable Uttara, bowed, stood to one side, and said to him: 

“Is\marginnote{5.1} it really true, sir, that you teach the mendicants in this way: ‘It’s good for a mendicant from time to time to check their own failings … the failings of others … their own successes … the successes of others’?” 

“Indeed,\marginnote{5.6} lord of gods.” 

“Sir,\marginnote{5.7} did this teaching come to you from your own inspiration, or was it spoken by the Blessed One, the perfected one, the fully awakened Buddha?” 

“Well\marginnote{5.8} then, lord of gods, I shall give you a simile. For by means of a simile some sensible people understand the meaning of what is said. 

Suppose\marginnote{6.1} there was a large heap of grain not far from a town or village. And a large crowd were to take away grain with carrying poles, baskets, hip sacks, or their cupped hands. If someone were to go to that crowd and ask them where they got the grain from, how should that crowd rightly reply?” 

“Sir,\marginnote{6.6} they should reply that they took it from the large heap of grain.” 

“In\marginnote{6.7} the same way, lord of gods, whatever is well spoken is spoken by the Blessed One, the perfected one, the fully awakened Buddha. Both myself and others rely completely on that when we speak.” 

“It’s\marginnote{7.1} incredible, sir, it’s amazing! How well this was said by Venerable Uttara! ‘Whatever is well spoken is spoken by the Blessed One, the perfected one, the fully awakened Buddha. Both myself and others rely completely on that when we speak.’ At one time the Buddha was staying near \textsanskrit{Rājagaha}, on the Vulture’s Peak Mountain, not long after Devadatta had left. There the Buddha spoke to the mendicants about Devadatta: 

‘Mendicants,\marginnote{8.1} it’s good for a mendicant from time to time to check their own failings … the failings of others … their own successes … the successes of others. Overcome and overwhelmed by eight things that oppose the true teaching, Devadatta is going to a place of loss, to hell, there to remain for an eon, irredeemable. What eight? Overcome and overwhelmed by gain … loss … fame … disgrace … honor … dishonor … wicked desires … bad friendship, Devadatta is going to a place of loss, to hell, there to remain for an eon, irredeemable. Overcome and overwhelmed by these eight things that oppose the true teaching, Devadatta is going to a place of loss, to hell, there to remain for an eon, irredeemable. 

It’s\marginnote{9.1} good for a mendicant, whenever they encounter it, to overcome gain … loss … fame … disgrace … honor … dishonor … wicked desires … bad friendship. 

What\marginnote{10.1} advantage does a mendicant gain by overcoming these eight things? 

The\marginnote{11.1} distressing and feverish defilements that might arise in someone who lives without overcoming these eight things do not arise when they have overcome them. This is the advantage that a mendicant gains by overcoming these eight things. 

So\marginnote{12.1} you should train like this: 

“Whenever\marginnote{12.2} we encounter it, we will overcome gain … loss … fame … disgrace … honor … dishonor … wicked desires … bad friendship.” That’s how you should train.’ 

Sir,\marginnote{13.1} Uttara, this exposition of the teaching is not established anywhere in the four assemblies—monks, nuns, laymen, and laywomen. Sir, learn this exposition of the teaching! Memorize this exposition of the teaching! Remember this exposition of the teaching! Sir, this exposition of the teaching is beneficial and relates to the fundamentals of the spiritual life.” 

%
\section*{{\suttatitleacronym AN 8.9}{\suttatitletranslation Nanda }{\suttatitleroot Nandasutta}}
\addcontentsline{toc}{section}{\tocacronym{AN 8.9} \toctranslation{Nanda } \tocroot{Nandasutta}}
\markboth{Nanda }{Nandasutta}
\extramarks{AN 8.9}{AN 8.9}

“Mendicants,\marginnote{1.1} you could rightly call Nanda ‘Gentleman’, ‘strong’, ‘lovely’, and ‘lustful’. How could he live the full and pure spiritual life unless he guards the sense doors, eats in moderation, is dedicated to wakefulness, and has mindfulness and situational awareness? 

This\marginnote{1.6} is how Nanda guards the sense doors. If he has to look to the east, he wholeheartedly concentrates before looking, thinking: ‘When I look to the east, bad, unskillful qualities of desire and aversion will not overwhelm me.’ In this way he’s aware of the situation. 

If\marginnote{2.1} he has to look to the west … north … south … up … down … If he has to survey the intermediate directions, he wholeheartedly concentrates before looking, thinking: ‘When I survey the intermediate directions, bad, unskillful qualities of desire and aversion will not overwhelm me.’ In this way he’s aware of the situation. This is how Nanda guards the sense doors. 

This\marginnote{3.1} is how Nanda eats in moderation. Nanda reflects properly on the food he eats: ‘Not for fun, indulgence, adornment, or decoration, but only to sustain this body, to avoid harm, and to support spiritual practice. In this way, I shall put an end to old discomfort and not give rise to new discomfort, and I will live blamelessly and at ease.’ This is how Nanda eats in moderation. 

This\marginnote{4.1} is how Nanda is committed to wakefulness. Nanda practices walking and sitting meditation by day, purifying his mind from obstacles. In the evening, he continues to practice walking and sitting meditation. In the middle of the night, he lies down in the lion’s posture—on the right side, placing one foot on top of the other—mindful and aware, and focused on the time of getting up. In the last part of the night, he gets up and continues to practice walking and sitting meditation, purifying his mind from obstacles. This is how Nanda is committed to wakefulness. 

This\marginnote{5.1} is how Nanda has mindfulness and situational awareness. Nanda knows feelings as they arise, as they remain, and as they go away. He knows perceptions as they arise, as they remain, and as they go away. He knows thoughts as they arise, as they remain, and as they go away. This is how Nanda has mindfulness and situational awareness. 

How\marginnote{6.1} could Nanda live the full and pure spiritual life unless he guards the sense doors, eats in moderation, is dedicated to wakefulness, and has mindfulness and situational awareness?” 

%
\section*{{\suttatitleacronym AN 8.10}{\suttatitletranslation Trash }{\suttatitleroot Kāraṇḍavasutta}}
\addcontentsline{toc}{section}{\tocacronym{AN 8.10} \toctranslation{Trash } \tocroot{Kāraṇḍavasutta}}
\markboth{Trash }{Kāraṇḍavasutta}
\extramarks{AN 8.10}{AN 8.10}

At\marginnote{1.1} one time the Buddha was staying near \textsanskrit{Campā} on the banks of the \textsanskrit{Gaggarā} Lotus Pond. Now at that time the mendicants accused a mendicant of an offense. The accused mendicant dodged the issue, distracted the discussion with irrelevant points, and displayed annoyance, hate, and bitterness. 

Then\marginnote{2.1} the Buddha said to the mendicants, “Mendicants, throw this person out! Throw this person out! This person should be shown the door. Why should you be vexed by an outsider? 

Take\marginnote{2.6} a case where a certain person looks just the same as other good-natured mendicants when going out and coming back, when looking ahead and aside, when bending and extending the limbs, and when bearing the outer robe, bowl and robes. That is, so long as the mendicants don’t notice his offense. But when the mendicants notice the offense, they know that he’s a corrupt ascetic, just useless trash. When they realize this they send him away. Why is that? So that he doesn’t corrupt good-natured mendicants. 

Suppose\marginnote{3.1} in a growing field of barley some bad barley appeared, just useless trash. Its roots, stem, and leaves would look just the same as the healthy barley. That is, so long as the head doesn’t appear. But when the head appears, they know that it’s bad barley, just useless trash. When they realize this they pull it up by the roots and throw it outside the field. Why is that? So that it doesn’t spoil the good barley. 

In\marginnote{4.1} the same way, take a case where a certain person looks just the same as other good-natured mendicants when going out and coming back, when looking ahead and aside, when bending and extending the limbs, and when bearing the outer robe, bowl and robes. That is, so long as the mendicants don’t notice his offense. But when the mendicants notice the offense, they know that he’s a corrupt ascetic, just useless trash. When they realize this they send him away. Why is that? So that he doesn’t corrupt good-natured mendicants. 

Suppose\marginnote{5.1} that a large heap of grain is being winnowed. The grains that are firm and substantial form a heap on one side. And the grains that are flimsy and insubstantial are blown over to the other side. Then the owners take a broom and sweep them even further away. Why is that? So that it doesn’t spoil the good grain. In the same way, take a case where a certain person looks just the same as other good-natured mendicants when going out and coming back, when looking ahead and aside, when bending and extending the limbs, and when bearing the outer robe, bowl and robes. That is, so long as the mendicants don’t notice his offense. But when the mendicants notice the offense, they know that he’s a corrupt ascetic, just useless trash. When they realize this they send him away. Why is that? So that he doesn’t corrupt good-natured mendicants. 

Suppose\marginnote{6.1} a man needs an irrigation gutter for a well. He’d take a sharp axe and enter the wood, where he’d knock various trees with the axe. The trees that were firm and substantial made a cracking sound. But the trees that were rotten inside, decomposing and decayed, made a thud. He’d cut down such a tree at the root, lop off the crown, and thoroughly clear out the insides. Then he’d use it as an irrigation gutter for the well. In the same way, take a case where a certain person looks just the same as other good-natured mendicants when going out and coming back, when looking ahead and aside, when bending and extending the limbs, and when bearing the outer robe, bowl and robes. That is, so long as the mendicants don’t notice his offense. But when the mendicants notice the offense, they know that he’s a corrupt ascetic, just useless trash. When they realize this they send him away. Why is that? So that he doesn’t corrupt good-natured mendicants. 

\begin{verse}%
By\marginnote{7.1} living together, know that \\
they’re irritable, with wicked desires, \\
offensive, stubborn, and contemptuous, \\
jealous, stingy, and devious. 

They\marginnote{8.1} speak to people with a voice \\
so smooth, just like an ascetic. \\
But they act in secret, with their bad views \\
and their lack of regard for others. 

You\marginnote{9.1} should recognize them for what they are: \\
a creep and liar. \\
Then having gathered in harmony, \\
you should expel them. 

Throw\marginnote{10.1} out the trash! \\
Get rid of the rubbish! \\
And sweep away the scraps—\\
they’re not ascetics, they just think they are. 

When\marginnote{11.1} you’ve thrown out those of wicked desires, \\
of bad behavior and alms-resort, \\
dwell in communion, ever mindful, \\
the pure with the pure. \\
Then in harmony, alert, \\
make an end of suffering.” 

%
\end{verse}

%
\addtocontents{toc}{\let\protect\contentsline\protect\nopagecontentsline}
\chapter*{The Great Chapter }
\addcontentsline{toc}{chapter}{\tocchapterline{The Great Chapter }}
\addtocontents{toc}{\let\protect\contentsline\protect\oldcontentsline}

%
\section*{{\suttatitleacronym AN 8.11}{\suttatitletranslation At Verañja }{\suttatitleroot Verañjasutta}}
\addcontentsline{toc}{section}{\tocacronym{AN 8.11} \toctranslation{At Verañja } \tocroot{Verañjasutta}}
\markboth{At Verañja }{Verañjasutta}
\extramarks{AN 8.11}{AN 8.11}

\scevam{So\marginnote{1.1} I have heard. }At one time the Buddha was staying in \textsanskrit{Verañja} at the root of a neem tree dedicated to \textsanskrit{Naḷeru}. Then the brahmin \textsanskrit{Verañja} went up to the Buddha, and exchanged greetings with him. When the greetings and polite conversation were over, he sat down to one side and said to the Buddha: 

“Master\marginnote{2.1} Gotama, I have heard that the ascetic Gotama doesn’t bow to old brahmins, the elderly and senior, who are advanced in years and have reached the final stage of life; nor does he rise in their presence or offer them a seat. And this is indeed the case, for Master Gotama does not bow to old brahmins, elderly and senior, who are advanced in years and have reached the final stage of life; nor does he rise in their presence or offer them a seat. This is not appropriate, Master Gotama.” 

“Brahmin,\marginnote{2.6} I don’t see anyone in this world—with its gods, \textsanskrit{Māras}, and \textsanskrit{Brahmās}, this population with its ascetics and brahmins, its gods and humans—for whom I should bow down or rise up or offer a seat. If the Realized One bowed down or rose up or offered a seat to anyone, their head would explode!” 

“Master\marginnote{3.1} Gotama lacks taste.” 

“There\marginnote{3.2} is, brahmin, a sense in which you could rightly say that I lack taste. For the Realized One has given up taste for sights, sounds, smells, tastes, and touches. It’s cut off at the root, made like a palm stump, obliterated, and unable to arise in the future. In this sense you could rightly say that I lack taste. But that’s not what you’re talking about.” 

“Master\marginnote{4.1} Gotama is indelicate.” 

“There\marginnote{4.2} is, brahmin, a sense in which you could rightly say that I’m indelicate. For the Realized One has given up delight in sights, sounds, smells, tastes, and touches. It’s cut off at the root, made like a palm stump, obliterated, and unable to arise in the future. In this sense you could rightly say that I’m indelicate. But that’s not what you’re talking about.” 

“Master\marginnote{5.1} Gotama is a teacher of inaction.” 

“There\marginnote{5.2} is, brahmin, a sense in which you could rightly say that I’m a teacher of inaction. For I teach inaction regarding bad bodily, verbal, and mental conduct, and the many kinds of unskillful things. In this sense you could rightly say that I’m a teacher of inaction. But that’s not what you’re talking about.” 

“Master\marginnote{6.1} Gotama is a teacher of annihilationism.” 

“There\marginnote{6.2} is, brahmin, a sense in which you could rightly say that I’m a teacher of annihilationism. For I teach the annihilation of greed, hate, and delusion, and the many kinds of unskillful things. In this sense you could rightly say that I’m a teacher of annihilationism. But that’s not what you’re talking about.” 

“Master\marginnote{7.1} Gotama is disgusted.” 

“There\marginnote{7.2} is, brahmin, a sense in which you could rightly say that I’m disgusted. For I’m disgusted by bad conduct by way of body, speech, and mind, and by attainment of the many kinds of unskillful things. In this sense you could rightly say that I’m disgusted. But that’s not what you’re talking about.” 

“Master\marginnote{8.1} Gotama is an exterminator.” 

“There\marginnote{8.2} is, brahmin, a sense in which you could rightly say that I’m an exterminator. For I teach the extermination of greed, hate, and delusion, and the many kinds of unskillful things. In this sense you could rightly say that I’m an exterminator. But that’s not what you’re talking about.” 

“Master\marginnote{9.1} Gotama is a mortifier.” 

“There\marginnote{9.2} is, brahmin, a sense in which you could rightly say that I’m a mortifier. For I say that bad conduct by way of body, speech, and mind should be mortified. I say that a mortifier is someone who has given up unskillful qualities that should be mortified. They’ve cut them off at the root, made them like a palm stump, obliterated them, so that they’re unable to arise in the future. The Realized One is someone who has given up unskillful qualities that should be mortified. He has cut them off at the root, made them like a palm stump, obliterated them, so that they’re unable to arise in the future. In this sense you could rightly say that I’m a mortifier. But that’s not what you’re talking about.” 

“Master\marginnote{10.1} Gotama is an abortionist.” 

“There\marginnote{10.2} is, brahmin, a sense in which you could rightly say that I’m an abortionist. I say that an abortionist is someone who has given up future wombs and rebirth into a new state of existence. They’ve cut them off at the root, made them like a palm stump, obliterated them, so that they’re unable to arise in the future. The Realized One has given up future wombs and rebirth into a new state of existence. He has cut them off at the root, made them like a palm stump, obliterated them, so that they’re unable to arise in the future. In this sense you could rightly say that I’m an abortionist. But that’s not what you’re talking about. 

Suppose,\marginnote{11.1} brahmin, there was a chicken with eight or ten or twelve eggs. And she properly sat on them to keep them warm and incubated. Now, the chick that is first to break out of the eggshell with its claws and beak and hatch safely: should that be called the eldest or the youngest?” 

“Master,\marginnote{11.4} Gotama, that should be called the eldest. For it is the eldest among them.” 

“In\marginnote{12.1} the same way, in this population lost in ignorance, trapped in their shells, I alone have broken open the egg of ignorance and realized the supreme perfect awakening. So, brahmin, I am the eldest and the first in the world. 

My\marginnote{13.1} energy was roused up and unflagging, my mindfulness was established and lucid, my body was tranquil and undisturbed, and my mind was immersed in \textsanskrit{samādhi}. Quite secluded from sensual pleasures, secluded from unskillful qualities, I entered and remained in the first absorption, which has the rapture and bliss born of seclusion, while placing the mind and keeping it connected. As the placing of the mind and keeping it connected were stilled, I entered and remained in the second absorption, which has the rapture and bliss born of immersion, with internal clarity and confidence, and unified mind, without placing the mind and keeping it connected. And with the fading away of rapture, I entered and remained in the third absorption, where I meditated with equanimity, mindful and aware, personally experiencing the bliss of which the noble ones declare, ‘Equanimous and mindful, one meditates in bliss.’ With the giving up of pleasure and pain, and the ending of former happiness and sadness, I entered and remained in the fourth absorption, without pleasure or pain, with pure equanimity and mindfulness. 

When\marginnote{14.1} my mind had immersed in \textsanskrit{samādhi} like this—purified, bright, flawless, rid of corruptions, pliable, workable, steady, and imperturbable—I extended it toward recollection of past lives. I recollected many kinds of past lives. That is: one, two, three, four, five, ten, twenty, thirty, forty, fifty, a hundred, a thousand, a hundred thousand rebirths; many eons of the world contracting, many eons of the world expanding, many eons of the world contracting and expanding. I remembered: ‘There, I was named this, my clan was that, I looked like this, and that was my food. This was how I felt pleasure and pain, and that was how my life ended. When I passed away from that place I was reborn somewhere else. There, too, I was named this, my clan was that, I looked like this, and that was my food. This was how I felt pleasure and pain, and that was how my life ended. When I passed away from that place I was reborn here.’ And so I recollected my many kinds of past lives, with features and details. 

This\marginnote{15.1} was the first knowledge, which I achieved in the first watch of the night. Ignorance was destroyed and knowledge arose; darkness was destroyed and light arose, as happens for a meditator who is diligent, keen, and resolute. This was my first breaking out, like a chick breaking out of the eggshell. 

When\marginnote{16.1} my mind had immersed in \textsanskrit{samādhi} like this—purified, bright, flawless, rid of corruptions, pliable, workable, steady, and imperturbable—I extended it toward knowledge of the death and rebirth of sentient beings. With clairvoyance that is purified and superhuman, I saw sentient beings passing away and being reborn—inferior and superior, beautiful and ugly, in a good place or a bad place. I understood how sentient beings are reborn according to their deeds: ‘These dear beings did bad things by way of body, speech, and mind. They spoke ill of the noble ones; they had wrong view; and they acted out of that wrong view. When their body breaks up, after death, they’re reborn in a place of loss, a bad place, the underworld, hell. These dear beings, however, did good things by way of body, speech, and mind. They never spoke ill of the noble ones; they had right view; and they acted out of that right view. When their body breaks up, after death, they’re reborn in a good place, a heavenly realm.’ And so, with clairvoyance that is purified and superhuman, I saw sentient beings passing away and being reborn—inferior and superior, beautiful and ugly, in a good place or a bad place. I understood how sentient beings are reborn according to their deeds. 

This\marginnote{17.1} was the second knowledge, which I achieved in the middle watch of the night. Ignorance was destroyed and knowledge arose; darkness was destroyed and light arose, as happens for a meditator who is diligent, keen, and resolute. This was my second breaking out, like a chick breaking out of the eggshell. 

When\marginnote{18.1} my mind had immersed in \textsanskrit{samādhi} like this—purified, bright, flawless, rid of corruptions, pliable, workable, steady, and imperturbable—I extended it toward knowledge of the ending of defilements. I truly understood: ‘This is suffering’ … ‘This is the origin of suffering’ … ‘This is the cessation of suffering’ … ‘This is the practice that leads to the cessation of suffering’. I truly understood: ‘These are defilements’ … ‘This is the origin of defilements’ … ‘This is the cessation of defilements’ … ‘This is the practice that leads to the cessation of defilements’. Knowing and seeing like this, my mind was freed from the defilements of sensuality, desire to be reborn, and ignorance. When it was freed, I knew it was freed. 

I\marginnote{18.6} understood: ‘Rebirth is ended; the spiritual journey has been completed; what had to be done has been done; there is no return to any state of existence.’ 

This\marginnote{19.1} was the third knowledge, which I achieved in the last watch of the night. Ignorance was destroyed and knowledge arose; darkness was destroyed and light arose, as happens for a meditator who is diligent, keen, and resolute. This was my third breaking out, like a chick breaking out of the eggshell.” 

When\marginnote{20.1} he said this, the brahmin \textsanskrit{Verañja} said to the Buddha: 

“Master\marginnote{20.2} Gotama is the eldest! Master Gotama is the best! Excellent, Master Gotama! Excellent! As if he were righting the overturned, or revealing the hidden, or pointing out the path to the lost, or lighting a lamp in the dark so people with good eyes can see what’s there, Master Gotama has made the teaching clear in many ways. I go for refuge to Master Gotama, to the teaching, and to the mendicant \textsanskrit{Saṅgha}. From this day forth, may Master Gotama remember me as a lay follower who has gone for refuge for life.” 

%
\section*{{\suttatitleacronym AN 8.12}{\suttatitletranslation With Sīha }{\suttatitleroot Sīhasutta}}
\addcontentsline{toc}{section}{\tocacronym{AN 8.12} \toctranslation{With Sīha } \tocroot{Sīhasutta}}
\markboth{With Sīha }{Sīhasutta}
\extramarks{AN 8.12}{AN 8.12}

At\marginnote{1.1} one time the Buddha was staying near \textsanskrit{Vesālī}, at the Great Wood, in the hall with the peaked roof. Now at that time several very prominent Licchavis were sitting together at the town hall, praising the Buddha, his teaching, and the \textsanskrit{Saṅgha} in many ways. 

Now\marginnote{2.1} at that time General \textsanskrit{Sīha}, a disciple of the Jains, was sitting in that assembly. He thought, “That Blessed One must without a doubt be a perfected one, a fully awakened Buddha. For several very prominent Licchavis are praising the Buddha, his teaching, and the \textsanskrit{Saṅgha} in many ways. Why don’t I go to see that Blessed One, the perfected one, the fully awakened Buddha!” 

Then\marginnote{2.5} General \textsanskrit{Sīha} went to \textsanskrit{Nigaṇṭha} \textsanskrit{Nātaputta} and said to him, “Sir, I’d like to go to see the ascetic Gotama.” 

“But\marginnote{3.1} \textsanskrit{Sīha}, you believe in the doctrine of action. Why should you go to see the ascetic Gotama, who teaches a doctrine of inaction? For the ascetic Gotama believes in a doctrine of inaction, he teaches inaction, and he guides his disciples in that way.” 

Then\marginnote{3.3} \textsanskrit{Sīha}’s determination to go and see the Buddha died down. 

For\marginnote{4.1} a second time, several prominent Licchavis were sitting together at the town hall, praising the Buddha, his teaching, and the \textsanskrit{Saṅgha} in many ways. And for a second time \textsanskrit{Sīha} thought: “Why don’t I go to see that Blessed One, the perfected one, the fully awakened Buddha!” 

Then\marginnote{4.5} General \textsanskrit{Sīha} went to \textsanskrit{Nigaṇṭha} \textsanskrit{Nātaputta} … 

Then\marginnote{5.3} for a second time \textsanskrit{Sīha}’s determination to go and see the Buddha died down. 

For\marginnote{6.1} a third time, several prominent Licchavis were sitting together at the town hall, praising the Buddha, his teaching, and the \textsanskrit{Saṅgha} in many ways. And for a third time \textsanskrit{Sīha} thought, “That Blessed One must without a doubt be a perfected one, a fully awakened Buddha. For several very prominent Licchavis are praising the Buddha, his teaching, and the \textsanskrit{Saṅgha} in many ways. What can these Jains do to me, whether I consult with them or not? Why don’t I, without consulting them, go to see that Blessed One, the perfected one, the fully awakened Buddha!” 

Then\marginnote{7.1} \textsanskrit{Sīha}, with around five hundred chariots, set out from \textsanskrit{Vesālī} in the middle of the day to see the Buddha. He went by carriage as far as the terrain allowed, then descended and went by foot. Then General \textsanskrit{Sīha} went up to the Buddha, bowed, sat down to one side, and said to him: 

“Sir,\marginnote{8.1} I have heard this: ‘The ascetic Gotama believes in a doctrine of inaction, he teaches inaction, and he guides his disciples in that way.’ I trust those who say this repeat what the Buddha has said, and do not misrepresent him with an untruth? Is their explanation in line with the teaching? Are there any legitimate grounds for rebuke and criticism? For we don’t want to misrepresent the Blessed One.” 

“There\marginnote{9.1} is, \textsanskrit{Sīha}, a sense in which you could rightly say that I believe in inaction, I teach inaction, and I guide my disciples in that way. 

And\marginnote{10.1} there is a sense in which you could rightly say that I believe in action, I teach action, and I guide my disciples in that way. 

And\marginnote{11.1} there is a sense in which you could rightly say that I believe in annihilationism, I teach annihilation, and I guide my disciples in that way. 

And\marginnote{12.1} there is a sense in which you could rightly say that I’m disgusted, I teach disgust, and I guide my disciples in that way. 

And\marginnote{13.1} there is a sense in which you could rightly say that I'm an exterminator, I teach extermination, and I guide my disciples in that way. 

And\marginnote{14.1} there is a sense in which you could rightly say that I’m a mortifier, I teach mortification, and I guide my disciples in that way. 

And\marginnote{15.1} there is a sense in which you could rightly say that I’m an abortionist, I teach abortion, and I guide my disciples in that way. 

And\marginnote{16.1} there is a sense in which you could rightly say that I’m ambitious, I teach ambition, and I guide my disciples in that way. 

And\marginnote{17.1} what’s the sense in which you could rightly say that I believe in inaction, I teach inaction, and I guide my disciples in that way? I teach inaction regarding bad bodily, verbal, and mental conduct, and the many kinds of unskillful things. In this sense you could rightly say that I teach inaction. 

And\marginnote{18.1} what’s the sense in which you could rightly say that I believe in action, I teach action, and I guide my disciples in that way? I teach action regarding good bodily, verbal, and mental conduct, and the many kinds of skillful things. In this sense you could rightly say that I teach action. 

And\marginnote{19.1} what’s the sense in which you could rightly say that I believe in annihilationism, I teach annihilation, and I guide my disciples in that way? I teach the annihilation of greed, hate, and delusion, and the many kinds of unskillful things. In this sense you could rightly say that I teach annihilationism. 

And\marginnote{20.1} what’s the sense in which you could rightly say that I’m disgusted, I teach disgust, and I guide my disciples in that way? I’m disgusted by bad conduct by way of body, speech, and mind, and by attainment of the many kinds of unskillful things. In this sense you could rightly say that I’m disgusted. 

And\marginnote{21.1} what’s the sense in which you could rightly say that I'm an exterminator, I teach extermination, and I guide my disciples in that way? I teach the extermination of greed, hate, and delusion, and the many kinds of unskillful things. In this sense you could rightly say that I’m an exterminator. 

And\marginnote{22.1} what’s the sense in which you could rightly say that I’m a mortifier, I teach mortification, and I guide my disciples in that way? I say that bad conduct by way of body, speech, and mind should be mortified. I say that a mortifier is someone who has given up unskillful qualities that should be mortified. They’ve cut them off at the root, made them like a palm stump, obliterated them, so that they’re unable to arise in the future. The Realized One is someone who has given up unskillful qualities that should be mortified. He has cut them off at the root, made them like a palm stump, obliterated them, so that they’re unable to arise in the future. In this sense you could rightly say that I’m a mortifier. 

And\marginnote{23.1} what’s the sense in which you could rightly say that I’m an abortionist, I teach abortion, and I guide my disciples in that way? I say that an abortionist is someone who has given up future wombs and rebirth into a new state of existence. They’ve cut them off at the root, made them like a palm stump, obliterated them, so that they’re unable to arise in the future. The Realized One has given up future wombs and rebirth into a new state of existence. He has cut them off at the root, made them like a palm stump, obliterated them, so that they’re unable to arise in the future. In this sense you could rightly say that I’m an abortionist. 

And\marginnote{24.1} what’s the sense in which you could rightly say that I’m ambitious, I teach ambition, and I guide my disciples in that way? I’m ambitious to offer solace, the highest solace, I teach solace, and I guide my disciples in that way. In this sense you could rightly say that I’m ambitious.” 

When\marginnote{25.1} he said this, General \textsanskrit{Sīha} said to the Buddha, “Excellent, sir! Excellent! From this day forth, may the Buddha remember me as a lay follower who has gone for refuge for life.” 

“\textsanskrit{Sīha},\marginnote{26.1} you should act after careful consideration. It’s good for well-known people such as yourself to act after careful consideration.” 

“Now\marginnote{26.2} I’m even more delighted and satisfied with the Buddha, since he tells me to act after careful consideration. For if the followers of other paths were to gain me as a disciple, they’d carry a banner all over \textsanskrit{Vesālī}, saying: ‘General \textsanskrit{Sīha} has become our disciple!’ And yet the Buddha tells me to act after careful consideration. For a second time, I go for refuge to the Buddha, to the teaching, and to the mendicant \textsanskrit{Saṅgha}. From this day forth, may the Buddha remember me as a lay follower who has gone for refuge for life.” 

“For\marginnote{27.1} a long time now, \textsanskrit{Sīha}, your family has been a well-spring of support for the Jain ascetics. You should consider giving to them when they come.” 

“Now\marginnote{27.2} I’m even more delighted and satisfied with the Buddha, since he tells me to consider giving to the Jain ascetics when they come. Sir, I have heard this: ‘The ascetic Gotama says, “Gifts should only be given to me, and to my disciples. Only what is given to me is very fruitful, not what is given to others. Only what is given to my disciples is very fruitful, not what is given to the disciples of others.”’ Yet the Buddha encourages me to give to the Jain ascetics. Well, sir, we’ll know the proper time for that. For a third time, I go for refuge to the Buddha, to the teaching, and to the mendicant \textsanskrit{Saṅgha}. From this day forth, may the Buddha remember me as a lay follower who has gone for refuge for life.” 

Then\marginnote{28.1} the Buddha taught \textsanskrit{Sīha} step by step, with a talk on giving, ethical conduct, and heaven. He explained the drawbacks of sensual pleasures, so sordid and corrupt, and the benefit of renunciation. And when the Buddha knew that \textsanskrit{Sīha}’s mind was ready, pliable, rid of hindrances, elated, and confident he explained the special teaching of the Buddhas: suffering, its origin, its cessation, and the path. Just as a clean cloth rid of stains would properly absorb dye, in that very seat the stainless, immaculate vision of the Dhamma arose in General \textsanskrit{Sīha}: “Everything that has a beginning has an end.” 

Then\marginnote{29.1} \textsanskrit{Sīha} saw, attained, understood, and fathomed the Dhamma. He went beyond doubt, got rid of indecision, and became self-assured and independent of others regarding the Teacher’s instructions. He said to the Buddha, “Sir, may the Buddha together with the mendicant \textsanskrit{Saṅgha} please accept tomorrow’s meal from me.” The Buddha consented in silence. Then, knowing that the Buddha had consented, \textsanskrit{Sīha} got up from his seat, bowed, and respectfully circled the Buddha, keeping him on his right, before leaving. 

Then\marginnote{30.1} \textsanskrit{Sīha} addressed a certain man, “Mister, please find out if there is any meat ready for sale.” And when the night had passed General \textsanskrit{Sīha} had a variety of delicious foods prepared in his own home. Then he had the Buddha informed of the time, saying, “Sir, it’s time. The meal is ready.” 

Then\marginnote{31.1} the Buddha robed up in the morning and, taking his bowl and robe, went to \textsanskrit{Sīha}’s home, where he sat on the seat spread out, together with the \textsanskrit{Saṅgha} of mendicants. Now at that time many Jain ascetics in \textsanskrit{Vesālī} went from street to street and from square to square, calling out with raised arms: “Today General \textsanskrit{Sīha} has slaughtered a fat calf for the ascetic Gotama’s meal. The ascetic Gotama knowingly eats meat prepared specially for him: this is a deed he caused.” 

Then\marginnote{32.1} a certain person went up to \textsanskrit{Sīha} and whispered in his ear, “Please sir, you should know this. Many Jain ascetics in \textsanskrit{Vesālī} are going from street to street and square to square, calling out with raised arms: ‘Today General \textsanskrit{Sīha} has slaughtered a fat calf for the ascetic Gotama’s meal. The ascetic Gotama knowingly eats meat prepared specially for him: this is a deed he caused.’” 

“Enough,\marginnote{32.6} sir. For a long time those venerables have wanted to discredit the Buddha, his teaching, and his \textsanskrit{Saṅgha}. They’ll never stop misrepresenting the Buddha with their false, hollow, lying, untruthful claims. We would never deliberately take the life of a living creature, not even for life’s sake.” 

Then\marginnote{33.1} \textsanskrit{Sīha} served and satisfied the mendicant \textsanskrit{Saṅgha} headed by the Buddha with his own hands with a variety of delicious foods. When the Buddha had eaten and washed his hand and bowl, \textsanskrit{Sīha} sat down to one side. Then the Buddha educated, encouraged, fired up, and inspired him with a Dhamma talk, after which he got up from his seat and left. 

%
\section*{{\suttatitleacronym AN 8.13}{\suttatitletranslation A Thoroughbred }{\suttatitleroot Assājānīyasutta}}
\addcontentsline{toc}{section}{\tocacronym{AN 8.13} \toctranslation{A Thoroughbred } \tocroot{Assājānīyasutta}}
\markboth{A Thoroughbred }{Assājānīyasutta}
\extramarks{AN 8.13}{AN 8.13}

“Mendicants,\marginnote{1.1} a fine royal thoroughbred with eight factors is worthy of a king, fit to serve a king, and considered a factor of kingship. What eight? 

It’s\marginnote{1.3} when a fine royal thoroughbred is well born on both the mother’s and the father’s sides. 

He’s\marginnote{1.5} bred in the region fine thoroughbreds come from. 

Whatever\marginnote{1.6} food he’s given, fresh or dry, he eats carefully, without making a mess. 

He’s\marginnote{1.9} disgusted by sitting or lying down in excrement or urine. 

He’s\marginnote{1.10} sweet-natured and pleasant to live with, and he doesn’t upset the other horses. 

He\marginnote{1.11} openly shows his tricks, bluffs, ruses, and feints to his trainer, so the trainer can try to subdue them. 

He\marginnote{1.13} carries his load, determining: ‘Whether or not the other horses carry their loads, I’ll carry mine.’ 

He\marginnote{1.15} always walks in a straight path. He’s strong, and stays strong even until death. 

A\marginnote{1.17} fine royal thoroughbred with these eight factors is worthy of a king. … 

In\marginnote{2.1} the same way, a mendicant with eight qualities is worthy of offerings dedicated to the gods, worthy of hospitality, worthy of a religious donation, worthy of greeting with joined palms, and is the supreme field of merit for the world. What eight? 

It’s\marginnote{2.3} when a mendicant is ethical, restrained in the monastic code, conducting themselves well and seeking alms in suitable places. Seeing danger in the slightest fault, they keep the rules they’ve undertaken. 

Whatever\marginnote{2.4} food they’re given, coarse or fine, they eat carefully, without bother. 

They're\marginnote{2.7} disgusted with bad conduct by way of body, speech, or mind, and by attainment of the many kinds of unskillful things. 

They're\marginnote{2.9} sweet-natured and pleasant to live with, and they doesn’t upset the other mendicants. 

They\marginnote{2.10} openly show their tricks, bluffs, ruses, and feints to their sensible spiritual companions, so they can try to subdue them. 

They\marginnote{2.12} do their training, determining: ‘Whether or not the other mendicants do their training, I’ll do mine.’ 

They\marginnote{2.14} always walk in a straight path. And here the straight path is right view, right thought, right speech, right action, right livelihood, right effort, right mindfulness, and right immersion. 

They’re\marginnote{2.17} energetic: ‘Gladly, let my skin, sinews, and bones remain! Let the blood and flesh waste away in my body! I will not stop trying until I have achieved what is possible by human strength, energy, and vigor.’ 

A\marginnote{2.19} mendicant with these eight qualities is worthy of offerings dedicated to the gods, worthy of hospitality, worthy of a religious donation, worthy of veneration with joined palms, and is the supreme field of merit for the world.” 

%
\section*{{\suttatitleacronym AN 8.14}{\suttatitletranslation A Wild Colt }{\suttatitleroot Assakhaḷuṅkasutta}}
\addcontentsline{toc}{section}{\tocacronym{AN 8.14} \toctranslation{A Wild Colt } \tocroot{Assakhaḷuṅkasutta}}
\markboth{A Wild Colt }{Assakhaḷuṅkasutta}
\extramarks{AN 8.14}{AN 8.14}

“Mendicants,\marginnote{1.1} I will teach you about eight wild colts and eight defects in horses, and about eight wild people and eight defects in people. Listen and pay close attention, I will speak.” 

“Yes,\marginnote{1.3} sir,” they replied. The Buddha said this: 

“And\marginnote{2.1} what, mendicants, are the eight wild colts and eight defects in horses? 

Firstly,\marginnote{2.2} when the trainer says ‘giddyup!’ and spurs and goads them on, some wild colts back right up and spin the chariot behind them. Some wild colts are like that. This is the first defect of a horse. 

Furthermore,\marginnote{3.1} when the trainer says ‘giddyup!’ and spurs and goads them on, some wild colts jump back, wreck the hub, and break the triple rod. Some wild colts are like that. This is the second defect of a horse. 

Furthermore,\marginnote{4.1} when the trainer says ‘giddyup!’ and spurs and goads them on, some wild colts shake the cart-pole off their thigh and trample it. Some wild colts are like that. This is the third defect of a horse. 

Furthermore,\marginnote{5.1} when the trainer says ‘giddyup!’ and spurs and goads them on, some wild colts take a wrong turn, sending the chariot off track. Some wild colts are like that. This is the fourth defect of a horse. 

Furthermore,\marginnote{6.1} when the trainer says ‘giddyup!’ and spurs and goads them on, some wild colts rear up and strike out with their fore-legs. Some wild colts are like that. This is the fifth defect of a horse. 

Furthermore,\marginnote{7.1} when the trainer says ‘giddyup!’ and spurs and goads them on, some wild colts ignore the trainer and the goad, spit out the bit, and go wherever they want. Some wild colts are like that. This is the sixth defect of a horse. 

Furthermore,\marginnote{8.1} when the trainer says ‘giddyup!’ and spurs and goads them on, some wild colts don’t step forward or turn back but stand right there still as a post. Some wild colts are like that. This is the seventh defect of a horse. 

Furthermore,\marginnote{9.1} when the trainer says ‘giddyup!’ and spurs and goads them on, some wild colts tuck in their fore-legs and hind-legs, and sit right down on their four legs. Some wild colts are like that. This is the eighth defect of a horse. These are the eight wild colts and the eight defects in horses. 

And\marginnote{10.1} what are the eight wild people and eight defects in people? 

Firstly,\marginnote{10.2} the mendicants accuse a mendicant of an offense. But the accused mendicant evades it by saying they don’t remember. I say that this person is comparable to the wild colts who, when the trainer says ‘giddyup!’ and spurs and goads them on, back right up and spin the chariot behind them. Some wild people are like that. This is the first defect of a person. 

Furthermore,\marginnote{11.1} the mendicants accuse a mendicant of an offense. But the accused mendicant objects to the accuser: ‘What has an incompetent fool like you got to say? How on earth could you imagine you’ve got something worth saying!’ I say that this person is comparable to the wild colts who, when the trainer says ‘giddyup!’ and spurs and goads them on, jump back, wreck the hub, and break the triple rod. Some wild people are like that. This is the second defect of a person. 

Furthermore,\marginnote{12.1} the mendicants accuse a mendicant of an offense. But the accused mendicant retorts to the accuser: ‘Well, you’ve fallen into such-and-such an offense. You should deal with that first.’ I say that this person is comparable to the wild colts who, when the trainer says ‘giddyup!’ and spurs and goads them on, shake the cart-pole off their thigh and trample it. Some wild people are like that. This is the third defect of a person. 

Furthermore,\marginnote{13.1} the mendicants accuse a mendicant of an offense. But the accused mendicant dodges the issue, distracts the discussion with irrelevant points, and displays annoyance, hate, and bitterness. I say that this person is comparable to the wild colts who, when the trainer says ‘giddyup!’ and spurs and goads them on, take a wrong turn, sending the chariot off track. Some wild people are like that. This is the fourth defect of a person. 

Furthermore,\marginnote{14.1} the mendicants accuse a mendicant of an offense. But the accused mendicant gesticulates while speaking in the middle of the \textsanskrit{Saṅgha}. I say that this person is comparable to the wild colts who, when the trainer says ‘giddyup!’ and spurs and goads them on, rear up and strike out with their fore-legs. Some wild people are like that. This is the fifth defect of a person. 

Furthermore,\marginnote{15.1} the mendicants accuse a mendicant of an offense. But the accused mendicant ignores the \textsanskrit{Saṅgha} and the accusation and, though still guilty of the offense, they go wherever they want. I say that this person is comparable to the wild colts who, when the trainer says ‘giddyup!’ and spurs and goads them on, ignore the trainer and the goad, spit out the bit, and go wherever they want. Some wild people are like that. This is the sixth defect of a person. 

Furthermore,\marginnote{16.1} the mendicants accuse a mendicant of an offense. But the accused mendicant neither confesses to the offense nor denies it, but frustrates the \textsanskrit{Saṅgha} by staying silent. I say that this person is comparable to the wild colts who, when the trainer says ‘giddyup!’ and spurs and goads them on, don’t step forward or turn back but stand right there still as a post. Some wild people are like that. This is the seventh defect of a person. 

Furthermore,\marginnote{17.1} the mendicants accuse a mendicant of an offense. But the accused mendicant says this: ‘Why are you venerables making so much of an issue over me? Now I’ll resign the training and return to a lesser life.’ When they have resigned the training, they say: ‘Well, venerables, are you happy now?’ I say that this person is comparable to the wild colts who, when the trainer says ‘giddyup!’ and spurs and goads them on, tuck in their fore-legs and hind-legs, and sit right down on their four legs. Some wild people are like that. This is the eighth defect of a person. 

These\marginnote{17.10} are the eight wild people and eight defects in people.” 

%
\section*{{\suttatitleacronym AN 8.15}{\suttatitletranslation Stains }{\suttatitleroot Malasutta}}
\addcontentsline{toc}{section}{\tocacronym{AN 8.15} \toctranslation{Stains } \tocroot{Malasutta}}
\markboth{Stains }{Malasutta}
\extramarks{AN 8.15}{AN 8.15}

“Mendicants,\marginnote{1.1} there are these eight stains. What eight? Not reciting is the stain of hymns. Neglect is the stain of houses. Laziness is the stain of beauty. Negligence is a guard’s stain. Misconduct is a woman’s stain. Stinginess is a giver’s stain. Bad, unskillful qualities are a stain in this world and the next. Worse than any of these is ignorance, the worst stain of all. These are the eight stains. 

\begin{verse}%
Not\marginnote{2.1} reciting is the stain of hymns. \\
The stain of houses is neglect. \\
Laziness is the stain of beauty. \\
A guard’s stain is negligence. 

Misconduct\marginnote{3.1} is a woman’s stain. \\
A giver’s stain is stinginess. \\
Bad qualities are a stain \\
in this world and the next. \\
But a worse stain than these \\
is ignorance, the worst stain of all.” 

%
\end{verse}

%
\section*{{\suttatitleacronym AN 8.16}{\suttatitletranslation Going on a Mission }{\suttatitleroot Dūteyyasutta}}
\addcontentsline{toc}{section}{\tocacronym{AN 8.16} \toctranslation{Going on a Mission } \tocroot{Dūteyyasutta}}
\markboth{Going on a Mission }{Dūteyyasutta}
\extramarks{AN 8.16}{AN 8.16}

“Mendicants,\marginnote{1.1} a mendicant with eight qualities is worthy of going on a mission. What eight? It’s a mendicant who learns and educates others. They memorize and remember. They understand and help others understand. They’re skilled at knowing what’s on topic and what isn’t. And they don’t cause quarrels. A mendicant with these eight qualities is worthy of going on a mission. 

Having\marginnote{1.5} eight qualities \textsanskrit{Sāriputta} is worthy of going on a mission. What eight? He learns and educates others. He memorizes and remembers. He understands and helps others understand. He’s skilled at knowing what’s on topic and what isn’t. And he doesn’t cause quarrels. Having these eight qualities \textsanskrit{Sāriputta} is worthy of going on a mission. 

\begin{verse}%
They\marginnote{2.1} don’t tremble when arriving \\
at an assembly of fierce debaters. \\
They don’t miss out any words, \\
or conceal the instructions. 

Their\marginnote{3.1} words aren’t poisoned, \\
and they don’t tremble when questioned. \\
Such a mendicant \\
is worthy of going on a mission.” 

%
\end{verse}

%
\section*{{\suttatitleacronym AN 8.17}{\suttatitletranslation Catching (1st) }{\suttatitleroot Paṭhamabandhanasutta}}
\addcontentsline{toc}{section}{\tocacronym{AN 8.17} \toctranslation{Catching (1st) } \tocroot{Paṭhamabandhanasutta}}
\markboth{Catching (1st) }{Paṭhamabandhanasutta}
\extramarks{AN 8.17}{AN 8.17}

“Mendicants,\marginnote{1.1} a woman catches a man using eight features. What eight? With weeping, laughing, speaking, appearance, gifts of wildflowers, scents, tastes, and touches. A woman catches a man using these eight features. But those beings who are caught by touch are well and truly caught.” 

%
\section*{{\suttatitleacronym AN 8.18}{\suttatitletranslation Catching (2nd) }{\suttatitleroot Dutiyabandhanasutta}}
\addcontentsline{toc}{section}{\tocacronym{AN 8.18} \toctranslation{Catching (2nd) } \tocroot{Dutiyabandhanasutta}}
\markboth{Catching (2nd) }{Dutiyabandhanasutta}
\extramarks{AN 8.18}{AN 8.18}

“Mendicants,\marginnote{1.1} a man catches a woman using eight features. What eight? With weeping, laughing, speaking, appearance, gifts of wildflowers, scents, tastes, and touches. A man catches a woman using these eight features. But those beings who are caught by touch are well and truly caught.” 

%
\section*{{\suttatitleacronym AN 8.19}{\suttatitletranslation With Pahārāda }{\suttatitleroot Pahārādasutta}}
\addcontentsline{toc}{section}{\tocacronym{AN 8.19} \toctranslation{With Pahārāda } \tocroot{Pahārādasutta}}
\markboth{With Pahārāda }{Pahārādasutta}
\extramarks{AN 8.19}{AN 8.19}

At\marginnote{1.1} one time the Buddha was staying in \textsanskrit{Verañja} at the root of a neem tree dedicated to \textsanskrit{Naḷeru}. 

Then\marginnote{1.2} \textsanskrit{Pahārāda}, lord of demons, went up to the Buddha, bowed, and stood to one side. The Buddha said to him, “Well, \textsanskrit{Pahārāda}, do the demons love the ocean?” 

“Sir,\marginnote{2.2} they do indeed.” 

“But\marginnote{2.3} seeing what incredible and amazing things do the demons love the ocean?” 

“Sir,\marginnote{2.4} seeing eight incredible and amazing things the demons love the ocean. What eight? The ocean gradually slants, slopes, and inclines, with no abrupt precipice. This is the first thing the demons love about the ocean. 

Furthermore,\marginnote{3.1} the ocean is consistent and doesn’t overflow its boundaries. This is the second thing the demons love about the ocean. 

Furthermore,\marginnote{4.1} the ocean doesn’t accommodate a corpse, but quickly carries it to the shore and strands it on the beach. This is the third thing the demons love about the ocean. 

Furthermore,\marginnote{5.1} when they reach the ocean, all the great rivers—that is, the Ganges, \textsanskrit{Yamunā}, \textsanskrit{Aciravatī}, \textsanskrit{Sarabhū}, and \textsanskrit{Mahī}—lose their names and clans and are simply considered ‘the ocean’. This is the fourth thing the demons love about the ocean. 

Furthermore,\marginnote{6.1} for all the world’s streams that reach it, and the rain that falls from the sky, the ocean never empties or fills up. This is the fifth thing the demons love about the ocean. 

Furthermore,\marginnote{7.1} the ocean has just one taste, the taste of salt. This is the sixth thing the demons love about the ocean. 

Furthermore,\marginnote{8.1} the ocean is full of many kinds of treasures, such as pearls, gems, beryl, conch, quartz, coral, silver, gold, rubies, and emeralds. This is the seventh thing the demons love about the ocean. 

Furthermore,\marginnote{9.1} many great beings live in the ocean, such as leviathans, leviathan-gulpers, leviathan-gulper-gulpers, demons, dragons, and fairies. In the ocean there are life-forms a hundred leagues long, or even two hundred, three hundred, four hundred, or five hundred leagues long. This is the eighth thing the demons love about the ocean. 

Seeing\marginnote{9.4} these eight incredible and amazing things the demons love the ocean. 

Well,\marginnote{10.1} sir, do the mendicants love this teaching and training?” 

“They\marginnote{10.2} do indeed, \textsanskrit{Pahārāda}.” 

“But\marginnote{10.3} seeing how many incredible and amazing things do the mendicants love this teaching and training?” 

“Seeing\marginnote{10.4} eight incredible and amazing things, \textsanskrit{Pahārāda}, the mendicants love this teaching and training. What eight? 

The\marginnote{10.6} ocean gradually slants, slopes, and inclines, with no abrupt precipice. In the same way in this teaching and training the penetration to enlightenment comes from gradual training, progress, and practice, not abruptly. This is the first thing the mendicants love about this teaching and training. 

The\marginnote{11.1} ocean is consistent and doesn’t overflow its boundaries. In the same way, when a training rule is laid down for my disciples they wouldn’t break it even for the sake of their own life. This is the second thing the mendicants love about this teaching and training. 

The\marginnote{12.1} ocean doesn’t accommodate a corpse, but quickly carries it to the shore and strands it on the beach. In the same way, the \textsanskrit{Saṅgha} doesn’t accommodate a person who is unethical, of bad qualities, filthy, with suspicious behavior, underhand, no true ascetic or spiritual practitioner—though claiming to be one—rotten inside, corrupt, and depraved. But they quickly gather and expel them. Even if such a person is sitting in the middle of the \textsanskrit{Saṅgha}, they’re far from the \textsanskrit{Saṅgha}, and the \textsanskrit{Saṅgha} is far from them. This is the third thing the mendicants love about this teaching and training. 

When\marginnote{14.1} they reach the ocean, all the great rivers—that is, the Ganges, \textsanskrit{Yamunā}, \textsanskrit{Aciravatī}, \textsanskrit{Sarabhū}, and \textsanskrit{Mahī}—lose their names and clans and are simply considered ‘the ocean’. In the same way, when they go forth from the lay life to homelessness, all four castes—aristocrats, brahmins, merchants, and workers—lose their former names and clans and are simply considered ‘Sakyan ascetics’. This is the fourth thing the mendicants love about this teaching and training. 

For\marginnote{15.1} all the world’s streams that reach it, and the rain that falls from the sky, the ocean never empties or fills up. In the same way, though several mendicants become fully extinguished through the element of extinguishment with nothing left over, the element of extinguishment never empties or fills up. This is the fifth thing the mendicants love about this teaching and training. 

The\marginnote{16.1} ocean has just one taste, the taste of salt. In the same way, this teaching and training has one taste, the taste of freedom. This is the sixth thing the mendicants love about this teaching and training. 

The\marginnote{17.1} ocean is full of many kinds of treasures, such as pearls, gems, beryl, conch, quartz, coral, silver, gold, rubies, and emeralds. In the same way, this teaching and training is full of many kinds of treasures, such as the four kinds of mindfulness meditation, the four right efforts, the four bases of psychic power, the five faculties, the five powers, the seven awakening factors, and the noble eightfold path. This is the seventh thing the mendicants love about this teaching and training. 

Many\marginnote{18.1} great beings live in the ocean, such as leviathans, leviathan-gulpers, leviathan-gulper-gulpers, demons, dragons, and fairies. In the ocean there are life-forms a hundred leagues long, or even two hundred, three hundred, four hundred, or five hundred leagues long. In the same way, great beings live in this teaching and training, and these are those beings. The stream-enterer and the one practicing to realize the fruit of stream-entry. The once-returner and the one practicing to realize the fruit of once-return. The non-returner and the one practicing to realize the fruit of non-return. The perfected one, and the one practicing for perfection. This is the eighth thing the mendicants love about this teaching and training. 

Seeing\marginnote{19.1} these eight incredible and amazing things, \textsanskrit{Pahārāda}, the mendicants love this teaching and training.” 

%
\section*{{\suttatitleacronym AN 8.20}{\suttatitletranslation Sabbath }{\suttatitleroot Uposathasutta}}
\addcontentsline{toc}{section}{\tocacronym{AN 8.20} \toctranslation{Sabbath } \tocroot{Uposathasutta}}
\markboth{Sabbath }{Uposathasutta}
\extramarks{AN 8.20}{AN 8.20}

At\marginnote{1.1} one time the Buddha was staying near \textsanskrit{Sāvatthī} in the Eastern Monastery, the stilt longhouse of \textsanskrit{Migāra}’s mother. 

Now,\marginnote{1.2} at that time it was the sabbath, and the Buddha was sitting surrounded by the \textsanskrit{Saṅgha} of monks. And then, as the night was getting late, in the first watch of the night, Venerable Ānanda got up from his seat, arranged his robe over one shoulder, raised his joined palms toward the Buddha and said, “Sir, the night is getting late. It is the first watch of the night, and the \textsanskrit{Saṅgha} has been sitting long. Please, sir, may the Buddha recite the monastic code to the mendicants.” 

But\marginnote{2.1} when he said this, the Buddha kept silent. 

For\marginnote{2.2} a second time, as the night was getting late, in the middle watch of the night, Ānanda got up from his seat, arranged his robe over one shoulder, raised his joined palms toward the Buddha and said, “Sir, the night is getting late. It is the middle watch of the night, and the \textsanskrit{Saṅgha} has been sitting long. Please, sir, may the Buddha recite the monastic code to the mendicants.” 

But\marginnote{2.5} for a second time the Buddha kept silent. 

For\marginnote{2.6} a third time, as the night was getting late, in the last watch of the night, as dawn stirred, bringing joy to the night, Ānanda got up from his seat, arranged his robe over one shoulder, raised his joined palms toward the Buddha and said, “Sir, the night is getting late. It is the last watch of the night and dawn stirs, bringing joy to the night. And the \textsanskrit{Saṅgha} has been sitting long. Please, sir, may the Buddha recite the monastic code to the mendicants.” 

“Ānanda,\marginnote{2.10} the assembly is not pure.” 

Then\marginnote{3.1} Venerable \textsanskrit{Mahāmoggallāna} thought, “Who is the Buddha talking about?” 

Then\marginnote{3.4} he focused on comprehending the minds of everyone in the \textsanskrit{Saṅgha}. He saw that unethical person, of bad qualities, filthy, with suspicious behavior, underhand, no true ascetic or spiritual practitioner—though claiming to be one—rotten inside, corrupt, and depraved, sitting in the middle of the \textsanskrit{Saṅgha}. 

When\marginnote{3.6} he saw him he got up from his seat, went up to him and said, “Get up, reverend. The Buddha has seen you. You can’t live in communion with the mendicants.” 

But\marginnote{4.1} when he said this, that person kept silent. 

For\marginnote{4.2} a second time 

and\marginnote{4.6} a third time, he asked that monk to leave. 

But\marginnote{4.9} for a third time that person kept silent. 

Then\marginnote{5.1} Venerable \textsanskrit{Mahāmoggallāna} took that person by the arm, ejected him out the gate, and bolted the door. Then he went up to the Buddha, and said to him, “I have ejected that person. The assembly is pure. Please, sir, may the Buddha recite the monastic code to the mendicants.” 

“It’s\marginnote{5.5} incredible, \textsanskrit{Moggallāna}, it’s amazing, how that silly man waited to be taken by the arm!” 

Then\marginnote{6.1} the Buddha said to the mendicants: 

“Now,\marginnote{6.2} mendicants, you should perform the sabbath and recite the monastic code. From this day forth, I will not perform the sabbath or recite the monastic code. It’s impossible, mendicants, it can’t happen that a Realized One could recite the monastic code in an impure assembly. 

Seeing\marginnote{7.1} these eight incredible and amazing things the demons love the ocean. What eight? The ocean gradually slants, slopes, and inclines, with no abrupt precipice. This is the first thing the demons love about the ocean. 

(Expand\marginnote{7.6} in detail as in the previous sutta.) 

Furthermore,\marginnote{8.1} many great beings live in the ocean, such as leviathans, leviathan-gulpers, leviathan-gulper-gulpers, demons, dragons, and fairies. In the ocean there are life-forms a hundred leagues long, or even two hundred, three hundred, four hundred, or five hundred leagues long. This is the eighth thing the demons love about the ocean. Seeing these eight incredible and amazing things the demons love the ocean. 

In\marginnote{9.1} the same way, seeing eight incredible and amazing things, mendicants, the mendicants love this teaching and training. What eight? 

The\marginnote{9.3} ocean gradually slants, slopes, and inclines, with no abrupt precipice. In the same way in this teaching and training the penetration to enlightenment comes from gradual training, progress, and practice, not abruptly. This is the first thing the mendicants love about this teaching and training. … 

Many\marginnote{9.7} great beings live in the ocean, such as leviathans, leviathan-gulpers, leviathan-gulper-gulpers, demons, dragons, and fairies. In the ocean there are life-forms a hundred leagues long, or even two hundred, three hundred, four hundred, or five hundred leagues long. In the same way, great beings live in this teaching and training, and these are those beings. The stream-enterer and the one practicing to realize the fruit of stream-entry. The once-returner and the one practicing to realize the fruit of once-return. The non-returner and the one practicing to realize the fruit of non-return. The perfected one, and the one practicing for perfection. This is the eighth thing the mendicants love about this teaching and training. 

Seeing\marginnote{9.11} these eight incredible and amazing things, the mendicants love this teaching and training.” 

%
\addtocontents{toc}{\let\protect\contentsline\protect\nopagecontentsline}
\chapter*{The Chapter on Householders }
\addcontentsline{toc}{chapter}{\tocchapterline{The Chapter on Householders }}
\addtocontents{toc}{\let\protect\contentsline\protect\oldcontentsline}

%
\section*{{\suttatitleacronym AN 8.21}{\suttatitletranslation With Ugga of Vesālī }{\suttatitleroot Paṭhamauggasutta}}
\addcontentsline{toc}{section}{\tocacronym{AN 8.21} \toctranslation{With Ugga of Vesālī } \tocroot{Paṭhamauggasutta}}
\markboth{With Ugga of Vesālī }{Paṭhamauggasutta}
\extramarks{AN 8.21}{AN 8.21}

At\marginnote{1.1} one time the Buddha was staying near \textsanskrit{Vesālī}, at the Great Wood, in the hall with the peaked roof. There the Buddha addressed the mendicants: “Mendicants, you should remember the householder Ugga of \textsanskrit{Vesālī} as someone who has eight amazing and incredible qualities.” 

That\marginnote{1.4} is what the Buddha said. When he had spoken, the Holy One got up from his seat and entered his dwelling. 

Then\marginnote{2.1} a certain mendicant robed up in the morning and, taking his bowl and robe, went to the home of the householder Ugga of \textsanskrit{Vesālī}, where he sat on the seat spread out. Then Ugga of \textsanskrit{Vesālī} went up to that mendicant, bowed, and sat down to one side. The mendicant said to him: 

“Householder,\marginnote{3.1} the Buddha declared that you have eight amazing and incredible qualities. What are the eight qualities that he spoke of?” 

“Sir,\marginnote{3.3} I don’t know what eight amazing and incredible qualities the Buddha was referring to. But these eight amazing and incredible qualities are found in me. Listen and pay close attention, I will speak.” 

“Yes,\marginnote{3.7} householder,” replied the mendicant. Ugga of \textsanskrit{Vesālī} said this: 

“Sir,\marginnote{3.9} when I first saw the Buddha off in the distance, my heart was inspired as soon as I saw him. This is the first incredible and amazing quality found in me. 

With\marginnote{4.1} confident heart I paid homage to the Buddha. The Buddha taught me step by step, with a talk on giving, ethical conduct, and heaven. He explained the drawbacks of sensual pleasures, so sordid and corrupt, and the benefit of renunciation. And when he knew that my mind was ready, pliable, rid of hindrances, elated, and confident he explained the special teaching of the Buddhas: suffering, its origin, its cessation, and the path. Just as a clean cloth rid of stains would properly absorb dye, in that very seat the stainless, immaculate vision of the Dhamma arose in me: ‘Everything that has a beginning has an end.’ I saw, attained, understood, and fathomed the Dhamma. I went beyond doubt, got rid of indecision, and became self-assured and independent of others regarding the Teacher’s instructions. Right there I went for refuge to the Buddha, his teaching, and the \textsanskrit{Saṅgha}. And I undertook the five training rules with celibacy as the fifth. This is the second incredible and amazing quality found in me. 

I\marginnote{5.1} had four teenage wives. And I went to them and said: ‘Sisters, I’ve undertaken the five training rules with celibacy as fifth. If you wish, you may stay here, enjoy my wealth, and do good deeds. Or you can return to your own families. Or would you prefer if I gave you to another man?’ When I said this, my eldest wife said to me: ‘Master, please give me to such-and-such a man.’ Then I summoned that man. Taking my wife with my left hand and a ceremonial vase with my right, I presented her to that man with the pouring of water. But I can’t recall getting upset while giving away my teenage wife. This is the third incredible and amazing quality found in me. 

And\marginnote{6.1} though my family has wealth, it’s shared without reserve with ethical people of good character. This is the fourth incredible and amazing quality found in me. 

When\marginnote{7.1} I pay homage to a mendicant, I do so carefully, not carelessly. This is the fifth incredible and amazing quality found in me. 

If\marginnote{8.1} that venerable teaches me the Dhamma, I listen carefully, not carelessly. But if they don’t teach me the Dhamma, I teach them. This is the sixth incredible and amazing quality found in me. 

It’s\marginnote{9.1} not unusual for deities to come to me and announce: ‘Householder, the Buddha’s teaching is well explained!’ When they say this I say to them: ‘The Buddha’s teaching is well explained, regardless of whether or not you deities say so!’ But I don’t recall getting too excited by the fact that the deities come to me, and I have a conversation with them. This is the seventh incredible and amazing quality found in me. 

Of\marginnote{10.1} the five lower fetters taught by the Buddha, I don’t see any that I haven’t given up. This is the eighth incredible and amazing quality found in me. 

These\marginnote{11.1} eight amazing and incredible qualities are found in me. But I don’t know what eight amazing and incredible qualities the Buddha was referring to.” 

Then\marginnote{12.1} that mendicant, after taking almsfood in Ugga of \textsanskrit{Vesālī}’s home, got up from his seat and left. Then after the meal, on his return from almsround, he went to the Buddha, bowed, and sat down to one side. He informed the Buddha of all he had discussed with the householder Ugga of \textsanskrit{Vesālī}. The Buddha said: 

“Good,\marginnote{13.1} good, mendicant! When I declared that the householder Ugga of \textsanskrit{Vesālī} was someone who has eight amazing and incredible qualities, I was referring to the same eight qualities that he rightly explained to you. You should remember the householder Ugga of \textsanskrit{Vesālī} as someone who has these eight amazing and incredible qualities.” 

%
\section*{{\suttatitleacronym AN 8.22}{\suttatitletranslation With Ugga of the Village of Hatthi }{\suttatitleroot Dutiyauggasutta}}
\addcontentsline{toc}{section}{\tocacronym{AN 8.22} \toctranslation{With Ugga of the Village of Hatthi } \tocroot{Dutiyauggasutta}}
\markboth{With Ugga of the Village of Hatthi }{Dutiyauggasutta}
\extramarks{AN 8.22}{AN 8.22}

At\marginnote{1.1} one time the Buddha was staying in the land of the Vajjis at the village of Hatthi. There the Buddha addressed the mendicants: “Mendicants, you should remember the householder Ugga of Hatthi as someone who has eight amazing and incredible qualities.” 

That\marginnote{1.4} is what the Buddha said. When he had spoken, the Holy One got up from his seat and entered his dwelling. 

Then\marginnote{2.1} a certain mendicant robed up in the morning and, taking his bowl and robe, went to the home of the householder Ugga of Hatthi, where he sat on the seat spread out. Then Ugga of Hatthi went up to that mendicant, bowed, and sat down to one side. The mendicant said to him: 

“Householder,\marginnote{2.3} the Buddha declared that you have eight amazing and incredible qualities. What are the eight qualities that he spoke of?” 

“Sir,\marginnote{3.1} I don’t know what eight amazing and incredible qualities the Buddha was referring to. But these eight amazing and incredible qualities are found in me. Listen and pay close attention, I will speak.” 

“Yes,\marginnote{3.5} householder,” replied the mendicant. Ugga of Hatthi said this: 

“Sir,\marginnote{3.7} when I first saw the Buddha off in the distance I was partying in the Dragon’s Park. My heart was inspired as soon as I saw him, and I sobered up. This is the first incredible and amazing quality found in me. 

With\marginnote{4.1} confident heart I paid homage to the Buddha. The Buddha taught me step by step, with a talk on giving, ethical conduct, and heaven. He explained the drawbacks of sensual pleasures, so sordid and corrupt, and the benefit of renunciation. And when he knew that my mind was ready, pliable, rid of hindrances, elated, and confident he explained the special teaching of the Buddhas: suffering, its origin, its cessation, and the path. Just as a clean cloth rid of stains would properly absorb dye, in that very seat the stainless, immaculate vision of the Dhamma arose in me: ‘Everything that has a beginning has an end.’ I saw, attained, understood, and fathomed the Dhamma. I went beyond doubt, got rid of indecision, and became self-assured and independent of others regarding the Teacher’s instructions. Right there I went for refuge to the Buddha, his teaching, and the \textsanskrit{Saṅgha}. And I undertook the five training rules with celibacy as the fifth. This is the second incredible and amazing quality found in me. 

I\marginnote{5.1} had four teenage wives. And I went to them and said: ‘Sisters, I’ve undertaken the five training rules with celibacy as fifth. If you wish, you may stay here, enjoy my wealth, and do good deeds. Or you can return to your own families. Or would you prefer if I gave you to another man?’ When I said this, my eldest wife said to me: ‘Master, please give me to such-and-such a man.’ Then I summoned that man. Taking my wife with my left hand and a ceremonial vase with my right, I presented her to that man with the pouring of water. But I can’t recall getting upset while giving away my teenage wife. This is the third incredible and amazing quality found in me. 

And\marginnote{6.1} though my family has wealth, it’s shared without reserve with ethical people of good character. This is the fourth incredible and amazing quality found in me. 

When\marginnote{7.1} I pay homage to a mendicant, I do so carefully, not carelessly. If that venerable teaches me the Dhamma, I listen carefully, not carelessly. But if they don’t teach me the Dhamma, I teach them. This is the fifth incredible and amazing quality found in me. 

It’s\marginnote{8.1} not unusual for deities to come to me when the \textsanskrit{Saṅgha} has been invited and announce: ‘Householder, that mendicant is freed both ways. That one is freed by wisdom. That one is a personal witness. That one is attained to view. That one is freed by faith. That one is a follower of the teachings. That one is a follower by faith. That one is ethical, of good character. That one is unethical, of bad character.’ But while I’m serving the \textsanskrit{Saṅgha} I don’t recall thinking: ‘Let me give this one just a little, and that one a lot.’ Rather, I give impartially. This is the sixth incredible and amazing quality found in me. 

It’s\marginnote{9.1} not unusual for deities to come to me and announce: ‘Householder, the Buddha’s teaching is well explained!’ When they say this I say to them: ‘The Buddha’s teaching is well explained, regardless of whether or not you deities say so!’ But I don’t recall getting too excited by the fact that the deities come to me, and I have a conversation with them. This is the seventh incredible and amazing quality found in me. 

If\marginnote{10.1} I pass away before the Buddha, it wouldn’t be surprising if the Buddha declares of me: ‘The householder Ugga of Hatthi is bound by no fetter that might return him to this world.’ This is the eighth incredible and amazing quality found in me. 

These\marginnote{11.1} eight amazing and incredible qualities are found in me. But I don’t know what eight amazing and incredible qualities the Buddha was referring to.” 

Then\marginnote{12.1} that mendicant, after taking almsfood in Ugga of Hatthi’s home, got up from his seat and left. Then after the meal, on his return from almsround, he went to the Buddha, bowed, and sat down to one side. He informed the Buddha of all he had discussed with the householder Ugga of the village of Hatthi. The Buddha said: 

“Good,\marginnote{13.1} good, mendicant! When I declared that the householder Ugga of the village of Hatthi was someone who has eight amazing and incredible qualities, I was referring to the same eight qualities that he rightly explained to you. You should remember the householder Ugga of Hatthi as someone who has these eight amazing and incredible qualities.” 

%
\section*{{\suttatitleacronym AN 8.23}{\suttatitletranslation With Hatthaka (1st) }{\suttatitleroot Paṭhamahatthakasutta}}
\addcontentsline{toc}{section}{\tocacronym{AN 8.23} \toctranslation{With Hatthaka (1st) } \tocroot{Paṭhamahatthakasutta}}
\markboth{With Hatthaka (1st) }{Paṭhamahatthakasutta}
\extramarks{AN 8.23}{AN 8.23}

At\marginnote{1.1} one time the Buddha was staying near \textsanskrit{Āḷavī}, at the \textsanskrit{Aggāḷava} Tree-shrine. There the Buddha addressed the mendicants: 

“Mendicants,\marginnote{1.3} you should remember the householder Hatthaka of \textsanskrit{Āḷavī} as someone who has seven amazing and incredible qualities. What seven? He’s faithful, ethical, conscientious, prudent, learned, generous, and wise. You should remember the householder Hatthaka of \textsanskrit{Āḷavī} as someone who has these seven amazing and incredible qualities.” 

That\marginnote{1.13} is what the Buddha said. When he had spoken, the Holy One got up from his seat and entered his dwelling. 

Then\marginnote{2.1} a certain mendicant robed up in the morning and, taking his bowl and robe, went to the home of the householder Hatthaka of \textsanskrit{Āḷavī}, where he sat on the seat spread out. Then Hatthaka went up to that mendicant, bowed, and sat down to one side. The mendicant said to Hatthaka: 

“Householder,\marginnote{3.1} the Buddha declared that you have seven amazing and incredible qualities. What seven? He said that you’re faithful, ethical, conscientious, prudent, learned, generous, and wise. The Buddha declared that you have these seven amazing and incredible qualities.” 

“But\marginnote{3.11} sir, I trust that no white-clothed lay people were present?” 

“No,\marginnote{3.12} there weren’t any white-clothed lay people present.” 

“That’s\marginnote{3.13} good, sir.” 

Then\marginnote{4.1} that mendicant, after taking almsfood in Hatthaka of \textsanskrit{Āḷavī}’s home, got up from his seat and left. Then after the meal, on his return from almsround, he went to the Buddha, bowed, sat down to one side, and told him of what he had discussed with the householder Hatthaka. The Buddha said: 

“Good,\marginnote{7.1} good, mendicant! That gentleman has few wishes. He doesn’t want his own good qualities to be made known to others. Well then, mendicant, you should remember the householder Hatthaka of \textsanskrit{Āḷavī} as someone who has this eighth amazing and incredible quality, that is, fewness of wishes.” 

%
\section*{{\suttatitleacronym AN 8.24}{\suttatitletranslation With Hatthaka (2nd) }{\suttatitleroot Dutiyahatthakasutta}}
\addcontentsline{toc}{section}{\tocacronym{AN 8.24} \toctranslation{With Hatthaka (2nd) } \tocroot{Dutiyahatthakasutta}}
\markboth{With Hatthaka (2nd) }{Dutiyahatthakasutta}
\extramarks{AN 8.24}{AN 8.24}

At\marginnote{1.1} one time the Buddha was staying near \textsanskrit{Āḷavī}, at the \textsanskrit{Aggāḷava} Tree-shrine. Then the householder Hatthaka of \textsanskrit{Āḷavī}, escorted by around five hundred lay followers, went up to the Buddha, bowed, and sat down to one side. The Buddha said to Hatthaka: 

“Hatthaka,\marginnote{1.3} you have a large congregation. How do you bring together such a large congregation?” 

“Sir,\marginnote{1.5} I bring together such a large congregation by using the four ways of being inclusive as taught by the Buddha. When I know that a person can be included by a gift, I include them by giving a gift. When I know that a person can be included by kindly words, I include them by kindly words. When I know that a person can be included by taking care of them, I include them by caring for them. When I know that a person can be included by equality, I include them by treating them equally. But also, sir, my family is wealthy. They wouldn’t think that a poor person was worth listening to in the same way.” 

“Good,\marginnote{1.16} good, Hatthaka! This is the right way to bring together a large congregation. Whether in the past, future, or present, all those who have brought together a large congregation have done so by using these four ways of being inclusive.” 

Then\marginnote{2.1} the Buddha educated, encouraged, fired up, and inspired Hatthaka of \textsanskrit{Āḷavī} with a Dhamma talk, after which he got up from his seat, bowed, and respectfully circled the Buddha before leaving. Then, not long after Hatthaka had left, the Buddha addressed the mendicants: “Mendicants, you should remember the householder Hatthaka of \textsanskrit{Āḷavī} as someone who has eight amazing and incredible qualities. What eight? He’s faithful, ethical, conscientious, prudent, learned, generous, wise, and has few wishes. You should remember the householder Hatthaka of \textsanskrit{Āḷavī} as someone who has these eight amazing and incredible qualities.” 

%
\section*{{\suttatitleacronym AN 8.25}{\suttatitletranslation With Mahānāma }{\suttatitleroot Mahānāmasutta}}
\addcontentsline{toc}{section}{\tocacronym{AN 8.25} \toctranslation{With Mahānāma } \tocroot{Mahānāmasutta}}
\markboth{With Mahānāma }{Mahānāmasutta}
\extramarks{AN 8.25}{AN 8.25}

At\marginnote{1.1} one time the Buddha was staying in the land of the Sakyans, near Kapilavatthu in the Banyan Tree Monastery. Then \textsanskrit{Mahānāma} the Sakyan went up to the Buddha, bowed, sat down to one side, and said to him: 

“Sir,\marginnote{1.3} how is a lay follower defined?” 

“\textsanskrit{Mahānāma},\marginnote{1.4} when you’ve gone for refuge to the Buddha, the teaching, and the \textsanskrit{Saṅgha}, you’re considered to be a lay follower.” 

“But\marginnote{2.1} how is an ethical lay follower defined?” 

“When\marginnote{2.2} a lay follower doesn’t kill living creatures, steal, commit sexual misconduct, lie, or use alcoholic drinks that cause negligence, they’re considered to be an ethical lay follower.” 

“But\marginnote{3.1} how do we define a lay follower who is practicing to benefit themselves, not others?” 

“A\marginnote{3.2} lay follower is accomplished in faith, but doesn’t encourage others to do the same. They’re accomplished in ethical conduct, but don’t encourage others to do the same. They’re accomplished in generosity, but don’t encourage others to do the same. They like to see the mendicants, but don’t encourage others to do the same. They like to hear the true teaching, but don’t encourage others to do the same. They readily memorize the teachings they’ve heard, but don’t encourage others to do the same. They examine the meaning of the teachings they’ve memorized, but don’t encourage others to do the same. Understanding the meaning and the teaching, they practice accordingly, but they don’t encourage others to do the same. That’s how we define a lay follower who is practicing to benefit themselves, not others.” 

“But\marginnote{4.1} how do we define a lay follower who is practicing to benefit both themselves and others?” 

“A\marginnote{4.2} lay follower is accomplished in faith and encourages others to do the same. They’re accomplished in ethical conduct and encourage others to do the same. They’re accomplished in generosity and encourage others to do the same. They like to see the mendicants and encourage others to do the same. They like to hear the true teaching and encourage others to do the same. They readily memorize the teachings they’ve heard and encourage others to do the same. They examine the meaning of the teachings they’ve memorized and encourage others to do the same. Understanding the meaning and the teaching, they practice accordingly and they encourage others to do the same. That’s how we define a lay follower who is practicing to benefit both themselves and others.” 

%
\section*{{\suttatitleacronym AN 8.26}{\suttatitletranslation With Jīvaka }{\suttatitleroot Jīvakasutta}}
\addcontentsline{toc}{section}{\tocacronym{AN 8.26} \toctranslation{With Jīvaka } \tocroot{Jīvakasutta}}
\markboth{With Jīvaka }{Jīvakasutta}
\extramarks{AN 8.26}{AN 8.26}

At\marginnote{1.1} one time the Buddha was staying near \textsanskrit{Rājagaha} in \textsanskrit{Jīvaka}’s Mango Grove. Then \textsanskrit{Jīvaka} \textsanskrit{Komārabhacca} went up to the Buddha, bowed, sat down to one side, and said to him, “Sir, how is a lay follower defined?” 

“\textsanskrit{Jīvaka},\marginnote{1.4} when you’ve gone for refuge to the Buddha, the teaching, and the \textsanskrit{Saṅgha}, you’re considered to be a lay follower.” 

“But\marginnote{2.1} how is an ethical lay follower defined?” 

“When\marginnote{2.2} a lay follower doesn’t kill living creatures, steal, commit sexual misconduct, lie, or use alcoholic drinks that cause negligence, they’re considered to be an ethical lay follower.” 

“But\marginnote{3.1} how do we define a lay follower who is practicing to benefit themselves, not others?” 

“A\marginnote{3.2} lay follower is accomplished in faith, but doesn’t encourage others to do the same. They’re accomplished in ethical conduct … they’re accomplished in generosity … they like to see the mendicants … they like to hear the true teaching … they memorize the teachings … they examine the meaning … Understanding the meaning and the teaching, they practice accordingly, but they don’t encourage others to do the same. That’s how we define a lay follower who is practicing to benefit themselves, not others.” 

“But\marginnote{4.1} how do we define a lay follower who is practicing to benefit both themselves and others?” 

“A\marginnote{4.2} lay follower is accomplished in faith and encourages others to do the same. They’re accomplished in ethical conduct and encourage others to do the same. They’re accomplished in generosity and encourage others to do the same. They like to see the mendicants and encourage others to do the same. They like to hear the true teaching and encourage others to do the same. They readily memorize the teachings they’ve heard and encourage others to do the same. They examine the meaning of the teachings they’ve memorized and encourage others to do the same. Understanding the meaning and the teaching, they practice accordingly and they encourage others to do the same. That’s how we define a lay follower who is practicing to benefit both themselves and others.” 

%
\section*{{\suttatitleacronym AN 8.27}{\suttatitletranslation Powers (1st) }{\suttatitleroot Paṭhamabalasutta}}
\addcontentsline{toc}{section}{\tocacronym{AN 8.27} \toctranslation{Powers (1st) } \tocroot{Paṭhamabalasutta}}
\markboth{Powers (1st) }{Paṭhamabalasutta}
\extramarks{AN 8.27}{AN 8.27}

“Mendicants,\marginnote{1.1} there are these eight powers. What eight? Crying is the power of babies. Anger is the power of females. Weapons are the power of bandits. Authority is the power of rulers. Complaining is the power of fools. Reason is the power of the astute. Reflection is the power of the learned. Patience is the power of ascetics and brahmins. These are the eight powers.” 

%
\section*{{\suttatitleacronym AN 8.28}{\suttatitletranslation Powers (2nd) }{\suttatitleroot Dutiyabalasutta}}
\addcontentsline{toc}{section}{\tocacronym{AN 8.28} \toctranslation{Powers (2nd) } \tocroot{Dutiyabalasutta}}
\markboth{Powers (2nd) }{Dutiyabalasutta}
\extramarks{AN 8.28}{AN 8.28}

Then\marginnote{1.1} Venerable \textsanskrit{Sāriputta} went up to the Buddha, bowed, and sat down to one side. The Buddha said to him: 

“\textsanskrit{Sāriputta},\marginnote{1.2} how many powers does a mendicant who has ended the defilements have that qualify them to claim: ‘My defilements have ended’?” 

“Sir,\marginnote{1.4} a mendicant who has ended the defilements has eight powers that qualify them to claim: ‘My defilements have ended.’ 

What\marginnote{2.1} eight? Firstly, a mendicant with defilements ended has clearly seen with right wisdom all conditions as truly impermanent. This is a power that a mendicant who has ended the defilements relies on to claim: ‘My defilements have ended.’ 

Furthermore,\marginnote{3.1} a mendicant with defilements ended has clearly seen with right wisdom that sensual pleasures are truly like a pit of glowing coals. This is a power that a mendicant who has ended the defilements relies on to claim: ‘My defilements have ended.’ 

Furthermore,\marginnote{4.1} the mind of a mendicant with defilements ended slants, slopes, and inclines to seclusion. They’re withdrawn, loving renunciation, and they’ve totally done with defiling influences. This is a power that a mendicant who has ended the defilements relies on to claim: ‘My defilements have ended.’ 

Furthermore,\marginnote{5.1} a mendicant with defilements ended has well developed the four kinds of mindfulness meditation. This is a power that a mendicant who has ended the defilements relies on to claim: ‘My defilements have ended.’ 

Furthermore,\marginnote{6.1} a mendicant with defilements ended has well developed the four bases of psychic power … the five faculties … the seven awakening factors … the noble eightfold path. This is a power that a mendicant who has ended the defilements relies on to claim: ‘My defilements have ended.’ 

A\marginnote{7.1} mendicant who has ended the defilements has these eight powers that qualify them to claim: ‘My defilements have ended.’” 

%
\section*{{\suttatitleacronym AN 8.29}{\suttatitletranslation Lost Opportunities }{\suttatitleroot Akkhaṇasutta}}
\addcontentsline{toc}{section}{\tocacronym{AN 8.29} \toctranslation{Lost Opportunities } \tocroot{Akkhaṇasutta}}
\markboth{Lost Opportunities }{Akkhaṇasutta}
\extramarks{AN 8.29}{AN 8.29}

“‘Now\marginnote{1.1} is the time! Now is the time!’ So says an unlearned ordinary person. But they don’t know whether it’s time or not. Mendicants, there are eight lost opportunities for spiritual practice. What eight? 

Firstly,\marginnote{1.4} a Realized One has arisen in the world—perfected, a fully awakened Buddha, accomplished in knowledge and conduct, holy, knower of the world, supreme guide for those who wish to train, teacher of gods and humans, awakened, blessed. He teaches the Dhamma leading to peace, extinguishment, awakening, as proclaimed by the Holy One. But a person has been reborn in hell. This is the first lost opportunity for spiritual practice. 

Furthermore,\marginnote{2.1} a Realized One has arisen in the world. But a person has been reborn in the animal realm. This is the second lost opportunity. 

Furthermore,\marginnote{3.1} a Realized One has arisen in the world. But a person has been reborn in the ghost realm. This is the third lost opportunity. 

Furthermore,\marginnote{4.1} a Realized One has arisen in the world. But a person has been reborn in one of the long-lived orders of gods. This is the fourth lost opportunity. 

Furthermore,\marginnote{5.1} a Realized One has arisen in the world. But a person has been reborn in the borderlands, among strange barbarian tribes, where monks, nuns, laymen, and laywomen do not go. This is the fifth lost opportunity … 

Furthermore,\marginnote{6.1} a Realized One has arisen in the world. And a person is reborn in a central country. But they have wrong view and distorted perspective: ‘There’s no meaning in giving, sacrifice, or offerings. There’s no fruit or result of good and bad deeds. There’s no afterlife. There’s no such thing as mother and father, or beings that are reborn spontaneously. And there’s no ascetic or brahmin who is well attained and practiced, and who describes the afterlife after realizing it with their own insight.’ This is the sixth lost opportunity … 

Furthermore,\marginnote{7.1} a Realized One has arisen in the world. And a person is reborn in a central country. But they’re witless, dull, stupid, and unable to distinguish what is well said from what is poorly said. This is the seventh lost opportunity … 

Furthermore,\marginnote{8.1} a Realized One has not arisen in the world … So he doesn’t teach the Dhamma leading to peace, extinguishment, awakening, as proclaimed by the Holy One. And a person is reborn in a central country. And they’re wise, bright, clever, and able to distinguish what is well said from what is poorly said. This is the eighth lost opportunity … 

There\marginnote{9.1} are these eight lost opportunities for spiritual practice. 

Mendicants,\marginnote{10.1} there is just one opportunity for spiritual practice. What is that one? It’s when a Realized One has arisen in the world, perfected, a fully awakened Buddha, accomplished in knowledge and conduct, holy, knower of the world, supreme guide for those who wish to train, teacher of gods and humans, awakened, blessed. He teaches the Dhamma leading to peace, extinguishment, awakening, as proclaimed by the Holy One. And a person is reborn in a central country. And they’re wise, bright, clever, and able to distinguish what is well said from what is poorly said. This is the one opportunity for spiritual practice. 

\begin{verse}%
When\marginnote{11.1} you’ve gained the human state, \\
and the true teaching has been so well proclaimed, \\
if you don’t seize the moment \\
it’ll pass you by. 

For\marginnote{12.1} many wrong times are spoken of, \\
which obstruct the path. \\
Only on rare occasions \\
do Realized Ones arise. 

If\marginnote{13.1} you find yourself in their presence, \\
so hard to find in the world, \\
and if you’ve gained a human birth, \\
and the teaching of the Dhamma; \\
that’s enough to make an effort, \\
for a person who loves themselves. 

How\marginnote{14.1} is the true teaching to be understood \\
so that the moment doesn’t pass you by? \\
For if you miss your moment \\
you’ll grieve when sent to hell. 

If\marginnote{15.1} you fail to achieve \\
certainty regarding the true teaching \\
you’ll regret it for a long time, \\
like a trader who loses a profit. 

A\marginnote{16.1} man shrouded by ignorance, \\
a failure in the true teaching, \\
will long undergo \\
transmigration through birth and death. 

Those\marginnote{17.1} who’ve gained the human state \\
when the true teaching has been so well proclaimed, \\
and have completed what the Teacher taught—\\
or will do so, or are doing so now—

have\marginnote{18.1} realized the right time in the world \\
for the supreme spiritual life. \\
You should live guarded, ever mindful, \\
not soaked with defilements, 

among\marginnote{19.1} those restrained ones \\
who have practiced the path \\
proclaimed by the Realized One, the one with vision, \\
and taught by the kinsman of the Sun. 

Having\marginnote{20.1} cut off all underlying tendencies \\
that follow those drifting in \textsanskrit{Māra}’s dominion, \\
they’re the ones in this world who’ve truly crossed over, \\
having reached the ending of defilements.” 

%
\end{verse}

%
\section*{{\suttatitleacronym AN 8.30}{\suttatitletranslation Anuruddha and the Great Thoughts }{\suttatitleroot Anuruddhamahāvitakkasutta}}
\addcontentsline{toc}{section}{\tocacronym{AN 8.30} \toctranslation{Anuruddha and the Great Thoughts } \tocroot{Anuruddhamahāvitakkasutta}}
\markboth{Anuruddha and the Great Thoughts }{Anuruddhamahāvitakkasutta}
\extramarks{AN 8.30}{AN 8.30}

At\marginnote{1.1} one time the Buddha was staying in the land of the Bhaggas on Crocodile Hill, in the deer park at \textsanskrit{Bhesakaḷā}’s Wood. And at that time Venerable Anuruddha was staying in the land of the \textsanskrit{Cetīs} in the Eastern Bamboo Park. Then as Anuruddha was in private retreat this thought came to his mind: 

“This\marginnote{1.4} teaching is for those of few wishes, not those of many wishes. It’s for the contented, not those who lack contentment. It’s for the secluded, not those who enjoy company. It’s for the energetic, not the lazy. It’s for the mindful, not the unmindful. It’s for those with immersion, not those without immersion. It’s for the wise, not the witless.” 

Then\marginnote{2.1} the Buddha knew what Anuruddha was thinking. As easily as a strong person would extend or contract their arm, he vanished from the deer park at \textsanskrit{Bhesakaḷā}’s Wood in the land of the Bhaggas and reappeared in front of Anuruddha in the Eastern Bamboo Park in the land of the \textsanskrit{Cetīs}, and sat on the seat spread out. Anuruddha bowed to the Buddha and sat down to one side. The Buddha said to him: 

“Good,\marginnote{3.1} good, Anuruddha! It’s good that you reflect on these thoughts of a great man: ‘This teaching is for those of few wishes, not those of many wishes. It’s for the contented, not those who lack contentment. It’s for the secluded, not those who enjoy company. It’s for the energetic, not the lazy. It’s for the mindful, not the unmindful. It’s for those with immersion, not those without immersion. It’s for the wise, not the witless.’ Well then, Anuruddha, you should also reflect on the following eighth thought of a great man: ‘This teaching is for those who don’t enjoy proliferating and don’t like to proliferate, not for those who enjoy proliferating and like to proliferate.’ 

First\marginnote{4.1} you’ll reflect on these eight thoughts of a great man. Then whenever you want, quite secluded from sensual pleasures, secluded from unskillful qualities, you’ll enter and remain in the first absorption, which has the rapture and bliss born of seclusion, while placing the mind and keeping it connected. 

You’ll\marginnote{5.1} enter and remain in the second absorption, which has the rapture and bliss born of immersion, with internal clarity and confidence, and unified mind, without placing the mind and keeping it connected. 

You’ll\marginnote{6.1} enter and remain in the third absorption, where you’ll meditate with equanimity, mindful and aware, personally experiencing the bliss of which the noble ones declare, ‘Equanimous and mindful, one meditates in bliss.’ 

Giving\marginnote{7.1} up pleasure and pain, and ending former happiness and sadness, you’ll enter and remain in the fourth absorption, without pleasure or pain, with pure equanimity and mindfulness. 

First\marginnote{8.1} you’ll reflect on these eight thoughts of a great man, and you’ll get the four absorptions—blissful meditations in the present life that belong to the higher mind—when you want, without trouble or difficulty. Then as you live contented your rag robe will seem to you like a chest full of garments of different colors seems to a householder or householder’s child. It will be for your enjoyment, relief, and comfort, and for alighting upon extinguishment. 

As\marginnote{9.1} you live contented your scraps of almsfood will seem to you like boiled fine rice with the dark grains picked out, served with many soups and sauces seems to a householder or householder’s child. It will be for your enjoyment, relief, and comfort, and for alighting upon extinguishment. 

As\marginnote{10.1} you live contented your lodging at the root of a tree will seem to you like a bungalow, plastered inside and out, draft-free, with latches fastened and windows shuttered seems to a householder or householder’s child. It will be for your enjoyment, relief, and comfort, and for alighting upon extinguishment. 

As\marginnote{11.1} you live contented your lodging at the root of a tree will seem to you like a couch spread with woolen covers—shag-piled, pure white, or embroidered with flowers—and spread with a fine deer hide, with a canopy above and red pillows at both ends seems to a householder or householder’s child. It will be for your enjoyment, relief, and comfort, and for alighting upon extinguishment. 

As\marginnote{12.1} you live contented your fermented urine as medicine will seem to you like various medicines—ghee, butter, oil, honey, and molasses—seem to a householder or householder’s child. It will be for your enjoyment, relief, and comfort, and for alighting upon extinguishment. Well then, Anuruddha, for the next rainy season residence you should stay right here in the land of the \textsanskrit{Cetīs} in the Eastern Bamboo Park.” 

“Yes,\marginnote{12.4} sir,” Anuruddha replied. 

After\marginnote{13.1} advising Anuruddha like this, the Buddha—as easily as a strong person would extend or contract their arm, vanished from the Eastern Bamboo Park in the land of the \textsanskrit{Cetīs} and reappeared in the deer park at \textsanskrit{Bhesakaḷā}’s Wood in the land of the Bhaggas. He sat on the seat spread out and addressed the mendicants: “Mendicants, I will teach you the eight thoughts of a great man. Listen … 

And\marginnote{13.5} what are the eight thoughts of a great man? This teaching is for those of few wishes, not those of many wishes. It’s for the contented, not those who lack contentment. It’s for the secluded, not those who enjoy company. It’s for the energetic, not the lazy. It’s for the mindful, not the unmindful. It’s for those with immersion, not those without immersion. It’s for the wise, not the witless. It’s for those who don’t enjoy proliferating and don’t like to proliferate, not for those who enjoy proliferating and like to proliferate. 

‘This\marginnote{14.1} teaching is for those of few wishes, not those of many wishes.’ That’s what I said, but why did I say it? A mendicant with few wishes doesn’t wish: ‘May they know me as having few wishes!’ When contented, they don’t wish: ‘May they know me as contented!’ When secluded, they don’t wish: ‘May they know me as secluded!’ When energetic, they don’t wish: ‘May they know me as energetic!’ When mindful, they don’t wish: ‘May they know me as mindful!’ When immersed, they don’t wish: ‘May they know me as immersed!’ When wise, they don’t wish: ‘May they know me as wise!’ When not enjoying proliferation, they don’t wish: ‘May they know me as one who doesn’t enjoy proliferating!’ ‘This teaching is for those of few wishes, not those of many wishes.’ That’s what I said, and this is why I said it. 

‘This\marginnote{15.1} teaching is for the contented, not those who lack contentment.’ That’s what I said, but why did I say it? It’s for a mendicant who’s content with any kind of robes, almsfood, lodgings, and medicines and supplies for the sick. ‘This teaching is for the contented, not those who lack contentment.’ That’s what I said, and this is why I said it. 

‘This\marginnote{16.1} teaching is for the secluded, not those who enjoy company.’ That’s what I said, but why did I say it? It’s for a mendicant who lives secluded. But monks, nuns, laymen, laywomen, rulers and their ministers, founders of religious sects and their disciples go to visit them. With a mind slanting, sloping, and inclining to seclusion, withdrawn, and loving renunciation, that mendicant invariably gives each of them a talk emphasizing the topic of dismissal. ‘This teaching is for the secluded, not those who enjoy company.’ That’s what I said, and this is why I said it. 

‘This\marginnote{17.1} teaching is for the energetic, not the lazy.’ That’s what I said, but why did I say it? It’s for a mendicant who lives with energy roused up for giving up unskillful qualities and embracing skillful qualities. They’re strong, staunchly vigorous, not slacking off when it comes to developing skillful qualities. ‘This teaching is for the energetic, not the lazy.’ That’s what I said, and this is why I said it. 

‘This\marginnote{18.1} teaching is for the mindful, not the unmindful.’ That’s what I said, but why did I say it? It’s for a mendicant who’s mindful. They have utmost mindfulness and alertness, and can remember and recall what was said and done long ago. ‘This teaching is for the mindful, not the unmindful.’ That’s what I said, and this is why I said it. 

‘This\marginnote{19.1} teaching is for those with immersion, not those without immersion.’ That’s what I said, but why did I say it? It’s for a mendicant who, quite secluded from sensual pleasures, secluded from unskillful qualities, enters and remains in the first absorption … second absorption … third absorption … fourth absorption. ‘This teaching is for those with immersion, not those without immersion.’ That’s what I said, and this is why I said it. 

‘This\marginnote{20.1} teaching is for the wise, not the witless.’ That’s what I said, but why did I say it? It’s for a mendicant who’s wise. They have the wisdom of arising and passing away which is noble, penetrative, and leads to the complete ending of suffering. ‘This teaching is for the wise, not the witless.’ That’s what I said, and this is why I said it. 

‘This\marginnote{21.1} teaching is for those who don’t enjoy proliferating and don’t like to proliferate, not for those who enjoy proliferating and like to proliferate.’ That’s what I said, but why did I say it? It’s for a mendicant whose mind is eager, confident, settled, and decided regarding the cessation of proliferation. ‘This teaching is for those who don’t enjoy proliferating and don’t like to proliferate, not for those who enjoy proliferating and like to proliferate.’ That’s what I said, and this is why I said it.” 

Then\marginnote{22.1} Anuruddha stayed the next rainy season residence right there in the land of the \textsanskrit{Cetīs} in the Eastern Bamboo Park. And Anuruddha, living alone, withdrawn, diligent, keen, and resolute, soon realized the supreme culmination of the spiritual path in this very life. He lived having achieved with his own insight the goal for which gentlemen rightly go forth from the lay life to homelessness. 

He\marginnote{22.3} understood: “Rebirth is ended; the spiritual journey has been completed; what had to be done has been done; there is no return to any state of existence.” And Venerable Anuruddha became one of the perfected. And on the occasion of attaining perfection he recited these verses: 

\begin{verse}%
“Knowing\marginnote{23.1} my thoughts, \\
the supreme Teacher in the world \\
came to me in a mind-made body, \\
using his psychic power. 

He\marginnote{24.1} taught me more \\
than I had thought of. \\
The Buddha who loves non-proliferation \\
taught me non-proliferation. 

Understanding\marginnote{25.1} that teaching, \\
I happily did his bidding. \\
I’ve attained the three knowledges, \\
and have fulfilled the Buddha’s instructions.” 

%
\end{verse}

%
\addtocontents{toc}{\let\protect\contentsline\protect\nopagecontentsline}
\chapter*{The Chapter on Giving }
\addcontentsline{toc}{chapter}{\tocchapterline{The Chapter on Giving }}
\addtocontents{toc}{\let\protect\contentsline\protect\oldcontentsline}

%
\section*{{\suttatitleacronym AN 8.31}{\suttatitletranslation Giving (1st) }{\suttatitleroot Paṭhamadānasutta}}
\addcontentsline{toc}{section}{\tocacronym{AN 8.31} \toctranslation{Giving (1st) } \tocroot{Paṭhamadānasutta}}
\markboth{Giving (1st) }{Paṭhamadānasutta}
\extramarks{AN 8.31}{AN 8.31}

“Mendicants,\marginnote{1.1} there are these eight gifts. What eight? A person might give a gift after insulting the recipient. Or they give out of fear. Or they give thinking, ‘They gave to me.’ Or they give thinking, ‘They’ll give to me.’ Or they give thinking, ‘It’s good to give.’ Or they give thinking, ‘I cook, they don’t. It wouldn’t be right for me to not give to them.’ Or they give thinking, ‘By giving this gift I’ll get a good reputation.’ Or they give thinking, ‘This is an adornment and requisite for the mind.’ These are the eight gifts.” 

%
\section*{{\suttatitleacronym AN 8.32}{\suttatitletranslation Giving (2nd) }{\suttatitleroot Dutiyadānasutta}}
\addcontentsline{toc}{section}{\tocacronym{AN 8.32} \toctranslation{Giving (2nd) } \tocroot{Dutiyadānasutta}}
\markboth{Giving (2nd) }{Dutiyadānasutta}
\extramarks{AN 8.32}{AN 8.32}

\begin{verse}%
“Faith,\marginnote{1.1} conscience, and skillful giving \\
are qualities good people follow. \\
For this, they say, is the path of the gods, \\
which leads to the heavenly realm.” 

%
\end{verse}

%
\section*{{\suttatitleacronym AN 8.33}{\suttatitletranslation Reasons to Give }{\suttatitleroot Dānavatthusutta}}
\addcontentsline{toc}{section}{\tocacronym{AN 8.33} \toctranslation{Reasons to Give } \tocroot{Dānavatthusutta}}
\markboth{Reasons to Give }{Dānavatthusutta}
\extramarks{AN 8.33}{AN 8.33}

“Mendicants,\marginnote{1.1} there are these eight grounds for giving. What eight? A person might give a gift out of favoritism or hostility or stupidity or cowardice. Or they give thinking, ‘Giving was practiced by my father and my father’s father. It would not be right for me to abandon this family tradition.’ Or they give thinking, ‘After I’ve given this gift, when my body breaks up, after death, I’ll be reborn in a good place, a heavenly realm.’ Or they give thinking, ‘When giving this gift my mind becomes clear, and I become happy and joyful.’ Or they give a gift thinking, ‘This is an adornment and requisite for the mind.’ These are the eight grounds for giving.” 

%
\section*{{\suttatitleacronym AN 8.34}{\suttatitletranslation A Field }{\suttatitleroot Khettasutta}}
\addcontentsline{toc}{section}{\tocacronym{AN 8.34} \toctranslation{A Field } \tocroot{Khettasutta}}
\markboth{A Field }{Khettasutta}
\extramarks{AN 8.34}{AN 8.34}

“Mendicants,\marginnote{1.1} when a field has eight factors a seed sown in it is not very fruitful or rewarding or productive. What eight factors does it have? It’s when a field has mounds and ditches. It has stones and gravel. It’s salty. It doesn’t have deep furrows. And it’s not equipped with water inlets, water outlets, irrigation channels, and boundaries. When a field has these eight factors a seed sown in it is not fruitful or rewarding or productive. 

In\marginnote{2.1} the same way, when an ascetic or brahmin has eight factors a gift given to them is not very fruitful or beneficial or splendid or bountiful. What eight factors do they have? It’s when an ascetic or brahmin has wrong view, wrong thought, wrong speech, wrong action, wrong livelihood, wrong effort, wrong mindfulness, and wrong immersion. When an ascetic or brahmin has these eight factors a gift given to them is not very fruitful or beneficial or splendid or bountiful. 

When\marginnote{3.1} a field has eight factors a seed sown in it is very fruitful and rewarding and productive. What eight factors does it have? It’s when a field doesn’t have mounds and ditches. It doesn’t have stones and gravel. It’s not salty. It has deep furrows. And it’s equipped with water inlets, water outlets, irrigation channels, and boundaries. When a field has these eight factors a seed sown in it is very fruitful and rewarding and productive. 

In\marginnote{4.1} the same way, when an ascetic or brahmin has eight factors a gift given to them is very fruitful and beneficial and splendid and bountiful. What eight factors do they have? It’s when an ascetic or brahmin has right view, right thought, right speech, right action, right livelihood, right effort, right mindfulness, and right immersion. When an ascetic or brahmin has these eight factors a gift given to them is very fruitful and beneficial and splendid and bountiful. 

\begin{verse}%
When\marginnote{5.1} the field is excellent, \\
and the seed sown in it is excellent, \\
and the rainfall is excellent, \\
the crop of grain will be excellent. 

Its\marginnote{6.1} health is excellent, \\
its growth is excellent, \\
its maturation is excellent, \\
and its fruit is excellent. 

So\marginnote{7.1} too, when you give excellent food \\
to those of excellent ethics, \\
it leads to many excellences, \\
for what you did was excellent. 

So\marginnote{8.1} if a person wants excellence, \\
let them excel in this. \\
You should frequent those with excellent wisdom, \\
so that your own excellence will flourish. 

Excelling\marginnote{9.1} in knowledge and conduct, \\
and having excellence of mind, \\
you perform excellent deeds, \\
and gain excellent benefits. 

Truly\marginnote{10.1} knowing the world, \\
and having attained excellence of view, \\
one who excels in mind proceeds, \\
relying on excellence in the path. 

Shaking\marginnote{11.1} off all stains, \\
and attaining the excellence of extinguishment, \\
you’re freed from all sufferings: \\
this is complete excellence.” 

%
\end{verse}

%
\section*{{\suttatitleacronym AN 8.35}{\suttatitletranslation Rebirth by Giving }{\suttatitleroot Dānūpapattisutta}}
\addcontentsline{toc}{section}{\tocacronym{AN 8.35} \toctranslation{Rebirth by Giving } \tocroot{Dānūpapattisutta}}
\markboth{Rebirth by Giving }{Dānūpapattisutta}
\extramarks{AN 8.35}{AN 8.35}

“Mendicants,\marginnote{1.1} there are these eight rebirths by giving. What eight? 

First,\marginnote{1.3} someone gives to ascetics or brahmins such things as food, drink, clothing, vehicles; garlands, fragrance, and makeup; and bed, house, and lighting. Whatever they give, they expect something back. They see a well-to-do aristocrat or brahmin or householder amusing themselves, supplied and provided with the five kinds of sensual stimulation. It occurs to them: ‘If only, when my body breaks up, after death, I would be reborn in the company of well-to-do aristocrats or brahmins or householders!’ They settle on that idea, concentrate on it and develop it. As they’ve settled for less and not developed further, their idea leads to rebirth there. When their body breaks up, after death, they’re reborn in the company of well-to-do aristocrats or brahmins or householders. But I say that this is only for those of ethical conduct, not for the unethical. The heart’s wish of an ethical person succeeds because of their purity. 

Next,\marginnote{2.1} someone gives to ascetics or brahmins … Whatever they give, they expect something back. And they’ve heard: ‘The Gods of the Four Great Kings are long-lived, beautiful, and very happy.’ It occurs to them: ‘If only, when my body breaks up, after death, I would be reborn in the company of the Gods of the Four Great Kings!’ … When their body breaks up, after death, they’re reborn in the company of the Gods of the Four Great Kings. But I say that this is only for those of ethical conduct, not for the unethical. The heart’s wish of an ethical person succeeds because of their purity. 

Next,\marginnote{3.1} someone gives to ascetics or brahmins … Whatever they give, they expect something back. And they’ve heard: ‘The Gods of the Thirty-Three …’ 

‘The\marginnote{3.5} Gods of Yama …’ 

‘The\marginnote{3.6} Joyful Gods …’ 

‘The\marginnote{3.7} Gods Who Love to Create …’ 

‘The\marginnote{3.8} Gods Who Control the Creations of Others are long-lived, beautiful, and very happy.’ It occurs to them: ‘If only, when my body breaks up, after death, I would be reborn in the company of the Gods Who Control the Creations of Others!’ They settle on that idea, concentrate on it and develop it. As they’ve settled for less and not developed further, their idea leads to rebirth there. When their body breaks up, after death, they’re reborn in the company of the Gods Who Control the Creations of Others. But I say that this is only for those of ethical conduct, not for the unethical. The heart’s wish of an ethical person succeeds because of their purity. 

Next,\marginnote{4.1} someone gives to ascetics or brahmins such things as food, drink, clothing, vehicles; garlands, fragrance, and makeup; and bed, house, and lighting. Whatever they give, they expect something back. And they’ve heard: ‘The Gods of \textsanskrit{Brahmā}’s Host are long-lived, beautiful, and very happy.’ It occurs to them: ‘If only, when my body breaks up, after death, I would be reborn in the company of the Gods of \textsanskrit{Brahmā}’s Host!’ They settle on that idea, concentrate on it and develop it. As they’ve settled for less and not developed further, their idea leads to rebirth there. When their body breaks up, after death, they’re reborn in the company of the Gods of \textsanskrit{Brahmā}’s Host. But I say that this is only for those of ethical conduct, not for the unethical. And for those free of desire, not those with desire. The heart’s wish of an ethical person succeeds because of their freedom from desire. 

These\marginnote{4.13} are the eight rebirths by giving.” 

%
\section*{{\suttatitleacronym AN 8.36}{\suttatitletranslation Grounds for Making Merit }{\suttatitleroot Puññakiriyavatthusutta}}
\addcontentsline{toc}{section}{\tocacronym{AN 8.36} \toctranslation{Grounds for Making Merit } \tocroot{Puññakiriyavatthusutta}}
\markboth{Grounds for Making Merit }{Puññakiriyavatthusutta}
\extramarks{AN 8.36}{AN 8.36}

“Mendicants,\marginnote{1.1} there are these three grounds for making merit. What three? Giving, ethical conduct, and meditation are all grounds for making merit. 

First,\marginnote{2.1} someone has practiced a little giving and ethical conduct as grounds for making merit, but they haven’t got as far as meditation as a ground for making merit. When their body breaks up, after death, they’re reborn among disadvantaged humans. 

Next,\marginnote{3.1} someone has practiced a moderate amount of giving and ethical conduct as grounds for making merit, but they haven’t got as far as meditation as a ground for making merit. When their body breaks up, after death, they’re reborn among well-off humans. 

Next,\marginnote{4.1} someone has practiced a lot of giving and ethical conduct as grounds for making merit, but they haven’t got as far as meditation as a ground for making merit. When their body breaks up, after death, they’re reborn in the company of the Gods of the Four Great Kings. There, the Four Great Kings themselves have practiced giving and ethical conduct as grounds for making merit to a greater degree than the other gods. So they surpass them in ten respects: divine life span, beauty, happiness, glory, sovereignty, sights, sounds, smells, tastes, and touches. 

Next,\marginnote{5.1} someone has practiced a lot of giving and ethical conduct as grounds for making merit, but they haven’t got as far as meditation as a ground for making merit. When their body breaks up, after death, they’re reborn in the company of the Gods of the Thirty-Three. There, Sakka, lord of gods, has practiced giving and ethical conduct as grounds for making merit to a greater degree than the other gods. So he surpasses them in ten respects … 

Next,\marginnote{6.1} someone has practiced a lot of giving and ethical conduct as grounds for making merit, but they haven’t got as far as meditation as a ground for making merit. When their body breaks up, after death, they’re reborn in the company of the Gods of Yama. There, the god \textsanskrit{Suyāma} has practiced giving and ethical conduct as grounds for making merit to a greater degree than the other gods. So he surpasses them in ten respects … 

Next,\marginnote{7.1} someone has practiced a lot of giving and ethical conduct as grounds for making merit, but they haven’t got as far as meditation as a ground for making merit. When their body breaks up, after death, they’re reborn in the company of the Joyful Gods. There, the god Santusita has practiced giving and ethical conduct as grounds for making merit to a greater degree than the other gods. So he surpasses them in ten respects … 

Next,\marginnote{8.1} someone has practiced a lot of giving and ethical conduct as grounds for making merit, but they haven’t got as far as meditation as a ground for making merit. When their body breaks up, after death, they’re reborn in the company of the Gods Who Love to Create. There, the god Sunimmita has practiced giving and ethical conduct as grounds for making merit to a greater degree than the other gods. So he surpasses them in ten respects … 

Next,\marginnote{9.1} someone has practiced a lot of giving and ethical conduct as grounds for making merit, but they haven’t got as far as meditation as a ground for making merit. When their body breaks up, after death, they’re reborn in the company of the Gods Who Control the Creations of Others. There, the god \textsanskrit{Vasavattī} has practiced giving and ethical conduct as grounds for making merit to a greater degree than the other gods. So he surpasses them in ten respects: divine life span, beauty, happiness, glory, sovereignty, sights, sounds, smells, tastes, and touches. 

These\marginnote{9.5} are the three grounds for making merit.” 

%
\section*{{\suttatitleacronym AN 8.37}{\suttatitletranslation Gifts of a Good Person }{\suttatitleroot Sappurisadānasutta}}
\addcontentsline{toc}{section}{\tocacronym{AN 8.37} \toctranslation{Gifts of a Good Person } \tocroot{Sappurisadānasutta}}
\markboth{Gifts of a Good Person }{Sappurisadānasutta}
\extramarks{AN 8.37}{AN 8.37}

“Mendicants,\marginnote{1.1} there are these eight gifts of a good person. What eight? Their gift is pure, good quality, timely, appropriate, intelligent, and regular. While giving their heart is confident, and afterwards they’re uplifted. These are the eight gifts of a good person. 

\begin{verse}%
He\marginnote{2.1} gives pure, good quality, and timely gifts \\
of appropriate food and drinks \\
regularly to spiritual practitioners \\
who are fertile fields of merit. 

They\marginnote{3.1} never regret \\
giving away many material things. \\
Discerning people praise \\
giving such gifts. 

An\marginnote{4.1} intelligent person sacrifices like this, \\
faithful, with a mind of letting go. \\
Such an astute person is reborn \\
in a happy, pleasing world.” 

%
\end{verse}

%
\section*{{\suttatitleacronym AN 8.38}{\suttatitletranslation A Good Person }{\suttatitleroot Sappurisasutta}}
\addcontentsline{toc}{section}{\tocacronym{AN 8.38} \toctranslation{A Good Person } \tocroot{Sappurisasutta}}
\markboth{A Good Person }{Sappurisasutta}
\extramarks{AN 8.38}{AN 8.38}

“Mendicants,\marginnote{1.1} a good person is born in a family for the benefit, welfare, and happiness of the people. For the benefit, welfare, and happiness of mother and father; children and partners; bondservants, workers, and staff; friends and colleagues; departed ancestors; the king; the deities; and ascetics and brahmins. 

It’s\marginnote{2.1} like a great rain cloud, which nourishes all the crops for the benefit, welfare, and happiness of the people. In the same way, a good person is born in a family for the benefit, welfare, and happiness of the people. … 

\begin{verse}%
A\marginnote{3.1} wise person living at home \\
benefits many people. \\
Neither by day or at night do they neglect \\
their mother, father, and ancestors. \\
They venerate them in accord with the teaching, \\
remembering what they have done. 

One\marginnote{4.3} of settled faith and good nature \\
venerates the homeless renunciates, \\
the mendicant spiritual practitioners, \\
knowing their good-hearted qualities. 

Good\marginnote{5.3} for the king, good for the gods, \\
and good for relatives and friends. 

In\marginnote{6.1} fact, they’re good for everyone, \\
well grounded in the true teaching. \\
Rid of the stain of stinginess, \\
they’ll enjoy a world of grace.” 

%
\end{verse}

%
\section*{{\suttatitleacronym AN 8.39}{\suttatitletranslation Overflowing Merit }{\suttatitleroot Abhisandasutta}}
\addcontentsline{toc}{section}{\tocacronym{AN 8.39} \toctranslation{Overflowing Merit } \tocroot{Abhisandasutta}}
\markboth{Overflowing Merit }{Abhisandasutta}
\extramarks{AN 8.39}{AN 8.39}

“Mendicants,\marginnote{1.1} there are these eight kinds of overflowing merit, overflowing goodness. They nurture happiness and are conducive to heaven, ripening in happiness and leading to heaven. They lead to what is likable, desirable, agreeable, to welfare and happiness. What eight? 

Firstly,\marginnote{1.3} a noble disciple has gone for refuge to the Buddha. This is the first kind of overflowing merit … 

Furthermore,\marginnote{2.1} a noble disciple has gone for refuge to the teaching. This is the second kind of overflowing merit … 

Furthermore,\marginnote{3.1} a noble disciple has gone for refuge to the \textsanskrit{Saṅgha}. This is the third kind of overflowing merit … 

Mendicants,\marginnote{3.3} these five gifts are great, primordial, long-standing, traditional, and ancient. They are uncorrupted, as they have been since the beginning. They’re not being corrupted now nor will they be. Sensible ascetics and brahmins don’t look down on them. What five? 

Firstly,\marginnote{3.5} a noble disciple gives up killing living creatures. By so doing they give to countless sentient beings the gift of freedom from fear, enmity, and ill will. And they themselves also enjoy unlimited freedom from fear, enmity, and ill will. This is the first gift that is a great offering, primordial, long-standing, traditional, and ancient. It is uncorrupted, as it has been since the beginning. It’s not being corrupted now nor will it be. Sensible ascetics and brahmins don’t look down on it. This is the fourth kind of overflowing merit … 

Furthermore,\marginnote{4.1} a noble disciple gives up stealing. … This is the fifth kind of overflowing merit … 

Furthermore,\marginnote{5.1} a noble disciple gives up sexual misconduct. … This is the sixth kind of overflowing merit … 

Furthermore,\marginnote{6.1} a noble disciple gives up lying. … This is the seventh kind of overflowing merit … 

Furthermore,\marginnote{7.1} a noble disciple gives up alcoholic drinks that cause negligence. By so doing they give to countless sentient beings the gift of freedom from fear, enmity, and ill will. And they themselves also enjoy unlimited freedom from fear, enmity, and ill will. This is the fifth gift that is a great offering, primordial, long-standing, traditional, and ancient. It is uncorrupted, as it has been since the beginning. It’s not being corrupted now nor will it be. Sensible ascetics and brahmins don’t look down on it. This is the eighth kind of overflowing merit … 

These\marginnote{8.1} are the eight kinds of overflowing merit, overflowing goodness. They nurture happiness and are conducive to heaven, ripening in happiness and leading to heaven. They lead to what is likable, desirable, agreeable, to welfare and happiness.” 

%
\section*{{\suttatitleacronym AN 8.40}{\suttatitletranslation The Results of Misconduct }{\suttatitleroot Duccaritavipākasutta}}
\addcontentsline{toc}{section}{\tocacronym{AN 8.40} \toctranslation{The Results of Misconduct } \tocroot{Duccaritavipākasutta}}
\markboth{The Results of Misconduct }{Duccaritavipākasutta}
\extramarks{AN 8.40}{AN 8.40}

“Mendicants,\marginnote{1.1} the killing of living creatures, when cultivated, developed, and practiced, leads to hell, the animal realm, or the ghost realm. The minimum result it leads to for a human being is a short life span. 

Stealing,\marginnote{2.1} when cultivated, developed, and practiced, leads to hell, the animal realm, or the ghost realm. The minimum result it leads to for a human being is loss of wealth. 

Sexual\marginnote{3.1} misconduct, when cultivated, developed, and practiced, leads to hell, the animal realm, or the ghost realm. The minimum result it leads to for a human being is rivalry and enmity. 

Lying,\marginnote{4.1} when cultivated, developed, and practiced, leads to hell, the animal realm, or the ghost realm. The minimum result it leads to for a human being is false accusations. 

Divisive\marginnote{5.1} speech, when cultivated, developed, and practiced, leads to hell, the animal realm, or the ghost realm. The minimum result it leads to for a human being is being divided against friends. 

Harsh\marginnote{6.1} speech, when cultivated, developed, and practiced, leads to hell, the animal realm, or the ghost realm. The minimum result it leads to for a human being is hearing disagreeable things. 

Talking\marginnote{7.1} nonsense, when cultivated, developed, and practiced, leads to hell, the animal realm, or the ghost realm. The minimum result it leads to for a human being is that no-one takes what you say seriously. 

Taking\marginnote{8.1} alcoholic drinks that cause negligence, when cultivated, developed, and practiced, leads to hell, the animal realm, or the ghost realm. The minimum result it leads to for a human being is madness.” 

%
\addtocontents{toc}{\let\protect\contentsline\protect\nopagecontentsline}
\chapter*{The Chapter on Sabbath }
\addcontentsline{toc}{chapter}{\tocchapterline{The Chapter on Sabbath }}
\addtocontents{toc}{\let\protect\contentsline\protect\oldcontentsline}

%
\section*{{\suttatitleacronym AN 8.41}{\suttatitletranslation The Sabbath With Eight Factors, In Brief }{\suttatitleroot Saṁkhittūposathasutta}}
\addcontentsline{toc}{section}{\tocacronym{AN 8.41} \toctranslation{The Sabbath With Eight Factors, In Brief } \tocroot{Saṁkhittūposathasutta}}
\markboth{The Sabbath With Eight Factors, In Brief }{Saṁkhittūposathasutta}
\extramarks{AN 8.41}{AN 8.41}

\scevam{So\marginnote{1.1} I have heard. }At one time the Buddha was staying near \textsanskrit{Sāvatthī} in Jeta’s Grove, \textsanskrit{Anāthapiṇḍika}’s monastery. There the Buddha addressed the mendicants, “Mendicants!” 

“Venerable\marginnote{1.5} sir,” they replied. The Buddha said this: 

“Mendicants,\marginnote{2.1} the observance of the sabbath with its eight factors is very fruitful and beneficial and splendid and bountiful. And how should it be observed? It’s when a noble disciple reflects: ‘As long as they live, the perfected ones give up killing living creatures, renouncing the rod and the sword. They are scrupulous and kind, and live full of compassion for all living beings. I, too, for this day and night will give up killing living creatures, renouncing the rod and the sword. I’ll be scrupulous and kind, and live full of compassion for all living beings. I will observe the sabbath by doing as the perfected ones do in this respect.’ This is its first factor. 

‘As\marginnote{3.1} long as they live, the perfected ones give up stealing. They take only what’s given, and expect only what’s given. They keep themselves clean by not thieving. I, too, for this day and night will give up stealing. I’ll take only what’s given, and expect only what’s given. I’ll keep myself clean by not thieving. I will observe the sabbath by doing as the perfected ones do in this respect.’ This is its second factor. 

‘As\marginnote{4.1} long as they live, the perfected ones give up unchastity. They are celibate, set apart, avoiding the common practice of sex. I, too, for this day and night will give up unchastity. I will be celibate, set apart, avoiding the common practice of sex. I will observe the sabbath by doing as the perfected ones do in this respect.’ This is its third factor. 

‘As\marginnote{5.1} long as they live, the perfected ones give up lying. They speak the truth and stick to the truth. They’re honest and trustworthy, and don’t trick the world with their words. I, too, for this day and night will give up lying. I’ll speak the truth and stick to the truth. I’ll be honest and trustworthy, and won’t trick the world with my words. I will observe the sabbath by doing as the perfected ones do in this respect.’ This is its fourth factor. 

‘As\marginnote{6.1} long as they live, the perfected ones give up alcoholic drinks that cause negligence. I, too, for this day and night will give up alcoholic drinks that cause negligence. I will observe the sabbath by doing as the perfected ones do in this respect.’ This is its fifth factor. 

‘As\marginnote{7.1} long as they live, the perfected ones eat in one part of the day, abstaining from eating at night and from food at the wrong time. I, too, for this day and night will eat in one part of the day, abstaining from eating at night and food at the wrong time. I will observe the sabbath by doing as the perfected ones do in this respect.’ This is its sixth factor. 

‘As\marginnote{8.1} long as they live, the perfected ones give up dancing, singing, music, and seeing shows; and beautifying and adorning themselves with garlands, fragrance, and makeup. I, too, for this day and night will give up dancing, singing, music, and seeing shows; and beautifying and adorning myself with garlands, fragrance, and makeup. I will observe the sabbath by doing as the perfected ones do in this respect.’ This is its seventh factor. 

‘As\marginnote{9.1} long as they live, the perfected ones give up high and luxurious beds. They sleep in a low place, either a small bed or a straw mat. I, too, for this day and night will give up high and luxurious beds. I’ll sleep in a low place, either a small bed or a straw mat. I will observe the sabbath by doing as the perfected ones do in this respect.’ This is its eighth factor. 

The\marginnote{10.1} observance of the sabbath with its eight factors in this way is very fruitful and beneficial and splendid and bountiful.” 

%
\section*{{\suttatitleacronym AN 8.42}{\suttatitletranslation The Sabbath With Eight Factors, In Detail }{\suttatitleroot Vitthatūposathasutta}}
\addcontentsline{toc}{section}{\tocacronym{AN 8.42} \toctranslation{The Sabbath With Eight Factors, In Detail } \tocroot{Vitthatūposathasutta}}
\markboth{The Sabbath With Eight Factors, In Detail }{Vitthatūposathasutta}
\extramarks{AN 8.42}{AN 8.42}

“Mendicants,\marginnote{1.1} the observance of the sabbath with its eight factors is very fruitful and beneficial and splendid and bountiful. And how should it be observed? 

It’s\marginnote{1.3} when a noble disciple reflects: ‘As long as they live, the perfected ones give up killing living creatures, renouncing the rod and the sword. They are scrupulous and kind, and live full of compassion for all living beings. I, too, for this day and night will give up killing living creatures, renouncing the rod and the sword. I’ll be scrupulous and kind, and live full of compassion for all living beings. I will observe the sabbath by doing as the perfected ones do in this respect.’ This is its first factor. … 

‘As\marginnote{2.1} long as they live, the perfected ones give up high and luxurious beds. They sleep in a low place, either a small bed or a straw mat. I, too, for this day and night will give up high and luxurious beds. I’ll sleep in a low place, either a small bed or a straw mat. I will observe the sabbath by doing as the perfected ones do in this respect.’ This is its eighth factor. The observance of the sabbath with its eight factors in this way is very fruitful and beneficial and splendid and bountiful. 

How\marginnote{3.1} much so? Suppose you were to rule as sovereign lord over these sixteen great countries—\textsanskrit{Aṅga}, Magadha, \textsanskrit{Kāsī}, Kosala, \textsanskrit{Vajjī}, Malla, Ceti, \textsanskrit{Vaṅga}, Kuru, \textsanskrit{Pañcāla}, Maccha, \textsanskrit{Sūrusena}, Assaka, Avanti, \textsanskrit{Gandhāra}, and Kamboja—full of the seven kinds of precious things. This wouldn’t be worth a sixteenth part of the sabbath with its eight factors. Why is that? Because human kingship is a poor thing compared to the happiness of the gods. 

Fifty\marginnote{4.1} years in the human realm is one day and night for the Gods of the Four Great Kings. Thirty such days make up a month. Twelve such months make up a year. The life span of the Gods of the Four Great Kings is five hundred of these divine years. It’s possible that a woman or man who has observed the eight-factored sabbath will—when their body breaks up, after death—be reborn in the company of the Gods of the Four Great Kings. This is what I was referring to when I said: ‘Human kingship is a poor thing compared to the happiness of the gods.’ 

A\marginnote{5.1} hundred years in the human realm is one day and night for the Gods of the Thirty-Three. Thirty such days make up a month. Twelve such months make up a year. The life span of the Gods of the Thirty-Three is a thousand of these divine years. It’s possible that a woman or man who has observed the eight-factored sabbath will—when their body breaks up, after death—be reborn in the company of the Gods of the Thirty-Three. This is what I was referring to when I said: ‘Human kingship is a poor thing compared to the happiness of the gods.’ 

Two\marginnote{6.1} hundred years in the human realm is one day and night for the Gods of Yama. Thirty such days make up a month. Twelve such months make up a year. The life span of the Gods of Yama is two thousand of these divine years. It’s possible that a woman or man who has observed the eight-factored sabbath will—when their body breaks up, after death—be reborn in the company of the Gods of Yama. This is what I was referring to when I said: ‘Human kingship is a poor thing compared to the happiness of the gods.’ 

Four\marginnote{7.1} hundred years in the human realm is one day and night for the Joyful Gods. Thirty such days make up a month. Twelve such months make up a year. The life span of the Joyful Gods is four thousand of these divine years. It’s possible that a woman or man who has observed the eight-factored sabbath will—when their body breaks up, after death—be reborn in the company of the Joyful Gods. This is what I was referring to when I said: ‘Human kingship is a poor thing compared to the happiness of the gods.’ 

Eight\marginnote{8.1} hundred years in the human realm is one day and night for the Gods Who Love to Create. Thirty such days make up a month. Twelve such months make up a year. The life span of the Gods Who Love to Create is eight thousand of these divine years. It’s possible that a woman or man who has observed the eight-factored sabbath will—when their body breaks up, after death—be reborn in the company of the Gods Who Love to Create. This is what I was referring to when I said: ‘Human kingship is a poor thing compared to the happiness of the gods.’ 

Sixteen\marginnote{9.1} hundred years in the human realm is one day and night for the Gods Who Control the Creations of Others. Thirty such days make up a month. Twelve such months make up a year. The life span of the Gods Who Control the Creations of Others is sixteen thousand of these divine years. It’s possible that a woman or man who has observed the eight-factored sabbath will—when their body breaks up, after death—be reborn in the company of the Gods Who Control the Creations of Others. This is what I was referring to when I said: ‘Human kingship is a poor thing compared to the happiness of the gods.’ 

\begin{verse}%
You\marginnote{10.1} shouldn’t kill living creatures, or steal, \\
or lie, or drink alcohol. \\
Be celibate, refraining from sex, \\
and don’t eat at night, the wrong time. 

Not\marginnote{11.1} wearing garlands or applying perfumes, \\
you should sleep on a low bed, or a mat on the ground. \\
This is the eight-factored sabbath, they say, \\
explained by the Buddha, who has gone to suffering’s end. 

The\marginnote{12.1} moon and sun are both fair to see, \\
radiating as far as they revolve. \\
Those shining ones in the sky light up the quarters, \\
dispelling the darkness as they traverse the heavens. 

All\marginnote{13.1} of the wealth that’s found in this realm—\\
pearls, gems, fine beryl too, \\
rose-gold or pure gold, \\
or natural gold dug up by marmots—

they’re\marginnote{14.1} not worth a sixteenth part \\
of the sabbath with its eight factors, \\
as starlight cannot rival the moon. 

So\marginnote{15.1} an ethical woman or man, \\
who has observed the eight-factored sabbath, \\
having made merit whose outcome is happiness, \\
blameless, they go to a heavenly place.” 

%
\end{verse}

%
\section*{{\suttatitleacronym AN 8.43}{\suttatitletranslation With Visākhā on the Sabbath }{\suttatitleroot Visākhāsutta}}
\addcontentsline{toc}{section}{\tocacronym{AN 8.43} \toctranslation{With Visākhā on the Sabbath } \tocroot{Visākhāsutta}}
\markboth{With Visākhā on the Sabbath }{Visākhāsutta}
\extramarks{AN 8.43}{AN 8.43}

At\marginnote{1.1} one time the Buddha was staying near \textsanskrit{Sāvatthī} in the Eastern Monastery, the stilt longhouse of \textsanskrit{Migāra}’s mother. Then \textsanskrit{Visākhā}, \textsanskrit{Migāra}’s mother, went up to the Buddha, bowed, and sat down to one side. The Buddha said to her: 

“\textsanskrit{Visākhā},\marginnote{1.3} the observance of the sabbath with its eight factors is very fruitful and beneficial and splendid and bountiful. And how should it be observed? It’s when a noble disciple reflects: ‘As long as they live, the perfected ones give up killing living creatures, renouncing the rod and the sword. They are scrupulous and kind, and live full of compassion for all living beings. I, too, for this day and night will give up killing living creatures, renouncing the rod and the sword. I’ll be scrupulous and kind, and live full of compassion for all living beings. I will observe the sabbath by doing as the perfected ones do in this respect.’ This is its first factor. … 

‘As\marginnote{2.1} long as they live, the perfected ones give up high and luxurious beds. They sleep in a low place, either a small bed or a straw mat. I, too, for this day and night will give up high and luxurious beds. I’ll sleep in a low place, either a small bed or a straw mat. I will observe the sabbath by doing as the perfected ones do in this respect.’ This is its eighth factor. The observance of the sabbath with its eight factors in this way is very fruitful and beneficial and splendid and bountiful. 

How\marginnote{3.1} much so? Suppose you were to rule as sovereign lord over these sixteen great countries—\textsanskrit{Aṅga}, Magadha, \textsanskrit{Kāsī}, Kosala, \textsanskrit{Vajjī}, Malla, Ceti, \textsanskrit{Vaṅga}, Kuru, \textsanskrit{Pañcāla}, Maccha, \textsanskrit{Sūrusena}, Assaka, Avanti, \textsanskrit{Gandhāra}, and Kamboja—full of the seven kinds of precious things. This wouldn’t be worth a sixteenth part of the sabbath with its eight factors. Why is that? Because human kingship is a poor thing compared to the happiness of the gods. 

Fifty\marginnote{4.1} years in the human realm is one day and night for the Gods of the Four Great Kings. Thirty such days make up a month. Twelve such months make up a year. The life span of the Gods of the Four Great Kings is five hundred of these divine years. It’s possible that a woman or man who has observed the eight-factored sabbath will—when their body breaks up, after death—be reborn in the company of the Gods of the Four Great Kings. This is what I was referring to when I said: ‘Human kingship is a poor thing compared to the happiness of the gods.’ 

A\marginnote{5.1} hundred years in the human realm is one day and night for the Gods of the Thirty-Three. Thirty such days make up a month. Twelve such months make up a year. The life span of the Gods of the Thirty-Three is a thousand of these divine years. It’s possible that a woman or man who has observed the eight-factored sabbath will—when their body breaks up, after death—be reborn in the company of the Gods of the Thirty-Three. This is what I was referring to when I said: ‘Human kingship is a poor thing compared to the happiness of the gods.’ 

Two\marginnote{6.1} hundred years in the human realm … 

Four\marginnote{6.2} hundred years in the human realm … 

Eight\marginnote{6.3} hundred years in the human realm … 

Sixteen\marginnote{6.4} hundred years in the human realm is one day and night for the Gods Who Control the Creations of Others. Thirty such days make up a month. Twelve such months make up a year. The life span of the Gods Who Control the Creations of Others is sixteen thousand of these divine years. It’s possible that a woman or man who has observed the eight-factored sabbath will—when their body breaks up, after death—be reborn in the company of the Gods Who Control the Creations of Others. This is what I was referring to when I said: ‘Human kingship is a poor thing compared to the happiness of the gods.’ 

\begin{verse}%
You\marginnote{7.1} shouldn’t kill living creatures, or steal, \\
or lie, or drink alcohol. \\
Be celibate, refraining from sex, \\
and don’t eat at night, the wrong time. 

Not\marginnote{8.1} wearing garlands or applying perfumes, \\
you should sleep on a low bed, or a mat on the ground. \\
This is the eight-factored sabbath, they say, \\
explained by the Buddha, who has gone to suffering’s end. 

The\marginnote{9.1} moon and sun are both fair to see, \\
radiating as far as they revolve. \\
Those shining ones in the sky light up the quarters, \\
dispelling the darkness as they traverse the heavens. 

All\marginnote{10.1} of the wealth that’s found in this realm—\\
pearls, gems, fine beryl too, \\
rose-gold or pure gold, \\
or natural gold dug up by marmots—

they’re\marginnote{11.1} not worth a sixteenth part \\
of the sabbath with its eight factors, \\
as starlight cannot rival the moon. 

So\marginnote{12.1} an ethical woman or man, \\
who has observed the eight-factored sabbath, \\
having made merit whose outcome is happiness, \\
blameless, they go to a heavenly place.” 

%
\end{verse}

%
\section*{{\suttatitleacronym AN 8.44}{\suttatitletranslation With Vāseṭṭha on the Sabbath }{\suttatitleroot Vāseṭṭhasutta}}
\addcontentsline{toc}{section}{\tocacronym{AN 8.44} \toctranslation{With Vāseṭṭha on the Sabbath } \tocroot{Vāseṭṭhasutta}}
\markboth{With Vāseṭṭha on the Sabbath }{Vāseṭṭhasutta}
\extramarks{AN 8.44}{AN 8.44}

At\marginnote{1.1} one time the Buddha was staying near \textsanskrit{Vesālī}, at the Great Wood, in the hall with the peaked roof. Then the layman \textsanskrit{Vāseṭṭha} went up to the Buddha, bowed, and sat down to one side. The Buddha said to him: 

“\textsanskrit{Vāseṭṭha},\marginnote{1.3} the observance of the sabbath with its eight factors is very fruitful and beneficial and splendid and bountiful … blameless, they go to a heavenly place.” 

When\marginnote{2.1} he said this, \textsanskrit{Vāseṭṭha} said to the Buddha: 

“If\marginnote{2.2} my loved ones—relatives and kin—were to observe this sabbath with its eight factors, it would be for their lasting welfare and happiness. If all the aristocrats, brahmins, merchants, and workers were to observe this sabbath with its eight factors, it would be for their lasting welfare and happiness.” 

“That’s\marginnote{3.1} so true, \textsanskrit{Vāseṭṭha}! That’s so true, \textsanskrit{Vāseṭṭha}! If all the aristocrats, brahmins, merchants, and workers were to observe this sabbath with its eight factors, it would be for their lasting welfare and happiness. If the whole world—with its gods, \textsanskrit{Māras} and \textsanskrit{Brahmās}, this population with its ascetics and brahmins, gods and humans—were to observe this sabbath with its eight factors, it would be for their lasting welfare and happiness. If these great sal trees were to observe this sabbath with its eight factors, it would be for their lasting welfare and happiness—if they were sentient. How much more then a human being!” 

%
\section*{{\suttatitleacronym AN 8.45}{\suttatitletranslation With Bojjhā on the Sabbath }{\suttatitleroot Bojjhasutta}}
\addcontentsline{toc}{section}{\tocacronym{AN 8.45} \toctranslation{With Bojjhā on the Sabbath } \tocroot{Bojjhasutta}}
\markboth{With Bojjhā on the Sabbath }{Bojjhasutta}
\extramarks{AN 8.45}{AN 8.45}

At\marginnote{1.1} one time the Buddha was staying near \textsanskrit{Sāvatthī} in Jeta’s Grove, \textsanskrit{Anāthapiṇḍika}’s monastery. Then the laywoman \textsanskrit{Bojjhā} went up to the Buddha, bowed, and sat down to one side. The Buddha said to her: 

“\textsanskrit{Bojjhā},\marginnote{2.1} the observance of the sabbath with its eight factors is very fruitful and beneficial and splendid and bountiful. And how should it be observed? 

It’s\marginnote{2.3} when a noble disciple reflects: ‘As long as they live, the perfected ones give up killing living creatures, renouncing the rod and the sword. They are scrupulous and kind, and live full of compassion for all living beings. I, too, for this day and night will give up killing living creatures, renouncing the rod and the sword. I’ll be scrupulous and kind, and live full of compassion for all living beings. I will observe the sabbath by doing as the perfected ones do in this respect.’ This is its first factor. … 

‘As\marginnote{3.1} long as they live, the perfected ones give up high and luxurious beds. They sleep in a low place, either a small bed or a straw mat. I, too, for this day and night will give up high and luxurious beds. I’ll sleep in a low place, either a small bed or a straw mat. I will observe the sabbath by doing as the perfected ones do in this respect.’ This is its eighth factor. The observance of the sabbath with its eight factors in this way is very fruitful and beneficial and splendid and bountiful. 

How\marginnote{4.1} much so? Suppose you were to rule as sovereign lord over these sixteen great countries—\textsanskrit{Aṅga}, Magadha, \textsanskrit{Kāsī}, Kosala, \textsanskrit{Vajjī}, Malla, Ceti, \textsanskrit{Vaṅga}, Kuru, \textsanskrit{Pañcāla}, Maccha, \textsanskrit{Sūrusena}, Assaka, Avanti, \textsanskrit{Gandhāra}, and Kamboja—full of the seven kinds of precious things. This wouldn’t be worth a sixteenth part of the sabbath with its eight factors. Why is that? Because human kingship is a poor thing compared to the happiness of the gods. 

Fifty\marginnote{5.1} years in the human realm is one day and night for the Gods of the Four Great Kings. Thirty such days make up a month. Twelve such months make up a year. The life span of the Gods of the Four Great Kings is five hundred of these divine years. It’s possible that a woman or man who has observed the eight-factored sabbath will—when their body breaks up, after death—be reborn in the company of the Gods of the Four Great Kings. This is what I was referring to when I said: ‘Human kingship is a poor thing compared to the happiness of the gods.’ 

A\marginnote{6.1} hundred years in the human realm … 

Two\marginnote{6.2} hundred years in the human realm … 

Four\marginnote{6.3} hundred years in the human realm … 

Eight\marginnote{6.4} hundred years in the human realm … 

Sixteen\marginnote{6.5} hundred years in the human realm is one day and night for the Gods Who Control the Creations of Others. Thirty such days make up a month. Twelve such months make up a year. The life span of the gods who control the creations of others is sixteen thousand of these divine years. It’s possible that a woman or man who has observed the eight-factored sabbath will—when their body breaks up, after death—be reborn in the company of the Gods Who Control the Creations of Others. This is what I was referring to when I said: ‘Human kingship is a poor thing compared to the happiness of the gods.’ 

\begin{verse}%
You\marginnote{7.1} shouldn’t kill living creatures, or steal, \\
or lie, or drink alcohol. \\
Be celibate, refraining from sex, \\
and don’t eat at night, the wrong time. 

Not\marginnote{8.1} wearing garlands or applying perfumes, \\
you should sleep on a low bed, or a mat on the ground. \\
This is the eight-factored sabbath, they say, \\
explained by the Buddha, who has gone to suffering’s end. 

The\marginnote{9.1} moon and sun are both fair to see, \\
radiating as far as they revolve. \\
Those shining ones in the sky light up the quarters, \\
dispelling the darkness as they traverse the heavens. 

All\marginnote{10.1} of the wealth that’s found in this realm—\\
pearls, gems, fine beryl too, \\
rose-gold or pure gold, \\
or natural gold dug up by marmots—

they’re\marginnote{11.1} not worth a sixteenth part \\
of the mind developed with love, \\
as starlight cannot rival the moon. 

So\marginnote{12.1} an ethical woman or man, \\
who has observed the eight-factored sabbath, \\
having made merit whose outcome is happiness, \\
blameless, they go to a heavenly place.” 

%
\end{verse}

%
\section*{{\suttatitleacronym AN 8.46}{\suttatitletranslation Anuruddha and the Agreeable Deities }{\suttatitleroot Anuruddhasutta}}
\addcontentsline{toc}{section}{\tocacronym{AN 8.46} \toctranslation{Anuruddha and the Agreeable Deities } \tocroot{Anuruddhasutta}}
\markboth{Anuruddha and the Agreeable Deities }{Anuruddhasutta}
\extramarks{AN 8.46}{AN 8.46}

At\marginnote{1.1} one time the Buddha was staying near Kosambi, in Ghosita’s Monastery. 

Now\marginnote{1.2} at that time Venerable Anuruddha had gone into retreat for the day’s meditation. Then several deities of the Lovable Host went up to Venerable Anuruddha, bowed, stood to one side, and said to him: 

“Sir,\marginnote{1.4} Anuruddha, we are the deities called ‘Loveable’. We wield authority and control over three things. We can turn any color we want on the spot. We can get any voice that we want on the spot. We can get any pleasure that we want on the spot. We are the deities called ‘Loveable’. We wield authority and control over these three things.” 

Then\marginnote{2.1} Venerable Anuruddha thought, “If only these deities would all turn blue, of blue color, clad in blue, adorned with blue!” Then those deities, knowing Anuruddha’s thought, all turned blue. 

Then\marginnote{3.1} Venerable Anuruddha thought, “If only these deities would all turn yellow …” 

“If\marginnote{3.3} only these gods would all turn red …” 

“If\marginnote{3.4} only these gods would all turn white …” Then those deities, knowing Anuruddha’s thought, all turned white. 

Then\marginnote{4.1} one of those deities sang, one danced, and one snapped her fingers. Suppose there was a quintet made up of skilled musicians who had practiced well and kept excellent rhythm. They’d sound graceful, tantalizing, sensuous, lovely, and intoxicating. In the same way the performance by those deities sounded graceful, tantalizing, sensuous, lovely, and intoxicating. But Venerable Anuruddha averted his senses. 

Then\marginnote{5.1} those deities, thinking “Master Anuruddha isn’t enjoying this,” vanished right there. Then in the late afternoon, Anuruddha came out of retreat and went to the Buddha, bowed, sat down to one side, and told him what had happened, adding: 

“How\marginnote{9.1} many qualities do females have so that—when their body breaks up, after death—they are reborn in company with the Gods of the Lovable Host?” 

“Anuruddha,\marginnote{10.1} when they have eight qualities females—when their body breaks up, after death—are reborn in company with the Gods of the Lovable Host. What eight? 

Take\marginnote{10.3} the case of a female whose mother and father give her to a husband wanting what’s best for her, out of kindness and compassion. She would get up before him and go to bed after him, and be obliging, behaving nicely and speaking politely. 

She\marginnote{11.1} honors, respects, esteems, and venerates those her husband respects, such as mother and father, and ascetics and brahmins. And when they arrive she serves them with a seat and water. 

She’s\marginnote{12.1} skilled and tireless in her husband’s household duties, such as knitting and sewing. She understands how to go about things in order to complete and organize the work. 

She\marginnote{13.1} knows what work her husband’s domestic bondservants, employees, and workers have completed, and what they’ve left incomplete. She knows who is sick, and who is fit or unwell. She distributes to each a fair portion of various foods. 

She\marginnote{14.1} ensures that any income her husband earns is guarded and protected, whether money, grain, silver, or gold. She doesn’t overspend, steal, waste, or lose it. 

She’s\marginnote{15.1} a lay follower who has gone for refuge to the Buddha, his teaching, and the \textsanskrit{Saṅgha}. 

She’s\marginnote{16.1} ethical. She doesn’t kill living creatures, steal, commit sexual misconduct, lie, or use alcoholic drinks that cause negligence. 

She’s\marginnote{17.1} generous. She lives at home rid of the stain of stinginess, freely generous, open-handed, loving to let go, committed to charity, loving to give and to share. 

When\marginnote{18.1} they have these eight qualities females—when their body breaks up, after death—are reborn in company with the Gods of the Lovable Host. 

\begin{verse}%
She’d\marginnote{19.1} never look down on her husband, \\
who’s always eager to work hard, \\
always looking after her, \\
and bringing whatever she wants. 

And\marginnote{20.1} a good woman never scolds her husband \\
with jealous words. \\
Being astute, she reveres \\
those respected by her husband. 

She\marginnote{21.1} gets up early, works tirelessly, \\
and manages the domestic help. \\
She’s loveable to her husband, \\
and preserves his wealth. 

A\marginnote{22.1} lady who fulfills these duties \\
according to her husband’s desire, \\
is reborn among the gods \\
called ‘Loveable’.” 

%
\end{verse}

%
\section*{{\suttatitleacronym AN 8.47}{\suttatitletranslation With Visākhā on the Loveable Gods }{\suttatitleroot Dutiyavisākhāsutta}}
\addcontentsline{toc}{section}{\tocacronym{AN 8.47} \toctranslation{With Visākhā on the Loveable Gods } \tocroot{Dutiyavisākhāsutta}}
\markboth{With Visākhā on the Loveable Gods }{Dutiyavisākhāsutta}
\extramarks{AN 8.47}{AN 8.47}

At\marginnote{1.1} one time the Buddha was staying near \textsanskrit{Sāvatthī} in the Eastern Monastery, the stilt longhouse of \textsanskrit{Migāra}’s mother. Then \textsanskrit{Visākhā}, \textsanskrit{Migāra}’s mother, went up to the Buddha, bowed, and sat down to one side. The Buddha said to her: 

“\textsanskrit{Visākhā},\marginnote{2.1} when they have eight qualities females—when their body breaks up, after death—are reborn in company with the Gods of the Lovable Host. What eight? Take the case of a female whose mother and father give her to a husband wanting what’s best for her, out of kindness and compassion. She would get up before him and go to bed after him, and be obliging, behaving nicely and speaking politely. … 

She’s\marginnote{3.1} generous. She lives at home rid of the stain of stinginess, freely generous, open-handed, loving to let go, committed to charity, loving to give and to share. When they have these eight qualities females—when their body breaks up, after death—are reborn in company with the Gods of the Lovable Host. 

\begin{verse}%
She’d\marginnote{4.1} never look down on her husband, \\
who’s always eager to work hard, \\
always looking after her, \\
and bringing whatever she wants. 

And\marginnote{5.1} a good woman never scolds her husband \\
with jealous words. \\
Being astute, she reveres \\
those respected by her husband. 

She\marginnote{6.1} gets up early, works tirelessly, \\
and manages the domestic help. \\
She’s loveable to her husband, \\
and preserves his wealth. 

A\marginnote{7.1} lady who fulfills these duties \\
according to her husband’s desire, \\
is reborn among the gods \\
called ‘Loveable’.” 

%
\end{verse}

%
\section*{{\suttatitleacronym AN 8.48}{\suttatitletranslation With Nakula’s Mother on the Loveable Gods }{\suttatitleroot Nakulamātāsutta}}
\addcontentsline{toc}{section}{\tocacronym{AN 8.48} \toctranslation{With Nakula’s Mother on the Loveable Gods } \tocroot{Nakulamātāsutta}}
\markboth{With Nakula’s Mother on the Loveable Gods }{Nakulamātāsutta}
\extramarks{AN 8.48}{AN 8.48}

At\marginnote{1.1} one time the Buddha was staying in the land of the Bhaggas on Crocodile Hill, in the deer park at \textsanskrit{Bhesakaḷā}’s Wood. Then the housewife Nakula’s mother went up to the Buddha, bowed, and sat down to one side. The Buddha said to her: 

“Nakula’s\marginnote{2.1} mother, when they have eight qualities females—when their body breaks up, after death—are reborn in company with the Gods of the Lovable Host. What eight? 

Take\marginnote{2.3} the case of a female whose mother and father give her to a husband wanting what’s best for her, out of kindness and compassion. She would get up before him and go to bed after him, and be obliging, behaving nicely and speaking politely. 

She\marginnote{3.1} honors, respects, esteems, and venerates those her husband respects, such as mother and father, and ascetics and brahmins. And when they arrive she serves them with a seat and water. 

She’s\marginnote{4.1} skilled and tireless in her husband’s household duties, such as knitting and sewing. She understands how to go about things in order to complete and organize the work. 

She\marginnote{5.1} knows what work her husband’s domestic bondservants, employees, and workers have completed, and what they’ve left incomplete. She knows who is sick, and who is fit or unwell. She distributes to each a fair portion of various foods. 

She\marginnote{6.1} ensures that any income her husband earns is guarded and protected, whether money, grain, silver, or gold. She doesn’t overspend, steal, waste, or lose it. 

She’s\marginnote{7.1} a lay follower who has gone for refuge to the Buddha, his teaching, and the \textsanskrit{Saṅgha}. 

She’s\marginnote{8.1} ethical. She doesn’t kill living creatures, steal, commit sexual misconduct, lie, or use alcoholic drinks that cause negligence. 

She’s\marginnote{9.1} generous. She lives at home rid of the stain of stinginess, freely generous, open-handed, loving to let go, committed to charity, loving to give and to share. 

When\marginnote{10.1} they have these eight qualities females—when their body breaks up, after death—are reborn in company with the Gods of the Lovable Host. 

\begin{verse}%
She’d\marginnote{11.1} never look down on her husband, \\
who’s always eager to work hard, \\
always looking after her, \\
and bringing whatever she wants. 

And\marginnote{12.1} a good woman never scolds her husband \\
with jealous words. \\
Being astute, she reveres \\
those respected by her husband. 

She\marginnote{13.1} gets up early, works tirelessly, \\
and manages the domestic help. \\
She’s loveable to her husband, \\
and preserves his wealth. 

A\marginnote{14.1} lady who fulfills these duties \\
according to her husband’s desire, \\
is reborn among the gods \\
called ‘Loveable’.” 

%
\end{verse}

%
\section*{{\suttatitleacronym AN 8.49}{\suttatitletranslation Winning in This Life (1st) }{\suttatitleroot Paṭhamaidhalokikasutta}}
\addcontentsline{toc}{section}{\tocacronym{AN 8.49} \toctranslation{Winning in This Life (1st) } \tocroot{Paṭhamaidhalokikasutta}}
\markboth{Winning in This Life (1st) }{Paṭhamaidhalokikasutta}
\extramarks{AN 8.49}{AN 8.49}

At\marginnote{1.1} one time the Buddha was staying near \textsanskrit{Sāvatthī} in the Eastern Monastery, the stilt longhouse of \textsanskrit{Migāra}’s mother. Then \textsanskrit{Visākhā}, \textsanskrit{Migāra}’s mother, went up to the Buddha, bowed, and sat down to one side. The Buddha said to her: 

“\textsanskrit{Visākhā},\marginnote{2.1} a female who has four qualities is practicing to win in this life, and she succeeds at it. What four? It’s when a female is well-organized at work, manages the domestic help, acts lovingly toward her husband, and preserves his earnings. 

And\marginnote{3.1} how is a female well-organized at work? It’s when she’s skilled and tireless in doing domestic duties for her husband, such as knitting and sewing. She understands how to go about things in order to complete and organize the work. That’s how a female is well-organized at work. 

And\marginnote{4.1} how does a female manage the domestic help? It’s when she knows what work her husband’s domestic bondservants, employees, and workers have completed, and what they’ve left incomplete. She knows who is sick, and who is fit or unwell. She distributes to each a fair portion of various foods. That’s how a female manages the domestic help. 

And\marginnote{5.1} how does a female act lovingly toward her husband? It’s when a female would not transgress in any way that her husband would not consider loveable, even for the sake of her own life. That’s how a female acts lovingly toward her husband. 

And\marginnote{6.1} how does a female preserve his earnings? It’s when she ensures that any income her husband earns is guarded and protected, whether money, grain, silver, or gold. She doesn’t overspend, steal, waste, or lose it. That’s how a female preserves his earnings. 

A\marginnote{6.4} female who has these four qualities is practicing to win in this life, and she succeeds at it. 

A\marginnote{7.1} female who has four qualities is practicing to win in the next life, and she succeeds at it. What four? It’s when a female is accomplished in faith, ethics, generosity, and wisdom. 

And\marginnote{8.1} how is a female accomplished in faith? It’s when a female has faith in the Realized One’s awakening: ‘That Blessed One is perfected, a fully awakened Buddha, accomplished in knowledge and conduct, holy, knower of the world, supreme guide for those who wish to train, teacher of gods and humans, awakened, blessed.’ That’s how a female is accomplished in faith. 

And\marginnote{9.1} how is a female accomplished in ethics? It’s when a female doesn’t kill living creatures, steal, commit sexual misconduct, lie, or consume alcoholic drinks that cause negligence. That’s how a female is accomplished in ethics. 

And\marginnote{10.1} how is a female accomplished in generosity? It’s when she lives at home rid of the stain of stinginess, freely generous, open-handed, loving to let go, committed to charity, loving to give and to share. That’s how a female is accomplished in generosity. 

And\marginnote{11.1} how is a female accomplished in wisdom? It’s when a female is wise. She has the wisdom of arising and passing away which is noble, penetrative, and leads to the complete ending of suffering. That’s how a female is accomplished in wisdom. 

A\marginnote{12.1} female who has these four qualities is practicing to win in the next life, and she succeeds at it. 

\begin{verse}%
She’s\marginnote{13.1} organized at work, \\
and manages the domestic help. \\
She’s loveable to her husband, \\
and preserves his wealth. 

Faithful,\marginnote{14.1} accomplished in ethics, \\
bountiful, rid of stinginess, \\
she always purifies the path \\
to well-being in lives to come. 

And\marginnote{15.1} so, a lady in whom \\
these eight qualities are found \\
is known as virtuous, \\
firm in principle, and truthful. 

Accomplished\marginnote{16.1} in sixteen respects, \\
complete with the eight factors, \\
a virtuous laywoman such as she \\
is reborn in the realm of the Loveable Gods.” 

%
\end{verse}

%
\section*{{\suttatitleacronym AN 8.50}{\suttatitletranslation Winning in This Life (2nd) }{\suttatitleroot Dutiyaidhalokikasutta}}
\addcontentsline{toc}{section}{\tocacronym{AN 8.50} \toctranslation{Winning in This Life (2nd) } \tocroot{Dutiyaidhalokikasutta}}
\markboth{Winning in This Life (2nd) }{Dutiyaidhalokikasutta}
\extramarks{AN 8.50}{AN 8.50}

“Mendicants,\marginnote{1.1} a female who has four qualities is practicing to win in this life, and she succeeds at it. What four? 

It’s\marginnote{1.3} when a female is well-organized at work, manages the domestic help, acts lovingly toward her husband, and preserves his earnings. 

And\marginnote{2.1} how is a female well-organized at work? It’s when she’s skilled and tireless in doing domestic duties for her husband … That’s how a female is well-organized at work. 

And\marginnote{3.1} how does a female manage the domestic help? It’s when she knows what work her husband’s domestic bondservants, employees, and workers have completed, and what they’ve left incomplete. She knows who is sick, and who is fit or unwell. She distributes to each a fair portion of various foods. That’s how a female manages the domestic help. 

And\marginnote{4.1} how does a female act lovingly toward her husband? It’s when a female would not transgress in any way that her husband would not consider loveable, even for the sake of her own life. That’s how a female acts lovingly toward her husband. 

And\marginnote{5.1} how does a female preserve his earnings? It’s when she tries to guard and protect any income her husband earns … That’s how a female preserves his earnings. 

A\marginnote{5.4} female who has these four qualities is practicing to win in this life, and she succeeds at it. 

A\marginnote{6.1} female who has four qualities is practicing to win in the next life, and she succeeds at it. What four? It’s when a female is accomplished in faith, ethics, generosity, and wisdom. 

And\marginnote{7.1} how is a female accomplished in faith? It’s when a female has faith in the Realized One’s awakening … That’s how a female is accomplished in faith. 

And\marginnote{8.1} how is a female accomplished in ethics? It’s when a female doesn’t kill living creatures, steal, commit sexual misconduct, lie, or consume alcoholic drinks that cause negligence. That’s how a female is accomplished in ethics. 

And\marginnote{9.1} how is a female accomplished in generosity? It’s when she lives at home rid of the stain of stinginess, freely generous, open-handed, loving to let go, committed to charity, loving to give and to share. That’s how a female is accomplished in generosity. 

And\marginnote{10.1} how is a female accomplished in wisdom? It’s when a female is wise. She has the wisdom of arising and passing away which is noble, penetrative, and leads to the complete ending of suffering. That’s how a female is accomplished in wisdom. 

A\marginnote{10.4} female who has these four qualities is practicing to win in the next life, and she succeeds at it. 

\begin{verse}%
She’s\marginnote{11.1} organized at work, \\
and manages the domestic help. \\
She’s loveable to her husband, \\
and preserves his wealth. 

Faithful,\marginnote{12.1} accomplished in ethics, \\
being bountiful and rid of stinginess. \\
She always purifies the path \\
to well-being in lives to come. 

And\marginnote{13.1} so, a lady in whom \\
these eight qualities are found \\
is known as virtuous, \\
firm in principle, and truthful. 

Accomplished\marginnote{14.1} in sixteen respects, \\
complete with the eight factors, \\
a virtuous laywoman such as she \\
is reborn in the realm of the Loveable Gods.” 

%
\end{verse}

%
\addtocontents{toc}{\let\protect\contentsline\protect\nopagecontentsline}
\pannasa{The Second Fifty }
\addcontentsline{toc}{pannasa}{The Second Fifty }
\markboth{}{}
\addtocontents{toc}{\let\protect\contentsline\protect\oldcontentsline}

%
\addtocontents{toc}{\let\protect\contentsline\protect\nopagecontentsline}
\chapter*{The Chapter on Gotamī }
\addcontentsline{toc}{chapter}{\tocchapterline{The Chapter on Gotamī }}
\addtocontents{toc}{\let\protect\contentsline\protect\oldcontentsline}

%
\section*{{\suttatitleacronym AN 8.51}{\suttatitletranslation With Gotamī }{\suttatitleroot Gotamīsutta}}
\addcontentsline{toc}{section}{\tocacronym{AN 8.51} \toctranslation{With Gotamī } \tocroot{Gotamīsutta}}
\markboth{With Gotamī }{Gotamīsutta}
\extramarks{AN 8.51}{AN 8.51}

At\marginnote{1.1} one time the Buddha was staying in the land of the Sakyans, near Kapilavatthu in the Banyan Tree Monastery. Then \textsanskrit{Mahāpajāpatī} \textsanskrit{Gotamī} went up to the Buddha, bowed, stood to one side, and said to him: 

“Sir,\marginnote{1.3} please let females gain the going forth from the lay life to homelessness in the teaching and training proclaimed by the Realized One.” 

“Enough,\marginnote{1.4} \textsanskrit{Gotamī}. Don’t advocate for females to gain the going forth from the lay life to homelessness in the teaching and training proclaimed by the Realized One.” 

For\marginnote{2.1} a second time … 

For\marginnote{3.1} a third time, \textsanskrit{Mahāpajāpatī} \textsanskrit{Gotamī} said to the Buddha: 

“Sir,\marginnote{3.2} please let females gain the going forth from the lay life to homelessness in the teaching and training proclaimed by the Realized One.” 

“Enough,\marginnote{3.3} \textsanskrit{Gotamī}. Don’t advocate for females to gain the going forth from the lay life to homelessness in the teaching and training proclaimed by the Realized One.” 

Then\marginnote{4.1} \textsanskrit{Mahāpajāpatī} \textsanskrit{Gotamī} thought, “The Buddha does not permit females to go forth.” Miserable and sad, weeping, with a tearful face, she bowed, and respectfully circled the Buddha, keeping him on her right, before leaving. 

When\marginnote{5.1} the Buddha had stayed in Kapilavatthu as long as he wished, he set out for \textsanskrit{Vesālī}. Traveling stage by stage, he arrived at \textsanskrit{Vesālī}, where he stayed at the Great Wood, in the hall with the peaked roof. Then \textsanskrit{Mahāpajāpatī} \textsanskrit{Gotamī} had her hair shaved, and dressed in ocher robes. Together with several Sakyan ladies she set out for \textsanskrit{Vesālī}. Traveling stage by stage, she arrived at \textsanskrit{Vesālī} and went to the Great Wood, the hall with the peaked roof. Then \textsanskrit{Mahāpajāpatī} \textsanskrit{Gotamī} stood crying outside the gate, her feet swollen, her limbs covered with dust, miserable and sad, with tearful face. 

Venerable\marginnote{6.1} Ānanda saw her standing there, and said to her, “\textsanskrit{Gotamī}, why do you stand crying outside the gate, your feet swollen, your limbs covered with dust, miserable and sad, with tearful face?” 

“Sir,\marginnote{6.4} Ānanda, it’s because the Buddha does not permit females to go forth in the teaching and training proclaimed by the Realized One.” 

“Well\marginnote{6.5} then, \textsanskrit{Gotamī}, wait here just a moment, while I ask the Buddha to grant the going forth for females.” 

Then\marginnote{7.1} Venerable Ānanda went up to the Buddha, bowed, sat down to one side, and said to him: 

“Sir,\marginnote{7.2} \textsanskrit{Mahāpajāpatī} \textsanskrit{Gotamī} is standing crying outside the gate, her feet swollen, her limbs covered with dust, miserable and sad, with tearful face. She says that it’s because the Buddha does not permit females to go forth. Sir, please let females gain the going forth from the lay life to homelessness in the teaching and training proclaimed by the Realized One.” 

“Enough,\marginnote{7.5} Ānanda. Don’t advocate for females to gain the going forth from the lay life to homelessness in the teaching and training proclaimed by the Realized One.” 

For\marginnote{8.1} a second time … 

For\marginnote{8.2} a third time, Ānanda said to the Buddha: 

“Sir,\marginnote{8.3} please let females gain the going forth from the lay life to homelessness in the teaching and training proclaimed by the Realized One.” 

“Enough,\marginnote{8.4} Ānanda. Don’t advocate for females to gain the going forth from the lay life to homelessness in the teaching and training proclaimed by the Realized One.” 

Then\marginnote{9.1} Venerable Ānanda thought, “The Buddha does not permit females to go forth. Why don’t I try another approach?” 

Then\marginnote{9.4} Venerable Ānanda said to the Buddha, “Sir, is a female able to realize the fruits of stream-entry, once-return, non-return, and perfection once she has gone forth?” 

“She\marginnote{9.6} is able, Ānanda.” 

“If\marginnote{9.7} a female is able to realize the fruits of stream-entry, once-return, non-return, and perfection once she has gone forth. Sir, \textsanskrit{Mahāpajāpatī} has been very helpful to the Buddha. She is his aunt who raised him, nurtured him, and gave him her milk. When the Buddha’s birth mother passed away, she nurtured him at her own breast. Sir, please let females gain the going forth from the lay life to homelessness in the teaching and training proclaimed by the Realized One.” 

“Ānanda,\marginnote{10.1} if \textsanskrit{Mahāpajāpatī} \textsanskrit{Gotamī} accepts these eight principles of respect, that will be her ordination. 

A\marginnote{11.1} nun, even if she has been ordained for a hundred years, should bow down to a monk who was ordained that very day. She should rise up for him, greet him with joined palms, and observe proper etiquette toward him. This principle should be honored, respected, esteemed, and venerated, and not transgressed so long as life lasts. 

A\marginnote{12.1} nun should not commence the rainy season residence in a monastery without monks. This principle should be honored, respected, esteemed, and venerated, and not transgressed so long as life lasts. 

Each\marginnote{13.1} fortnight the nuns should expect two things from the community of monks: the date of the sabbath, and visiting for advice. This principle should be honored, respected, esteemed, and venerated, and not transgressed so long as life lasts. 

After\marginnote{14.1} completing the rainy season residence the nuns should invite admonition from the communities of both monks and nuns in regard to anything that was seen, heard, or suspected. This principle should be honored, respected, esteemed, and venerated, and not transgressed so long as life lasts. 

A\marginnote{15.1} nun who has committed a grave offense should undergo penance in the communities of both monks and nuns for a fortnight. This principle should be honored, respected, esteemed, and venerated, and not transgressed so long as life lasts. 

A\marginnote{16.1} trainee nun who has trained in the six rules for two years should seek ordination from the communities of both monks and nuns. This principle should be honored, respected, esteemed, and venerated, and not transgressed so long as life lasts. 

A\marginnote{17.1} nun should not abuse or insult a monk in any way. This principle should be honored, respected, esteemed, and venerated, and not transgressed so long as life lasts. 

From\marginnote{18.1} this day forth it is forbidden for nuns to criticize monks, but it is not forbidden for monks to criticize nuns. This principle should be honored, respected, esteemed, and venerated, and not transgressed so long as life lasts. 

If\marginnote{19.1} \textsanskrit{Mahāpajāpatī} \textsanskrit{Gotamī} accepts these eight principles of respect, that will be her ordination.” 

Then\marginnote{20.1} Ānanda, having learned these eight principles of respect from the Buddha himself, went to \textsanskrit{Mahāpajāpatī} \textsanskrit{Gotamī} and said: 

“\textsanskrit{Gotamī},\marginnote{21.1} if you accept eight principles of respect, that will be your ordination. 

A\marginnote{22.1} nun, even if she has been ordained for a hundred years, should bow down to a monk who was ordained that very day. She should rise up for him, greet him with joined palms, and observe proper etiquette toward him. This principle should be honored, respected, esteemed, and venerated, and not transgressed so long as life lasts. … 

From\marginnote{23.1} this day forth it is forbidden for nuns to criticize monks, but it is not forbidden for monks to criticize nuns. This principle should be honored, respected, esteemed, and venerated, and not transgressed so long as life lasts. If you accept these eight principles of respect, that will be your ordination.” 

“Ānanda,\marginnote{24.1} suppose there was a woman or man who was young, youthful, and fond of adornments, and had bathed their head. After getting a garland of lotuses, jasmine, or liana flowers, they would take them in both hands and place them on the crown of the head. In the same way, sir, I accept these eight principles of respect as not to be transgressed so long as life lasts.” 

Then\marginnote{25.1} Venerable Ānanda went up to the Buddha, bowed, sat down to one side, and said to the Buddha: 

“Sir,\marginnote{25.2} \textsanskrit{Mahāpajāpatī} \textsanskrit{Gotamī} has accepted the eight principles of respect as not to be transgressed so long as life lasts.” 

“Ānanda,\marginnote{26.1} if females had not gained the going forth from the lay life to homelessness in the teaching and training proclaimed by the Realized One, the spiritual life would have lasted long. The true teaching would have remained for a thousand years. But since they have gained the going forth, now the spiritual life will not last long. The true teaching will remain only five hundred years. 

It’s\marginnote{27.1} like those families with many women and few men. They’re easy prey for bandits and thieves. In the same way, the spiritual life does not last long in a teaching and training where females gain the going forth. 

It’s\marginnote{28.1} like a field full of rice. Once the disease called ‘white bones’ attacks, it doesn’t last long. In the same way, the spiritual life does not last long in a teaching and training where females gain the going forth. 

It’s\marginnote{29.1} like a field full of sugar cane. Once the disease called ‘red rot’ attacks, it doesn’t last long. In the same way, the spiritual life does not last long in a teaching and training where females gain the going forth. 

As\marginnote{30.1} a man might build a dyke around a large lake as a precaution against the water overflowing, in the same way as a precaution I’ve prescribed the eight principles of respect as not to be transgressed so long as life lasts.” 

%
\section*{{\suttatitleacronym AN 8.52}{\suttatitletranslation An Adviser for Nuns }{\suttatitleroot Ovādasutta}}
\addcontentsline{toc}{section}{\tocacronym{AN 8.52} \toctranslation{An Adviser for Nuns } \tocroot{Ovādasutta}}
\markboth{An Adviser for Nuns }{Ovādasutta}
\extramarks{AN 8.52}{AN 8.52}

At\marginnote{1.1} one time the Buddha was staying near \textsanskrit{Vesālī}, at the Great Wood, in the hall with the peaked roof. Then Venerable Ānanda went up to the Buddha, bowed, sat down to one side, and said to the Buddha: 

“Sir,\marginnote{1.3} how many qualities should a monk have to be agreed on as an adviser for nuns?” 

“Ānanda,\marginnote{2.1} a monk with eight qualities may be agreed on as an adviser for nuns. What eight? 

Firstly,\marginnote{2.3} a monk is ethical, restrained in the code of conduct, conducting themselves well and seeking alms in suitable places. Seeing danger in the slightest fault, they keep the rules they’ve undertaken. 

They’re\marginnote{2.4} learned, remembering and keeping what they’ve learned. These teachings are good in the beginning, good in the middle, and good in the end, meaningful and well-phrased, describing a spiritual practice that’s totally full and pure. They are very learned in such teachings, remembering them, reciting them, mentally scrutinizing them, and comprehending them theoretically. 

Both\marginnote{2.5} monastic codes have been passed down to them in detail, well analyzed, well mastered, well judged in both the rules and accompanying material. 

They’re\marginnote{2.6} a good speaker. Their voice is polished, clear, articulate, and expresses the meaning. 

They’re\marginnote{2.7} able to educate, encourage, fire up, and inspire the community of nuns. 

They’re\marginnote{2.8} likable and agreeable to most of the nuns. 

They\marginnote{2.9} have never previously sexually harassed any woman wearing the ocher robe who has gone forth in the Buddha’s name. 

They\marginnote{2.10} have been ordained for twenty years or more. 

A\marginnote{2.11} monk with these eight qualities may be agreed on as an adviser for nuns.” 

%
\section*{{\suttatitleacronym AN 8.53}{\suttatitletranslation Brief Advice to Gotamī }{\suttatitleroot Saṁkhittasutta}}
\addcontentsline{toc}{section}{\tocacronym{AN 8.53} \toctranslation{Brief Advice to Gotamī } \tocroot{Saṁkhittasutta}}
\markboth{Brief Advice to Gotamī }{Saṁkhittasutta}
\extramarks{AN 8.53}{AN 8.53}

At\marginnote{1.1} one time the Buddha was staying near \textsanskrit{Vesālī}, at the Great Wood, in the hall with the peaked roof. Then \textsanskrit{Mahāpajāpatī} \textsanskrit{Gotamī} went up to the Buddha, bowed, stood to one side, and said to him: 

“Sir,\marginnote{2.1} may the Buddha please teach me Dhamma in brief. When I’ve heard it, I’ll live alone, withdrawn, diligent, keen, and resolute.” 

“\textsanskrit{Gotamī},\marginnote{2.2} you might know that certain things lead to passion, not dispassion; to being fettered, not to being unfettered; to accumulation, not dispersal; to more desires, not fewer; to lack of contentment, not contentment; to crowding, not seclusion; to laziness, not energy; to being burdensome, not being unburdensome. You should definitely bear in mind that these things are not the teaching, not the training, and not the Teacher’s instructions. 

You\marginnote{3.1} might know that certain things lead to dispassion, not passion; to being unfettered, not to being fettered; to dispersal, not accumulation; to fewer desires, not more; to contentment, not lack of contentment; to seclusion, not crowding; to energy, not laziness; to being unburdensome, not being burdensome. You should definitely bear in mind that these things are the teaching, the training, and the Teacher’s instructions.” 

%
\section*{{\suttatitleacronym AN 8.54}{\suttatitletranslation With Dīghajāṇu }{\suttatitleroot Dīghajāṇusutta}}
\addcontentsline{toc}{section}{\tocacronym{AN 8.54} \toctranslation{With Dīghajāṇu } \tocroot{Dīghajāṇusutta}}
\markboth{With Dīghajāṇu }{Dīghajāṇusutta}
\extramarks{AN 8.54}{AN 8.54}

At\marginnote{1.1} one time the Buddha was staying in the land of the Koliyans, where they have a town named Kakkarapatta. Then \textsanskrit{Dīghajāṇu} the Koliyan went up to the Buddha, bowed, sat down to one side, and said to the Buddha: 

“Sir,\marginnote{1.3} we are laypeople who enjoy sensual pleasures and living at home with our children. We use sandalwood imported from \textsanskrit{Kāsi}, we wear garlands, perfumes, and makeup, and we accept gold and money. May the Buddha please teach us the Dhamma in a way that leads to our welfare and happiness in this life and in future lives.” 

“Byagghapajja,\marginnote{2.1} these four things lead to the welfare and happiness of a gentleman in this life. What four? 

Accomplishment\marginnote{2.3} in initiative, protection, good friendship, and balanced finances. And what is accomplishment in initiative? It’s when a gentleman earns a living by means such as farming, trade, raising cattle, archery, government service, or one of the professions. He understands how to go about these things in order to complete and organize the work. This is called accomplishment in initiative. 

And\marginnote{3.1} what is accomplishment in protection? It’s when a gentleman owns legitimate wealth that he has earned by his own efforts and initiative, built up with his own hands, gathered by the sweat of the brow. He ensures it is guarded and protected, thinking: ‘How can I prevent my wealth from being taken by rulers or bandits, consumed by fire, swept away by flood, or taken by unloved heirs?’ This is called accomplishment in protection. 

And\marginnote{4.1} what is accomplishment in good friendship? It’s when a gentleman resides in a town or village. And in that place there are householders or their children who may be young or old, but are mature in conduct, accomplished in faith, ethics, generosity, and wisdom. He associates with them, converses and engages in discussion. And he emulates the same kind of accomplishment in faith, ethics, generosity, and wisdom. This is called accomplishment in good friendship. 

And\marginnote{5.1} what is accomplishment in balanced finances? It’s when a gentleman, knowing his income and expenditure, balances his finances, being neither too extravagant nor too frugal. He thinks, ‘In this way my income will exceed my expenditure, not the reverse.’ It’s like an appraiser or their apprentice who, holding up the scales, knows that it’s low by this much or high by this much. In the same way, a gentleman, knowing his income and expenditure, balances his finances, being neither too extravagant nor too frugal. He thinks, ‘In this way my income will exceed my expenditure, not the reverse.’ If a gentleman has little income but an opulent life, people will say: ‘This gentleman eats their wealth like a fig-eater!’ If a gentleman has a large income but a spartan life, people will say: ‘This gentleman is starving themselves to death!’ But a gentleman, knowing his income and expenditure, leads a balanced life, neither too extravagant nor too frugal, thinking, ‘In this way my income will exceed my expenditure, not the reverse.’ This is called accomplishment in balanced finances. 

There\marginnote{6.1} are four drains on wealth that has been gathered in this way. Womanizing, drinking, gambling, and having bad friends, companions, and associates. Suppose there was a large reservoir with four inlets and four drains. And someone was to open up the drains and close off the inlets, and the heavens don’t provide enough rain. You’d expect that large reservoir to dwindle, not expand. In the same way, there are four drains on wealth that has been gathered in this way. Womanizing, drinking, gambling, and having bad friends, companions, and associates. 

There\marginnote{7.1} are four inlets for wealth that has been gathered in this way. Not womanizing, drinking, or gambling, and having good friends, companions, and associates. Suppose there was a large reservoir with four inlets and four drains. And someone was to open up the inlets and close off the drains, and the heavens provide plenty of rain. You’d expect that large reservoir to expand, not dwindle. In the same way, there are four inlets for wealth that has been gathered in this way. Not womanizing, drinking, or gambling, and having good friends, companions, and associates. 

These\marginnote{8.1} are the four things that lead to the welfare and happiness of a gentleman in this life. 

These\marginnote{9.1} four things lead to the welfare and happiness of a gentleman in future lives. What four? Accomplishment in faith, ethics, generosity, and wisdom. 

And\marginnote{10.1} what is accomplishment in faith? It’s when a gentleman has faith in the Realized One’s awakening: ‘That Blessed One is perfected, a fully awakened Buddha, accomplished in knowledge and conduct, holy, knower of the world, supreme guide for those who wish to train, teacher of gods and humans, awakened, blessed.’ This is called accomplishment in faith. 

And\marginnote{11.1} what is accomplishment in ethics? It’s when a gentleman doesn’t kill living creatures, steal, commit sexual misconduct, lie, or consume alcoholic drinks that cause negligence. This is called accomplishment in ethics. 

And\marginnote{12.1} what is accomplishment in generosity? It’s when a gentleman lives at home rid of the stain of stinginess, freely generous, open-handed, loving to let go, committed to charity, loving to give and to share. This is called accomplishment in generosity. 

And\marginnote{13.1} what is accomplishment in wisdom? It’s when a gentleman is wise. He has the wisdom of arising and passing away which is noble, penetrative, and leads to the complete ending of suffering. This is called accomplishment in wisdom. 

These\marginnote{14.1} are the four things that lead to the welfare and happiness of a gentleman in future lives. 

\begin{verse}%
They’re\marginnote{15.1} enterprising in the workplace, \\
diligent in managing things, \\
they balance their finances, \\
and preserve their wealth. 

Faithful,\marginnote{16.1} accomplished in ethics, \\
bountiful, rid of stinginess, \\
they always purify the path \\
to well-being in lives to come. 

And\marginnote{17.1} so these eight qualities \\
of a faithful householder \\
are declared by the one who is truly named \\
to lead to happiness in both spheres, 

welfare\marginnote{18.1} and benefit in this life, \\
and happiness in the future lives. \\
This is how, for a householder, \\
merit grows by generosity.” 

%
\end{verse}

%
\section*{{\suttatitleacronym AN 8.55}{\suttatitletranslation With Ujjaya }{\suttatitleroot Ujjayasutta}}
\addcontentsline{toc}{section}{\tocacronym{AN 8.55} \toctranslation{With Ujjaya } \tocroot{Ujjayasutta}}
\markboth{With Ujjaya }{Ujjayasutta}
\extramarks{AN 8.55}{AN 8.55}

Then\marginnote{1.1} Ujjaya the brahmin went up to the Buddha, and exchanged greetings with him. When the greetings and polite conversation were over, he sat down to one side and said to the Buddha: 

“Master\marginnote{1.3} Gotama, we wish to travel abroad. May the Buddha please teach us the Dhamma in a way that leads to our welfare and happiness in this life and in future lives.” 

“Brahmin,\marginnote{2.1} these four things lead to the welfare and happiness of a gentleman in this life. What four? Accomplishment in initiative, protection, good friendship, and balanced finances. 

And\marginnote{3.1} what is accomplishment in initiative? A gentleman may earn a living by means such as farming, trade, raising cattle, archery, government service, or one of the professions. He understands how to go about these things in order to complete and organize the work. This is called accomplishment in initiative. 

And\marginnote{4.1} what is accomplishment in protection? It’s when a gentleman owns legitimate wealth that he has earned by his own efforts and initiative, built up with his own hands, gathered by the sweat of the brow. He ensures it is guarded and protected, thinking: ‘How can I prevent my wealth from being taken by rulers or bandits, consumed by fire, swept away by flood, or taken by unloved heirs?’ This is called accomplishment in protection. 

And\marginnote{5.1} what is accomplishment in good friendship? It’s when a gentleman resides in a town or village. And in that place there are householders or their children who may be young or old, but are mature in conduct, accomplished in faith, ethics, generosity, and wisdom. He associates with them, converses and engages in discussion. And he emulates the same kind of accomplishment in faith, ethics, generosity, and wisdom. This is called accomplishment in good friendship. 

And\marginnote{6.1} what is accomplishment in balanced finances? It’s when a gentleman, knowing his income and expenditure, balances his finances, being neither too extravagant nor too frugal. He thinks, ‘In this way my income will exceed my expenditure, not the reverse.’ It’s like an appraiser or their apprentice who, holding up the scales, knows that it’s low by this much or high by this much. In the same way, a gentleman, knowing his income and expenditure, balances his finances, being neither too extravagant nor too frugal. He thinks, ‘In this way my income will exceed my expenditure, not the reverse.’ If a gentleman has little income but an opulent life, people will say: ‘This gentleman eats their wealth like a fig-eater!’ If a gentleman has a large income but a spartan life, people will say: ‘This gentleman is starving themselves to death!’ But a gentleman, knowing his income and expenditure, leads a balanced life, neither too extravagant nor too frugal, thinking, ‘In this way my income will exceed my expenditure, not the reverse.’ This is called accomplishment in balanced finances. 

There\marginnote{7.1} are four drains on wealth that has been gathered in this way. Womanizing, drinking, gambling, and having bad friends, companions, and associates. Suppose there was a large reservoir with four inlets and four drains. And someone was to open up the drains and close off the inlets, and the heavens don’t provide enough rain. You’d expect that large reservoir to dwindle, not expand. In the same way, there are four drains on wealth that has been gathered in this way. Womanizing, drinking, gambling, and having bad friends, companions, and associates. 

There\marginnote{8.1} are four inlets for wealth that has been gathered in this way. Not womanizing, drinking, or gambling, and having good friends, companions, and associates. Suppose there was a large reservoir with four inlets and four drains. And someone was to open up the inlets and close off the drains, and the heavens provide plenty of rain. You’d expect that large reservoir to expand, not dwindle. In the same way, there are four inlets for wealth that has been gathered in this way. Not womanizing, drinking, or gambling, and having good friends, companions, and associates. 

These\marginnote{9.1} are the four things that lead to the welfare and happiness of a gentleman in this life. 

These\marginnote{10.1} four things lead to the welfare and happiness of a gentleman in future lives. What four? Accomplishment in faith, ethics, generosity, and wisdom. 

And\marginnote{10.4} what is accomplishment in faith? It’s when a gentleman has faith in the Realized One’s awakening: ‘That Blessed One is perfected, a fully awakened Buddha, accomplished in knowledge and conduct, holy, knower of the world, supreme guide for those who wish to train, teacher of gods and humans, awakened, blessed.’ This is called accomplishment in faith. 

And\marginnote{11.1} what is accomplishment in ethics? It’s when a gentleman doesn’t kill living creatures, steal, commit sexual misconduct, lie, or consume alcoholic drinks that cause negligence. This is called accomplishment in ethics. 

And\marginnote{12.1} what is accomplishment in generosity? It’s when a gentleman lives at home rid of the stain of stinginess, freely generous, open-handed, loving to let go, committed to charity, loving to give and to share. This is called accomplishment in generosity. 

And\marginnote{13.1} what is accomplishment in wisdom? It’s when a gentleman is wise. He has the wisdom of arising and passing away which is noble, penetrative, and leads to the complete ending of suffering. This is called accomplishment in wisdom. 

These\marginnote{14.1} are the four things that lead to the welfare and happiness of a gentleman in future lives. 

\begin{verse}%
They’re\marginnote{15.1} enterprising in the workplace, \\
diligent in managing things, \\
they balance their finances, \\
and preserve their wealth. 

Faithful,\marginnote{16.1} accomplished in ethics, \\
bountiful, rid of stinginess, \\
they always purify the path \\
to well-being in lives to come. 

And\marginnote{17.1} so these eight qualities \\
of a faithful householder \\
are declared by the one who is truly named \\
to lead to happiness in both spheres, 

welfare\marginnote{18.1} and benefit in this life, \\
and happiness in the next. \\
This is how, for a householder, \\
merit grows by generosity.” 

%
\end{verse}

%
\section*{{\suttatitleacronym AN 8.56}{\suttatitletranslation Danger }{\suttatitleroot Bhayasutta}}
\addcontentsline{toc}{section}{\tocacronym{AN 8.56} \toctranslation{Danger } \tocroot{Bhayasutta}}
\markboth{Danger }{Bhayasutta}
\extramarks{AN 8.56}{AN 8.56}

“Mendicants,\marginnote{1.1} ‘danger’ is a term for sensual pleasures. ‘Suffering’, ‘disease’, ‘boil’, ‘dart’, ‘snare’, ‘bog’, and ‘womb’ are terms for sensual pleasures. And why is ‘danger’ a term for sensual pleasures? Someone who is besotted by sensual greed and shackled by lustful desire is not freed from dangers in the present life or in lives to come. That is why ‘danger’ is a term for sensual pleasures. And why are ‘suffering’, ‘disease’, ‘boil’, ‘dart’, ‘snare’, ‘bog’, and ‘womb’ terms for sensual pleasures? Someone who is besotted by sensual greed and shackled by lustful desire is not freed from wombs in the present life or in lives to come. That is why ‘womb’ is a term for sensual pleasures. 

\begin{verse}%
Danger,\marginnote{2.1} suffering, and disease, \\
boil, dart, and snare, \\
and bogs and wombs both. \\
These describe the sensual pleasures \\
to which ordinary people are attached. 

Swamped\marginnote{3.1} by things that seem pleasant, \\
you go to another womb. \\
But when a mendicant is keen, \\
and doesn’t forget awareness, 

in\marginnote{4.1} this way they transcend \\
this grueling swamp. \\
They watch this population flounder, \\
fallen into rebirth and old age.” 

%
\end{verse}

%
\section*{{\suttatitleacronym AN 8.57}{\suttatitletranslation Worthy of Offerings Dedicated to the Gods (1st) }{\suttatitleroot Paṭhamaāhuneyyasutta}}
\addcontentsline{toc}{section}{\tocacronym{AN 8.57} \toctranslation{Worthy of Offerings Dedicated to the Gods (1st) } \tocroot{Paṭhamaāhuneyyasutta}}
\markboth{Worthy of Offerings Dedicated to the Gods (1st) }{Paṭhamaāhuneyyasutta}
\extramarks{AN 8.57}{AN 8.57}

“Mendicants,\marginnote{1.1} a mendicant with eight qualities is worthy of offerings dedicated to the gods, worthy of hospitality, worthy of a religious donation, worthy of veneration with joined palms, and is the supreme field of merit for the world. What eight? 

It’s\marginnote{1.3} when a mendicant is ethical, restrained in the code of conduct, conducting themselves well and seeking alms in suitable places. Seeing danger in the slightest fault, they keep the rules they’ve undertaken. 

They’re\marginnote{1.4} learned, remembering and keeping what they’ve learned. These teachings are good in the beginning, good in the middle, and good in the end, meaningful and well-phrased, describing a spiritual practice that’s totally full and pure. They are very learned in such teachings, remembering them, reciting them, mentally scrutinizing them, and comprehending them theoretically. 

They\marginnote{1.5} have good friends, companions, and associates. 

They\marginnote{1.6} have right view, possessing right perspective. 

They\marginnote{1.7} get the four absorptions—blissful meditations in the present life that belong to the higher mind—when they want, without trouble or difficulty. 

They\marginnote{1.8} recollect many kinds of past lives, with features and details. 

With\marginnote{1.9} clairvoyance that is purified and surpasses the human, they see how sentient beings are reborn according to their deeds. 

They\marginnote{1.10} realize the undefiled freedom of heart and freedom by wisdom in this very life. And they live having realized it with their own insight due to the ending of defilements. 

A\marginnote{1.11} mendicant with these eight qualities is worthy of offerings dedicated to the gods, worthy of hospitality, worthy of a religious donation, worthy of veneration with joined palms, and is the supreme field of merit for the world.” 

%
\section*{{\suttatitleacronym AN 8.58}{\suttatitletranslation Worthy of Offerings Dedicated to the Gods (2nd) }{\suttatitleroot Dutiyaāhuneyyasutta}}
\addcontentsline{toc}{section}{\tocacronym{AN 8.58} \toctranslation{Worthy of Offerings Dedicated to the Gods (2nd) } \tocroot{Dutiyaāhuneyyasutta}}
\markboth{Worthy of Offerings Dedicated to the Gods (2nd) }{Dutiyaāhuneyyasutta}
\extramarks{AN 8.58}{AN 8.58}

“A\marginnote{1.1} mendicant with eight qualities is worthy of offerings dedicated to the gods, worthy of hospitality, worthy of a religious donation, worthy of veneration with joined palms, and is the supreme field of merit for the world. What eight? 

It’s\marginnote{1.3} when a mendicant is ethical, restrained in the code of conduct, conducting themselves well and seeking alms in suitable places. Seeing danger in the slightest fault, they keep the rules they’ve undertaken. 

They’re\marginnote{1.4} learned, remembering and keeping what they’ve learned. These teachings are good in the beginning, good in the middle, and good in the end, meaningful and well-phrased, describing a spiritual practice that’s totally full and pure. They are very learned in such teachings, remembering them, reciting them, mentally scrutinizing them, and comprehending them theoretically. 

They\marginnote{1.5} live with energy roused up. They’re strong, staunchly vigorous, not slacking off when it comes to developing skillful qualities. 

They\marginnote{1.6} live in the wilderness, in remote lodgings. 

They\marginnote{1.7} prevail over desire and discontent, and live having mastered desire and discontent whenever they arose. 

They\marginnote{1.8} prevail over fear and dread, and live having mastered fear and dread whenever they arose. 

They\marginnote{1.9} get the four absorptions—blissful meditations in the present life that belong to the higher mind—when they want, without trouble or difficulty. 

They\marginnote{1.10} realize the undefiled freedom of heart and freedom by wisdom in this very life. And they live having realized it with their own insight due to the ending of defilements. 

A\marginnote{1.11} mendicant with these eight qualities is worthy of offerings dedicated to the gods, worthy of hospitality, worthy of a religious donation, worthy of veneration with joined palms, and is the supreme field of merit for the world.” 

%
\section*{{\suttatitleacronym AN 8.59}{\suttatitletranslation Eight People (1st) }{\suttatitleroot Paṭhamapuggalasutta}}
\addcontentsline{toc}{section}{\tocacronym{AN 8.59} \toctranslation{Eight People (1st) } \tocroot{Paṭhamapuggalasutta}}
\markboth{Eight People (1st) }{Paṭhamapuggalasutta}
\extramarks{AN 8.59}{AN 8.59}

“Mendicants,\marginnote{1.1} these eight people are worthy of offerings dedicated to the gods, worthy of hospitality, worthy of a religious donation, worthy of greeting with joined palms, and are the supreme field of merit for the world. What eight? The stream-enterer and the one practicing to realize the fruit of stream-entry. The once-returner and the one practicing to realize the fruit of once-return. The non-returner and the one practicing to realize the fruit of non-return. The perfected one, and the one practicing for perfection. These are the eight people who are worthy of offerings dedicated to the gods, worthy of hospitality, worthy of a religious donation, worthy of greeting with joined palms, and are the supreme field of merit for the world. 

\begin{verse}%
Four\marginnote{2.1} practicing the path, \\
and four established in the fruit. \\
This is the upright \textsanskrit{Saṅgha}, \\
with wisdom, ethics, and immersion. 

For\marginnote{3.1} humans, those merit-seeking creatures, \\
who sponsor sacrifices, \\
making worldly merit, \\
what is given to the \textsanskrit{Saṅgha} is very fruitful.” 

%
\end{verse}

%
\section*{{\suttatitleacronym AN 8.60}{\suttatitletranslation Eight People (2nd) }{\suttatitleroot Dutiyapuggalasutta}}
\addcontentsline{toc}{section}{\tocacronym{AN 8.60} \toctranslation{Eight People (2nd) } \tocroot{Dutiyapuggalasutta}}
\markboth{Eight People (2nd) }{Dutiyapuggalasutta}
\extramarks{AN 8.60}{AN 8.60}

“Mendicants,\marginnote{1.1} these eight people are worthy of offerings dedicated to the gods, worthy of hospitality, worthy of a religious donation, worthy of greeting with joined palms, and are the supreme field of merit for the world. What eight? The stream-enterer and the one practicing to realize the fruit of stream-entry. The once-returner and the one practicing to realize the fruit of once-return. The non-returner and the one practicing to realize the fruit of non-return. The perfected one, and the one practicing for perfection. These are the eight people who are worthy of offerings dedicated to the gods, worthy of hospitality, worthy of a religious donation, worthy of greeting with joined palms, and are the supreme field of merit for the world. 

\begin{verse}%
Four\marginnote{2.1} practicing the path, \\
and four established in the fruit. \\
This is the exalted \textsanskrit{Saṅgha}, \\
the eight people among sentient beings. 

For\marginnote{3.1} humans, those merit-seeking creatures, \\
who sponsor sacrifices, \\
making worldly merit, \\
what’s given here is very fruitful.” 

%
\end{verse}

%
\addtocontents{toc}{\let\protect\contentsline\protect\nopagecontentsline}
\chapter*{The Chapter on Earthquakes }
\addcontentsline{toc}{chapter}{\tocchapterline{The Chapter on Earthquakes }}
\addtocontents{toc}{\let\protect\contentsline\protect\oldcontentsline}

%
\section*{{\suttatitleacronym AN 8.61}{\suttatitletranslation Desire }{\suttatitleroot Icchāsutta}}
\addcontentsline{toc}{section}{\tocacronym{AN 8.61} \toctranslation{Desire } \tocroot{Icchāsutta}}
\markboth{Desire }{Icchāsutta}
\extramarks{AN 8.61}{AN 8.61}

“Mendicants,\marginnote{1.1} there are eight kinds of people found in the world. What eight? 

First,\marginnote{1.3} when a mendicant stays secluded, living independently, a desire arises for material possessions. They try hard, strive, and make an effort to get them. But material possessions don’t come to them. And so they sorrow and wail and lament, beating their breast and falling into confusion because they don’t get those material possessions. This is called a mendicant who lives desiring material possessions. They try hard, strive, and make an effort to get them. But when possessions don’t come to them, they sorrow and lament. They’ve fallen from the true teaching. 

Next,\marginnote{2.1} when a mendicant stays secluded, living independently, a desire arises for material possessions. They try hard, strive, and make an effort to get them. And material possessions do come to them. And so they become indulgent and fall into negligence regarding those material possessions. This is called a mendicant who lives desiring material possessions. They try hard, strive, and make an effort to get them. And when possessions come to them, they become intoxicated and negligent. They’ve fallen from the true teaching. 

Next,\marginnote{3.1} when a mendicant stays secluded, living independently, a desire arises for material possessions. They don’t try hard, strive, and make an effort to get them. And material possessions don’t come to them. And so they sorrow and wail and lament, beating their breast and falling into confusion because they don’t get those material possessions. This is called a mendicant who lives desiring material possessions. They don’t try hard, strive, and make an effort to get them. And when possessions don’t come to them, they sorrow and lament. They’ve fallen from the true teaching. 

Next,\marginnote{4.1} when a mendicant stays secluded, living independently, a desire arises for material possessions. They don’t try hard, strive, and make an effort to get them. But material possessions do come to them. And so they become indulgent and fall into negligence regarding those material possessions. This is called a mendicant who lives desiring material possessions. They don’t try hard, strive, and make an effort to get them. But when possessions come to them, they become intoxicated and negligent. They’ve fallen from the true teaching. 

Next,\marginnote{5.1} when a mendicant stays secluded, living independently, a desire arises for material possessions. They try hard, strive, and make an effort to get them. But material possessions don’t come to them. But they don’t sorrow and wail and lament, beating their breast and falling into confusion because they don’t get those material possessions. This is called a mendicant who lives desiring material possessions. They try hard, strive, and make an effort to get them. But when possessions don’t come to them, they don’t sorrow and lament. They haven’t fallen from the true teaching. 

Next,\marginnote{6.1} when a mendicant stays secluded, living independently, a desire arises for material possessions. They try hard, strive, and make an effort to get them. And material possessions do come to them. But they don’t become indulgent and fall into negligence regarding those material possessions. This is called a mendicant who lives desiring material possessions. They try hard, strive, and make an effort to get them. But when possessions come to them, they don’t become intoxicated and negligent. They haven’t fallen from the true teaching. 

Next,\marginnote{7.1} when a mendicant stays secluded, living independently, a desire arises for material possessions. They don’t try hard, strive, and make an effort to get them. And material possessions don’t come to them. But they don’t sorrow and wail and lament, beating their breast and falling into confusion because they don’t get those material possessions. This is called a mendicant who lives desiring material possessions. They don’t try hard, strive, and make an effort to get them. And when possessions don’t come to them, they don’t sorrow and lament. They haven’t fallen from the true teaching. 

Next,\marginnote{8.1} when a mendicant stays secluded, living independently, a desire arises for material possessions. They don’t try hard, strive, and make an effort to get them. But material possessions do come to them. But they don’t become indulgent and fall into negligence regarding those material possessions. This is called a mendicant who lives desiring material possessions. They don’t try hard, strive, and make an effort to get them. And when possessions come to them, they don’t become intoxicated and negligent. They haven’t fallen from the true teaching. 

These\marginnote{9.1} are the eight people found in the world.” 

%
\section*{{\suttatitleacronym AN 8.62}{\suttatitletranslation Good Enough }{\suttatitleroot Alaṁsutta}}
\addcontentsline{toc}{section}{\tocacronym{AN 8.62} \toctranslation{Good Enough } \tocroot{Alaṁsutta}}
\markboth{Good Enough }{Alaṁsutta}
\extramarks{AN 8.62}{AN 8.62}

“Mendicants,\marginnote{1.1} a mendicant with six qualities is good enough for themselves and others. What six? A mendicant is quick-witted when it comes to skillful teachings. They readily memorize the teachings they’ve heard. They examine the meaning of teachings they’ve memorized. Understanding the meaning and the teaching, they practice accordingly. They’re a good speaker. Their voice is polished, clear, articulate, and expresses the meaning. They educate, encourage, fire up, and inspire their spiritual companions. A mendicant with these six qualities is good enough for themselves and others. 

A\marginnote{2.1} mendicant with five qualities is good enough for themselves and others. What five? A mendicant is not quick-witted when it comes to skillful teachings. They readily memorize the teachings they’ve heard. They examine the meaning of teachings they’ve memorized. Understanding the meaning and the teaching, they practice accordingly. They’re a good speaker. Their voice is polished, clear, articulate, and expresses the meaning. They educate, encourage, fire up, and inspire their spiritual companions. A mendicant with these five qualities is good enough for themselves and others. 

A\marginnote{3.1} mendicant with four qualities is good enough for themselves but not for others. What four? A mendicant is quick-witted when it comes to skillful teachings. They readily memorize the teachings they’ve heard. They examine the meaning of teachings they’ve memorized. Understanding the meaning and the teaching, they practice accordingly. But they’re not a good speaker. Their voice isn’t polished, clear, articulate, and doesn’t express the meaning. They don’t educate, encourage, fire up, and inspire their spiritual companions. A mendicant with these four qualities is good enough for themselves but not for others. 

A\marginnote{4.1} mendicant with four qualities is good enough for others but not for themselves. What four? A mendicant is quick-witted when it comes to skillful teachings. They readily memorize the teachings they’ve heard. But they don’t examine the meaning of teachings they’ve memorized. Understanding the meaning and the teaching, they don’t practice accordingly. They’re a good speaker. Their voice is polished, clear, articulate, and expresses the meaning. They educate, encourage, fire up, and inspire their spiritual companions. A mendicant with these four qualities is good enough for others but not for themselves. 

A\marginnote{5.1} mendicant with three qualities is good enough for themselves but not for others. What three? A mendicant is not quick-witted when it comes to skillful teachings. They readily memorize the teachings they’ve heard. They examine the meaning of teachings they’ve memorized. Understanding the meaning and the teaching, they practice accordingly. But they’re not a good speaker. Their voice isn’t polished, clear, articulate, and doesn’t express the meaning. They don’t educate, encourage, fire up, and inspire their spiritual companions. A mendicant with these three qualities is good enough for themselves but not for others. 

A\marginnote{6.1} mendicant with three qualities is good enough for others but not for themselves. What three? A mendicant is not quick-witted when it comes to skillful teachings. They readily memorize the teachings they’ve heard. But they don’t examine the meaning of teachings they’ve memorized. Understanding the meaning and the teaching, they don’t practice accordingly. They’re a good speaker. Their voice is polished, clear, articulate, and expresses the meaning. They educate, encourage, fire up, and inspire their spiritual companions. A mendicant with these three qualities is good enough for others but not for themselves. 

A\marginnote{7.1} mendicant with two qualities is good enough for themselves but not for others. What two? A mendicant is not quick-witted when it comes to skillful teachings. And they don’t readily memorize the teachings they’ve heard. But they examine the meaning of teachings they have memorized. Understanding the meaning and the teaching, they practice accordingly. They’re not a good speaker. Their voice isn’t polished, clear, articulate, and doesn’t express the meaning. They don’t educate, encourage, fire up, and inspire their spiritual companions. A mendicant with these two qualities is good enough for themselves but not for others. 

A\marginnote{8.1} mendicant with two qualities is good enough for others but not for themselves. What two? A mendicant is not quick-witted when it comes to skillful teachings. And they don’t readily memorize the teachings they’ve heard. Nor do they examine the meaning of teachings they’ve memorized. Understanding the meaning and the teaching, they don’t practice accordingly. But they’re a good speaker. Their voice is polished, clear, articulate, and expresses the meaning. They educate, encourage, fire up, and inspire their spiritual companions. A mendicant with these two qualities is good enough for others but not for themselves.” 

%
\section*{{\suttatitleacronym AN 8.63}{\suttatitletranslation A Teaching in Brief }{\suttatitleroot Saṁkhittasutta}}
\addcontentsline{toc}{section}{\tocacronym{AN 8.63} \toctranslation{A Teaching in Brief } \tocroot{Saṁkhittasutta}}
\markboth{A Teaching in Brief }{Saṁkhittasutta}
\extramarks{AN 8.63}{AN 8.63}

Then\marginnote{1.1} a mendicant went up to the Buddha, bowed, sat down to one side, and said to him, “Sir, may the Buddha please teach me Dhamma in brief. When I’ve heard it, I’ll live alone, withdrawn, diligent, keen, and resolute.” 

“This\marginnote{1.3} is exactly how some foolish people ask me for something. But when the teaching has been explained they think only of following me around.” 

“Sir,\marginnote{1.5} may the Buddha please teach me Dhamma in brief! May the Holy One teach me the Dhamma in brief! Hopefully I can understand the meaning of what the Buddha says! Hopefully I can be an heir of the Buddha’s teaching!” 

“Well\marginnote{1.6} then, mendicant, you should train like this: ‘My mind will be steady and well settled internally. And bad, unskillful qualities that have arisen will not occupy my mind.’ That’s how you should train. 

When\marginnote{2.1} your mind is steady and well settled internally, and bad, unskillful qualities that have arisen don’t occupy your mind, then you should train like this: ‘I will develop the heart’s release by love. I’ll cultivate it, make it my vehicle and my basis, keep it up, consolidate it, and properly implement it.’ That’s how you should train. 

When\marginnote{3.1} this immersion is well developed and cultivated in this way, you should develop it while placing the mind and keeping it connected. You should develop it without placing the mind, but just keeping it connected. You should develop it without placing the mind or keeping it connected. You should develop it with rapture. You should develop it without rapture. You should develop it with pleasure. You should develop it with equanimity. 

When\marginnote{4.1} this immersion is well developed and cultivated in this way, you should train like this: ‘I will develop the heart’s release by compassion …’ … ‘I will develop the heart’s release by rejoicing …’ … ‘I will develop the heart’s release by equanimity. I’ll cultivate it, make it my vehicle and my basis, keep it up, consolidate it, and properly implement it.’ That’s how you should train. 

When\marginnote{5.1} this immersion is well developed and cultivated in this way, you should develop it while placing the mind and keeping it connected. You should develop it without placing the mind, but just keeping it connected. You should develop it without placing the mind or keeping it connected. You should develop it with rapture. You should develop it without rapture. You should develop it with pleasure. You should develop it with equanimity. 

When\marginnote{6.1} this immersion is well developed and cultivated in this way, you should train like this: ‘I’ll meditate observing an aspect of the body—keen, aware, and mindful, rid of desire and aversion for the world.’ That’s how you should train. 

When\marginnote{7.1} this immersion is well developed and cultivated in this way, you should develop it while placing the mind and keeping it connected. You should develop it without placing the mind, but just keeping it connected. You should develop it without placing the mind or keeping it connected. You should develop it with rapture. You should develop it without rapture. You should develop it with pleasure. You should develop it with equanimity. 

When\marginnote{8.1} this immersion is well developed and cultivated in this way, you should train like this: ‘I’ll meditate on an aspect of feelings …’ … ‘I’ll meditate on an aspect of the mind …’ … ‘I’ll meditate on an aspect of principles—keen, aware, and mindful, rid of desire and aversion for the world.’ That’s how you should train. 

When\marginnote{9.1} this immersion is well developed and cultivated in this way, you should develop it while placing the mind and keeping it connected. You should develop it without placing the mind, but just keeping it connected. You should develop it without placing the mind or keeping it connected. You should develop it with rapture. You should develop it without rapture. You should develop it with pleasure. You should develop it with equanimity. 

When\marginnote{10.1} this immersion is well developed and cultivated in this way, wherever you walk, you’ll walk comfortably. Wherever you stand, you’ll stand comfortably. Wherever you sit, you’ll sit comfortably. Wherever you lie down, you’ll lie down comfortably.” 

When\marginnote{11.1} that mendicant had been given this advice by the Buddha, he got up from his seat, bowed, and respectfully circled the Buddha, keeping him on his right, before leaving. 

Then\marginnote{11.2} that mendicant, living alone, withdrawn, diligent, keen, and resolute, soon realized the supreme culmination of the spiritual path in this very life. He lived having achieved with his own insight the goal for which gentlemen rightly go forth from the lay life to homelessness. 

He\marginnote{11.3} understood: “Rebirth is ended; the spiritual journey has been completed; what had to be done has been done; there is no return to any state of existence.” And that mendicant became one of the perfected. 

%
\section*{{\suttatitleacronym AN 8.64}{\suttatitletranslation At Gayā Head }{\suttatitleroot Gayāsīsasutta}}
\addcontentsline{toc}{section}{\tocacronym{AN 8.64} \toctranslation{At Gayā Head } \tocroot{Gayāsīsasutta}}
\markboth{At Gayā Head }{Gayāsīsasutta}
\extramarks{AN 8.64}{AN 8.64}

At\marginnote{1.1} one time the Buddha was staying near \textsanskrit{Gayā} on \textsanskrit{Gayā} Head. There the Buddha addressed the mendicants: 

“Mendicants,\marginnote{1.3} before my awakening—when I was still not awake but intent on awakening—I perceived light but did not see visions. 

Then\marginnote{2.1} it occurred to me, ‘What if I were to both perceive light and see visions? Then my knowledge and vision would become even more purified.’ 

So\marginnote{3.1} after some time, living alone, withdrawn, diligent, keen, and resolute, I perceived light and saw visions. But I didn’t associate with those deities, converse, or engage in discussion. 

Then\marginnote{4.1} it occurred to me, ‘What if I were to perceive light and see visions; and associate with those deities, converse, and engage in discussion? Then my knowledge and vision would become even more purified.’ 

So\marginnote{5.1} after some time … I perceived light and saw visions. And I associated with those deities, conversed, and engaged in discussion. But I didn’t know which orders of gods those deities came from. 

Then\marginnote{6.1} it occurred to me, ‘What if I were to perceive light and see visions; and associate with those deities, converse, and engage in discussion; and find out which orders of gods those deities come from? Then my knowledge and vision would become even more purified.’ 

So\marginnote{7.1} after some time … I perceived light and saw visions. And I associated with those deities … And I found out which orders of gods those deities came from. But I didn’t know what deeds caused those deities to be reborn there after passing away from here. 

So\marginnote{7.5} after some time … I found out what deeds caused those deities to be reborn there after passing away from here. But I didn’t know what deeds caused those deities to have such food and such an experience of pleasure and pain. 

So\marginnote{7.9} after some time … I found out what deeds caused those deities to have such food and such an experience of pleasure and pain. But I didn’t know that these deities have a life-span of such a length. 

So\marginnote{7.13} after some time … I found out that these deities have a life-span of such a length. But I didn’t know whether or not I had previously lived together with those deities. 

Then\marginnote{8.1} it occurred to me, ‘What if I were to perceive light and see visions; and associate with those deities, converse, and engage in discussion; and find out which orders of gods those deities come from; and what deeds caused those deities to be reborn there after passing away from here; and what deeds caused those deities to have such food and such an experience of pleasure and pain; and that these deities have a life-span of such a length; and whether or not I have previously lived together with those deities? Then my knowledge and vision would become even more purified.’ 

So\marginnote{9.1} after some time … I found out whether or not I have previously lived together with those deities. 

As\marginnote{10.1} long as my knowledge and vision about the deities was not fully purified from these eight perspectives, I didn’t announce my supreme perfect awakening in this world with its gods, \textsanskrit{Māras}, and \textsanskrit{Brahmās}, this population with its ascetics and brahmins, its gods and humans. 

But\marginnote{10.2} when my knowledge and vision about the deities was fully purified from these eight perspectives, I announced my supreme perfect awakening in this world with its gods, \textsanskrit{Māras}, and \textsanskrit{Brahmās}, this population with its ascetics and brahmins, its gods and humans. Knowledge and vision arose in me: ‘My freedom is unshakable; this is my last rebirth; now there’ll be no more future lives.’” 

%
\section*{{\suttatitleacronym AN 8.65}{\suttatitletranslation Dimensions of Mastery }{\suttatitleroot Abhibhāyatanasutta}}
\addcontentsline{toc}{section}{\tocacronym{AN 8.65} \toctranslation{Dimensions of Mastery } \tocroot{Abhibhāyatanasutta}}
\markboth{Dimensions of Mastery }{Abhibhāyatanasutta}
\extramarks{AN 8.65}{AN 8.65}

“Mendicants,\marginnote{1.1} there are these eight dimensions of mastery. What eight? 

Perceiving\marginnote{1.3} form internally, someone sees visions externally, limited, both pretty and ugly. Mastering them, they perceive: ‘I know and see.’ This is the first dimension of mastery. 

Perceiving\marginnote{2.1} form internally, someone sees visions externally, limitless, both pretty and ugly. Mastering them, they perceive: ‘I know and see.’ This is the second dimension of mastery. 

Not\marginnote{3.1} perceiving form internally, someone sees visions externally, limited, both pretty and ugly. Mastering them, they perceive: ‘I know and see.’ This is the third dimension of mastery. 

Not\marginnote{4.1} perceiving form internally, someone sees visions externally, limitless, both pretty and ugly. Mastering them, they perceive: ‘I know and see.’ This is the fourth dimension of mastery. 

Not\marginnote{5.1} perceiving form internally, someone sees visions externally, blue, with blue color, blue hue, and blue tint. Mastering them, they perceive: ‘I know and see.’ This is the fifth dimension of mastery. 

Not\marginnote{6.1} perceiving form internally, someone sees visions externally, yellow, with yellow color, yellow hue, and yellow tint. Mastering them, they perceive: ‘I know and see.’ This is the sixth dimension of mastery. 

Not\marginnote{7.1} perceiving form internally, someone sees visions externally, red, with red color, red hue, and red tint. Mastering them, they perceive: ‘I know and see.’ This is the seventh dimension of mastery. 

Not\marginnote{8.1} perceiving form internally, someone sees visions externally, white, with white color, white hue, and white tint. Mastering them, they perceive: ‘I know and see.’ This is the eighth dimension of mastery. 

These\marginnote{8.4} are the eight dimensions of mastery.” 

%
\section*{{\suttatitleacronym AN 8.66}{\suttatitletranslation Liberations }{\suttatitleroot Vimokkhasutta}}
\addcontentsline{toc}{section}{\tocacronym{AN 8.66} \toctranslation{Liberations } \tocroot{Vimokkhasutta}}
\markboth{Liberations }{Vimokkhasutta}
\extramarks{AN 8.66}{AN 8.66}

“Mendicants,\marginnote{1.1} there are these eight liberations. What eight? Having physical form, they see visions. This is the first liberation. 

Not\marginnote{2.1} perceiving form internally, they see visions externally. This is the second liberation. 

They’re\marginnote{3.1} focused only on beauty. This is the third liberation. 

Going\marginnote{4.1} totally beyond perceptions of form, with the ending of perceptions of impingement, not focusing on perceptions of diversity, aware that ‘space is infinite’, they enter and remain in the dimension of infinite space. This is the fourth liberation. 

Going\marginnote{5.1} totally beyond the dimension of infinite space, aware that ‘consciousness is infinite’, they enter and remain in the dimension of infinite consciousness. This is the fifth liberation. 

Going\marginnote{6.1} totally beyond the dimension of infinite consciousness, aware that ‘there is nothing at all’, they enter and remain in the dimension of nothingness. This is the sixth liberation. 

Going\marginnote{7.1} totally beyond the dimension of nothingness, they enter and remain in the dimension of neither perception nor non-perception. This is the seventh liberation. 

Going\marginnote{8.1} totally beyond the dimension of neither perception nor non-perception, they enter and remain in the cessation of perception and feeling. This is the eighth liberation. 

These\marginnote{8.3} are the eight liberations.” 

%
\section*{{\suttatitleacronym AN 8.67}{\suttatitletranslation Ignoble Expressions }{\suttatitleroot Anariyavohārasutta}}
\addcontentsline{toc}{section}{\tocacronym{AN 8.67} \toctranslation{Ignoble Expressions } \tocroot{Anariyavohārasutta}}
\markboth{Ignoble Expressions }{Anariyavohārasutta}
\extramarks{AN 8.67}{AN 8.67}

“Mendicants,\marginnote{1.1} there are these eight ignoble expressions. What eight? Saying you’ve seen, heard, thought, or known something, but you haven’t. And saying you haven’t seen, heard, thought, or known something, and you have. These are the eight ignoble expressions.” 

%
\section*{{\suttatitleacronym AN 8.68}{\suttatitletranslation Noble Expressions }{\suttatitleroot Ariyavohārasutta}}
\addcontentsline{toc}{section}{\tocacronym{AN 8.68} \toctranslation{Noble Expressions } \tocroot{Ariyavohārasutta}}
\markboth{Noble Expressions }{Ariyavohārasutta}
\extramarks{AN 8.68}{AN 8.68}

“Mendicants,\marginnote{1.1} there are these eight noble expressions. What eight? Saying you haven’t seen, heard, thought, or known something, and you haven’t. And saying you’ve seen, heard, thought, or known something, and you have. These are the eight noble expressions.” 

%
\section*{{\suttatitleacronym AN 8.69}{\suttatitletranslation Assemblies }{\suttatitleroot Parisāsutta}}
\addcontentsline{toc}{section}{\tocacronym{AN 8.69} \toctranslation{Assemblies } \tocroot{Parisāsutta}}
\markboth{Assemblies }{Parisāsutta}
\extramarks{AN 8.69}{AN 8.69}

“Mendicants,\marginnote{1.1} there are these eight assemblies. What eight? The assemblies of aristocrats, brahmins, householders, and ascetics. An assembly of the gods under the Four Great Kings. An assembly of the gods under the Thirty-Three. An assembly of \textsanskrit{Māras}. An assembly of \textsanskrit{Brahmās}. 

I\marginnote{1.4} recall having approached an assembly of hundreds of aristocrats. There I used to sit with them, converse, and engage in discussion. And my appearance and voice became just like theirs. I educated, encouraged, fired up, and inspired them with a Dhamma talk. But when I spoke they didn’t know: ‘Who is this that speaks? Is it a god or a human?’ And when my Dhamma talk was finished I vanished. But when I vanished they didn’t know: ‘Who was that who vanished? Was it a god or a human?’ 

I\marginnote{2.1} recall having approached an assembly of hundreds of brahmins … householders … ascetics … the gods under the Four Great Kings … the gods under the Thirty-Three … \textsanskrit{Māras} … \textsanskrit{Brahmās}. There too I used to sit with them, converse, and engage in discussion. And my appearance and voice became just like theirs. I educated, encouraged, fired up, and inspired them with a Dhamma talk. But when I spoke they didn’t know: ‘Who is this that speaks? Is it a god or a human?’ And when my Dhamma talk was finished I vanished. But when I vanished they didn’t know: ‘Who was that who vanished? Was it a god or a human?’ These are the eight assemblies.” 

%
\section*{{\suttatitleacronym AN 8.70}{\suttatitletranslation Earthquakes }{\suttatitleroot Bhūmicālasutta}}
\addcontentsline{toc}{section}{\tocacronym{AN 8.70} \toctranslation{Earthquakes } \tocroot{Bhūmicālasutta}}
\markboth{Earthquakes }{Bhūmicālasutta}
\extramarks{AN 8.70}{AN 8.70}

At\marginnote{1.1} one time the Buddha was staying near \textsanskrit{Vesālī}, at the Great Wood, in the hall with the peaked roof. 

Then\marginnote{1.2} the Buddha robed up in the morning and, taking his bowl and robe, entered \textsanskrit{Vesālī} for alms. Then, after the meal, on his return from almsround, he addressed Venerable Ānanda, “Ānanda, get your sitting cloth. Let’s go to the \textsanskrit{Cāpāla} shrine for the day’s meditation.” 

“Yes,\marginnote{1.6} sir,” replied Ānanda. Taking his sitting cloth he followed behind the Buddha. 

Then\marginnote{2.1} the Buddha went up to the \textsanskrit{Cāpāla} shrine, where he sat on the seat spread out. When he was seated he said to Venerable Ānanda: 

“Ānanda,\marginnote{3.1} \textsanskrit{Vesālī} is lovely. And the Udena, Gotamaka, Sattamba, Bahuputta, \textsanskrit{Sārandada}, and \textsanskrit{Cāpāla} Tree-shrines are all lovely. Whoever has developed and cultivated the four bases of psychic power—made them a vehicle and a basis, kept them up, consolidated them, and properly implemented them—may, if they wish, live on for the eon or what’s left of the eon. The Realized One has developed and cultivated the four bases of psychic power, made them a vehicle and a basis, kept them up, consolidated them, and properly implemented them. If he wished, the Realized One could live on for the eon or what’s left of the eon.” 

But\marginnote{3.4} Ānanda didn’t get it, even though the Buddha dropped such an obvious hint, such a clear sign. He didn’t beg the Buddha, “Sir, may the Blessed One please remain for the eon! May the Holy One please remain for the eon! That would be for the welfare and happiness of the people, out of compassion for the world, for the benefit, welfare, and happiness of gods and humans.” For his mind was as if possessed by \textsanskrit{Māra}. 

For\marginnote{4.1} a second time … 

And\marginnote{4.2} for a third time, the Buddha said to him: 

“Ānanda,\marginnote{4.3} \textsanskrit{Vesālī} is lovely. And the Udena, Gotamaka, Sattamba, Bahuputta, \textsanskrit{Sārandada}, and \textsanskrit{Cāpāla} Tree-shrines are all lovely. Whoever has developed and cultivated the four bases of psychic power—made them a vehicle and a basis, kept them up, consolidated them, and properly implemented them—may, if they wish, live on for the eon or what’s left of the eon. The Realized One has developed and cultivated the four bases of psychic power, made them a vehicle and a basis, kept them up, consolidated them, and properly implemented them. If he wished, the Realized One could live on for the eon or what’s left of the eon.” 

But\marginnote{4.6} Ānanda didn’t get it, even though the Buddha dropped such an obvious hint, such a clear sign. He didn’t beg the Buddha, “Sir, may the Blessed One please remain for the eon! May the Holy One please remain for the eon! That would be for the welfare and happiness of the people, out of compassion for the world, for the benefit, welfare, and happiness of gods and humans.” For his mind was as if possessed by \textsanskrit{Māra}. 

Then\marginnote{5.1} the Buddha said to Venerable Ānanda, “Go now, Ānanda, at your convenience.” 

“Yes,\marginnote{5.3} sir,” replied Ānanda. He rose from his seat, bowed, and respectfully circled the Buddha, keeping him on his right, before sitting at the root of a tree close by. 

And\marginnote{5.4} then, not long after Ānanda had left, \textsanskrit{Māra} the Wicked said to the Buddha: 

“Sir,\marginnote{6.1} may the Blessed One now become fully extinguished! May the Holy One now become fully extinguished! Now is the time for the Buddha to become fully extinguished. 

Sir,\marginnote{6.2} you once made this statement: ‘Wicked One, I will not become fully extinguished until I have monk disciples who are competent, educated, assured, learned, have memorized the teachings, and practice in line with the teachings; not until they practice appropriately, living in line with the teaching; not until they’ve learned their tradition, and explain, teach, assert, establish, clarify, analyze, and reveal; not until they can legitimately and completely refute the doctrines of others that come up, and teach with a demonstrable basis.’ Today you do have such monk disciples. 

May\marginnote{7.1} the Blessed One now become fully extinguished! May the Holy One now become fully extinguished! Now is the time for the Buddha to become fully extinguished. 

Sir,\marginnote{7.2} you once made this statement: ‘Wicked One, I will not become fully extinguished until I have nun disciples who are competent, educated, assured, learned …’ … 

‘Wicked\marginnote{7.4} One, I will not become fully extinguished until I have layman disciples who are competent, educated, assured, learned …’ … 

‘Wicked\marginnote{7.5} One, I will not become fully extinguished until I have laywoman disciples who are competent, educated, assured, learned …’ … Today you do have such laywoman disciples. 

Sir,\marginnote{8.1} may the Blessed One now become fully extinguished! May the Holy One become fully extinguished! Now is the time for the Buddha to become fully extinguished. Sir, you once made this statement: 

‘Wicked\marginnote{8.3} One, I will not become fully extinguished until my spiritual path is successful and prosperous, extensive, popular, widespread, and well proclaimed wherever there are gods and humans.’ Today your spiritual path is successful and prosperous, extensive, popular, widespread, and well proclaimed wherever there are gods and humans. 

Sir,\marginnote{9.1} may the Blessed One now become fully extinguished! May the Holy One become fully extinguished! Now is the time for the Buddha to become fully extinguished.” 

“Relax,\marginnote{9.2} Wicked One. The final extinguishment of the Realized One will be soon. Three months from now the Realized One will finally be extinguished.” 

So\marginnote{10.1} at the \textsanskrit{Cāpāla} Tree-shrine the Buddha, mindful and aware, surrendered the life force. When he did so there was a great earthquake, awe-inspiring and hair-raising, and thunder cracked the sky. Then, understanding this matter, on that occasion the Buddha expressed this heartfelt sentiment: 

\begin{verse}%
“Weighing\marginnote{10.4} up the incomparable against an extension of life, \\
the sage surrendered the life force. \\
Happy inside, serene, \\
he burst out of this self-made chain like a suit of armor.” 

%
\end{verse}

Then\marginnote{11.1} Venerable Ānanda thought, “That was a really big earthquake! That was really a very big earthquake; awe-inspiring and hair-raising, and thunder cracked the sky! What’s the cause, what’s the reason for a great earthquake?” 

Then\marginnote{13.1} Venerable Ānanda went up to the Buddha, bowed, sat down to one side, and said to him, “Sir, that was a really big earthquake! That was really a very big earthquake; awe-inspiring and hair-raising, and thunder cracked the sky! What’s the cause, what’s the reason for a great earthquake?” 

“Ānanda,\marginnote{14.1} there are these eight causes and reasons for a great earthquake. What eight? 

This\marginnote{14.3} great earth is grounded on water, the water is grounded on air, and the air stands in space. At a time when a great wind blows, it stirs the water, and the water stirs the earth. This is the first cause and reason for a great earthquake. 

Furthermore,\marginnote{15.1} there is an ascetic or brahmin with psychic power who has achieved mastery of the mind, or a god who is mighty and powerful. They’ve developed a limited perception of earth and a limitless perception of water. They make the earth shake and rock and tremble. This is the second cause and reason for a great earthquake. 

Furthermore,\marginnote{16.1} when the being intent on awakening passes away from the host of Joyful Gods, he’s conceived in his mother’s belly, mindful and aware. Then the earth shakes and rocks and trembles. This is the third cause and reason for a great earthquake. 

Furthermore,\marginnote{17.1} when the being intent on awakening comes out of his mother’s belly mindful and aware, the earth shakes and rocks and trembles. This is the fourth cause and reason for a great earthquake. 

Furthermore,\marginnote{18.1} when the Realized One realizes the supreme perfect awakening, the earth shakes and rocks and trembles. This is the fifth cause and reason for a great earthquake. 

Furthermore,\marginnote{19.1} when the Realized One rolls forth the supreme Wheel of Dhamma, the earth shakes and rocks and trembles. This is the sixth cause and reason for a great earthquake. 

Furthermore,\marginnote{20.1} when the Realized One, mindful and aware, surrenders the life force, the earth shakes and rocks and trembles. This is the seventh cause and reason for a great earthquake. 

Furthermore,\marginnote{21.1} when the Realized One becomes fully extinguished through the element of extinguishment with nothing left over, the earth shakes and rocks and trembles. This is the eighth cause and reason for a great earthquake. 

These\marginnote{21.3} are the eight causes and reasons for a great earthquake.” 

%
\addtocontents{toc}{\let\protect\contentsline\protect\nopagecontentsline}
\chapter*{The Chapter on Pairs }
\addcontentsline{toc}{chapter}{\tocchapterline{The Chapter on Pairs }}
\addtocontents{toc}{\let\protect\contentsline\protect\oldcontentsline}

%
\section*{{\suttatitleacronym AN 8.71}{\suttatitletranslation Inspiring All Around (1st) }{\suttatitleroot Paṭhamasaddhāsutta}}
\addcontentsline{toc}{section}{\tocacronym{AN 8.71} \toctranslation{Inspiring All Around (1st) } \tocroot{Paṭhamasaddhāsutta}}
\markboth{Inspiring All Around (1st) }{Paṭhamasaddhāsutta}
\extramarks{AN 8.71}{AN 8.71}

“Mendicants,\marginnote{1.1} a mendicant is faithful but not ethical. So they’re incomplete in that respect, and should fulfill it, thinking: ‘How can I become faithful and ethical?’ When the mendicant is faithful and ethical, they’re complete in that respect. 

A\marginnote{2.1} mendicant is faithful and ethical, but not learned. So they’re incomplete in that respect, and should fulfill it, thinking: ‘How can I become faithful, ethical, and learned?’ When the mendicant is faithful, ethical, and learned, they’re complete in that respect. 

A\marginnote{3.1} mendicant is faithful, ethical, and learned, but not a Dhamma speaker. … they don’t frequent assemblies … they don’t teach Dhamma to the assembly with assurance … they don’t get the four absorptions—blissful meditations in the present life that belong to the higher mind—when they want, without trouble or difficulty … they don’t realize the undefiled freedom of heart and freedom by wisdom in this very life, and live having realized it with their own insight due to the ending of defilements. So they’re incomplete in that respect, and should fulfill it, thinking: ‘How can I become faithful, ethical, and learned, a Dhamma speaker, one who frequents assemblies, one who teaches Dhamma to the assembly with assurance, one who gets the four absorptions when they want, and one who lives having realized the ending of defilements?’ 

When\marginnote{4.1} they’re faithful, ethical, and learned, a Dhamma speaker, one who frequents assemblies, one who teaches Dhamma to the assembly with assurance, one who gets the four absorptions when they want, and one who lives having realized the ending of defilements, they’re complete in that respect. A mendicant who has these eight qualities is inspiring all around, and is complete in every respect.” 

%
\section*{{\suttatitleacronym AN 8.72}{\suttatitletranslation Inspiring All Around (2nd) }{\suttatitleroot Dutiyasaddhāsutta}}
\addcontentsline{toc}{section}{\tocacronym{AN 8.72} \toctranslation{Inspiring All Around (2nd) } \tocroot{Dutiyasaddhāsutta}}
\markboth{Inspiring All Around (2nd) }{Dutiyasaddhāsutta}
\extramarks{AN 8.72}{AN 8.72}

“A\marginnote{1.1} mendicant is faithful, but not ethical. So they’re incomplete in that respect, and should fulfill it, thinking: ‘How can I become faithful and ethical?’ When the mendicant is faithful and ethical, they’re complete in that respect. 

A\marginnote{2.1} mendicant is faithful and ethical, but not learned. … they’re not a Dhamma speaker … they don’t frequent assemblies … they don’t teach Dhamma to the assembly with assurance … they don’t have direct meditative experience of the peaceful liberations that are formless, transcending form … they don’t realize the undefiled freedom of heart and freedom by wisdom in this very life, and live having realized it with their own insight due to the ending of defilements. So they’re incomplete in that respect, and should fulfill it, thinking: ‘How can I become faithful, ethical, and learned, a Dhamma speaker, one who frequents assemblies, one who teaches Dhamma to the assembly with assurance, one who gets the formless liberations, and one who lives having realized the ending of defilements?’ 

When\marginnote{3.1} they’re faithful, ethical, and learned, a Dhamma speaker, one who frequents assemblies, one who teaches Dhamma to the assembly with assurance, one who gets the formless liberations, and one who lives having realized the ending of defilements, they’re complete in that respect. A mendicant who has these eight qualities is inspiring all around, and is complete in every respect.” 

%
\section*{{\suttatitleacronym AN 8.73}{\suttatitletranslation Mindfulness of Death (1st) }{\suttatitleroot Paṭhamamaraṇassatisutta}}
\addcontentsline{toc}{section}{\tocacronym{AN 8.73} \toctranslation{Mindfulness of Death (1st) } \tocroot{Paṭhamamaraṇassatisutta}}
\markboth{Mindfulness of Death (1st) }{Paṭhamamaraṇassatisutta}
\extramarks{AN 8.73}{AN 8.73}

At\marginnote{1.1} one time the Buddha was staying at \textsanskrit{Nādika} in the brick house. There the Buddha addressed the mendicants, “Mendicants!” 

“Venerable\marginnote{1.4} sir,” they replied. The Buddha said this: 

“Mendicants,\marginnote{1.6} when mindfulness of death is developed and cultivated it’s very fruitful and beneficial. It culminates in the deathless and ends with the deathless. But do you develop mindfulness of death?” 

When\marginnote{2.1} he said this, one of the mendicants said to the Buddha, “Sir, I develop mindfulness of death.” 

“But\marginnote{2.3} mendicant, how do you develop it?” 

“In\marginnote{2.4} this case, sir, I think: ‘Oh, if I’d only live for another day and night, I’d focus on the Buddha’s instructions and I could really achieve a lot.’ That’s how I develop mindfulness of death.” 

Another\marginnote{3.1} mendicant said to the Buddha, “Sir, I too develop mindfulness of death.” 

“But\marginnote{3.3} mendicant, how do you develop it?” 

“In\marginnote{3.4} this case, sir, I think: ‘Oh, if I’d only live for another day, I’d focus on the Buddha’s instructions and I could really achieve a lot.’ That’s how I develop mindfulness of death.” 

Another\marginnote{4.1} mendicant said to the Buddha, “Sir, I too develop mindfulness of death.” 

“But\marginnote{4.3} mendicant, how do you develop it?” 

“In\marginnote{4.4} this case, sir, I think: ‘Oh, if I’d only live for half a day, I’d focus on the Buddha’s instructions and I could really achieve a lot.’ That’s how I develop mindfulness of death.” 

Another\marginnote{5.1} mendicant said to the Buddha, “Sir, I too develop mindfulness of death.” 

“But\marginnote{5.3} mendicant, how do you develop it?” 

“In\marginnote{5.4} this case, sir, I think: ‘Oh, if I’d only live as long as it takes to eat a single almsmeal, I’d focus on the Buddha’s instructions and I could really achieve a lot.’ That’s how I develop mindfulness of death.” 

Another\marginnote{6.1} mendicant said to the Buddha, “Sir, I too develop mindfulness of death.” 

“But\marginnote{6.3} mendicant, how do you develop it?” 

“In\marginnote{6.4} this case, sir, I think: ‘Oh, if I’d only live as long as it takes to eat half an almsmeal, I’d focus on the Buddha’s instructions and I could really achieve a lot.’ That’s how I develop mindfulness of death.” 

Another\marginnote{7.1} mendicant said to the Buddha, “Sir, I too develop mindfulness of death.” 

“But\marginnote{7.3} mendicant, how do you develop it?” 

“In\marginnote{7.4} this case, sir, I think: ‘Oh, if I’d only live as long as it takes to chew and swallow four or five mouthfuls, I’d focus on the Buddha’s instructions and I could really achieve a lot.’ That’s how I develop mindfulness of death.” 

Another\marginnote{8.1} mendicant said to the Buddha, “Sir, I too develop mindfulness of death.” 

“But\marginnote{8.3} mendicant, how do you develop it?” 

“In\marginnote{8.4} this case, sir, I think: ‘Oh, if I’d only live as long as it takes to chew and swallow a single mouthful, I’d focus on the Buddha’s instructions and I could really achieve a lot.’ That’s how I develop mindfulness of death.” 

Another\marginnote{9.1} mendicant said to the Buddha, “Sir, I too develop mindfulness of death.” 

“But\marginnote{9.3} mendicant, how do you develop it?” 

“In\marginnote{9.4} this case, sir, I think: ‘Oh, if I’d only live as long as it takes to breathe out after breathing in, or to breathe in after breathing out, I’d focus on the Buddha’s instructions and I could really achieve a lot.’ That’s how I develop mindfulness of death.” 

When\marginnote{10.1} this was said, the Buddha said to those mendicants: 

“The\marginnote{10.2} mendicants who develop mindfulness of death by wishing to live for a day and night … or to live for a day … or to live for half a day … or to live as long as it takes to eat a meal of almsfood … or to live as long as it takes to eat half a meal of almsfood … or to live as long as it takes to chew and swallow four or five mouthfuls … These are called mendicants who live negligently. They slackly develop mindfulness of death for the ending of defilements. 

But\marginnote{11.1} the mendicants who develop mindfulness of death by wishing to live as long as it takes to chew and swallow a single mouthful … or to live as long as it takes to breathe out after breathing in, or to breathe in after breathing out … These are called mendicants who live diligently. They keenly develop mindfulness of death for the ending of defilements. 

So\marginnote{12.1} you should train like this: ‘We will live diligently. We will keenly develop mindfulness of death for the ending of defilements.’ That’s how you should train.” 

%
\section*{{\suttatitleacronym AN 8.74}{\suttatitletranslation Mindfulness of Death (2nd) }{\suttatitleroot Dutiyamaraṇassatisutta}}
\addcontentsline{toc}{section}{\tocacronym{AN 8.74} \toctranslation{Mindfulness of Death (2nd) } \tocroot{Dutiyamaraṇassatisutta}}
\markboth{Mindfulness of Death (2nd) }{Dutiyamaraṇassatisutta}
\extramarks{AN 8.74}{AN 8.74}

At\marginnote{1.1} one time the Buddha was staying at \textsanskrit{Nādika} in the brick house. There the Buddha addressed the mendicants: “Mendicants, when mindfulness of death is developed and cultivated it’s very fruitful and beneficial. It culminates in the deathless and ends with the deathless. 

And\marginnote{2.1} how is mindfulness of death developed and cultivated to be very fruitful and beneficial, to culminate in the deathless and end with the deathless? As day passes by and night draws close, a mendicant reflects: ‘I might die of many causes. A snake might bite me, or a scorpion or centipede might sting me. And if I died from that it would stop my practice. Or I might stumble off a cliff, or get food poisoning, or suffer a disturbance of bile, phlegm, or piercing winds. Or I might be attacked by humans or non-humans. And if I died from that it would stop my practice.’ That mendicant should reflect: ‘Are there any bad, unskillful qualities that I haven’t given up, which might be an obstacle to me if I die tonight?’ 

Suppose\marginnote{3.1} that, upon checking, a mendicant knows that there are such bad, unskillful qualities. Then in order to give them up they should apply intense enthusiasm, effort, zeal, vigor, perseverance, mindfulness, and situational awareness. 

Suppose\marginnote{4.1} your clothes or head were on fire. In order to extinguish it, you’d apply intense enthusiasm, effort, zeal, vigor, perseverance, mindfulness, and situational awareness. In the same way, in order to give up those bad, unskillful qualities, that mendicant should apply intense enthusiasm … 

But\marginnote{5.1} suppose that, upon checking, a mendicant knows that there are no such bad, unskillful qualities. Then that mendicant should meditate with rapture and joy, training day and night in skillful qualities. 

Or\marginnote{6.1} else, as night passes by and day draws close, a mendicant reflects: ‘I might die of many causes. A snake might bite me, or a scorpion or centipede might sting me. And if I died from that it would stop my practice. Or I might stumble off a cliff, or get food poisoning, or suffer a disturbance of bile, phlegm, or piercing winds. Or I might be attacked by humans or non-humans. And if I died from that it would stop my practice.’ That mendicant should reflect: ‘Are there any bad, unskillful qualities that I haven’t given up, which might be an obstacle to me if I die today?’ 

Suppose\marginnote{7.1} that, upon checking, a mendicant knows that there are such bad, unskillful qualities. Then in order to give them up they should apply intense enthusiasm, effort, zeal, vigor, perseverance, mindfulness, and situational awareness. 

Suppose\marginnote{8.1} your clothes or head were on fire. In order to extinguish it, you’d apply intense enthusiasm, effort, zeal, vigor, perseverance, mindfulness, and situational awareness. In the same way, in order to give up those bad, unskillful qualities, that mendicant should apply intense enthusiasm … 

But\marginnote{9.1} suppose that, upon checking, a mendicant knows that there are no such bad, unskillful qualities. Then that mendicant should meditate with rapture and joy, training day and night in skillful qualities. Mindfulness of death, when developed and cultivated in this way, is very fruitful and beneficial. It culminates in the deathless and ends with the deathless.” 

%
\section*{{\suttatitleacronym AN 8.75}{\suttatitletranslation Accomplishments (1st) }{\suttatitleroot Paṭhamasampadāsutta}}
\addcontentsline{toc}{section}{\tocacronym{AN 8.75} \toctranslation{Accomplishments (1st) } \tocroot{Paṭhamasampadāsutta}}
\markboth{Accomplishments (1st) }{Paṭhamasampadāsutta}
\extramarks{AN 8.75}{AN 8.75}

“Mendicants,\marginnote{1.1} there are these eight accomplishments. What eight? Accomplishment in initiative, protection, good friendship, and balanced finances. And accomplishment in faith, ethics, generosity, and wisdom. These are the eight accomplishments. 

\begin{verse}%
They’re\marginnote{2.1} enterprising in the workplace, \\
diligent in managing things, \\
they balance their finances, \\
and preserve their wealth. 

Faithful,\marginnote{3.1} accomplished in ethics, \\
bountiful, rid of stinginess, \\
they always purify the path \\
to well-being in lives to come. 

And\marginnote{4.1} so these eight qualities \\
of a faithful householder \\
are declared by the one who is truly named \\
to lead to happiness in both spheres, 

welfare\marginnote{5.1} and benefit in this life, \\
and happiness in lives to come. \\
This is how, for a householder, \\
merit grows by generosity.” 

%
\end{verse}

%
\section*{{\suttatitleacronym AN 8.76}{\suttatitletranslation Accomplishments (2nd) }{\suttatitleroot Dutiyasampadāsutta}}
\addcontentsline{toc}{section}{\tocacronym{AN 8.76} \toctranslation{Accomplishments (2nd) } \tocroot{Dutiyasampadāsutta}}
\markboth{Accomplishments (2nd) }{Dutiyasampadāsutta}
\extramarks{AN 8.76}{AN 8.76}

“Mendicants,\marginnote{1.1} there are these eight accomplishments. What eight? Accomplishment in initiative, protection, good friendship, and balanced finances. And accomplishment in faith, ethics, generosity, and wisdom. 

And\marginnote{1.4} what is accomplishment in initiative? It’s when a gentleman earns a living by means such as farming, trade, raising cattle, archery, government service, or one of the professions. He understands how to go about these things in order to complete and organize the work. This is called accomplishment in initiative. 

And\marginnote{2.1} what is accomplishment in protection? It’s when a gentleman owns legitimate wealth that he has earned by his own efforts and initiative, built up with his own hands, gathered by the sweat of the brow. He ensures it is guarded and protected, thinking: ‘How can I prevent my wealth from being taken by rulers or bandits, consumed by fire, swept away by flood, or taken by unloved heirs?’ This is called accomplishment in protection. 

And\marginnote{3.1} what is accomplishment in good friendship? It’s when a gentleman resides in a town or village. And in that place there are householders or their children who may be young or old, but are mature in conduct, accomplished in faith, ethics, generosity, and wisdom. He associates with them, converses and engages in discussion. And he emulates the same kind of accomplishment in faith, ethics, generosity, and wisdom. This is called accomplishment in good friendship. 

And\marginnote{4.1} what is accomplishment in balanced finances? It’s when a gentleman, knowing his income and expenditure, balances his finances, being neither too extravagant nor too frugal. He thinks, ‘In this way my income will exceed my expenditure, not the reverse.’ It’s like an appraiser or their apprentice who, holding up the scales, knows that it’s low by this much or high by this much. In the same way, a gentleman, knowing his income and expenditure, balances his finances, being neither too extravagant nor too frugal. He thinks, ‘In this way my income will exceed my expenditure, not the reverse.’ If a gentleman has little income but an opulent life, people will say: ‘This gentleman eats their wealth like a fig-eater!’ If a gentleman has a large income but a spartan life, people will say: ‘This gentleman is starving themselves to death!’ But a gentleman, knowing his income and expenditure, leads a balanced life, neither too extravagant nor too frugal, thinking, ‘In this way my income will exceed my expenditure, not the reverse.’ This is called accomplishment in balanced finances. 

And\marginnote{5.1} what is accomplishment in faith? It’s when a gentleman has faith in the Realized One’s awakening: ‘That Blessed One is perfected, a fully awakened Buddha … teacher of gods and humans, awakened, blessed.’ This is called accomplishment in faith. 

And\marginnote{6.1} what is accomplishment in ethics? It’s when a gentleman doesn’t kill living creatures, steal, commit sexual misconduct, lie, or consume alcoholic drinks that cause negligence. This is called accomplishment in ethics. 

And\marginnote{7.1} what is accomplishment in generosity? It’s when a gentleman lives at home rid of the stain of stinginess, freely generous, open-handed, loving to let go, committed to charity, loving to give and to share. This is called accomplishment in generosity. 

And\marginnote{8.1} what is accomplishment in wisdom? It’s when a gentleman is wise. He has the wisdom of arising and passing away which is noble, penetrative, and leads to the complete ending of suffering. This is called accomplishment in wisdom. 

These\marginnote{9.1} are the eight accomplishments. 

\begin{verse}%
They’re\marginnote{10.1} enterprising in the workplace, \\
diligent in managing things, \\
they balance their finances, \\
and preserve their wealth. 

Faithful,\marginnote{11.1} accomplished in ethics, \\
bountiful, rid of stinginess, \\
they always purify the path \\
to well-being in lives to come. 

And\marginnote{12.1} so these eight qualities \\
of a faithful householder \\
are declared by the one who is truly named \\
to lead to happiness in both spheres, 

welfare\marginnote{13.1} and benefit in this life, \\
and happiness in the next. \\
This is how, for a householder, \\
merit grows by generosity.” 

%
\end{verse}

%
\section*{{\suttatitleacronym AN 8.77}{\suttatitletranslation Desires }{\suttatitleroot Icchāsutta}}
\addcontentsline{toc}{section}{\tocacronym{AN 8.77} \toctranslation{Desires } \tocroot{Icchāsutta}}
\markboth{Desires }{Icchāsutta}
\extramarks{AN 8.77}{AN 8.77}

There\marginnote{1.1} \textsanskrit{Sāriputta} addressed the mendicants: “Reverends, mendicants!” 

“Reverend,”\marginnote{1.3} they replied. \textsanskrit{Sāriputta} said this: 

“Reverends,\marginnote{2.1} these eight people are found in the world. What eight? 

First,\marginnote{2.3} when a mendicant stays secluded, living independently, a desire arises for material possessions. They try hard, strive, and make an effort to get them. But material possessions don’t come to them. And so they sorrow and wail and lament, beating their breast and falling into confusion because they don’t get those material possessions. This is called a mendicant who lives desiring material possessions. They try hard, strive, and make an effort to get them. But when possessions don’t come to them, they sorrow and lament. They’ve fallen from the true teaching. 

Next,\marginnote{3.1} when a mendicant stays secluded, living independently, a desire arises for material possessions. They try hard, strive, and make an effort to get them. And material possessions do come to them. And so they become indulgent and fall into negligence regarding those material possessions. This is called a mendicant who lives desiring material possessions. They try hard, strive, and make an effort to get them. And when possessions come to them, they become intoxicated and negligent. They’ve fallen from the true teaching. 

Next,\marginnote{4.1} when a mendicant stays secluded, living independently, a desire arises for material possessions. They don’t try hard, strive, and make an effort to get them. And material possessions don’t come to them. And so they sorrow and wail and lament, beating their breast and falling into confusion because they don’t get those material possessions. This is called a mendicant who lives desiring material possessions. They don’t try hard, strive, and make an effort to get them. But when possessions don’t come to them, they sorrow and lament. They’ve fallen from the true teaching. 

Next,\marginnote{5.1} when a mendicant stays secluded, living independently, a desire arises for material possessions. They don’t try hard, strive, and make an effort to get them. But material possessions do come to them. And so they become indulgent and fall into negligence regarding those material possessions. This is called a mendicant who lives desiring material possessions. They don’t try hard, strive, and make an effort to get them. But when possessions come to them, they become intoxicated and negligent. They’ve fallen from the true teaching. 

Next,\marginnote{6.1} when a mendicant stays secluded, living independently, a desire arises for material possessions. They try hard, strive, and make an effort to get them. But material possessions don’t come to them. But they don’t sorrow and wail and lament, beating their breast and falling into confusion because they don’t get those material possessions. This is called a mendicant who lives desiring material possessions. They try hard, strive, and make an effort to get them. But when possessions don’t come to them, they don’t sorrow and lament. They haven’t fallen from the true teaching. 

Next,\marginnote{7.1} when a mendicant stays secluded, living independently, a desire arises for material possessions. They try hard, strive, and make an effort to get them. And material possessions do come to them. But they don’t become indulgent and fall into negligence regarding those material possessions. This is called a mendicant who lives desiring material possessions. They try hard, strive, and make an effort to get them. But when possessions come to them, they don’t become intoxicated and negligent. They haven’t fallen from the true teaching. 

Next,\marginnote{8.1} when a mendicant stays secluded, living independently, a desire arises for material possessions. They don’t try hard, strive, and make an effort to get them. And material possessions don’t come to them. But they don’t sorrow and wail and lament, beating their breast and falling into confusion because they don’t get those material possessions. This is called a mendicant who lives desiring material possessions. They don’t try hard, strive, and make an effort to get them. And when possessions don’t come to them, they don’t sorrow and lament. They haven’t fallen from the true teaching. 

Next,\marginnote{9.1} when a mendicant stays secluded, living independently, a desire arises for material possessions. They don’t try hard, strive, and make an effort to get them. But material possessions do come to them. But they don’t become indulgent and fall into negligence regarding those material possessions. This is called a mendicant who lives desiring material possessions. They don’t try hard, strive, and make an effort to get them. And when possessions come to them, they don’t become intoxicated and negligent. They haven’t fallen from the true teaching. 

These\marginnote{9.6} eight people are found in the world.” 

%
\section*{{\suttatitleacronym AN 8.78}{\suttatitletranslation Good Enough }{\suttatitleroot Alaṁsutta}}
\addcontentsline{toc}{section}{\tocacronym{AN 8.78} \toctranslation{Good Enough } \tocroot{Alaṁsutta}}
\markboth{Good Enough }{Alaṁsutta}
\extramarks{AN 8.78}{AN 8.78}

There\marginnote{1.1} \textsanskrit{Sāriputta} addressed the mendicants: “Reverends, a mendicant with six qualities is good enough for themselves and others. What six? A mendicant is quick-witted when it comes to skillful teachings. They readily memorize the teachings they’ve heard. They examine the meaning of teachings they’ve memorized. Understanding the meaning and the teaching, they practice accordingly. They’re a good speaker. Their voice is polished, clear, articulate, and expresses the meaning. They educate, encourage, fire up, and inspire their spiritual companions. A mendicant with these six qualities is good enough for themselves and others. 

A\marginnote{2.1} mendicant with five qualities is good enough for themselves and others. What five? A mendicant is not quick-witted when it comes to skillful teachings. They readily memorize the teachings they’ve heard. They examine the meaning of teachings they’ve memorized. Understanding the meaning and the teaching, they practice accordingly. They’re a good speaker. Their voice is polished, clear, articulate, and expresses the meaning. They educate, encourage, fire up, and inspire their spiritual companions. A mendicant with these five qualities is good enough for themselves and others. 

A\marginnote{3.1} mendicant with four qualities is good enough for themselves but not for others. What four? A mendicant is quick-witted when it comes to skillful teachings. They readily memorize the teachings they’ve heard. They examine the meaning of teachings they’ve memorized. Understanding the meaning and the teaching, they practice accordingly. They’re not a good speaker. Their voice isn’t polished, clear, articulate, and doesn’t express the meaning. They don’t educate, encourage, fire up, and inspire their spiritual companions. A mendicant with these four qualities is good enough for themselves but not for others. 

A\marginnote{4.1} mendicant with four qualities is good enough for others but not for themselves. What four? A mendicant is quick-witted when it comes to skillful teachings. They readily memorize the teachings they’ve heard. But they don’t examine the meaning of teachings they’ve memorized. Understanding the meaning and the teaching, they don’t practice accordingly. They’re a good speaker. Their voice is polished, clear, articulate, and expresses the meaning. They educate, encourage, fire up, and inspire their spiritual companions. A mendicant with these four qualities is good enough for others but not for themselves. 

A\marginnote{5.1} mendicant with three qualities is good enough for themselves but not for others. What three? A mendicant is not quick-witted when it comes to skillful teachings. They readily memorize the teachings they’ve heard. They examine the meaning of teachings they’ve memorized. Understanding the meaning and the teaching, they practice accordingly. They’re not a good speaker. Their voice isn’t polished, clear, articulate, and doesn’t express the meaning. They don’t educate, encourage, fire up, and inspire their spiritual companions. A mendicant with these three qualities is good enough for themselves but not for others. 

A\marginnote{6.1} mendicant with three qualities is good enough for others but not for themselves. What three? A mendicant is not quick-witted when it comes to skillful teachings. They readily memorize the teachings they’ve heard. But they don’t examine the meaning of teachings they’ve memorized. Understanding the meaning and the teaching, they don’t practice accordingly. They’re a good speaker. Their voice is polished, clear, articulate, and expresses the meaning. They educate, encourage, fire up, and inspire their spiritual companions. A mendicant with these three qualities is good enough for others but not for themselves. 

A\marginnote{7.1} mendicant with two qualities is good enough for themselves but not for others. What two? A mendicant is not quick-witted when it comes to skillful teachings. And they don’t readily memorize the teachings they’ve heard. They examine the meaning of teachings they’ve memorized. Understanding the meaning and the teaching, they practice accordingly. They’re not a good speaker. Their voice isn’t polished, clear, articulate, and doesn’t express the meaning. They don’t educate, encourage, fire up, and inspire their spiritual companions. A mendicant with these two qualities is good enough for themselves but not for others. 

A\marginnote{8.1} mendicant with two qualities is good enough for others but not for themselves. What two? A mendicant is not quick-witted when it comes to skillful teachings. And they don’t readily memorize the teachings they’ve heard. Nor do they examine the meaning of teachings they’ve memorized. Understanding the meaning and the teaching, they don’t practice accordingly. They’re a good speaker. Their voice is polished, clear, articulate, and expresses the meaning. They educate, encourage, fire up, and inspire their spiritual companions. A mendicant with these two qualities is good enough for others but not for themselves.” 

%
\section*{{\suttatitleacronym AN 8.79}{\suttatitletranslation Decline }{\suttatitleroot Parihānasutta}}
\addcontentsline{toc}{section}{\tocacronym{AN 8.79} \toctranslation{Decline } \tocroot{Parihānasutta}}
\markboth{Decline }{Parihānasutta}
\extramarks{AN 8.79}{AN 8.79}

“These\marginnote{1.1} eight things lead to the decline of a mendicant trainee. What eight? They relish work, talk, sleep, and company. They don’t guard the sense doors and they eat too much. They relish closeness and proliferation. These eight things lead to the decline of a mendicant trainee. 

These\marginnote{2.1} eight things don’t lead to the decline of a mendicant trainee. What eight? They don’t relish work, talk, and sleep. They guard the sense doors, and they don’t eat too much. They don’t relish closeness and proliferation. These eight things don’t lead to the decline of a mendicant trainee.” 

%
\section*{{\suttatitleacronym AN 8.80}{\suttatitletranslation Grounds for Laziness and Arousing Energy }{\suttatitleroot Kusītārambhavatthusutta}}
\addcontentsline{toc}{section}{\tocacronym{AN 8.80} \toctranslation{Grounds for Laziness and Arousing Energy } \tocroot{Kusītārambhavatthusutta}}
\markboth{Grounds for Laziness and Arousing Energy }{Kusītārambhavatthusutta}
\extramarks{AN 8.80}{AN 8.80}

“Mendicants,\marginnote{1.1} there are eight grounds for laziness. What eight? 

Firstly,\marginnote{1.3} a mendicant has some work to do. They think: ‘I have some work to do. But while doing it my body will get tired. I’d better have a lie down.’ They lie down, and don’t rouse energy for attaining the unattained, achieving the unachieved, and realizing the unrealized. This is the first ground for laziness. 

Furthermore,\marginnote{2.1} a mendicant has done some work. They think: ‘I’ve done some work. But while working my body got tired. I’d better have a lie down.’ They lie down, and don’t rouse energy for attaining the unattained, achieving the unachieved, and realizing the unrealized. This is the second ground for laziness. 

Furthermore,\marginnote{3.1} a mendicant has to go on a journey. They think: ‘I have to go on a journey. But while walking my body will get tired. I’d better have a lie down.’ They lie down, and don’t rouse energy for attaining the unattained, achieving the unachieved, and realizing the unrealized. This is the third ground for laziness. 

Furthermore,\marginnote{4.1} a mendicant has gone on a journey. They think: ‘I’ve gone on a journey. But while walking my body got tired. I’d better have a lie down.’ They lie down, and don’t rouse energy for attaining the unattained, achieving the unachieved, and realizing the unrealized. This is the fourth ground for laziness. 

Furthermore,\marginnote{5.1} a mendicant has wandered for alms, but they didn’t get to fill up on as much food as they like, rough or fine. They think: ‘I’ve wandered for alms, but I didn’t get to fill up on as much food as I like, rough or fine. My body is tired and unfit for work. I’d better have a lie down.’ They lie down, and don’t rouse energy for achieving the unachieved, attaining the unattained, and realizing the unrealized. This is the fifth ground for laziness. 

Furthermore,\marginnote{6.1} a mendicant has wandered for alms, and they got to fill up on as much food as they like, rough or fine. They think: ‘I’ve wandered for alms, and I got to fill up on as much food as I like, rough or fine. My body is heavy, unfit for work, like I’ve just eaten a load of beans. I’d better have a lie down.’ They lie down, and don’t rouse energy for achieving the unachieved, attaining the unattained, and realizing the unrealized. This is the sixth ground for laziness. 

Furthermore,\marginnote{7.1} a mendicant feels a little sick. They think: ‘I feel a little sick. Lying down would be good for me. I’d better have a lie down.’ They lie down, and don’t rouse energy for achieving the unachieved, attaining the unattained, and realizing the unrealized. This is the seventh ground for laziness. 

Furthermore,\marginnote{8.1} a mendicant has recently recovered from illness. They think: ‘I’ve recently recovered from illness. My body is weak and unfit for work. I’d better have a lie down.’ They lie down, and don’t rouse energy for attaining the unattained, achieving the unachieved, and realizing the unrealized. This is the eighth ground for laziness. These are the eight grounds for laziness. 

There\marginnote{9.1} are eight grounds for arousing energy. What eight? Firstly, a mendicant has some work to do. They think: ‘I have some work to do. While working it’s not easy to focus on the instructions of the Buddhas. I’d better preemptively rouse up energy for attaining the unattained, achieving the unachieved, and realizing the unrealized.’ They rouse energy for attaining the unattained, achieving the unachieved, and realizing the unrealized. This is the first ground for arousing energy. 

Furthermore,\marginnote{10.1} a mendicant has done some work. They think: ‘I’ve done some work. While I was working I wasn’t able to focus on the instructions of the Buddhas. I’d better preemptively rouse up energy for attaining the unattained, achieving the unachieved, and realizing the unrealized.’ They rouse up energy … This is the second ground for arousing energy. 

Furthermore,\marginnote{11.1} a mendicant has to go on a journey. They think: ‘I have to go on a journey. While walking it’s not easy to focus on the instructions of the Buddhas. I’d better preemptively rouse up energy …’ … This is the third ground for arousing energy. 

Furthermore,\marginnote{12.1} a mendicant has gone on a journey. They think: ‘I’ve gone on a journey. While I was walking I wasn’t able to focus on the instructions of the Buddhas. I’d better preemptively rouse up energy …’ … This is the fourth ground for arousing energy. 

Furthermore,\marginnote{13.1} a mendicant has wandered for alms, but they didn’t get to fill up on as much food as they like, rough or fine. They think: ‘I’ve wandered for alms, but I didn’t get to fill up on as much food as I like, rough or fine. My body is light and fit for work. I’d better preemptively rouse up energy …’ … This is the fifth ground for arousing energy. 

Furthermore,\marginnote{14.1} a mendicant has wandered for alms, and they got to fill up on as much food as they like, rough or fine. They think: ‘I’ve wandered for alms, and I got to fill up on as much food as I like, rough or fine. My body is strong and fit for work. I’d better preemptively rouse up energy …’ … This is the sixth ground for arousing energy. 

Furthermore,\marginnote{15.1} a mendicant feels a little sick. They think: ‘I feel a little sick. It’s possible this illness will worsen. I’d better preemptively rouse up energy …’ … This is the seventh ground for arousing energy. 

Furthermore,\marginnote{16.1} a mendicant has recently recovered from illness. They think: ‘I’ve recently recovered from illness. It’s possible the illness will come back. I’d better preemptively rouse up energy for attaining the unattained, achieving the unachieved, and realizing the unrealized.’ They rouse energy for attaining the unattained, achieving the unachieved, and realizing the unrealized. This is the eighth ground for arousing energy. 

These\marginnote{17.1} are the eight grounds for arousing energy.” 

%
\addtocontents{toc}{\let\protect\contentsline\protect\nopagecontentsline}
\chapter*{The Chapter on Mindfulness }
\addcontentsline{toc}{chapter}{\tocchapterline{The Chapter on Mindfulness }}
\addtocontents{toc}{\let\protect\contentsline\protect\oldcontentsline}

%
\section*{{\suttatitleacronym AN 8.81}{\suttatitletranslation Mindfulness and Situational Awareness }{\suttatitleroot Satisampajaññasutta}}
\addcontentsline{toc}{section}{\tocacronym{AN 8.81} \toctranslation{Mindfulness and Situational Awareness } \tocroot{Satisampajaññasutta}}
\markboth{Mindfulness and Situational Awareness }{Satisampajaññasutta}
\extramarks{AN 8.81}{AN 8.81}

“Mendicants,\marginnote{1.1} when there is no mindfulness and situational awareness, one who lacks mindfulness and situational awareness has destroyed a vital condition for conscience and prudence. When there is no conscience and prudence, one who lacks conscience and prudence has destroyed a vital condition for sense restraint. When there is no sense restraint, one who lacks sense restraint has destroyed a vital condition for ethical conduct. When there is no ethical conduct, one who lacks ethics has destroyed a vital condition for right immersion. When there is no right immersion, one who lacks right immersion has destroyed a vital condition for true knowledge and vision. When there is no true knowledge and vision, one who lacks true knowledge and vision has destroyed a vital condition for disillusionment and dispassion. When there is no disillusionment and dispassion, one who lacks disillusionment and dispassion has destroyed a vital condition for knowledge and vision of freedom. 

Suppose\marginnote{1.8} there was a tree that lacked branches and foliage. Its shoots, bark, softwood, and heartwood would not grow to fullness. 

In\marginnote{1.9} the same way, when there is no mindfulness and situational awareness, one who lacks mindfulness and situational awareness has destroyed a vital condition for conscience and prudence. When there is no conscience and prudence … One who lacks disillusionment and dispassion has destroyed a vital condition for knowledge and vision of freedom. 

When\marginnote{2.1} there is mindfulness and situational awareness, one who has fulfilled mindfulness and situational awareness has fulfilled a vital condition for conscience and prudence. When there is conscience and prudence, a person who has fulfilled conscience and prudence has fulfilled a vital condition for sense restraint. When there is sense restraint, one who has sense restraint has fulfilled a vital condition for ethical conduct. When there is ethical conduct, one who has fulfilled ethical conduct has fulfilled a vital condition for right immersion. When there is right immersion, one who has fulfilled right immersion has fulfilled a vital condition for true knowledge and vision. When there is true knowledge and vision, one who has fulfilled true knowledge and vision has fulfilled a vital condition for disillusionment and dispassion. When there is disillusionment and dispassion, one who has fulfilled disillusionment and dispassion has fulfilled a vital condition for knowledge and vision of freedom. 

Suppose\marginnote{2.8} there was a tree that was complete with branches and foliage. Its shoots, bark, softwood, and heartwood would grow to fullness. 

In\marginnote{2.9} the same way, when there is mindfulness and situational awareness, one who has fulfilled mindfulness and situational awareness has fulfilled a vital condition for conscience and prudence. When there is conscience and prudence … One who has fulfilled disillusionment and dispassion has fulfilled a vital condition for knowledge and vision of freedom.” 

%
\section*{{\suttatitleacronym AN 8.82}{\suttatitletranslation With Puṇṇiya }{\suttatitleroot Puṇṇiyasutta}}
\addcontentsline{toc}{section}{\tocacronym{AN 8.82} \toctranslation{With Puṇṇiya } \tocroot{Puṇṇiyasutta}}
\markboth{With Puṇṇiya }{Puṇṇiyasutta}
\extramarks{AN 8.82}{AN 8.82}

Then\marginnote{1.1} Venerable \textsanskrit{Puṇṇiya} went up to the Buddha, bowed, sat down to one side, and said to him: 

“Sir,\marginnote{1.2} what is the cause, what is the reason why sometimes the Realized One feels inspired to teach, and other times not?” 

“\textsanskrit{Puṇṇiya},\marginnote{1.3} when a mendicant has faith but doesn’t approach, the Realized One doesn’t feel inspired to teach. But when a mendicant has faith and approaches, the Realized One feels inspired to teach. When a mendicant has faith and approaches, but doesn’t pay homage … they pay homage, but don’t ask questions … they ask questions, but don’t lend an ear … they lend an ear, but don’t remember the teaching they’ve heard … they remember the teaching they’ve heard, but don’t reflect on the meaning of the teachings they’ve remembered … they reflect on the meaning of the teachings they’ve remembered, but, having understood the meaning and the teaching, they don’t practice accordingly. The Realized One doesn’t feel inspired to teach. 

But\marginnote{2.1} when a mendicant has faith, approaches, pays homage, asks questions, lends an ear, remembers the teachings, reflects on the meaning, and practices accordingly, the Realized One feels inspired to teach. When someone has these eight qualities, the Realized One feels totally inspired to teach.” 

%
\section*{{\suttatitleacronym AN 8.83}{\suttatitletranslation Rooted }{\suttatitleroot Mūlakasutta}}
\addcontentsline{toc}{section}{\tocacronym{AN 8.83} \toctranslation{Rooted } \tocroot{Mūlakasutta}}
\markboth{Rooted }{Mūlakasutta}
\extramarks{AN 8.83}{AN 8.83}

“Mendicants,\marginnote{1.1} if wanderers who follow other paths were to ask: ‘Reverends, all things have what as their root? What produces them? What is their origin? What is their meeting place? What is their chief? What is their ruler? What is their overseer? What is their core?’ How would you answer them?” 

“Our\marginnote{1.3} teachings are rooted in the Buddha. He is our guide and our refuge. Sir, may the Buddha himself please clarify the meaning of this. The mendicants will listen and remember it.” 

“Well\marginnote{2.1} then, mendicants, I will teach it. Listen and pay close attention, I will speak.” 

“Yes,\marginnote{2.3} sir,” they replied. The Buddha said this: 

“Mendicants,\marginnote{2.5} if wanderers who follow other paths were to ask: ‘Reverends, all things have what as their root? What produces them? What is their origin? What is their meeting place? What is their chief? What is their ruler? What is their overseer? What is their core?’ You should answer them: ‘Reverends, all things are rooted in desire. Attention produces them. Contact is their origin. Feeling is their meeting place. Immersion is their chief. Mindfulness is their ruler. Wisdom is their overseer. Freedom is their core.’ When questioned by wanderers who follow other paths, that’s how you should answer them.” 

%
\section*{{\suttatitleacronym AN 8.84}{\suttatitletranslation A Master Thief }{\suttatitleroot Corasutta}}
\addcontentsline{toc}{section}{\tocacronym{AN 8.84} \toctranslation{A Master Thief } \tocroot{Corasutta}}
\markboth{A Master Thief }{Corasutta}
\extramarks{AN 8.84}{AN 8.84}

“Mendicants,\marginnote{1.1} a master thief with eight factors is soon executed, and doesn’t have long to live. What eight? He attacks unprovoked. He steals everything without exception. He kills a woman. He rapes a girl. He robs a monk. He robs the royal treasury. He works close to home. He’s not skilled at hiding his booty. A master thief with these eight factors is soon executed, and doesn’t have long to live. 

A\marginnote{2.1} master thief with eight factors is not soon executed, and lives long. What eight? He doesn’t attack unprovoked. He doesn’t steal everything without exception. He doesn’t kill a woman. He doesn’t rape a girl. He doesn’t rob a monk. He doesn’t rob the royal treasury. He doesn’t work close to home. He’s skilled at hiding his booty. A master thief with these eight factors is not soon executed, and lives long.” 

%
\section*{{\suttatitleacronym AN 8.85}{\suttatitletranslation Terms for the Realized One }{\suttatitleroot Samaṇasutta}}
\addcontentsline{toc}{section}{\tocacronym{AN 8.85} \toctranslation{Terms for the Realized One } \tocroot{Samaṇasutta}}
\markboth{Terms for the Realized One }{Samaṇasutta}
\extramarks{AN 8.85}{AN 8.85}

“‘Ascetic’\marginnote{1.1} is a term for the Realized One, the perfected one, the fully awakened Buddha. ‘Brahmin’, ‘Knowledge Master’, ‘Healer’, ‘Unstained’, ‘Immaculate’, ‘Knower’, and ‘Freed’ are terms for the Realized One, the perfected one, the fully awakened Buddha. 

\begin{verse}%
The\marginnote{2.1} supreme should be attained by an ascetic, \\
a brahmin who has lived the life; \\
it should be attained by a knowledge master, \\
a healer. 

The\marginnote{3.1} supreme should be attained by the unstained, \\
stainless and pure; \\
it should be attained by a knower, \\
who is free. 

I\marginnote{4.1} am victorious in battle! \\
Released, I release others from their chains. \\
I am a dragon completely tamed, \\
an adept, I am extinguished.” 

%
\end{verse}

%
\section*{{\suttatitleacronym AN 8.86}{\suttatitletranslation With Nāgita }{\suttatitleroot Yasasutta}}
\addcontentsline{toc}{section}{\tocacronym{AN 8.86} \toctranslation{With Nāgita } \tocroot{Yasasutta}}
\markboth{With Nāgita }{Yasasutta}
\extramarks{AN 8.86}{AN 8.86}

At\marginnote{1.1} one time the Buddha was wandering in the land of the Kosalans together with a large \textsanskrit{Saṅgha} of mendicants when he arrived at a village of the Kosalan brahmins named \textsanskrit{Icchānaṅgala}. He stayed in a forest near \textsanskrit{Icchānaṅgala}. The brahmins and householders of \textsanskrit{Icchānaṅgala} heard: 

“It\marginnote{1.4} seems the ascetic Gotama—a Sakyan, gone forth from a Sakyan family—has arrived at \textsanskrit{Icchānaṅgala}. He is staying in a forest near \textsanskrit{Icchānaṅgala}. He has this good reputation: ‘That Blessed One is perfected, a fully awakened Buddha, accomplished in knowledge and conduct, holy, knower of the world, supreme guide for those who wish to train, teacher of gods and humans, awakened, blessed.’ … It’s good to see such perfected ones.” 

Then,\marginnote{2.1} when the night had passed, they took many different foods and went to the forest near \textsanskrit{Icchānaṅgala}, where they stood outside the gates making a dreadful racket. 

Now,\marginnote{2.2} at that time Venerable \textsanskrit{Nāgita} was the Buddha’s attendant. Then the Buddha said to \textsanskrit{Nāgita}, “\textsanskrit{Nāgita}, who’s making that dreadful racket? You’d think it was fishermen hauling in a catch!” 

“Sir,\marginnote{2.5} it’s these brahmins and householders of \textsanskrit{Icchānaṅgala}. They’ve brought many different foods, and they’re standing outside the gates wanting to offer it specially to the Buddha and the mendicant \textsanskrit{Saṅgha}.” 

“\textsanskrit{Nāgita},\marginnote{2.6} may I never become famous. May fame not come to me. There are those who can’t get the bliss of renunciation, the bliss of seclusion, the bliss of peace, the bliss of awakening when they want, without trouble or difficulty like I can. Let them enjoy the filthy, lazy pleasure of possessions, honor, and popularity.” 

“Sir,\marginnote{3.1} may the Blessed One please relent now! May the Holy One relent! Now is the time for the Buddha to relent. Wherever the Buddha now goes, the brahmins and householders will incline the same way, as will the people of town and country. It’s like when it rains heavily and the water flows downhill. In the same way, wherever the Buddha now goes, the brahmins and householders will incline the same way, as will the people of town and country. Why is that? Because of the Buddha’s ethics and wisdom.” 

“\textsanskrit{Nāgita},\marginnote{4.1} may I never become famous. May fame not come to me. There are those who can’t get the bliss of renunciation, the bliss of seclusion, the bliss of peace, the bliss of awakening when they want, without trouble or difficulty like I can. Let them enjoy the filthy, lazy pleasure of possessions, honor, and popularity. 

Even\marginnote{5.1} some of the deities can’t get the bliss of renunciation, the bliss of seclusion, the bliss of peace, the bliss of awakening when they want, without trouble or difficulty like I can. When you all come together to enjoy each other’s company, I think: ‘These venerables mustn’t get the bliss of renunciation, the bliss of seclusion, the bliss of peace, the bliss of awakening when they want, without trouble or difficulty like I can. That must be because they come together to enjoy each other’s company.’ 

Take\marginnote{6.1} mendicants I see poking each other with their fingers, giggling and playing together. I think to myself: ‘These venerables mustn’t get the bliss of renunciation, the bliss of seclusion, the bliss of peace, the bliss of awakening when they want, without trouble or difficulty like I can. That must be because they’re poking each other with their fingers, giggling and playing together.’ 

Take\marginnote{7.1} mendicants I see eat as much as they like until their bellies are full, then indulge in the pleasures of sleeping, lying down, and drowsing. I think to myself: ‘These venerables mustn’t get the bliss of renunciation, the bliss of seclusion, the bliss of peace, the bliss of awakening when they want, without trouble or difficulty like I can. That must be because they eat as much as they like until their bellies are full, then indulge in the pleasures of sleeping, lying down, and drowsing.’ 

Take\marginnote{8.1} a mendicant living within a village who I see sitting immersed in \textsanskrit{samādhi}. I think to myself: ‘Now a monastery worker, a novice, or a fellow practitioner will make this venerable fall from immersion.’ So I’m not pleased that that mendicant is living within a village. 

Take\marginnote{9.1} a mendicant in the wilderness who I see sitting nodding in meditation. I think to myself: ‘Now this venerable, having dispelled that sleepiness and weariness, will focus just on the unified perception of wilderness.’ So I’m pleased that that mendicant is living in the wilderness. 

Take\marginnote{10.1} a mendicant in the wilderness who I see sitting without being immersed in \textsanskrit{samādhi}. I think to myself: ‘Now if this venerable’s mind is not immersed in \textsanskrit{samādhi} they will immerse it; or if it is immersed in \textsanskrit{samādhi}, they will preserve it.’ So I’m pleased that that mendicant is living in the wilderness. 

Take\marginnote{11.1} a mendicant in the wilderness who I see sitting immersed in \textsanskrit{samādhi}. I think to myself: ‘Now this venerable will free the unfreed mind or preserve the freed mind.’ So I’m pleased that that mendicant is living in the wilderness. 

Take\marginnote{12.1} a mendicant who I see living within a village receiving robes, almsfood, lodgings, and medicines and supplies for the sick. Enjoying possessions, honor, and popularity they neglect retreat, and they neglect remote lodgings in the wilderness and the forest. They come down to villages, towns and capital cities and make their homes there. So I’m not pleased that that mendicant is living within a village. 

Take\marginnote{13.1} a mendicant who I see in the wilderness receiving robes, almsfood, lodgings, and medicines and supplies for the sick. Fending off possessions, honor, and popularity they don’t neglect retreat, and they don’t neglect remote lodgings in the wilderness and the forest. So I’m pleased that that mendicant is living in the wilderness. 

\textsanskrit{Nāgita},\marginnote{14.1} when I’m walking along a road and I don’t see anyone ahead or behind I feel relaxed, even if I need to urinate or defecate.” 

%
\section*{{\suttatitleacronym AN 8.87}{\suttatitletranslation Turning the Bowl Upside Down }{\suttatitleroot Pattanikujjanasutta}}
\addcontentsline{toc}{section}{\tocacronym{AN 8.87} \toctranslation{Turning the Bowl Upside Down } \tocroot{Pattanikujjanasutta}}
\markboth{Turning the Bowl Upside Down }{Pattanikujjanasutta}
\extramarks{AN 8.87}{AN 8.87}

“Mendicants,\marginnote{1.1} the \textsanskrit{Saṅgha} may, if it wishes, turn the bowl upside down for a lay follower on eight grounds. What eight? They try to prevent the mendicants from getting material possessions. They try to harm mendicants. They try to drive mendicants from a monastery. They insult and abuse mendicants. They divide mendicants against each other. They criticize the Buddha, the teaching, and the \textsanskrit{Saṅgha}. The \textsanskrit{Saṅgha} may, if it wishes, turn the bowl upside down for a lay follower on these eight grounds. 

The\marginnote{2.1} \textsanskrit{Saṅgha} may, if it wishes, turn the bowl upright for a lay follower on eight grounds. What eight? They don’t try to prevent the mendicants from getting material possessions. They don’t try to harm mendicants. They don’t try to drive mendicants from a monastery. They don’t insult and abuse mendicants. They don’t divide mendicants against each other. They don’t criticize the Buddha, the teaching, and the \textsanskrit{Saṅgha}. The \textsanskrit{Saṅgha} may, if it wishes, turn the bowl upright for a lay follower on these eight grounds.” 

%
\section*{{\suttatitleacronym AN 8.88}{\suttatitletranslation A Proclamation of No Confidence }{\suttatitleroot Appasādapavedanīyasutta}}
\addcontentsline{toc}{section}{\tocacronym{AN 8.88} \toctranslation{A Proclamation of No Confidence } \tocroot{Appasādapavedanīyasutta}}
\markboth{A Proclamation of No Confidence }{Appasādapavedanīyasutta}
\extramarks{AN 8.88}{AN 8.88}

“Mendicants,\marginnote{1.1} the lay followers may, if they wish, make a proclamation of no confidence in a mendicant who has eight qualities. What eight? They try to prevent the lay people from getting material possessions. They try to harm lay people. They insult and abuse lay people. They divide lay people against each other. They criticize the Buddha, the teaching, and the \textsanskrit{Saṅgha}. They’re seen at an inappropriate place for collecting alms. The lay followers may, if they wish, make a proclamation of no confidence in a mendicant who has these eight qualities. 

The\marginnote{2.1} lay followers may, if they wish, make a proclamation of confidence in a mendicant who has eight qualities. What eight? They don’t try to prevent the lay people from getting material possessions. They don’t try to harm lay people. They don’t insult and abuse lay people. They don’t divide lay people against each other. They don’t criticize the Buddha, the teaching, and the \textsanskrit{Saṅgha}. They’re not seen at an inappropriate place for collecting alms. The lay followers may, if they wish, make a proclamation of confidence in a mendicant who has these eight qualities.” 

%
\section*{{\suttatitleacronym AN 8.89}{\suttatitletranslation Reconciliation }{\suttatitleroot Paṭisāraṇīyasutta}}
\addcontentsline{toc}{section}{\tocacronym{AN 8.89} \toctranslation{Reconciliation } \tocroot{Paṭisāraṇīyasutta}}
\markboth{Reconciliation }{Paṭisāraṇīyasutta}
\extramarks{AN 8.89}{AN 8.89}

“Mendicants,\marginnote{1.1} the \textsanskrit{Saṅgha} may, if it wishes, perform an act requiring that a mendicant who has eight qualities should pursue reconciliation. What eight? They try to prevent the lay people from getting material possessions. They try to harm lay people. They insult and abuse lay people. They divide lay people against each other. They criticize the Buddha, the teaching, and the \textsanskrit{Saṅgha}. They don’t keep a legitimate promise made to a lay person. The \textsanskrit{Saṅgha} may, if it wishes, perform an act requiring that a mendicant who has eight qualities should pursue reconciliation. 

The\marginnote{2.1} \textsanskrit{Saṅgha} may, if it wishes, revoke the act requiring that a mendicant who has eight qualities should pursue reconciliation. What eight? They don’t try to prevent the lay people from getting material possessions. They don’t try to harm lay people. They don’t insult and abuse lay people. They don’t divide lay people against each other. They don’t criticize the Buddha, the teaching, and the \textsanskrit{Saṅgha}. They keep a legitimate promise made to a lay person. The \textsanskrit{Saṅgha} may, if it wishes, revoke the act requiring that a mendicant who has eight qualities should pursue reconciliation.” 

%
\section*{{\suttatitleacronym AN 8.90}{\suttatitletranslation Proper Behavior in a Case of Aggravated Misconduct }{\suttatitleroot Sammāvattanasutta}}
\addcontentsline{toc}{section}{\tocacronym{AN 8.90} \toctranslation{Proper Behavior in a Case of Aggravated Misconduct } \tocroot{Sammāvattanasutta}}
\markboth{Proper Behavior in a Case of Aggravated Misconduct }{Sammāvattanasutta}
\extramarks{AN 8.90}{AN 8.90}

“Mendicants,\marginnote{1.1} a mendicant who has been convicted of aggravated misconduct must behave themselves properly in eight respects. They must not perform an ordination, give dependence, or be attended by a novice. They must not consent to being appointed as adviser for nuns, and if they are appointed they should not give such advice. They must not consent to any \textsanskrit{Saṅgha} appointment. They must not be put in an isolated place. They must not give rehabilitation in any offense similar to that which they transgressed. A mendicant who has been convicted of aggravated misconduct must behave themselves properly in these eight respects.” 

%
\addtocontents{toc}{\let\protect\contentsline\protect\nopagecontentsline}
\chapter*{The Chapter on Similarity }
\addcontentsline{toc}{chapter}{\tocchapterline{The Chapter on Similarity }}
\addtocontents{toc}{\let\protect\contentsline\protect\oldcontentsline}

%
\section*{{\suttatitleacronym AN 8.91–117}{\suttatitletranslation Untitled Discourses With Various Laywomen on the Sabbath }{\suttatitleroot Sāmaññavagga}}
\addcontentsline{toc}{section}{\tocacronym{AN 8.91–117} \toctranslation{Untitled Discourses With Various Laywomen on the Sabbath } \tocroot{Sāmaññavagga}}
\markboth{Untitled Discourses With Various Laywomen on the Sabbath }{Sāmaññavagga}
\extramarks{AN 8.91–117}{AN 8.91–117}

And\marginnote{1.1} then the lay woman \textsanskrit{Bojjhā} … \textsanskrit{Sirīmā} … \textsanskrit{Padumā} … \textsanskrit{Sutanā} … \textsanskrit{Manujā} … \textsanskrit{Uttarā} … \textsanskrit{Muttā} … \textsanskrit{Khemā} … \textsanskrit{Somā} … \textsanskrit{Rucī} … \textsanskrit{Cundī} … \textsanskrit{Bimbī} … \textsanskrit{Sumanā} … \textsanskrit{Mallikā} … \textsanskrit{Tissā} … \textsanskrit{Tissamātā} … \textsanskrit{Soṇā} … \textsanskrit{Soṇā}’s mother … \textsanskrit{Kāṇā} … \textsanskrit{Kāṇamātā} … \textsanskrit{Uttarā} Nanda’s mother … \textsanskrit{Visākhā} \textsanskrit{Migāra}’s mother … the lay woman \textsanskrit{Khujjuttarā} … the lay woman \textsanskrit{Sāmāvatī} … \textsanskrit{Suppavāsā} the Koliyan … the lay woman \textsanskrit{Suppiyā} … the housewife Nakula’s mother … 

%
\addtocontents{toc}{\let\protect\contentsline\protect\nopagecontentsline}
\chapter*{Abbreviated Texts Beginning With Greed }
\addcontentsline{toc}{chapter}{\tocchapterline{Abbreviated Texts Beginning With Greed }}
\addtocontents{toc}{\let\protect\contentsline\protect\oldcontentsline}

%
\section*{{\suttatitleacronym AN 8.118}{\suttatitletranslation Untitled Discourse on Greed (1st) }{\suttatitleroot \textasciitilde }}
\addcontentsline{toc}{section}{\tocacronym{AN 8.118} \toctranslation{Untitled Discourse on Greed (1st) } \tocroot{\textasciitilde }}
\markboth{Untitled Discourse on Greed (1st) }{\textasciitilde }
\extramarks{AN 8.118}{AN 8.118}

“For\marginnote{1.1} insight into greed, eight things should be developed. What eight? Right view, right thought, right speech, right action, right livelihood, right effort, right mindfulness, and right immersion. For insight into greed, these eight things should be developed.” 

%
\section*{{\suttatitleacronym AN 8.119}{\suttatitletranslation Untitled Discourse on Greed (2nd) }{\suttatitleroot \textasciitilde }}
\addcontentsline{toc}{section}{\tocacronym{AN 8.119} \toctranslation{Untitled Discourse on Greed (2nd) } \tocroot{\textasciitilde }}
\markboth{Untitled Discourse on Greed (2nd) }{\textasciitilde }
\extramarks{AN 8.119}{AN 8.119}

“For\marginnote{1.1} insight into greed, eight things should be developed. What eight? Perceiving form internally, they see visions externally, limited, both pretty and ugly. Mastering them, they perceive: ‘I know and see.’ Perceiving form internally, they see visions externally, limitless, both pretty and ugly. … Not perceiving form internally, they see visions externally, limited, both pretty and ugly. … Not perceiving form internally, they see visions externally, limitless, both pretty and ugly. … Not perceiving form internally, they see visions externally, blue, with blue color, blue hue, and blue tint. … yellow … red … Not perceiving form internally, they see visions externally, white, with white color, white hue, and white tint. Mastering them, they perceive: ‘I know and see.’ For insight into greed, these eight things should be developed.” 

%
\section*{{\suttatitleacronym AN 8.120}{\suttatitletranslation Untitled Discourse on Greed (3rd) }{\suttatitleroot \textasciitilde }}
\addcontentsline{toc}{section}{\tocacronym{AN 8.120} \toctranslation{Untitled Discourse on Greed (3rd) } \tocroot{\textasciitilde }}
\markboth{Untitled Discourse on Greed (3rd) }{\textasciitilde }
\extramarks{AN 8.120}{AN 8.120}

“For\marginnote{1.1} insight into greed, eight things should be developed. What eight? Having physical form, they see visions … not perceiving form internally, they see visions externally … they’re focused only on beauty … going totally beyond perceptions of form, with the ending of perceptions of impingement, not focusing on perceptions of diversity, aware that ‘space is infinite’, they enter and remain in the dimension of infinite space … going totally beyond the dimension of infinite space, aware that ‘consciousness is infinite’, they enter and remain in the dimension of infinite consciousness … going totally beyond the dimension of infinite consciousness, aware that ‘there is nothing at all’, they enter and remain in the dimension of nothingness … going totally beyond the dimension of nothingness, they enter and remain in the dimension of neither perception nor non-perception … going totally beyond the dimension of neither perception nor non-perception, they enter and remain in the cessation of perception and feeling … For insight into greed, these eight things should be developed.” 

%
\section*{{\suttatitleacronym AN 8.121–147}{\suttatitletranslation Untitled Discourses on Greed }{\suttatitleroot \textasciitilde }}
\addcontentsline{toc}{section}{\tocacronym{AN 8.121–147} \toctranslation{Untitled Discourses on Greed } \tocroot{\textasciitilde }}
\markboth{Untitled Discourses on Greed }{\textasciitilde }
\extramarks{AN 8.121–147}{AN 8.121–147}

“For\marginnote{1.1} the complete understanding of greed … complete ending … giving up … ending … vanishing … fading away … cessation … giving away … letting go … these eight things should be developed.” 

%
\section*{{\suttatitleacronym AN 8.148–627}{\suttatitletranslation Untitled Discourses on Hate, Etc. }{\suttatitleroot \textasciitilde }}
\addcontentsline{toc}{section}{\tocacronym{AN 8.148–627} \toctranslation{Untitled Discourses on Hate, Etc. } \tocroot{\textasciitilde }}
\markboth{Untitled Discourses on Hate, Etc. }{\textasciitilde }
\extramarks{AN 8.148–627}{AN 8.148–627}

“Of\marginnote{1.1} hate … delusion … anger … hostility … disdain … contempt … jealousy … stinginess … deceitfulness … deviousness … obstinacy … aggression … conceit … arrogance … vanity … for insight into negligence … complete understanding … complete ending … giving up … ending … vanishing … fading away … cessation … giving away … letting go of negligence these eight things should be developed.” 

\scendbook{The Book of the Eights is finished. }

%
\addtocontents{toc}{\let\protect\contentsline\protect\nopagecontentsline}
\part*{The Book of the Nines }
\addcontentsline{toc}{part}{The Book of the Nines }
\markboth{}{}
\addtocontents{toc}{\let\protect\contentsline\protect\oldcontentsline}

%
%
\addtocontents{toc}{\let\protect\contentsline\protect\nopagecontentsline}
\pannasa{The First Fifty }
\addcontentsline{toc}{pannasa}{The First Fifty }
\markboth{}{}
\addtocontents{toc}{\let\protect\contentsline\protect\oldcontentsline}

%
\addtocontents{toc}{\let\protect\contentsline\protect\nopagecontentsline}
\chapter*{The Chapter on Awakening }
\addcontentsline{toc}{chapter}{\tocchapterline{The Chapter on Awakening }}
\addtocontents{toc}{\let\protect\contentsline\protect\oldcontentsline}

%
\section*{{\suttatitleacronym AN 9.1}{\suttatitletranslation Awakening }{\suttatitleroot Sambodhisutta}}
\addcontentsline{toc}{section}{\tocacronym{AN 9.1} \toctranslation{Awakening } \tocroot{Sambodhisutta}}
\markboth{Awakening }{Sambodhisutta}
\extramarks{AN 9.1}{AN 9.1}

\scevam{So\marginnote{1.1} I have heard. }At one time the Buddha was staying near \textsanskrit{Sāvatthī} in Jeta’s Grove, \textsanskrit{Anāthapiṇḍika}’s monastery. There the Buddha addressed the mendicants: 

“Mendicants,\marginnote{2.1} if wanderers who follow other paths were to ask: ‘Reverends, what is the vital condition for the development of the awakening factors?’ How would you answer them?” 

“Our\marginnote{2.3} teachings are rooted in the Buddha. … The mendicants will listen and remember it.” 

“Well\marginnote{3.1} then, mendicants, listen and pay close attention, I will speak.” 

“Yes,\marginnote{3.2} sir,” they replied. The Buddha said this: 

“Mendicants,\marginnote{4.1} if wanderers who follow other paths were to ask: ‘Reverends, what is the vital condition for the development of the awakening factors?’ You should answer them: 

‘It’s\marginnote{5.1} when a mendicant has good friends, companions, and associates. This is the first vital condition for the development of the awakening factors. 

Furthermore,\marginnote{6.1} a mendicant is ethical, restrained in the monastic code, conducting themselves well and seeking alms in suitable places. Seeing danger in the slightest fault, they keep the rules they’ve undertaken. This is the second vital condition for the development of the awakening factors. 

Furthermore,\marginnote{7.1} a mendicant gets to take part in talk about self-effacement that helps open the heart, when they want, without trouble or difficulty. That is, talk about fewness of wishes, contentment, seclusion, aloofness, arousing energy, ethics, immersion, wisdom, freedom, and the knowledge and vision of freedom. This is the third vital condition for the development of the awakening factors. 

Furthermore,\marginnote{8.1} a mendicant lives with energy roused up for giving up unskillful qualities and embracing skillful qualities. They are strong, staunchly vigorous, not slacking off when it comes to developing skillful qualities. This is the fourth vital condition for the development of the awakening factors. 

Furthermore,\marginnote{9.1} a mendicant is wise. They have the wisdom of arising and passing away which is noble, penetrative, and leads to the complete ending of suffering. This is the fifth vital condition for the development of the awakening factors.’ 

A\marginnote{10.1} mendicant with good friends, companions, and associates can expect to be ethical … 

A\marginnote{11.1} mendicant with good friends, companions, and associates can expect to take part in talk about self-effacement that helps open the heart … 

A\marginnote{12.1} mendicant with good friends, companions, and associates can expect to live with energy roused up … 

A\marginnote{13.1} mendicant with good friends, companions, and associates can expect to be wise … 

But\marginnote{14.1} then, a mendicant grounded on these five things should develop four further things. They should develop the perception of ugliness to give up greed, love to give up hate, mindfulness of breathing to cut off thinking, and perception of impermanence to uproot the conceit ‘I am’. When you perceive impermanence, the perception of not-self becomes stabilized. Perceiving not-self, you uproot the conceit ‘I am’ and attain extinguishment in this very life.” 

%
\section*{{\suttatitleacronym AN 9.2}{\suttatitletranslation Supported }{\suttatitleroot Nissayasutta}}
\addcontentsline{toc}{section}{\tocacronym{AN 9.2} \toctranslation{Supported } \tocroot{Nissayasutta}}
\markboth{Supported }{Nissayasutta}
\extramarks{AN 9.2}{AN 9.2}

Then\marginnote{1.1} a mendicant went up to the Buddha, bowed, sat down to one side, and said to him: 

“Sir,\marginnote{1.2} they speak of being ‘supported’. How is a mendicant who is supported defined?” 

“Mendicant,\marginnote{1.4} if a mendicant supported by faith gives up the unskillful and develops the skillful, the unskillful is actually given up by them. 

If\marginnote{1.5} a mendicant supported by conscience … 

If\marginnote{1.6} a mendicant supported by prudence … 

If\marginnote{1.7} a mendicant supported by energy … 

If\marginnote{1.8} a mendicant supported by wisdom gives up the unskillful and develops the skillful, the unskillful is actually given up by them. What’s been given up is completely given up when it has been given up by seeing with noble wisdom. 

But\marginnote{2.1} then, a mendicant grounded on these five things should rely on four things. What four? After appraisal, a mendicant uses some things, endures some things, avoids some things, and gets rid of some things. That’s how a mendicant is supported.” 

%
\section*{{\suttatitleacronym AN 9.3}{\suttatitletranslation With Meghiya }{\suttatitleroot Meghiyasutta}}
\addcontentsline{toc}{section}{\tocacronym{AN 9.3} \toctranslation{With Meghiya } \tocroot{Meghiyasutta}}
\markboth{With Meghiya }{Meghiyasutta}
\extramarks{AN 9.3}{AN 9.3}

At\marginnote{1.1} one time the Buddha was staying near \textsanskrit{Cālikā}, on the \textsanskrit{Cālikā} mountain. 

Now,\marginnote{1.2} at that time Venerable Meghiya was the Buddha’s attendant. Then Venerable Meghiya went up to the Buddha, bowed, stood to one side, and said to him, “Sir, I’d like to enter Jantu village for alms.” 

“Please,\marginnote{1.5} Meghiya, go at your convenience.” 

Then\marginnote{2.1} Meghiya robed up in the morning and, taking his bowl and robe, entered Jantu village for alms. After the meal, on his return from almsround in Jantu village, he went to the shore of \textsanskrit{Kimikālā} river. As he was going for a walk along the shore of the river he saw a lovely and delightful mango grove. 

It\marginnote{2.4} occurred to him, “Oh, this mango grove is lovely and delightful! It’s truly good enough for meditation for a gentleman who wants to meditate. If the Buddha allows me, I’ll come back to this mango grove to meditate.” 

Then\marginnote{3.1} Venerable Meghiya went up to the Buddha, bowed, sat down to one side, and told him what had happened, adding, “If the Buddha allows me, I’ll go back to that mango grove to meditate.” 

“We’re\marginnote{4.9} alone, Meghiya. Wait until another mendicant comes.” 

For\marginnote{5.1} a second time Meghiya said to the Buddha, “Sir, the Buddha has nothing more to do, and nothing that needs improvement. But I have. If you allow me, I’ll go back to that mango grove to meditate.” 

“We’re\marginnote{5.5} alone, Meghiya. Wait until another mendicant comes.” 

For\marginnote{6.1} a third time Meghiya said to the Buddha, “Sir, the Buddha has nothing more to do, and nothing that needs improvement. But I have. If you allow me, I’ll go back to that mango grove to meditate.” 

“Meghiya,\marginnote{6.5} since you speak of meditation, what can I say? Please, Meghiya, go at your convenience.” 

Then\marginnote{7.1} Meghiya got up from his seat, bowed, and respectfully circled the Buddha, keeping him on his right. Then he went to that mango grove, and, having plunged deep into it, sat at the root of a certain tree for the day’s meditation. But while Meghiya was meditating in that mango grove he was beset mostly by three kinds of bad, unskillful thoughts, namely, sensual, malicious, and cruel thoughts. 

Then\marginnote{7.4} he thought, “It’s incredible, it’s amazing! I’ve gone forth from the lay life to homelessness out of faith, but I’m still harassed by these three kinds of bad, unskillful thoughts: sensual, malicious, and cruel thoughts.” 

Then\marginnote{8.1} Venerable Meghiya went up to the Buddha, bowed, sat down to one side, and told him what had happened. 

“Meghiya,\marginnote{10.1} when the heart’s release is not ripe, five things help it ripen. What five? 

Firstly,\marginnote{10.3} a mendicant has good friends, companions, and associates. This is the first thing … 

Furthermore,\marginnote{11.1} a mendicant is ethical, restrained in the monastic code, conducting themselves well and seeking alms in suitable places. Seeing danger in the slightest fault, they keep the rules they’ve undertaken. This is the second thing … 

Furthermore,\marginnote{12.1} a mendicant gets to take part in talk about self-effacement that helps open the heart, when they want, without trouble or difficulty. That is, talk about fewness of wishes, contentment, seclusion, aloofness, arousing energy, ethics, immersion, wisdom, freedom, and the knowledge and vision of freedom. This is the third thing … 

Furthermore,\marginnote{13.1} a mendicant lives with energy roused up for giving up unskillful qualities and embracing skillful qualities. They are strong, staunchly vigorous, not slacking off when it comes to developing skillful qualities. This is the fourth thing … 

Furthermore,\marginnote{14.1} a mendicant is wise. They have the wisdom of arising and passing away which is noble, penetrative, and leads to the complete ending of suffering. This is the fifth thing that, when the heart’s release is not ripe, helps it ripen. 

A\marginnote{15.1} mendicant with good friends, companions, and associates can expect to be ethical … 

A\marginnote{16.1} mendicant with good friends, companions, and associates can expect to take part in talk about self-effacement that helps open the heart … 

A\marginnote{17.1} mendicant with good friends, companions, and associates can expect to be energetic … 

A\marginnote{18.1} mendicant with good friends, companions, and associates can expect to be wise … 

But\marginnote{19.1} then, a mendicant grounded on these five things should develop four further things. They should develop the perception of ugliness to give up greed, love to give up hate, mindfulness of breathing to cut off thinking, and perception of impermanence to uproot the conceit ‘I am’. When you perceive impermanence, the perception of not-self becomes stabilized. Perceiving not-self, you uproot the conceit ‘I am’ and attain extinguishment in this very life.” 

%
\section*{{\suttatitleacronym AN 9.4}{\suttatitletranslation With Nandaka }{\suttatitleroot Nandakasutta}}
\addcontentsline{toc}{section}{\tocacronym{AN 9.4} \toctranslation{With Nandaka } \tocroot{Nandakasutta}}
\markboth{With Nandaka }{Nandakasutta}
\extramarks{AN 9.4}{AN 9.4}

At\marginnote{1.1} one time the Buddha was staying near \textsanskrit{Sāvatthī} in Jeta’s Grove, \textsanskrit{Anāthapiṇḍika}’s monastery. 

Now\marginnote{1.2} at that time Venerable Nandaka was educating, encouraging, firing up, and inspiring the mendicants in the assembly hall with a Dhamma talk. Then in the late afternoon, the Buddha came out of retreat and went to the assembly hall. He stood outside the door waiting for the talk to end. When he knew the talk had ended he cleared his throat and knocked with the latch. The mendicants opened the door for the Buddha, and he entered the assembly hall, where he sat on the seat spread out. 

He\marginnote{2.2} said to Nandaka, “Nandaka, that was a long exposition of the teaching you gave to the mendicants. My back was aching while I stood outside the door waiting for the talk to end.” 

When\marginnote{3.1} he said this, Nandaka felt embarrassed and said to the Buddha, “Sir, we didn’t know that the Buddha was standing outside the door. If we’d known, I wouldn’t have said so much.” 

Then\marginnote{4.1} the Buddha, knowing that Nandaka was embarrassed, said to him, “Good, good, Nandaka! It’s appropriate for gentlemen like you, who have gone forth in faith from the lay life to homelessness, to sit together for a Dhamma talk. When you’re sitting together you should do one of two things: discuss the teachings or keep noble silence. 

Nandaka,\marginnote{4.6} a mendicant is faithful but not ethical. So they’re incomplete in that respect, and should fulfill it, thinking, ‘How can I become faithful and ethical?’ When a mendicant is faithful and ethical, they’re complete in that respect. 

A\marginnote{5.1} mendicant is faithful and ethical, but does not get internal serenity of heart. So they’re incomplete in that respect, and should fulfill it, thinking, ‘How can I become faithful and ethical and get internal serenity of heart?’ When a mendicant is faithful and ethical and gets internal serenity of heart, they’re complete in that respect. 

A\marginnote{6.1} mendicant is faithful, ethical, and gets internal serenity of heart, but they don’t get the higher wisdom of discernment of principles. So they’re incomplete in that respect. Suppose, Nandaka, there was a four-footed animal that was lame and disabled. It would be incomplete in that respect. In the same way, a mendicant is faithful, ethical, and gets internal serenity of heart, but they don’t get the higher wisdom of discernment of principles. So they’re incomplete in that respect, and should fulfill it, thinking, ‘How can I become faithful and ethical and get internal serenity of heart and get the higher wisdom of discernment of principles?’ 

When\marginnote{7.1} a mendicant is faithful and ethical and gets internal serenity of heart and gets the higher wisdom of discernment of principles, they’re complete in that respect.” 

That\marginnote{7.2} is what the Buddha said. When he had spoken, the Holy One got up from his seat and entered his dwelling. 

Then\marginnote{8.1} soon after the Buddha left, Venerable Nandaka said to the mendicants, “Just now, reverends, the Buddha explained a spiritual practice that’s entirely full and pure in four statements, before getting up from his seat and entering his dwelling: 

‘Nandaka,\marginnote{8.3} a mendicant is faithful but not ethical. So they’re incomplete in that respect, and should fulfill it, thinking, “How can I become faithful and ethical?” When a mendicant is faithful and ethical, they’re complete in that respect. 

A\marginnote{8.8} mendicant is faithful and ethical, but does not get internal serenity of heart. … 

They\marginnote{8.9} get internal serenity of heart, but they don’t get the higher wisdom of discernment of principles. So they’re incomplete in that respect. Suppose, Nandaka, there was a four-footed animal that was lame and disabled. It would be incomplete in that respect. In the same way, a mendicant is faithful, ethical, and gets internal serenity of heart, but they don’t get the higher wisdom of discernment of principles. So they’re incomplete in that respect, and should fulfill it, thinking: “How can I become faithful and ethical and get internal serenity of heart and get the higher wisdom of discernment of principles?” When a mendicant is faithful and ethical and gets internal serenity of heart and gets the higher wisdom of discernment of principles, they’re complete in that respect.’ 

Reverends,\marginnote{9.1} there are these five benefits of listening to the teachings at the right time and discussing the teachings at the right time. What five? 

Firstly,\marginnote{9.3} a mendicant teaches the mendicants the Dhamma that’s good in the beginning, good in the middle, and good in the end, meaningful and well-phrased. And they reveal a spiritual practice that’s entirely full and pure. Whenever they do this, they become liked and approved by the Teacher, respected and admired. This is the first benefit … 

Furthermore,\marginnote{10.1} a mendicant teaches the mendicants the Dhamma … Whenever they do this, they feel inspired by the meaning and the teaching in that Dhamma. This is the second benefit … 

Furthermore,\marginnote{11.1} a mendicant teaches the mendicants the Dhamma … Whenever they do this, they see the meaning of a deep saying in that Dhamma with penetrating wisdom. This is the third benefit … 

Furthermore,\marginnote{12.1} a mendicant teaches the mendicants the Dhamma … Whenever they do this, their spiritual companions esteem them more highly, thinking, ‘For sure this venerable has attained or will attain.’ This is the fourth benefit … 

Furthermore,\marginnote{13.1} a mendicant teaches the mendicants the Dhamma … Whenever they do this, there may be trainee mendicants present, who haven’t achieved their heart’s desire, but live aspiring to the supreme sanctuary. Hearing that teaching, they rouse energy for attaining the unattained, achieving the unachieved, and realizing the unrealized. There may be perfected mendicants present, who have ended the defilements, completed the spiritual journey, done what had to be done, laid down the burden, achieved their own goal, utterly ended the fetters of rebirth, and are rightly freed through enlightenment. Hearing that teaching, they simply live happily in the present life. This is the fifth benefit … 

These\marginnote{13.5} are the five benefits of listening to the teachings at the right time and discussing the teachings at the right time.” 

%
\section*{{\suttatitleacronym AN 9.5}{\suttatitletranslation Powers }{\suttatitleroot Balasutta}}
\addcontentsline{toc}{section}{\tocacronym{AN 9.5} \toctranslation{Powers } \tocroot{Balasutta}}
\markboth{Powers }{Balasutta}
\extramarks{AN 9.5}{AN 9.5}

“Mendicants,\marginnote{1.1} there are these four powers. What four? The powers of wisdom, energy, blamelessness, and inclusiveness. 

And\marginnote{1.4} what is the power of wisdom? One has clearly seen and clearly contemplated with wisdom those qualities that are skillful and considered to be skillful; those that are unskillful … blameworthy … blameless … dark … bright … to be cultivated … not to be cultivated … not worthy of the noble ones … worthy of the noble ones and considered to be worthy of the noble ones. This is called the power of wisdom. 

And\marginnote{2.1} what is the power of energy? One generates enthusiasm, tries, makes an effort, exerts the mind, and strives to give up those qualities that are unskillful and considered to be unskillful; those that are blameworthy … dark … not to be cultivated … not worthy of the noble ones and considered to be not worthy of the noble ones. One generates enthusiasm, tries, makes an effort, exerts the mind, and strives to gain those qualities that are skillful and considered to be skillful; those that are blameless … bright … to be cultivated … worthy of the noble ones and considered to be worthy of the noble ones. This is called the power of energy. 

And\marginnote{3.1} what is the power of blamelessness? It’s when a mendicant has blameless conduct by way of body, speech, and mind. This is called the power of blamelessness. 

And\marginnote{4.1} what is the power of inclusiveness? There are these four ways of being inclusive. Giving, kindly words, taking care, and equality. The best of gifts is the gift of the teaching. The best sort of kindly speech is to teach the Dhamma again and again to someone who is engaged and who lends an ear. The best way of taking care is to encourage, settle, and ground the unfaithful in faith, the unethical in ethics, the stingy in generosity, and the ignorant in wisdom. The best kind of equality is the equality of a stream-enterer with another stream-enterer, a once-returner with another once-returner, a non-returner with another non-returner, and a perfected one with another perfected one. This is called the power of inclusiveness. These are the four powers. 

A\marginnote{5.1} noble disciple who has these four powers has got past five fears. What five? Fear regarding livelihood, disrepute, feeling insecure in an assembly, death, and bad rebirth. 

Then\marginnote{5.4} that noble disciple reflects: ‘I have no fear regarding livelihood. Why would I be afraid of that? I have these four powers: the powers of wisdom, energy, blamelessness, and inclusiveness. A witless person might fear for their livelihood. A lazy person might fear for their livelihood. A person who does blameworthy things by way of body, speech, and mind might fear for their livelihood. A person who does not include others might fear for their livelihood. I have no fear of disrepute … I have no fear about feeling insecure in an assembly … I have no fear of death … I have no fear of a bad rebirth. Why would I be afraid of that? I have these four powers: the powers of wisdom, energy, blamelessness, and inclusiveness. A witless person might be afraid of a bad rebirth. A lazy person might be afraid of a bad rebirth. A person who does blameworthy things by way of body, speech, and mind might be afraid of a bad rebirth. A person who does not include others might be afraid of a bad rebirth.’ 

A\marginnote{5.24} noble disciple who has these four powers has got past these five fears.” 

%
\section*{{\suttatitleacronym AN 9.6}{\suttatitletranslation Association }{\suttatitleroot Sevanāsutta}}
\addcontentsline{toc}{section}{\tocacronym{AN 9.6} \toctranslation{Association } \tocroot{Sevanāsutta}}
\markboth{Association }{Sevanāsutta}
\extramarks{AN 9.6}{AN 9.6}

There\marginnote{1.1} \textsanskrit{Sāriputta} addressed the mendicants: 

“Reverends,\marginnote{2.1} you should distinguish two kinds of people: those you should associate with, and those you shouldn’t associate with. You should distinguish two kinds of robes: those you should wear, and those you shouldn’t wear. You should distinguish two kinds of almsfood: that which you should eat, and that which you shouldn’t eat. You should distinguish two kinds of lodging: those you should frequent, and those you shouldn’t frequent. You should distinguish two kinds of market town: those you should frequent, and those you shouldn’t frequent. You should distinguish two kinds of country: those you should frequent, and those you shouldn’t frequent. 

You\marginnote{3.1} should distinguish two kinds of people: those you should associate with, and those you shouldn’t associate with.’ That’s what I said, but why did I say it? Well, should you know of a person: ‘When I associate with this person, unskillful qualities grow, and skillful qualities decline. And the necessities of life that a renunciate requires—robes, almsfood, lodgings, and medicines and supplies for the sick—are hard to come by. And the goal of the ascetic life for which I went forth from the lay life to homelessness is not being fully developed.’ In this case you should leave that person at that very time of the day or night, without asking. You shouldn’t follow them. 

Whereas,\marginnote{4.1} should you know of a person: ‘When I associate with this person, unskillful qualities grow, and skillful qualities decline. But the necessities of life that a renunciate requires—robes, almsfood, lodgings, and medicines and supplies for the sick—are easy to come by. However, the goal of the ascetic life for which I went forth from the lay life to homelessness is not being fully developed.’ In this case you should leave that person after reflecting, without asking. You shouldn’t follow them. 

Well,\marginnote{5.1} should you know of a person: ‘When I associate with this person, unskillful qualities decline, and skillful qualities grow. And the necessities of life that a renunciate requires—robes, almsfood, lodgings, and medicines and supplies for the sick—are hard to come by. But the goal of the ascetic life for which I went forth from the lay life to homelessness is being fully developed.’ In this case you should follow that person after appraisal. You shouldn’t leave them. 

Whereas,\marginnote{6.1} should you know of a person: ‘When I associate with this person, unskillful qualities decline, and skillful qualities grow. And the necessities of life that a renunciate requires—robes, almsfood, lodgings, and medicines and supplies for the sick—are easy to come by. And the goal of the ascetic life for which I went forth from the lay life to homelessness is being fully developed.’ In this case you should follow that person. You shouldn’t leave them, even if they send you away. ‘You should distinguish two kinds of people: those you should associate with, and those you shouldn’t associate with.’ That’s what I said, and this is why I said it. 

‘You\marginnote{7.1} should distinguish two kinds of robes: those you should wear, and those you shouldn’t wear.’ That’s what I said, but why did I say it? Well, should you know of a robe: ‘When I wear this robe, unskillful qualities grow, and skillful qualities decline.’ You should not wear that kind of robe. Whereas, should you know of a robe: ‘When I wear this robe, unskillful qualities decline, and skillful qualities grow.’ You should wear that kind of robe. ‘You should distinguish two kinds of robes: those you should wear, and those you shouldn’t wear.’ That’s what I said, and this is why I said it. 

‘You\marginnote{8.1} should distinguish two kinds of almsfood: that which you should eat, and that which you shouldn’t eat.’ That’s what I said, but why did I say it? Well, should you know of almsfood: ‘When I eat this almsfood, unskillful qualities grow, and skillful qualities decline.’ You should not eat that kind of almsfood. Whereas, should you know of almsfood: ‘When I eat this almsfood, unskillful qualities decline, and skillful qualities grow.’ You should eat that kind of almsfood. ‘You should distinguish two kinds of almsfood: that which you should eat, and that which you shouldn’t eat.’ That’s what I said, and this is why I said it. 

‘You\marginnote{9.1} should distinguish two kinds of lodging: those you should frequent, and those you shouldn’t frequent.’ That’s what I said, but why did I say it? Well, should you know of a lodging: ‘When I frequent this lodging, unskillful qualities grow, and skillful qualities decline.’ You should not frequent that kind of lodging. Whereas, should you know of a lodging: ‘When I frequent this lodging, unskillful qualities decline, and skillful qualities grow.’ You should frequent that kind of lodging. ‘You should distinguish two kinds of lodging: those you should frequent, and those you shouldn’t frequent.’ That’s what I said, and this is why I said it. 

‘You\marginnote{10.1} should distinguish two kinds of market town: those you should frequent, and those you shouldn’t frequent.’ That’s what I said, but why did I say it? Well, should you know of a market town: ‘When I frequent this market town, unskillful qualities grow, and skillful qualities decline.’ You should not frequent that kind of village or town. Whereas, should you know of a market town: ‘When I frequent this market town, unskillful qualities decline, and skillful qualities grow.’ You should frequent that kind of village or town. ‘You should distinguish two kinds of market town: those you should frequent, and those you shouldn’t frequent.’ That’s what I said, and this is why I said it. 

‘You\marginnote{11.1} should distinguish two kinds of country: those you should frequent, and those you shouldn’t frequent.’ That’s what I said, but why did I say it? Well, should you know of a country: ‘When I frequent this country, unskillful qualities grow, and skillful qualities decline.’ You should not frequent that kind of country. Whereas, should you know of a country: ‘When I frequent this country, unskillful qualities decline, and skillful qualities grow.’ You should frequent that kind of country. ‘You should distinguish two kinds of country: those you should frequent, and those you shouldn’t frequent.’ That’s what I said, and this is why I said it.” 

%
\section*{{\suttatitleacronym AN 9.7}{\suttatitletranslation With Sutavā the Wanderer }{\suttatitleroot Sutavāsutta}}
\addcontentsline{toc}{section}{\tocacronym{AN 9.7} \toctranslation{With Sutavā the Wanderer } \tocroot{Sutavāsutta}}
\markboth{With Sutavā the Wanderer }{Sutavāsutta}
\extramarks{AN 9.7}{AN 9.7}

At\marginnote{1.1} one time the Buddha was staying near \textsanskrit{Rājagaha}, on the Vulture’s Peak Mountain. Then the wanderer \textsanskrit{Sutavā} went up to the Buddha, and exchanged greetings with him. When the greetings and polite conversation were over, he sat down to one side and said to the Buddha: 

“Sir,\marginnote{2.1} one time the Buddha was staying right here in \textsanskrit{Rājagaha}, the Mountainfold. There I heard and learned this in the presence of the Buddha: ‘A mendicant who is perfected—with defilements ended, who has completed the spiritual journey, done what had to be done, laid down the burden, achieved their own true goal, utterly ended the fetters of rebirth, and is rightly freed through enlightenment—can’t transgress in five respects. A mendicant with defilements ended can’t deliberately take the life of a living creature, take something with the intention to steal, have sex, tell a deliberate lie, or store up goods for their own enjoyment like they did as a lay person.’ I trust I properly heard, learned, attended, and remembered that from the Buddha?” 

“Indeed,\marginnote{3.1} \textsanskrit{Sutavā}, you properly heard, learned, attended, and remembered that. In the past, as today, I say this: ‘A mendicant who is perfected—with defilements ended, who has completed the spiritual journey, done what had to be done, laid down the burden, achieved their own true goal, utterly ended the fetters of rebirth, and is rightly freed through enlightenment—can’t transgress in nine respects. A mendicant with defilements ended can’t deliberately take the life of a living creature, take something with the intention to steal, have sex, tell a deliberate lie, or store up goods for their own enjoyment like they did as a lay person. And they can’t make decisions prejudiced by favoritism, hostility, stupidity, or cowardice.’ In the past, as today, I say this: ‘A mendicant who is perfected—with defilements ended, who has completed the spiritual journey, done what had to be done, laid down the burden, achieved their own true goal, utterly ended the fetters of rebirth, and is rightly freed through enlightenment—can’t transgress in these nine respects.’” 

%
\section*{{\suttatitleacronym AN 9.8}{\suttatitletranslation With the Wanderer Sajjha }{\suttatitleroot Sajjhasutta}}
\addcontentsline{toc}{section}{\tocacronym{AN 9.8} \toctranslation{With the Wanderer Sajjha } \tocroot{Sajjhasutta}}
\markboth{With the Wanderer Sajjha }{Sajjhasutta}
\extramarks{AN 9.8}{AN 9.8}

At\marginnote{1.1} one time the Buddha was staying near \textsanskrit{Rājagaha}, on the Vulture’s Peak Mountain. Then the wanderer Sajjha went up to the Buddha, and exchanged greetings with him. When the greetings and polite conversation were over, he sat down to one side and said to the Buddha: 

“Sir,\marginnote{2.1} one time the Buddha was staying right here in \textsanskrit{Rājagaha}, the Mountainfold. There I heard and learned this in the presence of the Buddha: ‘A mendicant who is perfected—with defilements ended, who has completed the spiritual journey, done what had to be done, laid down the burden, achieved their own true goal, utterly ended the fetters of rebirth, and is rightly freed through enlightenment—can’t transgress in five respects. A mendicant with defilements ended can’t deliberately take the life of a living creature, take something with the intention to steal, have sex, tell a deliberate lie, or store up goods for their own enjoyment like they did as a lay person.’ I trust I properly heard, learned, attended, and remembered that from the Buddha?” 

“Indeed,\marginnote{3.1} Sajjha, you properly heard, learned, attended, and remembered that. In the past, as today, I say this: ‘A mendicant who is perfected—with defilements ended, who has completed the spiritual journey, done what had to be done, laid down the burden, achieved their own true goal, utterly ended the fetters of rebirth, and is rightly freed through enlightenment—can’t transgress in nine respects. A mendicant with defilements ended can’t deliberately kill a living creature, take something with the intention to steal, have sex, tell a deliberate lie, or store up goods for their own enjoyment like they did as a lay person. And they can’t abandon the Buddha, the teaching, the \textsanskrit{Saṅgha}, or the training.’ In the past, as today, I say this: ‘A mendicant who is perfected—with defilements ended, who has completed the spiritual journey, done what had to be done, laid down the burden, achieved their own true goal, utterly ended the fetters of rebirth, and is rightly freed through enlightenment—can’t transgress in these nine respects.’” 

%
\section*{{\suttatitleacronym AN 9.9}{\suttatitletranslation Persons }{\suttatitleroot Puggalasutta}}
\addcontentsline{toc}{section}{\tocacronym{AN 9.9} \toctranslation{Persons } \tocroot{Puggalasutta}}
\markboth{Persons }{Puggalasutta}
\extramarks{AN 9.9}{AN 9.9}

“Mendicants,\marginnote{1.1} these nine people are found in the world. What nine? The perfected one and the one practicing for perfection. The non-returner and the one practicing to realize the fruit of non-return. The once-returner and the one practicing to realize the fruit of once-return. The stream-enterer and the one practicing to realize the fruit of stream-entry. And the ordinary person. These are the nine people found in the world.” 

%
\section*{{\suttatitleacronym AN 9.10}{\suttatitletranslation Worthy of Offerings Dedicated to the Gods }{\suttatitleroot Āhuneyyasutta}}
\addcontentsline{toc}{section}{\tocacronym{AN 9.10} \toctranslation{Worthy of Offerings Dedicated to the Gods } \tocroot{Āhuneyyasutta}}
\markboth{Worthy of Offerings Dedicated to the Gods }{Āhuneyyasutta}
\extramarks{AN 9.10}{AN 9.10}

“Mendicants,\marginnote{1.1} these nine people are worthy of offerings dedicated to the gods, worthy of hospitality, worthy of a religious donation, worthy of greeting with joined palms, and are the supreme field of merit for the world. What nine? The perfected one and the one practicing for perfection. The non-returner and the one practicing to realize the fruit of non-return. The once-returner and the one practicing to realize the fruit of once-return. The stream-enterer and the one practicing to realize the fruit of stream-entry. And a member of the spiritual family. These are the nine people who are worthy of offerings dedicated to the gods, worthy of hospitality, worthy of a religious donation, worthy of greeting with joined palms, and are the supreme field of merit for the world.” 

%
\addtocontents{toc}{\let\protect\contentsline\protect\nopagecontentsline}
\chapter*{The Chapter on the Lion’s Roar }
\addcontentsline{toc}{chapter}{\tocchapterline{The Chapter on the Lion’s Roar }}
\addtocontents{toc}{\let\protect\contentsline\protect\oldcontentsline}

%
\section*{{\suttatitleacronym AN 9.11}{\suttatitletranslation Sāriputta’s Lion’s Roar }{\suttatitleroot Sīhanādasutta}}
\addcontentsline{toc}{section}{\tocacronym{AN 9.11} \toctranslation{Sāriputta’s Lion’s Roar } \tocroot{Sīhanādasutta}}
\markboth{Sāriputta’s Lion’s Roar }{Sīhanādasutta}
\extramarks{AN 9.11}{AN 9.11}

At\marginnote{1.1} one time the Buddha was staying near \textsanskrit{Sāvatthī} in Jeta’s Grove, \textsanskrit{Anāthapiṇḍika}’s monastery. 

Then\marginnote{1.2} Venerable \textsanskrit{Sāriputta} went up to the Buddha, bowed, sat down to one side, and said to him, “Sir, I have completed the rainy season residence at \textsanskrit{Sāvatthī}. I wish to depart to wander the countryside.” 

“Please,\marginnote{1.5} \textsanskrit{Sāriputta}, go at your convenience.” Then \textsanskrit{Sāriputta} got up from his seat, bowed, and respectfully circled the Buddha, keeping him on his right, before leaving. 

And\marginnote{1.7} then, not long after \textsanskrit{Sāriputta} had left, a certain monk said to the Buddha, “Sir, Venerable \textsanskrit{Sāriputta} attacked me and left without saying sorry.” 

So\marginnote{1.9} the Buddha addressed a certain monk, “Please, monk, in my name tell \textsanskrit{Sāriputta} that the teacher summons him.” 

“Yes,\marginnote{1.12} sir,” that monk replied. He went to \textsanskrit{Sāriputta} and said to him, “Reverend \textsanskrit{Sāriputta}, the teacher summons you.” 

“Yes,\marginnote{1.14} reverend,” \textsanskrit{Sāriputta} replied. 

Now\marginnote{2.1} at that time the venerables \textsanskrit{Mahāmoggallāna} and Ānanda took a key and went from dwelling to dwelling, saying: “Come forth, venerables! Come forth, venerables! Now Venerable \textsanskrit{Sāriputta} will roar his lion’s roar in the presence of the Buddha!” 

Then\marginnote{2.4} Venerable \textsanskrit{Sāriputta} went up to the Buddha, bowed, and sat down to one side. The Buddha said to him: 

“\textsanskrit{Sāriputta},\marginnote{2.5} one of your spiritual companions has made this complaint: ‘Venerable \textsanskrit{Sāriputta} attacked me and left without saying sorry.’” 

“Sir,\marginnote{3.1} someone who had not established mindfulness of the body might well attack one of their spiritual companions and leave without saying sorry. 

Suppose\marginnote{4.1} they were to toss both clean and unclean things on the earth, like feces, urine, spit, pus, and blood. The earth isn’t horrified, repelled, and disgusted because of this. In the same way, I live with a heart like the earth, abundant, expansive, limitless, free of enmity and ill will. Someone who had not established mindfulness of the body might well attack one of their spiritual companions and leave without saying sorry. 

Suppose\marginnote{5.1} they were to wash both clean and unclean things in water, like feces, urine, spit, pus, and blood. The water isn’t horrified, repelled, and disgusted because of this. In the same way, I live with a heart like water, abundant, expansive, limitless, free of enmity and ill will. Someone who had not established mindfulness of the body might well attack one of their spiritual companions and leave without saying sorry. 

Suppose\marginnote{6.1} a fire was to burn both clean and unclean things, like feces, urine, spit, pus, and blood. The fire isn’t horrified, repelled, and disgusted because of this. In the same way, I live with a heart like fire, abundant, expansive, limitless, free of enmity and ill will. Someone who had not established mindfulness of the body might well attack one of their spiritual companions and leave without saying sorry. 

Suppose\marginnote{7.1} the wind was to blow on both clean and unclean things, like feces, urine, spit, pus, and blood. The wind isn’t horrified, repelled, and disgusted because of this. In the same way, I live with a heart like the wind, abundant, expansive, limitless, free of enmity and ill will. Someone who had not established mindfulness of the body might well attack one of their spiritual companions and leave without saying sorry. 

Suppose\marginnote{8.1} a rag was to wipe up both clean and unclean things, like feces, urine, spit, pus, and blood. The rag isn’t horrified, repelled, and disgusted because of this. In the same way, I live with a heart like a rag, abundant, expansive, limitless, free of enmity and ill will. Someone who had not established mindfulness of the body might well attack one of their spiritual companions and leave without saying sorry. 

Suppose\marginnote{9.1} an outcaste boy or girl, holding a pot and clad in rags, were to enter a town or village. They’d enter with a humble mind. In the same way, I live with a heart like an outcaste boy or girl, abundant, expansive, limitless, free of enmity and ill will. Someone who had not established mindfulness of the body might well attack one of their spiritual companions and leave without saying sorry. 

Suppose\marginnote{10.1} there was a bull with his horns cut, gentle, well tamed and well trained. He’d wander from street to street and square to square without hurting anyone with his feet or horns. In the same way, I live with a heart like a bull with horns cut, abundant, expansive, limitless, free of enmity and ill will. Someone who had not established mindfulness of the body might well attack one of their spiritual companions and leave without saying sorry. 

Suppose\marginnote{11.1} there was a woman or man who was young, youthful, and fond of adornments, and had bathed their head. If the corpse of a snake or a dog or a human were hung around their neck, they’d be horrified, repelled, and disgusted. In the same way, I’m horrified, repelled, and disgusted by this rotten body. Someone who had not established mindfulness of the body might well attack one of their spiritual companions and leave without saying sorry. 

Suppose\marginnote{12.1} someone was to carry around a bowl of fat that was leaking and oozing from holes and cracks. In the same way, I carry around this body that’s leaking and oozing from holes and cracks. Someone who had not established mindfulness of the body might well attack one of their spiritual companions and leave without saying sorry.” 

Then\marginnote{13.1} that monk rose from his seat, placed his robe over one shoulder, bowed with his head at the Buddha’s feet, and said, “I have made a mistake, sir. It was foolish, stupid, and unskillful of me to speak ill of Venerable \textsanskrit{Sāriputta} with a false, hollow, lying, untruthful claim. Please, sir, accept my mistake for what it is, so I will restrain myself in future.” 

“Indeed,\marginnote{13.4} monk, you made a mistake. It was foolish, stupid, and unskillful of you to act in that way. But since you have recognized your mistake for what it is, and have dealt with it properly, I accept it. For it is growth in the training of the Noble One to recognize a mistake for what it is, deal with it properly, and commit to restraint in the future.” 

Then\marginnote{14.1} the Buddha said to Venerable \textsanskrit{Sāriputta}, “\textsanskrit{Sāriputta}, forgive that silly man before his head explodes into seven pieces right here.” 

“I\marginnote{14.3} will pardon that venerable if he asks me: ‘May the venerable please pardon me too.’” 

%
\section*{{\suttatitleacronym AN 9.12}{\suttatitletranslation With Something Left Over }{\suttatitleroot Saupādisesasutta}}
\addcontentsline{toc}{section}{\tocacronym{AN 9.12} \toctranslation{With Something Left Over } \tocroot{Saupādisesasutta}}
\markboth{With Something Left Over }{Saupādisesasutta}
\extramarks{AN 9.12}{AN 9.12}

At\marginnote{1.1} one time the Buddha was staying near \textsanskrit{Sāvatthī} in Jeta’s Grove, \textsanskrit{Anāthapiṇḍika}’s monastery. 

Then\marginnote{1.2} Venerable \textsanskrit{Sāriputta} robed up in the morning and, taking his bowl and robe, entered \textsanskrit{Sāvatthī} for alms. Then it occurred to him, “It’s too early to wander for alms in \textsanskrit{Sāvatthī}. Why don’t I go to the monastery of the wanderers who follow other paths?” Then he went to the monastery of the wanderers who follow other paths, and exchanged greetings with the wanderers there. When the greetings and polite conversation were over, he sat down to one side. 

Now\marginnote{2.1} at that time while those wanderers who follow other paths were sitting together this discussion came up among them: 

“Reverends,\marginnote{2.2} no-one who dies with something left over is exempt from hell, the animal realm, or the ghost realm. They’re not exempt from places of loss, bad places, the underworld.” 

\textsanskrit{Sāriputta}\marginnote{2.3} neither approved nor dismissed that statement of the wanderers who follow other paths. He got up from his seat, thinking, “I will learn the meaning of this statement from the Buddha himself.” 

Then\marginnote{2.6} \textsanskrit{Sāriputta} wandered for alms in \textsanskrit{Sāvatthī}. After the meal, on his return from almsround, he went to the Buddha, bowed, sat down to one side, and told him what had happened. 

“\textsanskrit{Sāriputta},\marginnote{4.1} these foolish, incompetent wanderers following other paths: who are they to know whether someone has something left over or not? 

There\marginnote{5.1} are these nine people who, dying with something left over, are exempt from hell, the animal realm, and the ghost realm. They’re exempt from places of loss, bad places, the underworld. What nine? 

There’s\marginnote{5.3} a person who has fulfilled ethics and immersion, but has limited wisdom. With the ending of the five lower fetters they’re extinguished between one life and the next. This is the first person … 

Furthermore,\marginnote{6.1} there’s a person who has fulfilled ethics and immersion, but has limited wisdom. With the ending of the five lower fetters they’re extinguished upon landing. This is the second person … 

With\marginnote{6.3} the ending of the five lower fetters they’re extinguished without extra effort. This is the third person … 

With\marginnote{6.5} the ending of the five lower fetters they’re extinguished with extra effort. This is the fourth person … 

With\marginnote{6.7} the ending of the five lower fetters they head upstream, going to the \textsanskrit{Akaniṭṭha} realm. This is the fifth person … 

Furthermore,\marginnote{7.1} there’s a person who has fulfilled ethics, but has limited immersion and wisdom. With the ending of three fetters, and the weakening of greed, hate, and delusion, they’re a once-returner. They come back to this world once only, then make an end of suffering. This is the sixth person … 

Furthermore,\marginnote{8.1} there’s a person who has fulfilled ethics, but has limited immersion and wisdom. With the ending of three fetters, they’re a one-seeder. They will be reborn just one time in a human existence, then make an end of suffering. This is the seventh person … 

Furthermore,\marginnote{9.1} there’s a person who has fulfilled ethics, but has limited immersion and wisdom. With the ending of three fetters, they go from family to family. They will transmigrate between two or three families and then make an end of suffering. This is the eighth person … 

Furthermore,\marginnote{10.1} there’s a person who has fulfilled ethics, but has limited immersion and wisdom. With the ending of three fetters, they have at most seven rebirths. They will transmigrate at most seven times among gods and humans and then make an end of suffering. This is the ninth person … 

These\marginnote{11.1} foolish, incompetent wanderers following other paths: who are they to know whether someone has something left over or not? These are the nine people who, dying with something left over, are exempt from hell, the animal realm, and the ghost realm. They’re exempt from places of loss, bad places, the underworld. 

Up\marginnote{11.3} until now, \textsanskrit{Sāriputta}, I have not felt the need to give this exposition of the teaching to the monks, nuns, laymen, and laywomen. Why is that? For I didn’t want those who heard it to introduce negligence. However, I have spoken it in order to answer your question.” 

%
\section*{{\suttatitleacronym AN 9.13}{\suttatitletranslation With Koṭṭhita }{\suttatitleroot Koṭṭhikasutta}}
\addcontentsline{toc}{section}{\tocacronym{AN 9.13} \toctranslation{With Koṭṭhita } \tocroot{Koṭṭhikasutta}}
\markboth{With Koṭṭhita }{Koṭṭhikasutta}
\extramarks{AN 9.13}{AN 9.13}

Then\marginnote{1.1} Venerable \textsanskrit{Mahākoṭṭhita} went up to Venerable \textsanskrit{Sāriputta}, and exchanged greetings with him. When the greetings and polite conversation were over, he sat down to one side and said to \textsanskrit{Sāriputta}: 

“Reverend\marginnote{1.3} \textsanskrit{Sāriputta}, is the spiritual life lived under the Buddha for this purpose: ‘May deeds to be experienced in this life be experienced by me in lives to come’?” 

“Certainly\marginnote{1.4} not, reverend.” 

“Then\marginnote{2.1} is the spiritual life lived under the Buddha for this purpose: ‘May deeds to be experienced in lives to come be experienced by me in this life’?” 

“Certainly\marginnote{2.2} not.” 

“Is\marginnote{3.1} the spiritual life lived under the Buddha for this purpose: ‘May deeds to be experienced as pleasant be experienced by me as painful’?” 

“Certainly\marginnote{3.2} not.” 

“Then\marginnote{4.1} is the spiritual life lived under the Buddha for this purpose: ‘May deeds to be experienced as painful be experienced by me as pleasant’?” 

“Certainly\marginnote{4.2} not.” 

“Is\marginnote{5.1} the spiritual life lived under the Buddha for this purpose: ‘May deeds to be experienced when ripe be experienced by me when unripe’?” 

“Certainly\marginnote{5.2} not.” 

“Then\marginnote{6.1} is the spiritual life lived under the Buddha for this purpose: ‘May deeds to be experienced when unripe be experienced by me when ripe’?” 

“Certainly\marginnote{6.2} not.” 

“Is\marginnote{7.1} the spiritual life lived under the Buddha for this purpose: ‘May deeds to be experienced a lot be experienced by me a little’?” 

“Certainly\marginnote{7.2} not.” 

“Then\marginnote{8.1} is the spiritual life lived under the Buddha for this purpose: ‘May deeds to be experienced a little be experienced by me a lot’?” 

“Certainly\marginnote{8.2} not.” 

“Is\marginnote{9.1} the spiritual life lived under the Buddha for this purpose: ‘May deeds to be experienced by me be not experienced’?” 

“Certainly\marginnote{9.2} not.” 

“Then\marginnote{10.1} is the spiritual life lived under the Buddha for this purpose: ‘May deeds not to be experienced be experienced’?” 

“Certainly\marginnote{10.2} not.” 

“Reverend\marginnote{11.1} \textsanskrit{Sāriputta}, when you were asked whether the spiritual life was lived under the Buddha so that deeds to be experienced in this life are experienced in lives to come, you said, ‘Certainly not’. 

When\marginnote{12.1} you were asked whether the spiritual life was lived under the Buddha so that deeds to be experienced in lives to come are experienced in this life … 

deeds\marginnote{13.1} to be experienced as pleasant are experienced as painful … 

deeds\marginnote{14.1} to be experienced as painful are experienced as pleasant … 

deeds\marginnote{15.1} to be experienced when ripe are experienced when unripe … 

deeds\marginnote{16.1} to be experienced when unripe are experienced when ripe … 

deeds\marginnote{17.1} to be experienced a lot are experienced a little … 

deeds\marginnote{18.1} to be experienced a little are experienced a lot … 

deeds\marginnote{19.1} to be experienced are not experienced … 

When\marginnote{20.1} you were asked whether the spiritual life was lived under the Buddha so that deeds not to be experienced are experienced, you said, ‘Certainly not.’ Then what exactly is the purpose of leading the spiritual life under the Buddha?” 

“Reverend,\marginnote{21.1} the spiritual life is lived under the Buddha to know, see, attain, realize, and comprehend that which is unknown, unseen, unattained, unrealized, and uncomprehended.” 

“But\marginnote{21.2} what is the unknown, unseen, unattained, unrealized, and uncomprehended?” 

“‘This\marginnote{21.3} is suffering.’ … ‘This is the origin of suffering.’ … ‘This is the cessation of suffering.’ … ‘This is the practice that leads to the cessation of suffering.’ … This is the unknown, unseen, unattained, unrealized, and uncomprehended. The spiritual life is lived under the Buddha to know, see, attain, realize, and comprehend this.” 

%
\section*{{\suttatitleacronym AN 9.14}{\suttatitletranslation With Samiddhi }{\suttatitleroot Samiddhisutta}}
\addcontentsline{toc}{section}{\tocacronym{AN 9.14} \toctranslation{With Samiddhi } \tocroot{Samiddhisutta}}
\markboth{With Samiddhi }{Samiddhisutta}
\extramarks{AN 9.14}{AN 9.14}

Then\marginnote{1.1} Venerable Samiddhi went up to Venerable \textsanskrit{Sāriputta}, bowed, and sat to one side. Venerable \textsanskrit{Sāriputta} said to him: 

“Samiddhi,\marginnote{2.1} based on what do thoughts arise in a person?” 

“Based\marginnote{2.2} on name and form, sir.” 

“Where\marginnote{3.1} do they become diversified?” 

“In\marginnote{3.2} the elements.” 

“What\marginnote{4.1} is their origin?” 

“Contact\marginnote{4.2} is their origin.” 

“What\marginnote{5.1} is their meeting place?” 

“Feeling\marginnote{5.2} is their meeting place.” 

“What\marginnote{6.1} is their chief?” 

“Immersion\marginnote{6.2} is their chief.” 

“What\marginnote{7.1} is their ruler?” 

“Mindfulness\marginnote{7.2} is their ruler.” 

“What\marginnote{8.1} is their overseer?” 

“Wisdom\marginnote{8.2} is their overseer.” 

“What\marginnote{9.1} is their core?” 

“Freedom\marginnote{9.2} is their core.” 

“What\marginnote{10.1} is their culmination?” 

“They\marginnote{10.2} culminate in the deathless.” 

“Samiddhi,\marginnote{11.1} when you were asked what is the basis on which thoughts arise in a person, you answered ‘name and form’. When you were asked … what is their culmination, you answered ‘the deathless’. Good, good, Samiddhi! It’s good that you answered each question. But don’t get conceited because of that.” 

%
\section*{{\suttatitleacronym AN 9.15}{\suttatitletranslation The Simile of the Boil }{\suttatitleroot Gaṇḍasutta}}
\addcontentsline{toc}{section}{\tocacronym{AN 9.15} \toctranslation{The Simile of the Boil } \tocroot{Gaṇḍasutta}}
\markboth{The Simile of the Boil }{Gaṇḍasutta}
\extramarks{AN 9.15}{AN 9.15}

“Mendicants,\marginnote{1.1} suppose there was a boil that was many years old. And that boil had nine orifices that were continually open wounds. Whatever oozed out of them would be filthy, stinking, and disgusting. Whatever leaked out them would be filthy, stinking, and disgusting. 

‘Boil’\marginnote{2.1} is a term for this body made up of the four primary elements, produced by mother and father, built up from rice and porridge, liable to impermanence, to wearing away and erosion, to breaking up and destruction. And that boil has nine orifices that are continually open wounds. Whatever oozes out of them is filthy, stinking, and disgusting. Whatever leaks out of them is filthy, stinking, and disgusting. So, mendicants, have no illusion about this body.” 

%
\section*{{\suttatitleacronym AN 9.16}{\suttatitletranslation Perceptions }{\suttatitleroot Saññāsutta}}
\addcontentsline{toc}{section}{\tocacronym{AN 9.16} \toctranslation{Perceptions } \tocroot{Saññāsutta}}
\markboth{Perceptions }{Saññāsutta}
\extramarks{AN 9.16}{AN 9.16}

“Mendicants,\marginnote{1.1} these nine perceptions, when developed and cultivated, are very fruitful and beneficial. They culminate in the deathless and end with the deathless. What nine? The perceptions of ugliness, death, repulsiveness of food, dissatisfaction with the whole world, impermanence, suffering in impermanence, not-self in suffering, giving up, and fading away. These nine perceptions, when developed and cultivated, are very fruitful and beneficial. They culminate in the deathless and end with the deathless.” 

%
\section*{{\suttatitleacronym AN 9.17}{\suttatitletranslation Families }{\suttatitleroot Kulasutta}}
\addcontentsline{toc}{section}{\tocacronym{AN 9.17} \toctranslation{Families } \tocroot{Kulasutta}}
\markboth{Families }{Kulasutta}
\extramarks{AN 9.17}{AN 9.17}

“Mendicants,\marginnote{1.1} visiting a family with nine factors is not worthwhile, or if you’ve already arrived, sitting down is not worthwhile. What nine? They don’t politely rise, bow, or offer a seat. They hide what they have. Even when they have much they give little. Even when they have refined things they give coarse things. They give carelessly, not carefully. They don’t sit nearby to listen to the teachings. When you’re speaking, they don’t listen well. Visiting a family with these nine factors is not worthwhile, or if you’ve already arrived, sitting down is not worthwhile. 

Visiting\marginnote{2.1} a family with nine factors is worthwhile, or if you’ve already arrived, sitting down is worthwhile. What nine? They politely rise, bow, and offer a seat. They don’t hide what they have. When they have much they give much. When they have refined things they give refined things. They give carefully, not carelessly. They sit nearby to listen to the teachings. When you’re speaking, they listen well. Visiting a family with these nine factors is worthwhile, or if you’ve already arrived, sitting down is worthwhile.” 

%
\section*{{\suttatitleacronym AN 9.18}{\suttatitletranslation The Sabbath with Nine Factors }{\suttatitleroot Navaṅguposathasutta}}
\addcontentsline{toc}{section}{\tocacronym{AN 9.18} \toctranslation{The Sabbath with Nine Factors } \tocroot{Navaṅguposathasutta}}
\markboth{The Sabbath with Nine Factors }{Navaṅguposathasutta}
\extramarks{AN 9.18}{AN 9.18}

“Mendicants,\marginnote{1.1} the observance of the sabbath with its nine factors is very fruitful and beneficial and splendid and bountiful. And how should it be observed? 

It’s\marginnote{1.3} when a noble disciple reflects: ‘As long as they live, the perfected ones give up killing living creatures, renouncing the rod and the sword. They are scrupulous and kind, and live full of compassion for all living beings. I, too, for this day and night will give up killing living creatures, renouncing the rod and the sword. I’ll be scrupulous and kind, and live full of compassion for all living beings. I will observe the sabbath by doing as the perfected ones do in this respect.’ This is its first factor. … 

‘As\marginnote{2.1} long as they live, the perfected ones give up high and luxurious beds. They sleep in a low place, either a small bed or a straw mat. I, too, for this day and night will give up high and luxurious beds. I’ll sleep in a low place, either a small bed or a straw mat. I will observe the sabbath by doing as the perfected ones do in this respect.’ This is its eighth factor. 

They\marginnote{3.1} meditate spreading a heart full of love to one direction, and to the second, and to the third, and to the fourth. In the same way above, below, across, everywhere, all around, they spread a heart full of love to the whole world—abundant, expansive, limitless, free of enmity and ill will. This is its ninth factor. 

The\marginnote{4.1} observance of the sabbath with its nine factors in this way is very fruitful and beneficial and splendid and bountiful.” 

%
\section*{{\suttatitleacronym AN 9.19}{\suttatitletranslation A Deity }{\suttatitleroot Devatāsutta}}
\addcontentsline{toc}{section}{\tocacronym{AN 9.19} \toctranslation{A Deity } \tocroot{Devatāsutta}}
\markboth{A Deity }{Devatāsutta}
\extramarks{AN 9.19}{AN 9.19}

“Mendicants,\marginnote{1.1} tonight, several glorious deities, lighting up the entire Jeta’s Grove, came to me, bowed, stood to one side, and said to me: ‘Sir, formerly when we were human beings, renunciates came to our homes. We politely rose for them, but we didn’t bow. And so, having not fulfilled our duty, full of remorse and regret, we were reborn in a lesser realm.’ 

Then\marginnote{2.1} several other deities came to me and said: ‘Sir, formerly when we were human beings, renunciates came to our homes. We politely rose for them and bowed, but we didn’t offer a seat. And so, having not fulfilled our duty, full of remorse and regret, we were reborn in a lesser realm.’ 

Then\marginnote{3.1} several other deities came to me and said: ‘Sir, formerly when we were human beings, renunciates came to our homes. We politely rose for them, bowed, and offered a seat, but we didn’t share as best we could. …’ 

‘…\marginnote{3.4} we didn’t sit nearby to listen to the teachings. …’ 

‘…\marginnote{3.5} we didn’t lend an ear to the teachings. …’ 

‘…\marginnote{3.6} we didn’t memorize the teachings. …’ 

‘…\marginnote{3.7} we didn’t examine the meaning of teachings we’d memorized. …’ 

‘…\marginnote{3.8} having understood the meaning and the teaching, we didn’t practice accordingly. And so, having not fulfilled our duty, full of remorse and regret, we were reborn in a lesser realm.’ 

Then\marginnote{4.1} several other deities came to me and said: ‘Sir, formerly when we were human beings, renunciates came to our homes. We politely rose, bowed, and offered them a seat. We shared as best we could. We sat nearby to listen to the teachings, lent an ear, memorized them, and examined their meaning. Understanding the teaching and the meaning we practiced accordingly. And so, having fulfilled our duty, free of remorse and regret, we were reborn in a superior realm.’ 

Here,\marginnote{4.5} mendicants, are these roots of trees, and here are these empty huts. Practice absorption, mendicants! Don’t be negligent! Don’t regret it later, like those former deities.” 

%
\section*{{\suttatitleacronym AN 9.20}{\suttatitletranslation About Velāma }{\suttatitleroot Velāmasutta}}
\addcontentsline{toc}{section}{\tocacronym{AN 9.20} \toctranslation{About Velāma } \tocroot{Velāmasutta}}
\markboth{About Velāma }{Velāmasutta}
\extramarks{AN 9.20}{AN 9.20}

At\marginnote{1.1} one time the Buddha was staying near \textsanskrit{Sāvatthī} in Jeta’s Grove, \textsanskrit{Anāthapiṇḍika}’s monastery. Then the householder \textsanskrit{Anāthapiṇḍika} went up to the Buddha, bowed, and sat down to one side. The Buddha said to him, “Householder, I wonder whether your family gives gifts?” 

“It\marginnote{2.2} does, sir. But only coarse gruel with pickles.” 

“Householder,\marginnote{2.4} someone might give a gift that’s either coarse or fine. But they give it carelessly, thoughtlessly, not with their own hand. They give the dregs, and they give without consideration for consequences. Then wherever the result of any such gift manifests, their mind doesn’t incline toward enjoyment of nice food, clothes, vehicles, or the five refined kinds of sensual stimulation. And their children, wives, bondservants, employees, and workers don’t want to listen to them. They don’t pay attention or try to understand. Why is that? Because that is the result of deeds done carelessly. 

Someone\marginnote{3.1} might give a gift that’s either coarse or fine. And they give it carefully, thoughtfully, with their own hand. They don’t give the dregs, and they give with consideration for consequences. Then wherever the result of any such gift manifests, their mind inclines toward enjoyment of nice food, clothes, vehicles, or the five refined kinds of sensual stimulation. And their children, wives, bondservants, employees, and workers want to listen. They pay attention and try to understand. Why is that? Because that is the result of deeds done carefully. 

Once\marginnote{4.1} upon a time, householder, there was a brahmin named \textsanskrit{Velāma}. He gave the following gift, a great offering. 84,000 gold bowls filled with silver. 84,000 silver bowls filled with gold. 84,000 bronze bowls filled with gold coins. 84,000 elephants with gold adornments and banners, covered with gold netting. 84,000 chariots upholstered with the hide of lions, tigers, and leopards, and cream rugs, with gold adornments and banners, covered with gold netting. 84,000 milk cows with silken reins and bronze pails. 84,000 maidens bedecked with jeweled earrings. 84,000 couches spread with woolen covers—shag-piled, pure white, or embroidered with flowers—and spread with a fine deer hide, with canopies above and red pillows at both ends. 8,400,000,000 fine cloths of linen, cotton, silk, and wool. And who can say how much food, drink, snacks, meals, refreshments, and beverages? It seemed like an overflowing river. 

Householder,\marginnote{5.1} you might think: ‘Surely the brahmin \textsanskrit{Velāma} must have been someone else at that time?’ But you should not see it like this. I myself was the brahmin \textsanskrit{Velāma} at that time. I gave that gift, a great offering. But at that event there was no-one worthy of a religious donation, and no-one to purify the religious donation. 

It\marginnote{6.1} would be more fruitful to feed one person accomplished in view than that great offering of \textsanskrit{Velāma}. 

It\marginnote{7.1} would be more fruitful to feed one once-returner than a hundred persons accomplished in view. 

It\marginnote{8.1} would be more fruitful to feed one non-returner than a hundred once-returners. 

It\marginnote{8.2} would be more fruitful to feed one perfected one than a hundred non-returners. 

It\marginnote{8.3} would be more fruitful to feed one Buddha awakened for themselves than a hundred perfected ones. 

It\marginnote{8.4} would be more fruitful to feed one Realized One, a perfected one, a fully awakened Buddha than a hundred Buddhas awakened for themselves. 

It\marginnote{8.5} would be more fruitful to feed the mendicant \textsanskrit{Saṅgha} headed by the Buddha than to feed one Realized One, a perfected one, a fully awakened Buddha. 

It\marginnote{8.6} would be more fruitful to build a dwelling especially for the \textsanskrit{Saṅgha} of the four quarters than to feed the mendicant \textsanskrit{Saṅgha} headed by the Buddha. 

It\marginnote{8.7} would be more fruitful to go for refuge to the Buddha, the teaching, and the \textsanskrit{Saṅgha} with a confident heart than to build a dwelling for the \textsanskrit{Saṅgha} of the four quarters. 

It\marginnote{8.8} would be more fruitful to undertake the training rules—not to kill living creatures, steal, commit sexual misconduct, lie, or take alcoholic drinks that cause negligence—than to go for refuge to the Buddha, the teaching, and the \textsanskrit{Saṅgha} with a confident heart. 

It\marginnote{8.9} would be more fruitful to develop a heart of love—even just as long as it takes to pull a cow’s udder—than to undertake the training rules. 

It\marginnote{9.1} would be more fruitful develop the perception of impermanence—even for as long as a finger-snap—than to do all of these things, including developing a heart of love for as long as it takes to pull a cow’s udder.” 

%
\addtocontents{toc}{\let\protect\contentsline\protect\nopagecontentsline}
\chapter*{The Chapter on Abodes of Sentient Beings }
\addcontentsline{toc}{chapter}{\tocchapterline{The Chapter on Abodes of Sentient Beings }}
\addtocontents{toc}{\let\protect\contentsline\protect\oldcontentsline}

%
\section*{{\suttatitleacronym AN 9.21}{\suttatitletranslation In Three Particulars }{\suttatitleroot Tiṭhānasutta}}
\addcontentsline{toc}{section}{\tocacronym{AN 9.21} \toctranslation{In Three Particulars } \tocroot{Tiṭhānasutta}}
\markboth{In Three Particulars }{Tiṭhānasutta}
\extramarks{AN 9.21}{AN 9.21}

“The\marginnote{1.1} humans of Uttarakuru surpass the Gods of the Thirty-Three and the humans of India in three particulars. What three? They’re selfless and not possessive. They have a fixed life span. They have a distinctive nature. The humans of Uttarakuru surpass the Gods of the Thirty-Three and the humans of India in these three particulars. 

The\marginnote{2.1} Gods of the Thirty-Three surpass the humans of Uttarakuru and India in three particulars. What three? Divine life span, beauty, and happiness. The Gods of the Thirty-Three surpass the humans of Uttarakuru and India in these three particulars. 

The\marginnote{3.1} humans of India surpass the humans of Uttarakuru and the Gods of the Thirty-Three in three particulars. What three? Bravery, mindfulness, and the spiritual life is lived here. The humans of India surpass the humans of Uttarakuru and the Gods of the Thirty-Three in these three particulars.” 

%
\section*{{\suttatitleacronym AN 9.22}{\suttatitletranslation A Wild Colt }{\suttatitleroot Assakhaḷuṅkasutta}}
\addcontentsline{toc}{section}{\tocacronym{AN 9.22} \toctranslation{A Wild Colt } \tocroot{Assakhaḷuṅkasutta}}
\markboth{A Wild Colt }{Assakhaḷuṅkasutta}
\extramarks{AN 9.22}{AN 9.22}

“Mendicants,\marginnote{1.1} I will teach you about three wild colts and three wild people; three excellent horses and three excellent people; and three fine thoroughbred horses and three fine thoroughbred people. Listen and pay close attention, I will speak. 

And\marginnote{2.1} what are the three wild colts? One wild colt is fast, but not beautiful or well proportioned. Another wild colt is fast and beautiful, but not well proportioned. While another wild colt is fast, beautiful, and well proportioned. These are the three wild colts. 

And\marginnote{3.1} what are the three wild people? One wild person is fast, but not beautiful or well proportioned. Another wild person is fast and beautiful, but not well proportioned. While another wild person is fast, beautiful, and well proportioned. 

And\marginnote{4.1} how is a wild person fast, but not beautiful or well proportioned? It’s when a mendicant truly understands: ‘This is suffering’ … ‘This is the origin of suffering’ … ‘This is the cessation of suffering’ … ‘This is the practice that leads to the cessation of suffering’. This is how they’re fast, I say. But when asked a question about the teaching or training, they falter without answering. This is how they’re not beautiful, I say. And they don’t receive robes, almsfood, lodgings, and medicines and supplies for the sick. This is how they’re not well proportioned, I say. This is how a wild person is fast, but not beautiful or well proportioned. 

And\marginnote{5.1} how is a wild person fast and beautiful, but not well proportioned? They truly understand: ‘This is suffering’ … ‘This is the origin of suffering’ … ‘This is the cessation of suffering’ … ‘This is the practice that leads to the cessation of suffering’. This is how they’re fast, I say. When asked a question about the teaching or training, they answer without faltering. This is how they’re beautiful, I say. But they don’t receive robes, almsfood, lodgings, and medicines and supplies for the sick. This is how they’re not well proportioned, I say. This is how a wild person is fast and beautiful, but not well proportioned. 

And\marginnote{6.1} how is a wild person fast, beautiful, and well proportioned? They truly understand: ‘This is suffering’ … ‘This is the origin of suffering’ … ‘This is the cessation of suffering’ … ‘This is the practice that leads to the cessation of suffering’. This is how they’re fast, I say. When asked a question about the teaching or training, they answer without faltering. This is how they’re beautiful, I say. They receive robes, almsfood, lodgings, and medicines and supplies for the sick. This is how they’re well proportioned, I say. This is how a wild person is fast, beautiful, and well proportioned. These are the three wild people. 

And\marginnote{7.1} what are the three excellent horses? One excellent horse … is fast, beautiful, and well proportioned. These are the three excellent horses. 

What\marginnote{8.1} are the three excellent people? One excellent person … is fast, beautiful, and well proportioned. 

And\marginnote{9.1} how is an excellent person … fast, beautiful, and well proportioned? It’s when a mendicant, with the ending of the five lower fetters, is reborn spontaneously. They’re extinguished there, and are not liable to return from that world. This is how they’re fast, I say. When asked a question about the teaching or training, they answer without faltering. This is how they’re beautiful, I say. They receive robes, almsfood, lodgings, and medicines and supplies for the sick. This is how they’re well proportioned, I say. This is how an excellent person fast, beautiful, and well proportioned. These are the three excellent people. 

And\marginnote{10.1} what are the three fine thoroughbred horses? One fine thoroughbred horse … is fast, beautiful, and well proportioned. These are the three fine thoroughbred horses. 

And\marginnote{11.1} what are the three fine thoroughbred people? One fine thoroughbred person … is fast, beautiful, and well proportioned. 

And\marginnote{12.1} how is a fine thoroughbred person … fast, beautiful, and well proportioned? It’s a mendicant who realizes the undefiled freedom of heart and freedom by wisdom in this very life. And they live having realized it with their own insight due to the ending of defilements. This is how they’re fast, I say. When asked a question about the teaching or training, they answer without faltering. This is how they’re beautiful, I say. They receive robes, almsfood, lodgings, and medicines and supplies for the sick. This is how they’re well proportioned, I say. This is how a fine thoroughbred person is fast, beautiful, and well proportioned. These are the three fine thoroughbred people.” 

%
\section*{{\suttatitleacronym AN 9.23}{\suttatitletranslation Rooted in Craving }{\suttatitleroot Taṇhāmūlakasutta}}
\addcontentsline{toc}{section}{\tocacronym{AN 9.23} \toctranslation{Rooted in Craving } \tocroot{Taṇhāmūlakasutta}}
\markboth{Rooted in Craving }{Taṇhāmūlakasutta}
\extramarks{AN 9.23}{AN 9.23}

“Mendicants,\marginnote{1.1} I will teach you about nine things rooted in craving. And what are the nine things rooted in craving? Craving is a cause of seeking. Seeking is a cause of gaining material possessions. Gaining material possessions is a cause of assessing. Assessing is a cause of desire and lust. Desire and lust is a cause of attachment. Attachment is a cause of ownership. Ownership is a cause of stinginess. Stinginess is a cause of safeguarding. Owing to safeguarding, many bad, unskillful things come to be: taking up the rod and the sword, quarrels, arguments, and fights, accusations, divisive speech, and lies. These are the nine things rooted in craving.” 

%
\section*{{\suttatitleacronym AN 9.24}{\suttatitletranslation Abodes of Sentient Beings }{\suttatitleroot Sattāvāsasutta}}
\addcontentsline{toc}{section}{\tocacronym{AN 9.24} \toctranslation{Abodes of Sentient Beings } \tocroot{Sattāvāsasutta}}
\markboth{Abodes of Sentient Beings }{Sattāvāsasutta}
\extramarks{AN 9.24}{AN 9.24}

“Mendicants,\marginnote{1.1} there are nine abodes of sentient beings. What nine? 

There\marginnote{1.3} are sentient beings that are diverse in body and diverse in perception, such as human beings, some gods, and some beings in the underworld. This is the first abode of sentient beings. 

There\marginnote{2.1} are sentient beings that are diverse in body and unified in perception, such as the gods reborn in \textsanskrit{Brahmā}’s Host through the first absorption. This is the second abode of sentient beings. 

There\marginnote{3.1} are sentient beings that are unified in body and diverse in perception, such as the gods of streaming radiance. This is the third abode of sentient beings. 

There\marginnote{4.1} are sentient beings that are unified in body and unified in perception, such as the gods replete with glory. This is the fourth abode of sentient beings. 

There\marginnote{5.1} are sentient beings that are non-percipient and do not experience anything, such as the gods who are non-percipient beings. This is the fifth abode of sentient beings. 

There\marginnote{6.1} are sentient beings that have gone totally beyond perceptions of form. With the ending of perceptions of impingement, not focusing on perceptions of diversity, aware that ‘space is infinite’, they have been reborn in the dimension of infinite space. This is the sixth abode of sentient beings. 

There\marginnote{7.1} are sentient beings that have gone totally beyond the dimension of infinite space. Aware that ‘consciousness is infinite’, they have been reborn in the dimension of infinite consciousness. This is the seventh abode of sentient beings. 

There\marginnote{8.1} are sentient beings that have gone totally beyond the dimension of infinite consciousness. Aware that ‘there is nothing at all’, they have been reborn in the dimension of nothingness. This is the eighth abode of sentient beings. 

There\marginnote{9.1} are sentient beings that have gone totally beyond the dimension of nothingness. They have been reborn in the dimension of neither perception nor non-perception. This is the ninth abode of sentient beings. 

These\marginnote{10.1} are the nine abodes of sentient beings.” 

%
\section*{{\suttatitleacronym AN 9.25}{\suttatitletranslation Consolidated by Wisdom }{\suttatitleroot Paññāsutta}}
\addcontentsline{toc}{section}{\tocacronym{AN 9.25} \toctranslation{Consolidated by Wisdom } \tocroot{Paññāsutta}}
\markboth{Consolidated by Wisdom }{Paññāsutta}
\extramarks{AN 9.25}{AN 9.25}

“Mendicants,\marginnote{1.1} when a mendicant’s mind has been well consolidated with wisdom it’s appropriate for them to say: ‘I understand: “Rebirth is ended, the spiritual journey has been completed, what had to be done has been done, there is no return to any state of existence.”’ 

And\marginnote{2.1} how is a mendicant’s mind well consolidated with wisdom? The mind is well consolidated with wisdom when they know: ‘My mind is without greed.’ … ‘My mind is without hate.’ … ‘My mind is without delusion.’ … ‘My mind is not liable to become greedy.’ … ‘My mind is not liable to become hateful.’ … ‘My mind is not liable to become deluded.’ … ‘My mind is not liable to return to rebirth in the sensual realm.’ … ‘My mind is not liable to return to rebirth in the realm of luminous form.’ … ‘My mind is not liable to return to rebirth in the formless realm.’ When a mendicant’s mind has been well consolidated with wisdom it’s appropriate for them to say: ‘I understand: “Rebirth is ended, the spiritual journey has been completed, what had to be done has been done, there is no return to any state of existence.”’” 

%
\section*{{\suttatitleacronym AN 9.26}{\suttatitletranslation The Simile of the Stone Pillar }{\suttatitleroot Silāyūpasutta}}
\addcontentsline{toc}{section}{\tocacronym{AN 9.26} \toctranslation{The Simile of the Stone Pillar } \tocroot{Silāyūpasutta}}
\markboth{The Simile of the Stone Pillar }{Silāyūpasutta}
\extramarks{AN 9.26}{AN 9.26}

So\marginnote{1.1} I have heard. At one time the Buddha was staying near \textsanskrit{Rājagaha}, in the Bamboo Grove, the squirrels’ feeding ground. 

There\marginnote{1.3} Venerable \textsanskrit{Candikāputta} addressed the mendicants, “Reverends, Devadatta teaches the mendicants like this: ‘When a mendicant’s mind is solidified by heart, it’s appropriate for them to say: “I understand: ‘Rebirth is ended, the spiritual journey has been completed, what had to be done has been done, there is no return to any state of existence.’”’” 

When\marginnote{2.1} he said this, Venerable \textsanskrit{Sāriputta} said to him, “Reverend \textsanskrit{Candikāputta}, Devadatta does not teach the mendicants like that. He teaches like this: ‘When a mendicant’s mind is well consolidated by heart, it’s appropriate for them to say: “I understand: ‘Rebirth is ended, the spiritual journey has been completed, what had to be done has been done, there is no return to any state of existence.’”’” 

For\marginnote{3.1} a second time … 

And\marginnote{4.1} for a third time Venerable \textsanskrit{Candikāputta} addressed the mendicants … 

And\marginnote{4.5} for a third time, \textsanskrit{Sāriputta} said to him, “Reverend \textsanskrit{Candikāputta}, Devadatta does not teach the mendicants like that. He teaches like this: ‘When a mendicant’s mind is well consolidated by heart, it’s appropriate for them to say: “I understand: ‘Rebirth is ended, the spiritual journey has been completed, what had to be done has been done, there is no return to any state of existence.’”’ 

And\marginnote{5.1} how is a mendicant’s mind well consolidated by heart? The mind is well consolidated by heart when they know: ‘My mind is without greed.’ … ‘My mind is without hate.’ … ‘My mind is without delusion.’ … ‘My mind is not liable to become greedy.’ … ‘My mind is not liable to become hateful.’ … ‘My mind is not liable to become deluded.’ … ‘My mind is not liable to return to rebirth in the sensual realm.’ … ‘My mind is not liable to return to rebirth in the realm of luminous form.’ … ‘My mind is not liable to return to rebirth in the formless realm.’ 

When\marginnote{5.11} a mendicant’s mind is rightly freed like this, even if compelling sights come into the range of vision they don’t occupy their mind. The mind remains unaffected. It is steady, imperturbable, observing disappearance. 

Suppose\marginnote{6.1} there was a stone pillar, sixteen feet long. Eight feet were buried underground, and eight above ground. And violent storms were to blow up out of the east, the west, the north, and the south. They couldn’t make it tremor and tremble and quake. Why is that? It’s because that stone pillar is firmly embedded, with deep foundations. In the same way, when a mendicant’s mind is rightly freed like this, even if compelling sights come into the range of vision they don’t occupy their mind. The mind remains unaffected. It is steady, imperturbable, observing disappearance. 

If\marginnote{7.1} even compelling sounds … smells … tastes … touches … and thoughts come into the range of the mind they don’t occupy the mind. The mind remains unaffected. It is steady, imperturbable, observing disappearance.” 

%
\section*{{\suttatitleacronym AN 9.27}{\suttatitletranslation Dangers and Threats (1st) }{\suttatitleroot Paṭhamaverasutta}}
\addcontentsline{toc}{section}{\tocacronym{AN 9.27} \toctranslation{Dangers and Threats (1st) } \tocroot{Paṭhamaverasutta}}
\markboth{Dangers and Threats (1st) }{Paṭhamaverasutta}
\extramarks{AN 9.27}{AN 9.27}

Then\marginnote{1.1} the householder \textsanskrit{Anāthapiṇḍika} went up to the Buddha, bowed, and sat down to one side. The Buddha said to him: 

“Householder,\marginnote{2.1} when a noble disciple has quelled five dangers and threats, and has the four factors of stream-entry, they may, if they wish, declare of themselves: ‘I’ve finished with rebirth in hell, the animal realm, and the ghost realm. I’ve finished with all places of loss, bad places, the underworld. I am a stream-enterer! I’m not liable to be reborn in the underworld, and am bound for awakening.’ 

What\marginnote{3.1} are the five dangers and threats they have quelled? Anyone who kills living creatures creates dangers and threats both in the present life and in lives to come, and experiences mental pain and sadness. Anyone who refrains from killing living creatures creates no dangers and threats either in the present life or in lives to come, and doesn’t experience mental pain and sadness. So that danger and threat is quelled for anyone who refrains from killing living creatures. 

Anyone\marginnote{4.1} who steals … 

Anyone\marginnote{4.2} who commits sexual misconduct … 

Anyone\marginnote{4.3} who lies … 

Anyone\marginnote{4.4} who uses alcoholic drinks that cause negligence creates dangers and threats both in the present life and in lives to come, and experiences mental pain and sadness. Anyone who refrains from using alcoholic drinks that cause negligence creates no dangers and threats either in the present life or in lives to come, and doesn’t experience mental pain and sadness. So that danger and threat is quelled for anyone who refrains from using alcoholic drinks that cause negligence. 

These\marginnote{5.1} are the five dangers and threats they have quelled. 

What\marginnote{6.1} are the four factors of stream-entry that they have? It’s when a noble disciple has experiential confidence in the Buddha: ‘That Blessed One is perfected, a fully awakened Buddha, accomplished in knowledge and conduct, holy, knower of the world, supreme guide for those who wish to train, teacher of gods and humans, awakened, blessed.’ 

They\marginnote{7.1} have experiential confidence in the teaching: ‘The teaching is well explained by the Buddha—visible in this very life, immediately effective, inviting inspection, relevant, so that sensible people can know it for themselves.’ 

They\marginnote{8.1} have experiential confidence in the \textsanskrit{Saṅgha}: ‘The \textsanskrit{Saṅgha} of the Buddha’s disciples is practicing the way that’s good, direct, methodical, and proper. It consists of the four pairs, the eight individuals. This is the \textsanskrit{Saṅgha} of the Buddha’s disciples that is worthy of offerings dedicated to the gods, worthy of hospitality, worthy of a religious donation, worthy of greeting with joined palms, and is the supreme field of merit for the world.’ 

And\marginnote{9.1} a noble disciple’s ethical conduct is loved by the noble ones, unbroken, impeccable, spotless, and unmarred, liberating, praised by sensible people, not mistaken, and leading to immersion. These are the four factors of stream-entry that they have. 

When\marginnote{10.1} a noble disciple has quelled these five dangers and threats, and has these four factors of stream-entry, they may, if they wish, declare of themselves: ‘I’ve finished with rebirth in hell, the animal realm, and the ghost realm. I’ve finished with all places of loss, bad places, the underworld. I am a stream-enterer! I’m not liable to be reborn in the underworld, and am bound for awakening.’” 

%
\section*{{\suttatitleacronym AN 9.28}{\suttatitletranslation Dangers and Threats (2nd) }{\suttatitleroot Dutiyaverasutta}}
\addcontentsline{toc}{section}{\tocacronym{AN 9.28} \toctranslation{Dangers and Threats (2nd) } \tocroot{Dutiyaverasutta}}
\markboth{Dangers and Threats (2nd) }{Dutiyaverasutta}
\extramarks{AN 9.28}{AN 9.28}

“Mendicants,\marginnote{1.1} when a noble disciple has quelled five dangers and threats, and has the four factors of stream-entry, they may, if they wish, declare of themselves: ‘I’ve finished with rebirth in hell, the animal realm, and the ghost realm. I’ve finished with all places of loss, bad places, the underworld. I am a stream-enterer! I’m not liable to be reborn in the underworld, and am bound for awakening.’ 

What\marginnote{2.1} are the five dangers and threats they have quelled? Anyone who kills living creatures creates dangers and threats both in the present life and in lives to come, and experiences mental pain and sadness. Anyone who refrains from killing living creatures creates no dangers and threats either in the present life or in lives to come, and doesn’t experience mental pain and sadness. So that danger and threat is quelled for anyone who refrains from killing living creatures. 

Anyone\marginnote{3.1} who steals … commits sexual misconduct … lies … Anyone who uses alcoholic drinks that cause negligence creates dangers and threats both in the present life and in lives to come, and experiences mental pain and sadness. Anyone who refrains from using alcoholic drinks that cause negligence creates no dangers and threats either in the present life or in lives to come, and doesn’t experience mental pain and sadness. So that danger and threat is quelled for anyone who refrains from using alcoholic drinks that cause negligence. These are the five dangers and threats they have quelled. 

What\marginnote{4.1} are the four factors of stream-entry that they have? When a noble disciple has experiential confidence in the Buddha … the teaching … the \textsanskrit{Saṅgha} … And a noble disciple’s ethical conduct is loved by the noble ones, unbroken, impeccable, spotless, and unmarred, liberating, praised by sensible people, not mistaken, and leading to immersion. These are the four factors of stream-entry that they have. 

When\marginnote{5.1} a noble disciple has quelled these five dangers and threats, and has these four factors of stream-entry, they may, if they wish, declare of themselves: ‘I’ve finished with rebirth in hell, the animal realm, and the ghost realm. I’ve finished with all places of loss, bad places, the underworld. I am a stream-enterer! I’m not liable to be reborn in the underworld, and am bound for awakening.’” 

%
\section*{{\suttatitleacronym AN 9.29}{\suttatitletranslation Grounds for Resentment }{\suttatitleroot Āghātavatthusutta}}
\addcontentsline{toc}{section}{\tocacronym{AN 9.29} \toctranslation{Grounds for Resentment } \tocroot{Āghātavatthusutta}}
\markboth{Grounds for Resentment }{Āghātavatthusutta}
\extramarks{AN 9.29}{AN 9.29}

“Mendicants,\marginnote{1.1} there are nine grounds for resentment. What nine? Thinking: ‘They did wrong to me,’ you harbor resentment. Thinking: ‘They are doing wrong to me’ … ‘They will do wrong to me’ … ‘They did wrong to someone I love’ … ‘They are doing wrong to someone I love’ … ‘They will do wrong to someone I love’ … ‘They helped someone I dislike’ … ‘They are helping someone I dislike’ … Thinking: ‘They will help someone I dislike,’ you harbor resentment. These are the nine grounds for resentment.” 

%
\section*{{\suttatitleacronym AN 9.30}{\suttatitletranslation Getting Rid of Resentment }{\suttatitleroot Āghātapaṭivinayasutta}}
\addcontentsline{toc}{section}{\tocacronym{AN 9.30} \toctranslation{Getting Rid of Resentment } \tocroot{Āghātapaṭivinayasutta}}
\markboth{Getting Rid of Resentment }{Āghātapaṭivinayasutta}
\extramarks{AN 9.30}{AN 9.30}

“Mendicants,\marginnote{1.1} there are these nine methods to get rid of resentment. What nine? Thinking: ‘They harmed me, but what can I possibly do?’ you get rid of resentment. Thinking: ‘They are harming me …’ … ‘They will harm me …’ … ‘They harmed someone I love …’ … ‘They are harming someone I love …’ ‘They will harm someone I love …’ … ‘They helped someone I dislike …’ … ‘They are helping someone I dislike …’ … Thinking: ‘They will help someone I dislike, but what can I possibly do?’ you get rid of resentment. These are the nine methods to get rid of resentment.” 

%
\section*{{\suttatitleacronym AN 9.31}{\suttatitletranslation Progressive Cessations }{\suttatitleroot Anupubbanirodhasutta}}
\addcontentsline{toc}{section}{\tocacronym{AN 9.31} \toctranslation{Progressive Cessations } \tocroot{Anupubbanirodhasutta}}
\markboth{Progressive Cessations }{Anupubbanirodhasutta}
\extramarks{AN 9.31}{AN 9.31}

“Mendicants,\marginnote{1.1} there are these nine progressive cessations. What nine? 

For\marginnote{1.3} someone who has attained the first absorption, sensual perceptions have ceased. 

For\marginnote{1.4} someone who has attained the second absorption, the placing of the mind and keeping it connected have ceased. 

For\marginnote{1.5} someone who has attained the third absorption, rapture has ceased. 

For\marginnote{1.6} someone who has attained the fourth absorption, breathing has ceased. 

For\marginnote{1.7} someone who has attained the dimension of infinite space, the perception of form has ceased. 

For\marginnote{1.8} someone who has attained the dimension of infinite consciousness, the perception of the dimension of infinite space has ceased. 

For\marginnote{1.9} someone who has attained the dimension of nothingness, the perception of the dimension of infinite consciousness has ceased. 

For\marginnote{1.10} someone who has attained the dimension of neither perception nor non-perception, the perception of the dimension of nothingness has ceased. 

For\marginnote{1.11} someone who has attained the cessation of perception and feeling, perception and feeling have ceased. 

These\marginnote{1.12} are the nine progressive cessations.” 

%
\addtocontents{toc}{\let\protect\contentsline\protect\nopagecontentsline}
\chapter*{The Great Chapter }
\addcontentsline{toc}{chapter}{\tocchapterline{The Great Chapter }}
\addtocontents{toc}{\let\protect\contentsline\protect\oldcontentsline}

%
\section*{{\suttatitleacronym AN 9.32}{\suttatitletranslation Progressive Meditations }{\suttatitleroot Anupubbavihārasutta}}
\addcontentsline{toc}{section}{\tocacronym{AN 9.32} \toctranslation{Progressive Meditations } \tocroot{Anupubbavihārasutta}}
\markboth{Progressive Meditations }{Anupubbavihārasutta}
\extramarks{AN 9.32}{AN 9.32}

“Mendicants,\marginnote{1.1} there are these nine progressive meditations. What nine? The first absorption, the second absorption, the third absorption, the fourth absorption, the dimension of infinite space, the dimension of infinite consciousness, the dimension of nothingness, the dimension of neither perception nor non-perception, and the cessation of perception and feeling. These are the nine progressive meditations.” 

%
\section*{{\suttatitleacronym AN 9.33}{\suttatitletranslation The Nine Progressive Meditative Attainments }{\suttatitleroot Anupubbavihārasamāpattisutta}}
\addcontentsline{toc}{section}{\tocacronym{AN 9.33} \toctranslation{The Nine Progressive Meditative Attainments } \tocroot{Anupubbavihārasamāpattisutta}}
\markboth{The Nine Progressive Meditative Attainments }{Anupubbavihārasamāpattisutta}
\extramarks{AN 9.33}{AN 9.33}

“Mendicants,\marginnote{1.1} I will teach you the nine progressive meditative attainments … And what are the nine progressive meditative attainments? 

Where\marginnote{1.3} sensual pleasures cease, and those who have thoroughly ended sensual pleasures meditate, I say: ‘Clearly those venerables are desireless, extinguished, crossed over, and gone beyond in that respect.’ If someone should say, ‘I do not know or see where sensual pleasures cease’, they should be told: ‘Reverend, it’s when a mendicant, quite secluded from sensual pleasures, secluded from unskillful qualities, enters and remains in the first absorption, which has the rapture and bliss born of seclusion, while placing the mind and keeping it connected. That’s where sensual pleasures cease.’ Clearly someone who is not devious or deceitful would approve and agree with that statement. They’d say ‘Good!’ and bowing down, they’d pay homage with joined palms. 

Where\marginnote{2.1} the placing of the mind and keeping it connected cease, and those who have thoroughly ended the placing of the mind and keeping it connected meditate, I say: ‘Clearly those venerables are desireless, extinguished, crossed over, and gone beyond in that respect.’ If someone should say, ‘I do not know or see where the placing of the mind and keeping it connected cease’, they should be told: ‘It’s when a mendicant, as the placing of the mind and keeping it connected are stilled, enters and remains in the second absorption, which has the rapture and bliss born of immersion, with internal clarity and confidence, and unified mind, without placing the mind and keeping it connected. That’s where the placing of the mind and keeping it connected cease.’ Clearly someone who is not devious or deceitful would approve and agree with that statement. They’d say ‘Good!’ and bowing down, they’d pay homage with joined palms. 

Where\marginnote{3.1} rapture ceases, and those who have thoroughly ended rapture meditate, I say: ‘Clearly those venerables are desireless, extinguished, crossed over, and gone beyond in that respect.’ If someone should say, ‘I do not know or see where rapture ceases’, they should be told: ‘It’s when a mendicant, with the fading away of rapture, enters and remains in the third absorption, where they meditate with equanimity, mindful and aware, personally experiencing the bliss of which the noble ones declare, “Equanimous and mindful, one meditates in bliss”. That’s where rapture ceases.’ Clearly someone who is not devious or deceitful would approve and agree with that statement. They’d say ‘Good!’ and bowing down, they’d pay homage with joined palms. 

Where\marginnote{4.1} equanimous bliss ceases, and those who have thoroughly ended equanimous bliss meditate, I say: ‘Clearly those venerables are desireless, extinguished, crossed over, and gone beyond in that respect.’ If someone should say, ‘I do not know or see where equanimous bliss ceases’, they should be told: ‘It’s when a mendicant, giving up pleasure and pain, and ending former happiness and sadness, enters and remains in the fourth absorption, without pleasure or pain, with pure equanimity and mindfulness. That’s where equanimous bliss ceases.’ Clearly someone who is not devious or deceitful would approve and agree with that statement. They’d say ‘Good!’ and bowing down, they’d pay homage with joined palms. 

Where\marginnote{5.1} perceptions of form ceases, and those who have thoroughly ended perceptions of form meditate, I say: ‘Clearly those venerables are desireless, extinguished, crossed over, and gone beyond in that respect.’ If someone should say, ‘I do not know or see where perceptions of form ceases’, they should be told: ‘It’s when a mendicant, going totally beyond perceptions of form, with the ending of perceptions of impingement, not focusing on perceptions of diversity, aware that “space is infinite”, enters and remains in the dimension of infinite space. That’s where perceptions of form cease.’ Clearly someone who is not devious or deceitful would approve and agree with that statement. They’d say ‘Good!’ and bowing down, they’d pay homage with joined palms. 

Where\marginnote{6.1} the perception of the dimension of infinite space ceases, and those who have thoroughly ended the perception of the dimension of infinite space meditate, I say: ‘Clearly those venerables are desireless, extinguished, crossed over, and gone beyond in that respect.’ If someone should say, ‘I do not know or see where the perception of the dimension of infinite space ceases’, they should be told: ‘It’s when a mendicant, going totally beyond the dimension of infinite space, aware that “consciousness is infinite”, enters and remains in the dimension of infinite consciousness. That’s where the perception of the dimension of infinite space ceases.’ Clearly someone who is not devious or deceitful would approve and agree with that statement. They’d say ‘Good!’ and bowing down, they’d pay homage with joined palms. 

Where\marginnote{7.1} the perception of the dimension of infinite consciousness ceases, and those who have thoroughly ended the perception of the dimension of infinite consciousness meditate, I say: ‘Clearly those venerables are desireless, extinguished, crossed over, and gone beyond in that respect.’ If someone should say, ‘I do not know or see where the perception of the dimension of infinite consciousness ceases’, they should be told: ‘It’s when a mendicant, going totally beyond the dimension of infinite consciousness, aware that “there is nothing at all”, enters and remains in the dimension of nothingness. That’s where the perception of the dimension of infinite consciousness ceases.’ Clearly someone who is not devious or deceitful would approve and agree with that statement. They’d say ‘Good!’ and bowing down, they’d pay homage with joined palms. 

Where\marginnote{8.1} the perception of the dimension of nothingness ceases, and those who have thoroughly ended the perception of the dimension of nothingness meditate, I say: ‘Clearly those venerables are desireless, extinguished, crossed over, and gone beyond in that respect.’ If someone should say, ‘I do not know or see where the perception of the dimension of nothingness ceases’, they should be told: ‘It’s when a mendicant, going totally beyond the dimension of nothingness, enters and remains in the dimension of neither perception nor non-perception. That’s where the perception of the dimension of nothingness ceases.’ Clearly someone who is not devious or deceitful would approve and agree with that statement. They’d say ‘Good!’ and bowing down, they’d pay homage with joined palms. 

Where\marginnote{9.1} the perception of the dimension of neither perception nor non-perception ceases, and those who have thoroughly ended the perception of the dimension of neither perception nor non-perception meditate, I say: ‘Clearly those venerables are desireless, extinguished, crossed over, and gone beyond in that respect.’ If someone should say, ‘I do not know or see where the perception of the dimension of neither perception nor non-perception ceases’, they should be told: ‘It’s when a mendicant, going totally beyond the dimension of neither perception nor non-perception, enters and remains in the cessation of perception and feeling. That’s where the perception of the dimension of neither perception nor non-perception ceases.’ Clearly someone who is not devious or deceitful would approve and agree with that statement. They’d say ‘Good!’ and bowing down, they’d pay homage with joined palms. 

These\marginnote{10.1} are the nine progressive meditative attainments.” 

%
\section*{{\suttatitleacronym AN 9.34}{\suttatitletranslation Extinguishment is Bliss }{\suttatitleroot Nibbānasukhasutta}}
\addcontentsline{toc}{section}{\tocacronym{AN 9.34} \toctranslation{Extinguishment is Bliss } \tocroot{Nibbānasukhasutta}}
\markboth{Extinguishment is Bliss }{Nibbānasukhasutta}
\extramarks{AN 9.34}{AN 9.34}

At\marginnote{1.1} one time Venerable \textsanskrit{Sāriputta} was staying near \textsanskrit{Rājagaha}, in the Bamboo Grove, the squirrels’ feeding ground. 

There\marginnote{1.2} he addressed the mendicants: “Reverends, extinguishment is bliss! Extinguishment is bliss!” 

When\marginnote{1.5} he said this, Venerable \textsanskrit{Udāyī} said to him, “But Reverend \textsanskrit{Sāriputta}, what’s blissful about it, since nothing is felt?” 

“The\marginnote{1.7} fact that nothing is felt is precisely what’s blissful about it. 

Reverend,\marginnote{1.8} there are these five kinds of sensual stimulation. What five? Sights known by the eye that are likable, desirable, agreeable, pleasant, sensual, and arousing. Sounds known by the ear … Smells known by the nose … Tastes known by the tongue … Touches known by the body that are likable, desirable, agreeable, pleasant, sensual, and arousing. These are the five kinds of sensual stimulation. The pleasure and happiness that arise from these five kinds of sensual stimulation is called sensual pleasure. 

First,\marginnote{2.1} take a mendicant who, quite secluded from sensual pleasures … enters and remains in the first absorption. While a mendicant is in such a meditation, should perceptions and attentions accompanied by sensual pleasures beset them, that’s an affliction for them. Suppose a happy person were to experience pain; that would be an affliction for them. In the same way, should perceptions and attentions accompanied by sensual pleasures beset them, that’s an affliction for them. And affliction has been called suffering by the Buddha. That’s the way to understand how extinguishment is bliss. 

Furthermore,\marginnote{3.1} take a mendicant who, as the placing of the mind and keeping it connected are stilled, enters and remains in the second absorption. While a mendicant is in such a meditation, should perceptions and attentions accompanied by placing of the mind and keeping it connected beset them, that’s an affliction for them. Suppose a happy person were to experience pain; that would be an affliction for them. In the same way, should perceptions and attentions accompanied by placing of the mind and keeping it connected beset them, that’s an affliction for them. And affliction has been called suffering by the Buddha. That too is a way to understand how extinguishment is bliss. 

Furthermore,\marginnote{4.1} take a mendicant who, with the fading away of rapture, enters and remains in the third absorption. While a mendicant is in such a meditation, should perceptions and attentions accompanied by rapture beset them, that’s an affliction for them. Suppose a happy person were to experience pain; that would be an affliction for them. In the same way, should perceptions and attentions accompanied by rapture beset them, that’s an affliction for them. And affliction has been called suffering by the Buddha. That too is a way to understand how extinguishment is bliss. 

Furthermore,\marginnote{5.1} take a mendicant who, giving up pleasure and pain, and ending former happiness and sadness, enters and remains in the fourth absorption. While a mendicant is in such a meditation, should perceptions and attentions accompanied by equanimous bliss beset them, that’s an affliction for them. Suppose a happy person were to experience pain; that would be an affliction for them. In the same way, should perceptions and attentions accompanied by equanimous bliss beset them, that’s an affliction for them. And affliction has been called suffering by the Buddha. That too is a way to understand how extinguishment is bliss. 

Furthermore,\marginnote{6.1} take a mendicant who, going totally beyond perceptions of form, with the ending of perceptions of impingement, not focusing on perceptions of diversity, aware that ‘space is infinite’, enters and remains in the dimension of infinite space. While a mendicant is in such a meditation, should perceptions and attentions accompanied by form beset them, that’s an affliction for them. Suppose a happy person were to experience pain; that would be an affliction for them. In the same way, should perceptions and attentions accompanied by form beset them, that’s an affliction for them. And affliction has been called suffering by the Buddha. That too is a way to understand how extinguishment is bliss. 

Furthermore,\marginnote{7.1} take a mendicant who, going totally beyond the dimension of infinite space, aware that ‘consciousness is infinite’, enters and remains in the dimension of infinite consciousness. While a mendicant is in such a meditation, should perceptions and attentions accompanied by the dimension of infinite space beset them, that’s an affliction for them. Suppose a happy person were to experience pain; that would be an affliction for them. In the same way, should perceptions and attentions accompanied by the dimension of infinite space beset them, that’s an affliction for them. And affliction has been called suffering by the Buddha. That too is a way to understand how extinguishment is bliss. 

Furthermore,\marginnote{8.1} take a mendicant who, going totally beyond the dimension of infinite consciousness, aware that ‘there is nothing at all’, enters and remains in the dimension of nothingness. While a mendicant is in such a meditation, should perceptions and attentions accompanied by the dimension of infinite consciousness beset them, that’s an affliction for them. Suppose a happy person were to experience pain; that would be an affliction for them. In the same way, should perceptions and attentions accompanied by the dimension of infinite consciousness beset them, that’s an affliction for them. And affliction has been called suffering by the Buddha. That too is a way to understand how extinguishment is bliss. 

Furthermore,\marginnote{9.1} take a mendicant who, going totally beyond the dimension of nothingness, enters and remains in the dimension of neither perception nor non-perception. While a mendicant is in such a meditation, should perceptions and attentions accompanied by the dimension of nothingness beset them, that’s an affliction for them. Suppose a happy person were to experience pain; that would be an affliction for them. In the same way, should perceptions and attentions accompanied by the dimension of nothingness beset them, that’s an affliction for them. And affliction has been called suffering by the Buddha. That too is a way to understand how extinguishment is bliss. 

Furthermore,\marginnote{10.1} take a mendicant who, going totally beyond the dimension of neither perception nor non-perception, enters and remains in the cessation of perception and feeling. And, having seen with wisdom, their defilements come to an end. 

That\marginnote{11.1} too is a way to understand how extinguishment is bliss.” 

%
\section*{{\suttatitleacronym AN 9.35}{\suttatitletranslation The Simile of the Cow }{\suttatitleroot Gāvīupamāsutta}}
\addcontentsline{toc}{section}{\tocacronym{AN 9.35} \toctranslation{The Simile of the Cow } \tocroot{Gāvīupamāsutta}}
\markboth{The Simile of the Cow }{Gāvīupamāsutta}
\extramarks{AN 9.35}{AN 9.35}

“Mendicants,\marginnote{1.1} suppose there was a mountain cow who was foolish, incompetent, unskillful, and lacked common sense when roaming on rugged mountains. She might think, ‘Why don’t I go somewhere I’ve never been before? I could eat grass and drink water that I’ve never tried before.’ She’d take a step with a fore-hoof; but before it was properly set down, she’d lift up a hind-hoof. She wouldn’t go somewhere she’d never been before, or eat grass and drink water that she’d never tried before. And she’d never return safely to the place she had started from. Why is that? Because that mountain cow was foolish, incompetent, unskillful, and lacked common sense when roaming on rugged mountains. 

In\marginnote{1.10} the same way, some foolish, incompetent, unskillful mendicant, lacking common sense, quite secluded from sensual pleasures, secluded from unskillful qualities, enters and remains in the first absorption, which has the rapture and bliss born of seclusion, while placing the mind and keeping it connected. But they don’t cultivate, develop, and make much of that foundation; they don’t ensure it is properly stabilized. 

They\marginnote{2.1} think, ‘Why don’t I, as the placing of the mind and keeping it connected are stilled, enter and remain in the second absorption, which has the rapture and bliss born of immersion, with internal clarity and confidence, and unified mind, without placing the mind and keeping it connected.’ But they’re not able to enter and remain in the second absorption. They think, ‘Why don’t I, quite secluded from sensual pleasures, secluded from unskillful qualities, enter and remain in the first absorption, which has the rapture and bliss born of seclusion, while placing the mind and keeping it connected.’ But they’re not able to enter and remain in the first absorption. This is called a mendicant who has slipped and fallen from both sides. They’re like the mountain cow who was foolish, incompetent, unskillful, and lacking in common sense when roaming on rugged mountains. 

Suppose\marginnote{3.1} there was a mountain cow who was astute, competent, skillful, and used common sense when roaming on rugged mountains. She might think, ‘Why don’t I go somewhere I’ve never been before? I could eat grass and drink water that I’ve never tried before.’ She’d take a step with a fore-hoof; and after it was properly set down, she’d lift up a hind-hoof. She’d go somewhere she’d never been before, and eat grass and drink water that she’d never tried before. And she’d return safely to the place she had started from. Why is that? Because that mountain cow was astute, competent, skillful, and used common sense when roaming on rugged mountains. In the same way, some astute, competent, skillful mendicant, using common sense, quite secluded from sensual pleasures, secluded from unskillful qualities, enters and remains in the first absorption, which has the rapture and bliss born of seclusion, while placing the mind and keeping it connected. They cultivate, develop, and make much of that foundation, ensuring that it’s properly stabilized. 

They\marginnote{4.1} think, ‘Why don’t I, as the placing of the mind and keeping it connected are stilled, enter and remain in the second absorption, which has the rapture and bliss born of immersion, with internal clarity and confidence, and unified mind, without placing the mind and keeping it connected.’ Without charging at the second absorption, as the placing of the mind and keeping it connected are stilled, they enter and remain in the second absorption. They cultivate, develop, and make much of that foundation, ensuring that it’s properly stabilized. 

They\marginnote{5.1} think, ‘Why don’t I, with the fading away of rapture, enter and remain in the third absorption, where I will meditate with equanimity, mindful and aware, personally experiencing the bliss of which the noble ones declare, “Equanimous and mindful, one meditates in bliss.”’ Without charging at the third absorption, with the fading away of rapture, they enter and remain in the third absorption. They cultivate, develop, and make much of that foundation, ensuring that it’s properly stabilized. 

They\marginnote{6.1} think, ‘Why don’t I, with the giving up of pleasure and pain, and the ending of former happiness and sadness, enter and remain in the fourth absorption, without pleasure or pain, with pure equanimity and mindfulness.’ Without charging at the fourth absorption, with the giving up of pleasure and pain, and the ending of former happiness and sadness, they enter and remain in the fourth absorption. They cultivate, develop, and make much of that foundation, ensuring that it’s properly stabilized. 

They\marginnote{7.1} think, ‘Why don’t I, going totally beyond perceptions of form, with the ending of perceptions of impingement, not focusing on perceptions of diversity, aware that “space is infinite”, enter and remain in the dimension of infinite space.’ Without charging at the dimension of infinite space, with the fading away of rapture, they enter and remain in the dimension of infinite space. They cultivate, develop, and make much of that foundation, ensuring that it’s properly stabilized. 

They\marginnote{8.1} think, ‘Why don’t I, going totally beyond the dimension of infinite space, aware that “consciousness is infinite”, enter and remain in the dimension of infinite consciousness.’ Without charging at the dimension of infinite consciousness, they enter and remain in the dimension of infinite consciousness. They cultivate, develop, and make much of that foundation, ensuring that it’s properly stabilized. 

They\marginnote{9.1} think, ‘Why don’t I, going totally beyond the dimension of infinite consciousness, aware that “there is nothing at all”, enter and remain in the dimension of nothingness.’ Without charging at the dimension of nothingness, they enter and remain in the dimension of nothingness. They cultivate, develop, and make much of that foundation, ensuring that it’s properly stabilized. 

They\marginnote{10.1} think, ‘Why don’t I, going totally beyond the dimension of nothingness, enter and remain in the dimension of neither perception nor non-perception.’ Without charging at the dimension of neither perception nor non-perception, they enter and remain in the dimension of neither perception nor non-perception. They cultivate, develop, and make much of that foundation, ensuring that it’s properly stabilized. 

They\marginnote{11.1} think, ‘Why don’t I, going totally beyond the dimension of neither perception nor non-perception, enter and remain in the cessation of perception and feeling.’ Without charging at the cessation of perception and feeling, they enter and remain in the cessation of perception and feeling. 

When\marginnote{12.1} a mendicant enters and emerges from all these attainments, their mind becomes pliable and workable. With a pliable and workable mind, their immersion becomes limitless and well developed. They become capable of realizing anything that can be realized by insight to which they extend the mind, in each and every case. 

They\marginnote{13.1} might wish: ‘May I wield the many kinds of psychic power: multiplying myself and becoming one again … controlling my body as far as the \textsanskrit{Brahmā} realm.’ They are capable of realizing it, in each and every case. 

They\marginnote{14.1} might wish: ‘With clairaudience that is purified and superhuman, may I hear both kinds of sounds, human and divine, whether near or far.’ They are capable of realizing it, in each and every case. 

They\marginnote{15.1} might wish: ‘May I understand the minds of other beings and individuals, having comprehended them with my mind. May I understand mind with greed as “mind with greed”, and mind without greed as “mind without greed”; mind with hate as “mind with hate”, and mind without hate as “mind without hate”; mind with delusion as “mind with delusion”, and mind without delusion as “mind without delusion”; constricted mind … scattered mind … expansive mind … unexpansive mind … mind that is not supreme … mind that is supreme … mind immersed in \textsanskrit{samādhi} … mind not immersed in \textsanskrit{samādhi} … freed mind … and unfreed mind as “unfreed mind”.’ They are capable of realizing it, in each and every case. 

They\marginnote{16.1} might wish: ‘May I recollect many kinds of past lives. That is: one, two, three, four, five, ten, twenty, thirty, forty, fifty, a hundred, a thousand, a hundred thousand rebirths; many eons of the world contracting, many eons of the world expanding, many eons of the world contracting and expanding. They remember: “There, I was named this, my clan was that, I looked like this, and that was my food. This was how I felt pleasure and pain, and that was how my life ended. When I passed away from that place I was reborn somewhere else. There, too, I was named this, my clan was that, I looked like this, and that was my food. This was how I felt pleasure and pain, and that was how my life ended. When I passed away from that place I was reborn here.” May I recollect my many past lives, with features and details.’ They’re capable of realizing it, in each and every case. 

They\marginnote{17.1} might wish: ‘With clairvoyance that is purified and superhuman, may I see sentient beings passing away and being reborn—inferior and superior, beautiful and ugly, in a good place or a bad place—and understand how sentient beings are reborn according to their deeds.’ They’re capable of realizing it, in each and every case. 

They\marginnote{18.1} might wish: ‘May I realize the undefiled freedom of heart and freedom by wisdom in this very life, and live having realized it with my own insight due to the ending of defilements.’ They’re capable of realizing it, in each and every case.” 

%
\section*{{\suttatitleacronym AN 9.36}{\suttatitletranslation Depending on Absorption }{\suttatitleroot Jhānasutta}}
\addcontentsline{toc}{section}{\tocacronym{AN 9.36} \toctranslation{Depending on Absorption } \tocroot{Jhānasutta}}
\markboth{Depending on Absorption }{Jhānasutta}
\extramarks{AN 9.36}{AN 9.36}

“Mendicants,\marginnote{1.1} I say that the first absorption is a basis for ending the defilements. The second absorption is also a basis for ending the defilements. The third absorption is also a basis for ending the defilements. The fourth absorption is also a basis for ending the defilements. The dimension of infinite space is also a basis for ending the defilements. The dimension of infinite consciousness is also a basis for ending the defilements. The dimension of nothingness is also a basis for ending the defilements. The dimension of neither perception nor non-perception is also a basis for ending the defilements. The cessation of perception and feeling is also a basis for ending the defilements. 

‘The\marginnote{2.1} first absorption is a basis for ending the defilements.’ That’s what I said, but why did I say it? Take a mendicant who, quite secluded from sensual pleasures, secluded from unskillful qualities, enters and remains in the first absorption. They contemplate the phenomena there—included in form, feeling, perception, choices, and consciousness—as impermanent, as suffering, as diseased, as a boil, as a dart, as misery, as an affliction, as alien, as falling apart, as empty, as not-self. They turn their mind away from those things, and apply it to the deathless: ‘This is peaceful; this is sublime—that is, the stilling of all activities, the letting go of all attachments, the ending of craving, fading away, cessation, extinguishment.’ Abiding in that they attain the ending of defilements. If they don’t attain the ending of defilements, with the ending of the five lower fetters they’re reborn spontaneously, because of their passion and love for that meditation. They are extinguished there, and are not liable to return from that world. 

It’s\marginnote{3.1} like an archer or their apprentice who first practices on a straw man or a clay model. At a later time they become a long-distance shooter, a marksman, who shatters large objects. In the same way a noble disciple, quite secluded from sensual pleasures, enters and remains in the first absorption. They contemplate the phenomena there—included in form, feeling, perception, choices, and consciousness—as impermanent, as suffering, as diseased, as a boil, as a dart, as misery, as an affliction, as alien, as falling apart, as empty, as not-self. They turn their mind away from those things, and apply it to the deathless: ‘This is peaceful; this is sublime—that is, the stilling of all activities, the letting go of all attachments, the ending of craving, fading away, cessation, extinguishment.’ Abiding in that they attain the ending of defilements. If they don’t attain the ending of defilements, with the ending of the five lower fetters they’re reborn spontaneously, because of their passion and love for that meditation. They are extinguished there, and are not liable to return from that world. ‘The first absorption is a basis for ending the defilements.’ That’s what I said, and this is why I said it. 

‘The\marginnote{4.1} second absorption is also a basis for ending the defilements.’ … 

‘The\marginnote{4.2} third absorption is also a basis for ending the defilements.’ … 

‘The\marginnote{4.3} fourth absorption is also a basis for ending the defilements.’ … 

‘The\marginnote{6.1} dimension of infinite space is also a basis for ending the defilements.’ That’s what I said, but why did I say it? Take a mendicant who, going totally beyond perceptions of form, with the ending of perceptions of impingement, not focusing on perceptions of diversity, aware that ‘space is infinite’, enters and remains in the dimension of infinite space. They contemplate the phenomena there—included in feeling, perception, choices, and consciousness—as impermanent, as suffering, as diseased, as a boil, as a dart, as misery, as an affliction, as alien, as falling apart, as empty, as not-self. They turn their mind away from those things, and apply it to the deathless: ‘This is peaceful; this is sublime—that is, the stilling of all activities, the letting go of all attachments, the ending of craving, fading away, cessation, extinguishment.’ Abiding in that they attain the ending of defilements. If they don’t attain the ending of defilements, with the ending of the five lower fetters they’re reborn spontaneously, because of their passion and love for that meditation. They are extinguished there, and are not liable to return from that world. 

It’s\marginnote{7.1} like an archer or their apprentice who first practices on a straw man or a clay model. At a later time they become a long-distance shooter, a marksman, who shatters large objects. In the same way, take a mendicant who enters and remains in the dimension of infinite space. … ‘The dimension of infinite space is a basis for ending the defilements.’ That’s what I said, and this is why I said it. 

‘The\marginnote{8.1} dimension of infinite consciousness is a basis for ending the defilements.’ … 

‘The\marginnote{8.2} dimension of nothingness is a basis for ending the defilements.’ That’s what I said, but why did I say it? Take a mendicant who, going totally beyond the dimension of infinite consciousness, aware that ‘there is nothing at all’, enters and remains in the dimension of nothingness. They contemplate the phenomena there—included in feeling, perception, choices, and consciousness—as impermanent, as suffering, as diseased, as a boil, as a dart, as misery, as an affliction, as alien, as falling apart, as empty, as not-self. They turn their mind away from those things, and apply it to the deathless: ‘This is peaceful; this is sublime—that is, the stilling of all activities, the letting go of all attachments, the ending of craving, fading away, cessation, extinguishment.’ Abiding in that they attain the ending of defilements. If they don’t attain the ending of defilements, with the ending of the five lower fetters they’re reborn spontaneously, because of their passion and love for that meditation. They are extinguished there, and are not liable to return from that world. 

It’s\marginnote{9.1} like an archer or their apprentice who first practices on a straw man or a clay model. At a later time they become a long-distance shooter, a marksman, who shatters large objects. In the same way, take a mendicant who, going totally beyond the dimension of infinite consciousness, aware that ‘there is nothing at all’, enters and remains in the dimension of nothingness. They contemplate the phenomena there—included in feeling, perception, choices, and consciousness—as impermanent, as suffering, as diseased, as a boil, as a dart, as misery, as an affliction, as alien, as falling apart, as empty, as not-self. They turn their mind away from those things, and apply it to the deathless: ‘This is peaceful; this is sublime—that is, the stilling of all activities, the letting go of all attachments, the ending of craving, fading away, cessation, extinguishment.’ Abiding in that they attain the ending of defilements. If they don’t attain the ending of defilements, with the ending of the five lower fetters they’re reborn spontaneously, because of their passion and love for that meditation. They are extinguished there, and are not liable to return from that world. ‘The dimension of nothingness is a basis for ending the defilements.’ That’s what I said, and this is why I said it. 

And\marginnote{10.1} so, mendicants, penetration to enlightenment extends as far as attainments with perception. But the two dimensions that depend on these—the dimension of neither perception nor non-perception, and the cessation of perception and feeling—are properly explained by mendicants who are skilled in these attainments and skilled in emerging from them, after they’ve entered them and emerged from them.” 

%
\section*{{\suttatitleacronym AN 9.37}{\suttatitletranslation By Ānanda }{\suttatitleroot Ānandasutta}}
\addcontentsline{toc}{section}{\tocacronym{AN 9.37} \toctranslation{By Ānanda } \tocroot{Ānandasutta}}
\markboth{By Ānanda }{Ānandasutta}
\extramarks{AN 9.37}{AN 9.37}

At\marginnote{1.1} one time Venerable Ānanda was staying near Kosambi, in Ghosita’s Monastery. There Ānanda addressed the mendicants: “Reverends, mendicants!” 

“Reverend,”\marginnote{1.4} they replied. Ānanda said this: 

“It’s\marginnote{2.1} incredible, reverends, it’s amazing! How this Blessed One who knows and sees, the perfected one, the fully awakened Buddha, has found an opening in a confined space. It’s in order to purify sentient beings, to get past sorrow and crying, to make an end of pain and sadness, to end the cycle of suffering, and to realize extinguishment. 

The\marginnote{2.3} eye itself is actually present, and so are those sights. Yet one will not experience that sense-field. The ear itself is actually present, and so are those sounds. Yet one will not experience that sense-field. The nose itself is actually present, and so are those smells. Yet one will not experience that sense-field. The tongue itself is actually present, and so are those tastes. Yet one will not experience that sense-field. The body itself is actually present, and so are those touches. Yet one will not experience that sense-field.” 

When\marginnote{3.1} he said this, Venerable \textsanskrit{Udāyī} said to Venerable Ānanda: 

“Reverend\marginnote{3.2} Ānanda, is one who doesn’t experience that sense-field actually percipient or not?” 

“Reverend,\marginnote{3.3} one who doesn’t experience that sense-field is actually percipient, not non-percipient.” 

“But\marginnote{4.1} what does one who doesn’t experience that sense-field perceive?” 

“It’s\marginnote{4.2} when a mendicant, going totally beyond perceptions of form, with the ending of perceptions of impingement, not focusing on perceptions of diversity, aware that ‘space is infinite’, enters and remains in the dimension of infinite space. One who doesn’t experience that sense-field perceives in this way. 

Furthermore,\marginnote{5.1} a mendicant, going totally beyond the dimension of infinite space, aware that ‘consciousness is infinite’, enters and remains in the dimension of infinite consciousness. One who doesn’t experience that sense-field perceives in this way. 

Furthermore,\marginnote{6.1} a mendicant, going totally beyond the dimension of infinite consciousness, aware that ‘there is nothing at all’, enters and remains in the dimension of nothingness. One who doesn’t experience that sense-field perceives in this way. 

Reverend,\marginnote{7.1} one time I was staying near \textsanskrit{Sāketa} in the deer park in \textsanskrit{Añjana} Wood. Then the nun \textsanskrit{Jaṭilagāhikā} came up to me, bowed, stood to one side, and said to me: ‘Sir, Ānanda, regarding the immersion that does not lean forward or pull back, and is not held in place by forceful suppression. Being free, it’s stable. Being stable, it’s content. Being content, one is not anxious. What did the Buddha say was the fruit of this immersion?’ 

When\marginnote{8.1} she said this, I said to her: ‘Sister, regarding the immersion that does not lean forward or pull back, and is not held in place by forceful suppression. Being free, it’s stable. Being stable, it’s content. Being content, one is not anxious. The Buddha said that the fruit of this immersion is enlightenment.’ One who doesn’t experience that sense-field perceives in this way, too.” 

%
\section*{{\suttatitleacronym AN 9.38}{\suttatitletranslation Brahmin Cosmologists }{\suttatitleroot Lokāyatikasutta}}
\addcontentsline{toc}{section}{\tocacronym{AN 9.38} \toctranslation{Brahmin Cosmologists } \tocroot{Lokāyatikasutta}}
\markboth{Brahmin Cosmologists }{Lokāyatikasutta}
\extramarks{AN 9.38}{AN 9.38}

Then\marginnote{1.1} two brahmin cosmologists went up to the Buddha, and exchanged greetings with him. When the greetings and polite conversation were over, they sat down to one side and said to the Buddha: 

“Master\marginnote{2.1} Gotama, \textsanskrit{Pūraṇa} Kassapa claims to be all-knowing and all-seeing, to know and see everything without exception, thus: ‘Knowledge and vision are constantly and continually present to me, while walking, standing, sleeping, and waking.’ He says: ‘With infinite knowledge I know and see that the world is infinite.’ And the Jain leader \textsanskrit{Nāṭaputta} also claims to be all-knowing and all-seeing, to know and see everything without exception, thus: ‘Knowledge and vision are constantly and continually present to me, while walking, standing, sleeping, and waking.’ He says: ‘With infinite knowledge I know and see that the world is finite.’ These two claim to speak from knowledge, but they directly contradict each other. Which one of them speaks the truth, and which falsehood?” 

“Enough,\marginnote{3.1} brahmins. Leave this aside: ‘These two claim to speak from knowledge, but they directly contradict each other. Which one of them speaks the truth, and which falsehood?’ I will teach you the Dhamma. Listen and pay close attention, I will speak.” 

“Yes\marginnote{3.6} sir,” those brahmins replied. The Buddha said this: 

“Suppose\marginnote{4.1} there were four men standing in the four quarters. Each of them was extremely fast, with an extremely mighty stride. They’re as fast as a light arrow easily shot across the shadow of a palm tree by a well-trained expert archer with a strong bow. Their stride was such that it spanned from the eastern ocean to the western ocean. Then the man standing in the east would say: ‘I will reach the end of the world by traveling.’ Though he’d travel for his whole lifespan of a hundred years—pausing only to eat and drink, go to the toilet, and sleep to dispel weariness—he’d die along the way, never reaching the end of the world. Then the man standing in the west … Then the man standing in the north … Then the man standing in the south would say: ‘I will reach the end of the world by traveling.’ Though he’d travel for his whole lifespan of a hundred years—pausing only to eat and drink, go to the toilet, and sleep to dispel weariness—he’d die along the way, never reaching the end of the world. Why is that? I say it’s not possible to know or see or reach the end of the world by running like this. But I also say there’s no making an end of suffering without reaching the end of the world. 

These\marginnote{5.1} five kinds of sensual stimulation are called the world in the training of the Noble One. What five? Sights known by the eye that are likable, desirable, agreeable, pleasant, sensual, and arousing. Sounds known by the ear … Smells known by the nose … Tastes known by the tongue … Touches known by the body that are likable, desirable, agreeable, pleasant, sensual, and arousing. These five kinds of sensual stimulation are called the world in the training of the Noble One. 

Take\marginnote{6.1} a mendicant who, quite secluded from sensual pleasures, secluded from unskillful qualities, enters and remains in the first absorption, which has the rapture and bliss born of seclusion, while placing the mind and keeping it connected. This is called a mendicant who, having gone to the end of the world, meditates at the end of the world. Others say of them: ‘They’re included in the world, and haven’t yet left the world.’ And I also say this: ‘They’re included in the world, and haven’t yet left the world.’ 

Furthermore,\marginnote{7.1} take a mendicant who, as the placing of the mind and keeping it connected are stilled, enters and remains in the second absorption … third absorption … fourth absorption. This is called a mendicant who, having gone to the end of the world, meditates at the end of the world. Others say of them: ‘They’re included in the world, and haven’t yet left the world.’ And I also say this: ‘They’re included in the world, and haven’t yet left the world.’ 

Furthermore,\marginnote{8.1} take a mendicant who, going totally beyond perceptions of form, with the ending of perceptions of impingement, not focusing on perceptions of diversity, aware that ‘space is infinite’, enters and remains in the dimension of infinite space. This is called a mendicant who, having gone to the end of the world, meditates at the end of the world. Others say of them: ‘They’re included in the world, and haven’t yet left the world.’ And I also say this: ‘They’re included in the world, and haven’t yet left the world.’ 

Furthermore,\marginnote{9.1} take a mendicant who enters and remains in the dimension of infinite consciousness. … the dimension of nothingness … the dimension of neither perception nor non-perception. This is called a mendicant who, having gone to the end of the world, meditates at the end of the world. Others say of them: ‘They’re included in the world, and haven’t yet left the world.’ And I also say this: ‘They’re included in the world, and haven’t yet left the world.’ 

Furthermore,\marginnote{10.1} take a mendicant who, going totally beyond the dimension of neither perception nor non-perception, enters and remains in the cessation of perception and feeling. And, having seen with wisdom, their defilements come to an end. This is called a mendicant who, having gone to the end of the world, meditates at the end of the world. And they’ve crossed over clinging to the world.” 

%
\section*{{\suttatitleacronym AN 9.39}{\suttatitletranslation The War Between the Gods and the Demons }{\suttatitleroot Devāsurasaṅgāmasutta}}
\addcontentsline{toc}{section}{\tocacronym{AN 9.39} \toctranslation{The War Between the Gods and the Demons } \tocroot{Devāsurasaṅgāmasutta}}
\markboth{The War Between the Gods and the Demons }{Devāsurasaṅgāmasutta}
\extramarks{AN 9.39}{AN 9.39}

“Once\marginnote{1.1} upon a time, mendicants, a battle was fought between the gods and the demons. In that battle the demons won and the gods lost. Defeated, the gods fled north with the demons in pursuit. 

Then\marginnote{1.4} the gods thought, ‘The demons are still in pursuit. Why don’t we engage them in battle a second time?’ And so a second battle was fought between the gods and the demons. And for a second time the demons won and the gods lost. Defeated, the gods fled north with the demons in pursuit. 

Then\marginnote{2.1} the gods thought, ‘The demons are still in pursuit. Why don’t we engage them in battle a third time?’ And so a third battle was fought between the gods and the demons. And for a third time the demons won and the gods lost. Defeated and terrified, the gods fled right into the castle of the gods. 

When\marginnote{2.7} they had entered their castle, they thought, ‘Now we’re in a secure location and the demons can’t do anything to us.’ The demons also thought, ‘Now the gods are in a secure location and we can’t do anything to them.’ 

Once\marginnote{3.1} upon a time, a battle was fought between the gods and the demons. In that battle the gods won and the demons lost. Defeated, the demons fled south with the gods in pursuit. 

Then\marginnote{3.4} the demons thought, ‘The gods are still in pursuit. Why don’t we engage them in battle a second time?’ And so a second battle was fought between the gods and the demons. And for a second time the gods won and the demons lost. Defeated, the demons fled south with the gods in pursuit. 

Then\marginnote{4.1} the demons thought, ‘The gods are still in pursuit. Why don’t we engage them in battle a third time?’ And so a third battle was fought between the gods and the demons. And for a third time the gods won and the demons lost. Defeated and terrified, the demons fled right into the citadel of the demons. 

When\marginnote{4.7} they had entered their citadel, they thought, ‘Now we’re in a secure location and the gods can’t do anything to us.’ And the gods also thought, ‘Now the demons are in a secure location and we can’t do anything to them.’ 

In\marginnote{5.1} the same way, a mendicant, quite secluded from sensual pleasures, secluded from unskillful qualities, enters and remains in the first absorption, which has the rapture and bliss born of seclusion, while placing the mind and keeping it connected. At such a time the mendicant thinks, ‘Now I’m in a secure location and \textsanskrit{Māra} can’t do anything to me.’ And \textsanskrit{Māra} the Wicked also thinks, ‘Now the mendicant is in a secure location and we can’t do anything to them.’ 

When,\marginnote{6.1} as the placing of the mind and keeping it connected are stilled, a mendicant enters and remains in the second absorption … third absorption … fourth absorption. At such a time the mendicant thinks, ‘Now I’m in a secure location and \textsanskrit{Māra} can’t do anything to me.’ And \textsanskrit{Māra} the Wicked also thinks, ‘Now the mendicant is in a secure location and we can’t do anything to them.’ 

A\marginnote{7.1} mendicant, going totally beyond perceptions of form, with the ending of perceptions of impingement, not focusing on perceptions of diversity, aware that ‘space is infinite’, enters and remains in the dimension of infinite space. At such a time they are called a mendicant who has blinded \textsanskrit{Māra}, put out his eyes without a trace, and gone where the Wicked One cannot see. 

A\marginnote{8.1} mendicant, going totally beyond the dimension of infinite space, aware that ‘consciousness is infinite’, enters and remains in the dimension of infinite consciousness. … Going totally beyond the dimension of infinite consciousness, aware that ‘there is nothing at all’, they enter and remain in the dimension of nothingness. … Going totally beyond the dimension of nothingness, they enter and remain in the dimension of neither perception nor non-perception. … 

Going\marginnote{8.4} totally beyond the dimension of neither perception nor non-perception, they enter and remain in the cessation of perception and feeling. And, having seen with wisdom, their defilements come to an end. At such a time they are called a mendicant who has blinded \textsanskrit{Māra}, put out his eyes without a trace, and gone where the Wicked One cannot see. And they’ve crossed over clinging to the world.” 

%
\section*{{\suttatitleacronym AN 9.40}{\suttatitletranslation The Simile of the Bull Elephant in the Forest }{\suttatitleroot Nāgasutta}}
\addcontentsline{toc}{section}{\tocacronym{AN 9.40} \toctranslation{The Simile of the Bull Elephant in the Forest } \tocroot{Nāgasutta}}
\markboth{The Simile of the Bull Elephant in the Forest }{Nāgasutta}
\extramarks{AN 9.40}{AN 9.40}

“Mendicants,\marginnote{1.1} when a wild bull elephant is engrossed in the pasture, but other elephants—males, females, younglings, or cubs—got there first and trampled the grass, the wild bull elephant is horrified, repelled, and disgusted by that. When the wild bull elephant is engrossed in the pasture, but other elephants—males, females, younglings, or cubs—eat the broken branches that he has dragged down, the wild bull elephant is horrified, repelled, and disgusted by that. When a wild bull elephant has plunged into the pool, but other elephants—males, females, younglings, or cubs—got there first and stirred up the water with their trunks, the wild bull elephant is horrified, repelled, and disgusted by that. When a wild bull elephant has come out of the pool and the female elephants bump into him, the wild bull elephant is horrified, repelled, and disgusted by that. 

At\marginnote{2.1} that time the wild bull elephant thinks: ‘These days I live crowded by other males, females, younglings, and cubs. I eat the grass they’ve trampled, and they eat the broken branches I’ve dragged down. I drink muddy water, and after my bath the female elephants bump into me. Why don’t I live alone, withdrawn from the herd?’ After some time he lives alone, withdrawn from the herd, and he eats untrampled grass, and other elephants don’t eat the broken branches he has dragged down. He doesn’t drink muddy water, and the female elephants don’t bump into him after his bath. 

At\marginnote{3.1} that time the wild bull elephant thinks: ‘Formerly I lived crowded by other males, females, younglings, and cubs. I ate the grass they’d trampled, and they ate the broken branches I’d dragged down. I drank muddy water, and after my bath the female elephants bumped into me. Now I live alone, and I’m free of all these things.’ He breaks off a branch and scratches his body, happily relieving his itches. 

In\marginnote{4.1} the same way, when a mendicant lives crowded by monks, nuns, laymen, and laywomen; by rulers and their ministers, and by teachers of other paths and their disciples, they think: ‘These days I live crowded by monks, nuns, laymen, and laywomen; by rulers and their ministers, and teachers of other paths and their disciples. Why don’t I live alone, withdrawn from the group?’ They frequent a secluded lodging—a wilderness, the root of a tree, a hill, a ravine, a mountain cave, a charnel ground, a forest, the open air, a heap of straw. Gone to a wilderness, or to the root of a tree, or to an empty hut, they sit down cross-legged, with their body straight, and establish mindfulness right there. 

Giving\marginnote{5.1} up desire for the world, they meditate with a heart rid of desire, cleansing the mind of desire. Giving up ill will and malevolence, they meditate with a mind rid of ill will, full of compassion for all living beings, cleansing the mind of ill will. Giving up dullness and drowsiness, they meditate with a mind free of dullness and drowsiness, perceiving light, mindful and aware, cleansing the mind of dullness and drowsiness. Giving up restlessness and remorse, they meditate without restlessness, their mind peaceful inside, cleansing the mind of restlessness and remorse. Giving up doubt, they meditate having gone beyond doubt, not undecided about skillful qualities, cleansing the mind of doubt. They give up these five hindrances, corruptions of the heart that weaken wisdom. Then, quite secluded from sensual pleasures, secluded from unskillful qualities, they enter and remain in the first absorption, which has the rapture and bliss born of seclusion, while placing the mind and keeping it connected. They happily relieve their itches. As the placing of the mind and keeping it connected are stilled, they enter and remain in the second absorption … third absorption … fourth absorption. They happily relieve their itches. 

Going\marginnote{6.1} totally beyond perceptions of form, with the ending of perceptions of impingement, not focusing on perceptions of diversity, aware that ‘space is infinite’, they enter and remain in the dimension of infinite space. They happily relieve their itches. Going totally beyond the dimension of infinite space, aware that ‘consciousness is infinite’, they enter and remain in the dimension of infinite consciousness. … Going totally beyond the dimension of infinite consciousness, aware that ‘there is nothing at all’, they enter and remain in the dimension of nothingness. … Going totally beyond the dimension of nothingness, they enter and remain in the dimension of neither perception nor non-perception. … Furthermore, take a mendicant who, going totally beyond the dimension of neither perception nor non-perception, enters and remains in the cessation of perception and feeling. And, having seen with wisdom, their defilements come to an end. They happily relieve their itches.” 

%
\section*{{\suttatitleacronym AN 9.41}{\suttatitletranslation With the Householder Tapussa }{\suttatitleroot Tapussasutta}}
\addcontentsline{toc}{section}{\tocacronym{AN 9.41} \toctranslation{With the Householder Tapussa } \tocroot{Tapussasutta}}
\markboth{With the Householder Tapussa }{Tapussasutta}
\extramarks{AN 9.41}{AN 9.41}

At\marginnote{1.1} one time the Buddha was staying in the land of the Mallas, near the Mallian town named Uruvelakappa. 

Then\marginnote{1.2} the Buddha robed up in the morning and, taking his bowl and robe, entered Uruvelakappa for alms. Then, after the meal, on his return from almsround, he addressed Venerable Ānanda, “Ānanda, you stay right here, while I plunge deep into the Great Wood for the day’s meditation.” 

“Yes,\marginnote{1.5} sir,” Ānanda replied. Then the Buddha plunged deep into the Great Wood and sat at the root of a tree for the day’s meditation. 

The\marginnote{2.1} householder Tapussa went up to Venerable Ānanda, bowed, sat down to one side, and said to him: 

“Sir,\marginnote{3.1} Ānanda, we are laypeople who enjoy sensual pleasures. We like sensual pleasures, we love them and take joy in them. But renunciation seems like an abyss. I have heard that in this teaching and training there are very young mendicants whose minds are eager for renunciation; they’re confident, settled, and decided about it. They see it as peaceful. Renunciation is the dividing line between the multitude and the mendicants in this teaching and training.” 

“Householder,\marginnote{4.1} we should see the Buddha about this matter. Come, let’s go to the Buddha and inform him about this. As he answers, so we’ll remember it.” 

“Yes,\marginnote{5.1} sir,” replied Tapussa. Then Ānanda together with Tapussa went to the Buddha, bowed, and sat down to one side. Ānanda told him what had happened. 

“That’s\marginnote{7.1} so true, Ānanda! That’s so true! Before my awakening—when I was still unawakened but intent on awakening—I too thought, ‘Renunciation is good! Seclusion is good!’ But my mind wasn’t eager for renunciation; it wasn’t confident, settled, and decided about it. I didn’t see it as peaceful. Then I thought, ‘What is the cause, what is the reason why my mind isn’t eager for renunciation, and not confident, settled, and decided about it? Why don’t I see it as peaceful?’ Then I thought, ‘I haven’t seen the drawbacks of sensual pleasures, and so I haven’t cultivated that. I haven’t realized the benefits of renunciation, and so I haven’t developed that. That’s why my mind isn’t eager for renunciation, and not confident, settled, and decided about it. And it’s why I don’t see it as peaceful.’ Then I thought, ‘Suppose that, seeing the drawbacks of sensual pleasures, I were to cultivate that. And suppose that, realizing the benefits of renunciation, I were to develop that. It’s possible that my mind would be eager for renunciation; it would be confident, settled, and decided about it. And I would see it as peaceful.’ And so, after some time, I saw the drawbacks of sensual pleasures and cultivated that, and I realized the benefits of renunciation and developed that. Then my mind was eager for renunciation; it was confident, settled, and decided about it. I saw it as peaceful. And so, quite secluded from sensual pleasures, secluded from unskillful qualities, I entered and remained in the first absorption, which has the rapture and bliss born of seclusion, while placing the mind and keeping it connected. While I was in that meditation, perceptions and attentions accompanied by sensual pleasures beset me, and that was an affliction for me. Suppose a happy person were to experience pain; that would be an affliction for them. In the same way, when perceptions and attentions accompanied by sensual pleasures beset me, that was an affliction for me. 

Then\marginnote{8.1} I thought, ‘Why don’t I, as the placing of the mind and keeping it connected are stilled … enter and remain in the second absorption?’ But my mind wasn’t eager to stop placing the mind; it wasn’t confident, settled, and decided about it. I didn’t see it as peaceful. Then I thought, ‘What is the cause, what is the reason why my mind isn’t eager to stop placing the mind, and not confident, settled, and decided about it? Why don’t I see it as peaceful?’ Then I thought, ‘I haven’t seen the drawbacks of placing the mind, and so I haven’t cultivated that. I haven’t realized the benefits of not placing the mind, and so I haven’t developed that. That’s why my mind isn’t eager to stop placing the mind, and not confident, settled, and decided about it. And it’s why I don’t see it as peaceful.’ Then I thought, ‘Suppose that, seeing the drawbacks of placing the mind, I were to cultivate that. And suppose that, realizing the benefits of not placing the mind, I were to develop that. It’s possible that my mind would be eager to stop placing the mind; it would be confident, settled, and decided about it. And I would see it as peaceful.’ And so, after some time, I saw the drawbacks of placing the mind and cultivated that, and I realized the benefits of not placing the mind and developed that. Then my mind was eager to stop placing the mind; it was confident, settled, and decided about it. I saw it as peaceful. And so, as the placing of the mind and keeping it connected were stilled … I entered and remained in the second absorption. While I was in that meditation, perceptions and attentions accompanied by placing the mind beset me, and that was an affliction for me. Suppose a happy person were to experience pain; that would be an affliction for them. In the same way, when perceptions and attentions accompanied by placing the mind and keeping it connected beset me, that was an affliction for me. 

Then\marginnote{9.1} I thought, ‘Why don’t I, with the fading away of rapture, enter and remain in the third absorption, where I will meditate with equanimity, mindful and aware, personally experiencing the bliss of which the noble ones declare, “Equanimous and mindful, one meditates in bliss”?’ But my mind wasn’t eager for freedom from rapture; it wasn’t confident, settled, and decided about it. I didn’t see it as peaceful. Then I thought, ‘What is the cause, what is the reason why my mind isn’t eager for freedom from rapture, and not confident, settled, and decided about it? Why don’t I see it as peaceful?’ Then I thought, ‘I haven’t seen the drawbacks of rapture, and so I haven’t cultivated that. I haven’t realized the benefits of freedom from rapture, and so I haven’t developed that. That’s why my mind isn’t eager for freedom from rapture, and not confident, settled, and decided about it. And it’s why I don’t see it as peaceful.’ Then I thought, ‘Suppose that, seeing the drawbacks of rapture, I were to cultivate that. And suppose that, realizing the benefits of freedom from rapture, I were to develop that. It’s possible that my mind would be eager to be free from rapture; it would be confident, settled, and decided about it. And I would see it as peaceful.’ And so, after some time, I saw the drawbacks of rapture and cultivated that, and I realized the benefits of freedom from rapture and developed that. Then my mind was eager for freedom from rapture; it was confident, settled, and decided about it. I saw it as peaceful. And so, with the fading away of rapture … I entered and remained in the third absorption. While I was in that meditation, perceptions and attentions accompanied by rapture beset me, and that was an affliction for me. Suppose a happy person were to experience pain; that would be an affliction for them. In the same way, when perceptions and attentions accompanied by rapture beset me, that was an affliction for me. 

Then\marginnote{10.1} I thought, ‘Why don’t I, with the giving up of pleasure and pain, and the ending of former happiness and sadness, enter and remain in the fourth absorption, without pleasure or pain, with pure equanimity and mindfulness?’ But my mind wasn’t eager to be without pleasure and pain; it wasn’t confident, settled, and decided about it. I didn’t see it as peaceful. Then I thought, ‘What is the cause, what is the reason why my mind isn’t eager to be without pleasure and pain, and not confident, settled, and decided about it? Why don’t I see it as peaceful?’ Then I thought, ‘I haven’t seen the drawbacks of equanimous bliss, and so I haven’t cultivated that. I haven’t realized the benefits of being without pleasure and pain, and so I haven’t developed that. That’s why my mind isn’t eager to be without pleasure and pain, and not confident, settled, and decided about it. And it’s why I don’t see it as peaceful.’ Then I thought, ‘Suppose that, seeing the drawbacks of equanimous bliss, I were to cultivate that. And suppose that, realizing the benefits of being without pleasure and pain, I were to develop that. It’s possible that my mind would be eager to be without pleasure and pain; it would be confident, settled, and decided about it. And I would see it as peaceful.’ And so, after some time, I saw the drawbacks of equanimous bliss and cultivated that, and I realized the benefits of being without pleasure and pain and developed that. Then my mind was eager to be without pleasure and pain; it was confident, settled, and decided about it. I saw it as peaceful. And so, giving up pleasure and pain … I entered and remained in the fourth absorption. While I was in that meditation, perceptions and attentions accompanied by equanimous bliss beset me, and that was an affliction for me. Suppose a happy person were to experience pain; that would be an affliction for them. In the same way, when perceptions and attentions accompanied by equanimous bliss beset me, that was an affliction for me. 

Then\marginnote{11.1} I thought, ‘Why don’t I, going totally beyond perceptions of form, with the ending of perceptions of impingement, not focusing on perceptions of diversity, aware that “space is infinite”, enter and remain in the dimension of infinite space?’ But my mind wasn’t eager for the dimension of infinite space; it wasn’t confident, settled, and decided about it. I didn’t see it as peaceful. Then I thought, ‘What is the cause, what is the reason why my mind isn’t eager for the dimension of infinite space, and not confident, settled, and decided about it? Why don’t I see it as peaceful?’ Then I thought, ‘I haven’t seen the drawbacks of forms, and so I haven’t cultivated that. I haven’t realized the benefits of the dimension of infinite space, and so I haven’t developed that. That’s why my mind isn’t eager for the dimension of infinite space, and not confident, settled, and decided about it. And it’s why I don’t see it as peaceful.’ Then I thought, ‘Suppose that, seeing the drawbacks of forms, I were to cultivate that. And suppose that, realizing the benefits of the dimension of infinite space, I were to develop that. It’s possible that my mind would be eager for the dimension of infinite space; it would be confident, settled, and decided about it. And I would see it as peaceful.’ And so, after some time, I saw the drawbacks of forms and cultivated that, and I realized the benefits of the dimension of infinite space and developed that. Then my mind was eager for the dimension of infinite space; it was confident, settled, and decided about it. I saw it as peaceful. And so, going totally beyond perceptions of form, with the ending of perceptions of impingement, not focusing on perceptions of diversity, aware that ‘space is infinite’, I entered and remained in the dimension of infinite space. While I was in that meditation, perceptions and attentions accompanied by forms beset me, and that was an affliction for me. Suppose a happy person were to experience pain; that would be an affliction for them. In the same way, when perceptions and attentions accompanied by forms beset me, that was an affliction for me. 

Then\marginnote{12.1} I thought, ‘Why don’t I, going totally beyond the dimension of infinite space, aware that “consciousness is infinite”, enter and remain in the dimension of infinite consciousness?’ But my mind wasn’t eager for the dimension of infinite consciousness; it wasn’t confident, settled, and decided about it. I didn’t see it as peaceful. Then I thought, ‘What is the cause, what is the reason why my mind isn’t eager for the dimension of infinite consciousness, and not confident, settled, and decided about it? Why don’t I see it as peaceful?’ Then I thought, ‘I haven’t seen the drawbacks of the dimension of infinite space, and so I haven’t cultivated that. I haven’t realized the benefits of the dimension of infinite consciousness, and so I haven’t developed that. That’s why my mind isn’t eager for the dimension of infinite consciousness, and not confident, settled, and decided about it. And it’s why I don’t see it as peaceful.’ Then I thought, ‘Suppose that, seeing the drawbacks of the dimension of infinite space, I were to cultivate that. And suppose that, realizing the benefits of the dimension of infinite consciousness, I were to develop that. It’s possible that my mind would be eager for the dimension of infinite consciousness; it would be confident, settled, and decided about it. And I would see it as peaceful.’ And so, after some time, I saw the drawbacks of the dimension of infinite space and cultivated that, and I realized the benefits of the dimension of infinite consciousness and developed that. Then my mind was eager for the dimension of infinite consciousness; it was confident, settled, and decided about it. I saw it as peaceful. And so, going totally beyond the dimension of infinite space, aware that ‘consciousness is infinite’, I entered and remained in the dimension of infinite consciousness. While I was in that meditation, perceptions and attentions accompanied by the dimension of infinite space beset me, and that was an affliction for me. Suppose a happy person were to experience pain; that would be an affliction for them. In the same way, when perceptions and attentions accompanied by the dimension of infinite space beset me, that was an affliction for me. 

Then\marginnote{13.1} I thought, ‘Why don’t I, going totally beyond the dimension of infinite consciousness, aware that “there is nothing at all”, enter and remain in the dimension of nothingness?’ But my mind wasn’t eager for the dimension of nothingness; it wasn’t confident, settled, and decided about it. I didn’t see it as peaceful. Then I thought, ‘What is the cause, what is the reason why my mind isn’t eager for the dimension of nothingness, and not confident, settled, and decided about it? Why don’t I see it as peaceful?’ Then I thought, ‘I haven’t seen the drawbacks of the dimension of infinite consciousness, and so I haven’t cultivated that. I haven’t realized the benefits of the dimension of nothingness, and so I haven’t developed that. That’s why my mind isn’t eager for the dimension of nothingness, and not confident, settled, and decided about it. And it’s why I don’t see it as peaceful.’ Then I thought, ‘Suppose that, seeing the drawbacks of the dimension of infinite consciousness, I were to cultivate that. And suppose that, realizing the benefits of the dimension of nothingness, I were to develop that. It’s possible that my mind would be eager for the dimension of nothingness; it would be confident, settled, and decided about it. And I would see it as peaceful.’ And so, after some time, I saw the drawbacks of the dimension of infinite consciousness and cultivated that, and I realized the benefits of the dimension of nothingness and developed that. Then my mind was eager for the dimension of nothingness; it was confident, settled, and decided about it. I saw it as peaceful. And so, going totally beyond the dimension of infinite consciousness, aware that ‘there is nothing at all’, I entered and remained in the dimension of nothingness. While I was in that meditation, perceptions and attentions accompanied by the dimension of infinite consciousness beset me, and that was an affliction for me. Suppose a happy person were to experience pain; that would be an affliction for them. In the same way, when perceptions and attentions accompanied by the dimension of infinite consciousness beset me, that was an affliction for me. 

Then\marginnote{14.1} I thought, ‘Why don’t I, going totally beyond the dimension of nothingness, enter and remain in the dimension of neither perception nor non-perception?’ But my mind wasn’t eager for the dimension of neither perception nor non-perception; it wasn’t confident, settled, and decided about it. I didn’t see it as peaceful. Then I thought, ‘What is the cause, what is the reason why my mind isn’t eager for the dimension of neither perception nor non-perception, and not confident, settled, and decided about it? Why don’t I see it as peaceful?’ Then I thought, ‘I haven’t seen the drawbacks of the dimension of nothingness, and so I haven’t cultivated that. I haven’t realized the benefits of the dimension of neither perception nor non-perception, and so I haven’t developed that. That’s why my mind isn’t eager for the dimension of neither perception nor non-perception, and not confident, settled, and decided about it. And it’s why I don’t see it as peaceful.’ Then I thought, ‘Suppose that, seeing the drawbacks of the dimension of nothingness, I were to cultivate that. And suppose that, realizing the benefits of the dimension of neither perception nor non-perception, I were to develop that. It’s possible that my mind would be eager for the dimension of neither perception nor non-perception; it would be confident, settled, and decided about it. And I would see it as peaceful.’ And so, after some time, I saw the drawbacks of the dimension of nothingness and cultivated that, and I realized the benefits of the dimension of neither perception nor non-perception and developed that. Then my mind was eager for the dimension of neither perception nor non-perception; it was confident, settled, and decided about it. I saw it as peaceful. And so, going totally beyond the dimension of nothingness, I entered and remained in the dimension of neither perception nor non-perception. While I was in that meditation, perceptions and attentions accompanied by the dimension of nothingness beset me, and that was an affliction for me. Suppose a happy person were to experience pain; that would be an affliction for them. In the same way, when perceptions and attentions accompanied by the dimension of nothingness beset me, that was an affliction for me. 

Then\marginnote{15.1} I thought, ‘Why don’t I, going totally beyond the dimension of neither perception nor non-perception, enter and remain in the cessation of perception and feeling?’ But my mind wasn’t eager for the cessation of perception and feeling; it wasn’t confident, settled, and decided about it. I didn’t see it as peaceful. Then I thought, ‘What is the cause, what is the reason why my mind isn’t eager for the cessation of perception and feeling, and not confident, settled, and decided about it? Why don’t I see it as peaceful?’ Then I thought, ‘I haven’t seen the drawbacks of the dimension of neither perception nor non-perception, and so I haven’t cultivated that. I haven’t realized the benefits of the cessation of perception and feeling, and so I haven’t developed that. That’s why my mind isn’t eager for the cessation of perception and feeling, and not confident, settled, and decided about it. And it’s why I don’t see it as peaceful.’ Then I thought, ‘Suppose that, seeing the drawbacks of the dimension of neither perception nor non-perception, I were to cultivate that. And suppose that, realizing the benefits of the cessation of perception and feeling, I were to develop that. It’s possible that my mind would be eager for cessation of perception and feeling; it would be confident, settled, and decided about it. And I would see it as peaceful.’ And so, after some time, I saw the drawbacks of the dimension of neither perception nor non-perception and cultivated that, and I realized the benefits of the cessation of perception and feeling and developed that. Then my mind was eager for the cessation of perception and feeling; it was confident, settled, and decided about it. I saw it as peaceful. And so, going totally beyond the dimension of neither perception nor non-perception, I entered and remained in the cessation of perception and feeling. And, having seen with wisdom, my defilements were ended. 

As\marginnote{16.1} long as I hadn’t entered into and withdrawn from these nine progressive meditative attainments in both forward and reverse order, I didn’t announce my supreme perfect awakening in this world with its gods, \textsanskrit{Māras}, and \textsanskrit{Brahmās}, this population with its ascetics and brahmins, its gods and humans. 

But\marginnote{16.2} when I had entered into and withdrawn from these nine progressive meditative attainments in both forward and reverse order, I announced my supreme perfect awakening in this world with its gods, \textsanskrit{Māras}, and \textsanskrit{Brahmās}, this population with its ascetics and brahmins, its gods and humans. 

Knowledge\marginnote{16.3} and vision arose in me: ‘My freedom is unshakable; this is my last rebirth; now there’ll be no more future lives.’” 

%
\addtocontents{toc}{\let\protect\contentsline\protect\nopagecontentsline}
\chapter*{The Chapter on Similarity }
\addcontentsline{toc}{chapter}{\tocchapterline{The Chapter on Similarity }}
\addtocontents{toc}{\let\protect\contentsline\protect\oldcontentsline}

%
\section*{{\suttatitleacronym AN 9.42}{\suttatitletranslation Cramped }{\suttatitleroot Sambādhasutta}}
\addcontentsline{toc}{section}{\tocacronym{AN 9.42} \toctranslation{Cramped } \tocroot{Sambādhasutta}}
\markboth{Cramped }{Sambādhasutta}
\extramarks{AN 9.42}{AN 9.42}

At\marginnote{1.1} one time Venerable Ānanda was staying near Kosambi, in Ghosita’s Monastery. Then Venerable \textsanskrit{Udāyī} went up to Venerable Ānanda and exchanged greetings with him. When the greetings and polite conversation were over, he sat down to one side and said to Ānanda, “Reverend, this was said by the god \textsanskrit{Pañcālacaṇḍa}: 

\begin{verse}%
‘The\marginnote{2.1} opening amid confinement \\
was discovered by the Buddha of vast intelligence, \\
who woke up to absorption, \\
the sage, the solitary bull.’ 

%
\end{verse}

But\marginnote{3.1} what is confinement, and what is the opening amid confinement that the Buddha spoke of?” 

“Reverend,\marginnote{3.2} these five kinds of sensual stimulation are called ‘confinement’ by the Buddha. What five? Sights known by the eye that are likable, desirable, agreeable, pleasant, sensual, and arousing. Sounds known by the ear … Smells known by the nose … Tastes known by the tongue … Touches known by the body that are likable, desirable, agreeable, pleasant, sensual, and arousing. These are the five kinds of sensual stimulation that are called ‘confinement’ by the Buddha. 

Now,\marginnote{4.1} take a mendicant who, quite secluded from sensual pleasures … enters and remains in the first absorption. To this extent the Buddha spoke of creating an opening amid confinement in a qualified sense. But it is still confined. Confined by what? Whatever placing of the mind and keeping it connected has not ceased is the confinement there. 

Furthermore,\marginnote{5.1} take a mendicant who, as the placing of the mind and keeping it connected are stilled … enters and remains in the second absorption. To this extent the Buddha spoke of creating an opening amid confinement in a qualified sense. But it is still confined. Confined by what? Whatever rapture has not ceased is the confinement there. 

Furthermore,\marginnote{6.1} take a mendicant who, with the fading away of rapture … enters and remains in the third absorption. To this extent the Buddha spoke of creating an opening amid confinement in a qualified sense. But it is still confined. Confined by what? Whatever equanimous bliss has not ceased is the confinement there. 

Furthermore,\marginnote{7.1} take a mendicant who, giving up pleasure and pain … enters and remains in the fourth absorption. To this extent the Buddha spoke of creating an opening amid confinement in a qualified sense. But it is still confined. Confined by what? Whatever perception of form has not ceased is the confinement there. 

Furthermore,\marginnote{8.1} take a mendicant who, going totally beyond perceptions of form, with the ending of perceptions of impingement, not focusing on perceptions of diversity, aware that ‘space is infinite’, enters and remains in the dimension of infinite space. To this extent the Buddha spoke of creating an opening amid confinement in a qualified sense. But it is still confined. Confined by what? Whatever perception of the dimension of infinite space has not ceased is the confinement there. 

Furthermore,\marginnote{9.1} a mendicant, going totally beyond the dimension of infinite space, aware that ‘consciousness is infinite’, enters and remains in the dimension of infinite consciousness. To this extent the Buddha spoke of creating an opening amid confinement in a qualified sense. But it is still confined. Confined by what? Whatever perception of the dimension of infinite consciousness has not ceased is the confinement there. 

Furthermore,\marginnote{10.1} a mendicant, going totally beyond the dimension of infinite consciousness, aware that ‘there is nothing at all’, enters and remains in the dimension of nothingness. To this extent the Buddha spoke of creating an opening amid confinement in a qualified sense. But it is still confined. Confined by what? Whatever perception of the dimension of nothingness has not ceased is the confinement there. 

Furthermore,\marginnote{11.1} take a mendicant who, going totally beyond the dimension of nothingness, enters and remains in the dimension of neither perception nor non-perception. To this extent the Buddha spoke of creating an opening amid confinement in a qualified sense. But it is still confined. Confined by what? Whatever perception of neither perception nor non-perception has not ceased is the confinement there. 

Furthermore,\marginnote{12.1} take a mendicant who, going totally beyond the dimension of neither perception nor non-perception, enters and remains in the cessation of perception and feeling. And, having seen with wisdom, their defilements come to an end. To this extent the Buddha spoke of creating an opening amid confinement in a definitive sense.” 

%
\section*{{\suttatitleacronym AN 9.43}{\suttatitletranslation A Personal Witness }{\suttatitleroot Kāyasakkhīsutta}}
\addcontentsline{toc}{section}{\tocacronym{AN 9.43} \toctranslation{A Personal Witness } \tocroot{Kāyasakkhīsutta}}
\markboth{A Personal Witness }{Kāyasakkhīsutta}
\extramarks{AN 9.43}{AN 9.43}

“Reverend,\marginnote{1.1} they speak of a person called ‘personal witness’. What is the personal witness that the Buddha spoke of?” 

“First,\marginnote{2.1} take a mendicant who, quite secluded from sensual pleasures … enters and remains in the first absorption. They meditate directly experiencing that dimension in every way. To this extent the Buddha spoke of the personal witness in a qualified sense. 

Furthermore,\marginnote{3.1} take a mendicant who, as the placing of the mind and keeping it connected are stilled, enters and remains in the second absorption … third absorption … fourth absorption. They meditate directly experiencing that dimension in every way. To this extent the Buddha spoke of the personal witness in a qualified sense. 

Furthermore,\marginnote{4.1} take a mendicant who, going totally beyond perceptions of form, with the ending of perceptions of impingement, not focusing on perceptions of diversity, aware that ‘space is infinite’, enters and remains in the dimension of infinite space. They meditate directly experiencing that dimension in every way. To this extent the Buddha spoke of the personal witness in a qualified sense. Furthermore, take a mendicant who enters and remains in the dimension of infinite consciousness … the dimension of nothingness … the dimension of neither perception nor non-perception … 

Furthermore,\marginnote{5.1} take a mendicant who, going totally beyond the dimension of neither perception nor non-perception, enters and remains in the cessation of perception and feeling. And, having seen with wisdom, their defilements come to an end. They meditate directly experiencing that dimension in every way. To this extent the Buddha spoke of the personal witness in a definitive sense.” 

%
\section*{{\suttatitleacronym AN 9.44}{\suttatitletranslation Freed by Wisdom }{\suttatitleroot Paññāvimuttasutta}}
\addcontentsline{toc}{section}{\tocacronym{AN 9.44} \toctranslation{Freed by Wisdom } \tocroot{Paññāvimuttasutta}}
\markboth{Freed by Wisdom }{Paññāvimuttasutta}
\extramarks{AN 9.44}{AN 9.44}

“Reverend,\marginnote{1.1} they speak of a person called ‘freed by wisdom’. What is the one freed by wisdom that the Buddha spoke of?” 

“First,\marginnote{2.1} take a mendicant who, quite secluded from sensual pleasures … enters and remains in the first absorption. And they understand that with wisdom. To this extent the Buddha spoke of the one freed by wisdom in a qualified sense. … 

Furthermore,\marginnote{3.1} take a mendicant who, going totally beyond the dimension of neither perception nor non-perception, enters and remains in the cessation of perception and feeling. And, having seen with wisdom, their defilements come to an end. And they understand that with wisdom. To this extent the Buddha spoke of the one freed by wisdom in a definitive sense.” 

%
\section*{{\suttatitleacronym AN 9.45}{\suttatitletranslation Freed Both Ways }{\suttatitleroot Ubhatobhāgavimuttasutta}}
\addcontentsline{toc}{section}{\tocacronym{AN 9.45} \toctranslation{Freed Both Ways } \tocroot{Ubhatobhāgavimuttasutta}}
\markboth{Freed Both Ways }{Ubhatobhāgavimuttasutta}
\extramarks{AN 9.45}{AN 9.45}

“Reverend,\marginnote{1.1} they speak of a person called ‘freed both ways’. What is the one freed both ways that the Buddha spoke of?” 

“First,\marginnote{2.1} take a mendicant who, quite secluded from sensual pleasures … enters and remains in the first absorption. They meditate directly experiencing that dimension in every way. And they understand that with wisdom. To this extent the Buddha spoke of the one freed both ways in a qualified sense. … 

Furthermore,\marginnote{3.1} take a mendicant who, going totally beyond the dimension of neither perception nor non-perception, enters and remains in the cessation of perception and feeling. And, having seen with wisdom, their defilements come to an end. They meditate directly experiencing that dimension in every way. And they understand that with wisdom. To this extent the Buddha spoke of the one freed both ways in a definitive sense.” 

%
\section*{{\suttatitleacronym AN 9.46}{\suttatitletranslation In This Very Life }{\suttatitleroot Sandiṭṭhikadhammasutta}}
\addcontentsline{toc}{section}{\tocacronym{AN 9.46} \toctranslation{In This Very Life } \tocroot{Sandiṭṭhikadhammasutta}}
\markboth{In This Very Life }{Sandiṭṭhikadhammasutta}
\extramarks{AN 9.46}{AN 9.46}

“Reverend,\marginnote{1.1} they speak of ‘a teaching visible in this very life’. In what way did the Buddha speak of a teaching visible in this very life?” 

“First,\marginnote{2.1} take a mendicant who, quite secluded from sensual pleasures … enters and remains in the first absorption. To this extent the Buddha spoke of the teaching visible in this very life in a qualified sense. … 

Furthermore,\marginnote{3.1} take a mendicant who, going totally beyond the dimension of neither perception nor non-perception, enters and remains in the cessation of perception and feeling. And, having seen with wisdom, their defilements come to an end. To this extent the Buddha spoke of the teaching visible in this very life in a definitive sense.” 

%
\section*{{\suttatitleacronym AN 9.47}{\suttatitletranslation Extinguishment Is Visible in This Very Life }{\suttatitleroot Sandiṭṭhikanibbānasutta}}
\addcontentsline{toc}{section}{\tocacronym{AN 9.47} \toctranslation{Extinguishment Is Visible in This Very Life } \tocroot{Sandiṭṭhikanibbānasutta}}
\markboth{Extinguishment Is Visible in This Very Life }{Sandiṭṭhikanibbānasutta}
\extramarks{AN 9.47}{AN 9.47}

“Reverend,\marginnote{1.1} they say that ‘extinguishment is visible in this very life’. In what way did the Buddha say extinguishment is visible in this very life?” 

“First,\marginnote{2.1} take a mendicant who, quite secluded from sensual pleasures … enters and remains in the first absorption. To this extent the Buddha said that extinguishment is visible in this very life in a qualified sense. … 

Furthermore,\marginnote{3.1} take a mendicant who, going totally beyond the dimension of neither perception nor non-perception, enters and remains in the cessation of perception and feeling. And, having seen with wisdom, their defilements come to an end. To this extent the Buddha said that extinguishment is visible in this very life in a definitive sense.” 

%
\section*{{\suttatitleacronym AN 9.48}{\suttatitletranslation Extinguishment }{\suttatitleroot Nibbānasutta}}
\addcontentsline{toc}{section}{\tocacronym{AN 9.48} \toctranslation{Extinguishment } \tocroot{Nibbānasutta}}
\markboth{Extinguishment }{Nibbānasutta}
\extramarks{AN 9.48}{AN 9.48}

“Reverend,\marginnote{1.1} they speak of ‘extinguishment’. …” 

%
\section*{{\suttatitleacronym AN 9.49}{\suttatitletranslation Final Extinguishment }{\suttatitleroot Parinibbānasutta}}
\addcontentsline{toc}{section}{\tocacronym{AN 9.49} \toctranslation{Final Extinguishment } \tocroot{Parinibbānasutta}}
\markboth{Final Extinguishment }{Parinibbānasutta}
\extramarks{AN 9.49}{AN 9.49}

“Reverend,\marginnote{1.1} they speak of ‘final extinguishment’. …” 

%
\section*{{\suttatitleacronym AN 9.50}{\suttatitletranslation Extinguishment in a Certain Respect }{\suttatitleroot Tadaṅganibbānasutta}}
\addcontentsline{toc}{section}{\tocacronym{AN 9.50} \toctranslation{Extinguishment in a Certain Respect } \tocroot{Tadaṅganibbānasutta}}
\markboth{Extinguishment in a Certain Respect }{Tadaṅganibbānasutta}
\extramarks{AN 9.50}{AN 9.50}

“Reverend,\marginnote{1.1} they speak of ‘extinguishment in a certain respect’. …” 

%
\section*{{\suttatitleacronym AN 9.51}{\suttatitletranslation Extinguishment in the Present Life }{\suttatitleroot Diṭṭhadhammanibbānasutta}}
\addcontentsline{toc}{section}{\tocacronym{AN 9.51} \toctranslation{Extinguishment in the Present Life } \tocroot{Diṭṭhadhammanibbānasutta}}
\markboth{Extinguishment in the Present Life }{Diṭṭhadhammanibbānasutta}
\extramarks{AN 9.51}{AN 9.51}

“Reverend,\marginnote{1.1} they speak of ‘extinguishment in the present life’. In what way did the Buddha speak of extinguishment in the present life?” 

“First,\marginnote{2.1} take a mendicant who, quite secluded from sensual pleasures … enters and remains in the first absorption. To this extent the Buddha spoke of extinguishment in the present life in a qualified sense. … 

Furthermore,\marginnote{3.1} take a mendicant who, going totally beyond the dimension of neither perception nor non-perception, enters and remains in the cessation of perception and feeling. And, having seen with wisdom, their defilements come to an end. To this extent the Buddha spoke of extinguishment in the present life in a definitive sense.” 

%
\addtocontents{toc}{\let\protect\contentsline\protect\nopagecontentsline}
\pannasa{The Second Fifty }
\addcontentsline{toc}{pannasa}{The Second Fifty }
\markboth{}{}
\addtocontents{toc}{\let\protect\contentsline\protect\oldcontentsline}

%
\addtocontents{toc}{\let\protect\contentsline\protect\nopagecontentsline}
\chapter*{The Chapter on a Safe Place }
\addcontentsline{toc}{chapter}{\tocchapterline{The Chapter on a Safe Place }}
\addtocontents{toc}{\let\protect\contentsline\protect\oldcontentsline}

%
\section*{{\suttatitleacronym AN 9.52}{\suttatitletranslation A Safe Place }{\suttatitleroot Khemasutta}}
\addcontentsline{toc}{section}{\tocacronym{AN 9.52} \toctranslation{A Safe Place } \tocroot{Khemasutta}}
\markboth{A Safe Place }{Khemasutta}
\extramarks{AN 9.52}{AN 9.52}

“Reverend,\marginnote{1.1} they speak of ‘a safe place’. In what way did the Buddha speak of a safe place?” 

“First,\marginnote{2.1} take a mendicant who, quite secluded from sensual pleasures … enters and remains in the first absorption. To this extent the Buddha spoke of a safe place in a qualified sense. … 

Furthermore,\marginnote{3.1} take a mendicant who, going totally beyond the dimension of neither perception nor non-perception, enters and remains in the cessation of perception and feeling. And, having seen with wisdom, their defilements come to an end. To this extent the Buddha spoke of a safe place in a definitive sense.” 

%
\section*{{\suttatitleacronym AN 9.53}{\suttatitletranslation Reaching a Safe Place }{\suttatitleroot Khemappattasutta}}
\addcontentsline{toc}{section}{\tocacronym{AN 9.53} \toctranslation{Reaching a Safe Place } \tocroot{Khemappattasutta}}
\markboth{Reaching a Safe Place }{Khemappattasutta}
\extramarks{AN 9.53}{AN 9.53}

“Reverend,\marginnote{1.1} they speak of ‘reaching a safe place’. …” 

%
\section*{{\suttatitleacronym AN 9.54}{\suttatitletranslation The Deathless }{\suttatitleroot Amatasutta}}
\addcontentsline{toc}{section}{\tocacronym{AN 9.54} \toctranslation{The Deathless } \tocroot{Amatasutta}}
\markboth{The Deathless }{Amatasutta}
\extramarks{AN 9.54}{AN 9.54}

“Reverend,\marginnote{1.1} they speak of ‘the deathless’. …” 

%
\section*{{\suttatitleacronym AN 9.55}{\suttatitletranslation Reaching the Deathless }{\suttatitleroot Amatappattasutta}}
\addcontentsline{toc}{section}{\tocacronym{AN 9.55} \toctranslation{Reaching the Deathless } \tocroot{Amatappattasutta}}
\markboth{Reaching the Deathless }{Amatappattasutta}
\extramarks{AN 9.55}{AN 9.55}

“Reverend,\marginnote{1.1} they speak of ‘reaching the deathless’. …” 

%
\section*{{\suttatitleacronym AN 9.56}{\suttatitletranslation A Place Without Fear }{\suttatitleroot Abhayasutta}}
\addcontentsline{toc}{section}{\tocacronym{AN 9.56} \toctranslation{A Place Without Fear } \tocroot{Abhayasutta}}
\markboth{A Place Without Fear }{Abhayasutta}
\extramarks{AN 9.56}{AN 9.56}

“Reverend,\marginnote{1.1} they speak of ‘a place without fear’. …” 

%
\section*{{\suttatitleacronym AN 9.57}{\suttatitletranslation Reaching a Place Without Fear }{\suttatitleroot Abhayappattasutta}}
\addcontentsline{toc}{section}{\tocacronym{AN 9.57} \toctranslation{Reaching a Place Without Fear } \tocroot{Abhayappattasutta}}
\markboth{Reaching a Place Without Fear }{Abhayappattasutta}
\extramarks{AN 9.57}{AN 9.57}

“Reverend,\marginnote{1.1} they speak of ‘reaching a place without fear’. …” 

%
\section*{{\suttatitleacronym AN 9.58}{\suttatitletranslation Tranquility }{\suttatitleroot Passaddhisutta}}
\addcontentsline{toc}{section}{\tocacronym{AN 9.58} \toctranslation{Tranquility } \tocroot{Passaddhisutta}}
\markboth{Tranquility }{Passaddhisutta}
\extramarks{AN 9.58}{AN 9.58}

“Reverend,\marginnote{1.1} they speak of ‘tranquility’. …” 

%
\section*{{\suttatitleacronym AN 9.59}{\suttatitletranslation Progressive Tranquility }{\suttatitleroot Anupubbapassaddhisutta}}
\addcontentsline{toc}{section}{\tocacronym{AN 9.59} \toctranslation{Progressive Tranquility } \tocroot{Anupubbapassaddhisutta}}
\markboth{Progressive Tranquility }{Anupubbapassaddhisutta}
\extramarks{AN 9.59}{AN 9.59}

“Reverend,\marginnote{1.1} they speak of ‘progressive tranquility’. …” 

%
\section*{{\suttatitleacronym AN 9.60}{\suttatitletranslation Cessation }{\suttatitleroot Nirodhasutta}}
\addcontentsline{toc}{section}{\tocacronym{AN 9.60} \toctranslation{Cessation } \tocroot{Nirodhasutta}}
\markboth{Cessation }{Nirodhasutta}
\extramarks{AN 9.60}{AN 9.60}

“Reverend,\marginnote{1.1} they speak of ‘cessation’. …” 

%
\section*{{\suttatitleacronym AN 9.61}{\suttatitletranslation Progressive Cessation }{\suttatitleroot Anupubbanirodhasutta}}
\addcontentsline{toc}{section}{\tocacronym{AN 9.61} \toctranslation{Progressive Cessation } \tocroot{Anupubbanirodhasutta}}
\markboth{Progressive Cessation }{Anupubbanirodhasutta}
\extramarks{AN 9.61}{AN 9.61}

“Reverend,\marginnote{1.1} they speak of ‘progressive cessation’. What is the progressive cessation that the Buddha spoke of?” 

“First,\marginnote{2.1} take a mendicant who, quite secluded from sensual pleasures … enters and remains in the first absorption. To this extent the Buddha spoke of progressive cessation in a qualified sense. … 

Furthermore,\marginnote{3.1} take a mendicant who, going totally beyond the dimension of neither perception nor non-perception, enters and remains in the cessation of perception and feeling. And, having seen with wisdom, their defilements come to an end. To this extent the Buddha spoke of progressive cessation in a definitive sense.” 

%
\section*{{\suttatitleacronym AN 9.62}{\suttatitletranslation Requirements for Perfection }{\suttatitleroot Abhabbasutta}}
\addcontentsline{toc}{section}{\tocacronym{AN 9.62} \toctranslation{Requirements for Perfection } \tocroot{Abhabbasutta}}
\markboth{Requirements for Perfection }{Abhabbasutta}
\extramarks{AN 9.62}{AN 9.62}

“Mendicants,\marginnote{1.1} without giving up nine things you can’t realize perfection. What nine? Greed, hate, delusion, anger, hostility, disdain, contempt, jealousy, and stinginess. Without giving up these nine things you can’t realize perfection. 

After\marginnote{2.1} giving up nine things you can realize perfection. What nine? Greed, hate, delusion, anger, hostility, disdain, contempt, jealousy, and stinginess. After giving up these nine things you can realize the fruit of perfection.” 

%
\addtocontents{toc}{\let\protect\contentsline\protect\nopagecontentsline}
\chapter*{The Chapter on Mindfulness Meditation }
\addcontentsline{toc}{chapter}{\tocchapterline{The Chapter on Mindfulness Meditation }}
\addtocontents{toc}{\let\protect\contentsline\protect\oldcontentsline}

%
\section*{{\suttatitleacronym AN 9.63}{\suttatitletranslation Weaknesses in Training and Mindfulness Meditation }{\suttatitleroot Sikkhādubbalyasutta}}
\addcontentsline{toc}{section}{\tocacronym{AN 9.63} \toctranslation{Weaknesses in Training and Mindfulness Meditation } \tocroot{Sikkhādubbalyasutta}}
\markboth{Weaknesses in Training and Mindfulness Meditation }{Sikkhādubbalyasutta}
\extramarks{AN 9.63}{AN 9.63}

“Mendicants,\marginnote{1.1} there are these five weaknesses when you’re training. What five? Killing living creatures, stealing, sexual misconduct, lying, and using alcoholic drinks that cause negligence. These are the five weaknesses when you’re training. 

To\marginnote{2.1} give up these five weaknesses in your training you should develop the four kinds of mindfulness meditation. What four? It’s when a mendicant meditates by observing an aspect of the body—keen, aware, and mindful, rid of desire and aversion for the world. They meditate observing an aspect of feelings … They meditate observing an aspect of the mind … They meditate observing an aspect of principles—keen, aware, and mindful, rid of desire and aversion for the world. To give up those five weaknesses in your training you should develop these four kinds of mindfulness meditation.” 

%
\section*{{\suttatitleacronym AN 9.64}{\suttatitletranslation Hindrances }{\suttatitleroot Nīvaraṇasutta}}
\addcontentsline{toc}{section}{\tocacronym{AN 9.64} \toctranslation{Hindrances } \tocroot{Nīvaraṇasutta}}
\markboth{Hindrances }{Nīvaraṇasutta}
\extramarks{AN 9.64}{AN 9.64}

“Mendicants,\marginnote{1.1} there are these five hindrances. What five? Sensual desire, ill will, dullness and drowsiness, restlessness and remorse, and doubt. These are the five hindrances. 

To\marginnote{2.1} give up these five hindrances you should develop the four kinds of mindfulness meditation. What four? It’s when a mendicant meditates by observing an aspect of the body—keen, aware, and mindful, rid of desire and aversion for the world. They meditate observing an aspect of feelings … They meditate observing an aspect of the mind … They meditate observing an aspect of principles—keen, aware, and mindful, rid of desire and aversion for the world. To give up those five hindrances you should develop these four kinds of mindfulness meditation.” 

%
\section*{{\suttatitleacronym AN 9.65}{\suttatitletranslation Kinds of Sensual Stimulation }{\suttatitleroot Kāmaguṇasutta}}
\addcontentsline{toc}{section}{\tocacronym{AN 9.65} \toctranslation{Kinds of Sensual Stimulation } \tocroot{Kāmaguṇasutta}}
\markboth{Kinds of Sensual Stimulation }{Kāmaguṇasutta}
\extramarks{AN 9.65}{AN 9.65}

“Mendicants,\marginnote{1.1} there are these five kinds of sensual stimulation. What five? Sights known by the eye that are likable, desirable, agreeable, pleasant, sensual, and arousing. Sounds known by the ear … Smells known by the nose … Tastes known by the tongue … Touches known by the body that are likable, desirable, agreeable, pleasant, sensual, and arousing. These are the five kinds of sensual stimulation. 

To\marginnote{2.1} give up these five kinds of sensual stimulation you should develop the four kinds of mindfulness meditation. …” 

%
\section*{{\suttatitleacronym AN 9.66}{\suttatitletranslation Grasping Aggregates }{\suttatitleroot Upādānakkhandhasutta}}
\addcontentsline{toc}{section}{\tocacronym{AN 9.66} \toctranslation{Grasping Aggregates } \tocroot{Upādānakkhandhasutta}}
\markboth{Grasping Aggregates }{Upādānakkhandhasutta}
\extramarks{AN 9.66}{AN 9.66}

“Mendicants,\marginnote{1.1} there are these five grasping aggregates. What five? The grasping aggregates of form, feeling, perception, choices, and consciousness. These are the five grasping aggregates. 

To\marginnote{2.1} give up these five grasping aggregates you should develop the four kinds of mindfulness meditation. …” 

%
\section*{{\suttatitleacronym AN 9.67}{\suttatitletranslation Lower Fetters }{\suttatitleroot Orambhāgiyasutta}}
\addcontentsline{toc}{section}{\tocacronym{AN 9.67} \toctranslation{Lower Fetters } \tocroot{Orambhāgiyasutta}}
\markboth{Lower Fetters }{Orambhāgiyasutta}
\extramarks{AN 9.67}{AN 9.67}

“Mendicants,\marginnote{1.1} there are five lower fetters. What five? Identity view, doubt, misapprehension of precepts and observances, sensual desire, and ill will. These are the five lower fetters. 

To\marginnote{2.1} give up these five lower fetters you should develop the four kinds of mindfulness meditation. …” 

%
\section*{{\suttatitleacronym AN 9.68}{\suttatitletranslation Places of Rebirth }{\suttatitleroot Gatisutta}}
\addcontentsline{toc}{section}{\tocacronym{AN 9.68} \toctranslation{Places of Rebirth } \tocroot{Gatisutta}}
\markboth{Places of Rebirth }{Gatisutta}
\extramarks{AN 9.68}{AN 9.68}

“Mendicants,\marginnote{1.1} there are five destinations. What five? Hell, the animal realm, the ghost realm, humanity, and the gods. These are the five destinations. 

To\marginnote{2.1} give up these five destinations you should develop the four kinds of mindfulness meditation. …” 

%
\section*{{\suttatitleacronym AN 9.69}{\suttatitletranslation Stinginess }{\suttatitleroot Macchariyasutta}}
\addcontentsline{toc}{section}{\tocacronym{AN 9.69} \toctranslation{Stinginess } \tocroot{Macchariyasutta}}
\markboth{Stinginess }{Macchariyasutta}
\extramarks{AN 9.69}{AN 9.69}

“Mendicants,\marginnote{1.1} there are these five kinds of stinginess. What five? Stinginess with dwellings, families, material possessions, praise, and the teaching. These are the five kinds of stinginess. 

To\marginnote{2.1} give up these five kinds of stinginess you should develop the four kinds of mindfulness meditation. …” 

%
\section*{{\suttatitleacronym AN 9.70}{\suttatitletranslation Higher Fetters }{\suttatitleroot Uddhambhāgiyasutta}}
\addcontentsline{toc}{section}{\tocacronym{AN 9.70} \toctranslation{Higher Fetters } \tocroot{Uddhambhāgiyasutta}}
\markboth{Higher Fetters }{Uddhambhāgiyasutta}
\extramarks{AN 9.70}{AN 9.70}

“Mendicants,\marginnote{1.1} there are five higher fetters. What five? Desire for rebirth in the realm of luminous form, desire for rebirth in the formless realm, conceit, restlessness, and ignorance. These are the five higher fetters. 

To\marginnote{2.1} give up these five higher fetters you should develop the four kinds of mindfulness meditation. …” 

%
\section*{{\suttatitleacronym AN 9.71}{\suttatitletranslation Emotional Barrenness }{\suttatitleroot Cetokhilasutta}}
\addcontentsline{toc}{section}{\tocacronym{AN 9.71} \toctranslation{Emotional Barrenness } \tocroot{Cetokhilasutta}}
\markboth{Emotional Barrenness }{Cetokhilasutta}
\extramarks{AN 9.71}{AN 9.71}

“Mendicants,\marginnote{1.1} there are five kinds of emotional barrenness. What five? Firstly, a mendicant has doubts about the Teacher. They’re uncertain, undecided, and lacking confidence. This being so, their mind doesn’t incline toward keenness, commitment, persistence, and striving. This is the first kind of emotional barrenness. 

Furthermore,\marginnote{2.1} a mendicant has doubts about the teaching … the \textsanskrit{Saṅgha} … the training … A mendicant is angry and upset with their spiritual companions, resentful and closed off. This being so, their mind doesn’t incline toward keenness, commitment, persistence, and striving. This is the fifth kind of emotional barrenness. 

To\marginnote{3.1} give up these five kinds of emotional barrenness you should develop the four kinds of mindfulness meditation. …” 

%
\section*{{\suttatitleacronym AN 9.72}{\suttatitletranslation Emotional Shackles }{\suttatitleroot Cetasovinibandhasutta}}
\addcontentsline{toc}{section}{\tocacronym{AN 9.72} \toctranslation{Emotional Shackles } \tocroot{Cetasovinibandhasutta}}
\markboth{Emotional Shackles }{Cetasovinibandhasutta}
\extramarks{AN 9.72}{AN 9.72}

“Mendicants,\marginnote{1.1} there are these five emotional shackles. What five? Firstly, a mendicant isn’t free of greed, desire, fondness, thirst, passion, and craving for sensual pleasures. This being so, their mind doesn’t incline toward keenness, commitment, persistence, and striving. This is the first emotional shackle. 

Furthermore,\marginnote{2.1} a mendicant isn’t free of greed for the body … They’re not free of greed for form … They eat as much as they like until their belly is full, then indulge in the pleasures of sleeping, lying down, and drowsing … They lead the spiritual life wishing to be reborn in one of the orders of gods: ‘By this precept or observance or mortification or spiritual life, may I become one of the gods!’ This being so, their mind doesn’t incline toward keenness, commitment, persistence, and striving. This is the fifth emotional shackle. These are the five emotional shackles. 

To\marginnote{3.1} give up these five emotional shackles you should develop the four kinds of mindfulness meditation. What four? It’s when a mendicant meditates by observing an aspect of the body—keen, aware, and mindful, rid of desire and aversion for the world. They meditate observing an aspect of feelings … They meditate observing an aspect of the mind … They meditate observing an aspect of principles—keen, aware, and mindful, rid of desire and aversion for the world. To give up these five emotional shackles you should develop these four kinds of mindfulness meditation.” 

%
\addtocontents{toc}{\let\protect\contentsline\protect\nopagecontentsline}
\chapter*{The Chapter on Right Efforts }
\addcontentsline{toc}{chapter}{\tocchapterline{The Chapter on Right Efforts }}
\addtocontents{toc}{\let\protect\contentsline\protect\oldcontentsline}

%
\section*{{\suttatitleacronym AN 9.73}{\suttatitletranslation Weaknesses in Training and Effort }{\suttatitleroot Sikkhasutta}}
\addcontentsline{toc}{section}{\tocacronym{AN 9.73} \toctranslation{Weaknesses in Training and Effort } \tocroot{Sikkhasutta}}
\markboth{Weaknesses in Training and Effort }{Sikkhasutta}
\extramarks{AN 9.73}{AN 9.73}

“Mendicants,\marginnote{1.1} there are these five weaknesses when you’re training. What five? Killing living creatures, stealing, sexual misconduct, lying, and using alcoholic drinks that cause negligence. These are the five weaknesses when you’re training. 

To\marginnote{2.1} give up these five weaknesses in your training you should develop the four right efforts. What four? It’s when a mendicant generates enthusiasm, tries, makes an effort, exerts the mind, and strives so that bad, unskillful qualities don’t arise. They generate enthusiasm, try, make an effort, exert the mind, and strive so that bad, unskillful qualities that have arisen are given up. They generate enthusiasm, try, make an effort, exert the mind, and strive so that skillful qualities arise. They generate enthusiasm, try, make an effort, exert the mind, and strive so that skillful qualities that have arisen remain, are not lost, but increase, mature, and are completed by development. To give up these five weaknesses in your training you should develop these four right efforts.” 

%
\section*{{\suttatitleacronym AN 9.74–81}{\suttatitletranslation Hindrances, Etc. }{\suttatitleroot Nīvaraṇasuttādi}}
\addcontentsline{toc}{section}{\tocacronym{AN 9.74–81} \toctranslation{Hindrances, Etc. } \tocroot{Nīvaraṇasuttādi}}
\markboth{Hindrances, Etc. }{Nīvaraṇasuttādi}
\extramarks{AN 9.74–81}{AN 9.74–81}

(This\marginnote{1.1} should be expanded in detail as in the chapter on mindfulness meditation.) 

%
\section*{{\suttatitleacronym AN 9.82}{\suttatitletranslation Emotional Shackles }{\suttatitleroot Cetasovinibandhasutta}}
\addcontentsline{toc}{section}{\tocacronym{AN 9.82} \toctranslation{Emotional Shackles } \tocroot{Cetasovinibandhasutta}}
\markboth{Emotional Shackles }{Cetasovinibandhasutta}
\extramarks{AN 9.82}{AN 9.82}

“Mendicants,\marginnote{1.1} there are these five emotional shackles. What five? Firstly, a mendicant isn’t free of greed, desire, fondness, thirst, passion, and craving for sensual pleasures. … These are the five emotional shackles. 

To\marginnote{2.1} give up these five emotional shackles you should develop the four right efforts. What four? It’s when a mendicant generates enthusiasm, tries, makes an effort, exerts the mind, and strives so that bad, unskillful qualities don’t arise. … so that unskillful qualities are given up … so that skillful qualities arise … They generate enthusiasm, try, make an effort, exert the mind, and strive so that skillful qualities that have arisen remain, are not lost, but increase, mature, and are completed by development. To give up these five emotional shackles you should develop these four right efforts.” 

%
\addtocontents{toc}{\let\protect\contentsline\protect\nopagecontentsline}
\chapter*{The Chapter on Bases of Psychic Power }
\addcontentsline{toc}{chapter}{\tocchapterline{The Chapter on Bases of Psychic Power }}
\addtocontents{toc}{\let\protect\contentsline\protect\oldcontentsline}

%
\section*{{\suttatitleacronym AN 9.83}{\suttatitletranslation Weaknesses in Training and the Bases of Psychic Power }{\suttatitleroot Sikkhasutta}}
\addcontentsline{toc}{section}{\tocacronym{AN 9.83} \toctranslation{Weaknesses in Training and the Bases of Psychic Power } \tocroot{Sikkhasutta}}
\markboth{Weaknesses in Training and the Bases of Psychic Power }{Sikkhasutta}
\extramarks{AN 9.83}{AN 9.83}

“Mendicants,\marginnote{1.1} there are these five weaknesses when you’re training. What five? Killing living creatures, stealing, sexual misconduct, lying, and using alcoholic drinks that cause negligence. These are the five weaknesses when you’re training. 

To\marginnote{2.1} give up these five weaknesses in your training you should develop the four bases of psychic power. What four? It’s when a mendicant develops the basis of psychic power that has immersion due to enthusiasm, and active effort. They develop the basis of psychic power that has immersion due to energy, and active effort. They develop the basis of psychic power that has immersion due to mental development, and active effort. They develop the basis of psychic power that has immersion due to inquiry, and active effort. To give up these five weaknesses in your training you should develop these four bases of psychic power.” 

%
\section*{{\suttatitleacronym AN 9.84–91}{\suttatitletranslation Hindrances, Etc. }{\suttatitleroot Nīvaraṇasuttādi}}
\addcontentsline{toc}{section}{\tocacronym{AN 9.84–91} \toctranslation{Hindrances, Etc. } \tocroot{Nīvaraṇasuttādi}}
\markboth{Hindrances, Etc. }{Nīvaraṇasuttādi}
\extramarks{AN 9.84–91}{AN 9.84–91}

(This\marginnote{1.1} should be expanded in detail as in the chapter on mindfulness meditation.) 

%
\section*{{\suttatitleacronym AN 9.92}{\suttatitletranslation Emotional Shackles }{\suttatitleroot Cetasovinibandhasutta}}
\addcontentsline{toc}{section}{\tocacronym{AN 9.92} \toctranslation{Emotional Shackles } \tocroot{Cetasovinibandhasutta}}
\markboth{Emotional Shackles }{Cetasovinibandhasutta}
\extramarks{AN 9.92}{AN 9.92}

“Mendicants,\marginnote{1.1} there are these five emotional shackles. What five? Firstly, a mendicant isn’t free of greed for sensual pleasures. … These are the five emotional shackles. 

To\marginnote{2.1} give up these five emotional shackles you should develop the four bases of psychic power. What four? It’s when a mendicant develops the basis of psychic power that has immersion due to enthusiasm, and active effort. They develop the basis of psychic power that has immersion due to energy, and active effort. They develop the basis of psychic power that has immersion due to mental development, and active effort. They develop the basis of psychic power that has immersion due to inquiry, and active effort. To give up these five emotional shackles you should develop these four bases of psychic power.” 

%
\addtocontents{toc}{\let\protect\contentsline\protect\nopagecontentsline}
\chapter*{Abbreviated Texts Beginning With Greed }
\addcontentsline{toc}{chapter}{\tocchapterline{Abbreviated Texts Beginning With Greed }}
\addtocontents{toc}{\let\protect\contentsline\protect\oldcontentsline}

%
\section*{{\suttatitleacronym AN 9.93}{\suttatitletranslation Untitled Discourse on Greed (1st) }{\suttatitleroot \textasciitilde }}
\addcontentsline{toc}{section}{\tocacronym{AN 9.93} \toctranslation{Untitled Discourse on Greed (1st) } \tocroot{\textasciitilde }}
\markboth{Untitled Discourse on Greed (1st) }{\textasciitilde }
\extramarks{AN 9.93}{AN 9.93}

“For\marginnote{1.1} insight into greed, nine things should be developed. What nine? The perceptions of ugliness, death, repulsiveness of food, dissatisfaction with the whole world, impermanence, suffering in impermanence, not-self in suffering, giving up, and fading away. For insight into greed, these nine things should be developed.” 

%
\section*{{\suttatitleacronym AN 9.94}{\suttatitletranslation Untitled Discourse on Greed (2nd) }{\suttatitleroot \textasciitilde }}
\addcontentsline{toc}{section}{\tocacronym{AN 9.94} \toctranslation{Untitled Discourse on Greed (2nd) } \tocroot{\textasciitilde }}
\markboth{Untitled Discourse on Greed (2nd) }{\textasciitilde }
\extramarks{AN 9.94}{AN 9.94}

“For\marginnote{1.1} insight into greed, nine things should be developed. What nine? The first absorption, the second absorption, the third absorption, the fourth absorption, the dimension of infinite space, the dimension of infinite consciousness, the dimension of nothingness, the dimension of neither perception nor non-perception, and the cessation of perception and feeling. For insight into greed, these nine things should be developed.” 

%
\section*{{\suttatitleacronym AN 9.95–112}{\suttatitletranslation Untitled Discourses on Greed }{\suttatitleroot \textasciitilde }}
\addcontentsline{toc}{section}{\tocacronym{AN 9.95–112} \toctranslation{Untitled Discourses on Greed } \tocroot{\textasciitilde }}
\markboth{Untitled Discourses on Greed }{\textasciitilde }
\extramarks{AN 9.95–112}{AN 9.95–112}

“For\marginnote{1.1} the complete understanding of greed … complete ending … giving up … ending … vanishing … fading away … cessation … giving away … letting go of greed … these nine things should be developed.” 

%
\section*{{\suttatitleacronym AN 9.113–432}{\suttatitletranslation Untitled Discourses on Hate, Etc. }{\suttatitleroot \textasciitilde }}
\addcontentsline{toc}{section}{\tocacronym{AN 9.113–432} \toctranslation{Untitled Discourses on Hate, Etc. } \tocroot{\textasciitilde }}
\markboth{Untitled Discourses on Hate, Etc. }{\textasciitilde }
\extramarks{AN 9.113–432}{AN 9.113–432}

“For\marginnote{1.1} insight into hate … delusion … anger … hostility … disdain … contempt … jealousy … stinginess … deceitfulness … deviousness … obstinacy … aggression … conceit … arrogance … vanity … for insight into negligence … complete understanding … complete ending … giving up … ending … vanishing … fading away … ceasing … giving away … letting go of negligence … these nine things should be developed.” 

\scendbook{The Book of the Nines is finished. }

%
\backmatter%
\chapter*{Colophon}
\addcontentsline{toc}{chapter}{Colophon}
\markboth{Colophon}{Colophon}

\section*{The Translator}

Bhikkhu Sujato was born as Anthony Aidan Best on 4/11/1966 in Perth, Western Australia. He grew up in the pleasant suburbs of Mt Lawley and Attadale alongside his sister Nicola, who was the good child. His mother, Margaret Lorraine Huntsman née Pinder, said “he’ll either be a priest or a poet”, while his father, Anthony Thomas Best, advised him to “never do anything for money”. He attended Aquinas College, a Catholic school, where he decided to become an atheist. At the University of WA he studied philosophy, aiming to learn what he wanted to do with his life. Finding that what he wanted to do was play guitar, he dropped out. His main band was named Martha’s Vineyard, which achieved modest success in the indie circuit. 

A seemingly random encounter with a roadside joey took him to Thailand, where he entered his first meditation retreat at Wat Ram Poeng, Chieng Mai in 1992. Feeling the call to the Buddha’s path, he took full ordination in Wat Pa Nanachat in 1994, where his teachers were Ajahn Pasanno and Ajahn Jayasaro. In 1997 he returned to Perth to study with Ajahn Brahm at Bodhinyana Monastery. 

He spent several years practicing in seclusion in Malaysia and Thailand before establishing Santi Forest Monastery in Bundanoon, NSW, in 2003. There he was instrumental in supporting the establishment of the Theravada bhikkhuni order in Australia and advocating for women’s rights. He continues to teach in Australia and globally, with a special concern for the moral implications of climate change and other forms of environmental destruction. He has published a series of books of original and groundbreaking research on early Buddhism. 

In 2005 he founded SuttaCentral together with Rod Bucknell and John Kelly. In 2015, seeing the need for a complete, accurate, plain English translation of the Pali texts, he undertook the task, spending nearly three years in isolation on the isle of Qi Mei off the coast of the nation of Taiwan. He completed the four main \textsanskrit{Nikāyas} in 2018, and the early books of the Khuddaka \textsanskrit{Nikāya} were complete by 2021. All this work is dedicated to the public domain and is entirely free of copyright encumbrance. 

In 2019 he returned to Sydney where he established Lokanta Vihara (The Monastery at the End of the World). 

\section*{Creation Process}

Primary source was the digital \textsanskrit{Mahāsaṅgīti} edition of the Pali \textsanskrit{Tipiṭaka}. Translated from the Pali, with reference to several English translations, especially those of Bhikkhu Bodhi.

\section*{The Translation}

This translation was part of a project to translate the four Pali \textsanskrit{Nikāyas} with the following aims: plain, approachable English; consistent terminology; accurate rendition of the Pali; free of copyright. It was made during 2016–2018 while Bhikkhu Sujato was staying in Qimei, Taiwan.

\section*{About SuttaCentral}

SuttaCentral publishes early Buddhist texts. Since 2005 we have provided root texts in Pali, Chinese, Sanskrit, Tibetan, and other languages, parallels between these texts, and translations in many modern languages. We build on the work of generations of scholars, and offer our contribution freely.

SuttaCentral is driven by volunteer contributions, and in addition we employ professional developers. We offer a sponsorship program for high quality translations from the original languages. Financial support for SuttaCentral is handled by the SuttaCentral Development Trust, a charitable trust registered in Australia.

\section*{About Bilara}

“Bilara” means “cat” in Pali, and it is the name of our Computer Assisted Translation (CAT) software. Bilara is a web app that enables translators to translate early Buddhist texts into their own language. These translations are published on SuttaCentral with the root text and translation side by side.

\section*{About SuttaCentral Editions}

The SuttaCentral Editions project makes high quality books from selected Bilara translations. These are published in formats including HTML, EPUB, PDF, and print.

If you want to print any of our Editions, please let us know and we will help prepare a file to your specifications.

%
\end{document}