\documentclass[12pt,openany]{book}%
\usepackage{lastpage}%
%
\usepackage[inner=1in, outer=1in, top=.7in, bottom=1in, papersize={6in,9in}, headheight=13pt]{geometry}
\usepackage{polyglossia}
\usepackage[12pt]{moresize}
\usepackage{soul}%
\usepackage{microtype}
\usepackage{tocbasic}
\usepackage{realscripts}
\usepackage{epigraph}%
\usepackage{setspace}%
\usepackage{sectsty}
\usepackage{fontspec}
\usepackage{marginnote}
\usepackage[bottom]{footmisc}
\usepackage{enumitem}
\usepackage{fancyhdr}
\usepackage{extramarks}
\usepackage{graphicx}
\usepackage{verse}
\usepackage{relsize}
\usepackage{etoolbox}
\usepackage[a-3u]{pdfx}

\hypersetup{
colorlinks=true,
urlcolor=black,
linkcolor=black,
citecolor=black
}

% use a small amount of tracking on small caps
\SetTracking[ spacing = {25*,166, } ]{ encoding = *, shape = sc }{ 25 }

% add a blank page
\newcommand{\blankpage}{
\newpage
\thispagestyle{empty}
\mbox{}
\newpage
}

% define languages
\setdefaultlanguage[]{english}
\setotherlanguage[script=Latin]{sanskrit}

%\usepackage{pagegrid}
%\pagegridsetup{top-left, step=.25in}

% define fonts
% use if arno sanskrit is unavailable
%\setmainfont{Gentium Plus}
%\newfontfamily\Semiboldsubheadfont[]{Gentium Plus}
%\newfontfamily\Semiboldnormalfont[]{Gentium Plus}
%\newfontfamily\Lightfont[]{Gentium Plus}
%\newfontfamily\Marginalfont[]{Gentium Plus}
%\newfontfamily\Allsmallcapsfont[RawFeature=+c2sc]{Gentium Plus}
%\newfontfamily\Noligaturefont[Renderer=Basic]{Gentium Plus}
%\newfontfamily\Noligaturecaptionfont[Renderer=Basic]{Gentium Plus}
%\newfontfamily\Fleuronfont[Ornament=1]{Gentium Plus}

% use if arno sanskrit is available. display is applied to \chapter and \part, subhead to \section and \subsection. When specifying semibold, the italic must be defined.
\setmainfont[Numbers=OldStyle]{Arno Pro}
\newfontfamily\Semibolddisplayfont[BoldItalicFont = Arno Pro Semibold Italic Display]{Arno Pro Semibold Display} %
\newfontfamily\Semiboldsubheadfont[BoldItalicFont = Arno Pro Semibold Italic Subhead]{Arno Pro Semibold Subhead}
\newfontfamily\Semiboldnormalfont[BoldItalicFont = Arno Pro Semibold Italic]{Arno Pro Semibold}
\newfontfamily\Marginalfont[RawFeature=+subs]{Arno Pro Regular}
\newfontfamily\Allsmallcapsfont[RawFeature=+c2sc]{Arno Pro}
\newfontfamily\Noligaturefont[Renderer=Basic]{Arno Pro}
\newfontfamily\Noligaturecaptionfont[Renderer=Basic]{Arno Pro Caption}

% chinese fonts
\newfontfamily\cjk{Noto Serif TC}
\newcommand*{\langlzh}[1]{\cjk{#1}\normalfont}%

% logo
\newfontfamily\Logofont{sclogo.ttf}
\newcommand*{\sclogo}[1]{\large\Logofont{#1}}

% use subscript numerals for margin notes
\renewcommand*{\marginfont}{\Marginalfont}

% ensure margin notes have consistent vertical alignment
\renewcommand*{\marginnotevadjust}{-.17em}

% use compact lists
\setitemize{noitemsep,leftmargin=1em}
\setenumerate{noitemsep,leftmargin=1em}
\setdescription{noitemsep, style=unboxed, leftmargin=0em}

% style ToC
\DeclareTOCStyleEntries[
  raggedentrytext,
  linefill=\hfill,
  pagenumberwidth=.5in,
  pagenumberformat=\normalfont,
  entryformat=\normalfont
]{tocline}{chapter,section}


  \setlength\topsep{0pt}%
  \setlength\parskip{0pt}%

% define new \centerpars command for use in ToC. This ensures centering, proper wrapping, and no page break after
\def\startcenter{%
  \par
  \begingroup
  \leftskip=0pt plus 1fil
  \rightskip=\leftskip
  \parindent=0pt
  \parfillskip=0pt
}
\def\stopcenter{%
  \par
  \endgroup
}
\long\def\centerpars#1{\startcenter#1\stopcenter}

% redefine part, so that it adds a toc entry without page number
\let\oldcontentsline\contentsline
\newcommand{\nopagecontentsline}[3]{\oldcontentsline{#1}{#2}{}}

    \makeatletter
\renewcommand*\l@part[2]{%
  \ifnum \c@tocdepth >-2\relax
    \addpenalty{-\@highpenalty}%
    \addvspace{0em \@plus\p@}%
    \setlength\@tempdima{3em}%
    \begingroup
      \parindent \z@ \rightskip \@pnumwidth
      \parfillskip -\@pnumwidth
      {\leavevmode
       \setstretch{.85}\large\scshape\centerpars{#1}\vspace*{-1em}\llap{#2}}\par
       \nobreak
         \global\@nobreaktrue
         \everypar{\global\@nobreakfalse\everypar{}}%
    \endgroup
  \fi}
\makeatother

\makeatletter
\def\@pnumwidth{2em}
\makeatother

% define new sectioning command, which is only used in volumes where the pannasa is found in some parts but not others, especially in an and sn

\newcommand*{\pannasa}[1]{\clearpage\thispagestyle{empty}\begin{center}\vspace*{14em}\setstretch{.85}\huge\itshape\scshape\MakeLowercase{#1}\end{center}}

    \makeatletter
\newcommand*\l@pannasa[2]{%
  \ifnum \c@tocdepth >-2\relax
    \addpenalty{-\@highpenalty}%
    \addvspace{.5em \@plus\p@}%
    \setlength\@tempdima{3em}%
    \begingroup
      \parindent \z@ \rightskip \@pnumwidth
      \parfillskip -\@pnumwidth
      {\leavevmode
       \setstretch{.85}\large\itshape\scshape\lowercase{\centerpars{#1}}\vspace*{-1em}\llap{#2}}\par
       \nobreak
         \global\@nobreaktrue
         \everypar{\global\@nobreakfalse\everypar{}}%
    \endgroup
  \fi}
\makeatother

% don't put page number on first page of toc (relies on etoolbox)
\patchcmd{\chapter}{plain}{empty}{}{}

% global line height
\setstretch{1.05}

% allow linebreak after em-dash
\catcode`\—=13
\protected\def—{\unskip\textemdash\allowbreak}

% style headings with secsty. chapter and section are defined per-edition
\partfont{\setstretch{.85}\normalfont\centering\textsc}
\subsectionfont{\setstretch{.85}\Semiboldsubheadfont}%
\subsubsectionfont{\setstretch{.85}\Semiboldnormalfont}

% style elements of suttatitle
\newcommand*{\suttatitleacronym}[1]{\smaller[2]{#1}\vspace*{.3em}}
\newcommand*{\suttatitletranslation}[1]{\linebreak{#1}}
\newcommand*{\suttatitleroot}[1]{\linebreak\smaller[2]\itshape{#1}}

\DeclareTOCStyleEntries[
  indent=3.3em,
  dynindent,
  beforeskip=.2em plus -2pt minus -1pt,
]{tocline}{section}

\DeclareTOCStyleEntries[
  indent=0em,
  dynindent,
  beforeskip=.4em plus -2pt minus -1pt,
]{tocline}{chapter}

\newcommand*{\tocacronym}[1]{\hspace*{-3.3em}{#1}\quad}
\newcommand*{\toctranslation}[1]{#1}
\newcommand*{\tocroot}[1]{(\textit{#1})}
\newcommand*{\tocchapterline}[1]{\bfseries\itshape{#1}}


% redefine paragraph and subparagraph headings to not be inline
\makeatletter
% Change the style of paragraph headings %
\renewcommand\paragraph{\@startsection{paragraph}{4}{\z@}%
            {-2.5ex\@plus -1ex \@minus -.25ex}%
            {1.25ex \@plus .25ex}%
            {\noindent\Semiboldnormalfont\normalsize}}

% Change the style of subparagraph headings %
\renewcommand\subparagraph{\@startsection{subparagraph}{5}{\z@}%
            {-2.5ex\@plus -1ex \@minus -.25ex}%
            {1.25ex \@plus .25ex}%
            {\noindent\Semiboldnormalfont\small}}
\makeatother

% use etoolbox to suppress page numbers on \part
\patchcmd{\part}{\thispagestyle{plain}}{\thispagestyle{empty}}
  {}{\errmessage{Cannot patch \string\part}}

% and to reduce margins on quotation
\patchcmd{\quotation}{\rightmargin}{\leftmargin 1.2em \rightmargin}{}{}
\AtBeginEnvironment{quotation}{\small}

% titlepage
\newcommand*{\titlepageTranslationTitle}[1]{{\begin{center}\begin{large}{#1}\end{large}\end{center}}}
\newcommand*{\titlepageCreatorName}[1]{{\begin{center}\begin{normalsize}{#1}\end{normalsize}\end{center}}}

% halftitlepage
\newcommand*{\halftitlepageTranslationTitle}[1]{\setstretch{2.5}{\begin{Huge}\uppercase{\so{#1}}\end{Huge}}}
\newcommand*{\halftitlepageTranslationSubtitle}[1]{\setstretch{1.2}{\begin{large}{#1}\end{large}}}
\newcommand*{\halftitlepageFleuron}[1]{{\begin{large}\Fleuronfont{{#1}}\end{large}}}
\newcommand*{\halftitlepageByline}[1]{{\begin{normalsize}\textit{{#1}}\end{normalsize}}}
\newcommand*{\halftitlepageCreatorName}[1]{{\begin{LARGE}{\textsc{#1}}\end{LARGE}}}
\newcommand*{\halftitlepageVolumeNumber}[1]{{\begin{normalsize}{\Allsmallcapsfont{\textsc{#1}}}\end{normalsize}}}
\newcommand*{\halftitlepageVolumeAcronym}[1]{{\begin{normalsize}{#1}\end{normalsize}}}
\newcommand*{\halftitlepageVolumeTranslationTitle}[1]{{\begin{Large}{\textsc{#1}}\end{Large}}}
\newcommand*{\halftitlepageVolumeRootTitle}[1]{{\begin{normalsize}{\Allsmallcapsfont{\textsc{\itshape #1}}}\end{normalsize}}}
\newcommand*{\halftitlepagePublisher}[1]{{\begin{large}{\Noligaturecaptionfont\textsc{#1}}\end{large}}}

% epigraph
\renewcommand{\epigraphflush}{center}
\renewcommand*{\epigraphwidth}{.85\textwidth}
\newcommand*{\epigraphTranslatedTitle}[1]{\vspace*{.5em}\footnotesize\textsc{#1}\\}%
\newcommand*{\epigraphRootTitle}[1]{\footnotesize\textit{#1}\\}%
\newcommand*{\epigraphReference}[1]{\footnotesize{#1}}%

% custom commands for html styling classes
\newcommand*{\scnamo}[1]{\begin{center}\textit{#1}\end{center}}
\newcommand*{\scendsection}[1]{\begin{center}\textit{#1}\end{center}}
\newcommand*{\scendsutta}[1]{\begin{center}\textit{#1}\end{center}}
\newcommand*{\scendbook}[1]{\begin{center}\uppercase{#1}\end{center}}
\newcommand*{\scendkanda}[1]{\begin{center}\textbf{#1}\end{center}}
\newcommand*{\scend}[1]{\begin{center}\textit{#1}\end{center}}
\newcommand*{\scuddanaintro}[1]{\textit{#1}}
\newcommand*{\scendvagga}[1]{\begin{center}\textbf{#1}\end{center}}
\newcommand*{\scrule}[1]{\textbf{#1}}
\newcommand*{\scadd}[1]{\textit{#1}}
\newcommand*{\scevam}[1]{\textsc{#1}}
\newcommand*{\scspeaker}[1]{\hspace{2em}\textit{#1}}
\newcommand*{\scbyline}[1]{\begin{flushright}\textit{#1}\end{flushright}\bigskip}

% custom command for thematic break = hr
\newcommand*{\thematicbreak}{\begin{center}\rule[.5ex]{6em}{.4pt}\begin{normalsize}\quad\Fleuronfont{•}\quad\end{normalsize}\rule[.5ex]{6em}{.4pt}\end{center}}

% manage and style page header and footer. "fancy" has header and footer, "plain" has footer only

\pagestyle{fancy}
\fancyhf{}
\fancyfoot[RE,LO]{\thepage}
\fancyfoot[LE,RO]{\footnotesize\lastleftxmark}
\fancyhead[CE]{\setstretch{.85}\Noligaturefont\MakeLowercase{\textsc{\firstrightmark}}}
\fancyhead[CO]{\setstretch{.85}\Noligaturefont\MakeLowercase{\textsc{\firstleftmark}}}
\renewcommand{\headrulewidth}{0pt}
\fancypagestyle{plain}{ %
\fancyhf{} % remove everything
\fancyfoot[RE,LO]{\thepage}
\fancyfoot[LE,RO]{\footnotesize\lastleftxmark}
\renewcommand{\headrulewidth}{0pt}
\renewcommand{\footrulewidth}{0pt}}

% style footnotes
\setlength{\skip\footins}{1em}

\makeatletter
\newcommand{\@makefntextcustom}[1]{%
    \parindent 0em%
    \thefootnote.\enskip #1%
}
\renewcommand{\@makefntext}[1]{\@makefntextcustom{#1}}
\makeatother

% hang quotes (requires microtype)
\microtypesetup{
  protrusion = true,
  expansion  = true,
  tracking   = true,
  factor     = 1000,
  patch      = all,
  final
}

% Custom protrusion rules to allow hanging punctuation
\SetProtrusion
{ encoding = *}
{
% char   right left
  {-} = {    , 500 },
  % Double Quotes
  \textquotedblleft
      = {1000,     },
  \textquotedblright
      = {    , 1000},
  \quotedblbase
      = {1000,     },
  % Single Quotes
  \textquoteleft
      = {1000,     },
  \textquoteright
      = {    , 1000},
  \quotesinglbase
      = {1000,     }
}

% make latex use actual font em for parindent, not Computer Modern Roman
\AtBeginDocument{\setlength{\parindent}{1em}}%
%

% Default values; a bit sloppier than normal
\tolerance 1414
\hbadness 1414
\emergencystretch 1.5em
\hfuzz 0.3pt
\clubpenalty = 10000
\widowpenalty = 10000
\displaywidowpenalty = 10000
\hfuzz \vfuzz
 \raggedbottom%

\title{Numbered Discourses}
\author{Bhikkhu Sujato}
\date{}%
% define a different fleuron for each edition
\newfontfamily\Fleuronfont[Ornament=18]{Arno Pro}

% Define heading styles per edition for chapter and section. Suttatitle can be either of these, depending on the volume. 

\let\oldfrontmatter\frontmatter
\renewcommand{\frontmatter}{%
\chapterfont{\setstretch{.85}\normalfont\centering}%
\sectionfont{\setstretch{.85}\Semiboldsubheadfont}%
\oldfrontmatter}

\let\oldmainmatter\mainmatter
\renewcommand{\mainmatter}{%
\chapterfont{\setstretch{.85}\normalfont\centering}%
\sectionfont{\setstretch{.85}\normalfont\centering}%
\oldmainmatter}

\let\oldbackmatter\backmatter
\renewcommand{\backmatter}{%
\chapterfont{\setstretch{.85}\normalfont\centering}%
\sectionfont{\setstretch{.85}\Semiboldsubheadfont}%
\oldbackmatter}
%
%
\begin{document}%
\normalsize%
\frontmatter%
\setlength{\parindent}{0cm}

\pagestyle{empty}

\maketitle

\blankpage%
\begin{center}

\vspace*{2.2em}

\halftitlepageTranslationTitle{Numbered Discourses}

\vspace*{1em}

\halftitlepageTranslationSubtitle{A sensible translation of the Aṅguttara Nikāya}

\vspace*{2em}

\halftitlepageFleuron{•}

\vspace*{2em}

\halftitlepageByline{translated and introduced by}

\vspace*{.5em}

\halftitlepageCreatorName{Bhikkhu Sujato}

\vspace*{4em}

\halftitlepageVolumeNumber{Volume 2}

\smallskip

\halftitlepageVolumeAcronym{AN 4}

\smallskip

\halftitlepageVolumeTranslationTitle{}

\smallskip

\halftitlepageVolumeRootTitle{}

\vspace*{\fill}

\sclogo{0}
 \halftitlepagePublisher{SuttaCentral}

\end{center}

\newpage
%
\setstretch{1.05}

\begin{footnotesize}

\textit{Numbered Discourses} is a translation of the Aṅguttaranikāya by Bhikkhu Sujato.

\medskip

Creative Commons Zero (CC0)

To the extent possible under law, Bhikkhu Sujato has waived all copyright and related or neighboring rights to \textit{Numbered Discourses}.

\medskip

This work is published from Australia.

\begin{center}
\textit{This translation is an expression of an ancient spiritual text that has been passed down by the Buddhist tradition for the benefit of all sentient beings. It is dedicated to the public domain via Creative Commons Zero (CC0). You are encouraged to copy, reproduce, adapt, alter, or otherwise make use of this translation. The translator respectfully requests that any use be in accordance with the values and principles of the Buddhist community.}
\end{center}

\medskip

\begin{description}
    \item[Web publication date] 2018
    \item[This edition] 2022-11-11 13:32:54
    \item[Publication type] paperback
    \item[Edition] ed5
    \item[Number of volumes] 5
    \item[Publication ISBN] 978-1-76132-037-8
    \item[Publication URL] https://suttacentral.net/editions/an/en/sujato
    \item[Source URL] https://github.com/suttacentral/bilara-data/tree/published/translation/en/sujato/sutta/an
    \item[Publication number] scpub5
\end{description}

\medskip

Published by SuttaCentral

\medskip

\textit{SuttaCentral,\\
c/o Alwis \& Alwis Pty Ltd\\
Kaurna Country,\\
Suite 12,\\
198 Greenhill Road,\\
Eastwood,\\
SA 5063,\\
Australia}

\end{footnotesize}

\newpage

\setlength{\parindent}{1.5em}%%
\tableofcontents
\newpage
\pagestyle{fancy}
%
\mainmatter%
\pagestyle{fancy}%
\addtocontents{toc}{\let\protect\contentsline\protect\nopagecontentsline}
\part*{The Book of the Fours }
\addcontentsline{toc}{part}{The Book of the Fours }
\markboth{}{}
\addtocontents{toc}{\let\protect\contentsline\protect\oldcontentsline}

%
%
\addtocontents{toc}{\let\protect\contentsline\protect\nopagecontentsline}
\pannasa{The First Fifty }
\addcontentsline{toc}{pannasa}{The First Fifty }
\markboth{}{}
\addtocontents{toc}{\let\protect\contentsline\protect\oldcontentsline}

%
\addtocontents{toc}{\let\protect\contentsline\protect\nopagecontentsline}
\chapter*{The Chapter at Bhaṇḍa Village }
\addcontentsline{toc}{chapter}{\tocchapterline{The Chapter at Bhaṇḍa Village }}
\addtocontents{toc}{\let\protect\contentsline\protect\oldcontentsline}

%
\section*{{\suttatitleacronym AN 4.1}{\suttatitletranslation Understood }{\suttatitleroot Anubuddhasutta}}
\addcontentsline{toc}{section}{\tocacronym{AN 4.1} \toctranslation{Understood } \tocroot{Anubuddhasutta}}
\markboth{Understood }{Anubuddhasutta}
\extramarks{AN 4.1}{AN 4.1}

\scevam{So\marginnote{1.1} I have heard. }At one time the Buddha was staying in the land of the Vajjis at the village of \textsanskrit{Bhaṇḍa}. There the Buddha addressed the mendicants, “Mendicants!” 

“Venerable\marginnote{1.5} sir,” they replied. The Buddha said this: 

“Mendicants,\marginnote{2.1} not understanding and not penetrating four things, both you and I have wandered and transmigrated for such a very long time. What four? Noble ethics, immersion, wisdom, and freedom. These noble ethics, immersion, wisdom, and freedom have been understood and comprehended. Craving for continued existence has been cut off; the conduit to rebirth is ended; now there’ll be no more future lives.” 

That\marginnote{3.1} is what the Buddha said. Then the Holy One, the Teacher, went on to say: 

\begin{verse}%
“Ethics,\marginnote{4.1} immersion, and wisdom, \\
and the supreme freedom: \\
these things have been understood \\
by Gotama the renowned. 

And\marginnote{5.1} so the Buddha, having insight, \\
explained this teaching to the mendicants. \\
The teacher made an end of suffering, \\
seeing clearly, he is extinguished.” 

%
\end{verse}

%
\section*{{\suttatitleacronym AN 4.2}{\suttatitletranslation Fallen }{\suttatitleroot Papatitasutta}}
\addcontentsline{toc}{section}{\tocacronym{AN 4.2} \toctranslation{Fallen } \tocroot{Papatitasutta}}
\markboth{Fallen }{Papatitasutta}
\extramarks{AN 4.2}{AN 4.2}

“Someone\marginnote{1.1} without four things is said to have ‘fallen from this teaching and training’. What four? Noble ethics, immersion, wisdom, and freedom. Someone without these four things is said to have ‘fallen from this teaching and training’. 

Someone\marginnote{2.1} with four things is said to be ‘secure in this teaching and training’. What four? Noble ethics, immersion, wisdom, and freedom. Someone with these four things is said to be ‘secure in this teaching and training’. 

\begin{verse}%
They\marginnote{3.1} fall, collapsed and fallen; \\
greedy, they return. \\
The work is done, the joyful is enjoyed, \\
happiness is found through happiness.” 

%
\end{verse}

%
\section*{{\suttatitleacronym AN 4.3}{\suttatitletranslation Broken (1st) }{\suttatitleroot Paṭhamakhatasutta}}
\addcontentsline{toc}{section}{\tocacronym{AN 4.3} \toctranslation{Broken (1st) } \tocroot{Paṭhamakhatasutta}}
\markboth{Broken (1st) }{Paṭhamakhatasutta}
\extramarks{AN 4.3}{AN 4.3}

“When\marginnote{1.1} a foolish, incompetent bad person has four qualities they keep themselves broken and damaged. They deserve to be blamed and criticized by sensible people, and they make much bad karma. What four? Without examining or scrutinizing, they praise those deserving of criticism, and they criticize those deserving of praise. They arouse faith in things that are dubious, and they don’t arouse faith in things that are inspiring. When a foolish, incompetent bad person has these four qualities they keep themselves broken and damaged. They deserve to be blamed and criticized by sensible people, and they make much bad karma. 

When\marginnote{2.1} an astute, competent good person has four qualities they keep themselves healthy and whole. They don’t deserve to be blamed and criticized by sensible people, and they make much merit. What four? After examining and scrutinizing, they criticize those deserving of criticism, and they praise those deserving of praise. They don’t arouse faith in things that are dubious, and they do arouse faith in things that are inspiring. When an astute, competent good person has these four qualities they keep themselves healthy and whole. They don’t deserve to be blamed and criticized by sensible people, and they make much merit. 

\begin{verse}%
When\marginnote{3.1} you praise someone worthy of criticism, \\
or criticize someone worthy of praise, \\
you choose bad luck with your own mouth: \\
you’ll never find happiness that way. 

Bad\marginnote{4.1} luck at dice is a trivial thing, \\
if all you lose is your money \\
and all you own, even yourself. \\
What’s really terrible luck \\
is to hate the holy ones. 

For\marginnote{5.1} more than two quinquadecillion years, \\
and another five quattuordecillion years, \\
a slanderer of noble ones goes to hell, \\
having aimed bad words and thoughts at them.” 

%
\end{verse}

%
\section*{{\suttatitleacronym AN 4.4}{\suttatitletranslation Broken (2nd) }{\suttatitleroot Dutiyakhatasutta}}
\addcontentsline{toc}{section}{\tocacronym{AN 4.4} \toctranslation{Broken (2nd) } \tocroot{Dutiyakhatasutta}}
\markboth{Broken (2nd) }{Dutiyakhatasutta}
\extramarks{AN 4.4}{AN 4.4}

“When\marginnote{1.1} a foolish, incompetent bad person acts wrongly toward four people they keep themselves broken and damaged. They deserve to be blamed and criticized by sensible people, and they make much bad karma. What four? Mother … father … a Realized One … and a disciple of a Realized One. When a foolish, incompetent bad person acts wrongly toward these four people they keep themselves broken and damaged. They deserve to be blamed and criticized by sensible people, and they make much bad karma. 

When\marginnote{2.1} an astute, competent good person acts rightly toward four people they keep themselves healthy and whole. They don’t deserve to be blamed and criticized by sensible people, and they make much merit. What four? Mother … father … a Realized One … and a disciple of a Realized One. When an astute, competent good person acts rightly toward these four people they keep themselves healthy and whole. They don’t deserve to be blamed and criticized by sensible people, and they make much merit. 

\begin{verse}%
A\marginnote{3.1} person who does wrong \\
by their mother or father, \\
or a Realized One, a Buddha, \\
or one of their disciples, \\
makes much bad karma. 

Because\marginnote{4.1} of their unprincipled conduct \\
toward their parents, \\
they’re criticized in this life by the astute, \\
and they depart to be reborn in a place of loss. 

A\marginnote{5.1} person who does right \\
by their mother and father, \\
or a Realized One, a Buddha, \\
or one of their disciples, \\
makes much merit. 

Because\marginnote{6.1} of their principled conduct \\
toward their parents, \\
they’re praised in this life by the astute, \\
and they depart to rejoice in heaven.” 

%
\end{verse}

%
\section*{{\suttatitleacronym AN 4.5}{\suttatitletranslation With the Stream }{\suttatitleroot Anusotasutta}}
\addcontentsline{toc}{section}{\tocacronym{AN 4.5} \toctranslation{With the Stream } \tocroot{Anusotasutta}}
\markboth{With the Stream }{Anusotasutta}
\extramarks{AN 4.5}{AN 4.5}

“These\marginnote{1.1} four people are found in the world. What four? A person who goes with the stream; a person who goes against the stream; a steadfast person; and a brahmin who has crossed over and stands on the far shore. 

And\marginnote{1.4} who is the person who goes with the stream? It’s a person who takes part in sensual pleasures and does bad deeds. This is called a person who goes with the stream. 

And\marginnote{2.1} who is the person who goes against the stream? It’s a person who doesn’t take part in sensual pleasures or do bad deeds. They live the full and pure spiritual life in pain and sadness, weeping, with tearful faces. This is called a person who goes against the stream. 

And\marginnote{3.1} who is the steadfast person? It’s a person who, with the ending of the five lower fetters, is reborn spontaneously. They’re extinguished there, and are not liable to return from that world. This is called a steadfast person. 

And\marginnote{4.1} who is a brahmin who has crossed over and stands on the far shore? It’s a person who realizes the undefiled freedom of heart and freedom by wisdom in this very life. And they live having realized it with their own insight due to the ending of defilements. This is called a brahmin who has crossed over and stands on the far shore. 

These\marginnote{4.4} are the four people found in the world. 

\begin{verse}%
All\marginnote{5.1} those people with unbridled sensuality, \\
not free of lust, enjoying sensual pleasures in this life: \\
again and again, they return to birth and old age; \\
those who go with the stream are sunk in craving. 

So\marginnote{6.1} a wise one in this life, with mindfulness established, \\
doesn’t take part in sensual pleasures and bad deeds. \\
In pain they’d give up sensual pleasures: \\
they call that person ‘one who goes against the stream’. 

Someone\marginnote{7.1} who’s given up five corruptions, \\
a perfect trainee, not liable to decline, \\
who’s mastered their mind, with faculties immersed in \textsanskrit{samādhi}, \\
that’s called ‘a steadfast person’. 

The\marginnote{8.1} sage who has comprehended all things, high and low, \\
cleared them and ended them, so they are no more; \\
they’ve completed the spiritual journey, and gone to the end of the world, \\
they’re called ‘one who has gone beyond’.” 

%
\end{verse}

%
\section*{{\suttatitleacronym AN 4.6}{\suttatitletranslation A Little Learning }{\suttatitleroot Appassutasutta}}
\addcontentsline{toc}{section}{\tocacronym{AN 4.6} \toctranslation{A Little Learning } \tocroot{Appassutasutta}}
\markboth{A Little Learning }{Appassutasutta}
\extramarks{AN 4.6}{AN 4.6}

“Mendicants,\marginnote{1.1} these four people are found in the world. What four? A person may have: 

\begin{enumerate}%
\item Little learning and not get the point of learning. %
\item Little learning but get the point of learning. %
\item Much learning but not get the point of learning. %
\item Much learning and get the point of learning. %
\end{enumerate}

And\marginnote{1.7} how has a person learned little and not got the point of learning? It’s when a person has learned little of the statements, songs, discussions, verses, inspired exclamations, legends, stories of past lives, amazing stories, and classifications. And with the little they’ve learned, they understand neither the meaning nor the text, nor do they practice in line with the teaching. That’s how a person has learned little and not got the point of learning. 

And\marginnote{2.1} how has a person learned little but has got the point of learning? It’s when a person has learned little of the statements, songs, discussions, verses, inspired exclamations, legends, stories of past lives, amazing stories, and classifications. But with the little they’ve learned, they understand the meaning and the text, and they practice in line with the teaching. That’s how a person has learned little but has got the point of learning. 

And\marginnote{3.1} how has a person learned much but hasn't got the point of learning? It’s when a person has learned much of the statements, songs, discussions, verses, inspired exclamations, legends, stories of past lives, amazing stories, and classifications. But even though they’ve learned much, they understand neither the meaning nor the text, nor do they practice in line with the teaching. That’s how a person has learned much but hasn't got the point of learning. 

And\marginnote{4.1} how has a person learned much and has got the point of learning? It’s when a person has learned much of the statements, songs, discussions, verses, inspired exclamations, legends, stories of past lives, amazing stories, and classifications. And with the large amount they’ve learned, they understand the meaning and the text, and they practice in line with the teaching. That’s how a person has learned much and has got the point of learning. 

These\marginnote{4.6} are the four people found in the world. 

\begin{verse}%
If\marginnote{5.1} you don’t learn much, \\
and aren’t steady in ethics, \\
they’ll criticize you on both counts, \\
for your ethics and your learning. 

If\marginnote{6.1} you don’t learn much, \\
and you are steady in ethics, \\
they’ll praise your ethical conduct, \\
since your learning has succeeded. 

If\marginnote{7.1} you learn much, \\
but aren’t steady in ethics, \\
they’ll criticize your ethical conduct, \\
for your learning hasn’t succeeded. 

If\marginnote{8.1} you learn much, \\
and you are steady in ethics, \\
they’ll praise you on both counts, \\
for your ethics and your learning. 

A\marginnote{9.1} wise disciple of the Buddha \\
who is learned and has memorized the teachings; \\
like a pendant of river gold, \\
who is worthy to criticize them? \\
Even the gods praise them, \\
and by \textsanskrit{Brahmā}, too, they’re praised.” 

%
\end{verse}

%
\section*{{\suttatitleacronym AN 4.7}{\suttatitletranslation Beautification }{\suttatitleroot Sobhanasutta}}
\addcontentsline{toc}{section}{\tocacronym{AN 4.7} \toctranslation{Beautification } \tocroot{Sobhanasutta}}
\markboth{Beautification }{Sobhanasutta}
\extramarks{AN 4.7}{AN 4.7}

“Mendicants,\marginnote{1.1} these four competent, educated, assured, learned people—who have memorized the teachings and practice in line with the teachings—beautify the \textsanskrit{Saṅgha}. What four? A monk, a nun, a layman, and a laywoman. 

These\marginnote{1.7} four competent, educated, assured, learned people—who have memorized the teachings and practice in line with the teachings—beautify the \textsanskrit{Saṅgha}. 

\begin{verse}%
Whoever\marginnote{2.1} is competent and assured, \\
learned, a memorizer of the teachings, \\
who lives in line with the teachings—\\
such a person is said to beautify the \textsanskrit{Saṅgha}. 

A\marginnote{3.1} monk accomplished in ethics, \\
and a learned nun, \\
a faithful layman, \\
and a faithful laywoman, too: \\
these beautify the \textsanskrit{Saṅgha}, \\
they are the beautifiers of the \textsanskrit{Saṅgha}.” 

%
\end{verse}

%
\section*{{\suttatitleacronym AN 4.8}{\suttatitletranslation Self-assured }{\suttatitleroot Vesārajjasutta}}
\addcontentsline{toc}{section}{\tocacronym{AN 4.8} \toctranslation{Self-assured } \tocroot{Vesārajjasutta}}
\markboth{Self-assured }{Vesārajjasutta}
\extramarks{AN 4.8}{AN 4.8}

“Mendicants,\marginnote{1.1} a Realized One has four kinds of self-assurance. With these he claims the bull’s place, roars his lion’s roar in the assemblies, and turns the holy wheel. What four? 

I\marginnote{1.3} see no reason for anyone—whether ascetic, brahmin, god, \textsanskrit{Māra}, or \textsanskrit{Brahmā}, or anyone else in the world—to legitimately scold me, saying: ‘You claim to be fully awakened, but you don’t understand these things.’ Since I see no such reason, I live secure, fearless, and assured. 

I\marginnote{2.1} see no reason for anyone—whether ascetic, brahmin, god, \textsanskrit{Māra}, or \textsanskrit{Brahmā}, or anyone else in the world—to legitimately scold me, saying: ‘You claim to have ended all defilements, but these defilements have not ended.’ Since I see no such reason, I live secure, fearless, and assured. 

I\marginnote{3.1} see no reason for anyone—whether ascetic, brahmin, god, \textsanskrit{Māra}, or \textsanskrit{Brahmā}, or anyone else in the world—to legitimately scold me, saying: ‘The acts that you say are obstructions are not really obstructions for the one who performs them.’ Since I see no such reason, I live secure, fearless, and assured. 

I\marginnote{4.1} see no reason for anyone—whether ascetic, brahmin, god, \textsanskrit{Māra}, or \textsanskrit{Brahmā}, or anyone else in the world—to legitimately scold me, saying: ‘Though you teach that this teaching leads to the goal of the complete ending of suffering, it doesn’t lead there for one who practices it.’ Since I see no such reason, I live secure, fearless, and assured. 

A\marginnote{4.3} Realized One has these four kinds of self-assurance. With these he claims the bull’s place, roars his lion’s roar in the assemblies, and turns the holy wheel. 

\begin{verse}%
The\marginnote{5.1} various grounds for criticism \\
that ascetics and brahmins rely on \\
vanish on reaching a Realized One, \\
assured, gone beyond grounds for criticism. 

He\marginnote{6.1} rolls forth the Wheel of Dhamma as a consummate one, \\
complete, compassionate for all living creatures. \\
Sentient beings revere him, first among gods and humans, \\
who has gone beyond rebirth.” 

%
\end{verse}

%
\section*{{\suttatitleacronym AN 4.9}{\suttatitletranslation The Arising of Craving }{\suttatitleroot Taṇhuppādasutta}}
\addcontentsline{toc}{section}{\tocacronym{AN 4.9} \toctranslation{The Arising of Craving } \tocroot{Taṇhuppādasutta}}
\markboth{The Arising of Craving }{Taṇhuppādasutta}
\extramarks{AN 4.9}{AN 4.9}

“Mendicants,\marginnote{1.1} there are four things that give rise to craving in a mendicant. What four? For the sake of robes, almsfood, lodgings, or rebirth in this or that state. 

These\marginnote{1.7} are the four things that give rise to craving in a mendicant. 

\begin{verse}%
Craving\marginnote{2.1} is a person’s partner \\
as they transmigrate on this long journey. \\
They go from this state to another, \\
but don’t escape transmigration. 

Knowing\marginnote{3.1} this drawback—\\
that craving is the cause of suffering—\\
rid of craving, free of grasping, \\
a mendicant would wander mindful.” 

%
\end{verse}

%
\section*{{\suttatitleacronym AN 4.10}{\suttatitletranslation Attachments }{\suttatitleroot Yogasutta}}
\addcontentsline{toc}{section}{\tocacronym{AN 4.10} \toctranslation{Attachments } \tocroot{Yogasutta}}
\markboth{Attachments }{Yogasutta}
\extramarks{AN 4.10}{AN 4.10}

“Mendicants,\marginnote{1.1} there are these four attachments. What four? The attachment to sensual pleasures, future lives, views, and ignorance. 

And\marginnote{1.4} what is the attachment to sensual pleasures? It’s when you don’t truly understand sensual pleasures’ origin, ending, gratification, drawback, and escape. So greed, relishing, affection, infatuation, thirst, passion, attachment, and craving for sensual pleasures linger on inside. This is called the attachment to sensual pleasures. Such is the attachment to sensual pleasures. 

And\marginnote{2.1} what is the attachment to future lives? It’s when you don’t truly understand future lives’ origin, ending, gratification, drawback, and escape. So greed, relishing, affection, infatuation, thirst, passion, attachment, and craving for continued existence linger on inside. This is called the attachment to future lives. Such are the attachments to sensual pleasures and future lives. 

And\marginnote{3.1} what is the attachment to views? It’s when you don’t truly understand views’ origin, ending, gratification, drawback, and escape. So greed, relishing, affection, infatuation, thirst, passion, attachment, and craving for views linger on inside. This is called the attachment to views. Such are the attachments to sensual pleasures, future lives, and views. 

And\marginnote{4.1} what is the attachment to ignorance? It’s when you don’t truly understand the six fields of contact’s origin, ending, gratification, drawback, and escape, so ignorance and unknowing of the six fields of contact linger on inside. This is called the attachment to ignorance. Such are the attachments to sensual pleasures, future lives, views, and ignorance. These are bad, unskillful qualities that are corrupting, leading to future lives, hurtful, resulting in suffering and future rebirth, old age, and death. That’s why someone attached to them is called: ‘one who has not found sanctuary from attachments’. 

These\marginnote{4.7} are the four attachments. 

There\marginnote{5.1} are these four kinds of detachment. What four? Detachment from sensual pleasures, future lives, views, and ignorance. 

And\marginnote{5.4} what is detachment from sensual pleasures? It’s when you truly understand sensual pleasures’ origin, ending, gratification, drawback, and escape. So greed, relishing, affection, infatuation, thirst, passion, attachment, and craving for sensual pleasures don’t linger on inside. This is called detachment from sensual pleasures. Such is detachment from sensual pleasures. 

And\marginnote{6.1} what is detachment from future lives? It’s when you truly understand future lives’ origin, ending, gratification, drawback, and escape. So greed, relishing, affection, infatuation, thirst, passion, attachment, and craving for continued existence don’t linger on inside. This is called detachment from future lives. Such is detachment from sensual pleasures and future lives. 

And\marginnote{7.1} what is detachment from views? It’s when you truly understand views’ origin, ending, gratification, drawback, and escape. So greed, relishing, affection, infatuation, thirst, passion, attachment, and craving for views don’t linger on inside. This is called detachment from views. Such is detachment from sensual pleasures, future lives, and views. 

And\marginnote{8.1} what is detachment from ignorance? It’s when you truly understand the six fields of contact’s origin, ending, gratification, drawback, and escape, so ignorance and unknowing of the six fields of contact don’t linger on inside. This is called detachment from ignorance. Such is detachment from sensual pleasures, future lives, views, and ignorance. These are bad, unskillful qualities that are corrupting, leading to future lives, hurtful, resulting in suffering and future rebirth, old age, and death. That’s why someone detached from them is called: ‘one who has found sanctuary from attachments’. 

These\marginnote{8.7} are the four kinds of detachment. 

\begin{verse}%
Attached\marginnote{9.1} to both sensual pleasures \\
and the desire to be reborn in a future life; \\
attached also to views, \\
and governed by ignorance, 

sentient\marginnote{10.1} beings continue to transmigrate, \\
with ongoing birth and death. \\
But those who completely understand sensual pleasures, \\
and the attachment to all future lives; 

with\marginnote{11.1} the attachment to views eradicated, \\
and ignorance dispelled, \\
detached from all attachments, \\
truly those sages have escaped their bonds.” 

%
\end{verse}

%
\addtocontents{toc}{\let\protect\contentsline\protect\nopagecontentsline}
\chapter*{The Chapter on Walking }
\addcontentsline{toc}{chapter}{\tocchapterline{The Chapter on Walking }}
\addtocontents{toc}{\let\protect\contentsline\protect\oldcontentsline}

%
\section*{{\suttatitleacronym AN 4.11}{\suttatitletranslation Walking }{\suttatitleroot Carasutta}}
\addcontentsline{toc}{section}{\tocacronym{AN 4.11} \toctranslation{Walking } \tocroot{Carasutta}}
\markboth{Walking }{Carasutta}
\extramarks{AN 4.11}{AN 4.11}

“Mendicants,\marginnote{1.1} suppose a mendicant has a sensual, malicious, or cruel thought while walking. They tolerate it and don’t give it up, get rid of it, eliminate it, and obliterate it. Such a mendicant is said to be ‘not keen or prudent, always lazy, and lacking energy’ when walking. 

Suppose\marginnote{2.1} a mendicant has a sensual, malicious, or cruel thought while standing … sitting … or when lying down while awake. They tolerate it and don’t give it up, get rid of it, eliminate it, and obliterate it. Such a mendicant is said to be ‘not keen or prudent, always lazy, and lacking energy’ when lying down while awake. 

Suppose\marginnote{5.1} a mendicant has a sensual, malicious, or cruel thought while walking. They don’t tolerate it, but give it up, get rid of it, eliminate it, and obliterate it. Such a mendicant is said to be ‘keen and prudent, always energetic and determined’ when walking. 

Suppose\marginnote{6.1} a mendicant has a sensual, malicious, or cruel thought while standing … sitting … or when lying down while awake. They don’t tolerate it, but give it up, get rid of it, eliminate it, and obliterate it. Such a mendicant is said to be ‘keen and prudent, always energetic and determined’ when lying down while awake.” 

\begin{verse}%
Whether\marginnote{9.1} walking or standing, \\
sitting or lying down, \\
if you think a bad thought \\
to do with the lay life, 

you’re\marginnote{10.1} on the wrong path, \\
lost among things that delude. \\
Such a mendicant is incapable \\
of touching the highest awakening. 

But\marginnote{11.1} one who, whether standing or walking, \\
sitting or lying down, \\
has calmed their thoughts, \\
loving peace of mind; \\
such a mendicant is capable \\
of touching the highest awakening.” 

%
\end{verse}

%
\section*{{\suttatitleacronym AN 4.12}{\suttatitletranslation Ethics }{\suttatitleroot Sīlasutta}}
\addcontentsline{toc}{section}{\tocacronym{AN 4.12} \toctranslation{Ethics } \tocroot{Sīlasutta}}
\markboth{Ethics }{Sīlasutta}
\extramarks{AN 4.12}{AN 4.12}

“Mendicants,\marginnote{1.1} live by the ethical precepts and the monastic code. Live restrained in the code of conduct, conducting yourselves well and seeking alms in suitable places. Seeing danger in the slightest fault, keep the rules you’ve undertaken. When you’ve done this, what more is there to do? 

Suppose\marginnote{2.1} a mendicant has got rid of desire and ill will while walking, and has given up dullness and drowsiness, restlessness and remorse, and doubt. Their energy is roused up and unflagging, their mindfulness is established and lucid, their body is tranquil and undisturbed, and their mind is immersed in \textsanskrit{samādhi}. Such a mendicant is said to be ‘keen and prudent, always energetic and determined’ when walking. 

Suppose\marginnote{3.1} a mendicant has got rid of desire and ill will while standing … sitting … and when lying down while awake, and has given up dullness and drowsiness, restlessness and remorse, and doubt. Their energy is roused up and unflagging, their mindfulness is established and lucid, their body is tranquil and undisturbed, and their mind is immersed in \textsanskrit{samādhi}. Such a mendicant is said to be ‘keen and prudent, always energetic and determined’ when lying down while awake. 

\begin{verse}%
Carefully\marginnote{6.1} walking, carefully standing, \\
carefully sitting, carefully lying; \\
a mendicant carefully bends their limbs, \\
and carefully extends them. 

Above,\marginnote{7.1} below, all round, \\
as far as the earth extends; \\
they scrutinize the rise and fall \\
of phenomena such as the aggregates. 

Training\marginnote{8.1} in what leads to serenity of heart, \\
always staying mindful; \\
they call such a mendicant \\
‘always determined’.” 

%
\end{verse}

%
\section*{{\suttatitleacronym AN 4.13}{\suttatitletranslation Effort }{\suttatitleroot Padhānasutta}}
\addcontentsline{toc}{section}{\tocacronym{AN 4.13} \toctranslation{Effort } \tocroot{Padhānasutta}}
\markboth{Effort }{Padhānasutta}
\extramarks{AN 4.13}{AN 4.13}

“Mendicants,\marginnote{1.1} there are these four right efforts. What four? 

A\marginnote{1.3} mendicant generates enthusiasm, tries, makes an effort, exerts the mind, and strives so that bad, unskillful qualities don’t arise. 

They\marginnote{1.4} generate enthusiasm, try, make an effort, exert the mind, and strive so that bad, unskillful qualities that have arisen are given up. 

They\marginnote{1.5} generate enthusiasm, try, make an effort, exert the mind, and strive so that skillful qualities arise. 

They\marginnote{1.6} generate enthusiasm, try, make an effort, exert the mind, and strive so that skillful qualities that have arisen remain, are not lost, but increase, mature, and are fulfilled by development. 

These\marginnote{1.7} are the four right efforts. 

\begin{verse}%
By\marginnote{2.1} rightly striving, they’ve crushed \textsanskrit{Māra}’s sovereignty; \\
unattached, they’ve gone beyond the danger of birth and death. \\
Contented and unstirred, they’ve vanquished \textsanskrit{Māra} and his mount; \\
now they’ve gone beyond all Namuci’s forces, they’re happy.” 

%
\end{verse}

%
\section*{{\suttatitleacronym AN 4.14}{\suttatitletranslation Restraint }{\suttatitleroot Saṁvarasutta}}
\addcontentsline{toc}{section}{\tocacronym{AN 4.14} \toctranslation{Restraint } \tocroot{Saṁvarasutta}}
\markboth{Restraint }{Saṁvarasutta}
\extramarks{AN 4.14}{AN 4.14}

“Mendicants,\marginnote{1.1} there are these four efforts. What four? The efforts to restrain, to give up, to develop, and to preserve. 

And\marginnote{1.4} what, mendicants, is the effort to restrain? When a mendicant sees a sight with their eyes, they don’t get caught up in the features and details. If the faculty of sight were left unrestrained, bad unskillful qualities of desire and aversion would become overwhelming. For this reason, they practice restraint, protecting the faculty of sight, and achieving its restraint. When they hear a sound with their ears … When they smell an odor with their nose … When they taste a flavor with their tongue … When they feel a touch with their body … When they know a thought with their mind, they don’t get caught up in the features and details. If the faculty of mind were left unrestrained, bad unskillful qualities of desire and aversion would become overwhelming. For this reason, they practice restraint, protecting the faculty of mind, and achieving its restraint. This is called the effort to restrain. 

And\marginnote{2.1} what, mendicants, is the effort to give up? It’s when a mendicant doesn’t tolerate a sensual, malicious, or cruel thought that’s arisen, but gives it up, gets rid of it, eliminates it, and obliterates it. They don’t tolerate any bad, unskillful qualities that have arisen, but give them up, get rid of them, eliminate them, and obliterate them. This is called the effort to give up. 

And\marginnote{3.1} what, mendicants, is the effort to develop? It’s when a mendicant develops the awakening factors of mindfulness, investigation of principles, energy, rapture, tranquility, immersion, and equanimity, which rely on seclusion, fading away, and cessation, and ripen as letting go. This is called the effort to develop. 

And\marginnote{4.1} what, mendicants, is the effort to preserve? It’s when a mendicant preserves a meditation subject that’s a fine foundation of immersion: the perception of a skeleton, a worm-infested corpse, a livid corpse, a split open corpse, or a bloated corpse. This is called the effort to preserve. 

These\marginnote{4.4} are the four efforts. 

\begin{verse}%
Restraint\marginnote{5.1} and giving up, \\
development and preservation: \\
these are the four efforts \\
taught by the kinsman of the Sun. \\
Any mendicant who keenly applies these \\
may attain the ending of suffering.” 

%
\end{verse}

%
\section*{{\suttatitleacronym AN 4.15}{\suttatitletranslation Regarded as Foremost }{\suttatitleroot Paññattisutta}}
\addcontentsline{toc}{section}{\tocacronym{AN 4.15} \toctranslation{Regarded as Foremost } \tocroot{Paññattisutta}}
\markboth{Regarded as Foremost }{Paññattisutta}
\extramarks{AN 4.15}{AN 4.15}

“Mendicants,\marginnote{1.1} these four are regarded as foremost. What four? The foremost in size of life-form is \textsanskrit{Rāhu}, lord of demons. The foremost sensualist is King \textsanskrit{Mandhātā}. The foremost in sovereignty is \textsanskrit{Māra} the Wicked. In this world—with its gods, \textsanskrit{Māras} and \textsanskrit{Brahmās}, this population with its ascetics and brahmins, gods and humans—a Realized One, the perfected one, the fully awakened Buddha is said to be the best. These are the four regarded as foremost. 

\begin{verse}%
\textsanskrit{Rāhu}\marginnote{2.1} is foremost in size of life-form, \\
\textsanskrit{Mandhātā} in enjoying sensual pleasures, \\
\textsanskrit{Māra} in sovereignty, \\
shining with power and glory. 

Above,\marginnote{3.1} below, all round, \\
as far as the earth extends; \\
in all the world with its gods, \\
the Buddha is declared foremost.” 

%
\end{verse}

%
\section*{{\suttatitleacronym AN 4.16}{\suttatitletranslation Subtlety }{\suttatitleroot Sokhummasutta}}
\addcontentsline{toc}{section}{\tocacronym{AN 4.16} \toctranslation{Subtlety } \tocroot{Sokhummasutta}}
\markboth{Subtlety }{Sokhummasutta}
\extramarks{AN 4.16}{AN 4.16}

“Mendicants,\marginnote{1.1} there are these four kinds of subtlety. What four? 

A\marginnote{1.3} mendicant has ultimate subtlety of form. They don’t see any other subtlety of form that’s better or finer than that, nor do they aim for it. 

A\marginnote{1.6} mendicant has ultimate subtlety of feeling. They don’t see any other subtlety of feeling that’s better or finer than that, nor do they aim for it. 

A\marginnote{1.9} mendicant has ultimate subtlety of perception. They don’t see any other subtlety of perception that’s better or finer than that, nor do they aim for it. 

A\marginnote{1.12} mendicant has ultimate subtlety of choices. They don’t see any other subtlety of choices that’s better or finer than that, nor do they aim for it. 

These\marginnote{1.15} are the four kinds of subtlety. 

\begin{verse}%
Knowing\marginnote{2.1} the subtlety of form, \\
the cause of feelings, \\
where perception comes from, \\
and where it ends; \\
and knowing choices as other, \\
as suffering and as not-self, 

that\marginnote{3.1} mendicant sees rightly, \\
peaceful, in love with the state of peace. \\
They bear their final body, \\
having vanquished \textsanskrit{Māra} and his mount.” 

%
\end{verse}

%
\section*{{\suttatitleacronym AN 4.17}{\suttatitletranslation Prejudice (1st) }{\suttatitleroot Paṭhamaagatisutta}}
\addcontentsline{toc}{section}{\tocacronym{AN 4.17} \toctranslation{Prejudice (1st) } \tocroot{Paṭhamaagatisutta}}
\markboth{Prejudice (1st) }{Paṭhamaagatisutta}
\extramarks{AN 4.17}{AN 4.17}

“Mendicants,\marginnote{1.1} there are these four ways of making prejudiced decisions. What four? Making decisions prejudiced by favoritism, hostility, stupidity, and cowardice. These are the four ways of making prejudiced decisions. 

\begin{verse}%
If\marginnote{2.1} you act against the teaching \\
out of favoritism, hostility, cowardice, or stupidity, \\
your fame fades away, \\
like the moon in the waning fortnight.” 

%
\end{verse}

%
\section*{{\suttatitleacronym AN 4.18}{\suttatitletranslation Prejudice (2nd) }{\suttatitleroot Dutiyaagatisutta}}
\addcontentsline{toc}{section}{\tocacronym{AN 4.18} \toctranslation{Prejudice (2nd) } \tocroot{Dutiyaagatisutta}}
\markboth{Prejudice (2nd) }{Dutiyaagatisutta}
\extramarks{AN 4.18}{AN 4.18}

“Mendicants,\marginnote{1.1} there are these four ways of making unprejudiced decisions. What four? Making decisions unprejudiced by favoritism, hostility, stupidity, and cowardice. These are the four ways of making unprejudiced decisions. 

\begin{verse}%
If\marginnote{2.1} you don’t act against the teaching \\
out of favoritism, hostility, cowardice, and stupidity, \\
your fame swells, \\
like the moon in the waxing fortnight.” 

%
\end{verse}

%
\section*{{\suttatitleacronym AN 4.19}{\suttatitletranslation Prejudice (3rd) }{\suttatitleroot Tatiyaagatisutta}}
\addcontentsline{toc}{section}{\tocacronym{AN 4.19} \toctranslation{Prejudice (3rd) } \tocroot{Tatiyaagatisutta}}
\markboth{Prejudice (3rd) }{Tatiyaagatisutta}
\extramarks{AN 4.19}{AN 4.19}

“Mendicants,\marginnote{1.1} there are these four ways of making prejudiced decisions. What four? Making decisions prejudiced by favoritism, hostility, stupidity, and cowardice. These are the four ways of making prejudiced decisions. 

There\marginnote{2.1} are these four ways of making unprejudiced decisions. What four? Making decisions unprejudiced by favoritism, hostility, stupidity, and cowardice. These are the four ways of making unprejudiced decisions. 

\begin{verse}%
If\marginnote{3.1} you act against the teaching \\
out of favoritism, hostility, cowardice, or stupidity, \\
your fame fades away, \\
like the moon in the waning fortnight. 

If\marginnote{4.1} you don’t act against the teaching \\
out of favoritism, hostility, cowardice, and stupidity, \\
your fame swells, \\
like the moon in the waxing fortnight.” 

%
\end{verse}

%
\section*{{\suttatitleacronym AN 4.20}{\suttatitletranslation A Meal Designator }{\suttatitleroot Bhattuddesakasutta}}
\addcontentsline{toc}{section}{\tocacronym{AN 4.20} \toctranslation{A Meal Designator } \tocroot{Bhattuddesakasutta}}
\markboth{A Meal Designator }{Bhattuddesakasutta}
\extramarks{AN 4.20}{AN 4.20}

“Mendicants,\marginnote{1.1} a meal designator who has four qualities is cast down to hell. What four? They make decisions prejudiced by favoritism, hostility, stupidity, and cowardice. A meal designator who has these four qualities is cast down to hell. 

A\marginnote{2.1} meal designator who has four qualities is raised up to heaven. What four? They make decisions unprejudiced by favoritism, hostility, stupidity, and cowardice. A meal designator who has these four qualities is raised up to heaven. 

\begin{verse}%
All\marginnote{3.1} those people with unbridled sensuality, \\
unprincipled, with no respect for principle, \\
led astray by favoritism, hatred, stupidity, or cowardice, \\
are called ‘an assembly of the dregs’: 

that’s\marginnote{4.1} what was said by the ascetic who knows. \\
And so those good, praiseworthy people, \\
standing on principle, doing nothing wrong, \\
not led astray by favoritism, hatred, stupidity, or cowardice, \\
are called ‘an assembly of the cream’: \\
that’s what was said by the ascetic who knows.” 

%
\end{verse}

%
\addtocontents{toc}{\let\protect\contentsline\protect\nopagecontentsline}
\chapter*{The Chapter at Uruvelā }
\addcontentsline{toc}{chapter}{\tocchapterline{The Chapter at Uruvelā }}
\addtocontents{toc}{\let\protect\contentsline\protect\oldcontentsline}

%
\section*{{\suttatitleacronym AN 4.21}{\suttatitletranslation At Uruvelā (1st) }{\suttatitleroot Paṭhamauruvelasutta}}
\addcontentsline{toc}{section}{\tocacronym{AN 4.21} \toctranslation{At Uruvelā (1st) } \tocroot{Paṭhamauruvelasutta}}
\markboth{At Uruvelā (1st) }{Paṭhamauruvelasutta}
\extramarks{AN 4.21}{AN 4.21}

\scevam{So\marginnote{1.1} I have heard. }At one time the Buddha was staying near \textsanskrit{Sāvatthī} in Jeta’s Grove, \textsanskrit{Anāthapiṇḍika}’s monastery. There the Buddha addressed the mendicants, “Mendicants!” 

“Venerable\marginnote{1.5} sir,” they replied. The Buddha said this: 

“Mendicants,\marginnote{2.1} this one time, when I was first awakened, I was staying near \textsanskrit{Uruvelā} at the goatherd’s banyan tree on the bank of the \textsanskrit{Nerañjarā} River. As I was in private retreat this thought came to mind: ‘One without respect and reverence lives in suffering. What ascetic or brahmin should I honor and respect and rely on?’ 

Then\marginnote{3.1} it occurred to me: ‘I would honor and respect and rely on another ascetic or brahmin so as to complete the entire spectrum of ethics, if it were incomplete. But I don’t see any other ascetic or brahmin in this world—with its gods, \textsanskrit{Māras}, and \textsanskrit{Brahmās}, this population with its ascetics and brahmins, its gods and humans—who is more accomplished than myself in ethics, who I should honor and respect and rely on. 

I\marginnote{4.1} would honor and respect and rely on another ascetic or brahmin so as to complete the entire spectrum of immersion, if it were incomplete. But I don’t see any other ascetic or brahmin … who is more accomplished than myself in immersion … 

I\marginnote{5.1} would honor and respect and rely on another ascetic or brahmin so as to complete the entrie spectrum of wisdom, if it were incomplete. But I don’t see any other ascetic or brahmin in this world … who is more accomplished than myself in wisdom … 

I\marginnote{6.1} would honor and respect and rely on another ascetic or brahmin so as to complete the entire spectrum of freedom, if it were incomplete. But I don’t see any other ascetic or brahmin in this world … who is more accomplished than myself in freedom …’ 

Then\marginnote{7.1} it occurred to me: ‘Why don’t I honor and respect and rely on the same teaching to which I was awakened?’ 

And\marginnote{8.1} then \textsanskrit{Brahmā} Sahampati, knowing what I was thinking, vanished from the \textsanskrit{Brahmā} realm and appeared in front of me, as easily as a strong man would extend or contract his arm. He arranged his robe over one shoulder, raised his joined palms toward me, and said: ‘That’s so true, Blessed One! That’s so true, Holy One! All the perfected ones, the fully awakened Buddhas who lived in the past honored and respected and relied on this same teaching. All the perfected ones, the fully awakened Buddhas who will live in the future will honor and respect and rely on this same teaching. May the Blessed One, who is the perfected one, the fully awakened Buddha at present, also honor and respect and rely on this same teaching.’ 

That’s\marginnote{8.7} what \textsanskrit{Brahmā} Sahampati said. Then he went on to say: 

\begin{verse}%
‘All\marginnote{9.1} Buddhas, whether in the past, \\
the Buddhas of the future, \\
and the Buddha at present—\\
destroyer of the sorrows of many—

respecting\marginnote{10.1} the true teaching \\
they did live, they do live, \\
and they also will live. \\
This is the nature of the Buddhas. 

Therefore\marginnote{11.1} someone who cares for their own welfare, \\
and wants to become the very best they can be, \\
should respect the true teaching, \\
remembering the instructions of the Buddhas.’ 

%
\end{verse}

That’s\marginnote{12.1} what \textsanskrit{Brahmā} Sahampati said. Then he bowed and respectfully circled me, keeping me on his right side, before vanishing right there. Then, knowing the request of \textsanskrit{Brahmā} and what was suitable for myself, I honored and respected and relied on the same teaching to which I was awakened. And since the \textsanskrit{Saṅgha} has also achieved greatness, I also respect the \textsanskrit{Saṅgha}.” 

%
\section*{{\suttatitleacronym AN 4.22}{\suttatitletranslation At Uruvelā (2nd) }{\suttatitleroot Dutiyauruvelasutta}}
\addcontentsline{toc}{section}{\tocacronym{AN 4.22} \toctranslation{At Uruvelā (2nd) } \tocroot{Dutiyauruvelasutta}}
\markboth{At Uruvelā (2nd) }{Dutiyauruvelasutta}
\extramarks{AN 4.22}{AN 4.22}

“Mendicants,\marginnote{1.1} this one time, when I was first awakened, I was staying near \textsanskrit{Uruvelā} at the goatherd’s banyan tree on the bank of the \textsanskrit{Nerañjarā} River. Then several old brahmins—elderly and senior, who were advanced in years and had reached the final stage of life—came up to me, and exchanged greetings with me. 

When\marginnote{1.3} the greetings and polite conversation were over, they sat down to one side, and said to me: ‘Master Gotama, we have heard this: 

“The\marginnote{1.5} ascetic Gotama does not bow to old brahmins, elderly and senior, who are advanced in years and have reached the final stage of life; nor does he rise in their presence or offer them a seat.” And this is indeed the case, for Master Gotama does not bow to old brahmins, elderly and senior, who are advanced in years and have reached the final stage of life; nor does he rise in their presence or offer them a seat. This is not appropriate, Master Gotama.’ 

Then\marginnote{2.1} it occurred to me, ‘These venerables don’t know what a senior is, or what qualities make you a senior.’ 

Mendicants,\marginnote{2.3} suppose you’re eighty, ninety, or a hundred years old. But your speech is untimely, false, meaningless, and against the teaching or training. You say things at the wrong time which are worthless, unreasonable, rambling, and unbeneficial. Then you’ll be considered a ‘childish senior’. 

Now\marginnote{3.1} suppose you’re a youth, young, black-haired, blessed with youth, in the prime of life. But your speech is timely, true, meaningful, and in line with the teaching and training. You say things at the right time which are valuable, reasonable, succinct, and beneficial. Then you’ll be considered an ‘astute senior’. 

There\marginnote{4.1} are these four qualities that make a senior. What four? A mendicant is ethical, restrained in the monastic code, conducting themselves well and seeking alms in suitable places. Seeing danger in the slightest fault, they keep the rules they’ve undertaken. 

They’re\marginnote{4.4} very learned, remembering and keeping what they’ve learned. These teachings are good in the beginning, good in the middle, and good in the end, meaningful and well-phrased, describing a spiritual practice that’s entirely full and pure. They are very learned in such teachings, remembering them, reinforcing them by recitation, mentally scrutinizing them, and comprehending them theoretically. 

They\marginnote{4.5} get the four absorptions—blissful meditations in the present life that belong to the higher mind—when they want, without trouble or difficulty. 

They\marginnote{4.6} realize the undefiled freedom of heart and freedom by wisdom in this very life. And they live having realized it with their own insight due to the ending of defilements. 

These\marginnote{4.7} are the four qualities that make a senior. 

\begin{verse}%
The\marginnote{5.1} creature with a restless mind \\
speaks a lot of nonsense. \\
Their thoughts are unsettled, \\
and they don’t like the true teaching. \\
They’re far from seniority, with their bad views \\
and their lack of regard for others. 

But\marginnote{6.1} one accomplished in ethics, \\
learned and eloquent, that wise one \\
is restrained when experiencing phenomena, \\
discerning the meaning with wisdom. 

Gone\marginnote{7.1} beyond all things, \\
kind, eloquent, \\
they’ve given up birth and death, \\
and have completed the spiritual journey. 

That’s\marginnote{8.1} who I call a senior, \\
who has no defilements. \\
With the ending of defilements, a mendicant \\
is declared a ‘senior’.” 

%
\end{verse}

%
\section*{{\suttatitleacronym AN 4.23}{\suttatitletranslation The World }{\suttatitleroot Lokasutta}}
\addcontentsline{toc}{section}{\tocacronym{AN 4.23} \toctranslation{The World } \tocroot{Lokasutta}}
\markboth{The World }{Lokasutta}
\extramarks{AN 4.23}{AN 4.23}

“Mendicants,\marginnote{1.1} the world has been understood by the Realized One; and he is detached from the world. The origin of the world has been understood by the Realized One; and he has given up the origin of the world. The cessation of the world has been understood by the Realized One; and he has realized the cessation of the world. The practice that leads to the cessation of the world has been understood by the Realized One; and he has developed the practice that leads to the cessation of the world. 

In\marginnote{2.1} this world—with its gods, \textsanskrit{Māras}, and \textsanskrit{Brahmās}, this population with its ascetics and brahmins, its gods and humans—whatever is seen, heard, thought, known, attained, sought, and explored by the mind, all that has been understood by the Realized One. That’s why he’s called the ‘Realized One’. 

From\marginnote{3.1} the night when the Realized One understands the supreme perfect awakening until the night he becomes fully extinguished—through the element of extinguishment with nothing left over—everything he speaks, says, and expresses is real, not otherwise. That’s why he’s called the ‘Realized One’. 

The\marginnote{4.1} Realized One does as he says, and says as he does. Since this is so, that’s why he’s called the ‘Realized One’. 

In\marginnote{5.1} this world—with its gods, \textsanskrit{Māras} and \textsanskrit{Brahmās}, this population with its ascetics and brahmins, gods and humans—the Realized One is the undefeated, the champion, the universal seer, the wielder of power. That’s why he’s called the ‘Realized One’. 

\begin{verse}%
Directly\marginnote{6.1} knowing the whole world as it is, \\
and everything in it, \\
he is detached from the whole world, \\
disengaged from the whole world. 

That\marginnote{7.1} wise one is the champion \\
who is released from all ties. \\
He has reached ultimate peace: \\
extinguishment, fearing nothing from any quarter. 

He\marginnote{8.1} is the Buddha, with defilements ended, \\
untroubled, with doubts cut off. \\
He has attained the end of all deeds, \\
freed with the end of attachments. 

That\marginnote{9.1} Blessed One is the Buddha, \\
he is the supreme lion, \\
in all the world with its gods, \\
he turns the holy wheel. 

And\marginnote{10.1} so those gods and humans, \\
who have gone to the Buddha for refuge, \\
come together and revere him, \\
great of heart and rid of naivety: 

‘Tamed,\marginnote{11.1} he is the best of tamers, \\
peaceful, he is the hermit among the peaceful, \\
liberated, he is the foremost of liberators, \\
crossed over, he is the most excellent of guides across.’ 

And\marginnote{12.1} so they revere him, \\
great of heart and rid of naivety. \\
In the world with its gods, \\
he has no counterpart.” 

%
\end{verse}

%
\section*{{\suttatitleacronym AN 4.24}{\suttatitletranslation At Kāḷaka’s Monastery }{\suttatitleroot Kāḷakārāmasutta}}
\addcontentsline{toc}{section}{\tocacronym{AN 4.24} \toctranslation{At Kāḷaka’s Monastery } \tocroot{Kāḷakārāmasutta}}
\markboth{At Kāḷaka’s Monastery }{Kāḷakārāmasutta}
\extramarks{AN 4.24}{AN 4.24}

At\marginnote{1.1} one time the Buddha was staying near \textsanskrit{Sāketa}, in \textsanskrit{Kāḷaka}’s monastery. There the Buddha addressed the mendicants, “Mendicants!” 

“Venerable\marginnote{1.4} sir,” they replied. The Buddha said this: 

“In\marginnote{2.1} this world—with its gods, \textsanskrit{Māras} and \textsanskrit{Brahmās}, this population with its ascetics and brahmins, its gods and humans—whatever is seen, heard, thought, known, attained, sought, and explored by the mind: that I know. 

In\marginnote{3.1} this world—with its gods, \textsanskrit{Māras}, and \textsanskrit{Brahmās}, this population with its ascetics and brahmins, its gods and humans—whatever is seen, heard, thought, known, attained, sought, and explored by the mind: that I have insight into. That has been known by a Realized One, but a Realized One is not subject to it. 

If\marginnote{4.1} I were to say that ‘I do not know … the world with its gods’, I would be lying. 

If\marginnote{5.1} I were to say that ‘I both know and do not know … the world with its gods’, that would be just the same. 

If\marginnote{6.1} I were to say that ‘I neither know nor do not know … the world with its gods’, that would be my fault. 

So\marginnote{7.1} a Realized One sees what is to be seen, but does not identify with what is seen, does not identify with what is unseen, does not identify with what is to be seen, and does not identify with a seer. He hears what is to be heard, but does not identify with what is heard, does not identify with what is unheard, does not identify with what is to be heard, and does not identify with a hearer. He thinks what is to be thought, but does not identify with what is thought, does not identify with what is not thought, does not identify with what is to be thought, and does not identify with a thinker. He knows what is to be known, but does not identify with what is known, does not identify with what is unknown, does not identify with what is to be known, and does not identify with a knower. 

Since\marginnote{7.5} a Realized One is poised in the midst of things seen, heard, thought, and known, he is the poised one. And I say that there is no better or finer poise than this. 

\begin{verse}%
Such\marginnote{8.1} a one does not take anything \\
seen, heard, or thought to be ultimately true or false. \\
But others get attached, thinking it’s the truth, \\
limited by their preconceptions. 

Since\marginnote{9.1} they’ve seen this dart \\
to which people are attached and cling, \\
saying, ‘I know, I see, that’s how it is’, \\
the Realized Ones have no attachments.” 

%
\end{verse}

%
\section*{{\suttatitleacronym AN 4.25}{\suttatitletranslation The Spiritual Life }{\suttatitleroot Brahmacariyasutta}}
\addcontentsline{toc}{section}{\tocacronym{AN 4.25} \toctranslation{The Spiritual Life } \tocroot{Brahmacariyasutta}}
\markboth{The Spiritual Life }{Brahmacariyasutta}
\extramarks{AN 4.25}{AN 4.25}

“Mendicants,\marginnote{1.1} this spiritual life is not lived for the sake of deceiving people or flattering them, nor for the benefit of possessions, honor, or popularity, nor for the benefit of winning debates, nor thinking, ‘So let people know about me!’ This spiritual life is lived for the sake of restraint, giving up, fading away, and cessation. 

\begin{verse}%
The\marginnote{2.1} Buddha taught the spiritual life \\
not because of tradition, \\
but for the sake of restraint and giving up, \\
and because it culminates in extinguishment. \\
This is the path followed by the great souls, \\
the great hermits. 

Those\marginnote{3.1} who practice it \\
as it was taught by the Buddha, \\
doing the teacher’s bidding, \\
make an end of suffering.” 

%
\end{verse}

%
\section*{{\suttatitleacronym AN 4.26}{\suttatitletranslation Deceivers }{\suttatitleroot Kuhasutta}}
\addcontentsline{toc}{section}{\tocacronym{AN 4.26} \toctranslation{Deceivers } \tocroot{Kuhasutta}}
\markboth{Deceivers }{Kuhasutta}
\extramarks{AN 4.26}{AN 4.26}

“Mendicants,\marginnote{1.1} those mendicants who are deceivers and flatterers, pompous and fake, insolent, and scattered: those mendicants are no followers of mine. They’ve left this teaching and training, and they don’t achieve growth, improvement, or maturity in this teaching and training. 

But\marginnote{1.3} those mendicants who are genuine, not flatterers, wise, amenable, and serene: those mendicants are followers of mine. They haven’t left this teaching and training, and they achieve growth, improvement, or maturity in this teaching and training. 

\begin{verse}%
Those\marginnote{2.1} who are deceivers and flatterers, pompous and fake, \\
insolent and scattered: \\
these don’t grow in the teaching \\
that was taught by the perfected Buddha. 

But\marginnote{3.1} those who are genuine, not flatterers, wise, \\
amenable, and serene: \\
these do grow in the teaching \\
that was taught by the perfected Buddha.” 

%
\end{verse}

%
\section*{{\suttatitleacronym AN 4.27}{\suttatitletranslation Contentment }{\suttatitleroot Santuṭṭhisutta}}
\addcontentsline{toc}{section}{\tocacronym{AN 4.27} \toctranslation{Contentment } \tocroot{Santuṭṭhisutta}}
\markboth{Contentment }{Santuṭṭhisutta}
\extramarks{AN 4.27}{AN 4.27}

“Mendicants,\marginnote{1.1} these four trifles are easy to find and are blameless. What four? Rag-robes … A lump of almsfood … Lodgings at the root of a tree … Fermented urine as medicine … 

These\marginnote{1.7} four trifles are easy to find and are blameless. When a mendicant is content with trifles that are easy to find, they have one of the factors of the ascetic life, I say. 

\begin{verse}%
When\marginnote{2.1} you’re content with what’s blameless, \\
trifling, and easy to find, \\
you don’t get upset \\
about lodgings, robes, \\
food, and drink, \\
and you’re not obstructed anywhere. 

These\marginnote{3.1} qualities are said to be \\
integral to the ascetic life. \\
They’re mastered by one who trains, \\
content and diligent.” 

%
\end{verse}

%
\section*{{\suttatitleacronym AN 4.28}{\suttatitletranslation The Noble Traditions }{\suttatitleroot Ariyavaṁsasutta}}
\addcontentsline{toc}{section}{\tocacronym{AN 4.28} \toctranslation{The Noble Traditions } \tocroot{Ariyavaṁsasutta}}
\markboth{The Noble Traditions }{Ariyavaṁsasutta}
\extramarks{AN 4.28}{AN 4.28}

“Mendicants,\marginnote{1.1} these four noble traditions are primordial, long-standing, traditional, and ancient. They are uncorrupted, as they have been since the beginning. They’re not being corrupted now, nor will they be. Sensible ascetics and brahmins don’t look down on them. What four? 

Firstly,\marginnote{1.3} a mendicant is content with any kind of robe, and praises such contentment. They don’t try to get hold of a robe in an improper way. They don’t get upset if they don’t get a robe. And if they do get a robe, they use it untied, uninfatuated, unattached, seeing the drawback, and understanding the escape. But they don’t glorify themselves or put others down on account of their contentment. A mendicant who is deft, tireless, aware, and mindful in this is said to stand in the ancient, primordial noble tradition. 

Furthermore,\marginnote{2.1} a mendicant is content with any kind of almsfood … 

Furthermore,\marginnote{3.1} a mendicant is content with any kind of lodgings … 

Furthermore,\marginnote{4.1} a mendicant enjoys meditation and loves to meditate. They enjoy giving up and love to give up. But they don’t glorify themselves or put down others on account of their love for meditation and giving up. A mendicant who is deft, tireless, aware, and mindful in this is said to stand in the ancient, primordial noble tradition. 

These\marginnote{4.4} four noble traditions are primordial, long-standing, traditional, and ancient. They are uncorrupted, as they have been since the beginning. They’re not being corrupted now nor will they be. Sensible ascetics and brahmins don’t look down on them. 

When\marginnote{5.1} a mendicant has these four noble traditions, if they live in the east they prevail over discontent, and discontent doesn’t prevail over them. If they live in the west … the north … the south, they prevail over discontent, and discontent doesn’t prevail over them. Why is that? Because a wise one prevails over desire and discontent. 

\begin{verse}%
Discontent\marginnote{6.1} doesn’t prevail over a wise one; \\
for the wise one is not beaten by discontent. \\
A wise one prevails over discontent, \\
for the wise one is a beater of discontent. 

Who\marginnote{7.1} can hold back the dispeller, \\
who’s thrown away all karma? \\
Like a pendant of river gold, \\
who is worthy to criticize them? \\
Even the gods praise them, \\
and by \textsanskrit{Brahmā}, too, they’re praised.” 

%
\end{verse}

%
\section*{{\suttatitleacronym AN 4.29}{\suttatitletranslation Footprints of the Dhamma }{\suttatitleroot Dhammapadasutta}}
\addcontentsline{toc}{section}{\tocacronym{AN 4.29} \toctranslation{Footprints of the Dhamma } \tocroot{Dhammapadasutta}}
\markboth{Footprints of the Dhamma }{Dhammapadasutta}
\extramarks{AN 4.29}{AN 4.29}

“Mendicants,\marginnote{1.1} these four footprints of the Dhamma are primordial, long-standing, traditional, and ancient. They are uncorrupted, as they have been since the beginning. They’re not being corrupted now nor will they be. Sensible ascetics and brahmins don’t look down on them. What four? Contentment, good will, right mindfulness, and right immersion. 

These\marginnote{4.2} four footprints of the Dhamma are primordial, long-standing, traditional, and ancient. They are uncorrupted, as they have been since the beginning. They’re not being corrupted now nor will they be. Sensible ascetics and brahmins don’t look down on them. 

\begin{verse}%
You\marginnote{5.1} should live with contentment, \\
and a heart of good will, \\
mindful, with unified mind, \\
serene within.” 

%
\end{verse}

%
\section*{{\suttatitleacronym AN 4.30}{\suttatitletranslation Wanderers }{\suttatitleroot Paribbājakasutta}}
\addcontentsline{toc}{section}{\tocacronym{AN 4.30} \toctranslation{Wanderers } \tocroot{Paribbājakasutta}}
\markboth{Wanderers }{Paribbājakasutta}
\extramarks{AN 4.30}{AN 4.30}

At\marginnote{1.1} one time the Buddha was staying near \textsanskrit{Rājagaha}, on the Vulture’s Peak Mountain. 

Now\marginnote{1.2} at that time several very well-known wanderers were residing in the monastery of the wanderers on the bank of the \textsanskrit{Sappinī} river. They included \textsanskrit{Annabhāra}, Varadhara, \textsanskrit{Sakuludāyī}, and other very well-known wanderers. Then in the late afternoon, the Buddha came out of retreat and went to the wanderer’s monastery on the banks of the \textsanskrit{Sappinī} river, He sat down on the seat spread out, and said to the wanderers: 

“Wanderers,\marginnote{2.1} these four footprints of the Dhamma are primordial, long-standing, traditional, and ancient. They are uncorrupted, as they have been since the beginning. They’re not being corrupted now nor will they be. Sensible ascetics and brahmins don’t look down on them. What four? Contentment … Good will … Right mindfulness … Right immersion … These four footprints of the Dhamma are primordial, long-standing, traditional, and ancient. They are uncorrupted, as they have been since the beginning. They’re not being corrupted now nor will they be. Sensible ascetics and brahmins don’t look down on them. 

Wanderers,\marginnote{3.1} if someone should say: ‘I’ll reject this Dhamma footprint of contentment, and describe a true ascetic or brahmin who covets sensual pleasures with acute lust.’ Then I’d say to them: ‘Let them come, speak, and discuss. We’ll see how powerful they are.’ It’s simply impossible to reject this Dhamma footprint of contentment, and point out a true ascetic or brahmin who covets sensual pleasures with acute lust. 

If\marginnote{4.1} someone should say: ‘I’ll reject this Dhamma footprint of good will, and describe a true ascetic or brahmin who has ill will and malicious intent.’ Then I’d say to them: ‘Let them come, speak, and discuss. We’ll see how powerful they are.’ It’s simply impossible to reject this Dhamma footprint of good will, and point out a true ascetic or brahmin who has ill will and malicious intent. 

If\marginnote{5.1} someone should say: ‘I’ll reject this Dhamma footprint of right mindfulness, and describe a true ascetic or brahmin who is unmindful, with no situational awareness.’ Then I’d say to them: ‘Let them come, speak, and discuss. We’ll see how powerful they are.’ It’s simply impossible to reject this Dhamma footprint of right mindfulness, and point out a true ascetic or brahmin who is unmindful, with no situational awareness. 

If\marginnote{6.1} someone should say: ‘I’ll reject this Dhamma footprint of right immersion, and describe a true ascetic or brahmin who is scattered, with straying mind.’ Then I’d say to them: ‘Let them come, speak, and discuss. We’ll see how powerful they are.’ It’s simply impossible to reject this Dhamma footprint of right immersion, and point out a true ascetic or brahmin who is scattered, with straying mind. 

If\marginnote{7.1} anyone imagines they can criticize and reject these four footprints of the Dhamma, they deserve rebuke and criticism on four legitimate grounds in the present life. What four? 

If\marginnote{7.3} you reject the Dhamma footprint of contentment, then you must honor and praise those ascetics and brahmins who covet sensual pleasures with acute lust. 

If\marginnote{7.4} you reject the Dhamma footprint of good will, you must honor and praise those ascetics and brahmins who have ill will and malicious intent. 

If\marginnote{7.5} you reject the Dhamma footprint of right mindfulness, then you must honor and praise those ascetics and brahmins who are unmindful, with no situational awareness. 

If\marginnote{7.6} you reject the Dhamma footprint of right immersion, you must honor and praise those ascetics and brahmins who are scattered, with straying minds. 

If\marginnote{8.1} anyone imagines they can criticize and reject these four footprints of the Dhamma, they deserve rebuke and criticism on four legitimate grounds in the present life. 

Even\marginnote{8.2} those wanderers of the past, Vassa and \textsanskrit{Bhañña} of \textsanskrit{Ukkalā}, who taught the doctrines of no-cause, inaction, and nihilism, didn’t imagine that these four footprints of the Dhamma should be criticized or rejected. Why is that? For fear of being blamed, criticized, and faulted. 

\begin{verse}%
One\marginnote{9.1} who has good will, ever mindful, \\
serene within, \\
training to remove desire, \\
is called ‘a diligent one’.” 

%
\end{verse}

%
\addtocontents{toc}{\let\protect\contentsline\protect\nopagecontentsline}
\chapter*{The Chapter on Situations }
\addcontentsline{toc}{chapter}{\tocchapterline{The Chapter on Situations }}
\addtocontents{toc}{\let\protect\contentsline\protect\oldcontentsline}

%
\section*{{\suttatitleacronym AN 4.31}{\suttatitletranslation Situations }{\suttatitleroot Cakkasutta}}
\addcontentsline{toc}{section}{\tocacronym{AN 4.31} \toctranslation{Situations } \tocroot{Cakkasutta}}
\markboth{Situations }{Cakkasutta}
\extramarks{AN 4.31}{AN 4.31}

“Mendicants,\marginnote{1.1} there are these four situations. When these situations come about, any god or human who takes advantage of them will soon acquire great and abundant wealth. What four? Living in a suitable region, relying on good people, being rightly resolved in oneself, and past merit. 

These\marginnote{1.4} are the four situations. When these situations come about, any god or human who takes advantage of them will soon acquire great and abundant wealth. 

\begin{verse}%
When\marginnote{2.1} a person lives in a suitable region, \\
making friends with noble ones, \\
possessing right resolve, \\
and having merit from the past, \\
grain, riches, fame, reputation, \\
and happiness come to them.” 

%
\end{verse}

%
\section*{{\suttatitleacronym AN 4.32}{\suttatitletranslation Inclusion }{\suttatitleroot Saṅgahasutta}}
\addcontentsline{toc}{section}{\tocacronym{AN 4.32} \toctranslation{Inclusion } \tocroot{Saṅgahasutta}}
\markboth{Inclusion }{Saṅgahasutta}
\extramarks{AN 4.32}{AN 4.32}

“Mendicants,\marginnote{1.1} there are these four ways of being inclusive. What four? Giving, kindly words, taking care, and equality. 

These\marginnote{1.4} are the four ways of being inclusive. 

\begin{verse}%
Giving\marginnote{2.1} and kindly words, \\
taking care here, \\
and equality in worldly conditions, \\
in each case as they deserve. \\
These ways of being inclusive in the world \\
are like a moving chariot’s linchpin. 

If\marginnote{3.1} there were no such ways of being inclusive, \\
neither mother nor father \\
would be respected and honored \\
for what they’ve done for their children. 

But\marginnote{4.1} since these ways of being inclusive do exist, \\
the astute do regard them well, \\
so they achieve greatness \\
and are praised.” 

%
\end{verse}

%
\section*{{\suttatitleacronym AN 4.33}{\suttatitletranslation The Lion }{\suttatitleroot Sīhasutta}}
\addcontentsline{toc}{section}{\tocacronym{AN 4.33} \toctranslation{The Lion } \tocroot{Sīhasutta}}
\markboth{The Lion }{Sīhasutta}
\extramarks{AN 4.33}{AN 4.33}

“Mendicants,\marginnote{1.1} towards evening the lion, king of beasts, emerges from his den, yawns, looks all around the four quarters, and roars his lion’s roar three times. Then he sets out on the hunt. And the animals who hear the roar of the lion, king of beasts, are typically filled with fear, awe, and terror. They return to their lairs, be they in a hole, the water, or a wood; and the birds take to the air. Even the royal elephants, bound with strong harnesses in the villages, towns, and capital cities, break apart their bonds, and urinate and defecate in terror as they flee here and there. That’s how powerful is the lion, king of beasts, among animals, how illustrious and mighty. 

In\marginnote{2.1} the same way, when a Realized One arises in the world—perfected, a fully awakened Buddha, accomplished in knowledge and conduct, holy, knower of the world, supreme guide for those who wish to train, teacher of gods and humans, awakened, blessed—he teaches the Dhamma: ‘Such is identity, such is the origin of identity, such is the cessation of identity, such is the practice that leads to the cessation of identity.’ 

Now,\marginnote{2.3} there are gods who are long-lived, beautiful, and very happy, lasting long in their divine palaces. When they hear this teaching by the Realized One, they’re typically filled with fear, awe, and terror. ‘Oh no! It turns out we’re impermanent, though we thought we were permanent! It turns out we don’t last, though we thought we were everlasting! It turns out we’re transient, though we thought we were eternal! It turns out that we’re impermanent, not lasting, transient, and included within identity.’ 

That’s\marginnote{2.8} how powerful is the Realized One in the world with its gods, how illustrious and mighty. 

\begin{verse}%
The\marginnote{3.1} Buddha, the teacher without a peer \\
in all the world with its gods, \\
rolls forth the Wheel of Dhamma \\
from his own insight: 

identity,\marginnote{4.1} its cessation, \\
the origin of identity, \\
and the noble eightfold path \\
that leads to the stilling of suffering. 

And\marginnote{5.1} then the long-lived gods, \\
so beautiful and famous, \\
are afraid and full of terror, \\
like the other beasts when they hear a lion. 

‘We\marginnote{6.1} haven’t transcended identity! \\
It turns out we’re impermanent!’ \\
So they say when they hear the word \\
of the perfected one, free and poised.” 

%
\end{verse}

%
\section*{{\suttatitleacronym AN 4.34}{\suttatitletranslation The Best Kinds of Confidence }{\suttatitleroot Aggappasādasutta}}
\addcontentsline{toc}{section}{\tocacronym{AN 4.34} \toctranslation{The Best Kinds of Confidence } \tocroot{Aggappasādasutta}}
\markboth{The Best Kinds of Confidence }{Aggappasādasutta}
\extramarks{AN 4.34}{AN 4.34}

“Mendicants,\marginnote{1.1} these four kinds of confidence are the best. What four? 

The\marginnote{1.3} Realized One, the perfected one, the fully awakened Buddha, is said to be the best of all sentient beings—be they footless, with two feet, four feet, or many feet; with form or formless; with perception or without perception or with neither perception nor non-perception. Those who have confidence in the Buddha have confidence in the best. Having confidence in the best, the result is the best. 

The\marginnote{2.1} noble eightfold path is said to be the best of all conditioned things. Those who have confidence in the noble eightfold path have confidence in the best. Having confidence in the best, the result is the best. 

Fading\marginnote{3.1} away is said to be the best of all things whether conditioned or unconditioned. That is, the quelling of vanity, the removing of thirst, the abolishing of clinging, the breaking of the round, the ending of craving, fading away, cessation, extinguishment. Those who have confidence in the teaching of fading away have confidence in the best. Having confidence in the best, the result is the best. 

The\marginnote{4.1} \textsanskrit{Saṅgha} of the Realized One’s disciples is said to be the best of all communities and groups. It consists of the four pairs, the eight individuals. This is the \textsanskrit{Saṅgha} of the Buddha’s disciples that is worthy of offerings dedicated to the gods, worthy of hospitality, worthy of a religious donation, worthy of greeting with joined palms, and is the supreme field of merit for the world. Those who have confidence in the \textsanskrit{Saṅgha} have confidence in the best. Having confidence in the best, the result is the best. 

These\marginnote{4.4} are the four best kinds of confidence. 

\begin{verse}%
For\marginnote{5.1} those who, knowing the best teaching, \\
base their confidence on the best—\\
confident in the best Awakened One, \\
supremely worthy of a religious donation; 

confident\marginnote{6.1} in the best teaching, \\
the bliss of fading and stilling; \\
confident in the best \textsanskrit{Saṅgha}, \\
the supreme field of merit—

giving\marginnote{7.1} gifts to the best, \\
the best of merit grows: \\
the best lifespan, beauty, \\
fame, reputation, happiness, and strength. 

An\marginnote{8.1} intelligent person gives to the best, \\
settled on the best teaching. \\
When they become a god or human, \\
they rejoice at reaching the best.” 

%
\end{verse}

%
\section*{{\suttatitleacronym AN 4.35}{\suttatitletranslation With Vassakāra }{\suttatitleroot Vassakārasutta}}
\addcontentsline{toc}{section}{\tocacronym{AN 4.35} \toctranslation{With Vassakāra } \tocroot{Vassakārasutta}}
\markboth{With Vassakāra }{Vassakārasutta}
\extramarks{AN 4.35}{AN 4.35}

At\marginnote{1.1} one time the Buddha was staying near \textsanskrit{Rājagaha}, in the Bamboo Grove, the squirrels’ feeding ground. Then \textsanskrit{Vassakāra} the brahmin, a chief minister of Magadha, went up to the Buddha, and exchanged greetings with him. When the greetings and polite conversation were over, he sat down to one side and said to the Buddha: 

“Master\marginnote{2.1} Gotama, when someone has four qualities we describe him as a great man with great wisdom. What four? 

They\marginnote{2.3} are very learned in diverse fields of learning. 

They\marginnote{2.4} understand the meaning of diverse statements, saying: ‘This is what that statement means; that is what this statement means.’ 

They\marginnote{2.5} are mindful, able to remember and recollect what was said and done long ago. 

They\marginnote{2.6} are deft and tireless in household duties, understanding how to go about things in order to complete and organize the work. 

When\marginnote{2.7} someone has these four qualities we describe him as a great man with great wisdom. If Master Gotama agrees with me, please say so. If he disagrees, please say so.” 

“Brahmin,\marginnote{3.1} I neither agree nor disagree with you, but when someone has four qualities I describe him as a great man with great wisdom. What four? 

It’s\marginnote{3.4} when someone practices for the welfare and happiness of the people. They’ve established many people in the noble method, that is, the principles of goodness and skillfulness. 

They\marginnote{3.6} think what they want to think, and don’t think what they don’t want to think. They consider what they want to consider, and don’t consider what they don’t want to consider. Thus they have achieved mental mastery of the paths of thought. 

They\marginnote{3.9} get the four absorptions—blissful meditations in the present life that belong to the higher mind—when they want, without trouble or difficulty. 

They\marginnote{3.10} realize the undefiled freedom of heart and freedom by wisdom in this very life. And they live having realized it with their own insight due to the ending of defilements. 

Brahmin,\marginnote{3.11} I neither agree nor disagree with you, but when someone has these four qualities I describe him as a great man with great wisdom.” 

“It’s\marginnote{4.1} incredible, Master Gotama, it’s amazing! How well said this was by Master Gotama! And we will remember Master Gotama as someone who has these four qualities. For Master Gotama practices for the welfare and happiness of the people … Master Gotama has achieved mental mastery of the paths of thought. Master Gotama gets the four absorptions … when he wants, without trouble or difficulty. Master Gotama has realized the undefiled freedom of heart and freedom by wisdom in this very life. He lives having realized it with his own insight due to the ending of defilements.” 

“Your\marginnote{5.1} words are clearly invasive and intrusive, brahmin. Nevertheless, I will answer you. For I do practice for the welfare and happiness of the people … I have achieved mental mastery of the paths of thought. I do get the four absorptions … when I want, without trouble or difficulty. I have realized the undefiled freedom of heart and freedom by wisdom in this very life. I live having realized it with my own insight due to the ending of defilements. 

\begin{verse}%
He\marginnote{6.1} discovered release from the snare of death \\
for all beings, \\
and explained the method of the teaching \\
for the welfare of gods and humans. \\
When they see him or hear him, \\
many people become confident. 

He\marginnote{7.1} is skilled in the variety of paths, \\
he has completed the task and is free of defilements. \\
The Buddha, bearing his final body, \\
is called ‘a great man, of great wisdom’.” 

%
\end{verse}

%
\section*{{\suttatitleacronym AN 4.36}{\suttatitletranslation Doṇa }{\suttatitleroot Doṇasutta}}
\addcontentsline{toc}{section}{\tocacronym{AN 4.36} \toctranslation{Doṇa } \tocroot{Doṇasutta}}
\markboth{Doṇa }{Doṇasutta}
\extramarks{AN 4.36}{AN 4.36}

At\marginnote{1.1} one time the Buddha was traveling along the road between \textsanskrit{Ukkaṭṭhā} and \textsanskrit{Setavyā}, as was the brahmin \textsanskrit{Doṇa}. 

\textsanskrit{Doṇa}\marginnote{1.3} saw that the Buddha’s footprints had thousand-spoked wheels, with rims and hubs, complete in every detail. It occurred to him, “It’s incredible, it’s amazing! Surely these couldn’t be the footprints of a human being?” 

The\marginnote{1.7} Buddha had left the road and sat at the root of a tree cross-legged, with his body straight and his mindfulness established right there. 

Then\marginnote{1.8} \textsanskrit{Doṇa}, following the Buddha’s footprints, saw him sitting at the tree root—impressive and inspiring, with peaceful faculties and mind, attained to the highest self-control and serenity, like an elephant with tamed, guarded, and controlled faculties. He went up to the Buddha and said to him: 

“Sir,\marginnote{2.1} might you be a god?” 

“I\marginnote{2.2} will not be a god, brahmin.” 

“Might\marginnote{2.3} you be a fairy?” 

“I\marginnote{2.4} will not be a fairy.” 

“Might\marginnote{2.5} you be a native spirit?” 

“I\marginnote{2.6} will not be a native spirit.” 

“Might\marginnote{2.7} you be a human?” 

“I\marginnote{2.8} will not be a human.” 

“When\marginnote{3.1} asked whether you might be a god, fairy, native spirit, or human, you answer that you will not be any of these. What then might you be?” 

“Brahmin,\marginnote{4.1} if I had not given up defilements I might have become a god … a fairy … a native spirit … or a human. But I have given up those defilements, cut them off at the root, made them like a palm stump, obliterated them so they are unable to arise in the future. 

Suppose\marginnote{4.3} there was a blue water lily, or a pink or white lotus. Though it sprouted and grew in the water, it would rise up above the water and stand with no water clinging to it. In the same way, though I was born and grew up in the world, I live having mastered the world, unsullied by the world. 

Remember\marginnote{4.5} me, brahmin, as a Buddha. 

\begin{verse}%
I\marginnote{5.1} could have been reborn as a god, \\
or as a fairy flying through the sky. \\
I could have become a native spirit, \\
or returned as a human. \\
But I’ve ended those defilements, \\
they’re blown away and mown down. 

Like\marginnote{6.1} a graceful lotus, \\
to which water does not cling, \\
the world doesn’t cling to me, \\
and so, brahmin, I am a Buddha.” 

%
\end{verse}

%
\section*{{\suttatitleacronym AN 4.37}{\suttatitletranslation Non-decline }{\suttatitleroot Aparihāniyasutta}}
\addcontentsline{toc}{section}{\tocacronym{AN 4.37} \toctranslation{Non-decline } \tocroot{Aparihāniyasutta}}
\markboth{Non-decline }{Aparihāniyasutta}
\extramarks{AN 4.37}{AN 4.37}

“Mendicants,\marginnote{1.1} a mendicant who has four qualities can’t decline, and has drawn near to extinguishment. What four? A mendicant is accomplished in ethics, guards the sense doors, eats in moderation, and is dedicated to wakefulness. 

And\marginnote{2.1} how is a mendicant accomplished in ethics? It’s when a mendicant is ethical, restrained in the monastic code, conducting themselves well and seeking alms in suitable places. Seeing danger in the slightest fault, they keep the rules they’ve undertaken. That’s how a mendicant is accomplished in ethics. 

And\marginnote{3.1} how does a mendicant guard the sense doors? When a mendicant sees a sight with their eyes, they don’t get caught up in the features and details. If the faculty of sight were left unrestrained, bad unskillful qualities of desire and aversion would become overwhelming. For this reason, they practice restraint, protecting the faculty of sight, and achieving restraint over it. Hearing a sound with their ears … Smelling an odor with their nose … Tasting a flavor with their tongue … Feeling a touch with their body … Knowing a thought with their mind, they don’t get caught up in the features and details. If the faculty of mind were left unrestrained, bad unskillful qualities of desire and aversion would become overwhelming. For this reason, they practice restraint, protecting the faculty of mind, and achieving restraint over it. That’s how a mendicant guards the sense doors. 

And\marginnote{4.1} how does a mendicant eat in moderation? It’s when a mendicant reflects properly on the food that they eat: ‘Not for fun, indulgence, adornment, or decoration, but only to sustain this body, to avoid harm, and to support spiritual practice. In this way, I shall put an end to old discomfort and not give rise to new discomfort, and I will live blamelessly and at ease.’ That’s how a mendicant eats in moderation. 

And\marginnote{5.1} how is a mendicant dedicated to wakefulness? It’s when a mendicant practices walking and sitting meditation by day, purifying their mind from obstacles. In the evening, they continue to practice walking and sitting meditation. In the middle of the night, they lie down in the lion’s posture—on the right side, placing one foot on top of the other—mindful and aware, and focused on the time of getting up. In the last part of the night, they get up and continue to practice walking and sitting meditation, purifying their mind from obstacles. This is how a mendicant is dedicated to wakefulness. A mendicant who has these four qualities can’t decline, and has drawn near to extinguishment. 

\begin{verse}%
Established\marginnote{6.1} in ethics, \\
restrained in the sense faculties, \\
eating in moderation, \\
and dedicated to wakefulness; 

a\marginnote{7.1} mendicant lives like this, with keen energy, \\
tireless all night and day, \\
developing skillful qualities, \\
for the sake of finding sanctuary. 

A\marginnote{8.1} mendicant who loves to be diligent, \\
seeing fear in negligence, \\
can’t decline, \\
and has drawn near to extinguishment.” 

%
\end{verse}

%
\section*{{\suttatitleacronym AN 4.38}{\suttatitletranslation Withdrawn }{\suttatitleroot Patilīnasutta}}
\addcontentsline{toc}{section}{\tocacronym{AN 4.38} \toctranslation{Withdrawn } \tocroot{Patilīnasutta}}
\markboth{Withdrawn }{Patilīnasutta}
\extramarks{AN 4.38}{AN 4.38}

“Mendicants,\marginnote{1.1} a mendicant has eliminated idiosyncratic interpretations of the truth, has totally given up searching, has stilled the physical process, and is said to be ‘withdrawn’. 

And\marginnote{1.2} how has a mendicant eliminated idiosyncratic interpretations of the truth? Different ascetics and brahmins have different idiosyncratic interpretations of the truth. For example: the cosmos is eternal, or not eternal, or finite, or infinite; the soul and the body are the same thing, or they are different things; after death, a Realized One exists, or doesn’t exist, or both exists and doesn’t exist, or neither exists nor doesn’t exist. A mendicant has dispelled, eliminated, thrown out, rejected, let go of, given up, and relinquished all these. That’s how a mendicant has eliminated idiosyncratic interpretations of the truth. 

And\marginnote{2.1} how has a mendicant totally given up searching? It’s when they’ve given up searching for sensual pleasures, for continued existence, and for a spiritual path. That’s how a mendicant has totally given up searching. 

And\marginnote{3.1} how has a mendicant stilled the physical process? It’s when, giving up pleasure and pain, and ending former happiness and sadness, they enter and remain in the fourth absorption, without pleasure or pain, with pure equanimity and mindfulness. That’s how a mendicant has stilled the physical process. 

And\marginnote{4.1} how is a mendicant withdrawn? It’s when they’ve given up the conceit ‘I am’, cut it off at the root, made it like a palm stump, obliterated it, so it’s unable to arise in the future. That’s how a mendicant is withdrawn. A mendicant has eliminated idiosyncratic interpretations of the truth, has totally given up searching, has stilled the physical process, and is said to be ‘withdrawn’. 

\begin{verse}%
The\marginnote{5.1} search for sensual pleasures, for a good rebirth, \\
and the search for a spiritual path; \\
the holding tight to the notion ‘this is the truth’, \\
and the mass of grounds for views—

for\marginnote{6.1} one detached from all lusts, \\
freed by the ending of craving, \\
that searching has been relinquished, \\
and those viewpoints eradicated. 

That\marginnote{7.1} mendicant is peaceful and mindful, \\
a tranquil champion. \\
And when they’re awakened by comprehending conceit, \\
they’re called ‘withdrawn’.” 

%
\end{verse}

%
\section*{{\suttatitleacronym AN 4.39}{\suttatitletranslation With Ujjaya }{\suttatitleroot Ujjayasutta}}
\addcontentsline{toc}{section}{\tocacronym{AN 4.39} \toctranslation{With Ujjaya } \tocroot{Ujjayasutta}}
\markboth{With Ujjaya }{Ujjayasutta}
\extramarks{AN 4.39}{AN 4.39}

Then\marginnote{1.1} Ujjaya the brahmin went up to the Buddha, and exchanged greetings with him. When the greetings and polite conversation were over, he sat down to one side and said to the Buddha: 

“Does\marginnote{1.3} Master Gotama praise sacrifice?” 

“Brahmin,\marginnote{1.4} I don’t praise all sacrifices. Nor do I criticize all sacrifices. Take the kind of sacrifice where cattle, goats and sheep, chickens and pigs, and various kinds of creatures are slaughtered. I criticize that kind of violent sacrifice. Why is that? Because neither perfected ones nor those who are on the path to perfection will attend such a violent sacrifice. 

But\marginnote{2.1} take the kind of sacrifice where cattle, goats and sheep, chickens and pigs, and various kinds of creatures are not slaughtered. I praise that kind of non-violent sacrifice; for example, a regular gift as an ongoing family sacrifice. Why is that? Because perfected ones and those who are on the path to perfection will attend such a non-violent sacrifice. 

\begin{verse}%
Horse\marginnote{3.1} sacrifice, human sacrifice, \\
the sacrifices of the ‘stick-casting’, \\
the ‘royal soma drinking’, and the ‘unbarred’—\\
these huge violent sacrifices yield no great fruit. 

The\marginnote{4.1} great sages of good conduct \\
don’t attend sacrifices \\
where goats, sheep, and cattle, \\
and various creatures are killed. 

But\marginnote{5.1} the great sages of good conduct \\
do attend non-violent sacrifices \\
of regular family tradition, \\
where goats, sheep, and cattle, \\
and various creatures aren’t killed. 

An\marginnote{6.1} intelligent person should sacrifice like this, \\
for this sacrifice is very fruitful. \\
For a sponsor of sacrifices like this, \\
things get better, not worse. \\
Such a sacrifice is truly abundant, \\
and even the deities are pleased.” 

%
\end{verse}

%
\section*{{\suttatitleacronym AN 4.40}{\suttatitletranslation With Udāyī }{\suttatitleroot Udāyīsutta}}
\addcontentsline{toc}{section}{\tocacronym{AN 4.40} \toctranslation{With Udāyī } \tocroot{Udāyīsutta}}
\markboth{With Udāyī }{Udāyīsutta}
\extramarks{AN 4.40}{AN 4.40}

Then\marginnote{1.1} \textsanskrit{Udāyī} the brahmin went up to the Buddha, … and asked him, “Does Master Gotama praise sacrifice?” 

“Brahmin,\marginnote{1.3} I don’t praise all sacrifices. Nor do I criticize all sacrifices. … Take the kind of sacrifice where cattle, goats and sheep, chickens and pigs, and various kinds of animals are slaughtered. I don’t praise that kind of violent sacrifice. 

But\marginnote{2.1} take the kind of sacrifice where cattle, goats and sheep, chickens and pigs, and various kinds of animals are not slaughtered. I do praise that kind of non-violent sacrifice; for example, a regular gift as an ongoing family sacrifice. 

\begin{verse}%
The\marginnote{3.1} kind of sacrifice that’s allowable and timely, \\
well prepared and non-violent, \\
is attended by \\
disciplined spiritual practitioners. 

The\marginnote{4.1} Buddhas—who have drawn back the veil from the world, \\
transcending time and rebirth—\\
praise this sacrifice, \\
as they are expert in sacrifice. 

When\marginnote{5.1} you’ve prepared a suitable offering, \\
whether as sacrifice or for ancestors, \\
sacrifice it with confident heart, \\
in the fertile field of spiritual practitioners. 

When\marginnote{6.1} it’s well-gotten, well-offered, and well-sacrificed, \\
to those worthy of a religious donation, \\
a sacrifice is truly abundant, \\
and even the deities are pleased. 

When\marginnote{7.1} an intelligent, faithful person, \\
sacrifices like this, with a mind of letting go, \\
that astute one is reborn \\
in a happy, pleasing world.” 

%
\end{verse}

%
\addtocontents{toc}{\let\protect\contentsline\protect\nopagecontentsline}
\chapter*{The Chapter with Rohitassa }
\addcontentsline{toc}{chapter}{\tocchapterline{The Chapter with Rohitassa }}
\addtocontents{toc}{\let\protect\contentsline\protect\oldcontentsline}

%
\section*{{\suttatitleacronym AN 4.41}{\suttatitletranslation Ways of Developing Immersion Further }{\suttatitleroot Samādhibhāvanāsutta}}
\addcontentsline{toc}{section}{\tocacronym{AN 4.41} \toctranslation{Ways of Developing Immersion Further } \tocroot{Samādhibhāvanāsutta}}
\markboth{Ways of Developing Immersion Further }{Samādhibhāvanāsutta}
\extramarks{AN 4.41}{AN 4.41}

“Mendicants,\marginnote{1.1} there are these four ways of developing immersion further. What four? There is a way of developing immersion further that leads to blissful meditation in the present life. There is a way of developing immersion further that leads to gaining knowledge and vision. There is a way of developing immersion further that leads to mindfulness and awareness. There is a way of developing immersion further that leads to the ending of defilements. 

And\marginnote{2.1} what is the way of developing immersion further that leads to blissful meditation in the present life? It’s when a mendicant, quite secluded from sensual pleasures, secluded from unskillful qualities, enters and remains in the first absorption … second absorption … third absorption … fourth absorption. This is the way of developing immersion further that leads to blissful meditation in the present life. 

And\marginnote{3.1} what is the way of developing immersion further that leads to gaining knowledge and vision? It’s when a mendicant focuses on the perception of light, concentrating on the perception of day, regardless of whether it’s night or day. And so, with an open and unenveloped heart, they develop a mind that’s full of radiance. This is the way of developing immersion further that leads to gaining knowledge and vision. 

And\marginnote{4.1} what is the way of developing immersion further that leads to mindfulness and awareness? It’s when a mendicant knows feelings as they arise, as they remain, and as they go away. They know perceptions as they arise, as they remain, and as they go away. They know thoughts as they arise, as they remain, and as they go away. This is the way of developing immersion further that leads to mindfulness and awareness. 

And\marginnote{5.1} what is the way of developing immersion further that leads to the ending of defilements? It’s when a mendicant meditates observing rise and fall in the five grasping aggregates. ‘Such is form, such is the origin of form, such is the ending of form. Such is feeling, such is the origin of feeling, such is the ending of feeling. Such is perception, such is the origin of perception, such is the ending of perception. Such are choices, such is the origin of choices, such is the ending of choices. Such is consciousness, such is the origin of consciousness, such is the ending of consciousness.’ This is the way of developing immersion further that leads to the ending of defilements. 

These\marginnote{5.9} are the four ways of developing immersion further. And it was in this connection that I said in ‘The Way to the Far Shore’, in ‘The Questions of \textsanskrit{Puṇṇaka}’: 

\begin{verse}%
‘Having\marginnote{6.1} assessed the world high and low, \\
there is nothing in the world that disturbs them. \\
Peaceful, unclouded, untroubled, with no need for hope, \\
they’ve crossed over rebirth and old age, I declare.’” 

%
\end{verse}

%
\section*{{\suttatitleacronym AN 4.42}{\suttatitletranslation Ways of Answering Questions }{\suttatitleroot Pañhabyākaraṇasutta}}
\addcontentsline{toc}{section}{\tocacronym{AN 4.42} \toctranslation{Ways of Answering Questions } \tocroot{Pañhabyākaraṇasutta}}
\markboth{Ways of Answering Questions }{Pañhabyākaraṇasutta}
\extramarks{AN 4.42}{AN 4.42}

“Mendicants,\marginnote{1.1} there are these four ways of answering questions. What four? There is a question that should be answered definitively. There is a question that should be answered analytically. There is a question that should be answered with a counter-question. There is a question that should be set aside. 

These\marginnote{1.7} are the four ways of answering questions. 

\begin{verse}%
One\marginnote{2.1} is stated definitively, \\
another analytically, \\
a third with a counter-question, \\
while a fourth is set aside. 

A\marginnote{3.1} mendicant who knows each of these, \\
in line with good principles, \\
is said to be skilled \\
in the four kinds of questions. 

They’re\marginnote{4.1} intimidating, hard to defeat, \\
deep, and hard to crush. \\
They’re expert in both \\
what the meaning is and what it isn’t. 

Rejecting\marginnote{5.1} what is not the meaning, \\
an astute person grasps the meaning. \\
A wise one, comprehending the meaning, \\
is said to be astute.” 

%
\end{verse}

%
\section*{{\suttatitleacronym AN 4.43}{\suttatitletranslation Valuing Anger }{\suttatitleroot Paṭhamakodhagarusutta}}
\addcontentsline{toc}{section}{\tocacronym{AN 4.43} \toctranslation{Valuing Anger } \tocroot{Paṭhamakodhagarusutta}}
\markboth{Valuing Anger }{Paṭhamakodhagarusutta}
\extramarks{AN 4.43}{AN 4.43}

“Mendicants,\marginnote{1.1} these four people are found in the world. What four? People who value anger, or denigration, or material possessions, or honor rather than the true teaching. These are the four people found in the world. 

These\marginnote{2.1} four people are found in the world. What four? People who value the true teaching rather than anger, or denigration, or material possessions, or honor. These are the four people found in the world. 

\begin{verse}%
Mendicants\marginnote{3.1} who value anger and denigration, \\
possessions and honor, \\
don’t grow in the teaching \\
that was taught by the perfected Buddha. 

But\marginnote{4.1} those who value the true teaching, \\
who have lived it, and are living it now, \\
these do grow in the teaching \\
that was taught by the perfected Buddha.” 

%
\end{verse}

%
\section*{{\suttatitleacronym AN 4.44}{\suttatitletranslation Valuing Anger (2nd) }{\suttatitleroot Dutiyakodhagarusutta}}
\addcontentsline{toc}{section}{\tocacronym{AN 4.44} \toctranslation{Valuing Anger (2nd) } \tocroot{Dutiyakodhagarusutta}}
\markboth{Valuing Anger (2nd) }{Dutiyakodhagarusutta}
\extramarks{AN 4.44}{AN 4.44}

“Mendicants,\marginnote{1.1} these four things oppose the true teaching. What four? Valuing anger, denigration, material possessions, and honor rather than the true teaching. These are the four things that oppose the true teaching. 

These\marginnote{2.1} four things are the true teaching. What four? Valuing the true teaching rather than anger, denigration, material possessions, and honor. These are the four things that are the true teaching. 

\begin{verse}%
A\marginnote{3.1} mendicant who values anger and denigration, \\
possessions and honor, \\
doesn’t grow in the true teaching, \\
like a rotten seed in a good field. 

But\marginnote{4.1} those who value the true teaching, \\
who have lived it, and are living it now, \\
these do grow in the teaching, \\
like well-watered herbs.” 

%
\end{verse}

%
\section*{{\suttatitleacronym AN 4.45}{\suttatitletranslation With Rohitassa }{\suttatitleroot Rohitassasutta}}
\addcontentsline{toc}{section}{\tocacronym{AN 4.45} \toctranslation{With Rohitassa } \tocroot{Rohitassasutta}}
\markboth{With Rohitassa }{Rohitassasutta}
\extramarks{AN 4.45}{AN 4.45}

At\marginnote{1.1} one time the Buddha was staying near \textsanskrit{Sāvatthī} in Jeta’s Grove, \textsanskrit{Anāthapiṇḍika}’s monastery. 

Then,\marginnote{1.2} late at night, the glorious god Rohitassa, lighting up the entire Jeta’s Grove, went up to the Buddha, bowed, stood to one side, and said to him: 

“Sir,\marginnote{2.1} is it possible to know or see or reach the end of the world by traveling to a place where there’s no being born, growing old, dying, passing away, or being reborn?” 

“Reverend,\marginnote{2.2} I say it’s not possible to know or see or reach the end of the world by traveling to a place where there’s no being born, growing old, dying, passing away, or being reborn.” 

“It’s\marginnote{3.1} incredible, sir, it’s amazing, how well said this was by the Buddha. 

Once\marginnote{4.1} upon a time, I was a hermit called Rohitassa, son of Bhoja. I was a sky-walker with psychic powers. I was as fast as a light arrow easily shot across the shadow of a palm tree by a well-trained expert archer with a strong bow. My stride was such that it could span from the eastern ocean to the western ocean. Having such speed and stride, this wish came to me: ‘I will reach the end of the world by traveling.’ I traveled for my whole lifespan of a hundred years—pausing only to eat and drink, go to the toilet, and sleep to dispel weariness—and I passed away along the way, never reaching the end of the world. 

It’s\marginnote{5.1} incredible, sir, it’s amazing, how well said this was by the Buddha.” 

“Reverend,\marginnote{6.1} I say it’s not possible to know or see or reach the end of the world by traveling to a place where there’s no being born, growing old, dying, passing away, or being reborn. But I also say there’s no making an end of suffering without reaching the end of the world. For it is in this fathom-long carcass with its perception and mind that I describe the world, its origin, its cessation, and the practice that leads to its cessation. 

\begin{verse}%
The\marginnote{7.1} end of the world can never \\
be reached by traveling. \\
But without reaching the end of the world, \\
there’s no release from suffering. 

So\marginnote{8.1} a clever person, understanding the world, \\
has completed the spiritual journey, and gone to the end of the world. \\
A peaceful one, knowing the end of the world, \\
does not long for this world or the next.” 

%
\end{verse}

%
\section*{{\suttatitleacronym AN 4.46}{\suttatitletranslation With Rohitassa (2nd) }{\suttatitleroot Dutiyarohitassasutta}}
\addcontentsline{toc}{section}{\tocacronym{AN 4.46} \toctranslation{With Rohitassa (2nd) } \tocroot{Dutiyarohitassasutta}}
\markboth{With Rohitassa (2nd) }{Dutiyarohitassasutta}
\extramarks{AN 4.46}{AN 4.46}

Then,\marginnote{1.1} when the night had passed, the Buddha addressed the mendicants: “Tonight, the glorious god Rohitassa, lighting up the entire Jeta’s Grove, came to me, bowed, stood to one side, and said to me: ‘Sir, is it possible to know or see or reach the end of the world by traveling to a place where there’s no being born, growing old, dying, passing away, or being reborn?’ … 

(The\marginnote{1.4} rest of this discourse is the same as the previous discourse, AN 4.45.) 

%
\section*{{\suttatitleacronym AN 4.47}{\suttatitletranslation Very Far Apart }{\suttatitleroot Suvidūrasutta}}
\addcontentsline{toc}{section}{\tocacronym{AN 4.47} \toctranslation{Very Far Apart } \tocroot{Suvidūrasutta}}
\markboth{Very Far Apart }{Suvidūrasutta}
\extramarks{AN 4.47}{AN 4.47}

“Mendicants,\marginnote{1.1} these four things are very far apart. What four? The sky and the earth. … The near and the far shore of the ocean. … Where the sun rises and where it sets. … The teaching of the virtuous and the teaching of the wicked. … These are the four things very far apart. 

\begin{verse}%
The\marginnote{2.1} sky is far from the earth; \\
they say the other shore of the ocean is far; \\
and where the sun rises is far \\
from where that beacon sets. \\
But even further apart than that, they say, \\
is the teaching of the virtuous from the wicked. 

The\marginnote{3.1} company of the virtuous is reliable; \\
as long as it remains, it stays the same. \\
But the company of the wicked is fickle, \\
and so the teaching of the virtuous is far from the wicked.” 

%
\end{verse}

%
\section*{{\suttatitleacronym AN 4.48}{\suttatitletranslation With Visākha, Pañcāli’s Son }{\suttatitleroot Visākhasutta}}
\addcontentsline{toc}{section}{\tocacronym{AN 4.48} \toctranslation{With Visākha, Pañcāli’s Son } \tocroot{Visākhasutta}}
\markboth{With Visākha, Pañcāli’s Son }{Visākhasutta}
\extramarks{AN 4.48}{AN 4.48}

At\marginnote{1.1} one time the Buddha was staying near \textsanskrit{Sāvatthī} in Jeta’s Grove, \textsanskrit{Anāthapiṇḍika}’s monastery. 

Now\marginnote{1.2} at that time Venerable \textsanskrit{Visākha}, \textsanskrit{Pañcāli}’s son, was educating, encouraging, firing up, and inspiring the mendicants in the assembly hall with a Dhamma talk. His words were polished, clear, articulate, expressing the meaning, comprehensive, and independent. 

Then\marginnote{1.3} in the late afternoon, the Buddha came out of retreat and went to the assembly hall. He sat down on the seat spread out, and addressed the mendicants, “Mendicants, who was educating, encouraging, firing up, and inspiring the mendicants in the assembly hall with a Dhamma talk?” 

“Sir,\marginnote{2.2} it was Venerable \textsanskrit{Visākha}, \textsanskrit{Pañcāli}’s son.” 

Then\marginnote{3.1} the Buddha said to \textsanskrit{Visākha}, “Good, good, \textsanskrit{Visākha}! It’s good that you educate, encourage, fire up, and inspire the mendicants in the assembly hall with a Dhamma talk, with words that are polished, clear, articulate, expressing the meaning, comprehensive, and independent. 

\begin{verse}%
Though\marginnote{4.1} an astute person is mixed up with fools, \\
they don’t know unless he speaks. \\
But when he speaks they know, \\
he’s teaching the deathless state. 

He\marginnote{5.1} should speak and illustrate the teaching, \\
holding up the banner of the hermits. \\
Words well spoken are the hermits’ banner, \\
for the teaching is the banner of the hermits.” 

%
\end{verse}

%
\section*{{\suttatitleacronym AN 4.49}{\suttatitletranslation Perversions }{\suttatitleroot Vipallāsasutta}}
\addcontentsline{toc}{section}{\tocacronym{AN 4.49} \toctranslation{Perversions } \tocroot{Vipallāsasutta}}
\markboth{Perversions }{Vipallāsasutta}
\extramarks{AN 4.49}{AN 4.49}

“Mendicants,\marginnote{1.1} there are these four perversions of perception, mind, and view. What four? 

\begin{enumerate}%
\item Taking impermanence as permanence. %
\item Taking suffering as happiness. %
\item Taking not-self as self. %
\item Taking ugliness as beauty. %
\end{enumerate}

These\marginnote{1.7} are the four perversions of perception, mind, and view. 

There\marginnote{2.1} are these four corrections of perception, mind, and view. What four? 

\begin{enumerate}%
\item Taking impermanence as impermanence. %
\item Taking suffering as suffering. %
\item Taking not-self as not-self. %
\item Taking ugliness as ugliness. %
\end{enumerate}

These\marginnote{2.7} are the four corrections of perception, mind, and view. 

\begin{verse}%
Perceiving\marginnote{3.1} impermanence as permanence, \\
suffering as happiness, \\
not-self as self, \\
and ugliness as beauty—\\
sentient beings are ruined by wrong view, \\
deranged, out of their mind. 

Yoked\marginnote{4.1} by \textsanskrit{Māra}’s yoke, these people \\
find no sanctuary from the yoke. \\
Sentient beings continue to transmigrate, \\
with ongoing birth and death. 

But\marginnote{5.1} when the Buddhas arise \\
in the world, those beacons \\
reveal this teaching, \\
that leads to the stilling of suffering. 

When\marginnote{6.1} a wise person hears them, \\
they get their mind back. \\
Seeing impermanence as impermanence, \\
suffering as suffering, 

not-self\marginnote{7.1} as not-self, \\
and ugliness as ugliness—\\
taking up right view, \\
they’ve risen above all suffering.” 

%
\end{verse}

%
\section*{{\suttatitleacronym AN 4.50}{\suttatitletranslation Corruptions }{\suttatitleroot Upakkilesasutta}}
\addcontentsline{toc}{section}{\tocacronym{AN 4.50} \toctranslation{Corruptions } \tocroot{Upakkilesasutta}}
\markboth{Corruptions }{Upakkilesasutta}
\extramarks{AN 4.50}{AN 4.50}

“Mendicants,\marginnote{1.1} these four corruptions obscure the sun and moon, so they don’t shine and glow and radiate. What four? Clouds … Fog … Smoke … An eclipse of \textsanskrit{Rāhu}, lord of demons … These are four corruptions that obscure the sun and moon, so they don’t shine and glow and radiate. 

In\marginnote{5.1} the same way, these four things corrupt ascetics and brahmins, so they don’t shine and glow and radiate. What four? 

There\marginnote{5.3} are some ascetics and brahmins who drink liquor, not avoiding drinking liquor. This is the first thing that corrupts ascetics and brahmins … 

There\marginnote{6.1} are some ascetics and brahmins who have sex, not avoiding sex. This is the second thing that corrupts ascetics and brahmins … 

There\marginnote{7.1} are some ascetics and brahmins who accept gold and money, not avoiding receiving gold and money. This is the third thing that corrupts ascetics and brahmins … 

There\marginnote{8.1} are some ascetics and brahmins who make a living the wrong way, not avoiding wrong livelihood. This is the fourth thing that corrupts ascetics and brahmins … 

These\marginnote{8.3} are four things that corrupt ascetics and brahmins, so they don’t shine and glow and radiate. 

\begin{verse}%
Some\marginnote{9.1} ascetics and brahmins \\
are plagued by greed and hate; \\
men shrouded by ignorance, \\
enjoying things that seem pleasant. 

Drinking\marginnote{10.1} liquor, \\
having sex, \\
accepting money and gold: \\
they’re ignorant. \\
Some ascetics and brahmins \\
make a living the wrong way. 

These\marginnote{11.1} corruptions were spoken of \\
by the Buddha, kinsman of the Sun. \\
When corrupted by these, \\
some ascetics and brahmins \\
don’t shine or glow. \\
Impure, dirty creatures, 

shrouded\marginnote{12.1} in darkness, \\
bondservants of craving, full of attachments, \\
swell the horrors of the charnel ground, \\
taking up future lives. 

%
\end{verse}

%
\addtocontents{toc}{\let\protect\contentsline\protect\nopagecontentsline}
\pannasa{The Second Fifty }
\addcontentsline{toc}{pannasa}{The Second Fifty }
\markboth{}{}
\addtocontents{toc}{\let\protect\contentsline\protect\oldcontentsline}

%
\addtocontents{toc}{\let\protect\contentsline\protect\nopagecontentsline}
\chapter*{The Chapter on Overflowing Merit }
\addcontentsline{toc}{chapter}{\tocchapterline{The Chapter on Overflowing Merit }}
\addtocontents{toc}{\let\protect\contentsline\protect\oldcontentsline}

%
\section*{{\suttatitleacronym AN 4.51}{\suttatitletranslation Overflowing Merit }{\suttatitleroot Paṭhamapuññābhisandasutta}}
\addcontentsline{toc}{section}{\tocacronym{AN 4.51} \toctranslation{Overflowing Merit } \tocroot{Paṭhamapuññābhisandasutta}}
\markboth{Overflowing Merit }{Paṭhamapuññābhisandasutta}
\extramarks{AN 4.51}{AN 4.51}

At\marginnote{1.1} \textsanskrit{Sāvatthī}. 

“Mendicants,\marginnote{1.2} there are these four kinds of overflowing merit, overflowing goodness. They nurture happiness and are conducive to heaven, ripening in happiness and leading to heaven. They lead to what is likable, desirable, agreeable, to welfare and happiness. What four? 

When\marginnote{1.4} a mendicant enters and remains in a limitless immersion of heart while using a robe, the overflowing of merit for the donor is limitless … 

When\marginnote{2.1} a mendicant enters and remains in a limitless immersion of heart while eating almsfood, the overflowing of merit for the donor is limitless … 

When\marginnote{3.1} a mendicant enters and remains in a limitless immersion of heart while using lodgings, the overflowing of merit for the donor is limitless … 

When\marginnote{4.1} a mendicant enters and remains in a limitless immersion of heart while using medicines and supplies for the sick, the overflowing of merit for the donor is limitless … 

These\marginnote{4.2} are the four kinds of overflowing merit, overflowing goodness. They nurture happiness and are conducive to heaven, ripening in happiness and leading to heaven. They lead to what is likable, desirable, agreeable, to welfare and happiness. 

When\marginnote{5.1} a noble disciple has these four kinds of overflowing merit and goodness, it’s not easy to grasp how much merit they have by saying that this is the extent of their overflowing merit … that leads to happiness. It’s simply reckoned as an incalculable, immeasurable, great mass of merit. 

It’s\marginnote{6.1} like trying to grasp how much water is in the ocean. It’s not easy to say how many gallons, how many hundreds, thousands, hundreds of thousands of gallons there are. It’s simply reckoned as an incalculable, immeasurable, great mass of water. 

In\marginnote{6.2} the same way, when a noble disciple has these four kinds of overflowing merit it’s simply reckoned as an incalculable, immeasurable, great mass of merit. 

\begin{verse}%
Hosts\marginnote{7.1} of people use the rivers, \\
and though the rivers are many, \\
all reach the great deep, the boundless ocean, \\
the cruel sea that’s home to precious gems. 

In\marginnote{8.1} the same way, when a person gives food, drink, and clothes; \\
and they’re a giver of beds, seats, and mats—\\
the streams of merit reach that astute person, \\
as the rivers bring their waters to the sea.” 

%
\end{verse}

%
\section*{{\suttatitleacronym AN 4.52}{\suttatitletranslation Overflowing Merit (2nd) }{\suttatitleroot Dutiyapuññābhisandasutta}}
\addcontentsline{toc}{section}{\tocacronym{AN 4.52} \toctranslation{Overflowing Merit (2nd) } \tocroot{Dutiyapuññābhisandasutta}}
\markboth{Overflowing Merit (2nd) }{Dutiyapuññābhisandasutta}
\extramarks{AN 4.52}{AN 4.52}

“Mendicants,\marginnote{1.1} there are these four kinds of overflowing merit, overflowing goodness. They nurture happiness and are conducive to heaven, ripening in happiness and leading to heaven. They lead to what is likable, desirable, agreeable, to welfare and happiness. What four? 

It’s\marginnote{1.3} when a noble disciple has experiential confidence in the Buddha: ‘That Blessed One is perfected, a fully awakened Buddha, accomplished in knowledge and conduct, holy, knower of the world, supreme guide for those who wish to train, teacher of gods and humans, awakened, blessed.’ This is the first kind of overflowing merit … 

Furthermore,\marginnote{2.1} a noble disciple has experiential confidence in the teaching: ‘The teaching is well explained by the Buddha—visible in this very life, immediately effective, inviting inspection, relevant, so that sensible people can know it for themselves.’ This is the second kind of overflowing merit … 

Furthermore,\marginnote{3.1} a noble disciple has experiential confidence in the \textsanskrit{Saṅgha}: ‘The \textsanskrit{Saṅgha} of the Buddha’s disciples is practicing the way that’s good, direct, methodical, and proper. It consists of the four pairs, the eight individuals. This is the \textsanskrit{Saṅgha} of the Buddha’s disciples that is worthy of offerings dedicated to the gods, worthy of hospitality, worthy of a religious donation, worthy of greeting with joined palms, and is the supreme field of merit for the world.’ This is the third kind of overflowing merit … 

Furthermore,\marginnote{4.1} a noble disciple’s ethical conduct is loved by the noble ones, unbroken, impeccable, spotless, and unmarred, liberating, praised by sensible people, not mistaken, and leading to immersion. This is the fourth kind of overflowing merit … 

These\marginnote{4.3} are the four kinds of overflowing merit, overflowing goodness. They nurture happiness and are conducive to heaven, ripening in happiness and leading to heaven. They lead to what is likable, desirable, agreeable, to welfare and happiness. 

\begin{verse}%
Whoever\marginnote{5.1} has faith in the Realized One, \\
unwavering and well grounded; \\
whose ethical conduct is good, \\
praised and loved by the noble ones; 

who\marginnote{6.1} has confidence in the \textsanskrit{Saṅgha}, \\
and correct view: \\
they’re said to be prosperous, \\
their life is not in vain. 

So\marginnote{7.1} let the wise devote themselves \\
to faith, ethical behavior, \\
confidence, and insight into the teaching, \\
remembering the instructions of the Buddhas. 

%
\end{verse}

%
\section*{{\suttatitleacronym AN 4.53}{\suttatitletranslation Living Together (1st) }{\suttatitleroot Paṭhamasaṁvāsasutta}}
\addcontentsline{toc}{section}{\tocacronym{AN 4.53} \toctranslation{Living Together (1st) } \tocroot{Paṭhamasaṁvāsasutta}}
\markboth{Living Together (1st) }{Paṭhamasaṁvāsasutta}
\extramarks{AN 4.53}{AN 4.53}

At\marginnote{1.1} one time the Buddha was traveling along the road between Madhura and \textsanskrit{Verañja}, as were several householders, both women and men. The Buddha left the road and sat at the root of a tree, where the householders saw him. 

They\marginnote{1.5} went up to the Buddha, bowed, and sat down to one side. The Buddha said to them: 

“Householders,\marginnote{2.1} there are four ways of living together. What four? 

\begin{enumerate}%
\item A male zombie living with a female zombie; %
\item a male zombie living with a goddess; %
\item a god living with a female zombie; %
\item a god living with a goddess. %
\end{enumerate}

And\marginnote{3.1} how does a male zombie live with a female zombie? It’s when the husband kills living creatures, steals, commits sexual misconduct, lies, and uses alcoholic drinks that cause negligence. He’s unethical, of bad character, living at home with his heart full of the stain of stinginess, abusing and insulting ascetics and brahmins. And the wife is also … unethical, of bad character … That’s how a male zombie lives with a female zombie. 

And\marginnote{4.1} how does a male zombie live with a goddess? It’s when the husband … is unethical, of bad character … But the wife doesn’t kill living creatures, steal, commit sexual misconduct, lie, or use alcoholic drinks that cause negligence. She’s ethical, of good character, living at home with her heart rid of the stain of stinginess, not abusing and insulting ascetics and brahmins. That’s how a male zombie lives with a goddess. 

And\marginnote{5.1} how does a god live with a female zombie? It’s when the husband … is ethical, of good character … But the wife … is unethical, of bad character … That’s how a god lives with a female zombie. 

And\marginnote{6.1} how does a god live with a goddess? It’s when the husband … is ethical, of good character … And the wife is also … ethical, of good character … That’s how a god lives with a goddess. 

These\marginnote{6.5} are the four ways of living together. 

\begin{verse}%
When\marginnote{7.1} both are unethical, \\
miserly and abusive, \\
then wife and husband \\
live together as zombies. 

When\marginnote{8.1} the husband is unethical, \\
miserly and abusive, \\
but the wife is ethical, \\
bountiful, rid of stinginess, \\
she’s a goddess living \\
with a zombie for a husband. 

When\marginnote{9.1} the husband is ethical, \\
bountiful, rid of stinginess, \\
but the wife is unethical, \\
miserly and abusive, \\
she’s a zombie living \\
with a god for a husband. 

When\marginnote{10.1} both are faithful and bountiful, \\
disciplined, living righteously, \\
then wife and husband \\
say nice things to each other. 

They\marginnote{11.1} get all the things they need, \\
so they live at ease. \\
Their enemies are downhearted, \\
when both are equal in ethics. 

Having\marginnote{12.1} practiced the teaching here, \\
both equal in precepts and observances, \\
they delight in the heavenly realm, \\
enjoying all the pleasures they desire.” 

%
\end{verse}

%
\section*{{\suttatitleacronym AN 4.54}{\suttatitletranslation Living Together (2nd) }{\suttatitleroot Dutiyasaṁvāsasutta}}
\addcontentsline{toc}{section}{\tocacronym{AN 4.54} \toctranslation{Living Together (2nd) } \tocroot{Dutiyasaṁvāsasutta}}
\markboth{Living Together (2nd) }{Dutiyasaṁvāsasutta}
\extramarks{AN 4.54}{AN 4.54}

“Mendicants,\marginnote{1.1} there are four ways of living together. What four? 

\begin{enumerate}%
\item A male zombie living with a female zombie; %
\item a male zombie living with a goddess; %
\item a god living with a female zombie; %
\item a god living with a goddess. %
\end{enumerate}

And\marginnote{2.1} how does a male zombie live with a female zombie? It’s when the husband kills living creatures, steals, commits sexual misconduct; he uses speech that’s false, divisive, harsh, or nonsensical; and he’s covetous, malicious, and has wrong view. He’s unethical, of bad character, living at home with his heart full of the stain of stinginess, abusing and insulting ascetics and brahmins. And the wife is also … unethical, of bad character … That’s how a male zombie lives with a female zombie. 

And\marginnote{3.1} how does a male zombie live with a goddess? It’s when the husband … is unethical, of bad character … But the wife doesn’t kill living creatures, steal, commit sexual misconduct, lie, or use alcoholic drinks that cause negligence. She doesn’t use speech that’s false, divisive, harsh, or nonsensical. And she’s contented, kind-hearted, with right view. She’s ethical, of good character, living at home with her heart rid of the stain of stinginess, not abusing and insulting ascetics and brahmins. That’s how a male zombie lives with a goddess. 

And\marginnote{4.1} how does a god live with a female zombie? It’s when the husband … is ethical, of good character … But the wife … is unethical, of bad character … That’s how a god lives with a female zombie. 

And\marginnote{5.1} how does a god live with a goddess? It’s when the husband … is ethical, of good character … And the wife is also … ethical, of good character … That’s how a god lives with a goddess. 

These\marginnote{5.5} are the four ways of living together.” … 

%
\section*{{\suttatitleacronym AN 4.55}{\suttatitletranslation Equality (1st) }{\suttatitleroot Paṭhamasamajīvīsutta}}
\addcontentsline{toc}{section}{\tocacronym{AN 4.55} \toctranslation{Equality (1st) } \tocroot{Paṭhamasamajīvīsutta}}
\markboth{Equality (1st) }{Paṭhamasamajīvīsutta}
\extramarks{AN 4.55}{AN 4.55}

\scevam{So\marginnote{1.1} I have heard. }At one time the Buddha was staying in the land of the Bhaggas on Crocodile Hill, in the deer park at \textsanskrit{Bhesakaḷā}’s Wood. 

Then\marginnote{1.3} the Buddha robed up in the morning and, taking his bowl and robe, went to the home of the householder Nakula’s father, where he sat on the seat spread out. 

Then\marginnote{1.4} the householder Nakula’s father and the housewife Nakula’s mother went up to the Buddha, bowed, and sat down to one side. Nakula’s father said to the Buddha, “Sir, ever since we were both young, and Nakula’s mother was given to me in marriage, I can’t recall betraying her even in thought, still less in deed. We want to see each other in both this life and the next.” 

Then\marginnote{2.3} Nakula’s mother said to the Buddha, “Sir, ever since we were both young, and I was given in marriage to Nakula’s father, I can’t recall betraying him even in thought, still less in deed. We want to see each other in both this life and the next.” 

“Householders,\marginnote{3.1} if wife and husband want to see each other in both this life and the next, they should be equals in faith, ethics, generosity, and wisdom. 

\begin{verse}%
When\marginnote{4.1} both are faithful and bountiful, \\
disciplined, living righteously, \\
then wife and husband \\
say nice things to each other. 

They\marginnote{5.1} get all the things they need, \\
so they live at ease. \\
Their enemies are downhearted, \\
when both are equal in ethics. 

Having\marginnote{6.1} practiced the teaching here, \\
both equal in precepts and observances, \\
they delight in the heavenly realm, \\
enjoying all the pleasures they desire.” 

%
\end{verse}

%
\section*{{\suttatitleacronym AN 4.56}{\suttatitletranslation Equality (2nd) }{\suttatitleroot Dutiyasamajīvīsutta}}
\addcontentsline{toc}{section}{\tocacronym{AN 4.56} \toctranslation{Equality (2nd) } \tocroot{Dutiyasamajīvīsutta}}
\markboth{Equality (2nd) }{Dutiyasamajīvīsutta}
\extramarks{AN 4.56}{AN 4.56}

“Mendicants,\marginnote{1.1} if wife and husband want to see each other in both this life and the next, they should be equals in faith, ethics, generosity, and wisdom. …” 

%
\section*{{\suttatitleacronym AN 4.57}{\suttatitletranslation Suppavāsā }{\suttatitleroot Suppavāsāsutta}}
\addcontentsline{toc}{section}{\tocacronym{AN 4.57} \toctranslation{Suppavāsā } \tocroot{Suppavāsāsutta}}
\markboth{Suppavāsā }{Suppavāsāsutta}
\extramarks{AN 4.57}{AN 4.57}

At\marginnote{1.1} one time the Buddha was staying in the land of the Koliyans, where they have a town named Pajjanika. 

Then\marginnote{1.2} the Buddha robed up in the morning and, taking his bowl and robe, went to the home of \textsanskrit{Suppavāsā} the Koliyan, where he sat on the seat spread out. Then \textsanskrit{Suppavāsā} served and satisfied the Buddha with her own hands with a variety of delicious foods. When the Buddha had eaten and washed his hand and bowl, she sat down to one side. The Buddha said to her: 

“\textsanskrit{Suppavāsā},\marginnote{2.1} when a noble disciple gives food, she gives the recipients four things. What four? Long life, beauty, happiness, and strength. Giving long life, she has long life as a god or human. Giving beauty, she has beauty as a god or human. Giving happiness, she has happiness as a god or human. Giving strength, she has strength as a god or human. When a noble disciple gives food, she gives the recipients these four things. 

\begin{verse}%
When\marginnote{3.1} she gives well-prepared food, \\
pure, fine, and full of flavor, \\
that offering—given to the upright, \\
who have good conduct, and are big-hearted—\\
joins merit to merit. It’s very fruitful, \\
and is praised by those who know the world. 

Those\marginnote{4.1} who recall such sacrifices, \\
live in the world full of inspiration. \\
They’ve driven out the stain of stinginess, root and all, \\
blameless, they go to a heavenly place.” 

%
\end{verse}

%
\section*{{\suttatitleacronym AN 4.58}{\suttatitletranslation Sudatta }{\suttatitleroot Sudattasutta}}
\addcontentsline{toc}{section}{\tocacronym{AN 4.58} \toctranslation{Sudatta } \tocroot{Sudattasutta}}
\markboth{Sudatta }{Sudattasutta}
\extramarks{AN 4.58}{AN 4.58}

Then\marginnote{1.1} the householder \textsanskrit{Anāthapiṇḍika} went up to the Buddha, bowed, and sat down to one side. The Buddha said to him: 

“Householder,\marginnote{2.1} when a noble disciple gives food, they give the recipients four things. What four? Long life, beauty, happiness, and strength. Giving long life, they have long life as a god or human. … Giving beauty … happiness … strength … When a noble disciple gives food, they give the recipients these four things. 

\begin{verse}%
Carefully\marginnote{3.1} giving food at the right time, \\
to those who are disciplined, eating only what others give, \\
you provide them with four things: \\
long life, beauty, happiness, and strength. 

A\marginnote{4.1} person who gives long life, beauty, \\
happiness, and strength, \\
has long life and fame \\
wherever they’re reborn.” 

%
\end{verse}

%
\section*{{\suttatitleacronym AN 4.59}{\suttatitletranslation Food }{\suttatitleroot Bhojanasutta}}
\addcontentsline{toc}{section}{\tocacronym{AN 4.59} \toctranslation{Food } \tocroot{Bhojanasutta}}
\markboth{Food }{Bhojanasutta}
\extramarks{AN 4.59}{AN 4.59}

“Mendicants,\marginnote{1.1} when a donor gives food, they give the recipients four things. What four? Long life, beauty, happiness, and strength. …” 

%
\section*{{\suttatitleacronym AN 4.60}{\suttatitletranslation Lay Practice }{\suttatitleroot Gihisāmīcisutta}}
\addcontentsline{toc}{section}{\tocacronym{AN 4.60} \toctranslation{Lay Practice } \tocroot{Gihisāmīcisutta}}
\markboth{Lay Practice }{Gihisāmīcisutta}
\extramarks{AN 4.60}{AN 4.60}

Then\marginnote{1.1} the householder \textsanskrit{Anāthapiṇḍika} went up to the Buddha, bowed, and sat down to one side. The Buddha said to him: 

“Householder,\marginnote{2.1} when a noble disciple does four things they are practicing appropriately for a layperson, which brings fame and leads to heaven. What four? It’s when a noble disciple serves the mendicant \textsanskrit{Saṅgha} with robes, almsfood, lodgings, and medicines and supplies for the sick. When a noble disciple does these four things they are practicing appropriately for a layperson, which brings fame and leads to heaven. 

\begin{verse}%
Those\marginnote{3.1} who are astute practice the way \\
that’s appropriate for laypeople. \\
They provide those who are ethical \\
and on the right path with robes, 

almsfood,\marginnote{4.1} lodgings, \\
and supplies for the sick. \\
Their merit always grows \\
by day and by night. \\
They pass on to a place in heaven, \\
having done excellent deeds.” 

%
\end{verse}

%
\addtocontents{toc}{\let\protect\contentsline\protect\nopagecontentsline}
\chapter*{The Chapter on Deeds of Substance }
\addcontentsline{toc}{chapter}{\tocchapterline{The Chapter on Deeds of Substance }}
\addtocontents{toc}{\let\protect\contentsline\protect\oldcontentsline}

%
\section*{{\suttatitleacronym AN 4.61}{\suttatitletranslation Deeds of Substance }{\suttatitleroot Pattakammasutta}}
\addcontentsline{toc}{section}{\tocacronym{AN 4.61} \toctranslation{Deeds of Substance } \tocroot{Pattakammasutta}}
\markboth{Deeds of Substance }{Pattakammasutta}
\extramarks{AN 4.61}{AN 4.61}

Then\marginnote{1.1} the householder \textsanskrit{Anāthapiṇḍika} went up to the Buddha, bowed, and sat down to one side. The Buddha said to him: 

“Householder,\marginnote{2.1} these four things that are likable, desirable, and agreeable are hard to get in the world. What four? The first thing is the wish: ‘May wealth come to me by legitimate means!’ 

The\marginnote{3.1} second thing, having got wealth by legitimate means, is the wish: ‘May fame come to me, together with my family and teachers.’ 

The\marginnote{4.1} third thing, having got wealth and fame, is the wish: ‘May I live long, keeping alive for a long time!’ 

The\marginnote{5.1} fourth thing, having got wealth, fame, and long life, is the wish: ‘When my body breaks up, after death, may I be reborn in a good place, a heavenly realm!’ These are the four things that are likable, desirable, and agreeable, but hard to get in the world. 

These\marginnote{6.1} next four things lead to the getting of those four things. What four? Accomplishment in faith, ethics, generosity, and wisdom. 

And\marginnote{7.1} what is accomplishment in faith? It’s when a noble disciple has faith in the Realized One’s awakening: ‘That Blessed One is perfected, a fully awakened Buddha, accomplished in knowledge and conduct, holy, knower of the world, supreme guide for those who wish to train, teacher of gods and humans, awakened, blessed.’ This is called accomplishment in faith. 

And\marginnote{8.1} what is accomplishment in ethics? It’s when a noble disciple doesn’t kill living creatures, steal, commit sexual misconduct, lie, or take alcoholic drinks that cause negligence. This is called accomplishment in ethics. 

And\marginnote{9.1} what is accomplishment in generosity? It’s when a noble disciple lives at home rid of the stain of stinginess, freely generous, open-handed, loving to let go, committed to charity, loving to give and to share. This is called accomplishment in generosity. 

And\marginnote{10.1} what is accomplishment in wisdom? When your heart is mastered by covetousness and immoral greed, you do what you shouldn’t, and fail to do what you should. Your fame and happiness fall to dust. When your heart is mastered by ill will … dullness and drowsiness … restlessness and remorse … doubt, you do what you shouldn’t, and fail to do what you should. Your fame and happiness fall to dust. 

Knowing\marginnote{11.1} that ‘covetousness and immoral greed are corruptions of the mind’, that noble disciple gives them up. Knowing that ‘ill will …’ … ‘dullness and drowsiness …’ … ‘restlessness and remorse …’ … ‘doubt is a corruption of the mind’, that noble disciple gives it up. 

When\marginnote{12.1} a noble disciple has given up these things, they’re called ‘a noble disciple of great wisdom, of widespread wisdom, who sees what matters, and is accomplished in wisdom’. This is called accomplishment in wisdom. These are the four things that lead to the getting of the four things that are likable, desirable, and agreeable, but hard to get in the world. 

There\marginnote{13.1} are four deeds of substance that a noble disciple does with the legitimate wealth he has earned by his efforts and initiative, built up with his own hands, gathered by the sweat of the brow. What four? 

To\marginnote{13.3} start with, with his legitimate wealth he makes himself happy and pleased, keeping himself properly happy. He makes his mother and father happy … He makes his children, partners, bondservants, workers, and staff happy … He makes his friends and colleagues happy … This is his first solid and substantive investment, used in the appropriate sphere. 

Furthermore,\marginnote{14.1} with his legitimate wealth he defends himself against threats from such things as fire, flood, rulers, bandits, or unloved heirs. He keeps himself safe. This is his second solid and substantive investment, used in the appropriate sphere. 

Furthermore,\marginnote{15.1} with his legitimate wealth he makes five spirit-offerings: to relatives, guests, ancestors, king, and deities. This is his third solid and substantive investment, used in the appropriate sphere. 

Furthermore,\marginnote{16.1} with his legitimate wealth he establishes an uplifting religious donation for ascetics and brahmins—those who avoid intoxication and negligence, are settled in patience and gentleness, and who tame, calm, and extinguish themselves—that’s conducive to heaven, ripens in happiness, and leads to heaven. This is his fourth solid and substantive investment, used in the appropriate sphere. 

These\marginnote{17.1} are the four deeds of substance that a noble disciple does with the legitimate wealth he has earned by his efforts and initiative, built up with his own hands, gathered by the sweat of the brow. 

Whatever\marginnote{17.2} wealth is spent on something other than these four deeds of substance is said to be not a solid or substantive investment, and not used in the appropriate sphere. But whatever wealth is spent on these four deeds of substance is said to be a solid and substantive investment, used in the appropriate sphere. 

\begin{verse}%
‘I’ve\marginnote{18.1} enjoyed my wealth, supporting those who depend on me; \\
I’ve overcome losses; \\
I’ve given uplifting religious donations; \\
and made the five spirit-offerings. \\
I have served the ethical and \\
disciplined spiritual practitioners. 

I’ve\marginnote{19.1} achieved the purpose \\
for which an astute lay person \\
wishes to gain wealth. \\
I don’t regret what I’ve done.’ 

A\marginnote{20.1} mortal person who recollects this \\
stands firm in the teaching of the noble ones. \\
They’re praised in this life, \\
and they depart to rejoice in heaven.” 

%
\end{verse}

%
\section*{{\suttatitleacronym AN 4.62}{\suttatitletranslation Debtlessness }{\suttatitleroot Ānaṇyasutta}}
\addcontentsline{toc}{section}{\tocacronym{AN 4.62} \toctranslation{Debtlessness } \tocroot{Ānaṇyasutta}}
\markboth{Debtlessness }{Ānaṇyasutta}
\extramarks{AN 4.62}{AN 4.62}

Then\marginnote{1.1} the householder \textsanskrit{Anāthapiṇḍika} went up to the Buddha, bowed, and sat down to one side. The Buddha said to him: 

“Householder,\marginnote{2.1} these four kinds of happiness can be earned by a layperson who enjoys sensual pleasures, depending on time and occasion. What four? The happiness of ownership, using wealth, debtlessness, and blamelessness. 

And\marginnote{3.1} what is the happiness of ownership? It’s when a gentleman owns legitimate wealth that he has earned by his own efforts and initiative, built up with his own hands, gathered by the sweat of the brow. When he reflects on this, he’s filled with pleasure and happiness. This is called ‘the happiness of ownership’. 

And\marginnote{4.1} what is the happiness of using wealth? It’s when a gentleman uses his legitimate wealth, and makes merit with it. When he reflects on this, he’s filled with pleasure and happiness. This is called ‘the happiness of using wealth’. 

And\marginnote{5.1} what is the happiness of debtlessness? It’s when a gentleman owes no debt, large or small, to anyone. When he reflects on this, he’s filled with pleasure and happiness. This is called ‘the happiness of debtlessness’. 

And\marginnote{6.1} what is the happiness of blamelessness? It’s when a noble disciple has blameless conduct by way of body, speech, and mind. When he reflects on this, he’s filled with pleasure and happiness. This is called ‘the happiness of blamelessness’. 

These\marginnote{6.5} four kinds of happiness can be earned by a layperson who enjoys sensual pleasures, depending on time and occasion. 

\begin{verse}%
Knowing\marginnote{7.1} the happiness of debtlessness, \\
and the extra happiness of possession, \\
a mortal enjoying the happiness of using wealth, \\
then sees clearly with wisdom. 

Seeing\marginnote{8.1} clearly, a clever person knows \\
both kinds of happiness: \\
the other kind is not worth a sixteenth part \\
of the happiness of blamelessness.” 

%
\end{verse}

%
\section*{{\suttatitleacronym AN 4.63}{\suttatitletranslation Living with Brahmā }{\suttatitleroot Brahmasutta}}
\addcontentsline{toc}{section}{\tocacronym{AN 4.63} \toctranslation{Living with Brahmā } \tocroot{Brahmasutta}}
\markboth{Living with Brahmā }{Brahmasutta}
\extramarks{AN 4.63}{AN 4.63}

“Mendicants,\marginnote{1.1} a family where the children honor their parents in their home is said to live with \textsanskrit{Brahmā}. A family where the children honor their parents in their home is said to live with the first teachers. A family where the children honor their parents in their home is said to live with the old deities. A family where the children honor their parents in their home is said to live with those worthy of offerings dedicated to the gods. 

‘\textsanskrit{Brahmā}’\marginnote{2.1} is a term for your parents. 

‘First\marginnote{2.2} teachers’ is a term for your parents. 

‘Old\marginnote{2.3} deities’ is a term for your parents. 

‘Worthy\marginnote{2.4} of an offering dedicated to the gods’ is a term for your parents. 

Why\marginnote{2.5} is that? Parents are very helpful to their children, they raise them, nurture them, and show them the world. 

\begin{verse}%
Parents\marginnote{3.1} are said to be ‘\textsanskrit{Brahmā}’ \\
and ‘first teachers’. \\
They’re worthy of offerings dedicated to the gods from their children, \\
for they love their offspring. 

Therefore\marginnote{4.1} an astute person \\
would revere them and honor them \\
with food and drink, \\
clothes and bedding, \\
by anointing and bathing, \\
and by washing their feet. 

Because\marginnote{5.1} they look after \\
their parents like this, \\
they’re praised in this life by the astute, \\
and they depart to rejoice in heaven.” 

%
\end{verse}

%
\section*{{\suttatitleacronym AN 4.64}{\suttatitletranslation Hell }{\suttatitleroot Nirayasutta}}
\addcontentsline{toc}{section}{\tocacronym{AN 4.64} \toctranslation{Hell } \tocroot{Nirayasutta}}
\markboth{Hell }{Nirayasutta}
\extramarks{AN 4.64}{AN 4.64}

“Mendicants,\marginnote{1.1} someone with four qualities is cast down to hell. What four? They kill living creatures, steal, commit sexual misconduct, and lie. Someone with these four qualities is cast down to hell. 

\begin{verse}%
Killing,\marginnote{2.1} stealing, \\
telling lies, \\
and visiting others’ wives: \\
astute people don’t praise these things.” 

%
\end{verse}

%
\section*{{\suttatitleacronym AN 4.65}{\suttatitletranslation Appearance }{\suttatitleroot Rūpasutta}}
\addcontentsline{toc}{section}{\tocacronym{AN 4.65} \toctranslation{Appearance } \tocroot{Rūpasutta}}
\markboth{Appearance }{Rūpasutta}
\extramarks{AN 4.65}{AN 4.65}

“Mendicants,\marginnote{1.1} these four people are found in the world. What four? There are those whose estimation of and confidence in others is based on appearance, on eloquence, on mortification, and on principle. 

These\marginnote{1.4} are the four people found in the world. 

\begin{verse}%
Those\marginnote{2.1} who judge on appearance, \\
and those swayed by a voice, \\
are full of desire and greed; \\
those people just don’t understand. 

Not\marginnote{3.1} knowing what’s inside, \\
nor seeing what’s outside, \\
the fool shut in on every side, \\
gets carried away by a voice. 

Not\marginnote{4.1} knowing what’s inside, \\
but seeing what’s outside, \\
seeing the fruit outside, \\
they’re also carried away by a voice. 

Understanding\marginnote{5.1} what’s inside, \\
and seeing what’s outside, \\
seeing without hindrances, \\
they don’t get carried away by a voice.” 

%
\end{verse}

%
\section*{{\suttatitleacronym AN 4.66}{\suttatitletranslation Greedy }{\suttatitleroot Sarāgasutta}}
\addcontentsline{toc}{section}{\tocacronym{AN 4.66} \toctranslation{Greedy } \tocroot{Sarāgasutta}}
\markboth{Greedy }{Sarāgasutta}
\extramarks{AN 4.66}{AN 4.66}

“Mendicants,\marginnote{1.1} these four people are found in the world. What four? The greedy, the hateful, the delusional, and the conceited. 

These\marginnote{1.4} are the four people found in the world. 

\begin{verse}%
Full\marginnote{2.1} of desire for desirable things, \\
enjoying things that seem pleasant, \\
beings veiled by ignorance, \\
only tighten their bonds. 

Born\marginnote{3.1} of greed, born of hate, \\
born of delusion: the ignorant \\
make bad karma \\
that afflicts and produces pain. 

If\marginnote{4.1} you act out of these qualities, that’s what you become. \\
But men shrouded by ignorance, \\
are blind, with no eyes to see, \\
and they never imagine that this could be so.” 

%
\end{verse}

%
\section*{{\suttatitleacronym AN 4.67}{\suttatitletranslation The Snake King }{\suttatitleroot Ahirājasutta}}
\addcontentsline{toc}{section}{\tocacronym{AN 4.67} \toctranslation{The Snake King } \tocroot{Ahirājasutta}}
\markboth{The Snake King }{Ahirājasutta}
\extramarks{AN 4.67}{AN 4.67}

At\marginnote{1.1} one time the Buddha was staying near \textsanskrit{Sāvatthī} in Jeta’s Grove, \textsanskrit{Anāthapiṇḍika}’s monastery. 

Now,\marginnote{1.2} at that time a monk in \textsanskrit{Sāvatthī} passed away due to a snake bite. Then several mendicants went up to the Buddha, bowed, sat down to one side, and said to him, “Sir, a monk in \textsanskrit{Sāvatthī} has passed away due to a snake bite.” 

“Mendicants,\marginnote{2.1} that monk mustn’t have spread a mind of love to the four royal snake families. If he had, he wouldn’t have died due to a snake bite. 

What\marginnote{3.1} four? The royal snake families of \textsanskrit{Virūpakkha}, \textsanskrit{Erāpatha}, \textsanskrit{Chabyāputta}, and \textsanskrit{Kaṇhāgotamaka}. … 

Mendicants,\marginnote{4.1} I urge you to spread a mind of love to the four royal snake families, for your own safety, security, and protection. 

\begin{verse}%
I\marginnote{5.1} love the \textsanskrit{Virūpakkhas}, \\
the \textsanskrit{Erāpathas} I love, \\
I love the \textsanskrit{Chabyāputtas}, \\
the \textsanskrit{Kaṇhāgotamakas} I love. 

I\marginnote{6.1} love the footless creatures, \\
the two-footed I love, \\
I love the four-footed, \\
the many-footed I love. 

May\marginnote{7.1} the footless not harm me! \\
May I not be harmed by the two-footed! \\
May the four-footed not harm me! \\
May I not be harmed by the many-footed! 

All\marginnote{8.1} sentient beings, all living things, \\
all creatures, every one: \\
may they see only nice things, \\
may bad not come to anyone. 

The\marginnote{9.1} Buddha is immeasurable, \\
the teaching is immeasurable, \\
the \textsanskrit{Saṅgha} is immeasurable. \\
But limited are crawling things, 

snakes\marginnote{10.1} and scorpions, centipedes, \\
spiders and lizards and mice. \\
I’ve made this safeguard, I’ve made this protection: \\
go away, creatures! \\
And so I revere the Blessed One, \\
I revere the seven perfectly awakened Buddhas.” 

%
\end{verse}

%
\section*{{\suttatitleacronym AN 4.68}{\suttatitletranslation Devadatta }{\suttatitleroot Devadattasutta}}
\addcontentsline{toc}{section}{\tocacronym{AN 4.68} \toctranslation{Devadatta } \tocroot{Devadattasutta}}
\markboth{Devadatta }{Devadattasutta}
\extramarks{AN 4.68}{AN 4.68}

At\marginnote{1.1} one time the Buddha was staying near \textsanskrit{Rājagaha}, on the Vulture’s Peak Mountain, not long after Devadatta had left. There the Buddha spoke to the mendicants about Devadatta: 

“Possessions,\marginnote{1.3} honor, and popularity came to Devadatta for his own ruin and downfall. 

It’s\marginnote{2.1} like a banana tree, or a bamboo, or a reed, all of which bear fruit to their own ruin and downfall … 

It’s\marginnote{5.1} like a mule, which becomes pregnant to its own ruin and downfall. In the same way, possessions, honor, and popularity came to Devadatta for his own ruin and downfall. 

\begin{verse}%
The\marginnote{6.1} banana tree is destroyed by its own fruit, \\
as are the bamboo and the reed. \\
Honor destroys a sinner, \\
as pregnancy destroys a mule.” 

%
\end{verse}

%
\section*{{\suttatitleacronym AN 4.69}{\suttatitletranslation Effort }{\suttatitleroot Padhānasutta}}
\addcontentsline{toc}{section}{\tocacronym{AN 4.69} \toctranslation{Effort } \tocroot{Padhānasutta}}
\markboth{Effort }{Padhānasutta}
\extramarks{AN 4.69}{AN 4.69}

“Mendicants,\marginnote{1.1} there are these four efforts. What four? The efforts to restrain, to give up, to develop, and to preserve. 

And\marginnote{1.4} what, mendicants, is the effort to restrain? It’s when you generate enthusiasm, try, make an effort, exert the mind, and strive so that bad, unskillful qualities don’t arise. This is called the effort to restrain. 

And\marginnote{2.1} what, mendicants, is the effort to give up? It’s when you generate enthusiasm, try, make an effort, exert the mind, and strive so that bad, unskillful qualities are given up. This is called the effort to give up. 

And\marginnote{3.1} what, mendicants, is the effort to develop? It’s when you generate enthusiasm, try, make an effort, exert the mind, and strive so that skillful qualities arise. This is called the effort to develop. 

And\marginnote{4.1} what, mendicants, is the effort to preserve? It’s when you generate enthusiasm, try, make an effort, exert the mind, and strive so that skillful qualities that have arisen remain, are not lost, but increase, mature, and are fulfilled by development. This is called the effort to preserve. 

These\marginnote{4.4} are the four efforts. 

\begin{verse}%
Restraint\marginnote{5.1} and giving up, \\
development and preservation: \\
these are the four efforts \\
taught by the kinsman of the Sun. \\
Any mendicant who keenly applies these \\
may attain the ending of suffering.” 

%
\end{verse}

%
\section*{{\suttatitleacronym AN 4.70}{\suttatitletranslation Unprincipled }{\suttatitleroot Adhammikasutta}}
\addcontentsline{toc}{section}{\tocacronym{AN 4.70} \toctranslation{Unprincipled } \tocroot{Adhammikasutta}}
\markboth{Unprincipled }{Adhammikasutta}
\extramarks{AN 4.70}{AN 4.70}

“At\marginnote{1.1} a time when kings are unprincipled, royal officials become unprincipled. When royal officials are unprincipled, brahmins and householders become unprincipled. When brahmins and householders are unprincipled, the people of town and country become unprincipled. When the people of town and country are unprincipled, the courses of the moon and sun become erratic. … the courses of the stars and constellations … the days and nights … the months and fortnights … the seasons and years become erratic. … the blowing of the winds becomes erratic and chaotic. … the deities are angered. … the heavens don’t provide enough rain. … the crops ripen erratically. When people eat crops that have ripened erratically, they become short-lived, ugly, weak, and sickly. 

At\marginnote{2.1} a time when kings are principled, royal officials become principled. … brahmins and householders … people of town and country become principled. When the people of town and country are principled, the courses of the sun and moon become regular. … the stars and constellations … the days and nights … the months and fortnights … the seasons and years become regular. … the blowing of the winds becomes regular and orderly. … the deities are not angered … … the heavens provide plenty of rain. When the heavens provide plenty of rain, the crops ripen well. When people eat crops that have ripened well, they become long-lived, beautiful, strong, and healthy. 

\begin{verse}%
When\marginnote{3.1} cattle ford a river, \\
if the bull goes off course, \\
they all go off course, \\
because their leader is off course. 

So\marginnote{4.1} it is for humans: \\
when the one agreed on as chief \\
behaves badly, \\
what do you expect the rest to do? \\
The whole country sleeps badly, \\
when the king is unprincipled. 

When\marginnote{5.1} cattle ford a river, \\
if the bull goes straight, \\
they all go straight, \\
because their leader is straight. 

So\marginnote{6.1} it is for humans: \\
when the one agreed on as chief \\
does the right thing, \\
what do you expect the rest to do? \\
The whole country sleeps at ease, \\
when the king is just.” 

%
\end{verse}

%
\addtocontents{toc}{\let\protect\contentsline\protect\nopagecontentsline}
\chapter*{The Chapter on Guaranteed }
\addcontentsline{toc}{chapter}{\tocchapterline{The Chapter on Guaranteed }}
\addtocontents{toc}{\let\protect\contentsline\protect\oldcontentsline}

%
\section*{{\suttatitleacronym AN 4.71}{\suttatitletranslation Effort }{\suttatitleroot Padhānasutta}}
\addcontentsline{toc}{section}{\tocacronym{AN 4.71} \toctranslation{Effort } \tocroot{Padhānasutta}}
\markboth{Effort }{Padhānasutta}
\extramarks{AN 4.71}{AN 4.71}

“Mendicants,\marginnote{1.1} when a mendicant has four things their practice is guaranteed, and they have laid the groundwork for ending the defilements. What four? It’s when a mendicant is ethical, learned, energetic, and wise. When a mendicant has these four things their practice is guaranteed, and they have laid the groundwork for ending the defilements.” 

%
\section*{{\suttatitleacronym AN 4.72}{\suttatitletranslation Right View }{\suttatitleroot Sammādiṭṭhisutta}}
\addcontentsline{toc}{section}{\tocacronym{AN 4.72} \toctranslation{Right View } \tocroot{Sammādiṭṭhisutta}}
\markboth{Right View }{Sammādiṭṭhisutta}
\extramarks{AN 4.72}{AN 4.72}

“Mendicants,\marginnote{1.1} when a mendicant has four things their practice is guaranteed, and they have laid the groundwork for ending the defilements. What four? Thoughts of renunciation, good will, and harmlessness; and right view. When a mendicant has these four things their practice is guaranteed, and they have laid the groundwork for ending the defilements.” 

%
\section*{{\suttatitleacronym AN 4.73}{\suttatitletranslation A Good Person }{\suttatitleroot Sappurisasutta}}
\addcontentsline{toc}{section}{\tocacronym{AN 4.73} \toctranslation{A Good Person } \tocroot{Sappurisasutta}}
\markboth{A Good Person }{Sappurisasutta}
\extramarks{AN 4.73}{AN 4.73}

“Mendicants,\marginnote{1.1} a bad person can be known by four qualities. What four? 

To\marginnote{1.3} start with, a bad person speaks ill of another even when not asked, let alone when asked. But when led on by questions they speak ill of another in full detail, not leaving anything out. That’s how to know that this is a bad person. 

Furthermore,\marginnote{2.1} a bad person doesn’t speak well of another even when asked, let alone when not asked. But when led on by questions they speak well of another without giving the full details, leaving many things out. That’s how to know that this is a bad person. 

Furthermore,\marginnote{3.1} a bad person doesn’t speak ill of themselves even when asked, let alone when not asked. But when led on by questions they speak ill of themselves without giving the full details, leaving many things out. That’s how to know that this is a bad person. 

Furthermore,\marginnote{4.1} a bad person speaks well of themselves even when not asked, let alone when asked. But when led on by questions they speak well of themselves in full detail, not leaving anything out. That’s how to know that this is a bad person. A bad person can be known by these four qualities. 

A\marginnote{5.1} good person can be known by four qualities. What four? 

To\marginnote{5.3} start with, a good person doesn’t speak ill of another even when asked, let alone when not asked. But when led on by questions they speak ill of another without giving the full details, leaving many things out. That’s how to know that this is a good person. 

Furthermore,\marginnote{6.1} a good person speaks well of another even when not asked, let alone when asked. But when led on by questions they speak well of another in full detail, not leaving anything out. That’s how to know that this is a good person. 

Furthermore,\marginnote{7.1} a good person speaks ill of themselves even when not asked, let alone when asked. But when led on by questions they speak ill of themselves in full detail, not leaving anything out. That’s how to know that this is a good person. 

Furthermore,\marginnote{8.1} a good person doesn’t speak well of themselves even when asked, let alone when not asked. But when led on by questions they speak well of themselves without giving the full details, leaving many things out. That’s how to know that this is a good person. A good person can be known by these four qualities. 

It’s\marginnote{9.1} like a bride on the day or night she’s first brought to her husband’s home. Right away she sets up a keen sense of conscience and prudence for her mother and father in law, her husband, and even the bondservants, workers, and staff. But after some time, because of living together and familiarity, she’ll even say to her mother and father in law, or to her husband: ‘Go away! What would you know?’ In the same way, on the day or night a mendicant first goes forth from the lay life to homelessness, right away they set up a keen sense of conscience and prudence for the monks, nuns, laymen, and laywomen, and even the monastery workers and novices. But after some time, because of living together and familiarity, they’ll even say to their teacher or mentor: ‘Go away! What would you know?’ 

So\marginnote{9.7} you should train like this: ‘We will live with hearts like that of a newly wedded bride.’ That’s how you should train.” 

%
\section*{{\suttatitleacronym AN 4.74}{\suttatitletranslation Best (1st) }{\suttatitleroot Paṭhamaaggasutta}}
\addcontentsline{toc}{section}{\tocacronym{AN 4.74} \toctranslation{Best (1st) } \tocroot{Paṭhamaaggasutta}}
\markboth{Best (1st) }{Paṭhamaaggasutta}
\extramarks{AN 4.74}{AN 4.74}

“Mendicants,\marginnote{1.1} these four things are the best. What four? The best ethics, immersion, wisdom, and freedom. These are the four things that are the best.” 

%
\section*{{\suttatitleacronym AN 4.75}{\suttatitletranslation Best (2nd) }{\suttatitleroot Dutiyaaggasutta}}
\addcontentsline{toc}{section}{\tocacronym{AN 4.75} \toctranslation{Best (2nd) } \tocroot{Dutiyaaggasutta}}
\markboth{Best (2nd) }{Dutiyaaggasutta}
\extramarks{AN 4.75}{AN 4.75}

“Mendicants,\marginnote{1.1} these four things are the best. What four? The best form, feeling, perception, and state of existence. These are the four things that are the best.” 

%
\section*{{\suttatitleacronym AN 4.76}{\suttatitletranslation At Kusinārā }{\suttatitleroot Kusinārasutta}}
\addcontentsline{toc}{section}{\tocacronym{AN 4.76} \toctranslation{At Kusinārā } \tocroot{Kusinārasutta}}
\markboth{At Kusinārā }{Kusinārasutta}
\extramarks{AN 4.76}{AN 4.76}

At\marginnote{1.1} one time the Buddha was staying between a pair of sal trees in the sal forest of the Mallas at Upavattana near \textsanskrit{Kusinārā} at the time of his final extinguishment. There the Buddha addressed the mendicants, “Mendicants!” 

“Venerable\marginnote{1.4} sir,” they replied. The Buddha said this: 

“Perhaps\marginnote{2.1} even a single mendicant has doubt or uncertainty regarding the Buddha, the teaching, the \textsanskrit{Saṅgha}, the path, or the practice. So ask, mendicants! Don’t regret it later, thinking: ‘We were in the Teacher’s presence and we weren’t able to ask the Buddha a question.’” When this was said, the mendicants kept silent. 

For\marginnote{3.1} a second time the Buddha addressed the mendicants: … For a second time, the mendicants kept silent. 

For\marginnote{4.1} a third time the Buddha addressed the mendicants: … For a third time, the mendicants kept silent. 

Then\marginnote{5.1} the Buddha said to the mendicants: 

“Mendicants,\marginnote{5.2} perhaps you don’t ask out of respect for the Teacher. So let a friend tell a friend.” When this was said, the mendicants kept silent. Then Venerable Ānanda said to the Buddha: 

“It’s\marginnote{5.5} incredible, sir, it’s amazing! I am quite confident that there’s not even a single mendicant in this \textsanskrit{Saṅgha} who has doubt or uncertainty regarding the Buddha, the teaching, the \textsanskrit{Saṅgha}, the path, or the practice.” 

“Ānanda,\marginnote{6.1} you speak from faith. But the Realized One knows that there’s not even a single mendicant in this \textsanskrit{Saṅgha} who has doubt or uncertainty regarding the Buddha, the teaching, the \textsanskrit{Saṅgha}, the path, or the practice. Even the last of these five hundred mendicants is a stream-enterer, not liable to be reborn in the underworld, bound for awakening.” 

%
\section*{{\suttatitleacronym AN 4.77}{\suttatitletranslation Unthinkable }{\suttatitleroot Acinteyyasutta}}
\addcontentsline{toc}{section}{\tocacronym{AN 4.77} \toctranslation{Unthinkable } \tocroot{Acinteyyasutta}}
\markboth{Unthinkable }{Acinteyyasutta}
\extramarks{AN 4.77}{AN 4.77}

“Mendicants,\marginnote{1.1} these four things are unthinkable. They should not be thought about, and anyone who tries to think about them will go mad or get frustrated. What four? 

The\marginnote{1.3} scope of the Buddhas … 

The\marginnote{1.5} scope of one in absorption … 

The\marginnote{1.7} results of deeds … 

Speculation\marginnote{1.9} about the world … 

These\marginnote{1.11} are the four unthinkable things. They should not be thought about, and anyone who tries to think about them will go mad or get frustrated.” 

%
\section*{{\suttatitleacronym AN 4.78}{\suttatitletranslation A Religious Donation }{\suttatitleroot Dakkhiṇasutta}}
\addcontentsline{toc}{section}{\tocacronym{AN 4.78} \toctranslation{A Religious Donation } \tocroot{Dakkhiṇasutta}}
\markboth{A Religious Donation }{Dakkhiṇasutta}
\extramarks{AN 4.78}{AN 4.78}

“Mendicants,\marginnote{1.1} there are these four ways of purifying a religious donation. What four? There’s a religious donation that’s purified by the giver, not the recipient. There’s a religious donation that’s purified by the recipient, not the giver. There’s a religious donation that’s purified by neither the giver nor the recipient. There’s a religious donation that’s purified by both the giver and the recipient. 

And\marginnote{2.1} how is a religious donation purified by the giver, not the recipient? It’s when the giver is ethical, of good character, but the recipient is unethical, of bad character. 

And\marginnote{3.1} how is a religious donation purified by the recipient, not the giver? It’s when the giver is unethical, of bad character, but the recipient is ethical, of good character. 

And\marginnote{4.1} how is a religious donation purified by neither the giver nor the recipient? It’s when both the giver and the recipient are unethical, of bad character. 

And\marginnote{5.1} how is a religious donation purified by both the giver and the recipient? It’s when both the giver and the recipient are ethical, of good character. 

These\marginnote{5.4} are the four ways of purifying a religious donation.” 

%
\section*{{\suttatitleacronym AN 4.79}{\suttatitletranslation Business }{\suttatitleroot Vaṇijjasutta}}
\addcontentsline{toc}{section}{\tocacronym{AN 4.79} \toctranslation{Business } \tocroot{Vaṇijjasutta}}
\markboth{Business }{Vaṇijjasutta}
\extramarks{AN 4.79}{AN 4.79}

Then\marginnote{1.1} Venerable \textsanskrit{Sāriputta} went up to the Buddha, bowed, sat down to one side, and said to him: 

“Sir,\marginnote{1.2} what is the cause, what is the reason why for different people the same kind of business undertaking might fail, while another doesn’t meet expectations, another meets expectations, and another exceeds expectations?” 

“\textsanskrit{Sāriputta},\marginnote{2.1} take a case where someone goes to an ascetic or brahmin and invites them to ask for what they need. But they fail to give what’s requested. When they’ve passed away from that life, if they’re reborn in this state of existence, whatever business they undertake fails. 

Take\marginnote{3.1} a case where someone goes to an ascetic or brahmin and invites them to ask for what they need. They give what’s requested, but don’t meet expectations. When they’ve passed away from that life, if they’re reborn in this state of existence, whatever business they undertake doesn’t meet expectations. 

Take\marginnote{4.1} a case where someone goes to an ascetic or brahmin and invites them to ask for what they need. They give what’s requested, meeting expectations. When they’ve passed away from that life, if they’re reborn in this state of existence, whatever business they undertake meets expectations. 

Take\marginnote{5.1} a case where someone goes to an ascetic or brahmin and invites them to ask for what they need. They give what’s requested, exceeding expectations. When they’ve passed away from that life, if they’re reborn in this state of existence, whatever business they undertake exceeds expectations. 

This\marginnote{6.1} is the cause, this is the reason why for different people the same kind of business enterprise might fail, while another doesn’t meet expectations, another meets expectations, and another exceeds expectations.” 

%
\section*{{\suttatitleacronym AN 4.80}{\suttatitletranslation Persia }{\suttatitleroot Kambojasutta}}
\addcontentsline{toc}{section}{\tocacronym{AN 4.80} \toctranslation{Persia } \tocroot{Kambojasutta}}
\markboth{Persia }{Kambojasutta}
\extramarks{AN 4.80}{AN 4.80}

At\marginnote{1.1} one time the Buddha was staying near Kosambi, in Ghosita’s Monastery. Then Venerable Ānanda went up to the Buddha, bowed, sat down to one side, and said to him: 

“Sir,\marginnote{2.1} what is the cause, what is the reason why females don’t attend council meetings, work for a living, or travel to Persia?” 

“Ānanda,\marginnote{2.2} females are irritable, jealous, stingy, and unintelligent. This is the cause, this is the reason why females don’t attend council meetings, work for a living, or travel to Persia.” 

%
\addtocontents{toc}{\let\protect\contentsline\protect\nopagecontentsline}
\chapter*{The Chapter on Confirmed }
\addcontentsline{toc}{chapter}{\tocchapterline{The Chapter on Confirmed }}
\addtocontents{toc}{\let\protect\contentsline\protect\oldcontentsline}

%
\section*{{\suttatitleacronym AN 4.81}{\suttatitletranslation Killing Living Creatures }{\suttatitleroot Pāṇātipātasutta}}
\addcontentsline{toc}{section}{\tocacronym{AN 4.81} \toctranslation{Killing Living Creatures } \tocroot{Pāṇātipātasutta}}
\markboth{Killing Living Creatures }{Pāṇātipātasutta}
\extramarks{AN 4.81}{AN 4.81}

“Mendicants,\marginnote{1.1} someone with four qualities is cast down to hell. What four? They kill living creatures, steal, commit sexual misconduct, and lie. Someone with these four qualities is cast down to hell. 

Someone\marginnote{2.1} with four qualities is raised up to heaven. What four? They don’t kill living creatures, steal, commit sexual misconduct, or lie. Someone with these four qualities is raised up to heaven.” 

%
\section*{{\suttatitleacronym AN 4.82}{\suttatitletranslation Lying }{\suttatitleroot Musāvādasutta}}
\addcontentsline{toc}{section}{\tocacronym{AN 4.82} \toctranslation{Lying } \tocroot{Musāvādasutta}}
\markboth{Lying }{Musāvādasutta}
\extramarks{AN 4.82}{AN 4.82}

“Mendicants,\marginnote{1.1} someone with four qualities is cast down to hell. What four? They use speech that’s false, divisive, harsh, or nonsensical. Someone with these four qualities is cast down to hell. 

Someone\marginnote{2.1} with four qualities is raised up to heaven. What four? They don’t use speech that’s false, divisive, harsh, or nonsensical. Someone with these four qualities is raised up to heaven.” 

%
\section*{{\suttatitleacronym AN 4.83}{\suttatitletranslation Where Criticism Takes You }{\suttatitleroot Avaṇṇārahasutta}}
\addcontentsline{toc}{section}{\tocacronym{AN 4.83} \toctranslation{Where Criticism Takes You } \tocroot{Avaṇṇārahasutta}}
\markboth{Where Criticism Takes You }{Avaṇṇārahasutta}
\extramarks{AN 4.83}{AN 4.83}

“Mendicants,\marginnote{1.1} someone with four qualities is cast down to hell. What four? Without examining or scrutinizing, they praise those deserving of criticism, and they criticize those deserving of praise. They arouse faith in things that are dubious, and they don’t arouse faith in things that are inspiring. Someone with these four qualities is cast down to hell. 

Someone\marginnote{2.1} with four qualities is raised up to heaven. What four? After examining and scrutinizing, they criticize those deserving of criticism, and they praise those deserving of praise. They don’t arouse faith in things that are dubious, and they do arouse faith in things that are inspiring. Someone with these four qualities is raised up to heaven.” 

%
\section*{{\suttatitleacronym AN 4.84}{\suttatitletranslation Valuing Anger }{\suttatitleroot Kodhagarusutta}}
\addcontentsline{toc}{section}{\tocacronym{AN 4.84} \toctranslation{Valuing Anger } \tocroot{Kodhagarusutta}}
\markboth{Valuing Anger }{Kodhagarusutta}
\extramarks{AN 4.84}{AN 4.84}

“Mendicants,\marginnote{1.1} someone with four qualities is cast down to hell. What four? They value anger, or denigration, or material possessions, or honor rather than the true teaching. Someone with these four qualities is cast down to hell. 

Someone\marginnote{2.1} with four qualities is raised up to heaven. What four? They value the true teaching rather than anger, or denigration, or material possessions, or honor. Someone with these four qualities is raised up to heaven.” 

%
\section*{{\suttatitleacronym AN 4.85}{\suttatitletranslation From Darkness to Darkness }{\suttatitleroot Tamotamasutta}}
\addcontentsline{toc}{section}{\tocacronym{AN 4.85} \toctranslation{From Darkness to Darkness } \tocroot{Tamotamasutta}}
\markboth{From Darkness to Darkness }{Tamotamasutta}
\extramarks{AN 4.85}{AN 4.85}

“Mendicants,\marginnote{1.1} these four people are found in the world. What four? 

\begin{enumerate}%
\item The dark bound for darkness, %
\item the dark bound for light, %
\item the light bound for darkness, and %
\item the light bound for light. %
\end{enumerate}

And\marginnote{2.1} how is a person dark and bound for darkness? It’s when someone is reborn in a low family—a family of outcastes, bamboo-workers, hunters, chariot-makers, or waste-collectors—poor, with little to eat or drink, where life is tough, and food and shelter are hard to find. And they’re ugly, unsightly, deformed, chronically ill—one-eyed, crippled, lame, or half-paralyzed. They don’t get to have food, drink, clothes, and vehicles; garlands, fragrance, and makeup; or bed, house, and lighting. And they do bad things by way of body, speech, and mind. When their body breaks up, after death, they’re reborn in a place of loss, a bad place, the underworld, hell. That’s how a person is dark and bound for darkness. 

And\marginnote{3.1} how is a person dark and bound for light? It’s when some person is reborn in a low family … But they do good things by way of body, speech, and mind. When their body breaks up, after death, they’re reborn in a good place, a heavenly realm. That’s how a person is dark and bound for light. 

And\marginnote{4.1} how is a person light and bound for darkness? It’s when some person is reborn in an eminent family—a well-to-do family of aristocrats, brahmins, or householders—rich, affluent, and wealthy, with lots of gold and silver, lots of property and assets, and lots of money and grain. And they’re attractive, good-looking, lovely, of surpassing beauty. They get to have food, drink, clothes, and vehicles; garlands, fragrance, and makeup; and bed, house, and lighting. But they do bad things by way of body, speech, and mind. When their body breaks up, after death, they’re reborn in a place of loss, a bad place, the underworld, hell. That’s how a person is light and bound for darkness. 

And\marginnote{5.1} how is a person light and bound for light? It’s when some person is reborn in an eminent family … And they do good things by way of body, speech, and mind. When their body breaks up, after death, they’re reborn in a good place, a heavenly realm. That’s how a person is light and bound for light. 

These\marginnote{5.7} are the four people found in the world.” 

%
\section*{{\suttatitleacronym AN 4.86}{\suttatitletranslation Sunk Low }{\suttatitleroot Oṇatoṇatasutta}}
\addcontentsline{toc}{section}{\tocacronym{AN 4.86} \toctranslation{Sunk Low } \tocroot{Oṇatoṇatasutta}}
\markboth{Sunk Low }{Oṇatoṇatasutta}
\extramarks{AN 4.86}{AN 4.86}

“These\marginnote{1.1} four people are found in the world. What four? 

\begin{enumerate}%
\item One sunk low who sinks lower, %
\item one sunk low who rises high, %
\item one risen high who sinks low, and %
\item one risen high who rises higher. %
\end{enumerate}

…\marginnote{1.7} These are the four people found in the world.” 

%
\section*{{\suttatitleacronym AN 4.87}{\suttatitletranslation The Son }{\suttatitleroot Puttasutta}}
\addcontentsline{toc}{section}{\tocacronym{AN 4.87} \toctranslation{The Son } \tocroot{Puttasutta}}
\markboth{The Son }{Puttasutta}
\extramarks{AN 4.87}{AN 4.87}

“Mendicants,\marginnote{1.1} these four people are found in the world. What four? The confirmed ascetic, the white lotus ascetic, the pink lotus ascetic, and the exquisite ascetic of ascetics. 

And\marginnote{2.1} how is a person a confirmed ascetic? It’s when a mendicant is a practicing trainee, who lives aspiring to the supreme sanctuary. It’s like the eldest son of an anointed aristocratic king. He has not yet been anointed, but is eligible, and has been confirmed in the succession. In the same way, a mendicant is a practicing trainee, who lives aspiring to the supreme sanctuary. That’s how a person is a confirmed ascetic. 

And\marginnote{3.1} how is a person a white lotus ascetic? It’s when a mendicant realizes the undefiled freedom of heart and freedom by wisdom in this very life. And they live having realized it with their own insight due to the ending of defilements. But they don’t have direct meditative experience of the eight liberations. That’s how a person is a white lotus ascetic. 

And\marginnote{4.1} how is a person a pink lotus ascetic? It’s when a mendicant realizes the undefiled freedom of heart and freedom by wisdom in this very life. … And they have direct meditative experience of the eight liberations. That’s how a person is a pink lotus ascetic. 

And\marginnote{5.1} how is a person an exquisite ascetic of ascetics? It’s when a mendicant usually uses only what they’ve been invited to accept—robes, almsfood, lodgings, and medicines and supplies for the sick—rarely using them without invitation. When living with other spiritual practitioners, they usually treat them agreeably by way of body, speech, and mind, and rarely disagreeably. And they usually present them with agreeable things, rarely with disagreeable ones. They’re healthy, so the various unpleasant feelings—stemming from disorders of bile, phlegm, wind, or their conjunction; or caused by change in weather, by not taking care of yourself, by overexertion, or as the result of past deeds—usually don’t come up. They get the four absorptions—blissful meditations in the present life that belong to the higher mind—when they want, without trouble or difficulty. And they realize the undefiled freedom of heart and freedom by wisdom in this very life. … That’s how a person is an exquisite ascetic of ascetics. 

And\marginnote{6.1} if anyone should be rightly called an exquisite ascetic of ascetics, it’s me. For I usually use only what I’ve been invited to accept … When living with other spiritual practitioners, I usually treat them agreeably … I’m healthy … I get the four absorptions when I want, without trouble or difficulty. And I’ve realized the undefiled freedom of heart and freedom by wisdom in this very life. … So if anyone should be rightly called an exquisite ascetic of ascetics, it’s me. 

These\marginnote{7.2} are the four people found in the world.” 

%
\section*{{\suttatitleacronym AN 4.88}{\suttatitletranslation Fetters }{\suttatitleroot Saṁyojanasutta}}
\addcontentsline{toc}{section}{\tocacronym{AN 4.88} \toctranslation{Fetters } \tocroot{Saṁyojanasutta}}
\markboth{Fetters }{Saṁyojanasutta}
\extramarks{AN 4.88}{AN 4.88}

“Mendicants,\marginnote{1.1} these four people are found in the world. What four? The confirmed ascetic, the white lotus ascetic, the pink lotus ascetic, and the exquisite ascetic of ascetics. 

And\marginnote{2.1} how is a person a confirmed ascetic? It’s when a mendicant—with the ending of three fetters—is a stream-enterer, not liable to be reborn in the underworld, bound for awakening. That’s how a person is a confirmed ascetic. 

And\marginnote{3.1} how is a person a white lotus ascetic? It’s when a mendicant—with the ending of three fetters, and the weakening of greed, hate, and delusion—is a once-returner. They come back to this world once only, then make an end of suffering. That’s how a person is a white lotus ascetic. 

And\marginnote{4.1} how is a person a pink lotus ascetic? It’s when a mendicant—with the ending of the five lower fetters—is reborn spontaneously. They’re extinguished there, and are not liable to return from that world. That’s how a person is a pink lotus ascetic. 

And\marginnote{5.1} how is a person an exquisite ascetic of ascetics? It’s when a mendicant realizes the undefiled freedom of heart and freedom by wisdom in this very life. And they live having realized it with their own insight due to the ending of defilements. That’s how a person is an exquisite ascetic of ascetics. 

These\marginnote{5.4} are the four people found in the world.” 

%
\section*{{\suttatitleacronym AN 4.89}{\suttatitletranslation Right View }{\suttatitleroot Sammādiṭṭhisutta}}
\addcontentsline{toc}{section}{\tocacronym{AN 4.89} \toctranslation{Right View } \tocroot{Sammādiṭṭhisutta}}
\markboth{Right View }{Sammādiṭṭhisutta}
\extramarks{AN 4.89}{AN 4.89}

“Mendicants,\marginnote{1.1} these four people are found in the world. What four? The confirmed ascetic, the white lotus ascetic, the pink lotus ascetic, and the exquisite ascetic of ascetics. 

And\marginnote{2.1} how is a person a confirmed ascetic? It’s when a mendicant has right view, right thought, right speech, right action, right livelihood, right effort, right mindfulness, and right immersion. That’s how a person is a confirmed ascetic. 

And\marginnote{3.1} how is a person a white lotus ascetic? It’s when they have right view, right thought, right speech, right action, right livelihood, right effort, right mindfulness, right immersion, right knowledge, and right freedom. But they don’t have direct meditative experience of the eight liberations. That’s how a person is a white lotus ascetic. 

And\marginnote{4.1} how is a person a pink lotus ascetic? It’s when they have right view … and right freedom. And they do have direct meditative experience of the eight liberations. That’s how a person is a pink lotus ascetic. 

And\marginnote{5.1} how is a person an exquisite ascetic of ascetics? It’s when a mendicant usually uses only what they’ve been invited to accept … And if anyone should be rightly called an exquisite ascetic of ascetics, it’s me. 

These\marginnote{5.3} are the four people found in the world.” 

%
\section*{{\suttatitleacronym AN 4.90}{\suttatitletranslation Aggregates }{\suttatitleroot Khandhasutta}}
\addcontentsline{toc}{section}{\tocacronym{AN 4.90} \toctranslation{Aggregates } \tocroot{Khandhasutta}}
\markboth{Aggregates }{Khandhasutta}
\extramarks{AN 4.90}{AN 4.90}

“Mendicants,\marginnote{1.1} these four people are found in the world. What four? The confirmed ascetic, the white lotus ascetic, the pink lotus ascetic, and the exquisite ascetic of ascetics. 

And\marginnote{2.1} how is a person a confirmed ascetic? It’s when a mendicant is a trainee who hasn’t achieved their heart’s desire, but lives aspiring to the supreme sanctuary. That’s how a person is a confirmed ascetic. 

And\marginnote{3.1} how is a person a white lotus ascetic? It’s when a mendicant meditates observing rise and fall in the five grasping aggregates. ‘Such is form, such is the origin of form, such is the ending of form. Such is feeling … Such is perception … Such are choices … Such is consciousness, such is the origin of consciousness, such is the ending of consciousness.’ But they don’t have direct meditative experience of the eight liberations. That’s how a person is a white lotus ascetic. 

And\marginnote{4.1} how is a person a pink lotus ascetic? It’s when a mendicant meditates observing rise and fall in the five grasping aggregates. ‘Such is form, such is the origin of form, such is the ending of form. Such is feeling … Such is perception … Such are choices … Such is consciousness, such is the origin of consciousness, such is the ending of consciousness.’ And they have direct meditative experience of the eight liberations. That’s how a person is a pink lotus ascetic. 

And\marginnote{5.1} how is a person an exquisite ascetic of ascetics? It’s when a mendicant usually uses only what they’ve been invited to accept … And if anyone should be rightly called an exquisite ascetic of ascetics, it’s me. 

These\marginnote{5.3} are the four people found in the world.” 

%
\addtocontents{toc}{\let\protect\contentsline\protect\nopagecontentsline}
\chapter*{The Chapter on Demons }
\addcontentsline{toc}{chapter}{\tocchapterline{The Chapter on Demons }}
\addtocontents{toc}{\let\protect\contentsline\protect\oldcontentsline}

%
\section*{{\suttatitleacronym AN 4.91}{\suttatitletranslation Demons }{\suttatitleroot Asurasutta}}
\addcontentsline{toc}{section}{\tocacronym{AN 4.91} \toctranslation{Demons } \tocroot{Asurasutta}}
\markboth{Demons }{Asurasutta}
\extramarks{AN 4.91}{AN 4.91}

“Mendicants,\marginnote{1.1} these four people are found in the world. What four? 

\begin{enumerate}%
\item A demon surrounded by demons, %
\item a demon surrounded by gods, %
\item a god surrounded by demons, and %
\item a god surrounded by gods. %
\end{enumerate}

And\marginnote{2.1} how is a person a demon surrounded by demons? It’s when a person is unethical, of bad character, and their followers are the same. That’s how a person is a demon surrounded by demons. 

And\marginnote{3.1} how is a person a demon surrounded by gods? It’s when a person is unethical, of bad character, but their followers are ethical, of good character. That’s how a person is a demon surrounded by gods. 

And\marginnote{4.1} how is a person a god surrounded by demons? It’s when a person is ethical, of good character, but their followers are unethical, of bad character. That’s how a person is a god surrounded by demons. 

And\marginnote{5.1} how is a person a god surrounded by gods? It’s when a person is ethical, of good character, and their followers are the same. That’s how a person is a god surrounded by gods. 

These\marginnote{5.4} are the four people found in the world.” 

%
\section*{{\suttatitleacronym AN 4.92}{\suttatitletranslation Immersion (1st) }{\suttatitleroot Paṭhamasamādhisutta}}
\addcontentsline{toc}{section}{\tocacronym{AN 4.92} \toctranslation{Immersion (1st) } \tocroot{Paṭhamasamādhisutta}}
\markboth{Immersion (1st) }{Paṭhamasamādhisutta}
\extramarks{AN 4.92}{AN 4.92}

“Mendicants,\marginnote{1.1} these four people are found in the world. What four? 

One\marginnote{1.3} person has internal serenity of heart, but not the higher wisdom of discernment of principles. 

One\marginnote{1.4} person has the higher wisdom of discernment of principles, but not internal serenity of heart. 

One\marginnote{1.5} person has neither internal serenity of heart, nor the higher wisdom of discernment of principles. 

One\marginnote{1.6} person has both internal serenity of heart, and the higher wisdom of discernment of principles. 

These\marginnote{1.7} are the four people found in the world.” 

%
\section*{{\suttatitleacronym AN 4.93}{\suttatitletranslation Immersion (2nd) }{\suttatitleroot Dutiyasamādhisutta}}
\addcontentsline{toc}{section}{\tocacronym{AN 4.93} \toctranslation{Immersion (2nd) } \tocroot{Dutiyasamādhisutta}}
\markboth{Immersion (2nd) }{Dutiyasamādhisutta}
\extramarks{AN 4.93}{AN 4.93}

“Mendicants,\marginnote{1.1} these four people are found in the world. What four? 

\begin{enumerate}%
\item One person has internal serenity of heart, but not the higher wisdom of discernment of principles. %
\item One person has the higher wisdom of discernment of principles, but not internal serenity of heart. %
\item One person has neither internal serenity of heart, nor the higher wisdom of discernment of principles. %
\item One person has both internal serenity of heart, and the higher wisdom of discernment of principles. %
\end{enumerate}

As\marginnote{2.1} for the person who has serenity but not discernment: grounded on serenity, they should practice meditation to get discernment. After some time they have both serenity and discernment. 

As\marginnote{3.1} for the person who has discernment but not serenity: grounded on discernment, they should practice meditation to get serenity. After some time they have both discernment and serenity. 

As\marginnote{4.1} for the person who has neither serenity nor discernment: in order to get those skillful qualities, they should apply intense enthusiasm, effort, zeal, vigor, perseverance, mindfulness, and situational awareness. Suppose your clothes or head were on fire. In order to extinguish it, you’d apply intense enthusiasm, effort, zeal, vigor, perseverance, mindfulness, and situational awareness. In the same way, in order to get those skillful qualities, that person should apply intense enthusiasm … After some time they have both serenity and discernment. 

As\marginnote{5.1} for the person who has both serenity and discernment: grounded on those skillful qualities, they should practice meditation further to end the defilements. 

These\marginnote{5.2} are the four people found in the world.” 

%
\section*{{\suttatitleacronym AN 4.94}{\suttatitletranslation Immersion (3rd) }{\suttatitleroot Tatiyasamādhisutta}}
\addcontentsline{toc}{section}{\tocacronym{AN 4.94} \toctranslation{Immersion (3rd) } \tocroot{Tatiyasamādhisutta}}
\markboth{Immersion (3rd) }{Tatiyasamādhisutta}
\extramarks{AN 4.94}{AN 4.94}

“Mendicants,\marginnote{1.1} these four people are found in the world. What four? 

One\marginnote{1.3} person has internal serenity of heart, but not the higher wisdom of discernment of principles. One person has the higher wisdom of discernment of principles, but not internal serenity of heart. One person has neither internal serenity of heart, nor the higher wisdom of discernment of principles. One person has both internal serenity of heart, and the higher wisdom of discernment of principles. 

As\marginnote{2.1} for the person who has serenity but not discernment: they should approach someone who has discernment and ask: ‘Reverend, how should conditions be seen? How should they be comprehended? How should they be discerned?’ That person would answer from their own experience: ‘This is how conditions should be seen, comprehended, and discerned.’ After some time they have both serenity and discernment. 

As\marginnote{3.1} for the person who has discernment but not serenity: they should approach someone who has serenity and ask: ‘Reverend, how should the mind be stilled? How should it be settled? How should it be unified? How should it be immersed in \textsanskrit{samādhi}?’ That person would answer from their own experience: ‘Reverend, this is how the mind should be stilled, settled, unified, and immersed in \textsanskrit{samādhi}.’ After some time they have both discernment and serenity. 

As\marginnote{4.1} for the person who has neither serenity nor discernment: they should approach someone who has serenity and discernment and ask: ‘Reverend, how should the mind be stilled? How should it be settled? How should it be unified? How should it be immersed in \textsanskrit{samādhi}?’ How should conditions be seen? How should they be comprehended? How should they be discerned?’ That person would answer as they’ve seen and known: ‘Reverend, this is how the mind should be stilled, settled, unified, and immersed in \textsanskrit{samādhi}. And this is how conditions should be seen, comprehended, and discerned.’ After some time they have both serenity and discernment. 

As\marginnote{5.1} for the person who has both serenity and discernment: grounded on those skillful qualities, they should practice meditation further to end the defilements. 

These\marginnote{5.2} are the four people found in the world.” 

%
\section*{{\suttatitleacronym AN 4.95}{\suttatitletranslation A Firebrand }{\suttatitleroot Chavālātasutta}}
\addcontentsline{toc}{section}{\tocacronym{AN 4.95} \toctranslation{A Firebrand } \tocroot{Chavālātasutta}}
\markboth{A Firebrand }{Chavālātasutta}
\extramarks{AN 4.95}{AN 4.95}

“Mendicants,\marginnote{1.1} these four people are found in the world. What four? 

\begin{enumerate}%
\item One who practices to benefit neither themselves nor others; %
\item one who practices to benefit others, but not themselves; %
\item one who practices to benefit themselves, but not others; and %
\item one who practices to benefit both themselves and others. %
\end{enumerate}

Suppose\marginnote{2.1} there was a firebrand for lighting a funeral pyre, burning at both ends, and smeared with dung in the middle. It couldn’t be used as timber either in the village or the wilderness. The person who practices to benefit neither themselves nor others is like this, I say. 

The\marginnote{3.1} person who practices to benefit others, but not themselves, is better than that. The person who practices to benefit themselves, but not others, is better than both of those. But the person who practices to benefit both themselves and others is the foremost, best, chief, highest, and finest of the four. 

From\marginnote{4.1} a cow comes milk, from milk comes curds, from curds come butter, from butter comes ghee, and from ghee comes cream of ghee. And the cream of ghee is said to be the best of these. In the same way, the person who practices to benefit both themselves and others is the foremost, best, chief, highest, and finest of the four. 

These\marginnote{4.3} are the four people found in the world.” 

%
\section*{{\suttatitleacronym AN 4.96}{\suttatitletranslation Removing Greed }{\suttatitleroot Rāgavinayasutta}}
\addcontentsline{toc}{section}{\tocacronym{AN 4.96} \toctranslation{Removing Greed } \tocroot{Rāgavinayasutta}}
\markboth{Removing Greed }{Rāgavinayasutta}
\extramarks{AN 4.96}{AN 4.96}

“Mendicants,\marginnote{1.1} these four people are found in the world. What four? 

\begin{enumerate}%
\item One who practices to benefit themselves, but not others; %
\item one who practices to benefit others, but not themselves; %
\item one who practices to benefit neither themselves nor others; and %
\item one who practices to benefit both themselves and others. %
\end{enumerate}

And\marginnote{2.1} how does a person practice to benefit themselves, but not others? It’s when a person practices to remove their own greed, hate, and delusion, but doesn’t encourage others to do the same. That’s how a person practices to benefit themselves, but not others. 

And\marginnote{3.1} how does a person practice to benefit others, but not themselves? It’s when a person doesn’t practice to remove their own greed, hate, and delusion, but encourages others to remove theirs. That’s how a person practices to benefit others, but not themselves. 

And\marginnote{4.1} how does a person practice to benefit neither themselves nor others? It’s when a person doesn’t practice to remove their own greed, hate, and delusion, nor do they encourage others to remove theirs. That’s how a person practices to benefit neither themselves nor others. 

And\marginnote{5.1} how does a person practice to benefit both themselves and others? It’s when a person practices to remove their own greed, hate, and delusion, and encourages others to remove theirs. That’s how a person practices to benefit both themselves and others. 

These\marginnote{5.6} are the four people found in the world.” 

%
\section*{{\suttatitleacronym AN 4.97}{\suttatitletranslation Quick-witted }{\suttatitleroot Khippanisantisutta}}
\addcontentsline{toc}{section}{\tocacronym{AN 4.97} \toctranslation{Quick-witted } \tocroot{Khippanisantisutta}}
\markboth{Quick-witted }{Khippanisantisutta}
\extramarks{AN 4.97}{AN 4.97}

“Mendicants,\marginnote{1.1} these four people are found in the world. What four? 

\begin{enumerate}%
\item One who practices to benefit themselves, but not others; %
\item one who practices to benefit others, but not themselves; %
\item one who practices to benefit neither themselves nor others; and %
\item one who practices to benefit both themselves and others. %
\end{enumerate}

And\marginnote{2.1} how does a person practice to benefit themselves, but not others? It’s when a person is quick-witted when it comes to skillful teachings. They readily memorize the teachings they’ve heard. They examine the meaning of teachings they’ve memorized. Understanding the meaning and the teaching, they practice accordingly. But they’re not a good speaker. Their voice isn’t polished, clear, articulate, and doesn’t express the meaning. They don’t educate, encourage, fire up, and inspire their spiritual companions. That’s how a person practices to benefit themselves, but not others. 

And\marginnote{3.1} how does a person practice to benefit others, but not themselves? It’s when a person is not quick-witted when it comes to skillful teachings. … But they’re a good speaker. … That’s how a person practices to benefit others, but not themselves. 

And\marginnote{4.1} how does a person practice to benefit neither themselves nor others? It’s when a person is not quick-witted when it comes to skillful teachings. … Nor are they a good speaker. … That’s how a person practices to benefit neither themselves nor others. 

And\marginnote{5.1} how does a person practice to benefit both themselves and others? It’s when a person is quick-witted when it comes to skillful teachings. … And they’re a good speaker. … That’s how a person practices to benefit both themselves and others. 

These\marginnote{5.5} are the four people found in the world.” 

%
\section*{{\suttatitleacronym AN 4.98}{\suttatitletranslation To Benefit Oneself }{\suttatitleroot Attahitasutta}}
\addcontentsline{toc}{section}{\tocacronym{AN 4.98} \toctranslation{To Benefit Oneself } \tocroot{Attahitasutta}}
\markboth{To Benefit Oneself }{Attahitasutta}
\extramarks{AN 4.98}{AN 4.98}

“Mendicants,\marginnote{1.1} these four people are found in the world. What four? 

\begin{enumerate}%
\item One who practices to benefit themselves, but not others; %
\item one who practices to benefit others, but not themselves; %
\item one who practices to benefit neither themselves nor others; and %
\item one who practices to benefit both themselves and others. %
\end{enumerate}

These\marginnote{1.7} are the four people found in the world.” 

%
\section*{{\suttatitleacronym AN 4.99}{\suttatitletranslation Training Rules }{\suttatitleroot Sikkhāpadasutta}}
\addcontentsline{toc}{section}{\tocacronym{AN 4.99} \toctranslation{Training Rules } \tocroot{Sikkhāpadasutta}}
\markboth{Training Rules }{Sikkhāpadasutta}
\extramarks{AN 4.99}{AN 4.99}

“Mendicants,\marginnote{1.1} these four people are found in the world. What four? 

\begin{enumerate}%
\item One who practices to benefit themselves, but not others; %
\item one who practices to benefit others, but not themselves; %
\item one who practices to benefit neither themselves nor others; and %
\item one who practices to benefit both themselves and others. %
\end{enumerate}

And\marginnote{2.1} how does a person practice to benefit themselves, but not others? It’s when a person doesn’t kill living creatures, steal, commit sexual misconduct, lie, or use alcoholic drinks that cause negligence. But they don’t encourage others to do the same. That’s how a person practices to benefit themselves, but not others. 

And\marginnote{3.1} how does a person practice to benefit others, but not themselves? It’s when a person kills living creatures, steals, commits sexual misconduct, lies, and uses alcoholic drinks that cause negligence. But they encourage others to not do these things. That’s how a person practices to benefit others, but not themselves. 

And\marginnote{4.1} how does a person practice to benefit neither themselves nor others? It’s when a person kills, etc. … and doesn’t encourage others to not do these things. That’s how a person practices to benefit neither themselves nor others. 

And\marginnote{5.1} how does a person practice to benefit both themselves and others? It’s when a person doesn’t kill, etc. … and encourages others to do the same. That’s how a person practices to benefit both themselves and others. 

These\marginnote{5.4} are the four people found in the world.” 

%
\section*{{\suttatitleacronym AN 4.100}{\suttatitletranslation With Potaliya the Wanderer }{\suttatitleroot Potaliyasutta}}
\addcontentsline{toc}{section}{\tocacronym{AN 4.100} \toctranslation{With Potaliya the Wanderer } \tocroot{Potaliyasutta}}
\markboth{With Potaliya the Wanderer }{Potaliyasutta}
\extramarks{AN 4.100}{AN 4.100}

Then\marginnote{1.1} the wanderer Potaliya went up to the Buddha, and exchanged greetings with him. When the greetings and polite conversation were over, he sat down to one side, and the Buddha said to him: 

“Potaliya,\marginnote{2.1} these four people are found in the world. What four? 

One\marginnote{2.3} person criticizes those deserving of criticism at the right time, truthfully and substantively. But they don’t praise those deserving of praise at the right time, truthfully and substantively. 

Another\marginnote{2.4} person praises those deserving of praise … But they don’t criticize those deserving of criticism … 

Another\marginnote{2.5} person doesn’t praise those deserving of praise … Nor do they criticize those deserving of criticism … 

Another\marginnote{2.6} person criticizes those deserving of criticism at the right time, truthfully and substantively. And they praise those deserving of praise at the right time, truthfully and substantively. 

These\marginnote{2.7} are the four people found in the world. Of these four people, who do you believe to be the finest?” 

“Master\marginnote{3.1} Gotama, of these four people, it is the person who neither praises those deserving of praise at the right time, truthfully and substantively; nor criticizes those deserving of criticism at the right time, truthfully and substantively. That is the person I believe to be the finest. Why is that? Because, Master Gotama, equanimity is the best.” 

“Potaliya,\marginnote{4.1} of these four people, it is the person who criticizes those deserving of criticism at the right time, truthfully and substantively; and praises those deserving of praise at the right time, truthfully and substantively. That is the person I consider to be the finest. Why is that? Because, Potaliya, understanding of time and context is the best.” 

“Master\marginnote{5.1} Gotama, of these four people, it is the person who criticizes those deserving of criticism at the right time, truthfully and substantively; and praises those deserving of praise at the right time, truthfully and substantively. That is the person I believe to be the finest. Why is that? Because, Master Gotama, understanding of time and context is the best. 

Excellent,\marginnote{6.1} Master Gotama! Excellent! As if he were righting the overturned, or revealing the hidden, or pointing out the path to the lost, or lighting a lamp in the dark so people with good eyes can see what’s there, Master Gotama has made the teaching clear in many ways. I go for refuge to Master Gotama, to the teaching, and to the mendicant \textsanskrit{Saṅgha}. From this day forth, may Master Gotama remember me as a lay follower who has gone for refuge for life.” 

%
\addtocontents{toc}{\let\protect\contentsline\protect\nopagecontentsline}
\pannasa{The Third Fifty }
\addcontentsline{toc}{pannasa}{The Third Fifty }
\markboth{}{}
\addtocontents{toc}{\let\protect\contentsline\protect\oldcontentsline}

%
\addtocontents{toc}{\let\protect\contentsline\protect\nopagecontentsline}
\chapter*{The Chapter on Gods of the Clouds }
\addcontentsline{toc}{chapter}{\tocchapterline{The Chapter on Gods of the Clouds }}
\addtocontents{toc}{\let\protect\contentsline\protect\oldcontentsline}

%
\section*{{\suttatitleacronym AN 4.101}{\suttatitletranslation Clouds (1st) }{\suttatitleroot Paṭhamavalāhakasutta}}
\addcontentsline{toc}{section}{\tocacronym{AN 4.101} \toctranslation{Clouds (1st) } \tocroot{Paṭhamavalāhakasutta}}
\markboth{Clouds (1st) }{Paṭhamavalāhakasutta}
\extramarks{AN 4.101}{AN 4.101}

\scevam{So\marginnote{1.1} I have heard. }At one time the Buddha was staying near \textsanskrit{Sāvatthī} in Jeta’s Grove, \textsanskrit{Anāthapiṇḍika}’s monastery. There the Buddha addressed the mendicants, “Mendicants!” 

“Venerable\marginnote{1.5} sir,” they replied. The Buddha said this: 

“Mendicants,\marginnote{2.1} there are these four kinds of clouds. What four? 

\begin{enumerate}%
\item One thunders but doesn’t rain, %
\item one rains but doesn’t thunder, %
\item one neither thunders nor rains, and %
\item one both rains and thunders. %
\end{enumerate}

These\marginnote{2.7} are the four kinds of clouds. In the same way, these four people similar to clouds are found in the world. What four? 

\begin{enumerate}%
\item One thunders but doesn’t rain, %
\item one rains but doesn’t thunder, %
\item one neither thunders nor rains, and %
\item one both rains and thunders. %
\end{enumerate}

And\marginnote{3.1} how does a person thunder but not rain? It’s when a person is a talker, not a doer. That’s how a person thunders but doesn’t rain. That person is like a cloud that thunders but doesn’t rain, I say. 

And\marginnote{4.1} how does a person rain but not thunder? It’s when a person is a doer, not a talker. … 

And\marginnote{5.1} how does a person neither thunder nor rain? It’s when a person is neither a talker nor a doer. … 

And\marginnote{6.1} how does a person both thunder and rain? It’s when a person is both a talker and a doer. … 

These\marginnote{6.6} four people similar to clouds are found in the world.” 

%
\section*{{\suttatitleacronym AN 4.102}{\suttatitletranslation Clouds (2nd) }{\suttatitleroot Dutiyavalāhakasutta}}
\addcontentsline{toc}{section}{\tocacronym{AN 4.102} \toctranslation{Clouds (2nd) } \tocroot{Dutiyavalāhakasutta}}
\markboth{Clouds (2nd) }{Dutiyavalāhakasutta}
\extramarks{AN 4.102}{AN 4.102}

“Mendicants,\marginnote{1.1} there are these four kinds of clouds. What four? 

\begin{enumerate}%
\item One thunders but doesn’t rain, %
\item one rains but doesn’t thunder, %
\item one neither thunders nor rains, and %
\item one both rains and thunders. %
\end{enumerate}

These\marginnote{1.7} are the four kinds of clouds. In the same way, these four people similar to clouds are found in the world. What four? 

\begin{enumerate}%
\item One thunders but doesn’t rain, %
\item one rains but doesn’t thunder, %
\item one neither thunders nor rains, and %
\item one both rains and thunders. %
\end{enumerate}

And\marginnote{2.1} how does a person thunder but not rain? It’s when a person memorizes the teaching—statements, songs, discussions, verses, inspired exclamations, legends, stories of past lives, amazing stories, and classifications. But they don’t truly understand: ‘This is suffering’ … ‘This is the origin of suffering’ … ‘This is the cessation of suffering’ … ‘This is the practice that leads to the cessation of suffering’. That’s how a person thunders but doesn’t rain. That person is like a cloud that thunders but doesn’t rain, I say. 

And\marginnote{3.1} how does a person rain but not thunder? It’s when a person doesn’t memorize the teaching … But they truly understand: ‘This is suffering’ … 

And\marginnote{4.1} how does a person neither thunder nor rain? It’s when a person doesn’t memorize the teaching … Nor do they truly understand: ‘This is suffering’ … 

And\marginnote{5.1} how does a person both thunder and rain? It’s when a person memorizes the teaching … And they truly understand: ‘This is suffering’ … 

These\marginnote{5.8} four people similar to clouds are found in the world.” 

%
\section*{{\suttatitleacronym AN 4.103}{\suttatitletranslation Pots }{\suttatitleroot Kumbhasutta}}
\addcontentsline{toc}{section}{\tocacronym{AN 4.103} \toctranslation{Pots } \tocroot{Kumbhasutta}}
\markboth{Pots }{Kumbhasutta}
\extramarks{AN 4.103}{AN 4.103}

“Mendicants,\marginnote{1.1} there are these four pots. What four? 

\begin{enumerate}%
\item Covered but hollow, %
\item uncovered but full, %
\item uncovered and hollow, and %
\item covered and full. %
\end{enumerate}

These\marginnote{1.7} are the four pots. In the same way, these four people similar to pots are found in the world. What four? 

\begin{enumerate}%
\item Covered but hollow, %
\item uncovered but full, %
\item uncovered and hollow, and %
\item covered and full. %
\end{enumerate}

And\marginnote{2.1} how is a person covered but hollow? It’s when a person is impressive when going out and coming back, when looking ahead and aside, when bending and extending the limbs, and when bearing the outer robe, bowl and robes. But they don’t truly understand: ‘This is suffering’ … ‘This is the origin of suffering’ … ‘This is the cessation of suffering’ … ‘This is the practice that leads to the cessation of suffering’. That’s how a person is covered but hollow. That person is like a pot that’s covered but hollow, I say. 

And\marginnote{3.1} how is a person uncovered but full? It’s when a person is not impressive … But they truly understand: ‘This is suffering’ … 

And\marginnote{4.1} how is a person uncovered and hollow? It’s when a person is not impressive … Nor do they truly understand: ‘This is suffering’ … 

And\marginnote{5.1} how is a person covered and full? It’s when a person is impressive … And they truly understand: ‘This is suffering’ … 

These\marginnote{5.7} four people similar to pots are found in the world.” 

%
\section*{{\suttatitleacronym AN 4.104}{\suttatitletranslation Lakes }{\suttatitleroot Udakarahadasutta}}
\addcontentsline{toc}{section}{\tocacronym{AN 4.104} \toctranslation{Lakes } \tocroot{Udakarahadasutta}}
\markboth{Lakes }{Udakarahadasutta}
\extramarks{AN 4.104}{AN 4.104}

“Mendicants,\marginnote{1.1} there are these four lakes. What four? 

\begin{enumerate}%
\item One is shallow but appears deep, %
\item one is deep but appears shallow, %
\item one is shallow and appears shallow, and %
\item one is deep and appears deep. %
\end{enumerate}

These\marginnote{1.7} are the four lakes. In the same way, these four people similar to lakes are found in the world. What four? 

\begin{enumerate}%
\item One is shallow but appears deep, %
\item one is deep but appears shallow, %
\item one is shallow and appears shallow, and %
\item one is deep and appears deep. %
\end{enumerate}

And\marginnote{2.1} how is a person shallow but appears deep? It’s when a person is impressive when going out and coming back, when looking ahead and aside, when bending and extending the limbs, and when bearing the outer robe, bowl and robes. But they don’t really understand: ‘This is suffering’ … ‘This is the origin of suffering’ … ‘This is the cessation of suffering’ … ‘This is the practice that leads to the cessation of suffering’. That’s how a person is shallow but appears deep. That person is like a lake that’s shallow but appears deep, I say. 

And\marginnote{3.1} how is a person deep but appears shallow? It’s when a person is not impressive … But they really understand: ‘This is suffering’ … 

And\marginnote{4.1} how is a person shallow and appears shallow? It’s when a person is not impressive … Nor do they really understand: ‘This is suffering’ … 

And\marginnote{5.1} how is a person deep and appears deep? It’s when a person is impressive … And they really understand: ‘This is suffering’ … 

These\marginnote{5.7} four people similar to lakes are found in the world.” 

%
\section*{{\suttatitleacronym AN 4.105}{\suttatitletranslation Mangoes }{\suttatitleroot Ambasutta}}
\addcontentsline{toc}{section}{\tocacronym{AN 4.105} \toctranslation{Mangoes } \tocroot{Ambasutta}}
\markboth{Mangoes }{Ambasutta}
\extramarks{AN 4.105}{AN 4.105}

“Mendicants,\marginnote{1.1} there are these four mangoes. What four? 

\begin{enumerate}%
\item One is unripe but seems ripe, %
\item one is ripe but seems unripe, %
\item one is unripe and seems unripe, and %
\item one is ripe and seems ripe. %
\end{enumerate}

These\marginnote{1.7} are the four mangoes. 

In\marginnote{1.8} the same way, these four people similar to mangoes are found in the world. What four? 

\begin{enumerate}%
\item One is unripe but seems ripe, %
\item one is ripe but seems unripe, %
\item one is unripe and seems unripe, and %
\item one is ripe and seems ripe. %
\end{enumerate}

And\marginnote{2.1} how is a person unripe but seems ripe? It’s when a person is impressive when going out and coming back, when looking ahead and aside, when bending and extending the limbs, and when bearing the outer robe, bowl and robes. But they don’t really understand: ‘This is suffering’ … ‘This is the origin of suffering’ … ‘This is the cessation of suffering’ … ‘This is the practice that leads to the cessation of suffering’. That’s how a person is unripe but seems ripe. That person is like a mango that’s unripe but seems ripe, I say. 

And\marginnote{3.1} how is a person ripe but seems unripe? It’s when a person is not impressive … But they really understand: ‘This is suffering’ … 

And\marginnote{4.1} how is a person unripe and seems unripe? It’s when a person is not impressive … Nor do they really understand: ‘This is suffering’ … 

And\marginnote{5.1} how is a person ripe and seems ripe? It’s when a person is impressive … And they really understand: ‘This is suffering’ … 

These\marginnote{5.7} four people similar to mangoes are found in the world.” 

%
\section*{{\suttatitleacronym AN 4.106}{\suttatitletranslation Mangoes (2nd) }{\suttatitleroot (Dutiyaambasutta)}}
\addcontentsline{toc}{section}{\tocacronym{AN 4.106} \toctranslation{Mangoes (2nd) } \tocroot{(Dutiyaambasutta)}}
\markboth{Mangoes (2nd) }{(Dutiyaambasutta)}
\extramarks{AN 4.106}{AN 4.106}

\textit{(This\marginnote{1.1} is a ghost sutta: there is no text for it in any available editions. However, the summary verse at the end of  the chapter (AN 4.110) mentions “two on mangoes”. The commentary notes the existence of a discourse here, but merely says that its meaning is clear. A note in the VRI edition says that there is a reading that has \textit{\textsanskrit{obhāsa}} where the previous sutta has \textit{\textsanskrit{vaṇṇa}}.) }

%
\section*{{\suttatitleacronym AN 4.107}{\suttatitletranslation Mice }{\suttatitleroot Mūsikasutta}}
\addcontentsline{toc}{section}{\tocacronym{AN 4.107} \toctranslation{Mice } \tocroot{Mūsikasutta}}
\markboth{Mice }{Mūsikasutta}
\extramarks{AN 4.107}{AN 4.107}

“Mendicants,\marginnote{1.1} there are these four kinds of mice. What four? 

\begin{enumerate}%
\item One makes a hole but doesn’t live in it, %
\item one lives in a hole but doesn’t make it, %
\item one neither makes a hole nor lives in it, and %
\item one both makes a hole and lives in it. %
\end{enumerate}

These\marginnote{1.7} are the four kinds of mice. In the same way, these four people similar to mice are found in the world. What four? 

\begin{enumerate}%
\item One makes a hole but doesn’t live in it, %
\item one lives in a hole but doesn’t make it, %
\item one neither makes a hole nor lives in it, and %
\item one both makes a hole and lives in it. %
\end{enumerate}

And\marginnote{2.1} how does a person make a hole but not live in it? It’s when a person memorizes the teaching—statements, songs, discussions, verses, inspired exclamations, legends, stories of past lives, amazing stories, and classifications. But they don’t really understand: ‘This is suffering’ … ‘This is the origin of suffering’ … ‘This is the cessation of suffering’ … ‘This is the practice that leads to the cessation of suffering’. That’s how a person makes a hole but doesn’t live in it. That person is like a mouse that makes a hole but doesn’t live in it, I say. 

And\marginnote{3.1} how does a person live in a hole but not make it? It’s when a person doesn’t memorize the teaching … But they really understand: ‘This is suffering’ … 

And\marginnote{4.1} how does a person neither make a hole nor live in it? It’s when a person doesn’t memorize the teaching … Nor do they really understand: ‘This is suffering’ … 

And\marginnote{5.1} how does a person both make a hole and live in it? It’s when a person memorizes the teaching … And they really understand: ‘This is suffering’ … 

These\marginnote{5.8} four people similar to mice are found in the world.” 

%
\section*{{\suttatitleacronym AN 4.108}{\suttatitletranslation Oxen }{\suttatitleroot Balībaddasutta}}
\addcontentsline{toc}{section}{\tocacronym{AN 4.108} \toctranslation{Oxen } \tocroot{Balībaddasutta}}
\markboth{Oxen }{Balībaddasutta}
\extramarks{AN 4.108}{AN 4.108}

“Mendicants,\marginnote{1.1} there are these four kinds of oxen. What four? 

\begin{enumerate}%
\item One hostile to its own herd, not others; %
\item one hostile to other herds, not its own; %
\item one hostile to both its own herd and others; and %
\item one hostile to neither its own herd nor others. %
\end{enumerate}

These\marginnote{1.7} are the four kinds of oxen. In the same way, these four people similar to oxen are found in the world. What four? 

\begin{enumerate}%
\item One hostile to their own herd, not others; %
\item one hostile to other herds, not their own; %
\item one hostile to both their own herd and others; and %
\item one hostile to neither their own herd nor others. %
\end{enumerate}

And\marginnote{2.1} how is a person hostile to their own herd, not others? It’s when a person intimidates their own followers, not the followers of others. That’s how a person is hostile to their own herd, not others. That person is like an ox that’s hostile to its own herd, not others. 

And\marginnote{3.1} how is a person hostile to other herds, not their own? It’s when a person intimidates the followers of others, not their own. … 

And\marginnote{4.1} how is a person hostile to both their own herd and others? It’s when a person intimidates their own followers and the followers of others. … 

And\marginnote{5.1} how is a person hostile to neither their own herd nor others? It’s when a person doesn’t intimidate their own followers or the followers of others. 

These\marginnote{5.6} four people similar to oxen are found in the world.” 

%
\section*{{\suttatitleacronym AN 4.109}{\suttatitletranslation Trees }{\suttatitleroot Rukkhasutta}}
\addcontentsline{toc}{section}{\tocacronym{AN 4.109} \toctranslation{Trees } \tocroot{Rukkhasutta}}
\markboth{Trees }{Rukkhasutta}
\extramarks{AN 4.109}{AN 4.109}

“Mendicants,\marginnote{1.1} there are these four kinds of tree. What four? 

\begin{enumerate}%
\item One is a softwood surrounded by softwoods, %
\item one is a softwood surrounded by hardwoods, %
\item one is a hardwood surrounded by softwoods, and %
\item one is a hardwood surrounded by hardwoods. %
\end{enumerate}

These\marginnote{1.7} are the four kinds of tree. In the same way, these four people similar to trees are found in the world. What four? 

\begin{enumerate}%
\item One is a softwood surrounded by softwoods, %
\item one is a softwood surrounded by hardwoods, %
\item one is a hardwood surrounded by softwoods, and %
\item one is a hardwood surrounded by hardwoods. %
\end{enumerate}

And\marginnote{2.1} how is a person a softwood surrounded by softwoods? It’s when a person is unethical, of bad character, and their followers are the same. That’s how a person is a softwood surrounded by softwoods. That person is like a softwood tree that’s surrounded by softwoods, I say. 

And\marginnote{3.1} how is a person a softwood surrounded by hardwoods? It’s when a person is unethical, of bad character, but their followers are ethical, of good character. … 

And\marginnote{4.1} how is a person a hardwood surrounded by softwoods? It’s when someone is ethical, of good character, but their followers are unethical, of bad character. … 

And\marginnote{5.1} how is a person a hardwood surrounded by hardwoods? It’s when someone is ethical, of good character, and their followers are the same. 

These\marginnote{5.7} four people similar to trees are found in the world.” 

%
\section*{{\suttatitleacronym AN 4.110}{\suttatitletranslation Vipers }{\suttatitleroot Āsīvisasutta}}
\addcontentsline{toc}{section}{\tocacronym{AN 4.110} \toctranslation{Vipers } \tocroot{Āsīvisasutta}}
\markboth{Vipers }{Āsīvisasutta}
\extramarks{AN 4.110}{AN 4.110}

“Mendicants,\marginnote{1.1} there are these four kinds of viper. What four? 

\begin{enumerate}%
\item One whose venom is fast-acting but not lethal, %
\item one whose venom is lethal but not fast-acting, %
\item one whose venom is both fast-acting and lethal, and %
\item one whose venom is neither fast-acting nor lethal. %
\end{enumerate}

These\marginnote{1.7} are the four kinds of viper. In the same way, these four people similar to vipers are found in the world. What four? 

\begin{enumerate}%
\item One whose venom is fast-acting but not lethal, %
\item one whose venom is lethal but not fast-acting, %
\item one whose venom is both fast-acting and lethal, and %
\item one whose venom is neither fast-acting nor lethal. %
\end{enumerate}

And\marginnote{2.1} how is a person’s venom fast-acting but not lethal? It’s when a person is often angry, but their anger doesn’t linger long. That’s how a person’s venom is fast-acting but not lethal. That person is like a viper whose venom is fast-acting but not lethal. 

And\marginnote{3.1} how is a person’s venom lethal but not fast-acting? It’s when a person is not often angry, but their anger lingers for a long time. 

And\marginnote{4.1} how is a person’s venom both fast-acting and lethal? It’s when a person is often angry, and their anger lingers for a long time. 

And\marginnote{5.1} how is a person’s venom neither fast-acting nor lethal? It’s when a person is not often angry, and their anger doesn’t linger long. 

These\marginnote{5.7} four people similar to vipers are found in the world.” 

%
\addtocontents{toc}{\let\protect\contentsline\protect\nopagecontentsline}
\chapter*{The Chapter with Kesi }
\addcontentsline{toc}{chapter}{\tocchapterline{The Chapter with Kesi }}
\addtocontents{toc}{\let\protect\contentsline\protect\oldcontentsline}

%
\section*{{\suttatitleacronym AN 4.111}{\suttatitletranslation With Kesi }{\suttatitleroot Kesisutta}}
\addcontentsline{toc}{section}{\tocacronym{AN 4.111} \toctranslation{With Kesi } \tocroot{Kesisutta}}
\markboth{With Kesi }{Kesisutta}
\extramarks{AN 4.111}{AN 4.111}

Then\marginnote{1.1} Kesi the horse trainer went up to the Buddha, bowed, and sat down to one side. The Buddha said to him, “Kesi, you’re known as a horse trainer. Just how do you guide a horse in training?” 

“Sir,\marginnote{1.4} I guide a horse in training sometimes gently, sometimes harshly, and sometimes both gently and harshly.” 

“Kesi,\marginnote{1.5} what do you do with a horse in training that doesn’t follow these forms of training?” 

“In\marginnote{1.6} that case, sir, I kill it. Why is that? So that I don’t disgrace my tradition. 

But\marginnote{2.1} sir, the Buddha is the supreme guide for those who wish to train. Just how do you guide a person in training?” 

“Kesi,\marginnote{2.3} I guide a person in training sometimes gently, sometimes harshly, and sometimes both gently and harshly. 

The\marginnote{2.4} gentle way is this: ‘This is good conduct by way of body, speech, and mind. This is the result of good conduct by way of body, speech, and mind. This is life as a god. This is life as a human.’ 

The\marginnote{2.6} harsh way is this: ‘This is bad conduct by way of body, speech, and mind. This is the result of bad conduct by way of body, speech, and mind. This is life in hell. This is life as an animal. This is life as a ghost.’ 

The\marginnote{3.1} both gentle and harsh way is this: ‘This is good conduct … this is bad conduct …’” 

“Sir,\marginnote{4.1} what do you do with a person in training who doesn’t follow these forms of training?” 

“In\marginnote{4.2} that case, Kesi, I kill them.” 

“Sir,\marginnote{4.3} it’s not appropriate for the Buddha to kill living creatures. And yet you say you kill them.” 

“It’s\marginnote{4.6} true, Kesi, it’s not appropriate for a Realized One to kill living creatures. But when a person in training doesn’t follow any of these forms of training, the Realized One doesn’t think they’re worth advising or instructing, and neither do their sensible spiritual companions. For it is killing in the training of the Noble One when the Realized One doesn’t think they’re worth advising or instructing, and neither do their sensible spiritual companions.” 

“Well,\marginnote{5.1} they’re definitely dead when the Realized One doesn’t think they’re worth advising or instructing, and neither do their sensible spiritual companions. Excellent, sir! … From this day forth, may the Buddha remember me as a lay follower who has gone for refuge for life.” 

%
\section*{{\suttatitleacronym AN 4.112}{\suttatitletranslation Speed }{\suttatitleroot Javasutta}}
\addcontentsline{toc}{section}{\tocacronym{AN 4.112} \toctranslation{Speed } \tocroot{Javasutta}}
\markboth{Speed }{Javasutta}
\extramarks{AN 4.112}{AN 4.112}

“Mendicants,\marginnote{1.1} a fine royal thoroughbred with four factors is worthy of a king, fit to serve a king, and considered a factor of kingship. What four? Integrity, speed, patience, and sweetness. A fine royal thoroughbred with these four factors is worthy of a king. … 

In\marginnote{2.1} the same way, a mendicant with four qualities is worthy of offerings dedicated to the gods, worthy of hospitality, worthy of a religious donation, worthy of veneration with joined palms, and is the supreme field of merit for the world. What four? Integrity, speed, patience, and sweetness. A mendicant with these four qualities … is the supreme field of merit for the world.” 

%
\section*{{\suttatitleacronym AN 4.113}{\suttatitletranslation The Goad }{\suttatitleroot Patodasutta}}
\addcontentsline{toc}{section}{\tocacronym{AN 4.113} \toctranslation{The Goad } \tocroot{Patodasutta}}
\markboth{The Goad }{Patodasutta}
\extramarks{AN 4.113}{AN 4.113}

“Mendicants,\marginnote{1.1} these four fine thoroughbreds are found in the world. What four? 

One\marginnote{1.3} fine thoroughbred is moved to act when it sees the shadow of the goad, thinking: ‘What task will the horse trainer have me do today? How should I respond?’ Some fine thoroughbreds are like that. This is the first fine thoroughbred found in the world. 

Furthermore,\marginnote{2.1} one fine thoroughbred isn’t moved to act when it sees the shadow of the goad, but only when its hairs are struck, thinking: ‘What task will the horse trainer have me do today? How should I respond?’ Some fine thoroughbreds are like that. This is the second fine thoroughbred found in the world. 

Furthermore,\marginnote{3.1} one fine thoroughbred isn’t moved to act when it sees the shadow of the goad, nor when its hairs are struck, but only when its hide is struck, thinking: ‘What task will the horse trainer have me do today? How should I respond?’ Some fine thoroughbreds are like that. This is the third fine thoroughbred found in the world. 

Furthermore,\marginnote{4.1} one fine thoroughbred isn’t moved to act when it sees the shadow of the goad, nor when its hairs are struck, nor when its hide is struck, but only when its bone is struck, thinking: ‘What task will the horse trainer have me do today? How should I respond?’ Some fine thoroughbreds are like that. This is the fourth fine thoroughbred found in the world. 

These\marginnote{4.5} are the four fine thoroughbreds found in the world. 

In\marginnote{5.1} the same way, these four fine thoroughbred people are found in the world. What four? 

One\marginnote{5.3} fine thoroughbred person hears about the suffering or death of a woman or man in such and such village or town. They’re moved to act by this, and strive effectively. Applying themselves, they directly realize the ultimate truth, and see it with penetrating wisdom. This person is like the fine thoroughbred that’s shaken when it sees the shadow of the goad. Some fine thoroughbred people are like that. This is the first fine thoroughbred person found in the world. 

Furthermore,\marginnote{6.1} one fine thoroughbred person doesn’t hear about the suffering or death of a woman or man in such and such village or town, but they see it themselves. They’re moved to act by this, and strive effectively. Applying themselves, they directly realize the ultimate truth, and see it with penetrating wisdom. This person is like the fine thoroughbred that’s moved to act when its hairs are struck. Some fine thoroughbred people are like that. This is the second fine thoroughbred person found in the world. 

Furthermore,\marginnote{7.1} one fine thoroughbred person doesn’t hear about the suffering or death of a woman or man in such and such village or town, nor do they see it themselves, but it happens to their own relative or family member. They’re moved to act by this, and strive effectively. Applying themselves, they directly realize the ultimate truth, and see it with penetrating wisdom. This person is like the fine thoroughbred that’s moved to act when its skin is struck. Some fine thoroughbred people are like that. This is the third fine thoroughbred person found in the world. 

Furthermore,\marginnote{8.1} one fine thoroughbred person doesn’t hear about the suffering or death of a woman or man in such and such village or town, nor do they see it themselves, nor does it happen to their own relative or family member, but they themselves are afflicted with physical pain—sharp, severe, acute, unpleasant, disagreeable, and life-threatening. They’re moved to act by this, and strive effectively. Applying themselves, they directly realize the ultimate truth, and see it with penetrating wisdom. This person is like the fine thoroughbred that’s moved to act when its bone is struck. Some fine thoroughbred people are like that. This is the fourth fine thoroughbred person found in the world. 

These\marginnote{8.10} are the four fine thoroughbred people found in the world.” 

%
\section*{{\suttatitleacronym AN 4.114}{\suttatitletranslation A Royal Elephant }{\suttatitleroot Nāgasutta}}
\addcontentsline{toc}{section}{\tocacronym{AN 4.114} \toctranslation{A Royal Elephant } \tocroot{Nāgasutta}}
\markboth{A Royal Elephant }{Nāgasutta}
\extramarks{AN 4.114}{AN 4.114}

“Mendicants,\marginnote{1.1} a royal bull elephant with four factors is worthy of a king, fit to serve a king, and is considered a factor of kingship. What four? A royal bull elephant listens, destroys, endures, and goes fast. 

And\marginnote{2.1} how does a royal bull elephant listen? It’s when a royal bull elephant pays heed, pays attention, engages wholeheartedly, and lends an ear to whatever task the elephant trainer has it do, whether or not it has done it before. That’s how a royal bull elephant listens. 

And\marginnote{3.1} how does a royal bull elephant destroy? It’s when a royal bull elephant in battle destroys elephants with their riders, horses with their riders, chariots and charioteers, and foot soldiers. That’s how a royal bull elephant destroys. 

And\marginnote{4.1} how does a royal bull elephant endure? It’s when a royal bull elephant in battle endures being struck by spears, swords, arrows, and axes; it endures the thunder of the drums, kettledrums, horns, and cymbals. That’s how a royal bull elephant endures. 

And\marginnote{5.1} how does a royal bull elephant go fast? It’s when a royal bull elephant swiftly goes in whatever direction the elephant trainer sends it, whether or not it has been there before. That’s how a royal bull elephant goes fast. A royal bull elephant with four factors is worthy of a king, fit to serve a king, and is considered a factor of kingship. 

In\marginnote{6.1} the same way, a mendicant with four qualities is worthy of offerings dedicated to the gods, worthy of hospitality, worthy of a religious donation, worthy of veneration with joined palms, and is the supreme field of merit for the world. What four? A mendicant listens, destroys, endures, and goes fast. 

And\marginnote{7.1} how does a mendicant listen? It’s when a mendicant pays heed, pays attention, engages wholeheartedly, and lends an ear when the teaching and training proclaimed by a Realized One is being taught. That’s how a mendicant listens. 

And\marginnote{8.1} how does a mendicant destroy? It’s when a mendicant doesn’t tolerate a sensual, malicious, or cruel thought. They don’t tolerate any bad, unskillful qualities that have arisen, but give them up, get rid of them, eliminate them, and obliterate them. That’s how a mendicant destroys. 

And\marginnote{9.1} how does a mendicant endure? It’s when a mendicant endures cold, heat, hunger, and thirst; the touch of flies, mosquitoes, wind, sun, and reptiles; rude and unwelcome criticism; and they put up with physical pain—sharp, severe, acute, unpleasant, disagreeable, and life-threatening. That’s how a mendicant endures. 

And\marginnote{10.1} how does a mendicant go fast? It’s when a mendicant swiftly goes in the direction they’ve never gone before in all this long time; that is, the stilling of all activities, the letting go of all attachments, the ending of craving, fading away, cessation, extinguishment. That’s how a mendicant goes fast. A mendicant with these four qualities … is the supreme field of merit for the world.” 

%
\section*{{\suttatitleacronym AN 4.115}{\suttatitletranslation Things }{\suttatitleroot Ṭhānasutta}}
\addcontentsline{toc}{section}{\tocacronym{AN 4.115} \toctranslation{Things } \tocroot{Ṭhānasutta}}
\markboth{Things }{Ṭhānasutta}
\extramarks{AN 4.115}{AN 4.115}

“Mendicants,\marginnote{1.1} there are these four things. What four? 

\begin{enumerate}%
\item There is a thing that’s unpleasant to do, and doing it proves harmful. %
\item There is a thing that’s unpleasant to do, but doing it proves beneficial. %
\item There is a thing that’s pleasant to do, but doing it proves harmful. %
\item There is a thing that’s pleasant to do, and doing it proves beneficial. %
\end{enumerate}

Take\marginnote{2.1} the thing that’s unpleasant to do, and doing it proves harmful. This is regarded as a thing that shouldn’t be done on both grounds: because it’s unpleasant, and because doing it proves harmful. This is regarded as a thing that shouldn’t be done on both grounds. 

Next,\marginnote{3.1} take the thing that’s unpleasant to do, but doing it proves beneficial. It is here that you can tell who is foolish and who is astute in regard to human strength, energy, and vigor. A fool doesn’t reflect: ‘Despite the fact that this thing is unpleasant to do, doing it still proves beneficial.’ They don’t do that thing, so that proves harmful. An astute person does reflect: ‘Despite the fact that this thing is unpleasant to do, doing it still proves beneficial.’ They do that thing, so that proves beneficial. 

Next,\marginnote{4.1} take the thing that’s pleasant to do, but doing it proves harmful. It is here that you can tell who is foolish and who is astute in regard to human strength, energy, and vigor. A fool doesn’t reflect: ‘Despite the fact that this thing is pleasant to do, doing it still proves harmful.’ They do that thing, and so that proves harmful. An astute person does reflect: ‘Despite the fact that this thing is pleasant to do, doing it still proves harmful.’ They don’t do that thing, so that proves beneficial. 

Next,\marginnote{5.1} take the thing that’s pleasant to do, and doing it proves beneficial. This is regarded as a thing that should be done on both grounds: because it’s pleasant, and because doing it proves beneficial. This is regarded as a thing that should be done on both grounds. 

These\marginnote{5.6} are the four things.” 

%
\section*{{\suttatitleacronym AN 4.116}{\suttatitletranslation Diligence }{\suttatitleroot Appamādasutta}}
\addcontentsline{toc}{section}{\tocacronym{AN 4.116} \toctranslation{Diligence } \tocroot{Appamādasutta}}
\markboth{Diligence }{Appamādasutta}
\extramarks{AN 4.116}{AN 4.116}

“Mendicants,\marginnote{1.1} you should be diligent in four situations. What four? Give up bad conduct by way of body, speech, and mind; and develop good conduct by way of body, speech, and mind. Don’t neglect these things. Give up wrong view; and develop right view. Don’t neglect this. 

A\marginnote{2.1} mendicant who has done these things does not fear death in lives to come.” 

%
\section*{{\suttatitleacronym AN 4.117}{\suttatitletranslation Guarding }{\suttatitleroot Ārakkhasutta}}
\addcontentsline{toc}{section}{\tocacronym{AN 4.117} \toctranslation{Guarding } \tocroot{Ārakkhasutta}}
\markboth{Guarding }{Ārakkhasutta}
\extramarks{AN 4.117}{AN 4.117}

“Mendicants,\marginnote{1.1} in your own way you should practice diligence, mindfulness, and guarding of the mind in four situations. What four? 

‘May\marginnote{1.3} my mind not be aroused by things that arouse greed.’ In your own way you should practice diligence, mindfulness, and guarding of the mind. 

‘May\marginnote{1.4} my mind not be angered by things that provoke hate.’ … 

‘May\marginnote{1.5} my mind not be deluded by things that promote delusion.’ … 

‘May\marginnote{1.6} my mind not be intoxicated by things that intoxicate.’ … 

When\marginnote{2.1} a mendicant’s mind is no longer affected by greed, hate, delusion, or intoxication because they’ve got rid of these things, they don’t shake, tremble, quake, or get nervous, nor are they persuaded by the teachings of other ascetics.” 

%
\section*{{\suttatitleacronym AN 4.118}{\suttatitletranslation Inspiring }{\suttatitleroot Saṁvejanīyasutta}}
\addcontentsline{toc}{section}{\tocacronym{AN 4.118} \toctranslation{Inspiring } \tocroot{Saṁvejanīyasutta}}
\markboth{Inspiring }{Saṁvejanīyasutta}
\extramarks{AN 4.118}{AN 4.118}

“Mendicants,\marginnote{1.1} a faithful gentleman should go to see these four inspiring places. What four? 

Thinking:\marginnote{1.3} ‘Here the Realized One was born!’—that is an inspiring place. 

Thinking:\marginnote{1.4} ‘Here the Realized One became awakened as a supreme fully awakened Buddha!’—that is an inspiring place. 

Thinking:\marginnote{1.5} ‘Here the Realized One rolled forth the supreme Wheel of Dhamma!’—that is an inspiring place. 

Thinking:\marginnote{1.6} ‘Here the Realized One became fully extinguished through the element of extinguishment, with nothing left over!’—that is an inspiring place. 

These\marginnote{1.7} are the four inspiring places that a faithful gentleman should go to see.” 

%
\section*{{\suttatitleacronym AN 4.119}{\suttatitletranslation Perils (1st) }{\suttatitleroot Paṭhamabhayasutta}}
\addcontentsline{toc}{section}{\tocacronym{AN 4.119} \toctranslation{Perils (1st) } \tocroot{Paṭhamabhayasutta}}
\markboth{Perils (1st) }{Paṭhamabhayasutta}
\extramarks{AN 4.119}{AN 4.119}

“Mendicants,\marginnote{1.1} there are these four perils. What four? The perils of rebirth, old age, sickness, and death. 

These\marginnote{1.4} are the four perils.” 

%
\section*{{\suttatitleacronym AN 4.120}{\suttatitletranslation Perils (2nd) }{\suttatitleroot Dutiyabhayasutta}}
\addcontentsline{toc}{section}{\tocacronym{AN 4.120} \toctranslation{Perils (2nd) } \tocroot{Dutiyabhayasutta}}
\markboth{Perils (2nd) }{Dutiyabhayasutta}
\extramarks{AN 4.120}{AN 4.120}

“Mendicants,\marginnote{1.1} there are these four perils. What four? The perils of fire, water, kings, and bandits. These are the four perils.” 

%
\addtocontents{toc}{\let\protect\contentsline\protect\nopagecontentsline}
\chapter*{The Chapter on Perils }
\addcontentsline{toc}{chapter}{\tocchapterline{The Chapter on Perils }}
\addtocontents{toc}{\let\protect\contentsline\protect\oldcontentsline}

%
\section*{{\suttatitleacronym AN 4.121}{\suttatitletranslation Guilt }{\suttatitleroot Attānuvādasutta}}
\addcontentsline{toc}{section}{\tocacronym{AN 4.121} \toctranslation{Guilt } \tocroot{Attānuvādasutta}}
\markboth{Guilt }{Attānuvādasutta}
\extramarks{AN 4.121}{AN 4.121}

“Mendicants,\marginnote{1.1} there are these four fears. What four? The fears of guilt, shame, punishment, and going to a bad place. 

And\marginnote{2.1} what, mendicants, is the fear of guilt? It’s when someone reflects: ‘If I were to do bad things by way of body, speech, and mind, wouldn’t I blame myself for my conduct?’ Being afraid of guilt, they give up bad conduct by way of body, speech, and mind, and develop good conduct by way of body, speech, and mind, keeping themselves pure. This is called the fear of guilt. 

And\marginnote{3.1} what, mendicants, is the fear of shame? It’s when someone reflects: ‘If I were to do bad things by way of body, speech, and mind, wouldn’t others blame me for my conduct?’ Being afraid of shame, they give up bad conduct by way of body, speech, and mind, and develop good conduct by way of body, speech, and mind, keeping themselves pure. This is called the fear of shame. 

And\marginnote{4.1} what, mendicants, is the fear of punishment? It’s when someone sees that the kings have arrested a bandit, a criminal, and subjected them to various punishments—whipping, caning, and clubbing; cutting off hands or feet, or both; cutting off ears or nose, or both; the ‘porridge pot’, the ‘shell-shave’, the ‘demon’s mouth’, the ‘garland of fire’, the ‘burning hand’, the ‘grass blades’, the ‘bark dress’, the ‘antelope’, the ‘meat hook’, the ‘coins’, the ‘caustic pickle’, the ‘twisting bar’, the ‘straw mat’; being splashed with hot oil, being fed to the dogs, being impaled alive, and being beheaded. 

They\marginnote{5.1} think: ‘If I were to do the same kind of bad deed, the kings would punish me in the same way.’ … Being afraid of punishment, they don’t steal the belongings of others. They give up bad conduct by way of body, speech, and mind, and develop good conduct by way of body, speech, and mind, keeping themselves pure. This is called the fear of punishment. 

And\marginnote{6.1} what, mendicants, is the fear of rebirth in a bad place? It’s when someone reflects: ‘Bad conduct of body, speech, or mind has a bad result in the next life. If I were to do such bad things, when my body breaks up, after death, I’d be reborn in a place of loss, a bad place, the underworld, hell.’ Being afraid of rebirth in a bad place, they give up bad conduct by way of body, speech, and mind, and develop good conduct by way of body, speech, and mind, keeping themselves pure. This is called the fear of rebirth in a bad place. 

These\marginnote{6.7} are the four fears.” 

%
\section*{{\suttatitleacronym AN 4.122}{\suttatitletranslation The Danger of Waves }{\suttatitleroot Ūmibhayasutta}}
\addcontentsline{toc}{section}{\tocacronym{AN 4.122} \toctranslation{The Danger of Waves } \tocroot{Ūmibhayasutta}}
\markboth{The Danger of Waves }{Ūmibhayasutta}
\extramarks{AN 4.122}{AN 4.122}

“Mendicants,\marginnote{1.1} anyone who enters the water should anticipate four dangers. What four? The dangers of waves, marsh crocodiles, whirlpools, and gharials. 

These\marginnote{1.4} are the four dangers that anyone who enters the water should anticipate. In the same way, a gentleman who goes forth from the lay life to homelessness in this teaching and training should anticipate four dangers. What four? The dangers of waves, marsh crocodiles, whirlpools, and gharials. 

And\marginnote{2.1} what, mendicants, is the danger of waves? It’s when a gentleman has gone forth from the lay life to homelessness, thinking: ‘I’m swamped by rebirth, old age, and death; by sorrow, lamentation, pain, sadness, and distress. I’m swamped by suffering, mired in suffering. Hopefully I can find an end to this entire mass of suffering.’ When they’ve gone forth, their spiritual companions advise and instruct them: ‘You should go out like this, and come back like that. You should look to the front like this, and to the side like that. You should contract your limbs like this, and extend them like that. This is how you should bear your outer robe, bowl, and robes.’ They think: ‘Formerly, as a lay person, I advised and instructed others. And now these mendicants—who you’d think were my children or grandchildren—imagine they can advise and instruct me!’ Angry and upset, they resign the training and return to a lesser life. This is called a mendicant who resigns the training and returns to a lesser life because they’re afraid of the danger of waves. ‘Danger of waves’ is a term for anger and distress. This is called the danger of waves. 

And\marginnote{3.1} what, mendicants, is the danger of marsh crocodiles? It’s when a gentleman has gone forth from the lay life to homelessness … When they’ve gone forth, their spiritual companions advise and instruct them: ‘You may eat, consume, taste, and drink these things, but not those. You may eat what’s allowable, but not what’s unallowable. You may eat at the right time, but not at the wrong time.’ They think: ‘Formerly, as a lay person, I used to eat, consume, taste, and drink what I wanted, not what I didn’t want. I ate and drank both allowable and unallowable things, at the right time and the wrong time. And these faithful householders give us a variety of delicious foods at the wrong time of day. But these mendicants imagine they can gag our mouths!’ Angry and upset, they resign the training and return to a lesser life. This is called a mendicant who resigns the training and returns to a lesser life because they’re afraid of the danger of marsh crocodiles. ‘Danger of marsh crocodiles’ is a term for gluttony. This is called the danger of marsh crocodiles. 

And\marginnote{4.1} what, mendicants, is the danger of whirlpools? It’s when a gentleman has gone forth from the lay life to homelessness … When they’ve gone forth, they robe up in the morning and, taking their bowl and robe, enter a village or town for alms without guarding body, speech, and mind, without establishing mindfulness, and without restraining the sense faculties. There they see a householder or their child amusing themselves, supplied and provided with the five kinds of sensual stimulation. They think: ‘Formerly, as a lay person, I amused myself, supplied and provided with the five kinds of sensual stimulation. And it’s true that my family is wealthy. I can both enjoy my wealth and make merit. Why don’t I resign the training and return to a lesser life, so I can enjoy my wealth and make merit?’ They resign the training and return to a lesser life. This is called a mendicant who resigns the training and returns to a lesser life because they’re afraid of the danger of whirlpools. ‘Danger of whirlpools’ is a term for the five kinds of sensual stimulation. This is called the danger of whirlpools. 

And\marginnote{5.1} what, mendicants, is the danger of gharials? It’s when a gentleman has gone forth from the lay life to homelessness … When they’ve gone forth, they robe up in the morning and, taking their bowl and robe, enter a village or town for alms without guarding body, speech, and mind, without establishing mindfulness, and without restraining the sense faculties. There they see a female scantily clad, with revealing clothes. Lust infects their mind, so they resign the training and return to a lesser life. This is called a mendicant who resigns the training and returns to a lesser life because they’re afraid of the danger of gharials. ‘Danger of gharials’ is a term for females. This is called the danger of gharials. 

These\marginnote{5.12} are the four dangers that a gentleman who goes forth from the lay life to homelessness in this teaching and training should anticipate.” 

%
\section*{{\suttatitleacronym AN 4.123}{\suttatitletranslation Difference (1st) }{\suttatitleroot Paṭhamanānākaraṇasutta}}
\addcontentsline{toc}{section}{\tocacronym{AN 4.123} \toctranslation{Difference (1st) } \tocroot{Paṭhamanānākaraṇasutta}}
\markboth{Difference (1st) }{Paṭhamanānākaraṇasutta}
\extramarks{AN 4.123}{AN 4.123}

“Mendicants,\marginnote{1.1} these four people are found in the world. What four? 

Firstly,\marginnote{1.3} a mendicant, quite secluded from sensual pleasures, secluded from unskillful qualities, enters and remains in the first absorption, which has the rapture and bliss born of seclusion, while placing the mind and keeping it connected. They enjoy it and like it and find it satisfying. If they abide in that, are committed to it, and meditate on it often without losing it, when they die they’re reborn in the company of the gods of \textsanskrit{Brahmā}’s Host. The lifespan of the gods of Brahma’s Host is one eon. An ordinary person stays there until the lifespan of those gods is spent, then they go to hell or the animal realm or the ghost realm. But a disciple of the Buddha stays there until the lifespan of those gods is spent, then they’re extinguished in that very life. This is the difference between a learned noble disciple and an unlearned ordinary person, that is, when there is a place of rebirth. 

As\marginnote{2.1} the placing of the mind and keeping it connected are stilled, they enter and remain in the second absorption, which has the rapture and bliss born of immersion, with internal clarity and confidence, and unified mind, without placing the mind and keeping it connected. They enjoy it and like it and find it satisfying. If they abide in that, are committed to it, and meditate on it often without losing it, when they die they’re reborn in the company of the gods of streaming radiance. The lifespan of the gods of streaming radiance is two eons. An ordinary person stays there until the lifespan of those gods is spent, then they go to hell or the animal realm or the ghost realm. But a disciple of the Buddha stays there until the lifespan of those gods is spent, then they’re extinguished in that very life. This is the difference between a learned noble disciple and an unlearned ordinary person, that is, when there is a place of rebirth. 

Furthermore,\marginnote{3.1} with the fading away of rapture, they enter and remain in the third absorption, where they meditate with equanimity, mindful and aware, personally experiencing the bliss of which the noble ones declare, ‘Equanimous and mindful, one meditates in bliss.’ They enjoy it and like it and find it satisfying. If they abide in that, are committed to it, and meditate on it often without losing it, when they die they’re reborn in the company of the gods replete with glory. The lifespan of the gods replete with glory is four eons. An ordinary person stays there until the lifespan of those gods is spent, then they go to hell or the animal realm or the ghost realm. But a disciple of the Buddha stays there until the lifespan of those gods is spent, then they’re extinguished in that very life. This is the difference between a learned noble disciple and an unlearned ordinary person, that is, when there is a place of rebirth. 

Furthermore,\marginnote{4.1} giving up pleasure and pain, and ending former happiness and sadness, they enter and remain in the fourth absorption, without pleasure or pain, with pure equanimity and mindfulness. They enjoy it and like it and find it satisfying. If they abide in that, are committed to it, and meditate on it often without losing it, when they die they’re reborn in the company of the gods of abundant fruit. The lifespan of the gods of abundant fruit is five hundred eons. An ordinary person stays there until the lifespan of those gods is spent, then they go to hell or the animal realm or the ghost realm. But a disciple of the Buddha stays there until the lifespan of those gods is spent, then they’re extinguished in that very life. This is the difference between a learned noble disciple and an unlearned ordinary person, that is, when there is a place of rebirth. 

These\marginnote{4.8} are the four people found in the world.” 

%
\section*{{\suttatitleacronym AN 4.124}{\suttatitletranslation Difference (2nd) }{\suttatitleroot Dutiyanānākaraṇasutta}}
\addcontentsline{toc}{section}{\tocacronym{AN 4.124} \toctranslation{Difference (2nd) } \tocroot{Dutiyanānākaraṇasutta}}
\markboth{Difference (2nd) }{Dutiyanānākaraṇasutta}
\extramarks{AN 4.124}{AN 4.124}

“Mendicants,\marginnote{1.1} these four people are found in the world. What four? 

Firstly,\marginnote{1.3} a person, quite secluded from sensual pleasures, secluded from unskillful qualities, enters and remains in the first absorption … They contemplate the phenomena there—included in form, feeling, perception, choices, and consciousness—as impermanent, as suffering, as diseased, as a boil, as a dart, as misery, as an affliction, as alien, as falling apart, as empty, as not-self. When their body breaks up, after death, they’re reborn in the company of the gods of the pure abodes. This rebirth is not shared with ordinary people. 

As\marginnote{2.1} the placing of the mind and keeping it connected are stilled, they enter and remain in the second absorption … third absorption … fourth absorption … They contemplate the phenomena there—included in form, feeling, perception, choices, and consciousness—as impermanent, as suffering, as diseased, as a boil, as a dart, as misery, as an affliction, as alien, as falling apart, as empty, as not-self. When their body breaks up, after death, they’re reborn in the company of the gods of the pure abodes. This rebirth is not shared with ordinary people. 

These\marginnote{2.5} are the four people found in the world.” 

%
\section*{{\suttatitleacronym AN 4.125}{\suttatitletranslation Love (1st) }{\suttatitleroot Paṭhamamettāsutta}}
\addcontentsline{toc}{section}{\tocacronym{AN 4.125} \toctranslation{Love (1st) } \tocroot{Paṭhamamettāsutta}}
\markboth{Love (1st) }{Paṭhamamettāsutta}
\extramarks{AN 4.125}{AN 4.125}

“Mendicants,\marginnote{1.1} these four people are found in the world. What four? 

Firstly,\marginnote{1.3} a person meditates spreading a heart full of love to one direction, and to the second, and to the third, and to the fourth. In the same way above, below, across, everywhere, all around, they spread a heart full of love to the whole world—abundant, expansive, limitless, free of enmity and ill will. They enjoy this and like it and find it satisfying. If they abide in that, are committed to it, and meditate on it often without losing it, when they die they’re reborn in the company of the gods of \textsanskrit{Brahmā}’s Host. The lifespan of the gods of Brahma’s Host is one eon. An ordinary person stays there until the lifespan of those gods is spent, then they go to hell or the animal realm or the ghost realm. But a disciple of the Buddha stays there until the lifespan of those gods is spent, then they’re extinguished in that very life. This is the difference between a learned noble disciple and an unlearned ordinary person, that is, when there is a place of rebirth. 

Furthermore,\marginnote{2.1} a person meditates spreading a heart full of compassion … rejoicing … equanimity to one direction, and to the second, and to the third, and to the fourth. In the same way above, below, across, everywhere, all around, they spread a heart full of equanimity to the whole world—abundant, expansive, limitless, free of enmity and ill will. They enjoy this and like it and find it satisfying. If they abide in that, are committed to it, and meditate on it often without losing it, when they die they’re reborn in the company of the gods of streaming radiance. The lifespan of the gods of streaming radiance is two eons. … they’re reborn in the company of the gods replete with glory. The lifespan of the gods replete with glory is four eons. … they’re reborn in the company of the gods of abundant fruit. The lifespan of the gods of abundant fruit is five hundred eons. An ordinary person stays there until the lifespan of those gods is spent, then they go to hell or the animal realm or the ghost realm. But a disciple of the Buddha stays there until the lifespan of those gods is spent, then they’re extinguished in that very life. This is the difference between a learned noble disciple and an unlearned ordinary person, that is, when there is a place of rebirth. 

These\marginnote{2.12} are the four people found in the world.” 

%
\section*{{\suttatitleacronym AN 4.126}{\suttatitletranslation Love (2nd) }{\suttatitleroot Dutiyamettāsutta}}
\addcontentsline{toc}{section}{\tocacronym{AN 4.126} \toctranslation{Love (2nd) } \tocroot{Dutiyamettāsutta}}
\markboth{Love (2nd) }{Dutiyamettāsutta}
\extramarks{AN 4.126}{AN 4.126}

“Mendicants,\marginnote{1.1} these four people are found in the world. What four? 

Firstly,\marginnote{1.3} a person meditates spreading a heart full of love to one direction, and to the second, and to the third, and to the fourth. In the same way above, below, across, everywhere, all around, they spread a heart full of love to the whole world—abundant, expansive, limitless, free of enmity and ill will. They contemplate the phenomena there—included in form, feeling, perception, choices, and consciousness—as impermanent, as suffering, as diseased, as a boil, as a dart, as misery, as an affliction, as alien, as falling apart, as empty, as not-self. When their body breaks up, after death, they’re reborn in the company of the gods of the pure abodes. This rebirth is not shared with ordinary people. 

Furthermore,\marginnote{2.1} a person meditates spreading a heart full of compassion … rejoicing … equanimity to one direction, and to the second, and to the third, and to the fourth. In the same way above, below, across, everywhere, all around, they spread a heart full of equanimity to the whole world—abundant, expansive, limitless, free of enmity and ill will. They contemplate the phenomena there—included in form, feeling, perception, choices, and consciousness—as impermanent, as suffering, as diseased, as a boil, as a dart, as misery, as an affliction, as alien, as falling apart, as empty, as not-self. When their body breaks up, after death, they’re reborn in the company of the gods of the pure abodes. This rebirth is not shared with ordinary people. 

These\marginnote{2.7} are the four people found in the world.” 

%
\section*{{\suttatitleacronym AN 4.127}{\suttatitletranslation Incredible Things About the Realized One (1st) }{\suttatitleroot Paṭhamatathāgataacchariyasutta}}
\addcontentsline{toc}{section}{\tocacronym{AN 4.127} \toctranslation{Incredible Things About the Realized One (1st) } \tocroot{Paṭhamatathāgataacchariyasutta}}
\markboth{Incredible Things About the Realized One (1st) }{Paṭhamatathāgataacchariyasutta}
\extramarks{AN 4.127}{AN 4.127}

“Mendicants,\marginnote{1.1} with the appearance of a Realized One, a perfected one, a fully awakened Buddha, four incredible and amazing things appear. What four? 

When\marginnote{1.3} the being intent on awakening passes away from the host of Joyful Gods, he’s conceived in his mother’s womb, mindful and aware. And then—in this world with its gods, \textsanskrit{Māras} and \textsanskrit{Brahmās}, this population with its ascetics and brahmins, gods and humans—an immeasurable, magnificent light appears, surpassing the glory of the gods. Even in the boundless desolation of interstellar space—so utterly dark that even the light of the moon and the sun, so mighty and powerful, makes no impression—an immeasurable, magnificent light appears, surpassing the glory of the gods. And the sentient beings reborn there recognize each other by that light: ‘So, it seems other sentient beings have been reborn here!’ This is the first incredible and amazing thing that appears with the appearance of a Realized One. 

Furthermore,\marginnote{2.1} the being intent on awakening emerges from his mother’s womb, mindful and aware. And then … an immeasurable, magnificent light appears … even in the boundless desolation of interstellar space … This is the second incredible and amazing thing that appears with the appearance of a Realized One. 

Furthermore,\marginnote{3.1} the Realized One understands the supreme perfect awakening. And then … an immeasurable, magnificent light appears … even in the boundless desolation of interstellar space … This is the third incredible and amazing thing that appears with the appearance of a Realized One. 

Furthermore,\marginnote{4.1} the Realized One rolls forth the supreme Wheel of Dhamma. And then … an immeasurable, magnificent light appears … even in the boundless desolation of interstellar space … This is the fourth incredible and amazing thing that appears with the appearance of a Realized One. 

With\marginnote{4.6} the appearance of a Realized One, the perfected one, the fully awakened Buddha, these four incredible and amazing things appear.” 

%
\section*{{\suttatitleacronym AN 4.128}{\suttatitletranslation Incredible Things About the Realized One (2nd) }{\suttatitleroot Dutiyatathāgataacchariyasutta}}
\addcontentsline{toc}{section}{\tocacronym{AN 4.128} \toctranslation{Incredible Things About the Realized One (2nd) } \tocroot{Dutiyatathāgataacchariyasutta}}
\markboth{Incredible Things About the Realized One (2nd) }{Dutiyatathāgataacchariyasutta}
\extramarks{AN 4.128}{AN 4.128}

“Mendicants,\marginnote{1.1} with the appearance of a Realized One, the perfected one, the fully awakened Buddha, four incredible and amazing things appear. What four? 

People\marginnote{1.3} like attachment, they love it and enjoy it. Yet when a Realized One is teaching the Dhamma of non-adherence, they want to listen, they lend an ear, and they apply their minds to understand it. This is the first incredible and amazing thing that appears with the appearance of a Realized One. 

People\marginnote{2.1} like conceit, they love it and enjoy it. Yet when a Realized One is teaching the Dhamma of removing conceit, they want to listen, they lend an ear, and they apply their minds to understand it. This is the second incredible and amazing thing that appears with the appearance of a Realized One. 

People\marginnote{3.1} like excitement, they love it and enjoy it. Yet when a Realized One is teaching the Dhamma of peace, they want to listen, they lend an ear, and they apply their minds to understand it. This is the third incredible and amazing thing that appears with the appearance of a Realized One. 

This\marginnote{4.1} population is lost in ignorance, trapped in their shells. Yet when a Realized One is teaching the Dhamma of removing ignorance, they want to listen, they lend an ear, and they apply their minds to understand it. This is the fourth incredible and amazing thing that appears with the appearance of a Realized One. 

With\marginnote{4.4} the appearance of a Realized One, the perfected one, the fully awakened Buddha, four incredible and amazing things appear.” 

%
\section*{{\suttatitleacronym AN 4.129}{\suttatitletranslation Incredible Things About Ānanda }{\suttatitleroot Ānandaacchariyasutta}}
\addcontentsline{toc}{section}{\tocacronym{AN 4.129} \toctranslation{Incredible Things About Ānanda } \tocroot{Ānandaacchariyasutta}}
\markboth{Incredible Things About Ānanda }{Ānandaacchariyasutta}
\extramarks{AN 4.129}{AN 4.129}

“Mendicants,\marginnote{1.1} there are these four incredible and amazing things about Ānanda. What four? 

If\marginnote{1.3} an assembly of monks goes to see Ānanda, they’re uplifted by seeing him and uplifted by hearing him speak. And when he falls silent, they’ve never had enough. 

If\marginnote{2.1} an assembly of nuns … laymen … or laywomen goes to see Ānanda, they’re uplifted by seeing him and uplifted by hearing him speak. And when he falls silent, they’ve never had enough. 

These\marginnote{4.4} are the four incredible and amazing things about Ānanda.” 

%
\section*{{\suttatitleacronym AN 4.130}{\suttatitletranslation Incredible Things About the Wheel-Turning Monarch }{\suttatitleroot Cakkavattiacchariyasutta}}
\addcontentsline{toc}{section}{\tocacronym{AN 4.130} \toctranslation{Incredible Things About the Wheel-Turning Monarch } \tocroot{Cakkavattiacchariyasutta}}
\markboth{Incredible Things About the Wheel-Turning Monarch }{Cakkavattiacchariyasutta}
\extramarks{AN 4.130}{AN 4.130}

“Mendicants,\marginnote{1.1} there are these four incredible and amazing things about a wheel-turning monarch. What four? 

If\marginnote{1.3} an assembly of aristocrats goes to see a wheel-turning monarch, they’re uplifted by seeing him and uplifted by hearing him speak. And when he falls silent, they’ve never had enough. 

If\marginnote{2.1} an assembly of brahmins … householders … or ascetics goes to see a wheel-turning monarch, they’re uplifted by seeing him and uplifted by hearing him speak. And when he falls silent, they’ve never had enough. 

These\marginnote{4.4} are the four incredible and amazing things about a wheel-turning monarch. 

In\marginnote{5.1} the same way, there are these four incredible and amazing things about Ānanda. What four? 

If\marginnote{5.3} an assembly of monks goes to see Ānanda, they’re uplifted by seeing him and uplifted by hearing him speak. And when he falls silent, they’ve never had enough. 

If\marginnote{6.1} an assembly of nuns … laymen … or laywomen goes to see Ānanda, they’re uplifted by seeing him and uplifted by hearing him speak. And when he falls silent, they’ve never had enough. 

These\marginnote{6.4} are the four incredible and amazing things about Ānanda.” 

%
\addtocontents{toc}{\let\protect\contentsline\protect\nopagecontentsline}
\chapter*{The Chapter on Persons }
\addcontentsline{toc}{chapter}{\tocchapterline{The Chapter on Persons }}
\addtocontents{toc}{\let\protect\contentsline\protect\oldcontentsline}

%
\section*{{\suttatitleacronym AN 4.131}{\suttatitletranslation Fetters }{\suttatitleroot Saṁyojanasutta}}
\addcontentsline{toc}{section}{\tocacronym{AN 4.131} \toctranslation{Fetters } \tocroot{Saṁyojanasutta}}
\markboth{Fetters }{Saṁyojanasutta}
\extramarks{AN 4.131}{AN 4.131}

“Mendicants,\marginnote{1.1} these four people are found in the world. What four? 

\begin{enumerate}%
\item One person hasn’t given up the lower fetters, the fetters for getting reborn, or the fetters for getting a continued existence. %
\item One person has given up the lower fetters, but not the fetters for getting reborn, or the fetters for getting a continued existence. %
\item One person has given up the lower fetters and the fetters for getting reborn, but not the fetters for getting a continued existence. %
\item One person has given up the lower fetters, the fetters for getting reborn, and the fetters for getting a continued existence. %
\end{enumerate}

What\marginnote{5.1} person hasn’t given up the lower fetters, the fetters for getting reborn, or the fetters for getting a continued existence? A once-returner. This is the person who hasn’t given up the lower fetters, the fetters for getting reborn, or the fetters for getting a continued existence. 

What\marginnote{6.1} person has given up the lower fetters, but not the fetters for getting reborn, or the fetters for getting a continued existence? One heading upstream, going to the \textsanskrit{Akaniṭṭha} realm. 

What\marginnote{7.1} person has given up the lower fetters and the fetters for getting reborn, but not the fetters for getting a continued existence? One extinguished between one life and the next. 

What\marginnote{8.1} person has given up the lower fetters, the fetters for getting reborn, and the fetters for getting a continued existence? A perfected one. 

These\marginnote{8.4} are the four people found in the world.” 

%
\section*{{\suttatitleacronym AN 4.132}{\suttatitletranslation Eloquence }{\suttatitleroot Paṭibhānasutta}}
\addcontentsline{toc}{section}{\tocacronym{AN 4.132} \toctranslation{Eloquence } \tocroot{Paṭibhānasutta}}
\markboth{Eloquence }{Paṭibhānasutta}
\extramarks{AN 4.132}{AN 4.132}

“Mendicants,\marginnote{1.1} these four people are found in the world. What four? 

\begin{enumerate}%
\item One who speaks on topic, but not fluently. %
\item One who speaks fluently, but not on topic. %
\item One who speaks on topic and fluently. %
\item One who speaks neither on topic nor fluently. %
\end{enumerate}

These\marginnote{1.7} are the four people found in the world.” 

%
\section*{{\suttatitleacronym AN 4.133}{\suttatitletranslation One Who Understands Immediately }{\suttatitleroot Ugghaṭitaññūsutta}}
\addcontentsline{toc}{section}{\tocacronym{AN 4.133} \toctranslation{One Who Understands Immediately } \tocroot{Ugghaṭitaññūsutta}}
\markboth{One Who Understands Immediately }{Ugghaṭitaññūsutta}
\extramarks{AN 4.133}{AN 4.133}

“Mendicants,\marginnote{1.1} these four people are found in the world. What four? One who understands immediately, one who understands after detailed explanation, one who needs education, and one who merely learns by rote. These are the four people found in the world.” 

%
\section*{{\suttatitleacronym AN 4.134}{\suttatitletranslation The Fruits of Initiative }{\suttatitleroot Uṭṭhānaphalasutta}}
\addcontentsline{toc}{section}{\tocacronym{AN 4.134} \toctranslation{The Fruits of Initiative } \tocroot{Uṭṭhānaphalasutta}}
\markboth{The Fruits of Initiative }{Uṭṭhānaphalasutta}
\extramarks{AN 4.134}{AN 4.134}

“These\marginnote{1.1} four people are found in the world. What four? 

\begin{enumerate}%
\item One who lives off the fruit of initiative, but not deeds; %
\item one who lives off the fruit of deeds, but not initiative; %
\item one who lives off the fruit of both deeds and initiative; %
\item one who lives off the fruit of neither initiative nor deeds. %
\end{enumerate}

These\marginnote{1.7} are the four people found in the world.” 

%
\section*{{\suttatitleacronym AN 4.135}{\suttatitletranslation Blameworthy }{\suttatitleroot Sāvajjasutta}}
\addcontentsline{toc}{section}{\tocacronym{AN 4.135} \toctranslation{Blameworthy } \tocroot{Sāvajjasutta}}
\markboth{Blameworthy }{Sāvajjasutta}
\extramarks{AN 4.135}{AN 4.135}

“Mendicants,\marginnote{1.1} these four people are found in the world. What four? The blameworthy, the mostly blameworthy, the slightly blameworthy, and the blameless. 

And\marginnote{2.1} how is a person blameworthy? It’s when a person does things by way of body, speech, and mind that are blameworthy. That’s how a person is blameworthy. 

And\marginnote{3.1} how is a person mostly blameworthy? It’s when a person does things by way of body, speech, and mind that are mostly blameworthy, but occasionally blameless. That’s how a person is mostly blameworthy. 

And\marginnote{4.1} how is a person slightly blameworthy? It’s when a person does things by way of body, speech, and mind that are mostly blameless, but occasionally blameworthy. That’s how a person is slightly blameworthy. 

And\marginnote{5.1} how is a person blameless? It’s when a person does things by way of body, speech, and mind that are blameless. That’s how a person is blameless. 

These\marginnote{5.4} are the four people found in the world.” 

%
\section*{{\suttatitleacronym AN 4.136}{\suttatitletranslation Ethics (1st) }{\suttatitleroot Paṭhamasīlasutta}}
\addcontentsline{toc}{section}{\tocacronym{AN 4.136} \toctranslation{Ethics (1st) } \tocroot{Paṭhamasīlasutta}}
\markboth{Ethics (1st) }{Paṭhamasīlasutta}
\extramarks{AN 4.136}{AN 4.136}

“Mendicants,\marginnote{1.1} these four people are found in the world. What four? One person has not fulfilled ethics, immersion, or wisdom. 

One\marginnote{2.1} person has fulfilled ethics, but not immersion or wisdom. 

One\marginnote{3.1} person has fulfilled ethics and immersion, but not wisdom. 

One\marginnote{4.1} person has fulfilled ethics, immersion, and wisdom. 

These\marginnote{4.2} are the four people found in the world.” 

%
\section*{{\suttatitleacronym AN 4.137}{\suttatitletranslation Ethics (2nd) }{\suttatitleroot Dutiyasīlasutta}}
\addcontentsline{toc}{section}{\tocacronym{AN 4.137} \toctranslation{Ethics (2nd) } \tocroot{Dutiyasīlasutta}}
\markboth{Ethics (2nd) }{Dutiyasīlasutta}
\extramarks{AN 4.137}{AN 4.137}

“Mendicants,\marginnote{1.1} these four people are found in the world. What four? 

\begin{enumerate}%
\item One person doesn’t value or submit to ethics, immersion, or wisdom. %
\item One person values and submits to ethics, but not to immersion or wisdom. %
\item One person values and submits to ethics and immersion, but not wisdom. %
\item One person values and submits to ethics, immersion, and wisdom. %
\end{enumerate}

These\marginnote{4.2} are the four people found in the world.” 

%
\section*{{\suttatitleacronym AN 4.138}{\suttatitletranslation Retreat }{\suttatitleroot Nikaṭṭhasutta}}
\addcontentsline{toc}{section}{\tocacronym{AN 4.138} \toctranslation{Retreat } \tocroot{Nikaṭṭhasutta}}
\markboth{Retreat }{Nikaṭṭhasutta}
\extramarks{AN 4.138}{AN 4.138}

“Mendicants,\marginnote{1.1} these four people are found in the world. What four? 

\begin{enumerate}%
\item One is on retreat in body, but not mind; %
\item one is on retreat in mind, but not body; %
\item one is on retreat in neither body nor mind; and %
\item one is on retreat in both body and mind. %
\end{enumerate}

And\marginnote{2.1} how is a person on retreat in body, but not mind? It’s when a person frequents remote lodgings in the wilderness and the forest. But they think sensual, malicious, and cruel thoughts. That’s how a person is on retreat in body, but not mind. 

And\marginnote{3.1} how is a person on retreat in mind, but not body? It’s when a person doesn’t frequent remote lodgings in the wilderness and the forest. But they think thoughts of renunciation, good will, and harmlessness. That’s how a person is on retreat in mind, but not body. 

And\marginnote{4.1} how is a person on retreat in neither body nor mind? It’s when a person doesn’t frequent remote lodgings in the wilderness and the forest. And they think sensual, malicious, and cruel thoughts. That’s how a person is on retreat in neither body nor mind. 

And\marginnote{5.1} how is a person on retreat in both body and mind? It’s when a person frequents remote lodgings in the wilderness and the forest. And they think thoughts of renunciation, good will, and harmlessness. That’s how a person is on retreat in both body and mind. 

These\marginnote{5.5} are the four people found in the world.” 

%
\section*{{\suttatitleacronym AN 4.139}{\suttatitletranslation Dhamma Speakers }{\suttatitleroot Dhammakathikasutta}}
\addcontentsline{toc}{section}{\tocacronym{AN 4.139} \toctranslation{Dhamma Speakers } \tocroot{Dhammakathikasutta}}
\markboth{Dhamma Speakers }{Dhammakathikasutta}
\extramarks{AN 4.139}{AN 4.139}

“Mendicants,\marginnote{1.1} there are these four Dhamma speakers. What four? 

One\marginnote{1.3} Dhamma speaker speaks little and off topic. And their assembly can’t tell what’s on topic and what’s off topic. Such an assembly regards such a Dhamma speaker simply as a Dhamma speaker. 

One\marginnote{2.1} Dhamma speaker speaks little but stays on topic. And their assembly can tell what’s on topic and what’s off topic. Such an assembly regards such a Dhamma speaker simply as a Dhamma speaker. 

One\marginnote{3.1} Dhamma speaker speaks much but off topic. And their assembly can’t tell what’s on topic and what’s off topic. Such an assembly regards such a Dhamma speaker simply as a Dhamma speaker. 

One\marginnote{4.1} Dhamma speaker speaks much and stays on topic. And their assembly can tell what’s on topic and what’s off topic. Such an assembly regards such a Dhamma speaker simply as a Dhamma speaker. 

These\marginnote{4.4} are the four Dhamma speakers.” 

%
\section*{{\suttatitleacronym AN 4.140}{\suttatitletranslation Speaker }{\suttatitleroot Vādīsutta}}
\addcontentsline{toc}{section}{\tocacronym{AN 4.140} \toctranslation{Speaker } \tocroot{Vādīsutta}}
\markboth{Speaker }{Vādīsutta}
\extramarks{AN 4.140}{AN 4.140}

“Mendicants,\marginnote{1.1} there are these four speakers. What four? 

\begin{enumerate}%
\item There’s a speaker who runs out of meaningful things to say, but not of ways of phrasing things. %
\item There’s a speaker who runs out of ways of phrasing things, but not of meaningful things to say. %
\item There’s a speaker who runs out of both meaningful things to say, and ways of phrasing things. %
\item There’s a speaker who never runs out of meaningful things to say, or ways of phrasing things. %
\end{enumerate}

These\marginnote{1.7} are the four speakers. It is impossible, it cannot happen that someone accomplished in the four kinds of textual analysis will ever run out of meaningful things to say, or ways of phrasing things.” 

%
\addtocontents{toc}{\let\protect\contentsline\protect\nopagecontentsline}
\chapter*{The Chapter on Brightness }
\addcontentsline{toc}{chapter}{\tocchapterline{The Chapter on Brightness }}
\addtocontents{toc}{\let\protect\contentsline\protect\oldcontentsline}

%
\section*{{\suttatitleacronym AN 4.141}{\suttatitletranslation Brightness }{\suttatitleroot Ābhāsutta}}
\addcontentsline{toc}{section}{\tocacronym{AN 4.141} \toctranslation{Brightness } \tocroot{Ābhāsutta}}
\markboth{Brightness }{Ābhāsutta}
\extramarks{AN 4.141}{AN 4.141}

“Mendicants,\marginnote{1.1} there are these four kinds of brightness. What four? The brightness of the moon, sun, fire, and wisdom. These are the four kinds of brightness. The best of these four kinds of brightness is the brightness of wisdom.” 

%
\section*{{\suttatitleacronym AN 4.142}{\suttatitletranslation Radiance }{\suttatitleroot Pabhāsutta}}
\addcontentsline{toc}{section}{\tocacronym{AN 4.142} \toctranslation{Radiance } \tocroot{Pabhāsutta}}
\markboth{Radiance }{Pabhāsutta}
\extramarks{AN 4.142}{AN 4.142}

“Mendicants,\marginnote{1.1} there are these four kinds of radiance. What four? The radiance of the moon, sun, fire, and wisdom. These are the four kinds of radiance. The best of these four kinds of radiance is the radiance of wisdom.” 

%
\section*{{\suttatitleacronym AN 4.143}{\suttatitletranslation Light }{\suttatitleroot Ālokasutta}}
\addcontentsline{toc}{section}{\tocacronym{AN 4.143} \toctranslation{Light } \tocroot{Ālokasutta}}
\markboth{Light }{Ālokasutta}
\extramarks{AN 4.143}{AN 4.143}

“Mendicants,\marginnote{1.1} there are these four lights. What four? The lights of the moon, sun, fire, and wisdom. These are the four lights. The best of these four lights is the light of wisdom.” 

%
\section*{{\suttatitleacronym AN 4.144}{\suttatitletranslation Shining }{\suttatitleroot Obhāsasutta}}
\addcontentsline{toc}{section}{\tocacronym{AN 4.144} \toctranslation{Shining } \tocroot{Obhāsasutta}}
\markboth{Shining }{Obhāsasutta}
\extramarks{AN 4.144}{AN 4.144}

“Mendicants,\marginnote{1.1} there are four kinds of shining. What four? The shining of the moon, sun, fire, and wisdom. These are the four kinds of shining. The best of these four kinds of shining is the shining of wisdom.” 

%
\section*{{\suttatitleacronym AN 4.145}{\suttatitletranslation Lamps }{\suttatitleroot Pajjotasutta}}
\addcontentsline{toc}{section}{\tocacronym{AN 4.145} \toctranslation{Lamps } \tocroot{Pajjotasutta}}
\markboth{Lamps }{Pajjotasutta}
\extramarks{AN 4.145}{AN 4.145}

“Mendicants,\marginnote{1.1} there are these four lamps. What four? The lamps of the moon, sun, fire, and wisdom. These are the four lamps. The best of these four lamps is the lamp of wisdom.” 

%
\section*{{\suttatitleacronym AN 4.146}{\suttatitletranslation Times (1st) }{\suttatitleroot Paṭhamakālasutta}}
\addcontentsline{toc}{section}{\tocacronym{AN 4.146} \toctranslation{Times (1st) } \tocroot{Paṭhamakālasutta}}
\markboth{Times (1st) }{Paṭhamakālasutta}
\extramarks{AN 4.146}{AN 4.146}

“Mendicants,\marginnote{1.1} there are these four times. What four? A time for listening to the teaching, a time for discussing the teaching, a time for serenity, and a time for discernment. These are the four times.” 

%
\section*{{\suttatitleacronym AN 4.147}{\suttatitletranslation Times (2nd) }{\suttatitleroot Dutiyakālasutta}}
\addcontentsline{toc}{section}{\tocacronym{AN 4.147} \toctranslation{Times (2nd) } \tocroot{Dutiyakālasutta}}
\markboth{Times (2nd) }{Dutiyakālasutta}
\extramarks{AN 4.147}{AN 4.147}

“Mendicants,\marginnote{1.1} when these four times are rightly developed and progressed, they gradually lead to the ending of defilements. What four? A time for listening to the teaching, a time for discussing the teaching, a time for serenity, and a time for discernment. 

It’s\marginnote{2.1} like when it rains heavily on a mountain top, and the water flows downhill to fill the hollows, crevices, and creeks. As they become full, they fill up the pools. The pools fill up the lakes, the lakes fill up the streams, and the streams fill up the rivers. And as the rivers become full, they fill up the ocean. 

In\marginnote{2.2} the same way, when these four times are rightly developed and progressed, they gradually lead to the ending of defilements.” 

%
\section*{{\suttatitleacronym AN 4.148}{\suttatitletranslation Bad Conduct }{\suttatitleroot Duccaritasutta}}
\addcontentsline{toc}{section}{\tocacronym{AN 4.148} \toctranslation{Bad Conduct } \tocroot{Duccaritasutta}}
\markboth{Bad Conduct }{Duccaritasutta}
\extramarks{AN 4.148}{AN 4.148}

“Mendicants,\marginnote{1.1} there are these four kinds of bad conduct by way of speech. What four? Speech that’s false, divisive, harsh, or nonsensical. These are the four kinds of bad conduct by way of speech.” 

%
\section*{{\suttatitleacronym AN 4.149}{\suttatitletranslation Good Conduct }{\suttatitleroot Sucaritasutta}}
\addcontentsline{toc}{section}{\tocacronym{AN 4.149} \toctranslation{Good Conduct } \tocroot{Sucaritasutta}}
\markboth{Good Conduct }{Sucaritasutta}
\extramarks{AN 4.149}{AN 4.149}

“Mendicants,\marginnote{1.1} there are these four kinds of good conduct by way of speech. What four? Speech that’s true, harmonious, gentle, and thoughtful. These are the four kinds of good conduct by way of speech.” 

%
\section*{{\suttatitleacronym AN 4.150}{\suttatitletranslation Essentials }{\suttatitleroot Sārasutta}}
\addcontentsline{toc}{section}{\tocacronym{AN 4.150} \toctranslation{Essentials } \tocroot{Sārasutta}}
\markboth{Essentials }{Sārasutta}
\extramarks{AN 4.150}{AN 4.150}

“Mendicants,\marginnote{1.1} there are these four essentials. What four? Ethics, immersion, wisdom, and freedom are essentials. These are the four essentials.” 

%
\addtocontents{toc}{\let\protect\contentsline\protect\nopagecontentsline}
\pannasa{The Fourth Fifty }
\addcontentsline{toc}{pannasa}{The Fourth Fifty }
\markboth{}{}
\addtocontents{toc}{\let\protect\contentsline\protect\oldcontentsline}

%
\addtocontents{toc}{\let\protect\contentsline\protect\nopagecontentsline}
\chapter*{The Chapter on Faculties }
\addcontentsline{toc}{chapter}{\tocchapterline{The Chapter on Faculties }}
\addtocontents{toc}{\let\protect\contentsline\protect\oldcontentsline}

%
\section*{{\suttatitleacronym AN 4.151}{\suttatitletranslation Faculties }{\suttatitleroot Indriyasutta}}
\addcontentsline{toc}{section}{\tocacronym{AN 4.151} \toctranslation{Faculties } \tocroot{Indriyasutta}}
\markboth{Faculties }{Indriyasutta}
\extramarks{AN 4.151}{AN 4.151}

“Mendicants,\marginnote{1.1} there are these four faculties. What four? The faculties of faith, energy, mindfulness, and immersion. These are the four faculties.” 

%
\section*{{\suttatitleacronym AN 4.152}{\suttatitletranslation The Power of Faith }{\suttatitleroot Saddhābalasutta}}
\addcontentsline{toc}{section}{\tocacronym{AN 4.152} \toctranslation{The Power of Faith } \tocroot{Saddhābalasutta}}
\markboth{The Power of Faith }{Saddhābalasutta}
\extramarks{AN 4.152}{AN 4.152}

“Mendicants,\marginnote{1.1} there are these four powers. What four? The powers of faith, energy, mindfulness, and immersion. These are the four powers.” 

%
\section*{{\suttatitleacronym AN 4.153}{\suttatitletranslation The Power of Wisdom }{\suttatitleroot Paññābalasutta}}
\addcontentsline{toc}{section}{\tocacronym{AN 4.153} \toctranslation{The Power of Wisdom } \tocroot{Paññābalasutta}}
\markboth{The Power of Wisdom }{Paññābalasutta}
\extramarks{AN 4.153}{AN 4.153}

“Mendicants,\marginnote{1.1} there are these four powers. What four? The powers of wisdom, energy, blamelessness, and inclusiveness. These are the four powers.” 

%
\section*{{\suttatitleacronym AN 4.154}{\suttatitletranslation The Power of Mindfulness }{\suttatitleroot Satibalasutta}}
\addcontentsline{toc}{section}{\tocacronym{AN 4.154} \toctranslation{The Power of Mindfulness } \tocroot{Satibalasutta}}
\markboth{The Power of Mindfulness }{Satibalasutta}
\extramarks{AN 4.154}{AN 4.154}

“Mendicants,\marginnote{1.1} there are these four powers. What four? The powers of mindfulness, immersion, blamelessness, and inclusiveness. These are the four powers.” 

%
\section*{{\suttatitleacronym AN 4.155}{\suttatitletranslation The Power of Reflection }{\suttatitleroot Paṭisaṅkhānabalasutta}}
\addcontentsline{toc}{section}{\tocacronym{AN 4.155} \toctranslation{The Power of Reflection } \tocroot{Paṭisaṅkhānabalasutta}}
\markboth{The Power of Reflection }{Paṭisaṅkhānabalasutta}
\extramarks{AN 4.155}{AN 4.155}

“Mendicants,\marginnote{1.1} there are these four powers. What four? The powers of reflection, development, blamelessness, and inclusiveness. These are the four powers.” 

%
\section*{{\suttatitleacronym AN 4.156}{\suttatitletranslation Eons }{\suttatitleroot Kappasutta}}
\addcontentsline{toc}{section}{\tocacronym{AN 4.156} \toctranslation{Eons } \tocroot{Kappasutta}}
\markboth{Eons }{Kappasutta}
\extramarks{AN 4.156}{AN 4.156}

“Mendicants,\marginnote{1.1} an eon contains four uncountable periods. What four? 

When\marginnote{1.3} an eon contracts, it’s not easy to calculate how many years, how many hundreds or thousands or hundreds of thousands of years it takes. 

When\marginnote{2.1} an eon remains fully contracted, it’s not easy to calculate how many years, how many hundreds or thousands or hundreds of thousands of years it takes. 

When\marginnote{3.1} an eon expands, it’s not easy to calculate how many years, how many hundreds or thousands or hundreds of thousands of years it takes. 

When\marginnote{4.1} an eon remains fully expanded, it’s not easy to calculate how many years, how many hundreds or thousands or hundreds of thousands of years it takes. 

These\marginnote{4.3} are the four uncountable periods of an eon.” 

%
\section*{{\suttatitleacronym AN 4.157}{\suttatitletranslation Illness }{\suttatitleroot Rogasutta}}
\addcontentsline{toc}{section}{\tocacronym{AN 4.157} \toctranslation{Illness } \tocroot{Rogasutta}}
\markboth{Illness }{Rogasutta}
\extramarks{AN 4.157}{AN 4.157}

“Mendicants,\marginnote{1.1} there are two kinds of illness. What two? Mental and physical. Some sentient beings are seen who can claim to be free of physical illness for a year, or two, or three years … even up to a hundred years or more. But it’s very hard to find any sentient beings in the world who can claim to be free of mental illness even for a moment, apart from those who have ended the defilements. 

There\marginnote{2.1} are four kinds of illness for those gone forth. What four? 

To\marginnote{2.3} start with, a mendicant has many wishes, is frustrated, and is not content with any kind of robes, almsfood, lodgings, and medicines and supplies for the sick. 

Because\marginnote{2.4} of this, they focus their corrupt wishes on being looked up to, and on getting material possessions, honor, and popularity. 

They\marginnote{2.5} try hard, strive, and make an effort to get these things. 

They\marginnote{2.6} have an ulterior motive when they visit families. They have an ulterior motive when they sit down, when they speak on Dhamma, and even when they hold it in when they need to go to the toilet. 

These\marginnote{2.7} are the four kinds of illness for those gone forth. 

So\marginnote{3.1} you should train like this: ‘We will not have many wishes or be frustrated. We will be content with any kind of robes, almsfood, lodgings, and medicines and supplies for the sick. We won’t focus our corrupt wishes on being looked up to, and on getting material possessions, honor, and popularity. We won’t try hard, strive, and make an effort to get these things. We will endure cold, heat, hunger, and thirst. We will endure the touch of flies, mosquitoes, wind, sun, and reptiles. We will endure rude and unwelcome criticism. We will put up with physical pain—sharp, severe, acute, unpleasant, disagreeable, and life-threatening.’ That’s how you should train.” 

%
\section*{{\suttatitleacronym AN 4.158}{\suttatitletranslation Decline }{\suttatitleroot Parihānisutta}}
\addcontentsline{toc}{section}{\tocacronym{AN 4.158} \toctranslation{Decline } \tocroot{Parihānisutta}}
\markboth{Decline }{Parihānisutta}
\extramarks{AN 4.158}{AN 4.158}

There\marginnote{1.1} \textsanskrit{Sāriputta} addressed the mendicants: “Reverends, mendicants!” 

“Reverend,”\marginnote{1.3} they replied. \textsanskrit{Sāriputta} said this: 

“Reverends,\marginnote{2.1} any monk or nun who sees four things inside themselves should conclude: ‘My skillful qualities are declining. For this is what the Buddha calls decline.’ What four? They have much greed, much hate, and much delusion; and their wisdom eye doesn’t go into the many deep matters. Any monk or nun who sees these four things inside themselves should conclude: ‘My skillful qualities are declining. For this is what the Buddha calls decline.’ 

Any\marginnote{3.1} monk or nun who sees four things inside themselves should conclude: ‘My skillful qualities are not declining. For this is what the Buddha calls non-decline.’ What four? Their greed, hate, and delusion grow less; and their wisdom eye goes into the many deep matters. Any monk or nun who sees these four things inside themselves should conclude: ‘My skillful qualities are not declining. For this is what the Buddha calls non-decline.’” 

%
\section*{{\suttatitleacronym AN 4.159}{\suttatitletranslation Nun }{\suttatitleroot Bhikkhunīsutta}}
\addcontentsline{toc}{section}{\tocacronym{AN 4.159} \toctranslation{Nun } \tocroot{Bhikkhunīsutta}}
\markboth{Nun }{Bhikkhunīsutta}
\extramarks{AN 4.159}{AN 4.159}

\scevam{So\marginnote{1.1} I have heard. }At one time Venerable Ānanda was staying near Kosambi, in Ghosita’s Monastery. 

And\marginnote{1.3} then a certain nun addressed a man, “Please, mister, go to Venerable Ānanda, and in my name bow with your head to his feet. Say to him: ‘Sir, the nun named so-and-so is sick, suffering, and gravely ill. She bows with her head to your feet.’ And then say: ‘Sir, please go to the nuns’ quarters to visit that nun out of compassion.’” 

“Yes,\marginnote{1.8} ma’am,” that man replied. He did as the nun asked. Ānanda consented in silence. 

Then\marginnote{3.1} Ānanda robed up and went to the nuns’ quarters to visit that nun, taking his bowl and robe. That nun saw Ānanda coming off in the distance. She wrapped herself up from head to foot and laid down on her cot. Then Venerable Ānanda went up to her, and sat down on the seat spread out. Then Ānanda said to the nun: 

“Sister,\marginnote{4.1} this body is produced by food. Relying on food, you should give up food. This body is produced by craving. Relying on craving, you should give up craving. This body is produced by conceit. Relying on conceit, you should give up conceit. This body is produced by sex. The Buddha spoke of breaking off everything to do with sex. 

‘This\marginnote{5.1} body is produced by food. Relying on food, you should give up food.’ This is what I said, but why did I say it? Take a mendicant who reflects properly on the food that they eat: ‘Not for fun, indulgence, adornment, or decoration, but only to sustain this body, to avoid harm, and to support spiritual practice. In this way, I shall put an end to old discomfort and not give rise to new discomfort, and I will live blamelessly and at ease.’ After some time, relying on food, they give up food. That’s why I said what I said. 

‘This\marginnote{6.1} body is produced by craving. Relying on craving, you should give up craving.’ This is what I said, but why did I say it? Take a mendicant who hears this: ‘They say that the mendicant named so-and-so has realized the undefiled freedom of heart and freedom by wisdom in this very life. And they live having realized it with their own insight due to the ending of defilements.’ They think: ‘Oh, when will I too realize the undefiled freedom of heart and freedom by wisdom in this very life. …’ After some time, relying on craving, they give up craving. That's why I said what I said. 

‘This\marginnote{7.1} body is produced by conceit. Relying on conceit, you should give up conceit.’ This is what I said, but why did I say it? Take a mendicant who hears this: ‘They say that the mendicant named so-and-so has realized the undefiled freedom of heart and freedom by wisdom in this very life. And they live having realized it with their own insight due to the ending of defilements.’ They think: ‘Well, that venerable can realize the undefiled freedom of heart and freedom by wisdom in this very life. … Why can’t I?’ After some time, relying on conceit, they give up conceit. That’s why I said what I said. 

‘This\marginnote{8.1} body is produced by sex. The Buddha spoke of breaking off everything to do with sex.’” 

Then\marginnote{9.1} that nun rose from her cot, placed her robe over one shoulder, bowed with her head at Ānanda’s feet, and said, “I have made a mistake, sir. It was foolish, stupid, and unskillful of me to act in that way. Please, sir, accept my mistake for what it is, so I can restrain myself in future.” 

“Indeed,\marginnote{9.4} sister, you made a mistake. It was foolish, stupid, and unskillful of you to act in that way. But since you have recognized your mistake for what it is, and have dealt with it properly, I accept it. For it is growth in the training of the Noble One to recognize a mistake for what it is, deal with it properly, and commit to restraint in the future.” 

%
\section*{{\suttatitleacronym AN 4.160}{\suttatitletranslation The Training of a Holy One }{\suttatitleroot Sugatavinayasutta}}
\addcontentsline{toc}{section}{\tocacronym{AN 4.160} \toctranslation{The Training of a Holy One } \tocroot{Sugatavinayasutta}}
\markboth{The Training of a Holy One }{Sugatavinayasutta}
\extramarks{AN 4.160}{AN 4.160}

“Mendicants,\marginnote{1.1} a Holy One or a Holy One’s training remain in the world for the welfare and happiness of the people, out of compassion for the world, for the benefit, welfare, and happiness of gods and humans. 

And\marginnote{2.1} who is a Holy One? It’s when a Realized One arises in the world, perfected, a fully awakened Buddha, accomplished in knowledge and conduct, holy, knower of the world, supreme guide for those who wish to train, teacher of gods and humans, awakened, blessed. This is a Holy One. 

And\marginnote{3.1} what is the training of a Holy One? He teaches Dhamma that’s good in the beginning, good in the middle, and good in the end, meaningful and well-phrased. And he reveals a spiritual practice that’s entirely full and pure. This is the training of a Holy One. This is how a Holy One or a Holy One’s training remain in the world for the welfare and happiness of the people, out of compassion for the world, for the benefit, welfare, and happiness of gods and humans. 

These\marginnote{4.1} four things lead to the decline and disappearance of the true teaching. What four? 

Firstly,\marginnote{4.3} the mendicants memorize discourses that they learned incorrectly, with misplaced words and phrases. When the words and phrases are misplaced, the meaning is misinterpreted. This is the first thing that leads to the decline and disappearance of the true teaching. 

Furthermore,\marginnote{5.1} the mendicants are hard to admonish, having qualities that make them hard to admonish. They’re impatient, and don’t take instruction respectfully. This is the second thing that leads to the decline and disappearance of the true teaching. 

Furthermore,\marginnote{6.1} the mendicants who are very learned—knowledgeable in the scriptures, who have memorized the teachings, the monastic law, and the outlines—don’t carefully make others recite the discourses. When they pass away, the discourses are cut off at the root, with no-one to preserve them. This is the third thing that leads to the decline and disappearance of the true teaching. 

Furthermore,\marginnote{7.1} the senior mendicants are indulgent and slack, leaders in backsliding, neglecting seclusion, not rousing energy for attaining the unattained, achieving the unachieved, and realizing the unrealized. Those who come after them follow their example. They too become indulgent and slack, leaders in backsliding, neglecting seclusion, not rousing energy for attaining the unattained, achieving the unachieved, and realizing the unrealized. This is the fourth thing that leads to the decline and disappearance of the true teaching. 

These\marginnote{7.5} are four things that lead to the decline and disappearance of the true teaching. 

These\marginnote{8.1} four things lead to the continuation, persistence, and enduring of the true teaching. What four? 

Firstly,\marginnote{8.3} the mendicants memorize discourses that have been learned correctly, with well placed words and phrases. When the words and phrases are well placed, the meaning is interpreted correctly. This is the first thing that leads to the continuation, persistence, and enduring of the true teaching. 

Furthermore,\marginnote{9.1} the mendicants are easy to admonish, having qualities that make them easy to admonish. They’re patient, and take instruction respectfully. This is the second thing that leads to the continuation, persistence, and enduring of the true teaching. 

Furthermore,\marginnote{10.1} the mendicants who are very learned—knowledgeable in the scriptures, who have memorized the teachings, the monastic law, and the outlines—carefully make others recite the discourses. When they pass away, the discourses aren’t cut off at the root, and they have someone to preserve them. This is the third thing that leads to the continuation, persistence, and enduring of the true teaching. 

Furthermore,\marginnote{11.1} the senior mendicants are not indulgent or slack, nor are they backsliders; instead, they take the lead in seclusion, rousing energy for attaining the unattained, achieving the unachieved, and realizing the unrealized. Those who come after them follow their example. They too aren’t indulgent or slack … This is the fourth thing that leads to the continuation, persistence, and enduring of the true teaching. 

These\marginnote{11.5} are four things that lead to the continuation, persistence, and enduring of the true teaching.” 

%
\addtocontents{toc}{\let\protect\contentsline\protect\nopagecontentsline}
\chapter*{The Chapter on Practice }
\addcontentsline{toc}{chapter}{\tocchapterline{The Chapter on Practice }}
\addtocontents{toc}{\let\protect\contentsline\protect\oldcontentsline}

%
\section*{{\suttatitleacronym AN 4.161}{\suttatitletranslation In Brief }{\suttatitleroot Saṁkhittasutta}}
\addcontentsline{toc}{section}{\tocacronym{AN 4.161} \toctranslation{In Brief } \tocroot{Saṁkhittasutta}}
\markboth{In Brief }{Saṁkhittasutta}
\extramarks{AN 4.161}{AN 4.161}

“Mendicants,\marginnote{1.1} there are four ways of practice. What four? 

\begin{enumerate}%
\item Painful practice with slow insight, %
\item painful practice with swift insight, %
\item pleasant practice with slow insight, and %
\item pleasant practice with swift insight. %
\end{enumerate}

These\marginnote{1.7} are the four ways of practice.” 

%
\section*{{\suttatitleacronym AN 4.162}{\suttatitletranslation In Detail }{\suttatitleroot Vitthārasutta}}
\addcontentsline{toc}{section}{\tocacronym{AN 4.162} \toctranslation{In Detail } \tocroot{Vitthārasutta}}
\markboth{In Detail }{Vitthārasutta}
\extramarks{AN 4.162}{AN 4.162}

“Mendicants,\marginnote{1.1} there are four ways of practice. What four? 

\begin{enumerate}%
\item Painful practice with slow insight, %
\item painful practice with swift insight, %
\item pleasant practice with slow insight, and %
\item pleasant practice with swift insight. %
\end{enumerate}

And\marginnote{2.1} what’s the painful practice with slow insight? It’s when someone is ordinarily full of acute greed, hate, and delusion. They often feel the pain and sadness that greed, hate, and delusion bring. These five faculties manifest in them weakly: faith, energy, mindfulness, immersion, and wisdom. Because of this, they only slowly attain the conditions for ending the defilements in the present life. This is called the painful practice with slow insight. 

And\marginnote{3.1} what’s the painful practice with swift insight? It’s when someone is ordinarily full of acute greed, hate, and delusion. They often feel the pain and sadness that greed, hate, and delusion bring. And these five faculties manifest in them strongly: faith, energy, mindfulness, immersion, and wisdom. Because of this, they swiftly attain the conditions for ending the defilements in the present life. This is called the painful practice with swift insight. 

And\marginnote{4.1} what’s pleasant practice with slow insight? It’s when someone is not ordinarily full of acute greed, hate, and delusion. They rarely feel the pain and sadness that greed, hate, and delusion bring. These five faculties manifest in them weakly: faith, energy, mindfulness, immersion, and wisdom. Because of this, they only slowly attain the conditions for ending the defilements in the present life. This is called the pleasant practice with slow insight. 

And\marginnote{5.1} what’s the pleasant practice with swift insight? It’s when someone is not ordinarily full of acute greed, hate, and delusion. They rarely feel the pain and sadness that greed, hate, and delusion bring. These five faculties manifest in them strongly: faith, energy, mindfulness, immersion, and wisdom. Because of this, they swiftly attain the conditions for ending the defilements in the present life. This is called the pleasant practice with swift insight. 

These\marginnote{5.9} are the four ways of practice.” 

%
\section*{{\suttatitleacronym AN 4.163}{\suttatitletranslation Ugly }{\suttatitleroot Asubhasutta}}
\addcontentsline{toc}{section}{\tocacronym{AN 4.163} \toctranslation{Ugly } \tocroot{Asubhasutta}}
\markboth{Ugly }{Asubhasutta}
\extramarks{AN 4.163}{AN 4.163}

“Mendicants,\marginnote{1.1} there are four ways of practice. What four? 

\begin{enumerate}%
\item Painful practice with slow insight, %
\item painful practice with swift insight, %
\item pleasant practice with slow insight, and %
\item pleasant practice with swift insight. %
\end{enumerate}

And\marginnote{2.1} what’s the painful practice with slow insight? It’s when a mendicant meditates observing the ugliness of the body, perceives the repulsiveness of food, perceives dissatisfaction with the whole world, observes the impermanence of all conditions, and has well established the perception of their own death. They rely on these five powers of a trainee: faith, conscience, prudence, energy, and wisdom. But these five faculties manifest in them weakly: faith, energy, mindfulness, immersion, and wisdom. Because of this, they only slowly attain the conditions for ending the defilements in the present life. This is called the painful practice with slow insight. 

And\marginnote{3.1} what’s the painful practice with swift insight? It’s when a mendicant meditates observing the ugliness of the body, perceives the repulsiveness of food, perceives dissatisfaction with the whole world, observes the impermanence of all conditions, and has well established the perception of their own death. They rely on these five powers of a trainee: faith, conscience, prudence, energy, and wisdom. And these five faculties manifest in them strongly: faith, energy, mindfulness, immersion, and wisdom. Because of this, they swiftly attain the conditions for ending the defilements in the present life. This is called the painful practice with swift insight. 

And\marginnote{4.1} what’s the pleasant practice with slow insight? It’s when a mendicant, quite secluded from sensual pleasures, secluded from unskillful qualities, enters and remains in the first absorption, which has the rapture and bliss born of seclusion, while placing the mind and keeping it connected. As the placing of the mind and keeping it connected are stilled, they enter and remain in the second absorption, which has the rapture and bliss born of immersion, with internal clarity and confidence, and unified mind, without placing the mind and keeping it connected. And with the fading away of rapture, they enter and remain in the third absorption, where they meditate with equanimity, mindful and aware, personally experiencing the bliss of which the noble ones declare, ‘Equanimous and mindful, one meditates in bliss.’ Giving up pleasure and pain, and ending former happiness and sadness, they enter and remain in the fourth absorption, without pleasure or pain, with pure equanimity and mindfulness. They rely on these five powers of a trainee: faith, conscience, prudence, energy, and wisdom. But these five faculties manifest in them weakly: faith, energy, mindfulness, immersion, and wisdom. Because of this, they only slowly attain the conditions for ending the defilements in the present life. This is called the pleasant practice with slow insight. 

And\marginnote{5.1} what’s the pleasant practice with swift insight? It’s when a mendicant … enters and remains in the first absorption … second absorption … third absorption … fourth absorption … They rely on these five powers of a trainee: faith, conscience, prudence, energy, and wisdom. And these five faculties manifest in them strongly: faith, energy, mindfulness, immersion, and wisdom. Because of this, they swiftly attain the conditions for ending the defilements in the present life. This is called the pleasant practice with swift insight. 

These\marginnote{5.9} are the four ways of practice.” 

%
\section*{{\suttatitleacronym AN 4.164}{\suttatitletranslation Patient (1st) }{\suttatitleroot Paṭhamakhamasutta}}
\addcontentsline{toc}{section}{\tocacronym{AN 4.164} \toctranslation{Patient (1st) } \tocroot{Paṭhamakhamasutta}}
\markboth{Patient (1st) }{Paṭhamakhamasutta}
\extramarks{AN 4.164}{AN 4.164}

“Mendicants,\marginnote{1.1} there are four ways of practice. What four? Impatient practice, patient practice, taming practice, and calming practice. 

And\marginnote{1.4} what’s the impatient practice? It’s when someone abuses, annoys, or argues with you, and you abuse, annoy, or argue right back at them. This is called the impatient practice. 

And\marginnote{2.1} what’s the patient practice? It’s when someone abuses, annoys, or argues with you, and you don’t abuse, annoy, or argue back at them. This is called the patient practice. 

And\marginnote{3.1} what’s the taming practice? When a mendicant sees a sight with their eyes, they don’t get caught up in the features and details. If the faculty of sight were left unrestrained, bad unskillful qualities of desire and aversion would become overwhelming. For this reason, they practice restraint, protecting the faculty of sight, and achieving restraint over it. When they hear a sound with their ears … When they smell an odor with their nose … When they taste a flavor with their tongue … When they feel a touch with their body … When they know a thought with their mind, they don’t get caught up in the features and details. If the faculty of mind were left unrestrained, bad unskillful qualities of desire and aversion would become overwhelming. For this reason, they practice restraint, protecting the faculty of mind, and achieving restraint over it. This is called the taming practice. 

And\marginnote{4.1} what’s the calming practice? It’s when a mendicant doesn’t tolerate a sensual, malicious, or cruel thought. They don’t tolerate any bad, unskillful qualities that have arisen, but give them up, get rid of them, calm them, eliminate them, and obliterate them. This is called the calming practice. 

These\marginnote{4.4} are the four ways of practice.” 

%
\section*{{\suttatitleacronym AN 4.165}{\suttatitletranslation Patience (2nd) }{\suttatitleroot Dutiyakhamasutta}}
\addcontentsline{toc}{section}{\tocacronym{AN 4.165} \toctranslation{Patience (2nd) } \tocroot{Dutiyakhamasutta}}
\markboth{Patience (2nd) }{Dutiyakhamasutta}
\extramarks{AN 4.165}{AN 4.165}

“Mendicants,\marginnote{1.1} there are four ways of practice. What four? Impatient practice, patient practice, taming practice, and calming practice. 

And\marginnote{2.1} what’s the impatient practice? It’s when a mendicant cannot endure cold, heat, hunger, and thirst. They cannot endure the touch of flies, mosquitoes, wind, sun, and reptiles. They cannot endure rude and unwelcome criticism. And they cannot put up with physical pain—sharp, severe, acute, unpleasant, disagreeable, and life-threatening. This is called the impatient practice. 

And\marginnote{3.1} what’s the patient practice? It’s when a mendicant endures cold, heat, hunger, and thirst. They endure the touch of flies, mosquitoes, wind, sun, and reptiles. They endure rude and unwelcome criticism. And they put up with physical pain—sharp, severe, acute, unpleasant, disagreeable, and life-threatening. This is called the patient practice. 

And\marginnote{4.1} what’s the taming practice? When a mendicant sees a sight with their eyes, they don’t get caught up in the features and details. … When they hear a sound with their ears … When they smell an odor with their nose … When they taste a flavor with their tongue … When they feel a touch with their body … When they know a thought with their mind, they don’t get caught up in the features and details. If the faculty of mind were left unrestrained, bad unskillful qualities of desire and aversion would become overwhelming. For this reason, they practice restraint, protecting the faculty of mind, and achieving restraint over it. This is called the taming practice. 

And\marginnote{5.1} what’s the calming practice? It’s when a mendicant doesn’t tolerate a sensual, malicious, or cruel thought. They don’t tolerate any bad, unskillful qualities that have arisen, but give them up, get rid of them, calm them, eliminate them, and obliterate them. This is called the calming practice. 

These\marginnote{5.4} are the four ways of practice.” 

%
\section*{{\suttatitleacronym AN 4.166}{\suttatitletranslation Both }{\suttatitleroot Ubhayasutta}}
\addcontentsline{toc}{section}{\tocacronym{AN 4.166} \toctranslation{Both } \tocroot{Ubhayasutta}}
\markboth{Both }{Ubhayasutta}
\extramarks{AN 4.166}{AN 4.166}

“Mendicants,\marginnote{1.1} there are four ways of practice. What four? 

\begin{enumerate}%
\item Painful practice with slow insight, %
\item painful practice with swift insight, %
\item pleasant practice with slow insight, and %
\item pleasant practice with swift insight. %
\end{enumerate}

Of\marginnote{2.1} these, the painful practice with slow insight is said to be inferior in both ways: because it’s painful and because it’s slow. This practice is said to be inferior in both ways. 

The\marginnote{3.1} painful practice with swift insight is said to be inferior because it’s painful. 

The\marginnote{4.1} pleasant practice with slow insight is said to be inferior because it’s slow. 

The\marginnote{5.1} pleasant practice with swift insight is said to be superior in both ways: because it’s pleasant, and because it’s swift. This practice is said to be superior in both ways. 

These\marginnote{5.3} are the four ways of practice.” 

%
\section*{{\suttatitleacronym AN 4.167}{\suttatitletranslation Moggallāna’s Practice }{\suttatitleroot Mahāmoggallānasutta}}
\addcontentsline{toc}{section}{\tocacronym{AN 4.167} \toctranslation{Moggallāna’s Practice } \tocroot{Mahāmoggallānasutta}}
\markboth{Moggallāna’s Practice }{Mahāmoggallānasutta}
\extramarks{AN 4.167}{AN 4.167}

Then\marginnote{1.1} Venerable \textsanskrit{Sāriputta} went up to Venerable \textsanskrit{Mahāmoggallāna}, and exchanged greetings with him. When the greetings and polite conversation were over, \textsanskrit{Sāriputta} sat down to one side and said to \textsanskrit{Mahāmoggallāna}: 

“Reverend\marginnote{2.1} \textsanskrit{Moggallāna}, there are four ways of practice. What four? 

\begin{enumerate}%
\item Painful practice with slow insight, %
\item painful practice with swift insight, %
\item pleasant practice with slow insight, and %
\item pleasant practice with swift insight. %
\end{enumerate}

These\marginnote{2.7} are the four ways of practice. Which one of these four ways of practice did you rely on to free your mind from defilements by not grasping?” 

“Reverend\marginnote{3.1} \textsanskrit{Sāriputta} … I relied on the painful practice with swift insight to free my mind from defilements by not grasping.” 

%
\section*{{\suttatitleacronym AN 4.168}{\suttatitletranslation Sāriputta’s Practice }{\suttatitleroot Sāriputtasutta}}
\addcontentsline{toc}{section}{\tocacronym{AN 4.168} \toctranslation{Sāriputta’s Practice } \tocroot{Sāriputtasutta}}
\markboth{Sāriputta’s Practice }{Sāriputtasutta}
\extramarks{AN 4.168}{AN 4.168}

Then\marginnote{1.1} Venerable \textsanskrit{Mahāmoggallāna} went up to Venerable \textsanskrit{Sāriputta}, and exchanged greetings with him. When the greetings and polite conversation were over, \textsanskrit{Mahāmoggallāna} sat down to one side, and said to \textsanskrit{Sāriputta}: 

“Reverend\marginnote{2.1} \textsanskrit{Sāriputta}, there are four ways of practice. What four? 

\begin{enumerate}%
\item Painful practice with slow insight, %
\item painful practice with swift insight, %
\item pleasant practice with slow insight, and %
\item pleasant practice with swift insight. %
\end{enumerate}

These\marginnote{2.7} are the four ways of practice. Which one of these four ways of practice did you rely on to free your mind from defilements by not grasping?” 

“Reverend\marginnote{3.1} \textsanskrit{Moggallāna} … I relied on the pleasant practice with swift insight to free my mind from defilements by not grasping.” 

%
\section*{{\suttatitleacronym AN 4.169}{\suttatitletranslation Extra Effort }{\suttatitleroot Sasaṅkhārasutta}}
\addcontentsline{toc}{section}{\tocacronym{AN 4.169} \toctranslation{Extra Effort } \tocroot{Sasaṅkhārasutta}}
\markboth{Extra Effort }{Sasaṅkhārasutta}
\extramarks{AN 4.169}{AN 4.169}

“Mendicants,\marginnote{1.1} these four people are found in the world. What four? 

\begin{enumerate}%
\item One person becomes fully extinguished in the present life by making extra effort. %
\item One person becomes fully extinguished when the body breaks up by making extra effort. %
\item One person becomes fully extinguished in the present life without making extra effort. %
\item One person becomes fully extinguished when the body breaks up without making extra effort. %
\end{enumerate}

And\marginnote{2.1} how does a person become fully extinguished in the present life by making extra effort? It’s when a mendicant meditates observing the ugliness of the body, perceives the repulsiveness of food, perceives dissatisfaction with the whole world, observes the impermanence of all conditions, and has well established the perception of their own death. They rely on these five powers of a trainee: faith, conscience, prudence, energy, and wisdom. And these five faculties manifest in them strongly: faith, energy, mindfulness, immersion, and wisdom. Because of the strength of the five faculties, they become fully extinguished in the present life by making extra effort. That’s how a person becomes fully extinguished in the present life by making extra effort. 

How\marginnote{3.1} does a person become fully extinguished when the body breaks up by making extra effort? It’s when a mendicant meditates observing the ugliness of the body, perceives the repulsiveness of food, perceives dissatisfaction with the whole world, observes the impermanence of all conditions, and has well established the perception of their own death. They rely on these five powers of a trainee: faith, conscience, prudence, energy, and wisdom. But these five faculties manifest in them weakly: faith, energy, mindfulness, immersion, and wisdom. Because of the weakness of the five faculties, they become fully extinguished when the body breaks up by making extra effort. That’s how a person becomes fully extinguished when the body breaks up by making extra effort. 

And\marginnote{4.1} how does a person become fully extinguished in the present life without making extra effort? It’s when a mendicant … enters and remains in the first absorption … second absorption … third absorption … fourth absorption … They rely on these five powers of a trainee: faith, conscience, prudence, energy, and wisdom. And these five faculties manifest in them strongly: faith, energy, mindfulness, immersion, and wisdom. Because of the strength of the five faculties, they become fully extinguished in the present life without making extra effort. That’s how a person becomes fully extinguished in the present life without making extra effort. 

And\marginnote{5.1} how does a person become fully extinguished when the body breaks up without making extra effort? It’s when a mendicant … enters and remains in the first absorption … second absorption … third absorption … fourth absorption … They rely on these five powers of a trainee: faith, conscience, prudence, energy, and wisdom. But these five faculties manifest in them weakly: faith, energy, mindfulness, immersion, and wisdom. Because of the weakness of the five faculties, they become fully extinguished when the body breaks up without making extra effort. That’s how a person becomes fully extinguished when the body breaks up without making extra effort. 

These\marginnote{5.8} are the four people found in the world.” 

%
\section*{{\suttatitleacronym AN 4.170}{\suttatitletranslation In Conjunction }{\suttatitleroot Yuganaddhasutta}}
\addcontentsline{toc}{section}{\tocacronym{AN 4.170} \toctranslation{In Conjunction } \tocroot{Yuganaddhasutta}}
\markboth{In Conjunction }{Yuganaddhasutta}
\extramarks{AN 4.170}{AN 4.170}

\scevam{So\marginnote{1.1} I have heard. }At one time Venerable Ānanda was staying near Kosambi, in Ghosita’s Monastery. There Ānanda addressed the mendicants: “Reverends, mendicants!” 

“Reverend,”\marginnote{1.5} they replied. Ānanda said this: 

“Reverends,\marginnote{2.1} all of the monks and nuns who declare in my presence that they have attained perfection, did so by one or other of four paths. 

What\marginnote{3.1} four? 

Take\marginnote{3.2} a mendicant who develops serenity before discernment. As they do so, the path is born in them. They cultivate, develop, and make much of it. By doing so, they give up the fetters and eliminate the underlying tendencies. 

Another\marginnote{4.1} mendicant develops discernment before serenity. As they do so, the path is born in them. They cultivate, develop, and make much of it. By doing so, they give up the fetters and eliminate the underlying tendencies. 

Another\marginnote{5.1} mendicant develops serenity and discernment in conjunction. As they do so, the path is born in them. They cultivate, develop, and make much of it. By doing so, they give up the fetters and eliminate the underlying tendencies. 

Another\marginnote{6.1} mendicant’s mind is seized by restlessness to realize the teaching. But there comes a time when their mind is stilled internally; it settles, unifies, and becomes immersed in \textsanskrit{samādhi}. The path is born in them. They cultivate, develop, and make much of it. By doing so, they give up the fetters and eliminate the underlying tendencies. 

All\marginnote{7.1} of the monks and nuns who declare in my presence that they have attained perfection, did so by one or other of these four paths.” 

%
\addtocontents{toc}{\let\protect\contentsline\protect\nopagecontentsline}
\chapter*{The Chapter on Intention }
\addcontentsline{toc}{chapter}{\tocchapterline{The Chapter on Intention }}
\addtocontents{toc}{\let\protect\contentsline\protect\oldcontentsline}

%
\section*{{\suttatitleacronym AN 4.171}{\suttatitletranslation Intention }{\suttatitleroot Cetanāsutta}}
\addcontentsline{toc}{section}{\tocacronym{AN 4.171} \toctranslation{Intention } \tocroot{Cetanāsutta}}
\markboth{Intention }{Cetanāsutta}
\extramarks{AN 4.171}{AN 4.171}

“Mendicants,\marginnote{1.1} as long as there’s a body, the intention that gives rise to bodily action causes pleasure and pain to arise in oneself. As long as there’s a voice, the intention that gives rise to verbal action causes pleasure and pain to arise in oneself. As long as there’s a mind, the intention that gives rise to mental action causes pleasure and pain to arise in oneself. But these only apply when conditioned by ignorance. 

By\marginnote{2.1} oneself one makes the choice that gives rise to bodily, verbal, and mental action, conditioned by which that pleasure and pain arise in oneself. Or else others make the choice … One consciously makes the choice … Or else one unconsciously makes the choice … 

Ignorance\marginnote{5.1} is included in all these things. But when ignorance fades away and ceases with nothing left over, there is no body and no voice and no mind, conditioned by which that pleasure and pain arise in oneself. There is no field, no ground, no scope, and no basis, conditioned by which that pleasure and pain arise in oneself. 

Mendicants,\marginnote{6.1} there are four kinds of reincarnation. What four? 

\begin{enumerate}%
\item There is a reincarnation where one’s own intention is effective, not that of others. %
\item There is a reincarnation where the intention of others is effective, not one’s own. %
\item There is a reincarnation where both one’s own and others’ intentions are effective. %
\item There is a reincarnation where neither one’s own nor others’ intentions are effective. %
\end{enumerate}

These\marginnote{6.7} are the four kinds of reincarnation.” 

When\marginnote{7.1} he said this, Venerable \textsanskrit{Sāriputta} said to the Buddha: 

“Sir,\marginnote{7.2} this is how I understand the detailed meaning of the Buddha’s brief statement. Take the case of the reincarnation where one’s own intention is effective, not that of others. Those sentient beings pass away from that realm due to their own intention. Take the case of the reincarnation where the intention of others is effective, not one’s own. Those sentient beings pass away from that realm due to the intention of others. Take the case of the reincarnation where both one’s own and others’ intentions are effective. Those sentient beings pass away from that realm due to both their own and others’ intentions. But sir, in the case of the reincarnation where neither one’s own nor others’ intentions are effective, what kind of gods does this refer to?” 

“\textsanskrit{Sāriputta},\marginnote{7.7} it refers to the gods reborn in the dimension of neither perception nor non-perception.” 

“What\marginnote{8.1} is the cause, sir, what is the reason why some sentient beings pass away from that realm as returners who come back to this state of existence, while others are non-returners who don’t come back?” 

“\textsanskrit{Sāriputta},\marginnote{8.3} take a person who hasn’t given up the lower fetters. In the present life they enter and abide in the dimension of neither perception nor non-perception. They enjoy it and like it and find it satisfying. If they abide in that, are committed to it, and meditate on it often without losing it, when they die they’re reborn in the company of the gods of the dimension of neither perception nor non-perception. When they pass away from there, they’re a returner, who comes back to this state of existence. 

\textsanskrit{Sāriputta},\marginnote{9.1} take a person who has given up the lower fetters. In the present life they enter and abide in the dimension of neither perception nor non-perception. They enjoy it and like it and find it satisfying. If they abide in that, are committed to it, and meditate on it often without losing it, when they die they’re reborn in the company of the gods of the dimension of neither perception nor non-perception. When they pass away from there, they’re a non-returner, not coming back to this state of existence. 

This\marginnote{10.1} is the cause, this is the reason why some sentient beings pass away from that realm as returners who come back to this state of existence, while others are non-returners who don’t come back.” 

%
\section*{{\suttatitleacronym AN 4.172}{\suttatitletranslation Sāriputta’s Attainment of Textual Analysis }{\suttatitleroot Vibhattisutta}}
\addcontentsline{toc}{section}{\tocacronym{AN 4.172} \toctranslation{Sāriputta’s Attainment of Textual Analysis } \tocroot{Vibhattisutta}}
\markboth{Sāriputta’s Attainment of Textual Analysis }{Vibhattisutta}
\extramarks{AN 4.172}{AN 4.172}

There\marginnote{1.1} \textsanskrit{Sāriputta} addressed the mendicants: “Reverends, mendicants!” 

“Reverend,”\marginnote{1.3} they replied. \textsanskrit{Sāriputta} said this: 

“Reverends,\marginnote{2.1} I realized the textual analysis of the meaning—piece by piece and expression by expression—a fortnight after I ordained. In many ways I explain, teach, assert, establish, clarify, analyze, and reveal it. Whoever has any doubt or uncertainty, let them ask me, I will answer. Our teacher is present, he who is so very skilled in our teachings. 

I\marginnote{3.1} realized the textual analysis of the text—piece by piece and expression by expression—a fortnight after I ordained. … 

I\marginnote{4.1} realized the textual analysis of terminology—piece by piece and expression by expression—a fortnight after I ordained. … 

I\marginnote{5.1} realized the textual analysis of eloquence—piece by piece and expression by expression—a fortnight after I ordained. In many ways I explain, teach, assert, establish, clarify, analyze, and reveal it. If anyone has any doubt or uncertainty, let them ask me, I will answer. Our teacher is present, he who is so very skilled in our teachings.” 

%
\section*{{\suttatitleacronym AN 4.173}{\suttatitletranslation With Mahākoṭṭhita }{\suttatitleroot Mahākoṭṭhikasutta}}
\addcontentsline{toc}{section}{\tocacronym{AN 4.173} \toctranslation{With Mahākoṭṭhita } \tocroot{Mahākoṭṭhikasutta}}
\markboth{With Mahākoṭṭhita }{Mahākoṭṭhikasutta}
\extramarks{AN 4.173}{AN 4.173}

Then\marginnote{1.1} Venerable \textsanskrit{Mahākoṭṭhita} went up to Venerable \textsanskrit{Sāriputta}, and exchanged greetings with him. When the greetings and polite conversation were over, \textsanskrit{Mahākoṭṭhita} sat down to one side, and said to \textsanskrit{Sāriputta}: 

“Reverend,\marginnote{2.1} when the six fields of contact have faded away and ceased with nothing left over, does something else exist?” 

“Don’t\marginnote{3.1} put it like that, reverend.” 

“Does\marginnote{4.1} nothing else exist?” 

“Don’t\marginnote{5.1} put it like that, reverend.” 

“Do\marginnote{6.1} both something else and nothing else exist?” 

“Don’t\marginnote{7.1} put it like that, reverend.” 

“Do\marginnote{8.1} neither something else nor nothing else exist?” 

“Don’t\marginnote{9.1} put it like that, reverend.” 

“Reverend,\marginnote{10.1} when asked whether—when the six fields of contact have faded away and ceased with nothing left over—something else exists, you say ‘don’t put it like that’. When asked whether nothing else exists, you say ‘don’t put it like that’. When asked whether both something else and nothing else exist, you say ‘don’t put it like that’. When asked whether neither something else nor nothing else exist, you say ‘don’t put it like that’. How then should we see the meaning of this statement?” 

“If\marginnote{11.1} you say that, ‘When the six fields of contact have faded away and ceased with nothing left over, something else exists’, you’re proliferating the unproliferated. If you say that ‘nothing else exists’, you’re proliferating the unproliferated. If you say that ‘both something else and nothing else exist’, you’re proliferating the unproliferated. If you say that ‘neither something else nor nothing else exists’, you’re proliferating the unproliferated. The scope of proliferation extends as far as the scope of the six fields of contact. The scope of the six fields of contact extends as far as the scope of proliferation. When the six fields of contact fade away and cease with nothing left over, proliferation stops and is stilled.” 

%
\section*{{\suttatitleacronym AN 4.174}{\suttatitletranslation With Ānanda }{\suttatitleroot Ānandasutta}}
\addcontentsline{toc}{section}{\tocacronym{AN 4.174} \toctranslation{With Ānanda } \tocroot{Ānandasutta}}
\markboth{With Ānanda }{Ānandasutta}
\extramarks{AN 4.174}{AN 4.174}

Then\marginnote{1.1} Venerable Ānanda went up to Venerable \textsanskrit{Mahākoṭṭhita}, and exchanged greetings with him. When the greetings and polite conversation were over, Ānanda sat down to one side, and said to \textsanskrit{Mahākoṭṭhita}: 

“Reverend,\marginnote{2.1} when these six fields of contact have faded away and ceased with nothing left over, does anything else exist?” 

“Don’t\marginnote{3.1} put it like that, reverend.” 

“Does\marginnote{4.1} nothing else exist?” 

“Don’t\marginnote{5.1} put it like that, reverend.” 

“Do\marginnote{6.1} both something else and nothing else exist?” 

“Don’t\marginnote{7.1} put it like that, reverend.” 

“Do\marginnote{8.1} neither something else nor nothing else exist?” 

“Don’t\marginnote{9.1} put it like that, reverend.” 

“Reverend,\marginnote{10.1} when asked these questions, you say ‘don’t put it like that’. … How then should we see the meaning of this statement?” 

“If\marginnote{11.1} you say that ‘when the six fields of contact have faded away and ceased with nothing left over, something else exists’, you’re proliferating the unproliferated. If you say that ‘nothing else exists’, you’re proliferating the unproliferated. If you say that ‘both something else and nothing else exist’, you’re proliferating the unproliferated. If you say that ‘neither something else nor nothing else exist’, you’re proliferating the unproliferated. The scope of proliferation extends as far as the scope of the six fields of contact. The scope of the six fields of contact extends as far as the scope of proliferation. When the six fields of contact fade away and cease with nothing left over, proliferation stops and is stilled.” 

%
\section*{{\suttatitleacronym AN 4.175}{\suttatitletranslation With Upavāṇa }{\suttatitleroot Upavāṇasutta}}
\addcontentsline{toc}{section}{\tocacronym{AN 4.175} \toctranslation{With Upavāṇa } \tocroot{Upavāṇasutta}}
\markboth{With Upavāṇa }{Upavāṇasutta}
\extramarks{AN 4.175}{AN 4.175}

Then\marginnote{1.1} Venerable \textsanskrit{Upavāṇa} went up to Venerable \textsanskrit{Sāriputta}, and exchanged greetings with him. When the greetings and polite conversation were over, \textsanskrit{Upavāṇa} sat down to one side, and said to \textsanskrit{Sāriputta}: 

“Reverend\marginnote{2.1} \textsanskrit{Sāriputta}, do you become a terminator because of knowledge?” 

“That’s\marginnote{3.1} not it, reverend.” 

“Do\marginnote{4.1} you become a terminator because of conduct?” 

“That’s\marginnote{5.1} not it, reverend.” 

“Do\marginnote{6.1} you become a terminator because of both knowledge and conduct?” 

“That’s\marginnote{7.1} not it, reverend.” 

“Do\marginnote{8.1} you become a terminator for some reason other than knowledge and conduct?” 

“That’s\marginnote{9.1} not it, reverend.” 

“Reverend\marginnote{10.1} \textsanskrit{Sāriputta}, when asked whether you become a terminator because of knowledge or conduct or knowledge and conduct, or for some other reason, you say ‘that’s not it’. How then do you become a terminator?” 

“Reverend,\marginnote{11.1} if you became a terminator because of knowledge, then even someone who still has grasping could be a terminator. If you became a terminator because of conduct, then even someone who still has grasping could be a terminator. If you became a terminator because of both knowledge and conduct, then even someone who still has grasping could be a terminator. If you became a terminator for some reason other than knowledge and conduct, then even an ordinary person could be a terminator. For an ordinary person lacks knowledge and conduct. Reverend, someone lacking good conduct does not know and see things as they are. Someone accomplished in good conduct knows and sees things as they are. Knowing and seeing things as they are, one is a terminator.” 

%
\section*{{\suttatitleacronym AN 4.176}{\suttatitletranslation Aspiration }{\suttatitleroot Āyācanasutta}}
\addcontentsline{toc}{section}{\tocacronym{AN 4.176} \toctranslation{Aspiration } \tocroot{Āyācanasutta}}
\markboth{Aspiration }{Āyācanasutta}
\extramarks{AN 4.176}{AN 4.176}

“Mendicants,\marginnote{1.1} a faithful monk would rightly aspire: ‘May I be like \textsanskrit{Sāriputta} and \textsanskrit{Moggallāna}!’ These are a standard and a measure for my monk disciples, that is, \textsanskrit{Sāriputta} and \textsanskrit{Moggallāna}. 

A\marginnote{2.1} faithful nun would rightly aspire: ‘May I be like the nuns \textsanskrit{Khemā} and \textsanskrit{Uppalavaṇṇā}!’ These are a standard and a measure for my nun disciples, that is, the nuns \textsanskrit{Khemā} and \textsanskrit{Uppalavaṇṇā}. 

A\marginnote{3.1} faithful layman would rightly aspire: ‘May I be like the householder Citta and Hatthaka of \textsanskrit{Ãḷavī}!’ These are a standard and a measure for my male lay disciples, that is, the householder Citta and Hatthaka of \textsanskrit{Ãḷavī}. 

A\marginnote{4.1} faithful laywoman would rightly aspire: ‘May I be like the laywomen \textsanskrit{Khujjuttarā} and \textsanskrit{Veḷukaṇṭakī}, Nanda’s mother!’ These are a standard and a measure for my female lay disciples, that is, the laywomen \textsanskrit{Khujjuttarā} and \textsanskrit{Veḷukaṇṭakī}, Nanda’s mother.” 

%
\section*{{\suttatitleacronym AN 4.177}{\suttatitletranslation With Rāhula }{\suttatitleroot Rāhulasutta}}
\addcontentsline{toc}{section}{\tocacronym{AN 4.177} \toctranslation{With Rāhula } \tocroot{Rāhulasutta}}
\markboth{With Rāhula }{Rāhulasutta}
\extramarks{AN 4.177}{AN 4.177}

Then\marginnote{1.1} Venerable \textsanskrit{Rāhula} went up to the Buddha, bowed, and sat down to one side. The Buddha said to him: 

“\textsanskrit{Rāhula},\marginnote{2.1} the interior earth element and the exterior earth element are just the earth element. This should be truly seen with right understanding like this: ‘This is not mine, I am not this, this is not my self.’ When you truly see with right understanding, you reject the earth element, detaching the mind from the earth element. 

The\marginnote{3.1} interior water element and the exterior water element are just the water element. This should be truly seen with right understanding like this: ‘This is not mine, I am not this, this is not my self.’ When you truly see with right understanding, you reject the water element, detaching the mind from the water element. 

The\marginnote{4.1} interior fire element and the exterior fire element are just the fire element. This should be truly seen with right understanding like this: ‘This is not mine, I am not this, this is not my self.’ When you truly see with right understanding, you reject the fire element, detaching the mind from the fire element. 

The\marginnote{5.1} interior air element and the exterior air element are just the air element. This should be truly seen with right understanding like this: ‘This is not mine, I am not this, this is not my self.’ When you truly see with right understanding, you reject the air element, detaching the mind from the air element. 

When\marginnote{6.1} a mendicant sees these four elements as neither self nor belonging to self, they’re called a mendicant who has cut off craving, untied the fetters, and by rightly comprehending conceit has made an end of suffering.” 

%
\section*{{\suttatitleacronym AN 4.178}{\suttatitletranslation Billabong }{\suttatitleroot Jambālīsutta}}
\addcontentsline{toc}{section}{\tocacronym{AN 4.178} \toctranslation{Billabong } \tocroot{Jambālīsutta}}
\markboth{Billabong }{Jambālīsutta}
\extramarks{AN 4.178}{AN 4.178}

“Mendicants,\marginnote{1.1} these four people are found in the world. What four? 

Take\marginnote{1.3} a mendicant who enters and remains in a peaceful release of the heart. They focus on the cessation of identification, but their mind isn’t eager, confident, settled, and decided about it. You wouldn’t expect that mendicant to stop identifying. Suppose a person were to grab a branch with a glue-smeared hand. Their hand would stick, hold, and bind to it. In the same way, take a mendicant who enters and remains in a peaceful release of the heart. They focus on the cessation of identification, but their mind isn’t eager, confident, settled, and decided about it. You wouldn’t expect that mendicant to stop identifying. 

Next,\marginnote{2.1} take a mendicant who enters and remains in a peaceful release of the heart. They focus on the cessation of identification, and their mind is eager, confident, settled, and decided about it. You’d expect that mendicant to stop identifying. Suppose a person were to grab a branch with a clean hand. Their hand wouldn’t stick, hold, or bind to it. In the same way, take a mendicant who enters and remains in a peaceful release of the heart. They focus on the cessation of identification, and their mind is eager, confident, settled, and decided about it. You’d expect that mendicant to stop identifying. 

Next,\marginnote{3.1} take a mendicant who enters and remains in a peaceful release of the heart. They focus on smashing ignorance, but their mind isn’t eager, confident, settled, and decided about it. You wouldn’t expect that mendicant to smash ignorance. Suppose there was a billabong that had been stagnant for many years. And someone was to close off the inlets and open up the drains, and the heavens didn’t provide enough rain. You wouldn’t expect that billabong to break its banks. In the same way, take a mendicant who enters and remains in a certain peaceful release of the heart. They focus on smashing ignorance, but their mind isn’t eager, confident, settled, and decided about it. You wouldn’t expect that mendicant to smash ignorance. 

Next,\marginnote{4.1} take a mendicant who enters and remains in a peaceful release of the heart. They focus on smashing ignorance, and their mind is eager, confident, settled, and decided about it. You’d expect that mendicant to smash ignorance. Suppose there was a billabong that had been stagnant for many years. And someone was to open up the inlets and close off the drains, and the heavens provided plenty of rain. You’d expect that billabong to break its banks. In the same way, take a mendicant who enters and remains in a certain peaceful release of the heart. They focus on smashing ignorance, and their mind is eager, confident, settled, and decided about it. You’d expect that mendicant to smash ignorance. 

These\marginnote{4.12} are the four people found in the world.” 

%
\section*{{\suttatitleacronym AN 4.179}{\suttatitletranslation Extinguishment }{\suttatitleroot Nibbānasutta}}
\addcontentsline{toc}{section}{\tocacronym{AN 4.179} \toctranslation{Extinguishment } \tocroot{Nibbānasutta}}
\markboth{Extinguishment }{Nibbānasutta}
\extramarks{AN 4.179}{AN 4.179}

Then\marginnote{1.1} Venerable Ānanda went up to Venerable \textsanskrit{Sāriputta}, and exchanged greetings with him. When the greetings and polite conversation were over, Ānanda sat down to one side, and said to \textsanskrit{Sāriputta}: 

“What\marginnote{1.3} is the cause, Reverend \textsanskrit{Sāriputta}, what is the reason why some sentient beings aren’t fully extinguished in the present life?” 

“Reverend\marginnote{2.1} Ānanda, it’s because some sentient beings don’t really understand which perceptions make things worse, which keep things steady, which lead to distinction, and which lead to penetration. That’s the cause, that’s the reason why some sentient beings aren’t fully extinguished in the present life.” 

“What\marginnote{3.1} is the cause, Reverend \textsanskrit{Sāriputta}, what is the reason why some sentient beings are fully extinguished in the present life?” 

“Reverend\marginnote{3.2} Ānanda, it’s because some sentient beings truly understand which perceptions make things worse, which keep things steady, which lead to distinction, and which lead to penetration. That’s the cause, that’s the reason why some sentient beings are fully extinguished in the present life.” 

%
\section*{{\suttatitleacronym AN 4.180}{\suttatitletranslation The Four Great References }{\suttatitleroot Mahāpadesasutta}}
\addcontentsline{toc}{section}{\tocacronym{AN 4.180} \toctranslation{The Four Great References } \tocroot{Mahāpadesasutta}}
\markboth{The Four Great References }{Mahāpadesasutta}
\extramarks{AN 4.180}{AN 4.180}

At\marginnote{1.1} one time the Buddha was staying near the city of Bhoga, at the Ānanda Tree-shrine. 

There\marginnote{1.2} the Buddha addressed the mendicants, “Mendicants!” 

“Venerable\marginnote{1.4} sir,” they replied. 

The\marginnote{1.5} Buddha said this: “Mendicants, I will teach you the four great references. Listen and pay close attention, I will speak.” 

“Yes,\marginnote{1.8} sir,” they replied. The Buddha said this: 

“Mendicants,\marginnote{2.1} what are the four great references? 

Take\marginnote{2.2} a mendicant who says: ‘Reverend, I have heard and learned this in the presence of the Buddha: this is the teaching, this is the monastic law, this is the Teacher’s instruction.’ You should neither approve nor dismiss that mendicant’s statement. Instead, you should carefully memorize those words and phrases, then check if they’re included in the discourses and found in the monastic law. If they’re not included in the discourses and found in the monastic law, you should draw the conclusion: ‘Clearly this is not the word of the Blessed One, the perfected one, the fully awakened Buddha. It has been incorrectly memorized by that mendicant.’ And so you should reject it. 

Take\marginnote{3.1} another mendicant who says: ‘Reverend, I have heard and learned this in the presence of the Buddha: this is the teaching, this is the monastic law, this is the Teacher’s instruction.’ You should neither approve nor dismiss that mendicant’s statement. Instead, you should carefully memorize those words and phrases, then check if they’re included in the discourses and found in the monastic law. If they are included in the discourses and found in the monastic law, you should draw the conclusion: ‘Clearly this is the word of the Blessed One, the perfected one, the fully awakened Buddha. It has been correctly memorized by that mendicant.’ You should remember it. This is the first great reference. 

Take\marginnote{4.1} another mendicant who says: ‘In such-and-such monastery lives a \textsanskrit{Saṅgha} with seniors and leaders. I’ve heard and learned this in the presence of that \textsanskrit{Saṅgha}: this is the teaching, this is the monastic law, this is the Teacher’s instruction.’ You should neither approve nor dismiss that mendicant’s statement. Instead, you should carefully memorize those words and phrases, then check if they’re included in the discourses or found in the monastic law. If they’re not included in the discourses or found in the monastic law, you should draw the conclusion: ‘Clearly this is not the word of the Blessed One, the perfected one, the fully awakened Buddha. It has been incorrectly memorized by that \textsanskrit{Saṅgha}.’ And so you should reject it. 

Take\marginnote{5.1} another mendicant who says: ‘In such-and-such monastery lives a \textsanskrit{Saṅgha} with seniors and leaders. I’ve heard and learned this in the presence of that \textsanskrit{Saṅgha}: this is the teaching, this is the monastic law, this is the Teacher’s instruction.’ You should neither approve nor dismiss that mendicant’s statement. Instead, you should carefully memorize those words and phrases, then check if they’re included in the discourses or found in the monastic law. If they are included in the discourses and found in the monastic law, you should draw the conclusion: ‘Clearly this is the word of the Blessed One, the perfected one, the fully awakened Buddha. It has been correctly memorized by that \textsanskrit{Saṅgha}.’ You should remember it. This is the second great reference. 

Take\marginnote{6.1} another mendicant who says: ‘In such-and-such monastery there are several senior mendicants who are very learned, knowledgeable in the scriptures, who remember the teachings, the monastic law, and the outlines. I’ve heard and learned this in the presence of those senior mendicants: this is the teaching, this is the monastic law, this is the Teacher’s instruction.’ You should neither approve nor dismiss that mendicant’s statement. Instead, you should carefully memorize those words and phrases, then check if they’re included in the discourses or found in the monastic law. If they’re not included in the discourses or found in the monastic law, you should draw the conclusion: ‘Clearly this is not the word of the Blessed One, the perfected one, the fully awakened Buddha. It has been incorrectly memorized by those senior mendicants.’ And so you should reject it. 

Take\marginnote{7.1} another mendicant who says: ‘In such-and-such monastery there are several senior mendicants who are very learned, knowledgeable in the scriptures, who remember the teachings, the monastic law, and the outlines. I’ve heard and learned this in the presence of those senior mendicants: this is the teaching, this is the monastic law, this is the Teacher’s instruction.’ You should neither approve nor dismiss that mendicant’s statement. Instead, you should carefully memorize those words and phrases, then check if they’re included in the discourses and found in the monastic law. If they are included in the discourses and found in the monastic law, you should draw the conclusion: ‘Clearly this is the word of the Blessed One, the perfected one, the fully awakened Buddha. It has been correctly memorized by those senior mendicants.’ You should remember it. This is the third great reference. 

Take\marginnote{8.1} another mendicant who says: ‘In such-and-such monastery there is a single senior mendicant who is very learned and knowledgeable in the scriptures, who has memorized the teachings, the monastic law, and the outlines. I’ve heard and learned this in the presence of that senior mendicant: this is the teaching, this is the monastic law, this is the Teacher’s instruction.’ You should neither approve nor dismiss that mendicant’s statement. Instead, you should carefully memorize those words and phrases, then check if they’re included in the discourses and found in the monastic law. If they’re not included in the discourses or found in the monastic law, you should draw the conclusion: ‘Clearly this is not the word of the Blessed One, the perfected one, the fully awakened Buddha. It has been incorrectly memorized by that senior mendicant.’ And so you should reject it. 

Take\marginnote{9.1} another mendicant who says: ‘In such-and-such monastery there is a single senior mendicant who is very learned and knowledgeable in the scriptures, who has memorized the teachings, the monastic law, and the outlines. I’ve heard and learned this in the presence of that senior mendicant: this is the teaching, this is the monastic law, this is the Teacher’s instruction.’ You should neither approve nor dismiss that mendicant’s statement. Instead, you should carefully memorize those words and phrases, then check if they’re included in the discourses and found in the monastic law. If they are included in the discourses and found in the monastic law, you should draw the conclusion: ‘Clearly this is the word of the Blessed One, the perfected one, the fully awakened Buddha. It has been correctly memorized by that senior mendicant.’ You should remember it. This is the fourth great reference. 

These\marginnote{9.11} are the four great references.” 

%
\addtocontents{toc}{\let\protect\contentsline\protect\nopagecontentsline}
\chapter*{The Chapter on Brahmins }
\addcontentsline{toc}{chapter}{\tocchapterline{The Chapter on Brahmins }}
\addtocontents{toc}{\let\protect\contentsline\protect\oldcontentsline}

%
\section*{{\suttatitleacronym AN 4.181}{\suttatitletranslation A Warrior }{\suttatitleroot Yodhājīvasutta}}
\addcontentsline{toc}{section}{\tocacronym{AN 4.181} \toctranslation{A Warrior } \tocroot{Yodhājīvasutta}}
\markboth{A Warrior }{Yodhājīvasutta}
\extramarks{AN 4.181}{AN 4.181}

“Mendicants,\marginnote{1.1} a warrior with four factors is worthy of a king, fit to serve a king, and is considered a factor of kingship. What four? He’s skilled in the basics, a long-distance shooter, a marksman, one who shatters large objects. A warrior with these four factors is worthy of a king, fit to serve a king, and is considered a factor of kingship. 

In\marginnote{1.5} the same way, a mendicant with four qualities is worthy of offerings dedicated to the gods, worthy of hospitality, worthy of a religious donation, worthy of veneration with joined palms, and is the supreme field of merit for the world. What four? He’s skilled in the basics, a long-distance shooter, a marksman, one who shatters large objects. 

And\marginnote{2.1} how is a mendicant skilled in the basics? It’s when a mendicant is ethical, restrained in the code of conduct, conducting themselves well and seeking alms in suitable places. Seeing danger in the slightest fault, they keep the rules they’ve undertaken. That’s how a mendicant is skilled in the basics. 

And\marginnote{3.1} how is a mendicant a long-distance shooter? It’s when a mendicant truly sees any kind of form at all—past, future, or present; internal or external; coarse or fine; inferior or superior; far or near: \emph{all} form—with right understanding: ‘This is not mine, I am not this, this is not my self.’ They truly see any kind of feeling … perception … choices … consciousness at all—past, future, or present; internal or external; coarse or fine; inferior or superior; far or near, \emph{all} consciousness—with right understanding: ‘This is not mine, I am not this, this is not my self.’ That’s how a mendicant is a long-distance shooter. 

And\marginnote{4.1} how is a mendicant a marksman? It’s when they truly understand: ‘This is suffering’ … ‘This is the origin of suffering’ … ‘This is the cessation of suffering’ … ‘This is the practice that leads to the cessation of suffering’. That’s how a mendicant is a marksman. 

And\marginnote{5.1} how does a mendicant shatter large objects? It’s when a mendicant shatters the great mass of ignorance. That’s how a mendicant shatters large objects. 

A\marginnote{5.4} mendicant with these four qualities … is the supreme field of merit for the world.” 

%
\section*{{\suttatitleacronym AN 4.182}{\suttatitletranslation Guarantee }{\suttatitleroot Pāṭibhogasutta}}
\addcontentsline{toc}{section}{\tocacronym{AN 4.182} \toctranslation{Guarantee } \tocroot{Pāṭibhogasutta}}
\markboth{Guarantee }{Pāṭibhogasutta}
\extramarks{AN 4.182}{AN 4.182}

“There\marginnote{1.1} are four things that no-one can guarantee—not an ascetic, a brahmin, a god, a \textsanskrit{Māra}, a \textsanskrit{Brahmā}, or anyone in the world. 

What\marginnote{2.1} four? No-one can guarantee that someone liable to old age will not grow old. No-one can guarantee that someone liable to sickness will not get sick. No-one can guarantee that someone liable to death will not die. No-one can guarantee that the bad deeds done in past lives—corrupting, leading to future lives, hurtful, resulting in suffering and future rebirth, old age, and death—will not produce their result. 

These\marginnote{3.1} are the four things that no-one can guarantee—not an ascetic, a brahmin, a god, a \textsanskrit{Māra}, a \textsanskrit{Brahmā}, or anyone in the world.” 

%
\section*{{\suttatitleacronym AN 4.183}{\suttatitletranslation Vassakāra on What is Heard }{\suttatitleroot Sutasutta}}
\addcontentsline{toc}{section}{\tocacronym{AN 4.183} \toctranslation{Vassakāra on What is Heard } \tocroot{Sutasutta}}
\markboth{Vassakāra on What is Heard }{Sutasutta}
\extramarks{AN 4.183}{AN 4.183}

At\marginnote{1.1} one time the Buddha was staying near \textsanskrit{Rājagaha}, in the Bamboo Grove, the squirrels’ feeding ground. Then \textsanskrit{Vassakāra} the brahmin, a chief minister of Magadha, went up to the Buddha, and exchanged greetings with him. When the greetings and polite conversation were over, he sat down to one side and said to the Buddha: 

“Master\marginnote{2.1} Gotama, this is my doctrine and view: There’s nothing wrong with talking about what you’ve seen, saying: ‘So I have seen.’ There’s nothing wrong with talking about what you’ve heard, saying: ‘So I have heard.’ There’s nothing wrong with talking about what you’ve thought, saying: ‘So I have thought.’ There’s nothing wrong with talking about what you’ve known, saying: ‘So I have known.’” 

“Brahmin,\marginnote{3.1} I don’t say you should talk about everything you see, hear, think, and know. But I also don’t say you should talk about nothing you see, hear, think, and know. 

When\marginnote{4.1} talking about certain things you’ve seen, heard, thought, or known, unskillful qualities grow while skillful qualities decline. I say that you shouldn’t talk about those things. When talking about other things you’ve seen, heard, thought, or known, unskillful qualities decline while skillful qualities grow. I say that you should talk about those things.” 

Then\marginnote{8.1} \textsanskrit{Vassakāra} the brahmin, having approved and agreed with what the Buddha said, got up from his seat and left. 

%
\section*{{\suttatitleacronym AN 4.184}{\suttatitletranslation Fearless }{\suttatitleroot Abhayasutta}}
\addcontentsline{toc}{section}{\tocacronym{AN 4.184} \toctranslation{Fearless } \tocroot{Abhayasutta}}
\markboth{Fearless }{Abhayasutta}
\extramarks{AN 4.184}{AN 4.184}

Then\marginnote{1.1} the brahmin \textsanskrit{Jāṇussoṇi} went up to the Buddha, and exchanged greetings with him. When the greetings and polite conversation were over, he sat down to one side and said to the Buddha: 

“Master\marginnote{2.1} Gotama, this is my doctrine and view: ‘All those liable to death are frightened and terrified of death.’” 

“Brahmin,\marginnote{2.3} some of those liable to death are frightened and terrified of death. But some of those liable to death are not frightened and terrified of death. 

Who\marginnote{3.1} are those frightened of death? It’s someone who isn’t free of greed, desire, fondness, thirst, passion, and craving for sensual pleasures. When they fall seriously ill, they think: ‘The sensual pleasures that I love so much will leave me, and I’ll leave them.’ They sorrow and wail and lament, beating their breast and falling into confusion. This is someone who is frightened of death. 

Furthermore,\marginnote{4.1} it’s someone who isn’t free of greed, desire, fondness, thirst, passion, and craving for the body. When they fall seriously ill, they think: ‘This body that I love so much will leave me, and I’ll leave it.’ They sorrow and wail and lament, beating their breast and falling into confusion. This, too, is someone who is frightened of death. 

Furthermore,\marginnote{5.1} it’s someone who hasn’t done good and skillful things that keep them safe, but has done bad, violent, and depraved things. When they fall seriously ill, they think: ‘Well, I haven’t done good and skillful things that keep me safe. And I have done bad, violent, and depraved things. When I depart, I’ll go to the place where people who’ve done such things go.’ They sorrow and wail and lament, beating their breast and falling into confusion. This, too, is someone who is frightened of death. 

Furthermore,\marginnote{6.1} it’s someone who’s doubtful, uncertain, and undecided about the true teaching. When they fall seriously ill, they think: ‘I’m doubtful, uncertain, and undecided about the true teaching.’ They sorrow and wail and lament, beating their breast and falling into confusion. This, too, is someone who is frightened of death. These are the four people liable to death who are frightened and terrified of death. 

Who\marginnote{7.1} are those not frightened of death? 

It’s\marginnote{7.2} someone who is rid of greed, desire, fondness, thirst, passion, and craving for sensual pleasures. When they fall seriously ill, they don’t think: ‘The sensual pleasures that I love so much will leave me, and I’ll leave them.’ They don’t sorrow and wail and lament, beating their breast and falling into confusion. This is someone who’s not frightened of death. 

Furthermore,\marginnote{8.1} it’s someone who is rid of greed, desire, fondness, thirst, passion, and craving for the body. When they fall seriously ill, they don’t think: ‘This body that I love so much will leave me, and I’ll leave it.’ They don’t sorrow and wail and lament, beating their breast and falling into confusion. This, too, is someone who’s not frightened of death. 

Furthermore,\marginnote{9.1} it’s someone who hasn’t done bad, violent, and corrupt deeds, but has done good and skillful deeds that keep them safe. When they fall seriously ill, they think: ‘Well, I haven’t done bad, violent, and depraved things. And I have done good and skillful deeds that keep me safe. When I depart, I’ll go to the place where people who’ve done such things go.’ They don’t sorrow and wail and lament, beating their breast and falling into confusion. This, too, is someone who’s not frightened of death. 

Furthermore,\marginnote{10.1} it’s someone who’s not doubtful, uncertain, or undecided about the true teaching. When they fall seriously ill, they think: ‘I’m not doubtful, uncertain, or undecided about the true teaching.’ They don’t sorrow and wail and lament, beating their breast and falling into confusion. This, too, is someone who’s not frightened of death. 

These\marginnote{10.7} are the four people liable to death who are not frightened and terrified of death.” 

“Excellent,\marginnote{11.1} Master Gotama! … From this day forth, may Master Gotama remember me as a lay follower who has gone for refuge for life.” 

%
\section*{{\suttatitleacronym AN 4.185}{\suttatitletranslation Truths of the Brahmins }{\suttatitleroot Brāhmaṇasaccasutta}}
\addcontentsline{toc}{section}{\tocacronym{AN 4.185} \toctranslation{Truths of the Brahmins } \tocroot{Brāhmaṇasaccasutta}}
\markboth{Truths of the Brahmins }{Brāhmaṇasaccasutta}
\extramarks{AN 4.185}{AN 4.185}

Once\marginnote{1.1} the Buddha was staying near \textsanskrit{Rājagaha}, on the Vulture’s Peak Mountain. 

Now\marginnote{1.2} at that time several very well-known wanderers were residing in the monastery of the wanderers on the bank of the \textsanskrit{Sappinī} river. They included \textsanskrit{Annabhāra}, Varadhara, \textsanskrit{Sakuludāyī}, and other very well-known wanderers. 

Then\marginnote{1.3} in the late afternoon, the Buddha came out of retreat and went to the wanderer’s monastery on the bank of the \textsanskrit{Sappinī} river. 

Now\marginnote{2.1} at that time this discussion came up while those wanderers who follow other paths were sitting together, “The truths of the brahmins are like this; the truths of the brahmins are like that.” 

Then\marginnote{2.3} the Buddha went up to those wanderers, sat down on the seat spread out, and said to them, “Wanderers, what were you sitting talking about just now? What conversation was left unfinished?” 

“Well,\marginnote{3.2} Master Gotama, this discussion came up among us while we were sitting together: ‘The truths of the brahmins are like this; the truths of the brahmins are like that.’” 

“Wanderers,\marginnote{4.1} I declare these four truths of the brahmins, having realized them with my own insight. What four? 

Take\marginnote{4.3} a brahmin who says: ‘No sentient beings should be killed.’ Saying this, a brahmin speaks the truth, not lies. But they don’t think of themselves as an ‘ascetic’ or ‘brahmin’ because of that. Nor do they think ‘I’m better’ or ‘I’m equal’ or ‘I’m worse’. Rather, they simply practice out of kindness and compassion for living creatures, having had insight into the truth of that. 

Take\marginnote{5.1} another brahmin who says: ‘All sensual pleasures are impermanent, suffering, and perishable.’ Saying this, a brahmin speaks the truth, not lies. But they don’t think of themselves as an ‘ascetic’ or ‘brahmin’ because of that. Nor do they think ‘I’m better’ or ‘I’m equal’ or ‘I’m worse’. Rather, they simply practice for disillusionment, dispassion, and cessation regarding sensual pleasures, having had insight into the truth of that. 

Take\marginnote{6.1} another brahmin who says: ‘All states of existence are impermanent, suffering, and perishable.’ … They simply practice for disillusionment, dispassion, and cessation regarding future lives, having had insight into the truth of that. 

Take\marginnote{7.1} another brahmin who says: ‘I don’t belong to anyone anywhere. And nothing belongs to me anywhere.’ Saying this, a brahmin speaks the truth, not lies. But they don’t think of themselves as an ‘ascetic’ or ‘brahmin’ because of that. Nor do they think ‘I’m better’ or ‘I’m equal’ or ‘I’m worse’. Rather, they simply practice the path of nothingness, having had insight into the truth of that. 

These\marginnote{7.6} are the four truths of the brahmins that I declare, having realized them with my own insight.” 

%
\section*{{\suttatitleacronym AN 4.186}{\suttatitletranslation Approach }{\suttatitleroot Ummaggasutta}}
\addcontentsline{toc}{section}{\tocacronym{AN 4.186} \toctranslation{Approach } \tocroot{Ummaggasutta}}
\markboth{Approach }{Ummaggasutta}
\extramarks{AN 4.186}{AN 4.186}

Then\marginnote{1.1} one of the mendicants went up to the Buddha, bowed, sat down to one side, and said to him: 

“Sir,\marginnote{1.2} what leads the world on? What drags it around? What arises and takes control?” 

“Good,\marginnote{2.1} good, mendicant! Your approach and articulation are excellent, and it’s a good question. For you asked: ‘What leads the world on? What drags it around? What arises and takes control?’” 

“Yes,\marginnote{2.5} sir.” 

“Mendicant,\marginnote{2.6} the mind leads the world on. The mind drags it around. When the mind arises, it takes control.” 

Saying\marginnote{3.1} “Good, sir”, that mendicant approved and agreed with what the Buddha said. Then he asked another question: 

“Sir,\marginnote{3.2} they speak of ‘a learned memorizer of the teaching’. How is a learned memorizer of the teaching defined?” 

“Good,\marginnote{4.1} good, mendicant! Your approach and articulation are excellent, and it’s a good question. … I have taught many teachings: statements, songs, discussions, verses, inspired exclamations, legends, stories of past lives, amazing stories, and classifications. But if anyone understands the meaning and the text of even a four-line verse, and if they practice in line with that teaching, they’re qualified to be called a ‘learned memorizer of the teaching’.” 

Saying\marginnote{5.1} “Good, sir”, that mendicant approved and agreed with what the Buddha said. Then he asked another question: 

“Sir,\marginnote{5.2} they speak of ‘a learned person with penetrating wisdom’. How is a learned person with penetrating wisdom defined?” 

“Good,\marginnote{6.1} good, mendicant! Your approach and articulation are excellent, and it’s a good question. … Take a mendicant who has heard: ‘This is suffering.’ They see what it means with penetrating wisdom. They’ve heard: ‘This is the origin of suffering’ … ‘This is the cessation of suffering’ … ‘This is the practice that leads to the cessation of suffering.’ They see what it means with penetrating wisdom. That’s how a person is learned, with penetrating wisdom.” 

Saying\marginnote{7.1} “Good, sir”, that mendicant approved and agreed with what the Buddha said. Then he asked another question: 

“Sir,\marginnote{7.2} they speak of ‘an astute person with great wisdom’. How is an astute person with great wisdom defined?” 

“Good,\marginnote{8.1} good, mendicant! Your approach and articulation are excellent, and it’s a good question. … An astute person with great wisdom is one who has no intention to hurt themselves, or to hurt others, or to hurt both. When they think, they only think of the benefit for themselves, for others, for both, and for the whole world. That’s how a person is astute, with great wisdom.” 

%
\section*{{\suttatitleacronym AN 4.187}{\suttatitletranslation With Vassakāra }{\suttatitleroot Vassakārasutta}}
\addcontentsline{toc}{section}{\tocacronym{AN 4.187} \toctranslation{With Vassakāra } \tocroot{Vassakārasutta}}
\markboth{With Vassakāra }{Vassakārasutta}
\extramarks{AN 4.187}{AN 4.187}

At\marginnote{1.1} one time the Buddha was staying near \textsanskrit{Rājagaha}, in the Bamboo Grove, the squirrels’ feeding ground. 

Then\marginnote{1.2} \textsanskrit{Vassakāra} the brahmin, a chief minister of Magadha, went up to the Buddha, and exchanged greetings with him. When the greetings and polite conversation were over, he sat down to one side and said to the Buddha: 

“Master\marginnote{2.1} Gotama, could a bad person know of a bad person: ‘This fellow is a bad person’?” 

“That’s\marginnote{2.3} impossible, brahmin, it can’t happen.” 

“Could\marginnote{2.5} a bad person know of a good person: ‘This fellow is a good person’?” 

“That\marginnote{2.7} too is impossible, it can’t happen.” 

“Master\marginnote{2.9} Gotama, could a good person know of a good person: ‘This fellow is a good person’?” 

“That,\marginnote{2.11} brahmin, is possible.” 

“Could\marginnote{2.13} a good person know of a bad person: ‘This fellow is a bad person’?” 

“That\marginnote{2.15} too is possible.” 

“It’s\marginnote{3.1} incredible, Master Gotama, it’s amazing, how well said this was by Master Gotama: ‘It’s impossible, it can’t happen, that a bad person could know … But it is possible that a good person could know …’ 

Once,\marginnote{4.1} members of the brahmin Todeyya’s assembly were going on complaining about others: ‘This King \textsanskrit{Eḷeyya} is a fool to be so devoted to \textsanskrit{Rāmaputta}. He even shows him the utmost deference by bowing down to him, rising up for him, greeting him with joined palms, and observing proper etiquette for him. And these king’s men are fools too—Yamaka, Moggalla, Ugga, \textsanskrit{Nāvindakī}, Gandhabba, and Aggivessa—for they show the same kind of deference to \textsanskrit{Rāmaputta}.’ Then the brahmin Todeyya reasoned with them like this: ‘What do you think, sirs? When it comes to the various duties and speeches, isn’t King \textsanskrit{Eḷeyya} astute, even better than the experts?’ ‘That’s true, sir.’ 

‘It’s\marginnote{5.1} because \textsanskrit{Rāmaputta} is even more astute and expert than King \textsanskrit{Eḷeyya} that the king is so devoted to him. That’s why he even shows \textsanskrit{Rāmaputta} the utmost deference by bowing down to him, rising up for him, greeting him with joined palms, and observing proper etiquette for him. 

What\marginnote{6.1} do you think, sirs? When it comes to the various duties and speeches, aren’t the king’s men—Yamaka, Moggalla, Ugga, \textsanskrit{Nāvindakī}, Gandhabba, and Aggivessa—astute, even better than the experts?’ ‘That’s true, sir.’ 

‘It’s\marginnote{7.1} because \textsanskrit{Rāmaputta} is even more astute and expert than the king’s men that they have such devotion to him. … It’s because \textsanskrit{Rāmaputta} is even more astute and expert than King \textsanskrit{Eḷeyya} that the king is so devoted to him. That’s why he even shows \textsanskrit{Rāmaputta} the utmost deference by bowing down to him, rising up for him, greeting him with joined palms, and observing proper etiquette for him.’ 

It’s\marginnote{8.1} incredible, Master Gotama, it’s amazing, how well said this was by Master Gotama: ‘It’s impossible, it can’t happen, that a bad person could know … But it is possible that a good person could know … Well, now, Master Gotama, I must go. I have many duties, and much to do.” 

“Please,\marginnote{8.13} brahmin, go at your convenience.” 

Then\marginnote{9.1} \textsanskrit{Vassakāra} the brahmin, having approved and agreed with what the Buddha said, got up from his seat and left. 

%
\section*{{\suttatitleacronym AN 4.188}{\suttatitletranslation With Upaka }{\suttatitleroot Upakasutta}}
\addcontentsline{toc}{section}{\tocacronym{AN 4.188} \toctranslation{With Upaka } \tocroot{Upakasutta}}
\markboth{With Upaka }{Upakasutta}
\extramarks{AN 4.188}{AN 4.188}

Once\marginnote{1.1} the Buddha was staying near \textsanskrit{Rājagaha}, on the Vulture’s Peak Mountain. Then Upaka the son of \textsanskrit{Maṇḍikā} went up to the Buddha, bowed, sat down to one side, and said to him: 

“Sir,\marginnote{2.1} this is my doctrine and view: ‘Whoever goes on complaining about others without giving any reasons is reprehensible and at fault.’” 

“Upaka,\marginnote{2.4} if someone goes on complaining about others without giving a reason, they’re reprehensible and at fault. But that’s what you do, so you’re reprehensible and at fault!” 

“Sir,\marginnote{2.6} like a fish caught in a big trap just as it rises, so the Buddha caught me in a big trap of words just as I rose up.” 

“Upaka,\marginnote{3.1} I’ve declared: ‘This is unskillful.’ And there are limitless words, phrases, and teachings of the Realized One about that: ‘This is another way of saying that this is unskillful.’ I’ve declared: ‘The unskillful should be given up.’ And there are limitless words, phrases, and teachings of the Realized One about that: ‘This is another way of saying that the unskillful should be given up.’ 

I’ve\marginnote{4.1} declared that: ‘This is skillful.’ And there are limitless words, phrases, and teachings of the Realized One about that: ‘This is another way of saying that this is skillful.’ I’ve declared: ‘The skillful should be developed.’ And there are limitless words, phrases, and teachings of the Realized One about that: ‘This is another way of saying that the skillful should be developed.’” 

And\marginnote{5.1} then Upaka son of \textsanskrit{Maṇḍikā} approved and agreed with what the Buddha said. He got up from his seat, bowed, and respectfully circled the Buddha, keeping him on his right. Then he went up to King \textsanskrit{Ajātasattu} Vedehiputta of Magadha. He told the King of all they had discussed. 

But\marginnote{5.2} \textsanskrit{Ajātasattu} became angry and upset, and said to Upaka, “How rude of this salt-maker’s boy! How scurrilous and impudent of him to imagine he could attack the Blessed One, the perfected one, the fully awakened Buddha! Get out, Upaka, go away! Don’t let me see you again.” 

%
\section*{{\suttatitleacronym AN 4.189}{\suttatitletranslation Things to be Realized }{\suttatitleroot Sacchikaraṇīyasutta}}
\addcontentsline{toc}{section}{\tocacronym{AN 4.189} \toctranslation{Things to be Realized } \tocroot{Sacchikaraṇīyasutta}}
\markboth{Things to be Realized }{Sacchikaraṇīyasutta}
\extramarks{AN 4.189}{AN 4.189}

“Mendicants,\marginnote{1.1} these four things should be realized. What four? 

There\marginnote{1.3} are things to be realized with direct meditative experience. There are things to be realized with recollection. There are things to be realized with vision. There are things to be realized with wisdom. 

What\marginnote{1.7} things are to be realized with direct meditative experience? The eight liberations. 

What\marginnote{2.1} things are to be realized with recollection? Past lives. 

What\marginnote{3.1} things are to be realized with vision? The passing away and rebirth of sentient beings. 

What\marginnote{4.1} things are to be realized with wisdom? The ending of defilements. 

These\marginnote{4.3} are the four things to be realized.” 

%
\section*{{\suttatitleacronym AN 4.190}{\suttatitletranslation Sabbath }{\suttatitleroot Uposathasutta}}
\addcontentsline{toc}{section}{\tocacronym{AN 4.190} \toctranslation{Sabbath } \tocroot{Uposathasutta}}
\markboth{Sabbath }{Uposathasutta}
\extramarks{AN 4.190}{AN 4.190}

At\marginnote{1.1} one time the Buddha was staying near \textsanskrit{Sāvatthī} in the Eastern Monastery, the stilt longhouse of \textsanskrit{Migāra}’s mother. 

Now,\marginnote{1.2} at that time it was the sabbath, and the Buddha was sitting surrounded by the \textsanskrit{Saṅgha} of monks. Then the Buddha looked around the \textsanskrit{Saṅgha} of monks, who were so very silent. He addressed them: 

“This\marginnote{2.1} assembly has no nonsense, mendicants, it’s free of nonsense. It consists purely of the essential core. Such is this \textsanskrit{Saṅgha} of monks, such is this assembly! An assembly such as this is rarely seen in the world. An assembly such as this is worthy of offerings dedicated to the gods, worthy of hospitality, worthy of a religious donation, worthy of greeting with joined palms, and is the supreme field of merit for the world. Even a small gift to an assembly such as this is plentiful, while giving more is even more plentiful. An assembly such as this is worth traveling many leagues to see, even if you have to carry your own provisions in a shoulder bag. 

There\marginnote{3.1} are monks staying in this \textsanskrit{Saṅgha} who have attained to the gods. There are monks staying in this \textsanskrit{Saṅgha} who have attained to \textsanskrit{Brahmā}. There are monks staying in this \textsanskrit{Saṅgha} who have attained to the imperturbable. There are monks staying in this \textsanskrit{Saṅgha} who have attained to nobility. 

And\marginnote{4.1} how has a monk attained to the gods? It’s when a monk, quite secluded from sensual pleasures, secluded from unskillful qualities, enters and remains in the first absorption … As the placing of the mind and keeping it connected are stilled, they enter and remain in the second absorption … third absorption … fourth absorption … That’s how a monk has attained to the gods. 

And\marginnote{5.1} how has a monk attained to \textsanskrit{Brahmā}? Firstly, a monk meditates spreading a heart full of love to one direction, and to the second, and to the third, and to the fourth. In the same way above, below, across, everywhere, all around, they spread a heart full of love to the whole world—abundant, expansive, limitless, free of enmity and ill will. Furthermore, a monk meditates spreading a heart full of compassion … rejoicing … equanimity to one direction, and to the second, and to the third, and to the fourth. In the same way above, below, across, everywhere, all around, they spread a heart full of equanimity to the whole world—abundant, expansive, limitless, free of enmity and ill will. That’s how a monk has attained to \textsanskrit{Brahmā}. 

And\marginnote{6.1} how has a monk attained to the imperturbable? It’s when a monk—going totally beyond perceptions of form, with the ending of perceptions of impingement, not focusing on perceptions of diversity—aware that ‘space is infinite’, enters and remains in the dimension of infinite space. Going totally beyond the dimension of infinite space, aware that ‘consciousness is infinite’, he enters and remains in the dimension of infinite consciousness. Going totally beyond the dimension of infinite consciousness, aware that ‘there is nothing at all’, he enters and remains in the dimension of nothingness. Going totally beyond the dimension of nothingness, he enters and remains in the dimension of neither perception nor non-perception. That’s how a monk has attained to the imperturbable. 

And\marginnote{7.1} how has a monk attained to nobility? It’s when they truly understand: ‘This is suffering’ … ‘This is the origin of suffering’ … ‘This is the cessation of suffering’ … ‘This is the practice that leads to the cessation of suffering’. That’s how a monk has attained to nobility.” 

%
\addtocontents{toc}{\let\protect\contentsline\protect\nopagecontentsline}
\chapter*{The Great Chapter }
\addcontentsline{toc}{chapter}{\tocchapterline{The Great Chapter }}
\addtocontents{toc}{\let\protect\contentsline\protect\oldcontentsline}

%
\section*{{\suttatitleacronym AN 4.191}{\suttatitletranslation Followed by Ear }{\suttatitleroot Sotānugatasutta}}
\addcontentsline{toc}{section}{\tocacronym{AN 4.191} \toctranslation{Followed by Ear } \tocroot{Sotānugatasutta}}
\markboth{Followed by Ear }{Sotānugatasutta}
\extramarks{AN 4.191}{AN 4.191}

“Mendicants,\marginnote{1.1} you can expect four benefits when the teachings have been followed by ear, reinforced by recitation, examined by the mind, and well comprehended theoretically. What four? 

Take\marginnote{1.3} a mendicant who memorizes the teaching—statements, songs, discussions, verses, inspired exclamations, legends, stories of past lives, amazing stories, and classifications. They’ve followed those teachings by ear, reinforced them by recitation, examined them by the mind, and well comprehended them theoretically. But they die unmindful and are reborn in one of the orders of gods. Being happy there, passages of the teaching come back to them. Memory comes up slowly, but then that being quickly reaches distinction. This is the first benefit you can expect when the teachings have been followed by ear, reinforced by recitation, examined by the mind, and well comprehended theoretically. 

Take\marginnote{2.1} another mendicant who memorizes the teaching—statements, songs, discussions, verses, inspired exclamations, legends, stories of past lives, amazing stories, and classifications. They’ve followed those teachings by ear, reinforced them by recitation, examined them by the mind, and well comprehended them theoretically. But they die unmindful and are reborn in one of the orders of gods. Though they’re happy there, passages of the teaching don’t come back to them. However, a mendicant with psychic powers, who has achieved mastery of the mind, teaches Dhamma to the assembly of gods. They think: ‘I used to lead the spiritual life in this same teaching and training.’ Memory comes up slowly, but then that being quickly reaches distinction. Suppose a person was skilled in the sound of drums. While traveling along a road they hear the sound of drums. They wouldn’t have any doubts or uncertainties about whether that was the sound of drums or not. They’d just conclude, ‘That’s the sound of drums.’ In the same way, take another mendicant who memorizes the teaching … But they die unmindful and are reborn in one of the orders of gods. … Memory comes up slowly, but then that being quickly reaches distinction. This is the second benefit you can expect when the teachings have been followed by ear, reinforced by recitation, examined by the mind, and well comprehended theoretically. 

Take\marginnote{3.1} another mendicant who memorizes the teaching—statements, songs, discussions, verses, inspired exclamations, legends, stories of past lives, amazing stories, and classifications. They’ve followed those teachings by ear, reinforced them by recitation, examined them by the mind, and well comprehended them theoretically. But they die unmindful and are reborn in one of the orders of gods. But passages of the teaching don’t come back to them when they’re happy, nor does a mendicant with psychic powers … teach Dhamma to the assembly of gods. However, a god teaches Dhamma to the assembly of gods. They think: ‘I used to lead the spiritual life in this same teaching and training.’ Memory comes up slowly, but then that being quickly reaches distinction. Suppose a person was skilled in the sound of horns. While traveling along a road they hear the sound of horns. They wouldn’t have any doubt about whether that was the sound of horns or not. They’d just conclude, ‘That’s the sound of horns.’ In the same way, take another mendicant who memorizes the teaching … But they die unmindful and are reborn in one of the orders of gods. … Memory comes up slowly, but then that being quickly reaches distinction. This is the third benefit you can expect when the teachings have been followed by ear, reinforced by recitation, examined by the mind, and well comprehended theoretically. 

Take\marginnote{4.1} another mendicant who memorizes the teaching—statements, songs, discussions, verses, inspired exclamations, legends, stories of past lives, amazing stories, and classifications. They’ve followed those teachings by ear, reinforced them by recitation, examined them by the mind, and well comprehended them theoretically. But they die unmindful and are reborn in one of the orders of gods. But passages of the teaching don’t come back to them when they’re happy, and neither a mendicant with psychic powers … nor a god teaches Dhamma to the assembly of gods. But a being who has been reborn spontaneously reminds another such being: ‘Do you remember, good sir? Do you remember where we used to lead the spiritual life?’ He says: ‘I remember, good sir, I remember!’ Memory comes up slowly, but then that being quickly reaches distinction. Suppose there were two friends who had played together in the sand. Some time or other they’d meet. And one friend would say to the other: ‘Do you remember this, friend? Do you remember that, friend?’ They’d say: ‘I remember, friend, I remember!’ In the same way, take another mendicant who memorizes the teaching … But they die unmindful and are reborn in one of the orders of gods. … Memory comes up slowly, but then that being quickly reaches distinction. This is the fourth benefit you can expect when the teachings have been followed by ear, reinforced by recitation, examined by the mind, and well comprehended theoretically. 

You\marginnote{4.30} can expect these four benefits when the teachings have been followed by ear, reinforced by recitation, examined by the mind, and well comprehended theoretically.” 

%
\section*{{\suttatitleacronym AN 4.192}{\suttatitletranslation Facts }{\suttatitleroot Ṭhānasutta}}
\addcontentsline{toc}{section}{\tocacronym{AN 4.192} \toctranslation{Facts } \tocroot{Ṭhānasutta}}
\markboth{Facts }{Ṭhānasutta}
\extramarks{AN 4.192}{AN 4.192}

“Mendicants,\marginnote{1.1} these four things can be known in four situations. What four? 

You\marginnote{1.3} can get to know a person’s ethics by living with them. But only after a long time, not casually; only when paying attention, not when inattentive; and only by the wise, not the witless. 

You\marginnote{1.4} can get to know a person’s purity by dealing with them. … 

You\marginnote{1.5} can get to know a person’s resilience in times of trouble. … 

You\marginnote{1.6} can get to know a person’s wisdom by discussion. But only after a long time, not casually; only when paying attention, not when inattentive; and only by the wise, not the witless. 

‘You\marginnote{2.1} can get to know a person’s ethics by living with them. But only after a long time, not casually; only when paying attention, not when inattentive; and only by the wise, not the witless.’ That’s what I said, but why did I say it? Take a person who’s living with someone else. They come to know: ‘For a long time this venerable’s deeds have been broken, tainted, spotty, and marred. Their deeds and behavior are inconsistent. This venerable is unethical, not ethical.’ 

Take\marginnote{3.1} another person who’s living with someone else. They come to know: ‘For a long time this venerable’s deeds have been unbroken, impeccable, spotless, and unmarred. Their deeds and behavior are consistent. This venerable is ethical, not unethical.’ That’s why I said that you can get to know a person’s ethics by living with them. But only after a long time, not a short time; only when paying attention, not when inattentive; and only by the wise, not the witless. 

‘You\marginnote{4.1} can get to know a person’s purity by dealing with them. …’ That’s what I said, but why did I say it? Take a person who has dealings with someone else. They come to know: ‘This venerable deals with one person in one way. Then they deal with two, three, or many people each in different ways. They’re not consistent from one deal to the next. This venerable’s dealings are impure, not pure.’ 

Take\marginnote{5.1} another person who has dealings with someone else. They come to know: ‘This venerable deals with one person in one way. Then they deal with two, three, or many people each in the same way. They’re consistent from one deal to the next. This venerable’s dealings are pure, not impure.’ That’s why I said that you can get to know a person’s purity by dealing with them. … 

‘You\marginnote{6.1} can get to know a person’s resilience in times of trouble. …’ That’s what I said, but why did I say it? Take a person who experiences loss of family, wealth, or health. But they don’t reflect: ‘The world’s like that. Reincarnation’s like that. That’s why the eight worldly conditions revolve around the world, and the world revolves around the eight worldly conditions: gain and loss, fame and disgrace, blame and praise, pleasure and pain.’ They sorrow and wail and lament, beating their breast and falling into confusion. 

Take\marginnote{7.1} another person who experiences loss of family, wealth, or health. But they reflect: ‘The world’s like that. Reincarnation’s like that. That’s why the eight worldly conditions revolve around the world, and the world revolves around the eight worldly conditions: gain and loss, fame and disgrace, blame and praise, pleasure and pain.’ They don’t sorrow or wail or lament, beating their breast and falling into confusion. That’s why I said that you can get to know a person’s resilience in times of trouble. … 

‘You\marginnote{8.1} can get to know a person’s wisdom by discussion. But only after a long time, not casually; only when paying attention, not when inattentive; and only by the wise, not the witless.’ That’s what I said, but why did I say it? Take a person who is discussing with someone else. They come to know: ‘Judging by this venerable’s approach, by what they’re getting at, and by how they discuss a question, they’re witless, not wise. Why is that? This venerable does not utter a deep and meaningful saying that is peaceful, sublime, beyond the scope of logic, subtle, comprehensible to the astute. When this venerable speaks on Dhamma they’re not able to explain the meaning, either briefly or in detail. They can’t teach it, assert it, establish it, clarify it, analyze it, or reveal it. This venerable is witless, not wise.’ 

Suppose\marginnote{9.1} a person with good eyesight was standing on the bank of a lake. They’d see a little fish rising, and think: ‘Judging by this fish’s approach, by the ripples it makes, and by its force, it’s a little fish, not a big one.’ In the same way, a person who is discussing with someone else would come to know: ‘Judging by this venerable’s approach, by what they’re getting at, and by how they discuss a question, they’re witless, not wise. …’ 

Take\marginnote{10.1} another person who is discussing with someone else. They come to know: ‘Judging by this venerable’s approach, by what they’re getting at, and by how they discuss a question, they’re wise, not witless. Why is that? This venerable utters a deep and meaningful saying that is peaceful, sublime, beyond the scope of logic, subtle, comprehensible to the astute. When this venerable speaks on Dhamma they’re able to explain the meaning, either briefly or in detail. They teach it, assert it, establish it, clarify it, analyze it, and reveal it. This venerable is wise, not witless.’ 

Suppose\marginnote{11.1} a man with good eyesight was standing on the bank of a lake. He’d see a big fish rising, and think: ‘Judging by this fish’s approach, by the ripples it makes, and by its force, it’s a big fish, not a little one.’ In the same way, a person who is discussing with someone else would come to know: ‘Judging by this venerable’s approach, by what they’re getting at, and by how they articulate a question, they’re wise, not witless. …’ 

That’s\marginnote{12.1} why I said that you can get to know a person’s wisdom by discussion. But only after a long time, not casually; only when paying attention, not when inattentive; and only by the wise, not the witless. 

These\marginnote{12.2} are the four things that can be known in four situations.” 

%
\section*{{\suttatitleacronym AN 4.193}{\suttatitletranslation With Bhaddiya }{\suttatitleroot Bhaddiyasutta}}
\addcontentsline{toc}{section}{\tocacronym{AN 4.193} \toctranslation{With Bhaddiya } \tocroot{Bhaddiyasutta}}
\markboth{With Bhaddiya }{Bhaddiyasutta}
\extramarks{AN 4.193}{AN 4.193}

At\marginnote{1.1} one time the Buddha was staying near \textsanskrit{Vesālī}, at the Great Wood, in the hall with the peaked roof. Then Bhaddiya the Licchavi went up to the Buddha, bowed, sat down to one side, and said to him: 

“Sir,\marginnote{2.1} I have heard this: ‘The ascetic Gotama is a magician. He knows a conversion magic, and uses it to convert the disciples of those who follow other paths.’ I trust that those who say this repeat what the Buddha has said, and do not misrepresent him with an untruth? Is their explanation in line with the teaching? Are there any legitimate grounds for rebuke and criticism?” 

“Please,\marginnote{3.1} Bhaddiya, don’t go by oral transmission, don’t go by lineage, don’t go by testament, don’t go by canonical authority, don’t rely on logic, don’t rely on inference, don’t go by reasoned contemplation, don’t go by the acceptance of a view after consideration, don’t go by the appearance of competence, and don’t think ‘The ascetic is our respected teacher.’ But when you know for yourselves: ‘These things are unskillful, blameworthy, criticized by sensible people, and when you undertake them, they lead to harm and suffering’, then you should give them up. 

What\marginnote{4.1} do you think, Bhaddiya? Does greed come up in a person for their welfare or harm?” 

“Harm,\marginnote{4.3} sir.” 

“A\marginnote{4.4} greedy individual—overcome by greed—kills living creatures, steals, commits adultery, lies, and encourages others to do the same. Is that for their lasting harm and suffering?” 

“Yes,\marginnote{4.5} sir.” 

“What\marginnote{5.1} do you think, Bhaddiya? Does hate … or delusion … or aggression come up in a person for their welfare or harm?” 

“Harm,\marginnote{5.3} sir.” 

“An\marginnote{5.4} aggressive individual kills living creatures, steals, commits adultery, lies, and encourages others to do the same. Is that for their lasting harm and suffering?” 

“Yes,\marginnote{5.5} sir.” 

“What\marginnote{6.1} do you think, Bhaddiya, are these things skillful or unskillful?” 

“Unskillful,\marginnote{6.2} sir.” 

“Blameworthy\marginnote{6.3} or blameless?” 

“Blameworthy,\marginnote{6.4} sir.” 

“Criticized\marginnote{6.5} or praised by sensible people?” 

“Criticized\marginnote{6.6} by sensible people, sir.” 

“When\marginnote{6.7} you undertake them, do they lead to harm and suffering, or not? Or how do you see this?” 

“When\marginnote{6.9} you undertake them, they lead to harm and suffering. That’s how we see it.” 

“So,\marginnote{7.1} Bhaddiya, when we said: ‘Please, Bhaddiya, don’t go by oral transmission, don’t go by lineage, don’t go by testament, don’t go by canonical authority, don’t rely on logic, don’t rely on inference, don’t go by reasoned contemplation, don’t go by the acceptance of a view after consideration, don’t go by the appearance of competence, and don’t think “The ascetic is our respected teacher.” But when you know for yourselves: “These things are unskillful, blameworthy, criticized by sensible people, and when you undertake them, they lead to harm and suffering”, then you should give them up.’ That’s what I said, and this is why I said it. 

Please,\marginnote{8.1} Bhaddiya, don’t rely on oral transmission … But when you know for yourselves: ‘These things are skillful, blameless, praised by sensible people, and when you undertake them, they lead to welfare and happiness’, then you should acquire them and keep them. 

What\marginnote{9.1} do you think, Bhaddiya? Does contentment … love … understanding … benevolence come up in a person for their welfare or harm?” 

“Welfare,\marginnote{10.2} sir.” 

“An\marginnote{10.3} individual who is benevolent—not overcome by aggression—doesn’t kill living creatures, steal, commit adultery, lie, or encourage others to do the same. Is that for their lasting welfare and happiness?” 

“Yes,\marginnote{10.4} sir.” 

“What\marginnote{11.1} do you think, Bhaddiya, are these things skillful or unskillful?” 

“Skillful,\marginnote{11.2} sir.” 

“Blameworthy\marginnote{11.3} or blameless?” 

“Blameless,\marginnote{11.4} sir.” 

“Criticized\marginnote{11.5} or praised by sensible people?” 

“Praised\marginnote{11.6} by sensible people, sir.” 

“When\marginnote{11.7} you undertake them, do they lead to welfare and happiness, or not? Or how do you see this?” 

“When\marginnote{11.9} you undertake them, they lead to welfare and happiness. That’s how we see it.” 

“So,\marginnote{12.1} Bhaddiya, when we said: ‘Please, Bhaddiya, don’t go by oral transmission, don’t go by lineage, don’t go by testament, don’t go by canonical authority, don’t rely on logic, don’t rely on inference, don’t go by reasoned contemplation, don’t go by the acceptance of a view after consideration, don’t go by the appearance of competence, and don’t think “The ascetic is our respected teacher.” But when you know for yourselves: “These things are skillful, blameless, praised by sensible people, and when you undertake them, they lead to welfare and happiness”, then you should acquire them and keep them.’ That’s what I said, and this is why I said it. 

The\marginnote{13.1} good people in the world encourage their disciples: ‘Please, mister, live rid of greed. Then you won’t act out of greed by way of body, speech, or mind. Live rid of hate … delusion … aggression. Then you won’t act out of hate … delusion … aggression by way of body, speech, or mind.” 

When\marginnote{14.1} he said this, Bhaddiya the Licchavi said to the Buddha, “Excellent, sir! … From this day forth, may the Buddha remember me as a lay follower who has gone for refuge for life.” 

“Well,\marginnote{15.1} Bhaddiya, did I say to you: ‘Please, Bhaddiya, be my disciple, and I will be your teacher’?” 

“No,\marginnote{15.4} sir.” 

“Though\marginnote{15.5} I speak and explain like this, certain ascetics and brahmins misrepresent me with the false, hollow, lying, untruthful claim: ‘The ascetic Gotama is a magician. He knows a conversion magic, and uses it to convert the disciples of those who follow other paths.’” 

“Sir,\marginnote{15.7} this conversion magic is excellent. This conversion magic is lovely! If my loved ones—relatives and kin—were to be converted by this, it would be for their lasting welfare and happiness. If all the aristocrats, brahmins, merchants, and workers were to be converted by this, it would be for their lasting welfare and happiness.” 

“That’s\marginnote{16.1} so true, Bhaddiya! That’s so true, Bhaddiya! If all the aristocrats, brahmins, merchants, and workers were to be converted by this, it would be for their lasting welfare and happiness. If the whole world—with its gods, \textsanskrit{Māras} and \textsanskrit{Brahmās}, this population with its ascetics and brahmins, gods and humans—were to be converted by this, for giving up unskillful qualities and embracing skillful qualities, it would be for their lasting welfare and happiness. If these great sal trees were to be converted by this, for giving up unskillful qualities and embracing skillful qualities, it would be for their lasting welfare and happiness—if they were sentient. How much more then a human being!” 

%
\section*{{\suttatitleacronym AN 4.194}{\suttatitletranslation At Sāpūga }{\suttatitleroot Sāmugiyasutta}}
\addcontentsline{toc}{section}{\tocacronym{AN 4.194} \toctranslation{At Sāpūga } \tocroot{Sāmugiyasutta}}
\markboth{At Sāpūga }{Sāmugiyasutta}
\extramarks{AN 4.194}{AN 4.194}

At\marginnote{1.1} one time Venerable Ānanda was staying in the land of the Koliyans, where they have a town named \textsanskrit{Sāpūga}. Then several Koliyans from \textsanskrit{Sāpūga} went up to Ānanda, bowed, and sat down to one side. Then Venerable Ānanda said to them: 

“Byagghapajjas,\marginnote{2.1} these four factors of trying to be pure have been rightly explained by the Blessed One, who knows and sees, the perfected one, the fully awakened Buddha. They are in order to purify sentient beings, to get past sorrow and crying, to make an end of pain and sadness, to end the cycle of suffering, and to realize extinguishment. What four? The factors of trying to be pure in ethics, mind, view, and freedom. 

And\marginnote{3.1} what is the factor of trying to be pure in ethics? It’s when a mendicant is ethical, restrained in the code of conduct, conducting themselves well and seeking alms in suitable places. Seeing danger in the slightest fault, they keep the rules they’ve undertaken. This is called purity of ethics. They think: ‘I will fulfill such purity of ethics, or, if it’s already fulfilled, I’ll support it in every situation by wisdom.’ Their enthusiasm for that—their effort, zeal, vigor, perseverance, mindfulness, and situational awareness—is called the factor of trying to be pure in ethics. 

And\marginnote{4.1} what is the factor of trying to be pure in mind? It’s when a mendicant, quite secluded from sensual pleasures, secluded from unskillful qualities, enters and remains in the first absorption … second absorption … third absorption … fourth absorption. This is called purity of mind. They think: ‘I will fulfill such purity of mind, or, if it’s already fulfilled, I’ll support it in every situation by wisdom.’ Their enthusiasm for that—their effort, zeal, vigor, perseverance, mindfulness, and situational awareness—is called the factor of trying to be pure in mind. 

And\marginnote{5.1} what is the factor of trying to be pure in view? Take a mendicant who truly understands: ‘This is suffering’ … ‘This is the origin of suffering’ … ‘This is the cessation of suffering’ … ‘This is the practice that leads to the cessation of suffering’. This is called purity of view. They think: ‘I will fulfill such purity of view, or, if it’s already fulfilled, I’ll support it in every situation by wisdom.’ Their enthusiasm for that—their effort, zeal, vigor, perseverance, mindfulness, and situational awareness—is called the factor of trying to be pure in view. 

And\marginnote{6.1} what is the factor of trying to be pure in freedom? That noble disciple—who has these factors of trying to be pure in ethics, mind, and view—detaches their mind from things that arouse greed, and frees their mind from things that it should be freed from. Doing so, they experience perfect freedom. This is called purity of freedom. They think: ‘I will fulfill such purity of freedom, or, if it’s already fulfilled, I’ll support it in every situation by wisdom.’ Their enthusiasm for that—their effort, zeal, vigor, perseverance, mindfulness, and situational awareness—is called the factor of trying to be pure in freedom. 

These\marginnote{7.1} four factors of trying to be pure have been rightly explained by the Blessed One, who knows and sees, the perfected one, the fully awakened Buddha. They are in order to purify sentient beings, to get past sorrow and crying, to make an end of pain and sadness, to end the cycle of suffering, and to realize extinguishment.” 

%
\section*{{\suttatitleacronym AN 4.195}{\suttatitletranslation With Vappa }{\suttatitleroot Vappasutta}}
\addcontentsline{toc}{section}{\tocacronym{AN 4.195} \toctranslation{With Vappa } \tocroot{Vappasutta}}
\markboth{With Vappa }{Vappasutta}
\extramarks{AN 4.195}{AN 4.195}

At\marginnote{1.1} one time the Buddha was staying in the land of the Sakyans, near Kapilavatthu in the Banyan Tree Monastery. Then Vappa of the Sakyans, a disciple of the Jains, went up to Venerable \textsanskrit{Mahāmoggallāna}, bowed, and sat down to one side. \textsanskrit{Mahāmoggallāna} said to him: 

“Vappa,\marginnote{2.1} take a person who is restrained in body, speech, and mind. When ignorance fades away and knowledge arises, do you see any reason why defilements giving rise to painful feelings would defile that person in the next life?” 

“Sir,\marginnote{2.3} I do see such a case. Take a person who did bad deeds in a past life. But the result of that has not yet ripened. For this reason defilements giving rise to painful feelings would defile that person in the next life.” But this conversation between \textsanskrit{Mahāmoggallāna} and Vappa was left unfinished. 

Then\marginnote{3.1} in the late afternoon, the Buddha came out of retreat and went to the assembly hall. He sat down on the seat spread out, and said to \textsanskrit{Mahāmoggallāna}, “\textsanskrit{Moggallāna}, what were you sitting talking about just now? What conversation was left unfinished?” 

\textsanskrit{Moggallāna}\marginnote{4.2} repeated the entire conversation to the Buddha, and concluded: “This was my conversation with Vappa that was unfinished when the Buddha arrived.” 

Then\marginnote{5.1} the Buddha said to Vappa, “Vappa, we can discuss this. But only if you allow what should be allowed, and reject what should be rejected. And if you ask me the meaning of anything you don’t understand, saying: ‘Sir, why is this? What’s the meaning of that?’” 

“Sir,\marginnote{5.4} let us discuss this. I will do as you say.” 

“What\marginnote{6.1} do you think, Vappa? There are distressing and feverish defilements that arise because of instigating bodily activity. These don’t occur in someone who avoids such bodily activity. They don’t perform any new deeds, and old deeds are eliminated by experiencing their results little by little. This wearing away is visible in this very life, immediately effective, inviting inspection, relevant, so that sensible people can know it for themselves. Do you see any reason why defilements giving rise to painful feelings would defile that person in the next life?” 

“No,\marginnote{6.5} sir.” 

“What\marginnote{7.1} do you think, Vappa? There are distressing and feverish defilements that arise because of instigating verbal activity. These don’t occur in someone who avoids such verbal activity. They don’t perform any new deeds, and old deeds are eliminated by experiencing their results little by little. This wearing away is visible in this very life, immediately effective, inviting inspection, relevant, so that sensible people can know it for themselves. Do you see any reason why defilements giving rise to painful feelings would defile that person in the next life?” 

“No,\marginnote{7.6} sir.” 

“What\marginnote{8.1} do you think, Vappa? There are distressing and feverish defilements that arise because of instigating mental activity. These don’t occur in someone who avoids such mental activity. They don’t perform any new deeds, and old deeds are eliminated by experiencing their results little by little. This wearing away is visible in this very life, immediately effective, inviting inspection, relevant, so that sensible people can know it for themselves. Do you see any reason why defilements giving rise to painful feelings would defile that person in the next life?” 

“No,\marginnote{8.6} sir.” 

“What\marginnote{9.1} do you think, Vappa? There are distressing and feverish defilements that arise because of ignorance. These don’t occur when ignorance fades away and knowledge arises. They don’t perform any new deeds, and old deeds are eliminated by experiencing their results little by little. This wearing away is visible in this very life, immediately effective, inviting inspection, relevant, so that sensible people can know it for themselves. Do you see any reason why defilements giving rise to painful feelings would defile that person in the next life?” 

“No,\marginnote{9.6} sir.” 

“A\marginnote{10.1} mendicant whose mind is rightly freed like this has achieved six consistent responses. Seeing a sight with the eye, they’re neither happy nor sad, but remain equanimous, mindful and aware. Hearing a sound with the ears … Smelling an odor with the nose … Tasting a flavor with the tongue … Feeling a touch with the body … Knowing a thought with the mind, they’re neither happy nor sad, but remain equanimous, mindful and aware. Feeling the end of the body approaching, they understand: ‘I feel the end of the body approaching.’ Feeling the end of life approaching, they understand: ‘I feel the end of life approaching.’ They understand: ‘When my body breaks up and my life has come to an end, everything that’s felt, being no longer relished, will become cool right here.’ 

Suppose\marginnote{11.1} there was a shadow cast by a sacrificial post. Then along comes a person with a spade and basket. They cut down the sacrificial post at its base, dig it up, and pull it out by its roots, right down to the fibers and stems. Then they split it apart, cut up the parts, and chop them into splinters. Next they dry the splinters in the wind and sun, burn them with fire, and reduce them to ashes. Then they sweep away the ashes in a strong wind, or float them away down a swift stream. And so the shadow cast by the post is cut off at the root, made like a palm stump, obliterated, and unable to arise in the future. 

In\marginnote{12.1} the same way, a mendicant whose mind is rightly freed like this has achieved six consistent responses. Seeing a sight with the eye, they’re neither happy nor sad, but remain equanimous, mindful and aware. Hearing a sound with the ears … Smelling an odor with the nose … Tasting a flavor with the tongue … Feeling a touch with the body … Knowing a thought with the mind, they’re neither happy nor sad, but remain equanimous, mindful and aware. Feeling the end of the body approaching, they understand: ‘I feel the end of the body approaching.’ Feeling the end of life approaching, they understand: ‘I feel the end of life approaching.’ They understand: ‘When my body breaks up and my life has come to an end, everything that’s felt, being no longer relished, will become cool right here.’” 

When\marginnote{13.1} he said this, Vappa the Sakyan, the disciple of the Jains, said to the Buddha: 

“Sir,\marginnote{13.2} suppose there was a man who raised commercial horses for profit. But he never made any profit, and instead just got weary and frustrated. In the same way, I paid homage to those Jain fools for profit. But I never made any profit, and instead just got weary and frustrated. From this day forth, any confidence I had in those Jain fools I sweep away as in a strong wind, or float away as down a swift stream. 

Excellent,\marginnote{13.7} sir! … From this day forth, may the Buddha remember me as a lay follower who has gone for refuge for life.” 

%
\section*{{\suttatitleacronym AN 4.196}{\suttatitletranslation With Sāḷha }{\suttatitleroot Sāḷhasutta}}
\addcontentsline{toc}{section}{\tocacronym{AN 4.196} \toctranslation{With Sāḷha } \tocroot{Sāḷhasutta}}
\markboth{With Sāḷha }{Sāḷhasutta}
\extramarks{AN 4.196}{AN 4.196}

At\marginnote{1.1} one time the Buddha was staying near \textsanskrit{Vesālī}, at the Great Wood, in the hall with the peaked roof. Then \textsanskrit{Sāḷha} and Abhaya the Licchavis went up to the Buddha, bowed, sat down to one side, and said to him: 

“There\marginnote{2.1} are, sir, some ascetics and brahmins who advocate crossing the flood by means of two things: purification of ethics, and mortification in disgust of sin. What does the Buddha say about this?” 

“\textsanskrit{Sāḷha},\marginnote{3.1} purification of ethics is one of the factors of the ascetic life, I say. But those ascetics and brahmins who teach mortification in disgust of sin—regarding it as essential and clinging to it—are incapable of crossing the flood. And those ascetics and brahmins whose livelihood and behavior by way of body, speech, and mind, is not pure are also incapable of knowing and seeing, of supreme awakening. 

Suppose\marginnote{4.1} a man who wanted to cross a river took a sharp axe into a wood. There he’d see a large green sal tree, straight and young and grown free of defects. He’d cut it down at the base, cut off the top, and completely strip off the branches and foliage. Then he’d trim it with axes and machetes, plane it, and sand it with a rock. Finally, he’d launch out on the river. 

What\marginnote{5.1} do you think, \textsanskrit{Sāḷha}? Is that man capable of crossing the river?” 

“No,\marginnote{5.3} sir. Why not? Because that green sal tree is well worked on the outside, but inside it’s still not cleared out. I’d expect that green sal tree to sink, and the man to come to ruin.” 

“In\marginnote{6.1} the same way, \textsanskrit{Sāḷha}, those ascetics and brahmins who teach mortification in disgust of sin—regarding it as essential and clinging to it—are incapable of crossing the flood. And those ascetics and brahmins whose livelihood and behavior by way of body, speech, and mind is not pure are also incapable of knowing and seeing, of supreme awakening. 

But\marginnote{7.1} those ascetics and brahmins who don’t teach mortification in disgust of sin—not regarding it as essential or clinging to it—are capable of crossing the flood. And those ascetics and brahmins whose behavior by way of body, speech, and mind is pure are also capable of knowing and seeing, of supreme awakening. 

Suppose\marginnote{8.1} a man who wanted to cross a river took a sharp axe into a wood. There he’d see a large green sal tree, straight and young and grown free of defects. He’d cut it down at the base, cut off the top, and completely strip off the branches and foliage. Then he’d trim it with axes and machetes. Then he’d take a chisel and completely clear it out inside. Then he’d plane it, sand it with a rock, and make it into a boat. Finally he’d fix it with oars and rudder, and launch out on the river. 

What\marginnote{9.1} do you think, \textsanskrit{Sāḷha}? Is that man capable of crossing the river?” 

“Yes,\marginnote{9.3} sir. Why is that? Because that green sal tree is well worked on the outside, cleared out on the inside, made into a boat, and fixed with oars and rudder. I’d expect that boat to not sink, and the man to safely make it to the far shore.” 

“In\marginnote{10.1} the same way, \textsanskrit{Sāḷha}, those ascetics and brahmins who don’t teach mortification in disgust of sin—not regarding it as essential or clinging to it—are capable of crossing the flood. And those ascetics and brahmins whose behavior by way of body, speech, and mind is pure are also capable of knowing and seeing, of supreme awakening. Suppose there was a warrior who knew lots of fancy archery tricks. It is only with these three factors that he becomes worthy of a king, fit to serve a king, and is considered a factor of kingship. What three? He’s a long-distance shooter, a marksman, and one who shatters large objects. 

Just\marginnote{11.1} as a warrior is a long-distance shooter, a noble disciple has right immersion. A noble disciple with right immersion truly sees any kind of form at all—past, future, or present; internal or external; coarse or fine; inferior or superior; far or near: \emph{all} form—with right understanding: ‘This is not mine, I am not this, this is not my self.’ They truly see any kind of feeling … perception … choices … consciousness at all—past, future, or present; internal or external; coarse or fine; inferior or superior; far or near, \emph{all} consciousness—with right understanding: ‘This is not mine, I am not this, this is not my self.’ 

Just\marginnote{12.1} as a warrior is a marksman, a noble disciple has right view. A noble disciple with right view truly understands: ‘This is suffering’ … ‘This is the origin of suffering’ … ‘This is the cessation of suffering’ … ‘This is the practice that leads to the cessation of suffering’. 

Just\marginnote{13.1} as a warrior shatters large objects, a noble disciple has right freedom. A noble disciple with right freedom shatters the great mass of ignorance.” 

%
\section*{{\suttatitleacronym AN 4.197}{\suttatitletranslation Queen Mallikā }{\suttatitleroot Mallikādevīsutta}}
\addcontentsline{toc}{section}{\tocacronym{AN 4.197} \toctranslation{Queen Mallikā } \tocroot{Mallikādevīsutta}}
\markboth{Queen Mallikā }{Mallikādevīsutta}
\extramarks{AN 4.197}{AN 4.197}

At\marginnote{1.1} one time the Buddha was staying near \textsanskrit{Sāvatthī} in Jeta’s Grove, \textsanskrit{Anāthapiṇḍika}’s monastery. Then Queen \textsanskrit{Mallikā} went up to the Buddha, bowed, sat down to one side, and said to him: 

“What\marginnote{2.1} is the cause, sir, what is the reason why in this life some females are ugly, unattractive, and bad-looking; and poor, with few assets and possessions; and insignificant? 

And\marginnote{3.1} why are some females ugly, unattractive, and bad-looking; but rich, affluent, wealthy, and illustrious? 

And\marginnote{4.1} why are some females attractive, good-looking, lovely, of surpassing beauty; but poor, with few assets and possessions; and insignificant? 

And\marginnote{5.1} why are some females attractive, good-looking, lovely, of surpassing beauty; and rich, affluent, wealthy, and illustrious?” 

“Take\marginnote{6.1} a female who is irritable and bad-tempered. Even when criticized a little bit she loses her temper, becoming annoyed, hostile, and hard-hearted, and displaying annoyance, hate, and bitterness. She doesn’t give to ascetics or brahmins such things as food, drink, clothing, vehicles; garlands, fragrance, and makeup; and bed, house, and lighting. And she’s jealous, envying, resenting, and begrudging the possessions, honor, respect, reverence, homage, and veneration given to others. If she comes back to this state of existence after passing away, wherever she is reborn she’s ugly, unattractive, and bad-looking; and poor, with few assets and possessions; and insignificant. 

Take\marginnote{7.1} another female who is irritable and bad-tempered. … But she does give to ascetics or brahmins … And she’s not jealous … If she comes back to this state of existence after passing away, wherever she is reborn she’s ugly, unattractive, and bad-looking; but rich, affluent, wealthy, and illustrious. 

Take\marginnote{8.1} another female who isn’t irritable and bad-tempered. … But she doesn’t give to ascetics or brahmins … And she’s jealous … If she comes back to this state of existence after passing away, wherever she is reborn she’s attractive, good-looking, lovely, of surpassing beauty; but poor, with few assets and possessions; and insignificant. 

Take\marginnote{9.1} another female who isn’t irritable and bad-tempered. … She gives to ascetics and brahmins … And she’s not jealous … If she comes back to this state of existence after passing away, wherever she is reborn she’s attractive, good-looking, lovely, of surpassing beauty; and rich, affluent, wealthy, and illustrious. 

This\marginnote{10.1} is why some females are ugly … and poor … and insignificant. And some females are ugly … but rich … and illustrious. And some females are attractive … but poor … and insignificant. And some females are attractive … and rich … and illustrious.” 

When\marginnote{11.1} this was said, Queen \textsanskrit{Mallikā} said to the Buddha: 

“Sir,\marginnote{11.2} in another life I must have been irritable and bad-tempered. Even when lightly criticized I must have lost my temper, becoming annoyed, hostile, and hard-hearted, and displaying annoyance, hate, and bitterness. For now I am ugly, unattractive, and bad-looking. 

In\marginnote{12.1} another life I must have given to ascetics or brahmins such things as food, drink, clothing, vehicles; garlands, fragrance, and makeup; and bed, house, and lighting. For now I am rich, affluent, and wealthy. 

In\marginnote{13.1} another life, I must not have been jealous, envying, resenting, and begrudging the possessions, honor, respect, reverence, homage, and veneration given to others. For now I am illustrious. In this royal court I command maidens of the aristocrats, brahmins, and householders. So, sir, from this day forth I will not be irritable and bad-tempered. Even when heavily criticized I won’t lose my temper, become annoyed, hostile, and hard-hearted, or display annoyance, hate, and bitterness. I will give to ascetics or brahmins such things as food, drink, clothing, vehicles; garlands, fragrance, and makeup; and bed, house, and lighting. I will not be jealous, envying, resenting, and begrudging the possessions, honor, respect, reverence, homage, and veneration given to others. 

Excellent,\marginnote{13.7} sir! … From this day forth, may the Buddha remember me as a lay follower who has gone for refuge for life.” 

%
\section*{{\suttatitleacronym AN 4.198}{\suttatitletranslation Self-mortification }{\suttatitleroot Attantapasutta}}
\addcontentsline{toc}{section}{\tocacronym{AN 4.198} \toctranslation{Self-mortification } \tocroot{Attantapasutta}}
\markboth{Self-mortification }{Attantapasutta}
\extramarks{AN 4.198}{AN 4.198}

“Mendicants,\marginnote{1.1} these four people are found in the world. What four? 

\begin{enumerate}%
\item One person mortifies themselves, pursuing the practice of mortifying themselves. %
\item One person mortifies others, pursuing the practice of mortifying others. %
\item One person mortifies themselves and others, pursuing the practice of mortifying themselves and others. %
\item One person neither mortifies themselves nor others, pursuing the practice of not mortifying themselves or others. They live without wishes in the present life, extinguished, cooled, experiencing bliss, having become holy in themselves. %
\end{enumerate}

And\marginnote{2.1} how does one person mortify themselves, pursuing the practice of mortifying themselves? It’s when someone goes naked, ignoring conventions. They lick their hands, and don’t come or wait when called. They don’t consent to food brought to them, or food prepared on purpose for them, or an invitation for a meal. They don’t receive anything from a pot or bowl; or from someone who keeps sheep, or who has a weapon or a shovel in their home; or where a couple is eating; or where there is a woman who is pregnant, breastfeeding, or who has a man in her home; or where there’s a dog waiting or flies buzzing. They accept no fish or meat or liquor or wine, and drink no beer. They go to just one house for alms, taking just one mouthful, or two houses and two mouthfuls, up to seven houses and seven mouthfuls. They feed on one saucer a day, two saucers a day, up to seven saucers a day. They eat once a day, once every second day, up to once a week, and so on, even up to once a fortnight. They live pursuing the practice of eating food at set intervals. 

They\marginnote{3.1} eat herbs, millet, wild rice, poor rice, water lettuce, rice bran, scum from boiling rice, sesame flour, grass, or cow dung. They survive on forest roots and fruits, or eating fallen fruit. 

They\marginnote{4.1} wear robes of sunn hemp, mixed hemp, corpse-wrapping cloth, rags, lodh tree bark, antelope hide (whole or in strips), kusa grass, bark, wood-chips, human hair, horse-tail hair, or owls’ wings. They tear out hair and beard, pursuing this practice. They constantly stand, refusing seats. They squat, committed to the endeavor of squatting. They lie on a mat of thorns, making a mat of thorns their bed. They pursue the practice of immersion in water three times a day, including the evening. And so they live pursuing these various ways of mortifying and tormenting the body. That’s how one person mortifies themselves, pursuing the practice of mortifying themselves. 

And\marginnote{5.1} how does one person mortify others, pursuing the practice of mortifying others? It’s when a person is a slaughterer of sheep, pigs, poultry, or deer, a hunter or fisher, a bandit, an executioner, a butcher of cattle, a jailer, or has some other cruel livelihood. That’s how one person mortifies others, pursuing the practice of mortifying others. 

And\marginnote{6.1} how does one person mortify themselves and others, pursuing the practice of mortifying themselves and others? It’s when a person is an anointed aristocratic king or a well-to-do brahmin. He has a new temple built to the east of the city. He shaves off his hair and beard, dresses in a rough antelope hide, and smears his body with ghee and oil. Scratching his back with antlers, he enters the temple with his chief queen and the brahmin high priest. There he lies on the bare ground strewn with grass. The king feeds on the milk from one teat of a cow that has a calf of the same color. The chief queen feeds on the milk from the second teat. The brahmin high priest feeds on the milk from the third teat. The milk from the fourth teat is served to the sacred flame. The calf feeds on the remainder. He says: ‘Slaughter this many bulls, bullocks, heifers, goats, rams, and horses for the sacrifice! Fell this many trees and reap this much grass for the sacrificial equipment!’ His bondservants, workers, and staff do their jobs under threat of punishment and danger, weeping, with tearful faces. That’s how one person mortifies themselves and others, pursuing the practice of mortifying themselves and others. 

And\marginnote{7.1} how does one person neither mortify themselves nor others, pursuing the practice of not mortifying themselves or others, living without wishes in the present life, extinguished, cooled, experiencing bliss, having become holy in themselves? It’s when a Realized One arises in the world, perfected, a fully awakened Buddha, accomplished in knowledge and conduct, holy, knower of the world, supreme guide for those who wish to train, teacher of gods and humans, awakened, blessed. He has realized with his own insight this world—with its gods, \textsanskrit{Māras} and \textsanskrit{Brahmās}, this population with its ascetics and brahmins, gods and humans—and he makes it known to others. He teaches Dhamma that’s good in the beginning, good in the middle, and good in the end, meaningful and well-phrased. And he reveals a spiritual practice that’s entirely full and pure. A householder hears that teaching, or a householder’s child, or someone reborn in some good family. They gain faith in the Realized One, and reflect: ‘Living in a house is cramped and dirty, but the life of one gone forth is wide open. It’s not easy for someone living at home to lead the spiritual life utterly full and pure, like a polished shell. Why don’t I shave off my hair and beard, dress in ocher robes, and go forth from the lay life to homelessness?’ After some time they give up a large or small fortune, and a large or small family circle. They shave off hair and beard, dress in ocher robes, and go forth from the lay life to homelessness. 

Once\marginnote{8.1} they’ve gone forth, they take up the training and livelihood of the mendicants. They give up killing living creatures, renouncing the rod and the sword. They’re scrupulous and kind, living full of compassion for all living beings. They give up stealing. They take only what’s given, and expect only what’s given. They keep themselves clean by not thieving. They give up unchastity. They are celibate, set apart, avoiding the common practice of sex. They give up lying. They speak the truth and stick to the truth. They’re honest and trustworthy, and don’t trick the world with their words. They give up divisive speech. They don’t repeat in one place what they heard in another so as to divide people against each other. Instead, they reconcile those who are divided, supporting unity, delighting in harmony, loving harmony, speaking words that promote harmony. They give up harsh speech. They speak in a way that’s mellow, pleasing to the ear, lovely, going to the heart, polite, likable, and agreeable to the people. They give up talking nonsense. Their words are timely, true, and meaningful, in line with the teaching and training. They say things at the right time which are valuable, reasonable, succinct, and beneficial. 

They\marginnote{9.1} refrain from injuring plants and seeds. They eat in one part of the day, abstaining from eating at night and food at the wrong time. They refrain from dancing, singing, music, and seeing shows. They refrain from beautifying and adorning themselves with garlands, fragrance, and makeup. They refrain from high and luxurious beds. They refrain from receiving gold and money, raw grains, raw meat, women and girls, male and female bondservants, goats and sheep, chickens and pigs, elephants, cows, horses, and mares, and fields and land. They refrain from running errands and messages; buying and selling; falsifying weights, metals, or measures; bribery, fraud, cheating, and duplicity; mutilation, murder, abduction, banditry, plunder, and violence. 

They’re\marginnote{10.1} content with robes to look after the body and almsfood to look after the belly. Wherever they go, they set out taking only these things. They’re like a bird: wherever it flies, wings are its only burden. In the same way, a mendicant is content with robes to look after the body and almsfood to look after the belly. Wherever they go, they set out taking only these things. When they have this entire spectrum of noble ethics, they experience a blameless happiness inside themselves. 

Seeing\marginnote{11.1} a sight with the eyes, they don’t get caught up in the features and details. If the faculty of sight were left unrestrained, bad unskillful qualities of desire and aversion would become overwhelming. For this reason, they practice restraint, protecting the faculty of sight, and achieving restraint over it. Hearing a sound with the ears … Smelling an odor with the nose … Tasting a flavor with the tongue … Feeling a touch with the body … Knowing a thought with the mind, they don’t get caught up in the features and details. If the faculty of mind were left unrestrained, bad unskillful qualities of desire and aversion would become overwhelming. For this reason, they practice restraint, protecting the faculty of mind, and achieving restraint over it. When they have this noble sense restraint, they experience an unsullied bliss inside themselves. 

They\marginnote{12.1} act with situational awareness when going out and coming back; when looking ahead and aside; when bending and extending the limbs; when bearing the outer robe, bowl and robes; when eating, drinking, chewing, and tasting; when urinating and defecating; when walking, standing, sitting, sleeping, waking, speaking, and keeping silent. 

When\marginnote{13.1} they have this noble spectrum of ethics, this noble sense restraint, and this noble mindfulness and situational awareness, they frequent a secluded lodging—a wilderness, the root of a tree, a hill, a ravine, a mountain cave, a charnel ground, a forest, the open air, a heap of straw. After the meal, they return from almsround, sit down cross-legged with their body straight, and establish mindfulness right there. Giving up desire for the world, they meditate with a heart rid of desire, cleansing the mind of desire. Giving up ill will and malevolence, they meditate with a mind rid of ill will, full of compassion for all living beings, cleansing the mind of ill will. Giving up dullness and drowsiness, they meditate with a mind rid of dullness and drowsiness, perceiving light, mindful and aware, cleansing the mind of dullness and drowsiness. Giving up restlessness and remorse, they meditate without restlessness, their mind peaceful inside, cleansing the mind of restlessness and remorse. Giving up doubt, they meditate having gone beyond doubt, not undecided about skillful qualities, cleansing the mind of doubt. They give up these five hindrances, corruptions of the heart that weaken wisdom. Then, quite secluded from sensual pleasures, secluded from unskillful qualities, they enter and remain in the first absorption … second absorption … third absorption … fourth absorption. 

When\marginnote{14.1} their mind has become immersed in \textsanskrit{samādhi} like this—purified, bright, flawless, rid of corruptions, pliable, workable, steady, and imperturbable—they extend it toward recollection of past lives … knowledge of the death and rebirth of sentient beings … knowledge of the ending of defilements. They truly understand: ‘This is suffering’ … ‘This is the origin of suffering’ … ‘This is the cessation of suffering’ … ‘This is the practice that leads to the cessation of suffering’. They truly understand: ‘These are defilements’ … ‘This is the origin of defilements’ … ‘This is the cessation of defilements’ … ‘This is the practice that leads to the cessation of defilements’. 

Knowing\marginnote{15.1} and seeing like this, their mind is freed from the defilements of sensuality, desire to be reborn, and ignorance. When they’re freed, they know they’re freed. 

They\marginnote{15.3} understand: ‘Rebirth is ended, the spiritual journey has been completed, what had to be done has been done, there is no return to any state of existence.’ That’s how one person neither mortifies themselves nor others, pursuing the practice of not mortifying themselves or others, living without wishes in the present life, extinguished, cooled, experiencing bliss, having become holy in themselves. 

These\marginnote{15.6} are the four people found in the world.” 

%
\section*{{\suttatitleacronym AN 4.199}{\suttatitletranslation Craving, the Weaver }{\suttatitleroot Taṇhāsutta}}
\addcontentsline{toc}{section}{\tocacronym{AN 4.199} \toctranslation{Craving, the Weaver } \tocroot{Taṇhāsutta}}
\markboth{Craving, the Weaver }{Taṇhāsutta}
\extramarks{AN 4.199}{AN 4.199}

The\marginnote{1.1} Buddha said this: 

“Mendicants,\marginnote{1.2} I will teach you about craving—the weaver, the migrant, the creeping, the clinging. This world is choked by it, engulfed by it. It makes the world tangled like yarn, knotted like a ball of thread, and matted like rushes and reeds, not escaping the places of loss, the bad places, the underworld, transmigration. Listen and pay close attention, I will speak.” 

“Yes,\marginnote{1.4} sir,” they replied. The Buddha said this: 

“And\marginnote{2.1} what is that craving …? There are eighteen currents of craving that derive from the interior, and eighteen that derive from the exterior. 

What\marginnote{3.1} are the eighteen currents of craving that derive from the interior? When there is the concept ‘I am’, there are the concepts ‘I am such’, ‘I am thus’, ‘I am otherwise’; ‘I am fleeting’, ‘I am lasting’; ‘mine’, ‘such is mine’, ‘thus is mine’, ‘otherwise is mine’; ‘also mine’, ‘such is also mine’, ‘thus is also mine’, ‘otherwise is also mine’; ‘I will be’, ‘I will be such’, ‘I will be thus’, ‘I will be otherwise’. These are the eighteen currents of craving that derive from the interior. 

What\marginnote{4.1} are the eighteen currents of craving that derive from the exterior? When there is the concept ‘I am because of this’, there are the concepts ‘I am such because of this’, ‘I am thus because of this’, ‘I am otherwise because of this’; ‘I am fleeting because of this’, ‘I am lasting because of this’; ‘mine because of this’, ‘such is mine because of this’, ‘thus is mine because of this’, ‘otherwise is mine because of this’; ‘also mine because of this’, ‘such is also mine because of this’, ‘thus is also mine because of this’, ‘otherwise is also mine because of this’; ‘I will be because of this’, ‘I will be such because of this’, ‘I will be thus because of this’, ‘I will be otherwise because of this’. These are the eighteen currents of craving that derive from the exterior. 

So\marginnote{5.1} there are eighteen currents of craving that derive from the interior, and eighteen that derive from the exterior. These are called the thirty-six currents of craving. Each of these pertain to the past, future, and present, making one hundred and eight currents of craving. 

This\marginnote{6.1} is that craving—the weaver, the migrant, the creeping, the clinging. This world is choked by it, engulfed by it. It makes the world tangled like yarn, knotted like a ball of thread, and matted like rushes and reeds, not escaping the places of loss, the bad places, the underworld, transmigration.” 

%
\section*{{\suttatitleacronym AN 4.200}{\suttatitletranslation Love and Hate }{\suttatitleroot Pemasutta}}
\addcontentsline{toc}{section}{\tocacronym{AN 4.200} \toctranslation{Love and Hate } \tocroot{Pemasutta}}
\markboth{Love and Hate }{Pemasutta}
\extramarks{AN 4.200}{AN 4.200}

“Mendicants,\marginnote{1.1} these four things are born of love and hate. What four? 

\begin{enumerate}%
\item Love is born of love, %
\item hate is born of love, %
\item love is born of hate, and %
\item hate is born of hate. %
\end{enumerate}

And\marginnote{2.1} how is love born of love? It’s when someone likes, loves, and cares for a person. Others treat that person with liking, love, and care. They think: ‘These others like the person I like.’ And so love for them springs up. That’s how love is born of love. 

And\marginnote{3.1} how is hate born of love? It’s when someone likes, loves, and cares for a person. Others treat that person with disliking, loathing, and detestation. They think: ‘These others dislike the person I like.’ And so hate for them springs up. That’s how hate is born of love. 

And\marginnote{4.1} how is love born of hate? It’s when someone dislikes, loathes, and detests a person. Others treat that person with disliking, loathing, and detestation. They think: ‘These others dislike the person I dislike.’ And so love for them springs up. That’s how love is born of hate. 

And\marginnote{5.1} how is hate born of hate? It’s when someone dislikes, loathes, and detests a person. Others treat that person with liking, love, and care. They think: ‘These others like the person I dislike.’ And so hate for them springs up. That’s how hate is born of hate. 

These\marginnote{5.8} are the four things that are born of love and hate. 

A\marginnote{6.1} time comes when a mendicant … enters and remains in the first absorption. At that time they have no love born of love, hate born of love, love born of hate, or hate born of hate. 

A\marginnote{7.1} time comes when a mendicant … enters and remains in the second absorption … third absorption … fourth absorption. At that time they have no love born of love, hate born of love, love born of hate, or hate born of hate. 

A\marginnote{8.1} time comes when a mendicant realizes the undefiled freedom of heart and freedom by wisdom in this very life. And they live having realized it with their own insight due to the ending of defilements. At that time any love born of love, hate born of love, love born of hate, or hate born of hate is given up, cut off at the root, made like a palm stump, obliterated, and unable to arise in the future. This is called a mendicant who doesn’t draw close or push back or fume or ignite or burn up. 

And\marginnote{9.1} how does a mendicant draw close? It’s when a mendicant regards form as self, self as having form, form in self, or self in form. They regard feeling as self, self as having feeling, feeling in self, or self in feeling. They regard perception as self, self as having perception, perception in self, or self in perception. They regard choices as self, self as having choices, choices in self, or self in choices. They regard consciousness as self, self as having consciousness, consciousness in self, or self in consciousness. That’s how a mendicant draws close. 

And\marginnote{10.1} how does a mendicant not draw close? It’s when a mendicant doesn’t regard form as self, self as having form, form in self, or self in form. They don’t regard feeling as self, self as having feeling, feeling in self, or self in feeling. They don’t regard perception as self, self as having perception, perception in self, or self in perception. They don’t regard choices as self, self as having choices, choices in self, or self in choices. They don’t regard consciousness as self, self as having consciousness, consciousness in self, or self in consciousness. That’s how a mendicant doesn’t draw close. 

And\marginnote{11.1} how does a mendicant push back? It’s when someone abuses, annoys, or argues with a mendicant, and the mendicant abuses, annoys, or argues back at them. That’s how a mendicant pushes back. 

And\marginnote{12.1} how does a mendicant not push back? It’s when someone abuses, annoys, or argues with a mendicant, and the mendicant doesn’t abuse, annoy, or argue back at them. That’s how a mendicant doesn’t push back. 

And\marginnote{13.1} how does a mendicant fume? When there is the concept ‘I am’, there are the concepts ‘I am such’, ‘I am thus’, ‘I am otherwise’; ‘I am fleeting’, ‘I am lasting’; ‘mine’, ‘such is mine’, ‘thus is mine’, ‘otherwise is mine’; ‘also mine’, ‘such is also mine’, ‘thus is also mine’, ‘otherwise is also mine’; ‘I will be’, ‘I will be such’, ‘I will be thus’, ‘I will be otherwise’. That’s how a mendicant fumes. 

And\marginnote{14.1} how does a mendicant not fume? When there is no concept ‘I am’, there are no concepts ‘I am such’, ‘I am thus’, ‘I am otherwise’; ‘I am fleeting’, ‘I am lasting’; ‘mine’, ‘such is mine’, ‘thus is mine’, ‘otherwise is mine’; ‘also mine’, ‘such is also mine’, ‘thus is also mine’, ‘otherwise is also mine’; ‘I will be’, ‘I will be such’, ‘I will be thus’, ‘I will be otherwise’. That’s how a mendicant doesn’t fume. 

And\marginnote{15.1} how is a mendicant ignited? When there is the concept ‘I am because of this’, there are the concepts ‘I am such because of this’, ‘I am thus because of this’, ‘I am otherwise because of this’; ‘I am fleeting because of this’, ‘I am lasting because of this’; ‘mine because of this’, ‘such is mine because of this’, ‘thus is mine because of this’, ‘otherwise is mine because of this’; ‘also mine because of this’, ‘such is also mine because of this’, ‘thus is also mine because of this’, ‘otherwise is also mine because of this’; ‘I will be because of this’, ‘I will be such because of this’, ‘I will be thus because of this’, ‘I will be otherwise because of this’. That’s how a mendicant is ignited. 

And\marginnote{16.1} how is a mendicant not ignited? When there is no concept ‘I am because of this’, there are no concepts ‘I am such because of this’, ‘I am thus because of this’, ‘I am otherwise because of this’; ‘I am fleeting because of this’, ‘I am lasting because of this’; ‘mine because of this’, ‘such is mine because of this’, ‘thus is mine because of this’, ‘otherwise is mine because of this’; ‘also mine because of this’, ‘such is also mine because of this’, ‘thus is also mine because of this’, ‘otherwise is also mine because of this’; ‘I will be because of this’, ‘I will be such because of this’, ‘I will be thus because of this’, ‘I will be otherwise because of this’. That’s how a mendicant is not ignited. 

And\marginnote{17.1} how does a mendicant burn up? It’s when a mendicant hasn’t given up the conceit ‘I am’, cut it off at the root, made it like a palm stump, obliterated it, so it’s unable to arise in the future. That’s how a mendicant is burned up. 

And\marginnote{18.1} how does a mendicant not burn up? It’s when a mendicant has given up the conceit ‘I am’, cut it off at the root, made it like a palm stump, obliterated it, so it’s unable to arise in the future. That’s how a mendicant is not burned up.” 

%
\addtocontents{toc}{\let\protect\contentsline\protect\nopagecontentsline}
\pannasa{The Fifth Fifty }
\addcontentsline{toc}{pannasa}{The Fifth Fifty }
\markboth{}{}
\addtocontents{toc}{\let\protect\contentsline\protect\oldcontentsline}

%
\addtocontents{toc}{\let\protect\contentsline\protect\nopagecontentsline}
\chapter*{The Chapter on a Good Person }
\addcontentsline{toc}{chapter}{\tocchapterline{The Chapter on a Good Person }}
\addtocontents{toc}{\let\protect\contentsline\protect\oldcontentsline}

%
\section*{{\suttatitleacronym AN 4.201}{\suttatitletranslation Training Rules }{\suttatitleroot Sikkhāpadasutta}}
\addcontentsline{toc}{section}{\tocacronym{AN 4.201} \toctranslation{Training Rules } \tocroot{Sikkhāpadasutta}}
\markboth{Training Rules }{Sikkhāpadasutta}
\extramarks{AN 4.201}{AN 4.201}

“Mendicants,\marginnote{1.1} I will teach you a bad person and a worse person, a good person and a better person. Listen and pay close attention, I will speak.” 

“Yes,\marginnote{1.4} sir,” they replied. The Buddha said this: 

“And\marginnote{2.1} what is a bad person? It’s someone who kills living creatures, steals, commits sexual misconduct, lies, and uses alcoholic drinks that cause negligence. This is called a bad person. 

And\marginnote{3.1} what is a worse person? It’s someone who kills living creatures, steals, commits sexual misconduct, lies, and uses alcoholic drinks that cause negligence. And they encourage others to do these things. This is called a worse person. 

And\marginnote{4.1} what is a good person? It’s someone who doesn’t kill living creatures, steal, commit sexual misconduct, lie, or use alcoholic drinks that cause negligence. This is called a good person. 

And\marginnote{5.1} what is a better person? It’s someone who doesn’t kill living creatures, steal, commit sexual misconduct, lie, or use alcoholic drinks that cause negligence. And they encourage others to avoid these things. This is called a better person.” 

%
\section*{{\suttatitleacronym AN 4.202}{\suttatitletranslation Faithless }{\suttatitleroot Assaddhasutta}}
\addcontentsline{toc}{section}{\tocacronym{AN 4.202} \toctranslation{Faithless } \tocroot{Assaddhasutta}}
\markboth{Faithless }{Assaddhasutta}
\extramarks{AN 4.202}{AN 4.202}

“Mendicants,\marginnote{1.1} I will teach you a bad person and a worse person, a good person and a better person. 

And\marginnote{2.1} what is a bad person? It’s someone who is faithless, shameless, imprudent, with little learning, lazy, unmindful, and witless. This is called a bad person. 

And\marginnote{3.1} what is a worse person? It’s someone who is faithless, shameless, imprudent, with little learning, lazy, confused, and witless. And they encourage others in these same qualities. This is called a worse person. 

And\marginnote{4.1} what is a good person? It’s someone who is faithful, conscientious, prudent, learned, energetic, mindful, and wise. This is called a good person. 

And\marginnote{5.1} what is a better person? It’s someone who is personally accomplished in faith, conscience, prudence, learning, energy, mindfulness, and wisdom. And they encourage others in these same qualities. This is called a better person.” 

%
\section*{{\suttatitleacronym AN 4.203}{\suttatitletranslation Seven Kinds of Deeds }{\suttatitleroot Sattakammasutta}}
\addcontentsline{toc}{section}{\tocacronym{AN 4.203} \toctranslation{Seven Kinds of Deeds } \tocroot{Sattakammasutta}}
\markboth{Seven Kinds of Deeds }{Sattakammasutta}
\extramarks{AN 4.203}{AN 4.203}

“Mendicants,\marginnote{1.1} I will teach you a bad person and a worse person, a good person and a better person. 

And\marginnote{2.1} what is a bad person? It’s someone who kills living creatures, steals, commits sexual misconduct, and uses speech that’s false, divisive, harsh, or nonsensical. This is called a bad person. 

And\marginnote{3.1} what is a worse person? It’s someone who kills living creatures, steals, commits sexual misconduct, and uses speech that’s false, divisive, harsh, or nonsensical. And they encourage others to do these things. This is called a worse person. 

And\marginnote{4.1} what is a good person? It’s someone who doesn’t kill living creatures, steal, commit sexual misconduct, or use speech that’s false, divisive, harsh, or nonsensical. This is called a good person. 

And\marginnote{5.1} what is a better person? It’s someone who doesn’t kill living creatures, steal, commit sexual misconduct, or use speech that’s false, divisive, harsh, or nonsensical. And they encourage others to avoid these things. This is called a better person.” 

%
\section*{{\suttatitleacronym AN 4.204}{\suttatitletranslation Ten Kinds of Deeds }{\suttatitleroot Dasakammasutta}}
\addcontentsline{toc}{section}{\tocacronym{AN 4.204} \toctranslation{Ten Kinds of Deeds } \tocroot{Dasakammasutta}}
\markboth{Ten Kinds of Deeds }{Dasakammasutta}
\extramarks{AN 4.204}{AN 4.204}

“Mendicants,\marginnote{1.1} I will teach you a bad person and a worse person, a good person and a better person. 

And\marginnote{2.1} what is a bad person? It’s someone who kills living creatures, steals, and commits sexual misconduct. They use speech that’s false, divisive, harsh, or nonsensical. And they’re covetous, malicious, with wrong view. This is called a bad person. 

And\marginnote{3.1} what is a worse person? It’s someone who kills living creatures, steals, and commits sexual misconduct. They use speech that’s false, divisive, harsh, or nonsensical. They’re covetous, malicious, with wrong view. And they encourage others to do these things. This is called a worse person. 

And\marginnote{4.1} what is a good person? It’s someone who doesn’t kill living creatures, steal, or commit sexual misconduct. They don’t use speech that’s false, divisive, harsh, or nonsensical. And they’re contented, kind-hearted, with right view. This is called a good person. 

And\marginnote{5.1} what is a better person? It’s someone who doesn’t kill living creatures, steal, or commit sexual misconduct. They don’t use speech that’s false, divisive, harsh, or nonsensical. They’re contented, kind-hearted, with right view. And they encourage others to do these things. This is called a better person.” 

%
\section*{{\suttatitleacronym AN 4.205}{\suttatitletranslation Eightfold }{\suttatitleroot Aṭṭhaṅgikasutta}}
\addcontentsline{toc}{section}{\tocacronym{AN 4.205} \toctranslation{Eightfold } \tocroot{Aṭṭhaṅgikasutta}}
\markboth{Eightfold }{Aṭṭhaṅgikasutta}
\extramarks{AN 4.205}{AN 4.205}

“Mendicants,\marginnote{1.1} I will teach you a bad person and a worse person, a good person and a better person. 

And\marginnote{2.1} what is a bad person? It’s someone who has wrong view, wrong thought, wrong speech, wrong action, wrong livelihood, wrong effort, wrong mindfulness, and wrong immersion. This is called a bad person. 

And\marginnote{3.1} what is a worse person? It’s someone who has wrong view, wrong thought, wrong speech, wrong action, wrong livelihood, wrong effort, wrong mindfulness, and wrong immersion. And they encourage others in these same qualities. This is called a worse person. 

And\marginnote{4.1} what is a good person? It’s someone who has right view, right thought, right speech, right action, right livelihood, right effort, right mindfulness, and right immersion. This is called a good person. 

And\marginnote{5.1} what is a better person? It’s someone who has right view, right thought, right speech, right action, right livelihood, right effort, right mindfulness, and right immersion. And they encourage others in these same qualities. This is called a better person.” 

%
\section*{{\suttatitleacronym AN 4.206}{\suttatitletranslation The Path with Ten Factors }{\suttatitleroot Dasamaggasutta}}
\addcontentsline{toc}{section}{\tocacronym{AN 4.206} \toctranslation{The Path with Ten Factors } \tocroot{Dasamaggasutta}}
\markboth{The Path with Ten Factors }{Dasamaggasutta}
\extramarks{AN 4.206}{AN 4.206}

“Mendicants,\marginnote{1.1} I will teach you a bad person and a worse person, a good person and a better person. 

And\marginnote{2.1} what is a bad person? It’s someone who has wrong view, wrong thought, wrong speech, wrong action, wrong livelihood, wrong effort, wrong mindfulness, wrong immersion, wrong knowledge, and wrong freedom. This is called a bad person. 

And\marginnote{3.1} what is a worse person? It’s someone who has wrong view, wrong thought, wrong speech, wrong action, wrong livelihood, wrong effort, wrong mindfulness, wrong immersion, wrong knowledge, and wrong freedom. And they encourage others in these same qualities. This is called a worse person. 

And\marginnote{4.1} what is a good person? It’s someone who has right view, right thought, right speech, right action, right livelihood, right effort, right mindfulness, right immersion, right knowledge, and right freedom. This is called a good person. 

And\marginnote{5.1} what is a better person? It’s someone who has right view, right thought, right speech, right action, right livelihood, right effort, right mindfulness, right immersion, right knowledge, and right freedom. And they encourage others in these same qualities. This is called a better person.” 

%
\section*{{\suttatitleacronym AN 4.207}{\suttatitletranslation Bad Character (1st) }{\suttatitleroot Paṭhamapāpadhammasutta}}
\addcontentsline{toc}{section}{\tocacronym{AN 4.207} \toctranslation{Bad Character (1st) } \tocroot{Paṭhamapāpadhammasutta}}
\markboth{Bad Character (1st) }{Paṭhamapāpadhammasutta}
\extramarks{AN 4.207}{AN 4.207}

“Mendicants,\marginnote{1.1} I will teach you who’s bad and who’s worse, who’s good and who’s better. 

And\marginnote{2.1} who’s bad? It’s someone who kills living creatures, steals, and commits sexual misconduct. They use speech that’s false, divisive, harsh, or nonsensical. And they’re covetous, malicious, with wrong view. This is called bad. 

And\marginnote{3.1} who’s worse? It’s someone who kills living creatures, steals, and commits sexual misconduct. They use speech that’s false, divisive, harsh, or nonsensical. They’re covetous, malicious, with wrong view. And they encourage others to do these things. This is called worse. 

And\marginnote{4.1} who’s good? It’s someone who doesn’t kill living creatures, steal, or commit sexual misconduct. They don’t use speech that’s false, divisive, harsh, or nonsensical. And they’re contented, kind-hearted, with right view. This is called good. 

And\marginnote{5.1} who’s better? It’s someone who doesn’t kill living creatures, steal, or commit sexual misconduct. They don’t use speech that’s false, divisive, harsh, or nonsensical. They’re contented, kind-hearted, with right view. And they encourage others to do these things. This is called better.” 

%
\section*{{\suttatitleacronym AN 4.208}{\suttatitletranslation Bad Character (2nd) }{\suttatitleroot Dutiyapāpadhammasutta}}
\addcontentsline{toc}{section}{\tocacronym{AN 4.208} \toctranslation{Bad Character (2nd) } \tocroot{Dutiyapāpadhammasutta}}
\markboth{Bad Character (2nd) }{Dutiyapāpadhammasutta}
\extramarks{AN 4.208}{AN 4.208}

“Mendicants,\marginnote{1.1} I will teach you who’s bad and who’s worse, who’s good and who’s better. 

And\marginnote{2.1} who’s bad? It’s someone who has wrong view, wrong thought, wrong speech, wrong action, wrong livelihood, wrong effort, wrong mindfulness, wrong immersion, wrong knowledge, and wrong freedom. This is called bad. 

And\marginnote{3.1} who’s worse? It’s someone who has wrong view, wrong thought, wrong speech, wrong action, wrong livelihood, wrong effort, wrong mindfulness, wrong immersion, wrong knowledge, and wrong freedom. And they encourage others in these same qualities. This is called worse. 

And\marginnote{4.1} who’s good? It’s someone who has right view, right thought, right speech, right action, right livelihood, right effort, right mindfulness, right immersion, right knowledge, and right freedom. This is called good. 

And\marginnote{5.1} who’s better? It’s someone who has right view, right thought, right speech, right action, right livelihood, right effort, right mindfulness, right immersion, right knowledge, and right freedom. And they encourage others in these same qualities. This is called better.” 

%
\section*{{\suttatitleacronym AN 4.209}{\suttatitletranslation Bad Character (3rd) }{\suttatitleroot Tatiyapāpadhammasutta}}
\addcontentsline{toc}{section}{\tocacronym{AN 4.209} \toctranslation{Bad Character (3rd) } \tocroot{Tatiyapāpadhammasutta}}
\markboth{Bad Character (3rd) }{Tatiyapāpadhammasutta}
\extramarks{AN 4.209}{AN 4.209}

“Mendicants,\marginnote{1.1} I will teach you bad character and worse character, good character and better character. 

And\marginnote{2.1} who has bad character? It’s someone who kills living creatures, steals, and commits sexual misconduct. They use speech that’s false, divisive, harsh, or nonsensical. And they’re covetous, malicious, with wrong view. This is called bad character. 

And\marginnote{3.1} who has worse character? It’s someone who kills living creatures, steals, and commits sexual misconduct. They use speech that’s false, divisive, harsh, or nonsensical. They’re covetous, malicious, with wrong view. And they encourage others to do these things. This is called worse character. 

And\marginnote{4.1} who has good character? It’s someone who doesn’t kill living creatures, steal, or commit sexual misconduct. They don’t use speech that’s false, divisive, harsh, or nonsensical. And they’re contented, kind-hearted, with right view. This is called good character. 

And\marginnote{5.1} who has better character? It’s someone who doesn’t kill living creatures, steal, or commit sexual misconduct. They don’t use speech that’s false, divisive, harsh, or nonsensical. They’re contented, kind-hearted, with right view. And they encourage others to do these things. This is called better character.” 

%
\section*{{\suttatitleacronym AN 4.210}{\suttatitletranslation Bad Character (4th) }{\suttatitleroot Catutthapāpadhammasutta}}
\addcontentsline{toc}{section}{\tocacronym{AN 4.210} \toctranslation{Bad Character (4th) } \tocroot{Catutthapāpadhammasutta}}
\markboth{Bad Character (4th) }{Catutthapāpadhammasutta}
\extramarks{AN 4.210}{AN 4.210}

“Mendicants,\marginnote{1.1} I will teach you bad character and worse character, good character and better character. 

And\marginnote{2.1} who has bad character? It’s someone who has wrong view, wrong thought, wrong speech, wrong action, wrong livelihood, wrong effort, wrong mindfulness, wrong immersion, wrong knowledge, and wrong freedom. This is called bad character. 

And\marginnote{3.1} who has worse character? It’s someone who has wrong view, wrong thought, wrong speech, wrong action, wrong livelihood, wrong effort, wrong mindfulness, wrong immersion, wrong knowledge, and wrong freedom. And they encourage others in these same qualities. This is called worse character. 

And\marginnote{4.1} who has good character? It’s someone who has right view, right thought, right speech, right action, right livelihood, right effort, right mindfulness, right immersion, right knowledge, and right freedom. This is called good character. 

And\marginnote{5.1} who has better character? It’s someone who has right view, right thought, right speech, right action, right livelihood, right effort, right mindfulness, right immersion, right knowledge, and right freedom. And they encourage others in these same qualities. This is called better character.” 

%
\addtocontents{toc}{\let\protect\contentsline\protect\nopagecontentsline}
\chapter*{The Chapter on Assemblies }
\addcontentsline{toc}{chapter}{\tocchapterline{The Chapter on Assemblies }}
\addtocontents{toc}{\let\protect\contentsline\protect\oldcontentsline}

%
\section*{{\suttatitleacronym AN 4.211}{\suttatitletranslation Assembly }{\suttatitleroot Parisāsutta}}
\addcontentsline{toc}{section}{\tocacronym{AN 4.211} \toctranslation{Assembly } \tocroot{Parisāsutta}}
\markboth{Assembly }{Parisāsutta}
\extramarks{AN 4.211}{AN 4.211}

“Mendicants,\marginnote{1.1} these four corrupt an assembly. What four? A monk, nun, layman, or laywoman who is unethical, of bad character. These are the four that corrupt an assembly. 

Mendicants,\marginnote{2.1} these four beautify an assembly. What four? A monk, nun, layman, or laywoman who is ethical, of good character. These are the four that beautify an assembly.” 

%
\section*{{\suttatitleacronym AN 4.212}{\suttatitletranslation View }{\suttatitleroot Diṭṭhisutta}}
\addcontentsline{toc}{section}{\tocacronym{AN 4.212} \toctranslation{View } \tocroot{Diṭṭhisutta}}
\markboth{View }{Diṭṭhisutta}
\extramarks{AN 4.212}{AN 4.212}

“Mendicants,\marginnote{1.1} someone with four qualities is cast down to hell. What four? Bad conduct by way of body, speech, and mind, and wrong view. Someone with these four qualities is cast down to hell. 

Someone\marginnote{2.1} with four qualities is raised up to heaven. What four? Good conduct by way of body, speech, and mind, and right view. Someone with these four qualities is raised up to heaven.” 

%
\section*{{\suttatitleacronym AN 4.213}{\suttatitletranslation Ungrateful }{\suttatitleroot Akataññutāsutta}}
\addcontentsline{toc}{section}{\tocacronym{AN 4.213} \toctranslation{Ungrateful } \tocroot{Akataññutāsutta}}
\markboth{Ungrateful }{Akataññutāsutta}
\extramarks{AN 4.213}{AN 4.213}

“Mendicants,\marginnote{1.1} someone with four qualities is cast down to hell. What four? Bad conduct by way of body, speech, and mind, and being ungrateful and thankless. Someone with these four qualities is cast down to hell. 

Someone\marginnote{2.1} with four qualities is raised up to heaven. What four? Good conduct by way of body, speech, and mind, and being grateful and thankful. Someone with these four qualities is raised up to heaven.” 

%
\section*{{\suttatitleacronym AN 4.214}{\suttatitletranslation Killing Living Creatures }{\suttatitleroot Pāṇātipātīsutta}}
\addcontentsline{toc}{section}{\tocacronym{AN 4.214} \toctranslation{Killing Living Creatures } \tocroot{Pāṇātipātīsutta}}
\markboth{Killing Living Creatures }{Pāṇātipātīsutta}
\extramarks{AN 4.214}{AN 4.214}

“Someone\marginnote{1.1} with four qualities is cast down to hell. … They kill living creatures, steal, commit sexual misconduct, and lie. … Someone with four qualities is raised up to heaven. … They don’t kill living creatures, steal, commit sexual misconduct, or lie. …” 

%
\section*{{\suttatitleacronym AN 4.215}{\suttatitletranslation Path (1st) }{\suttatitleroot Paṭhamamaggasutta}}
\addcontentsline{toc}{section}{\tocacronym{AN 4.215} \toctranslation{Path (1st) } \tocroot{Paṭhamamaggasutta}}
\markboth{Path (1st) }{Paṭhamamaggasutta}
\extramarks{AN 4.215}{AN 4.215}

“Someone\marginnote{1.1} with four qualities is cast down to hell. … wrong view, wrong thought, wrong speech, wrong action. … Someone with four qualities is raised up to heaven. … right view, right thought, right speech, right action. …” 

%
\section*{{\suttatitleacronym AN 4.216}{\suttatitletranslation Path (2nd) }{\suttatitleroot Dutiyamaggasutta}}
\addcontentsline{toc}{section}{\tocacronym{AN 4.216} \toctranslation{Path (2nd) } \tocroot{Dutiyamaggasutta}}
\markboth{Path (2nd) }{Dutiyamaggasutta}
\extramarks{AN 4.216}{AN 4.216}

“Someone\marginnote{1.1} with four qualities is cast down to hell. … wrong livelihood, wrong effort, wrong mindfulness, and wrong immersion. Someone with four qualities is raised up to heaven. … right livelihood, right effort, right mindfulness, and right immersion. …” 

%
\section*{{\suttatitleacronym AN 4.217}{\suttatitletranslation Kinds of Expression (1st) }{\suttatitleroot Paṭhamavohārapathasutta}}
\addcontentsline{toc}{section}{\tocacronym{AN 4.217} \toctranslation{Kinds of Expression (1st) } \tocroot{Paṭhamavohārapathasutta}}
\markboth{Kinds of Expression (1st) }{Paṭhamavohārapathasutta}
\extramarks{AN 4.217}{AN 4.217}

“Someone\marginnote{1.1} with four qualities is cast down to hell. … They say they’ve seen, heard, thought, or known something, but they haven’t. … Someone with four qualities is raised up to heaven. … They say they haven’t seen, heard, thought, or known something, and they haven’t. …” 

%
\section*{{\suttatitleacronym AN 4.218}{\suttatitletranslation Kinds of Expression (2nd) }{\suttatitleroot Dutiyavohārapathasutta}}
\addcontentsline{toc}{section}{\tocacronym{AN 4.218} \toctranslation{Kinds of Expression (2nd) } \tocroot{Dutiyavohārapathasutta}}
\markboth{Kinds of Expression (2nd) }{Dutiyavohārapathasutta}
\extramarks{AN 4.218}{AN 4.218}

“Someone\marginnote{1.1} with four qualities is cast down to hell. … They say they haven’t seen, heard, thought, or known something, but they have. … Someone with four qualities is raised up to heaven. … They say they’ve seen, heard, thought, or known something, and they have. …” 

%
\section*{{\suttatitleacronym AN 4.219}{\suttatitletranslation Imprudence }{\suttatitleroot Ahirikasutta}}
\addcontentsline{toc}{section}{\tocacronym{AN 4.219} \toctranslation{Imprudence } \tocroot{Ahirikasutta}}
\markboth{Imprudence }{Ahirikasutta}
\extramarks{AN 4.219}{AN 4.219}

“Someone\marginnote{1.1} with four qualities is cast down to hell. … They’re faithless, unethical, shameless, and imprudent. … Someone with four qualities is raised up to heaven. … They’re faithful, ethical, conscientious, and prudent. …” 

%
\section*{{\suttatitleacronym AN 4.220}{\suttatitletranslation Unethical }{\suttatitleroot Dussīlasutta}}
\addcontentsline{toc}{section}{\tocacronym{AN 4.220} \toctranslation{Unethical } \tocroot{Dussīlasutta}}
\markboth{Unethical }{Dussīlasutta}
\extramarks{AN 4.220}{AN 4.220}

“Mendicants,\marginnote{1.1} someone with four qualities is cast down to hell. What four? They’re faithless, unethical, lazy, and witless. Someone with these four qualities is cast down to hell. 

Someone\marginnote{2.1} with four qualities is raised up to heaven. What four? They’re faithful, ethical, energetic, and wise. Someone with these four qualities is raised up to heaven.” 

%
\addtocontents{toc}{\let\protect\contentsline\protect\nopagecontentsline}
\chapter*{The Chapter on Bad Conduct }
\addcontentsline{toc}{chapter}{\tocchapterline{The Chapter on Bad Conduct }}
\addtocontents{toc}{\let\protect\contentsline\protect\oldcontentsline}

%
\section*{{\suttatitleacronym AN 4.221}{\suttatitletranslation Verbal Conduct }{\suttatitleroot Duccaritasutta}}
\addcontentsline{toc}{section}{\tocacronym{AN 4.221} \toctranslation{Verbal Conduct } \tocroot{Duccaritasutta}}
\markboth{Verbal Conduct }{Duccaritasutta}
\extramarks{AN 4.221}{AN 4.221}

“Mendicants,\marginnote{1.1} there are these four kinds of bad conduct by way of speech. What four? Speech that’s false, divisive, harsh, or nonsensical. These are the four kinds of bad conduct by way of speech. 

There\marginnote{1.5} are these four kinds of good conduct by way of speech. What four? Speech that’s true, harmonious, gentle, and thoughtful. These are the four kinds of good conduct by way of speech.” 

%
\section*{{\suttatitleacronym AN 4.222}{\suttatitletranslation View }{\suttatitleroot Diṭṭhisutta}}
\addcontentsline{toc}{section}{\tocacronym{AN 4.222} \toctranslation{View } \tocroot{Diṭṭhisutta}}
\markboth{View }{Diṭṭhisutta}
\extramarks{AN 4.222}{AN 4.222}

“When\marginnote{1.1} a foolish, incompetent bad person has four qualities they keep themselves broken and damaged. They deserve to be blamed and criticized by sensible people, and they make much bad karma. What four? Bad conduct by way of body, speech, and mind, and wrong view. When a foolish, incompetent bad person has these four qualities they keep themselves broken and damaged. They deserve to be blamed and criticized by sensible people, and they make much bad karma. 

When\marginnote{2.1} an astute, competent good person has four qualities they keep themselves healthy and whole. They don’t deserve to be blamed and criticized by sensible people, and they make much merit. What four? Good conduct by way of body, speech, and mind, and right view. When an astute, competent good person has these four qualities they keep themselves healthy and whole. They don’t deserve to be blamed and criticized by sensible people, and they make much merit.” 

%
\section*{{\suttatitleacronym AN 4.223}{\suttatitletranslation Ungrateful }{\suttatitleroot Akataññutāsutta}}
\addcontentsline{toc}{section}{\tocacronym{AN 4.223} \toctranslation{Ungrateful } \tocroot{Akataññutāsutta}}
\markboth{Ungrateful }{Akataññutāsutta}
\extramarks{AN 4.223}{AN 4.223}

“When\marginnote{1.1} a foolish, incompetent bad person has four qualities they keep themselves broken and damaged. They deserve to be blamed and criticized by sensible people, and they make much bad karma. What four? Bad conduct by way of body, speech, and mind, and being ungrateful and thankless. An astute person … makes much merit. … Good conduct by way of body, speech, and mind, and being grateful and thankful. …” 

%
\section*{{\suttatitleacronym AN 4.224}{\suttatitletranslation Killing Living Creatures }{\suttatitleroot Pāṇātipātīsutta}}
\addcontentsline{toc}{section}{\tocacronym{AN 4.224} \toctranslation{Killing Living Creatures } \tocroot{Pāṇātipātīsutta}}
\markboth{Killing Living Creatures }{Pāṇātipātīsutta}
\extramarks{AN 4.224}{AN 4.224}

“A\marginnote{1.1} foolish person … makes much bad karma. … They kill living creatures, steal, commit sexual misconduct, and lie. … An astute person … makes much merit. … They don’t kill living creatures, steal, commit sexual misconduct, or lie. …” 

%
\section*{{\suttatitleacronym AN 4.225}{\suttatitletranslation Path (1st) }{\suttatitleroot Paṭhamamaggasutta}}
\addcontentsline{toc}{section}{\tocacronym{AN 4.225} \toctranslation{Path (1st) } \tocroot{Paṭhamamaggasutta}}
\markboth{Path (1st) }{Paṭhamamaggasutta}
\extramarks{AN 4.225}{AN 4.225}

“A\marginnote{1.1} foolish person … makes much bad karma. … wrong view, wrong thought, wrong speech, wrong action. … An astute person … makes much merit. … right view, right thought, right speech, right action. …” 

%
\section*{{\suttatitleacronym AN 4.226}{\suttatitletranslation Path (2nd) }{\suttatitleroot Dutiyamaggasutta}}
\addcontentsline{toc}{section}{\tocacronym{AN 4.226} \toctranslation{Path (2nd) } \tocroot{Dutiyamaggasutta}}
\markboth{Path (2nd) }{Dutiyamaggasutta}
\extramarks{AN 4.226}{AN 4.226}

“A\marginnote{1.1} foolish person … makes much bad karma. … wrong livelihood, wrong effort, wrong mindfulness, and wrong immersion. … An astute person … makes much merit. … right livelihood, right effort, right mindfulness, and right immersion. …” 

%
\section*{{\suttatitleacronym AN 4.227}{\suttatitletranslation Kinds of Expression (1st) }{\suttatitleroot Paṭhamavohārapathasutta}}
\addcontentsline{toc}{section}{\tocacronym{AN 4.227} \toctranslation{Kinds of Expression (1st) } \tocroot{Paṭhamavohārapathasutta}}
\markboth{Kinds of Expression (1st) }{Paṭhamavohārapathasutta}
\extramarks{AN 4.227}{AN 4.227}

“A\marginnote{1.1} foolish person … makes much bad karma. … They say they’ve seen, heard, thought, or known something, but they haven’t. … An astute person … makes much merit. … They say they haven’t seen, heard, thought, or known something, and they haven’t. …” 

%
\section*{{\suttatitleacronym AN 4.228}{\suttatitletranslation Kinds of Expression (2nd) }{\suttatitleroot Dutiyavohārapathasutta}}
\addcontentsline{toc}{section}{\tocacronym{AN 4.228} \toctranslation{Kinds of Expression (2nd) } \tocroot{Dutiyavohārapathasutta}}
\markboth{Kinds of Expression (2nd) }{Dutiyavohārapathasutta}
\extramarks{AN 4.228}{AN 4.228}

“A\marginnote{1.1} foolish person … makes much bad karma. … They say they haven’t seen, heard, thought, or known something, but they have. … An astute person … makes much merit. … They say they’ve seen, heard, thought, or known something, and they have. …” 

%
\section*{{\suttatitleacronym AN 4.229}{\suttatitletranslation Imprudence }{\suttatitleroot Ahirikasutta}}
\addcontentsline{toc}{section}{\tocacronym{AN 4.229} \toctranslation{Imprudence } \tocroot{Ahirikasutta}}
\markboth{Imprudence }{Ahirikasutta}
\extramarks{AN 4.229}{AN 4.229}

“A\marginnote{1.1} foolish person … makes much bad karma. … They’re faithless, unethical, shameless, and imprudent. … An astute person … makes much merit. … They’re faithful, ethical, conscientious, and prudent. …” 

%
\section*{{\suttatitleacronym AN 4.230}{\suttatitletranslation Witless }{\suttatitleroot Duppaññasutta}}
\addcontentsline{toc}{section}{\tocacronym{AN 4.230} \toctranslation{Witless } \tocroot{Duppaññasutta}}
\markboth{Witless }{Duppaññasutta}
\extramarks{AN 4.230}{AN 4.230}

“A\marginnote{1.1} foolish person … makes much bad karma. … They’re faithless, unethical, lazy, and witless. … An astute person … makes much merit. … They’re faithful, ethical, energetic, and wise. When an astute, competent good person has these four qualities they keep themselves healthy and whole. They don’t deserve to be blamed and criticized by sensible people, and they make much merit.” 

%
\section*{{\suttatitleacronym AN 4.231}{\suttatitletranslation Poets }{\suttatitleroot Kavisutta}}
\addcontentsline{toc}{section}{\tocacronym{AN 4.231} \toctranslation{Poets } \tocroot{Kavisutta}}
\markboth{Poets }{Kavisutta}
\extramarks{AN 4.231}{AN 4.231}

“Mendicants,\marginnote{1.1} there are these four poets. What four? 

A\marginnote{1.3} poet who thoughtfully composes their own work, a poet who repeats the oral transmission, a poet who educates, and a poet who improvises. 

These\marginnote{1.4} are the four poets.” 

%
\addtocontents{toc}{\let\protect\contentsline\protect\nopagecontentsline}
\chapter*{The Chapter on Deeds }
\addcontentsline{toc}{chapter}{\tocchapterline{The Chapter on Deeds }}
\addtocontents{toc}{\let\protect\contentsline\protect\oldcontentsline}

%
\section*{{\suttatitleacronym AN 4.232}{\suttatitletranslation Deeds In Brief }{\suttatitleroot Saṁkhittasutta}}
\addcontentsline{toc}{section}{\tocacronym{AN 4.232} \toctranslation{Deeds In Brief } \tocroot{Saṁkhittasutta}}
\markboth{Deeds In Brief }{Saṁkhittasutta}
\extramarks{AN 4.232}{AN 4.232}

“Mendicants,\marginnote{1.1} I declare these four kinds of deeds, having realized them with my own insight. What four? 

\begin{enumerate}%
\item There are dark deeds with dark results. %
\item There are bright deeds with bright results. %
\item There are dark and bright deeds with dark and bright results. %
\item There are neither dark nor bright deeds with neither dark nor bright results, which lead to the ending of deeds. %
\end{enumerate}

These\marginnote{1.7} are the four kinds of deeds that I declare, having realized them with my own insight.” 

%
\section*{{\suttatitleacronym AN 4.233}{\suttatitletranslation Deeds in Detail }{\suttatitleroot Vitthārasutta}}
\addcontentsline{toc}{section}{\tocacronym{AN 4.233} \toctranslation{Deeds in Detail } \tocroot{Vitthārasutta}}
\markboth{Deeds in Detail }{Vitthārasutta}
\extramarks{AN 4.233}{AN 4.233}

“Mendicants,\marginnote{1.1} I declare these four kinds of deeds, having realized them with my own insight. What four? 

\begin{enumerate}%
\item There are dark deeds with dark results; %
\item bright deeds with bright results; %
\item dark and bright deeds with dark and bright results; and %
\item neither dark nor bright deeds with neither dark nor bright results, which lead to the ending of deeds. %
\end{enumerate}

And\marginnote{2.1} what are dark deeds with dark results? It’s when someone makes hurtful choices by way of body, speech, and mind. Having made these choices, they’re reborn in a hurtful world, where hurtful contacts strike them. Touched by hurtful contacts, they experience hurtful feelings that are exclusively painful—like the beings in hell. These are called dark deeds with dark results. 

And\marginnote{3.1} what are bright deeds with bright results? It’s when someone makes pleasing choices by way of body, speech, and mind. Having made these choices, they’re reborn in a pleasing world, where pleasing contacts strike them. Touched by pleasing contacts, they experience pleasing feelings that are exclusively happy—like the gods replete with glory. These are called bright deeds with bright results. 

And\marginnote{4.1} what are dark and bright deeds with dark and bright results? It’s when someone makes both hurtful and pleasing choices by way of body, speech, and mind. Having made these choices, they are reborn in a world that is both hurtful and pleasing, where hurtful and pleasing contacts strike them. Touched by both hurtful and pleasing contacts, they experience both hurtful and pleasing feelings that are a mixture of pleasure and pain—like humans, some gods, and some beings in the underworld. These are called dark and bright deeds with dark and bright results. 

And\marginnote{5.1} what are neither dark nor bright deeds with neither dark nor bright results, which lead to the ending of deeds? It’s the intention to give up dark deeds with dark results, bright deeds with bright results, and both dark and bright deeds with both dark and bright results. These are called neither dark nor bright deeds with neither dark nor bright results, which lead to the ending of deeds. 

These\marginnote{5.4} are the four kinds of deeds that I declare, having realized them with my own insight.” 

%
\section*{{\suttatitleacronym AN 4.234}{\suttatitletranslation About Soṇakāyana }{\suttatitleroot Soṇakāyanasutta}}
\addcontentsline{toc}{section}{\tocacronym{AN 4.234} \toctranslation{About Soṇakāyana } \tocroot{Soṇakāyanasutta}}
\markboth{About Soṇakāyana }{Soṇakāyanasutta}
\extramarks{AN 4.234}{AN 4.234}

Then\marginnote{1.1} \textsanskrit{Sikhāmoggallāna} the brahmin went up to the Buddha, and exchanged greetings with him. When the greetings and polite conversation were over, \textsanskrit{Sikhāmoggallāna} sat down to one side, and said to the Buddha: 

“Master\marginnote{2.1} Gotama, a few days ago the student \textsanskrit{Soṇakāyana} came to me and said: ‘The ascetic Gotama advocates not doing any deeds. So he teaches the annihilation of the world!’ The world exists through deeds, and it remains because deeds are undertaken.” 

“Brahmin,\marginnote{3.1} I can’t recall even seeing the student \textsanskrit{Soṇakāyana}, so how could we possibly have had such a discussion? I declare these four kinds of deeds, having realized them with my own insight. What four? 

\begin{enumerate}%
\item There are dark deeds with dark results; %
\item bright deeds with bright results; %
\item dark and bright deeds with dark and bright results; and %
\item neither dark nor bright deeds with neither dark nor bright results, which lead to the ending of deeds. %
\end{enumerate}

And\marginnote{4.1} what are the dark deeds with dark results? It’s when someone makes hurtful choices by way of body, speech, and mind. … Touched by hurtful contacts, they experience hurtful feelings that are exclusively painful—like the beings in hell. These are called dark deeds with dark results. 

And\marginnote{5.1} what are bright deeds with bright results? It’s when someone makes pleasing choices by way of body, speech, and mind. … Touched by pleasing contacts, they experience pleasing feelings that are exclusively happy—like the gods replete with glory. These are called bright deeds with bright results. 

And\marginnote{6.1} what are dark and bright deeds with dark and bright results? It’s when someone makes both hurtful and pleasing choices by way of body, speech, and mind. … Touched by both hurtful and pleasing contacts, they experience both hurtful and pleasing feelings that are a mixture of pleasure and pain—like humans, some gods, and some beings in the underworld. These are called dark and bright deeds with dark and bright results. 

And\marginnote{7.1} what are neither dark nor bright deeds with neither dark nor bright results, which lead to the ending of deeds? It’s the intention to give up dark deeds with dark results, bright deeds with bright results, and both dark and bright deeds with both dark and bright results. These are called neither dark nor bright deeds with neither dark nor bright results, which lead to the ending of deeds. 

These\marginnote{7.4} are the four kinds of deeds that I declare, having realized them with my own insight.” 

%
\section*{{\suttatitleacronym AN 4.235}{\suttatitletranslation Training Rules (1st) }{\suttatitleroot Paṭhamasikkhāpadasutta}}
\addcontentsline{toc}{section}{\tocacronym{AN 4.235} \toctranslation{Training Rules (1st) } \tocroot{Paṭhamasikkhāpadasutta}}
\markboth{Training Rules (1st) }{Paṭhamasikkhāpadasutta}
\extramarks{AN 4.235}{AN 4.235}

“Mendicants,\marginnote{1.1} I declare these four kinds of deeds, having realized them with my own insight. What four? 

\begin{enumerate}%
\item There are dark deeds with dark results; %
\item bright deeds with bright results; %
\item dark and bright deeds with dark and bright results; and %
\item neither dark nor bright deeds with neither dark nor bright results, which lead to the ending of deeds. %
\end{enumerate}

And\marginnote{1.7} what are the dark deeds with dark results? It’s when someone kills living creatures, steals, commits sexual misconduct, lies, and uses alcoholic drinks that cause negligence. These are called dark deeds with dark results. 

And\marginnote{2.1} what are bright deeds with bright results? It’s when someone doesn’t kill living creatures, steal, commit sexual misconduct, lie, or use alcoholic drinks that cause negligence. These are called bright deeds with bright results. 

And\marginnote{3.1} what are dark and bright deeds with dark and bright results? It’s when someone makes both hurtful and pleasing choices by way of body, speech, and mind. These are called dark and bright deeds with dark and bright results. 

And\marginnote{4.1} what are neither dark nor bright deeds with neither dark nor bright results, which lead to the ending of deeds? It’s the intention to give up dark deeds with dark results, bright deeds with bright results, and both dark and bright deeds with both dark and bright results. These are called neither dark nor bright deeds with neither dark nor bright results, which lead to the ending of deeds. 

These\marginnote{4.3} are the four kinds of deeds that I declare, having realized them with my own insight.” 

%
\section*{{\suttatitleacronym AN 4.236}{\suttatitletranslation Training Rules (2nd) }{\suttatitleroot Dutiyasikkhāpadasutta}}
\addcontentsline{toc}{section}{\tocacronym{AN 4.236} \toctranslation{Training Rules (2nd) } \tocroot{Dutiyasikkhāpadasutta}}
\markboth{Training Rules (2nd) }{Dutiyasikkhāpadasutta}
\extramarks{AN 4.236}{AN 4.236}

“Mendicants,\marginnote{1.1} I declare these four kinds of deeds, having realized them with my own insight. What four? 

\begin{enumerate}%
\item There are dark deeds with dark results; %
\item bright deeds with bright results; %
\item dark and bright deeds with dark and bright results; and %
\item neither dark nor bright deeds with neither dark nor bright results, which lead to the ending of deeds. %
\end{enumerate}

And\marginnote{2.1} what are dark deeds with dark results? It's when someone murders their mother or father or a perfected one. They maliciously shed the blood of a Realized One. Or they cause a schism in the \textsanskrit{Saṅgha}. These are called dark deeds with dark results. 

And\marginnote{3.1} what are bright deeds with bright results? It’s when someone doesn’t kill living creatures, steal, or commit sexual misconduct. They don’t use speech that’s false, divisive, harsh, or nonsensical. And they’re content, kind-hearted, with right view. These are called bright deeds with bright results. 

And\marginnote{4.1} what are dark and bright deeds with dark and bright results? It’s when someone makes both hurtful and pleasing choices by way of body, speech, and mind. These are called dark and bright deeds with dark and bright results. 

And\marginnote{5.1} what are neither dark nor bright deeds with neither dark nor bright results, which lead to the ending of deeds? It’s the intention to give up dark deeds with dark results, bright deeds with bright results, and both dark and bright deeds with both dark and bright results. These are called neither dark nor bright deeds with neither dark nor bright results, which lead to the ending of deeds. 

These\marginnote{5.3} are the four kinds of deeds that I declare, having realized them with my own insight.” 

%
\section*{{\suttatitleacronym AN 4.237}{\suttatitletranslation The Noble Path }{\suttatitleroot Ariyamaggasutta}}
\addcontentsline{toc}{section}{\tocacronym{AN 4.237} \toctranslation{The Noble Path } \tocroot{Ariyamaggasutta}}
\markboth{The Noble Path }{Ariyamaggasutta}
\extramarks{AN 4.237}{AN 4.237}

“Mendicants,\marginnote{1.1} I declare these four kinds of deeds, having realized them with my own insight. What four? 

\begin{enumerate}%
\item There are dark deeds with dark results; %
\item bright deeds with bright results; %
\item dark and bright deeds with dark and bright results; and %
\item neither dark nor bright deeds with neither dark nor bright results, which lead to the ending of deeds. %
\end{enumerate}

And\marginnote{2.1} what are dark deeds with dark results? It’s when someone makes hurtful choices by way of body, speech, and mind. These are called dark deeds with dark results. 

And\marginnote{3.1} what are bright deeds with bright results? It’s when someone makes pleasing choices by way of body, speech, and mind. These are called bright deeds with bright results. 

And\marginnote{4.1} what are dark and bright deeds with dark and bright results? It’s when someone makes both hurtful and pleasing choices by way of body, speech, and mind. These are called dark and bright deeds with dark and bright results. 

And\marginnote{5.1} what are neither dark nor bright deeds with neither dark nor bright results, which lead to the ending of deeds? Right view, right thought, right speech, right action, right livelihood, right effort, right mindfulness, and right immersion. These are called neither dark nor bright deeds with neither dark nor bright results, which lead to the ending of deeds. 

These\marginnote{5.4} are the four kinds of deeds that I declare, having realized them with my own insight.” 

%
\section*{{\suttatitleacronym AN 4.238}{\suttatitletranslation Awakening Factors }{\suttatitleroot Bojjhaṅgasutta}}
\addcontentsline{toc}{section}{\tocacronym{AN 4.238} \toctranslation{Awakening Factors } \tocroot{Bojjhaṅgasutta}}
\markboth{Awakening Factors }{Bojjhaṅgasutta}
\extramarks{AN 4.238}{AN 4.238}

“Mendicants,\marginnote{1.1} I declare these four kinds of deeds, having realized them with my own insight…. 

And\marginnote{1.2} what are dark deeds with dark results? It’s when someone makes hurtful choices by way of body, speech, and mind. These are called dark deeds with dark results. 

And\marginnote{2.1} what are bright deeds with bright results? It’s when someone makes pleasing choices by way of body, speech, and mind. These are called bright deeds with bright results. 

And\marginnote{3.1} what are dark and bright deeds with dark and bright results? It’s when someone makes both hurtful and pleasing choices by way of body, speech, and mind. These are called dark and bright deeds with dark and bright results. 

And\marginnote{4.1} what are neither dark nor bright deeds with neither dark nor bright results, which lead to the ending of deeds? The awakening factors of mindfulness, investigation of principles, energy, rapture, tranquility, immersion, and equanimity. These are called neither dark nor bright deeds with neither dark nor bright results, which lead to the ending of deeds. 

These\marginnote{4.4} are the four kinds of deeds that I declare, having realized them with my own insight.” 

%
\section*{{\suttatitleacronym AN 4.239}{\suttatitletranslation Blameworthy }{\suttatitleroot Sāvajjasutta}}
\addcontentsline{toc}{section}{\tocacronym{AN 4.239} \toctranslation{Blameworthy } \tocroot{Sāvajjasutta}}
\markboth{Blameworthy }{Sāvajjasutta}
\extramarks{AN 4.239}{AN 4.239}

“Mendicants,\marginnote{1.1} someone with four qualities is cast down to hell. What four? Blameworthy deeds by way of body, speech, and mind, and blameworthy view. Someone with these four qualities is cast down to hell. 

Someone\marginnote{2.1} with four qualities is raised up to heaven. What four? Blameless deeds by way of body, speech, and mind, and blameless view. Someone with these four qualities is raised up to heaven.” 

%
\section*{{\suttatitleacronym AN 4.240}{\suttatitletranslation Pleasing }{\suttatitleroot Abyābajjhasutta}}
\addcontentsline{toc}{section}{\tocacronym{AN 4.240} \toctranslation{Pleasing } \tocroot{Abyābajjhasutta}}
\markboth{Pleasing }{Abyābajjhasutta}
\extramarks{AN 4.240}{AN 4.240}

“Mendicants,\marginnote{1.1} someone with four qualities is cast down to hell. What four? Hurtful deeds by way of body, speech, and mind, and hurtful view. Someone with these four qualities is cast down to hell. 

Someone\marginnote{2.1} with four qualities is raised up to heaven. What four? Pleasing deeds by way of body, speech, and mind, and pleasing view. Someone with these four qualities is raised up to heaven.” 

%
\section*{{\suttatitleacronym AN 4.241}{\suttatitletranslation Ascetics }{\suttatitleroot Samaṇasutta}}
\addcontentsline{toc}{section}{\tocacronym{AN 4.241} \toctranslation{Ascetics } \tocroot{Samaṇasutta}}
\markboth{Ascetics }{Samaṇasutta}
\extramarks{AN 4.241}{AN 4.241}

“‘Only\marginnote{1.1} here is there a first ascetic, here a second ascetic, here a third ascetic, and here a fourth ascetic. Other sects are empty of ascetics.’ This, mendicants, is how you should rightly roar your lion’s roar. 

And\marginnote{2.1} who is the first ascetic? It’s a mendicant who—with the ending of three fetters—is a stream-enterer, not liable to be reborn in the underworld, bound for awakening. This is the first ascetic. 

And\marginnote{3.1} who is the second ascetic? It’s a mendicant who—with the ending of three fetters, and the weakening of greed, hate, and delusion—is a once-returner. They come back to this world once only, then make an end of suffering. This is the second ascetic. 

And\marginnote{4.1} who is the third ascetic? It’s a mendicant who—with the ending of the five lower fetters—is reborn spontaneously. They’re extinguished there, and are not liable to return from that world. This is the third ascetic. 

And\marginnote{5.1} who is the fourth ascetic? It’s a mendicant who realizes the undefiled freedom of heart and freedom by wisdom in this very life. And they live having realized it with their own insight due to the ending of defilements. This is the fourth ascetic. 

‘Only\marginnote{6.1} here is there a first ascetic, here a second ascetic, here a third ascetic, and here a fourth ascetic. Other sects are empty of ascetics.’ This, mendicants, is how you should rightly roar your lion’s roar.” 

%
\section*{{\suttatitleacronym AN 4.242}{\suttatitletranslation Benefits of a Good Person }{\suttatitleroot Sappurisānisaṁsasutta}}
\addcontentsline{toc}{section}{\tocacronym{AN 4.242} \toctranslation{Benefits of a Good Person } \tocroot{Sappurisānisaṁsasutta}}
\markboth{Benefits of a Good Person }{Sappurisānisaṁsasutta}
\extramarks{AN 4.242}{AN 4.242}

“Mendicants,\marginnote{1.1} you can expect four benefits from relying on a good person. What four? Growth in noble ethics, immersion, wisdom, and freedom. You can expect these four benefits from relying on a good person.” 

%
\addtocontents{toc}{\let\protect\contentsline\protect\nopagecontentsline}
\chapter*{The Chapter on Perils of Offenses }
\addcontentsline{toc}{chapter}{\tocchapterline{The Chapter on Perils of Offenses }}
\addtocontents{toc}{\let\protect\contentsline\protect\oldcontentsline}

%
\section*{{\suttatitleacronym AN 4.243}{\suttatitletranslation Schism in the Saṅgha }{\suttatitleroot Saṁghabhedakasutta}}
\addcontentsline{toc}{section}{\tocacronym{AN 4.243} \toctranslation{Schism in the Saṅgha } \tocroot{Saṁghabhedakasutta}}
\markboth{Schism in the Saṅgha }{Saṁghabhedakasutta}
\extramarks{AN 4.243}{AN 4.243}

At\marginnote{1.1} one time the Buddha was staying near Kosambi, in Ghosita’s Monastery. Then Venerable Ānanda went up to the Buddha, bowed, and sat down to one side. The Buddha said to him, “Well, Ānanda, has that disciplinary issue been settled yet?” 

“How\marginnote{1.4} could it be, sir? Venerable Anuruddha’s pupil \textsanskrit{Bāhiya} remains entirely committed to creating a schism in the \textsanskrit{Saṅgha}. But Anuruddha doesn’t think to say a single word about it.” 

“But\marginnote{2.1} Ānanda, since when has Anuruddha been involved in disciplinary issues in the midst of the \textsanskrit{Saṅgha}? Shouldn’t you, together with \textsanskrit{Sāriputta} and \textsanskrit{Moggallāna}, settle all disciplinary issues that come up? 

A\marginnote{3.1} bad monk sees four reasons to relish schism in the \textsanskrit{Saṅgha}. What four? Take an unethical monk, of bad qualities, filthy, with suspicious behavior, underhand, no true ascetic or spiritual practitioner—though claiming to be one—rotten inside, corrupt, and depraved. He thinks: ‘Suppose the monks know that I’m a bad monk … If they’re in harmony, they’ll expel me, but if they’re divided they won’t.’ A bad monk sees this as the first reason to relish schism in the \textsanskrit{Saṅgha}. 

Furthermore,\marginnote{4.1} a bad monk has wrong view, he’s attached to an extremist view. He thinks: ‘Suppose the monks know that I have wrong view … If they’re in harmony they’ll expel me, but if they’re divided they won’t.’ A bad monk sees this as the second reason to relish schism in the \textsanskrit{Saṅgha}. 

Furthermore,\marginnote{5.1} a bad monk has wrong livelihood and earns a living by wrong livelihood. He thinks: ‘Suppose the monks know that I have wrong livelihood … If they’re in harmony they’ll expel me, but if they’re divided they won’t.’ A bad monk sees this as the third reason to relish schism in the \textsanskrit{Saṅgha}. 

Furthermore,\marginnote{6.1} a bad monk desires material possessions, honor, and admiration. He thinks: ‘Suppose the monks know that I desire material possessions, honor, and admiration. If they’re in harmony they won’t honor, respect, revere, or venerate me, but if they’re divided they will.’ A bad monk sees this as the fourth reason to relish schism in the \textsanskrit{Saṅgha}. 

A\marginnote{6.7} bad monk sees these four reasons to relish schism in the \textsanskrit{Saṅgha}.” 

%
\section*{{\suttatitleacronym AN 4.244}{\suttatitletranslation Perils of Offenses }{\suttatitleroot Āpattibhayasutta}}
\addcontentsline{toc}{section}{\tocacronym{AN 4.244} \toctranslation{Perils of Offenses } \tocroot{Āpattibhayasutta}}
\markboth{Perils of Offenses }{Āpattibhayasutta}
\extramarks{AN 4.244}{AN 4.244}

“Mendicants,\marginnote{1.1} there are these four perils of offenses. What four? 

Suppose\marginnote{1.3} they were to arrest a bandit, a criminal and present him to the king, saying: ‘Your Majesty, this is a bandit, a criminal. May Your Majesty punish them!’ The king would say: ‘Go, my men, and tie this man’s arms tightly behind his back with a strong rope. Shave his head and march him from street to street and square to square to the beating of a harsh drum. Then take him out the south gate and there, to the south of the city, chop off his head.’ The king’s men would do as they were told. Then a bystander might think: ‘This man must have done a truly bad and reprehensible deed, a capital offense. There’s no way I’d ever do such a bad and reprehensible deed, a capital offense.’ In the same way, take any monk or nun who has set up such an acute perception of peril regarding expulsion offenses. It can be expected that if they haven’t committed an expulsion offense they won’t, and if they committed one they will deal with it properly. 

Suppose\marginnote{2.1} a man was to put on a black cloth, mess up his hair, and put a club on his shoulder. Then he approaches a large crowd and says: ‘Sirs, I’ve done a bad and reprehensible deed, deserving of clubbing. I submit to your pleasure.’ Then a bystander might think: ‘This man must have done a truly bad and reprehensible deed, deserving of clubbing. … There’s no way I’d ever do such a bad and reprehensible deed, deserving of clubbing.’ In the same way, take any monk or nun who has set up such an acute perception of peril regarding suspension offenses. It can be expected that if they haven’t committed a suspension offense they won’t, and if they committed one they will deal with it properly. 

Suppose\marginnote{3.1} a man was to put on a black cloth, mess up his hair, and put a sack of ashes on his shoulder. Then he approaches a large crowd and says: ‘Sirs, I’ve done a bad and reprehensible deed, deserving of a sack of ashes. I submit to your pleasure.’ Then a bystander might think: ‘This man must have done a truly bad and reprehensible deed, deserving of a sack of ashes. … There’s no way I’d ever do such a bad and reprehensible deed, deserving of a sack of ashes.’ In the same way, take any monk or nun who has set up such an acute perception of peril regarding confessable offenses. It can be expected that if they haven’t committed a confessable offense they won’t, and if they committed one they will deal with it properly. 

Suppose\marginnote{4.1} a man was to put on a black cloth and mess up his hair. Then he approaches a large crowd and says: ‘Sirs, I’ve done a bad and reprehensible deed, deserving of criticism. I submit to your pleasure.’ Then a bystander might think: ‘This man must have done a truly bad and reprehensible deed, deserving of criticism. … There’s no way I’d ever do such a bad and reprehensible deed, deserving of criticism.’ In the same way, take any monk or nun who has set up such an acute perception of peril regarding acknowledgable offenses. It can be expected that if they haven’t committed an acknowledgeable offense they won’t, and if they committed one they will deal with it properly. 

These\marginnote{4.11} are the four perils of offenses.” 

%
\section*{{\suttatitleacronym AN 4.245}{\suttatitletranslation The Benefits of Training }{\suttatitleroot Sikkhānisaṁsasutta}}
\addcontentsline{toc}{section}{\tocacronym{AN 4.245} \toctranslation{The Benefits of Training } \tocroot{Sikkhānisaṁsasutta}}
\markboth{The Benefits of Training }{Sikkhānisaṁsasutta}
\extramarks{AN 4.245}{AN 4.245}

“Mendicants,\marginnote{1.1} this spiritual life is lived with training as its benefit, with wisdom as its overseer, with freedom as its core, and with mindfulness as its ruler. 

And\marginnote{1.2} how is training its benefit? Firstly, I laid down for my disciples the training that deals with supplementary regulations in order to inspire confidence in those without it and to increase confidence in those who have it. They undertake whatever supplementary regulations I have laid down, keeping them unbroken, impeccable, spotless, and unmarred. 

Furthermore,\marginnote{2.1} I laid down for my disciples the training that deals with the fundamentals of the spiritual life in order to rightly end suffering in every way. They undertake whatever training that deals with the fundamentals of the spiritual life I have laid down, keeping it unbroken, impeccable, spotless, and unmarred. That’s how training is its benefit. 

And\marginnote{3.1} how is wisdom its overseer? I taught the Dhamma to my disciples in order to rightly end suffering in every way. They examine with wisdom any teachings I taught them. That’s how wisdom is its overseer. 

And\marginnote{4.1} how is freedom its core? I taught the Dhamma to my disciples in order to rightly end suffering in every way. They experience through freedom any teachings I taught them. That’s how freedom is its core. 

And\marginnote{5.1} how is mindfulness its ruler? Mindfulness is well established in oneself: ‘In this way I’ll fulfill the training dealing with supplementary regulations, or support with wisdom in every situation the training dealing with supplementary regulations I’ve already fulfilled.’ Mindfulness is well established in oneself: ‘In this way I’ll fulfill the training dealing with the fundamentals of the spiritual life, or support with wisdom in every situation the training dealing with the fundamentals of the spiritual life I’ve already fulfilled.’ Mindfulness is well established in oneself: ‘In this way I’ll examine with wisdom the teaching that I haven’t yet examined, or support with wisdom in every situation the teaching I’ve already examined.’ Mindfulness is well established in oneself: ‘In this way I’ll experience through freedom the teaching that I haven’t yet experienced, or support with wisdom in every situation the teaching I’ve already experienced.’ That’s how mindfulness is its ruler. 

‘This\marginnote{5.7} spiritual life is lived with training as its benefit, with wisdom as its overseer, with freedom as its core, and with mindfulness as its ruler.’ That’s what I said, and this is why I said it.” 

%
\section*{{\suttatitleacronym AN 4.246}{\suttatitletranslation Lying Postures }{\suttatitleroot Seyyāsutta}}
\addcontentsline{toc}{section}{\tocacronym{AN 4.246} \toctranslation{Lying Postures } \tocroot{Seyyāsutta}}
\markboth{Lying Postures }{Seyyāsutta}
\extramarks{AN 4.246}{AN 4.246}

“Mendicants,\marginnote{1.1} there are these four ways of lying down. What four? The ways a corpse, a pleasure seeker, a lion, and a Realized One lie down. 

And\marginnote{1.4} how does a corpse lie down? Corpses usually lie flat on their backs. This is called the way a corpse lies down. 

And\marginnote{2.1} how does a pleasure seeker lie down? Pleasure seekers usually lie down on their left side. This is called the way a pleasure seeker lies down. 

And\marginnote{3.1} how does a lion lie down? The lion, king of beasts, lies down on the right side, placing one foot on top of the other, with his tail tucked between his thighs. When he wakes, he lifts his front quarters and checks his hind quarters. If he sees that any part of his body is disordered or displaced, he is displeased. But if he sees that no part of his body is disordered or displaced, he is pleased. This is called the way a lion lies down. 

And\marginnote{4.1} how does a Realized One lie down? It’s when a Realized One, quite secluded from sensual pleasures, secluded from unskillful qualities, enters and remains in the first absorption … second absorption … third absorption … fourth absorption. This is called the way a Realized One lies down. 

These\marginnote{4.4} are the four ways of lying down.” 

%
\section*{{\suttatitleacronym AN 4.247}{\suttatitletranslation Worthy of a Monument }{\suttatitleroot Thūpārahasutta}}
\addcontentsline{toc}{section}{\tocacronym{AN 4.247} \toctranslation{Worthy of a Monument } \tocroot{Thūpārahasutta}}
\markboth{Worthy of a Monument }{Thūpārahasutta}
\extramarks{AN 4.247}{AN 4.247}

“Mendicants,\marginnote{1.1} these four are worthy of a monument. What four? A Realized One, a perfected one, a fully awakened Buddha; a Buddha awakened for themselves; a disciple of a Realized One; and a wheel-turning monarch. These four are worthy of a monument.” 

%
\section*{{\suttatitleacronym AN 4.248}{\suttatitletranslation The Growth of Wisdom }{\suttatitleroot Paññāvuddhisutta}}
\addcontentsline{toc}{section}{\tocacronym{AN 4.248} \toctranslation{The Growth of Wisdom } \tocroot{Paññāvuddhisutta}}
\markboth{The Growth of Wisdom }{Paññāvuddhisutta}
\extramarks{AN 4.248}{AN 4.248}

“Mendicants,\marginnote{1.1} these four things lead to the growth of wisdom. What four? Associating with good people, listening to the true teaching, proper attention, and practicing in line with the teaching. These four things lead to the growth of wisdom.” 

%
\section*{{\suttatitleacronym AN 4.249}{\suttatitletranslation Very Helpful }{\suttatitleroot Bahukārasutta}}
\addcontentsline{toc}{section}{\tocacronym{AN 4.249} \toctranslation{Very Helpful } \tocroot{Bahukārasutta}}
\markboth{Very Helpful }{Bahukārasutta}
\extramarks{AN 4.249}{AN 4.249}

“Mendicants,\marginnote{1.1} these four things are very helpful to a human being. What four? Associating with good people, listening to the true teaching, proper attention, and practicing in line with the teaching. These four things are very helpful to a human being.” 

%
\section*{{\suttatitleacronym AN 4.250}{\suttatitletranslation Expressions (1st) }{\suttatitleroot Paṭhamavohārasutta}}
\addcontentsline{toc}{section}{\tocacronym{AN 4.250} \toctranslation{Expressions (1st) } \tocroot{Paṭhamavohārasutta}}
\markboth{Expressions (1st) }{Paṭhamavohārasutta}
\extramarks{AN 4.250}{AN 4.250}

“Mendicants,\marginnote{1.1} there are these four ignoble expressions. What four? Saying you’ve seen, heard, thought, or known something, but you haven’t. These are the four ignoble expressions.” 

%
\section*{{\suttatitleacronym AN 4.251}{\suttatitletranslation Expressions (2nd) }{\suttatitleroot Dutiyavohārasutta}}
\addcontentsline{toc}{section}{\tocacronym{AN 4.251} \toctranslation{Expressions (2nd) } \tocroot{Dutiyavohārasutta}}
\markboth{Expressions (2nd) }{Dutiyavohārasutta}
\extramarks{AN 4.251}{AN 4.251}

“Mendicants,\marginnote{1.1} there are these four noble expressions. What four? Saying you haven’t seen, heard, thought, or known something, and you haven’t. These are the four noble expressions.” 

%
\section*{{\suttatitleacronym AN 4.252}{\suttatitletranslation Expressions (3rd) }{\suttatitleroot Tatiyavohārasutta}}
\addcontentsline{toc}{section}{\tocacronym{AN 4.252} \toctranslation{Expressions (3rd) } \tocroot{Tatiyavohārasutta}}
\markboth{Expressions (3rd) }{Tatiyavohārasutta}
\extramarks{AN 4.252}{AN 4.252}

“Mendicants,\marginnote{1.1} there are these four ignoble expressions. What four? Saying you haven’t seen, heard, thought, or known something, and you have. These are the four ignoble expressions.” 

%
\section*{{\suttatitleacronym AN 4.253}{\suttatitletranslation Expressions (4th) }{\suttatitleroot Catutthavohārasutta}}
\addcontentsline{toc}{section}{\tocacronym{AN 4.253} \toctranslation{Expressions (4th) } \tocroot{Catutthavohārasutta}}
\markboth{Expressions (4th) }{Catutthavohārasutta}
\extramarks{AN 4.253}{AN 4.253}

“Mendicants,\marginnote{1.1} there are these four noble expressions. What four? Saying you’ve seen, heard, thought, or known something, and you have. These are the four noble expressions.” 

%
\addtocontents{toc}{\let\protect\contentsline\protect\nopagecontentsline}
\chapter*{The Chapter on Direct Knowledges }
\addcontentsline{toc}{chapter}{\tocchapterline{The Chapter on Direct Knowledges }}
\addtocontents{toc}{\let\protect\contentsline\protect\oldcontentsline}

%
\section*{{\suttatitleacronym AN 4.254}{\suttatitletranslation Insight }{\suttatitleroot Abhiññāsutta}}
\addcontentsline{toc}{section}{\tocacronym{AN 4.254} \toctranslation{Insight } \tocroot{Abhiññāsutta}}
\markboth{Insight }{Abhiññāsutta}
\extramarks{AN 4.254}{AN 4.254}

Mendicants,\marginnote{1.1} there are these four things. What four? There are things that should be completely understood by direct knowledge. There are things that should be given up by direct knowledge. There are things that should be developed by direct knowledge. There are things that should be realized by direct knowledge. 

And\marginnote{2.1} what are the things that should be completely understood by direct knowledge? The five grasping aggregates. These are called the things that should be completely understood by direct knowledge. 

And\marginnote{3.1} what are the things that should be given up by direct knowledge? Ignorance and craving for continued existence. These are called the things that should be given up by direct knowledge. 

And\marginnote{4.1} what are the things that should be developed by direct knowledge? Serenity and discernment. These are called the things that should be developed by direct knowledge. 

And\marginnote{5.1} what are the things that should be realized by direct knowledge? Knowledge and freedom. These are called the things that should be realized by direct knowledge. 

These\marginnote{5.4} are the four things.” 

%
\section*{{\suttatitleacronym AN 4.255}{\suttatitletranslation Searches }{\suttatitleroot Pariyesanāsutta}}
\addcontentsline{toc}{section}{\tocacronym{AN 4.255} \toctranslation{Searches } \tocroot{Pariyesanāsutta}}
\markboth{Searches }{Pariyesanāsutta}
\extramarks{AN 4.255}{AN 4.255}

“Mendicants,\marginnote{1.1} there are these four ignoble searches. What four? Someone liable to old age searches only for what grows old. Someone liable to sickness searches only for what gets sick. Someone liable to death searches only for what dies. Someone whose nature is defiled searches only for what is defiled. These are the four ignoble searches. 

There\marginnote{2.1} are these four noble searches. What four? Someone who is liable to grow old, knowing the drawback in what grows old, searches for the unaging supreme sanctuary, extinguishment. Someone who is liable to get sick, knowing the drawback in what gets sick, searches for the sickness-free supreme sanctuary, extinguishment. Someone who is liable to die, knowing the drawback in what dies, searches for the deathless supreme sanctuary, extinguishment. Someone whose nature is defiled, knowing the drawback in what is defiled, searches for the undefiled supreme sanctuary, extinguishment. These are the four noble searches.” 

%
\section*{{\suttatitleacronym AN 4.256}{\suttatitletranslation Ways of Being Inclusive }{\suttatitleroot Saṅgahavatthusutta}}
\addcontentsline{toc}{section}{\tocacronym{AN 4.256} \toctranslation{Ways of Being Inclusive } \tocroot{Saṅgahavatthusutta}}
\markboth{Ways of Being Inclusive }{Saṅgahavatthusutta}
\extramarks{AN 4.256}{AN 4.256}

“Mendicants,\marginnote{1.1} there are these four ways of being inclusive. What four? Giving, kindly words, taking care, and equality. These are the four ways of being inclusive.” 

%
\section*{{\suttatitleacronym AN 4.257}{\suttatitletranslation With Māluṅkyaputta }{\suttatitleroot Mālukyaputtasutta}}
\addcontentsline{toc}{section}{\tocacronym{AN 4.257} \toctranslation{With Māluṅkyaputta } \tocroot{Mālukyaputtasutta}}
\markboth{With Māluṅkyaputta }{Mālukyaputtasutta}
\extramarks{AN 4.257}{AN 4.257}

Then\marginnote{1.1} Venerable \textsanskrit{Māluṅkyaputta} went up to the Buddha, bowed, sat down to one side, and said to him: 

“Sir,\marginnote{2.1} may the Buddha please teach me Dhamma in brief. When I’ve heard it, I’ll live alone, withdrawn, diligent, keen, and resolute.” 

“Well\marginnote{2.2} now, \textsanskrit{Māluṅkyaputta}, what are we to say to the young monks, when even an old man like you, elderly and senior, asks the Realized One for brief advice?” 

“Sir,\marginnote{2.4} may the Buddha please teach me Dhamma in brief! May the Holy One teach me the Dhamma in brief! Hopefully I can understand the meaning of what the Buddha says! Hopefully I can be an heir of the Buddha’s teaching!” 

“\textsanskrit{Māluṅkyaputta},\marginnote{3.1} there are four things that give rise to craving in a mendicant. What four? For the sake of robes, almsfood, lodgings, or rebirth in this or that state. These are the four things that give rise to craving in a mendicant. That craving is given up by a mendicant, cut off at the root, made like a palm stump, obliterated, and unable to arise in the future. Then they’re called a mendicant who has cut off craving, untied the fetters, and by rightly comprehending conceit has made an end of suffering.” 

When\marginnote{4.1} \textsanskrit{Māluṅkyaputta} had been given this advice by the Buddha, he got up from his seat, bowed, and respectfully circled the Buddha, keeping him on his right, before leaving. Then \textsanskrit{Māluṅkyaputta}, living alone, withdrawn, diligent, keen, and resolute, soon realized the supreme culmination of the spiritual path in this very life. He lived having achieved with his own insight the goal for which gentlemen rightly go forth from the lay life to homelessness. 

He\marginnote{4.3} understood: “Rebirth is ended; the spiritual journey has been completed; what had to be done has been done; there is no return to any state of existence.” And Venerable \textsanskrit{Māluṅkyaputta} became one of the perfected. 

%
\section*{{\suttatitleacronym AN 4.258}{\suttatitletranslation Families }{\suttatitleroot Kulasutta}}
\addcontentsline{toc}{section}{\tocacronym{AN 4.258} \toctranslation{Families } \tocroot{Kulasutta}}
\markboth{Families }{Kulasutta}
\extramarks{AN 4.258}{AN 4.258}

“Mendicants,\marginnote{1.1} when families don’t stay wealthy for long, it’s always for one or more of these four reasons. What four? They don’t look for what’s lost; they don’t fix old things; they eat and drink too much; or they put an unethical woman or man in charge. When families don’t stay wealthy for long, it’s always for one or more of these four reasons. 

When\marginnote{2.1} families do stay wealthy for long, it’s always for one or more of these four reasons. What four? They look for what’s lost; they fix old things; they eat and drink in moderation; and they put an ethical woman or man in charge. When families do stay wealthy for long, it’s always for one or more of these four reasons.” 

%
\section*{{\suttatitleacronym AN 4.259}{\suttatitletranslation A Thoroughbred (1st) }{\suttatitleroot Paṭhamaājānīyasutta}}
\addcontentsline{toc}{section}{\tocacronym{AN 4.259} \toctranslation{A Thoroughbred (1st) } \tocroot{Paṭhamaājānīyasutta}}
\markboth{A Thoroughbred (1st) }{Paṭhamaājānīyasutta}
\extramarks{AN 4.259}{AN 4.259}

“Mendicants,\marginnote{1.1} a fine royal thoroughbred with four factors is worthy of a king, fit to serve a king, and considered a factor of kingship. What four? It’s when a fine royal thoroughbred is beautiful, strong, fast, and well-proportioned. A fine royal thoroughbred with these four factors is worthy of a king. … 

In\marginnote{2.1} the same way, a mendicant with four qualities is worthy of offerings dedicated to the gods, worthy of hospitality, worthy of a religious donation, worthy of veneration with joined palms, and is the supreme field of merit for the world. What four? It’s when a mendicant is beautiful, strong, fast, and well proportioned. 

And\marginnote{3.1} how is a mendicant beautiful? It’s when a mendicant is ethical, restrained in the code of conduct, conducting themselves well and seeking alms in suitable places. Seeing danger in the slightest fault, they keep the rules they’ve undertaken. That’s how a mendicant is beautiful. 

And\marginnote{4.1} how is a mendicant strong? It’s when a mendicant lives with energy roused up for giving up unskillful qualities and embracing skillful qualities. They are strong, staunchly vigorous, not slacking off when it comes to developing skillful qualities. That’s how a mendicant is strong. 

And\marginnote{5.1} how is a mendicant fast? It’s when they truly understand: ‘This is suffering’ … ‘This is the origin of suffering’ … ‘This is the cessation of suffering’ … ‘This is the practice that leads to the cessation of suffering’. That’s how a mendicant is fast. 

And\marginnote{6.1} how is a mendicant well proportioned? It’s when a mendicant receives robes, almsfood, lodgings, and medicines and supplies for the sick. That’s how a mendicant is well proportioned. 

A\marginnote{7.1} mendicant with these four qualities … is the supreme field of merit for the world.” 

%
\section*{{\suttatitleacronym AN 4.260}{\suttatitletranslation A Thoroughbred (2nd) }{\suttatitleroot Dutiyaājānīyasutta}}
\addcontentsline{toc}{section}{\tocacronym{AN 4.260} \toctranslation{A Thoroughbred (2nd) } \tocroot{Dutiyaājānīyasutta}}
\markboth{A Thoroughbred (2nd) }{Dutiyaājānīyasutta}
\extramarks{AN 4.260}{AN 4.260}

“Mendicants,\marginnote{1.1} a fine royal thoroughbred with four factors is worthy of a king, fit to serve a king, and considered a factor of kingship. What four? It’s when a fine royal thoroughbred is beautiful, strong, fast, and well-proportioned. A fine royal thoroughbred with these four factors is worthy of a king. … 

In\marginnote{2.1} the same way, a mendicant with four qualities is worthy of offerings dedicated to the gods, worthy of hospitality, worthy of a religious donation, worthy of veneration with joined palms, and is the supreme field of merit for the world. What four? It’s when a mendicant is beautiful, strong, fast, and well proportioned. 

And\marginnote{3.1} how is a mendicant beautiful? It’s when a mendicant is ethical, restrained in the code of conduct, conducting themselves well and seeking alms in suitable places. Seeing danger in the slightest fault, they keep the rules they’ve undertaken. That’s how a mendicant is beautiful. 

And\marginnote{4.1} how is a mendicant strong? It’s when a mendicant lives with energy roused up for giving up unskillful qualities and embracing skillful qualities. They are strong, staunchly vigorous, not slacking off when it comes to developing skillful qualities. That’s how a mendicant is strong. 

And\marginnote{5.1} how is a mendicant fast? It’s when a mendicant realizes the undefiled freedom of heart and freedom by wisdom in this very life. And they live having realized it with their own insight due to the ending of defilements. That’s how a mendicant is fast. 

And\marginnote{6.1} how is a mendicant well proportioned? It’s when a mendicant receives robes, almsfood, lodgings, and medicines and supplies for the sick. That’s how a mendicant is well proportioned. 

A\marginnote{7.1} mendicant with these four qualities … is the supreme field of merit for the world.” 

%
\section*{{\suttatitleacronym AN 4.261}{\suttatitletranslation Powers }{\suttatitleroot Balasutta}}
\addcontentsline{toc}{section}{\tocacronym{AN 4.261} \toctranslation{Powers } \tocroot{Balasutta}}
\markboth{Powers }{Balasutta}
\extramarks{AN 4.261}{AN 4.261}

“Mendicants,\marginnote{1.1} there are these four powers. What four? The powers of energy, mindfulness, immersion, and wisdom. These are the four powers.” 

%
\section*{{\suttatitleacronym AN 4.262}{\suttatitletranslation Wilderness }{\suttatitleroot Araññasutta}}
\addcontentsline{toc}{section}{\tocacronym{AN 4.262} \toctranslation{Wilderness } \tocroot{Araññasutta}}
\markboth{Wilderness }{Araññasutta}
\extramarks{AN 4.262}{AN 4.262}

“Mendicants,\marginnote{1.1} when a mendicant has four qualities they’re not ready to frequent remote lodgings in the wilderness and the forest. What four? They have sensual, malicious, and cruel thoughts; or they’re witless, dull, and stupid. When a mendicant has these four qualities they’re not ready to frequent remote lodgings in the wilderness and the forest. 

When\marginnote{2.1} a mendicant has four qualities they’re ready to frequent remote lodgings in the wilderness and the forest. What four? They have thoughts of renunciation, good will, and harmlessness; and they’re wise, bright, and clever. When a mendicant has these four qualities they’re ready to frequent remote lodgings in the wilderness and the forest.” 

%
\section*{{\suttatitleacronym AN 4.263}{\suttatitletranslation Deeds }{\suttatitleroot Kammasutta}}
\addcontentsline{toc}{section}{\tocacronym{AN 4.263} \toctranslation{Deeds } \tocroot{Kammasutta}}
\markboth{Deeds }{Kammasutta}
\extramarks{AN 4.263}{AN 4.263}

“When\marginnote{1.1} a foolish, incompetent bad person has four qualities they keep themselves broken and damaged. They deserve to be blamed and criticized by sensible people, and they make much bad karma. What four? Blameworthy deeds by way of body, speech, and mind, and blameworthy view. When a foolish, incompetent bad person has these four qualities they keep themselves broken and damaged. They deserve to be blamed and criticized by sensible people, and they make much bad karma. 

When\marginnote{2.1} an astute, competent good person has four qualities they keep themselves healthy and whole. They don’t deserve to be blamed and criticized by sensible people, and they make much merit. What four? Blameless deeds by way of body, speech, and mind, and blameless view. When an astute, competent good person has these four qualities they keep themselves healthy and whole. They don’t deserve to be blamed and criticized by sensible people, and they make much merit.” 

%
\addtocontents{toc}{\let\protect\contentsline\protect\nopagecontentsline}
\chapter*{The Chapter on Ways of Performing Deeds }
\addcontentsline{toc}{chapter}{\tocchapterline{The Chapter on Ways of Performing Deeds }}
\addtocontents{toc}{\let\protect\contentsline\protect\oldcontentsline}

%
\section*{{\suttatitleacronym AN 4.264}{\suttatitletranslation Killing Living Creatures }{\suttatitleroot Pāṇātipātīsutta}}
\addcontentsline{toc}{section}{\tocacronym{AN 4.264} \toctranslation{Killing Living Creatures } \tocroot{Pāṇātipātīsutta}}
\markboth{Killing Living Creatures }{Pāṇātipātīsutta}
\extramarks{AN 4.264}{AN 4.264}

“Mendicants,\marginnote{1.1} someone with four qualities is cast down to hell. What four? They themselves kill living creatures; they encourage others to kill living creatures; they approve of killing living creatures; and they praise killing living creatures. Someone with these four qualities is cast down to hell. 

Someone\marginnote{2.1} with four qualities is raised up to heaven. What four? They don’t themselves kill living creatures; they encourage others to not kill living creatures; they approve of not killing living creatures; and they praise not killing living creatures. Someone with these four qualities is raised up to heaven.” 

%
\section*{{\suttatitleacronym AN 4.265}{\suttatitletranslation Stealing }{\suttatitleroot Adinnādāyīsutta}}
\addcontentsline{toc}{section}{\tocacronym{AN 4.265} \toctranslation{Stealing } \tocroot{Adinnādāyīsutta}}
\markboth{Stealing }{Adinnādāyīsutta}
\extramarks{AN 4.265}{AN 4.265}

“Mendicants,\marginnote{1.1} someone with four qualities is cast down to hell. What four? They themselves steal … Someone with four qualities is raised up to heaven. … 

They\marginnote{2.1} don’t themselves steal … 

%
\section*{{\suttatitleacronym AN 4.266}{\suttatitletranslation Misconduct }{\suttatitleroot Micchācārīsutta}}
\addcontentsline{toc}{section}{\tocacronym{AN 4.266} \toctranslation{Misconduct } \tocroot{Micchācārīsutta}}
\markboth{Misconduct }{Micchācārīsutta}
\extramarks{AN 4.266}{AN 4.266}

…\marginnote{1.1} They themselves commit sexual misconduct … 

They\marginnote{2.1} themselves don’t commit sexual misconduct … 

%
\section*{{\suttatitleacronym AN 4.267}{\suttatitletranslation Lying }{\suttatitleroot Musāvādīsutta}}
\addcontentsline{toc}{section}{\tocacronym{AN 4.267} \toctranslation{Lying } \tocroot{Musāvādīsutta}}
\markboth{Lying }{Musāvādīsutta}
\extramarks{AN 4.267}{AN 4.267}

…\marginnote{1.1} They themselves lie … 

…\marginnote{2.1} They themselves don’t lie … 

%
\section*{{\suttatitleacronym AN 4.268}{\suttatitletranslation Divisive Speech }{\suttatitleroot Pisuṇavācāsutta}}
\addcontentsline{toc}{section}{\tocacronym{AN 4.268} \toctranslation{Divisive Speech } \tocroot{Pisuṇavācāsutta}}
\markboth{Divisive Speech }{Pisuṇavācāsutta}
\extramarks{AN 4.268}{AN 4.268}

…\marginnote{1.1} They themselves speak divisively … 

…\marginnote{2.1} They themselves don’t speak divisively … 

%
\section*{{\suttatitleacronym AN 4.269}{\suttatitletranslation Harsh Speech }{\suttatitleroot Pharusavācāsutta}}
\addcontentsline{toc}{section}{\tocacronym{AN 4.269} \toctranslation{Harsh Speech } \tocroot{Pharusavācāsutta}}
\markboth{Harsh Speech }{Pharusavācāsutta}
\extramarks{AN 4.269}{AN 4.269}

…\marginnote{1.1} They themselves speak harshly … 

…\marginnote{2.1} They themselves don’t speak harshly … 

%
\section*{{\suttatitleacronym AN 4.270}{\suttatitletranslation Talking Nonsense }{\suttatitleroot Samphappalāpasutta}}
\addcontentsline{toc}{section}{\tocacronym{AN 4.270} \toctranslation{Talking Nonsense } \tocroot{Samphappalāpasutta}}
\markboth{Talking Nonsense }{Samphappalāpasutta}
\extramarks{AN 4.270}{AN 4.270}

…\marginnote{1.1} They themselves talk nonsense … 

…\marginnote{2.1} They themselves don’t talk nonsense … 

%
\section*{{\suttatitleacronym AN 4.271}{\suttatitletranslation Covetousness }{\suttatitleroot Abhijjhālusutta}}
\addcontentsline{toc}{section}{\tocacronym{AN 4.271} \toctranslation{Covetousness } \tocroot{Abhijjhālusutta}}
\markboth{Covetousness }{Abhijjhālusutta}
\extramarks{AN 4.271}{AN 4.271}

…\marginnote{1.1} They themselves are covetous … 

…\marginnote{2.1} They themselves are content … 

%
\section*{{\suttatitleacronym AN 4.272}{\suttatitletranslation Ill Will }{\suttatitleroot Byāpannacittasutta}}
\addcontentsline{toc}{section}{\tocacronym{AN 4.272} \toctranslation{Ill Will } \tocroot{Byāpannacittasutta}}
\markboth{Ill Will }{Byāpannacittasutta}
\extramarks{AN 4.272}{AN 4.272}

…\marginnote{1.1} They themselves have ill will … 

…\marginnote{2.1} They themselves have good will … 

%
\section*{{\suttatitleacronym AN 4.273}{\suttatitletranslation Wrong View }{\suttatitleroot Micchādiṭṭhisutta}}
\addcontentsline{toc}{section}{\tocacronym{AN 4.273} \toctranslation{Wrong View } \tocroot{Micchādiṭṭhisutta}}
\markboth{Wrong View }{Micchādiṭṭhisutta}
\extramarks{AN 4.273}{AN 4.273}

…\marginnote{1.1} They themselves have wrong view … 

They\marginnote{2.1} themselves have right view; they encourage others to have right view; they approve of right view; and they praise right view. Someone with these four qualities is raised up to heaven.” 

%
\addtocontents{toc}{\let\protect\contentsline\protect\nopagecontentsline}
\chapter*{Abbreviated Texts Beginning with Greed }
\addcontentsline{toc}{chapter}{\tocchapterline{Abbreviated Texts Beginning with Greed }}
\addtocontents{toc}{\let\protect\contentsline\protect\oldcontentsline}

%
\section*{{\suttatitleacronym AN 4.274}{\suttatitletranslation Mindfulness Meditation }{\suttatitleroot Satipaṭṭhānasutta}}
\addcontentsline{toc}{section}{\tocacronym{AN 4.274} \toctranslation{Mindfulness Meditation } \tocroot{Satipaṭṭhānasutta}}
\markboth{Mindfulness Meditation }{Satipaṭṭhānasutta}
\extramarks{AN 4.274}{AN 4.274}

“For\marginnote{1.1} insight into greed, four things should be developed. What four? Firstly, a mendicant meditates by observing an aspect of the body—keen, aware, and mindful, rid of desire and aversion for the world. They meditate observing an aspect of feelings … mind … principles—keen, aware, and mindful, rid of desire and aversion for the world. For insight into greed, these four things should be developed.” 

%
\section*{{\suttatitleacronym AN 4.275}{\suttatitletranslation Right Efforts }{\suttatitleroot Sammappadhānasutta}}
\addcontentsline{toc}{section}{\tocacronym{AN 4.275} \toctranslation{Right Efforts } \tocroot{Sammappadhānasutta}}
\markboth{Right Efforts }{Sammappadhānasutta}
\extramarks{AN 4.275}{AN 4.275}

“For\marginnote{1.1} insight into greed, four things should be developed. What four? Firstly, a mendicant generates enthusiasm, tries, makes an effort, exerts the mind, and strives so that bad, unskillful qualities don’t arise. …so that unskillful qualities that have arisen are given up … so that skillful qualities arise … so that skillful qualities that have arisen remain, are not lost, but increase, mature, and are fulfilled by development. For insight into greed, these four things should be developed.” 

%
\section*{{\suttatitleacronym AN 4.276}{\suttatitletranslation Bases of Psychic Power }{\suttatitleroot Iddhipādasutta}}
\addcontentsline{toc}{section}{\tocacronym{AN 4.276} \toctranslation{Bases of Psychic Power } \tocroot{Iddhipādasutta}}
\markboth{Bases of Psychic Power }{Iddhipādasutta}
\extramarks{AN 4.276}{AN 4.276}

“For\marginnote{1.1} insight into greed, four things should be developed. What four? It’s when a mendicant develops the basis of psychic power that has immersion due to enthusiasm, and active effort. They develop the basis of psychic power that has immersion due to energy … mental development … inquiry, and active effort. For insight into greed, these four things should be developed.” 

%
\section*{{\suttatitleacronym AN 4.277–303}{\suttatitletranslation Complete Understanding, Etc. }{\suttatitleroot Pariññādisutta}}
\addcontentsline{toc}{section}{\tocacronym{AN 4.277–303} \toctranslation{Complete Understanding, Etc. } \tocroot{Pariññādisutta}}
\markboth{Complete Understanding, Etc. }{Pariññādisutta}
\extramarks{AN 4.277–303}{AN 4.277–303}

“For\marginnote{1.1} the complete understanding … finishing … giving up … ending … vanishing … fading away … cessation … giving away … letting go of greed, four things should be developed.” 

%
\section*{{\suttatitleacronym AN 4.304–783}{\suttatitletranslation Insight into Hate, Etc. }{\suttatitleroot Dosaabhiññādisutta}}
\addcontentsline{toc}{section}{\tocacronym{AN 4.304–783} \toctranslation{Insight into Hate, Etc. } \tocroot{Dosaabhiññādisutta}}
\markboth{Insight into Hate, Etc. }{Dosaabhiññādisutta}
\extramarks{AN 4.304–783}{AN 4.304–783}

“Of\marginnote{1.1} hate … delusion … anger … hostility … disdain … contempt … jealousy … stinginess … deceit … deviousness … obstinacy … aggression … conceit … arrogance … vanity … negligence … for insight … complete understanding … finishing … giving up … ending … vanishing … fading away … cessation … giving away … letting go … four things should be developed.” 

\scendbook{The Book of the Fours is finished. }

%
\backmatter%
\chapter*{Colophon}
\addcontentsline{toc}{chapter}{Colophon}
\markboth{Colophon}{Colophon}

\section*{The Translator}

Bhikkhu Sujato was born as Anthony Aidan Best on 4/11/1966 in Perth, Western Australia. He grew up in the pleasant suburbs of Mt Lawley and Attadale alongside his sister Nicola, who was the good child. His mother, Margaret Lorraine Huntsman née Pinder, said “he’ll either be a priest or a poet”, while his father, Anthony Thomas Best, advised him to “never do anything for money”. He attended Aquinas College, a Catholic school, where he decided to become an atheist. At the University of WA he studied philosophy, aiming to learn what he wanted to do with his life. Finding that what he wanted to do was play guitar, he dropped out. His main band was named Martha’s Vineyard, which achieved modest success in the indie circuit. Then it broke up, because everyone thought they personally were reason for the success, which, oddly enough, turns out not to have been the case. 

A seemingly random encounter with a roadside joey took him to Thailand, where he entered his first meditation retreat at Wat Ram Poeng, Chieng Mai in 1992. He decided to devote himself to the Buddha’s path, and took full ordination in Wat Pa Nanachat in 1994, where his teachers were Ajahn Pasanno and Ajahn Jayasaro. In 1997 he returned to Perth to study with Ajahn Brahm at Bodhinyana Monastery. 

He spent several years practicing in seclusion in Malaysia and Thailand before establishing Santi Forest Monastery in Bundanoon, NSW, in 2003. There he was instrumental in supporting the establishment of the Theravada bhikkhuni order in Australia and advocating for women’s rights. He continues to teach in Australia and globally, with a special concern for the moral implications of climate change and other forms of environmental destruction. He has published a series of books of original and groundbreaking research on early Buddhism. 

In 2005 he founded SuttaCentral together with Rod Bucknell and John Kelly. In 2015, seeing the need for a complete, accurate, plain English translation of the Pali texts, he undertook the task, spending nearly three years in isolation on the isle of Qi Mei off the coast of the nation of Taiwan. He completed the four main \textsanskrit{Nikāyas} in 2018, and the early books of the Khuddaka \textsanskrit{Nikāya} were complete by 2021. All this work is dedicated to the public domain and is entirely free of copyright encumbrance. 

In 2019 he returned to Sydney where, together with Bhikkhu Akaliko, he established Lokanta Vihara (The Monastery at the End of the World). 

\section*{Creation Process}

Primary source was the digital \textsanskrit{Mahāsaṅgīti} edition of the Pali \textsanskrit{Tipiṭaka}. Translated from the Pali, with reference to several English translations, especially those of Bhikkhu Bodhi.

\section*{The Translation}

This translation was part of a project to translate the four Pali \textsanskrit{Nikāyas} with the following aims: plain, approachable English; consistent terminology; accurate rendition of the Pali; free of copyright. It was made during 2016–2018 while Bhikkhu Sujato was staying in Qimei, Taiwan.

\section*{About SuttaCentral}

SuttaCentral publishes early Buddhist texts. Since 2005 we have provided root texts in Pali, Chinese, Sanskrit, Tibetan, and other languages, parallels between these texts, and translations in many modern languages. We build on the work of generations of scholars, and offer our contribution freely.

SuttaCentral is driven by volunteer contributions, and in addition we employ professional developers. We offer a sponsorship program for high quality translations from the original languages. Financial support for SuttaCentral is handled by the SuttaCentral Development Trust, a charitable trust registered in Australia.

\section*{About Bilara}

“Bilara” means “cat” in Pali, and it is the name of our Computer Assisted Translation (CAT) software. Bilara is a web app that enables translators to translate early Buddhist texts into their own language. These translations are published on SuttaCentral with the root text and translation side by side.

\section*{About SuttaCentral Editions}

The SuttaCentral Editions project makes high quality books from selected Bilara translations. These are published in formats including HTML, EPUB, PDF, and print.

If you want to print any of our Editions, please let us know and we will help prepare a file to your specifications.

%
\end{document}