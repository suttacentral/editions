\documentclass[12pt,openany]{book}%
\usepackage{lastpage}%
%
\usepackage[inner=1in, outer=1in, top=.7in, bottom=1in, papersize={6in,9in}, headheight=13pt]{geometry}
\usepackage{polyglossia}
\usepackage[12pt]{moresize}
\usepackage{soul}%
\usepackage{microtype}
\usepackage{tocbasic}
\usepackage{realscripts}
\usepackage{epigraph}%
\usepackage{setspace}%
\usepackage{sectsty}
\usepackage{fontspec}
\usepackage{marginnote}
\usepackage[bottom]{footmisc}
\usepackage{enumitem}
\usepackage{fancyhdr}
\usepackage{extramarks}
\usepackage{graphicx}
\usepackage{verse}
\usepackage{relsize}
\usepackage{etoolbox}
\usepackage[a-3u]{pdfx}

\hypersetup{
colorlinks=true,
urlcolor=black,
linkcolor=black,
citecolor=black
}

% use a small amount of tracking on small caps
\SetTracking[ spacing = {25*,166, } ]{ encoding = *, shape = sc }{ 25 }

% add a blank page
\newcommand{\blankpage}{
\newpage
\thispagestyle{empty}
\mbox{}
\newpage
}

% define languages
\setdefaultlanguage[]{english}
\setotherlanguage[script=Latin]{sanskrit}

%\usepackage{pagegrid}
%\pagegridsetup{top-left, step=.25in}

% define fonts
% use if arno sanskrit is unavailable
%\setmainfont{Gentium Plus}
%\newfontfamily\Semiboldsubheadfont[]{Gentium Plus}
%\newfontfamily\Semiboldnormalfont[]{Gentium Plus}
%\newfontfamily\Lightfont[]{Gentium Plus}
%\newfontfamily\Marginalfont[]{Gentium Plus}
%\newfontfamily\Allsmallcapsfont[RawFeature=+c2sc]{Gentium Plus}
%\newfontfamily\Noligaturefont[Renderer=Basic]{Gentium Plus}
%\newfontfamily\Noligaturecaptionfont[Renderer=Basic]{Gentium Plus}
%\newfontfamily\Fleuronfont[Ornament=1]{Gentium Plus}

% use if arno sanskrit is available. display is applied to \chapter and \part, subhead to \section and \subsection. When specifying semibold, the italic must be defined.
\setmainfont[Numbers=OldStyle]{Arno Pro}
\newfontfamily\Semibolddisplayfont[BoldItalicFont = Arno Pro Semibold Italic Display]{Arno Pro Semibold Display} %
\newfontfamily\Semiboldsubheadfont[BoldItalicFont = Arno Pro Semibold Italic Subhead]{Arno Pro Semibold Subhead}
\newfontfamily\Semiboldnormalfont[BoldItalicFont = Arno Pro Semibold Italic]{Arno Pro Semibold}
\newfontfamily\Marginalfont[RawFeature=+subs]{Arno Pro Regular}
\newfontfamily\Allsmallcapsfont[RawFeature=+c2sc]{Arno Pro}
\newfontfamily\Noligaturefont[Renderer=Basic]{Arno Pro}
\newfontfamily\Noligaturecaptionfont[Renderer=Basic]{Arno Pro Caption}

% chinese fonts
\newfontfamily\cjk{Noto Serif TC}
\newcommand*{\langlzh}[1]{\cjk{#1}\normalfont}%

% logo
\newfontfamily\Logofont{sclogo.ttf}
\newcommand*{\sclogo}[1]{\large\Logofont{#1}}

% use subscript numerals for margin notes
\renewcommand*{\marginfont}{\Marginalfont}

% ensure margin notes have consistent vertical alignment
\renewcommand*{\marginnotevadjust}{-.17em}

% use compact lists
\setitemize{noitemsep,leftmargin=1em}
\setenumerate{noitemsep,leftmargin=1em}
\setdescription{noitemsep, style=unboxed, leftmargin=0em}

% style ToC
\DeclareTOCStyleEntries[
  raggedentrytext,
  linefill=\hfill,
  pagenumberwidth=.5in,
  pagenumberformat=\normalfont,
  entryformat=\normalfont
]{tocline}{chapter,section}


  \setlength\topsep{0pt}%
  \setlength\parskip{0pt}%

% define new \centerpars command for use in ToC. This ensures centering, proper wrapping, and no page break after
\def\startcenter{%
  \par
  \begingroup
  \leftskip=0pt plus 1fil
  \rightskip=\leftskip
  \parindent=0pt
  \parfillskip=0pt
}
\def\stopcenter{%
  \par
  \endgroup
}
\long\def\centerpars#1{\startcenter#1\stopcenter}

% redefine part, so that it adds a toc entry without page number
\let\oldcontentsline\contentsline
\newcommand{\nopagecontentsline}[3]{\oldcontentsline{#1}{#2}{}}

    \makeatletter
\renewcommand*\l@part[2]{%
  \ifnum \c@tocdepth >-2\relax
    \addpenalty{-\@highpenalty}%
    \addvspace{0em \@plus\p@}%
    \setlength\@tempdima{3em}%
    \begingroup
      \parindent \z@ \rightskip \@pnumwidth
      \parfillskip -\@pnumwidth
      {\leavevmode
       \setstretch{.85}\large\scshape\centerpars{#1}\vspace*{-1em}\llap{#2}}\par
       \nobreak
         \global\@nobreaktrue
         \everypar{\global\@nobreakfalse\everypar{}}%
    \endgroup
  \fi}
\makeatother

\makeatletter
\def\@pnumwidth{2em}
\makeatother

% define new sectioning command, which is only used in volumes where the pannasa is found in some parts but not others, especially in an and sn

\newcommand*{\pannasa}[1]{\clearpage\thispagestyle{empty}\begin{center}\vspace*{14em}\setstretch{.85}\huge\itshape\scshape\MakeLowercase{#1}\end{center}}

    \makeatletter
\newcommand*\l@pannasa[2]{%
  \ifnum \c@tocdepth >-2\relax
    \addpenalty{-\@highpenalty}%
    \addvspace{.5em \@plus\p@}%
    \setlength\@tempdima{3em}%
    \begingroup
      \parindent \z@ \rightskip \@pnumwidth
      \parfillskip -\@pnumwidth
      {\leavevmode
       \setstretch{.85}\large\itshape\scshape\lowercase{\centerpars{#1}}\vspace*{-1em}\llap{#2}}\par
       \nobreak
         \global\@nobreaktrue
         \everypar{\global\@nobreakfalse\everypar{}}%
    \endgroup
  \fi}
\makeatother

% don't put page number on first page of toc (relies on etoolbox)
\patchcmd{\chapter}{plain}{empty}{}{}

% global line height
\setstretch{1.05}

% allow linebreak after em-dash
\catcode`\—=13
\protected\def—{\unskip\textemdash\allowbreak}

% style headings with secsty. chapter and section are defined per-edition
\partfont{\setstretch{.85}\normalfont\centering\textsc}
\subsectionfont{\setstretch{.85}\Semiboldsubheadfont}%
\subsubsectionfont{\setstretch{.85}\Semiboldnormalfont}

% style elements of suttatitle
\newcommand*{\suttatitleacronym}[1]{\smaller[2]{#1}\vspace*{.3em}}
\newcommand*{\suttatitletranslation}[1]{\linebreak{#1}}
\newcommand*{\suttatitleroot}[1]{\linebreak\smaller[2]\itshape{#1}}

\DeclareTOCStyleEntries[
  indent=3.3em,
  dynindent,
  beforeskip=.2em plus -2pt minus -1pt,
]{tocline}{section}

\DeclareTOCStyleEntries[
  indent=0em,
  dynindent,
  beforeskip=.4em plus -2pt minus -1pt,
]{tocline}{chapter}

\newcommand*{\tocacronym}[1]{\hspace*{-3.3em}{#1}\quad}
\newcommand*{\toctranslation}[1]{#1}
\newcommand*{\tocroot}[1]{(\textit{#1})}
\newcommand*{\tocchapterline}[1]{\bfseries\itshape{#1}}


% redefine paragraph and subparagraph headings to not be inline
\makeatletter
% Change the style of paragraph headings %
\renewcommand\paragraph{\@startsection{paragraph}{4}{\z@}%
            {-2.5ex\@plus -1ex \@minus -.25ex}%
            {1.25ex \@plus .25ex}%
            {\noindent\Semiboldnormalfont\normalsize}}

% Change the style of subparagraph headings %
\renewcommand\subparagraph{\@startsection{subparagraph}{5}{\z@}%
            {-2.5ex\@plus -1ex \@minus -.25ex}%
            {1.25ex \@plus .25ex}%
            {\noindent\Semiboldnormalfont\small}}
\makeatother

% use etoolbox to suppress page numbers on \part
\patchcmd{\part}{\thispagestyle{plain}}{\thispagestyle{empty}}
  {}{\errmessage{Cannot patch \string\part}}

% and to reduce margins on quotation
\patchcmd{\quotation}{\rightmargin}{\leftmargin 1.2em \rightmargin}{}{}
\AtBeginEnvironment{quotation}{\small}

% titlepage
\newcommand*{\titlepageTranslationTitle}[1]{{\begin{center}\begin{large}{#1}\end{large}\end{center}}}
\newcommand*{\titlepageCreatorName}[1]{{\begin{center}\begin{normalsize}{#1}\end{normalsize}\end{center}}}

% halftitlepage
\newcommand*{\halftitlepageTranslationTitle}[1]{\setstretch{2.5}{\begin{Huge}\uppercase{\so{#1}}\end{Huge}}}
\newcommand*{\halftitlepageTranslationSubtitle}[1]{\setstretch{1.2}{\begin{large}{#1}\end{large}}}
\newcommand*{\halftitlepageFleuron}[1]{{\begin{large}\Fleuronfont{{#1}}\end{large}}}
\newcommand*{\halftitlepageByline}[1]{{\begin{normalsize}\textit{{#1}}\end{normalsize}}}
\newcommand*{\halftitlepageCreatorName}[1]{{\begin{LARGE}{\textsc{#1}}\end{LARGE}}}
\newcommand*{\halftitlepageVolumeNumber}[1]{{\begin{normalsize}{\Allsmallcapsfont{\textsc{#1}}}\end{normalsize}}}
\newcommand*{\halftitlepageVolumeAcronym}[1]{{\begin{normalsize}{#1}\end{normalsize}}}
\newcommand*{\halftitlepageVolumeTranslationTitle}[1]{{\begin{Large}{\textsc{#1}}\end{Large}}}
\newcommand*{\halftitlepageVolumeRootTitle}[1]{{\begin{normalsize}{\Allsmallcapsfont{\textsc{\itshape #1}}}\end{normalsize}}}
\newcommand*{\halftitlepagePublisher}[1]{{\begin{large}{\Noligaturecaptionfont\textsc{#1}}\end{large}}}

% epigraph
\renewcommand{\epigraphflush}{center}
\renewcommand*{\epigraphwidth}{.85\textwidth}
\newcommand*{\epigraphTranslatedTitle}[1]{\vspace*{.5em}\footnotesize\textsc{#1}\\}%
\newcommand*{\epigraphRootTitle}[1]{\footnotesize\textit{#1}\\}%
\newcommand*{\epigraphReference}[1]{\footnotesize{#1}}%

% custom commands for html styling classes
\newcommand*{\scnamo}[1]{\begin{center}\textit{#1}\end{center}}
\newcommand*{\scendsection}[1]{\begin{center}\textit{#1}\end{center}}
\newcommand*{\scendsutta}[1]{\begin{center}\textit{#1}\end{center}}
\newcommand*{\scendbook}[1]{\begin{center}\uppercase{#1}\end{center}}
\newcommand*{\scendkanda}[1]{\begin{center}\textbf{#1}\end{center}}
\newcommand*{\scend}[1]{\begin{center}\textit{#1}\end{center}}
\newcommand*{\scuddanaintro}[1]{\textit{#1}}
\newcommand*{\scendvagga}[1]{\begin{center}\textbf{#1}\end{center}}
\newcommand*{\scrule}[1]{\textbf{#1}}
\newcommand*{\scadd}[1]{\textit{#1}}
\newcommand*{\scevam}[1]{\textsc{#1}}
\newcommand*{\scspeaker}[1]{\hspace{2em}\textit{#1}}
\newcommand*{\scbyline}[1]{\begin{flushright}\textit{#1}\end{flushright}\bigskip}

% custom command for thematic break = hr
\newcommand*{\thematicbreak}{\begin{center}\rule[.5ex]{6em}{.4pt}\begin{normalsize}\quad\Fleuronfont{•}\quad\end{normalsize}\rule[.5ex]{6em}{.4pt}\end{center}}

% manage and style page header and footer. "fancy" has header and footer, "plain" has footer only

\pagestyle{fancy}
\fancyhf{}
\fancyfoot[RE,LO]{\thepage}
\fancyfoot[LE,RO]{\footnotesize\lastleftxmark}
\fancyhead[CE]{\setstretch{.85}\Noligaturefont\MakeLowercase{\textsc{\firstrightmark}}}
\fancyhead[CO]{\setstretch{.85}\Noligaturefont\MakeLowercase{\textsc{\firstleftmark}}}
\renewcommand{\headrulewidth}{0pt}
\fancypagestyle{plain}{ %
\fancyhf{} % remove everything
\fancyfoot[RE,LO]{\thepage}
\fancyfoot[LE,RO]{\footnotesize\lastleftxmark}
\renewcommand{\headrulewidth}{0pt}
\renewcommand{\footrulewidth}{0pt}}

% style footnotes
\setlength{\skip\footins}{1em}

\makeatletter
\newcommand{\@makefntextcustom}[1]{%
    \parindent 0em%
    \thefootnote.\enskip #1%
}
\renewcommand{\@makefntext}[1]{\@makefntextcustom{#1}}
\makeatother

% hang quotes (requires microtype)
\microtypesetup{
  protrusion = true,
  expansion  = true,
  tracking   = true,
  factor     = 1000,
  patch      = all,
  final
}

% Custom protrusion rules to allow hanging punctuation
\SetProtrusion
{ encoding = *}
{
% char   right left
  {-} = {    , 500 },
  % Double Quotes
  \textquotedblleft
      = {1000,     },
  \textquotedblright
      = {    , 1000},
  \quotedblbase
      = {1000,     },
  % Single Quotes
  \textquoteleft
      = {1000,     },
  \textquoteright
      = {    , 1000},
  \quotesinglbase
      = {1000,     }
}

% make latex use actual font em for parindent, not Computer Modern Roman
\AtBeginDocument{\setlength{\parindent}{1em}}%
%

% Default values; a bit sloppier than normal
\tolerance 1414
\hbadness 1414
\emergencystretch 1.5em
\hfuzz 0.3pt
\clubpenalty = 10000
\widowpenalty = 10000
\displaywidowpenalty = 10000
\hfuzz \vfuzz
 \raggedbottom%

\title{So It Was Said}
\author{Bhikkhu Sujato}
\date{}%
% define a different fleuron for each edition
\newfontfamily\Fleuronfont[Ornament=20]{Arno Pro}

% Define heading styles per edition for chapter and section. Suttatitle can be either of these, depending on the volume. 

\let\oldfrontmatter\frontmatter
\renewcommand{\frontmatter}{%
\chapterfont{\setstretch{.85}\normalfont\centering}%
\sectionfont{\setstretch{.85}\Semiboldsubheadfont}%
\oldfrontmatter}

\let\oldmainmatter\mainmatter
\renewcommand{\mainmatter}{%
\chapterfont{\setstretch{.85}\normalfont\centering}%
\sectionfont{\setstretch{.85}\normalfont\centering}%
\oldmainmatter}

\let\oldbackmatter\backmatter
\renewcommand{\backmatter}{%
\chapterfont{\setstretch{.85}\normalfont\centering}%
\sectionfont{\setstretch{.85}\Semiboldsubheadfont}%
\oldbackmatter}
%
%
\begin{document}%
\normalsize%
\frontmatter%
\setlength{\parindent}{0cm}

\pagestyle{empty}

\maketitle

\blankpage%
\begin{center}

\vspace*{2.2em}

\halftitlepageTranslationTitle{So It Was Said}

\vspace*{1em}

\halftitlepageTranslationSubtitle{A delectable translation of the Itivuttaka}

\vspace*{2em}

\halftitlepageFleuron{•}

\vspace*{2em}

\halftitlepageByline{translated and introduced by}

\vspace*{.5em}

\halftitlepageCreatorName{Bhikkhu Sujato}

\vspace*{4em}

\halftitlepageVolumeNumber{}

\smallskip

\halftitlepageVolumeAcronym{Iti}

\smallskip

\halftitlepageVolumeTranslationTitle{}

\smallskip

\halftitlepageVolumeRootTitle{}

\vspace*{\fill}

\sclogo{0}
 \halftitlepagePublisher{SuttaCentral}

\end{center}

\newpage
%
\setstretch{1.05}

\begin{footnotesize}

\textit{So It Was Said} is a translation of the Itivuttaka by Bhikkhu Sujato.

\medskip

Creative Commons Zero (CC0)

To the extent possible under law, Bhikkhu Sujato has waived all copyright and related or neighboring rights to \textit{So It Was Said}.

\medskip

This work is published from Australia.

\begin{center}
\textit{This translation is an expression of an ancient spiritual text that has been passed down by the Buddhist tradition for the benefit of all sentient beings. It is dedicated to the public domain via Creative Commons Zero (CC0). You are encouraged to copy, reproduce, adapt, alter, or otherwise make use of this translation. The translator respectfully requests that any use be in accordance with the values and principles of the Buddhist community.}
\end{center}

\medskip

\begin{description}
    \item[Web publication date] 2020
    \item[This edition] 2022-12-05 01:25:21
    \item[Publication type] paperback
    \item[Edition] ed5
    \item[Number of volumes] 1
    \item[Publication ISBN] 978-1-76132-077-4
    \item[Publication URL] https://suttacentral.net/editions/iti/en/sujato
    \item[Source URL] https://github.com/suttacentral/bilara-data/tree/published/translation/en/sujato/sutta/kn/iti
    \item[Publication number] scpub16
\end{description}

\medskip

Published by SuttaCentral

\medskip

\textit{SuttaCentral,\\
c/o Alwis \& Alwis Pty Ltd\\
Kaurna Country,\\
Suite 12,\\
198 Greenhill Road,\\
Eastwood,\\
SA 5063,\\
Australia}

\end{footnotesize}

\newpage

\setlength{\parindent}{1.5em}%%
\newpage

\vspace*{\fill}

\begin{center}
\epigraph{The Realized One, compassionate for all living creatures,\\
unstintingly offers up teaching.\\
Sentient beings revere him, first among gods and humans,\\
who has gone beyond rebirth.}
{
\epigraphTranslatedTitle{The Holy Offering of the Teaching}
\epigraphRootTitle{\textsanskrit{Brāhmaṇadhammayāgasutta}}
\epigraphReference{Itivuttaka 100}
}
\end{center}

\vspace*{2in}

\vspace*{\fill}

\blankpage%

\setlength{\parindent}{1em}
%
\tableofcontents
\newpage
\pagestyle{fancy}
%
\chapter*{The SuttaCentral Editions Series}
\addcontentsline{toc}{chapter}{The SuttaCentral Editions Series}
\markboth{The SuttaCentral Editions Series}{The SuttaCentral Editions Series}

Since 2005 SuttaCentral has provided access to the texts, translations, and parallels of early Buddhist texts. In 2018 we started creating and publishing our own translations of these seminal spiritual classics. The “Editions” series now makes selected translations available as books in various forms, including print, PDF, and EPUB.

Editions are selected from our most complete, well-crafted, and reliable translations. They aim to bring these texts to a wider audience in forms that reward mindful reading. Care is taken with every detail of the production, and we aim to meet or exceed professional best standards in every way. These are the core scriptures underlying the entire Buddhist tradition, and we believe that they deserve to be preserved and made available in highest quality without compromise.

SuttaCentral is a charitable organization. Our work is accomplished by volunteers and through the generosity of our donors. Everything we create is offered to all of humanity free of any copyright or licensing restrictions. 

%
\chapter*{Preface to the Itivuttaka}
\addcontentsline{toc}{chapter}{Preface to the Itivuttaka}
\markboth{Preface to the Itivuttaka}{Preface to the Itivuttaka}

For the slightest of the early Buddhist books, I feel I should write a slight preface. There is a beauty in simplicity and a strength in humility. Spiritual teachings need not always tell us something new, nor need they dazzle us with complicated philosophy or intricate arguments. Sometimes the most powerful lesson is that which you have always known.

It is a discomforting thought, but when I think of those who most effortlessly embody the Buddha’s teachings, it is rarely those who have mastered the scriptures or who debate the meaning of abstruse points. It is in the simple offering, the quiet devotion, the humble wisdom, the unnoticed kindness that I see the flowering of grace.

To be all that we can be is to be less than we have become. If all that the suttas did was to fill us with information, they would be as useful as an ancient Wikipedia (which would be no bad thing). The real wisdom of the suttas, I have come to learn, lies in their delicate balance: saying what need be said and not saying what need not be said.

A full meal is best enjoyed on an empty stomach. And the suttas will fill our minds and hearts, but only if we respect their empty spaces.

%
\chapter*{So It Was Said: summary sayings}
\addcontentsline{toc}{chapter}{So It Was Said: summary sayings}
\markboth{So It Was Said: summary sayings}{So It Was Said: summary sayings}

\scbyline{Bhikkhu Sujato, 2022}

The Itivuttaka or “So it Was Said” is the fourth book of the Pali Khuddhaka \textsanskrit{Nikāya}, the “Minor Collection”. It is a short book, with 112 discourses in mixed prose and verse.  The teachings in the Itivuttaka are by and large simple and straightforward, as is the style of both prose and verse.

The text is arranged in the \textsanskrit{Aṅguttara} style of incremental numbering, which here goes from the Ones through Fours. Within each of these major sections the texts are rather arbitrarily divided into chapters (\emph{vagga}) of ten. The exceptions are the concluding chapters of the Ones, Twos, and Fours, which contain 7, 12, and 13 discourses respectively.

\section*{The Chinese Collection of Itivuttakas}

As with most of the early texts in the Pali Canon, there exists a corresponding Chinese version: \langlzh{本事經} (Taishō vol. 17, sutra 765), which was translated by Xuanzang in CE 650. This was studied by K. Watanabe in his \textit{A Chinese Collection of Itivuttakas} (Journal of the Pali Text Society V, 1906–7, pp. 44–49).

The Chinese text contains framing statements that are similar to the Pali, except the final statement is omitted. This is, I think, a significant detail, to which I will return below.

While the first two sections are similar to the Pali, three-fifths of the third and all of the fourth sections are missing. That this is the result of an incomplete text, rather than a shorter recension, is supported by two details. The text lacks a concluding \emph{\textsanskrit{uddāna}}, the summary verse or “resumé” that is normally found at the end of every section. And the content of the Suttas, while still remaining within the scope of the early Buddhist teachings, is somewhat developed compared to the Pali. It seems, then, that the Chinese text represents an incomplete text of a somewhat later version of an Itivuttaka.

It is not clear why a scholar as able as Xuanzang would leave the work incomplete. Perhaps he had only a partial manuscript to work with, or perhaps it was simply that other demands took his time. In the introduction to his translation of the Itivuttaka, Ven. Ṭhanissaro remarks that Xuanzang’s translation “dates from the last months of his life”. This is mistaken, and apparently caused by a misreading of Watanabe’s article. Xuanzang did not die until fourteen years later, in 664.

\section*{The Formation of the Itivuttaka}

Each Sutta is introduced with a distinct phrase saying the text was “said” (\emph{vutta}) by the Buddha, and it appears that this tag is what gives the collection its name. The tag is more than just an introduction; it is a full template that frames each discourse.

\begin{itemize}%
\item Start the prose — \emph{This was said by the Buddha, the Perfected One: that is what I heard.}%
\item End the prose — \emph{The Buddha spoke this matter.}%
\item Start the verse — \emph{On this it is said:}%
\item End the verse — \emph{This too is a matter that was spoken by the Blessed One: that is what I heard.}%
\end{itemize}

This framework, more formalized and consistent than the standard forms, is adhered to rigorously throughout without variation.

Note that the tag lines assume that the verse comments on the prose (\emph{tattha}). Note too that here, as in the \textsanskrit{Udāna}, \emph{attha} has the sense “matter, substance, content” rather than “meaning”.

The opening compound, \emph{\textsanskrit{vuttañhetaṁ}} contains the particle \emph{hi}, which most translators ignore, but which Masefield perhaps over-renders as “unquestionably”. This exact idiom is not used elsewhere in the early texts, but it is quite common in the Niddesa, where it seems to act as a logical connection. A doctrine is stated, and it is supported with additional quotations. Perhaps then we should translate, “For this was said by the Buddha …”.

Unusually, there is no mention of the setting or other background details. Additionally, there are few personal names in the text. Apart from the Buddha, who is referred to by many epithets, only the Buddha’s antagonists Devadatta (\href{https://suttacentral.net/iti89/en/sujato}{Iti 89}) and \textsanskrit{Māra} (\href{https://suttacentral.net/iti58/en/sujato}{Iti 58}, \href{https://suttacentral.net/iti68/en/sujato}{Iti 68}, \href{https://suttacentral.net/iti82/en/sujato}{Iti 82}, \href{https://suttacentral.net/iti83/en/sujato}{Iti 83}, \href{https://suttacentral.net/iti93/en/sujato}{Iti 93}) are mentioned by name. As for places, only the Vulture’s Peak in \textsanskrit{Rāgaha} is mentioned (\href{https://suttacentral.net/iti24/en/sujato}{Iti 24}). All this adds up to an oddly abstracted and spare text, almost Abhidhammic in style. It suggests that the collection was compiled from reports of what the Buddha said rather than from first-hand recollections.

This is, in fact, the position of the commentary, which explains that the Itivuttaka, alone among the texts of the Pali canon, was not compiled primarily by the monks, but by the laywoman \textsanskrit{Khujjuttarā}. In the \textsanskrit{Aṅguttara} \textsanskrit{Nikāya} she was extolled by the Buddha as the foremost in learning among the laywomen, and is frequently held up as an exemplary laywoman. While the Suttas do not say how she earned that title, the commentary tells of how, as handmaid to Queen \textsanskrit{Sāmāvatī} of Kosambi, she became the respected teacher of Dhamma for the ladies of the court. This story is only part of a much longer and very dramatic series of events known as the Kosambi Cycle.

It seems that the Queen entrusted her with procuring flowers for the court, but she would save some of the money each day. One day, she overheard the Buddha teaching the Dhamma to the gardener Sumana, and right away entered the stream. In celebration, she spent her saved money on flowers, prompting the queen to ask where they all came from. And when \textsanskrit{Khujjuttarā} told her, the queen showered her with honors, bathed her in perfumed water, and became her student. \textsanskrit{Khujjuttarā} continued to listen and memorize the Dhamma from the Buddha and would convey it for all the court ladies, who became stream-enterers in turn. These teachings were compiled into the Itivuttaka, which is why they do not have the usual prose opening formula.

It is difficult to reconcile this story with the Itivuttaka as it stands. The texts are clearly organized in a pattern of numbers from one to four, and it seems improbable that \textsanskrit{Khujjuttarā} just happened to hear texts that would be amenable to such an arrangement. It could be that she did hear many more teachings, but selected certain texts and arranged them for the ladies of the court. However there is no special emphasis on teachings suitable for court ladies, and a number of quite difficult texts that would be hard to make sense of without a broader context.

On the other hand, there seems no reason why the monks would invent such a story, which sidelines their own role in the creation of this text, and ascribes it instead to the elevated slave of a doomed dynasty. As so often in such matters, we are left without definite conclusions. The absence of certainty does not, however, imply the presence of ignorance. In such matters, it is usually the case that there is something to the stories; they are rarely fabricated out of thin air. We should not reject knowledge simply because it is unconfirmed or hard to understand. The truth is surely more complicated than we know. Yet if we were to abandon our few clues because they are not as certain as we would like, we would be like someone who, lost in the darkness with only a candle to light the way, blows out the candle because it is not a torch.

A text called Itivuttaka is mentioned in the list of the nine \emph{\textsanskrit{aṅgas}} (sections) of the Dhamma that is found throughout the early texts. And as with so many of the \emph{\textsanskrit{aṅgas}}, it is not easy to determine the extent to which the Itivuttaka as it exists today is the same text referred to as an \emph{\textsanskrit{aṅga}}.

Normally the \emph{\textsanskrit{aṅgas}} define a genre of text with an identifiable style. If, however, we accept the Theravadin account, the Itivuttaka is unusual in that there is no intrinsic relationship between the name of the \emph{\textsanskrit{aṅga}} and the style of text. The discourses are, leaving the unique framing aside, essentially no different from those found in the \textsanskrit{Aṅguttara}. Perhaps, like most \emph{\textsanskrit{aṅgas}}, \emph{itivuttaka} originally referred to a distinct genre of early Buddhist literature.

The name itself is perhaps misleadingly malleable. By that I mean that it compounds two very common words (\emph{iti} “thus” and \emph{vutta} “spoke”) and hence may be applied very generally. However in idiomatic usage, the terms have a more specific and stronger sense: something that is quoted or passed down from the past. The name is reminiscent of the class of Vedic literature called \emph{\textsanskrit{itihāsa}}, “Thus It Was”, i.e. “stories of the past, legendary histories”, which is sometimes equated with the \textsanskrit{Mahābhārata} and the \textsanskrit{Ramāyana}. Similarly, \emph{\textsanskrit{itikirā}} “So It Seems”, though sometimes translated as “hearsay”, is grouped with words referring to the transmission of texts, and must mean something like “testament”. We also find \emph{\textsanskrit{itihītiha}} “So and So It Was” used of knowledge that has been passed down from the past.

By analogy, \emph{itivuttaka} would mean “sayings of the past”, which could refer to the legendary histories that are found in the \textsanskrit{Nikāyas}, such as the \textsanskrit{Aggaññasutta} (\href{https://suttacentral.net/dn27/en/sujato}{DN 27}) and the \textsanskrit{Cakkavattisīhanādasutta} (\href{https://suttacentral.net/dn26/en/sujato}{DN 26}), as well as legendary lore such as the 32 marks of the great man, or the occasional sayings reported to have been passed down from teachers of old (eg. \href{https://suttacentral.net/mn75/en/sujato\#19.11}{MN 75:19.11}). This theory finds support from the great \textsanskrit{Mahāyāna} commentator \textsanskrit{Asaṅga}, who in his Abhidharmasamuccaya says \emph{itivuttaka} “narrates the former existences of the noble disciples”, while in the \textsanskrit{Śrāvakabhūmi} of the \textsanskrit{Yogacārabhūmiśāstra} he says it refers to “whatever is connected with previous practice”.

Still further support may be adduced in that part of the framing tags of the Itivuttaka—specifically, those that connect the prose and verse—are found in one other text of the Pali canon in nearly identical form (lacking only a connecting \emph{iti}: \emph{\textsanskrit{tatthetaṁ} vuccati} rather than \emph{\textsanskrit{tatthetaṁ} iti vuccati}). This is in \href{https://suttacentral.net/dn30/en/sujato}{DN 30} \textsanskrit{Mahāpurisalakkhaṇasutta}, an extended elaboration of the so-called “marks of the great man”, which are consistently said in the Suttas to be a traditional lore handed down among the brahmins. The verses especially are among the latest additions to the four \textsanskrit{Nikāyas}, and in this case it is clear that that the tag line serves to add verses on to a pre-existing prose text. The commentary, in fact, says that some “elders of old” explain \emph{\textsanskrit{tatthetaṁ} vuccati} as indicating that the verses were added by Ānanda.

If this reasoning is cogent, then it seems that \emph{itivuttaka} originally referred to the various legendary accounts that are currently scattered in the four \textsanskrit{Nikāyas}. At some point—the First Council, perhaps—the material organized in the nine  \emph{\textsanskrit{aṅgas}} was rearranged for the convenience of memorization into the \textsanskrit{Nikāyas}. With the legendary texts absorbed in the \textsanskrit{Nikāyas}, the name \emph{itivuttaka} was floating unused, and was adopted to frame this small selection of \textsanskrit{Aṅguttara} style texts.

If this is true it may be easily reconciled with the traditional account. There is no reason why, if the collection the collection as we have it is due to \textsanskrit{Khujjuttarā}, it should not have been titled Itivuttaka at a later date. All this is, of course, speculative.

\section*{The Purpose of the Itivuttaka}

Given its minor position within the Pali Canon, it is probably safe to assume that most modern readers will pick up the Itivuttaka when already familiar with the Suttas from the \textsanskrit{Nikāyas}. It may be that this is the wrong approach.

The text begins with a series of teachings on the “one thing” that must be given up. To read a series of multiple “one things” is a bit odd. Are they all “one thing”? What is the relation between them? Does any one of these imply all the rest? The text begins by speaking of realizing “non-return”, assuming the audience already knows what this is and wants to attain it.

Take the first discourse: the one thing to be given up is greed. As a reader, this discourse can be skimmed in a couple of minutes, and it provides no new information or perspective. But perhaps that is not its purpose. Perhaps the text was meant to be approached as a meditation structured for those who are familiar with the basics, and are undertaking the process of internalizing the theory.

One could learn just this one discourse and take it as a theme for meditation. Focus only on greed, and how it drags the mind to unworthy places. The \emph{idea} is simple, but to truly digest it can be a long and complex work. By giving only the simplest of outlines, the text leaves the details to the individual, who is left to explore their own relation to greed.

Such a process might take days, weeks, or months. But conservatively, one might take such a short discourse as the theme of contemplation for a day. The next day it is not greed, but hate. The details, differences, and relations between greed and hate are not spelled out; they are realized by the meditator, informed by their prior study and experience.

And so on through the different qualities. Then the round repeats with a slight variation. Then new topics are introduced, each one a self-contained reflection.

I can’t prove that this is why the Itivuttaka is that way it is, but I do think this would be a fruitful way of approaching the text.

\section*{Relation Between Prose and Verse}

Unlike the \textsanskrit{Udāna}, where the verse is the culminating purpose of the narrative, here the verses serve to repeat and amplify the prose teachings, again in a style similar to certain portions of the \textsanskrit{Aṅguttara}. In line with the systematic tendency of the Itivuttaka, this pattern occurs in all Suttas.

The framing text asserts that both prose and verse portions were spoken by the Buddha. However it is careful to qualify this by saying “so I have heard”, indicating that the speaker was not present when the teachings were given, but rather is passing down an oral tradition. It is quite possible that this is correct, and that both prose and verse were spoken in this form by the Buddha. However there are a number of indications that this is not always the case.

It is quite common for Buddhist texts to have verse and prose portions that are loosely coupled. Sometimes the same verse has a different prose background. Sometimes the connection between the two seems distant or arbitrary. Sometimes the texts attribute the different portions to different authors. So it would be no great surprise for the Itivuttaka to follow a similar pattern.

A fruitful approach is to look at the implicit speaker in the text, rather than the speaker assigned by the framework. Now, the framing portions must have been added by redactors at some point, possibly the First Council. They are in third person, reporting what the Buddha said.

In the prose teaching portions, by contrast, the Buddha refers to himself in the first person (eg. \href{https://suttacentral.net/iti14/en/sujato}{Iti 14} \emph{\textsanskrit{ahaṁ}}; \href{https://suttacentral.net/iti103/en/sujato}{Iti 103} \emph{na me te}). Appropriately, he addresses the mendicants in the second person (eg. \href{https://suttacentral.net/iti22/en/sujato}{Iti 22} \emph{\textsanskrit{mā}, bhikkhave, \textsanskrit{puññānaṁ} \textsanskrit{bhāyittha}}; \href{https://suttacentral.net/iti38/en/sujato}{Iti 38} \emph{tumhepi \textsanskrit{abyāpajjhārāmā} viharatha}; \href{https://suttacentral.net/iti111/en/sujato}{Iti 111} \emph{\textsanskrit{sampannasīlā}, bhikkhave, viharatha}). I’m setting aside here the vocative form of direct address found in every Sutta (\emph{bhikkhave}), as this could be regarded as a mere convention.

In the verses, however, we typically find the third person used for both the Buddha and the mendicants, in a manner that is more similar to the framing portions.

For example, in \href{https://suttacentral.net/iti26/en/sujato}{Iti 26} the prose is in first person “as I understand” \emph{\textsanskrit{yathāhaṁ} \textsanskrit{jānāmi}}, while the verse reports the words of the Buddha, “as taught by the great hermit” (\emph{\textsanskrit{yathāvuttaṁ} \textsanskrit{mahesinā}}). Not only is this in third person, but the passive instrumental construction is identical with that used in the frame (\emph{vutto \textsanskrit{bhagavatā}}). The Buddha is also referred to in third person in a similar way in \href{https://suttacentral.net/iti36/en/sujato}{Iti 36} (\emph{adesayi so \textsanskrit{bhagavā}}, \emph{\textsanskrit{yathā} buddhena \textsanskrit{desitaṁ}}).

In \href{https://suttacentral.net/iti85/en/sujato}{Iti 85} the mendicants are addressed in second person in the prose (\emph{viharatha}), but the third person is used in the verse (\emph{vimuccati}).

In \href{https://suttacentral.net/iti70/en/sujato}{Iti 70}, \href{https://suttacentral.net/iti71/en/sujato}{Iti 71}, and \href{https://suttacentral.net/iti81/en/sujato}{Iti 81} the Buddha speaks of how good and bad kamma results in good or bad rebirths, insisting that this is something that he has seen for himself (\emph{\textsanskrit{diṭṭhā} \textsanskrit{mayā}}), and has not learned from any other ascetic or brahmin. The verses again are in third person.

\href{https://suttacentral.net/iti92/en/sujato}{Iti 92} has one of the Itivuttaka’s rare moments of intimacy. The Buddha speaks of a poorly-behaved mendicant who might follow him around holding the corner of his robe, yet they remain “far from me, and I from them” (\emph{\textsanskrit{ārakāva} \textsanskrit{mayhaṁ}, \textsanskrit{ahañca} tassa}). This is unlike one who has well-practiced the Dhamma, to whom the Buddha is always close. The verses once more avoid the personal touch here. The same pattern holds true for \href{https://suttacentral.net/iti100/en/sujato}{Iti 100}, where the Buddha says to his students that “you are my children” (\emph{me tumhe \textsanskrit{puttā}}); and in \href{https://suttacentral.net/iti107/en/sujato}{Iti 107} where the mendicants are enjoined to be grateful for the things that the lay folk offer “to you”, while in both cases the verses shift to the more distant third person.

Rarely in analysis of ancient texts do we find that a pattern admits of no exceptions. It is true, admittedly, that the prose text sometimes has the Buddha referring to himself in third person as the “Realized One” (eg. \href{https://suttacentral.net/iti38/en/sujato}{Iti 38}, \href{https://suttacentral.net/iti39/en/sujato}{Iti 39}, \href{https://suttacentral.net/iti84/en/sujato}{Iti 84}). But this is a common feature of prose Suttas. The Buddha speaks in this way when evoking the profound nature of his state of transcendent realization (\href{https://suttacentral.net/iti112/en/sujato}{Iti 112}).

In \href{https://suttacentral.net/iti47/en/sujato}{Iti 47} the verses are in a direct second person. This, however, turns out to be the exception that proves the rule, for the subject here is wakefulness. The text is designed to jolt awake the sleepy, so the direct address of the second person is called for. Amid the almost brutalist plainness of the Itivuttaka, this qualifies as a flourish of literary style.

A further exception is in \href{https://suttacentral.net/iti99/en/sujato}{Iti 99}, where both the prose and verse employ the first person. The verses are not unique to this Sutta, however, for they are found in a similar context in \href{https://suttacentral.net/an3.58/en/sujato}{AN 3.58} and \href{https://suttacentral.net/an3.59/en/sujato}{AN 3.59}. There, the Buddha is responding to a brahmin who challenges him on the true meaning of a “brahmin who is master of the three knowledges”. The brahmin defines this in terms of knowing the Vedic literary tradition, while the Buddha redefines it, as always, in terms of the gradual training and the realization of the three higher knowledges. The text of the Itivuttaka reads like the \textsanskrit{Aṅguttara} texts with narrative removed, and has probably been adapted from there. This is an interesting case, because it agrees very nicely with the traditional origin story of the Itivuttaka. A text with a narrative context has been repeated in a slightly adapted and stripped-down form, taking a step towards an almost Abhidhammic plainness.

Finally, in two further Suttas we find the first person \emph{\textsanskrit{brūmi}} “I say” used in verse (\href{https://suttacentral.net/iti38/en/sujato\#7.5}{Iti 38:7.5}, \href{https://suttacentral.net/iti46/en/sujato\#3.4}{Iti 46:3.4}). These lines are unique to the Itivuttaka, so they constitute a genuine exception to the rule.

In sum, there is a strong tendency for the prose sections to be presented as the Buddha speaking directly to the monks, while the verses read as a third-hand rephrasing of the same teachings, sometimes summarizing or expanding. The few occasions where this pattern does not hold are mostly explained by the specific context. This pattern suggests that, on the whole, the prose portions are relatively direct reports of the Buddha’s words, while the verses were added by redactors.

I noted above that the Chinese version, though otherwise appearing somewhat more developed than the Pali, lacks the final tag phrase of the frame (\emph{ayampi attho vutto \textsanskrit{bhagavatā}}). It is in this line that the Pali text asserts that the verses were spoken by the Buddha. The tag phrase that starts the verses says simply “on this it is said” (\emph{\textsanskrit{tatthetaṁ} iti vuccati}), which might easily refer to an addition made by redactors. Recall that a nearly-identical tag is used in \href{https://suttacentral.net/dn30/en/sujato}{DN 30} where it clearly indicates a later addition.

This provides, I think, concrete support for the conclusion that the verses are for the most part a later addition to the prose. Originally they were simply presented as such, but at some point the redactors claimed that, like the prose, the verses were “also” (\emph{pi}) spoken by the Buddha. As we have seen, they were not entirely wrong, for some of the verses do present as the direct words of the Buddha. And in many other cases there is no real evidence either way.

A case like this is not so much an attempt to misrepresent the material as it is an outcome of a process of systematization. Material of diverse sources is flattened and simplified, and certain nuances get lost along the way. The Theravada tradition is usually very scrupulous about such details, and less inclined to adapt older texts to later needs. Yet this case proves the exception, as the northern—possibly \textsanskrit{Sarvāstivādin}—text retains a clue to an earlier form.

\section*{Similes}

Moore, in the introduction to his 1908 translation, counts a round fifty similes in the Itivuttaka, and the following is based on his analysis. Nature provides most of the similes, starting with water in its many forms.

Water is a decidedly ambiguous element in the Itivuttaka. It often figures in powerfully negative ways. The realm of desire and suffering is called a flood (\emph{ogha}, \href{https://suttacentral.net/iti107/en/sujato}{Iti 107}), an ocean (\emph{samudda}, \href{https://suttacentral.net/iti69/en/sujato}{Iti 69}), a river (\emph{\textsanskrit{nadī}}) or  a stream, (\emph{sota}, \href{https://suttacentral.net/iti109/en/sujato}{Iti 109}), or even a treacherous whirlpool (\emph{\textsanskrit{āvaṭṭa}}, \href{https://suttacentral.net/iti109/en/sujato\#7.1}{Iti 109:7.1}). A perfected one crosses (\emph{tarati}, occurring about a dozen times) beyond all these, going to the far side (\emph{\textsanskrit{pāra}}, \href{https://suttacentral.net/iti69/en/sujato\#2.4}{Iti 69:2.4}).

On the other hand, the bursting of a rain cloud is like one whose generosity covers all quarters (\href{https://suttacentral.net/iti75/en/sujato}{Iti 75}), while one who understands the Dhamma is like a still lake unruffled by the winds (\href{https://suttacentral.net/iti92/en/sujato}{Iti 92}).

Metaphors based upon light are almost as numerous as the aqueous metaphors, but unlike the water’s ambiguity, they are invariably positive, contrasting with the darkness of ignorance. The awakened mendicants are the “torch-bearers” for those still in darkness (\href{https://suttacentral.net/iti104/en/sujato\#2.9}{Iti 104:2.9}).  Of the heavenly bodies, we find the sun (\href{https://suttacentral.net/iti59/en/sujato\#3.4}{Iti 59:3.4}, \href{https://suttacentral.net/iti88/en/sujato\#11.4}{Iti 88:11.4}), the moon (\href{https://suttacentral.net/iti27/en/sujato\#9.3}{Iti 27:9.3}, \href{https://suttacentral.net/iti74/en/sujato\#7.5}{Iti 74:7.5}), and the morning-star, (\emph{osadhitaraka}, \href{https://suttacentral.net/iti27/en/sujato\#5.1}{Iti 27:5.1}) as images of wisdom and freedom.

Fire, like water, is often negative. The impressive and relatively modern technology of smelting iron, based on the capcity to focus and amplify heat beyond anything experienced in nature, provides a suitable metaphor for the fate awaiting evildoers in hell (\href{https://suttacentral.net/iti48/en/sujato\#5.1}{Iti 48:5.1}, \href{https://suttacentral.net/iti91/en/sujato\#5.1}{Iti 91:5.1}).

The highest goal of \textsanskrit{Nibbāna}, of course, is the quenching or extinguishment of a flame, the fundamental image of all Buddhism. One who has attained such is far from those who are still burning (\href{https://suttacentral.net/iti92/en/sujato\#4.4}{Iti 92:4.4}).

Ambiguously, Devadatta before his fall had glory that shone forth like the crest of a flame (\href{https://suttacentral.net/iti89/en/sujato\#4.3)}{Iti 89:4.3}).

Despite its rather austere style, at times the Itivuttaka builds images in way that accumulate beauty. A gentle series of similes describes one who is able to let go of greed, like the water that rolls off a lotus leaf; hate, like a palm-leaf falling from its stem; and delusion, like the rising sun banishing the dark (\href{https://suttacentral.net/iti88/en/sujato}{Iti 88}).

Few animals are mentioned in the Itivuttaka, but of those that are we find the Buddha compared to a lion, \emph{\textsanskrit{sīha}} (\href{https://suttacentral.net/iti112/en/sujato\#10.2}{Iti 112:10.2}); and a man who wraps stinky fish  \emph{maccha} (\href{https://suttacentral.net/iti76/en/sujato\#7.1}{Iti 76:7.1}).

One who enters the water is at risk of being devoured by the saltwater crocodile (\emph{gaha}, \href{https://suttacentral.net/iti69/en/sujato\#2.2}{Iti 69:2.2}, \href{https://suttacentral.net/iti109/en/sujato\#2.3}{Iti 109:2.3}). Science tells us that seven species of crocodile flourished in ancient India, sadly reduced to three in the present day. Pali offers us at least five words for these: \emph{\textsanskrit{suṁsumārā}}, \emph{\textsanskrit{susukā}}, \emph{nakka}, \emph{gaha} (\emph{\textsanskrit{gāha}},  \emph{gahaka}), and \emph{\textsanskrit{kumbhīlā}}. It is not clear which words apply to which species, or even if they correlate to different species at all. Only the \emph{gaha} appears in the Itivuttaka, and it notably is a denizen of both oceans and large lakes, so I think it must be a saltwater crocodile. Since the  \emph{\textsanskrit{kumbhīlā}} and \emph{\textsanskrit{susukā}} do not appear to inhabit the ocean, I call them “marsh crocodiles” and “gharials” respectivelly, while \emph{\textsanskrit{suṁsumārā}} is generically a “crocodile”.

There is, I believe, an implicit metaphor in the term \emph{\textsanskrit{siṅgī}}, “fraud”, that is applied to a bad monk in \href{https://suttacentral.net/iti108/en/sujato}{Iti 108}. This is a word for either “horn” or “gold”. I believe the latter is meant here, for there are hints that the kind of gold meant is adulterated, perhaps a form of rose gold admixed with copper to form jewellery. The robes accepted by the Buddha shortly before his \textsanskrit{Parinibbāna} are this color, where they are meant to contrast palely with the true gold of the Buddha’s skin. I think \emph{\textsanskrit{siṅgī}} became an idiomatic term for “false gold” and hence a bad monk.

A few miscellaneous similes are worthy of note. The poisoned arrow that contaminates its quiver (\href{https://suttacentral.net/iti76/en/sujato}{Iti 76}) is like a bad person who infects those around them. The striking image occurs of the conduit to rebirth (\emph{netti}, \href{https://suttacentral.net/iti43/en/sujato\#4.1}{Iti 43:4.1}), psychologically explained as craving, but metaphorically evocative of a line or a link that leads from one life to the next. Finally, the piling up of a person’s bones in their countless rebirths would reach higher than the great mountain of Vulture Peak (\href{https://suttacentral.net/iti24/en/sujato}{Iti 24}).

\section*{A Few Remarks on the Discourses}

\subsection*{The Ones}

The first chapter begins with a series of teachings on the “one thing”. This follows the pattern of the \textsanskrit{Aṅguttara}, although the specifics are different. Rather than opening with the overcoming of sensual desire, here we begin with what must be given up in order to guarantee non-return. This is the third of the four stages of awakening commonly taught in the Buddhist texts:

\begin{itemize}%
\item Stream-entry (\emph{\textsanskrit{sotāpatti}})%
\item Once-return (\emph{\textsanskrit{sakadāgāmitā}})%
\item Non-return (\emph{\textsanskrit{anāgāmitā}})%
\item Perfection (\emph{\textsanskrit{arahattā}})%
\end{itemize}

The qualities spoken of, however, don’t always sit easily with this ideal. A non-returner has given up greed (\href{https://suttacentral.net/iti1/en/sujato}{Iti 1}) and hate (\href{https://suttacentral.net/iti2/en/sujato}{Iti 2}), but they have not given up delusion (\href{https://suttacentral.net/iti3/en/sujato}{Iti 3}) or conceit (\href{https://suttacentral.net/iti6/en/sujato}{Iti 6}). Perhaps the text has been overly-systematized, since these details are repeated in the next section where they fit better. But this is the kind of detail that the Pali texts are normally very careful with.

The next series graduates from non-return to speak of ending suffering through complete understanding, which implies arahantship. This pattern crosses over the boundary of the second chapter, which suggests that the texts were grouped together prior to being somewhat arbitrarily organized in groups of ten. Likewise, the pair on the “corrupted mind” (\href{https://suttacentral.net/iti20/en/sujato}{Iti 20}) and “pure mind” (\href{https://suttacentral.net/iti21/en/sujato}{Iti 21}) also cross the chapter boundary. Again, we find a similar phenomenon in the \textsanskrit{Aṅguttara}, where for example the discourses on the radiant or corrupted mind (\href{https://suttacentral.net/1.49/en/sujato}{1.49}–52) cross the boundary of the fifth and sixth chapters.

These details are not very important in themselves, but they do indicate the struggles of the redactors to formalize the organization of texts. If we are alert to these issues, we guard against reading undue significance into mere editorial choices.

The benefits of the meditation on love are extolled in (\href{https://suttacentral.net/iti22/en/sujato}{Iti 22}), which details some of the Buddha’s own past life practices, and (\href{https://suttacentral.net/iti27/en/sujato}{Iti 27}), which is adorned by a series of glorious metaphors. This relatively extended and exalted text forms a suitable conclusion to the first part of the book. This pattern repeats throughout the Itivuttaka, as the final discourse of each of the numbers deals with deep matters in a solemn and serious tone.

\subsection*{The Twos}

Continuing a similar approach, the second chapter speaks of sets of “two things” that lead to happiness or suffering, or else practices that lead to one of two good results.

In a break from the practical ethics of most of the Itivuttaka, \href{https://suttacentral.net/iti43/en/sujato}{Iti 43} speaks of \textsanskrit{Nibbāna} as the “unborn”, in a passage shared with \href{https://suttacentral.net/ud8.3/en/sujato}{Ud 8.3}. Here an extra set of verses is added, adding to the impression that the Itivuttaka is compiled from earlier texts, sometimes with additions.

In \href{https://suttacentral.net/iti44/en/sujato}{Iti 44} we find one of the few distinct doctrinal contributions of the Itivuttaka. It introduces the distinction between “the element of extinguishment with something left over” (\emph{\textsanskrit{saupādisesā} ca \textsanskrit{nibbānadhātu}}) and “the element of extinguishment with nothing left over” (\emph{\textsanskrit{anupādisesā} ca \textsanskrit{nibbānadhātu}}). The first refers to an arahant who has abandoned all defilements, yet who continues to live and experience pleasure and pain. The second refers to an arahant for whom “everything that’s felt, being no longer relished, will become cool right here”. This presumably refers to the time of death, an inference that is confirmed in the verses. The idea of the “element of extinguishment with nothing left over” is found elsewhere in the Suttas in the same sense, but here it is more clearly defined. And while the contrast with what “has something left over” is found elsewhere in the Suttas, nowhere is this said to be an “element of exitinguishment”. A fine distinction to be sure, but it indicates that the Itivuttaka is not solely a remix of teachings from elsewhere in the canon. This distinction went on to become a fundamental aspect of the Theravadin teachings on \textsanskrit{Nibbāna}.

\subsection*{The Threes}

Rather than contrasting pairs, the Threes begins with the enumeration of various sets of three principles, such as greed, hate, and delusion, or the three feelings.

While most of the teachings are familiar from elsewhere in the Suttas, we find a few unique presentations. For example, in \href{https://suttacentral.net/iti74/en/sujato}{Iti 74} a child is said to better, equal, or fail their birth, while the famous simile of the generous giver who is like a rainstorm over all quarters is found in \href{https://suttacentral.net/iti75/en/sujato}{Iti 75}. One who wisely wishes for even the worldly aims of wealth, praise, and heaven should guard their morality \href{https://suttacentral.net/iti76/en/sujato}{Iti 76}.

At \href{https://suttacentral.net/iti77/en/sujato}{Iti 77} we find a rather blunt assessment of the fragility of the body, consciousness, and all attachments. And while it is commonly said that a heavenly rebirth is a reward for good deeds, the aspiration to heaven is put to question by the fact that even the gods celebrate a mendicant going forth (\href{https://suttacentral.net/iti82/en/sujato}{Iti 82}), and the end of their all-too-temporary lives is foreshadowed by five signs (\href{https://suttacentral.net/iti83/en/sujato}{Iti 83}).

Among the straightforward, didactic texts of the Itivuttaka, we find an occasional passage of a more subtle philosophical nature. Such is \href{https://suttacentral.net/iti63/en/sujato}{Iti 63} on the three “periods” of the past, future, and present. According to Buddhist philosophy, the use of language embeds notions of time in the very pathways of thought. Thus those who are still trapped in the “communicable” (\emph{akkheyya}) do not find the peace that is beyond time and reckoning (\emph{\textsanskrit{saṅkhyaṁ} nopeti}).

\subsection*{The Fours}

The discourses of the Fours are often held to be later than the other numbers; I don’t know that I am completely persuaded by this, but certainly the section is notable for its brevity. While the exact forms of the discourses are sometimes unique to the Itivuttaka, there is nothing in the teachings that would not be familiar to a student of the four \textsanskrit{Nikāyas}.

\section*{A Brief Textual History}

A latin-script edition of the Itivuttaka was published in 1889 by the Pali Text Society. It was edited by the handsomely-bearded Ernst Windisch, who was a professor of Sanskrit and comparative linguistics at the University of Leipzig. He made use of three manuscripts in Sinhalese script and four in Burmese, as well as a copy of the commentary. He notes that the Sinhalese manuscripts appear to have been influenced by Burmese script, an indication that they were copied from Burmese sources. His primary source was a Burmese manuscript held in the India Office Library, which he describes as “beautifully written”, and which almost always held the more correct reading. Windisch gave each Sutta a number in simple sequence, a numbering system that is still used by SuttaCentral. His discussion of the manuscripts is exemplary, and well worth a read to see the process by which modern editions are created. The edition is extensively footnoted, and is praised by Ireland and by Moore, who calls it “admirable”. Masefield, however, draws attention to the “poor quality of many readings” in this edition, for which he supplies emendations.

The first English translation was published by Columbia University Press in 1908 by Justin Hartley Moore under the title \textit{Sayings of Buddha}. Moore undertook the translation for his Phd program at Columbia University, a task he described as “a dive into unfathomed waters”. Moore’s introductory essay remains one of the more complete surveys of the text. And in addition, he published \textit{Metrical Analysis of the \textsanskrit{Pāli} Iti-vuttaka, a Collection of Discourses of Buddha} (Journal of the American Oriental Society, vol. 28 1907), which was an early contribution to the difficult and still understudied field of Pali metre.  On the question of authorship, Moore suggests that the verses may be older, while the prose portions “bear all the ear-marks of a short commentary on the succeeding verses”. I find his argumentation here curiously unpersuasive; he presents a couple of examples in support, but I fail to see how they relate to his argument. As I mentioned above, I think it is more likely that to the extent that the prose and verse have separate origins, the verses were added to the prose.

F.L. Woodward was the next to translate the text into English, under the title \textit{As It Was Said}. It was published by the Pali Text Society in 1935 together with his translation of the \textsanskrit{Udāna} with the collective title, \textit{Minor Anthologies of the Pali Canon, Part II}. Woodward endorses Moore’s view that the prose is a commentary on the verses. His translation is unfortunately marred by the then-fashionable tendency to render religious text with deliberate archaisms. Time has not been kind to these stylings.

As is usual with English translations from the Pali, there is a marked leap from the first or second-generation translations, and those completed after the work of Bhikkhu \textsanskrit{Ñāṇamoḷi} in the 1950s. All later translators aspired to his consistency, clarity, and straight-forwardness of diction. The first modern translation of the Itivuttaka was that of John Ireland, originally published through the Buddhist Publications Society in 1991, and subsequently reprinted together with his equally readable translation of the \textsanskrit{Udāna}. It contains a brief introduction and notes.

Peter Masefield published a highly literal translation in 2000 with the Pali Text Society. This was a companion to his translation of the commentary, and is intended to present the text as understood by the commentator. It was completed while the translator was at the University of Sydney, which makes my translation the second to be done in Sydney. And since Woodward made his while in Tasmania, mine is the third translation of the Itivuttaka to be completed in Australia.

Bhikkhu \textsanskrit{Ṭhānissaro} published a translation in 2001 under the title \textit{This Was Said by the Buddha}, revised with a new introduction in 2013. In 2018 Anagarika Mahendra (AKA \textsanskrit{Sāmaṇera} Mahinda) published a “contemporary” translation with both Pali and English under the title \textit{Book of This Was Said} through Dhamma Publishers. And a simple English version is made available by Ven. Gnanananda Thero on his Sutta Friends website.

%
\chapter*{Acknowledgements}
\addcontentsline{toc}{chapter}{Acknowledgements}
\markboth{Acknowledgements}{Acknowledgements}

I remember with gratitude all those from whom I have learned the Dhamma, especially Ajahn Brahm and Bhikkhu Bodhi, the two monks who more than anyone else showed me the depth, meaning, and practical value of the Suttas.

Special thanks to Dustin and Keiko Cheah and family, who sponsored my stay in Qi Mei while I made this translation.

Thanks also for Blake Walshe, who provided essential software support for my translation work.

Throughout the process of translation, I have frequently sought feedback and suggestions from the community on the SuttaCentral community on our forum, “Discuss and Discover”. I want to thank all those who have made suggestions and contributed to my understanding, as well as to the moderators who have made the forum possible. A special thanks is due to \textsanskrit{Sabbamittā}, a true friend of all, who has tirelessly and precisely checked my work.

Finally my everlasting thanks to all those people, far too many to mention, who have supported SuttaCentral, and those who have supported my life as a monastic. None of this would be possible without you.

%
\mainmatter%
\pagestyle{fancy}%
\addtocontents{toc}{\let\protect\contentsline\protect\nopagecontentsline}
\part*{The Book of the Ones }
\addcontentsline{toc}{part}{The Book of the Ones }
\markboth{}{}
\addtocontents{toc}{\let\protect\contentsline\protect\oldcontentsline}

%
\addtocontents{toc}{\let\protect\contentsline\protect\nopagecontentsline}
\chapter*{Chapter One }
\addcontentsline{toc}{chapter}{\tocchapterline{Chapter One }}
\addtocontents{toc}{\let\protect\contentsline\protect\oldcontentsline}

%
\section*{{\suttatitleacronym Iti 1}{\suttatitletranslation Greed }{\suttatitleroot Lobhasutta}}
\addcontentsline{toc}{section}{\tocacronym{Iti 1} \toctranslation{Greed } \tocroot{Lobhasutta}}
\markboth{Greed }{Lobhasutta}
\extramarks{Iti 1}{Iti 1}

This\marginnote{1.1} was said by the Buddha, the Perfected One: that is what I heard. 

“Mendicants,\marginnote{2.1} give up one thing and I guarantee you non-return. What one thing? Greed is the one thing. Give it up, and I guarantee you non-return.” 

The\marginnote{2.6} Buddha spoke this matter. On this it is said: 

\begin{verse}%
“When\marginnote{3.1} overcome by greed \\
beings go to a bad place. \\
Having rightly understood that greed, \\
the discerning give it up. \\
Once they’ve given it up, \\
they never return to this world.” 

%
\end{verse}

This\marginnote{4.1} too is a matter that was spoken by the Blessed One: that is what I heard. 

%
\section*{{\suttatitleacronym Iti 2}{\suttatitletranslation Hate }{\suttatitleroot Dosasutta}}
\addcontentsline{toc}{section}{\tocacronym{Iti 2} \toctranslation{Hate } \tocroot{Dosasutta}}
\markboth{Hate }{Dosasutta}
\extramarks{Iti 2}{Iti 2}

This\marginnote{1.1} was said by the Buddha, the Perfected One: that is what I heard. 

“Mendicants,\marginnote{2.1} give up one thing and I guarantee you non-return. What one thing? Hate is the one thing. Give it up, and I guarantee you non-return.” 

The\marginnote{2.6} Buddha spoke this matter. On this it is said: 

\begin{verse}%
“When\marginnote{3.1} overcome by hate \\
beings go to a bad place. \\
Having rightly understood that hate, \\
the discerning give it up. \\
Once they’ve given it up, \\
they never return to this world.” 

%
\end{verse}

This\marginnote{4.1} too is a matter that was spoken by the Blessed One: that is what I heard. 

%
\section*{{\suttatitleacronym Iti 3}{\suttatitletranslation Delusion }{\suttatitleroot Mohasutta}}
\addcontentsline{toc}{section}{\tocacronym{Iti 3} \toctranslation{Delusion } \tocroot{Mohasutta}}
\markboth{Delusion }{Mohasutta}
\extramarks{Iti 3}{Iti 3}

This\marginnote{1.1} was said by the Buddha, the Perfected One: that is what I heard. 

“Mendicants,\marginnote{2.1} give up one thing and I guarantee you non-return. What one thing? Delusion is the one thing. Give it up, and I guarantee you non-return.” 

The\marginnote{2.6} Buddha spoke this matter. On this it is said: 

\begin{verse}%
“When\marginnote{3.1} overcome by delusion \\
beings go to a bad place. \\
Having rightly understood that delusion, \\
the discerning give it up. \\
Once they’ve given it up, \\
they never return to this world.” 

%
\end{verse}

This\marginnote{4.1} too is a matter that was spoken by the Blessed One: that is what I heard. 

%
\section*{{\suttatitleacronym Iti 4}{\suttatitletranslation Anger }{\suttatitleroot Kodhasutta}}
\addcontentsline{toc}{section}{\tocacronym{Iti 4} \toctranslation{Anger } \tocroot{Kodhasutta}}
\markboth{Anger }{Kodhasutta}
\extramarks{Iti 4}{Iti 4}

This\marginnote{1.1} was said by the Buddha, the Perfected One: that is what I heard. 

“Mendicants,\marginnote{2.1} give up one thing and I guarantee you non-return. What one thing? Anger is the one thing. Give it up, and I guarantee you non-return.” 

The\marginnote{2.6} Buddha spoke this matter. On this it is said: 

\begin{verse}%
“When\marginnote{3.1} overcome by anger \\
beings go to a bad place. \\
Having rightly understood that anger, \\
the discerning give it up. \\
Once they’ve given it up, \\
they never return to this world.” 

%
\end{verse}

This\marginnote{4.1} too is a matter that was spoken by the Blessed One: that is what I heard. 

%
\section*{{\suttatitleacronym Iti 5}{\suttatitletranslation Disdain }{\suttatitleroot Makkhasutta}}
\addcontentsline{toc}{section}{\tocacronym{Iti 5} \toctranslation{Disdain } \tocroot{Makkhasutta}}
\markboth{Disdain }{Makkhasutta}
\extramarks{Iti 5}{Iti 5}

This\marginnote{1.1} was said by the Buddha, the Perfected One: that is what I heard. 

“Mendicants,\marginnote{2.1} give up one thing. and I guarantee you non-return. What one thing? Disdain is the one thing. Give it up, and I guarantee you non-return.” The Buddha spoke this matter. On this it is said: 

\begin{verse}%
“When\marginnote{3.1} overcome by disdain \\
beings go to a bad place. \\
Having rightly understood that disdain, \\
the discerning give it up. \\
Once they’ve given it up, \\
they never return to this world.” 

%
\end{verse}

This\marginnote{4.1} too is a matter that was spoken by the Blessed One: that is what I heard. 

%
\section*{{\suttatitleacronym Iti 6}{\suttatitletranslation Conceit }{\suttatitleroot Mānasutta}}
\addcontentsline{toc}{section}{\tocacronym{Iti 6} \toctranslation{Conceit } \tocroot{Mānasutta}}
\markboth{Conceit }{Mānasutta}
\extramarks{Iti 6}{Iti 6}

This\marginnote{1.1} was said by the Buddha, the Perfected One: that is what I heard. 

“Mendicants,\marginnote{2.1} give up one thing and I guarantee you non-return. What one thing? Conceit is the one thing. Give it up, and I guarantee you non-return.” 

The\marginnote{2.6} Buddha spoke this matter. On this it is said: 

\begin{verse}%
“Drunk\marginnote{3.1} on conceit, \\
beings go to a bad place. \\
Having rightly understood that conceit, \\
the discerning give it up. \\
Once they’ve given it up, \\
they never return to this world.” 

%
\end{verse}

This\marginnote{4.1} too is a matter that was spoken by the Blessed One: that is what I heard. 

%
\section*{{\suttatitleacronym Iti 7}{\suttatitletranslation Complete Understanding of All }{\suttatitleroot Sabbapariññāsutta}}
\addcontentsline{toc}{section}{\tocacronym{Iti 7} \toctranslation{Complete Understanding of All } \tocroot{Sabbapariññāsutta}}
\markboth{Complete Understanding of All }{Sabbapariññāsutta}
\extramarks{Iti 7}{Iti 7}

This\marginnote{1.1} was said by the Buddha, the Perfected One: that is what I heard. 

“Mendicants,\marginnote{2.1} without directly knowing and completely understanding the all, without dispassion for it and giving it up, you can’t end suffering. By directly knowing and completely understanding the all, having dispassion for it and giving it up, you can end suffering.” 

That\marginnote{2.3} is what the Buddha said. On this it is said: 

\begin{verse}%
“Those\marginnote{3.1} who know the all as all, \\
are not attracted to anything. \\
They completely understand all, \\
and have risen above all suffering.” 

%
\end{verse}

This\marginnote{4.1} too is a matter that was spoken by the Blessed One: that is what I heard. 

%
\section*{{\suttatitleacronym Iti 8}{\suttatitletranslation Complete Understanding of Conceit }{\suttatitleroot Mānapariññāsutta}}
\addcontentsline{toc}{section}{\tocacronym{Iti 8} \toctranslation{Complete Understanding of Conceit } \tocroot{Mānapariññāsutta}}
\markboth{Complete Understanding of Conceit }{Mānapariññāsutta}
\extramarks{Iti 8}{Iti 8}

This\marginnote{1.1} was said by the Buddha, the Perfected One: that is what I heard. 

“Mendicants,\marginnote{2.1} without directly knowing and completely understanding conceit, without dispassion for it and giving it up, you can’t end suffering. By directly knowing and completely understanding conceit, having dispassion for it and giving it up, you can end suffering.” 

The\marginnote{2.3} Buddha spoke this matter. On this it is said: 

\begin{verse}%
“These\marginnote{3.1} folk are caught up in conceit, \\
tied by conceit, delighting in existence. \\
Not completely understanding conceit, \\
they return in future lives.” 

Those\marginnote{4.1} who have given up conceit, \\
freed in the ending of conceit, \\
vanquishers of the tie of conceit, \\
have risen above all suffering.” 

%
\end{verse}

This\marginnote{5.1} too is a matter that was spoken by the Blessed One: that is what I heard. 

%
\section*{{\suttatitleacronym Iti 9}{\suttatitletranslation Complete Understanding of Greed }{\suttatitleroot Lobhapariññāsutta}}
\addcontentsline{toc}{section}{\tocacronym{Iti 9} \toctranslation{Complete Understanding of Greed } \tocroot{Lobhapariññāsutta}}
\markboth{Complete Understanding of Greed }{Lobhapariññāsutta}
\extramarks{Iti 9}{Iti 9}

This\marginnote{1.1} was said by the Buddha, the Perfected One: that is what I heard. 

“Mendicants,\marginnote{2.1} without directly knowing and completely understanding greed, without dispassion for it and giving it up, you can’t end suffering. By directly knowing and completely understanding greed, having dispassion for it and giving it up, you can end suffering.” 

The\marginnote{2.3} Buddha spoke this matter. On this it is said: 

\begin{verse}%
“When\marginnote{3.1} overcome by greed \\
beings go to a bad place. \\
Having rightly understood that greed, \\
the discerning give it up. \\
Once they’ve given it up, \\
they never return to this world.” 

%
\end{verse}

This\marginnote{4.1} too is a matter that was spoken by the Blessed One: that is what I heard. 

%
\section*{{\suttatitleacronym Iti 10}{\suttatitletranslation Complete Understanding of Hate }{\suttatitleroot Dosapariññāsutta}}
\addcontentsline{toc}{section}{\tocacronym{Iti 10} \toctranslation{Complete Understanding of Hate } \tocroot{Dosapariññāsutta}}
\markboth{Complete Understanding of Hate }{Dosapariññāsutta}
\extramarks{Iti 10}{Iti 10}

This\marginnote{1.1} was said by the Buddha, the Perfected One: that is what I heard. 

“Mendicants,\marginnote{2.1} without directly knowing and completely understanding hate, without dispassion for it and giving it up, you can’t end suffering. By directly knowing and completely understanding hate, having dispassion for it and giving it up, you can end suffering.” 

The\marginnote{2.3} Buddha spoke this matter. On this it is said: 

\begin{verse}%
“When\marginnote{3.1} overcome by hate \\
beings go to a bad place. \\
Having rightly understood that hate, \\
the discerning give it up. \\
Once they’ve given it up, \\
they never return to this world.” 

%
\end{verse}

This\marginnote{4.1} too is a matter that was spoken by the Blessed One: that is what I heard. 

%
\addtocontents{toc}{\let\protect\contentsline\protect\nopagecontentsline}
\chapter*{Chapter Two }
\addcontentsline{toc}{chapter}{\tocchapterline{Chapter Two }}
\addtocontents{toc}{\let\protect\contentsline\protect\oldcontentsline}

%
\section*{{\suttatitleacronym Iti 11}{\suttatitletranslation Complete Understanding of Delusion }{\suttatitleroot Mohapariññāsutta}}
\addcontentsline{toc}{section}{\tocacronym{Iti 11} \toctranslation{Complete Understanding of Delusion } \tocroot{Mohapariññāsutta}}
\markboth{Complete Understanding of Delusion }{Mohapariññāsutta}
\extramarks{Iti 11}{Iti 11}

This\marginnote{1.1} was said by the Buddha, the Perfected One: that is what I heard. 

“Mendicants,\marginnote{2.1} without directly knowing and completely understanding delusion, without dispassion for it and giving it up, you can’t end suffering. By directly knowing and completely understanding delusion, having dispassion for it and giving it up, you can end suffering.” 

The\marginnote{2.3} Buddha spoke this matter. On this it is said: 

\begin{verse}%
“When\marginnote{3.1} overcome by delusion \\
beings go to a bad place. \\
Having rightly understood that delusion, \\
the discerning give it up. \\
Once they’ve given it up, \\
they never return to this world.” 

%
\end{verse}

This\marginnote{4.1} too is a matter that was spoken by the Blessed One: that is what I heard. 

%
\section*{{\suttatitleacronym Iti 12}{\suttatitletranslation Complete Understanding of Anger }{\suttatitleroot Kodhapariññāsutta}}
\addcontentsline{toc}{section}{\tocacronym{Iti 12} \toctranslation{Complete Understanding of Anger } \tocroot{Kodhapariññāsutta}}
\markboth{Complete Understanding of Anger }{Kodhapariññāsutta}
\extramarks{Iti 12}{Iti 12}

This\marginnote{1.1} was said by the Buddha, the Perfected One: that is what I heard. 

“Mendicants,\marginnote{2.1} without directly knowing and completely understanding anger, without dispassion for it and giving it up, you can’t end suffering. By directly knowing and completely understanding anger, having dispassion for it and giving it up, you can end suffering. 

The\marginnote{2.3} Buddha spoke this matter. On this it is said: 

\begin{verse}%
“When\marginnote{3.1} overcome by anger \\
beings go to a bad place. \\
Having rightly understood that anger, \\
the discerning give it up. \\
Once they’ve given it up, \\
they never return to this world.” 

%
\end{verse}

This\marginnote{4.1} too is a matter that was spoken by the Blessed One: that is what I heard. 

%
\section*{{\suttatitleacronym Iti 13}{\suttatitletranslation Complete Understanding of Disdain }{\suttatitleroot Makkhapariññāsutta}}
\addcontentsline{toc}{section}{\tocacronym{Iti 13} \toctranslation{Complete Understanding of Disdain } \tocroot{Makkhapariññāsutta}}
\markboth{Complete Understanding of Disdain }{Makkhapariññāsutta}
\extramarks{Iti 13}{Iti 13}

This\marginnote{1.1} was said by the Buddha, the Perfected One: that is what I heard. 

“Mendicants,\marginnote{2.1} without directly knowing and completely understanding disdain, without dispassion for it and giving it up, you can’t end suffering. By directly knowing and completely understanding disdain, having dispassion for it and giving it up, you can end suffering.” 

The\marginnote{2.3} Buddha spoke this matter. On this it is said: 

\begin{verse}%
“When\marginnote{3.1} overcome by disdain \\
beings go to a bad place. \\
Having rightly understood that disdain, \\
the discerning give it up. \\
Once they’ve given it up, \\
they never return to this world.” 

%
\end{verse}

This\marginnote{4.1} too is a matter that was spoken by the Blessed One: that is what I heard. 

%
\section*{{\suttatitleacronym Iti 14}{\suttatitletranslation The Shroud of Ignorance }{\suttatitleroot Avijjānīvaraṇasutta}}
\addcontentsline{toc}{section}{\tocacronym{Iti 14} \toctranslation{The Shroud of Ignorance } \tocroot{Avijjānīvaraṇasutta}}
\markboth{The Shroud of Ignorance }{Avijjānīvaraṇasutta}
\extramarks{Iti 14}{Iti 14}

This\marginnote{1.1} was said by the Buddha, the Perfected One: that is what I heard. 

“Mendicants,\marginnote{2.1} I do not see a single shroud, shrouded by which people wander and transmigrate for a long time like the shroud of ignorance. Shrouded by ignorance, people wander and transmigrate for a long time.” 

The\marginnote{2.3} Buddha spoke this matter. On this it is said: 

\begin{verse}%
“There\marginnote{3.1} is no other thing \\
that shrouds people like ignorance. \\
Veiled by delusion, \\
they transmigrate day and night. 

Those\marginnote{4.1} who have given up delusion, \\
shattering the mass of darkness, \\
wander no more, \\
the root is not found in them.” 

%
\end{verse}

This\marginnote{5.1} too is a matter that was spoken by the Blessed One: that is what I heard. 

%
\section*{{\suttatitleacronym Iti 15}{\suttatitletranslation The Fetter of Craving }{\suttatitleroot Taṇhāsaṁyojanasutta}}
\addcontentsline{toc}{section}{\tocacronym{Iti 15} \toctranslation{The Fetter of Craving } \tocroot{Taṇhāsaṁyojanasutta}}
\markboth{The Fetter of Craving }{Taṇhāsaṁyojanasutta}
\extramarks{Iti 15}{Iti 15}

This\marginnote{1.1} was said by the Buddha, the Perfected One: that is what I heard. 

“Mendicants,\marginnote{2.1} I do not see a single fetter, fettered by which people wander and transmigrate for a long time like the fetter of craving. Fettered by craving, people wander and transmigrate for a long time.” 

The\marginnote{2.3} Buddha spoke this matter. On this it is said: 

\begin{verse}%
“Craving\marginnote{3.1} is a person’s partner \\
as they transmigrate on this long journey. \\
They go from this state to another, \\
but don’t escape transmigration. 

Knowing\marginnote{4.1} this danger, \\
that craving is the cause of suffering—\\
rid of craving, free of grasping, \\
a mendicant would wander mindful.” 

%
\end{verse}

This\marginnote{5.1} too is a matter that was spoken by the Blessed One: that is what I heard. 

%
\section*{{\suttatitleacronym Iti 16}{\suttatitletranslation A Trainee (1st) }{\suttatitleroot Paṭhamasekhasutta}}
\addcontentsline{toc}{section}{\tocacronym{Iti 16} \toctranslation{A Trainee (1st) } \tocroot{Paṭhamasekhasutta}}
\markboth{A Trainee (1st) }{Paṭhamasekhasutta}
\extramarks{Iti 16}{Iti 16}

This\marginnote{1.1} was said by the Buddha, the Perfected One: that is what I heard. 

“Taking\marginnote{2.1} into account interior factors, mendicants, I do not see a single one that is so very helpful as proper attention for a trainee mendicant who hasn’t achieved their heart’s desire, but lives aspiring to the supreme sanctuary. A mendicant paying proper attention gives up the unskillful and develops the skillful.” 

The\marginnote{2.3} Buddha spoke this matter. On this it is said: 

\begin{verse}%
“There\marginnote{3.1} is nothing so helpful \\
for a trainee mendicant \\
aspiring for the ultimate goal \\
as proper attention. \\
Striving properly, a mendicant \\
attains the end of suffering.” 

%
\end{verse}

This\marginnote{4.1} too is a matter that was spoken by the Blessed One: that is what I heard. 

%
\section*{{\suttatitleacronym Iti 17}{\suttatitletranslation A Trainee (2nd) }{\suttatitleroot Dutiyasekhasutta}}
\addcontentsline{toc}{section}{\tocacronym{Iti 17} \toctranslation{A Trainee (2nd) } \tocroot{Dutiyasekhasutta}}
\markboth{A Trainee (2nd) }{Dutiyasekhasutta}
\extramarks{Iti 17}{Iti 17}

This\marginnote{1.1} was said by the Buddha, the Perfected One: that is what I heard. 

“Taking\marginnote{2.1} into account exterior factors, mendicants, I do not see a single one that is so very helpful as good friendship for a trainee mendicant who hasn’t achieved their heart’s desire, but lives aspiring to the supreme sanctuary. A mendicant who has good friends gives up the unskillful and develops the skillful.” 

The\marginnote{2.3} Buddha spoke this matter. On this it is said: 

\begin{verse}%
“A\marginnote{3.1} mendicant with good friends \\
is reverential and respectful \\
when their friends are speaking, \\
aware and mindful. \\
Gradually they would attain \\
the ending of all fetters.” 

%
\end{verse}

This\marginnote{4.1} too is a matter that was spoken by the Blessed One: that is what I heard. 

%
\section*{{\suttatitleacronym Iti 18}{\suttatitletranslation Schism in the Saṅgha }{\suttatitleroot Saṁghabhedasutta}}
\addcontentsline{toc}{section}{\tocacronym{Iti 18} \toctranslation{Schism in the Saṅgha } \tocroot{Saṁghabhedasutta}}
\markboth{Schism in the Saṅgha }{Saṁghabhedasutta}
\extramarks{Iti 18}{Iti 18}

This\marginnote{1.1} was said by the Buddha, the Perfected One: that is what I heard. 

“One\marginnote{2.1} thing, mendicants, arises in the world for the hurt and unhappiness of the people, for the harm, hurt, and suffering of gods and humans. What one thing? Schism in the \textsanskrit{Saṅgha}. When the \textsanskrit{Saṅgha} is split, they argue, insult, block, and reject each other. This doesn’t inspire confidence in those without it, and it causes some with confidence to change their minds.” 

The\marginnote{2.6} Buddha spoke this matter. On this it is said: 

\begin{verse}%
“A\marginnote{3.1} schismatic remains for the eon \\
in a place of loss, in hell. \\
Taking a stand against the teaching, \\
favoring factions, they destroy their sanctuary. \\
After causing schism in a harmonious \textsanskrit{Saṅgha}, \\
they burn in hell for an eon.” 

%
\end{verse}

This\marginnote{4.1} too is a matter that was spoken by the Blessed One: that is what I heard. 

%
\section*{{\suttatitleacronym Iti 19}{\suttatitletranslation Harmony in the Saṅgha }{\suttatitleroot Saṁghasāmaggīsutta}}
\addcontentsline{toc}{section}{\tocacronym{Iti 19} \toctranslation{Harmony in the Saṅgha } \tocroot{Saṁghasāmaggīsutta}}
\markboth{Harmony in the Saṅgha }{Saṁghasāmaggīsutta}
\extramarks{Iti 19}{Iti 19}

This\marginnote{1.1} was said by the Buddha, the Perfected One: that is what I heard. 

“One\marginnote{2.1} thing, mendicants, arises in the world for the welfare and happiness of the people, for the benefit, welfare, and happiness of gods and humans. What one thing? Harmony in the \textsanskrit{Saṅgha}. When the \textsanskrit{Saṅgha} is in harmony, they don’t argue, insult, block, or reject each other. This inspires confidence in those without it, and increases confidence in those who have it.” 

The\marginnote{2.6} Buddha spoke this matter. On this it is said: 

\begin{verse}%
“A\marginnote{3.1} \textsanskrit{Saṅgha} in harmony is happy, \\
as is support for those in harmony. \\
Taking a stand on the teaching, \\
favoring harmony, they ruin no sanctuary. \\
After creating harmony in the \textsanskrit{Saṅgha}, \\
they rejoice in heaven for an eon.” 

%
\end{verse}

This\marginnote{4.1} too is a matter that was spoken by the Blessed One: that is what I heard. 

%
\section*{{\suttatitleacronym Iti 20}{\suttatitletranslation A Corrupted Mind }{\suttatitleroot Paduṭṭhacittasutta}}
\addcontentsline{toc}{section}{\tocacronym{Iti 20} \toctranslation{A Corrupted Mind } \tocroot{Paduṭṭhacittasutta}}
\markboth{A Corrupted Mind }{Paduṭṭhacittasutta}
\extramarks{Iti 20}{Iti 20}

This\marginnote{1.1} was said by the Buddha, the Perfected One: that is what I heard. 

“Mendicants,\marginnote{2.1} when I’ve comprehended the mind of a person whose mind is corrupted, I understand: ‘If this person were to die right now, they would be cast down to hell.’ Why is that? Because their mind is corrupted. Corruption of mind is the reason why some sentient beings, when their body breaks up, after death, are reborn in a place of loss, a bad place, the underworld, hell.” 

The\marginnote{2.6} Buddha spoke this matter. On this it is said: 

\begin{verse}%
“Knowing\marginnote{3.1} a person’s \\
corrupted mind, \\
the Buddha explained this matter \\
in the mendicants’ presence. 

If\marginnote{4.1} that person \\
were to die at this time, \\
they’d be reborn in hell, \\
for their mind is corrupted. 

Such\marginnote{5.1} a person is cast down as surely \\
as if they’d been carried off and put there. \\
For corruption of mind is the reason \\
sentient beings go to a bad place.” 

%
\end{verse}

This\marginnote{6.1} too is a matter that was spoken by the Blessed One: that is what I heard. 

%
\addtocontents{toc}{\let\protect\contentsline\protect\nopagecontentsline}
\chapter*{Chapter Three }
\addcontentsline{toc}{chapter}{\tocchapterline{Chapter Three }}
\addtocontents{toc}{\let\protect\contentsline\protect\oldcontentsline}

%
\section*{{\suttatitleacronym Iti 21}{\suttatitletranslation A Pure Mind }{\suttatitleroot Pasannacittasutta}}
\addcontentsline{toc}{section}{\tocacronym{Iti 21} \toctranslation{A Pure Mind } \tocroot{Pasannacittasutta}}
\markboth{A Pure Mind }{Pasannacittasutta}
\extramarks{Iti 21}{Iti 21}

This\marginnote{1.1} was said by the Buddha, the Perfected One: that is what I heard. 

“Mendicants,\marginnote{2.1} when I’ve comprehended the mind of a person whose mind is pure, I understand: ‘If this person were to die right now, they would be raised up to heaven.’ Why is that? Because their mind is pure. Purity of mind is the reason why some sentient beings, when their body breaks up, after death, are reborn in a good place, a heavenly realm.” 

The\marginnote{2.6} Buddha spoke this matter. On this it is said: 

\begin{verse}%
“Knowing\marginnote{3.1} a person’s \\
pure mind, \\
the Buddha explained this matter \\
in the mendicants’ presence. 

If\marginnote{4.1} that person \\
were to die at this time, \\
they’d be reborn in heaven, \\
for their mind is pure. 

Such\marginnote{5.1} a person is raised up as surely \\
as if they’d been carried and put there. \\
For purity of mind is the reason \\
sentient beings go to a good place.” 

%
\end{verse}

This\marginnote{6.1} too is a matter that was spoken by the Blessed One: that is what I heard. 

%
\section*{{\suttatitleacronym Iti 22}{\suttatitletranslation The Benefits of Love }{\suttatitleroot Mettasutta}}
\addcontentsline{toc}{section}{\tocacronym{Iti 22} \toctranslation{The Benefits of Love } \tocroot{Mettasutta}}
\markboth{The Benefits of Love }{Mettasutta}
\extramarks{Iti 22}{Iti 22}

This\marginnote{1.1} was said by the Buddha, the Perfected One: that is what I heard. 

“Mendicants,\marginnote{2.1} don’t fear good deeds. For ‘good deeds’ is a term for happiness, for what is likable, desirable, and agreeable. I recall undergoing for a long time the likable, desirable, and agreeable results of good deeds performed over a long time. As a result, for seven eons of the cosmos contracting and expanding I didn’t return to this world again. As the eon contracted I went to the realm of streaming radiance. As it expanded I was reborn in an empty mansion of \textsanskrit{Brahmā}. 

There\marginnote{3.1} I was \textsanskrit{Brahmā}, the Great \textsanskrit{Brahmā}, the undefeated, the champion, the universal seer, the wielder of power. I was Sakka, lord of gods, thirty-six times. Many hundreds of times I was a king, a wheel-turning monarch, a just and principled king. My dominion extended to all four sides, I achieved stability in the country, and I possessed the seven treasures. Not to mention regional kingship! 

Then\marginnote{4.1} I thought, ‘Of what deed of mine is this the fruit and result, that I am now so mighty and powerful?’ Then I thought, ‘It is the fruit and result of three kinds of deeds: giving, self-control, and restraint.’” 

The\marginnote{4.6} Buddha spoke this matter. On this it is said: 

\begin{verse}%
“One\marginnote{5.1} should practice only good deeds, \\
whose happy outcome stretches ahead. \\
Giving and moral conduct, \\
developing a mind of love: 

having\marginnote{6.1} developed these \\
three things yielding happiness, \\
that astute one is reborn \\
in a happy, pleasing world.” 

%
\end{verse}

This\marginnote{7.1} too is a matter that was spoken by the Blessed One: that is what I heard. 

%
\section*{{\suttatitleacronym Iti 23}{\suttatitletranslation Both Kinds of Benefit }{\suttatitleroot Ubhayatthasutta}}
\addcontentsline{toc}{section}{\tocacronym{Iti 23} \toctranslation{Both Kinds of Benefit } \tocroot{Ubhayatthasutta}}
\markboth{Both Kinds of Benefit }{Ubhayatthasutta}
\extramarks{Iti 23}{Iti 23}

This\marginnote{1.1} was said by the Buddha, the Perfected One: that is what I heard. 

“This\marginnote{2.1} one thing, mendicants, when developed and cultivated, secures benefits for both the present life and lives to come. What one thing? Diligence in skillful qualities. This is the one thing that, when developed and cultivated, secures benefits for both the present life and lives to come.” 

The\marginnote{2.7} Buddha spoke this matter. On this it is said: 

\begin{verse}%
“The\marginnote{3.1} astute praise diligence \\
in making merit. \\
Being diligent, an astute person \\
secures both benefits: 

the\marginnote{4.1} benefit in this life, \\
and in lives to come. \\
A wise one, comprehending the meaning, \\
is said to be astute.” 

%
\end{verse}

This\marginnote{5.1} too is a matter that was spoken by the Blessed One: that is what I heard. 

%
\section*{{\suttatitleacronym Iti 24}{\suttatitletranslation A Heap of Bones }{\suttatitleroot Aṭṭhipuñjasutta}}
\addcontentsline{toc}{section}{\tocacronym{Iti 24} \toctranslation{A Heap of Bones } \tocroot{Aṭṭhipuñjasutta}}
\markboth{A Heap of Bones }{Aṭṭhipuñjasutta}
\extramarks{Iti 24}{Iti 24}

This\marginnote{1.1} was said by the Buddha, the Perfected One: that is what I heard. 

“Mendicants,\marginnote{2.1} one person roaming and transmigrating for an eon would amass a heap of bones the size of this Mount Vepulla, if they were gathered together and not lost.” 

The\marginnote{2.2} Buddha spoke this matter. On this it is said: 

\begin{verse}%
“If\marginnote{3.1} the bones of a single person \\
for a single eon were gathered up, \\
they’d make a pile the size of a mountain: \\
so said the great hermit. 

And\marginnote{4.1} this is declared to be \\
as huge as Mount Vepulla, \\
higher than the Vulture’s Peak \\
near the Mountainfold of the Magadhans. 

But\marginnote{5.1} then, with right understanding, \\
a person sees the noble truths—\\
suffering, suffering’s origin, \\
suffering’s transcendence, \\
and the noble eightfold path \\
that leads to the stilling of suffering. 

After\marginnote{6.1} roaming on seven times at most, \\
that person \\
makes an end of suffering, \\
with the ending of all fetters.” 

%
\end{verse}

This\marginnote{7.1} too is a matter that was spoken by the Blessed One: that is what I heard. 

%
\section*{{\suttatitleacronym Iti 25}{\suttatitletranslation Lying }{\suttatitleroot Musāvādasutta}}
\addcontentsline{toc}{section}{\tocacronym{Iti 25} \toctranslation{Lying } \tocroot{Musāvādasutta}}
\markboth{Lying }{Musāvādasutta}
\extramarks{Iti 25}{Iti 25}

This\marginnote{1.1} was said by the Buddha, the Perfected One: that is what I heard. 

“Mendicants,\marginnote{2.1} for an individual who transgresses in one thing, there is no bad deed they would not do, I say. What one thing? It is this: telling a deliberate lie.” 

The\marginnote{2.4} Buddha spoke this matter. On this it is said: 

\begin{verse}%
“When\marginnote{3.1} a person, spurning the hereafter, \\
transgresses in just one thing—\\
lying—\\
there is no evil they would not do.” 

%
\end{verse}

This\marginnote{4.1} too is a matter that was spoken by the Blessed One: that is what I heard. 

%
\section*{{\suttatitleacronym Iti 26}{\suttatitletranslation Giving }{\suttatitleroot Dānasutta}}
\addcontentsline{toc}{section}{\tocacronym{Iti 26} \toctranslation{Giving } \tocroot{Dānasutta}}
\markboth{Giving }{Dānasutta}
\extramarks{Iti 26}{Iti 26}

This\marginnote{1.1} was said by the Buddha, the Perfected One: that is what I heard. 

“Mendicants,\marginnote{2.1} if sentient beings only knew, as I do, the fruit of giving and sharing, they would not eat without first giving, and the stain of stinginess would not occupy their minds. They would not eat without sharing even their last mouthful, their last morsel, so long as there was someone to receive it. It is because sentient beings do not know, as I do, the fruit of giving and sharing, that they eat without first giving, and the stain of stinginess occupies their minds.” 

The\marginnote{2.4} Buddha spoke this matter. On this it is said: 

\begin{verse}%
“If\marginnote{3.1} sentient beings only knew \\
how great is the fruit \\
of giving and sharing \\
as taught by the great hermit! 

Rid\marginnote{4.1} of the stain of stinginess, \\
with clear and confident heart, \\
they would duly give to the noble ones, \\
where a gift is very fruitful. 

Having\marginnote{5.1} given food in abundance \\
to those worthy of a religious donation, \\
after passing from the human realm, \\
the givers go to heaven. 

And\marginnote{6.1} when they have arrived there in heaven, \\
they enjoy all the pleasures they desire. \\
The generous enjoy the \\
fruit of giving and sharing.” 

%
\end{verse}

This\marginnote{7.1} too is a matter that was spoken by the Blessed One: that is what I heard. 

%
\section*{{\suttatitleacronym Iti 27}{\suttatitletranslation The Meditation on Love }{\suttatitleroot Mettābhāvanāsutta}}
\addcontentsline{toc}{section}{\tocacronym{Iti 27} \toctranslation{The Meditation on Love } \tocroot{Mettābhāvanāsutta}}
\markboth{The Meditation on Love }{Mettābhāvanāsutta}
\extramarks{Iti 27}{Iti 27}

This\marginnote{1.1} was said by the Buddha, the Perfected One: that is what I heard. 

“Mendicants,\marginnote{2.1} of all the grounds for making worldly merit, none are worth a sixteenth part of the heart’s release by love. Surpassing them, the heart’s release by love shines and glows and radiates. 

It’s\marginnote{3.1} like how the radiance of all the stars is not worth a sixteenth part of the moon’s radiance. Surpassing them, the moon’s radiance shines and glows and radiates. In the same way, of all the grounds for making worldly merit, none are worth a sixteenth part of the heart’s release by love. Surpassing them, the heart’s release by love shines and glows and radiates. 

It’s\marginnote{4.1} like the time after the rainy season when the sky is clear and cloudless. And when the sun rises, it dispels all the darkness from the sky as it shines and glows and radiates. In the same way, of all the grounds for making worldly merit, none are worth a sixteenth part of the heart’s release by love. Surpassing them, the heart’s release by love shines and glows and radiates. 

It’s\marginnote{5.1} like how after the rainy season the sky is clear and cloudless. At the crack of dawn, the Morning Star shines and glows and radiates. In the same way, of all the grounds for making worldly merit, none are worth a sixteenth part of the heart’s release by love. Surpassing them, the heart’s release by love shines and glows and radiates.” 

The\marginnote{5.3} Buddha spoke this matter. On this it is said: 

\begin{verse}%
“A\marginnote{6.1} mindful one who develops \\
limitless love \\
weakens the fetters, \\
seeing the ending of attachments. 

Loving\marginnote{7.1} just one creature with a hateless heart \\
makes you a good person. \\
Compassionate for all creatures, \\
a noble one creates abundant merit. 

The\marginnote{8.1} royal potentates conquered this land \\
and traveled around sponsoring sacrifices—\\
horse sacrifice, human sacrifice, \\
the sacrifices of the ‘stick-casting’, the ‘royal soma drinking’, and the ‘unbarred’. 

These\marginnote{9.1} are not worth a sixteenth part \\
of the mind developed with love, \\
as starlight cannot rival the moon. 

Don’t\marginnote{10.1} kill or cause others to kill, \\
don’t conquer or encourage others to conquer, \\
with love for all living creatures—\\
you’ll have no enmity for anyone.” 

%
\end{verse}

This\marginnote{11.1} too is a matter that was spoken by the Blessed One: that is what I heard. 

%
\addtocontents{toc}{\let\protect\contentsline\protect\nopagecontentsline}
\part*{The Book of the Twos }
\addcontentsline{toc}{part}{The Book of the Twos }
\markboth{}{}
\addtocontents{toc}{\let\protect\contentsline\protect\oldcontentsline}

%
\addtocontents{toc}{\let\protect\contentsline\protect\nopagecontentsline}
\chapter*{Chapter One }
\addcontentsline{toc}{chapter}{\tocchapterline{Chapter One }}
\addtocontents{toc}{\let\protect\contentsline\protect\oldcontentsline}

%
\section*{{\suttatitleacronym Iti 28}{\suttatitletranslation Living in Suffering }{\suttatitleroot Dukkhavihārasutta}}
\addcontentsline{toc}{section}{\tocacronym{Iti 28} \toctranslation{Living in Suffering } \tocroot{Dukkhavihārasutta}}
\markboth{Living in Suffering }{Dukkhavihārasutta}
\extramarks{Iti 28}{Iti 28}

This\marginnote{1.1} was said by the Buddha, the Perfected One: that is what I heard. 

“Mendicants,\marginnote{2.1} when a mendicant has two qualities they live unhappily in the present life—with distress, anguish, and fever—and when the body breaks up, after death, they can expect a bad rebirth. What two? Not guarding the sense doors and eating too much. When a mendicant has these two qualities they live unhappily in the present life—with distress, anguish, and fever—and when the body breaks up, after death, they can expect a bad rebirth.” 

The\marginnote{2.7} Buddha spoke this matter. On this it is said: 

\begin{verse}%
“Eye,\marginnote{3.1} ear, nose, \\
tongue, body, and likewise mind: \\
a mendicant who leaves these \\
sense doors unguarded—

immoderate\marginnote{4.1} in eating, \\
sense faculties unrestrained—\\
reaps suffering \\
both physical and mental. 

Burning\marginnote{5.1} in body, \\
burning in mind, \\
by day or by night \\
such a person lives in suffering.” 

%
\end{verse}

This\marginnote{6.1} too is a matter that was spoken by the Blessed One: that is what I heard. 

%
\section*{{\suttatitleacronym Iti 29}{\suttatitletranslation Living in Happiness }{\suttatitleroot Sukhavihārasutta}}
\addcontentsline{toc}{section}{\tocacronym{Iti 29} \toctranslation{Living in Happiness } \tocroot{Sukhavihārasutta}}
\markboth{Living in Happiness }{Sukhavihārasutta}
\extramarks{Iti 29}{Iti 29}

This\marginnote{1.1} was said by the Buddha, the Perfected One: that is what I heard. 

“Mendicants,\marginnote{2.1} when a mendicant has two qualities they live happily in the present life—without distress, anguish, and fever—and when the body breaks up, after death, they can expect a good rebirth. What two? Guarding the sense doors and moderation in eating. When a mendicant has these two qualities they live happily in the present life—without distress, anguish, and fever—and when the body breaks up, after death, they can expect a good rebirth.” 

The\marginnote{2.7} Buddha spoke this matter. On this it is said: 

\begin{verse}%
“Eye,\marginnote{3.1} ear, nose, \\
tongue, body, and likewise mind: \\
a mendicant who makes these \\
sense doors well guarded—

eating\marginnote{4.1} in moderation, \\
restrained in the sense faculties—\\
reaps happiness \\
both physical and mental. 

Not\marginnote{5.1} burning in body, \\
not burning in mind, \\
by day or by night \\
such a person lives in happiness.” 

%
\end{verse}

This\marginnote{6.1} too is a matter that was spoken by the Blessed One: that is what I heard. 

%
\section*{{\suttatitleacronym Iti 30}{\suttatitletranslation Mortifying }{\suttatitleroot Tapanīyasutta}}
\addcontentsline{toc}{section}{\tocacronym{Iti 30} \toctranslation{Mortifying } \tocroot{Tapanīyasutta}}
\markboth{Mortifying }{Tapanīyasutta}
\extramarks{Iti 30}{Iti 30}

This\marginnote{1.1} was said by the Buddha, the Perfected One: that is what I heard. 

“These\marginnote{2.1} two things, mendicants, are mortifying. What two? It’s when someone hasn’t done good and skillful things that keep them safe, but has done bad, violent, and depraved things. Thinking, ‘I haven’t done good things’, they’re mortified. Thinking, ‘I have done bad things’, they’re mortified. These are the two things that are mortifying.” 

The\marginnote{2.6} Buddha spoke this matter. On this it is said: 

\begin{verse}%
“Having\marginnote{3.1} done bad things \\
by way of body, \\
speech, and mind, \\
and whatever else is corrupt; 

not\marginnote{4.1} having done good deeds, \\
and having done many bad, \\
when their body breaks up, that witless person \\
is reborn in hell.” 

%
\end{verse}

This\marginnote{5.1} too is a matter that was spoken by the Blessed One: that is what I heard. 

%
\section*{{\suttatitleacronym Iti 31}{\suttatitletranslation Not Mortifying }{\suttatitleroot Atapanīyasutta}}
\addcontentsline{toc}{section}{\tocacronym{Iti 31} \toctranslation{Not Mortifying } \tocroot{Atapanīyasutta}}
\markboth{Not Mortifying }{Atapanīyasutta}
\extramarks{Iti 31}{Iti 31}

This\marginnote{1.1} was said by the Buddha, the Perfected One: that is what I heard. 

“These\marginnote{2.1} two things, mendicants, are not mortifying. What two? It’s when someone has done good and skillful things that keep them safe, but has not done bad, violent, and depraved things. Thinking, ‘I have done good things’, they’re not mortified. Thinking, ‘I haven’t done bad things’, they’re not mortified. These are the two things that are not mortifying.” 

The\marginnote{2.6} Buddha spoke this matter. On this it is said: 

\begin{verse}%
“Having\marginnote{3.1} given up bad conduct \\
by way of body, \\
speech, and mind, \\
and whatever else is corrupt; 

not\marginnote{4.1} having done bad deeds, \\
and having done many good, \\
when their body breaks up, that wise person \\
is reborn in heaven.” 

%
\end{verse}

This\marginnote{5.1} too is a matter that was spoken by the Blessed One: that is what I heard. 

%
\section*{{\suttatitleacronym Iti 32}{\suttatitletranslation Ethics (1st) }{\suttatitleroot Paṭhamasīlasutta}}
\addcontentsline{toc}{section}{\tocacronym{Iti 32} \toctranslation{Ethics (1st) } \tocroot{Paṭhamasīlasutta}}
\markboth{Ethics (1st) }{Paṭhamasīlasutta}
\extramarks{Iti 32}{Iti 32}

This\marginnote{1.1} was said by the Buddha, the Perfected One: that is what I heard. 

“Mendicants,\marginnote{2.1} a person with two qualities is cast down to hell. What two? Bad conduct and bad view. A person who has these two qualities is cast down to hell.” 

The\marginnote{2.5} Buddha spoke this matter. On this it is said: 

\begin{verse}%
“If\marginnote{3.1} a person possesses \\
these two qualities—\\
bad conduct \\
and bad views—\\
when their body breaks up, that witless person \\
is reborn in hell.” 

%
\end{verse}

This\marginnote{4.1} too is a matter that was spoken by the Blessed One: that is what I heard. 

%
\section*{{\suttatitleacronym Iti 33}{\suttatitletranslation Ethics (2nd) }{\suttatitleroot Dutiyasīlasutta}}
\addcontentsline{toc}{section}{\tocacronym{Iti 33} \toctranslation{Ethics (2nd) } \tocroot{Dutiyasīlasutta}}
\markboth{Ethics (2nd) }{Dutiyasīlasutta}
\extramarks{Iti 33}{Iti 33}

This\marginnote{1.1} was said by the Buddha, the Perfected One: that is what I heard. 

“Mendicants,\marginnote{2.1} a person with two qualities is raised up to heaven. What two? Excellent conduct and excellent view. A person who has these two qualities is is raised up to heaven.” 

The\marginnote{2.5} Buddha spoke this matter. On this it is said: 

\begin{verse}%
“If\marginnote{3.1} a person possesses \\
these two qualities—\\
excellent conduct \\
and excellent views—\\
when their body breaks up, that wise person \\
is reborn in heaven.” 

%
\end{verse}

This\marginnote{4.1} too is a matter that was spoken by the Blessed One: that is what I heard. 

%
\section*{{\suttatitleacronym Iti 34}{\suttatitletranslation Keen }{\suttatitleroot Ātāpīsutta}}
\addcontentsline{toc}{section}{\tocacronym{Iti 34} \toctranslation{Keen } \tocroot{Ātāpīsutta}}
\markboth{Keen }{Ātāpīsutta}
\extramarks{Iti 34}{Iti 34}

This\marginnote{1.1} was said by the Buddha, the Perfected One: that is what I heard. 

“Mendicants,\marginnote{2.1} without being keen and prudent a mendicant can’t achieve awakening, extinguishment, and the supreme sanctuary. But if a mendicant is keen and prudent they can achieve awakening, extinguishment, and the supreme sanctuary.” 

The\marginnote{2.3} Buddha spoke this matter. On this it is said: 

\begin{verse}%
“Neither\marginnote{3.1} keen nor prudent, \\
lazy, lacking energy, \\
full of dullness and drowsiness, \\
unconscientious, lacking regard for others, \\
such a mendicant is incapable \\
of touching the highest awakening. 

One\marginnote{4.1} who is mindful, alert, meditative, \\
keen, prudent, and diligent, \\
having cut the fetter of birth and old age, \\
would realize supreme awakening in this very life.” 

%
\end{verse}

This\marginnote{5.1} too is a matter that was spoken by the Blessed One: that is what I heard. 

%
\section*{{\suttatitleacronym Iti 35}{\suttatitletranslation Deceit and Flattery }{\suttatitleroot Paṭhamajananakuhanasutta}}
\addcontentsline{toc}{section}{\tocacronym{Iti 35} \toctranslation{Deceit and Flattery } \tocroot{Paṭhamajananakuhanasutta}}
\markboth{Deceit and Flattery }{Paṭhamajananakuhanasutta}
\extramarks{Iti 35}{Iti 35}

This\marginnote{1.1} was said by the Buddha, the Perfected One: that is what I heard. 

“Mendicants,\marginnote{2.1} this spiritual life is not lived for the sake of deceiving people or flattering them, nor for the benefit of possessions, honor, or popularity, nor thinking, ‘So let people know about me!’ This spiritual life is lived for the sake of restraint and giving up.” 

The\marginnote{2.3} Buddha spoke this matter. On this it is said: 

\begin{verse}%
“The\marginnote{3.1} Buddha taught the spiritual life \\
not because of tradition, \\
but for the sake of restraint and giving up, \\
and because it culminates in extinguishment. 

This\marginnote{4.1} is the path followed by the great souls, \\
the great hermits. \\
Those who practice it \\
as it was taught by the Buddha \\
doing the teacher’s bidding, \\
make an end of suffering.” 

%
\end{verse}

This\marginnote{5.1} too is a matter that was spoken by the Blessed One: that is what I heard. 

%
\section*{{\suttatitleacronym Iti 36}{\suttatitletranslation Deceit and Flattery }{\suttatitleroot Dutiyajananakuhanasutta}}
\addcontentsline{toc}{section}{\tocacronym{Iti 36} \toctranslation{Deceit and Flattery } \tocroot{Dutiyajananakuhanasutta}}
\markboth{Deceit and Flattery }{Dutiyajananakuhanasutta}
\extramarks{Iti 36}{Iti 36}

This\marginnote{1.1} was said by the Buddha, the Perfected One: that is what I heard. 

“Mendicants,\marginnote{2.1} this spiritual life is not lived for the sake of deceiving people or flattering them, nor for the benefit of possessions, honor, or popularity, nor thinking, ‘So let people know about me!’ This spiritual life is lived for the sake of direct knowledge and complete understanding.” 

The\marginnote{2.3} Buddha spoke this matter. On this it is said: 

\begin{verse}%
“The\marginnote{3.1} Buddha taught the spiritual life \\
not because of tradition, \\
but for the sake of insight and understanding, \\
and because it culminates in extinguishment. 

This\marginnote{4.1} is the path followed by the great souls, \\
the great hermits. \\
Those who practice it \\
as it was taught by the Buddha \\
doing the teacher’s bidding, \\
make an end of suffering.” 

%
\end{verse}

This\marginnote{5.1} too is a matter that was spoken by the Blessed One: that is what I heard. 

%
\section*{{\suttatitleacronym Iti 37}{\suttatitletranslation Happiness }{\suttatitleroot Somanassasutta}}
\addcontentsline{toc}{section}{\tocacronym{Iti 37} \toctranslation{Happiness } \tocroot{Somanassasutta}}
\markboth{Happiness }{Somanassasutta}
\extramarks{Iti 37}{Iti 37}

This\marginnote{1.1} was said by the Buddha, the Perfected One: that is what I heard. 

“Mendicants,\marginnote{2.1} when a mendicant has two qualities they’re full of joy and happiness in the present life, and they have laid the groundwork for ending the defilements. What two? Being inspired at inspiring places, and making a suitable effort when inspired. When a mendicant has these two qualities they’re full of joy and happiness in the present life, and they have laid the groundwork for ending the defilements.” 

The\marginnote{2.5} Buddha spoke this matter. On this it is said: 

\begin{verse}%
“At\marginnote{3.1} inspiring places \\
an astute person should be inspired; \\
a keen and alert mendicant \\
should examine with wisdom. 

A\marginnote{4.1} mendicant living like this, with keen energy, \\
peaceful and stable, \\
devoted to serenity of heart, \\
attains the ending of suffering.” 

%
\end{verse}

This\marginnote{5.1} too is a matter that was spoken by the Blessed One: that is what I heard. 

%
\addtocontents{toc}{\let\protect\contentsline\protect\nopagecontentsline}
\chapter*{Chapter Two }
\addcontentsline{toc}{chapter}{\tocchapterline{Chapter Two }}
\addtocontents{toc}{\let\protect\contentsline\protect\oldcontentsline}

%
\section*{{\suttatitleacronym Iti 38}{\suttatitletranslation Thoughts }{\suttatitleroot Vitakkasutta}}
\addcontentsline{toc}{section}{\tocacronym{Iti 38} \toctranslation{Thoughts } \tocroot{Vitakkasutta}}
\markboth{Thoughts }{Vitakkasutta}
\extramarks{Iti 38}{Iti 38}

This\marginnote{1.1} was said by the Buddha, the Perfected One: that is what I heard. 

“Two\marginnote{2.1} thoughts, mendicants, often occur to the Realized One, the perfected one, the fully awakened Buddha: the thought of sanctuary, and that of seclusion. The Realized One loves kindness and delights in it, so this thought often occurs to him: ‘Through this behavior, I shall not hurt any creature firm or frail.’ 

The\marginnote{3.1} Realized One loves seclusion and delights in it, so this thought often occurs to him: ‘What is unskillful has been given up.’ 

So,\marginnote{4.1} mendicants, you too should love kindness and delight in it, then this thought will often occur to you: ‘Through this behavior, I shall not hurt any creature firm or frail.’ 

You\marginnote{5.1} too should love seclusion and delight in it, then this thought will often occur to you: ‘What is unskillful? What is not given up? What should I give up?’” 

The\marginnote{5.4} Buddha spoke this matter. On this it is said: 

\begin{verse}%
“Two\marginnote{6.1} thoughts occur to him, \\
the Realized One, the bearer of the unbearable: \\
first mentioned was thought of sanctuary, \\
then the second made clear was seclusion. 

Dispeller\marginnote{7.1} of darkness, the great hermit has crossed over: \\
the attained, the master, the undefiled. \\
In the midst of it all, he is freed in the ending of craving; \\
that sage bears his final body. \\
He has disposed of \textsanskrit{Māra}, I declare, and gone beyond old age. 

Standing\marginnote{8.1} high on a rocky mountain, \\
you can see the people all around. \\
In just the same way, the all-seer, wise one, \\
having ascended the Temple of Truth, \\
rid of sorrow, looks upon the people \\
swamped with sorrow, oppressed by rebirth and old age.” 

%
\end{verse}

This\marginnote{9.1} too is a matter that was spoken by the Blessed One: that is what I heard. 

%
\section*{{\suttatitleacronym Iti 39}{\suttatitletranslation Teaching }{\suttatitleroot Desanāsutta}}
\addcontentsline{toc}{section}{\tocacronym{Iti 39} \toctranslation{Teaching } \tocroot{Desanāsutta}}
\markboth{Teaching }{Desanāsutta}
\extramarks{Iti 39}{Iti 39}

This\marginnote{1.1} was said by the Buddha, the Perfected One: that is what I heard. 

“Mendicants,\marginnote{2.1} the Realized One, the perfected one, the fully awakened Buddha has two approaches to teaching Dhamma. What two? ‘See evil as evil’—this is the first approach to teaching Dhamma. ‘Having seen evil as evil, be disillusioned, dispassionate, and freed from it’—this is the second approach to teaching Dhamma. The Realized One, the perfected one, the fully awakened Buddha has these two approaches to teaching Dhamma.” 

The\marginnote{2.8} Buddha spoke this matter. On this it is said: 

\begin{verse}%
“See\marginnote{3.1} the two approaches for \\
explaining the Dhamma \\
used by the Realized One, the Buddha, \\
compassionate for all beings: 

see\marginnote{4.1} that that is evil, \\
and be dispassionate towards it. \\
Then, with a mind free of desire, \\
you will make an end of suffering.” 

%
\end{verse}

This\marginnote{5.1} too is a matter that was spoken by the Blessed One: that is what I heard. 

%
\section*{{\suttatitleacronym Iti 40}{\suttatitletranslation Knowledge }{\suttatitleroot Vijjāsutta}}
\addcontentsline{toc}{section}{\tocacronym{Iti 40} \toctranslation{Knowledge } \tocroot{Vijjāsutta}}
\markboth{Knowledge }{Vijjāsutta}
\extramarks{Iti 40}{Iti 40}

This\marginnote{1.1} was said by the Buddha, the Perfected One: that is what I heard. 

“Mendicants,\marginnote{2.1} ignorance precedes the attainment of unskillful qualities, with lack of conscience and prudence following along. Knowledge precedes the attainment of skillful qualities, with conscience and prudence following along.” 

The\marginnote{2.3} Buddha spoke this matter. On this it is said: 

\begin{verse}%
“Bad\marginnote{3.1} destinies of whatever kind, \\
in this world or the next, \\
are all rooted in ignorance, \\
compounded of greed and desire. 

Since\marginnote{4.1} one of wicked desires is \\
unconscientious, lacking regard for others, \\
they make much bad karma, \\
which sends them to a place of loss. 

Therefore,\marginnote{5.1} dispelling desire \\
and greed and ignorance, \\
a mendicant arousing knowledge \\
would cast off all bad destinies.” 

%
\end{verse}

This\marginnote{6.1} too is a matter that was spoken by the Blessed One: that is what I heard. 

%
\section*{{\suttatitleacronym Iti 41}{\suttatitletranslation Bereft of Wisdom }{\suttatitleroot Paññāparihīnasutta}}
\addcontentsline{toc}{section}{\tocacronym{Iti 41} \toctranslation{Bereft of Wisdom } \tocroot{Paññāparihīnasutta}}
\markboth{Bereft of Wisdom }{Paññāparihīnasutta}
\extramarks{Iti 41}{Iti 41}

This\marginnote{1.1} was said by the Buddha, the Perfected One: that is what I heard. 

“Those\marginnote{2.1} sentient beings are truly bereft, mendicants, who are bereft of noble wisdom. They live unhappily in the present life—with distress, anguish, and fever—and when the body breaks up, after death, they can expect a bad rebirth. Those sentient beings are not bereft who are not bereft of noble wisdom. In the present life they’re happy—free of anguish, distress, and fever—and when the body breaks up, after death, they can expect a good rebirth.” 

The\marginnote{2.7} Buddha spoke this matter. On this it is said: 

\begin{verse}%
“See\marginnote{3.1} the world with its gods, \\
bereft of wisdom, \\
habituated to name and form, \\
imagining this is truth. 

Wisdom\marginnote{4.1} is best in the world, \\
as it leads to penetration, \\
through which one rightly understands \\
the ending of rebirth and continued existence. 

Gods\marginnote{5.1} and humans envy them, \\
the Buddhas, ever mindful, \\
of laughing wisdom, \\
bearing their final body.” 

%
\end{verse}

This\marginnote{6.1} too is a matter that was spoken by the Blessed One: that is what I heard. 

%
\section*{{\suttatitleacronym Iti 42}{\suttatitletranslation Bright Things }{\suttatitleroot Sukkadhammasutta}}
\addcontentsline{toc}{section}{\tocacronym{Iti 42} \toctranslation{Bright Things } \tocroot{Sukkadhammasutta}}
\markboth{Bright Things }{Sukkadhammasutta}
\extramarks{Iti 42}{Iti 42}

This\marginnote{1.1} was said by the Buddha, the Perfected One: that is what I heard. 

“These\marginnote{2.1} two bright things, mendicants, protect the world. What two? Conscience and prudence. If these two bright things did not protect the world, there would be no recognition of the status of mother, aunts, or wives and partners of teachers and respected people. The world would become promiscuous, like goats and sheep, chickens and pigs, and dogs and jackals. But because the two bright things protect the world, there is recognition of the status of mother, aunts, and wives and partners of teachers and respected people.” 

The\marginnote{2.7} Buddha spoke this matter. On this it is said: 

\begin{verse}%
“Those\marginnote{3.1} in whom conscience and shame \\
are never found at all, \\
have lost their bright roots, \\
and fare on in birth and death. 

Those\marginnote{4.1} in whom conscience and shame \\
are always rightly established, \\
thrive in the spiritual life; \\
being at peace, they will not be reborn again.” 

%
\end{verse}

This\marginnote{5.1} too is a matter that was spoken by the Blessed One: that is what I heard. 

%
\section*{{\suttatitleacronym Iti 43}{\suttatitletranslation Unborn }{\suttatitleroot Ajātasutta}}
\addcontentsline{toc}{section}{\tocacronym{Iti 43} \toctranslation{Unborn } \tocroot{Ajātasutta}}
\markboth{Unborn }{Ajātasutta}
\extramarks{Iti 43}{Iti 43}

This\marginnote{1.1} was said by the Buddha, the Perfected One: that is what I heard. 

“There\marginnote{2.1} is, mendicants, an unborn, unproduced, unmade, and unconditioned. If there were no unborn, unproduced, unmade, and unconditioned, then you would find no escape here from the born, produced, made, and conditioned. But since there is an unborn, unproduced, unmade, and unconditioned, an escape is found from the born, produced, made, and conditioned.” 

The\marginnote{2.4} Buddha spoke this matter. On this it is said: 

\begin{verse}%
“What’s\marginnote{3.1} born, produced, and arisen, \\
made, conditioned, not lasting, \\
wrapped in old age and death, \\
frail, a nest of disease, 

generated\marginnote{4.1} by food and the conduit to rebirth: \\
that’s not fit to delight in. \\
The escape from that is peaceful, \\
beyond the scope of logic, everlasting, 

unborn\marginnote{5.1} and unarisen, \\
the sorrowless, stainless state, \\
the cessation of all painful things, \\
the stilling of conditions, bliss.” 

%
\end{verse}

This\marginnote{6.1} too is a matter that was spoken by the Blessed One: that is what I heard. 

%
\section*{{\suttatitleacronym Iti 44}{\suttatitletranslation Elements of Extinguishment }{\suttatitleroot Nibbānadhātusutta}}
\addcontentsline{toc}{section}{\tocacronym{Iti 44} \toctranslation{Elements of Extinguishment } \tocroot{Nibbānadhātusutta}}
\markboth{Elements of Extinguishment }{Nibbānadhātusutta}
\extramarks{Iti 44}{Iti 44}

This\marginnote{1.1} was said by the Buddha, the Perfected One: that is what I heard. 

“There\marginnote{2.1} are, mendicants, these two elements of extinguishment. What two? The element of extinguishment with something left over, and the element of extinguishment with nothing left over. 

And\marginnote{3.1} what is the element of extinguishment with something left over? It’s when a mendicant is a perfected one, with defilements ended, who has completed the spiritual journey, done what had to be done, laid down the burden, achieved their own true goal, utterly ended the fetters of rebirth, and is rightly freed through enlightenment. Their five sense faculties still remain. So long as their senses have not gone they continue to experience the agreeable and disagreeable, to feel pleasure and pain. The ending of greed, hate, and delusion in them is called the element of extinguishment with something left over. 

And\marginnote{4.1} what is the element of extinguishment with nothing left over? It’s when a mendicant is a perfected one, with defilements ended, who has completed the spiritual journey, done what had to be done, laid down the burden, achieved their own true goal, utterly ended the fetters of rebirth, and is rightly freed through enlightenment. For them, everything that’s felt, being no longer relished, will become cool right here. This is called the element of extinguishment with nothing left over. These are the two elements of extinguishment.” 

The\marginnote{4.6} Buddha spoke this matter. On this it is said: 

\begin{verse}%
“These\marginnote{5.1} two elements of extinguishment have been made clear \\
by the seer, the unattached, the poised. \\
One element pertains to the present life—\\
what is left over when the conduit to rebirth has ended. \\
What has nothing left over pertains to what follows this life, \\
where all states of existence cease. 

Those\marginnote{6.1} who have fully understood the unconditioned state—\\
their minds freed, the conduit to rebirth ended—\\
attained to the heart of the Dhamma, they delight in ending, \\
the poised ones have given up all states of existence.” 

%
\end{verse}

%
\section*{{\suttatitleacronym Iti 45}{\suttatitletranslation Retreat }{\suttatitleroot Paṭisallānasutta}}
\addcontentsline{toc}{section}{\tocacronym{Iti 45} \toctranslation{Retreat } \tocroot{Paṭisallānasutta}}
\markboth{Retreat }{Paṭisallānasutta}
\extramarks{Iti 45}{Iti 45}

This\marginnote{1.1} was said by the Buddha, the Perfected One: that is what I heard. 

“Enjoy\marginnote{2.1} retreat, mendicants, love retreat. Be committed to inner serenity of the heart, don’t neglect absorption, be endowed with discernment, and frequent empty huts. A mendicant who enjoys retreat can expect one of two results: enlightenment in the present life, or if there’s something left over, non-return.” 

The\marginnote{2.4} Buddha spoke this matter. On this it is said: 

\begin{verse}%
“With\marginnote{3.1} minds at peace, alert, \\
mindful and meditative, \\
they rightly discern the Dhamma, \\
unconcerned for sensual pleasures. 

Delighting\marginnote{4.1} in diligence, peaceful, \\
seeing fear in negligence, \\
such a one can’t decline, \\
and has drawn near to quenching.” 

%
\end{verse}

This\marginnote{5.1} too is a matter that was spoken by the Blessed One: that is what I heard. 

%
\section*{{\suttatitleacronym Iti 46}{\suttatitletranslation The Benefits of Training }{\suttatitleroot Sikkhānisaṁsasutta}}
\addcontentsline{toc}{section}{\tocacronym{Iti 46} \toctranslation{The Benefits of Training } \tocroot{Sikkhānisaṁsasutta}}
\markboth{The Benefits of Training }{Sikkhānisaṁsasutta}
\extramarks{Iti 46}{Iti 46}

This\marginnote{1.1} was said by the Buddha, the Perfected One: that is what I heard. 

“Mendicants,\marginnote{2.1} live with training as benefit, with wisdom as overseer, with freedom as core, and with mindfulness as ruler. A mendicant who lives in this way can expect one of two results: enlightenment in the present life, or if there’s something left over, non-return.” 

The\marginnote{2.4} Buddha spoke this matter. On this it is said: 

\begin{verse}%
“The\marginnote{3.1} training fulfilled, not liable to decline, \\
overseen by wisdom, seer of rebirth’s end; \\
that sage bears their final body; \\
they have disposed of \textsanskrit{Māra}, I declare, and gone beyond old age. 

Therefore\marginnote{4.1} be always enjoying absorption, immersed in \textsanskrit{samādhi}, \\
energetic, seers of rebirth’s end. \\
Having overcome \textsanskrit{Māra} and his armies, mendicants, \\
go beyond birth and death.” 

%
\end{verse}

This\marginnote{5.1} too is a matter that was spoken by the Blessed One: that is what I heard. 

%
\section*{{\suttatitleacronym Iti 47}{\suttatitletranslation Wake Up }{\suttatitleroot Jāgariyasutta}}
\addcontentsline{toc}{section}{\tocacronym{Iti 47} \toctranslation{Wake Up } \tocroot{Jāgariyasutta}}
\markboth{Wake Up }{Jāgariyasutta}
\extramarks{Iti 47}{Iti 47}

This\marginnote{1.1} was said by the Buddha, the Perfected One: that is what I heard. 

“Meditate\marginnote{2.1} wakeful, mendicants, mindful and aware, joyful and clear, and at times discern the skillful qualities in that state. A mendicant who meditates in this way can expect one of two results: enlightenment in the present life, or if there’s something left over, non-return.” 

The\marginnote{2.4} Buddha spoke this matter. On this it is said: 

\begin{verse}%
“Listen\marginnote{3.1} up, wakeful ones! \\
And those asleep, wake up! \\
Wakefulness is better than sleep, \\
the wakeful have nothing to fear. 

Those\marginnote{4.1} who are wakeful, mindful and aware, \\
immersed in \textsanskrit{samādhi}, joyful and clear, \\
rightly investigating the Dhamma in good time, \\
unified, would banish the darkness. 

That’s\marginnote{5.1} why you should apply yourself to wakefulness. \\
A keen and alert mendicant, possessing absorption, \\
having cut the fetter of birth and old age, \\
would touch the highest awakening right here.” 

%
\end{verse}

This\marginnote{6.1} too is a matter that was spoken by the Blessed One: that is what I heard. 

%
\section*{{\suttatitleacronym Iti 48}{\suttatitletranslation Bound for Loss }{\suttatitleroot Āpāyikasutta}}
\addcontentsline{toc}{section}{\tocacronym{Iti 48} \toctranslation{Bound for Loss } \tocroot{Āpāyikasutta}}
\markboth{Bound for Loss }{Āpāyikasutta}
\extramarks{Iti 48}{Iti 48}

This\marginnote{1.1} was said by the Buddha, the Perfected One: that is what I heard. 

“Mendicants,\marginnote{2.1} two kinds of people are bound for a place of loss, bound for hell, if they don’t give up this fault. What two? Someone who is unchaste, but claims to be celibate; and someone who makes a groundless accusation of unchastity against a person whose celibacy is pure. These are the two kinds of people bound for a place of loss, bound for hell, if they don’t give up this fault.” 

The\marginnote{2.5} Buddha spoke this matter. On this it is said: 

\begin{verse}%
“A\marginnote{3.1} liar goes to hell, \\
as does one who denies what they did. \\
Both are equal in the hereafter, \\
those men of base deeds. 

Many\marginnote{4.1} who wrap their necks in ocher robes \\
are unrestrained and wicked. \\
Being wicked, they are reborn in hell \\
due to their bad deeds. 

It’d\marginnote{5.1} be better for the immoral and unrestrained \\
to eat an iron ball, \\
scorching, like a burning flame, \\
than to eat the nation’s alms.” 

%
\end{verse}

This\marginnote{6.1} too is a matter that was spoken by the Blessed One: that is what I heard. 

%
\section*{{\suttatitleacronym Iti 49}{\suttatitletranslation Misconceptions }{\suttatitleroot Diṭṭhigatasutta}}
\addcontentsline{toc}{section}{\tocacronym{Iti 49} \toctranslation{Misconceptions } \tocroot{Diṭṭhigatasutta}}
\markboth{Misconceptions }{Diṭṭhigatasutta}
\extramarks{Iti 49}{Iti 49}

This\marginnote{1.1} was said by the Buddha, the Perfected One: that is what I heard. 

“Overcome\marginnote{2.1} by two misconceptions, mendicants, some gods and humans get stuck, some overreach, while those with vision see. 

And\marginnote{3.1} how do some get stuck? Because of love, delight, and enjoyment of existence, when the Dhamma is being taught for the cessation of existence, the minds of some gods and humans are not eager, confident, settled, and decided. That is how some get stuck. 

And\marginnote{4.1} how do some overreach? Some, becoming horrified, repelled, and disgusted with existence, delight in ending existence: ‘When this self is annihilated and destroyed when the body breaks up, and doesn’t exist after death: that is peaceful, that is sublime, that is reality.’ That is how some overreach. 

And\marginnote{5.1} how do those with vision see? It’s when a mendicant sees what has come to be as having come to be. Seeing this, they are practicing for disillusionment, dispassion, and cessation regarding what has come to be. That is how those with vision see.” 

The\marginnote{5.5} Buddha spoke this matter. On this it is said: 

\begin{verse}%
“Those\marginnote{6.1} who see what has come to be as having come to be, \\
transcending what has come to be, \\
are freed in accord with the truth, \\
with the ending of craving for continued existence. 

They\marginnote{7.1} completely understand what has come to be, \\
rid of craving for rebirth in this or that state, \\
with the disappearance of what has come to be, \\
a mendicant does not come back to future lives.” 

%
\end{verse}

This\marginnote{8.1} too is a matter that was spoken by the Blessed One: that is what I heard. 

%
\addtocontents{toc}{\let\protect\contentsline\protect\nopagecontentsline}
\part*{The Book of the Threes }
\addcontentsline{toc}{part}{The Book of the Threes }
\markboth{}{}
\addtocontents{toc}{\let\protect\contentsline\protect\oldcontentsline}

%
\addtocontents{toc}{\let\protect\contentsline\protect\nopagecontentsline}
\chapter*{Chapter One }
\addcontentsline{toc}{chapter}{\tocchapterline{Chapter One }}
\addtocontents{toc}{\let\protect\contentsline\protect\oldcontentsline}

%
\section*{{\suttatitleacronym Iti 50}{\suttatitletranslation Roots }{\suttatitleroot Mūlasutta}}
\addcontentsline{toc}{section}{\tocacronym{Iti 50} \toctranslation{Roots } \tocroot{Mūlasutta}}
\markboth{Roots }{Mūlasutta}
\extramarks{Iti 50}{Iti 50}

This\marginnote{1.1} was said by the Buddha, the Perfected One: that is what I heard. 

“Mendicants,\marginnote{2.1} there are these three unskillful roots. What three? Greed, hate, and delusion. These are the three unskillful roots. ” 

The\marginnote{2.5} Buddha spoke this matter. On this it is said: 

\begin{verse}%
“When\marginnote{3.1} greed, hate, and delusion, \\
have arisen inside oneself, \\
they harm a person of wicked heart, \\
as a reed is destroyed by its own fruit.” 

%
\end{verse}

This\marginnote{4.1} too is a matter that was spoken by the Blessed One: that is what I heard. 

%
\section*{{\suttatitleacronym Iti 51}{\suttatitletranslation Elements }{\suttatitleroot Dhātusutta}}
\addcontentsline{toc}{section}{\tocacronym{Iti 51} \toctranslation{Elements } \tocroot{Dhātusutta}}
\markboth{Elements }{Dhātusutta}
\extramarks{Iti 51}{Iti 51}

This\marginnote{1.1} was said by the Buddha, the Perfected One: that is what I heard. 

“Mendicants,\marginnote{2.1} there are these three elements. What three? The elements of form, formlessness, and cessation. These are the three elements.” 

The\marginnote{2.5} Buddha spoke this matter. On this it is said: 

\begin{verse}%
“Comprehending\marginnote{3.1} the element of form, \\
not stuck in the formless, \\
those who are released in cessation, \\
they are conquerors of death. 

Having\marginnote{4.1} directly experienced the deathless element, \\
free of attachments; \\
having realised relinquishment \\
of attachments, the undefiled \\
fully awakened Buddha teaches \\
the sorrowless, stainless state.” 

%
\end{verse}

This\marginnote{5.1} too is a matter that was spoken by the Blessed One: that is what I heard. 

%
\section*{{\suttatitleacronym Iti 52}{\suttatitletranslation Feelings (1st) }{\suttatitleroot Paṭhamavedanāsutta}}
\addcontentsline{toc}{section}{\tocacronym{Iti 52} \toctranslation{Feelings (1st) } \tocroot{Paṭhamavedanāsutta}}
\markboth{Feelings (1st) }{Paṭhamavedanāsutta}
\extramarks{Iti 52}{Iti 52}

This\marginnote{1.1} was said by the Buddha, the Perfected One: that is what I heard. 

“Mendicants,\marginnote{2.1} there are these three feelings. What three? Pleasant, painful, and neutral feeling. These are the three feelings.” 

The\marginnote{2.5} Buddha spoke this matter. On this it is said: 

\begin{verse}%
“Stilled,\marginnote{3.1} aware, \\
a mindful disciple of the Buddha \\
understands feelings, \\
the cause of feelings, 

where\marginnote{4.1} they cease, \\
and the path that leads to their ending. \\
With the ending of feelings, a mendicant \\
is hungerless, extinguished.” 

%
\end{verse}

This\marginnote{5.1} too is a matter that was spoken by the Blessed One: that is what I heard. 

%
\section*{{\suttatitleacronym Iti 53}{\suttatitletranslation Feelings (2nd) }{\suttatitleroot Dutiyavedanāsutta}}
\addcontentsline{toc}{section}{\tocacronym{Iti 53} \toctranslation{Feelings (2nd) } \tocroot{Dutiyavedanāsutta}}
\markboth{Feelings (2nd) }{Dutiyavedanāsutta}
\extramarks{Iti 53}{Iti 53}

This\marginnote{1.1} was said by the Buddha, the Perfected One: that is what I heard. 

“Mendicants,\marginnote{2.1} there are these three feelings. What three? Pleasant, painful, and neutral feeling. Pleasant feeling should be seen as suffering. Painful feeling should be seen as a dart. Neutral feeling should be seen as impermanent. When a mendicant has seen these three feelings in this way, they’re called a mendicant who has cut off craving, untied the fetters, and by rightly comprehending conceit has made an end of suffering.” 

The\marginnote{2.9} Buddha spoke this matter. On this it is said: 

\begin{verse}%
“A\marginnote{3.1} mendicant who sees pleasure as pain, \\
and suffering as a dart, \\
and that peaceful, neutral feeling \\
as impermanent 

sees\marginnote{4.1} rightly. \\
And when freed in regards to that, \\
that peaceful sage, with perfect insight, \\
has truly escaped their bonds.” 

%
\end{verse}

This\marginnote{5.1} too is a matter that was spoken by the Blessed One: that is what I heard. 

%
\section*{{\suttatitleacronym Iti 54}{\suttatitletranslation Searches (1st) }{\suttatitleroot Paṭhamaesanāsutta}}
\addcontentsline{toc}{section}{\tocacronym{Iti 54} \toctranslation{Searches (1st) } \tocroot{Paṭhamaesanāsutta}}
\markboth{Searches (1st) }{Paṭhamaesanāsutta}
\extramarks{Iti 54}{Iti 54}

This\marginnote{1.1} was said by the Buddha, the Perfected One: that is what I heard. 

“Mendicants,\marginnote{2.1} there are these three searches. What three? The search for sensual pleasures, the search for continued existence, and the search for a spiritual path. These are the three searches. 

The\marginnote{2.5} Buddha spoke this matter. On this it is said: 

\begin{verse}%
“Stilled,\marginnote{3.1} aware, \\
a mindful disciple of the Buddha \\
understands searches, \\
the cause of searches, 

where\marginnote{4.1} they cease, \\
and the path that leads to their ending. \\
With the ending of searches, a mendicant \\
is hungerless, extinguished.” 

%
\end{verse}

This\marginnote{5.1} too is a matter that was spoken by the Blessed One: that is what I heard. 

%
\section*{{\suttatitleacronym Iti 55}{\suttatitletranslation Searches (2nd) }{\suttatitleroot Dutiyaesanāsutta}}
\addcontentsline{toc}{section}{\tocacronym{Iti 55} \toctranslation{Searches (2nd) } \tocroot{Dutiyaesanāsutta}}
\markboth{Searches (2nd) }{Dutiyaesanāsutta}
\extramarks{Iti 55}{Iti 55}

This\marginnote{1.1} was said by the Buddha, the Perfected One: that is what I heard. 

“Mendicants,\marginnote{2.1} there are these three searches. What three? The search for sensual pleasures, the search for continued existence, and the search for a spiritual path. These are the three searches.” 

The\marginnote{2.5} Buddha spoke this matter. On this it is said: 

\begin{verse}%
“The\marginnote{3.1} search for sensual pleasures, for a good rebirth, \\
and the search for a spiritual path; \\
the holding tight to the notion ‘this is the truth’, \\
and the mass of grounds for views—

for\marginnote{4.1} one detached from all lusts, \\
freed by the ending of craving, \\
that searching has been relinquished, \\
and those viewpoints eradicated. \\
With the ending of searches, a mendicant \\
is free of hope and indecision.” 

%
\end{verse}

This\marginnote{5.1} too is a matter that was spoken by the Blessed One: that is what I heard. 

%
\section*{{\suttatitleacronym Iti 56}{\suttatitletranslation Defilements (1st) }{\suttatitleroot Paṭhamaāsavasutta}}
\addcontentsline{toc}{section}{\tocacronym{Iti 56} \toctranslation{Defilements (1st) } \tocroot{Paṭhamaāsavasutta}}
\markboth{Defilements (1st) }{Paṭhamaāsavasutta}
\extramarks{Iti 56}{Iti 56}

This\marginnote{1.1} was said by the Buddha, the Perfected One: that is what I heard. 

“Mendicants,\marginnote{2.1} there are these three defilements. What three? The defilements of sensuality, desire to be reborn, and ignorance. These are the three defilements.” 

The\marginnote{2.5} Buddha spoke this matter. On this it is said: 

\begin{verse}%
“Stilled,\marginnote{3.1} aware, \\
a mindful disciple of the Buddha \\
understands defilements, \\
the cause of defilements, 

where\marginnote{4.1} they cease, \\
and the path that leads to their ending. \\
With the ending of defilements, a mendicant \\
is hungerless, extinguished.” 

%
\end{verse}

This\marginnote{5.1} too is a matter that was spoken by the Blessed One: that is what I heard. 

%
\section*{{\suttatitleacronym Iti 57}{\suttatitletranslation Defilements (2nd) }{\suttatitleroot Dutiyaāsavasutta}}
\addcontentsline{toc}{section}{\tocacronym{Iti 57} \toctranslation{Defilements (2nd) } \tocroot{Dutiyaāsavasutta}}
\markboth{Defilements (2nd) }{Dutiyaāsavasutta}
\extramarks{Iti 57}{Iti 57}

This\marginnote{1.1} was said by the Buddha, the Perfected One: that is what I heard. 

“Mendicants,\marginnote{2.1} there are these three defilements. What three? The defilements of sensuality, desire to be reborn, and ignorance. These are the three defilements.” 

The\marginnote{2.5} Buddha spoke this matter. On this it is said: 

\begin{verse}%
“One\marginnote{3.1} who has ended the defilement of sensuality, \\
whose ignorance has faded away, \\
and whose desire to be reborn is finished—\\
liberated, free of attachments, \\
they bear their final body, \\
having vanquished \textsanskrit{Māra} and his mount.” 

%
\end{verse}

This\marginnote{4.1} too is a matter that was spoken by the Blessed One: that is what I heard. 

%
\section*{{\suttatitleacronym Iti 58}{\suttatitletranslation Craving }{\suttatitleroot Taṇhāsutta}}
\addcontentsline{toc}{section}{\tocacronym{Iti 58} \toctranslation{Craving } \tocroot{Taṇhāsutta}}
\markboth{Craving }{Taṇhāsutta}
\extramarks{Iti 58}{Iti 58}

This\marginnote{1.1} was said by the Buddha, the Perfected One: that is what I heard. 

“Mendicants,\marginnote{2.1} there are these three cravings. What three? Craving for sensual pleasures, craving to continue existence, and craving to end existence. These are the three cravings.” 

The\marginnote{2.5} Buddha spoke this matter. On this it is said: 

\begin{verse}%
“Bound\marginnote{3.1} by craving, minds full of desire \\
for rebirth in this or that state, \\
yoked by \textsanskrit{Māra}’s yoke, these people \\
find no sanctuary from the yoke. \\
Sentient beings continue to transmigrate, \\
with ongoing birth and death. 

Those\marginnote{4.1} who have given up craving, \\
rid of craving for rebirth in this or that state, \\
they’re the ones in this world who’ve truly crossed over, \\
having reached the ending of defilements.” 

%
\end{verse}

This\marginnote{5.1} too is a matter that was spoken by the Blessed One: that is what I heard. 

%
\section*{{\suttatitleacronym Iti 59}{\suttatitletranslation Māra’s Sway }{\suttatitleroot Māradheyyasutta}}
\addcontentsline{toc}{section}{\tocacronym{Iti 59} \toctranslation{Māra’s Sway } \tocroot{Māradheyyasutta}}
\markboth{Māra’s Sway }{Māradheyyasutta}
\extramarks{Iti 59}{Iti 59}

This\marginnote{1.1} was said by the Buddha, the Perfected One: that is what I heard. 

“Mendicants,\marginnote{2.1} a mendicant with these three qualities has slipped free of \textsanskrit{Māra}’s sway and shines like the sun. What three? It’s when they have the entire spectrum of the master’s ethics, immersion, and wisdom. A mendicant with these three qualities has slipped free of \textsanskrit{Māra}’s sway and shines like the sun.” 

The\marginnote{2.5} Buddha spoke this matter. On this it is said: 

\begin{verse}%
“Whoever\marginnote{3.1} has well developed \\
ethics, immersion, and wisdom \\
has slipped free of \textsanskrit{Māra}’s sway \\
and shines like the sun.” 

%
\end{verse}

This\marginnote{4.1} too is a matter that was spoken by the Blessed One: that is what I heard. 

%
\addtocontents{toc}{\let\protect\contentsline\protect\nopagecontentsline}
\chapter*{Chapter Two }
\addcontentsline{toc}{chapter}{\tocchapterline{Chapter Two }}
\addtocontents{toc}{\let\protect\contentsline\protect\oldcontentsline}

%
\section*{{\suttatitleacronym Iti 60}{\suttatitletranslation Grounds for Making Merit }{\suttatitleroot Puññakiriyavatthusutta}}
\addcontentsline{toc}{section}{\tocacronym{Iti 60} \toctranslation{Grounds for Making Merit } \tocroot{Puññakiriyavatthusutta}}
\markboth{Grounds for Making Merit }{Puññakiriyavatthusutta}
\extramarks{Iti 60}{Iti 60}

This\marginnote{1.1} was said by the Buddha, the Perfected One: that is what I heard. 

“Mendicants,\marginnote{2.1} there are these three grounds for making merit. What three? Giving, ethical conduct, and meditation are all grounds for making merit. These are the three grounds for making merit.” 

The\marginnote{2.5} Buddha spoke this matter. On this it is said: 

\begin{verse}%
“One\marginnote{3.1} should practice only good deeds, \\
whose happy outcome stretches ahead. \\
Giving and moral conduct, \\
developing a mind of love: 

having\marginnote{4.1} developed these \\
three things yielding happiness, \\
that astute one is reborn \\
in a happy, pleasing world.” 

%
\end{verse}

This\marginnote{5.1} too is a matter that was spoken by the Blessed One: that is what I heard. 

%
\section*{{\suttatitleacronym Iti 61}{\suttatitletranslation The Eye }{\suttatitleroot Cakkhusutta}}
\addcontentsline{toc}{section}{\tocacronym{Iti 61} \toctranslation{The Eye } \tocroot{Cakkhusutta}}
\markboth{The Eye }{Cakkhusutta}
\extramarks{Iti 61}{Iti 61}

This\marginnote{1.1} was said by the Buddha, the Perfected One: that is what I heard. 

“Mendicants,\marginnote{2.1} there are these three eyes. What three? the eye of the flesh, the eye of clairvoyance, and the eye of wisdom. These are the three eyes.” 

The\marginnote{2.5} Buddha spoke this matter. On this it is said: 

\begin{verse}%
“The\marginnote{3.1} eye of flesh, the clairvoyant eye, \\
the supreme eye of wisdom: \\
these three eyes \\
were taught by the supreme person. 

The\marginnote{4.1} arising of the eye of flesh \\
is the path to the clairvoyant eye. \\
When knowledge arises—\\
the supreme eye of wisdom—\\
one who gains that eye \\
is released from all suffering.” 

%
\end{verse}

This\marginnote{5.1} too is a matter that was spoken by the Blessed One: that is what I heard. 

%
\section*{{\suttatitleacronym Iti 62}{\suttatitletranslation Faculties }{\suttatitleroot Indriyasutta}}
\addcontentsline{toc}{section}{\tocacronym{Iti 62} \toctranslation{Faculties } \tocroot{Indriyasutta}}
\markboth{Faculties }{Indriyasutta}
\extramarks{Iti 62}{Iti 62}

This\marginnote{1.1} was said by the Buddha, the Perfected One: that is what I heard. 

“Mendicants,\marginnote{2.1} there are these three faculties. What three? The faculty of understanding that one’s enlightenment is imminent. The faculty of enlightenment. The faculty of one who is enlightened. These are the three faculties.” 

The\marginnote{2.5} Buddha spoke this matter. On this it is said: 

\begin{verse}%
“As\marginnote{3.1} a trainee trains, \\
following the straight road, \\
first they know about ending; \\
enlightenment follows in the same lifetime. 

Then\marginnote{4.1} the knowledge comes \\
to such a one, freed through enlightenment, \\
with the end of the fetters of rebirth: \\
‘My freedom is unshakable.’ 

Accomplished\marginnote{5.1} in the faculties, \\
peaceful, in love with the state of peace, \\
they bear their final body, \\
having vanquished \textsanskrit{Māra} and his mount.” 

%
\end{verse}

This\marginnote{6.1} too is a matter that was spoken by the Blessed One: that is what I heard. 

%
\section*{{\suttatitleacronym Iti 63}{\suttatitletranslation Periods }{\suttatitleroot Addhāsutta}}
\addcontentsline{toc}{section}{\tocacronym{Iti 63} \toctranslation{Periods } \tocroot{Addhāsutta}}
\markboth{Periods }{Addhāsutta}
\extramarks{Iti 63}{Iti 63}

This\marginnote{1.1} was said by the Buddha, the Perfected One: that is what I heard. 

“Mendicants,\marginnote{2.1} there are these three periods. What three? Past, future, and present. These are the three periods.” 

The\marginnote{2.5} Buddha spoke this matter. On this it is said: 

\begin{verse}%
“Sentient\marginnote{3.1} beings who perceive the communicable, \\
become established in the communicable. \\
Not understanding the communicable, \\
they fall under the yoke of Death. 

But\marginnote{4.1} having fully understood the communicable, \\
they don’t identify as a communicator, \\
as they’ve touched liberation with their mind, \\
the supreme state of peace. 

Accomplished\marginnote{5.1} in the communicable, \\
peaceful, in love with the state of peace; \\
making use after reflection, firm in principle, \\
a knowledge master cannot be reckoned.” 

%
\end{verse}

This\marginnote{6.1} too is a matter that was spoken by the Blessed One: that is what I heard. 

%
\section*{{\suttatitleacronym Iti 64}{\suttatitletranslation Bad Conduct }{\suttatitleroot Duccaritasutta}}
\addcontentsline{toc}{section}{\tocacronym{Iti 64} \toctranslation{Bad Conduct } \tocroot{Duccaritasutta}}
\markboth{Bad Conduct }{Duccaritasutta}
\extramarks{Iti 64}{Iti 64}

This\marginnote{1.1} was said by the Buddha, the Perfected One: that is what I heard. 

“Mendicants,\marginnote{2.1} there are these three kinds of bad conduct. What three? Bad conduct by way of body, speech, and mind. These are the three kinds of bad conduct.” 

The\marginnote{2.5} Buddha spoke this matter. On this it is said: 

\begin{verse}%
“Having\marginnote{3.1} done bad things \\
by way of body, \\
speech, and mind, \\
and whatever else is corrupt; 

not\marginnote{4.1} having done good deeds, \\
and having done many bad, \\
when their body breaks up, that witless person \\
is reborn in hell.” 

%
\end{verse}

This\marginnote{5.1} too is a matter that was spoken by the Blessed One: that is what I heard. 

%
\section*{{\suttatitleacronym Iti 65}{\suttatitletranslation Good Conduct }{\suttatitleroot Sucaritasutta}}
\addcontentsline{toc}{section}{\tocacronym{Iti 65} \toctranslation{Good Conduct } \tocroot{Sucaritasutta}}
\markboth{Good Conduct }{Sucaritasutta}
\extramarks{Iti 65}{Iti 65}

This\marginnote{1.1} was said by the Buddha, the Perfected One: that is what I heard. 

“Mendicants,\marginnote{2.1} there are these kinds of good conduct. What three? Good conduct by way of body, speech, and mind. These are the three kinds of good conduct.” The Buddha spoke this matter. On this it is said: 

\begin{verse}%
“Having\marginnote{3.1} given up bad conduct \\
by way of body, \\
speech, and mind, \\
and whatever else is corrupt; 

not\marginnote{4.1} having done bad deeds, \\
and having done many good, \\
when their body breaks up, that wise person \\
is reborn in heaven.” 

%
\end{verse}

This\marginnote{5.1} too is a matter that was spoken by the Blessed One: that is what I heard. 

%
\section*{{\suttatitleacronym Iti 66}{\suttatitletranslation Purity }{\suttatitleroot Soceyyasutta}}
\addcontentsline{toc}{section}{\tocacronym{Iti 66} \toctranslation{Purity } \tocroot{Soceyyasutta}}
\markboth{Purity }{Soceyyasutta}
\extramarks{Iti 66}{Iti 66}

This\marginnote{1.1} was said by the Buddha, the Perfected One: that is what I heard. 

“Mendicants,\marginnote{2.1} there are these three kinds of purity. What three? Purity of body, speech, and mind. These are the three kinds of purity.” 

The\marginnote{2.5} Buddha spoke this matter. On this it is said: 

\begin{verse}%
“Purity\marginnote{3.1} of body, purity of speech, \\
and undefiled purity of heart. \\
A pure person, blessed with purity, \\
has given up everything, they say.” 

%
\end{verse}

This\marginnote{4.1} too is a matter that was spoken by the Blessed One: that is what I heard. 

%
\section*{{\suttatitleacronym Iti 67}{\suttatitletranslation Sagacity }{\suttatitleroot Moneyyasutta}}
\addcontentsline{toc}{section}{\tocacronym{Iti 67} \toctranslation{Sagacity } \tocroot{Moneyyasutta}}
\markboth{Sagacity }{Moneyyasutta}
\extramarks{Iti 67}{Iti 67}

This\marginnote{1.1} was said by the Buddha, the Perfected One: that is what I heard. 

“Mendicants,\marginnote{2.1} there are these three kinds of sagacity. What three? Sagacity of body, speech, and mind. These are the three kinds of sagacity.” The Buddha spoke this matter. On this it is said: 

\begin{verse}%
“A\marginnote{3.1} sage in body, a sage in speech, \\
a sage undefiled in mind; \\
a sage, blessed with sagacity, \\
has washed off all bad things, they say.” 

%
\end{verse}

This\marginnote{4.1} too is a matter that was spoken by the Blessed One: that is what I heard. 

%
\section*{{\suttatitleacronym Iti 68}{\suttatitletranslation Greed (1st) }{\suttatitleroot Paṭhamarāgasutta}}
\addcontentsline{toc}{section}{\tocacronym{Iti 68} \toctranslation{Greed (1st) } \tocroot{Paṭhamarāgasutta}}
\markboth{Greed (1st) }{Paṭhamarāgasutta}
\extramarks{Iti 68}{Iti 68}

This\marginnote{1.1} was said by the Buddha, the Perfected One: that is what I heard. 

“Whoever\marginnote{2.1} has not given up greed, hate, and delusion is said to be trapped. They’re caught in \textsanskrit{Māra}’s snare, and the Wicked One can do with them what he wants. Whoever has given up greed, hate, and delusion is said to be free. They’re caught in \textsanskrit{Māra}’s snare, and the Wicked One can do with them what he wants.” 

The\marginnote{2.5} Buddha spoke this matter. On this it is said: 

\begin{verse}%
“Those\marginnote{3.1} in whom greed, hate, and ignorance \\
have faded away, \\
belong with those who are evolved; \\
holy, realized, awakened, \\
beyond enmity and fear, \\
they’ve given up everything, they say.” 

%
\end{verse}

This\marginnote{4.1} too is a matter that was spoken by the Blessed One: that is what I heard. 

%
\section*{{\suttatitleacronym Iti 69}{\suttatitletranslation Greed (2nd) }{\suttatitleroot Dutiyarāgasutta}}
\addcontentsline{toc}{section}{\tocacronym{Iti 69} \toctranslation{Greed (2nd) } \tocroot{Dutiyarāgasutta}}
\markboth{Greed (2nd) }{Dutiyarāgasutta}
\extramarks{Iti 69}{Iti 69}

This\marginnote{1.1} was said by the Buddha, the Perfected One: that is what I heard. 

“Whoever\marginnote{2.1} has not given up greed, hate, and delusion is said to have not crossed over the ocean with its waves and whirlpools, its saltwater crocodiles and monsters. Whoever has given up greed, hate, and delusion is said to have crossed over the ocean with its waves and whirlpools, its saltwater crocodiles and monsters. Crossed over and gone beyond, that brahmin stands on the shore.” 

The\marginnote{2.5} Buddha spoke this matter. On this it is said: 

\begin{verse}%
“Those\marginnote{3.1} in whom greed, hate, and ignorance \\
have faded away, \\
have crossed the ocean so hard to cross, \\
with its saltwater crocodiles and monsters, its waves and dangers. 

They’ve\marginnote{4.1} escaped their chains, given up death, and have no attachments. \\
They’ve given up suffering, so there are no more future lives. \\
They’ve come to an end, and cannot be measured; \\
and they’ve confounded the King of Death, I say.” 

%
\end{verse}

This\marginnote{5.1} too is a matter that was spoken by the Blessed One: that is what I heard. 

%
\addtocontents{toc}{\let\protect\contentsline\protect\nopagecontentsline}
\chapter*{Chapter Three }
\addcontentsline{toc}{chapter}{\tocchapterline{Chapter Three }}
\addtocontents{toc}{\let\protect\contentsline\protect\oldcontentsline}

%
\section*{{\suttatitleacronym Iti 70}{\suttatitletranslation Having Wrong View }{\suttatitleroot Micchādiṭṭhikasutta}}
\addcontentsline{toc}{section}{\tocacronym{Iti 70} \toctranslation{Having Wrong View } \tocroot{Micchādiṭṭhikasutta}}
\markboth{Having Wrong View }{Micchādiṭṭhikasutta}
\extramarks{Iti 70}{Iti 70}

This\marginnote{1.1} was said by the Buddha, the Perfected One: that is what I heard. 

“Mendicants,\marginnote{2.1} I’ve seen beings who engaged in misconduct by body, speech, and mind, who abused the noble ones, who had wrong views and acted accordingly. At the breaking up of the body, after death, they were reborn in a lower realm, a bad destination, a world of misery, hell. 

Now,\marginnote{3.1} I don’t say this because I’ve heard it from some other ascetic or brahmin. I only say it because I’ve known, seen, and realized it for myself.” 

The\marginnote{4.3} Buddha spoke this matter. On this it is said: 

\begin{verse}%
“When\marginnote{5.1} the mind has been misdirected, \\
and words wrongly spoken, \\
and wrong bodily deeds have been done, \\
a person here 

unlearned,\marginnote{6.1} doer of bad deeds, \\
though their life may be short, \\
when their body breaks up, that witless person \\
is reborn in hell.” 

%
\end{verse}

This\marginnote{7.1} too is a matter that was spoken by the Blessed One: that is what I heard. 

%
\section*{{\suttatitleacronym Iti 71}{\suttatitletranslation Having Right View }{\suttatitleroot Sammādiṭṭhikasutta}}
\addcontentsline{toc}{section}{\tocacronym{Iti 71} \toctranslation{Having Right View } \tocroot{Sammādiṭṭhikasutta}}
\markboth{Having Right View }{Sammādiṭṭhikasutta}
\extramarks{Iti 71}{Iti 71}

This\marginnote{1.1} was said by the Buddha, the Perfected One: that is what I heard. 

“Mendicants,\marginnote{2.1} I’ve seen beings who engaged in good conduct of body, speech, and mind, who did not abuse the noble ones, who held right view and acted accordingly. At the breaking up of the body, after death, they were reborn in a good destination, a heavenly realm. 

Now,\marginnote{3.1} I don’t say this because I’ve heard it from some other ascetic or brahmin. I only say it because I’ve known, seen, and realized it for myself.” 

The\marginnote{4.3} Buddha spoke this matter. On this it is said: 

\begin{verse}%
“When\marginnote{5.1} the mind has been directed right, \\
and words rightly spoken, \\
and right bodily deeds have been done, \\
a person here 

learned,\marginnote{6.1} doer of good deeds, \\
though their life may be short, \\
when their body breaks up, that wise person \\
is reborn in heaven.” 

%
\end{verse}

This\marginnote{7.1} too is a matter that was spoken by the Blessed One: that is what I heard. 

%
\section*{{\suttatitleacronym Iti 72}{\suttatitletranslation Elements of Escape }{\suttatitleroot Nissaraṇiyasutta}}
\addcontentsline{toc}{section}{\tocacronym{Iti 72} \toctranslation{Elements of Escape } \tocroot{Nissaraṇiyasutta}}
\markboth{Elements of Escape }{Nissaraṇiyasutta}
\extramarks{Iti 72}{Iti 72}

This\marginnote{1.1} was said by the Buddha, the Perfected One: that is what I heard. 

“Mendicants,\marginnote{2.1} there are these three elements of escape. What three? Renunciation is the escape from sensual pleasures. Formlessness is the escape from form. Cessation is the escape from whatever is created, conditioned, and dependently originated. These are the three elements of escape.” 

The\marginnote{2.5} Buddha spoke this matter. On this it is said: 

\begin{verse}%
“Knowing\marginnote{3.1} the escape from sensuality, \\
and form’s transcendence, \\
one always keen touches \\
the stilling of all activities. 

That\marginnote{4.1} mendicant sees rightly, \\
and when freed in regards to that, \\
that peaceful sage, with perfect insight, \\
has truly escaped their bonds.” 

%
\end{verse}

This\marginnote{5.1} too is a matter that was spoken by the Blessed One: that is what I heard. 

%
\section*{{\suttatitleacronym Iti 73}{\suttatitletranslation More Peaceful }{\suttatitleroot Santatarasutta}}
\addcontentsline{toc}{section}{\tocacronym{Iti 73} \toctranslation{More Peaceful } \tocroot{Santatarasutta}}
\markboth{More Peaceful }{Santatarasutta}
\extramarks{Iti 73}{Iti 73}

This\marginnote{1.1} was said by the Buddha, the Perfected One: that is what I heard. 

“Mendicants,\marginnote{2.1} formless states are more peaceful than states of form; cessation is more peaceful than formless states.” 

The\marginnote{2.2} Buddha spoke this matter. On this it is said: 

\begin{verse}%
“There\marginnote{3.1} are beings in the realm of luminous form, \\
and others stuck in the formless. \\
Not understanding cessation, \\
they return in future lives. 

But\marginnote{4.1} the people who completely understand form, \\
not stuck in the formless, \\
released in cessation—\\
they are conquerors of death. 

Having\marginnote{5.1} directly experienced the deathless element, \\
free of attachments; \\
having realised relinquishment \\
of attachments, the undefiled \\
fully awakened Buddha teaches \\
the sorrowless, stainless state.” 

%
\end{verse}

This\marginnote{6.1} too is a matter that was spoken by the Blessed One: that is what I heard. 

%
\section*{{\suttatitleacronym Iti 74}{\suttatitletranslation A Child }{\suttatitleroot Puttasutta}}
\addcontentsline{toc}{section}{\tocacronym{Iti 74} \toctranslation{A Child } \tocroot{Puttasutta}}
\markboth{A Child }{Puttasutta}
\extramarks{Iti 74}{Iti 74}

This\marginnote{1.1} was said by the Buddha, the Perfected One: that is what I heard. 

“These\marginnote{2.1} three kinds of children are found in the world. What three? One who betters their birth, one who equals their birth, one who fails their birth. 

And\marginnote{3.1} how does a child better their birth? It’s when a child’s parents have not gone for refuge to the Buddha, the teaching, and the \textsanskrit{Saṅgha}. They kill living creatures, steal, commit sexual misconduct, lie, and use alcoholic drinks that cause negligence. They’re immoral, of bad character. But their child has gone for refuge to the Buddha, the teaching, and the \textsanskrit{Saṅgha}. They don’t kill living creatures, steal, commit sexual misconduct, lie, or take alcoholic drinks that cause negligence. They’re ethical, of good character. That’s how a child betters their birth. 

And\marginnote{4.1} how does a child equal their birth? It’s when a child’s parents have gone for refuge to the Buddha, the teaching, and the \textsanskrit{Saṅgha}. They don’t kill living creatures, steal, commit sexual misconduct, lie, or take alcoholic drinks that cause negligence. They’re ethical, of good character. And their child has gone for refuge to the Buddha, the teaching, and the \textsanskrit{Saṅgha}. They don’t kill living creatures, steal, commit sexual misconduct, lie, or take alcoholic drinks that cause negligence. They’re ethical, of good character. That’s how a child equals their birth. 

And\marginnote{5.1} how does a child fail their birth? It’s when a child’s parents have gone for refuge to the Buddha, the teaching, and the \textsanskrit{Saṅgha}. They don’t kill living creatures, steal, commit sexual misconduct, lie, or take alcoholic drinks that cause negligence. They’re ethical, of good character. But their child has not gone for refuge to the Buddha, the teaching, and the \textsanskrit{Saṅgha}. They kill living creatures, steal, commit sexual misconduct, lie, and use alcoholic drinks that cause negligence. They’re immoral, of bad character. That’s how a child fails their birth. These are the three kinds of children found in the world.” 

The\marginnote{5.10} Buddha spoke this matter. On this it is said: 

\begin{verse}%
“The\marginnote{6.1} astute wish for a child \\
who betters or equals their birth; \\
not one who fails their birth, \\
disgracing their family. 

These\marginnote{7.1} are the children in the world \\
who become lay devotees; \\
faithful, accomplished in ethics, \\
bountiful, rid of stinginess. \\
Like the moon freed from a cloud, \\
they shine in the assemblies.” 

%
\end{verse}

This\marginnote{8.1} too is a matter that was spoken by the Blessed One: that is what I heard. 

%
\section*{{\suttatitleacronym Iti 75}{\suttatitletranslation A Rainless Cloud }{\suttatitleroot Avuṭṭhikasutta}}
\addcontentsline{toc}{section}{\tocacronym{Iti 75} \toctranslation{A Rainless Cloud } \tocroot{Avuṭṭhikasutta}}
\markboth{A Rainless Cloud }{Avuṭṭhikasutta}
\extramarks{Iti 75}{Iti 75}

This\marginnote{1.1} was said by the Buddha, the Perfected One: that is what I heard. 

“Mendicants,\marginnote{2.1} these three people are found in the world. What three? One like a rainless cloud, one who rains locally, one who rains all over. 

And\marginnote{3.1} how is a person like a rainless cloud? It’s when some person doesn’t give to anyone at all—whether ascetics and brahmins, paupers, vagrants, travelers, or beggars—such things as food, drink, clothing, vehicles; garlands, perfumes, and makeup; and bed, house, and lighting. That’s how a person is like a rainless cloud. 

And\marginnote{4.1} how does a person rain locally? It’s when some person gives to some but not to others—whether ascetics and brahmins, paupers, vagrants, travelers, or beggars—such things as food, drink, clothing, vehicles; garlands, perfumes, and makeup; and bed, house, and lighting. That’s how a person rains locally. 

And\marginnote{5.1} how does a person rain all over? It’s when some person gives to everyone—whether ascetics and brahmins, paupers, vagrants, travelers, or beggars—such things as food, drink, clothing, vehicles; garlands, perfumes, and makeup; and bed, house, and lighting. That’s how a person rains all over. These are the three people found in the world.” 

The\marginnote{5.5} Buddha spoke this matter. On this it is said: 

\begin{verse}%
“They\marginnote{6.1} don’t share the food and drink \\
they have acquired \\
with ascetics or brahmins, \\
with paupers, vagrants, or travelers. \\
They’re like a rainless cloud, \\
they say, the meanest of men. 

They\marginnote{7.1} don’t give to some, \\
to some they provide. \\
They rain locally, \\
so say the wise. 

Compassionate\marginnote{8.1} for all beings, \\
that person distributes \\
abundant food upon request, \\
saying, “Give! Give!” 

The\marginnote{9.1} rain cloud rains forth, \\
having thundered and roared, \\
drenching the earth with water, \\
soaking the uplands and valleys. 

Even\marginnote{10.1} so, such a person, \\
having accumulated wealth \\
by legitimate means, \\
through their own hard work, \\
rightly satisfies with food and drink \\
those fallen to destitution.” 

%
\end{verse}

This\marginnote{11.1} too is a matter that was spoken by the Blessed One: that is what I heard. 

%
\section*{{\suttatitleacronym Iti 76}{\suttatitletranslation Wishing for Happiness }{\suttatitleroot Sukhapatthanāsutta}}
\addcontentsline{toc}{section}{\tocacronym{Iti 76} \toctranslation{Wishing for Happiness } \tocroot{Sukhapatthanāsutta}}
\markboth{Wishing for Happiness }{Sukhapatthanāsutta}
\extramarks{Iti 76}{Iti 76}

This\marginnote{1.1} was said by the Buddha, the Perfected One: that is what I heard. 

“Mendicants,\marginnote{2.1} an astute person who wishes for three kinds of happiness should take care of their ethics. What three? “May I be be praised!” “May I become rich!” “When my body breaks up, after death, may I be reborn in a good place, a heavenly realm!” An astute person who wishes for these three kinds of happiness should protect their precepts.” 

The\marginnote{2.5} Buddha spoke this matter. On this it is said: 

\begin{verse}%
Wishing\marginnote{3.1} for three kinds of happiness—\\
praise, prosperity, \\
and to delight in heaven after passing away—\\
the wise would take care of their ethics. 

Though\marginnote{4.1} you do no wrong, \\
if you associate with one who does, \\
you’re suspected of wrong, \\
and your disrepute grows. 

Whatever\marginnote{5.1} kind of friend you make, \\
with whom you associate, \\
that’s how you become, \\
for so it is when you share your life. 

The\marginnote{6.1} one who associates and the one associated with, \\
the one contacted and the one who contacts another, \\
are like an arrow smeared with poison \\
that contaminates the quiver. \\
A wise one, fearing contamination, \\
would never have wicked friends. 

A\marginnote{7.1} man who wraps \\
putrid fish in blades of grass \\
makes the grass stink—\\
so it is when associating with fools. 

But\marginnote{8.1} one who wraps \\
sandalwood incense in leaves \\
makes the leaves fragrant—\\
so it is when associating with the wise. 

So,\marginnote{9.1} knowing they’ll end up \\
like the wrapping, the astute \\
would shun the wicked, \\
and befriend the good. \\
The wicked lead you to hell, \\
the good help you to a good place.” 

%
\end{verse}

This\marginnote{10.1} too is a matter that was spoken by the Blessed One: that is what I heard. 

%
\section*{{\suttatitleacronym Iti 77}{\suttatitletranslation Fragile }{\suttatitleroot Bhidurasutta}}
\addcontentsline{toc}{section}{\tocacronym{Iti 77} \toctranslation{Fragile } \tocroot{Bhidurasutta}}
\markboth{Fragile }{Bhidurasutta}
\extramarks{Iti 77}{Iti 77}

This\marginnote{1.1} was said by the Buddha, the Perfected One: that is what I heard. 

“This\marginnote{2.1} body is fragile, mendicants, consciousness is liable to fade away, and all attachments are impermanent, suffering, and perishable.” 

The\marginnote{2.2} Buddha spoke this matter. On this it is said: 

\begin{verse}%
“Knowing\marginnote{3.1} that the body is fragile, \\
that consciousness fades away, \\
and seeing the danger in attachments, \\
they go beyond birth and death. \\
Having attained ultimate peace, \\
evolved, they bide their time.” 

%
\end{verse}

This\marginnote{4.1} too is a matter that was spoken by the Blessed One: that is what I heard. 

%
\section*{{\suttatitleacronym Iti 78}{\suttatitletranslation Converging Elements }{\suttatitleroot Dhātusosaṁsandanasutta}}
\addcontentsline{toc}{section}{\tocacronym{Iti 78} \toctranslation{Converging Elements } \tocroot{Dhātusosaṁsandanasutta}}
\markboth{Converging Elements }{Dhātusosaṁsandanasutta}
\extramarks{Iti 78}{Iti 78}

This\marginnote{1.1} was said by the Buddha, the Perfected One: that is what I heard. 

“Mendicants,\marginnote{2.1} sentient beings come together and converge because of an element: Those who have bad convictions come together and converge with those who have bad convictions. Those who have good convictions come together and converge with those who have good convictions. 

In\marginnote{3.1} the past … 

In\marginnote{4.1} the future … 

At\marginnote{5.1} present, too, sentient beings come together and converge because of an element. Those who have bad convictions come together and converge with those who have bad convictions. Those who have good convictions come together and converge with those who have good convictions.” 

The\marginnote{5.4} Buddha spoke this matter. On this it is said: 

\begin{verse}%
“Socializing\marginnote{6.1} promotes entanglements; \\
they’re cut off by being aloof. \\
If you’re lost in the middle of a great sea, \\
and you clamber up on a little log, you’ll sink. 

So\marginnote{7.1} too, a person who lives well \\
sinks by relying on a lazy person. \\
Hence you should avoid such \\
a lazy person who lacks energy. 

Dwell\marginnote{8.1} with the noble ones \\
who are secluded and determined \\
constantly energetic, \\
the astute who practice absorption.” 

%
\end{verse}

This\marginnote{9.1} too is a matter that was spoken by the Blessed One: that is what I heard. 

%
\section*{{\suttatitleacronym Iti 79}{\suttatitletranslation Decline }{\suttatitleroot Parihānasutta}}
\addcontentsline{toc}{section}{\tocacronym{Iti 79} \toctranslation{Decline } \tocroot{Parihānasutta}}
\markboth{Decline }{Parihānasutta}
\extramarks{Iti 79}{Iti 79}

This\marginnote{1.1} was said by the Buddha, the Perfected One: that is what I heard. 

“These\marginnote{2.1} three things lead to the decline of a mendicant trainee. What three? It’s when a mendicant relishes work, talk, and sleep. These three things lead to the decline of a mendicant trainee. 

These\marginnote{3.1} three things don’t lead to the decline of a mendicant trainee. What three? It’s when a mendicant doesn’t relish work, talk, and sleep. These three things don’t lead to the decline of a mendicant trainee.” 

The\marginnote{3.7} Buddha spoke this matter. On this it is said: 

\begin{verse}%
“Restless,\marginnote{4.1} they relish \\
work, talk, and sleep. \\
Such a mendicant is incapable \\
of touching the highest awakening. 

That’s\marginnote{5.1} why one ought have few duties, \\
being wakeful and stable. \\
Such a mendicant is capable \\
of touching the highest awakening.” 

%
\end{verse}

This\marginnote{6.1} too is a matter that was spoken by the Blessed One: that is what I heard. 

%
\addtocontents{toc}{\let\protect\contentsline\protect\nopagecontentsline}
\chapter*{Chapter Four }
\addcontentsline{toc}{chapter}{\tocchapterline{Chapter Four }}
\addtocontents{toc}{\let\protect\contentsline\protect\oldcontentsline}

%
\section*{{\suttatitleacronym Iti 80}{\suttatitletranslation Thoughts }{\suttatitleroot Vitakkasutta}}
\addcontentsline{toc}{section}{\tocacronym{Iti 80} \toctranslation{Thoughts } \tocroot{Vitakkasutta}}
\markboth{Thoughts }{Vitakkasutta}
\extramarks{Iti 80}{Iti 80}

This\marginnote{1.1} was said by the Buddha, the Perfected One: that is what I heard. 

“Mendicants,\marginnote{2.1} there are these three unskillful thoughts. What three? The thought of being looked up to; of getting material possessions, honor, and popularity; and of fondness for others. These are the three unskillful thoughts.” 

The\marginnote{2.5} Buddha spoke this matter. On this it is said: 

\begin{verse}%
“One\marginnote{3.1} concerned with being looked up to, \\
with possessions, honor, and respect, \\
with sharing joys with friends, \\
is far from the ending of fetters. 

But\marginnote{4.1} one who gives up children and herds, \\
marriage and acquisitions—\\
such a mendicant is capable \\
of touching the highest awakening.” 

%
\end{verse}

This\marginnote{5.1} too is a matter that was spoken by the Blessed One: that is what I heard. 

%
\section*{{\suttatitleacronym Iti 81}{\suttatitletranslation Esteem }{\suttatitleroot Sakkārasutta}}
\addcontentsline{toc}{section}{\tocacronym{Iti 81} \toctranslation{Esteem } \tocroot{Sakkārasutta}}
\markboth{Esteem }{Sakkārasutta}
\extramarks{Iti 81}{Iti 81}

This\marginnote{1.1} was said by the Buddha, the Perfected One: that is what I heard. 

“I’ve\marginnote{2.1} seen, mendicants, sentient beings whose minds are overcome and overwhelmed by honor. When their body breaks up, after death, they’re reborn in a place of loss, a bad place, the underworld, hell. 

I’ve\marginnote{3.1} seen sentient beings whose minds are overcome and overwhelmed by not being honored. When their body breaks up, after death, they’re reborn in a place of loss, a bad place, the underworld, hell. 

I’ve\marginnote{4.1} seen sentient beings whose minds are overcome and overwhelmed by both honor and by not being honored. When their body breaks up, after death, they’re reborn in a place of loss, a bad place, the underworld, hell. 

Now,\marginnote{5.1} I don’t say this because I’ve heard it from some other ascetic or brahmin. I only say it because I’ve known, seen, and realized it for myself.” 

The\marginnote{8.2} Buddha spoke this matter. On this it is said: 

\begin{verse}%
“Whether\marginnote{9.1} they’re honored \\
or not honored, or both, \\
their immersion doesn’t waver \\
as they live diligently. 

They\marginnote{10.1} persistently practice absorption \\
with subtle view and discernment. \\
Rejoicing in the ending of grasping, \\
they’re said to be a good person.” 

%
\end{verse}

This\marginnote{11.1} too is a matter that was spoken by the Blessed One: that is what I heard. 

%
\section*{{\suttatitleacronym Iti 82}{\suttatitletranslation The Cry of the Gods }{\suttatitleroot Devasaddasutta}}
\addcontentsline{toc}{section}{\tocacronym{Iti 82} \toctranslation{The Cry of the Gods } \tocroot{Devasaddasutta}}
\markboth{The Cry of the Gods }{Devasaddasutta}
\extramarks{Iti 82}{Iti 82}

This\marginnote{1.1} was said by the Buddha, the Perfected One: that is what I heard. 

“Mendicants,\marginnote{2.1} these three cries are uttered among the gods on occasion. What three? When a noble disciple shaves off their hair and beard, dresses in ocher robes, and goes forth from the lay life to homelessness, the gods cry out: ‘This noble disciple intends to join battle with \textsanskrit{Māra}!’ This is the first occasion a cry is uttered among the gods. 

Furthermore,\marginnote{3.1} when a noble disciple meditates pursuing the development of the seven qualities that lead to awakening, the gods cry out: ‘This noble disciple is joining battle with \textsanskrit{Māra}!’ This is the second occasion a cry is uttered among the gods. 

Furthermore,\marginnote{4.1} when a noble disciple realizes the undefiled freedom of heart and freedom by wisdom in this very life, and they live having realized it with their own insight due to the ending of defilements, the gods cry out: ‘This noble disciple has won victory in battle, establishing himself as foremost in battle!’ This is the third occasion a cry is uttered among the gods. These are the three cries that are uttered among the gods on occasion.” 

The\marginnote{4.5} Buddha spoke this matter. On this it is said: 

\begin{verse}%
“Seeing\marginnote{5.1} the winner of the battle—\\
a disciple of the Buddha, \\
a great one, rid of naivety—\\
even the deities revere them: 

‘Homage\marginnote{6.1} to you, O thoroughbred! \\
You won a battle hard to win! \\
Having defeated the army of death, \\
your liberation is unobstructed.’ 

And\marginnote{7.1} so the deities revere the one, \\
who has achieved their heart’s desire. \\
For they see nothing in them by means of which \\
they might fall under the sway of Death.” 

%
\end{verse}

This\marginnote{8.1} too is a matter that was spoken by the Blessed One: that is what I heard. 

%
\section*{{\suttatitleacronym Iti 83}{\suttatitletranslation Five Warning Signs }{\suttatitleroot Pañcapubbanimittasutta}}
\addcontentsline{toc}{section}{\tocacronym{Iti 83} \toctranslation{Five Warning Signs } \tocroot{Pañcapubbanimittasutta}}
\markboth{Five Warning Signs }{Pañcapubbanimittasutta}
\extramarks{Iti 83}{Iti 83}

This\marginnote{1.1} was said by the Buddha, the Perfected One: that is what I heard. 

“Mendicants,\marginnote{2.1} when a god is due to pass away from the realm of the gods, five warning signs appear. Their flower-garlands wither; their clothes become soiled; they sweat from the armpits; their physical appearance deteriorates; and they no longer delight in their heavenly throne. When the other gods know that that god is due to pass away, they wish them well in three ways: ‘Sir, may you go from here to a good place! 

When\marginnote{3.1} you have gone to a good place, may you be blessed with good fortune! 

When\marginnote{4.1} you have been blessed with good fortune, may you become well grounded!’ 

When\marginnote{5.1} he said this, one of the mendicants said to the Buddha, “Sir, what do the gods reckon to be going to a good place? 

What\marginnote{6.1} do they reckon to be blessed with good fortune? 

What\marginnote{7.1} do they reckon to become well grounded?” 

“It\marginnote{8.1} is human existence, mendicant, that the gods reckon to be going to a good place. 

When\marginnote{9.1} a human being gains faith in the teaching and training proclaimed by the Realized One, that is what the gods reckon to be blessed with good fortune. 

When\marginnote{10.1} that faith in the Realized One is settled, rooted, and planted deep; when it’s strong and can’t be shifted by any ascetic or brahmin or god or \textsanskrit{Māra} or \textsanskrit{Brahmā} or by anyone in the world, that is what the gods reckon to become well grounded.” 

The\marginnote{11.1} Buddha spoke this matter. On this it is said: 

\begin{verse}%
“When,\marginnote{12.1} with the fading of life, \\
a god passes from the realm of the gods, \\
the gods utter three cries \\
of well-wishing: 

‘Sir,\marginnote{13.1} go from here to a good place, \\
in the company of humans. \\
As a human being, gain supreme faith \\
in the true teaching. 

May\marginnote{14.1} that faith of yours be settled, \\
with roots planted deep, \\
unfaltering all life long \\
in the true teaching so well proclaimed. 

Having\marginnote{15.1} given up bad conduct \\
by way of body, \\
speech, and mind, \\
and whatever else is corrupt; 

and\marginnote{16.1} having done much good, \\
by way of body, \\
speech, and mind, \\
limitless, free of attachments; 

then,\marginnote{17.1} having made much worldly merit \\
by giving gifts, \\
establish other colleagues \\
in the true teaching, the spiritual life.’ 

It\marginnote{18.1} is due to such compassion \\
that when the gods know a god \\
is due to pass away, they wish them well: \\
‘Come back, god, again and again!’ 

%
\end{verse}

This\marginnote{19.1} too is a matter that was spoken by the Blessed One: that is what I heard. 

%
\section*{{\suttatitleacronym Iti 84}{\suttatitletranslation For the Welfare of the People }{\suttatitleroot Bahujanahitasutta}}
\addcontentsline{toc}{section}{\tocacronym{Iti 84} \toctranslation{For the Welfare of the People } \tocroot{Bahujanahitasutta}}
\markboth{For the Welfare of the People }{Bahujanahitasutta}
\extramarks{Iti 84}{Iti 84}

This\marginnote{1.1} was said by the Buddha, the Perfected One: that is what I heard. 

“Three\marginnote{2.1} people, mendicants, arise in the world for the welfare and happiness of the people, out of compassion for the world, for the benefit, welfare, and happiness of gods and humans. What three? It’s when a Realized One arises in the world, perfected, a fully awakened Buddha, accomplished in knowledge and conduct, holy, knower of the world, supreme guide for those who wish to train, teacher of gods and humans, awakened, blessed. He teaches Dhamma that’s good in the beginning, good in the middle, and good in the end, meaningful and well-phrased. And he reveals a spiritual practice that’s entirely full and pure. This is the first person who arises in the world for the welfare and happiness of the people, out of compassion for the world, for the benefit, welfare, and happiness of gods and humans. 

Furthermore,\marginnote{3.1} it’s when a mendicant is a perfected one, with defilements ended, who has completed the spiritual journey, done what had to be done, laid down the burden, achieved their own true goal, utterly ended the fetters of rebirth, and is rightly freed through enlightenment. They teach Dhamma that’s good in the beginning, good in the middle, and good in the end, meaningful and well-phrased. And they reveal a spiritual practice that’s entirely full and pure. This is the second person who arises in the world for the welfare and happiness of the people, out of compassion for the world, for the benefit, welfare, and happiness of gods and humans. 

Furthermore,\marginnote{4.1} it’s when a disciple of that Teacher is a trainee, a learned practitioner with precepts and observances intact. They teach Dhamma that’s good in the beginning, good in the middle, and good in the end, meaningful and well-phrased. And they reveal a spiritual practice that’s entirely full and pure. This is the third person who arises in the world for the welfare and happiness of the people, out of compassion for the world, for the benefit, welfare, and happiness of gods and humans. These are the three people who arise in the world for the welfare and happiness of the people, out of compassion for the world, for the benefit, welfare, and happiness of gods and humans.” 

The\marginnote{4.5} Buddha spoke this matter. On this it is said: 

\begin{verse}%
“The\marginnote{5.1} Teacher is the first, the great hermit, \\
following whom is the disciple of developed self, \\
and then a trainee, a practitioner, \\
learned, with precepts and observances intact. 

These\marginnote{6.1} three are first among gods and humans, \\
beacons proclaiming the teaching! \\
They fling open the door to the deathless, \\
freeing many from bondage. 

Following\marginnote{7.1} the path so well taught \\
by the unsurpassed caravan leader, \\
those who are diligent in the Holy One’s teaching \\
make an end of suffering in this very life.” 

%
\end{verse}

This\marginnote{8.1} too is a matter that was spoken by the Blessed One: that is what I heard. 

%
\section*{{\suttatitleacronym Iti 85}{\suttatitletranslation Observing Ugliness }{\suttatitleroot Asubhānupassīsutta}}
\addcontentsline{toc}{section}{\tocacronym{Iti 85} \toctranslation{Observing Ugliness } \tocroot{Asubhānupassīsutta}}
\markboth{Observing Ugliness }{Asubhānupassīsutta}
\extramarks{Iti 85}{Iti 85}

This\marginnote{1.1} was said by the Buddha, the Perfected One: that is what I heard. 

“Mendicants,\marginnote{2.1} meditate observing the ugliness of the body. Let mindfulness of breathing be well-established right there inside you. Meditate observing the impermanence of all conditions. As you meditate observing the ugliness of the body, you will give up desire for the body. When mindfulness of breathing is well-established right there inside you, there will be no distressing external thoughts or wishes. When you meditate observing the impermanence of all conditions, ignorance is given up and knowledge arises.” 

The\marginnote{2.7} Buddha spoke this matter. On this it is said: 

\begin{verse}%
“Observing\marginnote{3.1} the ugliness of the body, \\
mindful of the breath, \\
one always keen sees \\
the stilling of all activities. 

That\marginnote{4.1} mendicant sees rightly, \\
and when freed in regards to that, \\
that peaceful sage, with perfect insight, \\
has truly escaped their bonds.” 

%
\end{verse}

This\marginnote{5.1} too is a matter that was spoken by the Blessed One: that is what I heard. 

%
\section*{{\suttatitleacronym Iti 86}{\suttatitletranslation Practicing In Line With the Teaching }{\suttatitleroot Dhammānudhammapaṭipannasutta}}
\addcontentsline{toc}{section}{\tocacronym{Iti 86} \toctranslation{Practicing In Line With the Teaching } \tocroot{Dhammānudhammapaṭipannasutta}}
\markboth{Practicing In Line With the Teaching }{Dhammānudhammapaṭipannasutta}
\extramarks{Iti 86}{Iti 86}

This\marginnote{1.1} was said by the Buddha, the Perfected One: that is what I heard. 

“Regarding\marginnote{2.1} a mendicant practicing in line with the teaching, it is in line with the teaching to declare that this is what it means to practice in line with the teaching. When speaking, they speak in line with the teaching, not against it. When thinking, they think in line with the teaching, not against it. And rejecting both, they meditate staying equanimous, mindful and aware.” 

The\marginnote{2.3} Buddha spoke this matter. On this it is said: 

\begin{verse}%
“Delighting\marginnote{3.1} in the teaching, enjoying the teaching, \\
contemplating the teaching, \\
a mendicant who recollects the teaching \\
doesn’t decline in the true teaching. 

Whether\marginnote{4.1} walking or standing, \\
sitting or lying down, \\
with mind collected inside, \\
they attain only peace.” 

%
\end{verse}

This\marginnote{5.1} too is a matter that was spoken by the Blessed One: that is what I heard. 

%
\section*{{\suttatitleacronym Iti 87}{\suttatitletranslation Destroyers of Sight }{\suttatitleroot Andhakaraṇasutta}}
\addcontentsline{toc}{section}{\tocacronym{Iti 87} \toctranslation{Destroyers of Sight } \tocroot{Andhakaraṇasutta}}
\markboth{Destroyers of Sight }{Andhakaraṇasutta}
\extramarks{Iti 87}{Iti 87}

This\marginnote{1.1} was said by the Buddha, the Perfected One: that is what I heard. 

“Mendicants,\marginnote{2.1} these three unskillful thoughts are destroyers of sight, vision, and knowledge. They block wisdom, they’re on the side of anguish, and they don’t lead to extinguishment. What three? Thoughts of sensuality, malice, and cruelty. These are the three unskillful thoughts that are destroyers of sight, vision, and knowledge. They block wisdom, they’re on the side of anguish, and they don’t lead to extinguishment. 

These\marginnote{3.1} three skillful thoughts are creators of sight, vision, and knowledge. They grow wisdom, they’re on the side of solace, and they lead to extinguishment. What three? Thoughts of renunciation, good will, and harmlessness. These are the three skillful thoughts that are creators of sight, vision, and knowledge. They grow wisdom, they’re on the side of solace, and they lead to extinguishment.” 

The\marginnote{3.7} Buddha spoke this matter. On this it is said: 

\begin{verse}%
“Think\marginnote{4.1} the three skillful thoughts, \\
and get rid of the unskillful. \\
Quelling such thoughts and considerations, \\
like rain on the dust, \\
with a heart calmed of thought, \\
you’ll touch the state of peace right here.” 

%
\end{verse}

This\marginnote{5.1} too is a matter that was spoken by the Blessed One: that is what I heard. 

%
\section*{{\suttatitleacronym Iti 88}{\suttatitletranslation Inner Stains }{\suttatitleroot Antarāmalasutta}}
\addcontentsline{toc}{section}{\tocacronym{Iti 88} \toctranslation{Inner Stains } \tocroot{Antarāmalasutta}}
\markboth{Inner Stains }{Antarāmalasutta}
\extramarks{Iti 88}{Iti 88}

This\marginnote{1.1} was said by the Buddha, the Perfected One: that is what I heard. 

“Mendicants,\marginnote{2.1} there are these three inner stains, inner foes, inner enemies, inner killers, and inner adversaries. What three? Greed, hate, and delusion. These three are inner stains, inner foes, inner enemies, inner killers, and inner adversaries.” 

The\marginnote{2.7} Buddha spoke this matter. On this it is said: 

\begin{verse}%
“Greed\marginnote{3.1} creates harm; \\
greed upsets the mind. \\
That person doesn’t recognize \\
the danger that arises within. 

A\marginnote{4.1} greedy person doesn’t know the good. \\
A greedy person doesn’t see the truth. \\
When a person is beset by greed, \\
only blind darkness is left. 

Those\marginnote{5.1} who have given up greed, \\
don’t get greedy even when provoked. \\
Greed falls off them \\
like a drop from a lotus-leaf. 

Hate\marginnote{6.1} creates harm; \\
hate upsets the mind. \\
That person doesn’t recognize \\
the danger that arises within. 

A\marginnote{7.1} hateful person doesn’t know the good. \\
A hateful person doesn’t see the truth. \\
When a person is beset by hate, \\
only blind darkness is left. 

Those\marginnote{8.1} who have given up hate, \\
don’t get angry even when provoked. \\
Hate falls off them \\
like a palm-leaf from its stem. 

Delusion\marginnote{9.1} creates harm; \\
delusion upsets the mind. \\
That person doesn’t recognize \\
the danger that arises within. 

A\marginnote{10.1} deluded person doesn’t know the good. \\
A deluded person doesn’t see the truth. \\
When a person is beset by delusion, \\
only blind darkness is left. 

Those\marginnote{11.1} who have given up delusion, \\
don’t get deluded even when provoked. \\
They banish all delusion, \\
as the rising sun the dark.” 

%
\end{verse}

This\marginnote{12.1} too is a matter that was spoken by the Blessed One: that is what I heard. 

%
\section*{{\suttatitleacronym Iti 89}{\suttatitletranslation About Devadatta }{\suttatitleroot Devadattasutta}}
\addcontentsline{toc}{section}{\tocacronym{Iti 89} \toctranslation{About Devadatta } \tocroot{Devadattasutta}}
\markboth{About Devadatta }{Devadattasutta}
\extramarks{Iti 89}{Iti 89}

This\marginnote{1.1} was said by the Buddha, the Perfected One: that is what I heard. 

“Mendicants,\marginnote{2.1} overcome and overwhelmed by three things that oppose the true teaching, Devadatta is going to a place of loss, to hell, there to remain for an eon, irredeemable. What three? Wicked desires … Bad friendship … When there is still more to be done, stopping half-way after achieving some insignificant distinction. Overcome and overwhelmed by these three things that oppose the true teaching, Devadatta is going to a place of loss, to hell, there to remain for an eon, irredeemable.” 

The\marginnote{2.7} Buddha spoke this matter. On this it is said: 

\begin{verse}%
“Surely,\marginnote{3.1} none of wicked desire \\
are reborn into this world. \\
And by this too you should know \\
the place where those of wicked desires go. 

He\marginnote{4.1} once was considered astute, \\
regarded as evolved, \\
his glory stood forth like a flame, \\
the renowned Devadatta. 

Seduced\marginnote{5.1} by heedlessness, \\
he attacked the Realized One. \\
He has fallen to \textsanskrit{Avīci} hell, \\
four-doored and terrifying. 

When\marginnote{6.1} someone betrays the innocent, \\
who have done no wrong, \\
their bad deeds impact the one \\
with corrupt heart, lacking regard for others. 

One\marginnote{7.1} might think to pollute \\
the ocean with a pot of poison, \\
but it wouldn’t work, \\
for the sea is terribly large. 

So\marginnote{8.1} too when someone attacks \\
with words the Realized One—\\
consummate, of peaceful mind—\\
the words don’t take. 

The\marginnote{9.1} astute would befriend one like this, \\
and follow them around. \\
A mendicant who walks the path \\
attains the ending of suffering.” 

%
\end{verse}

This\marginnote{10.1} too is a matter that was spoken by the Blessed One: that is what I heard. 

%
\addtocontents{toc}{\let\protect\contentsline\protect\nopagecontentsline}
\chapter*{Chapter Five }
\addcontentsline{toc}{chapter}{\tocchapterline{Chapter Five }}
\addtocontents{toc}{\let\protect\contentsline\protect\oldcontentsline}

%
\section*{{\suttatitleacronym Iti 90}{\suttatitletranslation The Best Kinds of Confidence }{\suttatitleroot Aggappasādasutta}}
\addcontentsline{toc}{section}{\tocacronym{Iti 90} \toctranslation{The Best Kinds of Confidence } \tocroot{Aggappasādasutta}}
\markboth{The Best Kinds of Confidence }{Aggappasādasutta}
\extramarks{Iti 90}{Iti 90}

This\marginnote{1.1} was said by the Buddha, the Perfected One: that is what I heard. 

“Mendicants,\marginnote{2.1} these three kinds of confidence are the best. What three? Mendicants, the Realized One, the perfected one, the fully awakened Buddha, is said to be the best of all sentient beings—be they footless, with two feet, four feet, or many feet; with form or formless; with perception or without perception or with neither perception nor non-perception. Those who have confidence in the Buddha have confidence in the best. Having confidence in the best, the result is the best. 

Fading\marginnote{3.1} away is said to be the best of all things whether conditioned or unconditioned. That is, the quelling of vanity, the removing of thirst, the abolishing of clinging, the breaking of the round, the ending of craving, fading away, cessation, extinguishment. Those who have confidence in the teaching of fading away have confidence in the best. Having confidence in the best, the result is the best. 

The\marginnote{4.1} \textsanskrit{Saṅgha} of the Realized One’s disciples is said to be the best of all communities and groups. It consists of the four pairs, the eight individuals. This is the \textsanskrit{Saṅgha} of the Buddha’s disciples that is worthy of offerings dedicated to the gods, worthy of hospitality, worthy of a religious donation, worthy of greeting with joined palms, and is the supreme field of merit for the world. Those who have confidence in the \textsanskrit{Saṅgha} have confidence in the best. Having confidence in the best, the result is the best. These are the three best kinds of confidence.” 

The\marginnote{4.5} Buddha spoke this matter. On this it is said: 

\begin{verse}%
“For\marginnote{5.1} those who, knowing the best teaching, \\
base their confidence on the best—\\
confident in the best Awakened One, \\
supremely worthy of a religious donation; 

confident\marginnote{6.1} in the best teaching, \\
the bliss of fading and stilling; \\
confident in the best \textsanskrit{Saṅgha}, \\
the supreme field of merit—

giving\marginnote{7.1} gifts to the best, \\
the best of merit grows: \\
the best lifespan, beauty, \\
fame, reputation, happiness, and strength. 

An\marginnote{8.1} intelligent person gives to the best, \\
settled on the best teaching. \\
When they become a god or human, \\
they rejoice at reaching the best.” 

%
\end{verse}

This\marginnote{9.1} too is a matter that was spoken by the Blessed One: that is what I heard. 

%
\section*{{\suttatitleacronym Iti 91}{\suttatitletranslation Lifestyle }{\suttatitleroot Jīvikasutta}}
\addcontentsline{toc}{section}{\tocacronym{Iti 91} \toctranslation{Lifestyle } \tocroot{Jīvikasutta}}
\markboth{Lifestyle }{Jīvikasutta}
\extramarks{Iti 91}{Iti 91}

This\marginnote{1.1} was said by the Buddha, the Perfected One: that is what I heard. 

“Mendicants,\marginnote{2.1} this relying on alms is an extreme lifestyle. The world curses you: ‘You beggar, walking bowl in hand!’ Yet earnest gentlemen take it up for a good reason. Not because they’ve been forced to by kings or bandits, or because they’re in debt or threatened, or to earn a living. Rather, because they think: ‘I’m swamped by rebirth, old age, and death; by sorrow, lamentation, pain, sadness, and distress. I’m swamped by suffering, mired in suffering. Hopefully I can find an end to this entire mass of suffering.’ That’s how this gentleman has gone forth. Yet they covet sensual pleasures; they’re infatuated, full of ill will and malicious intent. They are unmindful, lacking situational awareness and immersion, with straying mind and undisciplined faculties. Suppose there was a firebrand for lighting a funeral pyre, burning at both ends, and smeared with dung in the middle. It couldn’t be used as timber either in the village or the wilderness. I say that person is just like this. They’ve missed out on the pleasures of the lay life, and haven’t fulfilled the goal of the ascetic life. 

The\marginnote{2.11} Buddha spoke this matter. On this it is said: 

\begin{verse}%
“They’ve\marginnote{3.1} left behind the pleasures of the lay life, \\
and miss out on the goal of the ascetic life. \\
Ruining it, they throw it away, \\
and perish like a funeral firebrand. 

Many\marginnote{4.1} who wrap their necks in ocher robes \\
are unrestrained and wicked. \\
Being wicked, they are reborn in hell \\
due to their bad deeds. 

It’d\marginnote{5.1} be better for the immoral and unrestrained \\
to eat an iron ball, \\
scorching, like a burning flame, \\
than to eat the nation’s alms.” 

%
\end{verse}

This\marginnote{6.1} too is a matter that was spoken by the Blessed One: that is what I heard. 

%
\section*{{\suttatitleacronym Iti 92}{\suttatitletranslation The Corner of the Cloak }{\suttatitleroot Saṅghāṭikaṇṇasutta}}
\addcontentsline{toc}{section}{\tocacronym{Iti 92} \toctranslation{The Corner of the Cloak } \tocroot{Saṅghāṭikaṇṇasutta}}
\markboth{The Corner of the Cloak }{Saṅghāṭikaṇṇasutta}
\extramarks{Iti 92}{Iti 92}

This\marginnote{1.1} was said by the Buddha, the Perfected One: that is what I heard. 

“Mendicants,\marginnote{2.1} suppose a mendicant were to hold the corner of my cloak and follow behind me step by step. Yet they covet sensual pleasures; they’re infatuated, full of ill will and malicious intent. They are unmindful, lacking situational awareness and immersion, with straying mind and undisciplined faculties. Then they are far from me, and I from them. Why is that? Because that mendicant does not see the teaching. Not seeing the teaching, they do not see me. 

Suppose\marginnote{3.1} a mendicant were to live a hundred leagues away. Yet they do not covet sensual pleasures; they’re not infatuated, or full of ill will and malicious intent. They have established mindfulness, situational awareness and immersion, with unified mind and restrained faculties. Then they are close to me, and I to them. Why is that? Because that mendicant sees the teaching. Seeing the teaching, they see me.” 

The\marginnote{3.7} Buddha spoke this matter. On this it is said: 

\begin{verse}%
“One\marginnote{4.1} full of desire and distress \\
may follow close behind, \\
yet see how distant they are—\\
the stirred from the still, \\
the burning from the quenched, \\
the greedy from the greedless. 

An\marginnote{5.1} astute person who has understood \\
and directly known the teaching, \\
grows calm, \\
like a lake unstirred by the wind. 

See\marginnote{6.1} how close they are—\\
the still to the still, \\
the quenched to the quenched, \\
the greedless to the greedless.” 

%
\end{verse}

This\marginnote{7.1} too is a matter that was spoken by the Blessed One: that is what I heard. 

%
\section*{{\suttatitleacronym Iti 93}{\suttatitletranslation Fire }{\suttatitleroot Aggisutta}}
\addcontentsline{toc}{section}{\tocacronym{Iti 93} \toctranslation{Fire } \tocroot{Aggisutta}}
\markboth{Fire }{Aggisutta}
\extramarks{Iti 93}{Iti 93}

This\marginnote{1.1} was said by the Buddha, the Perfected One: that is what I heard. 

“Mendicants,\marginnote{2.1} there are these three fires. What three? The fires of greed, hate, and delusion. These are the three fires.” 

The\marginnote{2.5} Buddha spoke this matter. On this it is said: 

\begin{verse}%
“The\marginnote{3.1} fire of greed burns a mortal, \\
lustful, infatuated by sensual pleasures; \\
while, fallen in the fire of hate, \\
a person kills living creatures; 

and,\marginnote{4.1} bewildered by the fire of delusion, \\
they miss the teaching of the noble ones. \\
Not recognizing these three fires, \\
people are caught up in identity. 

They\marginnote{5.1} fill the ranks of hell, \\
of birth as an animal, \\
or of demons and ghosts, \\
not freed from \textsanskrit{Māra}’s bonds. 

But\marginnote{6.1} as to those committed day and night \\
to the teaching of the Buddha: \\
they quench the fire of greed, \\
always perceiving ugliness; 

while\marginnote{7.1} those supreme persons \\
quench the fire of hate with love; \\
and the fire of delusion with the wisdom \\
that leads to penetration. 

Having\marginnote{8.1} quenched these fires, alert, \\
tireless all day and night, \\
they become completely quenched, \\
completely transcending suffering. 

Seers\marginnote{9.1} of the noble truths, knowledge masters, \\
the astute, understanding rightly, \\
directly know the ending of rebirth, \\
they come not back to future lives.” 

%
\end{verse}

This\marginnote{10.1} too is a matter that was spoken by the Blessed One: that is what I heard. 

%
\section*{{\suttatitleacronym Iti 94}{\suttatitletranslation Examination }{\suttatitleroot Upaparikkhasutta}}
\addcontentsline{toc}{section}{\tocacronym{Iti 94} \toctranslation{Examination } \tocroot{Upaparikkhasutta}}
\markboth{Examination }{Upaparikkhasutta}
\extramarks{Iti 94}{Iti 94}

This\marginnote{1.1} was said by the Buddha, the Perfected One: that is what I heard. 

“Mendicants,\marginnote{2.1} a mendicant should examine in any such a way that their consciousness is neither scattered and diffused externally nor stuck internally, and they are not anxious because of grasping. When this is the case and they are no longer anxious, there is for them no coming to be of the origin of suffering—of rebirth, old age, and death in the future.” 

The\marginnote{2.3} Buddha spoke this matter. On this it is said: 

\begin{verse}%
“For\marginnote{3.1} one who has given up seven chains, \\
a mendicant who has cut the cord, \\
transmigration through births is finished, \\
now there’ll be no more future lives.” 

%
\end{verse}

This\marginnote{4.1} too is a matter that was spoken by the Blessed One: that is what I heard. 

%
\section*{{\suttatitleacronym Iti 95}{\suttatitletranslation Provided With Pleasure }{\suttatitleroot Kāmūpapattisutta}}
\addcontentsline{toc}{section}{\tocacronym{Iti 95} \toctranslation{Provided With Pleasure } \tocroot{Kāmūpapattisutta}}
\markboth{Provided With Pleasure }{Kāmūpapattisutta}
\extramarks{Iti 95}{Iti 95}

This\marginnote{1.1} was said by the Buddha, the Perfected One: that is what I heard. 

“Mendicants,\marginnote{2.1} there are these three ways of being provided with sensual pleasures. What three? Some sensual pleasures are simply present; some are for those who love to create; and some are for those who control the creations of others. These are the three ways of being provided with sensual pleasures.” 

The\marginnote{2.5} Buddha spoke this matter. On this it is said: 

\begin{verse}%
“Sensual\marginnote{3.1} pleasures that are simply present, \\
Gods Who Control the Creations of Others, \\
Gods Who Love to Create, \\
and others who indulge in sensual pleasures—\\
They go from this state to another, \\
but don’t escape transmigration. 

Knowing\marginnote{4.1} this danger \\
in sensual indulgence, an astute person \\
would reject all sensual pleasures, \\
both human and divine. 

Having\marginnote{5.1} cut the stream so hard to pass, \\
that’s tied to pleasant seeming things, \\
they become completely quenched, \\
completely transcending suffering. 

Seers\marginnote{6.1} of the noble truths, knowledge masters, \\
the astute, understanding rightly, \\
directly know the ending of rebirth, \\
they come not back to future lives.” 

%
\end{verse}

This\marginnote{7.1} too is a matter that was spoken by the Blessed One: that is what I heard. 

%
\section*{{\suttatitleacronym Iti 96}{\suttatitletranslation Attached to Sensual Pleasures }{\suttatitleroot Kāmayogasutta}}
\addcontentsline{toc}{section}{\tocacronym{Iti 96} \toctranslation{Attached to Sensual Pleasures } \tocroot{Kāmayogasutta}}
\markboth{Attached to Sensual Pleasures }{Kāmayogasutta}
\extramarks{Iti 96}{Iti 96}

This\marginnote{1.1} was said by the Buddha, the Perfected One: that is what I heard. 

“Mendicants,\marginnote{2.1} one attached to both sensual pleasures and rebirth is a returner, who comes back to this state of existence. One detached from sensual pleasures but still attached to rebirth is a non-returner, who comes not back to this state of existence. One detached from both sensual pleasures and rebirth is a perfected one, who has ended defilements.” 

The\marginnote{2.4} Buddha spoke this matter. On this it is said: 

\begin{verse}%
“Attached\marginnote{3.1} to both sensual pleasures \\
and the desire to be reborn in a future life; \\
sentient beings continue to transmigrate, \\
with ongoing birth and death. 

Those\marginnote{4.1} who’ve given up sensual pleasures \\
without attaining the end of defilements, \\
and are still attached to being reborn, \\
are said to be non-returners. 

Those\marginnote{5.1} who have cut off doubt, \\
and ended conceit and future lives, \\
they’re the ones in this world who’ve truly crossed over, \\
having reached the ending of defilements.” 

%
\end{verse}

This\marginnote{6.1} too is a matter that was spoken by the Blessed One: that is what I heard. 

%
\section*{{\suttatitleacronym Iti 97}{\suttatitletranslation Good Morals }{\suttatitleroot Kalyāṇasīlasutta}}
\addcontentsline{toc}{section}{\tocacronym{Iti 97} \toctranslation{Good Morals } \tocroot{Kalyāṇasīlasutta}}
\markboth{Good Morals }{Kalyāṇasīlasutta}
\extramarks{Iti 97}{Iti 97}

This\marginnote{1.1} was said by the Buddha, the Perfected One: that is what I heard. 

“Mendicants,\marginnote{2.1} in this teaching and training a mendicant of good morals, good practice, and good wisdom is called consummate, accomplished, a supreme person. 

And\marginnote{3.1} how does a mendicant have good morals? It’s when a mendicant is ethical, restrained in the monastic code, conducting themselves well and seeking alms in suitable places. Seeing danger in the slightest fault, they keep the rules they’ve undertaken. That’s how a mendicant has good morals. Such is one of good morality. 

And\marginnote{4.1} how does one have good practice? It’s when a mendicant meditates pursuing the development of the seven qualities that lead to awakening. That’s how a mendicant has good practice. Such is one of good morality and good practice. 

And\marginnote{5.1} how does one have good wisdom? It’s when a mendicant realizes the undefiled freedom of heart and freedom by wisdom in this very life. And they live having realized it with their own insight due to the ending of defilements. That’s how a mendicant has good wisdom; 

Such\marginnote{6.1} is one of good morals, good practice, and good wisdom, who in this teaching and training is called consummate, accomplished, a supreme person. 

The\marginnote{6.2} Buddha spoke this matter. On this it is said: 

\begin{verse}%
“Who\marginnote{7.1} does nothing wrong \\
by body, speech or mind, \\
is said to be one good morals, \\
a conscientious mendicant. 

Who\marginnote{8.1} has well developed the seven \\
factors that lead to awakening \\
is said to be one good practice, \\
a humble mendicant. 

Who\marginnote{9.1} understands for themselves \\
the end of suffering in this life \\
is said to be one good wisdom, \\
an undefiled mendicant. 

One\marginnote{10.1} accomplished in these three things, \\
untroubled, with doubts cut off, \\
unattached to anything in the world, \\
has given up everything, they say.” 

%
\end{verse}

This\marginnote{11.1} too is a matter that was spoken by the Blessed One: that is what I heard. 

%
\section*{{\suttatitleacronym Iti 98}{\suttatitletranslation Giving }{\suttatitleroot Dānasutta}}
\addcontentsline{toc}{section}{\tocacronym{Iti 98} \toctranslation{Giving } \tocroot{Dānasutta}}
\markboth{Giving }{Dānasutta}
\extramarks{Iti 98}{Iti 98}

This\marginnote{1.1} was said by the Buddha, the Perfected One: that is what I heard. 

“There\marginnote{2.1} are, mendicants, these two gifts. A gift of material things and a gift of the teaching. The better of these two gifts is the gift of the teaching. 

There\marginnote{3.1} are these two kinds of sharing. Sharing material things and sharing the teaching. The better of these two kinds of sharing is sharing the teaching. 

There\marginnote{4.1} are these two kinds of support. Support in material things and support in the teaching. The better of these two kinds of support is support in the teaching.” 

The\marginnote{4.4} Buddha spoke this matter. On this it is said: 

\begin{verse}%
“It\marginnote{5.1} is said to be the supreme, ultimate gift, \\
and the sharing praised by the Buddha; \\
what wise and sensible person, confident in the best of fields, \\
would not sow a such timely gift? 

For\marginnote{6.1} those who are diligent in the dispensation of the Holy One, \\
both those who speak and those who listen, \\
confident in the dispensation of the Holy One, \\
such a gift purifies the highest goal.” 

%
\end{verse}

This\marginnote{7.1} too is a matter that was spoken by the Blessed One: that is what I heard. 

%
\section*{{\suttatitleacronym Iti 99}{\suttatitletranslation The Three Knowledges }{\suttatitleroot Tevijjasutta}}
\addcontentsline{toc}{section}{\tocacronym{Iti 99} \toctranslation{The Three Knowledges } \tocroot{Tevijjasutta}}
\markboth{The Three Knowledges }{Tevijjasutta}
\extramarks{Iti 99}{Iti 99}

This\marginnote{1.1} was said by the Buddha, the Perfected One: that is what I heard. 

“Mendicants,\marginnote{2.1} I describe a brahmin who is master of the three Vedic knowledges in terms of the teaching, not by mere oral recitation. 

How\marginnote{3.1} so? It’s when a mendicant recollects many kinds of past lives. That is: one, two, three, four, five, ten, twenty, thirty, forty, fifty, a hundred, a thousand, a hundred thousand rebirths; many eons of the world contracting, many eons of the world expanding, many eons of the world contracting and expanding. They remember: ‘There, I was named this, my clan was that, I looked like this, and that was my food. This was how I felt pleasure and pain, and that was how my life ended. When I passed away from that place I was reborn somewhere else. There, too, I was named this, my clan was that, I looked like this, and that was my food. This was how I felt pleasure and pain, and that was how my life ended. When I passed away from that place I was reborn here.’ And so they recollect their many kinds of past lives, with features and details. This was the first knowledge they achieved. Ignorance was destroyed and knowledge arose; darkness was destroyed and light arose, as happens for a meditator who is diligent, keen, and resolute. 

Furthermore,\marginnote{4.1} with clairvoyance that is purified and superhuman, a mendicant sees sentient beings passing away and being reborn—inferior and superior, beautiful and ugly, in a good place or a bad place. They understand how sentient beings are reborn according to their deeds: ‘These dear beings did bad things by way of body, speech, and mind. They spoke ill of the noble ones; they had wrong view; and they chose to act out of that wrong view. When their body breaks up, after death, they’re reborn in a place of loss, a bad place, the underworld, hell. These dear beings, however, did good things by way of body, speech, and mind. They never spoke ill of the noble ones; they had right view; and they chose to act out of that right view. When their body breaks up, after death, they’re reborn in a good place, a heavenly realm.’ And so, with clairvoyance that is purified and superhuman, they see sentient beings passing away and being reborn—inferior and superior, beautiful and ugly, in a good place or a bad place. They understand how sentient beings are reborn according to their deeds. This was the second knowledge they achieved. Ignorance was destroyed and knowledge arose; darkness was destroyed and light arose, as happens for a meditator who is diligent, keen, and resolute. 

Furthermore,\marginnote{5.1} a mendicant realizes the undefiled freedom of heart and freedom by wisdom in this very life, and they live having realized it with their own insight due to the ending of defilements. This was the third knowledge which they achieved. Ignorance was destroyed and knowledge arose; darkness was destroyed and light arose, as happens for a meditator who is diligent, keen, and resolute. That’s how I describe a brahmin who is master of the three Vedic knowledges in terms of the teaching, not by mere oral recitation.” 

The\marginnote{5.4} Buddha spoke this matter. On this it is said: 

\begin{verse}%
“They\marginnote{6.1} know their past lives, \\
seeing heaven and places of loss, \\
and have attained the end of rebirth; \\
that sage has perfect insight. 

Because\marginnote{7.1} of these three knowledges \\
a brahmin is a master of the three knowledges. \\
That’s who I call a three-knowledge master, \\
and not some mere reciter.” 

%
\end{verse}

This\marginnote{8.1} too is a matter that was spoken by the Blessed One: that is what I heard. 

%
\addtocontents{toc}{\let\protect\contentsline\protect\nopagecontentsline}
\part*{The Book of the Fours }
\addcontentsline{toc}{part}{The Book of the Fours }
\markboth{}{}
\addtocontents{toc}{\let\protect\contentsline\protect\oldcontentsline}

%
\addtocontents{toc}{\let\protect\contentsline\protect\nopagecontentsline}
\chapter*{The Chapter on the Holy Offering of the Teaching }
\addcontentsline{toc}{chapter}{\tocchapterline{The Chapter on the Holy Offering of the Teaching }}
\addtocontents{toc}{\let\protect\contentsline\protect\oldcontentsline}

%
\section*{{\suttatitleacronym Iti 100}{\suttatitletranslation The Holy Offering of the Teaching }{\suttatitleroot Brāhmaṇadhammayāgasutta}}
\addcontentsline{toc}{section}{\tocacronym{Iti 100} \toctranslation{The Holy Offering of the Teaching } \tocroot{Brāhmaṇadhammayāgasutta}}
\markboth{The Holy Offering of the Teaching }{Brāhmaṇadhammayāgasutta}
\extramarks{Iti 100}{Iti 100}

This\marginnote{1.1} was said by the Buddha, the Perfected One: that is what I heard. 

“I,\marginnote{2.1} Mendicants, am a brahmin, committed to charity, always open-handed, bearing my final body, a healer, a surgeon. You are my rightful children, born of my mouth, born of the teaching, created by the teaching, heirs in the teaching, not in material things. 

There\marginnote{3.1} are these two gifts. A gift of material things and a gift of the teaching. The better of these two gifts is the gift of the teaching. 

There\marginnote{4.1} are these two kinds of sharing. Sharing material things and sharing the teaching. The better of these two kinds of sharing is sharing the teaching. 

There\marginnote{5.1} are these two kinds of support. Support in material things and support in the teaching. The better of these two kinds of support is support in the teaching. 

There\marginnote{6.1} are these two offerings. An offering of material things and an offering of the teaching. The better of these two offerings is an offering of the teaching.” 

The\marginnote{6.4} Buddha spoke this matter. On this it is said: 

\begin{verse}%
“The\marginnote{7.1} Realized One, compassionate for all living creatures, \\
unstintingly offers up teaching. \\
Sentient beings revere him, first among gods and humans, \\
who has gone beyond rebirth.” 

%
\end{verse}

This\marginnote{8.1} too is a matter that was spoken by the Blessed One: that is what I heard. 

%
\section*{{\suttatitleacronym Iti 101}{\suttatitletranslation Easy to Find }{\suttatitleroot Sulabhasutta}}
\addcontentsline{toc}{section}{\tocacronym{Iti 101} \toctranslation{Easy to Find } \tocroot{Sulabhasutta}}
\markboth{Easy to Find }{Sulabhasutta}
\extramarks{Iti 101}{Iti 101}

This\marginnote{1.1} was said by the Buddha, the Perfected One: that is what I heard. 

“Mendicants,\marginnote{2.1} these four trifles are easy to find and are blameless. What four? Rag-robes … A lump of almsfood … Lodgings at the root of a tree … Fermented urine as medicine … These four trifles are easy to find and are blameless. When a mendicant is content with trifles that are easy to find, they have one of the factors of the ascetic life, I say.” 

The\marginnote{2.9} Buddha spoke this matter. On this it is said: 

\begin{verse}%
“When\marginnote{3.1} one is content with what’s blameless, \\
trifling, and easy to find, \\
they don’t get upset \\
about lodgings, robes, \\
food, and drink, \\
and they’re not obstructed anywhere. 

These\marginnote{4.1} qualities are said to be \\
integral to the ascetic life. \\
They’re mastered by a mendicant, \\
content and diligent.” 

%
\end{verse}

This\marginnote{5.1} too is a matter that was spoken by the Blessed One: that is what I heard. 

%
\section*{{\suttatitleacronym Iti 102}{\suttatitletranslation The Ending of Defilements }{\suttatitleroot Āsavakkhayasutta}}
\addcontentsline{toc}{section}{\tocacronym{Iti 102} \toctranslation{The Ending of Defilements } \tocroot{Āsavakkhayasutta}}
\markboth{The Ending of Defilements }{Āsavakkhayasutta}
\extramarks{Iti 102}{Iti 102}

This\marginnote{1.1} was said by the Buddha, the Perfected One: that is what I heard. 

“Mendicants,\marginnote{2.1} I say that the ending of defilements is for one who knows and sees, not for one who does not know or see. For one who knows and sees what? The ending of defilements is for one who knows and sees suffering, its origin, its cessation, and the path. The ending of the defilements is for one who knows and sees this.” 

The\marginnote{2.5} Buddha spoke this matter. On this it is said: 

\begin{verse}%
“As\marginnote{3.1} a trainee trains, \\
following the straight road, \\
first they know about ending; \\
enlightenment follows in the same lifetime. 

Then\marginnote{4.1} to one freed through enlightenment \\
the knowledge of ending arises, \\
the supreme knowledge of freedom, \\
with the ending of the fetters. 

This\marginnote{5.1} is not for the lazy, \\
the fools don’t understand, \\
extinguishment is realized \\
with release from all ties.” 

%
\end{verse}

This\marginnote{6.1} too is a matter that was spoken by the Blessed One: that is what I heard. 

%
\section*{{\suttatitleacronym Iti 103}{\suttatitletranslation Ascetics and Brahmins }{\suttatitleroot Samaṇabrāhmaṇasutta}}
\addcontentsline{toc}{section}{\tocacronym{Iti 103} \toctranslation{Ascetics and Brahmins } \tocroot{Samaṇabrāhmaṇasutta}}
\markboth{Ascetics and Brahmins }{Samaṇabrāhmaṇasutta}
\extramarks{Iti 103}{Iti 103}

This\marginnote{1.1} was said by the Buddha, the Perfected One: that is what I heard. 

“Mendicants,\marginnote{2.1} there are ascetics and brahmins who don’t truly understand about suffering, its origin, its cessation, and the path. I don’t regard them as true ascetics and brahmins. Those venerables don’t realize the goal of life as an ascetic or brahmin, and don’t live having realized it with their own insight. 

There\marginnote{3.1} are ascetics and brahmins who truly understand about suffering, its origin, its cessation, and the path. 

\begin{verse}%
“There\marginnote{4.1} are those who don’t understand suffering \\
and suffering’s cause, \\
and where all suffering \\
cease with nothing left over; \\
And they don’t know the path \\
that leads to the stilling of suffering. 

They\marginnote{5.1} lack the heart’s release, \\
as well as the release by wisdom. \\
Unable to make an end, \\
they continue to be reborn and grow old. 

But\marginnote{6.1} there are those who understand suffering \\
and suffering’s cause, \\
and where all suffering \\
cease with nothing left over; \\
And they understand the path \\
that leads to the stilling of suffering. 

They’re\marginnote{7.1} endowed with the heart’s release, \\
as well as the release by wisdom. \\
Able to make an end, \\
they don’t continue to be reborn and grow old.” 

%
\end{verse}

This\marginnote{8.1} too is a matter that was spoken by the Blessed One: that is what I heard. 

%
\section*{{\suttatitleacronym Iti 104}{\suttatitletranslation Accomplished in Ethics }{\suttatitleroot Sīlasampannasutta}}
\addcontentsline{toc}{section}{\tocacronym{Iti 104} \toctranslation{Accomplished in Ethics } \tocroot{Sīlasampannasutta}}
\markboth{Accomplished in Ethics }{Sīlasampannasutta}
\extramarks{Iti 104}{Iti 104}

This\marginnote{1.1} was said by the Buddha, the Perfected One: that is what I heard. 

“Mendicants,\marginnote{2.1} take a mendicant who is accomplished in ethics, immersion, wisdom, freedom, and the knowledge and vision of freedom. They advise and instruct. They educate, encourage, fire up, and inspire, and can rightly explain the true teaching. Even the sight of those mendicants is very helpful, I say. Even to hear them, approach them, pay homage to them, recollect them, or go forth after them is very helpful, I say. For one who frequents and associates with such mendicants, their incomplete spectrum of ethics is completed. Their incomplete spectrum of immersion … wisdom … freedom … knowledge and vision of freedom is completed. Such mendicants are called ‘teachers’, ‘leaders of the caravan’, ‘vice-discarders’, ‘dispellers of darkness’, ‘bringers of light’, ‘luminaries’, ‘lamps’, ‘candlebearers’, ‘beacons’, ‘noble ones’, and ‘seers’. 

The\marginnote{2.10} Buddha spoke this matter. On this it is said: 

\begin{verse}%
“This\marginnote{3.1} is a reason for joy \\
for those who understand: \\
that is, those who are evolved, \\
the noble ones living righteously. 

They\marginnote{4.1} illuminate the true teaching, \\
beacons beaming light, \\
the wise ones bringing light, \\
seers with vices discarded. 

Having\marginnote{5.1} heard their instruction, \\
the astute, understanding rightly, \\
directly know the ending of rebirth, \\
they come not back to future lives.” 

%
\end{verse}

This\marginnote{6.1} too is a matter that was spoken by the Blessed One: that is what I heard. 

%
\section*{{\suttatitleacronym Iti 105}{\suttatitletranslation The Arising of Craving }{\suttatitleroot Taṇhuppādasutta}}
\addcontentsline{toc}{section}{\tocacronym{Iti 105} \toctranslation{The Arising of Craving } \tocroot{Taṇhuppādasutta}}
\markboth{The Arising of Craving }{Taṇhuppādasutta}
\extramarks{Iti 105}{Iti 105}

This\marginnote{1.1} was said by the Buddha, the Perfected One: that is what I heard. 

“Mendicants,\marginnote{2.1} there are four things that give rise to craving in a mendicant. What four? For the sake of robes, almsfood, lodgings, or rebirth in this or that state. These are the four things that give rise to craving in a mendicant.” 

The\marginnote{2.8} Buddha spoke this matter. On this it is said: 

\begin{verse}%
“Craving\marginnote{3.1} is a person’s partner \\
as they transmigrate on this long journey. \\
They go from this state to another, \\
but don’t escape transmigration. 

Knowing\marginnote{4.1} this danger, \\
that craving is the cause of suffering—\\
rid of craving, free of grasping, \\
a mendicant would wander mindful.” 

%
\end{verse}

This\marginnote{5.1} too is a matter that was spoken by the Blessed One: that is what I heard. 

%
\section*{{\suttatitleacronym Iti 106}{\suttatitletranslation With Brahmā }{\suttatitleroot Sabrahmakasutta}}
\addcontentsline{toc}{section}{\tocacronym{Iti 106} \toctranslation{With Brahmā } \tocroot{Sabrahmakasutta}}
\markboth{With Brahmā }{Sabrahmakasutta}
\extramarks{Iti 106}{Iti 106}

This\marginnote{1.1} was said by the Buddha, the Perfected One: that is what I heard. 

“Mendicants,\marginnote{2.1} a family where the children honor their parents in their home is said to live with \textsanskrit{Brahmā}. A family where the children honor their parents in their home is said to live with the old deities. A family where the children honor their parents in their home is said to live with the first teachers. A family where the children honor their parents in their home is said to live with those worthy of offerings dedicated to the gods. 

‘\textsanskrit{Brahmā}’\marginnote{3.1} is a term for your parents. ‘Old deities’ is a term for your parents. ‘First teachers’ is a term for your parents. ‘Worthy of offerings dedicated to the gods’ is a term for your parents. Why is that? Parents are very helpful to their children, they raise them, nurture them, and show them the world.” 

The\marginnote{3.7} Buddha spoke this matter. On this it is said: 

\begin{verse}%
“Parents\marginnote{4.1} are said to be ‘\textsanskrit{Brahmā}’ \\
and ‘first teachers’. \\
They’re worthy of offerings dedicated to the gods from their children, \\
for they love their offspring. 

Therefore\marginnote{5.1} an astute person \\
would revere them and honor them \\
with food and drink, \\
clothes and bedding, \\
anointing and bathing, \\
and by washing their feet. 

Because\marginnote{6.1} they look after \\
their parents like this, \\
they’re praised in this life by the astute, \\
and they depart to rejoice in heaven.” 

%
\end{verse}

This\marginnote{7.1} too is a matter that was spoken by the Blessed One: that is what I heard. 

%
\section*{{\suttatitleacronym Iti 107}{\suttatitletranslation Very Helpful }{\suttatitleroot Bahukārasutta}}
\addcontentsline{toc}{section}{\tocacronym{Iti 107} \toctranslation{Very Helpful } \tocroot{Bahukārasutta}}
\markboth{Very Helpful }{Bahukārasutta}
\extramarks{Iti 107}{Iti 107}

This\marginnote{1.1} was said by the Buddha, the Perfected One: that is what I heard. 

“Mendicants,\marginnote{2.1} brahmins and householders are very helpful to you, as they provide you with robes, almsfood, lodgings, and medicines and supplies for the sick. And you are very helpful to brahmins and householders, as you teach them the Dhamma that’s good in the beginning, good in the middle, and good in the end, meaningful and well-phrased. And you reveal a spiritual practice that’s entirely full and pure. That is how this spiritual path is lived in mutual dependence in order to cross over the flood and make a complete end of suffering.” 

The\marginnote{2.4} Buddha spoke this matter. On this it is said: 

\begin{verse}%
“The\marginnote{3.1} home-dweller and the homeless, \\
depending on each other, \\
find success in the true teaching, \\
the supreme sanctuary. 

The\marginnote{4.1} homeless receive requisites \\
from the home-dwellers: \\
robes and lodgings \\
to shelter from harsh conditions. 

Relying\marginnote{5.1} on the Holy One, \\
home-loving layfolk \\
place faith in the perfected ones, \\
meditators of noble wisdom. 

Having\marginnote{6.1} practiced the teaching here, \\
the path that goes to a good place, \\
they delight in the heavenly realm, \\
enjoying all the pleasures they desire.” 

%
\end{verse}

This\marginnote{7.1} too is a matter that was spoken by the Blessed One: that is what I heard. 

%
\section*{{\suttatitleacronym Iti 108}{\suttatitletranslation Deceivers }{\suttatitleroot Kuhasutta}}
\addcontentsline{toc}{section}{\tocacronym{Iti 108} \toctranslation{Deceivers } \tocroot{Kuhasutta}}
\markboth{Deceivers }{Kuhasutta}
\extramarks{Iti 108}{Iti 108}

This\marginnote{1.1} was said by the Buddha, the Perfected One: that is what I heard. 

“Mendicants,\marginnote{2.1} those mendicants who are deceivers and flatterers, pompous and fake, insolent and scattered: those mendicants are no followers of mine. They’ve left this teaching and training, and they don’t achieve growth, improvement, or maturity in this teaching and training. But those mendicants who are genuine, not flatterers, wise, amenable, and serene: those mendicants are followers of mine. They haven’t left this teaching and training, and they do achieve growth, improvement, or maturity in this teaching and training.” 

The\marginnote{2.5} Buddha spoke this matter. On this it is said: 

\begin{verse}%
“Those\marginnote{3.1} who are deceivers and flatterers, pompous and fake, \\
insolent and scattered: \\
these don’t grow in the teaching \\
that was taught by the perfected Buddha. 

But\marginnote{4.1} those who are genuine, not flatterers, wise, \\
amenable, and serene: \\
these do grow in the teaching \\
that was taught by the perfected Buddha.” 

%
\end{verse}

This\marginnote{5.1} too is a matter that was spoken by the Blessed One: that is what I heard. 

%
\section*{{\suttatitleacronym Iti 109}{\suttatitletranslation A River }{\suttatitleroot Nadīsotasutta}}
\addcontentsline{toc}{section}{\tocacronym{Iti 109} \toctranslation{A River } \tocroot{Nadīsotasutta}}
\markboth{A River }{Nadīsotasutta}
\extramarks{Iti 109}{Iti 109}

This\marginnote{1.1} was said by the Buddha, the Perfected One: that is what I heard. 

“Suppose\marginnote{2.1} a person was being carried along by a river current that seemed nice and pleasant. If a person with good eyesight saw them, they’d say: ‘Mister, even though the river current carrying you along seems nice and pleasant, downstream there is a lake with waves and whirlpools, saltwater crocodiles and monsters. When you reach there it will result in death or deadly pain.’ Then, when they heard what was said, they’d paddle against the stream using their hands and feet. 

I’ve\marginnote{3.1} made up this simile to make a point. And this is the point. ‘Stream’ is a term for craving. 

‘Seeming\marginnote{4.1} nice and pleasant’ is a term for the six interior sense fields. 

‘A\marginnote{5.1} downstream lake’ is a term for the five lower fetters. 

‘Danger\marginnote{6.1} of waves’ is a term for anger and distress. 

‘Whirlpool’\marginnote{7.1} is a term for the five kinds of sensual stimulation. 

‘Saltwater\marginnote{8.1} crocodiles and monsters’ is a term for females. 

‘Against\marginnote{9.1} the stream’ is a term for renunciation. 

‘Paddling\marginnote{10.1} with hands and feet’ is a term for being energetic. 

‘A\marginnote{11.1} person with good eyesight’ is a term for the Realized One, the perfected one, the fully awakened Buddha.” 

The\marginnote{11.2} Buddha spoke this matter. On this it is said: 

\begin{verse}%
“In\marginnote{12.1} pain they’d give up sensual pleasures, \\
aspiring to the future sanctuary. \\
With deep understanding and heart well-freed, \\
they’d experience universal liberation. \\
That knowledge master who has completed the spiritual journey, \\
and gone to the end of the world, is called ‘one who has gone beyond’.” 

%
\end{verse}

This\marginnote{13.1} too is a matter that was spoken by the Blessed One: that is what I heard. 

%
\section*{{\suttatitleacronym Iti 110}{\suttatitletranslation Walking }{\suttatitleroot Carasutta}}
\addcontentsline{toc}{section}{\tocacronym{Iti 110} \toctranslation{Walking } \tocroot{Carasutta}}
\markboth{Walking }{Carasutta}
\extramarks{Iti 110}{Iti 110}

This\marginnote{1.1} was said by the Buddha, the Perfected One: that is what I heard. 

“Mendicants,\marginnote{2.1} suppose a mendicant has a sensual, malicious, or cruel thought while walking. They tolerate it and don’t give it up, get rid of it, eliminate it, and obliterate it. Such a mendicant is said to be ‘not keen or prudent, always lazy, and lacking energy’ when walking. 

Suppose\marginnote{3.1} a mendicant has a sensual, malicious, or cruel thought while standing … sitting … or when lying down while awake. They tolerate it and don’t give it up, get rid of it, eliminate it, and obliterate it. Such a mendicant is said to be ‘not keen or prudent, always lazy, and lacking energy’ when lying down while awake. 

Suppose\marginnote{6.1} a mendicant has a sensual, malicious, or cruel thought while walking. They don’t tolerate them, but give them up, get rid of them, eliminate them, and obliterate them. Such a mendicant is said to be ‘keen and prudent, always energetic and determined’ when walking. 

Suppose\marginnote{7.1} a mendicant has a sensual, malicious, or cruel thought while standing … sitting … or when lying down while awake. They don’t tolerate it, but give it up, get rid of it, eliminate it, and obliterate it. Such a mendicant is said to be ‘keen and prudent, always energetic and determined’ when lying down while awake.” 

The\marginnote{9.4} Buddha spoke this matter. On this it is said: 

\begin{verse}%
“Whether\marginnote{10.1} walking or standing, \\
sitting or lying down, \\
one who thinks a bad thought \\
to do with the lay life 

is\marginnote{11.1} practicing the wrong way, \\
lost among things that delude; \\
such a mendicant is incapable \\
of touching the highest awakening. 

But\marginnote{12.1} one who, whether standing or walking, \\
sitting or lying down, \\
has calmed their thoughts, \\
loving peace of mind; \\
such a mendicant is capable \\
of touching the highest awakening.” 

%
\end{verse}

This\marginnote{13.1} too is a matter that was spoken by the Blessed One: that is what I heard. 

%
\section*{{\suttatitleacronym Iti 111}{\suttatitletranslation Accomplishment in Ethics }{\suttatitleroot Sampannasīlasutta}}
\addcontentsline{toc}{section}{\tocacronym{Iti 111} \toctranslation{Accomplishment in Ethics } \tocroot{Sampannasīlasutta}}
\markboth{Accomplishment in Ethics }{Sampannasīlasutta}
\extramarks{Iti 111}{Iti 111}

This\marginnote{1.1} was said by the Buddha, the Perfected One: that is what I heard. 

“Mendicants,\marginnote{2.1} live by the ethical precepts and the monastic code. Live restrained in the code of conduct, conducting yourselves well and seeking alms in suitable places. Seeing danger in the slightest fault, keep the rules you’ve undertaken. 

When\marginnote{3.1} you’ve done this, what more is there to do? 

Suppose\marginnote{4.1} a mendicant has got rid of desire and ill will while walking, and has given up dullness and drowsiness, restlessness and remorse, and doubt. Their energy is roused up and unflagging, their mindfulness is established and lucid, their body is tranquil and undisturbed, and their mind is immersed in \textsanskrit{samādhi}. Such a mendicant is said to be ‘keen and prudent, always energetic and determined’ when walking. 

Suppose\marginnote{5.1} a mendicant has got rid of desire and ill will while standing … 

sitting\marginnote{6.1} … 

or\marginnote{7.1} when lying down while awake. Such a mendicant is said to be ‘keen and prudent, always energetic and determined’ when lying down while awake.” 

The\marginnote{7.3} Buddha spoke this matter. On this it is said: 

\begin{verse}%
“Carefully\marginnote{8.1} walking, carefully standing, \\
carefully sitting, carefully lying; \\
a mendicant carefully bends their limbs, \\
and carefully extends them. 

Above,\marginnote{9.1} below, all round, \\
as far as the earth extends; \\
they scrutinize the rise and fall \\
of phenomena such as the aggregates. 

Meditating\marginnote{10.1} diligently like this, \\
peaceful and stable, \\
training in what leads to serenity of heart, \\
always staying mindful; \\
they call such a mendicant \\
‘always determined’.” 

%
\end{verse}

This\marginnote{11.1} too is a matter that was spoken by the Blessed One: that is what I heard. 

%
\section*{{\suttatitleacronym Iti 112}{\suttatitletranslation The World }{\suttatitleroot Lokasutta}}
\addcontentsline{toc}{section}{\tocacronym{Iti 112} \toctranslation{The World } \tocroot{Lokasutta}}
\markboth{The World }{Lokasutta}
\extramarks{Iti 112}{Iti 112}

This\marginnote{1.1} was said by the Buddha, the Perfected One: that is what I heard. 

“Mendicants,\marginnote{2.1} the world has been understood by the Realized One; and he is detached from the world. The origin of the world has been understood by the Realized One; and he has given up the origin of the world. The cessation of the world has been understood by the Realized One; and he has realized the cessation of the world. The practice that leads to the cessation of the world has been understood by the Realized One; and he has developed the practice that leads to the cessation of the world. 

In\marginnote{3.1} this world—with its gods, \textsanskrit{Māras}, and \textsanskrit{Brahmās}, this population with its ascetics and brahmins, its gods and humans—whatever is seen, heard, thought, known, attained, sought, and explored by the mind, all that has been understood by the Realized One. That’s why he’s called the ‘Realized One’. 

From\marginnote{4.1} the night when the Realized One understands the supreme perfect awakening until the night he becomes fully extinguished—through the element of extinguishment with nothing left over—everything he speaks, says, and expresses is real, not otherwise. That’s why he’s called the ‘Realized One’. 

The\marginnote{5.1} Realized One does as he says, and says as he does. Since this is so, that’s why he’s called the ‘Realized One’. 

In\marginnote{6.1} this world—with its gods, \textsanskrit{Māras} and \textsanskrit{Brahmās}, this population with its ascetics and brahmins, gods and humans—the Realized One is the undefeated, the champion, the universal seer, the wielder of power. That’s why he’s called the ‘Realized One’.” 

The\marginnote{6.3} Buddha spoke this matter. On this it is said: 

\begin{verse}%
“Directly\marginnote{7.1} knowing the whole world as it is, \\
and everything in it, \\
he is detached from the whole world, \\
disengaged from the whole world. 

That\marginnote{8.1} wise one is the champion \\
who is released from all ties. \\
He has reached ultimate peace: \\
extinguishment, fearing nothing from any quarter. 

He\marginnote{9.1} is the Buddha, with defilements ended, \\
untroubled, with doubts cut off. \\
He has attained the end of all deeds, \\
freed with the ending of attachments. 

That\marginnote{10.1} Blessed One is the Buddha, \\
he is the supreme lion, \\
in all the world with its gods, \\
he turns the holy wheel. 

And\marginnote{11.1} so those gods and humans, \\
who have gone to the Buddha for refuge, \\
come together and revere him, \\
even the deities revere him: 

‘Tamed,\marginnote{12.1} he is the best of tamers, \\
peaceful, he is the hermit among the peaceful, \\
liberated, he is the foremost of liberators, \\
crossed over, he is the most excellent of guides across.’ 

And\marginnote{13.1} so they revere him, \\
great of heart and rid of naivety. \\
In the world with its gods, \\
he has no counterpart.” 

%
\end{verse}

This\marginnote{14.1} too is a matter that was spoken by the Blessed One: that is what I heard. 

%
\backmatter%
\chapter*{Colophon}
\addcontentsline{toc}{chapter}{Colophon}
\markboth{Colophon}{Colophon}

\section*{The Translator}

Bhikkhu Sujato was born as Anthony Aidan Best on 4/11/1966 in Perth, Western Australia. He grew up in the pleasant suburbs of Mt Lawley and Attadale alongside his sister Nicola, who was the good child. His mother, Margaret Lorraine Huntsman née Pinder, said “he’ll either be a priest or a poet”, while his father, Anthony Thomas Best, advised him to “never do anything for money”. He attended Aquinas College, a Catholic school, where he decided to become an atheist. At the University of WA he studied philosophy, aiming to learn what he wanted to do with his life. Finding that what he wanted to do was play guitar, he dropped out. His main band was named Martha’s Vineyard, which achieved modest success in the indie circuit. 

A seemingly random encounter with a roadside joey took him to Thailand, where he entered his first meditation retreat at Wat Ram Poeng, Chieng Mai in 1992. Feeling the call to the Buddha’s path, he took full ordination in Wat Pa Nanachat in 1994, where his teachers were Ajahn Pasanno and Ajahn Jayasaro. In 1997 he returned to Perth to study with Ajahn Brahm at Bodhinyana Monastery. 

He spent several years practicing in seclusion in Malaysia and Thailand before establishing Santi Forest Monastery in Bundanoon, NSW, in 2003. There he was instrumental in supporting the establishment of the Theravada bhikkhuni order in Australia and advocating for women’s rights. He continues to teach in Australia and globally, with a special concern for the moral implications of climate change and other forms of environmental destruction. He has published a series of books of original and groundbreaking research on early Buddhism. 

In 2005 he founded SuttaCentral together with Rod Bucknell and John Kelly. In 2015, seeing the need for a complete, accurate, plain English translation of the Pali texts, he undertook the task, spending nearly three years in isolation on the isle of Qi Mei off the coast of the nation of Taiwan. He completed the four main \textsanskrit{Nikāyas} in 2018, and the early books of the Khuddaka \textsanskrit{Nikāya} were complete by 2021. All this work is dedicated to the public domain and is entirely free of copyright encumbrance. 

In 2019 he returned to Sydney where he established Lokanta Vihara (The Monastery at the End of the World). 

\section*{Creation Process}

Translated from the Pali. Primary source was the \textsanskrit{Mahāsaṅgīti} edition, with reference to several English translations, especially those of John Ireland.

\section*{The Translation}

This translation aims to make a clear, readable, and accurate rendering of the Itivuttaka, preserving consistency with Sujato’s renderings of the main \textsanskrit{nikāyas}. 

\section*{About SuttaCentral}

SuttaCentral publishes early Buddhist texts. Since 2005 we have provided root texts in Pali, Chinese, Sanskrit, Tibetan, and other languages, parallels between these texts, and translations in many modern languages. We build on the work of generations of scholars, and offer our contribution freely.

SuttaCentral is driven by volunteer contributions, and in addition we employ professional developers. We offer a sponsorship program for high quality translations from the original languages. Financial support for SuttaCentral is handled by the SuttaCentral Development Trust, a charitable trust registered in Australia.

\section*{About Bilara}

“Bilara” means “cat” in Pali, and it is the name of our Computer Assisted Translation (CAT) software. Bilara is a web app that enables translators to translate early Buddhist texts into their own language. These translations are published on SuttaCentral with the root text and translation side by side.

\section*{About SuttaCentral Editions}

The SuttaCentral Editions project makes high quality books from selected Bilara translations. These are published in formats including HTML, EPUB, PDF, and print.

If you want to print any of our Editions, please let us know and we will help prepare a file to your specifications.

%
\end{document}