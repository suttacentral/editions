\documentclass[12pt,openany]{book}%
\usepackage{lastpage}%
%
\usepackage{ragged2e}
\usepackage{verse}
\usepackage[a-3u]{pdfx}
\usepackage[inner=1in, outer=1in, top=.7in, bottom=1in, papersize={6in,9in}, headheight=13pt]{geometry}
\usepackage{polyglossia}
\usepackage[12pt]{moresize}
\usepackage{soul}%
\usepackage{microtype}
\usepackage{tocbasic}
\usepackage{realscripts}
\usepackage{epigraph}%
\usepackage{setspace}%
\usepackage{sectsty}
\usepackage{fontspec}
\usepackage{marginnote}
\usepackage[bottom]{footmisc}
\usepackage{enumitem}
\usepackage{fancyhdr}
\usepackage{emptypage}
\usepackage{extramarks}
\usepackage{graphicx}
\usepackage{relsize}
\usepackage{etoolbox}

% improve ragged right headings by suppressing hyphenation and orphans. spaceskip plus and minus adjust interword spacing; increase rightskip stretch to make it want to push a word on the first line(s) to the next line; reduce parfillskip stretch to make line length more equal . spacefillskip and xspacefillskip can be deleted to use defaults.
\protected\def\BalancedRagged{
\leftskip     0pt
\rightskip    0pt plus 10em
\spaceskip=1\fontdimen2\font plus .5\fontdimen3\font minus 1.5\fontdimen4\font
\xspaceskip=1\fontdimen2\font plus 1\fontdimen3\font minus 1\fontdimen4\font
\parfillskip  0pt plus 15em
\relax
}

\hypersetup{
colorlinks=true,
urlcolor=black,
linkcolor=black,
citecolor=black,
allcolors=black
}

% use a small amount of tracking on small caps
\SetTracking[ spacing = {25*,166, } ]{ encoding = *, shape = sc }{ 25 }

% add a blank page
\newcommand{\blankpage}{
\newpage
\thispagestyle{empty}
\mbox{}
\newpage
}

% define languages
\setdefaultlanguage[]{english}
\setotherlanguage[script=Latin]{sanskrit}

%\usepackage{pagegrid}
%\pagegridsetup{top-left, step=.25in}

% define fonts
% use if arno sanskrit is unavailable
%\setmainfont{Gentium Plus}
%\newfontfamily\Marginalfont[]{Gentium Plus}
%\newfontfamily\Allsmallcapsfont[RawFeature=+c2sc]{Gentium Plus}
%\newfontfamily\Noligaturefont[Renderer=Basic]{Gentium Plus}
%\newfontfamily\Noligaturecaptionfont[Renderer=Basic]{Gentium Plus}
%\newfontfamily\Fleuronfont[Ornament=1]{Gentium Plus}

% use if arno sanskrit is available. display is applied to \chapter and \part, subhead to \section and \subsection.
\setmainfont[
  FontFace={sb}{n}{Font = {Arno Pro Semibold}},
  FontFace={sb}{it}{Font = {Arno  Pro Semibold Italic}}
]{Arno Pro}

% create commands for using semibold
\DeclareRobustCommand{\sbseries}{\fontseries{sb}\selectfont}
\DeclareTextFontCommand{\textsb}{\sbseries}

\newfontfamily\Marginalfont[RawFeature=+subs]{Arno Pro Regular}
\newfontfamily\Allsmallcapsfont[RawFeature=+c2sc]{Arno Pro}
\newfontfamily\Noligaturefont[Renderer=Basic]{Arno Pro}
\newfontfamily\Noligaturecaptionfont[Renderer=Basic]{Arno Pro Caption}

% chinese fonts
\newfontfamily\cjk{Noto Serif TC}
\newcommand*{\langlzh}[1]{\cjk{#1}\normalfont}%

% logo
\newfontfamily\Logofont{sclogo.ttf}
\newcommand*{\sclogo}[1]{\large\Logofont{#1}}

% use subscript numerals for margin notes
\renewcommand*{\marginfont}{\Marginalfont}

% ensure margin notes have consistent vertical alignment
\renewcommand*{\marginnotevadjust}{-.17em}

% use compact lists
\setitemize{noitemsep,leftmargin=1em}
\setenumerate{noitemsep,leftmargin=1em}
\setdescription{noitemsep, style=unboxed, leftmargin=1em}

% style ToC
\DeclareTOCStyleEntries[
  raggedentrytext,
  linefill=\hfill,
  pagenumberwidth=.5in,
  pagenumberformat=\normalfont,
  entryformat=\normalfont
]{tocline}{chapter,section}


  \setlength\topsep{0pt}%
  \setlength\parskip{0pt}%

% define new \centerpars command for use in ToC. This ensures centering, proper wrapping, and no page break after
\def\startcenter{%
  \par
  \begingroup
  \leftskip=0pt plus 1fil
  \rightskip=\leftskip
  \parindent=0pt
  \parfillskip=0pt
}
\def\stopcenter{%
  \par
  \endgroup
}
\long\def\centerpars#1{\startcenter#1\stopcenter}

% redefine part, so that it adds a toc entry without page number
\let\oldcontentsline\contentsline
\newcommand{\nopagecontentsline}[3]{\oldcontentsline{#1}{#2}{}}

    \makeatletter
\renewcommand*\l@part[2]{%
  \ifnum \c@tocdepth >-2\relax
    \addpenalty{-\@highpenalty}%
    \addvspace{0em \@plus\p@}%
    \setlength\@tempdima{3em}%
    \begingroup
      \parindent \z@ \rightskip \@pnumwidth
      \parfillskip -\@pnumwidth
      {\leavevmode
       \setstretch{.85}\large\scshape\centerpars{#1}\vspace*{-1em}\llap{#2}}\par
       \nobreak
         \global\@nobreaktrue
         \everypar{\global\@nobreakfalse\everypar{}}%
    \endgroup
  \fi}
\makeatother

\makeatletter
\def\@pnumwidth{2em}
\makeatother

% define new sectioning command, which is only used in volumes where the pannasa is found in some parts but not others, especially in an and sn

\newcommand*{\pannasa}[1]{\clearpage\thispagestyle{empty}\begin{center}\vspace*{14em}\setstretch{.85}\huge\itshape\scshape\MakeLowercase{#1}\end{center}}

    \makeatletter
\newcommand*\l@pannasa[2]{%
  \ifnum \c@tocdepth >-2\relax
    \addpenalty{-\@highpenalty}%
    \addvspace{.5em \@plus\p@}%
    \setlength\@tempdima{3em}%
    \begingroup
      \parindent \z@ \rightskip \@pnumwidth
      \parfillskip -\@pnumwidth
      {\leavevmode
       \setstretch{.85}\large\itshape\scshape\lowercase{\centerpars{#1}}\vspace*{-1em}\llap{#2}}\par
       \nobreak
         \global\@nobreaktrue
         \everypar{\global\@nobreakfalse\everypar{}}%
    \endgroup
  \fi}
\makeatother

% don't put page number on first page of toc (relies on etoolbox)
\patchcmd{\chapter}{plain}{empty}{}{}

% global line height
\setstretch{1.05}

% allow linebreak after em-dash
\catcode`\—=13
\protected\def—{\unskip\textemdash\allowbreak}

% style headings with secsty. chapter and section are defined per-edition
\partfont{\setstretch{.85}\normalfont\centering\textsc}
\subsectionfont{\setstretch{.95}\normalfont\BalancedRagged}%
\subsubsectionfont{\setstretch{1}\normalfont\itshape\BalancedRagged}

% style elements of suttatitle
\newcommand*{\suttatitleacronym}[1]{\smaller[2]{#1}\vspace*{.3em}}
\newcommand*{\suttatitletranslation}[1]{\linebreak{#1}}
\newcommand*{\suttatitleroot}[1]{\linebreak\smaller[2]\itshape{#1}}

\DeclareTOCStyleEntries[
  indent=3.3em,
  dynindent,
  beforeskip=.2em plus -2pt minus -1pt,
]{tocline}{section}

\DeclareTOCStyleEntries[
  indent=0em,
  dynindent,
  beforeskip=.4em plus -2pt minus -1pt,
]{tocline}{chapter}

\newcommand*{\tocacronym}[1]{\hspace*{-3.3em}{#1}\quad}
\newcommand*{\toctranslation}[1]{#1}
\newcommand*{\tocroot}[1]{(\textit{#1})}
\newcommand*{\tocchapterline}[1]{\bfseries\itshape{#1}}


% redefine paragraph and subparagraph headings to not be inline
\makeatletter
% Change the style of paragraph headings %
\renewcommand\paragraph{\@startsection{paragraph}{4}{\z@}%
            {-2.5ex\@plus -1ex \@minus -.25ex}%
            {1.25ex \@plus .25ex}%
            {\noindent\normalfont\itshape\small}}

% Change the style of subparagraph headings %
\renewcommand\subparagraph{\@startsection{subparagraph}{5}{\z@}%
            {-2.5ex\@plus -1ex \@minus -.25ex}%
            {1.25ex \@plus .25ex}%
            {\noindent\normalfont\itshape\footnotesize}}
\makeatother

% use etoolbox to suppress page numbers on \part
\patchcmd{\part}{\thispagestyle{plain}}{\thispagestyle{empty}}
  {}{\errmessage{Cannot patch \string\part}}

% and to reduce margins on quotation
\patchcmd{\quotation}{\rightmargin}{\leftmargin 1.2em \rightmargin}{}{}
\AtBeginEnvironment{quotation}{\small}

% titlepage
\newcommand*{\titlepageTranslationTitle}[1]{{\begin{center}\begin{large}{#1}\end{large}\end{center}}}
\newcommand*{\titlepageCreatorName}[1]{{\begin{center}\begin{normalsize}{#1}\end{normalsize}\end{center}}}

% halftitlepage
\newcommand*{\halftitlepageTranslationTitle}[1]{\setstretch{2.5}{\begin{Huge}\uppercase{\so{#1}}\end{Huge}}}
\newcommand*{\halftitlepageTranslationSubtitle}[1]{\setstretch{1.2}{\begin{large}{#1}\end{large}}}
\newcommand*{\halftitlepageFleuron}[1]{{\begin{large}\Fleuronfont{{#1}}\end{large}}}
\newcommand*{\halftitlepageByline}[1]{{\begin{normalsize}\textit{{#1}}\end{normalsize}}}
\newcommand*{\halftitlepageCreatorName}[1]{{\begin{LARGE}{\textsc{#1}}\end{LARGE}}}
\newcommand*{\halftitlepageVolumeNumber}[1]{{\begin{normalsize}{\Allsmallcapsfont{\textsc{#1}}}\end{normalsize}}}
\newcommand*{\halftitlepageVolumeAcronym}[1]{{\begin{normalsize}{#1}\end{normalsize}}}
\newcommand*{\halftitlepageVolumeTranslationTitle}[1]{{\begin{Large}{\textsc{#1}}\end{Large}}}
\newcommand*{\halftitlepageVolumeRootTitle}[1]{{\begin{normalsize}{\Allsmallcapsfont{\textsc{\itshape #1}}}\end{normalsize}}}
\newcommand*{\halftitlepagePublisher}[1]{{\begin{large}{\Noligaturecaptionfont\textsc{#1}}\end{large}}}

% epigraph
\renewcommand{\epigraphflush}{center}
\renewcommand*{\epigraphwidth}{.85\textwidth}
\newcommand*{\epigraphTranslatedTitle}[1]{\vspace*{.5em}\footnotesize\textsc{#1}\\}%
\newcommand*{\epigraphRootTitle}[1]{\footnotesize\textit{#1}\\}%
\newcommand*{\epigraphReference}[1]{\footnotesize{#1}}%

% map
\newsavebox\IBox

% custom commands for html styling classes
\newcommand*{\scnamo}[1]{\begin{Center}\textit{#1}\end{Center}\bigskip}
\newcommand*{\scendsection}[1]{\begin{Center}\begin{small}\textit{#1}\end{small}\end{Center}\addvspace{1em}}
\newcommand*{\scendsutta}[1]{\begin{Center}\textit{#1}\end{Center}\addvspace{1em}}
\newcommand*{\scendbook}[1]{\bigskip\begin{Center}\uppercase{#1}\end{Center}\addvspace{1em}}
\newcommand*{\scendkanda}[1]{\begin{Center}\textbf{#1}\end{Center}\addvspace{1em}} % use for ending vinaya rule sections and also samyuttas %
\newcommand*{\scend}[1]{\begin{Center}\begin{small}\textit{#1}\end{small}\end{Center}\addvspace{1em}}
\newcommand*{\scendvagga}[1]{\begin{Center}\textbf{#1}\end{Center}\addvspace{1em}}
\newcommand*{\scrule}[1]{\textsb{#1}}
\newcommand*{\scadd}[1]{\textit{#1}}
\newcommand*{\scevam}[1]{\textsc{#1}}
\newcommand*{\scspeaker}[1]{\hspace{2em}\textit{#1}}
\newcommand*{\scbyline}[1]{\begin{flushright}\textit{#1}\end{flushright}\bigskip}
\newcommand*{\scexpansioninstructions}[1]{\begin{small}\textit{#1}\end{small}}
\newcommand*{\scuddanaintro}[1]{\medskip\noindent\begin{footnotesize}\textit{#1}\end{footnotesize}\smallskip}

\newenvironment{scuddana}{%
\setlength{\stanzaskip}{.5\baselineskip}%
  \vspace{-1em}\begin{verse}\begin{footnotesize}%
}{%
\end{footnotesize}\end{verse}
}%

% custom command for thematic break = hr
\newcommand*{\thematicbreak}{\begin{center}\rule[.5ex]{6em}{.4pt}\begin{normalsize}\quad\Fleuronfont{•}\quad\end{normalsize}\rule[.5ex]{6em}{.4pt}\end{center}}

% manage and style page header and footer. "fancy" has header and footer, "plain" has footer only

\pagestyle{fancy}
\fancyhf{}
\fancyfoot[RE,LO]{\thepage}
\fancyfoot[LE,RO]{\footnotesize\lastleftxmark}
\fancyhead[CE]{\setstretch{.85}\Noligaturefont\MakeLowercase{\textsc{\firstrightmark}}}
\fancyhead[CO]{\setstretch{.85}\Noligaturefont\MakeLowercase{\textsc{\firstleftmark}}}
\renewcommand{\headrulewidth}{0pt}
\fancypagestyle{plain}{ %
\fancyhf{} % remove everything
\fancyfoot[RE,LO]{\thepage}
\fancyfoot[LE,RO]{\footnotesize\lastleftxmark}
\renewcommand{\headrulewidth}{0pt}
\renewcommand{\footrulewidth}{0pt}}
\fancypagestyle{plainer}{ %
\fancyhf{} % remove everything
\fancyfoot[RE,LO]{\thepage}
\renewcommand{\headrulewidth}{0pt}
\renewcommand{\footrulewidth}{0pt}}

% style footnotes
\setlength{\skip\footins}{1em}

\makeatletter
\newcommand{\@makefntextcustom}[1]{%
    \parindent 0em%
    \thefootnote.\enskip #1%
}
\renewcommand{\@makefntext}[1]{\@makefntextcustom{#1}}
\makeatother

% hang quotes (requires microtype)
\microtypesetup{
  protrusion = true,
  expansion  = true,
  tracking   = true,
  factor     = 1000,
  patch      = all,
  final
}

% Custom protrusion rules to allow hanging punctuation
\SetProtrusion
{ encoding = *}
{
% char   right left
  {-} = {    , 500 },
  % Double Quotes
  \textquotedblleft
      = {1000,     },
  \textquotedblright
      = {    , 1000},
  \quotedblbase
      = {1000,     },
  % Single Quotes
  \textquoteleft
      = {1000,     },
  \textquoteright
      = {    , 1000},
  \quotesinglbase
      = {1000,     }
}

% make latex use actual font em for parindent, not Computer Modern Roman
\AtBeginDocument{\setlength{\parindent}{1em}}%
%

% Default values; a bit sloppier than normal
\tolerance 1414
\hbadness 1414
\emergencystretch 1.5em
\hfuzz 0.3pt
\clubpenalty = 10000
\widowpenalty = 10000
\displaywidowpenalty = 10000
\hfuzz \vfuzz
 \raggedbottom%

\title{Linked Discourses}
\author{Bhikkhu Sujato}
\date{}%
% define a different fleuron for each edition
\newfontfamily\Fleuronfont[Ornament=40]{Arno Pro}

% Define heading styles per edition for chapter and section. Suttatitle can be either of these, depending on the volume. 

\let\oldfrontmatter\frontmatter
\renewcommand{\frontmatter}{%
\chapterfont{\setstretch{.85}\normalfont\centering}%
\sectionfont{\setstretch{.85}\normalfont\BalancedRagged}%
\oldfrontmatter}

\let\oldmainmatter\mainmatter
\renewcommand{\mainmatter}{%
\chapterfont{\setstretch{.85}\normalfont\centering}%
\sectionfont{\setstretch{.85}\normalfont\centering}%
\oldmainmatter}

\let\oldbackmatter\backmatter
\renewcommand{\backmatter}{%
\chapterfont{\setstretch{.85}\normalfont\centering}%
\sectionfont{\setstretch{.85}\normalfont\BalancedRagged}%
\pagestyle{plainer}%
\oldbackmatter}
%
%
\begin{document}%
\normalsize%
\frontmatter%
\setlength{\parindent}{0cm}

\pagestyle{empty}

\maketitle

\blankpage%
\begin{center}

\vspace*{2.2em}

\halftitlepageTranslationTitle{Linked Discourses}

\vspace*{1em}

\halftitlepageTranslationSubtitle{A plain translation of the Saṁyutta Nikāya}

\vspace*{2em}

\halftitlepageFleuron{•}

\vspace*{2em}

\halftitlepageByline{translated and introduced by}

\vspace*{.5em}

\halftitlepageCreatorName{Bhikkhu Sujato}

\vspace*{4em}

\halftitlepageVolumeNumber{Volume 2}

\smallskip

\halftitlepageVolumeAcronym{SN 12–21}

\smallskip

\halftitlepageVolumeTranslationTitle{The Group of Linked Discourses Beginning With Causation}

\smallskip

\halftitlepageVolumeRootTitle{Nidānavaggasaṁyutta}

\vspace*{\fill}

\sclogo{0}
 \halftitlepagePublisher{SuttaCentral}

\end{center}

\newpage
%
\setstretch{1.05}

\begin{footnotesize}

\textit{Linked Discourses} is a translation of the Saṁyuttanikāya by Bhikkhu Sujato.

\medskip

Creative Commons Zero (CC0)

To the extent possible under law, Bhikkhu Sujato has waived all copyright and related or neighboring rights to \textit{Linked Discourses}.

\medskip

This work is published from Australia.

\begin{center}
\textit{This translation is an expression of an ancient spiritual text that has been passed down by the Buddhist tradition for the benefit of all sentient beings. It is dedicated to the public domain via Creative Commons Zero (CC0). You are encouraged to copy, reproduce, adapt, alter, or otherwise make use of this translation. The translator respectfully requests that any use be in accordance with the values and principles of the Buddhist community.}
\end{center}

\medskip

\begin{description}
    \item[Web publication date] 2018
    \item[This edition] 2025-01-13 01:01:43
    \item[Publication type] hardcover
    \item[Edition] ed3
    \item[Number of volumes] 5
    \item[Publication ISBN] 978-1-76132-078-1
    \item[Volume ISBN] 978-1-76132-080-4
    \item[Publication URL] \href{https://suttacentral.net/editions/sn/en/sujato}{https://suttacentral.net/editions/sn/en/sujato}
    \item[Source URL] \href{https://github.com/suttacentral/bilara-data/tree/published/translation/en/sujato/sutta/sn}{https://github.com/suttacentral/bilara-data/tree/published/translation/en/sujato/sutta/sn}
    \item[Publication number] scpub4
\end{description}

\medskip

Map of Jambudīpa is by Jonas David Mitja Lang, and is released by him under Creative Commons Zero (CC0).

\medskip

Published by SuttaCentral

\medskip

\textit{SuttaCentral,\\
c/o Alwis \& Alwis Pty Ltd\\
Kaurna Country,\\
Suite 12,\\
198 Greenhill Road,\\
Eastwood,\\
SA 5063,\\
Australia}

\end{footnotesize}

\newpage

\setlength{\parindent}{1em}%%
\tableofcontents
\newpage
\pagestyle{fancy}
%
\mainmatter%
\pagestyle{fancy}%
%
%
\addtocontents{toc}{\let\protect\contentsline\protect\nopagecontentsline}
\part*{Linked Discourses on Causation }
\addcontentsline{toc}{part}{Linked Discourses on Causation }
\markboth{}{}
\addtocontents{toc}{\let\protect\contentsline\protect\oldcontentsline}

%
\addtocontents{toc}{\let\protect\contentsline\protect\nopagecontentsline}
\chapter*{The Chapter on the Buddhas }
\addcontentsline{toc}{chapter}{\tocchapterline{The Chapter on the Buddhas }}
\addtocontents{toc}{\let\protect\contentsline\protect\oldcontentsline}

%
\section*{{\suttatitleacronym SN 12.1}{\suttatitletranslation Dependent Origination }{\suttatitleroot Paṭiccasamuppādasutta}}
\addcontentsline{toc}{section}{\tocacronym{SN 12.1} \toctranslation{Dependent Origination } \tocroot{Paṭiccasamuppādasutta}}
\markboth{Dependent Origination }{Paṭiccasamuppādasutta}
\extramarks{SN 12.1}{SN 12.1}

\scevam{So\marginnote{1.1} I have heard. }At one time the Buddha was staying near \textsanskrit{Sāvatthī} in Jeta’s Grove, \textsanskrit{Anāthapiṇḍika}’s monastery. There the Buddha addressed the mendicants, “Mendicants!” 

“Venerable\marginnote{1.5} sir,” they replied. The Buddha said this: 

“Mendicants,\marginnote{1.7} I will teach you dependent origination. Listen and apply your mind well, I will speak.” 

“Yes,\marginnote{1.9} sir,” they replied. The Buddha said this: 

“And\marginnote{2.1} what is dependent origination? Ignorance is a condition for choices. Choices are a condition for consciousness. Consciousness is a condition for name and form. Name and form are conditions for the six sense fields. The six sense fields are conditions for contact. Contact is a condition for feeling. Feeling is a condition for craving. Craving is a condition for grasping. Grasping is a condition for continued existence. Continued existence is a condition for rebirth. Rebirth is a condition for old age and death, sorrow, lamentation, pain, sadness, and distress to come to be. That is how this entire mass of suffering originates. This is called dependent origination. 

When\marginnote{3.1} ignorance fades away and ceases with nothing left over, choices cease. When choices cease, consciousness ceases. When consciousness ceases, name and form cease. When name and form cease, the six sense fields cease. When the six sense fields cease, contact ceases. When contact ceases, feeling ceases. When feeling ceases, craving ceases. When craving ceases, grasping ceases. When grasping ceases, continued existence ceases. When continued existence ceases, rebirth ceases. When rebirth ceases, old age and death, sorrow, lamentation, pain, sadness, and distress cease. That is how this entire mass of suffering ceases.” 

That\marginnote{3.13} is what the Buddha said. Satisfied, the mendicants approved what the Buddha said. 

%
\section*{{\suttatitleacronym SN 12.2}{\suttatitletranslation Analysis }{\suttatitleroot Vibhaṅgasutta}}
\addcontentsline{toc}{section}{\tocacronym{SN 12.2} \toctranslation{Analysis } \tocroot{Vibhaṅgasutta}}
\markboth{Analysis }{Vibhaṅgasutta}
\extramarks{SN 12.2}{SN 12.2}

At\marginnote{1.1} \textsanskrit{Sāvatthī}. 

“Mendicants,\marginnote{1.2} I will teach and analyze for you dependent origination. Listen and apply your mind well, I will speak.” 

“Yes,\marginnote{1.4} sir,” they replied. The Buddha said this: 

“And\marginnote{2.1} what is dependent origination? Ignorance is a condition for choices. Choices are a condition for consciousness. Consciousness is a condition for name and form. Name and form are conditions for the six sense fields. The six sense fields are conditions for contact. Contact is a condition for feeling. Feeling is a condition for craving. Craving is a condition for grasping. Grasping is a condition for continued existence. Continued existence is a condition for rebirth. Rebirth is a condition for old age and death, sorrow, lamentation, pain, sadness, and distress to come to be. That is how this entire mass of suffering originates. 

And\marginnote{3.1} what is old age and death? The old age, decrepitude, broken teeth, grey hair, wrinkly skin, diminished vitality, and failing faculties of the various sentient beings in the various orders of sentient beings. This is called old age. The passing away, passing on, disintegration, demise, mortality, death, decease, breaking up of the aggregates, and laying to rest of the corpse of the various sentient beings in the various orders of sentient beings. This is called death. Such is old age, and such is death. This is called old age and death. 

And\marginnote{4.1} what is rebirth? The rebirth, inception, conception, reincarnation, manifestation of the aggregates, and acquisition of the sense fields of the various sentient beings in the various orders of sentient beings. This is called rebirth. 

And\marginnote{5.1} what is continued existence? There are these three states of existence. Existence in the sensual realm, the realm of luminous form, and the formless realm. This is called continued existence. 

And\marginnote{6.1} what is grasping? There are these four kinds of grasping. Grasping at sensual pleasures, views, precepts and observances, and theories of a self. This is called grasping. 

And\marginnote{7.1} what is craving? There are these six classes of craving. Craving for sights, sounds, smells, tastes, touches, and ideas. This is called craving. 

And\marginnote{8.1} what is feeling? There are these six classes of feeling. Feeling born of contact through the eye, ear, nose, tongue, body, and mind. This is called feeling. 

And\marginnote{9.1} what is contact? There are these six classes of contact. Contact through the eye, ear, nose, tongue, body, and mind. This is called contact. 

And\marginnote{10.1} what are the six sense fields? The sense fields of the eye, ear, nose, tongue, body, and mind. These are called the six sense fields. 

And\marginnote{11.1} what are name and form? Feeling, perception, intention, contact, and application of mind. This is called name. The four principal states, and form derived from the four principal states. This is called form. Such is name and such is form. These are called name and form. 

And\marginnote{12.1} what is consciousness? There are these six classes of consciousness. Eye, ear, nose, tongue, body, and mind consciousness. This is called consciousness. 

And\marginnote{13.1} what are choices? There are three kinds of choices. Choices by way of body, speech, and mind. These are called choices. 

And\marginnote{14.1} what is ignorance? Not knowing about suffering, the origin of suffering, the cessation of suffering, and the practice that leads to the cessation of suffering. This is called ignorance. 

And\marginnote{15.1} so, ignorance is a condition for choices. 

Choices\marginnote{15.2} are a condition for consciousness. … That is how this entire mass of suffering originates. When ignorance fades away and ceases with nothing left over, choices cease. When choices cease, consciousness ceases. … That is how this entire mass of suffering ceases.” 

%
\section*{{\suttatitleacronym SN 12.3}{\suttatitletranslation Practice }{\suttatitleroot Paṭipadāsutta}}
\addcontentsline{toc}{section}{\tocacronym{SN 12.3} \toctranslation{Practice } \tocroot{Paṭipadāsutta}}
\markboth{Practice }{Paṭipadāsutta}
\extramarks{SN 12.3}{SN 12.3}

At\marginnote{1.1} \textsanskrit{Sāvatthī}. 

“Mendicants,\marginnote{1.2} I will teach you the wrong practice and the right practice. Listen and apply your mind well, I will speak.” 

“Yes,\marginnote{1.4} sir,” they replied. The Buddha said this: 

“And\marginnote{2.1} what’s the wrong practice? Ignorance is a condition for choices. 

Choices\marginnote{2.3} are a condition for consciousness. … That is how this entire mass of suffering originates. This is called the wrong practice. 

And\marginnote{3.1} what’s the right practice? When ignorance fades away and ceases with nothing left over, choices cease. When choices cease, consciousness ceases. … That is how this entire mass of suffering ceases. This is called the right practice.” 

%
\section*{{\suttatitleacronym SN 12.4}{\suttatitletranslation About Vipassī }{\suttatitleroot Vipassīsutta}}
\addcontentsline{toc}{section}{\tocacronym{SN 12.4} \toctranslation{About Vipassī } \tocroot{Vipassīsutta}}
\markboth{About Vipassī }{Vipassīsutta}
\extramarks{SN 12.4}{SN 12.4}

At\marginnote{1.1} \textsanskrit{Sāvatthī}. 

“Mendicants,\marginnote{1.2} \textsanskrit{Vipassī} the Blessed One, the perfected one, the fully awakened Buddha had this thought before his awakening, when he was still unawakened but intent on awakening: ‘Alas, this world has fallen into trouble. It’s born, grows old, dies, passes away, and is reborn, yet it doesn’t understand how to escape from this suffering, from old age and death. Oh, when will an escape be found from this suffering, from old age and death?’ 

Then\marginnote{2.1} \textsanskrit{Vipassī}, the one intent on awakening, thought: ‘When what exists is there old age and death? What is a condition for old age and death?’ Then, through rational application of mind, \textsanskrit{Vipassī} comprehended with wisdom: ‘When rebirth exists there’s old age and death. Rebirth is a condition for old age and death.’ 

Then\marginnote{3.1} \textsanskrit{Vipassī} thought: ‘When what exists is there rebirth? What is a condition for rebirth?’ Then, through rational application of mind, \textsanskrit{Vipassī} comprehended with wisdom: ‘When continued existence exists there’s rebirth. Continued existence is a condition for rebirth.’ 

Then\marginnote{4.1} \textsanskrit{Vipassī} thought: ‘When what exists is there continued existence? What is a condition for continued existence?’ Then, through rational application of mind, \textsanskrit{Vipassī} comprehended with wisdom: ‘When grasping exists there’s continued existence. Grasping is a condition for continued existence.’ 

Then\marginnote{5.1} \textsanskrit{Vipassī} thought: ‘When what exists is there grasping? What is a condition for grasping?’ Then, through rational application of mind, \textsanskrit{Vipassī} comprehended with wisdom: ‘When craving exists there’s grasping. Craving is a condition for grasping.’ 

Then\marginnote{6.1} \textsanskrit{Vipassī} thought: ‘When what exists is there craving? What is a condition for craving?’ Then, through rational application of mind, \textsanskrit{Vipassī} comprehended with wisdom: ‘When feeling exists there’s craving. Feeling is a condition for craving.’ 

Then\marginnote{7.1} \textsanskrit{Vipassī} thought: ‘When what exists is there feeling? What is a condition for feeling?’ Then, through rational application of mind, \textsanskrit{Vipassī} comprehended with wisdom: ‘When contact exists there’s feeling. Contact is a condition for feeling.’ 

Then\marginnote{8.1} \textsanskrit{Vipassī} thought: ‘When what exists is there contact? What is a condition for contact?’ Then, through rational application of mind, \textsanskrit{Vipassī} comprehended with wisdom: ‘When the six sense fields exist there’s contact. The six sense fields are a condition for contact.’ 

Then\marginnote{9.1} \textsanskrit{Vipassī} thought: ‘When what exists are there the six sense fields? What is a condition for the six sense fields?’ Then, through rational application of mind, \textsanskrit{Vipassī} comprehended with wisdom: ‘When name and form exist there are the six sense fields. Name and form are a condition for the six sense fields.’ 

Then\marginnote{10.1} \textsanskrit{Vipassī} thought: ‘When what exists are there name and form? What is a condition for name and form?’ Then, through rational application of mind, \textsanskrit{Vipassī} comprehended with wisdom: ‘When consciousness exists there are name and form. Consciousness is a condition for name and form.’ 

Then\marginnote{11.1} \textsanskrit{Vipassī} thought: ‘When what exists is there consciousness? What is a condition for consciousness?’ Then, through rational application of mind, \textsanskrit{Vipassī} comprehended with wisdom: ‘When choices exist there’s consciousness. Choices are a condition for consciousness.’ 

Then\marginnote{12.1} \textsanskrit{Vipassī} thought: ‘When what exists are there choices? What is a condition for choices?’ Then, through rational application of mind, \textsanskrit{Vipassī} comprehended with wisdom: ‘When ignorance exists there are choices. Ignorance is a condition for choices.’ 

And\marginnote{13.1} so, ignorance is a condition for choices. 

Choices\marginnote{13.2} are a condition for consciousness. … That is how this entire mass of suffering originates. ‘Origination, origination.’ While \textsanskrit{Vipassī} was intent on awakening, such was the vision, knowledge, wisdom, realization, and light that arose in him regarding teachings not learned before from another. 

Then\marginnote{14.1} \textsanskrit{Vipassī} thought: ‘When what doesn’t exist is there no old age and death? When what ceases do old age and death cease?’ Then, through rational application of mind, \textsanskrit{Vipassī} comprehended with wisdom: ‘When rebirth doesn’t exist there’s no old age and death. When rebirth ceases, old age and death cease.’ 

Then\marginnote{15.1} \textsanskrit{Vipassī} thought: ‘When what doesn’t exist is there no rebirth? When what ceases does rebirth cease?’ Then, through rational application of mind, \textsanskrit{Vipassī} comprehended with wisdom: ‘When continued existence doesn’t exist there’s no rebirth. When continued existence ceases, rebirth ceases.’ 

Then\marginnote{16.1} \textsanskrit{Vipassī} thought: ‘When what doesn’t exist is there no continued existence? When what ceases does continued existence cease?’ Then, through rational application of mind, \textsanskrit{Vipassī} comprehended with wisdom: ‘When grasping doesn’t exist there’s no continued existence. When grasping ceases, continued existence ceases.’ 

Then\marginnote{17.1} \textsanskrit{Vipassī} thought: ‘When what doesn’t exist is there no grasping? When what ceases does grasping cease?’ Then, through rational application of mind, \textsanskrit{Vipassī} comprehended with wisdom: ‘When craving doesn’t exist there’s no grasping. When craving ceases, grasping ceases.’ 

Then\marginnote{18.1} \textsanskrit{Vipassī} thought: ‘When what doesn’t exist is there no craving? When what ceases does craving cease?’ Then, through rational application of mind, \textsanskrit{Vipassī} comprehended with wisdom: ‘When feeling doesn’t exist there’s no craving. When feeling ceases, craving ceases.’ 

Then\marginnote{19.1} \textsanskrit{Vipassī} thought: ‘When what doesn’t exist is there no feeling? When what ceases does feeling cease?’ Then, through rational application of mind, \textsanskrit{Vipassī} comprehended with wisdom: ‘When contact doesn’t exist there’s no feeling. When contact ceases, feeling ceases.’ 

Then\marginnote{20.1} \textsanskrit{Vipassī} thought: ‘When what doesn’t exist is there no contact? When what ceases does contact cease?’ Then, through rational application of mind, \textsanskrit{Vipassī} comprehended with wisdom: ‘When the six sense fields don’t exist there’s no contact. When the six sense fields cease, contact ceases.’ 

Then\marginnote{21.1} \textsanskrit{Vipassī} thought: ‘When what doesn’t exist are there no six sense fields? When what ceases do the six sense fields cease?’ Then, through rational application of mind, \textsanskrit{Vipassī} comprehended with wisdom: ‘When name and form don’t exist there are no six sense fields. When name and form cease, the six sense fields cease.’ 

Then\marginnote{22.1} \textsanskrit{Vipassī} thought: ‘When what doesn’t exist are there no name and form? When what ceases do name and form cease?’ Then, through rational application of mind, \textsanskrit{Vipassī} comprehended with wisdom: ‘When consciousness doesn’t exist there are no name and form. When consciousness ceases, name and form cease.’ 

Then\marginnote{23.1} \textsanskrit{Vipassī} thought: ‘When what doesn’t exist is there no consciousness? When what ceases does consciousness cease?’ Then, through rational application of mind, \textsanskrit{Vipassī} comprehended with wisdom: ‘When choices don’t exist there’s no consciousness. When choices cease, consciousness ceases.’ 

Then\marginnote{24.1} \textsanskrit{Vipassī} thought: ‘When what doesn’t exist are there no choices? When what ceases do choices cease?’ Then, through rational application of mind, \textsanskrit{Vipassī} comprehended with wisdom: ‘When ignorance doesn’t exist there are no choices. When ignorance ceases, choices cease.’ 

And\marginnote{25.1} so, when ignorance ceases, choices cease. When choices cease, consciousness ceases. … That is how this entire mass of suffering ceases. ‘Cessation, cessation.’ Such was the vision, knowledge, wisdom, realization, and light that arose in \textsanskrit{Vipassī}, the one intent on awakening, regarding teachings not learned before from another.” 

\scexpansioninstructions{(Tell in full  for each of the seven Buddhas.) }

%
\section*{{\suttatitleacronym SN 12.5}{\suttatitletranslation Sikhī }{\suttatitleroot Sikhīsutta}}
\addcontentsline{toc}{section}{\tocacronym{SN 12.5} \toctranslation{Sikhī } \tocroot{Sikhīsutta}}
\markboth{Sikhī }{Sikhīsutta}
\extramarks{SN 12.5}{SN 12.5}

“\textsanskrit{Sikhī},\marginnote{1.1} the Blessed One, the perfected one, the fully awakened Buddha …” 

%
\section*{{\suttatitleacronym SN 12.6}{\suttatitletranslation Vessabhū }{\suttatitleroot Vessabhūsutta}}
\addcontentsline{toc}{section}{\tocacronym{SN 12.6} \toctranslation{Vessabhū } \tocroot{Vessabhūsutta}}
\markboth{Vessabhū }{Vessabhūsutta}
\extramarks{SN 12.6}{SN 12.6}

“\textsanskrit{Vessabhū},\marginnote{1.1} the Blessed One, the perfected one, the fully awakened Buddha …” 

%
\section*{{\suttatitleacronym SN 12.7}{\suttatitletranslation Kakusandha }{\suttatitleroot Kakusandhasutta}}
\addcontentsline{toc}{section}{\tocacronym{SN 12.7} \toctranslation{Kakusandha } \tocroot{Kakusandhasutta}}
\markboth{Kakusandha }{Kakusandhasutta}
\extramarks{SN 12.7}{SN 12.7}

“Kakusandha,\marginnote{1.1} the Blessed One, the perfected one, the fully awakened Buddha …” 

%
\section*{{\suttatitleacronym SN 12.8}{\suttatitletranslation Koṇāgamana }{\suttatitleroot Koṇāgamanasutta}}
\addcontentsline{toc}{section}{\tocacronym{SN 12.8} \toctranslation{Koṇāgamana } \tocroot{Koṇāgamanasutta}}
\markboth{Koṇāgamana }{Koṇāgamanasutta}
\extramarks{SN 12.8}{SN 12.8}

“\textsanskrit{Koṇāgamana},\marginnote{1.1} the Blessed One, the perfected one, the fully awakened Buddha …” 

%
\section*{{\suttatitleacronym SN 12.9}{\suttatitletranslation Kassapa }{\suttatitleroot Kassapasutta}}
\addcontentsline{toc}{section}{\tocacronym{SN 12.9} \toctranslation{Kassapa } \tocroot{Kassapasutta}}
\markboth{Kassapa }{Kassapasutta}
\extramarks{SN 12.9}{SN 12.9}

“Kassapa,\marginnote{1.1} the Blessed One, the perfected one, the fully awakened Buddha …” 

%
\section*{{\suttatitleacronym SN 12.10}{\suttatitletranslation Gotama }{\suttatitleroot Gotamasutta}}
\addcontentsline{toc}{section}{\tocacronym{SN 12.10} \toctranslation{Gotama } \tocroot{Gotamasutta}}
\markboth{Gotama }{Gotamasutta}
\extramarks{SN 12.10}{SN 12.10}

“Mendicants,\marginnote{1.1} before my awakening—when I was still unawakened but intent on awakening—I thought: ‘Alas, this world has fallen into trouble. It’s born, grows old, dies, passes away, and is reborn, yet it doesn’t understand how to escape from this suffering, from old age and death. Oh, when will an escape be found from this suffering, from old age and death?’ 

Then\marginnote{2.1} it occurred to me: ‘When what exists is there old age and death? What is a condition for old age and death?’ Then, through rational application of mind, I comprehended with wisdom: ‘When rebirth exists there’s old age and death. Rebirth is a condition for old age and death.’ 

Then\marginnote{3.1} it occurred to me: ‘When what exists is there rebirth? … continued existence … grasping … craving … feeling … contact … the six sense fields … name and form … consciousness … ‘When what exists are there choices? What is a condition for choices?’ Then, through rational application of mind, I comprehended with wisdom: ‘When ignorance exists there are choices. Ignorance is a condition for choices.’ 

And\marginnote{4.1} so, ignorance is a condition for choices. 

Choices\marginnote{4.2} are a condition for consciousness. … That is how this entire mass of suffering originates. ‘Origination, origination.’ Such was the vision, knowledge, wisdom, realization, and light that arose in me regarding teachings not learned before from another. 

Then\marginnote{5.1} it occurred to me: ‘When what doesn’t exist is there no old age and death? When what ceases do old age and death cease?’ Then, through rational application of mind, I comprehended with wisdom: ‘When rebirth doesn’t exist there’s no old age and death. When rebirth ceases, old age and death cease.’ 

Then\marginnote{6.1} it occurred to me: ‘When what doesn’t exist is there no rebirth? … continued existence … grasping … craving … feeling … contact … the six sense fields … name and form … consciousness … ‘When what doesn’t exist are there no choices? When what ceases do choices cease?’ Then, through rational application of mind, I comprehended with wisdom: ‘When ignorance doesn’t exist there are no choices. When ignorance ceases, choices cease.’ 

And\marginnote{7.1} so, when ignorance ceases, choices cease. When choices cease, consciousness ceases. … That is how this entire mass of suffering ceases. ‘Cessation, cessation.’ Such was the vision, knowledge, wisdom, realization, and light that arose in me regarding teachings not learned before from another.” 

%
\addtocontents{toc}{\let\protect\contentsline\protect\nopagecontentsline}
\chapter*{The Chapter on Fuel }
\addcontentsline{toc}{chapter}{\tocchapterline{The Chapter on Fuel }}
\addtocontents{toc}{\let\protect\contentsline\protect\oldcontentsline}

%
\section*{{\suttatitleacronym SN 12.11}{\suttatitletranslation Fuel }{\suttatitleroot Āhārasutta}}
\addcontentsline{toc}{section}{\tocacronym{SN 12.11} \toctranslation{Fuel } \tocroot{Āhārasutta}}
\markboth{Fuel }{Āhārasutta}
\extramarks{SN 12.11}{SN 12.11}

\scevam{So\marginnote{1.1} I have heard. }At one time the Buddha was staying near \textsanskrit{Sāvatthī} in Jeta’s Grove, \textsanskrit{Anāthapiṇḍika}’s monastery. … 

“Mendicants,\marginnote{1.3} there are these four fuels. They maintain sentient beings that have been born and help those that are about to be born. What four? Solid food, whether solid or subtle; contact is the second, mental intention the third, and consciousness the fourth. These are the four fuels that maintain sentient beings that have been born and help those that are about to be born. 

What\marginnote{2.1} is the source, origin, birthplace, and inception of these four fuels? Craving. And what is the source, origin, birthplace, and inception of craving? Feeling. And what is the source of feeling? Contact. And what is the source of contact? The six sense fields. And what is the source of the six sense fields? Name and form. And what is the source of name and form? Consciousness. And what is the source of consciousness? Choices. And what is the source of choices? Ignorance. 

And\marginnote{3.1} so, ignorance is a condition for choices. 

Choices\marginnote{3.2} are a condition for consciousness. … That is how this entire mass of suffering originates. When ignorance fades away and ceases with nothing left over, choices cease. When choices cease, consciousness ceases. … That is how this entire mass of suffering ceases.” 

%
\section*{{\suttatitleacronym SN 12.12}{\suttatitletranslation Phagguna of the Top-Knot }{\suttatitleroot Moḷiyaphaggunasutta}}
\addcontentsline{toc}{section}{\tocacronym{SN 12.12} \toctranslation{Phagguna of the Top-Knot } \tocroot{Moḷiyaphaggunasutta}}
\markboth{Phagguna of the Top-Knot }{Moḷiyaphaggunasutta}
\extramarks{SN 12.12}{SN 12.12}

At\marginnote{1.1} \textsanskrit{Sāvatthī}. 

“Mendicants,\marginnote{1.2} there are these four fuels. They maintain sentient beings that have been born and help those that are about to be born. What four? Solid food, whether solid or subtle; contact is the second, mental intention the third, and consciousness the fourth. These are the four fuels that maintain sentient beings that have been born and help those that are about to be born.” 

When\marginnote{2.1} he said this, Venerable Phagguna of the Top-Knot said to the Buddha, “But sir, who consumes the fuel for consciousness?” 

“That’s\marginnote{2.3} not a fitting question,” said the Buddha. 

“I\marginnote{2.4} don’t speak of one who consumes. If I were to speak of one who consumes, then it would be fitting to ask who consumes. But I don’t speak like that. Hence it would be fitting to ask: ‘Consciousness is a fuel for what?’ And a fitting answer to this would be: ‘Consciousness is a fuel that conditions rebirth into a new state of existence in the future. When that which has been reborn is present, there are the six sense fields. The six sense fields are a condition for contact.’” 

“But\marginnote{3.1} sir, who contacts?” 

“That’s\marginnote{3.2} not a fitting question,” said the Buddha. 

“I\marginnote{3.3} don’t speak of one who contacts. If I were to speak of one who contacts, then it would be fitting to ask who contacts. But I don’t speak like that. Hence it would be fitting to ask: ‘What is a condition for contact?’ And a fitting answer to this would be: ‘The six sense fields are a condition for contact. Contact is a condition for feeling.’” 

“But\marginnote{4.1} sir, who feels?” 

“That’s\marginnote{4.2} not a fitting question,” said the Buddha. 

“I\marginnote{4.3} don’t speak of one who feels. If I were to speak of one who feels, then it would be fitting to ask who feels. But I don’t speak like that. Hence it would be fitting to ask: ‘What is a condition for feeling?’ And a fitting answer to this would be: ‘Contact is a condition for feeling. Feeling is a condition for craving.’” 

“But\marginnote{5.1} sir, who craves?” 

“That’s\marginnote{5.2} not a fitting question,” said the Buddha. 

“I\marginnote{5.3} don’t speak of one who craves. If I were to speak of one who craves, then it would be fitting to ask who craves. But I don’t speak like that. Hence it would be fitting to ask: ‘What is a condition for craving?’ And a fitting answer to this would be: ‘Feeling is a condition for craving. Craving is a condition for grasping.’” 

“But\marginnote{6.1} sir, who grasps?” 

“That’s\marginnote{6.2} not a fitting question,” said the Buddha. 

“I\marginnote{6.3} don’t speak of one who grasps. If I were to speak of one who grasps, then it would be fitting to ask who grasps. But I don’t speak like that. Hence it would be fitting to ask: ‘What is a condition for grasping?’ And a fitting answer to this would be: ‘Craving is a condition for grasping. Grasping is a condition for continued existence.’ … That is how this entire mass of suffering originates. 

When\marginnote{7.1} the six fields of contact fade away and cease with nothing left over, contact ceases. When contact ceases, feeling ceases. When feeling ceases, craving ceases. When craving ceases, grasping ceases. When grasping ceases, continued existence ceases. When continued existence ceases, rebirth ceases. When rebirth ceases, old age and death, sorrow, lamentation, pain, sadness, and distress cease. That is how this entire mass of suffering ceases.” 

%
\section*{{\suttatitleacronym SN 12.13}{\suttatitletranslation Ascetics and Brahmins }{\suttatitleroot Samaṇabrāhmaṇasutta}}
\addcontentsline{toc}{section}{\tocacronym{SN 12.13} \toctranslation{Ascetics and Brahmins } \tocroot{Samaṇabrāhmaṇasutta}}
\markboth{Ascetics and Brahmins }{Samaṇabrāhmaṇasutta}
\extramarks{SN 12.13}{SN 12.13}

At\marginnote{1.1} \textsanskrit{Sāvatthī}. 

“Mendicants,\marginnote{1.2} there are ascetics and brahmins who don’t understand old age and death, their origin, their cessation, and the practice that leads to their cessation. They don’t understand rebirth … continued existence … grasping … craving … feeling … contact … the six sense fields … name and form … consciousness … They don’t understand choices, their origin, their cessation, and the practice that leads to their cessation. I don’t deem them as true ascetics and brahmins. Those venerables don’t realize the goal of life as an ascetic or brahmin, and don’t live having realized it with their own insight. 

There\marginnote{2.1} are ascetics and brahmins who do understand old age and death, their origin, their cessation, and the practice that leads to their cessation. They understand rebirth … continued existence … grasping … craving … feeling … contact … the six sense fields … name and form … consciousness … They understand choices, their origin, their cessation, and the practice that leads to their cessation. I deem them as true ascetics and brahmins. Those venerables realize the goal of life as an ascetic or brahmin, and live having realized it with their own insight.” 

%
\section*{{\suttatitleacronym SN 12.14}{\suttatitletranslation Ascetics and Brahmins (2nd) }{\suttatitleroot Dutiyasamaṇabrāhmaṇasutta}}
\addcontentsline{toc}{section}{\tocacronym{SN 12.14} \toctranslation{Ascetics and Brahmins (2nd) } \tocroot{Dutiyasamaṇabrāhmaṇasutta}}
\markboth{Ascetics and Brahmins (2nd) }{Dutiyasamaṇabrāhmaṇasutta}
\extramarks{SN 12.14}{SN 12.14}

At\marginnote{1.1} \textsanskrit{Sāvatthī}. 

“Mendicants,\marginnote{1.2} there are ascetics and brahmins who don’t understand these things, their origin, their cessation, and the practice that leads to their cessation. What things don’t they understand? 

They\marginnote{2.1} don’t understand old age and death, their origin, their cessation, and the practice that leads to their cessation. They don’t understand rebirth … continued existence … grasping … craving … feeling … contact … the six sense fields … name and form … consciousness … They don’t understand choices, their origin, their cessation, and the practice that leads to their cessation. They don’t understand these things, their origin, their cessation, and the practice that leads to their cessation. I don’t deem them as true ascetics and brahmins. Those venerables don’t realize the goal of life as an ascetic or brahmin, and don’t live having realized it with their own insight. 

There\marginnote{3.1} are ascetics and brahmins who do understand these things, their origin, their cessation, and the practice that leads to their cessation. What things do they understand? 

They\marginnote{4.1} understand old age and death, their origin, their cessation, and the practice that leads to their cessation. They understand rebirth … continued existence … grasping … craving … feeling … contact … the six sense fields … name and form … consciousness … They understand choices, their origin, their cessation, and the practice that leads to their cessation. They understand these things, their origin, their cessation, and the practice that leads to their cessation. I deem them as true ascetics and brahmins. Those venerables realize the goal of life as an ascetic or brahmin, and live having realized it with their own insight.” 

%
\section*{{\suttatitleacronym SN 12.15}{\suttatitletranslation Kaccānagotta }{\suttatitleroot Kaccānagottasutta}}
\addcontentsline{toc}{section}{\tocacronym{SN 12.15} \toctranslation{Kaccānagotta } \tocroot{Kaccānagottasutta}}
\markboth{Kaccānagotta }{Kaccānagottasutta}
\extramarks{SN 12.15}{SN 12.15}

At\marginnote{1.1} \textsanskrit{Sāvatthī}. 

Then\marginnote{1.2} Venerable \textsanskrit{Kaccānagotta} went up to the Buddha, bowed, sat down to one side, and said to him: 

“Sir,\marginnote{1.3} they speak of this thing called ‘right view’. How is right view defined?” 

“\textsanskrit{Kaccāna},\marginnote{2.1} this world mostly relies on the dual notions of existence and non-existence. 

But\marginnote{2.2} when you truly see the origin of the world with right understanding, the concept of non-existence regarding the world does not occur. And when you truly see the cessation of the world with right understanding, the concept of existence regarding the world does not occur. 

The\marginnote{2.4} world is for the most part shackled by attraction, grasping, and insisting. 

But\marginnote{2.5} if—when it comes to this attraction, grasping, mental fixation, insistence, and underlying tendency—you don’t get attracted, grasp, and commit to the thought, ‘my self’, you’ll have no doubt or uncertainty that what arises is just suffering arising, and what ceases is just suffering ceasing. Your knowledge about this is independent of others. 

This\marginnote{2.7} is how right view is defined. 

‘All\marginnote{3.1} exists’: this is one extreme. 

‘All\marginnote{3.2} does not exist’: this is the second extreme. 

Avoiding\marginnote{3.3} these two extremes, the Realized One teaches by the middle way: 

‘Ignorance\marginnote{3.4} is a condition for choices. Choices are a condition for consciousness. … That is how this entire mass of suffering originates. 

When\marginnote{3.7} ignorance fades away and ceases with nothing left over, choices cease. When choices cease, consciousness ceases. … That is how this entire mass of suffering ceases.’” 

%
\section*{{\suttatitleacronym SN 12.16}{\suttatitletranslation A Dhamma Speaker }{\suttatitleroot Dhammakathikasutta}}
\addcontentsline{toc}{section}{\tocacronym{SN 12.16} \toctranslation{A Dhamma Speaker } \tocroot{Dhammakathikasutta}}
\markboth{A Dhamma Speaker }{Dhammakathikasutta}
\extramarks{SN 12.16}{SN 12.16}

At\marginnote{1.1} \textsanskrit{Sāvatthī}. 

Then\marginnote{1.2} a mendicant went up to the Buddha, bowed, sat down to one side, and said to him: 

“Sir,\marginnote{1.3} they speak of a ‘Dhamma speaker’. How is a Dhamma speaker defined?” 

“If\marginnote{2.1} a mendicant teaches Dhamma for disillusionment, dispassion, and cessation regarding old age and death, they’re qualified to be called a ‘mendicant who speaks on Dhamma’. If they practice for disillusionment, dispassion, and cessation regarding old age and death, they’re qualified to be called a ‘mendicant who practices in line with the teaching’. If they’re freed by not grasping, by disillusionment, dispassion, and cessation regarding old age and death, they’re qualified to be called a ‘mendicant who has attained extinguishment in this very life’. 

If\marginnote{3.1} a mendicant teaches Dhamma for disillusionment regarding rebirth … continued existence … grasping … craving … feeling … contact … the six sense fields … name and form … consciousness … choices … If a mendicant teaches Dhamma for disillusionment, dispassion, and cessation regarding ignorance, they’re qualified to be called a ‘mendicant who speaks on Dhamma’. If they practice for disillusionment, dispassion, and cessation regarding ignorance, they’re qualified to be called a ‘mendicant who practices in line with the teaching’. If they’re freed by not grasping, by disillusionment, dispassion, and cessation regarding ignorance, they’re qualified to be called a ‘mendicant who has attained extinguishment in this very life’.” 

%
\section*{{\suttatitleacronym SN 12.17}{\suttatitletranslation With Kassapa, the Naked Ascetic }{\suttatitleroot Acelakassapasutta}}
\addcontentsline{toc}{section}{\tocacronym{SN 12.17} \toctranslation{With Kassapa, the Naked Ascetic } \tocroot{Acelakassapasutta}}
\markboth{With Kassapa, the Naked Ascetic }{Acelakassapasutta}
\extramarks{SN 12.17}{SN 12.17}

\scevam{So\marginnote{1.1} I have heard. }At one time the Buddha was staying near \textsanskrit{Rājagaha}, in the Bamboo Grove, the squirrels’ feeding ground. 

Then\marginnote{1.3} the Buddha robed up in the morning and, taking his bowl and robe, entered \textsanskrit{Rājagaha} for alms. The naked ascetic Kassapa saw the Buddha coming off in the distance. He went up to the Buddha, and exchanged greetings with him. 

When\marginnote{1.6} the greetings and polite conversation were over, he stood to one side and said to the Buddha, 

“I’d\marginnote{1.7} like to ask Mister Gotama about a certain point, if you’d take the time to answer.” 

“Kassapa,\marginnote{1.8} it’s the wrong time for questions. We’ve entered an inhabited area.” 

A\marginnote{2.1} second time, and a third time, Kassapa spoke to the Buddha and the Buddha replied. When this was said, Kassapa said to the Buddha, 

“I\marginnote{2.8} don’t want to ask much.” 

“Ask\marginnote{2.9} what you wish, Kassapa.” 

“Well,\marginnote{3.1} Mister Gotama, is suffering made by oneself?” 

“Not\marginnote{3.2} so, Kassapa,” said the Buddha. 

“Then\marginnote{3.3} is suffering made by another?” 

“Not\marginnote{3.4} so, Kassapa,” said the Buddha. 

“Well,\marginnote{3.5} is suffering made by both oneself and another?” 

“Not\marginnote{3.6} so, Kassapa,” said the Buddha. 

“Then\marginnote{3.7} does suffering arise by chance, not made by oneself or another?” 

“Not\marginnote{3.8} so, Kassapa,” said the Buddha. 

“Well,\marginnote{3.9} is there no such thing as suffering?” 

“It’s\marginnote{3.10} not that there’s no such thing as suffering. Suffering is real.” 

“Then\marginnote{3.12} does Mister Gotama not know or see suffering?” 

“It’s\marginnote{3.13} not that I don’t know or see suffering. I do know suffering, I do see suffering.” 

“Mister\marginnote{4.1} Gotama, when asked these questions, you say ‘not so’. Yet you say that there is such a thing as suffering. And you say that you do know suffering, and you do see suffering. Sir, explain suffering to me! Teach me about suffering!” 

“Suppose\marginnote{5.1} that the person who does the deed experiences the result. Then for one who has existed since the beginning, suffering is made by oneself. This statement leans toward eternalism. Suppose that one person does the deed and another experiences the result. Then for one stricken by feeling, suffering is made by another. This statement leans toward annihilationism. Avoiding these two extremes, the Realized One teaches by the middle way: ‘Ignorance is a condition for choices. 

Choices\marginnote{5.5} are a condition for consciousness. … That is how this entire mass of suffering originates. When ignorance fades away and ceases with nothing left over, choices cease. When choices cease, consciousness ceases. … That is how this entire mass of suffering ceases.’” 

When\marginnote{6.1} this was said, Kassapa said to the Buddha, “Excellent, sir! Excellent! As if he were righting the overturned, or revealing the hidden, or pointing out the path to the lost, or lighting a lamp in the dark so people with clear eyes can see what’s there, the Buddha has made the teaching clear in many ways. I go for refuge to the Buddha, to the teaching, and to the mendicant \textsanskrit{Saṅgha}. Sir, may I receive the going forth, the ordination in the Buddha’s presence?” 

“Kassapa,\marginnote{7.1} if someone formerly ordained in another sect wishes to take the going forth, the ordination in this teaching and training, they must spend four months on probation. When four months have passed, if the mendicants are satisfied, they’ll give the going forth, the ordination into monkhood. However, I have recognized individual differences.” 

“Sir,\marginnote{8.1} if four months probation are required in such a case, I’ll spend four years on probation. When four years have passed, if the mendicants are satisfied, let them give me the going forth, the ordination into monkhood.” 

And\marginnote{9.1} the naked ascetic Kassapa received the going forth, the ordination in the Buddha’s presence. Not long after his ordination, Venerable Kassapa, living alone, withdrawn, diligent, keen, and resolute, soon realized the supreme culmination of the spiritual path in this very life. He lived having achieved with his own insight the goal for which gentlemen rightly go forth from the lay life to homelessness. 

He\marginnote{9.3} understood: “Rebirth is ended; the spiritual journey has been completed; what had to be done has been done; there is nothing further for this place.” And Venerable Kassapa became one of the perfected. 

%
\section*{{\suttatitleacronym SN 12.18}{\suttatitletranslation With Timbaruka }{\suttatitleroot Timbarukasutta}}
\addcontentsline{toc}{section}{\tocacronym{SN 12.18} \toctranslation{With Timbaruka } \tocroot{Timbarukasutta}}
\markboth{With Timbaruka }{Timbarukasutta}
\extramarks{SN 12.18}{SN 12.18}

At\marginnote{1.1} \textsanskrit{Sāvatthī}. 

Then\marginnote{1.2} the wanderer Timbaruka went up to the Buddha, and exchanged greetings with him. When the greetings and polite conversation were over, he sat down to one side and said to the Buddha: 

“Well,\marginnote{2.1} Mister Gotama, are pleasure and pain made by oneself?” 

“Not\marginnote{2.2} so, Timbaruka,” said the Buddha. 

“Then\marginnote{2.3} are pleasure and pain made by another?” 

“Not\marginnote{2.4} so, Timbaruka,” said the Buddha. 

“Well,\marginnote{2.5} are pleasure and pain made by both oneself and another?” 

“Not\marginnote{2.6} so, Timbaruka,” said the Buddha. 

“Then\marginnote{2.7} do pleasure and pain arise by chance, not made by oneself or another?” 

“Not\marginnote{2.8} so, Timbaruka,” said the Buddha. 

“Well,\marginnote{2.9} is there no such thing as pleasure and pain?” 

“It’s\marginnote{2.10} not that there’s no such thing as pleasure and pain. Pleasure and pain are real.” 

“Then\marginnote{2.12} does Mister Gotama not know or see suffering?” 

“It’s\marginnote{2.13} not that I don’t know or see pleasure and pain. I do know pleasure and pain, I do see pleasure and pain.” 

“Mister\marginnote{3.1} Gotama, when asked these questions, you say ‘not so’. Yet you say that there is such a thing as pleasure and pain. And you say that you do know pleasure and pain, and you do see pleasure and pain. Sir, explain pleasure and pain to me! Teach me about pleasure and pain!” 

“Suppose\marginnote{4.1} that the feeling and the one who feels it are the same thing. Then for one who has existed since the beginning, pleasure and pain is made by oneself. I don’t say this. Suppose that the feeling is one thing and the one who feels it is another. Then for one stricken by feeling, pleasure and pain is made by another. I don’t say this. Avoiding these two extremes, the Realized One teaches by the middle way: ‘Ignorance is a condition for choices. 

Choices\marginnote{4.5} are a condition for consciousness. … That is how this entire mass of suffering originates. When ignorance fades away and ceases with nothing left over, choices cease. When choices cease, consciousness ceases. … That is how this entire mass of suffering ceases.’” 

When\marginnote{5.1} he said this, the wanderer Timbaruka said to the Buddha, “Excellent, Mister Gotama! Excellent! … I go for refuge to Mister Gotama, to the teaching, and to the mendicant \textsanskrit{Saṅgha}. From this day forth, may Mister Gotama remember me as a lay follower who has gone for refuge for life.” 

%
\section*{{\suttatitleacronym SN 12.19}{\suttatitletranslation The Astute and the Foolish }{\suttatitleroot Bālapaṇḍitasutta}}
\addcontentsline{toc}{section}{\tocacronym{SN 12.19} \toctranslation{The Astute and the Foolish } \tocroot{Bālapaṇḍitasutta}}
\markboth{The Astute and the Foolish }{Bālapaṇḍitasutta}
\extramarks{SN 12.19}{SN 12.19}

At\marginnote{1.1} \textsanskrit{Sāvatthī}. 

“Mendicants,\marginnote{1.2} for a fool shrouded by ignorance and fettered by craving, this body has been produced. So there is the duality of this body and external name and form. Contact depends on this duality. When contacted through one or other of the six sense fields, the fool experiences pleasure and pain. 

For\marginnote{2.1} an astute person shrouded by ignorance and fettered by craving, this body has been produced. So there is the duality of this body and external name and form. Contact depends on this duality. When contacted through one or other of the six sense fields, the astute person experiences pleasure and pain. 

What,\marginnote{3.1} then, is the difference between the foolish and the astute?” 

“Our\marginnote{3.2} teachings are rooted in the Buddha. He is our guide and our refuge. Sir, may the Buddha himself please clarify the meaning of this. The mendicants will listen and remember it.” 

“Well\marginnote{4.1} then, mendicants, listen and apply your mind well, I will speak.” 

“Yes,\marginnote{4.2} sir,” they replied. The Buddha said this: 

“For\marginnote{5.1} a fool shrouded by ignorance and fettered by craving, this body has been produced. But the fool has not given up that ignorance or finished that craving. Why is that? The fool has not completed the spiritual journey for the complete ending of suffering. Therefore, when their body breaks up, the fool is reborn in another body. When reborn in another body, they’re not freed from rebirth, old age, and death, from sorrow, lamentation, pain, sadness, and distress. They’re not freed from suffering, I say. 

For\marginnote{6.1} an astute person shrouded by ignorance and fettered by craving, this body has been produced. But the astute person has given up that ignorance and finished that craving. Why is that? The astute person has completed the spiritual journey for the complete ending of suffering. Therefore, when their body breaks up, the astute person is not reborn in another body. Not being reborn in another body, they’re freed from rebirth, old age, and death, from sorrow, lamentation, pain, sadness, and distress. They’re freed from suffering, I say. This is the difference here between the foolish and the astute, that is, leading the spiritual life.” 

%
\section*{{\suttatitleacronym SN 12.20}{\suttatitletranslation Conditions }{\suttatitleroot Paccayasutta}}
\addcontentsline{toc}{section}{\tocacronym{SN 12.20} \toctranslation{Conditions } \tocroot{Paccayasutta}}
\markboth{Conditions }{Paccayasutta}
\extramarks{SN 12.20}{SN 12.20}

At\marginnote{1.1} \textsanskrit{Sāvatthī}. 

“Mendicants,\marginnote{1.2} I will teach you dependent origination and dependently originated phenomena. Listen and apply your mind well, I will speak.” 

“Yes,\marginnote{1.4} sir,” they replied. The Buddha said this: 

“And\marginnote{2.1} what is dependent origination? Rebirth is a condition for old age and death. Whether Realized Ones arise or not, this law of nature persists, this regularity of natural principles, this invariance of natural principles, specific conditionality. A Realized One understands this and comprehends it, then he explains, teaches, asserts, establishes, clarifies, analyzes, and reveals it. ‘Look,’ he says, ‘Rebirth is a condition for old age and death.’ 

Continued\marginnote{3.1} existence is a condition for rebirth … Grasping is a condition for continued existence … Craving is a condition for grasping … 

Feeling\marginnote{3.4} is a condition for craving … Contact is a condition for feeling … The six sense fields are a condition for contact … Name and form are conditions for the six sense fields … 

Consciousness\marginnote{3.8} is a condition for name and form … 

Choices\marginnote{3.9} are a condition for consciousness … Ignorance is a condition for choices. Whether Realized Ones arise or not, this law of nature persists, this regularity of natural principles, this invariance of natural principles, specific conditionality. A Realized One understands this and comprehends it, then he explains, teaches, asserts, establishes, clarifies, analyzes, and reveals it. ‘Look,’ he says, ‘Ignorance is a condition for choices.’ So the fact that this is real, not unreal, not otherwise; the specific conditionality of it: this is called dependent origination. 

And\marginnote{4.1} what are the dependently originated phenomena? Old age and death are impermanent, conditioned, dependently originated, liable to end, vanish, fade away, and cease. Rebirth … Continued existence … Grasping … Craving … 

Feeling\marginnote{4.7} … Contact … The six sense fields … Name and form … 

Consciousness\marginnote{4.11} … 

Choices\marginnote{4.12} … Ignorance is impermanent, conditioned, dependently originated, liable to end, vanish, fade away, and cease. These are called the dependently originated phenomena. 

When\marginnote{5.1} a noble disciple has clearly seen with right wisdom this dependent origination and these dependently originated phenomena as they are, it is quite impossible for them to turn back to the past, thinking: ‘Did I exist in the past? Did I not exist in the past? What was I in the past? How was I in the past? After being what, what did I become in the past?’ Or to turn forward to the future, thinking: ‘Will I exist in the future? Will I not exist in the future? What will I be in the future? How will I be in the future? After being what, what will I become in the future?’ Or to be undecided about the present, thinking: ‘Am I? Am I not? What am I? How am I? This sentient being—where did it come from? And where will it go?’ Why is that? Because that noble disciple has clearly seen with right wisdom this dependent origination and these dependently originated phenomena as they are.” 

%
\addtocontents{toc}{\let\protect\contentsline\protect\nopagecontentsline}
\chapter*{The Chapter on the Ten Powers }
\addcontentsline{toc}{chapter}{\tocchapterline{The Chapter on the Ten Powers }}
\addtocontents{toc}{\let\protect\contentsline\protect\oldcontentsline}

%
\section*{{\suttatitleacronym SN 12.21}{\suttatitletranslation The Ten Powers }{\suttatitleroot Dasabalasutta}}
\addcontentsline{toc}{section}{\tocacronym{SN 12.21} \toctranslation{The Ten Powers } \tocroot{Dasabalasutta}}
\markboth{The Ten Powers }{Dasabalasutta}
\extramarks{SN 12.21}{SN 12.21}

At\marginnote{1.1} \textsanskrit{Sāvatthī}. 

“Mendicants,\marginnote{1.2} a Realized One has ten powers and four kinds of self-assurance. With these he claims the bull’s place, roars his lion’s roar in the assemblies, and turns the divine wheel. 

Such\marginnote{1.3} is form, such is the origin of form, such is the ending of form. Such is feeling, such is the origin of feeling, such is the ending of feeling. Such is perception, such is the origin of perception, such is the ending of perception. Such are choices, such is the origin of choices, such is the ending of choices. Such is consciousness, such is the origin of consciousness, such is the ending of consciousness. 

When\marginnote{1.8} this exists, that is; due to the arising of this, that arises. When this doesn’t exist, that is not; due to the cessation of this, that ceases. That is: 

Ignorance\marginnote{1.10} is a condition for choices. 

Choices\marginnote{1.11} are a condition for consciousness. … That is how this entire mass of suffering originates. When ignorance fades away and ceases with nothing left over, choices cease. When choices cease, consciousness ceases. … That is how this entire mass of suffering ceases.” 

%
\section*{{\suttatitleacronym SN 12.22}{\suttatitletranslation The Ten Powers (2nd) }{\suttatitleroot Dutiyadasabalasutta}}
\addcontentsline{toc}{section}{\tocacronym{SN 12.22} \toctranslation{The Ten Powers (2nd) } \tocroot{Dutiyadasabalasutta}}
\markboth{The Ten Powers (2nd) }{Dutiyadasabalasutta}
\extramarks{SN 12.22}{SN 12.22}

At\marginnote{1.1} \textsanskrit{Sāvatthī}. 

“Mendicants,\marginnote{1.2} a Realized One has ten powers and four kinds of self-assurance. With these he claims the bull’s place, roars his lion’s roar in the assemblies, and turns the divine wheel. 

Such\marginnote{1.3} is form, such is the origin of form, such is the ending of form. Such is feeling, such is the origin of feeling, such is the ending of feeling. Such is perception, such is the origin of perception, such is the ending of perception. Such are choices, such is the origin of choices, such is the ending of choices. Such is consciousness, such is the origin of consciousness, such is the ending of consciousness. 

When\marginnote{1.8} this exists, that is; due to the arising of this, that arises. When this doesn’t exist, that is not; due to the cessation of this, that ceases. That is: 

Ignorance\marginnote{1.10} is a condition for choices. 

Choices\marginnote{1.11} are a condition for consciousness. … That is how this entire mass of suffering originates. When ignorance fades away and ceases with nothing left over, choices cease. When choices cease, consciousness ceases. … That is how this entire mass of suffering ceases. 

So\marginnote{2.1} the teaching has been well explained by me, made clear, opened, illuminated, and stripped of patchwork. Just this much is quite enough for a gentleman who has gone forth out of faith to rouse up his energy. ‘Gladly, let only skin, sinews, and tendons remain! Let the flesh and blood waste away in my body! I will not stop trying until I have achieved what is possible by human strength, energy, and vigor.’ 

A\marginnote{3.1} lazy person lives in suffering, mixed up with bad, unskillful qualities, and ruins a great deal of their own good. An energetic person lives happily, secluded from bad, unskillful qualities, and fulfills a great deal of their own good. 

The\marginnote{3.3} best isn’t reached by the worst. The best is reached by the best. This spiritual life is the cream, mendicants, and the Teacher is before you. 

So\marginnote{3.6} you should rouse up energy for attaining the unattained, achieving the unachieved, and realizing the unrealized, thinking: ‘In this way our going forth will not be wasted, but will be fruitful and fertile. And our use of robes, almsfood, lodgings, and medicines and supplies for the sick shall be of great fruit and benefit for those who offered them.’ That’s how you should train. 

Considering\marginnote{3.10} what is good for yourself, mendicants, is quite enough for you to persist with diligence. Considering what is good for others is quite enough for you to persist with diligence. Considering what is good for both is quite enough for you to persist with diligence.” 

%
\section*{{\suttatitleacronym SN 12.23}{\suttatitletranslation Vital Conditions }{\suttatitleroot Upanisasutta}}
\addcontentsline{toc}{section}{\tocacronym{SN 12.23} \toctranslation{Vital Conditions } \tocroot{Upanisasutta}}
\markboth{Vital Conditions }{Upanisasutta}
\extramarks{SN 12.23}{SN 12.23}

At\marginnote{1.1} \textsanskrit{Sāvatthī}. 

“Mendicants,\marginnote{1.2} I say that the ending of defilements is for one who knows and sees, not for one who does not know or see. For one who knows and sees what? ‘Such is form, such is the origin of form, such is the ending of form. Such is feeling … Such is perception … Such are choices … Such is consciousness, such is the origin of consciousness, such is the ending of consciousness.’ The ending of the defilements is for one who knows and sees this. 

I\marginnote{2.1} say that this knowledge of ending has a vital condition, it doesn’t lack a vital condition. And what is it? You should say: ‘Freedom.’ I say that freedom has a vital condition, it doesn’t lack a vital condition. And what is it? You should say: ‘Dispassion.’ I say that dispassion has a vital condition. And what is it? You should say: ‘Disillusionment.’ I say that disillusionment has a vital condition. And what is it? You should say: ‘Truly knowing and seeing.’ I say that truly knowing and seeing has a vital condition. And what is it? You should say: ‘Immersion.’ I say that immersion has a vital condition. 

And\marginnote{3.1} what is it? You should say: ‘Bliss.’ I say that bliss has a vital condition. And what is it? You should say: ‘Tranquility.’ I say that tranquility has a vital condition. And what is it? You should say: ‘Rapture.’ I say that rapture has a vital condition. And what is it? You should say: ‘Joy.’ I say that joy has a vital condition. And what is it? You should say: ‘Faith.’ I say that faith has a vital condition. 

And\marginnote{4.1} what is it? You should say: ‘Suffering.’ I say that suffering has a vital condition. And what is it? You should say: ‘Rebirth.’ I say that rebirth has a vital condition. And what is it? You should say: ‘Continued existence.’ I say that continued existence has a vital condition. And what is it? You should say: ‘Grasping.’ I say that grasping has a vital condition. And what is it? You should say: ‘Craving.’ I say that craving has a vital condition. 

And\marginnote{5.1} what is it? You should say: ‘Feeling.’ … You should say: ‘Contact.’ … You should say: ‘The six sense fields.’ … You should say: ‘Name and form.’ … You should say: ‘Consciousness.’ … You should say: ‘Choices.’ … I say that choices have a vital condition, they don’t lack a vital condition. And what is the vital condition for choices? You should say: ‘Ignorance.’ 

So\marginnote{6.1} ignorance is a vital condition for choices. Choices are a vital condition for consciousness. Consciousness is a vital condition for name and form. Name and form are vital conditions for the six sense fields. The six sense fields are vital conditions for contact. Contact is a vital condition for feeling. Feeling is a vital condition for craving. Craving is a vital condition for grasping. Grasping is a vital condition for continued existence. Continued existence is a vital condition for rebirth. Rebirth is a vital condition for suffering. Suffering is a vital condition for faith. Faith is a vital condition for joy. Joy is a vital condition for rapture. Rapture is a vital condition for tranquility. Tranquility is a vital condition for bliss. Bliss is a vital condition for immersion. Immersion is a vital condition for truly knowing and seeing. Truly knowing and seeing is a vital condition for disillusionment. Disillusionment is a vital condition for dispassion. Dispassion is a vital condition for freedom. Freedom is a vital condition for the knowledge of ending. 

It’s\marginnote{7.1} like when the heavens rain heavily on a mountain top, and the water flows downhill to fill the hollows, crevices, and creeks. As they become full, they fill up the pools. The pools fill up the lakes, the lakes fill up the streams, and the streams fill up the rivers. And as the rivers become full, they fill up the ocean. 

In\marginnote{8.1} the same way, ignorance is a vital condition for choices. … Freedom is a vital condition for the knowledge of ending.” 

%
\section*{{\suttatitleacronym SN 12.24}{\suttatitletranslation Followers of Other Religions }{\suttatitleroot Aññatitthiyasutta}}
\addcontentsline{toc}{section}{\tocacronym{SN 12.24} \toctranslation{Followers of Other Religions } \tocroot{Aññatitthiyasutta}}
\markboth{Followers of Other Religions }{Aññatitthiyasutta}
\extramarks{SN 12.24}{SN 12.24}

Near\marginnote{1.1} \textsanskrit{Rājagaha}, in the Bamboo Grove. Then Venerable \textsanskrit{Sāriputta} robed up in the morning and, taking his bowl and robe, entered \textsanskrit{Rājagaha} for alms. Then it occurred to him, “It’s too early to wander for alms in \textsanskrit{Rājagaha}. Why don’t I visit the monastery of the wanderers of other religions?” 

Then\marginnote{2.1} he went to the monastery of the wanderers of other religions and exchanged greetings with the wanderers there. When the greetings and polite conversation were over, he sat down to one side. The wanderers said to him: 

“Reverend\marginnote{3.1} \textsanskrit{Sāriputta}, there are ascetics and brahmins who teach the efficacy of deeds. Some of them declare that suffering is made by oneself. Some of them declare that suffering is made by another. Some of them declare that suffering is made by both oneself and another. Some of them declare that suffering arises by chance, not made by oneself or another. What does the ascetic Gotama say about this? How does he explain it? How should we answer so as to repeat what the ascetic Gotama has said, and not misrepresent him with an untruth? How should we explain in line with his teaching, with no legitimate grounds for rebuttal and criticism?” 

“Reverends,\marginnote{4.1} the Buddha said that suffering is dependently originated. Dependent on what? Dependent on contact. If you said this you would repeat what the Buddha has said, and not misrepresent him with an untruth. You would explain in line with his teaching, and there would be no legitimate grounds for rebuttal and criticism. 

Consider\marginnote{5.1} the ascetics and brahmins who teach the efficacy of deeds. In the case of those who declare that suffering is made by oneself, that’s conditioned by contact. In the case of those who declare that suffering is made by another, that’s also conditioned by contact. In the case of those who declare that suffering is made by oneself and another, that’s also conditioned by contact. In the case of those who declare that suffering arises by chance, not made by oneself or another, that’s also conditioned by contact. 

Consider\marginnote{6.1} the ascetics and brahmins who teach the efficacy of deeds. In the case of those who declare that suffering is made by oneself, it’s impossible that they will experience that without contact. In the case of those who declare that suffering is made by another, it’s impossible that they will experience that without contact. In the case of those who declare that suffering is made by oneself and another, it’s impossible that they will experience that without contact. In the case of those who declare that suffering arises by chance, not made by oneself or another, it’s impossible that they will experience that without contact.” 

Venerable\marginnote{7.1} Ānanda heard this discussion between Venerable \textsanskrit{Sāriputta} and those wanderers of other religions. Then Ānanda wandered for alms in \textsanskrit{Rājagaha}. After the meal, on his return from almsround, he went to the Buddha, bowed, sat down to one side, and informed the Buddha of all they had discussed. 

“Good,\marginnote{8.1} good, Ānanda! It’s just as \textsanskrit{Sāriputta} has so rightly explained. I have said that suffering is dependently originated. Dependent on what? Dependent on contact. Saying this you would repeat what I have said, and not misrepresent me with an untruth. You would explain in line with my teaching, and there would be no legitimate grounds for rebuttal and criticism. 

Consider\marginnote{9.1} the ascetics and brahmins who teach the efficacy of deeds. In the case of those who declare that suffering is made by oneself, that’s conditioned by contact. … In the case of those who declare that suffering arises by chance, not made by oneself or another, that’s also conditioned by contact. 

In\marginnote{10.1} the case of those who declare that suffering is made by oneself, it’s impossible that they will experience that without contact. … In the case of those who declare that suffering arises by chance, not made by oneself or another, it’s impossible that they will experience that without contact. 

Ānanda,\marginnote{11.1} this one time I was staying near \textsanskrit{Rājagaha}, in the Bamboo Grove, the squirrels’ feeding ground. Then I robed up in the morning and, taking my bowl and robe, entered \textsanskrit{Rājagaha} for alms. Then I thought: ‘It’s too early to wander for alms in \textsanskrit{Rājagaha}. Why don’t I visit the monastery of the wanderers of other religions?’ 

Then\marginnote{12.1} I went to the monastery of the wanderers of other religions and exchanged greetings with the wanderers there. When the greetings and polite conversation were over, I sat down to one side. …” 

(The\marginnote{13.1} wanderers asked the Buddha the very same questions, and he gave the same answers.) 

“It’s\marginnote{17.1} incredible, sir, it’s amazing, how the whole matter is stated with one phrase. Could there be a detailed explanation of this matter that is both deep and appears deep?” 

“Well\marginnote{18.1} then, Ānanda, clarify this matter yourself.” 

“Sir,\marginnote{18.2} suppose they were to ask me: ‘Reverend Ānanda, what is the source, origin, birthplace, and inception of old age and death?’ I’d answer like this: ‘Reverends, rebirth is the source, origin, birthplace, and inception of old age and death.’ That’s how I’d answer such a question. 

Suppose\marginnote{19.1} they were to ask me: ‘What is the source of rebirth?’ I’d answer like this: ‘Continued existence is the source of rebirth.’ That’s how I’d answer such a question. 

Suppose\marginnote{20.1} they were to ask me: ‘What is the source of continued existence?’ I’d answer like this: ‘Grasping is the source of continued existence.’ That’s how I’d answer such a question. 

Suppose\marginnote{21.1} they were to ask me: ‘What is the source of grasping?’ … craving … feeling … Suppose they were to ask me: ‘What is the source of contact?’ I’d answer like this: ‘The six sense fields are the source, origin, birthplace, and inception of contact.’ ‘When the six fields of contact fade away and cease with nothing left over, contact ceases. When contact ceases, feeling ceases. When feeling ceases, craving ceases. When craving ceases, grasping ceases. When grasping ceases, continued existence ceases. When continued existence ceases, rebirth ceases. When rebirth ceases, old age and death, sorrow, lamentation, pain, sadness, and distress cease. That is how this entire mass of suffering ceases.’ That’s how I’d answer such a question.” 

%
\section*{{\suttatitleacronym SN 12.25}{\suttatitletranslation With Bhūmija }{\suttatitleroot Bhūmijasutta}}
\addcontentsline{toc}{section}{\tocacronym{SN 12.25} \toctranslation{With Bhūmija } \tocroot{Bhūmijasutta}}
\markboth{With Bhūmija }{Bhūmijasutta}
\extramarks{SN 12.25}{SN 12.25}

At\marginnote{1.1} \textsanskrit{Sāvatthī}. 

Then\marginnote{1.2} in the late afternoon, Venerable \textsanskrit{Bhūmija} came out of retreat, went to Venerable \textsanskrit{Sāriputta}, and exchanged greetings with him. When the greetings and polite conversation were over, he sat down to one side and said to him: 

“Reverend\marginnote{2.1} \textsanskrit{Sāriputta}, there are ascetics and brahmins who teach the efficacy of deeds. Some of them declare that pleasure and pain are made by oneself. Some of them declare that pleasure and pain are made by another. Some of them declare that pleasure and pain are made by both oneself and another. Some of them declare that pleasure and pain arise by chance, not made by oneself or another. What does the Buddha say about this? How does he explain it? How should we answer so as to repeat what the Buddha has said, and not misrepresent him with an untruth? How should we explain in line with his teaching, with no legitimate grounds for rebuttal and criticism?” 

“Reverend,\marginnote{3.1} the Buddha said that pleasure and pain are dependently originated. Dependent on what? Dependent on contact. If you said this you would repeat what the Buddha has said, and not misrepresent him with an untruth. You would explain in line with his teaching, and there would be no legitimate grounds for rebuttal and criticism. 

Consider\marginnote{4.1} the ascetics and brahmins who teach the efficacy of deeds. In the case of those who declare that pleasure and pain are made by oneself, that’s conditioned by contact. … In the case of those who declare that pleasure and pain arise by chance, not made by oneself or another, that’s also conditioned by contact. 

Consider\marginnote{5.1} the ascetics and brahmins who teach the efficacy of deeds. In the case of those who declare that pleasure and pain are made by oneself, it’s impossible that they will experience that without contact. … In the case of those who declare that pleasure and pain arise by chance, not made by oneself or another, it’s impossible that they will experience that without contact.” 

Venerable\marginnote{6.1} Ānanda heard this discussion between Venerable \textsanskrit{Sāriputta} and Venerable \textsanskrit{Bhūmija}. Then Venerable Ānanda went up to the Buddha, bowed, sat down to one side, and informed the Buddha of all they had discussed. 

“Good,\marginnote{7.1} good, Ānanda! It’s just as \textsanskrit{Sāriputta} has so rightly explained. I have said that pleasure and pain are dependently originated. Dependent on what? Dependent on contact. Saying this you would repeat what I have said, and not misrepresent me with an untruth. You would explain in line with my teaching, and there would be no legitimate grounds for rebuttal and criticism. 

Consider\marginnote{8.1} the ascetics and brahmins who teach the efficacy of deeds. In the case of those who declare that pleasure and pain are made by oneself, that’s conditioned by contact. … In the case of those who declare that pleasure and pain arise by chance, not made by oneself or another, that’s also conditioned by contact. 

Consider\marginnote{9.1} the ascetics and brahmins who teach the efficacy of deeds. In the case of those who declare that pleasure and pain are made by oneself, it’s impossible that they will experience that without contact. … In the case of those who declare that pleasure and pain arise by chance, not made by oneself or another, it’s impossible that they will experience that without contact. 

Ānanda,\marginnote{10.1} as long as there’s a body, the intention that gives rise to bodily action causes pleasure and pain to arise in oneself. As long as there’s a voice, the intention that gives rise to verbal action causes pleasure and pain to arise in oneself. As long as there’s a mind, the intention that gives rise to mental action causes pleasure and pain to arise in oneself. But these only apply when conditioned by ignorance. 

By\marginnote{11.1} oneself one instigates the choice that gives rise to bodily, verbal, and mental action, conditioned by which that pleasure and pain arise in oneself. Or else others instigate the choice … One consciously instigates the choice … Or else one unconsciously instigates the choice … 

Ignorance\marginnote{14.1} is included in all these things. But when ignorance fades away and ceases with nothing left over, there is no body and no voice and no mind, conditioned by which that pleasure and pain arise in oneself. There is no field, no ground, no scope, no basis, conditioned by which that pleasure and pain arise in oneself.” 

%
\section*{{\suttatitleacronym SN 12.26}{\suttatitletranslation With Upavāna }{\suttatitleroot Upavāṇasutta}}
\addcontentsline{toc}{section}{\tocacronym{SN 12.26} \toctranslation{With Upavāna } \tocroot{Upavāṇasutta}}
\markboth{With Upavāna }{Upavāṇasutta}
\extramarks{SN 12.26}{SN 12.26}

At\marginnote{1.1} \textsanskrit{Sāvatthī}. 

Then\marginnote{1.2} Venerable \textsanskrit{Upavāna} went up to the Buddha, bowed, sat down to one side, and said to him: 

“Sir,\marginnote{2.1} there are some ascetics and brahmins who declare that suffering is made by oneself. There are some who declare that suffering is made by another. There are some who declare that suffering is made by both oneself and another. There are some who declare that suffering arises by chance, not made by oneself or another. 

What\marginnote{2.5} does the Buddha say about this? How does he explain it? How should we answer so as to repeat what the Buddha has said, and not misrepresent him with an untruth? How should we explain in line with his teaching, with no legitimate grounds for rebuttal and criticism?” 

“\textsanskrit{Upavāna},\marginnote{3.1} I have said that suffering is dependently originated. Dependent on what? Dependent on contact. Saying this you would repeat what I have said, and not misrepresent me with an untruth. You would explain in line with my teaching, and there would be no legitimate grounds for rebuttal and criticism. 

In\marginnote{4.1} the case of those ascetics and brahmins who declare that suffering is made by oneself, that’s conditioned by contact. … In the case of those who declare that suffering arises by chance, not made by oneself or another, that’s also conditioned by contact. 

In\marginnote{5.1} the case of those ascetics and brahmins who declare that suffering is made by oneself, it’s impossible that they will experience that without contact. … In the case of those who declare that suffering arises by chance, not made by oneself or another, it’s impossible that they will experience that without contact.” 

%
\section*{{\suttatitleacronym SN 12.27}{\suttatitletranslation Conditions }{\suttatitleroot Paccayasutta}}
\addcontentsline{toc}{section}{\tocacronym{SN 12.27} \toctranslation{Conditions } \tocroot{Paccayasutta}}
\markboth{Conditions }{Paccayasutta}
\extramarks{SN 12.27}{SN 12.27}

At\marginnote{1.1} \textsanskrit{Sāvatthī}. 

“Ignorance\marginnote{1.2} is a condition for choices. 

Choices\marginnote{1.3} are a condition for consciousness. … That is how this entire mass of suffering originates. 

And\marginnote{2.1} what is old age and death? The old age, decrepitude, broken teeth, grey hair, wrinkly skin, diminished vitality, and failing faculties of the various sentient beings in the various orders of sentient beings. This is called old age. The passing away, passing on, disintegration, demise, mortality, death, decease, breaking up of the aggregates, and laying to rest of the corpse of the various sentient beings in the various orders of sentient beings. This is called death. Such is old age, and such is death. This is called old age and death. Rebirth is the origin of old age and death. When rebirth ceases, old age and death cease. The practice that leads to the cessation of old age and death is simply this noble eightfold path, that is: right view, right thought, right speech, right action, right livelihood, right effort, right mindfulness, and right immersion. 

And\marginnote{3.1} what is rebirth? … And what is continued existence? … And what is grasping? … And what is craving? … And what is feeling? … And what is contact? … And what are the six sense fields? … And what are name and form? … And what is consciousness? … 

And\marginnote{4.1} what are choices? There are three kinds of choices. Choices by way of body, speech, and mind. These are called choices. Ignorance is the origin of choices. When ignorance ceases, choices cease. The practice that leads to the cessation of choices is simply this noble eightfold path, that is: right view, right thought, right speech, right action, right livelihood, right effort, right mindfulness, and right immersion. 

A\marginnote{5.1} noble disciple understands conditions, their origin, their cessation, and the practice that leads to their cessation. Such a noble disciple is one who is called ‘one accomplished in view’, ‘one accomplished in vision’, ‘one who has come to the true teaching’, ‘one who sees this true teaching’, ‘one endowed with a trainee’s knowledge’, ‘one who has entered the stream of the teaching’, ‘a noble one with penetrative wisdom’, and also ‘one who stands pushing open the door to freedom from death’.” 

%
\section*{{\suttatitleacronym SN 12.28}{\suttatitletranslation A Mendicant }{\suttatitleroot Bhikkhusutta}}
\addcontentsline{toc}{section}{\tocacronym{SN 12.28} \toctranslation{A Mendicant } \tocroot{Bhikkhusutta}}
\markboth{A Mendicant }{Bhikkhusutta}
\extramarks{SN 12.28}{SN 12.28}

At\marginnote{1.1} \textsanskrit{Sāvatthī}. 

“A\marginnote{1.2} mendicant understands old age and death, their origin, their cessation, and the practice that leads to their cessation. They understand rebirth … continued existence … grasping … craving … feeling … contact … the six sense fields … name and form … consciousness … They understand choices, their origin, their cessation, and the practice that leads to their cessation. 

And\marginnote{2.1} what is old age and death? The old age, decrepitude, broken teeth, grey hair, wrinkly skin, diminished vitality, and failing faculties of the various sentient beings in the various orders of sentient beings. This is called old age. The passing away, passing on, disintegration, demise, mortality, death, decease, breaking up of the aggregates, and laying to rest of the corpse of the various sentient beings in the various orders of sentient beings. This is called death. Such is old age, and such is death. This is called old age and death. Rebirth is the origin of old age and death. When rebirth ceases, old age and death cease. The practice that leads to the cessation of old age and death is simply this noble eightfold path, that is: right view, right thought, right speech, right action, right livelihood, right effort, right mindfulness, and right immersion. 

And\marginnote{3.1} what is rebirth? … And what is continued existence? … And what is grasping? … And what is craving? … feeling … contact … the six sense fields … name and form … consciousness … 

And\marginnote{4.1} what are choices? There are three kinds of choices. Choices by way of body, speech, and mind. These are called choices. Ignorance is the origin of choices. When ignorance ceases, choices cease. The practice that leads to the cessation of choices is simply this noble eightfold path, that is: right view, right thought, right speech, right action, right livelihood, right effort, right mindfulness, and right immersion. 

A\marginnote{5.1} mendicant understands old age and death, their origin, their cessation, and the practice that leads to their cessation. They understand rebirth … continued existence … grasping … craving … feeling … contact … the six sense fields … name and form … consciousness … They understand choices, their origin, their cessation, and the practice that leads to their cessation. Such a mendicant is one who is called ‘one accomplished in view’, ‘one accomplished in vision’, ‘one who has come to the true teaching’, ‘one who sees this true teaching’, ‘one endowed with a trainee’s knowledge’, ‘one who has entered the stream of the teaching’, ‘a noble one with penetrative wisdom’, and also ‘one who stands pushing open the door to freedom from death’.” 

%
\section*{{\suttatitleacronym SN 12.29}{\suttatitletranslation Ascetics and Brahmins }{\suttatitleroot Samaṇabrāhmaṇasutta}}
\addcontentsline{toc}{section}{\tocacronym{SN 12.29} \toctranslation{Ascetics and Brahmins } \tocroot{Samaṇabrāhmaṇasutta}}
\markboth{Ascetics and Brahmins }{Samaṇabrāhmaṇasutta}
\extramarks{SN 12.29}{SN 12.29}

At\marginnote{1.1} \textsanskrit{Sāvatthī}. 

“There\marginnote{1.2} are ascetics and brahmins who don’t completely understand old age and death, their origin, their cessation, and the practice that leads to their cessation. They don’t completely understand rebirth … continued existence … grasping … craving … feeling … contact … the six sense fields … name and form … consciousness … They don’t completely understand choices, their origin, their cessation, and the practice that leads to their cessation. I don’t deem them as true ascetics and brahmins. Those venerables don’t realize the goal of life as an ascetic or brahmin, and don’t live having realized it with their own insight. 

There\marginnote{2.1} are ascetics and brahmins who completely understand old age and death, their origin, their cessation, and the practice that leads to their cessation. They completely understand rebirth … continued existence … grasping … craving … feeling … contact … the six sense fields … name and form … consciousness … They completely understand choices, their origin, their cessation, and the practice that leads to their cessation. I deem them as true ascetics and brahmins. Those venerables realize the goal of life as an ascetic or brahmin, and live having realized it with their own insight.” 

%
\section*{{\suttatitleacronym SN 12.30}{\suttatitletranslation Ascetics and Brahmins (2nd) }{\suttatitleroot Dutiyasamaṇabrāhmaṇasutta}}
\addcontentsline{toc}{section}{\tocacronym{SN 12.30} \toctranslation{Ascetics and Brahmins (2nd) } \tocroot{Dutiyasamaṇabrāhmaṇasutta}}
\markboth{Ascetics and Brahmins (2nd) }{Dutiyasamaṇabrāhmaṇasutta}
\extramarks{SN 12.30}{SN 12.30}

At\marginnote{1.1} \textsanskrit{Sāvatthī}. 

“Mendicants,\marginnote{1.2} there are ascetics and brahmins who don’t understand old age and death, their origin, their cessation, and the practice that leads to their cessation. It’s impossible that they will abide having transcended old age and death. They don’t understand rebirth … continued existence … grasping … craving … feeling … contact … the six sense fields … name and form … consciousness … They don’t understand choices, their origin, their cessation, and the practice that leads to their cessation. It’s impossible that they will abide having transcended choices. 

There\marginnote{2.1} are ascetics and brahmins who do understand old age and death, their origin, their cessation, and the practice that leads to their cessation. It’s possible that they will abide having transcended old age and death. They understand rebirth … continued existence … grasping … craving … feeling … contact … the six sense fields … name and form … consciousness … They understand choices, their origin, their cessation, and the practice that leads to their cessation. It’s possible that they will abide having transcended choices.” 

%
\addtocontents{toc}{\let\protect\contentsline\protect\nopagecontentsline}
\chapter*{The Chapter with Kaḷāra the Aristocrat }
\addcontentsline{toc}{chapter}{\tocchapterline{The Chapter with Kaḷāra the Aristocrat }}
\addtocontents{toc}{\let\protect\contentsline\protect\oldcontentsline}

%
\section*{{\suttatitleacronym SN 12.31}{\suttatitletranslation What Has Come to Be }{\suttatitleroot Bhūtasutta}}
\addcontentsline{toc}{section}{\tocacronym{SN 12.31} \toctranslation{What Has Come to Be } \tocroot{Bhūtasutta}}
\markboth{What Has Come to Be }{Bhūtasutta}
\extramarks{SN 12.31}{SN 12.31}

At\marginnote{1.1} one time the Buddha was staying near \textsanskrit{Sāvatthī}. 

Then\marginnote{1.2} the Buddha said to Venerable \textsanskrit{Sāriputta}, “\textsanskrit{Sāriputta}, this was said in ‘The Way to the Far Shore’, in ‘The Questions of Ajita’: 

\begin{verse}%
‘There\marginnote{2.1} are those who have appraised the teaching, \\
and many kinds of trainees here. \\
Tell me about their behavior, good sir, \\
when asked, for you are alert.’ 

%
\end{verse}

How\marginnote{3.1} should we see the detailed meaning of this brief statement?” 

When\marginnote{3.2} he said this, \textsanskrit{Sāriputta} kept silent. 

For\marginnote{4.1} a second time … 

For\marginnote{4.3} a third time … 

\textsanskrit{Sāriputta}\marginnote{6.1} kept silent. 

“\textsanskrit{Sāriputta},\marginnote{7.1} do you see that this has come to be?” 

“Sir,\marginnote{7.2} one truly sees with right wisdom that this has come to be. Seeing this, one is practicing for disillusionment, dispassion, and cessation regarding what has come to be. One truly sees with right wisdom that it originated with that as fuel. Seeing this, one is practicing for disillusionment, dispassion, and cessation regarding the fuel for its origination. One truly sees with right wisdom that when that fuel ceases, what has come to be is liable to cease. Seeing this, one is practicing for disillusionment, dispassion, and cessation regarding what is liable to cease. In this way one is a trainee. 

And\marginnote{8.1} what, sir, is one who has appraised the teaching? Sir, one truly sees with right wisdom that this has come to be. Seeing this, one is freed by not grasping through disillusionment, dispassion, and cessation regarding what has come to be. One truly sees with right wisdom that it originated with that as fuel. Seeing this, one is freed by not grasping through disillusionment, dispassion, and cessation regarding the fuel for its origination. One truly sees with right wisdom that when that fuel ceases, what has come to be is liable to cease. Seeing this, one is freed by not grasping through disillusionment, dispassion, and cessation regarding what is liable to cease. In this way one has appraised the teaching. 

Sir,\marginnote{8.9} regarding what was said in ‘The Way to the Far Shore’, in ‘The Questions of Ajita’: 

\begin{verse}%
‘There\marginnote{9.1} are those who have appraised the teaching, \\
and many kinds of trainees here. \\
Tell me about their behavior, good sir, \\
when asked, for you are alert.’ 

%
\end{verse}

This\marginnote{10.1} is how I understand the detailed meaning of what was said in brief.” 

“Good,\marginnote{11.1} good, \textsanskrit{Sāriputta}!” (The Buddha repeated all of \textsanskrit{Sāriputta}’s explanation, concluding:) 

This\marginnote{14.1} is how to understand the detailed meaning of what was said in brief.” 

%
\section*{{\suttatitleacronym SN 12.32}{\suttatitletranslation With Kaḷāra the Aristocrat }{\suttatitleroot Kaḷārasutta}}
\addcontentsline{toc}{section}{\tocacronym{SN 12.32} \toctranslation{With Kaḷāra the Aristocrat } \tocroot{Kaḷārasutta}}
\markboth{With Kaḷāra the Aristocrat }{Kaḷārasutta}
\extramarks{SN 12.32}{SN 12.32}

At\marginnote{1.1} \textsanskrit{Sāvatthī}. 

Then\marginnote{1.2} the mendicant \textsanskrit{Kaḷāra} the Aristocrat went up to Venerable \textsanskrit{Sāriputta} and exchanged greetings with him. When the greetings and polite conversation were over, he sat down to one side and said to him, “Reverend \textsanskrit{Sāriputta}, the mendicant Phagguna of the Top-Knot has resigned the training and returned to a lesser life.” 

“That\marginnote{1.5} venerable mustn’t have got any satisfaction in this teaching and training.” 

“Well\marginnote{2.1} then, has Venerable \textsanskrit{Sāriputta} found satisfaction in this teaching and training?” 

“Reverend,\marginnote{3.1} I have no uncertainty.” 

“But\marginnote{3.2} what of the future?” 

“I\marginnote{4.1} have no doubt.” 

Then\marginnote{5.1} \textsanskrit{Kaḷāra} the Aristocrat went up to the Buddha, bowed, sat down to one side, and said to him, “Sir, Venerable \textsanskrit{Sāriputta} has declared enlightenment: ‘I understand: “Rebirth is ended, the spiritual journey has been completed, what had to be done has been done, there is nothing further for this place.”’” 

So\marginnote{6.1} the Buddha addressed one of the monks, “Please, monk, in my name tell \textsanskrit{Sāriputta} that the teacher summons him.” 

“Yes,\marginnote{6.4} sir,” that monk replied. He went to \textsanskrit{Sāriputta} and said to him, “Reverend \textsanskrit{Sāriputta}, the teacher summons you.” 

“Yes,\marginnote{6.6} reverend,” replied \textsanskrit{Sāriputta}. He went to the Buddha, bowed, and sat down to one side. The Buddha said to him, “\textsanskrit{Sāriputta}, is it really true that you have declared enlightenment: ‘I understand: “Rebirth is ended, the spiritual journey has been completed, what had to be done has been done, there is nothing further for this place”’?” 

“Sir,\marginnote{6.9} I did not state the matter in these words and phrases.” 

“\textsanskrit{Sāriputta},\marginnote{6.10} no matter how a gentleman declares enlightenment, what he has declared should be regarded as such.” 

“Sir,\marginnote{6.11} did I not also say that I did not state the matter in these words and phrases?” 

“\textsanskrit{Sāriputta},\marginnote{7.1} suppose they were to ask you: ‘But Reverend \textsanskrit{Sāriputta}, how have you known and seen so that you’ve declared enlightenment: “I understand: ‘Rebirth is ended, the spiritual journey has been completed, what had to be done has been done, there is nothing further for this place.’”’ How would you answer?” 

“Sir,\marginnote{8.1} if they were to ask me this, I would answer: ‘Reverends, because of the ending of the source of rebirth, when it ended, I knew “it is ended”. Knowing this, I understand: “Rebirth is ended, the spiritual journey has been completed, what had to be done has been done, there is nothing further for this place.”’ That’s how I’d answer such a question.” 

“But\marginnote{9.1} \textsanskrit{Sāriputta}, suppose they were to ask you: ‘But what is the source, origin, birthplace, and inception of rebirth?’ How would you answer?” 

“Sir,\marginnote{9.4} if they were to ask me this, I would answer: ‘Continued existence is the source, origin, birthplace, and inception of rebirth.’ That’s how I’d answer such a question.” 

“But\marginnote{10.1} \textsanskrit{Sāriputta}, suppose they were to ask you: ‘What is the source of continued existence?’ How would you answer?” 

“Sir,\marginnote{10.4} if they were to ask me this, I’d answer: ‘Grasping is the source of continued existence.’ That’s how I’d answer such a question.” 

“But\marginnote{11.1} \textsanskrit{Sāriputta}, suppose they were to ask you: ‘What is the source of grasping?’ … But \textsanskrit{Sāriputta}, suppose they were to ask you: ‘What is the source of craving?’ How would you answer?” 

“Sir,\marginnote{11.6} if they were to ask me this, I’d answer: ‘Feeling is the source of craving.’ That’s how I’d answer such a question.” 

“But\marginnote{12.1} \textsanskrit{Sāriputta}, suppose they were to ask you: ‘But how have you known and seen so that the relishing of feelings is no longer present?’ How would you answer?” 

“Sir,\marginnote{12.4} if they were to ask me this, I’d answer: ‘Reverends, there are three feelings. What three? Pleasant, painful, and neutral feeling. These three feelings are impermanent, and what’s impermanent is suffering. When I understood this, the relishing of feelings was no longer present.’ That’s how I’d answer such a question.” 

“Good,\marginnote{13.1} good, \textsanskrit{Sāriputta}! The same point may also be briefly explained in this way: ‘Suffering includes whatever is felt.’ 

But\marginnote{14.1} \textsanskrit{Sāriputta}, suppose they were to ask you: ‘But Reverend, how have you been released that you declare enlightenment: “I understand: ‘Rebirth is ended, the spiritual journey has been completed, what had to be done has been done, there is nothing further for this place.’”?’ How would you answer?” 

“Sir,\marginnote{14.5} if they were to ask me this, I’d answer: ‘Because of an inner release with the ending of all grasping, I live mindfully so that defilements don’t defile me and I don’t look down on myself.’ That’s how I’d answer such a question.” 

“Good,\marginnote{15.1} good, \textsanskrit{Sāriputta}! The same point may also be briefly explained in this way: ‘I have no uncertainty regarding the defilements spoken of by the ascetic. I have no doubt that I’ve given them up.’” 

That\marginnote{15.4} is what the Buddha said. When he had spoken, the Holy One got up from his seat and entered his dwelling. 

Then\marginnote{16.1} soon after the Buddha left, Venerable \textsanskrit{Sāriputta} said to the mendicants, “Reverends, the first question that the Buddha asked me was something that I’d not previously considered, so I hesitated. But when the Buddha agreed with my answer, I thought: ‘If the Buddha were to question me all day on this matter in different words and ways, I could answer all day with different words and ways. If he were to question me all night, all day and night, for two days and nights, for three, four, five, six, or seven days and nights, I could answer in different words and ways for seven days and nights.’” 

Then\marginnote{17.1} \textsanskrit{Kaḷāra} the Aristocrat went up to the Buddha, bowed, sat down to one side, and said to him, “Sir, Venerable \textsanskrit{Sāriputta} has roared his lion’s roar!” (And he told the Buddha all that \textsanskrit{Sāriputta} had said.) 

“Mendicant,\marginnote{18.1} \textsanskrit{Sāriputta} has clearly comprehended the principle of the teachings, so that he could answer any questions I might ask him in different words and ways up to the seventh day and night.” 

%
\section*{{\suttatitleacronym SN 12.33}{\suttatitletranslation Grounds for Knowledge }{\suttatitleroot Ñāṇavatthusutta}}
\addcontentsline{toc}{section}{\tocacronym{SN 12.33} \toctranslation{Grounds for Knowledge } \tocroot{Ñāṇavatthusutta}}
\markboth{Grounds for Knowledge }{Ñāṇavatthusutta}
\extramarks{SN 12.33}{SN 12.33}

At\marginnote{1.1} \textsanskrit{Sāvatthī}. 

“Mendicants,\marginnote{1.2} I will teach forty-four grounds for knowledge. Listen and apply your mind well, I will speak.” 

“Yes,\marginnote{1.4} sir,” they replied. The Buddha said this: 

“And\marginnote{2.1} what are the forty-four grounds for knowledge? Knowledge of old age and death, knowledge of the origin of old age and death, knowledge of the cessation of old age and death, and knowledge of the practice that leads to the cessation of old age and death. Knowledge of rebirth … Knowledge of continued existence … Knowledge of grasping … Knowledge of craving … Knowledge of feeling … Knowledge of contact … Knowledge of the six sense fields … Knowledge of name and form … Knowledge of consciousness … Knowledge of choices, knowledge of the origin of choices, knowledge of the cessation of choices, and knowledge of the practice that leads to the cessation of choices. These are called the forty-four grounds for knowledge. 

And\marginnote{3.1} what is old age and death? The old age, decrepitude, broken teeth, grey hair, wrinkly skin, diminished vitality, and failing faculties of the various sentient beings in the various orders of sentient beings. This is called old age. The passing away, passing on, disintegration, demise, mortality, death, decease, breaking up of the aggregates, and laying to rest of the corpse of the various sentient beings in the various orders of sentient beings. This is called death. Such is old age, and such is death. This is called old age and death. 

Rebirth\marginnote{4.1} is the origin of old age and death. When rebirth ceases, old age and death cease. The practice that leads to the cessation of old age and death is simply this noble eightfold path, that is: right view, right thought, right speech, right action, right livelihood, right effort, right mindfulness, and right immersion. 

A\marginnote{5.1} noble disciple understands old age and death, their origin, their cessation, and the practice that leads to their cessation. This is their knowledge of the present phenomenon. With this present phenomenon that is seen, known, immediate, attained, and fathomed, they infer to the past and future. 

Whatever\marginnote{6.1} ascetics and brahmins in the past directly knew old age and death, their origin, their cessation, and the practice that leads to their cessation, all of them directly knew these things in exactly the same way that I do now. 

Whatever\marginnote{7.1} ascetics and brahmins in the future will directly know old age and death, their origin, their cessation, and the practice that leads to their cessation, all of them will directly know these things in exactly the same way that I do now. This is their inferential knowledge. 

A\marginnote{8.1} noble disciple has purified and cleansed these two knowledges—knowledge of the present phenomena, and inferential knowledge. When a noble disciple has done this, they’re one who is called ‘one accomplished in view’, ‘one accomplished in vision’, ‘one who has come to the true teaching’, ‘one who sees this true teaching’, ‘one endowed with a trainee’s knowledge’, ‘one who has entered the stream of the teaching’, ‘a noble one with penetrative wisdom’, and also ‘one who stands pushing open the door to freedom from death’. 

And\marginnote{9.1} what is rebirth? … And what is continued existence? … And what is grasping? … And what is craving? … And what is feeling? … And what is contact? … And what are the six sense fields? … And what are name and form? … And what is consciousness? … And what are choices? There are three kinds of choices. Choices by way of body, speech, and mind. These are called choices. 

Ignorance\marginnote{10.1} is the origin of choices. When ignorance ceases, choices cease. The practice that leads to the cessation of choices is simply this noble eightfold path, that is: right view, right thought, right speech, right action, right livelihood, right effort, right mindfulness, and right immersion. 

A\marginnote{11.1} noble disciple understands choices, their origin, their cessation, and the practice that leads to their cessation. This is their knowledge of the present phenomenon. With this present phenomenon that is seen, known, immediate, attained, and fathomed, they infer to the past and future. 

Whatever\marginnote{12.1} ascetics and brahmins in the past directly knew choices, their origin, their cessation, and the practice that leads to their cessation, all of them directly knew these things in exactly the same way that I do now. 

Whatever\marginnote{13.1} ascetics and brahmins in the future will directly know choices, their origin, their cessation, and the practice that leads to their cessation, all of them will directly know these things in exactly the same way that I do now. This is their inferential knowledge. 

A\marginnote{14.1} noble disciple has purified and cleansed these two knowledges—knowledge of the present phenomena, and inferential knowledge. When a noble disciple has done this, they’re one who is called ‘one accomplished in view’, ‘one accomplished in vision’, ‘one who has come to the true teaching’, ‘one who sees this true teaching’, ‘one endowed with a trainee’s knowledge’, ‘one who has entered the stream of the teaching’, ‘a noble one with penetrative wisdom’, and also ‘one who stands pushing open the door to freedom from death’.” 

%
\section*{{\suttatitleacronym SN 12.34}{\suttatitletranslation Grounds for Knowledge (2nd) }{\suttatitleroot Dutiyañāṇavatthusutta}}
\addcontentsline{toc}{section}{\tocacronym{SN 12.34} \toctranslation{Grounds for Knowledge (2nd) } \tocroot{Dutiyañāṇavatthusutta}}
\markboth{Grounds for Knowledge (2nd) }{Dutiyañāṇavatthusutta}
\extramarks{SN 12.34}{SN 12.34}

At\marginnote{1.1} \textsanskrit{Sāvatthī}. 

“Mendicants,\marginnote{1.2} I will teach seventy-seven grounds for knowledge. Listen and apply your mind well, I will speak.” 

“Yes,\marginnote{1.4} sir,” they replied. The Buddha said this: 

“And\marginnote{2.1} what are the seventy-seven grounds for knowledge? The knowledge that rebirth is a condition for old age and death, and the knowledge that when rebirth doesn’t exist, there is no old age and death. Also regarding the past: the knowledge that rebirth is a condition for old age and death, and the knowledge that when rebirth doesn’t exist, there is no old age and death. Also regarding the future: the knowledge that rebirth is a condition for old age and death, and the knowledge that when rebirth doesn’t exist, there is no old age and death. And also their knowledge that even this knowledge of the stability of natural principles is liable to end, vanish, fade away, and cease. 

The\marginnote{3.1} knowledge that continued existence is a condition for rebirth … The knowledge that ignorance is a condition for choices, and the knowledge that when ignorance doesn’t exist, there are no choices. Also regarding the past: the knowledge that ignorance is a condition for choices, and the knowledge that when ignorance doesn’t exist, there are no choices. Also regarding the future: the knowledge that ignorance is a condition for choices, and the knowledge that when ignorance doesn’t exist, there are no choices. And also their knowledge that even this knowledge of the stability of natural principles is liable to end, vanish, fade away, and cease. These are called the seventy-seven grounds for knowledge.” 

%
\section*{{\suttatitleacronym SN 12.35}{\suttatitletranslation Ignorance is a Condition }{\suttatitleroot Avijjāpaccayasutta}}
\addcontentsline{toc}{section}{\tocacronym{SN 12.35} \toctranslation{Ignorance is a Condition } \tocroot{Avijjāpaccayasutta}}
\markboth{Ignorance is a Condition }{Avijjāpaccayasutta}
\extramarks{SN 12.35}{SN 12.35}

At\marginnote{1.1} \textsanskrit{Sāvatthī}. 

“Ignorance\marginnote{1.2} is a condition for choices. 

Choices\marginnote{1.3} are a condition for consciousness. … That is how this entire mass of suffering originates.” 

When\marginnote{1.5} this was said, one of the mendicants asked the Buddha, “What are old age and death, sir, and who do they belong to?” 

“That’s\marginnote{1.7} not a fitting question,” said the Buddha. “You might say, ‘What are old age and death, and who do they belong to?’ Or you might say, ‘Old age and death are one thing, who they belong to is another.’ But both of these mean the same thing, only the phrasing differs. Mendicant, if you have the view that the soul and the body are the same thing, there is no living of the spiritual life. If you have the view that the soul and the body are different things, there is no living of the spiritual life. Avoiding these two extremes, the Realized One teaches by the middle way: ‘Rebirth is a condition for old age and death.’” 

“What\marginnote{2.1} is rebirth, sir, and who does it belong to?” 

“That’s\marginnote{2.2} not a fitting question,” said the Buddha. “You might say, ‘What is rebirth, and who does it belong to?’ Or you might say, ‘Rebirth is one thing, who it belongs to is another.’ But both of these mean the same thing, only the phrasing differs. Mendicant, if you have the view that the soul and the body are the same thing, there is no living of the spiritual life. If you have the view that the soul and the body are different things, there is no living of the spiritual life. Avoiding these two extremes, the Realized One teaches by the middle way: ‘Continued existence is a condition for rebirth.’” 

“What\marginnote{3.1} is continued existence, sir, and who is it for?” 

“That’s\marginnote{3.2} not a fitting question,” said the Buddha. “You might say, ‘What is continued existence, and who does it belong to?’ Or you might say, ‘Continued existence is one thing, who it belongs to is another.’ But both of these mean the same thing, only the phrasing differs. Mendicant, if you have the view that the soul and the body are identical, there is no living of the spiritual life. If you have the view that the soul and the body are different things, there is no living of the spiritual life. Avoiding these two extremes, the Realized One teaches by the middle way: ‘Grasping is a condition for continued existence.’ … ‘Craving is a condition for grasping.’ … ‘Feeling is a condition for craving.’ … ‘Contact is a condition for feeling.’ … ‘The six sense fields are conditions for contact.’ … ‘Name and form are conditions for the six sense fields.’ … ‘Consciousness is a condition for name and form.’ … ‘Choices are a condition for consciousness.’” 

“What\marginnote{4.1} are choices, sir, and who do they belong to?” 

“That’s\marginnote{4.2} not a fitting question,” said the Buddha. “You might say, ‘What are choices, and who do they belong to?’ Or you might say, ‘Choices are one thing, who they belong to is another.’ But both of these mean the same thing, only the phrasing differs. Mendicant, if you have the view that the soul and the body are the same thing, there is no living of the spiritual life. If you have the view that the soul and the body are different things, there is no living of the spiritual life. Avoiding these two extremes, the Realized One teaches by the middle way: ‘Ignorance is a condition for choices.’ 

When\marginnote{5.1} ignorance fades away and ceases with nothing left over, then any twists, ducks, and dodges are given up: ‘What are old age and death, and who do they belong to?’ or ‘old age and death are one thing, who they belong to is another’, or ‘the soul and the body are the same thing’, or ‘the soul and the body are different things.’ These are all cut off at the root, made like a palm stump, obliterated, and unable to arise in the future. 

When\marginnote{6.1} ignorance fades away and ceases with nothing left over, then any twists, ducks, and dodges are given up: ‘What is rebirth, and who does it belong to?’ or ‘rebirth is one thing, who it belongs to is another’, or ‘the soul and the body are the same thing’, or ‘the soul and the body are different things.’ These are all cut off at the root, made like a palm stump, obliterated, and unable to arise in the future. 

When\marginnote{7.1} ignorance fades away and ceases with nothing left over, then any twists, ducks, and dodges are given up: ‘What is continued existence …’ ‘What is grasping …’ ‘What is craving …’ ‘What is feeling …’ ‘What is contact …’ ‘What are the six sense fields …’ ‘What are name and form …’ ‘What is consciousness …’ 

When\marginnote{8.1} ignorance fades away and ceases with nothing left over, then any twists, ducks, and dodges are given up: ‘What are choices, and who do they belong to?’ or ‘choices are one thing, who they belong to is another’, or ‘the soul and the body are the same thing’, or ‘the soul and the body are different things.’ These are all cut off at the root, made like a palm stump, obliterated, and unable to arise in the future.” 

%
\section*{{\suttatitleacronym SN 12.36}{\suttatitletranslation Ignorance is a Condition (2nd) }{\suttatitleroot Dutiyaavijjāpaccayasutta}}
\addcontentsline{toc}{section}{\tocacronym{SN 12.36} \toctranslation{Ignorance is a Condition (2nd) } \tocroot{Dutiyaavijjāpaccayasutta}}
\markboth{Ignorance is a Condition (2nd) }{Dutiyaavijjāpaccayasutta}
\extramarks{SN 12.36}{SN 12.36}

At\marginnote{1.1} \textsanskrit{Sāvatthī}. 

“Ignorance\marginnote{1.2} is a condition for choices. 

Choices\marginnote{1.3} are a condition for consciousness. … That is how this entire mass of suffering originates. 

Mendicants,\marginnote{2.1} you might say, ‘What are old age and death, and who do they belong to?’ Or you might say, ‘Old age and death are one thing, who they belong to is another.’ But both of these mean the same thing, only the phrasing differs. If you have the view that the soul and the body are the same thing, there is no living of the spiritual life. If you have the view that the soul and the body are different things, there is no living of the spiritual life. Avoiding these two extremes, the Realized One teaches by the middle way: ‘Rebirth is a condition for old age and death.’ 

‘What\marginnote{3.1} is rebirth …’ ‘What is continued existence …’ ‘What is grasping …’ ‘What is craving …’ ‘What is feeling …’ ‘What is contact …’ ‘What are the six sense fields …’ ‘What are name and form …’ ‘What is consciousness …’ You might say, ‘What are choices, and who do they belong to?’ Or you might say, ‘Choices are one thing, who they belong to is another.’ But both of these mean the same thing, only the phrasing differs. If you have the view that the soul and the body are identical, there is no living of the spiritual life. If you have the view that the soul and the body are different things, there is no living of the spiritual life. Avoiding these two extremes, the Realized One teaches by the middle way: ‘Ignorance is a condition for choices.’ 

When\marginnote{4.1} ignorance fades away and ceases with nothing left over, then any twists, ducks, and dodges are given up: ‘What are old age and death, and who do they belong to?’ or ‘old age and death are one thing, who they belong to is another’, or ‘the soul and the body are identical’, or ‘the soul and the body are different things’. These are all cut off at the root, made like a palm stump, obliterated, and unable to arise in the future. 

When\marginnote{5.1} ignorance fades away and ceases with nothing left over, then any twists, ducks, and dodges are given up: ‘What is rebirth …’ ‘What is continued existence …’ ‘What is grasping …’ ‘What is craving …’ ‘What is feeling …’ ‘What is contact …’ ‘What are the six sense fields …’ ‘What are name and form …’ ‘What is consciousness …’ ‘What are choices, and who do they belong to?’ or ‘choices are one thing, who they belong to is another’, or ‘the soul and the body are identical’, or ‘the soul and the body are different things’. These are all cut off at the root, made like a palm stump, obliterated, and unable to arise in the future.” 

%
\section*{{\suttatitleacronym SN 12.37}{\suttatitletranslation Not Yours }{\suttatitleroot Natumhasutta}}
\addcontentsline{toc}{section}{\tocacronym{SN 12.37} \toctranslation{Not Yours } \tocroot{Natumhasutta}}
\markboth{Not Yours }{Natumhasutta}
\extramarks{SN 12.37}{SN 12.37}

At\marginnote{1.1} \textsanskrit{Sāvatthī}. 

“Mendicants,\marginnote{1.2} this body doesn’t belong to you or to anyone else. It’s old deeds, and should be seen as produced by choices and intentions, as something to be felt. 

A\marginnote{2.1} learned noble disciple carefully and rationally applies the mind to dependent origination itself: ‘When this exists, that is; due to the arising of this, that arises. When this doesn’t exist, that is not; due to the cessation of this, that ceases. That is: Ignorance is a condition for choices. 

Choices\marginnote{2.5} are a condition for consciousness. … That is how this entire mass of suffering originates. When ignorance fades away and ceases with nothing left over, choices cease. When choices cease, consciousness ceases. … That is how this entire mass of suffering ceases.’” 

%
\section*{{\suttatitleacronym SN 12.38}{\suttatitletranslation Intention }{\suttatitleroot Cetanāsutta}}
\addcontentsline{toc}{section}{\tocacronym{SN 12.38} \toctranslation{Intention } \tocroot{Cetanāsutta}}
\markboth{Intention }{Cetanāsutta}
\extramarks{SN 12.38}{SN 12.38}

At\marginnote{1.1} \textsanskrit{Sāvatthī}. 

“Mendicants,\marginnote{1.2} what you intend or plan, and what you have underlying tendencies for become a support for the continuation of consciousness. When this support exists, consciousness becomes established. When consciousness is established and grows, there is rebirth into a new state of existence in the future. When there is rebirth into a new state of existence in the future, future rebirth, old age, and death come to be, as do sorrow, lamentation, pain, sadness, and distress. That is how this entire mass of suffering originates. 

If\marginnote{2.1} you don’t intend or plan, but still have underlying tendencies, this becomes a support for the continuation of consciousness. When this support exists, consciousness becomes established. When consciousness is established and grows, there is rebirth into a new state of existence in the future. When there is rebirth into a new state of existence in the future, future rebirth, old age, and death come to be, as do sorrow, lamentation, pain, sadness, and distress. That is how this entire mass of suffering originates. 

If\marginnote{3.1} you don’t intend or plan or have underlying tendencies, this doesn’t become a support for the continuation of consciousness. With no support, consciousness is not established. When consciousness is not established and doesn’t grow, there’s no rebirth into a new state of existence in the future. When there is no rebirth into a new state of existence in the future, future rebirth, old age, and death cease, as do sorrow, lamentation, pain, sadness, and distress. That is how this entire mass of suffering ceases.” 

%
\section*{{\suttatitleacronym SN 12.39}{\suttatitletranslation Intention (2nd) }{\suttatitleroot Dutiyacetanāsutta}}
\addcontentsline{toc}{section}{\tocacronym{SN 12.39} \toctranslation{Intention (2nd) } \tocroot{Dutiyacetanāsutta}}
\markboth{Intention (2nd) }{Dutiyacetanāsutta}
\extramarks{SN 12.39}{SN 12.39}

At\marginnote{1.1} \textsanskrit{Sāvatthī}. 

“Mendicants,\marginnote{1.2} what you intend or plan, and what you have underlying tendencies for become a support for the continuation of consciousness. When this support exists, consciousness becomes established. When consciousness is established, name and form are conceived. Name and form are conditions for the six sense fields. The six sense fields are conditions for contact. Contact is a condition for feeling. … craving … grasping … continued existence … rebirth … old age and death, sorrow, lamentation, pain, sadness, and distress come to be. That is how this entire mass of suffering originates. 

If\marginnote{2.1} you don’t intend or plan, but still have underlying tendencies, this becomes a support for the continuation of consciousness. When this support exists, consciousness becomes established. When consciousness is established, name and form are conceived. Name and form are conditions for the six sense fields. … That is how this entire mass of suffering originates. 

If\marginnote{3.1} you don’t intend or plan or have underlying tendencies, this doesn’t become a support for the continuation of consciousness. With no support, consciousness is not established. When consciousness is not established and doesn’t grow, name and form are not conceived. When name and form cease, the six sense fields cease. … That is how this entire mass of suffering ceases.” 

%
\section*{{\suttatitleacronym SN 12.40}{\suttatitletranslation Intention (3rd) }{\suttatitleroot Tatiyacetanāsutta}}
\addcontentsline{toc}{section}{\tocacronym{SN 12.40} \toctranslation{Intention (3rd) } \tocroot{Tatiyacetanāsutta}}
\markboth{Intention (3rd) }{Tatiyacetanāsutta}
\extramarks{SN 12.40}{SN 12.40}

At\marginnote{1.1} \textsanskrit{Sāvatthī}. 

“Mendicants,\marginnote{1.2} what you intend or plan, and what you have underlying tendencies for become a support for the continuation of consciousness. When this support exists, consciousness becomes established. When consciousness is established and grows, there is an inclination. When there is an inclination, there is coming and going. When there is coming and going, there is passing away and reappearing. When there is passing away and reappearing, future rebirth, old age, and death come to be, as do sorrow, lamentation, pain, sadness, and distress. That is how this entire mass of suffering originates. 

If\marginnote{2.1} you don’t intend or plan, but still have underlying tendencies, this becomes a support for the continuation of consciousness. When this support exists, consciousness becomes established. When consciousness is established and grows, there is an inclination. When there is an inclination, there is coming and going. When there is coming and going, there is passing away and reappearing. When there is passing away and reappearing, future rebirth, old age, and death come to be, as do sorrow, lamentation, pain, sadness, and distress. That is how this entire mass of suffering originates. 

If\marginnote{3.1} you don’t intend or plan or have underlying tendencies, this doesn’t become a support for the continuation of consciousness. With no support, consciousness is not established. When consciousness is not established and doesn’t grow, there’s no inclination. When there’s no inclination, there’s no coming and going. When there’s no coming and going, there’s no passing away and reappearing. When there’s no passing away and reappearing, future rebirth, old age, and death cease, as do sorrow, lamentation, pain, sadness, and distress. That is how this entire mass of suffering ceases.” 

%
\addtocontents{toc}{\let\protect\contentsline\protect\nopagecontentsline}
\chapter*{The Chapter on Householders }
\addcontentsline{toc}{chapter}{\tocchapterline{The Chapter on Householders }}
\addtocontents{toc}{\let\protect\contentsline\protect\oldcontentsline}

%
\section*{{\suttatitleacronym SN 12.41}{\suttatitletranslation Dangers and Threats }{\suttatitleroot Pañcaverabhayasutta}}
\addcontentsline{toc}{section}{\tocacronym{SN 12.41} \toctranslation{Dangers and Threats } \tocroot{Pañcaverabhayasutta}}
\markboth{Dangers and Threats }{Pañcaverabhayasutta}
\extramarks{SN 12.41}{SN 12.41}

At\marginnote{1.1} \textsanskrit{Sāvatthī}. 

Then\marginnote{1.2} the householder \textsanskrit{Anāthapiṇḍika} went up to the Buddha, bowed, and sat down to one side. Seated to one side, the Buddha said to the householder \textsanskrit{Anāthapiṇḍika}: 

“Householder,\marginnote{2.1} when a noble disciple has quelled five dangers and threats, has the four factors of stream-entry, and has clearly seen and comprehended the noble system with wisdom, they may, if they wish, declare of themselves: ‘I’ve finished with rebirth in hell, the animal realm, and the ghost realm. I’ve finished with all places of loss, bad places, the underworld. I am a stream-enterer! I’m not liable to be reborn in the underworld, and am bound for awakening.’ 

What\marginnote{3.1} are the five dangers and threats they have quelled? Anyone who kills living creatures creates dangers and threats both in this life and in lives to come, and experiences mental pain and sadness. That danger and threat is quelled for anyone who refrains from killing living creatures. 

Anyone\marginnote{4.1} who steals creates dangers and threats both in this life and in lives to come, and experiences mental pain and sadness. That danger and threat is quelled for anyone who refrains from stealing. 

Anyone\marginnote{5.1} who commits sexual misconduct creates dangers and threats both in this life and in lives to come, and experiences mental pain and sadness. That danger and threat is quelled for anyone who refrains from committing sexual misconduct. 

Anyone\marginnote{6.1} who lies creates dangers and threats both in this life and in lives to come, and experiences mental pain and sadness. That danger and threat is quelled for anyone who refrains from lying. 

Anyone\marginnote{7.1} who consumes beer, wine, and liquor intoxicants creates dangers and threats both in this life and in lives to come, and experiences mental pain and sadness. That danger and threat is quelled for anyone who refrains from consuming beer, wine, and liquor intoxicants. These are the five dangers and threats they have quelled. 

What\marginnote{8.1} are the four factors of stream-entry that they have? It’s when a noble disciple has experiential confidence in the Buddha: ‘That Blessed One is perfected, a fully awakened Buddha, accomplished in knowledge and conduct, holy, knower of the world, supreme guide for those who wish to train, teacher of gods and humans, awakened, blessed.’ 

They\marginnote{9.1} have experiential confidence in the teaching: ‘The teaching is well explained by the Buddha—apparent in the present life, immediately effective, inviting inspection, relevant, so that sensible people can know it for themselves.’ 

They\marginnote{10.1} have experiential confidence in the \textsanskrit{Saṅgha}: ‘The \textsanskrit{Saṅgha} of the Buddha’s disciples is practicing the way that’s good, sincere, systematic, and proper. It consists of the four pairs, the eight individuals. This is the \textsanskrit{Saṅgha} of the Buddha’s disciples that is worthy of offerings dedicated to the gods, worthy of hospitality, worthy of a religious donation, worthy of greeting with joined palms, and is the supreme field of merit for the world.’ 

And\marginnote{11.1} a noble disciple’s ethical conduct is loved by the noble ones, unbroken, impeccable, spotless, and unmarred, liberating, praised by sensible people, not mistaken, and leading to immersion. These are the four factors of stream-entry that they have. 

And\marginnote{12.1} what is the noble system that they have clearly seen and comprehended with wisdom? A noble disciple carefully and rationally applies the mind to dependent origination itself: ‘When this exists, that is; when this doesn’t exist, that is not. Due to the arising of this, that arises; due to the cessation of this, that ceases. Ignorance is a condition for choices. 

Choices\marginnote{12.6} are a condition for consciousness. … That is how this entire mass of suffering originates. When ignorance fades away and ceases with nothing left over, choices cease. When choices cease, consciousness ceases. … That is how this entire mass of suffering ceases.’ This is the noble system that they have clearly seen and comprehended with wisdom. 

When\marginnote{13.1} a noble disciple has quelled five dangers and threats, has the four factors of stream-entry, and has clearly seen and comprehended the noble system with wisdom, they may, if they wish, declare of themselves: ‘I’ve finished with rebirth in hell, the animal realm, and the ghost realm. I’ve finished with all places of loss, bad places, the underworld. I am a stream-enterer! I’m not liable to be reborn in the underworld, and am bound for awakening.’” 

%
\section*{{\suttatitleacronym SN 12.42}{\suttatitletranslation Dangers and Threats (2nd) }{\suttatitleroot Dutiyapañcaverabhayasutta}}
\addcontentsline{toc}{section}{\tocacronym{SN 12.42} \toctranslation{Dangers and Threats (2nd) } \tocroot{Dutiyapañcaverabhayasutta}}
\markboth{Dangers and Threats (2nd) }{Dutiyapañcaverabhayasutta}
\extramarks{SN 12.42}{SN 12.42}

At\marginnote{1.1} \textsanskrit{Sāvatthī}. 

“Mendicants,\marginnote{1.2} when a noble disciple has quelled five dangers and threats, has the four factors of stream-entry, and has clearly seen and comprehended the noble system with wisdom, they may, if they wish, declare of themselves: ‘I’ve finished with rebirth in hell, the animal realm, and the ghost realm. I’ve finished with all places of loss, bad places, the underworld. I am a stream-enterer! I’m not liable to be reborn in the underworld, and am bound for awakening.’ 

What\marginnote{2.1} are the five dangers and threats they have quelled? Killing living creatures … stealing … sexual misconduct … lying … consuming beer, wine, and liquor intoxicants … These are the five dangers and threats they have quelled. 

What\marginnote{3.1} are the four factors of stream-entry that they have? It’s when a noble disciple has experiential confidence in the Buddha … the teaching … the \textsanskrit{Saṅgha} … and their ethical conduct is loved by the noble ones. These are the four factors of stream-entry that they have. 

And\marginnote{4.1} what is the noble system that they have clearly seen and comprehended with wisdom? A noble disciple carefully and rationally applies the mind to dependent origination itself … This is the noble system that they have clearly seen and comprehended with wisdom. 

When\marginnote{5.1} a noble disciple has quelled five dangers and threats, has the four factors of stream-entry, and has clearly seen and comprehended the noble system with wisdom, they may, if they wish, declare of themselves: ‘I’ve finished with rebirth in hell, the animal realm, and the ghost realm. I’ve finished with all places of loss, bad places, the underworld. I am a stream-enterer! I’m not liable to be reborn in the underworld, and am bound for awakening.’” 

%
\section*{{\suttatitleacronym SN 12.43}{\suttatitletranslation Suffering }{\suttatitleroot Dukkhasutta}}
\addcontentsline{toc}{section}{\tocacronym{SN 12.43} \toctranslation{Suffering } \tocroot{Dukkhasutta}}
\markboth{Suffering }{Dukkhasutta}
\extramarks{SN 12.43}{SN 12.43}

At\marginnote{1.1} \textsanskrit{Sāvatthī}. 

“Mendicants,\marginnote{1.2} I will teach you the origin and ending of suffering. Listen and apply your mind well, I will speak.” 

“Yes,\marginnote{1.4} sir,” they replied. The Buddha said this: 

“And\marginnote{2.1} what, mendicants, is the origin of suffering? Eye consciousness arises dependent on the eye and sights. The meeting of the three is contact. Contact is a condition for feeling. Feeling is a condition for craving. This is the origin of suffering. 

Ear\marginnote{3.1} consciousness arises dependent on the ear and sounds. … Nose consciousness arises dependent on the nose and smells. … Tongue consciousness arises dependent on the tongue and tastes. … Body consciousness arises dependent on the body and touches. … Mind consciousness arises dependent on the mind and ideas. The meeting of the three is contact. Contact is a condition for feeling. Feeling is a condition for craving. This is the origin of suffering. 

And\marginnote{4.1} what is the ending of suffering? Eye consciousness arises dependent on the eye and sights. The meeting of the three is contact. Contact is a condition for feeling. Feeling is a condition for craving. When that craving fades away and ceases with nothing left over, grasping ceases. When grasping ceases, continued existence ceases. When continued existence ceases, rebirth ceases. When rebirth ceases, old age and death, sorrow, lamentation, pain, sadness, and distress cease. That is how this entire mass of suffering ceases. This is the ending of suffering. 

Ear\marginnote{5.1} consciousness arises dependent on the ear and sounds. … Nose consciousness arises dependent on the nose and smells. … Tongue consciousness arises dependent on the tongue and tastes. … Body consciousness arises dependent on the body and touches. … Mind consciousness arises dependent on the mind and ideas. The meeting of the three is contact. Contact is a condition for feeling. Feeling is a condition for craving. When that craving fades away and ceases with nothing left over, grasping ceases. When grasping ceases, continued existence ceases. When continued existence ceases, rebirth ceases. When rebirth ceases, old age and death, sorrow, lamentation, pain, sadness, and distress cease. That is how this entire mass of suffering ceases. This is the ending of suffering.” 

%
\section*{{\suttatitleacronym SN 12.44}{\suttatitletranslation The World }{\suttatitleroot Lokasutta}}
\addcontentsline{toc}{section}{\tocacronym{SN 12.44} \toctranslation{The World } \tocroot{Lokasutta}}
\markboth{The World }{Lokasutta}
\extramarks{SN 12.44}{SN 12.44}

At\marginnote{1.1} \textsanskrit{Sāvatthī}. 

“Mendicants,\marginnote{1.2} I will teach you the origin and ending of the world. Listen and apply your mind well, I will speak.” 

“Yes,\marginnote{1.4} sir,” they replied. The Buddha said this: 

“And\marginnote{2.1} what, mendicants, is the origin of the world? Eye consciousness arises dependent on the eye and sights. The meeting of the three is contact. Contact is a condition for feeling. Feeling is a condition for craving. Craving is a condition for grasping. Grasping is a condition for continued existence. Continued existence is a condition for rebirth. Rebirth is a condition for old age and death, sorrow, lamentation, pain, sadness, and distress to come to be. This is the origin of the world. 

Ear\marginnote{3.1} consciousness arises dependent on the ear and sounds. … Nose consciousness arises dependent on the nose and smells. … Tongue consciousness arises dependent on the tongue and tastes. … Body consciousness arises dependent on the body and touches. … Mind consciousness arises dependent on the mind and ideas. The meeting of the three is contact. Contact is a condition for feeling. … Rebirth is a condition for old age and death, sorrow, lamentation, pain, sadness, and distress to come to be. This is the origin of the world. 

And\marginnote{4.1} what is the ending of the world? Eye consciousness arises dependent on the eye and sights. The meeting of the three is contact. Contact is a condition for feeling. Feeling is a condition for craving. When that craving fades away and ceases with nothing left over, grasping ceases. When grasping ceases, continued existence ceases. … That is how this entire mass of suffering ceases. This is the ending of the world. 

Ear\marginnote{5.1} consciousness arises dependent on the ear and sounds. … Nose consciousness arises dependent on the nose and smells. … Tongue consciousness arises dependent on the tongue and tastes. … Body consciousness arises dependent on the body and touches. … Mind consciousness arises dependent on the mind and ideas. The meeting of the three is contact. Contact is a condition for feeling. Feeling is a condition for craving. When that craving fades away and ceases with nothing left over, grasping ceases. When grasping ceases, continued existence ceases. … That is how this entire mass of suffering ceases. This is the ending of the world.” 

%
\section*{{\suttatitleacronym SN 12.45}{\suttatitletranslation At Ñātika }{\suttatitleroot Ñātikasutta}}
\addcontentsline{toc}{section}{\tocacronym{SN 12.45} \toctranslation{At Ñātika } \tocroot{Ñātikasutta}}
\markboth{At Ñātika }{Ñātikasutta}
\extramarks{SN 12.45}{SN 12.45}

\scevam{So\marginnote{1.1} I have heard. }At one time the Buddha was staying at \textsanskrit{Ñātika} in the brick house. Then while the Buddha was in private retreat he spoke this exposition of the teaching: 

“Eye\marginnote{2.1} consciousness arises dependent on the eye and sights. The meeting of the three is contact. Contact is a condition for feeling. Feeling is a condition for craving. Craving is a condition for grasping. … That is how this entire mass of suffering originates. 

Ear\marginnote{3.1} consciousness arises dependent on the ear and sounds. … Nose consciousness arises dependent on the nose and smells. … Tongue consciousness arises dependent on the tongue and tastes. … Body consciousness arises dependent on the body and touches. … Mind consciousness arises dependent on the mind and ideas. The meeting of the three is contact. Contact is a condition for feeling. Feeling is a condition for craving. Craving is a condition for grasping. … That is how this entire mass of suffering originates. 

Eye\marginnote{4.1} consciousness arises dependent on the eye and sights. The meeting of the three is contact. Contact is a condition for feeling. Feeling is a condition for craving. When that craving fades away and ceases with nothing left over, grasping ceases. When grasping ceases, continued existence ceases. … That is how this entire mass of suffering ceases. 

Ear\marginnote{5.1} consciousness arises dependent on the ear and sounds. … Mind consciousness arises dependent on the mind and ideas. The meeting of the three is contact. Contact is a condition for feeling. Feeling is a condition for craving. When that craving fades away and ceases with nothing left over, grasping ceases. When grasping ceases, continued existence ceases. … That is how this entire mass of suffering ceases.” 

Now\marginnote{6.1} at that time a certain monk was standing listening in on the Buddha. The Buddha saw him and said, “Monk, did you hear that exposition of the teaching?” 

“Yes,\marginnote{6.5} sir.” 

“Learn\marginnote{6.6} that exposition of the teaching, memorize it, and remember it. That exposition of the teaching is beneficial and relates to the fundamentals of the spiritual life.” 

%
\section*{{\suttatitleacronym SN 12.46}{\suttatitletranslation A Certain Brahmin }{\suttatitleroot Aññatarabrāhmaṇasutta}}
\addcontentsline{toc}{section}{\tocacronym{SN 12.46} \toctranslation{A Certain Brahmin } \tocroot{Aññatarabrāhmaṇasutta}}
\markboth{A Certain Brahmin }{Aññatarabrāhmaṇasutta}
\extramarks{SN 12.46}{SN 12.46}

At\marginnote{1.1} \textsanskrit{Sāvatthī}. 

Then\marginnote{1.2} a certain brahmin went up to the Buddha, and exchanged greetings with him. When the greetings and polite conversation were over, he sat down to one side and said to the Buddha: 

“Mister\marginnote{2.1} Gotama, does the person who does the deed experience the result?” 

“‘The\marginnote{2.2} person who does the deed experiences the result’: this is one extreme, brahmin.” 

“Then\marginnote{3.1} does one person do the deed and another experience the result?” 

“‘One\marginnote{3.2} person does the deed and another experiences the result’: this is the second extreme. 

Avoiding\marginnote{3.3} these two extremes, the Realized One teaches by the middle way: ‘Ignorance is a condition for choices. 

Choices\marginnote{3.5} are a condition for consciousness. … That is how this entire mass of suffering originates. When ignorance fades away and ceases with nothing left over, choices cease. When choices cease … That is how this entire mass of suffering ceases.’” 

When\marginnote{4.1} he said this, the brahmin said to the Buddha, “Excellent, Mister Gotama! Excellent! … From this day forth, may Mister Gotama remember me as a lay follower who has gone for refuge for life.” 

%
\section*{{\suttatitleacronym SN 12.47}{\suttatitletranslation Jānussoṇi }{\suttatitleroot Jāṇussoṇisutta}}
\addcontentsline{toc}{section}{\tocacronym{SN 12.47} \toctranslation{Jānussoṇi } \tocroot{Jāṇussoṇisutta}}
\markboth{Jānussoṇi }{Jāṇussoṇisutta}
\extramarks{SN 12.47}{SN 12.47}

At\marginnote{1.1} \textsanskrit{Sāvatthī}. 

Then\marginnote{1.2} the brahmin \textsanskrit{Jānussoṇi} went up to the Buddha, and exchanged greetings with him. Seated to one side he said to the Buddha: 

“Mister\marginnote{2.1} Gotama, does all exist?” 

“‘All\marginnote{2.2} exists’: this is one extreme, brahmin.” 

“Then\marginnote{3.1} does all not exist?” 

“‘All\marginnote{3.2} does not exist’: this is the second extreme. 

Avoiding\marginnote{3.3} these two extremes, the Realized One teaches by the middle way: ‘Ignorance is a condition for choices. 

Choices\marginnote{3.5} are a condition for consciousness. … That is how this entire mass of suffering originates. When ignorance fades away and ceases with nothing left over, choices cease. When choices cease, consciousness ceases. … That is how this entire mass of suffering ceases.’” 

When\marginnote{4.1} he said this, the brahmin \textsanskrit{Jānussoṇi} said to the Buddha, “Excellent, Mister Gotama! Excellent! … From this day forth, may Mister Gotama remember me as a lay follower who has gone for refuge for life.” 

%
\section*{{\suttatitleacronym SN 12.48}{\suttatitletranslation A Cosmologist }{\suttatitleroot Lokāyatikasutta}}
\addcontentsline{toc}{section}{\tocacronym{SN 12.48} \toctranslation{A Cosmologist } \tocroot{Lokāyatikasutta}}
\markboth{A Cosmologist }{Lokāyatikasutta}
\extramarks{SN 12.48}{SN 12.48}

At\marginnote{1.1} \textsanskrit{Sāvatthī}. 

Then\marginnote{1.2} a brahmin cosmologist went up to the Buddha … Seated to one side he said to the Buddha: 

“Mister\marginnote{2.1} Gotama, does all exist?” 

“‘All\marginnote{2.2} exists’: this is the oldest cosmology, brahmin.” 

“Then\marginnote{3.1} does all not exist?” 

“‘All\marginnote{3.2} does not exist’: this is the second cosmology. 

“Well,\marginnote{4.1} is all a unity?” 

“‘All\marginnote{4.2} is a unity’: this is the third cosmology. 

“Then\marginnote{5.1} is all a plurality?” 

“‘All\marginnote{5.2} is a plurality’: this is the fourth cosmology. 

Avoiding\marginnote{6.1} these two extremes, the Realized One teaches by the middle way: ‘Ignorance is a condition for choices. 

Choices\marginnote{6.3} are a condition for consciousness. … That is how this entire mass of suffering originates. When ignorance fades away and ceases with nothing left over, choices cease. When choices cease, consciousness ceases. … That is how this entire mass of suffering ceases.’” 

When\marginnote{7.1} he said this, the brahmin cosmologist said to the Buddha, “Excellent, Mister Gotama! Excellent! … From this day forth, may Mister Gotama remember me as a lay follower who has gone for refuge for life.” 

%
\section*{{\suttatitleacronym SN 12.49}{\suttatitletranslation A Noble Disciple }{\suttatitleroot Ariyasāvakasutta}}
\addcontentsline{toc}{section}{\tocacronym{SN 12.49} \toctranslation{A Noble Disciple } \tocroot{Ariyasāvakasutta}}
\markboth{A Noble Disciple }{Ariyasāvakasutta}
\extramarks{SN 12.49}{SN 12.49}

At\marginnote{1.1} \textsanskrit{Sāvatthī}. 

“Mendicants,\marginnote{1.2} a learned noble disciple doesn’t think: ‘When what exists, what is? Due to the arising of what, what arises? When what exists do name and form come to be? When what exists do the six sense fields … contact … feeling … craving … grasping … continued existence … rebirth … old age and death come to be?’ 

Rather,\marginnote{2.1} a learned noble disciple has only knowledge about this that is independent of others: ‘When this exists, that is; due to the arising of this, that arises. When ignorance exists choices come to be. When choices exist consciousness comes to be. When consciousness exists name and form come to be. When name and form exist the six sense fields come to be. When the six sense fields exist contact comes to be. When contact exists feeling comes to be. When feeling exists craving comes to be. When craving exists grasping comes to be. When grasping exists continued existence comes to be. When continued existence exists rebirth comes to be. When rebirth exists old age and death come to be.’ They understand: ‘This is the origin of the world.’ 

A\marginnote{3.1} learned noble disciple doesn’t think: ‘When what doesn’t exist, what is not? Due to the cessation of what, what ceases? When what doesn’t exist do choices not come to be? When what doesn’t exist do name and form not come to be? When what doesn’t exist do the six sense fields … contact … feeling … craving … grasping … continued existence … rebirth … old age and death not come to be?’ 

Rather,\marginnote{4.1} a learned noble disciple has only knowledge about this that is independent of others: ‘When this doesn’t exist, that is not; due to the cessation of this, that ceases. When ignorance doesn’t exist choices don’t come to be. When choices don't exist consciousness doesn’t come to be. When consciousness doesn’t exist name and form don’t come to be. When name and form don’t exist the six sense fields don’t come to be. … continued existence doesn’t come to be … rebirth doesn’t come to be … When rebirth doesn’t exist old age and death don’t come to be.’ They understand: ‘This is the cessation of the world.’ 

A\marginnote{5.1} noble disciple comes to understand the world, its origin, its cessation, and the practice that leads to its cessation. Such a noble disciple is one who is called ‘one accomplished in view’, ‘one accomplished in vision’, ‘one who has come to the true teaching’, ‘one who sees this true teaching’, ‘one endowed with a trainee’s knowledge’, ‘one who has entered the stream of the teaching’, ‘a noble one with penetrative wisdom’, and also ‘one who stands knocking at the door to freedom from death’.” 

%
\section*{{\suttatitleacronym SN 12.50}{\suttatitletranslation A Noble Disciple (2nd) }{\suttatitleroot Dutiyaariyasāvakasutta}}
\addcontentsline{toc}{section}{\tocacronym{SN 12.50} \toctranslation{A Noble Disciple (2nd) } \tocroot{Dutiyaariyasāvakasutta}}
\markboth{A Noble Disciple (2nd) }{Dutiyaariyasāvakasutta}
\extramarks{SN 12.50}{SN 12.50}

At\marginnote{1.1} \textsanskrit{Sāvatthī}. 

“Mendicants,\marginnote{1.2} a learned noble disciple doesn’t think: ‘When what exists, what is? Due to the arising of what, what arises? When what exists do choices come to be? When what exists does consciousness come to be? When what exists do name and form … the six sense fields … contact … feeling … craving … grasping … continued existence … rebirth … old age and death come to be?’ 

Rather,\marginnote{2.1} a learned noble disciple has only knowledge about this that is independent of others: ‘When this exists, that is; due to the arising of this, that arises. When ignorance exists, choices come to be. When choices exist consciousness comes to be. When consciousness exists name and form come to be. When name and form exist the six sense fields come to be. When the six sense fields exist contact comes to be. When contact exists feeling comes to be. When feeling exists craving comes to be. When craving exists grasping comes to be. When grasping exists continued existence comes to be. When continued existence exists rebirth comes to be. When rebirth exists old age and death come to be.’ They understand: ‘This is the origin of the world.’ 

A\marginnote{3.1} learned noble disciple doesn’t think: ‘When what doesn’t exist, what is not? Due to the cessation of what, what ceases? When what doesn’t exist do choices not come to be? When what doesn’t exist does consciousness not come to be? When what doesn’t exist do name and form … the six sense fields … contact … feeling … craving … grasping … continued existence … rebirth … old age and death not come to be?’ 

Rather,\marginnote{4.1} a learned noble disciple has only knowledge about this that is independent of others: ‘When this doesn’t exist, that is not; due to the cessation of this, that ceases. That is: When ignorance doesn’t exists, choices don’t come to be. When choices don’t exist consciousness doesn’t come to be. When consciousness doesn’t exist name and form don’t come to be. When name and form don’t exist the six sense fields don’t come to be. … When rebirth doesn’t exist old age and death don’t come to be.’ They understand: ‘This is the cessation of the world.’ 

A\marginnote{5.1} noble disciple comes to understand the world, its origin, its cessation, and the practice that leads to its cessation. Such a noble disciple is one who is called ‘one accomplished in view’, ‘one accomplished in vision’, ‘one who has come to the true teaching’, ‘one who sees this true teaching’, ‘one endowed with a trainee’s knowledge’, ‘one who has entered the stream of the teaching’, ‘a noble one with penetrative wisdom’, and also ‘one who stands pushing open the door to freedom from death’.” 

%
\addtocontents{toc}{\let\protect\contentsline\protect\nopagecontentsline}
\chapter*{The Chapter on Suffering }
\addcontentsline{toc}{chapter}{\tocchapterline{The Chapter on Suffering }}
\addtocontents{toc}{\let\protect\contentsline\protect\oldcontentsline}

%
\section*{{\suttatitleacronym SN 12.51}{\suttatitletranslation An Inquiry }{\suttatitleroot Parivīmaṁsanasutta}}
\addcontentsline{toc}{section}{\tocacronym{SN 12.51} \toctranslation{An Inquiry } \tocroot{Parivīmaṁsanasutta}}
\markboth{An Inquiry }{Parivīmaṁsanasutta}
\extramarks{SN 12.51}{SN 12.51}

\scevam{So\marginnote{1.1} I have heard. }At one time the Buddha was staying near \textsanskrit{Sāvatthī} in Jeta’s Grove, \textsanskrit{Anāthapiṇḍika}’s monastery. There the Buddha addressed the mendicants, “Mendicants!” 

“Venerable\marginnote{1.5} sir,” they replied. The Buddha said this: 

“Mendicants,\marginnote{2.1} when a mendicant is inquiring, how do you define when they are inquiring for the complete ending of suffering?” 

“Our\marginnote{2.2} teachings are rooted in the Buddha. He is our guide and our refuge. Sir, may the Buddha himself please clarify the meaning of this. The mendicants will listen and remember it.” 

“Well\marginnote{2.3} then, mendicants, listen and apply your mind well, I will speak.” 

“Yes,\marginnote{2.4} sir,” they replied. The Buddha said this: 

“Mendicants,\marginnote{3.1} take a mendicant who makes an inquiry: ‘The suffering that arises in the world starting with old age and death takes many and diverse forms. What is the source, origin, birthplace, and inception of this suffering? When what exists do old age and death come to be? When what does not exist do old age and death not come to be?’ While making an inquiry they understand: ‘The suffering that arises in the world starting with old age and death takes many and diverse forms. The source of this suffering is rebirth. When rebirth exists, old age and death come to be. When rebirth doesn’t exist, old age and death don’t come to be.’ 

They\marginnote{4.1} understand old age and death, their origin, their cessation, and the fitting practice for their cessation. And they practice in line with that path. This is called a mendicant who is practicing for the complete ending of suffering, for the cessation of old age and death. 

Then\marginnote{5.1} they inquire further: ‘But what is the source of this rebirth? When what exists does rebirth come to be? And when what does not exist does rebirth not come to be?’ While making an inquiry they understand: ‘Continued existence is the source of rebirth. When continued existence exists, rebirth comes to be. When continued existence does not exist, rebirth doesn’t come to be.’ 

They\marginnote{6.1} understand rebirth, its origin, its cessation, and the fitting practice for its cessation. And they practice in line with that path. This is called a mendicant who is practicing for the complete ending of suffering, for the cessation of rebirth. 

Then\marginnote{7.1} they inquire further: ‘But what is the source of this continued existence? …’ … ‘But what is the source of this grasping? …’ … ‘But what is the source of this craving? …’ … ‘But what is the source of this feeling? …’ … ‘But what is the source of this contact? …’ … ‘But what is the source of these six sense fields? …’ … ‘But what is the source of this name and form? …’ … ‘But what is the source of this consciousness? …’ … ‘But what is the source of these choices? When what exists do choices come to be? When what does not exist do choices not come to be?’ While making an inquiry they understand: ‘Ignorance is the source of choices. When ignorance exists, choices come to be. When ignorance does not exist, choices don’t come to be.’ 

They\marginnote{8.1} understand choices, their origin, their cessation, and the fitting practice for their cessation. And they practice in line with that path. This is called a mendicant who is practicing for the complete ending of suffering, for the cessation of choices. 

If\marginnote{9.1} an ignorant individual makes a good choice, their consciousness enters a good realm. If they make a bad choice, their consciousness enters a bad realm. If they make an imperturbable choice, their consciousness enters an imperturbable realm. When a mendicant has given up ignorance and given rise to knowledge, they don’t make a good choice, a bad choice, or an imperturbable choice. Not choosing or intending, they don’t grasp at anything in the world. Not grasping, they’re not anxious. Not being anxious, they personally become extinguished. 

They\marginnote{9.7} understand: ‘Rebirth is ended, the spiritual journey has been completed, what had to be done has been done, there is nothing further for this place.’ 

If\marginnote{10.1} they feel a pleasant feeling, they understand that it’s impermanent, that they’re not attached to it, and that they don’t take pleasure in it. If they feel a painful feeling, they understand that it’s impermanent, that they’re not attached to it, and that they don’t take pleasure in it. If they feel a neutral feeling, they understand that it’s impermanent, that they’re not attached to it, and that they don’t take pleasure in it. If they feel a pleasant feeling, they feel it detached. If they feel a painful feeling, they feel it detached. If they feel a neutral feeling, they feel it detached. 

Feeling\marginnote{11.1} the end of the body approaching, they understand: ‘I feel the end of the body approaching.’ Feeling the end of life approaching, they understand: ‘I feel the end of life approaching.’ They understand: ‘When my body breaks up and my life has come to an end, everything that’s felt, since I no longer take pleasure in it, will become cool right here. Only bodily remains will be left.’ 

Suppose\marginnote{12.1} a person were to remove a hot clay pot from a potter’s kiln and place it down on level ground. Its heat would dissipate right there, and the shards would be left behind. 

In\marginnote{12.3} the same way, feeling the end of the body approaching, they understand: ‘I feel the end of the body approaching.’ Feeling the end of life approaching, they understand: ‘I feel the end of life approaching.’ They understand: ‘When my body breaks up and my life has come to an end, everything that’s felt, since I no longer take pleasure in it, will become cool right here. Only bodily remains will be left.’ 

What\marginnote{13.1} do you think, mendicants? Would a mendicant who has ended the defilements still make good choices, bad choices, or imperturbable choices?” 

“No,\marginnote{13.3} sir.” 

“And\marginnote{14.1} when there are no choices at all, with the cessation of choices, would consciousness still be found?” 

“No,\marginnote{14.2} sir.” 

“And\marginnote{15.1} when there’s no consciousness at all, would name and form still be found?” 

“No,\marginnote{15.2} sir.” 

“And\marginnote{16.1} when there are no name and form at all, would the six sense fields still be found?” 

“No,\marginnote{16.2} sir.” 

“And\marginnote{17.1} when there are no six sense fields at all, would contact still be found?” 

“No,\marginnote{17.2} sir.” 

“And\marginnote{18.1} when there’s no contact at all, would feeling still be found?” 

“No,\marginnote{18.2} sir.” 

“And\marginnote{19.1} when there’s no feeling at all, would craving still be found?” 

“No,\marginnote{19.2} sir.” 

“And\marginnote{20.1} when there’s no craving at all, would grasping still be found?” 

“No,\marginnote{20.2} sir.” 

“And\marginnote{21.1} when there’s no grasping at all, would continued existence still be found?” 

“No,\marginnote{21.2} sir.” 

“And\marginnote{22.1} when there’s no continued existence at all, would rebirth still be found?” 

“No,\marginnote{22.2} sir.” 

“And\marginnote{23.1} when there’s no rebirth at all, would old age and death still be found?” 

“No,\marginnote{23.2} sir.” 

“Good,\marginnote{24.1} good, mendicants! That’s how it is, not otherwise. Trust me on this, mendicants; be convinced. Have no doubts or uncertainties in this matter. Just this is the end of suffering.” 

%
\section*{{\suttatitleacronym SN 12.52}{\suttatitletranslation Grasping }{\suttatitleroot Upādānasutta}}
\addcontentsline{toc}{section}{\tocacronym{SN 12.52} \toctranslation{Grasping } \tocroot{Upādānasutta}}
\markboth{Grasping }{Upādānasutta}
\extramarks{SN 12.52}{SN 12.52}

At\marginnote{1.1} \textsanskrit{Sāvatthī}. 

“There\marginnote{1.2} are things that fuel grasping. When you concentrate on the gratification provided by these things, your craving grows. Craving is a condition for grasping. Grasping is a condition for continued existence. Continued existence is a condition for rebirth. Rebirth is a condition for old age and death, sorrow, lamentation, pain, sadness, and distress to come to be. That is how this entire mass of suffering originates. 

Suppose\marginnote{2.1} a great mass of fire was burning with ten, twenty, thirty, or forty loads of wood. And from time to time someone would toss in dry grass, cow dung, or wood. Fed and fuelled by that, the bonfire would burn for a long time. 

In\marginnote{2.4} the same way, there are things that fuel grasping. When you concentrate on the gratification provided by these things, your craving grows. Craving is a condition for grasping. … That is how this entire mass of suffering originates. 

There\marginnote{3.1} are things that fuel grasping. When you concentrate on the drawbacks of these things, your craving ceases. When craving ceases, grasping ceases. When grasping ceases, continued existence ceases. When continued existence ceases, rebirth ceases. When rebirth ceases, old age and death, sorrow, lamentation, pain, sadness, and distress cease. That is how this entire mass of suffering ceases. 

Suppose\marginnote{4.1} a great mass of fire was burning with ten, twenty, thirty, or forty loads of wood. And no-one would toss in dry grass, cow dung, or wood from time to time. As the original fuel is used up and no more is added, the great mass of fire would be extinguished due to not being fed. 

In\marginnote{4.4} the same way, there are things that fuel grasping. When you concentrate on the drawbacks of these things, your craving ceases. When craving ceases, grasping ceases. … That is how this entire mass of suffering ceases.” 

%
\section*{{\suttatitleacronym SN 12.53}{\suttatitletranslation Fetters }{\suttatitleroot Saṁyojanasutta}}
\addcontentsline{toc}{section}{\tocacronym{SN 12.53} \toctranslation{Fetters } \tocroot{Saṁyojanasutta}}
\markboth{Fetters }{Saṁyojanasutta}
\extramarks{SN 12.53}{SN 12.53}

At\marginnote{1.1} \textsanskrit{Sāvatthī}. 

“There\marginnote{1.2} are things that are prone to being fettered. When you concentrate on the gratification provided by these things, your craving grows. Craving is a condition for grasping. Grasping is a condition for continued existence. Continued existence is a condition for rebirth. Rebirth is a condition for old age and death, sorrow, lamentation, pain, sadness, and distress to come to be. That is how this entire mass of suffering originates. 

Suppose\marginnote{2.1} an oil lamp depended on oil and a wick to burn. And from time to time someone would pour oil in and adjust the wick. Fed and fuelled by that, the oil lamp would burn for a long time. 

In\marginnote{2.4} the same way, there are things that are prone to being fettered. When you concentrate on the gratification provided by these things, your craving grows. Craving is a condition for grasping. Grasping is a condition for continued existence. Continued existence is a condition for rebirth. Rebirth is a condition for old age and death, sorrow, lamentation, pain, sadness, and distress to come to be. That is how this entire mass of suffering originates. 

There\marginnote{3.1} are things that are prone to being fettered. When you concentrate on the drawbacks of these things, your craving ceases. When craving ceases, grasping ceases. When grasping ceases, continued existence ceases. When continued existence ceases, rebirth ceases. When rebirth ceases, old age and death, sorrow, lamentation, pain, sadness, and distress cease. That is how this entire mass of suffering ceases. 

Suppose\marginnote{4.1} an oil lamp depended on oil and a wick to burn. And no-one would pour oil in and adjust the wick from time to time. As the original fuel is used up and no more is added, the oil lamp would be extinguished due to not being fed. 

In\marginnote{4.4} the same way, there are things that are prone to being fettered. When you concentrate on the drawbacks of these things, your craving ceases. When craving ceases, grasping ceases. … That is how this entire mass of suffering ceases.” 

%
\section*{{\suttatitleacronym SN 12.54}{\suttatitletranslation Fetters (2nd) }{\suttatitleroot Dutiyasaṁyojanasutta}}
\addcontentsline{toc}{section}{\tocacronym{SN 12.54} \toctranslation{Fetters (2nd) } \tocroot{Dutiyasaṁyojanasutta}}
\markboth{Fetters (2nd) }{Dutiyasaṁyojanasutta}
\extramarks{SN 12.54}{SN 12.54}

At\marginnote{1.1} \textsanskrit{Sāvatthī}. 

“Mendicants,\marginnote{1.2} suppose an oil lamp depended on oil and a wick to burn. And from time to time someone would pour oil in and adjust the wick. Fed and fuelled by that, the oil lamp would burn for a long time. 

In\marginnote{1.5} the same way, there are things that are prone to being fettered. When you concentrate on the gratification provided by these things, your craving grows. Craving is a condition for grasping. … That is how this entire mass of suffering originates. 

Suppose\marginnote{2.1} an oil lamp depended on oil and a wick to burn. And no-one would pour oil in and adjust the wick from time to time. As the original fuel is used up and no more is added, the oil lamp would be extinguished due to not being fed. 

In\marginnote{2.4} the same way, there are things that are prone to being fettered. When you concentrate on the drawbacks of these things, your craving ceases. When craving ceases, grasping ceases. … That is how this entire mass of suffering ceases.” 

%
\section*{{\suttatitleacronym SN 12.55}{\suttatitletranslation A Great Tree }{\suttatitleroot Mahārukkhasutta}}
\addcontentsline{toc}{section}{\tocacronym{SN 12.55} \toctranslation{A Great Tree } \tocroot{Mahārukkhasutta}}
\markboth{A Great Tree }{Mahārukkhasutta}
\extramarks{SN 12.55}{SN 12.55}

At\marginnote{1.1} \textsanskrit{Sāvatthī}. 

“There\marginnote{1.2} are things that fuel grasping. When you concentrate on the gratification provided by these things, your craving grows. Craving is a condition for grasping. Grasping is a condition for continued existence. … That is how this entire mass of suffering originates. 

Suppose\marginnote{2.1} there was a great tree. And its roots going downwards and across all draw the sap upwards. Fed and fuelled by that, the great tree would stand for a long time. 

In\marginnote{2.4} the same way, there are things that fuel grasping. When you concentrate on the gratification provided by these things, your craving grows. Craving is a condition for grasping. … That is how this entire mass of suffering originates. 

There\marginnote{3.1} are things that fuel grasping. When you concentrate on the drawbacks of these things, your craving ceases. When craving ceases, grasping ceases. When grasping ceases, continued existence ceases. … That is how this entire mass of suffering ceases. 

Suppose\marginnote{4.1} there was a great tree. Then a person comes along with a spade and basket. They’d cut the tree down at the roots, dig it up, and pull the roots out, down to the fibers and stems. They’d cut the tree apart, cut up the parts, and chop it into splinters. They’d dry the splinters in the wind and sun, burn them with fire, and reduce them to ashes. Then they’d winnow the ashes in a strong wind, or float them away down a swift stream. In this way the great tree is cut off at the root, made like a palm stump, obliterated, and unable to arise in the future. 

In\marginnote{4.7} the same way, there are things that fuel grasping. When you concentrate on the drawbacks of these things, your craving ceases. When craving ceases, grasping ceases. When grasping ceases, continued existence ceases. … That is how this entire mass of suffering ceases.” 

%
\section*{{\suttatitleacronym SN 12.56}{\suttatitletranslation A Great Tree (2nd) }{\suttatitleroot Dutiyamahārukkhasutta}}
\addcontentsline{toc}{section}{\tocacronym{SN 12.56} \toctranslation{A Great Tree (2nd) } \tocroot{Dutiyamahārukkhasutta}}
\markboth{A Great Tree (2nd) }{Dutiyamahārukkhasutta}
\extramarks{SN 12.56}{SN 12.56}

At\marginnote{1.1} \textsanskrit{Sāvatthī}. 

“Mendicants,\marginnote{1.2} suppose there was a great tree. And its roots going downwards and across all draw the sap upwards. Fed and fuelled by that, the great tree would stand for a long time. 

In\marginnote{1.5} the same way, there are things that fuel grasping. When you concentrate on the gratification provided by these things, your craving grows. Craving is a condition for grasping. … That is how this entire mass of suffering originates. 

Suppose\marginnote{2.1} there was a great tree. Then a person comes along with a spade and basket. They’d cut the tree down at the roots, dig them up, and pull them out, down to the fibers and stems. They’d cut the tree apart, cut up the parts, and chop it into splinters. They’d dry the splinters in the wind and sun, burn them with fire, and reduce them to ashes. Then they’d winnow the ashes in a strong wind, or float them away down a swift stream. In this way the great tree is cut off at the root, made like a palm stump, obliterated, and unable to arise in the future. 

In\marginnote{2.6} the same way, there are things that fuel grasping. When you concentrate on the drawbacks of these things, your craving ceases. When craving ceases, grasping ceases. … That is how this entire mass of suffering ceases.” 

%
\section*{{\suttatitleacronym SN 12.57}{\suttatitletranslation A Sapling }{\suttatitleroot Taruṇarukkhasutta}}
\addcontentsline{toc}{section}{\tocacronym{SN 12.57} \toctranslation{A Sapling } \tocroot{Taruṇarukkhasutta}}
\markboth{A Sapling }{Taruṇarukkhasutta}
\extramarks{SN 12.57}{SN 12.57}

At\marginnote{1.1} \textsanskrit{Sāvatthī}. 

“There\marginnote{1.2} are things that are prone to being fettered. When you concentrate on the gratification provided by these things, your craving grows. Craving is a condition for grasping. … That is how this entire mass of suffering originates. 

Suppose\marginnote{2.1} there was a sapling. And from time to time someone would clear around the roots, supply soil, and water it. Fed and fuelled by that, the sapling would grow, increase, and mature. 

In\marginnote{2.4} the same way, there are things that are prone to being fettered. When you concentrate on the gratification provided by these things, your craving grows. Craving is a condition for grasping. … That is how this entire mass of suffering originates. 

There\marginnote{3.1} are things that are prone to being fettered. When you concentrate on the drawbacks of these things, your craving ceases. When craving ceases, grasping ceases. … That is how this entire mass of suffering ceases. 

Suppose\marginnote{4.1} there was a sapling. Then a person comes along with a spade and basket. … They’d cut the sapling apart, cut up the parts, and chop it into splinters. They’d dry the splinters in the wind and sun, burn them with fire, and reduce them to ashes. Then they’d winnow the ashes in a strong wind, or float them away down a swift stream. In this way the sapling is cut off at the root, made like a palm stump, obliterated, and unable to arise in the future. 

In\marginnote{4.5} the same way, there are things that are prone to being fettered. When you concentrate on the drawbacks of these things, your craving ceases. When craving ceases, grasping ceases. … That is how this entire mass of suffering ceases.” 

%
\section*{{\suttatitleacronym SN 12.58}{\suttatitletranslation Name and Form }{\suttatitleroot Nāmarūpasutta}}
\addcontentsline{toc}{section}{\tocacronym{SN 12.58} \toctranslation{Name and Form } \tocroot{Nāmarūpasutta}}
\markboth{Name and Form }{Nāmarūpasutta}
\extramarks{SN 12.58}{SN 12.58}

At\marginnote{1.1} \textsanskrit{Sāvatthī}. 

“There\marginnote{1.2} are things that are prone to being fettered. When you concentrate on the gratification provided by these things, name and form are conceived. Name and form are conditions for the six sense fields. … That is how this entire mass of suffering originates. 

Suppose\marginnote{2.1} there was a great tree. And its roots going downwards and across all draw the sap upwards. Fed and fuelled by that, the great tree would stand for a long time. 

In\marginnote{2.4} the same way, there are things that are prone to being fettered. When you concentrate on the gratification provided by these things, name and form are conceived. … 

There\marginnote{3.1} are things that are prone to being fettered. When you concentrate on the drawbacks of these things, name and form are not conceived. When name and form cease, the six sense fields cease. … That is how this entire mass of suffering ceases. 

Suppose\marginnote{4.1} there was a great tree. Then a person comes along with a spade and basket. … In this way the great tree is cut off at the root, made like a palm stump, obliterated, and unable to arise in the future. 

In\marginnote{4.4} the same way, there are things that are prone to being fettered. When you concentrate on the drawbacks of these things, name and form are not conceived. When name and form cease, the six sense fields cease. … That is how this entire mass of suffering ceases.” 

%
\section*{{\suttatitleacronym SN 12.59}{\suttatitletranslation Consciousness }{\suttatitleroot Viññāṇasutta}}
\addcontentsline{toc}{section}{\tocacronym{SN 12.59} \toctranslation{Consciousness } \tocroot{Viññāṇasutta}}
\markboth{Consciousness }{Viññāṇasutta}
\extramarks{SN 12.59}{SN 12.59}

At\marginnote{1.1} \textsanskrit{Sāvatthī}. 

“There\marginnote{1.2} are things that are prone to being fettered. When you concentrate on the gratification provided by these things, consciousness is conceived. 

Consciousness\marginnote{1.3} is a condition for name and form. … That is how this entire mass of suffering originates. 

Suppose\marginnote{2.1} there was a great tree. And its roots going downwards and across all draw the sap upwards. … 

In\marginnote{2.3} the same way, there are things that are prone to being fettered. When you concentrate on the gratification provided by these things, consciousness is conceived. … 

There\marginnote{3.1} are things that are prone to being fettered. When you concentrate on the drawbacks of these things, consciousness is not conceived. When consciousness ceases, name and form cease. … That is how this entire mass of suffering ceases. 

Suppose\marginnote{4.1} there was a great tree. Then a person comes along with a spade and basket. … In this way the great tree is cut off at the root, made like a palm stump, obliterated, and unable to arise in the future. 

In\marginnote{4.4} the same way, there are things that are prone to being fettered. When you concentrate on the drawbacks of these things, consciousness is not conceived. When consciousness ceases, name and form cease. … That is how this entire mass of suffering ceases.” 

%
\section*{{\suttatitleacronym SN 12.60}{\suttatitletranslation Sources }{\suttatitleroot Nidānasutta}}
\addcontentsline{toc}{section}{\tocacronym{SN 12.60} \toctranslation{Sources } \tocroot{Nidānasutta}}
\markboth{Sources }{Nidānasutta}
\extramarks{SN 12.60}{SN 12.60}

At\marginnote{1.1} one time the Buddha was staying in the land of the Kurus, near the Kuru town named \textsanskrit{Kammāsadamma}. Then Venerable Ānanda went up to the Buddha, bowed, sat down to one side, and said to the Buddha: 

“It’s\marginnote{1.3} incredible, sir! It’s amazing, in that this dependent origination is deep and appears deep, yet to me it seems as plain as can be.” 

“Not\marginnote{2.1} so, Ānanda! Not so, Ānanda! This dependent origination is deep and appears deep. It is because of not understanding and not penetrating this teaching that this population has become tangled like string, knotted like a ball of thread, and matted like rushes and reeds, and it doesn’t escape the places of loss, the bad places, the underworld, transmigration. 

There\marginnote{3.1} are things that fuel grasping. When you concentrate on the gratification provided by these things, your craving grows. Craving is a condition for grasping. Grasping is a condition for continued existence. Continued existence is a condition for rebirth. Rebirth is a condition for old age and death, sorrow, lamentation, pain, sadness, and distress to come to be. That is how this entire mass of suffering originates. 

Suppose\marginnote{4.1} there was a great tree. And its roots going downwards and across all draw the sap upwards. Fed and fuelled by that, the great tree would stand for a long time. 

In\marginnote{4.4} the same way, there are things that fuel grasping. When you concentrate on the gratification provided by these things, your craving grows. Craving is a condition for grasping. Grasping is a condition for continued existence. … That is how this entire mass of suffering originates. 

There\marginnote{5.1} are things that fuel grasping. When you concentrate on the drawbacks of these things, your craving ceases. When craving ceases, grasping ceases. When grasping ceases, continued existence ceases. … That is how this entire mass of suffering ceases. 

Suppose\marginnote{6.1} there was a great tree. Then a person comes along with a spade and basket. They’d cut the tree down at the roots, dig them up, and pull them out, down to the fibers and stems. Then they’d split the tree apart, cut up the parts, and chop it into splinters. They’d dry the splinters in the wind and sun, burn them with fire, and reduce them to ashes. Then they’d winnow the ashes in a strong wind, or float them away down a swift stream. In this way the great tree is cut off at the root, made like a palm stump, obliterated, and unable to arise in the future. 

In\marginnote{6.8} the same way, there are things that fuel grasping. When you concentrate on the drawbacks of these things, your craving ceases. When craving ceases, grasping ceases. When grasping ceases, continued existence ceases. When continued existence ceases, rebirth ceases. When rebirth ceases, old age and death, sorrow, lamentation, pain, sadness, and distress cease. That is how this entire mass of suffering ceases.” 

%
\addtocontents{toc}{\let\protect\contentsline\protect\nopagecontentsline}
\chapter*{The Great Chapter }
\addcontentsline{toc}{chapter}{\tocchapterline{The Great Chapter }}
\addtocontents{toc}{\let\protect\contentsline\protect\oldcontentsline}

%
\section*{{\suttatitleacronym SN 12.61}{\suttatitletranslation Unlearned }{\suttatitleroot Assutavāsutta}}
\addcontentsline{toc}{section}{\tocacronym{SN 12.61} \toctranslation{Unlearned } \tocroot{Assutavāsutta}}
\markboth{Unlearned }{Assutavāsutta}
\extramarks{SN 12.61}{SN 12.61}

\scevam{So\marginnote{1.1} I have heard. }At one time the Buddha was staying near \textsanskrit{Sāvatthī} in Jeta’s Grove, \textsanskrit{Anāthapiṇḍika}’s monastery. … 

“Mendicants,\marginnote{1.3} when it comes to this body made up of the four principal states, an unlearned ordinary person might become disillusioned, dispassionate, and freed. Why is that? This body made up of the four principal states is seen to accumulate and disperse, to be taken up and laid to rest. That’s why, when it comes to this body, an unlearned ordinary person might become disillusioned, dispassionate, and freed. 

But\marginnote{2.1} when it comes to that which is called ‘mind’ and also ‘sentience’ and also ‘consciousness’, an unlearned ordinary person is unable to become disillusioned, dispassionate, or freed. Why is that? Because for a long time they’ve been attached to it, thought of it as their own, and mistaken it: ‘This is mine, I am this, this is my self.’ That’s why, when it comes to this mind, an unlearned ordinary person is unable to become disillusioned, dispassionate, and freed. 

But\marginnote{3.1} an unlearned ordinary person would be better off taking this body made up of the four principal states to be their self, rather than the mind. Why is that? This body made up of the four principal states is seen to last for a year, or for two, three, four, five, ten, twenty, thirty, forty, fifty, or a hundred years, or even longer. 

But\marginnote{4.1} that which is called ‘mind’ and also ‘sentience’ and also ‘consciousness’ arises as one thing and ceases as another all day and all night. It’s like a monkey moving through the forest. It grabs hold of one branch, lets it go, and grabs another; then it lets that go and grabs yet another. In the same way, that which is called ‘mind’ and also ‘sentience’ and also ‘consciousness’ arises as one thing and ceases as another all day and all night. 

In\marginnote{5.1} this case, a learned noble disciple carefully and rationally applies the mind to dependent origination itself: ‘When this exists, that is; due to the arising of this, that arises. When this doesn’t exist, that is not; due to the cessation of this, that ceases. That is: Ignorance is a condition for choices. 

Choices\marginnote{5.5} are a condition for consciousness. … That is how this entire mass of suffering originates. When ignorance fades away and ceases with nothing left over, choices cease. When choices cease, consciousness ceases. … That is how this entire mass of suffering ceases.’ 

Seeing\marginnote{6.1} this, a learned noble disciple grows disillusioned with form, feeling, perception, choices, and consciousness. Being disillusioned, desire fades away. When desire fades away they’re freed. When they’re freed, they know they’re freed. 

They\marginnote{6.3} understand: ‘Rebirth is ended, the spiritual journey has been completed, what had to be done has been done, there is nothing further for this place.’” 

%
\section*{{\suttatitleacronym SN 12.62}{\suttatitletranslation Unlearned (2nd) }{\suttatitleroot Dutiyaassutavāsutta}}
\addcontentsline{toc}{section}{\tocacronym{SN 12.62} \toctranslation{Unlearned (2nd) } \tocroot{Dutiyaassutavāsutta}}
\markboth{Unlearned (2nd) }{Dutiyaassutavāsutta}
\extramarks{SN 12.62}{SN 12.62}

At\marginnote{1.1} \textsanskrit{Sāvatthī}. 

“Mendicants,\marginnote{1.2} when it comes to this body made up of the four principal states, an unlearned ordinary person might become disillusioned, dispassionate, and freed. Why is that? This body made up of the four principal states is seen to accumulate and disperse, to be taken up and laid to rest. That’s why, when it comes to this body, an unlearned ordinary person might become disillusioned, dispassionate, and freed. But when it comes to that which is called ‘mind’ and also ‘sentience’ and also ‘consciousness’, an unlearned ordinary person is unable to become disillusioned, dispassionate, or freed. Why is that? Because for a long time they’ve been attached to it, thought of it as their own, and mistaken it: ‘This is mine, I am this, this is my self.’ That’s why, when it comes to this mind, an unlearned ordinary person is unable to become disillusioned, dispassionate, and freed. 

But\marginnote{2.1} an unlearned ordinary person would be better off taking this body made up of the four principal states to be their self, rather than the mind. Why is that? This body made up of the four principal states is seen to last for a year, or for two, three, four, five, ten, twenty, thirty, forty, fifty, or a hundred years, or even longer. But that which is called ‘mind’ and also ‘sentience’ and also ‘consciousness’ arises as one thing and ceases as another all day and all night. 

In\marginnote{3.1} this case, a learned noble disciple carefully and rationally applies the mind to dependent origination itself: ‘When this exists, that is; due to the arising of this, that arises. When this doesn’t exist, that is not; due to the cessation of this, that ceases. That is: Pleasant feeling arises dependent on a contact to be experienced as pleasant. With the cessation of that contact to be experienced as pleasant, the corresponding pleasant feeling ceases and stops. Painful feeling arises dependent on a contact to be experienced as painful. With the cessation of that contact to be experienced as painful, the corresponding painful feeling ceases and stops. Neutral feeling arises dependent on a contact to be experienced as neutral. With the cessation of that contact to be experienced as neutral, the corresponding neutral feeling ceases and stops. 

When\marginnote{4.1} you rub two sticks together, heat is generated and fire is produced. But when you part the sticks and lay them aside, any corresponding heat ceases and stops. In the same way, pleasant feeling arises dependent on a contact to be experienced as pleasant. With the cessation of that contact to be experienced as pleasant, the corresponding pleasant feeling ceases and stops. Painful feeling … Neutral feeling arises dependent on a contact to be experienced as neutral. With the cessation of that contact to be experienced as neutral, the corresponding neutral feeling ceases and stops. 

Seeing\marginnote{5.1} this, a learned noble disciple grows disillusioned with form, feeling, perception, choices, and consciousness. Being disillusioned, desire fades away. When desire fades away they’re freed. When they’re freed, they know they’re freed. 

They\marginnote{5.3} understand: ‘Rebirth is ended, the spiritual journey has been completed, what had to be done has been done, there is nothing further for this place.’” 

%
\section*{{\suttatitleacronym SN 12.63}{\suttatitletranslation A Child’s Flesh }{\suttatitleroot Puttamaṁsasutta}}
\addcontentsline{toc}{section}{\tocacronym{SN 12.63} \toctranslation{A Child’s Flesh } \tocroot{Puttamaṁsasutta}}
\markboth{A Child’s Flesh }{Puttamaṁsasutta}
\extramarks{SN 12.63}{SN 12.63}

At\marginnote{1.1} \textsanskrit{Sāvatthī}. 

“Mendicants,\marginnote{1.2} there are these four fuels. They maintain sentient beings that have been born and help those that are about to be born. What four? Solid food, whether solid or subtle; contact is the second, mental intention the third, and consciousness the fourth. These are the four fuels that maintain sentient beings that have been born and help those that are about to be born. 

And\marginnote{2.1} how should you regard solid food? Suppose a couple who were husband and wife set out to cross a desert, taking limited supplies. They had an only child, dear and beloved. As the couple were crossing the desert their limited quantity of supplies would run out, and they’d still have the rest of the desert to cross. Then it would occur to that couple: ‘Our limited quantity of supplies has run out, and we still have the rest of the desert to cross. Why don’t we kill our only child, so dear and beloved, and prepare dried and spiced meat? Then we can make it across the desert by eating our child’s flesh. Let not all three perish.’ Then that couple would kill their only child, so dear and beloved, and prepare dried and spiced meat. They’d make it across the desert by eating their child’s flesh. And as they’d eat their child’s flesh, they’d beat their breasts and cry: ‘Where are you, our only child? Where are you, our only child?’ 

What\marginnote{3.1} do you think, mendicants? Would they eat that food for fun, indulgence, adornment, or decoration?” 

“No,\marginnote{3.3} sir.” 

“Wouldn’t\marginnote{3.4} they eat that food just so they could make it across the desert?” 

“Yes,\marginnote{3.5} sir.” 

“I\marginnote{3.6} say that this is how you should regard solid food. When solid food is completely understood, desire for the five kinds of sensual stimulation is completely understood. When desire for the five kinds of sensual stimulation is completely understood, a noble disciple is bound by no fetter that might return them again to this world. 

And\marginnote{4.1} how should you regard contact as fuel? Suppose there was a flayed cow. If she stands by a wall, the creatures on the wall bite her. If she stands under a tree, the creatures in the tree bite her. If she stands in some water, the creatures in the water bite her. If she stands in the open, the creatures in the open bite her. Wherever that flayed cow stands, the creatures there would bite her. I say that this is how you should regard contact as fuel. When contact as fuel is completely understood, the three feelings are completely understood. When the three feelings are completely understood, a noble disciple has nothing further to do, I say. 

And\marginnote{5.1} how should you regard mental intention as fuel? Suppose there was a pit of glowing coals deeper than a man’s height, filled with glowing coals that neither flamed nor smoked. Then a person would come along who wants to live and doesn’t want to die, who wants to be happy and recoils from pain. Two strong men would grab them by the arms and drag them towards the pit of glowing coals. Then that person’s intention, aim, and wish would be to get far away. Why is that? Because that person would think: ‘If I fall in that pit of glowing coals, that will result in my death or deadly pain.’ I say that this is how you should regard mental intention as fuel. When mental intention as fuel is completely understood, the three cravings are completely understood. When the three cravings are completely understood, a noble disciple has nothing further to do, I say. 

And\marginnote{6.1} how should you regard consciousness as fuel? Suppose they were to arrest a bandit, a criminal and present him to the king, saying: ‘Your Majesty, this is a bandit, a criminal. Punish him as you will.’ The king would say: ‘Go, my men, and strike this man in the morning with a hundred spears!’ The king’s men did as they were told. Then at midday the king would say: ‘My men, how is that man?’ ‘He’s still alive, Your Majesty.’ The king would say: ‘Go, my men, and strike this man in the middle of the day with a hundred spears!’ The king’s men did as they were told. Then late in the afternoon the king would say: ‘My men, how is that man?’ ‘He’s still alive, Your Majesty.’ The king would say: ‘Go, my men, and strike this man in the late afternoon with a hundred spears!’ The king’s men did as they were told. 

What\marginnote{6.19} do you think, mendicants? Would that man experience pain and distress from being struck with three hundred spears a day?” 

“Sir,\marginnote{6.21} that man would experience pain and distress from being struck with one spear, let alone three hundred spears!” 

“I\marginnote{6.23} say that this is how you should regard consciousness as fuel. When consciousness as fuel is completely understood, name and form is completely understood. When name and form are completely understood, a noble disciple has nothing further to do, I say.” 

%
\section*{{\suttatitleacronym SN 12.64}{\suttatitletranslation If There Is Desire }{\suttatitleroot Atthirāgasutta}}
\addcontentsline{toc}{section}{\tocacronym{SN 12.64} \toctranslation{If There Is Desire } \tocroot{Atthirāgasutta}}
\markboth{If There Is Desire }{Atthirāgasutta}
\extramarks{SN 12.64}{SN 12.64}

At\marginnote{1.1} \textsanskrit{Sāvatthī}. 

“Mendicants,\marginnote{1.2} there are these four fuels. They maintain sentient beings that have been born and help those that are about to be born. What four? Solid food, whether solid or subtle; contact is the second, mental intention the third, and consciousness the fourth. These are the four fuels that maintain sentient beings that have been born and help those that are about to be born. 

If\marginnote{2.1} there is desire, relishing, and craving for solid food, consciousness becomes established there and grows. Where consciousness is established and grows, name and form are conceived. Where name and form are conceived, there is the growth of choices. Where choices grow, there is rebirth into a new state of existence in the future. Where there is rebirth into a new state of existence in the future, there is rebirth, old age, and death in the future. Where there is rebirth, old age, and death in the future, I say this is full of sorrow, anguish, and distress. 

If\marginnote{3.1} there is desire, relishing, and craving for contact as fuel … If there is desire, relishing, and craving for mental intention as fuel … If there is desire, relishing, and craving for consciousness as fuel, consciousness becomes established there and grows. Where consciousness is established and grows, name and form are conceived. Where name and form are conceived, there is the growth of choices. Where choices grow, there is rebirth into a new state of existence in the future. Where there is rebirth into a new state of existence in the future, there is rebirth, old age, and death in the future. Where there is rebirth, old age, and death in the future, I say this is full of sorrow, anguish, and distress. 

Suppose\marginnote{4.1} an artist or painter had some dye, red lac, turmeric, indigo, or rose madder. And on a polished plank or a wall or a canvas they’d create the form of a woman or a man, whole in its major and minor limbs. 

In\marginnote{4.2} the same way, if there is desire, relishing, and craving for solid food, consciousness becomes established there and grows. Where consciousness is established and grows, name and form are conceived. Where name and form are conceived, there is the growth of choices. Where choices grow, there is rebirth into a new state of existence in the future. Where there is rebirth into a new state of existence in the future, there is rebirth, old age, and death in the future. Where there is rebirth, old age, and death in the future, I say this is full of sorrow, anguish, and distress. 

If\marginnote{5.1} there is desire, relishing, and craving for contact as fuel … If there is desire, relishing, and craving for mental intention as fuel … If there is desire, relishing, and craving for consciousness as fuel, consciousness becomes established there and grows. Where consciousness is established and grows, name and form are conceived. Where name and form are conceived, there is the growth of choices. Where choices grow, there is rebirth into a new state of existence in the future. Where there is rebirth into a new state of existence in the future, there is rebirth, old age, and death in the future. Where there is rebirth, old age, and death in the future, I say this is full of sorrow, anguish, and distress. 

If\marginnote{6.1} there is no desire, relishing, and craving for solid food, consciousness does not become established there and doesn’t grow. Where consciousness is not established and doesn’t grow, name and form are not conceived. Where name and form are not conceived, there is no growth of choices. Where choices don’t grow, there is no rebirth into a new state of existence in the future. Where there is no rebirth into a new state of existence in the future, there is no rebirth, old age, and death in the future. Where there is no rebirth, old age, and death in the future, I say there’s no sorrow, anguish, and distress. 

If\marginnote{7.1} there is no desire, relishing, and craving for contact as fuel … If there is no desire, relishing, and craving for mental intention as fuel … If there is no desire, relishing, and craving for consciousness as fuel, consciousness doesn’t become established there and doesn’t grow. Where consciousness is not established and doesn’t grow, name and form are not conceived. Where name and form are not conceived, there is no growth of choices. Where choices don’t grow, there is no rebirth into a new state of existence in the future. Where there is no rebirth into a new state of existence in the future, there is no rebirth, old age, and death in the future. Where there is no rebirth, old age, and death in the future, I say there’s no sorrow, anguish, and distress. 

Suppose\marginnote{8.1} there was a bungalow or a hall with a peaked roof, with windows on the northern, southern, or eastern side. When the sun rises and a ray of light enters through a window, where would it land?” 

“On\marginnote{8.2} the western wall, sir.” 

“If\marginnote{8.3} there was no western wall, where would it land?” 

“On\marginnote{8.4} the ground, sir.” 

“If\marginnote{8.5} there was no ground, where would it land?” 

“In\marginnote{8.6} water, sir.” 

“If\marginnote{8.7} there was no water, where would it land?” 

“It\marginnote{8.8} wouldn’t land, sir.” 

“In\marginnote{8.9} the same way, if there is no desire, relishing, and craving for solid food, consciousness does not become established there and doesn’t grow. … 

If\marginnote{9.1} there is no desire, relishing, and craving for contact as fuel … If there is no desire, relishing, and craving for mental intention as fuel … If there is no desire, relishing, and craving for consciousness as fuel, consciousness doesn’t become established there and doesn’t grow. Where consciousness is not established and doesn’t grow, name and form are not conceived. Where name and form are not conceived, there is no growth of choices. Where choices don’t grow, there is no rebirth into a new state of existence in the future. Where there is no rebirth into a new state of existence in the future, there is no rebirth, old age, and death in the future. Where there is no rebirth, old age, and death in the future, I say there’s no sorrow, anguish, and distress.” 

%
\section*{{\suttatitleacronym SN 12.65}{\suttatitletranslation The City }{\suttatitleroot Nagarasutta}}
\addcontentsline{toc}{section}{\tocacronym{SN 12.65} \toctranslation{The City } \tocroot{Nagarasutta}}
\markboth{The City }{Nagarasutta}
\extramarks{SN 12.65}{SN 12.65}

At\marginnote{1.1} \textsanskrit{Sāvatthī}. 

“Mendicants,\marginnote{1.2} before my awakening—when I was still unawakened but intent on awakening—I thought: ‘Alas, this world has fallen into trouble. It’s born, grows old, dies, passes away, and is reborn, yet it doesn’t understand how to escape from this suffering, from old age and death. Oh, when will an escape be found from this suffering, from old age and death?’ Then it occurred to me: ‘When what exists is there old age and death? What is a condition for old age and death?’ Then, through rational application of mind, I comprehended with wisdom: ‘When rebirth exists there’s old age and death. Rebirth is a condition for old age and death.’ 

Then\marginnote{2.1} it occurred to me: ‘When what exists is there rebirth? … continued existence … grasping … craving … feeling … contact … the six sense fields … name and form … What is a condition for name and form?’ Then, through rational application of mind, I comprehended with wisdom: ‘When consciousness exists there are name and form. Consciousness is a condition for name and form.’ Then it occurred to me: ‘When what exists is there consciousness? What is a condition for consciousness?’ Then, through rational application of mind, I comprehended with wisdom: ‘When name and form exist there’s consciousness. Name and form are a condition for consciousness.’ 

Then\marginnote{3.1} it occurred to me: This consciousness turns back from name and form, and doesn’t go beyond that. This is the extent to which one may be reborn, grow old, die, pass away, or reappear. That is: name and form are conditions for consciousness. Consciousness is a condition for name and form. Name and form are conditions for the six sense fields. The six sense fields are conditions for contact. … That is how this entire mass of suffering originates. ‘Origination, origination.’ Such was the vision, knowledge, wisdom, realization, and light that arose in me regarding teachings not learned before from another. 

Then\marginnote{4.1} it occurred to me: ‘When what doesn’t exist is there no old age and death? When what ceases do old age and death cease?’ Then, through rational application of mind, I comprehended with wisdom: ‘When rebirth doesn’t exist there is no old age and death. When rebirth ceases old age and death cease.’ Then it occurred to me: ‘When what doesn’t exist is there no rebirth … continued existence … grasping … craving … feeling … contact … six sense fields … name and form? When what ceases do name and form cease?’ Then, through rational application of mind, I comprehended with wisdom: ‘When consciousness doesn’t exist there is no name and form. When consciousness ceases name and form cease.’ 

Then\marginnote{5.1} it occurred to me: ‘When what doesn’t exist is there no consciousness? When what ceases does consciousness cease?’ Then, through rational application of mind, I comprehended with wisdom: ‘When name and form don’t exist, there is no consciousness. When name and form cease, consciousness ceases.’ 

Then\marginnote{6.1} it occurred to me: I have discovered the path to awakening. That is: When name and form cease, consciousness ceases. When consciousness ceases, name and form cease. When name and form cease, the six sense fields cease. When the six sense fields cease, contact ceases. … That is how this entire mass of suffering ceases. ‘Cessation, cessation.’ Such was the vision, knowledge, wisdom, realization, and light that arose in me regarding teachings not learned before from another. 

Suppose\marginnote{7.1} a person was walking through a forest. They’d see an ancient path, an ancient route traveled by humans in the past. Following it along, they’d see an ancient city, an ancient capital, inhabited by humans in the past. It was lovely, complete with parks, groves, lotus ponds, and embankments. Then that person would inform a king or their chief minister: ‘Please sir, you should know this. While walking through a forest I saw an ancient path, an ancient route traveled by humans in the past. Following it along I saw an ancient city, an ancient capital, inhabited by humans in the past. It was lovely, complete with parks, groves, lotus ponds, and embankments. Sir, you should rebuild that city!’ Then that king or their chief minister would have that city rebuilt. And after some time that city was successful and prosperous, populous, full of people, attained to growth and expansion. In the same way, I saw an ancient path, an ancient route traveled by fully awakened Buddhas in the past. 

And\marginnote{8.1} what is that ancient path, the ancient route traveled by fully awakened Buddhas in the past? It is simply this noble eightfold path, that is: right view, right thought, right speech, right action, right livelihood, right effort, right mindfulness, and right immersion. This is that ancient path, the ancient route traveled by fully awakened Buddhas in the past. 

Following\marginnote{8.5} it along, I directly knew old age and death, their origin, their cessation, and the practice that leads to their cessation. Following it along, I directly knew rebirth … continued existence … grasping … craving … feeling … contact … the six sense fields … name and form … consciousness … Following it along, I directly knew choices, their origin, their cessation, and the practice that leads to their cessation. 

Having\marginnote{8.24} directly known this, I told the monks, nuns, laymen, and laywomen. And that’s how this spiritual life has become successful and prosperous, extensive, popular, widespread, and well proclaimed wherever there are gods and humans.” 

%
\section*{{\suttatitleacronym SN 12.66}{\suttatitletranslation Self-examination }{\suttatitleroot Sammasasutta}}
\addcontentsline{toc}{section}{\tocacronym{SN 12.66} \toctranslation{Self-examination } \tocroot{Sammasasutta}}
\markboth{Self-examination }{Sammasasutta}
\extramarks{SN 12.66}{SN 12.66}

\scevam{So\marginnote{1.1} I have heard. }At one time the Buddha was staying in the land of the Kurus, near the Kuru town named \textsanskrit{Kammāsadamma}. There the Buddha addressed the mendicants, “Mendicants!” 

“Venerable\marginnote{1.5} sir,” they replied. The Buddha said this: 

“Mendicants,\marginnote{1.7} do you perform inner self-examination?” 

When\marginnote{1.8} he said this, one of the mendicants said to the Buddha, “Sir, I perform inner self-examination.” 

“But\marginnote{1.10} mendicant, how do you perform inner self-examination?” Then that mendicant answered, but the Buddha was not happy with the answer. 

When\marginnote{2.1} he had spoken, Venerable Ānanda said to the Buddha, “Now is the time, Blessed One! Now is the time, Holy One! Let the Buddha speak of the inner self-examination. The mendicants will listen and remember it.” 

“Well\marginnote{2.4} then, Ānanda, listen and apply your mind well, I will speak.” 

“Yes,\marginnote{2.5} sir,” they replied. The Buddha said this: 

“Take\marginnote{3.1} a mendicant who performs inner self-examination: ‘The suffering that arises in the world starting with old age and death takes many and diverse forms. But what is the source of this suffering? When what exists do old age and death come to be? And when what does not exist do old age and death not come to be?’ While examining they know: ‘The suffering that arises in the world starting with old age and death takes many and diverse forms. The source of this suffering is attachment. When attachments exist old age and death come to be. And when attachments do not exist old age and death don’t come to be.’ They understand old age and death, their origin, their cessation, and the fitting practice for their cessation. And they practice in line with that path. This is called a mendicant who is practicing for the complete ending of suffering, for the cessation of old age and death. 

They\marginnote{4.1} perform further inner self-examination: ‘But what is the source of this attachment? When what exists does attachment come to be? And when what does not exist does attachment not come to be?’ While examining they know: ‘The source of this attachment is craving. When craving exists attachments come to be. And when craving doesn’t exist attachments don’t come to be.’ They understand attachments, their origin, their cessation, and the fitting practice for their cessation. And they practice in line with that path. This is called a mendicant who is practicing for the complete ending of suffering, for the cessation of attachments. 

They\marginnote{5.1} perform further inner self-examination: ‘But where does that craving arise and where does it settle?’ While examining they know: ‘That craving arises and settles on whatever in the world seems nice and pleasant. And what in the world seems nice and pleasant? The eye in the world seems nice and pleasant, and it is there that craving arises and settles. The ear … nose … tongue … body … mind in the world seems nice and pleasant, and it is there that craving arises and settles.’ 

There\marginnote{6.1} were ascetics and brahmins of the past who saw the things that seem nice and pleasant in the world as permanent, as pleasurable, as self, as healthy, and as safe. Their craving grew. As their craving grew, their attachments grew. As their attachments grew, their suffering grew. And as their suffering grew, they were not freed from rebirth, old age, and death, from sorrow, lamentation, pain, sadness, and distress. They were not freed from suffering, I say. 

There\marginnote{7.1} will be ascetics and brahmins in the future who will see the things that seem nice and pleasant in the world as permanent, as pleasurable, as self, as healthy, and as safe. Their craving will grow. As their craving grows, their attachments will grow. As their attachments grow, their suffering will grow. And as their suffering grows, they will not be freed from rebirth, old age, and death, from sorrow, lamentation, pain, sadness, and distress. They will not be freed from suffering, I say. 

There\marginnote{8.1} are ascetics and brahmins in the present who see the things that seem nice and pleasant in the world as permanent, as pleasurable, as self, as healthy, and as safe. Their craving grows. As their craving grows, their attachments grow. As their attachments grow, their suffering grows. And as their suffering grows, they are not freed from rebirth, old age, and death, from sorrow, lamentation, pain, sadness, and distress. They are not freed from suffering, I say. 

Suppose\marginnote{9.1} there was a bronze goblet of beverage that had a nice color, aroma, and flavor. But it was mixed with poison. Then along comes a man struggling in the oppressive heat, weary, thirsty, and parched. They’d say to him: ‘Here, mister, this bronze goblet of beverage has a nice color, aroma, and flavor. But it’s mixed with poison. Drink it if you like. If you drink it, the color, aroma, and flavor will be appetizing, but it will result in death or deadly pain.’ He wouldn’t reject that beverage. Hastily, without reflection, he’d drink it, resulting in death or deadly pain. 

In\marginnote{9.11} the same way, there are ascetics and brahmins of the past … future … There are ascetics and brahmins in the present who see the things that seem nice and pleasant in the world as permanent, as pleasurable, as self, as healthy, and as safe. Their craving grows. As their craving grows, their attachments grow. As their attachments grow, their suffering grows. And as their suffering grows, they are not freed from rebirth, old age, and death, from sorrow, lamentation, pain, sadness, and distress. They are not freed from suffering, I say. 

There\marginnote{10.1} were ascetics and brahmins of the past who saw the things that seem nice and pleasant in the world as impermanent, as suffering, as not-self, as diseased, and as dangerous. They gave up craving. Giving up craving, they gave up attachments. Giving up attachments, they gave up suffering. Giving up suffering, they were freed from rebirth, old age, and death, from sorrow, lamentation, pain, sadness, and distress. They were freed from suffering, I say. 

There\marginnote{11.1} will be ascetics and brahmins in the future who will see the things that seem nice and pleasant in the world as impermanent, as suffering, as not-self, as diseased, and as dangerous. They will give up craving. Giving up craving … they will be freed from suffering, I say. 

There\marginnote{12.1} are ascetics and brahmins in the present who see the things that seem nice and pleasant in the world as impermanent, as suffering, as not-self, as diseased, and as dangerous. They give up craving. Giving up craving, they give up attachments. Giving up attachments, they give up suffering. Giving up suffering, they are freed from rebirth, old age, and death, from sorrow, lamentation, pain, sadness, and distress. They are freed from suffering, I say. 

Suppose\marginnote{13.1} there was a bronze goblet of beverage that had a nice color, aroma, and flavor. But it was mixed with poison. Then along comes a man struggling in the oppressive heat, weary, thirsty, and parched. They’d say to him: ‘Here, mister, this bronze goblet of beverage has a nice color, aroma, and flavor. But it’s mixed with poison. Drink it if you like. If you drink it, its nice color, aroma, and flavor will refresh you. But drinking it will result in death or deadly pain.’ Then that man might think: ‘I could quench my thirst with water, whey, seasoned drink, or broth. But I shouldn’t drink that beverage, for it would be for my lasting harm and suffering.’ He’d reject that beverage. After appraisal, he wouldn’t drink it, and it wouldn’t result in death or deadly pain. 

In\marginnote{13.13} the same way, there were ascetics and brahmins of the past who saw the things that seem nice and pleasant in the world as impermanent, as suffering, as not-self, as diseased, and as dangerous. They gave up craving. Giving up craving, they gave up attachments. Giving up attachments, they gave up suffering. Giving up suffering, they were freed from rebirth, old age, and death, from sorrow, lamentation, pain, sadness, and distress. They were freed from suffering, I say. 

There\marginnote{14.1} will be ascetics and brahmins in the future … There are ascetics and brahmins in the present who see the things that seem nice and pleasant in the world as impermanent, as suffering, as not-self, as diseased, and as dangerous. They give up craving. Giving up craving, they give up attachments. Giving up attachments, they give up suffering. Giving up suffering, they are freed from rebirth, old age, and death, from sorrow, lamentation, pain, sadness, and distress. They are freed from suffering, I say.” 

%
\section*{{\suttatitleacronym SN 12.67}{\suttatitletranslation Bundles of Reeds }{\suttatitleroot Naḷakalāpīsutta}}
\addcontentsline{toc}{section}{\tocacronym{SN 12.67} \toctranslation{Bundles of Reeds } \tocroot{Naḷakalāpīsutta}}
\markboth{Bundles of Reeds }{Naḷakalāpīsutta}
\extramarks{SN 12.67}{SN 12.67}

At\marginnote{1.1} one time Venerable \textsanskrit{Sāriputta} and Venerable \textsanskrit{Mahākoṭṭhita} were staying near Varanasi, in the deer park at Isipatana. 

Then\marginnote{1.2} in the late afternoon, Venerable \textsanskrit{Mahākoṭṭhita} came out of retreat, went to Venerable \textsanskrit{Sāriputta}, and exchanged greetings with him. When the greetings and polite conversation were over, he sat down to one side and said to \textsanskrit{Sāriputta}: 

“Well,\marginnote{1.4} Reverend \textsanskrit{Sāriputta}, are old age and death made by oneself? Or by another? Or by both oneself and another? Or do they arise by chance, not made by oneself or another?” 

“No,\marginnote{1.5} Reverend \textsanskrit{Koṭṭhita}, old age and death are not made by oneself, nor by another, nor by both oneself and another, nor do they arise by chance, not made by oneself or another. Rather, rebirth is a condition for old age and death.” 

“Well,\marginnote{2.1} Reverend \textsanskrit{Sāriputta}, is rebirth made by oneself? Or by another? Or by both oneself and another? Or does it arise by chance, not made by oneself or another?” 

“No,\marginnote{2.2} Reverend \textsanskrit{Koṭṭhita}, rebirth is not made by oneself, nor by another, nor by both oneself and another, nor does it arise by chance, not made by oneself or another. Rather, continued existence is a condition for rebirth.” 

“Well,\marginnote{3.1} Reverend \textsanskrit{Sāriputta}, is continued existence made by oneself? …” … “Is grasping made by oneself? …” … “Is craving made by oneself? …” … “Is feeling made by oneself? …” … “Is contact made by oneself? …” … “Are the six sense fields made by oneself? …” … “Well, Reverend \textsanskrit{Sāriputta}, are name and form made by oneself? Or by another? Or by both oneself and another? Or do they arise by chance, not made by oneself or another?” 

“No,\marginnote{3.8} Reverend \textsanskrit{Koṭṭhita}, name and form are not made by oneself, nor by another, nor by both oneself and another, nor do they arise by chance, not made by oneself or another. Rather, consciousness is a condition for name and form.” 

“Well,\marginnote{4.1} Reverend \textsanskrit{Sāriputta}, is consciousness made by oneself? Or by another? Or by both oneself and another? Or does it arise by chance, not made by oneself or another?” 

“No,\marginnote{4.2} Reverend \textsanskrit{Koṭṭhita}, consciousness is not made by oneself, nor by another, nor by both oneself and another, nor does it arise by chance, not made by oneself or another. Rather, name and form are conditions for consciousness.” 

“Just\marginnote{5.1} now I understood you to say: ‘No, Reverend \textsanskrit{Koṭṭhita}, name and form are not made by oneself, nor by another, nor by both oneself and another, nor do they arise by chance, not made by oneself or another. Rather, consciousness is a condition for name and form.’ 

But\marginnote{6.1} I also understood you to say: ‘No, Reverend \textsanskrit{Koṭṭhita}, consciousness is not made by oneself, nor by another, nor by both oneself and another, nor does it arise by chance, not made by oneself or another. Rather, name and form are conditions for consciousness.’ 

How\marginnote{7.1} then should we see the meaning of this statement?” 

“Well\marginnote{7.2} then, reverend, I shall give you a simile. For by means of a simile some sensible people understand the meaning of what is said. Suppose there were two bundles of reeds leaning up against each other. 

In\marginnote{7.5} the same way, name and form are conditions for consciousness. Consciousness is a condition for name and form. Name and form are conditions for the six sense fields. The six sense fields are conditions for contact. … That is how this entire mass of suffering originates. If the first of those bundles of reeds were to be pulled away, the other would collapse. And if the other were to be pulled away, the first would collapse. 

In\marginnote{7.12} the same way, when name and form cease, consciousness ceases. When consciousness ceases, name and form cease. When name and form cease, the six sense fields cease. When the six sense fields cease, contact ceases. … That is how this entire mass of suffering ceases.” 

“It’s\marginnote{8.1} incredible, Reverend \textsanskrit{Sāriputta}, it’s amazing! How well spoken this was by Venerable \textsanskrit{Sāriputta}! And we can express our agreement with Venerable \textsanskrit{Sāriputta}’s statement on these thirty-six grounds. 

If\marginnote{9.1} a mendicant teaches Dhamma for disillusionment, dispassion, and cessation regarding old age and death, they’re qualified to be called a ‘mendicant who speaks on Dhamma’. If they practice for disillusionment, dispassion, and cessation regarding old age and death, they’re qualified to be called a ‘mendicant who practices in line with the teaching’. If they’re freed by not grasping by disillusionment, dispassion, and cessation regarding old age and death, they’re qualified to be called a ‘mendicant who has attained extinguishment in this very life’. If a mendicant teaches Dhamma for disillusionment regarding rebirth … continued existence … grasping … craving … feeling … contact … the six sense fields … name and form … consciousness … choices … If a mendicant teaches Dhamma for disillusionment, dispassion, and cessation regarding ignorance, they’re qualified to be called a ‘mendicant who speaks on Dhamma’. If they practice for disillusionment, dispassion, and cessation regarding ignorance, they’re qualified to be called a ‘mendicant who practices in line with the teaching’. If they’re freed by not grasping by disillusionment, dispassion, and cessation regarding ignorance, they’re qualified to be called a ‘mendicant who has attained extinguishment in this very life’.” 

%
\section*{{\suttatitleacronym SN 12.68}{\suttatitletranslation At Kosambī }{\suttatitleroot Kosambisutta}}
\addcontentsline{toc}{section}{\tocacronym{SN 12.68} \toctranslation{At Kosambī } \tocroot{Kosambisutta}}
\markboth{At Kosambī }{Kosambisutta}
\extramarks{SN 12.68}{SN 12.68}

At\marginnote{1.1} one time the venerables \textsanskrit{Musīla}, \textsanskrit{Saviṭṭha}, \textsanskrit{Nārada}, and Ānanda were staying near \textsanskrit{Kosambī} in Ghosita’s monastery. Then Venerable \textsanskrit{Saviṭṭha} said to Venerable \textsanskrit{Musīla}: 

“Reverend\marginnote{1.3} \textsanskrit{Musīla}, apart from faith, endorsement, oral transmission, reasoned train of thought, or acceptance of a view after deliberation, do you know for yourself that rebirth is a condition for old age and death?” 

“Reverend\marginnote{1.5} \textsanskrit{Saviṭṭha}, apart from faith, endorsement, oral transmission, reasoned train of thought, or acceptance of a view after deliberation, I know and see that rebirth is a condition for old age and death.” 

“Reverend\marginnote{2.1} \textsanskrit{Musīla}, apart from faith, endorsement, oral transmission, reasoned train of thought, or acceptance of a view after deliberateation, do you know for yourself that continued existence is a condition for rebirth … grasping is a condition for continued existence … craving is a condition for grasping … feeling is a condition for craving … contact is a condition for feeling … the six sense fields are conditions for contact … name and form are conditions for the six sense fields … consciousness is a condition for name and form … choices are a condition for consciousness … ignorance is a condition for choices?” 

“Reverend\marginnote{2.12} \textsanskrit{Saviṭṭha}, apart from faith, endorsement, oral transmission, reasoned train of thought, or acceptance of a view after deliberation, I know and see that ignorance is a condition for choices.” 

“Reverend\marginnote{3.1} \textsanskrit{Musīla}, apart from faith, endorsement, oral transmission, reasoned train of thought, or acceptance of a view after deliberation, do you know for yourself that when rebirth ceases, old age and death cease?” 

“Reverend\marginnote{3.3} \textsanskrit{Saviṭṭha}, apart from faith, endorsement, oral transmission, reasoned train of thought, or acceptance of a view after deliberation, I know and see that when rebirth ceases, old age and death cease.” 

“Reverend\marginnote{4.1} \textsanskrit{Musīla}, apart from faith, endorsement, oral transmission, reasoned train of thought, or acceptance of a view after deliberation, do you know for yourself that when continued existence ceases, rebirth ceases … when grasping ceases, continued existence ceases … when craving ceases, grasping ceases … when feeling ceases, craving ceases … when contact ceases, feeling ceases … when the six sense fields cease, contact ceases … when name and form cease, the six sense fields cease … when consciousness ceases name and form cease … when choices cease consciousness ceases … when ignorance ceases, choices cease?” 

“Reverend\marginnote{4.12} \textsanskrit{Saviṭṭha}, apart from faith, endorsement, oral transmission, reasoned train of thought, or acceptance of a view after deliberation, I know and see that when ignorance ceases, choices cease.” 

“Reverend\marginnote{5.1} \textsanskrit{Musīla}, apart from faith, endorsement, oral transmission, reasoned train of thought, or acceptance of a view after deliberation, do you know for yourself that the cessation of continued existence is extinguishment?” 

“Reverend\marginnote{5.3} \textsanskrit{Saviṭṭha}, apart from faith, endorsement, oral transmission, reasoned train of thought, or acceptance of a view after deliberation, I know and see that the cessation of continued existence is extinguishment.” 

“Then\marginnote{6.1} Venerable \textsanskrit{Musīla} is a perfected one, with defilements ended.” When he said this, \textsanskrit{Musīla} kept silent. 

Then\marginnote{7.1} Venerable \textsanskrit{Nārada} said to Venerable \textsanskrit{Saviṭṭha}, “Reverend \textsanskrit{Saviṭṭha}, please let me answer these questions. Ask me and I will answer them for you.” 

“By\marginnote{7.5} all means, Venerable \textsanskrit{Nārada}, try these questions. I’ll ask you and you can answer them for me.” 

(\textsanskrit{Saviṭṭha}\marginnote{8.1} repeats exactly the same series of questions, and \textsanskrit{Nārada} answers just as \textsanskrit{Musīla} did.) 

“Reverend\marginnote{12.1} \textsanskrit{Nārada}, apart from faith, endorsement, oral transmission, reasoned train of thought, or acceptance of a view after deliberation, do you know for yourself that the cessation of continued existence is extinguishment?” 

“Reverend\marginnote{12.3} \textsanskrit{Saviṭṭha}, apart from faith, endorsement, oral transmission, reasoned train of thought, or acceptance of a view after deliberation, I know and see that the cessation of continued existence is extinguishment.” 

“Then\marginnote{13.1} Venerable \textsanskrit{Nārada} is a perfected one, with defilements ended.” 

“I\marginnote{13.2} have truly seen clearly with right wisdom that the cessation of continued existence is extinguishment. Yet I am not a perfected one. Suppose there was a well on a desert road that had neither rope nor bucket. Then along comes a person struggling in the oppressive heat, weary, thirsty, and parched. They’d know that there was water, but they couldn’t physically touch it. 

In\marginnote{13.6} the same way, I have truly seen clearly with right wisdom that the cessation of continued existence is extinguishment. Yet I am not a perfected one.” 

When\marginnote{14.1} he said this, Venerable Ānanda said to Venerable \textsanskrit{Saviṭṭha}, “Reverend \textsanskrit{Saviṭṭha}, what do you have to say to Venerable \textsanskrit{Nārada} when he speaks like this?” 

“Reverend\marginnote{14.3} Ānanda, I have nothing to say to Venerable \textsanskrit{Nārada} when he speaks like this, except what is good and wholesome.” 

%
\section*{{\suttatitleacronym SN 12.69}{\suttatitletranslation Surge }{\suttatitleroot Upayantisutta}}
\addcontentsline{toc}{section}{\tocacronym{SN 12.69} \toctranslation{Surge } \tocroot{Upayantisutta}}
\markboth{Surge }{Upayantisutta}
\extramarks{SN 12.69}{SN 12.69}

\scevam{So\marginnote{1.1} I have heard. }At one time the Buddha was staying near \textsanskrit{Sāvatthī} in Jeta’s Grove, \textsanskrit{Anāthapiṇḍika}’s monastery. 

“Mendicants,\marginnote{1.4} when the ocean surges it makes the rivers surge. When the rivers surge they make the streams surge. When the streams surge they make the lakes surge. When the lakes surge they make the ponds surge. 

In\marginnote{1.5} the same way, when ignorance surges it makes choices surge. When choices surge they make consciousness surge. When consciousness surges it makes name and form surge. When name and form surge they make the six sense fields surge. When the six sense fields surge they make contact surge. When contact surges it makes feeling surge. When feeling surges it makes craving surge. When craving surges it makes grasping surge. When grasping surges it makes continued existence surge. When continued existence surges it makes rebirth surge. When rebirth surges it makes old age and death surge. 

When\marginnote{2.1} the ocean recedes it makes the rivers recede. When the rivers recede they make the streams recede. When the streams recede they make the lakes recede. When the lakes recede they make the ponds recede. 

In\marginnote{2.2} the same way, when ignorance recedes it makes choices recede. When choices recede they make consciousness recede. When consciousness recedes it makes name and form recede. When name and form recede they make the six sense fields recede. When the six sense fields recede they make contact recede. When contact recedes it makes feeling recede. When feeling recedes it makes craving recede. When craving recedes it makes grasping recede. When grasping recedes it makes continued existence recede. When continued existence recedes it makes rebirth recede. When rebirth recedes it makes old age and death recede.” 

%
\section*{{\suttatitleacronym SN 12.70}{\suttatitletranslation The Wanderer Susīma }{\suttatitleroot Susimaparibbājakasutta}}
\addcontentsline{toc}{section}{\tocacronym{SN 12.70} \toctranslation{The Wanderer Susīma } \tocroot{Susimaparibbājakasutta}}
\markboth{The Wanderer Susīma }{Susimaparibbājakasutta}
\extramarks{SN 12.70}{SN 12.70}

\scevam{So\marginnote{1.1} I have heard. }At one time the Buddha was staying near \textsanskrit{Rājagaha}, in the Bamboo Grove, the squirrels’ feeding ground. 

Now\marginnote{1.3} at that time the Buddha was honored, respected, revered, venerated, and esteemed. And he received robes, almsfood, lodgings, and medicines and supplies for the sick. And the mendicant \textsanskrit{Saṅgha} was also honored, respected, revered, venerated, and esteemed. And they received robes, almsfood, lodgings, and medicines and supplies for the sick. But the wanderers who followed other religions were not honored, respected, revered, venerated, and esteemed. And they didn’t receive robes, almsfood, lodgings, and medicines and supplies for the sick. 

Now\marginnote{2.1} at that time the wanderer \textsanskrit{Susīma} was residing at \textsanskrit{Rājagaha} together with a large community of wanderers. 

Then\marginnote{2.2} his community said to \textsanskrit{Susīma}, “Reverend \textsanskrit{Susīma}, please lead the spiritual life with the ascetic Gotama. Memorize that teaching and have us recite it with you. When we’ve memorized it we’ll recite it to the laity. In this way we too will be honored, respected, revered, venerated, and esteemed. And we’ll receive robes, almsfood, lodgings, and medicines and supplies for the sick.” 

“Yes,\marginnote{2.7} reverends,” replied \textsanskrit{Susīma}. Then he went to Venerable Ānanda, and exchanged greetings with him. When the greetings and polite conversation were over, he sat down to one side and said to Ānanda, “Reverend Ānanda, I wish to lead the spiritual life in this teaching and training.” 

Then\marginnote{3.1} Ānanda took \textsanskrit{Susīma} to the Buddha, bowed, sat down to one side, and said to him, “Sir, this wanderer \textsanskrit{Susīma} says that he wishes to lead the spiritual life in this teaching and training.” 

“Well\marginnote{3.4} then, Ānanda, give \textsanskrit{Susīma} the going forth.” And the wanderer \textsanskrit{Susīma} received the going forth, the ordination in the Buddha’s presence. 

Now\marginnote{4.1} at that time several mendicants had declared their enlightenment in the Buddha’s presence: “We understand: ‘Rebirth is ended, the spiritual journey has been completed, what had to be done has been done, there is nothing further for this place.’” 

Venerable\marginnote{4.3} \textsanskrit{Susīma} heard about this. He went up to those mendicants, and exchanged greetings with them. When the greetings and polite conversation were over, he sat down to one side and said to those mendicants, “Is it really true that the venerables have declared enlightenment in the Buddha’s presence?” 

“Yes,\marginnote{4.10} reverend.” 

“But\marginnote{5.1} knowing and seeing thus, do you wield the many kinds of psychic power? That is, multiplying yourselves and becoming one again; appearing and disappearing; going unobstructed through a wall, a rampart, or a mountain as if through space; diving in and out of the earth as if it were water; walking on water as if it were earth; flying cross-legged through the sky like a bird; touching and stroking with the hand the sun and moon, so mighty and powerful. Do you control the body as far as the realm of divinity?” 

“No,\marginnote{5.2} reverend.” 

“Well,\marginnote{6.1} knowing and seeing thus do you, with clairaudience that is purified and superhuman, hear both kinds of sounds, human and heavenly, whether near or far?” 

“No,\marginnote{6.2} reverend.” 

“Well,\marginnote{7.1} knowing and seeing thus, do you understand the minds of other beings and individuals, having comprehended them with your mind? Do you understand mind with greed as ‘mind with greed’, and mind without greed as ‘mind without greed’? Do you understand mind with hate as ‘mind with hate’, and mind without hate as ‘mind without hate’? Do you understand mind with delusion as ‘mind with delusion’, and mind without delusion as ‘mind without delusion’? Do you understand constricted mind as ‘constricted mind’, and scattered mind as ‘scattered mind’? Do you understand expansive mind as ‘expansive mind’, and unexpansive mind as ‘unexpansive mind’? Do you understand mind that is not supreme as ‘mind that is not supreme’, and mind that is supreme as ‘mind that is supreme’? Do you understand mind immersed in \textsanskrit{samādhi} as ‘mind immersed in \textsanskrit{samādhi}’, and mind not immersed in \textsanskrit{samādhi} as ‘mind not immersed in \textsanskrit{samādhi}’? Do you understand freed mind as ‘freed mind’, and unfreed mind as ‘unfreed mind’?” 

“No,\marginnote{7.18} reverend.” 

“Well,\marginnote{8.1} knowing and seeing thus, do you recollect many kinds of past lives. That is: one, two, three, four, five, ten, twenty, thirty, forty, fifty, a hundred, a thousand, a hundred thousand rebirths; many eons of the world contracting, many eons of the world expanding, many eons of the world contracting and expanding? Do you remember: ‘There, I was named this, my clan was that, I looked like this, and that was my food. This was how I felt pleasure and pain, and that was how my life ended. When I passed away from that place I was reborn somewhere else. There, too, I was named this, my clan was that, I looked like this, and that was my food. This was how I felt pleasure and pain, and that was how my life ended. When I passed away from that place I was reborn here.’ Do you recollect your many kinds of past lives, with features and details?” 

“No,\marginnote{8.2} reverend.” 

“Well,\marginnote{9.1} knowing and seeing thus, do you, with clairvoyance that is purified and superhuman, see sentient beings passing away and being reborn—inferior and superior, beautiful and ugly, in a good place or a bad place—and understand how sentient beings are reborn according to their deeds? ‘These dear beings did bad things by way of body, speech, and mind. They denounced the noble ones; they had wrong view; and they chose to act out of that wrong view. When their body breaks up, after death, they’re reborn in a place of loss, a bad place, the underworld, hell. These dear beings, however, did good things by way of body, speech, and mind. They never denounced the noble ones; they had right view; and they chose to act out of that right view. When their body breaks up, after death, they’re reborn in a good place, a heavenly realm.’ And so, with clairvoyance that is purified and superhuman, do you see sentient beings passing away and being reborn—inferior and superior, beautiful and ugly, in a good place or a bad place. And do you understand how sentient beings are reborn according to their deeds?” 

“No,\marginnote{9.2} reverend.” 

“Well,\marginnote{10.1} knowing and seeing thus, do you have direct meditative experience of the peaceful liberations that are formless, transcending form?” 

“No,\marginnote{10.2} reverend.” 

“Well\marginnote{11.1} now, venerables, how could there be such a declaration when these things are not attained?” 

“Reverend\marginnote{11.2} \textsanskrit{Susīma}, we are freed by wisdom.” 

“I\marginnote{12.1} don’t understand the detailed meaning of what you have said in brief. Please teach me this matter so I can understand the detailed meaning.” 

“Reverend\marginnote{12.3} \textsanskrit{Susīma}, whether you understand or not, we are freed by wisdom.” 

Then\marginnote{13.1} \textsanskrit{Susīma} went to the Buddha, bowed, sat down to one side, and informed the Buddha of all he had discussed with those mendicants. 

“\textsanskrit{Susīma},\marginnote{13.3} first comes knowledge of the stability of natural principles. Afterwards there is knowledge of extinguishment.” 

“Sir,\marginnote{14.1} I don’t understand the detailed meaning of what you have said in brief. Please teach me this matter so I can understand the detailed meaning.” 

“\textsanskrit{Susīma},\marginnote{14.3} whether you understand or not, first comes knowledge of the stability of natural principles. Afterwards there is knowledge of extinguishment. 

What\marginnote{15.1} do you think, \textsanskrit{Susīma}? Is form permanent or impermanent?” 

“Impermanent,\marginnote{15.3} sir.” 

“But\marginnote{15.4} if it’s impermanent, is it suffering or happiness?” 

“Suffering,\marginnote{15.5} sir.” 

“But\marginnote{15.6} if it’s impermanent, suffering, and perishable, is it fit to be regarded thus: ‘This is mine, I am this, this is my self’?” 

“No,\marginnote{15.8} sir.” 

“Is\marginnote{15.9} feeling permanent or impermanent?” 

“Impermanent,\marginnote{15.10} sir.” 

“But\marginnote{15.11} if it’s impermanent, is it suffering or happiness?” 

“Suffering,\marginnote{15.12} sir.” 

“But\marginnote{15.13} if it’s impermanent, suffering, and perishable, is it fit to be regarded thus: ‘This is mine, I am this, this is my self’?” 

“No,\marginnote{15.15} sir.” 

“Is\marginnote{15.16} perception permanent or impermanent?” 

“Impermanent,\marginnote{15.17} sir.” … 

“Are\marginnote{15.18} choices permanent or impermanent?” 

“Impermanent,\marginnote{15.19} sir.” 

“But\marginnote{15.20} if they’re impermanent, are they suffering or happiness?” 

“Suffering,\marginnote{15.21} sir.” 

“But\marginnote{15.22} if they’re impermanent, suffering, and perishable, are they fit to be regarded thus: ‘This is mine, I am this, this is my self’?” 

“No,\marginnote{15.24} sir.” 

“Is\marginnote{15.25} consciousness permanent or impermanent?” 

“Impermanent,\marginnote{15.26} sir.” 

“But\marginnote{15.27} if it’s impermanent, is it suffering or happiness?” 

“Suffering,\marginnote{15.28} sir.” 

“But\marginnote{15.29} if it’s impermanent, suffering, and perishable, is it fit to be regarded thus: ‘This is mine, I am this, this is my self’?” 

“No,\marginnote{15.31} sir.” 

“So,\marginnote{16.1} \textsanskrit{Susīma}, you should truly see any kind of form at all—past, future, or present; internal or external; solid or subtle; inferior or superior; far or near: \emph{all} form—with right understanding: ‘This is not mine, I am not this, this is not my self.’ You should truly see any kind of feeling at all—past, future, or present; internal or external; solid or subtle; inferior or superior; far or near: \emph{all} feeling—with right understanding: ‘This is not mine, I am not this, this is not my self.’ You should truly see any kind of perception at all—past, future, or present; internal or external; solid or subtle; inferior or superior; far or near: \emph{all} perception—with right understanding: ‘This is not mine, I am not this, this is not my self.’ You should truly see any kind of choices at all—past, future, or present; internal or external; solid or subtle; inferior or superior; far or near: \emph{all} choices—with right understanding: ‘This is not mine, I am not this, this is not my self.’ You should truly see any kind of consciousness at all—past, future, or present; internal or external; solid or subtle; inferior or superior; far or near: \emph{all} consciousness—with right understanding: ‘This is not mine, I am not this, this is not my self.’ 

Seeing\marginnote{17.1} this, a learned noble disciple grows disillusioned with form, feeling, perception, choices, and consciousness. Being disillusioned, desire fades away. When desire fades away they’re freed. When they’re freed, they know they’re freed. 

They\marginnote{17.3} understand: ‘Rebirth is ended, the spiritual journey has been completed, what had to be done has been done, there is nothing further for this place.’ 

\textsanskrit{Susīma},\marginnote{18.1} do you see that rebirth is a condition for old age and death?” 

“Yes,\marginnote{18.2} sir.” 

“Do\marginnote{18.3} you see that continued existence is a condition for rebirth?” 

“Yes,\marginnote{18.4} sir.” 

“Do\marginnote{18.5} you see that grasping is a condition for continued existence?” 

“Yes,\marginnote{18.6} sir.” 

“Do\marginnote{18.7} you see that craving is a condition for grasping?” 

“Yes,\marginnote{18.8} sir.” 

“Do\marginnote{18.9} you see that feeling is a condition for craving … contact is a condition for feeling … the six sense fields are conditions for contact … name and form are conditions for the six sense fields … consciousness is a condition for name and form … choices are a condition for consciousness … ignorance is a condition for choices?” 

“Yes,\marginnote{18.16} sir.” 

“Do\marginnote{19.1} you see that when rebirth ceases old age and death cease?” 

“Yes,\marginnote{19.2} sir.” 

“Do\marginnote{19.3} you see that when continued existence ceases rebirth ceases?” 

“Yes,\marginnote{19.4} sir.” 

“Do\marginnote{19.5} you see that when grasping ceases continued existence ceases … when craving ceases, grasping ceases … when feeling ceases, craving ceases … when contact ceases, feeling ceases … when the six sense fields cease, contact ceases … when name and form cease, the six sense fields cease … when consciousness ceases name and form cease … when choices cease consciousness ceases … when ignorance ceases choices cease?” 

“Yes,\marginnote{19.14} sir.” 

“But\marginnote{20.1} knowing and seeing thus, do you wield the many kinds of psychic power? …” 

“No,\marginnote{20.2} sir.” 

“Well,\marginnote{21.1} knowing and seeing thus do you, with clairaudience that is purified and superhuman, hear both kinds of sounds, human and heavenly, whether near or far?” 

“No,\marginnote{21.2} sir.” 

“Well,\marginnote{22.1} knowing and seeing thus, do you understand the minds of other beings and individuals, having comprehended them with your mind? …” 

“No,\marginnote{22.4} sir.” 

“Well,\marginnote{23.1} knowing and seeing thus, do you recollect many kinds of past lives, with features and details?” 

“No,\marginnote{23.2} sir.” 

“Well,\marginnote{24.1} knowing and seeing thus, do you, with clairvoyance that is purified and superhuman, see sentient beings passing away and being reborn … according to their deeds?” 

“No,\marginnote{24.2} sir.” 

“Well,\marginnote{25.1} knowing and seeing thus, do you have direct meditative experience of the peaceful liberations that are formless, transcending form?” 

“No,\marginnote{25.2} sir.” 

“Well\marginnote{26.1} now, \textsanskrit{Susīma}, how could there be such a declaration when these things are not attained?” 

Then\marginnote{27.1} Venerable \textsanskrit{Susīma} bowed with his head at the Buddha’s feet and said, “I have made a mistake, sir. It was foolish, stupid, and unskillful of me to go forth as a thief in such a well-explained teaching and training. Please, sir, accept my mistake for what it is, so I will restrain myself in future.” 

“Indeed,\marginnote{28.1} \textsanskrit{Susīma}, you made a mistake. It was foolish, stupid, and unskillful of you to go forth as a thief in such a well-explained teaching and training. Suppose they were to arrest a bandit, a criminal and present him to the king, saying: ‘Your Majesty, this is a bandit, a criminal. Punish him as you will.’ The king would say: ‘Go, my men, and tie this man’s arms tightly behind his back with a strong rope. Shave his head and march him from street to street and from square to square to the beating of a harsh drum. Then take him out the south gate and there, to the south of the city, chop off his head.’ The king’s men would do as they were told. What do you think, \textsanskrit{Susīma}? Wouldn’t that man experience pain and distress because of that?” 

“Yes,\marginnote{28.9} sir.” 

“Although\marginnote{29.1} that man would experience pain and distress because of that, going forth as a thief in such a well-explained teaching and training has a more painful and bitter result. And it even leads to the underworld. But since you have recognized your mistake for what it is, and have dealt with it properly, I accept it. For it is growth in the training of the Noble One to recognize a mistake for what it is, deal with it properly, and commit to restraint in the future.” 

%
\addtocontents{toc}{\let\protect\contentsline\protect\nopagecontentsline}
\chapter*{The Chapter on Ascetics and Brahmins }
\addcontentsline{toc}{chapter}{\tocchapterline{The Chapter on Ascetics and Brahmins }}
\addtocontents{toc}{\let\protect\contentsline\protect\oldcontentsline}

%
\section*{{\suttatitleacronym SN 12.71}{\suttatitletranslation Old Age and Death }{\suttatitleroot Jarāmaraṇasutta}}
\addcontentsline{toc}{section}{\tocacronym{SN 12.71} \toctranslation{Old Age and Death } \tocroot{Jarāmaraṇasutta}}
\markboth{Old Age and Death }{Jarāmaraṇasutta}
\extramarks{SN 12.71}{SN 12.71}

\scevam{So\marginnote{1.1} I have heard. }At one time the Buddha was staying near \textsanskrit{Sāvatthī} in Jeta’s Grove, \textsanskrit{Anāthapiṇḍika}’s monastery. 

“Mendicants,\marginnote{1.4} there are ascetics and brahmins who don’t understand old age and death, their origin, their cessation, and the practice that leads to their cessation. I don’t deem them as true ascetics and brahmins. Those venerables don’t realize the goal of life as an ascetic or brahmin, and don’t live having realized it with their own insight. 

There\marginnote{2.1} are ascetics and brahmins who do understand old age and death, their origin, their cessation, and the practice that leads to their cessation. I deem them as true ascetics and brahmins. Those venerables realize the goal of life as an ascetic or brahmin, and live having realized it with their own insight.” 

%
\section*{{\suttatitleacronym SN 12.72–81}{\suttatitletranslation A Set of Ten on Rebirth, Etc. }{\suttatitleroot Jātisuttādidasaka}}
\addcontentsline{toc}{section}{\tocacronym{SN 12.72–81} \toctranslation{A Set of Ten on Rebirth, Etc. } \tocroot{Jātisuttādidasaka}}
\markboth{A Set of Ten on Rebirth, Etc. }{Jātisuttādidasaka}
\extramarks{SN 12.72–81}{SN 12.72–81}

At\marginnote{1.1} \textsanskrit{Sāvatthī}. 

“…\marginnote{1.2} they don’t understand rebirth …” 

“…\marginnote{2.1} continued existence …” 

“…\marginnote{3.1} grasping …” 

“…\marginnote{4.1} craving …” 

“…\marginnote{5.1} feeling …” 

“…\marginnote{6.1} contact …” 

“…\marginnote{7.1} the six sense fields …” 

“…\marginnote{8.1} name and form …” 

“…\marginnote{9.1} consciousness …” 

“…\marginnote{10.1} choices … 

…\marginnote{10.2} they understand …” 

%
\addtocontents{toc}{\let\protect\contentsline\protect\nopagecontentsline}
\pannasa{Consecutive Repetitions }
\addcontentsline{toc}{pannasa}{Consecutive Repetitions }
\markboth{}{}
\addtocontents{toc}{\let\protect\contentsline\protect\oldcontentsline}

%
\addtocontents{toc}{\let\protect\contentsline\protect\nopagecontentsline}
\chapter*{The Teacher, etc. }
\addcontentsline{toc}{chapter}{\tocchapterline{The Teacher, etc. }}
\addtocontents{toc}{\let\protect\contentsline\protect\oldcontentsline}

%
\section*{{\suttatitleacronym SN 12.82}{\suttatitletranslation The Teacher }{\suttatitleroot Satthusutta}}
\addcontentsline{toc}{section}{\tocacronym{SN 12.82} \toctranslation{The Teacher } \tocroot{Satthusutta}}
\markboth{The Teacher }{Satthusutta}
\extramarks{SN 12.82}{SN 12.82}

At\marginnote{1.1} \textsanskrit{Sāvatthī}. 

“Mendicants,\marginnote{1.2} one who does not truly know or see old age and death should seek the Teacher so as to truly know old age and death. One who does not truly know or see the origin of old age and death should seek the Teacher so as to truly know the origin of old age and death. One who does not truly know or see the cessation of old age and death should seek the Teacher so as to truly know the cessation of old age and death. One who does not truly know or see the practice that leads to the cessation of old age and death should seek the Teacher so as to truly know the practice that leads to the cessation of old age and death.” 

\scexpansioninstructions{(All the abbreviated texts should be told in full.) }

%
\section*{{\suttatitleacronym SN 12.83–92}{\suttatitletranslation The Teacher (2nd) }{\suttatitleroot Dutiyasatthusuttādidasaka}}
\addcontentsline{toc}{section}{\tocacronym{SN 12.83–92} \toctranslation{The Teacher (2nd) } \tocroot{Dutiyasatthusuttādidasaka}}
\markboth{The Teacher (2nd) }{Dutiyasatthusuttādidasaka}
\extramarks{SN 12.83–92}{SN 12.83–92}

“Mendicants,\marginnote{1.1} one who does not truly know or see rebirth …” 

“…\marginnote{1.1} continued existence …” 

“…\marginnote{1.1} grasping …” 

“…\marginnote{1.1} craving …” 

“…\marginnote{1.1} feeling …” 

“…\marginnote{1.1} contact …” 

“…\marginnote{1.1} the six sense fields …” 

“…\marginnote{1.1} name and form …” 

“…\marginnote{1.1} consciousness …” 

“…\marginnote{1.1} choices …” 

\scexpansioninstructions{(All should be treated according to the four truths.) }

%
\addtocontents{toc}{\let\protect\contentsline\protect\nopagecontentsline}
\chapter*{Sets of Eleven on Training, Etc. }
\addcontentsline{toc}{chapter}{\tocchapterline{Sets of Eleven on Training, Etc. }}
\addtocontents{toc}{\let\protect\contentsline\protect\oldcontentsline}

%
\section*{{\suttatitleacronym SN 12.93–213}{\suttatitletranslation Sets of Eleven on Training, Etc. }{\suttatitleroot Sikkhāsuttādipeyyālaekādasaka}}
\addcontentsline{toc}{section}{\tocacronym{SN 12.93–213} \toctranslation{Sets of Eleven on Training, Etc. } \tocroot{Sikkhāsuttādipeyyālaekādasaka}}
\markboth{Sets of Eleven on Training, Etc. }{Sikkhāsuttādipeyyālaekādasaka}
\extramarks{SN 12.93–213}{SN 12.93–213}

“Mendicants,\marginnote{1.1} one who does not truly know or see old age and death should train so as to truly know old age and death. …” 

“…\marginnote{1.1} practice meditation …” 

“…\marginnote{1.1} rouse up enthusiasm …” 

“…\marginnote{1.1} try vigorously …” 

“…\marginnote{1.1} persevere …” 

“…\marginnote{1.1} be keen …” 

“…\marginnote{1.1} rouse up energy …” 

“…\marginnote{1.1} persist …” 

“…\marginnote{1.1} be mindful …” 

“…\marginnote{1.1} use situational awareness …” 

“…\marginnote{1.1} be diligent …” 

\scendkanda{The Linked Discourses on causality are complete. }

%
\addtocontents{toc}{\let\protect\contentsline\protect\nopagecontentsline}
\part*{Linked Discourses on Comprehension }
\addcontentsline{toc}{part}{Linked Discourses on Comprehension }
\markboth{}{}
\addtocontents{toc}{\let\protect\contentsline\protect\oldcontentsline}

%
\addtocontents{toc}{\let\protect\contentsline\protect\nopagecontentsline}
\chapter*{The Chapter on Comprehension }
\addcontentsline{toc}{chapter}{\tocchapterline{The Chapter on Comprehension }}
\addtocontents{toc}{\let\protect\contentsline\protect\oldcontentsline}

%
\section*{{\suttatitleacronym SN 13.1}{\suttatitletranslation A Fingernail }{\suttatitleroot Nakhasikhāsutta}}
\addcontentsline{toc}{section}{\tocacronym{SN 13.1} \toctranslation{A Fingernail } \tocroot{Nakhasikhāsutta}}
\markboth{A Fingernail }{Nakhasikhāsutta}
\extramarks{SN 13.1}{SN 13.1}

\scevam{So\marginnote{1.1} I have heard. }At one time the Buddha was staying near \textsanskrit{Sāvatthī} in Jeta’s Grove, \textsanskrit{Anāthapiṇḍika}’s monastery. 

Then\marginnote{1.3} the Buddha, picking up a little bit of dirt under his fingernail, addressed the mendicants: “What do you think, mendicants? Which is more: the little bit of dirt under my fingernail, or this great earth?” 

“Sir,\marginnote{2.1} the great earth is far more. The little bit of dirt under your fingernail is tiny. Compared to the great earth, it’s not nearly a hundredth, a thousandth, or a hundred thousandth part.” 

“In\marginnote{2.4} the same way, for a noble disciple accomplished in view, a person with comprehension, the suffering that’s over and done with is more, what’s left is tiny. Compared to the mass of suffering in the past that’s over and done with, it’s not nearly a hundredth, a thousandth, or a hundred thousandth part, since there are at most seven more lives. That’s how very beneficial it is to comprehend the teaching and gain the vision of the teaching.” 

%
\section*{{\suttatitleacronym SN 13.2}{\suttatitletranslation A Lotus Pond }{\suttatitleroot Pokkharaṇīsutta}}
\addcontentsline{toc}{section}{\tocacronym{SN 13.2} \toctranslation{A Lotus Pond } \tocroot{Pokkharaṇīsutta}}
\markboth{A Lotus Pond }{Pokkharaṇīsutta}
\extramarks{SN 13.2}{SN 13.2}

At\marginnote{1.1} \textsanskrit{Sāvatthī}. 

“Mendicants,\marginnote{1.2} suppose there was a lotus pond that was fifty leagues long, fifty leagues wide, and fifty leagues deep, full to the brim so a crow could drink from it. Then a person would pick up some water on the tip of a blade of grass. 

What\marginnote{1.4} do you think, mendicants? Which is more: the water on the tip of the blade of grass, or the water in the lotus pond?” 

“Sir,\marginnote{2.1} the water in the lotus pond is certainly more. The water on the tip of a blade of grass is tiny. Compared to the water in the lotus pond, it’s not nearly a hundredth, a thousandth, or a hundred thousandth part.” 

“In\marginnote{3.1} the same way, for a person with comprehension, a noble disciple accomplished in view, the suffering that’s over and done with is more, what’s left is tiny. Compared to the mass of suffering in the past that’s over and done with, it’s not nearly a hundredth, a thousandth, or a hundred thousandth part, since there are at most seven more lives. That’s how very beneficial it is to comprehend the teaching and gain the vision of the teaching.” 

%
\section*{{\suttatitleacronym SN 13.3}{\suttatitletranslation Where the Waters Flow Together }{\suttatitleroot Sambhejjaudakasutta}}
\addcontentsline{toc}{section}{\tocacronym{SN 13.3} \toctranslation{Where the Waters Flow Together } \tocroot{Sambhejjaudakasutta}}
\markboth{Where the Waters Flow Together }{Sambhejjaudakasutta}
\extramarks{SN 13.3}{SN 13.3}

At\marginnote{1.1} \textsanskrit{Sāvatthī}. 

“Mendicants,\marginnote{1.2} there are places where the great rivers—the Ganges, Yamuna, \textsanskrit{Aciravatī}, \textsanskrit{Sarabhū}, and \textsanskrit{Mahī}—come together and converge. Suppose a person was to draw two or three drops of water from such a place. 

What\marginnote{1.4} do you think, mendicants? Which is more: the two or three drops drawn out or the water in the confluence?” 

“Sir,\marginnote{2.1} the water in the confluence is certainly more. The two or three drops drawn out are tiny. Compared to the water in the confluence, it’s not nearly a hundredth, a thousandth, or a hundred thousandth part.” 

“In\marginnote{2.4} the same way, for a noble disciple, the suffering that’s over and done with is more …” 

%
\section*{{\suttatitleacronym SN 13.4}{\suttatitletranslation Where the Waters Flow Together (2nd) }{\suttatitleroot Dutiyasambhejjaudakasutta}}
\addcontentsline{toc}{section}{\tocacronym{SN 13.4} \toctranslation{Where the Waters Flow Together (2nd) } \tocroot{Dutiyasambhejjaudakasutta}}
\markboth{Where the Waters Flow Together (2nd) }{Dutiyasambhejjaudakasutta}
\extramarks{SN 13.4}{SN 13.4}

At\marginnote{1.1} \textsanskrit{Sāvatthī}. 

“Mendicants,\marginnote{1.2} there are places where the great rivers—the Ganges, Yamuna, \textsanskrit{Aciravatī}, \textsanskrit{Sarabhū}, and \textsanskrit{Mahī}—come together and converge. Suppose that water dried up and evaporated except for two or three drops. 

What\marginnote{1.4} do you think, mendicants? Which is more: the water in the confluence that has dried up and evaporated, or the two or three drops left?” 

“Sir,\marginnote{2.1} the water in the confluence that has dried up and evaporated is certainly more. The two or three drops left are tiny. Compared to the water in the confluence that has dried up and evaporated, it’s not nearly a hundredth, a thousandth, or a hundred thousandth part.” 

“In\marginnote{2.4} the same way, for a noble disciple, the suffering that’s over and done with is more …” 

%
\section*{{\suttatitleacronym SN 13.5}{\suttatitletranslation The Earth }{\suttatitleroot Pathavīsutta}}
\addcontentsline{toc}{section}{\tocacronym{SN 13.5} \toctranslation{The Earth } \tocroot{Pathavīsutta}}
\markboth{The Earth }{Pathavīsutta}
\extramarks{SN 13.5}{SN 13.5}

At\marginnote{1.1} \textsanskrit{Sāvatthī}. 

“Mendicants,\marginnote{1.2} suppose a person was to place seven clay balls the size of jujube seeds on the great earth. 

What\marginnote{1.3} do you think, mendicants? Which is more: the seven clay balls the size of jujube seeds, or the great earth?” 

“Sir,\marginnote{2.1} the great earth is certainly more. The seven clay balls the size of jujube seeds are tiny. Compared to the great earth, it’s not nearly a hundredth, a thousandth, or a hundred thousandth part.” 

“In\marginnote{2.4} the same way, for a noble disciple, the suffering that’s over and done with is more …” 

%
\section*{{\suttatitleacronym SN 13.6}{\suttatitletranslation The Earth (2nd) }{\suttatitleroot Dutiyapathavīsutta}}
\addcontentsline{toc}{section}{\tocacronym{SN 13.6} \toctranslation{The Earth (2nd) } \tocroot{Dutiyapathavīsutta}}
\markboth{The Earth (2nd) }{Dutiyapathavīsutta}
\extramarks{SN 13.6}{SN 13.6}

At\marginnote{1.1} \textsanskrit{Sāvatthī}. 

“Mendicants,\marginnote{1.2} suppose the great earth was worn away and eroded except for seven clay balls the size of jujube seeds. 

What\marginnote{1.3} do you think, mendicants? Which is more: the great earth that has been worn away and eroded, or the seven clay balls the size of jujube seeds that are left?” 

“Sir,\marginnote{2.1} the great earth that has been worn away and eroded is certainly more. The seven clay balls the size of jujube seeds are tiny. Compared to the great earth that has been worn away and eroded, it’s not nearly a hundredth, a thousandth, or a hundred thousandth part.” 

“In\marginnote{2.4} the same way, for a noble disciple, the suffering that’s over and done with is more …” 

%
\section*{{\suttatitleacronym SN 13.7}{\suttatitletranslation The Ocean }{\suttatitleroot Samuddasutta}}
\addcontentsline{toc}{section}{\tocacronym{SN 13.7} \toctranslation{The Ocean } \tocroot{Samuddasutta}}
\markboth{The Ocean }{Samuddasutta}
\extramarks{SN 13.7}{SN 13.7}

At\marginnote{1.1} \textsanskrit{Sāvatthī}. 

“Mendicants,\marginnote{1.2} suppose a man was to draw up two or three drops of water from the ocean. 

What\marginnote{1.3} do you think, mendicants? Which is more: the two or three drops drawn out or the water in the ocean?” 

“Sir,\marginnote{2.1} the water in the ocean is certainly more. The two or three drops drawn out are tiny. Compared to the water in the ocean, it’s not nearly a hundredth, a thousandth, or a hundred thousandth part.” 

“In\marginnote{2.4} the same way, for a noble disciple, the suffering that’s over and done with is more …” 

%
\section*{{\suttatitleacronym SN 13.8}{\suttatitletranslation The Ocean (2nd) }{\suttatitleroot Dutiyasamuddasutta}}
\addcontentsline{toc}{section}{\tocacronym{SN 13.8} \toctranslation{The Ocean (2nd) } \tocroot{Dutiyasamuddasutta}}
\markboth{The Ocean (2nd) }{Dutiyasamuddasutta}
\extramarks{SN 13.8}{SN 13.8}

At\marginnote{1.1} \textsanskrit{Sāvatthī}. 

“Mendicants,\marginnote{1.2} suppose the water in the ocean dried up and evaporated except for two or three drops. 

What\marginnote{1.3} do you think, mendicants? Which is more: the water in the ocean that has dried up and evaporated, or the two or three drops left?” 

“Sir,\marginnote{2.1} the water in the ocean that has dried up and evaporated is certainly more. The two or three drops left are tiny. Compared to the water in the ocean that has dried up and evaporated, it’s not nearly a hundredth, a thousandth, or a hundred thousandth part.” 

“In\marginnote{2.4} the same way, for a noble disciple, the suffering that’s over and done with is more …” 

%
\section*{{\suttatitleacronym SN 13.9}{\suttatitletranslation A Mountain }{\suttatitleroot Pabbatasutta}}
\addcontentsline{toc}{section}{\tocacronym{SN 13.9} \toctranslation{A Mountain } \tocroot{Pabbatasutta}}
\markboth{A Mountain }{Pabbatasutta}
\extramarks{SN 13.9}{SN 13.9}

At\marginnote{1.1} \textsanskrit{Sāvatthī}. 

“Mendicants,\marginnote{1.2} suppose a person was to place seven pebbles the size of mustard seeds on the Himalayas, the king of mountains. 

What\marginnote{1.3} do you think, mendicants? Which is more: the seven pebbles the size of mustard seeds, or the Himalayas, the king of mountains?” 

“Sir,\marginnote{2.1} the Himalayas, the king of mountains, is certainly more. The seven pebbles the size of mustard seeds are tiny. Compared to the Himalayas, it’s not nearly a hundredth, a thousandth, or a hundred thousandth part.” 

“In\marginnote{2.4} the same way, for a noble disciple, the suffering that’s over and done with is more …” 

%
\section*{{\suttatitleacronym SN 13.10}{\suttatitletranslation A Mountain (2nd) }{\suttatitleroot Dutiyapabbatasutta}}
\addcontentsline{toc}{section}{\tocacronym{SN 13.10} \toctranslation{A Mountain (2nd) } \tocroot{Dutiyapabbatasutta}}
\markboth{A Mountain (2nd) }{Dutiyapabbatasutta}
\extramarks{SN 13.10}{SN 13.10}

At\marginnote{1.1} \textsanskrit{Sāvatthī}. 

“Mendicants,\marginnote{1.2} suppose the Himalayas, the king of mountains, was worn away and eroded except for seven pebbles the size of mustard seeds. 

What\marginnote{1.3} do you think, mendicants? Which is more: the portion of the Himalayas, the king of mountains, that has been worn away and eroded, or the seven pebbles the size of mustard seeds that are left?” 

“Sir,\marginnote{2.1} the portion of the Himalayas, the king of mountains, that has been worn away and eroded is certainly more. The seven pebbles the size of mustard seeds are tiny. Compared to the Himalayas, it’s not nearly a hundredth, a thousandth, or a hundred thousandth part.” 

“In\marginnote{3.1} the same way, for a noble disciple accomplished in view, a person with comprehension, the suffering that’s over and done with is more, what’s left is tiny. Compared to the mass of suffering in the past that’s over and done with, it’s not nearly a hundredth, a thousandth, or a hundred thousandth part, since there are at most seven more lives. That’s how very beneficial it is to comprehend the teaching and gain the vision of the teaching.” 

%
\section*{{\suttatitleacronym SN 13.11}{\suttatitletranslation A Mountain (3rd) }{\suttatitleroot Tatiyapabbatasutta}}
\addcontentsline{toc}{section}{\tocacronym{SN 13.11} \toctranslation{A Mountain (3rd) } \tocroot{Tatiyapabbatasutta}}
\markboth{A Mountain (3rd) }{Tatiyapabbatasutta}
\extramarks{SN 13.11}{SN 13.11}

At\marginnote{1.1} \textsanskrit{Sāvatthī}. 

“Mendicants,\marginnote{1.2} suppose a person was to place down on Sineru, the king of mountains, seven pebbles the size of mung beans. 

What\marginnote{1.3} do you think, mendicants? Which is more: the seven pebbles the size of mung beans, or Sineru, the king of mountains?” 

“Sir,\marginnote{2.1} Sineru, the king of mountains, is certainly more. The seven pebbles the size of mung beans are tiny. Compared to Sineru, it’s not nearly a hundredth, a thousandth, or a hundred thousandth part.” 

“In\marginnote{3.1} the same way, compared with the achievements of a noble disciple accomplished in view, the achievements of the ascetics, brahmins, and wanderers of other religions is not nearly a hundredth, a thousandth, or a hundred thousandth part. So great is the achievement of the person accomplished in view, so great is their direct knowledge.” 

\scendkanda{The Linked Discourses on comprehension are complete. }

%
\addtocontents{toc}{\let\protect\contentsline\protect\nopagecontentsline}
\part*{Linked Discourses on the Elements }
\addcontentsline{toc}{part}{Linked Discourses on the Elements }
\markboth{}{}
\addtocontents{toc}{\let\protect\contentsline\protect\oldcontentsline}

%
\addtocontents{toc}{\let\protect\contentsline\protect\nopagecontentsline}
\chapter*{The Chapter on Diversity }
\addcontentsline{toc}{chapter}{\tocchapterline{The Chapter on Diversity }}
\addtocontents{toc}{\let\protect\contentsline\protect\oldcontentsline}

%
\section*{{\suttatitleacronym SN 14.1}{\suttatitletranslation Diversity of Elements }{\suttatitleroot Dhātunānattasutta}}
\addcontentsline{toc}{section}{\tocacronym{SN 14.1} \toctranslation{Diversity of Elements } \tocroot{Dhātunānattasutta}}
\markboth{Diversity of Elements }{Dhātunānattasutta}
\extramarks{SN 14.1}{SN 14.1}

At\marginnote{1.1} \textsanskrit{Sāvatthī}. 

“Mendicants,\marginnote{1.2} I will teach you the diversity of elements. Listen and apply your mind well, I will speak.” 

“Yes,\marginnote{1.4} sir,” they replied. The Buddha said this: 

“And\marginnote{2.1} what is the diversity of elements? The eye element, sight element, and eye consciousness element. The ear element, sound element, and ear consciousness element. The nose element, smell element, and nose consciousness element. The tongue element, taste element, and tongue consciousness element. The body element, touch element, and body consciousness element. The mind element, idea element, and mind consciousness element. This is called the diversity of elements.” 

%
\section*{{\suttatitleacronym SN 14.2}{\suttatitletranslation Diversity of Contacts }{\suttatitleroot Phassanānattasutta}}
\addcontentsline{toc}{section}{\tocacronym{SN 14.2} \toctranslation{Diversity of Contacts } \tocroot{Phassanānattasutta}}
\markboth{Diversity of Contacts }{Phassanānattasutta}
\extramarks{SN 14.2}{SN 14.2}

At\marginnote{1.1} \textsanskrit{Sāvatthī}. 

“Mendicants,\marginnote{1.2} diversity of elements gives rise to diversity of contacts. And what is the diversity of elements? The eye element, ear element, nose element, tongue element, body element, and mind element. This is called the diversity of elements. 

And\marginnote{2.1} how does diversity of elements give rise to diversity of contacts? The eye element gives rise to eye contact. The ear element … nose … tongue … body … The mind element gives rise to mind contact. That’s how diversity of elements gives rise to diversity of contacts.” 

%
\section*{{\suttatitleacronym SN 14.3}{\suttatitletranslation Not Diversity of Contacts }{\suttatitleroot Nophassanānattasutta}}
\addcontentsline{toc}{section}{\tocacronym{SN 14.3} \toctranslation{Not Diversity of Contacts } \tocroot{Nophassanānattasutta}}
\markboth{Not Diversity of Contacts }{Nophassanānattasutta}
\extramarks{SN 14.3}{SN 14.3}

At\marginnote{1.1} \textsanskrit{Sāvatthī}. 

“Mendicants,\marginnote{1.2} diversity of elements gives rise to diversity of contacts. Diversity of contacts doesn’t give rise to diversity of elements. And what is the diversity of elements? The eye element, ear element, nose element, tongue element, body element, and mind element. This is called the diversity of elements. 

And\marginnote{2.1} how does diversity of elements give rise to diversity of contacts, while diversity of contacts doesn’t give rise to diversity of elements? The eye element gives rise to eye contact. Eye contact doesn’t give rise to the eye element. … The mind element gives rise to mind contact. Mind contact doesn’t give rise to the mind element. That’s how diversity of elements gives rise to diversity of contacts, while diversity of contacts doesn’t give rise to diversity of elements.” 

%
\section*{{\suttatitleacronym SN 14.4}{\suttatitletranslation Diversity of Feelings }{\suttatitleroot Vedanānānattasutta}}
\addcontentsline{toc}{section}{\tocacronym{SN 14.4} \toctranslation{Diversity of Feelings } \tocroot{Vedanānānattasutta}}
\markboth{Diversity of Feelings }{Vedanānānattasutta}
\extramarks{SN 14.4}{SN 14.4}

At\marginnote{1.1} \textsanskrit{Sāvatthī}. 

“Mendicants,\marginnote{1.2} diversity of elements gives rise to diversity of contacts, and diversity of contacts gives rise to diversity of feelings. And what is the diversity of elements? The eye element, ear element, nose element, tongue element, body element, and mind element. This is called the diversity of elements. 

And\marginnote{2.1} how does diversity of elements give rise to diversity of contacts, and diversity of contacts gives rise to diversity of feelings? The eye element gives rise to eye contact. Eye contact gives rise to the feeling born of eye contact. … The mind element gives rise to mind contact. Mind contact gives rise to the feeling born of mind contact. That’s how diversity of elements gives rise to diversity of contacts, and diversity of contacts gives rise to diversity of feelings.” 

%
\section*{{\suttatitleacronym SN 14.5}{\suttatitletranslation Diversity of Feelings (2nd) }{\suttatitleroot Dutiyavedanānānattasutta}}
\addcontentsline{toc}{section}{\tocacronym{SN 14.5} \toctranslation{Diversity of Feelings (2nd) } \tocroot{Dutiyavedanānānattasutta}}
\markboth{Diversity of Feelings (2nd) }{Dutiyavedanānānattasutta}
\extramarks{SN 14.5}{SN 14.5}

At\marginnote{1.1} \textsanskrit{Sāvatthī}. 

“Mendicants,\marginnote{1.2} diversity of elements gives rise to diversity of contacts. Diversity of contacts gives rise to diversity of feelings. Diversity of feelings doesn’t give rise to diversity of contacts. Diversity of contacts doesn’t give rise to diversity of elements. And what is the diversity of elements? The eye element, ear element, nose element, tongue element, body element, and mind element. This is called the diversity of elements. 

And\marginnote{2.1} how does diversity of elements give rise to diversity of contacts, and diversity of contacts give rise to diversity of feelings, while diversity of feelings doesn’t give rise to diversity of contacts, and diversity of contacts doesn’t give rise to diversity of elements? The eye element gives rise to eye contact. Eye contact gives rise to feeling born of eye contact. Feeling born of eye contact doesn’t give rise to eye contact. Eye contact doesn’t give rise to the eye element. … The mind element gives rise to mind contact. Mind contact gives rise to feeling born of mind contact. Feeling born of mind contact doesn’t give rise to mind contact. Mind contact doesn’t give rise to the mind element. That’s how diversity of elements gives rise to diversity of contacts, and diversity of contacts gives rise to diversity of feelings, while diversity of feelings doesn’t give rise to diversity of contacts, and diversity of contacts doesn’t give rise to diversity of elements.” 

%
\section*{{\suttatitleacronym SN 14.6}{\suttatitletranslation External Diversity of Elements }{\suttatitleroot Bāhiradhātunānattasutta}}
\addcontentsline{toc}{section}{\tocacronym{SN 14.6} \toctranslation{External Diversity of Elements } \tocroot{Bāhiradhātunānattasutta}}
\markboth{External Diversity of Elements }{Bāhiradhātunānattasutta}
\extramarks{SN 14.6}{SN 14.6}

At\marginnote{1.1} \textsanskrit{Sāvatthī}. 

“Mendicants,\marginnote{1.2} I will teach you the diversity of elements. And what is the diversity of elements? The sight element, the sound element, the smell element, the taste element, the touch element, and the idea element. This is called the diversity of elements.” 

%
\section*{{\suttatitleacronym SN 14.7}{\suttatitletranslation Diversity of Perceptions }{\suttatitleroot Saññānānattasutta}}
\addcontentsline{toc}{section}{\tocacronym{SN 14.7} \toctranslation{Diversity of Perceptions } \tocroot{Saññānānattasutta}}
\markboth{Diversity of Perceptions }{Saññānānattasutta}
\extramarks{SN 14.7}{SN 14.7}

At\marginnote{1.1} \textsanskrit{Sāvatthī}. 

“Mendicants,\marginnote{1.2} diversity of elements gives rise to diversity of perceptions. Diversity of perceptions gives rise to diversity of thoughts. Diversity of thoughts gives rise to diversity of desires. Diversity of desires gives rise to diversity of passions. Diversity of passions gives rise to diversity of searches. And what is the diversity of elements? The sight element, the sound element, the smell element, the taste element, the touch element, and the idea element. This is called the diversity of elements. 

And\marginnote{2.1} how does diversity of elements give rise to diversity of perceptions, and diversity of perceptions give rise to diversity of thoughts, and diversity of thoughts give rise to diversity of desires, and diversity of desires give rise to diversity of passions, and diversity of passions give rise to diversity of searches? 

The\marginnote{3.1} sight element gives rise to the perception of sights. The perception of sights gives rise to thoughts about sights. Thoughts about sights give rise to the desire for sights. The desire for sights gives rise to the passion for sights. The passion for sights gives rise to searching for sights. … The idea element gives rise to the perception of ideas. The perception of ideas gives rise to thoughts about ideas. Thoughts about ideas give rise to the desire for ideas. The desire for ideas gives rise to the passion for ideas. The passion for ideas gives rise to searching for ideas. 

That’s\marginnote{4.1} how diversity of elements gives rise to diversity of perceptions, and diversity of perceptions gives rise to diversity of thoughts, and diversity of thoughts gives rise to diversity of desires, and diversity of desires gives rise to diversity of passions, and diversity of passions gives rise to diversity of searches.” 

%
\section*{{\suttatitleacronym SN 14.8}{\suttatitletranslation No Diversity of Searches }{\suttatitleroot Nopariyesanānānattasutta}}
\addcontentsline{toc}{section}{\tocacronym{SN 14.8} \toctranslation{No Diversity of Searches } \tocroot{Nopariyesanānānattasutta}}
\markboth{No Diversity of Searches }{Nopariyesanānānattasutta}
\extramarks{SN 14.8}{SN 14.8}

At\marginnote{1.1} \textsanskrit{Sāvatthī}. 

“Mendicants,\marginnote{1.2} diversity of elements gives rise to diversity of perceptions. Diversity of perceptions gives rise to diversity of thoughts. Diversity of thoughts gives rise to diversity of desires. Diversity of desires gives rise to diversity of passions. Diversity of passions gives rise to diversity of searches. Diversity of searches doesn’t give rise to diversity of passions. Diversity of passions doesn’t give rise to diversity of desires. Diversity of desires doesn’t give rise to diversity of thoughts. Diversity of thoughts doesn’t give rise to diversity of perceptions. Diversity of perceptions doesn’t give rise to diversity of elements. And what is the diversity of elements? The sight element, the sound element, the smell element, the taste element, the touch element, and the idea element. This is called the diversity of elements. 

And\marginnote{2.1} how does diversity of elements give rise to diversity of perceptions … diversity of perceptions doesn’t give rise to diversity of elements? 

The\marginnote{3.1} sight element gives rise to the perception of sights … The idea element gives rise to the perception of ideas … the search for ideas. The search for ideas doesn’t give rise to the passion for ideas. The passion for ideas doesn’t give rise to the desire for ideas. The desire for ideas doesn’t give rise to thoughts about ideas. Thoughts about ideas don’t give rise to perceptions of ideas. Perceptions of ideas don’t give rise to the idea element. 

That’s\marginnote{4.1} how diversity of elements gives rise to diversity of perceptions … diversity of perceptions doesn’t give rise to diversity of elements.” 

%
\section*{{\suttatitleacronym SN 14.9}{\suttatitletranslation Diversity of Gains }{\suttatitleroot Bāhiraphassanānattasutta}}
\addcontentsline{toc}{section}{\tocacronym{SN 14.9} \toctranslation{Diversity of Gains } \tocroot{Bāhiraphassanānattasutta}}
\markboth{Diversity of Gains }{Bāhiraphassanānattasutta}
\extramarks{SN 14.9}{SN 14.9}

At\marginnote{1.1} \textsanskrit{Sāvatthī}. 

“Mendicants,\marginnote{1.2} diversity of elements gives rise to diversity of perceptions. Diversity of perceptions gives rise to diversity of thoughts. Diversity of thoughts gives rise to diversity of contacts. Diversity of contacts gives rise to diversity of feelings. Diversity of feelings gives rise to diversity of desires. Diversity of desires gives rise to diversity of passions. Diversity of passions gives rise to diversity of searches. Diversity of searches gives rise to diversity of gains. And what is the diversity of elements? The sight element, the sound element, the smell element, the taste element, the touch element, and the idea element. This is called the diversity of elements. 

And\marginnote{2.1} how does diversity of elements give rise to diversity of perceptions … diversity of searches give rise to diversity of gains? 

The\marginnote{3.1} sight element gives rise to the perception of sights. The perception of sights gives rise to thoughts about sights. Thoughts about sights give rise to sight contact. Sight contact gives rise to feeling born of sight contact. Feeling born of sight contact gives rise to the desire for sights. The desire for sights gives rise to the passion for sights. The passion for sights gives rise to searching for sights. Searching for sights gives rise to gaining sights … The idea element gives rise to the perception of ideas. The perception of ideas gives rise to thoughts about ideas. Thoughts about ideas give rise to idea contact. Idea contact gives rise to feeling born of idea contact. Feeling born of idea contact gives rise to the desire for ideas. The desire for ideas gives rise to the passion for ideas. The passion for ideas gives rise to searching for ideas. Searching for ideas gives rise to gaining ideas. 

That’s\marginnote{4.1} how diversity of elements gives rise to diversity of perceptions … diversity of searches gives rise to diversity of gains.” 

%
\section*{{\suttatitleacronym SN 14.10}{\suttatitletranslation No Diversity of Gains }{\suttatitleroot Dutiyabāhiraphassanānattasutta}}
\addcontentsline{toc}{section}{\tocacronym{SN 14.10} \toctranslation{No Diversity of Gains } \tocroot{Dutiyabāhiraphassanānattasutta}}
\markboth{No Diversity of Gains }{Dutiyabāhiraphassanānattasutta}
\extramarks{SN 14.10}{SN 14.10}

At\marginnote{1.1} \textsanskrit{Sāvatthī}. 

“Mendicants,\marginnote{1.2} diversity of elements gives rise to diversity of perceptions. Diversity of perceptions gives rise to diversity of thoughts. … contacts … feelings … desires … passions … Diversity of searches gives rise to diversity of gains. Diversity of gains doesn’t give rise to diversity of searches. Diversity of searches doesn’t give rise to diversity of passions. … desires … feelings … contacts … thoughts … Diversity of perceptions doesn’t give rise to diversity of elements. And what is the diversity of elements? The sight element, the sound element, the smell element, the taste element, the touch element, and the idea element. This is called the diversity of elements. 

And\marginnote{2.1} how does diversity of elements give rise to diversity of perceptions, and diversity of perceptions give rise to diversity of thoughts? contacts … feelings … desires … passions … searches … gains … while diversity of gains doesn’t give rise to diversity of searches … passions … desires … feelings … contacts … thoughts … perceptions … elements? 

The\marginnote{3.1} sight element gives rise to the perception of sights … The idea element gives rise to the perception of ideas … The search for ideas gives rise to gaining ideas. The gaining of ideas doesn’t give rise to the search for ideas. The search for ideas doesn’t give rise to the passion for ideas. The passion for ideas doesn’t give rise to the desire for ideas. The desire for ideas doesn’t give rise to feeling born of idea contact. Feeling born of idea contact doesn’t give rise to idea contact. Idea contact doesn’t give rise to thoughts about ideas. Thoughts about ideas don’t give rise to perceptions of ideas. Perceptions of ideas don’t give rise to the idea element. 

That’s\marginnote{4.1} how diversity of elements gives rise to diversity of perceptions … diversity of perceptions doesn’t give rise to diversity of elements.” 

%
\addtocontents{toc}{\let\protect\contentsline\protect\nopagecontentsline}
\chapter*{Chapter Two }
\addcontentsline{toc}{chapter}{\tocchapterline{Chapter Two }}
\addtocontents{toc}{\let\protect\contentsline\protect\oldcontentsline}

%
\section*{{\suttatitleacronym SN 14.11}{\suttatitletranslation Seven Elements }{\suttatitleroot Sattadhātusutta}}
\addcontentsline{toc}{section}{\tocacronym{SN 14.11} \toctranslation{Seven Elements } \tocroot{Sattadhātusutta}}
\markboth{Seven Elements }{Sattadhātusutta}
\extramarks{SN 14.11}{SN 14.11}

At\marginnote{1.1} \textsanskrit{Sāvatthī}. 

“Mendicants,\marginnote{1.2} there are these seven elements. What seven? The element of light, the element of beauty, the element of the dimension of infinite space, the element of the dimension of infinite consciousness, the element of the dimension of nothingness, the element of the dimension of neither perception nor non-perception, and the element of the cessation of perception and feeling. These are the seven elements.” 

When\marginnote{2.1} he said this, one of the mendicants asked the Buddha, “Sir, due to what does each of these elements appear?” 

“Mendicant,\marginnote{3.1} the element of light appears due to the element of darkness. The element of beauty appears due to the element of ugliness. The element of the dimension of infinite space appears due to the element of form. The element of the dimension of infinite consciousness appears due to the element of the dimension of infinite space. The element of the dimension of nothingness appears due to the element of the dimension of infinite consciousness. The element of the dimension of neither perception nor non-perception appears due to the element of the dimension of nothingness. The element of the cessation of perception and feeling appears due to the element of cessation.” 

“Sir,\marginnote{4.1} how is each of these elements to be attained?” 

“The\marginnote{5.1} elements of light, beauty, the dimension of infinite space, the dimension of infinite consciousness, and the dimension of nothingness are attainments with perception. The element of the dimension of neither perception nor non-perception is an attainment with only a residue of conditioned phenomena. The element of the cessation of perception and feeling is an attainment of cessation.” 

%
\section*{{\suttatitleacronym SN 14.12}{\suttatitletranslation With a Cause }{\suttatitleroot Sanidānasutta}}
\addcontentsline{toc}{section}{\tocacronym{SN 14.12} \toctranslation{With a Cause } \tocroot{Sanidānasutta}}
\markboth{With a Cause }{Sanidānasutta}
\extramarks{SN 14.12}{SN 14.12}

At\marginnote{1.1} \textsanskrit{Sāvatthī}. 

“Mendicants,\marginnote{1.2} sensual, malicious, and cruel thoughts arise for a reason, not without reason. 

And\marginnote{2.1} how do sensual, malicious, and cruel thoughts arise for a reason, not without reason? The element of sensuality gives rise to sensual perceptions. Sensual perceptions give rise to sensual thoughts. Sensual thoughts give rise to sensual desires. Sensual desires give rise to sensual passions. Sensual passions give rise to searches for sensual pleasures. An unlearned ordinary person on a search for sensual pleasures behaves badly in three ways: by body, speech, and mind. 

The\marginnote{3.1} element of malice gives rise to malicious perceptions. Malicious perceptions give rise to malicious thoughts. … malicious desires … malicious passions … malicious searches … An unlearned ordinary person on a malicious search behaves badly in three ways: by body, speech, and mind. 

The\marginnote{4.1} element of cruelty gives rise to cruel perceptions. Cruel perceptions give rise to cruel thoughts. … cruel desires … cruel passions … cruel searches … An unlearned ordinary person on a cruel search behaves badly in three ways: by body, speech, and mind. 

Suppose\marginnote{5.1} a person was to drop a burning torch in a thicket of dry grass. If they don’t quickly extinguish it with their hands and feet, the creatures living in the grass and wood would come to ruin. 

In\marginnote{5.2} the same way, a corrupt perception might arise in an ascetic or brahmin. If they don’t quickly give it up, get rid of it, eliminate it, and obliterate it, they’ll suffer in the present life, with distress, anguish, and fever. And when the body breaks up, after death, they can expect to be reborn in a bad place. 

Thoughts\marginnote{6.1} of renunciation, good will, and harmlessness arise for a reason, not without reason. 

And\marginnote{7.1} how do thoughts of renunciation, good will, and harmlessness arise for a reason, not without reason? The element of renunciation gives rise to perceptions of renunciation. Perceptions of renunciation give rise to thoughts of renunciation. Thoughts of renunciation give rise to enthusiasm for renunciation. Enthusiasm for renunciation gives rise to fervor for renunciation. Fervor for renunciation gives rise to the search for renunciation. A learned noble disciple on a search for renunciation behaves well in three ways: by body, speech, and mind. 

The\marginnote{8.1} element of good will gives rise to perceptions of good will. Perceptions of good will give rise to thoughts of good will. … enthusiasm for good will … fervor for good will … the search for good will. A learned noble disciple on a search for good will behaves well in three ways: by body, speech, and mind. 

The\marginnote{9.1} element of harmlessness gives rise to perceptions of harmlessness. Perceptions of harmlessness give rise to thoughts of harmlessness. … enthusiasm for harmlessness … fervor for harmlessness … the search for harmlessness. A learned noble disciple on a search for harmlessness behaves well in three ways: by body, speech, and mind. 

Suppose\marginnote{10.1} a person was to drop a burning torch in a thicket of dry grass. If they were to quickly extinguish it with their hands and feet, the creatures living in the grass and wood wouldn’t come to ruin. 

In\marginnote{10.2} the same way, a corrupt perception might arise in an ascetic or brahmin. If they quickly give it up, get rid of it, eliminate it, and obliterate it, they’ll be happy in the present life, free of distress, anguish, and fever. And when the body breaks up, after death, they can expect to be reborn in a good place.” 

%
\section*{{\suttatitleacronym SN 14.13}{\suttatitletranslation In the Brick Hall }{\suttatitleroot Giñjakāvasathasutta}}
\addcontentsline{toc}{section}{\tocacronym{SN 14.13} \toctranslation{In the Brick Hall } \tocroot{Giñjakāvasathasutta}}
\markboth{In the Brick Hall }{Giñjakāvasathasutta}
\extramarks{SN 14.13}{SN 14.13}

At\marginnote{1.1} one time the Buddha was staying at \textsanskrit{Ñātika} in the brick house. There the Buddha addressed the mendicants, “Mendicants!” 

“Venerable\marginnote{1.4} sir,” they replied. The Buddha said this: 

“Mendicants,\marginnote{2.1} an element gives rise to a perception, a view, and a thought.” 

When\marginnote{2.2} he said this, Venerable \textsanskrit{Kaccāna} said to the Buddha, “Sir, regarding those who are not fully awakened Buddhas, there is a view that they are in fact fully awakened Buddhas. Due to what does this view appear?” 

“It’s\marginnote{3.1} a mighty thing, \textsanskrit{Kaccāna}, the element of ignorance. An inferior element gives rise to inferior perceptions, inferior views, inferior thoughts, inferior intentions, inferior aims, inferior wishes, an inferior person, and inferior speech. One explains, teaches, asserts, establishes, clarifies, analyzes, and reveals the inferior. I say that their rebirth is inferior. 

A\marginnote{4.1} middling element gives rise to middling perceptions, middling views, middling thoughts, middling intentions, middling aims, middling wishes, a middling person, and middling speech. One explains, teaches, asserts, establishes, clarifies, analyzes, and reveals the middling. I say that their rebirth is middling. 

A\marginnote{5.1} superior element gives rise to superior perceptions, superior views, superior thoughts, superior intentions, superior aims, superior wishes, a superior person, and superior speech. One explains, teaches, asserts, establishes, clarifies, analyzes, and reveals the superior. I say that their rebirth is superior.” 

%
\section*{{\suttatitleacronym SN 14.14}{\suttatitletranslation Bad Convictions }{\suttatitleroot Hīnādhimuttikasutta}}
\addcontentsline{toc}{section}{\tocacronym{SN 14.14} \toctranslation{Bad Convictions } \tocroot{Hīnādhimuttikasutta}}
\markboth{Bad Convictions }{Hīnādhimuttikasutta}
\extramarks{SN 14.14}{SN 14.14}

At\marginnote{1.1} \textsanskrit{Sāvatthī}. 

“Mendicants,\marginnote{1.2} sentient beings come together and converge because of an element. Those who have bad convictions come together and converge with those who have bad convictions. Those who have good convictions come together and converge with those who have good convictions. 

In\marginnote{2.1} the past, too, sentient beings came together and converged because of an element. … 

In\marginnote{3.1} the future, too, sentient beings will come together and converge because of an element. … 

At\marginnote{4.1} present, too, sentient beings come together and converge because of an element. Those who have bad convictions come together and converge with those who have bad convictions. Those who have good convictions come together and converge with those who have good convictions.” 

%
\section*{{\suttatitleacronym SN 14.15}{\suttatitletranslation Walking Together }{\suttatitleroot Caṅkamasutta}}
\addcontentsline{toc}{section}{\tocacronym{SN 14.15} \toctranslation{Walking Together } \tocroot{Caṅkamasutta}}
\markboth{Walking Together }{Caṅkamasutta}
\extramarks{SN 14.15}{SN 14.15}

At\marginnote{1.1} one time the Buddha was staying near \textsanskrit{Rājagaha}, on the Vulture’s Peak Mountain. Now at that time Venerable \textsanskrit{Sāriputta} was walking together with several mendicants not far from the Buddha. Venerable \textsanskrit{Mahāmoggallāna} was doing likewise, as were Venerable \textsanskrit{Mahākassapa}, Venerable Anuruddha, Venerable \textsanskrit{Puṇṇa} son of \textsanskrit{Mantāṇī}, Venerable \textsanskrit{Upāli}, Venerable Ānanda, and Devadatta. 

Then\marginnote{2.1} the Buddha said to the mendicants, “Mendicants, do you see \textsanskrit{Sāriputta} walking together with several mendicants?” 

“Yes,\marginnote{2.3} sir.” 

“All\marginnote{2.4} of those mendicants have great wisdom. Do you see \textsanskrit{Moggallāna} walking together with several mendicants?” 

“Yes,\marginnote{2.6} sir.” 

“All\marginnote{2.7} of those mendicants have great psychic power. Do you see Kassapa walking together with several mendicants?” 

“Yes,\marginnote{2.9} sir.” 

“All\marginnote{2.10} of those mendicants advocate austerities. Do you see Anuruddha walking together with several mendicants?” 

“Yes,\marginnote{2.12} sir.” 

“All\marginnote{2.13} of those mendicants have clairvoyance. Do you see \textsanskrit{Puṇṇa} son of \textsanskrit{Mantāṇī} walking together with several mendicants?” 

“Yes,\marginnote{2.15} sir.” 

“All\marginnote{2.16} of those mendicants are Dhamma speakers. Do you see \textsanskrit{Upāli} walking together with several mendicants?” 

“Yes,\marginnote{2.18} sir.” 

“All\marginnote{2.19} of those mendicants are experts in monastic law. Do you see Ānanda walking together with several mendicants?” 

“Yes,\marginnote{2.21} sir.” 

“All\marginnote{2.22} of those mendicants are very learned. Do you see Devadatta walking together with several mendicants?” 

“Yes,\marginnote{2.24} sir.” 

“All\marginnote{2.25} of those mendicants have corrupt wishes. 

Sentient\marginnote{3.1} beings come together and converge because of an element. Those who have bad convictions come together and converge with those who have bad convictions. Those who have good convictions come together and converge with those who have good convictions. 

In\marginnote{3.4} the past, in the future, and also in the present, sentient beings come together and converge because of an element. Those who have bad convictions come together and converge with those who have bad convictions. Those who have good convictions come together and converge with those who have good convictions.” 

%
\section*{{\suttatitleacronym SN 14.16}{\suttatitletranslation With Verses }{\suttatitleroot Sagāthāsutta}}
\addcontentsline{toc}{section}{\tocacronym{SN 14.16} \toctranslation{With Verses } \tocroot{Sagāthāsutta}}
\markboth{With Verses }{Sagāthāsutta}
\extramarks{SN 14.16}{SN 14.16}

At\marginnote{1.1} \textsanskrit{Sāvatthī}. 

“Mendicants,\marginnote{1.2} sentient beings come together and converge because of an element. Those who have bad convictions come together and converge with those who have bad convictions. In the past … 

In\marginnote{2.1} the future … 

At\marginnote{3.1} present, too, sentient beings come together and converge because of an element. Those who have bad convictions come together and converge with those who have bad convictions. 

It’s\marginnote{4.1} like how dung comes together with dung, urine with urine, spit with spit, pus with pus, and blood with blood. In the same way, sentient beings come together and converge because of an element. Those who have bad convictions come together and converge with those who have bad convictions. In the past … In the future … At present, too, sentient beings come together and converge because of an element. Those who have bad convictions come together and converge with those who have bad convictions. 

Sentient\marginnote{5.1} beings come together and converge because of an element. Those who have good convictions come together and converge with those who have good convictions. In the past … 

In\marginnote{6.1} the future … At present, too, sentient beings come together and converge because of an element. Those who have good convictions come together and converge with those who have good convictions. 

It’s\marginnote{7.1} like how milk comes together with milk, oil with oil, ghee with ghee, honey with honey, and molasses with molasses. In the same way, sentient beings come together and converge because of an element. Those who have good convictions come together and converge with those who have good convictions. In the past … In the future … At present, too, sentient beings come together and converge because of an element. Those who have good convictions come together and converge with those who have good convictions.” 

That\marginnote{8.1} is what the Buddha said. Then the Holy One, the Teacher, went on to say: 

\begin{verse}%
“Socializing\marginnote{9.1} promotes entanglement; \\
they’re cut off by being aloof. \\
If you’re lost in the middle of a great sea, \\
and you clamber up on a little log, you’ll sink. 

So\marginnote{10.1} too, a person who lives well \\
sinks by relying on a lazy person. \\
Hence you should avoid such \\
a lazy person who lacks energy. 

Dwell\marginnote{11.1} with the noble ones \\
who are secluded and determined \\
and always energetic; \\
the astute who practice absorption.” 

%
\end{verse}

%
\section*{{\suttatitleacronym SN 14.17}{\suttatitletranslation Faithless }{\suttatitleroot Assaddhasaṁsandanasutta}}
\addcontentsline{toc}{section}{\tocacronym{SN 14.17} \toctranslation{Faithless } \tocroot{Assaddhasaṁsandanasutta}}
\markboth{Faithless }{Assaddhasaṁsandanasutta}
\extramarks{SN 14.17}{SN 14.17}

At\marginnote{1.1} \textsanskrit{Sāvatthī}. 

“Mendicants,\marginnote{1.2} sentient beings come together and converge because of an element: the faithless with the faithless, the unconscientious with the unconscientious, the imprudent with the imprudent, the unlearned with the unlearned, the lazy with the lazy, the unmindful with the unmindful, and the witless with the witless. 

In\marginnote{2.1} the past, too, sentient beings came together and converged because of an element. … 

In\marginnote{3.1} the future, too, sentient beings will come together and converge because of an element. … 

At\marginnote{4.1} present, too, sentient beings come together and converge because of an element. … 

Sentient\marginnote{5.1} beings come together and converge because of an element: the faithful with the faithful, the conscientious with the conscientious, the prudent with the prudent, the learned with the learned, the energetic with the energetic, the mindful with the mindful, and the wise with the wise. In the past … In the future … At present, too, sentient beings come together and converge because of an element. …” 

%
\section*{{\suttatitleacronym SN 14.18}{\suttatitletranslation Beginning With the Faithless }{\suttatitleroot Assaddhamūlakasutta}}
\addcontentsline{toc}{section}{\tocacronym{SN 14.18} \toctranslation{Beginning With the Faithless } \tocroot{Assaddhamūlakasutta}}
\markboth{Beginning With the Faithless }{Assaddhamūlakasutta}
\extramarks{SN 14.18}{SN 14.18}

At\marginnote{1.1} \textsanskrit{Sāvatthī}. 

“Mendicants,\marginnote{1.2} sentient beings come together and converge because of an element: the faithless with the faithless, the unconscientious with the unconscientious, the witless with the witless, the faithful with the faithful, the conscientious with the conscientious, and the wise with the wise. In the past … In the future … 

At\marginnote{2.1} present, too, sentient beings come together and converge because of an element. … 

Sentient\marginnote{3.1} beings come together and converge because of an element: the faithless with the faithless, the imprudent with the imprudent, the witless with the witless, the faithful with the faithful, the prudent with the prudent, and the wise with the wise. 

\scexpansioninstructions{(The following should be told in full like the first section.) }

Sentient\marginnote{4.1} beings come together because of an element: the faithless … unlearned … witless … the faithful … 

Sentient\marginnote{5.1} beings come together because of an element: the faithless … lazy … witless. the faithful … energetic … wise. 

Sentient\marginnote{6.1} beings come together because of an element: the faithless … unmindful … witless. the faithful … mindful … wise.” 

%
\section*{{\suttatitleacronym SN 14.19}{\suttatitletranslation Beginning With the Shameless }{\suttatitleroot Ahirikamūlakasutta}}
\addcontentsline{toc}{section}{\tocacronym{SN 14.19} \toctranslation{Beginning With the Shameless } \tocroot{Ahirikamūlakasutta}}
\markboth{Beginning With the Shameless }{Ahirikamūlakasutta}
\extramarks{SN 14.19}{SN 14.19}

At\marginnote{1.1} \textsanskrit{Sāvatthī}. 

“Mendicants,\marginnote{1.2} sentient beings come together and converge because of an element: the unconscientious with the unconscientious … imprudent … witless. The conscientious with the conscientious … prudent … wise. 

…\marginnote{2.1} shameless … unlearned … witless. The conscientious with the conscientious … learned … wise. 

…\marginnote{3.1} shameless … lazy … witless. The conscientious with the conscientious … energetic … wise. 

…\marginnote{4.1} shameless … unmindful … witless. The conscientious with the conscientious … mindful … wise.” 

%
\section*{{\suttatitleacronym SN 14.20}{\suttatitletranslation Beginning With Imprudence }{\suttatitleroot Anottappamūlakasutta}}
\addcontentsline{toc}{section}{\tocacronym{SN 14.20} \toctranslation{Beginning With Imprudence } \tocroot{Anottappamūlakasutta}}
\markboth{Beginning With Imprudence }{Anottappamūlakasutta}
\extramarks{SN 14.20}{SN 14.20}

At\marginnote{1.1} \textsanskrit{Sāvatthī}. 

“Mendicants,\marginnote{1.2} sentient beings come together and converge because of an element: the imprudent with the imprudent … unlearned … witless. The prudent with the prudent … learned … wise. 

…\marginnote{2.1} the imprudent with the imprudent … lazy … witless. The prudent with the prudent … energetic … wise. 

…\marginnote{3.1} the imprudent with the imprudent … unmindful … witless. The prudent with the prudent … mindful … wise.” 

%
\section*{{\suttatitleacronym SN 14.21}{\suttatitletranslation Beginning With the Unlearned }{\suttatitleroot Appassutamūlakasutta}}
\addcontentsline{toc}{section}{\tocacronym{SN 14.21} \toctranslation{Beginning With the Unlearned } \tocroot{Appassutamūlakasutta}}
\markboth{Beginning With the Unlearned }{Appassutamūlakasutta}
\extramarks{SN 14.21}{SN 14.21}

At\marginnote{1.1} \textsanskrit{Sāvatthī}. 

“Mendicants,\marginnote{1.2} sentient beings come together and converge because of an element: the unlearned with the unlearned … lazy … witless. The learned with the learned … energetic … wise. 

…\marginnote{2.1} the unlearned with the unlearned … unmindful … witless. The learned with the learned … mindful … wise.” 

%
\section*{{\suttatitleacronym SN 14.22}{\suttatitletranslation Beginning With the Lazy }{\suttatitleroot Kusītamūlakasutta}}
\addcontentsline{toc}{section}{\tocacronym{SN 14.22} \toctranslation{Beginning With the Lazy } \tocroot{Kusītamūlakasutta}}
\markboth{Beginning With the Lazy }{Kusītamūlakasutta}
\extramarks{SN 14.22}{SN 14.22}

At\marginnote{1.1} \textsanskrit{Sāvatthī}. 

“Mendicants,\marginnote{1.2} sentient beings come together and converge because of an element: the lazy with the lazy … unmindful … witless … energetic … mindful … wise …” 

\scexpansioninstructions{(Tell all in full for the past, future, and present.) }

%
\addtocontents{toc}{\let\protect\contentsline\protect\nopagecontentsline}
\chapter*{The Chapter on Ways of Performing Deeds }
\addcontentsline{toc}{chapter}{\tocchapterline{The Chapter on Ways of Performing Deeds }}
\addtocontents{toc}{\let\protect\contentsline\protect\oldcontentsline}

%
\section*{{\suttatitleacronym SN 14.23}{\suttatitletranslation Lacking Immersion }{\suttatitleroot Asamāhitasutta}}
\addcontentsline{toc}{section}{\tocacronym{SN 14.23} \toctranslation{Lacking Immersion } \tocroot{Asamāhitasutta}}
\markboth{Lacking Immersion }{Asamāhitasutta}
\extramarks{SN 14.23}{SN 14.23}

At\marginnote{1.1} \textsanskrit{Sāvatthī}. 

“Mendicants,\marginnote{1.2} sentient beings come together and converge because of an element: the faithless with the faithless … shameless … imprudent … lacking immersion … witless … 

The\marginnote{2.1} faithful with the faithful … conscientious … prudent … possessing immersion … and the wise with the wise.” 

%
\section*{{\suttatitleacronym SN 14.24}{\suttatitletranslation Unethical }{\suttatitleroot Dussīlasutta}}
\addcontentsline{toc}{section}{\tocacronym{SN 14.24} \toctranslation{Unethical } \tocroot{Dussīlasutta}}
\markboth{Unethical }{Dussīlasutta}
\extramarks{SN 14.24}{SN 14.24}

At\marginnote{1.1} \textsanskrit{Sāvatthī}. 

“Mendicants,\marginnote{1.2} sentient beings come together and converge because of an element: the faithless with the faithless … shameless … imprudent … unethical … witless … 

The\marginnote{2.1} faithful with the faithful … conscientious … prudent … ethical … and the wise with the wise.” 

%
\section*{{\suttatitleacronym SN 14.25}{\suttatitletranslation The Five Precepts }{\suttatitleroot Pañcasikkhāpadasutta}}
\addcontentsline{toc}{section}{\tocacronym{SN 14.25} \toctranslation{The Five Precepts } \tocroot{Pañcasikkhāpadasutta}}
\markboth{The Five Precepts }{Pañcasikkhāpadasutta}
\extramarks{SN 14.25}{SN 14.25}

At\marginnote{1.1} \textsanskrit{Sāvatthī}. 

“Mendicants,\marginnote{1.2} sentient beings come together and converge because of an element: those who kill living creatures with those who kill living creatures, those who steal … commit sexual misconduct … lie … consume beer, wine, and liquor intoxicants … 

Those\marginnote{2.1} who refrain from killing living creatures … who refrain from stealing … who refrain from sexual misconduct … who refrain from lying … those who refrain from consuming beer, wine, and liquor intoxicants with those who refrain from consuming beer, wine, and liquor intoxicants.” 

%
\section*{{\suttatitleacronym SN 14.26}{\suttatitletranslation Seven Ways of Performing Deeds }{\suttatitleroot Sattakammapathasutta}}
\addcontentsline{toc}{section}{\tocacronym{SN 14.26} \toctranslation{Seven Ways of Performing Deeds } \tocroot{Sattakammapathasutta}}
\markboth{Seven Ways of Performing Deeds }{Sattakammapathasutta}
\extramarks{SN 14.26}{SN 14.26}

At\marginnote{1.1} \textsanskrit{Sāvatthī}. 

“Mendicants,\marginnote{1.2} sentient beings come together and converge because of an element: those who kill living creatures with those who kill living creatures, those who steal … commit sexual misconduct … lie … speak divisively … speak harshly … talk nonsense … 

Those\marginnote{2.1} who refrain from killing living creatures. … who refrain from stealing … who refrain from sexual misconduct … who refrain from lying … who refrain from divisive speech … who refrain from harsh speech … who refrain from talking nonsense with those who refrain from talking nonsense.” 

%
\section*{{\suttatitleacronym SN 14.27}{\suttatitletranslation Ten Ways of Performing Deeds }{\suttatitleroot Dasakammapathasutta}}
\addcontentsline{toc}{section}{\tocacronym{SN 14.27} \toctranslation{Ten Ways of Performing Deeds } \tocroot{Dasakammapathasutta}}
\markboth{Ten Ways of Performing Deeds }{Dasakammapathasutta}
\extramarks{SN 14.27}{SN 14.27}

At\marginnote{1.1} \textsanskrit{Sāvatthī}. 

“Mendicants,\marginnote{1.2} sentient beings come together and converge because of an element: those who kill living creatures with those who kill living creatures, those who steal … commit sexual misconduct … lie … speak divisively … speak harshly … talk nonsense … are covetous … are malicious … have wrong view … 

Those\marginnote{2.1} who refrain from killing living creatures … who refrain from stealing … who refrain from sexual misconduct … who refrain from lying … who refrain from divisive speech … who refrain from harsh speech … who refrain from talking nonsense … are not covetous … are not malicious … have right view with those who have right view.” 

%
\section*{{\suttatitleacronym SN 14.28}{\suttatitletranslation The Eightfold Path }{\suttatitleroot Aṭṭhaṅgikasutta}}
\addcontentsline{toc}{section}{\tocacronym{SN 14.28} \toctranslation{The Eightfold Path } \tocroot{Aṭṭhaṅgikasutta}}
\markboth{The Eightfold Path }{Aṭṭhaṅgikasutta}
\extramarks{SN 14.28}{SN 14.28}

At\marginnote{1.1} \textsanskrit{Sāvatthī}. 

“Mendicants,\marginnote{1.2} sentient beings come together and converge because of an element: those of wrong view with those of wrong view … wrong thought … wrong speech … wrong action … wrong livelihood … wrong effort … wrong mindfulness … wrong immersion … 

Those\marginnote{2.1} who have right view … right thought … right speech … right action … right livelihood … right effort … right mindfulness … right immersion with those who have right immersion.” 

%
\section*{{\suttatitleacronym SN 14.29}{\suttatitletranslation Ten Factored Path }{\suttatitleroot Dasaṅgasutta}}
\addcontentsline{toc}{section}{\tocacronym{SN 14.29} \toctranslation{Ten Factored Path } \tocroot{Dasaṅgasutta}}
\markboth{Ten Factored Path }{Dasaṅgasutta}
\extramarks{SN 14.29}{SN 14.29}

At\marginnote{1.1} \textsanskrit{Sāvatthī}. 

“Mendicants,\marginnote{1.2} sentient beings come together and converge because of an element: those of wrong view with those of wrong view … wrong thought … wrong speech … wrong action … wrong livelihood … wrong effort … wrong mindfulness … wrong immersion … wrong knowledge … wrong freedom … 

Those\marginnote{2.1} who have right view … right thought … right speech … right action … right livelihood … right effort … right mindfulness … right immersion … right knowledge … right freedom with those who have right freedom.” 

\scexpansioninstructions{(Tell all in full for the past, future, and present.) }

%
\addtocontents{toc}{\let\protect\contentsline\protect\nopagecontentsline}
\chapter*{Chapter Four }
\addcontentsline{toc}{chapter}{\tocchapterline{Chapter Four }}
\addtocontents{toc}{\let\protect\contentsline\protect\oldcontentsline}

%
\section*{{\suttatitleacronym SN 14.30}{\suttatitletranslation Four Elements }{\suttatitleroot Catudhātusutta}}
\addcontentsline{toc}{section}{\tocacronym{SN 14.30} \toctranslation{Four Elements } \tocroot{Catudhātusutta}}
\markboth{Four Elements }{Catudhātusutta}
\extramarks{SN 14.30}{SN 14.30}

At\marginnote{1.1} one time the Buddha was staying near \textsanskrit{Sāvatthī} in Jeta’s Grove, \textsanskrit{Anāthapiṇḍika}’s monastery. … “Mendicants, there are these four elements. What four? The elements of earth, water, fire, and air. These are the four elements.” 

%
\section*{{\suttatitleacronym SN 14.31}{\suttatitletranslation Before Awakening }{\suttatitleroot Pubbesambodhasutta}}
\addcontentsline{toc}{section}{\tocacronym{SN 14.31} \toctranslation{Before Awakening } \tocroot{Pubbesambodhasutta}}
\markboth{Before Awakening }{Pubbesambodhasutta}
\extramarks{SN 14.31}{SN 14.31}

At\marginnote{1.1} \textsanskrit{Sāvatthī}. 

“Mendicants,\marginnote{1.2} before my awakening—when I was still unawakened but intent on awakening—I thought: ‘What’s the gratification, the drawback, and the escape when it comes to the earth element … the water element … the fire element … and the air element?’ 

Then\marginnote{2.1} it occurred to me: ‘The pleasure and happiness that arise from the earth element: this is its gratification. That the earth element is impermanent, suffering, and perishable: this is its drawback. Removing and giving up desire and greed for the earth element: this is its escape. The pleasure and happiness that arise from the water element … The pleasure and happiness that arise from the fire element … The pleasure and happiness that arise from the air element: this is its gratification. That the air element is impermanent, suffering, and perishable: this is its drawback. Removing and giving up desire and greed for the air element: this is its escape.’ 

As\marginnote{3.1} long as I didn’t truly understand these four elements’ gratification, drawback, and escape in this way for what they are, I didn’t announce my supreme perfect awakening in this world with its gods, \textsanskrit{Māras}, and Divinities, this population with its ascetics and brahmins, its gods and humans. 

But\marginnote{4.1} when I did truly understand these four elements’ gratification, drawback, and escape in this way for what they are, I announced my supreme perfect awakening in this world with its gods, \textsanskrit{Māras}, and Divinities, this population with its ascetics and brahmins, its gods and humans. 

Knowledge\marginnote{4.2} and vision arose in me: ‘My freedom is unshakable; this is my last rebirth; now there’ll be no more future lives.’” 

%
\section*{{\suttatitleacronym SN 14.32}{\suttatitletranslation In Search of Gratification }{\suttatitleroot Acariṁsutta}}
\addcontentsline{toc}{section}{\tocacronym{SN 14.32} \toctranslation{In Search of Gratification } \tocroot{Acariṁsutta}}
\markboth{In Search of Gratification }{Acariṁsutta}
\extramarks{SN 14.32}{SN 14.32}

At\marginnote{1.1} \textsanskrit{Sāvatthī}. 

“Mendicants,\marginnote{1.2} I went in search of the earth element’s gratification, and I found it. I’ve seen clearly with wisdom the full extent of gratification in the earth element. I went in search of the earth element’s drawback, and I found it. I’ve seen clearly with wisdom the full extent of the drawback in the earth element. I went in search of escape from the earth element, and I found it. I’ve seen clearly with wisdom the full extent of escape from the earth element. 

I\marginnote{2.1} went in search of the water element’s gratification … I went in search of the fire element’s gratification … I went in search of the air element’s gratification … 

As\marginnote{3.1} long as I didn’t truly understand these four elements’ gratification, drawback, and escape for what they are, I didn’t announce my supreme perfect awakening in this world with its gods, \textsanskrit{Māras}, and Divinities, this population with its ascetics and brahmins, its gods and humans. 

But\marginnote{4.1} when I did truly understand the four elements’ gratification, drawback, and escape for what they are, I announced my supreme perfect awakening in this world with its gods, \textsanskrit{Māras}, and Divinities, this population with its ascetics and brahmins, its gods and humans. 

Knowledge\marginnote{4.2} and vision arose in me: ‘My freedom is unshakable; this is my last rebirth; now there’ll be no more future lives.’” 

%
\section*{{\suttatitleacronym SN 14.33}{\suttatitletranslation If There Was No }{\suttatitleroot Nocedaṁsutta}}
\addcontentsline{toc}{section}{\tocacronym{SN 14.33} \toctranslation{If There Was No } \tocroot{Nocedaṁsutta}}
\markboth{If There Was No }{Nocedaṁsutta}
\extramarks{SN 14.33}{SN 14.33}

At\marginnote{1.1} \textsanskrit{Sāvatthī}. 

“Mendicants,\marginnote{1.2} if there were no gratification in the earth element, sentient beings wouldn’t be aroused by it. But since there is gratification in the earth element, sentient beings are aroused by it. If the earth element had no drawback, sentient beings wouldn’t grow disillusioned with it. But since the earth element has a drawback, sentient beings do grow disillusioned with it. If there were no escape from the earth element, sentient beings wouldn’t escape from it. But since there is an escape from the earth element, sentient beings do escape from it. 

If\marginnote{2.1} there were no gratification in the water element … If there were no gratification in the fire element … If there were no gratification in the air element … 

As\marginnote{3.1} long as sentient beings don’t truly understand these four elements’ gratification, drawback, and escape for what they are, they haven’t escaped from this world—with its gods, \textsanskrit{Māras}, and Divinities, this population with its ascetics and brahmins, its gods and humans—and they don’t live detached, liberated, with a mind free of limits. 

But\marginnote{4.1} when sentient beings truly understand these four elements’ gratification, drawback, and escape for what they are, they’ve escaped from this world—with its gods, \textsanskrit{Māras}, and Divinities, this population with its ascetics and brahmins, its gods and humans—and they live detached, liberated, with a mind free of limits.” 

%
\section*{{\suttatitleacronym SN 14.34}{\suttatitletranslation Exclusively Painful }{\suttatitleroot Ekantadukkhasutta}}
\addcontentsline{toc}{section}{\tocacronym{SN 14.34} \toctranslation{Exclusively Painful } \tocroot{Ekantadukkhasutta}}
\markboth{Exclusively Painful }{Ekantadukkhasutta}
\extramarks{SN 14.34}{SN 14.34}

At\marginnote{1.1} \textsanskrit{Sāvatthī}. 

“Mendicants,\marginnote{1.2} if the earth element were exclusively painful—soaked and steeped in pain and not steeped in pleasure—sentient beings wouldn’t be aroused by it. But because the earth element is pleasurable—soaked and steeped in pleasure and not steeped in pain—sentient beings are aroused by it. 

If\marginnote{2.1} the water element … If the fire element … If the air element … 

If\marginnote{3.1} the earth element was perfectly pleasurable—soaked and steeped in pleasure and not steeped in pain—sentient beings wouldn’t grow disillusioned with it. But because the earth element is painful—soaked and steeped in pain and not steeped in pleasure—sentient beings do grow disillusioned with it. 

If\marginnote{4.1} the water element … If the fire element … If the air element was perfectly pleasurable—soaked and steeped in pleasure and not steeped in pain—sentient beings wouldn’t grow disillusioned with it. But because the air element is painful—soaked and steeped in pain and not steeped in pleasure—sentient beings do grow disillusioned with it.” 

%
\section*{{\suttatitleacronym SN 14.35}{\suttatitletranslation Taking Pleasure }{\suttatitleroot Abhinandasutta}}
\addcontentsline{toc}{section}{\tocacronym{SN 14.35} \toctranslation{Taking Pleasure } \tocroot{Abhinandasutta}}
\markboth{Taking Pleasure }{Abhinandasutta}
\extramarks{SN 14.35}{SN 14.35}

At\marginnote{1.1} \textsanskrit{Sāvatthī}. 

“Mendicants,\marginnote{1.2} if you take pleasure in the earth element, you take pleasure in suffering. If you take pleasure in suffering, I say you’re not exempt from suffering. If you take pleasure in the water element … If you take pleasure in the fire element … If you take pleasure in the air element, you take pleasure in suffering. If you take pleasure in suffering, I say you’re not exempt from suffering. 

If\marginnote{2.1} you don’t take pleasure in the earth element, you don’t take pleasure in suffering. If you don’t take pleasure in suffering, I say you’re exempt from suffering. If you don’t take pleasure in the water element … If you don’t take pleasure in the fire element … If you don’t take pleasure in the air element, you don’t take pleasure in suffering. If you don’t take pleasure in suffering, I say you’re exempt from suffering.” 

%
\section*{{\suttatitleacronym SN 14.36}{\suttatitletranslation Arising }{\suttatitleroot Uppādasutta}}
\addcontentsline{toc}{section}{\tocacronym{SN 14.36} \toctranslation{Arising } \tocroot{Uppādasutta}}
\markboth{Arising }{Uppādasutta}
\extramarks{SN 14.36}{SN 14.36}

At\marginnote{1.1} \textsanskrit{Sāvatthī}. 

“Mendicants,\marginnote{1.2} the arising, continuation, rebirth, and manifestation of the earth element is the arising of suffering, the continuation of diseases, and the manifestation of old age and death. The arising, continuation, rebirth, and manifestation of the water element … The arising, continuation, rebirth, and manifestation of the fire element … The arising, continuation, rebirth, and manifestation of the air element is the arising of suffering, the continuation of diseases, and the manifestation of old age and death. 

The\marginnote{2.1} cessation, settling, and ending of the earth element is the cessation of suffering, the settling of diseases, and the ending of old age and death. The cessation, settling, and ending of the water element … The cessation, settling, and ending of the fire element … The cessation, settling, and ending of the air element is the cessation of suffering, the settling of diseases, and the ending of old age and death.” 

%
\section*{{\suttatitleacronym SN 14.37}{\suttatitletranslation Ascetics and Brahmins }{\suttatitleroot Samaṇabrāhmaṇasutta}}
\addcontentsline{toc}{section}{\tocacronym{SN 14.37} \toctranslation{Ascetics and Brahmins } \tocroot{Samaṇabrāhmaṇasutta}}
\markboth{Ascetics and Brahmins }{Samaṇabrāhmaṇasutta}
\extramarks{SN 14.37}{SN 14.37}

At\marginnote{1.1} \textsanskrit{Sāvatthī}. 

“Mendicants,\marginnote{1.2} there are these four elements. What four? The elements of earth, water, fire, and air. There are ascetics and brahmins who don’t truly understand these four elements’ gratification, drawback, and escape. I don’t deem them as true ascetics and brahmins. Those venerables don’t realize the goal of life as an ascetic or brahmin, and don’t live having realized it with their own insight. 

There\marginnote{2.1} are ascetics and brahmins who do truly understand these four elements’ gratification, drawback, and escape. I deem them as true ascetics and brahmins. Those venerables realize the goal of life as an ascetic or brahmin, and live having realized it with their own insight.” 

%
\section*{{\suttatitleacronym SN 14.38}{\suttatitletranslation Ascetics and Brahmins (2nd) }{\suttatitleroot Dutiyasamaṇabrāhmaṇasutta}}
\addcontentsline{toc}{section}{\tocacronym{SN 14.38} \toctranslation{Ascetics and Brahmins (2nd) } \tocroot{Dutiyasamaṇabrāhmaṇasutta}}
\markboth{Ascetics and Brahmins (2nd) }{Dutiyasamaṇabrāhmaṇasutta}
\extramarks{SN 14.38}{SN 14.38}

At\marginnote{1.1} \textsanskrit{Sāvatthī}. 

“Mendicants,\marginnote{1.2} there are these four elements. What four? The elements of earth, water, fire, and air. There are ascetics and brahmins who don’t truly understand these four elements’ origin, ending, gratification, drawback, and escape … 

\scexpansioninstructions{(Tell all in full.) }

There\marginnote{1.7} are ascetics and brahmins who do truly understand these four elements’ origin, ending, gratification, drawback, and escape …” 

%
\section*{{\suttatitleacronym SN 14.39}{\suttatitletranslation Ascetics and Brahmins (3rd) }{\suttatitleroot Tatiyasamaṇabrāhmaṇasutta}}
\addcontentsline{toc}{section}{\tocacronym{SN 14.39} \toctranslation{Ascetics and Brahmins (3rd) } \tocroot{Tatiyasamaṇabrāhmaṇasutta}}
\markboth{Ascetics and Brahmins (3rd) }{Tatiyasamaṇabrāhmaṇasutta}
\extramarks{SN 14.39}{SN 14.39}

At\marginnote{1.1} \textsanskrit{Sāvatthī}. 

“Mendicants,\marginnote{1.2} there are ascetics and brahmins who don’t understand the earth element, its origin, its cessation, and the practice that leads to its cessation. They don’t understand the water element … fire element … air element … Those venerables don’t realize the goal of life as an ascetic or brahmin, and don’t live having realized it with their own insight. 

There\marginnote{2.1} are ascetics and brahmins who do understand the earth element, its origin, its cessation, and the practice that leads to its cessation. They do understand the water element … the fire element … the air element … Those venerables realize the goal of life as an ascetic or brahmin, and live having realized it with their own insight.” 

\scendkanda{The Linked Discourses on the elements are complete. }

%
\addtocontents{toc}{\let\protect\contentsline\protect\nopagecontentsline}
\part*{Linked Discourses on the Unknowable Beginning }
\addcontentsline{toc}{part}{Linked Discourses on the Unknowable Beginning }
\markboth{}{}
\addtocontents{toc}{\let\protect\contentsline\protect\oldcontentsline}

%
\addtocontents{toc}{\let\protect\contentsline\protect\nopagecontentsline}
\chapter*{Chapter One }
\addcontentsline{toc}{chapter}{\tocchapterline{Chapter One }}
\addtocontents{toc}{\let\protect\contentsline\protect\oldcontentsline}

%
\section*{{\suttatitleacronym SN 15.1}{\suttatitletranslation Grass and Sticks }{\suttatitleroot Tiṇakaṭṭhasutta}}
\addcontentsline{toc}{section}{\tocacronym{SN 15.1} \toctranslation{Grass and Sticks } \tocroot{Tiṇakaṭṭhasutta}}
\markboth{Grass and Sticks }{Tiṇakaṭṭhasutta}
\extramarks{SN 15.1}{SN 15.1}

\scevam{So\marginnote{1.1} I have heard. }At one time the Buddha was staying near \textsanskrit{Sāvatthī} in Jeta’s Grove, \textsanskrit{Anāthapiṇḍika}’s monastery. There the Buddha addressed the mendicants, “Mendicants!” 

“Venerable\marginnote{1.5} sir,” they replied. The Buddha said this: 

“Mendicants,\marginnote{1.7} transmigration has no known beginning. No first point is found of sentient beings roaming and transmigrating, shrouded by ignorance and fettered by craving. Suppose a person was to strip all the grass, sticks, branches, and leaves in the Black Plum Tree Land, gather them together into one pile, and chop them each into four inch pieces. They’d lay them down, saying: ‘This is my mother, this is my grandmother.’ The grass, sticks, branches, and leaves of the Black Plum Tree Land would run out before that person’s mothers and grandmothers. 

Why\marginnote{1.11} is that? Transmigration has no known beginning. No first point is found of sentient beings roaming and transmigrating, shrouded by ignorance and fettered by craving. For such a long time you have undergone suffering, agony, and disaster, swelling the cemeteries. This is quite enough for you to become disillusioned, dispassionate, and freed regarding all conditions.” 

%
\section*{{\suttatitleacronym SN 15.2}{\suttatitletranslation The Earth }{\suttatitleroot Pathavīsutta}}
\addcontentsline{toc}{section}{\tocacronym{SN 15.2} \toctranslation{The Earth } \tocroot{Pathavīsutta}}
\markboth{The Earth }{Pathavīsutta}
\extramarks{SN 15.2}{SN 15.2}

At\marginnote{1.1} \textsanskrit{Sāvatthī}. 

“Mendicants,\marginnote{1.2} transmigration has no known beginning. No first point is found of sentient beings roaming and transmigrating, shrouded by ignorance and fettered by craving. Suppose a person was to make the whole earth into clay balls the size of jujube seeds. They’d lay them down, saying: ‘This is my father, this is my grandfather.’ The whole earth would run out before that person’s fathers and grandfathers. 

Why\marginnote{1.6} is that? Transmigration has no known beginning. No first point is found of sentient beings roaming and transmigrating, shrouded by ignorance and fettered by craving. For such a long time you have undergone suffering, agony, and disaster, swelling the cemeteries. This is quite enough for you to become disillusioned, dispassionate, and freed regarding all conditions.” 

%
\section*{{\suttatitleacronym SN 15.3}{\suttatitletranslation Tears }{\suttatitleroot Assusutta}}
\addcontentsline{toc}{section}{\tocacronym{SN 15.3} \toctranslation{Tears } \tocroot{Assusutta}}
\markboth{Tears }{Assusutta}
\extramarks{SN 15.3}{SN 15.3}

At\marginnote{1.1} \textsanskrit{Sāvatthī}. 

“Mendicants,\marginnote{1.2} transmigration has no known beginning. No first point is found of sentient beings roaming and transmigrating, shrouded by ignorance and fettered by craving. What do you think? Which is more: the flow of tears you’ve shed while roaming and transmigrating for such a very long time—weeping and wailing from being united with the unloved and separated from the loved—or the water in the four oceans?” 

“As\marginnote{1.5} we understand the Buddha’s teaching, the flow of tears we’ve shed while roaming and transmigrating is more than the water in the four oceans.” 

“Good,\marginnote{2.1} good, mendicants! It’s good that you understand my teaching like this. The flow of tears you’ve shed while roaming and transmigrating is indeed more than the water in the four oceans. For a long time you’ve undergone the death of a mother … father … brother … sister … son … daughter … loss of relatives … loss of wealth … or loss through illness. From being united with the unloved and separated from the loved, the flow of tears you’ve shed while roaming and transmigrating is indeed more than the water in the four oceans. 

Why\marginnote{2.13} is that? Transmigration has no known beginning. … This is quite enough for you to become disillusioned, dispassionate, and freed regarding all conditions.” 

%
\section*{{\suttatitleacronym SN 15.4}{\suttatitletranslation Mother’s Milk }{\suttatitleroot Khīrasutta}}
\addcontentsline{toc}{section}{\tocacronym{SN 15.4} \toctranslation{Mother’s Milk } \tocroot{Khīrasutta}}
\markboth{Mother’s Milk }{Khīrasutta}
\extramarks{SN 15.4}{SN 15.4}

At\marginnote{1.1} \textsanskrit{Sāvatthī}. 

“Mendicants,\marginnote{1.2} transmigration has no known beginning. No first point is found of sentient beings roaming and transmigrating, shrouded by ignorance and fettered by craving. 

What\marginnote{1.4} do you think? Which is more: the mother’s milk you’ve drunk while roaming and transmigrating for such a very long time, or the water in the four oceans?” 

“As\marginnote{1.5} we understand the Buddha’s teaching, the mother’s milk we’ve drunk while roaming and transmigrating is more than the water in the four oceans.” 

“Good,\marginnote{2.1} good, mendicants! It’s good that you understand my teaching like this. The mother’s milk you’ve drunk while roaming and transmigrating for such a very long time is more than the water in the four oceans. 

Why\marginnote{2.3} is that? Transmigration has no known beginning. … This is quite enough for you to become disillusioned, dispassionate, and freed regarding all conditions.” 

%
\section*{{\suttatitleacronym SN 15.5}{\suttatitletranslation A Mountain }{\suttatitleroot Pabbatasutta}}
\addcontentsline{toc}{section}{\tocacronym{SN 15.5} \toctranslation{A Mountain } \tocroot{Pabbatasutta}}
\markboth{A Mountain }{Pabbatasutta}
\extramarks{SN 15.5}{SN 15.5}

At\marginnote{1.1} \textsanskrit{Sāvatthī}. 

Then\marginnote{1.2} a mendicant went up to the Buddha, bowed, sat down to one side, and asked him, “Sir, how long is an eon?” 

“Mendicant,\marginnote{1.4} an eon is long. It’s not easy to calculate how many years, how many hundreds or thousands or hundreds of thousands of years it lasts.” 

“But\marginnote{2.1} sir, is it possible to give a simile?” 

“It’s\marginnote{2.2} possible,” said the Buddha. 

“Suppose\marginnote{2.3} there was a huge stone mountain, a league long, a league wide, and a league high, with no cracks or holes, one solid mass. And as each century passed someone would stroke it once with a fine cloth from \textsanskrit{Kāsi}. By this means the huge stone mountain would be worn away before the eon comes to an end. That’s how long an eon is. And we’ve transmigrated through many such eons, many hundreds, many thousands, many hundreds of thousands. 

Why\marginnote{2.8} is that? Transmigration has no known beginning. … This is quite enough for you to become disillusioned, dispassionate, and freed regarding all conditions.” 

%
\section*{{\suttatitleacronym SN 15.6}{\suttatitletranslation A Mustard Seed }{\suttatitleroot Sāsapasutta}}
\addcontentsline{toc}{section}{\tocacronym{SN 15.6} \toctranslation{A Mustard Seed } \tocroot{Sāsapasutta}}
\markboth{A Mustard Seed }{Sāsapasutta}
\extramarks{SN 15.6}{SN 15.6}

At\marginnote{1.1} \textsanskrit{Sāvatthī}. 

Then\marginnote{1.2} a mendicant went up to the Buddha … and asked him, “Sir, how long is an eon?” 

“Mendicant,\marginnote{1.4} an eon is long. It’s not easy to calculate how many years, how many hundreds or thousands or hundreds of thousands of years it lasts.” 

“But\marginnote{2.1} sir, is it possible to give a simile?” 

“It’s\marginnote{2.2} possible,” said the Buddha. 

“Suppose\marginnote{2.3} there was an iron city, a league long, a league wide, and a league high, full of mustard seeds pressed into balls. And as each century passed someone would remove a single mustard seed. By this means the huge heap of mustard seeds would be used up before the eon comes to an end. That’s how long an eon is. And we’ve transmigrated through many such eons, many hundreds, many thousands, many hundreds of thousands. 

Why\marginnote{2.8} is that? Transmigration has no known beginning. … This is quite enough for you to become disillusioned, dispassionate, and freed regarding all conditions.” 

%
\section*{{\suttatitleacronym SN 15.7}{\suttatitletranslation Disciples }{\suttatitleroot Sāvakasutta}}
\addcontentsline{toc}{section}{\tocacronym{SN 15.7} \toctranslation{Disciples } \tocroot{Sāvakasutta}}
\markboth{Disciples }{Sāvakasutta}
\extramarks{SN 15.7}{SN 15.7}

At\marginnote{1.1} \textsanskrit{Sāvatthī}. 

Then\marginnote{1.2} several mendicants went up to the Buddha … and asked him, “Sir, how many eons have passed?” 

“Mendicants,\marginnote{1.4} many eons have passed. It’s not easy to calculate how many eons have passed, how many hundreds or thousands or hundreds of thousands of eons.” 

“But\marginnote{2.1} sir, is it possible to give a simile?” 

“It’s\marginnote{2.2} possible,” said the Buddha. 

“Suppose\marginnote{2.3} there were four disciples with a lifespan of a hundred years. And each day they would each recollect a hundred thousand eons. Those four disciples would pass away after a hundred years and there would still be eons that they haven’t recollected. That’s how many eons have passed. It’s not easy to calculate how many eons have passed, how many hundreds or thousands or hundreds of thousands of eons. 

Why\marginnote{2.9} is that? Transmigration has no known beginning. … This is quite enough for you to become disillusioned, dispassionate, and freed regarding all conditions.” 

%
\section*{{\suttatitleacronym SN 15.8}{\suttatitletranslation The Ganges }{\suttatitleroot Gaṅgāsutta}}
\addcontentsline{toc}{section}{\tocacronym{SN 15.8} \toctranslation{The Ganges } \tocroot{Gaṅgāsutta}}
\markboth{The Ganges }{Gaṅgāsutta}
\extramarks{SN 15.8}{SN 15.8}

Near\marginnote{1.1} \textsanskrit{Rājagaha}, in the Bamboo Grove. Then a certain brahmin went up to the Buddha, and exchanged greetings with him. 

When\marginnote{1.3} the greetings and polite conversation were over, he sat down to one side and asked the Buddha, “Mister Gotama, how many eons have passed?” 

“Brahmin,\marginnote{1.5} many eons have passed. It’s not easy to calculate how many eons have passed, how many hundreds or thousands or hundreds of thousands of eons.” 

“But\marginnote{2.1} Mister Gotama, can you give a simile?” 

“I\marginnote{2.2} can,” said the Buddha. 

“Consider\marginnote{2.3} the Ganges river from where it originates to where it meets the ocean. Between these places it’s not easy to calculate how many grains of sand there are, how many hundreds or thousands or hundreds of thousands of grains of sand. The eons that have passed are more than this. It’s not easy to calculate how many eons have passed, how many hundreds or thousands or hundreds of thousands of eons. 

Why\marginnote{2.8} is that? Transmigration has no known beginning. No first point is found of sentient beings roaming and transmigrating, shrouded by ignorance and fettered by craving. For such a long time you have undergone suffering, agony, and disaster, swelling the cemeteries. This is quite enough for you to become disillusioned, dispassionate, and freed regarding all conditions.” 

When\marginnote{3.1} he said this, the brahmin said to the Buddha, “Excellent, Mister Gotama! Excellent! … From this day forth, may Mister Gotama remember me as a lay follower who has gone for refuge for life.” 

%
\section*{{\suttatitleacronym SN 15.9}{\suttatitletranslation A Stick }{\suttatitleroot Daṇḍasutta}}
\addcontentsline{toc}{section}{\tocacronym{SN 15.9} \toctranslation{A Stick } \tocroot{Daṇḍasutta}}
\markboth{A Stick }{Daṇḍasutta}
\extramarks{SN 15.9}{SN 15.9}

At\marginnote{1.1} \textsanskrit{Sāvatthī}. 

“Mendicants,\marginnote{1.2} transmigration has no known beginning. No first point is found of sentient beings roaming and transmigrating, shrouded by ignorance and fettered by craving. Suppose a stick was tossed up in the air. Sometimes it’d fall on its bottom, sometimes the middle, and sometimes the end. It’s the same for sentient beings roaming and transmigrating, shrouded by ignorance and fettered by craving. Sometimes they go from this world to the other world, and sometimes they come from the other world to this world. 

Why\marginnote{1.6} is that? Transmigration has no known beginning. … This is quite enough for you to become disillusioned, dispassionate, and freed regarding all conditions.” 

%
\section*{{\suttatitleacronym SN 15.10}{\suttatitletranslation A Single Person }{\suttatitleroot Puggalasutta}}
\addcontentsline{toc}{section}{\tocacronym{SN 15.10} \toctranslation{A Single Person } \tocroot{Puggalasutta}}
\markboth{A Single Person }{Puggalasutta}
\extramarks{SN 15.10}{SN 15.10}

At\marginnote{1.1} one time the Buddha was staying near \textsanskrit{Rājagaha}, on the Vulture’s Peak Mountain. There the Buddha addressed the mendicants, “Mendicants!” 

“Venerable\marginnote{1.4} sir,” they replied. The Buddha said this: 

“Mendicants,\marginnote{2.1} transmigration has no known beginning. … One person roaming and transmigrating for an eon would amass a heap of bones the size of this Mount Vepulla, if they were gathered together and not lost. 

Why\marginnote{2.3} is that? Transmigration has no known beginning. … This is quite enough for you to become disillusioned, dispassionate, and freed regarding all conditions.” 

That\marginnote{3.1} is what the Buddha said. Then the Holy One, the Teacher, went on to say: 

\begin{verse}%
“If\marginnote{4.1} the bones of a single person \\
for a single eon were gathered up, \\
they’d make a pile the size of a mountain: \\
so said the great seer. 

And\marginnote{5.1} this is declared to be \\
as huge as Mount Vepulla, \\
higher than the Vulture’s Peak \\
in the Magadhan mountain range. 

But\marginnote{6.1} then, with right understanding, \\
a person sees the noble truths—\\
suffering, suffering’s origin, \\
suffering’s transcendence, \\
and the noble eightfold path \\
that leads to the stilling of suffering. 

After\marginnote{7.1} roaming on seven times at most, \\
that person \\
makes an end of suffering, \\
with the ending of all fetters.” 

%
\end{verse}

%
\addtocontents{toc}{\let\protect\contentsline\protect\nopagecontentsline}
\chapter*{Chapter Two }
\addcontentsline{toc}{chapter}{\tocchapterline{Chapter Two }}
\addtocontents{toc}{\let\protect\contentsline\protect\oldcontentsline}

%
\section*{{\suttatitleacronym SN 15.11}{\suttatitletranslation In a Sorry State }{\suttatitleroot Duggatasutta}}
\addcontentsline{toc}{section}{\tocacronym{SN 15.11} \toctranslation{In a Sorry State } \tocroot{Duggatasutta}}
\markboth{In a Sorry State }{Duggatasutta}
\extramarks{SN 15.11}{SN 15.11}

At\marginnote{1.1} one time the Buddha was staying near \textsanskrit{Sāvatthī}. 

“Mendicants,\marginnote{1.3} transmigration has no known beginning. No first point is found of sentient beings roaming and transmigrating, shrouded by ignorance and fettered by craving. When you see someone in a sorry state, in distress, you should conclude: ‘In all this long time, we too have undergone the same thing.’ Why is that? Transmigration has no known beginning. … This is quite enough for you to become disillusioned, dispassionate, and freed regarding all conditions.” 

%
\section*{{\suttatitleacronym SN 15.12}{\suttatitletranslation In a Good Way }{\suttatitleroot Sukhitasutta}}
\addcontentsline{toc}{section}{\tocacronym{SN 15.12} \toctranslation{In a Good Way } \tocroot{Sukhitasutta}}
\markboth{In a Good Way }{Sukhitasutta}
\extramarks{SN 15.12}{SN 15.12}

At\marginnote{1.1} \textsanskrit{Sāvatthī}. 

“Mendicants,\marginnote{1.2} transmigration has no known beginning. … When you see someone in a good way, in a happy state, you should conclude: ‘In all this long time, we too have undergone the same thing.’ 

Why\marginnote{1.5} is that? Transmigration has no known beginning. … This is quite enough for you to become disillusioned, dispassionate, and freed regarding all conditions.” 

%
\section*{{\suttatitleacronym SN 15.13}{\suttatitletranslation Thirty Mendicants }{\suttatitleroot Tiṁsamattasutta}}
\addcontentsline{toc}{section}{\tocacronym{SN 15.13} \toctranslation{Thirty Mendicants } \tocroot{Tiṁsamattasutta}}
\markboth{Thirty Mendicants }{Tiṁsamattasutta}
\extramarks{SN 15.13}{SN 15.13}

Near\marginnote{1.1} \textsanskrit{Rājagaha}, in the Bamboo Grove. Then thirty mendicants from \textsanskrit{Pāvā} went to the Buddha. All of them lived in the wilderness, ate only almsfood, wore rag robes, and owned just three robes; yet they all still had fetters. They bowed to the Buddha and sat down to one side. 

Then\marginnote{1.3} it occurred to the Buddha, “These thirty mendicants from \textsanskrit{Pāvā} live in the wilderness, eat only almsfood, wear rag robes, and own just three robes; yet they all still have fetters. Why don’t I teach them the Dhamma in such a way that their minds are freed from defilements by not grasping while sitting in this very seat?” 

Then\marginnote{1.6} the Buddha said to the mendicants, “Mendicants!” 

“Venerable\marginnote{1.8} sir,” they replied. The Buddha said this: 

“Mendicants,\marginnote{2.1} transmigration has no known beginning. No first point is found of sentient beings roaming and transmigrating, shrouded by ignorance and fettered by craving. 

What\marginnote{2.3} do you think? Which is more: the flow of blood you’ve shed when your head was chopped off while roaming and transmigrating for such a very long time, or the water in the four oceans?” 

“As\marginnote{2.4} we understand the Buddha’s teaching, the flow of blood we’ve shed when our head was chopped off while roaming and transmigrating is more than the water in the four oceans.” 

“Good,\marginnote{3.1} good, mendicants! It’s good that you understand my teaching like this. The flow of blood you’ve shed when your head was chopped off while roaming and transmigrating is indeed more than the water in the four oceans. For a long time you’ve been cows, and the flow of blood you’ve shed when your head was chopped off as a cow is more than the water in the four oceans. For a long time you’ve been buffalo … rams … goats … deer … chickens … pigs … For a long time you’ve been bandits, arrested for raiding villages, highway robbery, or adultery. And the flow of blood you’ve shed when your head was chopped off as a bandit is more than the water in the four oceans. 

Why\marginnote{3.13} is that? Transmigration has no known beginning. … This is quite enough for you to become disillusioned, dispassionate, and freed regarding all conditions.” 

That\marginnote{4.1} is what the Buddha said. Satisfied, the mendicants approved what the Buddha said. And while this discourse was being spoken, the minds of the thirty mendicants from \textsanskrit{Pāvā} were freed from defilements by not grasping. 

%
\section*{{\suttatitleacronym SN 15.14}{\suttatitletranslation Mother }{\suttatitleroot Mātusutta}}
\addcontentsline{toc}{section}{\tocacronym{SN 15.14} \toctranslation{Mother } \tocroot{Mātusutta}}
\markboth{Mother }{Mātusutta}
\extramarks{SN 15.14}{SN 15.14}

At\marginnote{1.1} \textsanskrit{Sāvatthī}. 

“Mendicants,\marginnote{1.2} transmigration has no known beginning. … It’s not easy to find a sentient being who in all this long time has not previously been your mother. 

Why\marginnote{1.4} is that? Transmigration has no known beginning. … This is quite enough for you to become disillusioned, dispassionate, and freed regarding all conditions.” 

%
\section*{{\suttatitleacronym SN 15.15}{\suttatitletranslation Father }{\suttatitleroot Pitusutta}}
\addcontentsline{toc}{section}{\tocacronym{SN 15.15} \toctranslation{Father } \tocroot{Pitusutta}}
\markboth{Father }{Pitusutta}
\extramarks{SN 15.15}{SN 15.15}

At\marginnote{1.1} \textsanskrit{Sāvatthī}. 

“Mendicants,\marginnote{1.2} transmigration has no known beginning. … It’s not easy to find a sentient being who in all this long time has not previously been your father. … This is quite enough for you to become disillusioned, dispassionate, and freed regarding all conditions.” 

%
\section*{{\suttatitleacronym SN 15.16}{\suttatitletranslation Brother }{\suttatitleroot Bhātusutta}}
\addcontentsline{toc}{section}{\tocacronym{SN 15.16} \toctranslation{Brother } \tocroot{Bhātusutta}}
\markboth{Brother }{Bhātusutta}
\extramarks{SN 15.16}{SN 15.16}

At\marginnote{1.1} \textsanskrit{Sāvatthī}. 

“It’s\marginnote{1.2} not easy to find a sentient being who in all this long time has not previously been your brother. … This is quite enough for you to become disillusioned, dispassionate, and freed regarding all conditions.” 

%
\section*{{\suttatitleacronym SN 15.17}{\suttatitletranslation Sister }{\suttatitleroot Bhaginisutta}}
\addcontentsline{toc}{section}{\tocacronym{SN 15.17} \toctranslation{Sister } \tocroot{Bhaginisutta}}
\markboth{Sister }{Bhaginisutta}
\extramarks{SN 15.17}{SN 15.17}

At\marginnote{1.1} \textsanskrit{Sāvatthī}. 

“It’s\marginnote{1.2} not easy to find a sentient being who in all this long time has not previously been your sister. … This is quite enough for you to become disillusioned, dispassionate, and freed regarding all conditions.” 

%
\section*{{\suttatitleacronym SN 15.18}{\suttatitletranslation Son }{\suttatitleroot Puttasutta}}
\addcontentsline{toc}{section}{\tocacronym{SN 15.18} \toctranslation{Son } \tocroot{Puttasutta}}
\markboth{Son }{Puttasutta}
\extramarks{SN 15.18}{SN 15.18}

At\marginnote{1.1} \textsanskrit{Sāvatthī}. 

“It’s\marginnote{1.2} not easy to find a sentient being who in all this long time has not previously been your son. … This is quite enough for you to become disillusioned, dispassionate, and freed regarding all conditions.” 

%
\section*{{\suttatitleacronym SN 15.19}{\suttatitletranslation Daughter }{\suttatitleroot Dhītusutta}}
\addcontentsline{toc}{section}{\tocacronym{SN 15.19} \toctranslation{Daughter } \tocroot{Dhītusutta}}
\markboth{Daughter }{Dhītusutta}
\extramarks{SN 15.19}{SN 15.19}

At\marginnote{1.1} \textsanskrit{Sāvatthī}. 

“Mendicants,\marginnote{1.2} transmigration has no known beginning. No first point is found of sentient beings roaming and transmigrating, shrouded by ignorance and fettered by craving. It’s not easy to find a sentient being who in all this long time has not previously been your daughter. 

Why\marginnote{1.5} is that? Transmigration has no known beginning. No first point is found of sentient beings roaming and transmigrating, shrouded by ignorance and fettered by craving. For such a long time you have undergone suffering, agony, and disaster, swelling the cemeteries. This is quite enough for you to become disillusioned, dispassionate, and freed regarding all conditions.” 

%
\section*{{\suttatitleacronym SN 15.20}{\suttatitletranslation Mount Vepulla }{\suttatitleroot Vepullapabbatasutta}}
\addcontentsline{toc}{section}{\tocacronym{SN 15.20} \toctranslation{Mount Vepulla } \tocroot{Vepullapabbatasutta}}
\markboth{Mount Vepulla }{Vepullapabbatasutta}
\extramarks{SN 15.20}{SN 15.20}

At\marginnote{1.1} one time the Buddha was staying near \textsanskrit{Rājagaha}, on the Vulture’s Peak Mountain. There the Buddha addressed the mendicants, “Mendicants!” 

“Venerable\marginnote{1.4} sir,” they replied. The Buddha said this: 

“Mendicants,\marginnote{2.1} transmigration has no known beginning. No first point is found of sentient beings roaming and transmigrating, shrouded by ignorance and fettered by craving. Once upon a time, mendicants, this Mount Vepulla was known as \textsanskrit{Pācīnavaṁsa}. And at that time people were known as Tivaras. The lifespan of the Tivaras was 40,000 years. It took them four days to climb Mount Vepulla, and four days to descend. At that time Kakusandha, the Blessed One, the perfected one, the fully awakened Buddha arose in the world. Kakusandha had a fine pair of chief disciples named Vidhura and \textsanskrit{Sañjīva}. See, mendicants! This mountain’s name has vanished, those people have passed away, and that Buddha has become fully quenched. So impermanent are conditions, so unstable are conditions, so unreliable are conditions. This is quite enough for you to become disillusioned, dispassionate, and freed regarding all conditions. 

Once\marginnote{3.1} upon a time this Mount Vepulla was known as \textsanskrit{Vaṅkaka}. And at that time people were known as Rohitassas. The lifespan of the Rohitassas was 30,000 years. It took them three days to climb Mount Vepulla, and three days to descend. At that time \textsanskrit{Koṇāgamana}, the Blessed One, the perfected one, the fully awakened Buddha arose in the world. \textsanskrit{Koṇāgamana} had a fine pair of chief disciples named Bhiyyosa and Uttara. See, mendicants! This mountain’s name has vanished, those people have passed away, and that Buddha has become fully quenched. So impermanent are conditions … 

Once\marginnote{4.1} upon a time this Mount Vepulla was known as Supassa. And at that time people were known as Suppiyas. The lifespan of the Suppiyas was 20,000 years. It took them two days to climb Mount Vepulla, and two days to descend. At that time Kassapa, the Blessed One, the perfected one, the fully awakened Buddha arose in the world. Kassapa had a fine pair of chief disciples named Tissa and \textsanskrit{Bhāradvāja}. See, mendicants! This mountain’s name has vanished, those people have passed away, and that Buddha has become fully quenched. So impermanent are conditions … 

These\marginnote{5.1} days this Mount Vepulla is known as Vepulla. And these people are known as Magadhans. The lifespan of the Magadhans is short, brief, and fleeting. A long life is a hundred years or a little more. It takes the Magadhans an hour to climb Mount Vepulla, and an hour to descend. And now I am the Blessed One, the perfected one, the fully awakened Buddha who has arisen in the world. I have a fine pair of chief disciples named \textsanskrit{Sāriputta} and \textsanskrit{Moggallāna}. There will come a time when this mountain’s name will vanish, those people will die, and I will be fully extinguished. So impermanent are conditions, so unstable are conditions, so unreliable are conditions. This is quite enough for you to become disillusioned, dispassionate, and freed regarding all conditions.” 

That\marginnote{6.1} is what the Buddha said. Then the Holy One, the Teacher, went on to say: 

\begin{verse}%
“For\marginnote{7.1} the Tivaras it was \textsanskrit{Pācīnavaṁsa}, \\
for the Rohitassas, \textsanskrit{Vaṅkaka}, \\
Supassa for the Suppiyas, \\
and Vepulla for the Magadhans. 

Oh!\marginnote{8.1} Conditions are impermanent, \\
their nature is to rise and fall; \\
having arisen, they cease; \\
their stilling is blissful.” 

%
\end{verse}

\scendkanda{The Linked Discourses on the unknown beginning are complete. }

%
\addtocontents{toc}{\let\protect\contentsline\protect\nopagecontentsline}
\part*{Linked Discourses with Kassapa }
\addcontentsline{toc}{part}{Linked Discourses with Kassapa }
\markboth{}{}
\addtocontents{toc}{\let\protect\contentsline\protect\oldcontentsline}

%
\addtocontents{toc}{\let\protect\contentsline\protect\nopagecontentsline}
\chapter*{The Chapter with Kassapa }
\addcontentsline{toc}{chapter}{\tocchapterline{The Chapter with Kassapa }}
\addtocontents{toc}{\let\protect\contentsline\protect\oldcontentsline}

%
\section*{{\suttatitleacronym SN 16.1}{\suttatitletranslation Content }{\suttatitleroot Santuṭṭhasutta}}
\addcontentsline{toc}{section}{\tocacronym{SN 16.1} \toctranslation{Content } \tocroot{Santuṭṭhasutta}}
\markboth{Content }{Santuṭṭhasutta}
\extramarks{SN 16.1}{SN 16.1}

At\marginnote{1.1} \textsanskrit{Sāvatthī}. 

“Mendicants,\marginnote{1.2} Kassapa is content with any kind of robe, and praises such contentment. He doesn’t try to get hold of a robe in an improper way. He doesn’t get upset if he doesn’t get a robe. And if he does get a robe, he uses it untied, uninfatuated, unattached, seeing the drawback, and understanding the escape. 

Kassapa\marginnote{2.1} is content with any kind of almsfood … 

Kassapa\marginnote{3.1} is content with any kind of lodging … 

Kassapa\marginnote{4.1} is content with any kind of medicines and supplies for the sick … 

So\marginnote{5.1} you should train like this: ‘We will be content with any kind of robe, and praise such contentment. We won’t try to get hold of a robe in an improper way. We won’t get upset if we don’t get a robe. And if we do get a robe, we’ll use it untied, uninfatuated, unattached, seeing the drawback, and understanding the escape.’ 

\scexpansioninstructions{(All should be told in full the same way.) }

‘We\marginnote{6.1} will be content with any kind of almsfood …’ ‘We will be content with any kind of lodging …’ ‘We will be content with any kind of medicines and supplies for the sick …’ That’s how you should train. I will exhort you with the example of Kassapa or someone like him. You should practice accordingly.” 

%
\section*{{\suttatitleacronym SN 16.2}{\suttatitletranslation Imprudent }{\suttatitleroot Anottappīsutta}}
\addcontentsline{toc}{section}{\tocacronym{SN 16.2} \toctranslation{Imprudent } \tocroot{Anottappīsutta}}
\markboth{Imprudent }{Anottappīsutta}
\extramarks{SN 16.2}{SN 16.2}

\scevam{So\marginnote{1.1} I have heard. }At one time Venerable \textsanskrit{Mahākassapa} and Venerable \textsanskrit{Sāriputta} were staying near Varanasi, in the deer park at Isipatana. 

Then\marginnote{1.3} in the late afternoon, Venerable \textsanskrit{Sāriputta} came out of retreat, went to Venerable \textsanskrit{Mahākassapa}, and exchanged greetings with him. When the greetings and polite conversation were over, he sat down to one side and said to \textsanskrit{Mahākassapa}: 

“Reverend\marginnote{1.5} Kassapa, it’s said that without being keen and prudent you can’t achieve awakening, extinguishment, and the supreme sanctuary from the yoke. But if you’re keen and prudent you can achieve awakening, extinguishment, and the supreme sanctuary from the yoke. 

To\marginnote{2.1} what extent is this the case?” 

“Reverend,\marginnote{2.3} take a mendicant who doesn’t foster keenness by thinking: ‘If unarisen unskillful qualities arise in me, they’ll lead to harm.’ ‘If I don’t give up arisen unskillful qualities, they’ll lead to harm.’ ‘If I don’t give rise to unarisen skillful qualities, they’ll lead to harm.’ ‘If arisen skillful qualities cease in me, they’ll lead to harm.’ That’s how you’re not keen. 

And\marginnote{3.1} how are you not prudent? Take a mendicant who doesn’t foster prudence by thinking: ‘If unarisen unskillful qualities arise in me, they’ll lead to harm.’ ‘If I don’t give up arisen unskillful qualities, they’ll lead to harm.’ ‘If I don’t give rise to unarisen skillful qualities, they’ll lead to harm.’ ‘If arisen skillful qualities cease in me, they’ll lead to harm.’ That’s how you’re not prudent. That’s how without being keen and prudent you can’t achieve awakening, extinguishment, and the supreme sanctuary from the yoke. 

And\marginnote{4.1} how are you keen? Take a mendicant who fosters keenness by thinking: ‘If unarisen unskillful qualities arise in me, they’ll lead to harm.’ ‘If I don’t give up arisen unskillful qualities, they’ll lead to harm.’ ‘If I don’t give rise to unarisen skillful qualities, they’ll lead to harm.’ ‘If arisen skillful qualities cease in me, they’ll lead to harm.’ That’s how you’re keen. 

And\marginnote{5.1} how are you prudent? Take a mendicant who fosters prudence by thinking: ‘If unarisen unskillful qualities arise in me, they’ll lead to harm.’ ‘If I don’t give up arisen unskillful qualities, they’ll lead to harm.’ ‘If I don’t give rise to unarisen skillful qualities, they’ll lead to harm.’ ‘If arisen skillful qualities cease in me, they’ll lead to harm.’ That’s how you’re prudent. That’s how if you’re keen and prudent you can achieve awakening, extinguishment, and the supreme sanctuary from the yoke.” 

%
\section*{{\suttatitleacronym SN 16.3}{\suttatitletranslation Like the Moon }{\suttatitleroot Candūpamāsutta}}
\addcontentsline{toc}{section}{\tocacronym{SN 16.3} \toctranslation{Like the Moon } \tocroot{Candūpamāsutta}}
\markboth{Like the Moon }{Candūpamāsutta}
\extramarks{SN 16.3}{SN 16.3}

At\marginnote{1.1} \textsanskrit{Sāvatthī}. 

“Mendicants,\marginnote{1.2} you should approach families like the moon: withdrawn in body and mind, always the newcomer, and never rude. Suppose a person were to look down at an old well, a rugged cliff, or an inaccessible riverland. They’d withdraw their body and mind. In the same way, you should approach families like the moon: withdrawn in body and mind, always the newcomer, and never rude. 

Kassapa\marginnote{2.1} approaches families like the moon: withdrawn in body and mind, always the newcomer, and never rude. 

What\marginnote{2.3} do you think, mendicants? What kind of mendicant is worthy of approaching families?” 

“Our\marginnote{2.5} teachings are rooted in the Buddha. He is our guide and our refuge. Sir, may the Buddha himself please clarify the meaning of this. The mendicants will listen and remember it.” 

Then\marginnote{3.1} the Buddha waved his hand in space. 

“Mendicants,\marginnote{3.2} this hand is not stuck or held or caught in space. In the same way, when approaching families, a mendicant’s mind is not stuck or held or caught, thinking: ‘May those who want material things get them, and may those who want merit make merits!’ They’re just as pleased and happy when others get something as they are when they get something. This kind of mendicant is worthy of approaching families. 

When\marginnote{4.1} Kassapa approaches families, his mind is not stuck or held or caught, thinking: ‘May those who want material things get them, and may those who want merit make merits!’ He’s just as pleased and happy when others get something as he is when he gets something. 

What\marginnote{5.1} do you think, mendicants? What kind of mendicant’s teaching is pure, and what kind is impure?” 

“Our\marginnote{5.3} teachings are rooted in the Buddha. He is our guide and our refuge. Sir, may the Buddha himself please clarify the meaning of this. The mendicants will listen and remember it.” 

“Well\marginnote{5.4} then, mendicants, listen and apply your mind well, I will speak.” 

“Yes,\marginnote{5.5} sir,” they replied. The Buddha said this: 

“Whoever\marginnote{6.1} teaches Dhamma to others with the thought: ‘Oh! May they listen to the teaching from me. When they’ve heard it, may they gain confidence in the teaching and demonstrate their confidence to me.’ Such a mendicant’s teaching is impure. 

Whoever\marginnote{7.1} teaches Dhamma to others with the thought: ‘The teaching is well explained by the Buddha—apparent in the present life, immediately effective, inviting inspection, relevant, so that sensible people can know it for themselves. Oh! May they listen to the teaching from me. When they’ve heard it, may they understand the teaching and practice accordingly.’ So they teach others because of the natural excellence of the teaching, out of compassion, kindness, and sympathy. Such a mendicant’s teaching is pure. 

Kassapa\marginnote{8.1} teaches Dhamma to others with the thought: ‘The teaching is well explained by the Buddha—apparent in the present life, immediately effective, inviting inspection, relevant, so that sensible people can know it for themselves. Oh! May they listen to the teaching from me. When they’ve heard it, may they understand the teaching and practice accordingly.’ 

Thus\marginnote{8.4} he teaches others because of the natural excellence of the teaching, out of sympathy, kindness, and sympathy. I will exhort you with the example of Kassapa or someone like him. You should practice accordingly.” 

%
\section*{{\suttatitleacronym SN 16.4}{\suttatitletranslation Visiting Families }{\suttatitleroot Kulūpakasutta}}
\addcontentsline{toc}{section}{\tocacronym{SN 16.4} \toctranslation{Visiting Families } \tocroot{Kulūpakasutta}}
\markboth{Visiting Families }{Kulūpakasutta}
\extramarks{SN 16.4}{SN 16.4}

At\marginnote{1.1} \textsanskrit{Sāvatthī}. 

“What\marginnote{1.2} do you think, mendicants? What kind of mendicant is worthy of visiting families? And what kind of mendicant is not worthy of visiting families?” 

“Our\marginnote{1.4} teachings are rooted in the Buddha. …” The Buddha said this: 

“Whoever\marginnote{2.1} visits families with the thought: ‘May they give to me, may they not fail to give. May they give a lot, not a little. May they give me fine things, not coarse. May they give quickly, not slowly. May they give carefully, not carelessly.’ If a mendicant with such a thought approaches a family and they don’t give, the mendicant feels slighted. And they experience pain and sadness because of that. If they give only a little … if they give coarse things … if they give slowly … if they give carelessly, the mendicant feels slighted. And they experience pain and sadness because of that. That kind of mendicant is not worthy of visiting families. 

Whoever\marginnote{3.1} visits families with the thought: ‘When among other families, how could I possibly think: 

“May\marginnote{3.3} they give to me, may they not fail to give. May they give a lot, not a little. May they give me fine things, not coarse. May they give quickly, not slowly. May they give carefully, not carelessly.”’ If a mendicant with such a thought approaches a family and they don’t give, the mendicant doesn’t feel slighted. And they don’t experience pain and sadness because of that. If they give only a little … if they give coarse things … if they give slowly … if they give carelessly, the mendicant doesn’t feel slighted. And they don’t experience pain and sadness because of that. That kind of mendicant is worthy of visiting families. 

Kassapa\marginnote{4.1} visits families with the thought: ‘When among other families, how could I possibly think: 

“May\marginnote{4.3} they give to me, may they not fail to give. May they give a lot, not a little. May they give me fine things, not coarse. May they give quickly, not slowly. May they give carefully, not carelessly.”’ With such a thought, if he approaches a family and they don’t give, he doesn’t feel slighted. And he doesn’t experience pain and sadness because of that. If they give only a little … if they give coarse things … if they give slowly … if they give carelessly, he doesn’t feel slighted. And he doesn’t experience pain and sadness because of that. I will exhort you with the example of Kassapa or someone like him. You should practice accordingly.” 

%
\section*{{\suttatitleacronym SN 16.5}{\suttatitletranslation Old Age }{\suttatitleroot Jiṇṇasutta}}
\addcontentsline{toc}{section}{\tocacronym{SN 16.5} \toctranslation{Old Age } \tocroot{Jiṇṇasutta}}
\markboth{Old Age }{Jiṇṇasutta}
\extramarks{SN 16.5}{SN 16.5}

So\marginnote{1.1} I have heard. Near \textsanskrit{Rājagaha}, in the Bamboo Grove. Then Venerable \textsanskrit{Mahākassapa} went up to the Buddha, bowed, and sat down to one side. The Buddha said to him: 

“You’re\marginnote{1.4} old now, Kassapa. Those worn-out hempen rag robes must be a burden for you. So Kassapa, you should wear clothes given by householders, accept invitations for the meal, and stay in my presence.” 

“For\marginnote{2.1} a long time, sir, I’ve lived in the wilderness, eaten only almsfood, worn rag robes, and owned just three robes; and I’ve praised these things. I’ve been one of few wishes, content, secluded, aloof, and energetic; and I’ve praised these things.” 

“But\marginnote{3.1} seeing what benefit, Kassapa, have you long practiced these things?” 

“Sir,\marginnote{4.1} seeing two benefits I have long practiced these things. 

I\marginnote{5.1} see happiness for myself in the this life. And I have sympathy for future generations, thinking: ‘Hopefully those who come after might follow my example.’ For they may think: ‘It seems that the awakened disciples of the Buddha for a long time lived in the wilderness, ate only almsfood, wore rag robes, and owned just three robes; and they praised these things. They were of few wishes, content, secluded, aloof, and energetic; and they praised these things.’ They’ll practice accordingly, which will be for their lasting welfare and happiness. 

Seeing\marginnote{6.1} these two benefits I have long practiced these things.” 

“Good,\marginnote{7.1} good, Kassapa! You’re acting for the welfare and happiness of the people, out of sympathy for the world, for the benefit, welfare, and happiness of gods and humans. So Kassapa, wear worn-out hempen rag robes, walk for alms, and stay in the wilderness.” 

%
\section*{{\suttatitleacronym SN 16.6}{\suttatitletranslation Advice }{\suttatitleroot Ovādasutta}}
\addcontentsline{toc}{section}{\tocacronym{SN 16.6} \toctranslation{Advice } \tocroot{Ovādasutta}}
\markboth{Advice }{Ovādasutta}
\extramarks{SN 16.6}{SN 16.6}

Near\marginnote{1.1} \textsanskrit{Rājagaha}, in the Bamboo Grove. Then Venerable \textsanskrit{Mahākassapa} went up to the Buddha, bowed, and sat down to one side. The Buddha said to him: 

“Kassapa,\marginnote{1.3} advise the mendicants! Give them a Dhamma talk! Either you or I should advise the mendicants and give them a Dhamma talk.” 

“Sir,\marginnote{2.1} the mendicants these days are hard to admonish, having qualities that make them hard to admonish. They’re impatient, and don’t take instruction respectfully. 

Take\marginnote{2.2} the monk called \textsanskrit{Bhaṇḍa}, Ānanda’s protégé. He’s been competing in studies with the monk called \textsanskrit{Abhiñjika}, Anuruddha’s protégé. They say: ‘Come on, monk, who can recite more? Who can recite better? Who can recite longer?’” 

So\marginnote{3.1} the Buddha addressed one of the monks, “Please, monk, in my name tell the monk called \textsanskrit{Bhaṇḍa}, Ānanda’s protégé, and the monk called \textsanskrit{Abhiñjika}, Anuruddha’s protégé that the teacher summons them.” 

“Yes,\marginnote{3.4} sir,” that monk replied. He went to those monks and said, “Venerables, the teacher summons you.” 

“Yes,\marginnote{4.1} reverend,” those monks replied. They went to the Buddha, bowed, and sat down to one side. The Buddha said to them: 

“Is\marginnote{4.2} it really true, monks, that you’ve been competing in studies, saying: ‘Come on, monk, who can recite more? Who can recite better? Who can recite longer?’” 

“Yes,\marginnote{4.4} sir.” 

“Have\marginnote{4.5} you ever known me to teach the Dhamma like this: ‘Please mendicants, compete in studies to see who can recite more and better and longer’?” 

“No,\marginnote{4.8} sir.” 

“If\marginnote{4.9} you’ve never known me to teach the Dhamma like this, then what exactly do you know and see, you futile men, that after going forth in such a well explained teaching and training you’d compete in studies to see who can recite more and better and longer?” 

Then\marginnote{5.1} those monks bowed with their heads at the Buddha’s feet and said, “We have made a mistake, sir. It was foolish, stupid, and unskillful of us in that after going forth in such a well explained teaching and training we competed in studies to see who can recite more and better and longer. Please, sir, accept our mistake for what it is, so we will restrain ourselves in future.” 

“Indeed,\marginnote{6.1} monks, you made a mistake. It was foolish, stupid, and unskillful of you to act in that way. But since you have recognized your mistake for what it is, and have dealt with it properly, I accept it. For it is growth in the training of the Noble One to recognize a mistake for what it is, deal with it properly, and commit to restraint in the future.” 

%
\section*{{\suttatitleacronym SN 16.7}{\suttatitletranslation Advice (2nd) }{\suttatitleroot Dutiyaovādasutta}}
\addcontentsline{toc}{section}{\tocacronym{SN 16.7} \toctranslation{Advice (2nd) } \tocroot{Dutiyaovādasutta}}
\markboth{Advice (2nd) }{Dutiyaovādasutta}
\extramarks{SN 16.7}{SN 16.7}

Near\marginnote{1.1} \textsanskrit{Rājagaha}, in the Bamboo Grove. Then Venerable \textsanskrit{Mahākassapa} went up to the Buddha, bowed, and sat down to one side. 

The\marginnote{1.3} Buddha said to him, “Kassapa, advise the mendicants! Give them a Dhamma talk! Either you or I should advise the mendicants and give them a Dhamma talk.” 

“Sir,\marginnote{2.1} the mendicants these days are hard to admonish, having qualities that make them hard to admonish. They’re impatient, and don’t take instruction respectfully. 

Sir,\marginnote{2.2} whoever has no faith, conscience, prudence, energy, and wisdom when it comes to skillful qualities can expect decline, not growth, in skillful qualities, whether by day or by night. 

It’s\marginnote{3.1} like the moon in the waning fortnight. Whether by day or by night, its beauty, roundness, light, and diameter and circumference only decline. 

In\marginnote{3.2} the same way, whoever has no faith, conscience, prudence, energy, and wisdom when it comes to skillful qualities can expect decline, not growth, in skillful qualities, whether by day or by night. 

A\marginnote{4.1} faithless individual is in decline. An individual with no conscience is in decline. An imprudent individual is in decline. A lazy individual is in decline. A witless individual is in decline. An irritable individual is in decline. An acrimonious individual is in decline. When there are no mendicant advisers there is decline. 

Sir,\marginnote{5.1} whoever has faith, conscience, prudence, energy, and wisdom when it comes to skillful qualities can expect growth, not decline, in skillful qualities, whether by day or by night. 

It’s\marginnote{6.1} like the moon in the waxing fortnight. Whether by day or by night, its beauty, roundness, light, and diameter and circumference only grow. 

In\marginnote{6.2} the same way, whoever has faith, conscience, prudence, energy, and wisdom when it comes to skillful qualities can expect growth, not decline, in skillful qualities, whether by day or by night. 

A\marginnote{7.1} faithful individual doesn’t decline. An individual with a conscience doesn’t decline. A prudent individual doesn’t decline. An energetic individual doesn’t decline. A wise individual doesn’t decline. A loving individual doesn’t decline. A kind individual doesn’t decline. When there are mendicant advisers there is no decline.” 

“Good,\marginnote{8.1} good, Kassapa! Whoever has no faith, conscience, prudence, energy, and wisdom when it comes to skillful qualities can expect decline, not growth … 

When\marginnote{9.3} there are no mendicant advisers there is decline. 

Whoever\marginnote{10.1} has faith, conscience, prudence, energy, and wisdom when it comes to skillful qualities can expect growth, not decline … 

When\marginnote{12.1} there are mendicant advisers there is no decline.” 

%
\section*{{\suttatitleacronym SN 16.8}{\suttatitletranslation Advice (3rd) }{\suttatitleroot Tatiyaovādasutta}}
\addcontentsline{toc}{section}{\tocacronym{SN 16.8} \toctranslation{Advice (3rd) } \tocroot{Tatiyaovādasutta}}
\markboth{Advice (3rd) }{Tatiyaovādasutta}
\extramarks{SN 16.8}{SN 16.8}

Near\marginnote{1.1} \textsanskrit{Rājagaha}, in the squirrels’ feeding ground. Then Venerable \textsanskrit{Mahākassapa} went up to the Buddha, bowed, and sat down to one side. The Buddha said to him: 

“Kassapa,\marginnote{1.3} advise the mendicants! Give them a Dhamma talk! Either you or I should advise the mendicants and give them a Dhamma talk.” 

“Sir,\marginnote{2.1} the mendicants these days are hard to admonish, having qualities that make them hard to admonish. They’re impatient, and don’t take instruction respectfully.” 

“Kassapa,\marginnote{2.2} that’s because formerly the senior mendicants lived in the wilderness, ate only almsfood, wore rag robes, and owned just three robes; and they praised these things. They were of few wishes, content, secluded, aloof, and energetic; and they praised these things. 

The\marginnote{3.1} senior mendicants invite such a mendicant to a seat, saying: ‘Welcome, mendicant! What is this mendicant’s name? This mendicant is good-natured; he really wants to train. Please, mendicant, take a seat.’ 

Then\marginnote{4.1} the junior mendicants think: ‘It seems that when a mendicant lives in the wilderness … and is energetic, and praises these things, senior mendicants invite them to a seat …’ They practice accordingly. That is for their lasting welfare and happiness. 

But\marginnote{5.1} these days, Kassapa, the senior mendicants don’t live in the wilderness … and aren’t energetic; and they don’t praise these things. 

When\marginnote{6.1} a mendicant is well-known and famous, a recipient of robes, almsfood, lodgings, and medicines and supplies for the sick, senior mendicants invite them to a seat: ‘Welcome, mendicant! What is this mendicant’s name? This mendicant is good-natured; he really likes his fellow monks. Please, mendicant, take a seat.’ 

Then\marginnote{7.1} the junior mendicants think: ‘It seems that when a mendicant is well-known and famous, a recipient of robes, almsfood, lodgings, and medicines and supplies for the sick, senior mendicants invite them to a seat …’ They practice accordingly. That is for their lasting harm and suffering. And if it could ever be rightly said that spiritual practitioners are imperiled by the peril of a spiritual practitioner, and vanquished by the vanquishing of a spiritual practitioner, it is these days that this could be rightly said.” 

%
\section*{{\suttatitleacronym SN 16.9}{\suttatitletranslation Absorptions and Insights }{\suttatitleroot Jhānābhiññasutta}}
\addcontentsline{toc}{section}{\tocacronym{SN 16.9} \toctranslation{Absorptions and Insights } \tocroot{Jhānābhiññasutta}}
\markboth{Absorptions and Insights }{Jhānābhiññasutta}
\extramarks{SN 16.9}{SN 16.9}

At\marginnote{1.1} \textsanskrit{Sāvatthī}. 

“Mendicants,\marginnote{1.2} whenever I want, quite secluded from sensual pleasures, secluded from unskillful qualities, I enter and remain in the first absorption, which has the rapture and bliss born of seclusion, while placing the mind and keeping it connected. And so does Kassapa. 

Whenever\marginnote{2.1} I want, as the placing of the mind and keeping it connected are stilled, I enter and remain in the second absorption, which has the rapture and bliss born of immersion, with internal clarity and mind at one, without placing the mind and keeping it connected. And so does Kassapa. 

Whenever\marginnote{3.1} I want, with the fading away of rapture, I enter and remain in the third absorption, where I meditate with equanimity, mindful and aware, personally experiencing the bliss of which the noble ones declare, ‘Equanimous and mindful, one meditates in bliss.’ And so does Kassapa. 

Whenever\marginnote{4.1} I want, with the giving up of pleasure and pain, and the ending of former happiness and sadness, I enter and remain in the fourth absorption, without pleasure or pain, with pure equanimity and mindfulness. And so does Kassapa. 

Whenever\marginnote{5.1} I want, going totally beyond perceptions of form, with the ending of perceptions of impingement, not focusing on perceptions of diversity, aware that ‘space is infinite’, I enter and remain in the dimension of infinite space. And so does Kassapa. 

Whenever\marginnote{6.1} I want, going totally beyond the dimension of infinite space, aware that ‘consciousness is infinite’, I enter and remain in the dimension of infinite consciousness. And so does Kassapa. 

Whenever\marginnote{7.1} I want, going totally beyond the dimension of infinite consciousness, aware that ‘there is nothing at all’, I enter and remain in the dimension of nothingness. And so does Kassapa. 

Whenever\marginnote{8.1} I want, going totally beyond the dimension of nothingness, I enter and remain in the dimension of neither perception nor non-perception. And so does Kassapa. 

Whenever\marginnote{9.1} I want, going totally beyond the dimension of neither perception nor non-perception, I enter and remain in the cessation of perception and feeling. And so does Kassapa. 

Whenever\marginnote{10.1} I want, I wield the many kinds of psychic power: multiplying myself and becoming one again; appearing and disappearing; going unobstructed through a wall, a rampart, or a mountain as if through space; diving in and out of the earth as if it were water; walking on water as if it were earth; flying cross-legged through the sky like a bird; touching and stroking with the hand the sun and moon, so mighty and powerful. I control the body as far as the realm of divinity. And so does Kassapa. 

Whenever\marginnote{11.1} I want, with clairaudience that is purified and superhuman, I hear both kinds of sounds, human and heavenly, whether near or far. And so does Kassapa. 

Whenever\marginnote{12.1} I want, I understand the minds of other beings and individuals, having comprehended them with my mind. I understand mind with greed as ‘mind with greed’, and mind without greed as ‘mind without greed’; mind with hate … mind without hate … mind with delusion … mind without delusion … constricted mind … scattered mind … expansive mind … unexpansive mind … mind that is not supreme … mind that is supreme … mind immersed in \textsanskrit{samādhi} … mind not immersed in \textsanskrit{samādhi} … freed mind … unfreed mind … And so does Kassapa. 

Whenever\marginnote{13.1} I want, I recollect my many kinds of past lives. That is: one, two, three, four, five, ten, twenty, thirty, forty, fifty, a hundred, a thousand, a hundred thousand rebirths; many eons of the world contracting, many eons of the world expanding, many eons of the world contracting and expanding. I remember: ‘There, I was named this, my clan was that, I looked like this, and that was my food. This was how I felt pleasure and pain, and that was how my life ended. When I passed away from that place I was reborn somewhere else. There, too, I was named this, my clan was that, I looked like this, and that was my food. This was how I felt pleasure and pain, and that was how my life ended. When I passed away from that place I was reborn here.’ And so I recollect my many kinds of past lives, with features and details. And so does Kassapa. 

Whenever\marginnote{14.1} I want, with clairvoyance that is purified and superhuman, I see sentient beings passing away and being reborn—inferior and superior, beautiful and ugly, in a good place or a bad place. I understand how sentient beings are reborn according to their deeds. ‘These dear beings did bad things by way of body, speech, and mind. They denounced the noble ones; they had wrong view; and they chose to act out of that wrong view. When their body breaks up, after death, they’re reborn in a place of loss, a bad place, the underworld, hell. These dear beings, however, did good things by way of body, speech, and mind. They never denounced the noble ones; they had right view; and they chose to act out of that right view. When their body breaks up, after death, they’re reborn in a good place, a heavenly realm.’ And so, with clairvoyance that is purified and superhuman, I see sentient beings passing away and being reborn—inferior and superior, beautiful and ugly, in a good place or a bad place. I understand how sentient beings are reborn according to their deeds. And so does Kassapa. 

I\marginnote{15.1} have realized the undefiled freedom of heart and freedom by wisdom in this very life. And I live having realized it with my own insight due to the ending of defilements. And so does Kassapa.” 

%
\section*{{\suttatitleacronym SN 16.10}{\suttatitletranslation The Nuns’ Quarters }{\suttatitleroot Upassayasutta}}
\addcontentsline{toc}{section}{\tocacronym{SN 16.10} \toctranslation{The Nuns’ Quarters } \tocroot{Upassayasutta}}
\markboth{The Nuns’ Quarters }{Upassayasutta}
\extramarks{SN 16.10}{SN 16.10}

\scevam{So\marginnote{1.1} I have heard. }At one time Venerable \textsanskrit{Mahākassapa} was staying near \textsanskrit{Sāvatthī} in Jeta’s Grove, \textsanskrit{Anāthapiṇḍika}’s monastery. Then Venerable Ānanda robed up in the morning and, taking his bowl and robe, went to \textsanskrit{Mahākassapa} and said, “Kassapa, come, sir. Let’s go to one of the nuns’ quarters.” 

“You\marginnote{1.5} go, Reverend Ānanda. You have many duties and responsibilities.” 

And\marginnote{1.6} a second time … 

And\marginnote{1.9} a third time, Ānanda said, “Come, Honorable Kassapa. Let’s go to one of the nuns’ quarters.” 

Then\marginnote{2.1} Venerable \textsanskrit{Mahākassapa} robed up in the morning and, taking his bowl and robe, went with Venerable Ānanda as his second monk to one of the nuns’ quarters, where he sat on the seat spread out. And then several nuns went up to \textsanskrit{Mahākassapa}, bowed, and sat down to one side. \textsanskrit{Mahākassapa} educated, encouraged, fired up, and inspired those nuns with a Dhamma talk, after which he got up from his seat and left. 

But\marginnote{3.1} the nun \textsanskrit{Thullatissā} was upset and blurted out, “What is Mister \textsanskrit{Mahākassapa} thinking, that he’d teach Dhamma in front of Mister Ānanda, the Videhan sage? He’s like a needle seller who thinks they can sell a needle to a needle maker!” 

\textsanskrit{Mahākassapa}\marginnote{4.1} heard \textsanskrit{Thullatissā} say these words, and he said to Ānanda, “Is that right, Reverend Ānanda? Am I the needle seller and you the needle maker? Or am I the needle maker and you the needle seller?” 

“Forgive\marginnote{4.5} her, sir. The woman’s a fool.” 

“Hold\marginnote{4.6} on, Reverend Ānanda! Don’t make the \textsanskrit{Saṅgha} investigate you further! 

What\marginnote{5.1} do you think, Reverend Ānanda? Was it you who the Buddha brought up before the \textsanskrit{Saṅgha} of mendicants, saying: ‘Mendicants, whenever I want, quite secluded from sensual pleasures, secluded from unskillful qualities, I enter and remain in the first absorption, which has the rapture and bliss born of seclusion, while placing the mind and keeping it connected. And so does Ānanda’?” 

“No,\marginnote{5.5} sir.” 

“I\marginnote{6.1} was the one the Buddha brought up before the \textsanskrit{Saṅgha} of mendicants, saying: ‘Mendicants, whenever I want, quite secluded from sensual pleasures, secluded from unskillful qualities, I enter and remain in the first absorption, which has the rapture and bliss born of seclusion, while placing the mind and keeping it connected. And so does Kassapa. …’ 

\scexpansioninstructions{(The nine progressive meditations and the five insights should be told in full.) }

What\marginnote{7.1} do you think, Reverend Ānanda? Was it you who the Buddha brought up before the \textsanskrit{Saṅgha} of mendicants, saying: ‘I have realized the undefiled freedom of heart and freedom by wisdom in this very life. And I live having realized it with my own insight due to the ending of defilements. And so does Ānanda’?” 

“No,\marginnote{7.5} sir.” 

“I\marginnote{8.1} was the one the Buddha brought up before the \textsanskrit{Saṅgha} of mendicants, saying: ‘I have realized the undefiled freedom of heart and freedom by wisdom in this very life. And I live having realized it with my own insight due to the ending of defilements. And so does Kassapa.’ 

Reverend,\marginnote{9.1} you might as well think to hide a bull elephant that’s three or three and a half meters tall behind a palm leaf as to hide my six insights.” 

But\marginnote{10.1} the nun \textsanskrit{Thullatissā} fell from the spiritual life. 

%
\section*{{\suttatitleacronym SN 16.11}{\suttatitletranslation Robes }{\suttatitleroot Cīvarasutta}}
\addcontentsline{toc}{section}{\tocacronym{SN 16.11} \toctranslation{Robes } \tocroot{Cīvarasutta}}
\markboth{Robes }{Cīvarasutta}
\extramarks{SN 16.11}{SN 16.11}

At\marginnote{1.1} one time Venerable \textsanskrit{Mahākassapa} was staying near \textsanskrit{Rājagaha}, in the Bamboo Grove, the squirrels’ feeding ground. Now at that time Venerable Ānanda was wandering in the Southern Hills together with a large \textsanskrit{Saṅgha} of mendicants. 

And\marginnote{2.1} at that time thirty of Ānanda’s mendicant protégés resigned the training and returned to a lesser life. Most of them were youths. 

When\marginnote{2.2} Venerable Ānanda had wandered in the Southern Hills as long as he pleased, he set out for \textsanskrit{Rājagaha}, to the Bamboo Grove, the squirrels’ feeding ground. He went up to Venerable \textsanskrit{Mahākassapa}, bowed, and sat down to one side. \textsanskrit{Mahākassapa} said to him: 

“Reverend\marginnote{2.3} Ānanda, for how many reasons did the Buddha lay down a rule against eating in groups of more than three among families?” 

“Sir,\marginnote{3.1} the Buddha laid down that rule for three reasons. For keeping difficult persons in check and for the comfort of good-hearted mendicants. To prevent those of corrupt wishes from taking sides and dividing the \textsanskrit{Saṅgha}. And out of consideration for families. These are the three reasons why the Buddha laid down that rule.” 

“So\marginnote{4.1} what exactly are you doing, wandering together with these junior mendicants? They don’t guard their sense doors, they eat too much, and they’re not committed to wakefulness. It’s like you’re wandering about wrecking crops and ruining families! Your following is falling apart, Reverend Ānanda, and those just getting started are slipping away. Yet this boy knows no bounds!” 

“Though\marginnote{5.1} there are grey hairs on my head, I still can’t escape being called a boy by Venerable \textsanskrit{Mahākassapa}.” 

“It’s\marginnote{5.3} because you wander with these junior mendicants. … Your following is falling apart, Reverend Ānanda, and those just getting started are slipping away. Yet this boy knows no bounds!” 

The\marginnote{6.1} nun \textsanskrit{Thullanandā} heard a rumor that Mister \textsanskrit{Mahākassapa} had rebuked Mister Ānanda the Videhan sage by calling him a boy. 

She\marginnote{7.1} was upset and blurted out, “How can Mister \textsanskrit{Mahākassapa}, who formerly followed another religion, presume to rebuke Mister Ānanda the Videhan sage by calling him a boy?” 

\textsanskrit{Mahākassapa}\marginnote{7.3} heard \textsanskrit{Thullanandā} say these words, and he said to Ānanda, “Indeed, Reverend Ānanda, the nun \textsanskrit{Thullanandā} spoke rashly and without reflection. 

Since\marginnote{8.3} I shaved off my hair and beard, dressed in ocher robes, and went forth from the lay life to homelessness, I don’t recall acknowledging any other teacher apart from the Blessed One, the perfected one, the fully awakened Buddha. 

Formerly\marginnote{8.4} when I was still a layman, I thought: ‘Life at home is cramped and dirty, life gone forth is wide open. It’s not easy for someone living at home to lead the spiritual life utterly full and pure, like a polished shell. Why don’t I shave off my hair and beard, dress in ocher robes, and go forth from the lay life to homelessness?’ After some time I made an outer robe of patches and, in the name of the perfected ones in the world, I shaved off my hair and beard, dressed in ocher robes, and went forth from the lay life to homelessness. 

When\marginnote{9.1} I had gone forth, I traveled along the road between \textsanskrit{Rājagaha} and \textsanskrit{Nāḷandā}, where I saw the Buddha sitting at the Many Sons Shrine. Seeing him, I thought: ‘If I’m ever to see a Teacher, it would be this Blessed One! If I’m ever to see a Holy One, it would be this Blessed One! If I’m ever to see a fully awakened Buddha, it would be this Blessed One!’ 

Then\marginnote{9.7} I bowed with my head at the Buddha’s feet and said: ‘Sir, the Buddha is my Teacher, I am his disciple! The Buddha is my Teacher, I am his disciple!’ 

The\marginnote{9.10} Buddha said to me, ‘Kassapa, if anyone was to say to such a wholehearted disciple that they know when they don’t know, or that they see when they don’t see, their head would explode. But Kassapa, when I say that I know and see I really do know and see. 

So\marginnote{10.1} you should train like this: “I will set up a keen sense of conscience and prudence for seniors, juniors, and those in the middle.” That’s how you should train. 

And\marginnote{11.1} you should train like this: “Whenever I hear a teaching connected with what’s skillful, I will pay attention, apply the mind, concentrate wholeheartedly, and actively listen to that teaching.” That’s how you should train. 

And\marginnote{12.1} you should train like this: “I will never neglect mindfulness of the body that is full of pleasure.” That’s how you should train.’ 

And\marginnote{13.1} when the Buddha had given me this advice he got up from his seat and left. For seven days I ate the nation’s almsfood as a debtor. On the eighth day I was enlightened. 

And\marginnote{14.1} then the Buddha left the road and went to the root of a certain tree. So I spread out my outer robe of patches folded in four and said to him, ‘Sir, sit here. That would be for my lasting welfare and happiness.’ 

The\marginnote{14.4} Buddha sat on the seat spread out and said to me, ‘Kassapa, this outer robe of patches is soft.’ 

‘Sir,\marginnote{14.7} please accept my outer robe of patches out of sympathy.’ 

‘In\marginnote{14.8} that case, Kassapa, will you wear my worn-out hempen rag robe?’ 

‘I\marginnote{14.9} will wear it, sir.’ 

And\marginnote{14.10} so I presented my outer robe of patches to the Buddha, and the Buddha presented me with his worn-out hempen rag robe. 

For\marginnote{15.1} if anyone should be rightly called the Buddha’s true-born son, born from his mouth, born of the teaching, created by the teaching, heir to the teaching, and receiver of his worn-out hempen rag robes, it’s me. 

Whenever\marginnote{16.1} I want, quite secluded from sensual pleasures, secluded from unskillful qualities, I enter and remain in the first absorption, which has the rapture and bliss born of seclusion, while placing the mind and keeping it connected. … 

\scexpansioninstructions{(The nine progressive meditations and the five insights should be told in full.) }

I\marginnote{17.1} have realized the undefiled freedom of heart and freedom by wisdom in this very life. And I live having realized it with my own insight due to the ending of defilements. 

Reverend,\marginnote{17.2} you might as well think to hide a bull elephant that’s three or three and a half meters tall behind a palm leaf as to hide my six insights.” 

But\marginnote{18.1} the nun \textsanskrit{Thullanandā} fell from the spiritual life. 

%
\section*{{\suttatitleacronym SN 16.12}{\suttatitletranslation The Realized One After Death }{\suttatitleroot Paraṁmaraṇasutta}}
\addcontentsline{toc}{section}{\tocacronym{SN 16.12} \toctranslation{The Realized One After Death } \tocroot{Paraṁmaraṇasutta}}
\markboth{The Realized One After Death }{Paraṁmaraṇasutta}
\extramarks{SN 16.12}{SN 16.12}

At\marginnote{1.1} one time Venerable \textsanskrit{Mahākassapa} and Venerable \textsanskrit{Sāriputta} were staying near Varanasi, in the deer park at Isipatana. 

Then\marginnote{1.2} in the late afternoon, Venerable \textsanskrit{Sāriputta} came out of retreat, went to Venerable \textsanskrit{Mahākassapa}, and exchanged greetings with him. When the greetings and polite conversation were over, he sat down to one side and said to \textsanskrit{Mahākassapa}: 

“Reverend\marginnote{1.4} Kassapa, does a realized one still exist after death?” 

“Reverend,\marginnote{1.5} this has not been declared by the Buddha.” 

“Well\marginnote{1.7} then, does a realized one no longer exist after death?” 

“This\marginnote{1.8} too has not been declared by the Buddha.” 

“Well\marginnote{1.10} then, does a realized one both still exist and no longer exist after death?” 

“This\marginnote{1.11} too has not been declared by the Buddha.” 

“Well\marginnote{1.13} then, does a realized one neither still exist nor no longer exist after death?” 

“This\marginnote{1.14} too has not been declared by the Buddha.” 

“And\marginnote{1.16} why has this not been declared by the Buddha?” 

“Because\marginnote{1.17} it’s not beneficial or relevant to the fundamentals of the spiritual life. It doesn’t lead to disillusionment, dispassion, cessation, peace, insight, awakening, and extinguishment. That’s why it has not been declared by the Buddha.” 

“So\marginnote{2.1} what now has been declared by the Buddha?” “‘This is suffering’ has been declared by the Buddha. ‘This is the origin of suffering’ … ‘This is the cessation of suffering’ … ‘This is the practice that leads to the cessation of suffering’ has been declared by the Buddha.” 

“And\marginnote{2.6} why has this been declared by the Buddha?” 

“Because\marginnote{2.7} it’s beneficial and relevant to the fundamentals of the spiritual life. It leads to disillusionment, dispassion, cessation, peace, insight, awakening, and extinguishment. That’s why it has been declared by the Buddha.” 

%
\section*{{\suttatitleacronym SN 16.13}{\suttatitletranslation The Counterfeit of the True Teaching }{\suttatitleroot Saddhammappatirūpakasutta}}
\addcontentsline{toc}{section}{\tocacronym{SN 16.13} \toctranslation{The Counterfeit of the True Teaching } \tocroot{Saddhammappatirūpakasutta}}
\markboth{The Counterfeit of the True Teaching }{Saddhammappatirūpakasutta}
\extramarks{SN 16.13}{SN 16.13}

\scevam{So\marginnote{1.1} I have heard. }At one time the Buddha was staying near \textsanskrit{Sāvatthī} in Jeta’s Grove, \textsanskrit{Anāthapiṇḍika}’s monastery. Then Venerable \textsanskrit{Mahākassapa} went up to the Buddha, bowed, sat down to one side, and said to him: 

“What\marginnote{1.4} is the cause, sir, what is the reason why there used to be fewer training rules but more enlightened mendicants? And what is the cause, what is the reason why these days there are more training rules and fewer enlightened mendicants?” 

“That’s\marginnote{1.6} how it is, Kassapa. When sentient beings are in decline and the true teaching is disappearing there are more training rules and fewer enlightened mendicants. The true teaching doesn’t disappear as long the counterfeit of the true teaching hasn’t appeared in the world. But when the counterfeit of the true teaching appears in the world then the true teaching disappears. 

It’s\marginnote{2.1} like native gold, which doesn’t disappear as long as counterfeit gold hasn’t appeared in the world. But when counterfeit gold appears in the world then native gold disappears. 

In\marginnote{2.3} the same way, the true teaching doesn’t disappear as long the counterfeit of the true teaching hasn’t appeared in the world. But when the counterfeit of the true teaching appears in the world then the true teaching disappears. 

It’s\marginnote{3.1} not the elements of earth, water, fire, or air that make the true teaching disappear. Rather, it’s the silly people who appear right here that make the true teaching disappear. The true teaching doesn’t disappear like a ship that sinks all at once. 

There\marginnote{4.1} are five detrimental things that lead to the decline and disappearance of the true teaching. What five? It’s when the monks, nuns, laymen, and laywomen lack respect and reverence for the Teacher, the teaching, the \textsanskrit{Saṅgha}, the training, and immersion. These five detrimental things lead to the decline and disappearance of the true teaching. 

There\marginnote{5.1} are five things that lead to the continuation, persistence, and enduring of the true teaching. What five? It’s when the monks, nuns, laymen, and laywomen maintain respect and reverence for the Teacher, the teaching, the \textsanskrit{Saṅgha}, the training, and immersion. These five things lead to the continuation, persistence, and enduring of the true teaching.” 

\scendkanda{The Linked Discourses with Kassapa are complete. }

%
\addtocontents{toc}{\let\protect\contentsline\protect\nopagecontentsline}
\part*{Linked Discourses on Gains and Honor }
\addcontentsline{toc}{part}{Linked Discourses on Gains and Honor }
\markboth{}{}
\addtocontents{toc}{\let\protect\contentsline\protect\oldcontentsline}

%
\addtocontents{toc}{\let\protect\contentsline\protect\nopagecontentsline}
\chapter*{Chapter One }
\addcontentsline{toc}{chapter}{\tocchapterline{Chapter One }}
\addtocontents{toc}{\let\protect\contentsline\protect\oldcontentsline}

%
\section*{{\suttatitleacronym SN 17.1}{\suttatitletranslation Brutal }{\suttatitleroot Dāruṇasutta}}
\addcontentsline{toc}{section}{\tocacronym{SN 17.1} \toctranslation{Brutal } \tocroot{Dāruṇasutta}}
\markboth{Brutal }{Dāruṇasutta}
\extramarks{SN 17.1}{SN 17.1}

\scevam{So\marginnote{1.1} I have heard. }At one time the Buddha was staying near \textsanskrit{Sāvatthī} in Jeta’s Grove, \textsanskrit{Anāthapiṇḍika}’s monastery. There the Buddha addressed the mendicants, “Mendicants!” 

“Venerable\marginnote{1.5} sir,” they replied. The Buddha said this: 

“Possessions,\marginnote{2.1} honor, and popularity are brutal, bitter, and harsh. They’re an obstacle to reaching the supreme sanctuary from the yoke. 

So\marginnote{2.2} you should train like this: ‘We will give up arisen possessions, honor, and popularity, and we won’t let them occupy our minds.’ That’s how you should train.” 

%
\section*{{\suttatitleacronym SN 17.2}{\suttatitletranslation A Hook }{\suttatitleroot Baḷisasutta}}
\addcontentsline{toc}{section}{\tocacronym{SN 17.2} \toctranslation{A Hook } \tocroot{Baḷisasutta}}
\markboth{A Hook }{Baḷisasutta}
\extramarks{SN 17.2}{SN 17.2}

At\marginnote{1.1} \textsanskrit{Sāvatthī}. 

“Possessions,\marginnote{1.2} honor, and popularity are brutal, bitter, and harsh. They’re an obstacle to reaching the supreme sanctuary from the yoke. 

Suppose\marginnote{1.3} a fisherman was to cast a baited hook into a deep lake. Seeing the bait, a fish would swallow it. And so the fish that swallowed the hook would meet with tragedy and disaster, and the fisherman can do what he wants with it. 

‘Fisherman’\marginnote{2.1} is a term for \textsanskrit{Māra} the Wicked. ‘Hook’ is a term for possessions, honor, and popularity. Whoever enjoys and likes arisen possessions, honor, and popularity is called a mendicant who has swallowed \textsanskrit{Māra}’s hook. They’ve met with tragedy and disaster, and the Wicked One can do with them what he wants. 

So\marginnote{2.4} brutal are possessions, honor, and popularity—bitter and harsh, an obstacle to reaching the supreme sanctuary from the yoke. 

So\marginnote{2.5} you should train like this: ‘We will give up arisen possessions, honor, and popularity, and we won’t let them occupy our minds.’ That’s how you should train.” 

%
\section*{{\suttatitleacronym SN 17.3}{\suttatitletranslation A Turtle }{\suttatitleroot Kummasutta}}
\addcontentsline{toc}{section}{\tocacronym{SN 17.3} \toctranslation{A Turtle } \tocroot{Kummasutta}}
\markboth{A Turtle }{Kummasutta}
\extramarks{SN 17.3}{SN 17.3}

At\marginnote{1.1} \textsanskrit{Sāvatthī}. 

“Possessions,\marginnote{1.2} honor, and popularity are brutal … 

Once\marginnote{1.3} upon a time in a certain lake there was a large family of turtles that had lived there for a long time. Then one of the turtles said to another, ‘My dear turtle, don’t you go to that place.’ 

But\marginnote{1.6} that turtle did go to that place, and a hunter pierced her with a harpoon. 

Then\marginnote{1.8} that turtle went back to the other turtle. When the other turtle saw her coming off in the distance, he said, ‘My dear turtle, I hope you didn’t go to that place!’ 

‘I\marginnote{1.12} did.’ 

‘But\marginnote{1.13} my dear turtle, I hope you’re not hurt or injured!’ 

‘I’m\marginnote{1.14} not hurt or injured. But this cord keeps dragging behind me.’ 

‘Indeed,\marginnote{1.15} my dear turtle, you’re hurt and injured! Your father and grandfather met with tragedy and disaster because of such a cord. Go now, you are no longer one of us.’ 

‘Hunter’\marginnote{2.1} is a term for \textsanskrit{Māra} the Wicked. 

‘Harpoon’\marginnote{2.2} is a term for possessions, honor, and popularity. 

‘Cord’\marginnote{2.3} is a term for greed and relishing. 

Whoever\marginnote{2.4} enjoys and likes arisen possessions, honor, and popularity is called a mendicant who has been pierced with a harpoon. They’ve met with tragedy and disaster, and the Wicked One can do with them what he wants. 

So\marginnote{2.6} brutal are possessions, honor, and popularity. …” 

%
\section*{{\suttatitleacronym SN 17.4}{\suttatitletranslation A Fleecy Sheep }{\suttatitleroot Dīghalomikasutta}}
\addcontentsline{toc}{section}{\tocacronym{SN 17.4} \toctranslation{A Fleecy Sheep } \tocroot{Dīghalomikasutta}}
\markboth{A Fleecy Sheep }{Dīghalomikasutta}
\extramarks{SN 17.4}{SN 17.4}

At\marginnote{1.1} \textsanskrit{Sāvatthī}. 

“Possessions,\marginnote{1.2} honor, and popularity are brutal … 

Suppose\marginnote{1.3} a fleecy sheep was to enter a briar patch. She’d get caught, snagged, and trapped at every turn, coming to ruin. 

In\marginnote{1.5} the same way, take a certain mendicant whose mind is overcome and overwhelmed by possessions, honor, and popularity. They robe up in the morning and, taking their bowl and robe, enter the village or town for alms. They get caught, snagged, and trapped at every turn, coming to ruin. 

So\marginnote{1.7} brutal are possessions, honor, and popularity. …” 

%
\section*{{\suttatitleacronym SN 17.5}{\suttatitletranslation A Dung Beetle }{\suttatitleroot Mīḷhakasutta}}
\addcontentsline{toc}{section}{\tocacronym{SN 17.5} \toctranslation{A Dung Beetle } \tocroot{Mīḷhakasutta}}
\markboth{A Dung Beetle }{Mīḷhakasutta}
\extramarks{SN 17.5}{SN 17.5}

At\marginnote{1.1} \textsanskrit{Sāvatthī}. 

“Possessions,\marginnote{1.2} honor, and popularity are brutal … 

Suppose\marginnote{1.3} there was a dung-eating beetle full of dung, stuffed with dung, and before her was a huge pile of dung. She’d look down on other beetles, thinking: ‘For I am a dung-eating beetle full of dung, stuffed with dung, and before me is a huge pile of dung.’ 

In\marginnote{1.6} the same way, take a certain mendicant whose mind is overcome and overwhelmed by possessions, honor, and popularity. They robe up in the morning and, taking their bowl and robe, enter the village or town for alms. There they eat as much as they like, get invited back tomorrow, and have plenty of almsfood. When they get back to the monastery, they boast in the middle of a group of mendicants: ‘I ate as much as I liked, got invited back tomorrow, and had plenty of almsfood. I get robes, almsfood, lodgings, and medicines and supplies for the sick. But these other mendicants have little merit or significance, so they don’t get these things.’ With a mind overcome and overwhelmed by possessions, honor, and popularity, they look down on other good-hearted mendicants. This will be for their lasting harm and suffering. 

So\marginnote{1.12} brutal are possessions, honor, and popularity. …” 

%
\section*{{\suttatitleacronym SN 17.6}{\suttatitletranslation A Bolt of Lightning }{\suttatitleroot Asanisutta}}
\addcontentsline{toc}{section}{\tocacronym{SN 17.6} \toctranslation{A Bolt of Lightning } \tocroot{Asanisutta}}
\markboth{A Bolt of Lightning }{Asanisutta}
\extramarks{SN 17.6}{SN 17.6}

At\marginnote{1.1} \textsanskrit{Sāvatthī}. 

“Possessions,\marginnote{1.2} honor, and popularity are brutal … 

Who\marginnote{1.3} should be struck by lightning? A trainee who comes into possessions, honor, and popularity before they achieve their heart’s desire. 

‘Lightning\marginnote{2.1} strike’ is a term for possessions, honor, and popularity. 

So\marginnote{2.2} brutal are possessions, honor, and popularity. …” 

%
\section*{{\suttatitleacronym SN 17.7}{\suttatitletranslation A Poisoned Arrow }{\suttatitleroot Diddhasutta}}
\addcontentsline{toc}{section}{\tocacronym{SN 17.7} \toctranslation{A Poisoned Arrow } \tocroot{Diddhasutta}}
\markboth{A Poisoned Arrow }{Diddhasutta}
\extramarks{SN 17.7}{SN 17.7}

At\marginnote{1.1} \textsanskrit{Sāvatthī}. 

“Possessions,\marginnote{1.2} honor, and popularity are brutal … 

Who\marginnote{1.3} should be pierced by a poisoned arrow? A trainee who comes into possessions, honor, and popularity before they achieve their heart’s desire. 

‘Arrow’\marginnote{2.1} is a term for possessions, honor, and popularity. 

So\marginnote{2.2} brutal are possessions, honor, and popularity. …” 

%
\section*{{\suttatitleacronym SN 17.8}{\suttatitletranslation A Jackal }{\suttatitleroot Siṅgālasutta}}
\addcontentsline{toc}{section}{\tocacronym{SN 17.8} \toctranslation{A Jackal } \tocroot{Siṅgālasutta}}
\markboth{A Jackal }{Siṅgālasutta}
\extramarks{SN 17.8}{SN 17.8}

At\marginnote{1.1} \textsanskrit{Sāvatthī}. 

“Possessions,\marginnote{1.2} honor, and popularity are brutal … 

Mendicants,\marginnote{1.3} did you hear an old jackal howling at the crack of dawn?” 

“Yes,\marginnote{1.4} sir.” 

“That\marginnote{1.5} old jackal has the disease called mange. He’s not happy in his den, or at the root of a tree, or out in the open. Wherever he goes, stands, sits, or lies down he meets with tragedy and disaster. 

In\marginnote{1.7} the same way, take a certain mendicant whose mind is overcome and overwhelmed by possessions, honor, and popularity. They’re not happy in an empty hut, at the root of a tree, or out in the open. Wherever they go, stand, sit, or lie down they meet with tragedy and disaster. 

So\marginnote{1.9} brutal are possessions, honor, and popularity. …” 

%
\section*{{\suttatitleacronym SN 17.9}{\suttatitletranslation Gale-force Winds }{\suttatitleroot Verambhasutta}}
\addcontentsline{toc}{section}{\tocacronym{SN 17.9} \toctranslation{Gale-force Winds } \tocroot{Verambhasutta}}
\markboth{Gale-force Winds }{Verambhasutta}
\extramarks{SN 17.9}{SN 17.9}

At\marginnote{1.1} \textsanskrit{Sāvatthī}. 

“Possessions,\marginnote{1.2} honor, and popularity are brutal … 

High\marginnote{1.3} in the atmosphere there are gale-force winds blowing. Any bird that flies there is flung about by those gale-force winds. Their feet go one way, their wings another, their head another, and their body another. 

In\marginnote{1.6} the same way, take a certain monk whose mind is overcome and overwhelmed by possessions, honor, and popularity. He robes up in the morning and, taking his bowl and robe, enters the village or town for alms without guarding body, speech, and mind, without establishing mindfulness, and without restraining the sense faculties. There he sees a female scantily clad, with revealing clothes. Lust infects his mind. He rejects the training and returns to a lesser life. Some take his robe, others his bowl, others his sitting cloth, others his needle case, just like the bird flung about by the gale-force winds. 

So\marginnote{1.11} brutal are possessions, honor, and popularity. …” 

%
\section*{{\suttatitleacronym SN 17.10}{\suttatitletranslation With Verses }{\suttatitleroot Sagāthakasutta}}
\addcontentsline{toc}{section}{\tocacronym{SN 17.10} \toctranslation{With Verses } \tocroot{Sagāthakasutta}}
\markboth{With Verses }{Sagāthakasutta}
\extramarks{SN 17.10}{SN 17.10}

At\marginnote{1.1} \textsanskrit{Sāvatthī}. 

“Possessions,\marginnote{1.2} honor, and popularity are brutal … 

Take\marginnote{1.3} a case where I see a certain person whose mind is overcome and overwhelmed by honor. When their body breaks up, after death, they’re reborn in a place of loss, a bad place, the underworld, hell. 

Take\marginnote{1.4} another case where I see a certain person whose mind is overcome and overwhelmed by lack of honor. When their body breaks up, after death, they’re reborn in a place of loss, a bad place, the underworld, hell. 

And\marginnote{1.5} take another case where I see a certain person whose mind is overcome and overwhelmed by both honor and lack of honor. When their body breaks up, after death, they’re reborn in a place of loss, a bad place, the underworld, hell. 

So\marginnote{1.6} brutal are possessions, honor, and popularity. …” 

That\marginnote{2.1} is what the Buddha said. Then the Holy One, the Teacher, went on to say: 

\begin{verse}%
“Whether\marginnote{3.1} they’re honored \\
or not honored, or both, \\
their immersion doesn’t waver \\
as they live diligently. 

They\marginnote{4.1} persistently meditate \\
with subtle view and discernment. \\
Rejoicing in the ending of grasping, \\
they’re said to be a true person.” 

%
\end{verse}

%
\addtocontents{toc}{\let\protect\contentsline\protect\nopagecontentsline}
\chapter*{Chapter Two }
\addcontentsline{toc}{chapter}{\tocchapterline{Chapter Two }}
\addtocontents{toc}{\let\protect\contentsline\protect\oldcontentsline}

%
\section*{{\suttatitleacronym SN 17.11}{\suttatitletranslation A Gold Cup }{\suttatitleroot Suvaṇṇapātisutta}}
\addcontentsline{toc}{section}{\tocacronym{SN 17.11} \toctranslation{A Gold Cup } \tocroot{Suvaṇṇapātisutta}}
\markboth{A Gold Cup }{Suvaṇṇapātisutta}
\extramarks{SN 17.11}{SN 17.11}

At\marginnote{1.1} \textsanskrit{Sāvatthī}. 

“Possessions,\marginnote{1.2} honor, and popularity are brutal … 

When\marginnote{1.3} I’ve comprehended the mind of a certain person, I understand: ‘This venerable would not tell a deliberate lie even for the sake of a gold cup filled with silver powder.’ But some time later I see them tell a deliberate lie because their mind is overcome and overwhelmed by possessions, honor, and popularity. 

So\marginnote{1.6} brutal are possessions, honor, and popularity. …” 

%
\section*{{\suttatitleacronym SN 17.12}{\suttatitletranslation A Silver Pot }{\suttatitleroot Rūpiyapātisutta}}
\addcontentsline{toc}{section}{\tocacronym{SN 17.12} \toctranslation{A Silver Pot } \tocroot{Rūpiyapātisutta}}
\markboth{A Silver Pot }{Rūpiyapātisutta}
\extramarks{SN 17.12}{SN 17.12}

At\marginnote{1.1} \textsanskrit{Sāvatthī}. 

“Possessions,\marginnote{1.2} honor, and popularity are brutal … 

When\marginnote{1.3} I’ve comprehended the mind of a certain person, I understand: ‘This venerable would not tell a deliberate lie even for the sake of a silver pot filled with gold powder.’ But some time later I see them tell a deliberate lie because their mind is overcome and overwhelmed by possessions, honor, and popularity. 

So\marginnote{1.6} brutal are possessions, honor, and popularity. …” 

%
\section*{{\suttatitleacronym SN 17.13–20}{\suttatitletranslation A Gold Ingot, Etc. }{\suttatitleroot Suvaṇṇanikkhasuttādiaṭṭhaka}}
\addcontentsline{toc}{section}{\tocacronym{SN 17.13–20} \toctranslation{A Gold Ingot, Etc. } \tocroot{Suvaṇṇanikkhasuttādiaṭṭhaka}}
\markboth{A Gold Ingot, Etc. }{Suvaṇṇanikkhasuttādiaṭṭhaka}
\extramarks{SN 17.13–20}{SN 17.13–20}

At\marginnote{1.1} \textsanskrit{Sāvatthī}. 

“Mendicants,\marginnote{1.2} when I’ve comprehended the mind of a certain person, I understand: ‘This venerable would not tell a deliberate lie even for the sake of a gold ingot.’ …” 

“‘…\marginnote{1.4} for the sake of a hundred gold ingots.’ …” 

“‘…\marginnote{1.5} for the sake of a mountain of gold.’ …” 

“‘…\marginnote{1.6} for the sake of a hundred mountains of gold.’ …” 

“‘…\marginnote{1.7} for the sake of the whole earth full of gold.’ …” 

“‘…\marginnote{1.8} for any kind of material reward.’ …” 

“‘…\marginnote{1.9} for the sake of life.’ …” 

“‘…\marginnote{1.10} for the sake of the finest lady in the land.’ But some time later I see them tell a deliberate lie because their mind is overcome and overwhelmed by possessions, honor, and popularity. 

So\marginnote{1.12} brutal are possessions, honor, and popularity. …” 

%
\addtocontents{toc}{\let\protect\contentsline\protect\nopagecontentsline}
\chapter*{Chapter Three }
\addcontentsline{toc}{chapter}{\tocchapterline{Chapter Three }}
\addtocontents{toc}{\let\protect\contentsline\protect\oldcontentsline}

%
\section*{{\suttatitleacronym SN 17.21}{\suttatitletranslation A Female }{\suttatitleroot Mātugāmasutta}}
\addcontentsline{toc}{section}{\tocacronym{SN 17.21} \toctranslation{A Female } \tocroot{Mātugāmasutta}}
\markboth{A Female }{Mātugāmasutta}
\extramarks{SN 17.21}{SN 17.21}

At\marginnote{1.1} \textsanskrit{Sāvatthī}. 

“Possessions,\marginnote{1.2} honor, and popularity are brutal … 

Even\marginnote{1.3} if you’re alone with a female she might not occupy your mind, but possessions, honor and popularity would still occupy your mind. 

So\marginnote{1.4} brutal are possessions, honor, and popularity. …” 

%
\section*{{\suttatitleacronym SN 17.22}{\suttatitletranslation The Finest Lady in the Land }{\suttatitleroot Kalyāṇīsutta}}
\addcontentsline{toc}{section}{\tocacronym{SN 17.22} \toctranslation{The Finest Lady in the Land } \tocroot{Kalyāṇīsutta}}
\markboth{The Finest Lady in the Land }{Kalyāṇīsutta}
\extramarks{SN 17.22}{SN 17.22}

At\marginnote{1.1} \textsanskrit{Sāvatthī}. 

“Possessions,\marginnote{1.2} honor, and popularity are brutal … 

Even\marginnote{1.3} if you’re alone with the finest lady in the land she might not occupy your mind, but possessions, honor and popularity would still occupy your mind. 

So\marginnote{1.4} brutal are possessions, honor, and popularity. …” 

%
\section*{{\suttatitleacronym SN 17.23}{\suttatitletranslation An Only Son }{\suttatitleroot Ekaputtakasutta}}
\addcontentsline{toc}{section}{\tocacronym{SN 17.23} \toctranslation{An Only Son } \tocroot{Ekaputtakasutta}}
\markboth{An Only Son }{Ekaputtakasutta}
\extramarks{SN 17.23}{SN 17.23}

At\marginnote{1.1} \textsanskrit{Sāvatthī}. 

“Possessions,\marginnote{1.2} honor, and popularity are brutal … 

A\marginnote{1.3} faithful laywoman with a dear and beloved only son would rightly appeal to him, ‘My darling, please be like the householder Citta and Hatthaka of \textsanskrit{Ãḷavī}.’ 

These\marginnote{1.5} are a standard and a measure for my male lay disciples, that is, the householder Citta and Hatthaka of \textsanskrit{Ãḷavī}. 

‘But\marginnote{1.6} my darling, if you go forth from the lay life to homelessness, please be like \textsanskrit{Sāriputta} and \textsanskrit{Moggallāna}.’ 

These\marginnote{1.8} are a standard and a measure for my monk disciples, that is, \textsanskrit{Sāriputta} and \textsanskrit{Moggallāna}. 

‘And\marginnote{1.9} my darling, may you not come into possessions, honor, and popularity while you’re still a trainee and haven’t achieved your heart’s desire.’ 

If\marginnote{1.10} a trainee who hasn’t achieved their heart’s desire comes into possessions, honor, and popularity it’s an obstacle for them. 

So\marginnote{1.11} brutal are possessions, honor, and popularity. …” 

%
\section*{{\suttatitleacronym SN 17.24}{\suttatitletranslation An Only Daughter }{\suttatitleroot Ekadhītusutta}}
\addcontentsline{toc}{section}{\tocacronym{SN 17.24} \toctranslation{An Only Daughter } \tocroot{Ekadhītusutta}}
\markboth{An Only Daughter }{Ekadhītusutta}
\extramarks{SN 17.24}{SN 17.24}

At\marginnote{1.1} \textsanskrit{Sāvatthī}. 

“Possessions,\marginnote{1.2} honor, and popularity are brutal … 

A\marginnote{1.3} faithful laywoman with a dear and beloved only daughter would rightly appeal to her, ‘My darling, please be like the laywomen \textsanskrit{Khujjuttarā} and \textsanskrit{Veḷukaṇṭakī}, Nanda’s mother.’ 

These\marginnote{1.5} are a standard and a measure for my female lay disciples, that is, the laywomen \textsanskrit{Khujjuttarā} and \textsanskrit{Veḷukaṇṭakī}, Nanda’s mother. 

‘But\marginnote{1.6} my darling, if you go forth from the lay life to homelessness, please be like the nuns \textsanskrit{Khemā} and \textsanskrit{Uppalavaṇṇā}.’ 

These\marginnote{1.8} are a standard and a measure for my nun disciples, that is, the nuns \textsanskrit{Khemā} and \textsanskrit{Uppalavaṇṇā}. 

‘And\marginnote{1.9} my darling, may you not come into possessions, honor, and popularity while you’re still a trainee and haven’t achieved your heart’s desire.’ If a trainee who hasn’t achieved their heart’s desire comes into possessions, honor, and popularity it’s an obstacle for them. 

So\marginnote{1.11} brutal are possessions, honor, and popularity. …” 

%
\section*{{\suttatitleacronym SN 17.25}{\suttatitletranslation Ascetics and Brahmins }{\suttatitleroot Samaṇabrāhmaṇasutta}}
\addcontentsline{toc}{section}{\tocacronym{SN 17.25} \toctranslation{Ascetics and Brahmins } \tocroot{Samaṇabrāhmaṇasutta}}
\markboth{Ascetics and Brahmins }{Samaṇabrāhmaṇasutta}
\extramarks{SN 17.25}{SN 17.25}

At\marginnote{1.1} \textsanskrit{Sāvatthī}. 

“Mendicants,\marginnote{1.2} there are ascetics and brahmins who don’t truly understand the gratification, drawback, and escape when it comes to possessions, honor, and popularity. I don’t deem them as true ascetics and brahmins. Those venerables don’t realize the goal of life as an ascetic or brahmin, and don’t live having realized it with their own insight. 

There\marginnote{1.4} are ascetics and brahmins who do truly understand the gratification, drawback, and escape when it comes to possessions, honor, and popularity. I deem them as true ascetics and brahmins. Those venerables realize the goal of life as an ascetic or brahmin, and live having realized it with their own insight.” 

%
\section*{{\suttatitleacronym SN 17.26}{\suttatitletranslation Ascetics and Brahmins (2nd) }{\suttatitleroot Dutiyasamaṇabrāhmaṇasutta}}
\addcontentsline{toc}{section}{\tocacronym{SN 17.26} \toctranslation{Ascetics and Brahmins (2nd) } \tocroot{Dutiyasamaṇabrāhmaṇasutta}}
\markboth{Ascetics and Brahmins (2nd) }{Dutiyasamaṇabrāhmaṇasutta}
\extramarks{SN 17.26}{SN 17.26}

At\marginnote{1.1} \textsanskrit{Sāvatthī}. 

“There\marginnote{1.2} are ascetics and brahmins who don’t truly understand the origin, ending, gratification, drawback, and escape when it comes to possessions, honor, and popularity … 

There\marginnote{1.3} are ascetics and brahmins who do truly understand …” 

%
\section*{{\suttatitleacronym SN 17.27}{\suttatitletranslation Ascetics and Brahmins (3rd) }{\suttatitleroot Tatiyasamaṇabrāhmaṇasutta}}
\addcontentsline{toc}{section}{\tocacronym{SN 17.27} \toctranslation{Ascetics and Brahmins (3rd) } \tocroot{Tatiyasamaṇabrāhmaṇasutta}}
\markboth{Ascetics and Brahmins (3rd) }{Tatiyasamaṇabrāhmaṇasutta}
\extramarks{SN 17.27}{SN 17.27}

At\marginnote{1.1} \textsanskrit{Sāvatthī}. 

“There\marginnote{1.2} are ascetics and brahmins who don’t truly understand possessions, honor, and popularity, their origin, their cessation, and the path that leads to their cessation … 

There\marginnote{1.3} are ascetics and brahmins who do truly understand …” 

%
\section*{{\suttatitleacronym SN 17.28}{\suttatitletranslation Skin }{\suttatitleroot Chavisutta}}
\addcontentsline{toc}{section}{\tocacronym{SN 17.28} \toctranslation{Skin } \tocroot{Chavisutta}}
\markboth{Skin }{Chavisutta}
\extramarks{SN 17.28}{SN 17.28}

At\marginnote{1.1} \textsanskrit{Sāvatthī}. 

“Possessions,\marginnote{1.2} honor, and popularity are brutal … 

They\marginnote{1.3} cut through the outer skin, the inner skin, the flesh, sinews, and bones, until they reach the marrow and keep pushing. 

So\marginnote{1.4} brutal are possessions, honor, and popularity. …” 

%
\section*{{\suttatitleacronym SN 17.29}{\suttatitletranslation A Rope }{\suttatitleroot Rajjusutta}}
\addcontentsline{toc}{section}{\tocacronym{SN 17.29} \toctranslation{A Rope } \tocroot{Rajjusutta}}
\markboth{A Rope }{Rajjusutta}
\extramarks{SN 17.29}{SN 17.29}

At\marginnote{1.1} \textsanskrit{Sāvatthī}. 

“Possessions,\marginnote{1.2} honor, and popularity are brutal … 

They\marginnote{1.3} cut through the outer skin, the inner skin, the flesh, sinews, and bones, until they reach the marrow and keep pushing. 

Suppose\marginnote{2.1} a strong man was to twist a tough horse-hair rope around your shin and tighten it. It would cut through the outer skin, the inner skin, the flesh, sinews, and bones, until it reached the marrow and kept pushing. 

In\marginnote{2.3} the same way, possessions, honor, and popularity cut through the outer skin, the inner skin, the flesh, sinews, and bones, until they reach the marrow and keep pushing. 

So\marginnote{2.4} brutal are possessions, honor, and popularity. …” 

%
\section*{{\suttatitleacronym SN 17.30}{\suttatitletranslation A Mendicant With Defilements Ended }{\suttatitleroot Bhikkhusutta}}
\addcontentsline{toc}{section}{\tocacronym{SN 17.30} \toctranslation{A Mendicant With Defilements Ended } \tocroot{Bhikkhusutta}}
\markboth{A Mendicant With Defilements Ended }{Bhikkhusutta}
\extramarks{SN 17.30}{SN 17.30}

At\marginnote{1.1} \textsanskrit{Sāvatthī}. 

“Mendicants,\marginnote{1.2} possessions, honor, and popularity are an obstacle even for a mendicant who is perfected, with defilements ended.” 

When\marginnote{1.3} he said this, Venerable Ānanda said to the Buddha, “Sir, what do possessions, honor, and popularity obstruct for a mendicant with defilements ended?” 

“Ānanda,\marginnote{1.5} I don’t say that possessions, honor, and popularity obstruct the unshakable freedom of heart. But I do say that possessions, honor, and popularity obstruct the achievement of blissful meditations in this life for a meditator who is diligent, keen, and resolute. 

So\marginnote{1.7} brutal are possessions, honor, and popularity—bitter and harsh, an obstacle to reaching the supreme sanctuary from the yoke. 

So\marginnote{1.8} you should train like this: ‘We will give up arisen possessions, honor, and popularity, and we won’t let them occupy our minds.’ That’s how you should train.” 

%
\addtocontents{toc}{\let\protect\contentsline\protect\nopagecontentsline}
\chapter*{Chapter Four }
\addcontentsline{toc}{chapter}{\tocchapterline{Chapter Four }}
\addtocontents{toc}{\let\protect\contentsline\protect\oldcontentsline}

%
\section*{{\suttatitleacronym SN 17.31}{\suttatitletranslation Schism }{\suttatitleroot Bhindisutta}}
\addcontentsline{toc}{section}{\tocacronym{SN 17.31} \toctranslation{Schism } \tocroot{Bhindisutta}}
\markboth{Schism }{Bhindisutta}
\extramarks{SN 17.31}{SN 17.31}

At\marginnote{1.1} \textsanskrit{Sāvatthī}. 

“Possessions,\marginnote{1.2} honor, and popularity are brutal … 

Devadatta\marginnote{1.3} split the \textsanskrit{Saṅgha} because his mind was overcome and overwhelmed by possessions, honor, and popularity. 

So\marginnote{1.4} brutal are possessions, honor, and popularity. …” 

%
\section*{{\suttatitleacronym SN 17.32}{\suttatitletranslation Skillful Root }{\suttatitleroot Kusalamūlasutta}}
\addcontentsline{toc}{section}{\tocacronym{SN 17.32} \toctranslation{Skillful Root } \tocroot{Kusalamūlasutta}}
\markboth{Skillful Root }{Kusalamūlasutta}
\extramarks{SN 17.32}{SN 17.32}

At\marginnote{1.1} \textsanskrit{Sāvatthī}. 

“Possessions,\marginnote{1.2} honor, and popularity are brutal … 

Devadatta\marginnote{1.3} cut off his skillful root because his mind was overcome and overwhelmed by possessions, honor, and popularity. 

So\marginnote{1.4} brutal are possessions, honor, and popularity. …” 

%
\section*{{\suttatitleacronym SN 17.33}{\suttatitletranslation Skillful Quality }{\suttatitleroot Kusaladhammasutta}}
\addcontentsline{toc}{section}{\tocacronym{SN 17.33} \toctranslation{Skillful Quality } \tocroot{Kusaladhammasutta}}
\markboth{Skillful Quality }{Kusaladhammasutta}
\extramarks{SN 17.33}{SN 17.33}

At\marginnote{1.1} \textsanskrit{Sāvatthī}. 

“Possessions,\marginnote{1.2} honor, and popularity are brutal … 

Devadatta\marginnote{1.3} cut off his skillful quality because his mind was overcome and overwhelmed by possessions, honor, and popularity. 

So\marginnote{1.4} brutal are possessions, honor, and popularity. …” 

%
\section*{{\suttatitleacronym SN 17.34}{\suttatitletranslation Bright Quality }{\suttatitleroot Sukkadhammasutta}}
\addcontentsline{toc}{section}{\tocacronym{SN 17.34} \toctranslation{Bright Quality } \tocroot{Sukkadhammasutta}}
\markboth{Bright Quality }{Sukkadhammasutta}
\extramarks{SN 17.34}{SN 17.34}

At\marginnote{1.1} \textsanskrit{Sāvatthī}. 

“Possessions,\marginnote{1.2} honor, and popularity are brutal … 

Devadatta\marginnote{1.3} cut off his bright quality because his mind was overcome and overwhelmed by possessions, honor, and popularity. 

So\marginnote{1.4} brutal are possessions, honor, and popularity. …” 

%
\section*{{\suttatitleacronym SN 17.35}{\suttatitletranslation Shortly After He Left }{\suttatitleroot Acirapakkantasutta}}
\addcontentsline{toc}{section}{\tocacronym{SN 17.35} \toctranslation{Shortly After He Left } \tocroot{Acirapakkantasutta}}
\markboth{Shortly After He Left }{Acirapakkantasutta}
\extramarks{SN 17.35}{SN 17.35}

At\marginnote{1.1} one time the Buddha was staying near \textsanskrit{Rājagaha}, on the Vulture’s Peak Mountain, not long after Devadatta had left. There the Buddha spoke to the mendicants about Devadatta: 

“Possessions,\marginnote{1.3} honor, and popularity came to Devadatta for his own ruin and downfall. 

It’s\marginnote{2.1} like a banana tree … or a bamboo … or a reed, all of which bear fruit to their own ruin and downfall … 

It’s\marginnote{5.1} like a mule, which becomes pregnant to its own ruin and downfall. In the same way, possessions, honor, and popularity came to Devadatta for his own ruin and downfall. 

So\marginnote{5.3} brutal are possessions, honor, and popularity. That’s how you should train.” 

That\marginnote{6.1} is what the Buddha said. Then the Holy One, the Teacher, went on to say: 

\begin{verse}%
“The\marginnote{7.1} banana tree is destroyed by its own fruit, \\
as are the bamboo and the reed. \\
Honor destroys a sinner, \\
as pregnancy destroys a mule.” 

%
\end{verse}

%
\section*{{\suttatitleacronym SN 17.36}{\suttatitletranslation Five Hundred Carts }{\suttatitleroot Pañcarathasatasutta}}
\addcontentsline{toc}{section}{\tocacronym{SN 17.36} \toctranslation{Five Hundred Carts } \tocroot{Pañcarathasatasutta}}
\markboth{Five Hundred Carts }{Pañcarathasatasutta}
\extramarks{SN 17.36}{SN 17.36}

Near\marginnote{1.1} \textsanskrit{Rājagaha}, in the Bamboo Grove, the squirrels’ feeding ground. 

Now\marginnote{1.2} at that time Prince \textsanskrit{Ajātasattu} was going with five hundred carts in the morning and the evening to attend on Devadatta, presenting him with an offering of five hundred servings of food. 

Then\marginnote{1.3} several mendicants went up to the Buddha, bowed, sat down to one side, and said to him, “Sir, Prince \textsanskrit{Ajātasattu} is going with five hundred carts in the morning and the evening to attend on Devadatta, presenting him with an offering of five hundred servings of food.” 

“Mendicants,\marginnote{1.5} don’t envy Devadatta’s possessions, honor, and popularity. As long as Prince \textsanskrit{Ajātasattu} goes with five hundred carts in the morning and the evening to attend on Devadatta, presenting him with an offering of five hundred servings of food, Devadatta can expect decline, not growth, in skillful qualities. 

If\marginnote{2.1} they were to pop a boil on a wild dog’s nose, it would get even wilder. In the same way, as long as Prince \textsanskrit{Ajātasattu} goes with five hundred carts in the morning and the evening to attend on Devadatta, presenting him with an offering of five hundred servings of food, Devadatta can expect decline, not growth, in skillful qualities. 

So\marginnote{2.3} brutal are possessions, honor, and popularity. …” 

%
\section*{{\suttatitleacronym SN 17.37}{\suttatitletranslation Mother }{\suttatitleroot Mātusutta}}
\addcontentsline{toc}{section}{\tocacronym{SN 17.37} \toctranslation{Mother } \tocroot{Mātusutta}}
\markboth{Mother }{Mātusutta}
\extramarks{SN 17.37}{SN 17.37}

At\marginnote{1.1} \textsanskrit{Sāvatthī}. 

“Possessions,\marginnote{1.2} honor, and popularity are brutal, bitter, and harsh. They’re an obstacle to reaching the supreme sanctuary from the yoke. When I’ve comprehended the mind of a certain person, I understand: ‘This venerable would not tell a deliberate lie even for the sake of their mother.’ But some time later I see them tell a deliberate lie because their mind is overcome and overwhelmed by possessions, honor, and popularity. 

So\marginnote{1.6} brutal are possessions, honor, and popularity—bitter and harsh, an obstacle to reaching the supreme sanctuary from the yoke. 

So\marginnote{1.7} you should train like this: ‘We will give up arisen possessions, honor, and popularity, and we won’t let them occupy our minds.’ That’s how you should train.” 

%
\section*{{\suttatitleacronym SN 17.38–43}{\suttatitletranslation Father, Etc. }{\suttatitleroot Pitusuttādichakka}}
\addcontentsline{toc}{section}{\tocacronym{SN 17.38–43} \toctranslation{Father, Etc. } \tocroot{Pitusuttādichakka}}
\markboth{Father, Etc. }{Pitusuttādichakka}
\extramarks{SN 17.38–43}{SN 17.38–43}

At\marginnote{1.1} \textsanskrit{Sāvatthī}. 

“Possessions,\marginnote{1.2} honor, and popularity are brutal, bitter, and harsh. They’re an obstacle to reaching the supreme sanctuary from the yoke. When I’ve comprehended the mind of a certain person, I understand: ‘This venerable would not tell a deliberate lie even for the sake of their father.’ …” 

\scexpansioninstructions{(Tell in full as in SN 17.37.) }

“‘…\marginnote{1.6} brother.’ …” 

“‘…\marginnote{1.7} sister.’ …” 

“‘…\marginnote{1.8} son.’ …” 

“‘…\marginnote{1.9} daughter.’ …” 

“‘…\marginnote{1.10} wife.’ But some time later I see them tell a deliberate lie because their mind is overcome and overwhelmed by possessions, honor, and popularity. 

So\marginnote{1.12} brutal are possessions, honor, and popularity—bitter and harsh, an obstacle to reaching the supreme sanctuary from the yoke. 

So\marginnote{1.13} you should train like this: ‘We will give up arisen possessions, honor, and popularity, and we won’t let them occupy our minds.’ That’s how you should train.” 

\scendkanda{The Linked Discourses on possessions, honor, and popularity are complete. }

%
\addtocontents{toc}{\let\protect\contentsline\protect\nopagecontentsline}
\part*{Linked Discourses with Rāhula }
\addcontentsline{toc}{part}{Linked Discourses with Rāhula }
\markboth{}{}
\addtocontents{toc}{\let\protect\contentsline\protect\oldcontentsline}

%
\addtocontents{toc}{\let\protect\contentsline\protect\nopagecontentsline}
\chapter*{Chapter One }
\addcontentsline{toc}{chapter}{\tocchapterline{Chapter One }}
\addtocontents{toc}{\let\protect\contentsline\protect\oldcontentsline}

%
\section*{{\suttatitleacronym SN 18.1}{\suttatitletranslation The Eye, Etc. }{\suttatitleroot Cakkhusutta}}
\addcontentsline{toc}{section}{\tocacronym{SN 18.1} \toctranslation{The Eye, Etc. } \tocroot{Cakkhusutta}}
\markboth{The Eye, Etc. }{Cakkhusutta}
\extramarks{SN 18.1}{SN 18.1}

\scevam{So\marginnote{1.1} I have heard. }At one time the Buddha was staying near \textsanskrit{Sāvatthī} in Jeta’s Grove, \textsanskrit{Anāthapiṇḍika}’s monastery. 

Then\marginnote{1.3} Venerable \textsanskrit{Rāhula} went up to the Buddha, bowed, sat down to one side, and said to him, “Sir, may the Buddha please teach me Dhamma in brief. When I’ve heard it, I’ll live alone, withdrawn, diligent, keen, and resolute.” 

“What\marginnote{2.1} do you think, \textsanskrit{Rāhula}? Is the eye permanent or impermanent?” 

“Impermanent,\marginnote{2.3} sir.” 

“But\marginnote{2.4} if it’s impermanent, is it suffering or happiness?” 

“Suffering,\marginnote{2.5} sir.” 

“But\marginnote{2.6} if it’s impermanent, suffering, and perishable, is it fit to be regarded thus: ‘This is mine, I am this, this is my self’?” 

“No,\marginnote{2.8} sir.” 

“Is\marginnote{3.1} the ear permanent or impermanent?” 

“Impermanent,\marginnote{3.2} sir.” … 

“Is\marginnote{3.3} the nose permanent or impermanent?” 

“Impermanent,\marginnote{3.4} sir.” … 

“Is\marginnote{3.5} the tongue permanent or impermanent?” 

“Impermanent,\marginnote{3.6} sir.” … 

“Is\marginnote{3.7} the body permanent or impermanent?” 

“Impermanent,\marginnote{3.8} sir.” … 

“Is\marginnote{3.9} the mind permanent or impermanent?” 

“Impermanent,\marginnote{3.10} sir.” 

“But\marginnote{3.11} if it’s impermanent, is it suffering or happiness?” 

“Suffering,\marginnote{3.12} sir.” 

“But\marginnote{3.13} if it’s impermanent, suffering, and perishable, is it fit to be regarded thus: ‘This is mine, I am this, this is my self’?” 

“No,\marginnote{3.15} sir.” 

“Seeing\marginnote{4.1} this, a learned noble disciple grows disillusioned with the eye, the ear, the nose, the tongue, the body, and the mind. Being disillusioned, desire fades away. When desire fades away they’re freed. When they’re freed, they know they’re freed. 

They\marginnote{4.3} understand: ‘Rebirth is ended, the spiritual journey has been completed, what had to be done has been done, there is nothing further for this place.’” 

\scexpansioninstructions{(The ten discourses of this series should be told in full the same way.) }

%
\section*{{\suttatitleacronym SN 18.2}{\suttatitletranslation Sights, Etc. }{\suttatitleroot Rūpasutta}}
\addcontentsline{toc}{section}{\tocacronym{SN 18.2} \toctranslation{Sights, Etc. } \tocroot{Rūpasutta}}
\markboth{Sights, Etc. }{Rūpasutta}
\extramarks{SN 18.2}{SN 18.2}

At\marginnote{1.1} \textsanskrit{Sāvatthī}. 

“What\marginnote{1.2} do you think, \textsanskrit{Rāhula}? Are sights permanent or impermanent?” 

“Impermanent,\marginnote{1.4} sir.” … “… sounds … smells … tastes … touches … Are ideas permanent or impermanent?” 

“Impermanent,\marginnote{1.10} sir.” … 

“Seeing\marginnote{1.11} this, a learned noble disciple grows disillusioned with sights, sounds, smells, tastes, touches, and ideas. Being disillusioned, desire fades away. …” 

%
\section*{{\suttatitleacronym SN 18.3}{\suttatitletranslation Consciousness }{\suttatitleroot Viññāṇasutta}}
\addcontentsline{toc}{section}{\tocacronym{SN 18.3} \toctranslation{Consciousness } \tocroot{Viññāṇasutta}}
\markboth{Consciousness }{Viññāṇasutta}
\extramarks{SN 18.3}{SN 18.3}

At\marginnote{1.1} \textsanskrit{Sāvatthī}. 

“What\marginnote{1.2} do you think, \textsanskrit{Rāhula}? Is eye consciousness permanent or impermanent?” 

“Impermanent,\marginnote{1.4} sir.” … 

“…\marginnote{1.5} ear consciousness … nose consciousness … tongue consciousness … body consciousness … Is mind consciousness permanent or impermanent?” 

“Impermanent,\marginnote{1.10} sir.” … 

“Seeing\marginnote{1.11} this, a learned noble disciple grows disillusioned with eye consciousness, ear consciousness, nose consciousness, tongue consciousness, body consciousness, and mind consciousness. Being disillusioned, desire fades away. …” 

%
\section*{{\suttatitleacronym SN 18.4}{\suttatitletranslation Contact }{\suttatitleroot Samphassasutta}}
\addcontentsline{toc}{section}{\tocacronym{SN 18.4} \toctranslation{Contact } \tocroot{Samphassasutta}}
\markboth{Contact }{Samphassasutta}
\extramarks{SN 18.4}{SN 18.4}

At\marginnote{1.1} \textsanskrit{Sāvatthī}. 

“What\marginnote{1.2} do you think, \textsanskrit{Rāhula}? Is eye contact permanent or impermanent?” 

“Impermanent,\marginnote{1.4} sir.” … 

“…\marginnote{1.5} ear contact … nose contact … tongue contact … body contact … Is mind contact permanent or impermanent?” 

“Impermanent,\marginnote{1.10} sir.” … 

“Seeing\marginnote{1.11} this, a learned noble disciple grows disillusioned with eye contact, ear contact, nose contact, tongue contact, body contact, and mind contact. Being disillusioned, desire fades away. …” 

%
\section*{{\suttatitleacronym SN 18.5}{\suttatitletranslation Feeling }{\suttatitleroot Vedanāsutta}}
\addcontentsline{toc}{section}{\tocacronym{SN 18.5} \toctranslation{Feeling } \tocroot{Vedanāsutta}}
\markboth{Feeling }{Vedanāsutta}
\extramarks{SN 18.5}{SN 18.5}

At\marginnote{1.1} \textsanskrit{Sāvatthī}. 

“What\marginnote{1.2} do you think, \textsanskrit{Rāhula}? Is feeling born of eye contact permanent or impermanent?” 

“Impermanent,\marginnote{1.4} sir.” … 

“…\marginnote{1.5} feeling born of ear contact … feeling born of nose contact … feeling born of tongue contact … feeling born of body contact … Is feeling born of mind contact permanent or impermanent?” 

“Impermanent,\marginnote{1.10} sir.” … 

“Seeing\marginnote{1.11} this, a learned noble disciple grows disillusioned with feeling born of eye contact, ear contact, nose contact, tongue contact, body contact, and mind contact. …” 

%
\section*{{\suttatitleacronym SN 18.6}{\suttatitletranslation Perceptions }{\suttatitleroot Saññāsutta}}
\addcontentsline{toc}{section}{\tocacronym{SN 18.6} \toctranslation{Perceptions } \tocroot{Saññāsutta}}
\markboth{Perceptions }{Saññāsutta}
\extramarks{SN 18.6}{SN 18.6}

At\marginnote{1.1} \textsanskrit{Sāvatthī}. 

“What\marginnote{1.2} do you think, \textsanskrit{Rāhula}? Is perception of sights permanent or impermanent?” 

“Impermanent,\marginnote{1.4} sir.” … 

“…\marginnote{1.5} perception of sounds … perception of smells … perception of tastes … perception of touches … Is perception of ideas permanent or impermanent?” 

“Impermanent,\marginnote{1.10} sir.” … 

“Seeing\marginnote{1.11} this, a learned noble disciple grows disillusioned with the perception of sights, sounds, smells, tastes, touches, and ideas. …” 

%
\section*{{\suttatitleacronym SN 18.7}{\suttatitletranslation Intention }{\suttatitleroot Sañcetanāsutta}}
\addcontentsline{toc}{section}{\tocacronym{SN 18.7} \toctranslation{Intention } \tocroot{Sañcetanāsutta}}
\markboth{Intention }{Sañcetanāsutta}
\extramarks{SN 18.7}{SN 18.7}

At\marginnote{1.1} \textsanskrit{Sāvatthī}. 

“What\marginnote{1.2} do you think, \textsanskrit{Rāhula}? Is intention regarding sights permanent or impermanent?” 

“Impermanent,\marginnote{1.4} sir.” … 

“…\marginnote{1.5} intention regarding sounds … intention regarding smells … intention regarding tastes … intention regarding touches … Is intention regarding ideas permanent or impermanent?” 

“Impermanent,\marginnote{1.10} sir.” … 

“Seeing\marginnote{1.11} this, a learned noble disciple grows disillusioned with intention regarding sights, sounds, smells, tastes, touches, and ideas. …” 

%
\section*{{\suttatitleacronym SN 18.8}{\suttatitletranslation Craving }{\suttatitleroot Taṇhāsutta}}
\addcontentsline{toc}{section}{\tocacronym{SN 18.8} \toctranslation{Craving } \tocroot{Taṇhāsutta}}
\markboth{Craving }{Taṇhāsutta}
\extramarks{SN 18.8}{SN 18.8}

At\marginnote{1.1} \textsanskrit{Sāvatthī}. 

“What\marginnote{1.2} do you think, \textsanskrit{Rāhula}? Is craving for sights permanent or impermanent?” 

“Impermanent,\marginnote{1.4} sir.” … 

“…\marginnote{1.5} craving for sounds … craving for smells … craving for tastes … craving for touches … Is craving for ideas permanent or impermanent?” 

“Impermanent,\marginnote{1.10} sir.” … 

“Seeing\marginnote{1.11} this, a learned noble disciple grows disillusioned with craving for sights, sounds, smells, tastes, touches, and ideas. …” 

%
\section*{{\suttatitleacronym SN 18.9}{\suttatitletranslation Elements }{\suttatitleroot Dhātusutta}}
\addcontentsline{toc}{section}{\tocacronym{SN 18.9} \toctranslation{Elements } \tocroot{Dhātusutta}}
\markboth{Elements }{Dhātusutta}
\extramarks{SN 18.9}{SN 18.9}

At\marginnote{1.1} \textsanskrit{Sāvatthī}. 

“What\marginnote{1.2} do you think, \textsanskrit{Rāhula}? Is the earth element permanent or impermanent?” 

“Impermanent,\marginnote{1.4} sir.” … 

“…\marginnote{1.5} the water element … the fire element … the air element … the space element … Is the consciousness element permanent or impermanent?” 

“Impermanent,\marginnote{1.10} sir.” … 

“Seeing\marginnote{1.11} this, a learned noble disciple grows disillusioned with the earth element, water element, fire element, air element, space element, and consciousness element …” 

%
\section*{{\suttatitleacronym SN 18.10}{\suttatitletranslation The Aggregates }{\suttatitleroot Khandhasutta}}
\addcontentsline{toc}{section}{\tocacronym{SN 18.10} \toctranslation{The Aggregates } \tocroot{Khandhasutta}}
\markboth{The Aggregates }{Khandhasutta}
\extramarks{SN 18.10}{SN 18.10}

At\marginnote{1.1} \textsanskrit{Sāvatthī}. 

“What\marginnote{1.2} do you think, \textsanskrit{Rāhula}? Is form permanent or impermanent?” 

“Impermanent,\marginnote{1.4} sir.” … 

“…\marginnote{1.5} feeling … perception … choices … Is consciousness permanent or impermanent?” 

“Impermanent,\marginnote{1.9} sir.” … 

“Seeing\marginnote{1.10} this, a learned noble disciple grows disillusioned with form, feeling, perception, choices, and consciousness. Being disillusioned, desire fades away. When desire fades away they’re freed. When they’re freed, they know they’re freed. 

They\marginnote{1.12} understand: ‘Rebirth is ended, the spiritual journey has been completed, what had to be done has been done, there is nothing further for this place.’” 

%
\addtocontents{toc}{\let\protect\contentsline\protect\nopagecontentsline}
\chapter*{Chapter Two }
\addcontentsline{toc}{chapter}{\tocchapterline{Chapter Two }}
\addtocontents{toc}{\let\protect\contentsline\protect\oldcontentsline}

%
\section*{{\suttatitleacronym SN 18.11}{\suttatitletranslation The Eye, Etc. }{\suttatitleroot Cakkhusutta}}
\addcontentsline{toc}{section}{\tocacronym{SN 18.11} \toctranslation{The Eye, Etc. } \tocroot{Cakkhusutta}}
\markboth{The Eye, Etc. }{Cakkhusutta}
\extramarks{SN 18.11}{SN 18.11}

\scevam{So\marginnote{1.1} I have heard. }At one time the Buddha was staying near \textsanskrit{Sāvatthī}. Then Venerable \textsanskrit{Rāhula} went up to the Buddha, bowed, and sat down to one side. The Buddha said to him: 

“What\marginnote{1.4} do you think, \textsanskrit{Rāhula}? Is the eye permanent or impermanent?” 

“Impermanent,\marginnote{1.6} sir.” 

“But\marginnote{1.7} if it’s impermanent, is it suffering or happiness?” 

“Suffering,\marginnote{1.8} sir.” 

“But\marginnote{1.9} if it’s impermanent, suffering, and perishable, is it fit to be regarded thus: ‘This is mine, I am this, this is my self’?” 

“No,\marginnote{1.11} sir.” 

“…\marginnote{1.12} the ear … the nose … the tongue … the body … Is the mind permanent or impermanent?” 

“Impermanent,\marginnote{1.17} sir.” 

“But\marginnote{1.18} if it’s impermanent, is it suffering or happiness?” 

“Suffering,\marginnote{1.19} sir.” 

“But\marginnote{1.20} if it’s impermanent, suffering, and perishable, is it fit to be regarded thus: ‘This is mine, I am this, this is my self’?” 

“No,\marginnote{1.22} sir.” 

“Seeing\marginnote{1.23} this, a learned noble disciple grows disillusioned with the eye, the ear, the nose, the tongue, the body, and the mind. Being disillusioned, desire fades away. When desire fades away they’re freed. When they’re freed, they know they’re freed. 

They\marginnote{1.25} understand: ‘Rebirth is ended, the spiritual journey has been completed, what had to be done has been done, there is nothing further for this place.’” 

\scexpansioninstructions{(The ten discourses of this series should be told in full the same way.) }

%
\section*{{\suttatitleacronym SN 18.12–20}{\suttatitletranslation The Nine Discourses on Sights, Etc. }{\suttatitleroot Rūpādisuttanavaka}}
\addcontentsline{toc}{section}{\tocacronym{SN 18.12–20} \toctranslation{The Nine Discourses on Sights, Etc. } \tocroot{Rūpādisuttanavaka}}
\markboth{The Nine Discourses on Sights, Etc. }{Rūpādisuttanavaka}
\extramarks{SN 18.12–20}{SN 18.12–20}

At\marginnote{1.1} \textsanskrit{Sāvatthī}. 

“What\marginnote{1.2} do you think, \textsanskrit{Rāhula}? Are sights permanent or impermanent?” 

“Impermanent,\marginnote{1.4} sir.” … “… sounds … smells … tastes … touches … ideas …” 

“…\marginnote{2.1} eye consciousness … ear consciousness … nose consciousness … tongue consciousness … body consciousness … mind consciousness …” 

“…\marginnote{3.1} eye contact … ear contact … nose contact … tongue contact … body contact … mind contact …” 

“…\marginnote{4.1} feeling born of eye contact … feeling born of ear contact … feeling born of nose contact … feeling born of tongue contact … feeling born of body contact … feeling born of mind contact …” 

“…\marginnote{5.1} perception of sights … perception of sounds … perception of smells … perception of tastes … perception of touches … perception of ideas …” 

“…\marginnote{6.1} intention regarding sights … intention regarding sounds … intention regarding smells … intention regarding tastes … intention regarding touches … intention regarding ideas …” 

“…\marginnote{7.1} craving for sights … craving for sounds … craving for smells … craving for tastes … craving for touches … craving for ideas …” 

“…\marginnote{8.1} the earth element … the water element … the fire element … the air element … the space element … the consciousness element …” 

“…\marginnote{9.1} form … feeling … perception … choices … Is consciousness permanent or impermanent?” 

“Impermanent,\marginnote{9.6} sir.” … 

“Seeing\marginnote{9.7} this … They understand: ‘… there is nothing further for this place.’” 

%
\section*{{\suttatitleacronym SN 18.21}{\suttatitletranslation Tendency }{\suttatitleroot Anusayasutta}}
\addcontentsline{toc}{section}{\tocacronym{SN 18.21} \toctranslation{Tendency } \tocroot{Anusayasutta}}
\markboth{Tendency }{Anusayasutta}
\extramarks{SN 18.21}{SN 18.21}

At\marginnote{1.1} \textsanskrit{Sāvatthī}. 

Then\marginnote{1.2} Venerable \textsanskrit{Rāhula} went up to the Buddha, bowed, sat down to one side, and said to him: 

“Sir,\marginnote{1.3} how does one know and see so that there’s no I-making, mine-making, or underlying tendency to conceit for this conscious body and all external stimuli?” 

“\textsanskrit{Rāhula},\marginnote{1.4} one truly sees any kind of form at all—past, future, or present; internal or external; solid or subtle; inferior or superior; far or near: \emph{all} form—with right understanding: ‘This is not mine, I am not this, this is not my self.’ One truly sees any kind of feeling … perception … choices … consciousness at all—past, future, or present; internal or external; solid or subtle; inferior or superior; far or near: \emph{all} consciousness—with right understanding: ‘This is not mine, I am not this, this is not my self.’ 

That’s\marginnote{1.9} how to know and see so that there’s no I-making, mine-making, or underlying tendency to conceit for this conscious body and all external stimuli.” 

%
\section*{{\suttatitleacronym SN 18.22}{\suttatitletranslation Rid of Conceit }{\suttatitleroot Apagatasutta}}
\addcontentsline{toc}{section}{\tocacronym{SN 18.22} \toctranslation{Rid of Conceit } \tocroot{Apagatasutta}}
\markboth{Rid of Conceit }{Apagatasutta}
\extramarks{SN 18.22}{SN 18.22}

At\marginnote{1.1} \textsanskrit{Sāvatthī}. 

Then\marginnote{1.2} Venerable \textsanskrit{Rāhula} went up to the Buddha, bowed, sat down to one side, and said to him: 

“Sir,\marginnote{1.3} how does one know and see so that the mind is rid of I-making, mine-making, and conceit for this conscious body and all external stimuli; and going beyond discrimination, it’s peaceful and well freed?” 

“\textsanskrit{Rāhula},\marginnote{1.4} when one truly sees any kind of form at all—past, future, or present; internal or external; solid or subtle; inferior or superior; far or near: \emph{all} form—with right understanding: ‘This is not mine, I am not this, this is not my self,’ one is freed by not grasping. 

When\marginnote{2.1} one truly sees any kind of feeling … perception … choices … When one truly sees any kind of consciousness at all—past, future, or present; internal or external; solid or subtle; inferior or superior; far or near: \emph{all} consciousness—with right understanding: ‘This is not mine, I am not this, this is not my self,’ one is freed by not grasping. 

That’s\marginnote{2.5} how to know and see so that the mind is rid of I-making, mine-making, and conceit for this conscious body and all external stimuli; and going beyond discrimination, it’s peaceful and well freed.” 

\scendkanda{The Linked Discourses with \textsanskrit{Rāhula} are complete. }

%
\addtocontents{toc}{\let\protect\contentsline\protect\nopagecontentsline}
\part*{Linked Discourses with Lakkhaṇa }
\addcontentsline{toc}{part}{Linked Discourses with Lakkhaṇa }
\markboth{}{}
\addtocontents{toc}{\let\protect\contentsline\protect\oldcontentsline}

%
\addtocontents{toc}{\let\protect\contentsline\protect\nopagecontentsline}
\chapter*{Chapter One }
\addcontentsline{toc}{chapter}{\tocchapterline{Chapter One }}
\addtocontents{toc}{\let\protect\contentsline\protect\oldcontentsline}

%
\section*{{\suttatitleacronym SN 19.1}{\suttatitletranslation A Skeleton }{\suttatitleroot Aṭṭhisutta}}
\addcontentsline{toc}{section}{\tocacronym{SN 19.1} \toctranslation{A Skeleton } \tocroot{Aṭṭhisutta}}
\markboth{A Skeleton }{Aṭṭhisutta}
\extramarks{SN 19.1}{SN 19.1}

\scevam{So\marginnote{1.1} I have heard. }At one time the Buddha was staying near \textsanskrit{Rājagaha}, in the Bamboo Grove, the squirrels’ feeding ground. 

Now\marginnote{1.3} at that time Venerable \textsanskrit{Lakkhaṇa} and Venerable \textsanskrit{Mahāmoggallāna} were staying on the Vulture’s Peak Mountain. Then \textsanskrit{Mahāmoggallāna} robed up in the morning and, taking his bowl and robe, went to \textsanskrit{Lakkhaṇa} and said to him, “Come, Reverend \textsanskrit{Lakkhaṇa}, let’s enter \textsanskrit{Rājagaha} for alms.” 

“Yes,\marginnote{1.6} reverend,” \textsanskrit{Lakkhaṇa} replied. 

As\marginnote{1.7} \textsanskrit{Mahāmoggallāna} was descending from Vulture’s Peak Mountain he smiled at a certain spot. So \textsanskrit{Lakkhaṇa} said to \textsanskrit{Mahāmoggallāna}, “What is the cause, Reverend \textsanskrit{Moggallāna}, what is the reason you smiled?” 

“Reverend\marginnote{1.10} \textsanskrit{Lakkhaṇa}, it’s the wrong time for this question. Ask me when we’re in the Buddha’s presence.” 

Then\marginnote{2.1} \textsanskrit{Lakkhaṇa} and \textsanskrit{Mahāmoggallāna} wandered for alms in \textsanskrit{Rājagaha}. After the meal, on their return from almsround, they went to the Buddha, bowed, and sat down to one side. \textsanskrit{Lakkhaṇa} said to \textsanskrit{Mahāmoggallāna}: 

“Just\marginnote{2.2} now, as \textsanskrit{Mahāmoggallāna} was descending from Vulture’s Peak Mountain he smiled at a certain spot. What is the cause, Reverend \textsanskrit{Moggallāna}, what is the reason you smiled?” 

“Just\marginnote{3.1} now, reverend, as I was descending from Vulture’s Peak Mountain I saw a skeleton flying through the air. Vultures, crows, and hawks kept chasing it, pecking, clawing, and stabbing it in the ribs as it screeched in pain. It occurred to me: ‘Oh, how incredible, how amazing! That there can be such a sentient being, such an entity, such an incarnation!’” 

Then\marginnote{4.1} the Buddha said to the mendicants: 

“Mendicants,\marginnote{4.2} there are disciples who live full of vision and knowledge, since a disciple knows, sees, and witnesses such a thing. 

Formerly,\marginnote{4.4} I too saw that being, but I did not speak of it. For if I had spoken of it others would not have believed me, which would be for their lasting harm and suffering. 

That\marginnote{4.7} being used to be a cattle butcher right here in \textsanskrit{Rājagaha}. As a result of that deed he burned in hell for many years, many hundreds, many thousands, many hundreds of thousands of years. Now he experiences the residual result of that deed in such an incarnation.” 

\scexpansioninstructions{(Tell all these discourses in full like this.) }

%
\section*{{\suttatitleacronym SN 19.2}{\suttatitletranslation A Piece of Meat }{\suttatitleroot Pesisutta}}
\addcontentsline{toc}{section}{\tocacronym{SN 19.2} \toctranslation{A Piece of Meat } \tocroot{Pesisutta}}
\markboth{A Piece of Meat }{Pesisutta}
\extramarks{SN 19.2}{SN 19.2}

“Just\marginnote{1.1} now, reverend, as I was descending from Vulture’s Peak Mountain I saw a scrap of meat flying through the air. Vultures, crows, and hawks kept chasing it, pecking and clawing as it screeched in pain. …” … 

“That\marginnote{1.4} being used to be a cattle butcher right here in \textsanskrit{Rājagaha}. …” 

%
\section*{{\suttatitleacronym SN 19.3}{\suttatitletranslation A Piece of Flesh }{\suttatitleroot Piṇḍasutta}}
\addcontentsline{toc}{section}{\tocacronym{SN 19.3} \toctranslation{A Piece of Flesh } \tocroot{Piṇḍasutta}}
\markboth{A Piece of Flesh }{Piṇḍasutta}
\extramarks{SN 19.3}{SN 19.3}

“Just\marginnote{1.1} now, reverend, as I was descending from Vulture’s Peak Mountain I saw a piece of flesh flying through the air. Vultures, crows, and hawks kept chasing it, pecking and clawing as it screeched in pain. …” … 

“That\marginnote{1.4} being used to be a bird hunter right here in \textsanskrit{Rājagaha}. …” 

%
\section*{{\suttatitleacronym SN 19.4}{\suttatitletranslation A Flayed Man }{\suttatitleroot Nicchavisutta}}
\addcontentsline{toc}{section}{\tocacronym{SN 19.4} \toctranslation{A Flayed Man } \tocroot{Nicchavisutta}}
\markboth{A Flayed Man }{Nicchavisutta}
\extramarks{SN 19.4}{SN 19.4}

“Just\marginnote{1.1} now, reverend, as I was descending from Vulture’s Peak Mountain I saw a flayed man flying through the air. Vultures, crows, and hawks kept chasing it, pecking and clawing as he screamed in pain. …” … 

“That\marginnote{1.4} being used to be a sheep butcher right here in \textsanskrit{Rājagaha}. …” 

%
\section*{{\suttatitleacronym SN 19.5}{\suttatitletranslation Sword Hairs }{\suttatitleroot Asilomasutta}}
\addcontentsline{toc}{section}{\tocacronym{SN 19.5} \toctranslation{Sword Hairs } \tocroot{Asilomasutta}}
\markboth{Sword Hairs }{Asilomasutta}
\extramarks{SN 19.5}{SN 19.5}

“Just\marginnote{1.1} now, reverend, as I was descending from Vulture’s Peak Mountain I saw a man whose body hairs were swords flying through the air. And those swords kept rising up and falling on his body as he screamed in pain. …” … 

“That\marginnote{1.4} being used to be a pig butcher right here in \textsanskrit{Rājagaha}. …” 

%
\section*{{\suttatitleacronym SN 19.6}{\suttatitletranslation Spear Hairs }{\suttatitleroot Sattisutta}}
\addcontentsline{toc}{section}{\tocacronym{SN 19.6} \toctranslation{Spear Hairs } \tocroot{Sattisutta}}
\markboth{Spear Hairs }{Sattisutta}
\extramarks{SN 19.6}{SN 19.6}

“Just\marginnote{1.1} now, reverend, as I was descending from Vulture’s Peak Mountain I saw a man whose body hairs were spears flying through the air. And those spears kept rising up and falling on his body as he screamed in pain. …” … 

“That\marginnote{1.4} being used to be a deer hunter right here in \textsanskrit{Rājagaha}. …” 

%
\section*{{\suttatitleacronym SN 19.7}{\suttatitletranslation Arrow Hairs }{\suttatitleroot Usulomasutta}}
\addcontentsline{toc}{section}{\tocacronym{SN 19.7} \toctranslation{Arrow Hairs } \tocroot{Usulomasutta}}
\markboth{Arrow Hairs }{Usulomasutta}
\extramarks{SN 19.7}{SN 19.7}

“Just\marginnote{1.1} now, reverend, as I was descending from Vulture’s Peak Mountain I saw a man whose body hairs were arrows flying through the air. And those arrows kept rising up and falling on his body as he screamed in pain. …” … 

“That\marginnote{1.4} being used to be a torturer right here in \textsanskrit{Rājagaha}. …” 

%
\section*{{\suttatitleacronym SN 19.8}{\suttatitletranslation Needle Hairs }{\suttatitleroot Sūcilomasutta}}
\addcontentsline{toc}{section}{\tocacronym{SN 19.8} \toctranslation{Needle Hairs } \tocroot{Sūcilomasutta}}
\markboth{Needle Hairs }{Sūcilomasutta}
\extramarks{SN 19.8}{SN 19.8}

“Just\marginnote{1.1} now, reverend, as I was descending from Vulture’s Peak Mountain I saw a man whose body hairs were needles flying through the air. And those needles kept rising up and falling on his body as he screamed in pain. …” … 

“That\marginnote{1.4} being used to be a war herald right here in \textsanskrit{Rājagaha}. …” 

%
\section*{{\suttatitleacronym SN 19.9}{\suttatitletranslation Needle Hairs (2nd) }{\suttatitleroot Dutiyasūcilomasutta}}
\addcontentsline{toc}{section}{\tocacronym{SN 19.9} \toctranslation{Needle Hairs (2nd) } \tocroot{Dutiyasūcilomasutta}}
\markboth{Needle Hairs (2nd) }{Dutiyasūcilomasutta}
\extramarks{SN 19.9}{SN 19.9}

“Just\marginnote{1.1} now, reverend, as I was descending from Vulture’s Peak Mountain I saw a man whose body hairs were needles flying through the air. The needles bored into his head and out his mouth, into his mouth and out his chest, into his chest and out his belly, into his belly and out his thighs, into his thighs and out his calves, and into his calves and out his feet. And he screamed in pain. …” … 

“That\marginnote{1.9} being used to be an informant right here in \textsanskrit{Rājagaha}. …” 

%
\section*{{\suttatitleacronym SN 19.10}{\suttatitletranslation Pot Balls }{\suttatitleroot Kumbhaṇḍasutta}}
\addcontentsline{toc}{section}{\tocacronym{SN 19.10} \toctranslation{Pot Balls } \tocroot{Kumbhaṇḍasutta}}
\markboth{Pot Balls }{Kumbhaṇḍasutta}
\extramarks{SN 19.10}{SN 19.10}

“Just\marginnote{1.1} now, reverend, as I was descending from Vulture’s Peak Mountain I saw a man with testicles as big as pots flying through the air. When he was walking he had to lift his testicles on to his shoulder. And when he sat down, he sat right on them. Vultures, crows, and hawks kept chasing him, pecking and clawing as he screamed in pain. …” … 

“That\marginnote{1.6} being used to be a corrupt official right here in \textsanskrit{Rājagaha}. …” 

%
\addtocontents{toc}{\let\protect\contentsline\protect\nopagecontentsline}
\chapter*{Chapter Two }
\addcontentsline{toc}{chapter}{\tocchapterline{Chapter Two }}
\addtocontents{toc}{\let\protect\contentsline\protect\oldcontentsline}

%
\section*{{\suttatitleacronym SN 19.11}{\suttatitletranslation Over His Head }{\suttatitleroot Sasīsakasutta}}
\addcontentsline{toc}{section}{\tocacronym{SN 19.11} \toctranslation{Over His Head } \tocroot{Sasīsakasutta}}
\markboth{Over His Head }{Sasīsakasutta}
\extramarks{SN 19.11}{SN 19.11}

\scevam{So\marginnote{1.1} I have heard. }At one time near \textsanskrit{Rājagaha} in the Bamboo Grove … 

“Just\marginnote{1.3} now, reverend, as I was descending from Vulture’s Peak Mountain I saw a man sunk over his head in a sewer. …” … 

“That\marginnote{1.4} being used to be an adulterer right here in \textsanskrit{Rājagaha}. …” 

%
\section*{{\suttatitleacronym SN 19.12}{\suttatitletranslation A Dung Eater }{\suttatitleroot Gūthakhādasutta}}
\addcontentsline{toc}{section}{\tocacronym{SN 19.12} \toctranslation{A Dung Eater } \tocroot{Gūthakhādasutta}}
\markboth{A Dung Eater }{Gūthakhādasutta}
\extramarks{SN 19.12}{SN 19.12}

“Just\marginnote{1.1} now, reverend, as I was descending from Vulture’s Peak Mountain I saw a man sunk in a sewer, eating dung with both hands. …” … 

“That\marginnote{1.2} being used to be a nasty brahmin right here in \textsanskrit{Rājagaha}. In the time of the Buddha Kassapa’s dispensation he invited the \textsanskrit{Saṅgha} of mendicants for a meal. He filled a trough with dung and said: ‘My good men, eat as much as you like, and take what’s left.’ …” 

%
\section*{{\suttatitleacronym SN 19.13}{\suttatitletranslation A Flayed Woman }{\suttatitleroot Nicchavitthisutta}}
\addcontentsline{toc}{section}{\tocacronym{SN 19.13} \toctranslation{A Flayed Woman } \tocroot{Nicchavitthisutta}}
\markboth{A Flayed Woman }{Nicchavitthisutta}
\extramarks{SN 19.13}{SN 19.13}

“Just\marginnote{1.1} now, reverend, as I was descending from Vulture’s Peak Mountain I saw a flayed woman flying through the air. Vultures, crows, and hawks kept chasing her, pecking and clawing as she screamed in pain. …” … 

“That\marginnote{1.4} woman used to be an adulteress right here in \textsanskrit{Rājagaha}. …” 

%
\section*{{\suttatitleacronym SN 19.14}{\suttatitletranslation A Fishwife }{\suttatitleroot Maṅgulitthisutta}}
\addcontentsline{toc}{section}{\tocacronym{SN 19.14} \toctranslation{A Fishwife } \tocroot{Maṅgulitthisutta}}
\markboth{A Fishwife }{Maṅgulitthisutta}
\extramarks{SN 19.14}{SN 19.14}

“Just\marginnote{1.1} now, reverend, as I was descending from Vulture’s Peak Mountain I saw a stinking fishwife flying through the air. Vultures, crows, and hawks kept chasing her, pecking and clawing as she screamed in pain. …” … 

“That\marginnote{1.4} woman used to be a fortune-teller right here in \textsanskrit{Rājagaha}. …” 

%
\section*{{\suttatitleacronym SN 19.15}{\suttatitletranslation A Sweltering Woman }{\suttatitleroot Okilinīsutta}}
\addcontentsline{toc}{section}{\tocacronym{SN 19.15} \toctranslation{A Sweltering Woman } \tocroot{Okilinīsutta}}
\markboth{A Sweltering Woman }{Okilinīsutta}
\extramarks{SN 19.15}{SN 19.15}

“Just\marginnote{1.1} now, reverend, as I was descending from Vulture’s Peak Mountain I saw a scorched woman, sooty and sweaty, flying through the air, as she screamed in pain. …” … 

“That\marginnote{1.3} woman used to be the king of \textsanskrit{Kaliṅga}’s chief queen. She was of jealous nature, and poured a brazier of hot coals over her co-wife. …” … 

%
\section*{{\suttatitleacronym SN 19.16}{\suttatitletranslation A Headless Trunk }{\suttatitleroot Asīsakasutta}}
\addcontentsline{toc}{section}{\tocacronym{SN 19.16} \toctranslation{A Headless Trunk } \tocroot{Asīsakasutta}}
\markboth{A Headless Trunk }{Asīsakasutta}
\extramarks{SN 19.16}{SN 19.16}

“Just\marginnote{1.1} now, reverend, as I was descending from Vulture’s Peak Mountain I saw a headless trunk flying through the air. Its eyes and mouth were on its chest. Vultures, crows, and hawks kept chasing it, pecking and clawing as it screamed in pain. …” … 

“That\marginnote{1.5} being used to be an executioner called \textsanskrit{Hārika} right here in \textsanskrit{Rājagaha}. …” 

%
\section*{{\suttatitleacronym SN 19.17}{\suttatitletranslation A Bad Monk }{\suttatitleroot Pāpabhikkhusutta}}
\addcontentsline{toc}{section}{\tocacronym{SN 19.17} \toctranslation{A Bad Monk } \tocroot{Pāpabhikkhusutta}}
\markboth{A Bad Monk }{Pāpabhikkhusutta}
\extramarks{SN 19.17}{SN 19.17}

“Just\marginnote{1.1} now, reverend, as I was descending from Vulture’s Peak Mountain I saw a monk flying through the air. His outer robe, bowl, belt, and body were burning, blazing, and glowing as he screamed in pain. …” … 

“That\marginnote{1.4} monk used to be a bad monk in the time of Buddha Kassapa’s dispensation. …” 

%
\section*{{\suttatitleacronym SN 19.18}{\suttatitletranslation A Bad Nun }{\suttatitleroot Pāpabhikkhunīsutta}}
\addcontentsline{toc}{section}{\tocacronym{SN 19.18} \toctranslation{A Bad Nun } \tocroot{Pāpabhikkhunīsutta}}
\markboth{A Bad Nun }{Pāpabhikkhunīsutta}
\extramarks{SN 19.18}{SN 19.18}

“I\marginnote{1.1} saw a nun flying through the air. Her outer robe was burning …” … 

“She\marginnote{1.3} used to be a bad nun …” 

%
\section*{{\suttatitleacronym SN 19.19}{\suttatitletranslation A Bad Trainee Nun }{\suttatitleroot Pāpasikkhamānasutta}}
\addcontentsline{toc}{section}{\tocacronym{SN 19.19} \toctranslation{A Bad Trainee Nun } \tocroot{Pāpasikkhamānasutta}}
\markboth{A Bad Trainee Nun }{Pāpasikkhamānasutta}
\extramarks{SN 19.19}{SN 19.19}

“I\marginnote{1.1} saw a trainee nun flying through the air. Her outer robe was burning …” … 

“She\marginnote{1.3} used to be a bad trainee nun …” 

%
\section*{{\suttatitleacronym SN 19.20}{\suttatitletranslation A Bad Novice Monk }{\suttatitleroot Pāpasāmaṇerasutta}}
\addcontentsline{toc}{section}{\tocacronym{SN 19.20} \toctranslation{A Bad Novice Monk } \tocroot{Pāpasāmaṇerasutta}}
\markboth{A Bad Novice Monk }{Pāpasāmaṇerasutta}
\extramarks{SN 19.20}{SN 19.20}

“I\marginnote{1.1} saw a novice monk flying through the air. His outer robe was burning …” … 

“He\marginnote{1.3} used to be a bad novice monk …” 

%
\section*{{\suttatitleacronym SN 19.21}{\suttatitletranslation A Bad Novice Nun }{\suttatitleroot Pāpasāmaṇerīsutta}}
\addcontentsline{toc}{section}{\tocacronym{SN 19.21} \toctranslation{A Bad Novice Nun } \tocroot{Pāpasāmaṇerīsutta}}
\markboth{A Bad Novice Nun }{Pāpasāmaṇerīsutta}
\extramarks{SN 19.21}{SN 19.21}

“Just\marginnote{1.1} now, reverend, as I was descending from Vulture’s Peak Mountain I saw a novice nun flying through the air. Her outer robe, bowl, belt, and body were burning, blazing, and glowing as she screamed in pain. It occurred to me: ‘Oh, how incredible, how amazing! That there can be such a sentient being, such an entity, such an incarnation!’” 

Then\marginnote{2.1} the Buddha said to the mendicants: 

“Mendicants,\marginnote{2.2} there are disciples who live full of vision and knowledge, since a disciple knows, sees, and witnesses such a thing. 

Formerly,\marginnote{2.4} I too saw that novice nun, but I did not speak of it. For if I had spoken of it others would not have believed me, which would be for their lasting harm and suffering. 

That\marginnote{2.8} female novice used to be a bad novice nun in the time of the Buddha Kassapa’s dispensation. As a result of that deed she burned in hell for many years, many hundreds, many thousands, many hundreds of thousands of years. Now she experiences the residual result of that deed in such an incarnation.” 

\scendkanda{The Linked Discourses with \textsanskrit{Lakkhaṇa} are complete. }

%
\addtocontents{toc}{\let\protect\contentsline\protect\nopagecontentsline}
\part*{Linked Discourses with Similes }
\addcontentsline{toc}{part}{Linked Discourses with Similes }
\markboth{}{}
\addtocontents{toc}{\let\protect\contentsline\protect\oldcontentsline}

%
\addtocontents{toc}{\let\protect\contentsline\protect\nopagecontentsline}
\chapter*{The Chapter on the Similes }
\addcontentsline{toc}{chapter}{\tocchapterline{The Chapter on the Similes }}
\addtocontents{toc}{\let\protect\contentsline\protect\oldcontentsline}

%
\section*{{\suttatitleacronym SN 20.1}{\suttatitletranslation A Roof Peak }{\suttatitleroot Kūṭasutta}}
\addcontentsline{toc}{section}{\tocacronym{SN 20.1} \toctranslation{A Roof Peak } \tocroot{Kūṭasutta}}
\markboth{A Roof Peak }{Kūṭasutta}
\extramarks{SN 20.1}{SN 20.1}

\scevam{So\marginnote{1.1} I have heard. }At one time the Buddha was staying near \textsanskrit{Sāvatthī} in Jeta’s Grove, \textsanskrit{Anāthapiṇḍika}’s monastery. 

There\marginnote{1.3} the Buddha … said: 

“Mendicants,\marginnote{1.4} the rafters of a bungalow all lean to the peak and meet at the peak, and when the peak is demolished they’re all demolished too. In the same way all unskillful qualities are rooted in ignorance and meet in ignorance, and when ignorance is demolished they’re all demolished too. 

So\marginnote{1.6} you should train like this: ‘We will stay diligent.’ That’s how you should train.” 

%
\section*{{\suttatitleacronym SN 20.2}{\suttatitletranslation A Fingernail }{\suttatitleroot Nakhasikhasutta}}
\addcontentsline{toc}{section}{\tocacronym{SN 20.2} \toctranslation{A Fingernail } \tocroot{Nakhasikhasutta}}
\markboth{A Fingernail }{Nakhasikhasutta}
\extramarks{SN 20.2}{SN 20.2}

At\marginnote{1.1} \textsanskrit{Sāvatthī}. 

Then\marginnote{1.2} the Buddha, picking up a little bit of dirt under his fingernail, addressed the mendicants: “What do you think, mendicants? Which is more: the little bit of dirt under my fingernail, or this great earth?” 

“Sir,\marginnote{1.5} the great earth is far more. The little bit of dirt under your fingernail is tiny. Compared to the great earth, it doesn’t count, there’s no comparison, it’s not worth a fraction.” 

“In\marginnote{1.8} the same way the sentient beings reborn as humans are few, while those not reborn as humans are many. 

So\marginnote{1.10} you should train like this: ‘We will stay diligent.’ That’s how you should train.” 

%
\section*{{\suttatitleacronym SN 20.3}{\suttatitletranslation Families }{\suttatitleroot Kulasutta}}
\addcontentsline{toc}{section}{\tocacronym{SN 20.3} \toctranslation{Families } \tocroot{Kulasutta}}
\markboth{Families }{Kulasutta}
\extramarks{SN 20.3}{SN 20.3}

At\marginnote{1.1} \textsanskrit{Sāvatthī}. 

“Mendicants,\marginnote{1.2} those families with many women and few men are easy prey for bandits and thieves. In the same way any mendicant who has not developed and cultivated the heart’s release by love is easy prey for non-humans. Those families with few women and many men are hard prey for bandits and thieves. In the same way a mendicant who has developed and cultivated the heart’s release by love is hard prey for non-humans. 

So\marginnote{1.6} you should train like this: ‘We will develop the heart’s release by love. We’ll cultivate it, make it our vehicle and our basis, keep it up, consolidate it, and properly implement it.’ That’s how you should train.” 

%
\section*{{\suttatitleacronym SN 20.4}{\suttatitletranslation Rice Pots }{\suttatitleroot Okkhāsutta}}
\addcontentsline{toc}{section}{\tocacronym{SN 20.4} \toctranslation{Rice Pots } \tocroot{Okkhāsutta}}
\markboth{Rice Pots }{Okkhāsutta}
\extramarks{SN 20.4}{SN 20.4}

At\marginnote{1.1} \textsanskrit{Sāvatthī}. 

“Mendicants,\marginnote{1.2} suppose one person was to give a gift of a hundred pots of rice in the morning, at midday, and in the evening. And someone else was to develop a heart of love, even just as long as it takes to pull a cow’s udder. The latter would be more fruitful. 

So\marginnote{1.3} you should train like this: ‘We will develop the heart’s release by love. We’ll cultivate it, make it our vehicle and our basis, keep it up, consolidate it, and properly implement it.’ That’s how you should train.” 

%
\section*{{\suttatitleacronym SN 20.5}{\suttatitletranslation A Spear }{\suttatitleroot Sattisutta}}
\addcontentsline{toc}{section}{\tocacronym{SN 20.5} \toctranslation{A Spear } \tocroot{Sattisutta}}
\markboth{A Spear }{Sattisutta}
\extramarks{SN 20.5}{SN 20.5}

At\marginnote{1.1} \textsanskrit{Sāvatthī}. 

“Mendicants,\marginnote{1.2} suppose there was a sharp-pointed spear. And a man came along and thought, ‘With my hand or fist I’ll fold this sharp spear over, crumple it, and bend it back!’ 

What\marginnote{1.5} do you think, mendicants? Is that man capable of doing so?” 

“No,\marginnote{1.7} sir. Why not? Because it’s not easy to fold that sharp spear over, crumple it, and bend it back with the hand or fist. That man will eventually get weary and frustrated.” 

“In\marginnote{2.1} the same way, suppose a mendicant has developed the heart’s release by love, has cultivated it, made it a vehicle and a basis, kept it up, consolidated it, and properly implemented it. Should any non-human think to overthrow their mind, they’ll eventually get weary and frustrated. 

So\marginnote{2.3} you should train like this: ‘We will develop the heart’s release by love. We’ll cultivate it, make it our vehicle and our basis, keep it up, consolidate it, and properly implement it.’ That’s how you should train.” 

%
\section*{{\suttatitleacronym SN 20.6}{\suttatitletranslation The Archers }{\suttatitleroot Dhanuggahasutta}}
\addcontentsline{toc}{section}{\tocacronym{SN 20.6} \toctranslation{The Archers } \tocroot{Dhanuggahasutta}}
\markboth{The Archers }{Dhanuggahasutta}
\extramarks{SN 20.6}{SN 20.6}

At\marginnote{1.1} \textsanskrit{Sāvatthī}. 

“Mendicants,\marginnote{1.2} suppose there were four well-trained expert archers with strong bows standing in the four quarters. And a man came along and thought, ‘When these four well-trained expert archers shoot arrows in four quarters, I’ll catch them before they reach the ground, and then I’ll bring them back.’ 

What\marginnote{1.5} do you think, mendicants? Are they qualified to be called ‘a speedster, with ultimate speed’?” 

“If\marginnote{2.1} he could catch an arrow shot by just one well-trained expert archer before it reaches the ground and bring it back, he’d be qualified to be called ‘a speedster, with ultimate speed’. How much more so arrows shot by four archers!” 

“As\marginnote{3.1} fast as that man is, the sun and moon are faster. As fast as that man is, as fast as the sun and moon are, and as fast as the deities that run before the sun and moon are, the waning of the life forces is faster. 

So\marginnote{3.3} you should train like this: ‘We will stay diligent.’ That’s how you should train.” 

%
\section*{{\suttatitleacronym SN 20.7}{\suttatitletranslation The Drum Peg }{\suttatitleroot Āṇisutta}}
\addcontentsline{toc}{section}{\tocacronym{SN 20.7} \toctranslation{The Drum Peg } \tocroot{Āṇisutta}}
\markboth{The Drum Peg }{Āṇisutta}
\extramarks{SN 20.7}{SN 20.7}

At\marginnote{1.1} \textsanskrit{Sāvatthī}. 

“Once\marginnote{1.2} upon a time, mendicants, the \textsanskrit{Dasārahas} had a clay drum called the Commander. Each time the Commander split they repaired it by inserting another peg. But there came a time when the clay drum Commander’s original wooden rim disappeared and only a mass of pegs remained. 

In\marginnote{1.6} the same way, in a future time there will be mendicants who won’t want to listen when discourses spoken by the Realized One—deep, profound, transcendent, dealing with emptiness—are being recited. They won’t actively listen or try to understand, nor will they think those teachings are worth learning and memorizing. 

But\marginnote{2.1} when discourses composed by poets—poetry, with fancy words and phrases, composed by outsiders or spoken by disciples—are being recited they will want to listen. They’ll actively listen and try to understand, and they’ll think those teachings are worth learning and memorizing. And that is how the discourses spoken by the Realized One—deep, profound, transcendent, dealing with emptiness—will disappear. 

So\marginnote{2.3} you should train like this: ‘When discourses spoken by the Realized One—deep, profound, transcendent, dealing with emptiness—are being recited we will want to listen. We will actively listen and trying to understand, and we will think those teachings are worth learning and memorizing.’ That’s how you should train.” 

%
\section*{{\suttatitleacronym SN 20.8}{\suttatitletranslation Wood Blocks }{\suttatitleroot Kaliṅgarasutta}}
\addcontentsline{toc}{section}{\tocacronym{SN 20.8} \toctranslation{Wood Blocks } \tocroot{Kaliṅgarasutta}}
\markboth{Wood Blocks }{Kaliṅgarasutta}
\extramarks{SN 20.8}{SN 20.8}

\scevam{So\marginnote{1.1} I have heard. }At one time the Buddha was staying near \textsanskrit{Vesālī}, at the Great Wood, in the hall with the peaked roof. There the Buddha addressed the mendicants, “Mendicants!” 

“Venerable\marginnote{1.5} sir,” they replied. The Buddha said this: 

“Mendicants,\marginnote{2.1} these days the Licchavis live using wood blocks as pillows, and they exercise diligently and keenly. King \textsanskrit{Ajātasattu} of Magadha, son of the princess of Videha, finds no vulnerability, he’s got no foothold. But in the future the Licchavis will become delicate, with soft and tender hands and feet. They’ll sleep on soft beds with down pillows until the sun comes up. King \textsanskrit{Ajātasattu} of Magadha, son of the princess of Videha, will find a vulnerability, he’ll get his foothold. 

These\marginnote{3.1} days the mendicants live using wood blocks as pillows, and they meditate diligently and keenly. \textsanskrit{Māra} the Wicked finds no vulnerability, he's got no foothold. But in the future the mendicants will become delicate, with soft and tender hands and feet. They’ll sleep on soft beds with down pillows until the sun comes up. \textsanskrit{Māra} the Wicked will find a vulnerability and will get a foothold. 

So\marginnote{3.6} you should train like this: ‘We will live using wood blocks as pillows, and we will meditate diligently and keenly.’ That’s how you should train.” 

%
\section*{{\suttatitleacronym SN 20.9}{\suttatitletranslation A Bull Elephant }{\suttatitleroot Nāgasutta}}
\addcontentsline{toc}{section}{\tocacronym{SN 20.9} \toctranslation{A Bull Elephant } \tocroot{Nāgasutta}}
\markboth{A Bull Elephant }{Nāgasutta}
\extramarks{SN 20.9}{SN 20.9}

\scevam{So\marginnote{1.1} I have heard. }At one time the Buddha was staying near \textsanskrit{Sāvatthī} in Jeta’s Grove, \textsanskrit{Anāthapiṇḍika}’s monastery. Now at that time a certain junior mendicant went to visit families too often. 

The\marginnote{1.4} mendicants said to him, “Venerable, don’t go to visit families too often.” 

But\marginnote{1.6} that mendicant, when spoken to by the mendicants, said this, “But these senior mendicants think they can go to visit families, so why can’t I?” 

And\marginnote{2.1} then several mendicants went up to the Buddha, bowed, sat down to one side, and told him what had happened. The Buddha said: 

“Once\marginnote{3.1} upon a time, mendicants, there was a great lake in the jungle, with bull elephants living nearby. They’d plunge into the lake and pull up lotus bulbs with their trunks. They’d wash them thoroughly until they were free of mud before chewing and swallowing them. That was good for their appearance and strength, and wouldn’t result in death or deadly pain. 

The\marginnote{3.5} young cubs, following the example of the great bull elephants, plunged into the lake and pulled up lotus bulbs with their trunks. But they didn’t wash them thoroughly, and while they were still muddy they chewed and swallowed them. That was not good for their appearance and strength, and resulted in death or deadly pain. 

In\marginnote{4.1} the same way, there are senior mendicants who robe up in the morning and, taking their bowl and robe, enter the town or village for alms. There they speak on the teachings, and lay people demonstrate their confidence in them. And when they get things, they use them untied, uninfatuated, unattached, seeing the drawbacks, and understanding the escape. That’s good for their appearance and strength, and doesn’t result in death or deadly pain. 

Junior\marginnote{4.6} mendicants, following the example of the senior mendicants, robe up in the morning and, taking their bowl and robe, enter the town or village for alms. There they speak on the teachings, and lay people demonstrate their confidence in them. But when they get things, they use them tied, infatuated, attached, blind to the drawbacks, not understanding the escape. That’s not good for their appearance and strength, and results in death or deadly pain. 

So\marginnote{4.11} you should train like this: ‘When we get things, we will use them untied, uninfatuated, unattached, seeing the drawbacks, and understanding the escape.’ That’s how you should train.” 

%
\section*{{\suttatitleacronym SN 20.10}{\suttatitletranslation A Cat }{\suttatitleroot Biḷārasutta}}
\addcontentsline{toc}{section}{\tocacronym{SN 20.10} \toctranslation{A Cat } \tocroot{Biḷārasutta}}
\markboth{A Cat }{Biḷārasutta}
\extramarks{SN 20.10}{SN 20.10}

At\marginnote{1.1} \textsanskrit{Sāvatthī}. 

Now\marginnote{1.2} at that time a certain mendicant socialized with families too often. 

The\marginnote{1.3} mendicants said to him, “Venerable, don’t socialize with families too often.” 

But\marginnote{1.5} that mendicant, when spoken to by the mendicants, did not stop. 

And\marginnote{1.6} then several mendicants went up to the Buddha, bowed, sat down to one side, and told him what had happened. The Buddha said: 

“Once\marginnote{2.1} upon a time, mendicants, a cat was standing by an alley or a drain or a dustbin hunting a little mouse, thinking, ‘When that little mouse comes out to feed, I’ll catch it right there and eat it!’ And then that little mouse came out to feed. The cat caught it and hastily swallowed it without chewing. But that little mouse ate its intestines and mesentery, resulting in death and deadly pain. 

In\marginnote{3.1} the same way, take a certain monk who robes up in the morning and, taking his bowl and robe, enters the village or town for alms without guarding body, speech, and mind, without establishing mindfulness, and without restraining the sense faculties. There he sees a female scantily clad, with revealing clothes. Lust infects his mind, resulting in death or deadly pain. 

For\marginnote{3.5} it is death in the training of the Noble One to reject the training and return to a lesser life. And it is deadly pain to commit one of the corrupt offenses for which resolution is possible. 

So\marginnote{3.8} you should train like this: ‘We will enter the village or town for alms guarding body, speech, and mind, establishing mindfulness, and restraining the sense faculties.’ That’s how you should train.” 

%
\section*{{\suttatitleacronym SN 20.11}{\suttatitletranslation A Jackal }{\suttatitleroot Siṅgālasutta}}
\addcontentsline{toc}{section}{\tocacronym{SN 20.11} \toctranslation{A Jackal } \tocroot{Siṅgālasutta}}
\markboth{A Jackal }{Siṅgālasutta}
\extramarks{SN 20.11}{SN 20.11}

At\marginnote{1.1} \textsanskrit{Sāvatthī}. 

“Mendicants,\marginnote{1.2} did you hear an old jackal howling at the crack of dawn?” 

“Yes,\marginnote{1.3} sir.” 

“That\marginnote{1.4} old jackal has the disease called mange. Yet it still goes where it wants, stands where it wants, sits where it wants, and lies down where it wants. And the cool breeze still blows on it. A certain person here who claims to follow the Sakyan would be lucky to experience even such an incarnation. 

So\marginnote{1.11} you should train like this: ‘We will stay diligent.’ That’s how you should train.” 

%
\section*{{\suttatitleacronym SN 20.12}{\suttatitletranslation A Jackal (2nd) }{\suttatitleroot Dutiyasiṅgālasutta}}
\addcontentsline{toc}{section}{\tocacronym{SN 20.12} \toctranslation{A Jackal (2nd) } \tocroot{Dutiyasiṅgālasutta}}
\markboth{A Jackal (2nd) }{Dutiyasiṅgālasutta}
\extramarks{SN 20.12}{SN 20.12}

At\marginnote{1.1} \textsanskrit{Sāvatthī}. 

“Mendicants,\marginnote{1.2} did you hear an old jackal howling at the crack of dawn?” 

“Yes,\marginnote{1.3} sir.” 

“There\marginnote{1.4} might be some gratitude and thankfulness in that old jackal, but there is none in a certain person here who claims to follow the Sakyan. 

So\marginnote{1.5} you should train like this: ‘We will be grateful and thankful. We won’t forget even a small thing done for us.’ That’s how you should train.” 

\scendkanda{The Linked Discourses with similes are complete. }

%
\addtocontents{toc}{\let\protect\contentsline\protect\nopagecontentsline}
\part*{Linked Discourses with Monks }
\addcontentsline{toc}{part}{Linked Discourses with Monks }
\markboth{}{}
\addtocontents{toc}{\let\protect\contentsline\protect\oldcontentsline}

%
\addtocontents{toc}{\let\protect\contentsline\protect\nopagecontentsline}
\chapter*{The Chapter on Monks }
\addcontentsline{toc}{chapter}{\tocchapterline{The Chapter on Monks }}
\addtocontents{toc}{\let\protect\contentsline\protect\oldcontentsline}

%
\section*{{\suttatitleacronym SN 21.1}{\suttatitletranslation With Kolita }{\suttatitleroot Kolitasutta}}
\addcontentsline{toc}{section}{\tocacronym{SN 21.1} \toctranslation{With Kolita } \tocroot{Kolitasutta}}
\markboth{With Kolita }{Kolitasutta}
\extramarks{SN 21.1}{SN 21.1}

\scevam{So\marginnote{1.1} I have heard. }At one time the Buddha was staying near \textsanskrit{Sāvatthī} in Jeta’s Grove, \textsanskrit{Anāthapiṇḍika}’s monastery. There Venerable \textsanskrit{Mahāmoggallāna} addressed the mendicants: “Reverends, mendicants!” 

“Reverend,”\marginnote{1.5} they replied. 

Venerable\marginnote{2.1} \textsanskrit{Mahāmoggallāna} said this: 

“Just\marginnote{2.2} now, reverends, as I was in private retreat this thought came to mind: ‘They speak of this thing called “noble silence”. What then is this noble silence?’ 

It\marginnote{2.5} occurred to me: ‘As the placing of the mind and keeping it connected are stilled, a mendicant enters and remains in the second absorption, which has the rapture and bliss born of immersion, with internal clarity and mind at one, without placing the mind and keeping it connected. This is called noble silence.’ 

And\marginnote{2.8} so, as the placing of the mind and keeping it connected were stilled, I was entering and remaining in the second absorption, which has the rapture and bliss born of immersion, with internal clarity and mind at one, without placing the mind and keeping it connected. 

While\marginnote{2.9} I was in that meditation, perception and focus accompanied by placing the mind beset me. 

Then\marginnote{3.1} the Buddha came up to me with his psychic power and said, ‘\textsanskrit{Moggallāna}, \textsanskrit{Moggallāna}! Don’t neglect noble silence, brahmin! Settle your mind in noble silence; unify your mind and bring it to immersion in noble silence.’ 

And\marginnote{3.3} so, after some time, as the placing of the mind and keeping it connected were stilled, I entered and remained in the second absorption … 

So\marginnote{3.4} if anyone should be rightly called a disciple who attained to great direct knowledge with help from the Teacher, it’s me.” 

%
\section*{{\suttatitleacronym SN 21.2}{\suttatitletranslation With Upatissa }{\suttatitleroot Upatissasutta}}
\addcontentsline{toc}{section}{\tocacronym{SN 21.2} \toctranslation{With Upatissa } \tocroot{Upatissasutta}}
\markboth{With Upatissa }{Upatissasutta}
\extramarks{SN 21.2}{SN 21.2}

At\marginnote{1.1} \textsanskrit{Sāvatthī}. 

There\marginnote{1.2} \textsanskrit{Sāriputta} addressed the mendicants: “Reverends, mendicants!” 

“Reverend,”\marginnote{1.4} they replied. \textsanskrit{Sāriputta} said this: 

“Just\marginnote{2.1} now, reverends, as I was in private retreat this thought came to mind: ‘Is there anything in the world whose decay and perishing would give rise to sorrow, lamentation, pain, sadness, and distress in me?’ It occurred to me: ‘There is nothing in the world whose decay and perishing would give rise to sorrow, lamentation, pain, sadness, and distress in me.’” 

When\marginnote{3.1} he said this, Venerable Ānanda said to him, “Even if the Teacher were to decay and perish? Wouldn’t that give rise to sorrow, lamentation, pain, sadness, and distress in you?” 

“Even\marginnote{3.3} if the Teacher were to decay and perish, that wouldn’t give rise to sorrow, lamentation, pain, sadness, and distress in me. Still, I would think: ‘Alas, the illustrious Teacher, so mighty and powerful, has vanished! If the Buddha was to remain for a long time, that would be for the welfare and happiness of the people, out of sympathy for the world, for the benefit, welfare, and happiness of gods and humans.’” 

“That\marginnote{3.6} must be because Venerable \textsanskrit{Sāriputta} has long ago totally eradicated I-making, mine-making, and the underlying tendency to conceit. So even if the Teacher were to decay and perish, it wouldn’t give rise to sorrow, lamentation, pain, sadness, and distress in him.” 

%
\section*{{\suttatitleacronym SN 21.3}{\suttatitletranslation A Mound of Salt }{\suttatitleroot Ghaṭasutta}}
\addcontentsline{toc}{section}{\tocacronym{SN 21.3} \toctranslation{A Mound of Salt } \tocroot{Ghaṭasutta}}
\markboth{A Mound of Salt }{Ghaṭasutta}
\extramarks{SN 21.3}{SN 21.3}

\scevam{So\marginnote{1.1} I have heard. }At one time the Buddha was staying near \textsanskrit{Sāvatthī} in Jeta’s Grove, \textsanskrit{Anāthapiṇḍika}’s monastery. 

At\marginnote{1.3} that time Venerables \textsanskrit{Sāriputta} and \textsanskrit{Moggallāna} were staying near \textsanskrit{Rājagaha}, in the Bamboo Grove, the squirrels’ feeding ground. Then in the late afternoon, Venerable \textsanskrit{Sāriputta} came out of retreat, went to Venerable \textsanskrit{Moggallāna}, and exchanged greetings with him. When the greetings and polite conversation were over, \textsanskrit{Sāriputta} sat down to one side, and said to \textsanskrit{Mahāmoggallāna}: 

“Reverend\marginnote{2.1} \textsanskrit{Moggallāna}, your faculties are so very clear, and your complexion is pure and bright. Have you spent the day in a peaceful meditation?” 

“Reverend,\marginnote{2.3} I’ve spent the day in a coarse meditation. But I have had some Dhamma talk.” 

“Who\marginnote{2.5} did you have a Dhamma talk with?” 

“With\marginnote{2.6} the Buddha.” 

“But\marginnote{2.7} Reverend, the Buddha is far away. He’s staying near \textsanskrit{Sāvatthī} in Jeta’s Grove, \textsanskrit{Anāthapiṇḍika}’s monastery. Did you go to him with your psychic power, or did he come to you?” 

“No\marginnote{2.10} reverend, I didn’t go to him with my psychic power, nor did he come to me. Rather, the Buddha cleared his clairvoyance and clairaudience towards me, and I cleared my clairvoyance and clairaudience towards him.” 

“But\marginnote{2.14} what manner of Dhamma talk did you have together?” 

“Well,\marginnote{3.1} reverend, I said to the Buddha, ‘Sir, they speak of one who is energetic. How is an energetic person defined?’ 

When\marginnote{3.4} I said this, the Buddha said, ‘\textsanskrit{Moggallāna}, it’s when a mendicant lives with energy roused up: “Gladly, let only skin, sinews, and bones remain! Let the flesh and blood waste away in my body! I will not stop trying until I have achieved what is possible by human strength, energy, and vigor.” That’s how a person is energetic.’ 

That’s\marginnote{3.8} the Dhamma talk I had together with the Buddha.” 

“Reverend,\marginnote{4.1} next to Venerable \textsanskrit{Mahāmoggallāna} I’m like a few pieces of gravel next to the Himalayas, the king of mountains. Venerable \textsanskrit{Mahāmoggallāna} is so mighty and powerful he could, if he wished, live on for the proper lifespan.” 

“Reverend,\marginnote{5.1} next to Venerable \textsanskrit{Sāriputta} I’m like a few grains of salt next to a mound of salt. Venerable \textsanskrit{Sāriputta} has been commended, complimented, and praised by the Buddha: 

\begin{verse}%
‘\textsanskrit{Sāriputta}\marginnote{6.1} is full of wisdom, \\
ethics, and peace. \\
Even a mendicant who has crossed over \\
might at best equal him.’” 

%
\end{verse}

And\marginnote{7.1} so these two spiritual giants agreed with each others’ fine words. 

%
\section*{{\suttatitleacronym SN 21.4}{\suttatitletranslation A Junior Mendicant }{\suttatitleroot Navasutta}}
\addcontentsline{toc}{section}{\tocacronym{SN 21.4} \toctranslation{A Junior Mendicant } \tocroot{Navasutta}}
\markboth{A Junior Mendicant }{Navasutta}
\extramarks{SN 21.4}{SN 21.4}

At\marginnote{1.1} \textsanskrit{Sāvatthī}. 

Now\marginnote{1.2} at that time a certain junior monk, after his meal, on his return from almsround, entered his dwelling, where he adhered to passivity and silence. And he didn’t help the mendicants out when it was time to sew robes. Then several mendicants went up to the Buddha, bowed, sat down to one side, and told him what had happened. 

So\marginnote{2.1} the Buddha addressed one of the monks, “Please, monk, in my name tell that monk that the Teacher summons him.” 

“Yes,\marginnote{2.3} sir,” that monk replied. He went to that monk and said to him, “Reverend, the teacher summons you.” 

“Yes,\marginnote{2.5} reverend,” that monk replied. He went to the Buddha, bowed, and sat down to one side. The Buddha said to him: 

“Is\marginnote{2.6} it really true, monk, that after your meal, on your return from almsround, you entered your dwelling, where you adhered to passivity and silence, and you didn’t help the mendicants out when it was time to sew robes?” 

“Sir,\marginnote{2.7} I am doing my own work.” 

Then\marginnote{3.1} the Buddha, knowing that monk’s train of thought, addressed the mendicants: “Mendicants, don’t complain about this monk. This monk gets the four absorptions—blissful meditations in this life that belong to the higher mind—when he wants, without trouble or difficulty. He has realized the supreme culmination of the spiritual path in this very life, and lives having achieved with his own insight the goal for which gentlemen rightly go forth from the lay life to homelessness.” 

That\marginnote{4.1} is what the Buddha said. Then the Holy One, the Teacher, went on to say: 

\begin{verse}%
“Not\marginnote{5.1} by being slack, \\
or with little strength \\
is extinguishment realized, \\
the freedom from all suffering. 

This\marginnote{6.1} young monk, \\
this best of men, \\
bears his final body, \\
having vanquished \textsanskrit{Māra} and his mount.” 

%
\end{verse}

%
\section*{{\suttatitleacronym SN 21.5}{\suttatitletranslation With Sujāta }{\suttatitleroot Sujātasutta}}
\addcontentsline{toc}{section}{\tocacronym{SN 21.5} \toctranslation{With Sujāta } \tocroot{Sujātasutta}}
\markboth{With Sujāta }{Sujātasutta}
\extramarks{SN 21.5}{SN 21.5}

At\marginnote{1.1} \textsanskrit{Sāvatthī}. 

Then\marginnote{1.2} Venerable \textsanskrit{Sujāta} went to see the Buddha. 

The\marginnote{1.3} Buddha saw him coming off in the distance, and addressed the mendicants: “This gentleman is beautiful in both ways. He’s attractive, good-looking, lovely, of surpassing beauty. And he has realized the supreme end of the spiritual path in this very life. He lives having achieved with his own insight the goal for which gentlemen rightly go forth from the lay life to homelessness.” 

That\marginnote{1.7} is what the Buddha said. Then the Holy One, the Teacher, went on to say: 

\begin{verse}%
“This\marginnote{2.1} mendicant is truly beautiful. \\
His heart is sincere, \\
he’s unfettered, detached, \\
quenched by not grasping. \\
He bears his final body, \\
having vanquished \textsanskrit{Māra} and his mount.” 

%
\end{verse}

%
\section*{{\suttatitleacronym SN 21.6}{\suttatitletranslation With Bhaddiya the Dwarf }{\suttatitleroot Lakuṇḍakabhaddiyasutta}}
\addcontentsline{toc}{section}{\tocacronym{SN 21.6} \toctranslation{With Bhaddiya the Dwarf } \tocroot{Lakuṇḍakabhaddiyasutta}}
\markboth{With Bhaddiya the Dwarf }{Lakuṇḍakabhaddiyasutta}
\extramarks{SN 21.6}{SN 21.6}

At\marginnote{1.1} \textsanskrit{Sāvatthī}. 

Then\marginnote{1.2} Venerable Bhaddiya the Dwarf went to see the Buddha. 

The\marginnote{1.3} Buddha saw him coming off in the distance, and addressed the mendicants: “Mendicants, do you see this monk coming—ugly, unsightly, deformed, and despised by the mendicants?” 

“Yes,\marginnote{1.6} sir.” 

“That\marginnote{1.7} mendicant is very mighty and powerful. It’s not easy to find an attainment that he has not already attained. And he has realized the supreme end of the spiritual path in this very life. He lives having achieved with his own insight the goal for which gentlemen rightly go forth from the lay life to homelessness.” 

That\marginnote{1.9} is what the Buddha said. Then the Holy One, the Teacher, went on to say: 

\begin{verse}%
“Geese,\marginnote{2.1} herons, and peacocks, \\
elephants and spotted deer—\\
though their bodies are not equal, \\
they all fear the lion. 

So\marginnote{3.1} it is for humans—\\
if a little person is wise, \\
they’re the truly great one, \\
not the fool with a good body.” 

%
\end{verse}

%
\section*{{\suttatitleacronym SN 21.7}{\suttatitletranslation With Visākha, Pañcāli’s Son }{\suttatitleroot Visākhasutta}}
\addcontentsline{toc}{section}{\tocacronym{SN 21.7} \toctranslation{With Visākha, Pañcāli’s Son } \tocroot{Visākhasutta}}
\markboth{With Visākha, Pañcāli’s Son }{Visākhasutta}
\extramarks{SN 21.7}{SN 21.7}

\scevam{So\marginnote{1.1} I have heard. }At one time the Buddha was staying near \textsanskrit{Vesālī}, at the Great Wood, in the hall with the peaked roof. 

Now\marginnote{1.3} at that time Venerable \textsanskrit{Visākha}, \textsanskrit{Pañcāli}’s son, was educating, encouraging, firing up, and inspiring the mendicants in the assembly hall with a Dhamma talk. His words were polished, clear, articulate, expressing the meaning, comprehensive, and independent. 

Then\marginnote{2.1} in the late afternoon, the Buddha came out of retreat and went to the assembly hall. He sat down on the seat spread out, and addressed the mendicants: “Mendicants, who was educating, encouraging, firing up, and inspiring the mendicants in the assembly hall with a Dhamma talk?” 

“Sir,\marginnote{2.4} it was Venerable \textsanskrit{Visākha}, \textsanskrit{Pañcāli}’s son.” 

Then\marginnote{3.1} the Buddha said to \textsanskrit{Visākha}: 

“Good,\marginnote{3.2} good, \textsanskrit{Visākha}! It’s good that you educate, encourage, fire up, and inspire the mendicants in the assembly hall with a Dhamma talk, with words that are polished, clear, articulate, expressing the meaning, comprehensive, and independent.” 

That\marginnote{4.1} is what the Buddha said. Then the Holy One, the Teacher, went on to say: 

\begin{verse}%
“Though\marginnote{5.1} an astute person is mixed up with fools, \\
they don’t know unless he speaks. \\
But when he speaks they know, \\
he’s teaching the state free of death. 

He\marginnote{6.1} should speak and illustrate the teaching, \\
holding up the banner of the seers. \\
Words well spoken are the seers’ banner, \\
for the teaching is the banner of the seers.” 

%
\end{verse}

%
\section*{{\suttatitleacronym SN 21.8}{\suttatitletranslation With Nanda }{\suttatitleroot Nandasutta}}
\addcontentsline{toc}{section}{\tocacronym{SN 21.8} \toctranslation{With Nanda } \tocroot{Nandasutta}}
\markboth{With Nanda }{Nandasutta}
\extramarks{SN 21.8}{SN 21.8}

At\marginnote{1.1} \textsanskrit{Sāvatthī}. 

Then\marginnote{1.2} Venerable Nanda—the Buddha’s cousin on his mother’s side—dressed in nicely pressed and ironed robes, applied eyeshadow, and took a polished black bowl. He went to the Buddha, bowed, and sat down to one side. The Buddha said to him: 

“Nanda,\marginnote{1.3} as a gentleman who has gone forth out of faith from the lay life to homelessness, it’s not appropriate for you to dress in nicely pressed and ironed robes, apply eyeshadow, and carry a polished black bowl. It’s appropriate for you to stay in the wilderness, eat only almsfood, wear rag robes, and live without concern for sensual pleasures.” 

That\marginnote{1.5} is what the Buddha said. Then the Holy One, the Teacher, went on to say: 

\begin{verse}%
“When\marginnote{2.1} will I see Nanda \\
in the wilderness, wearing rag robes, \\
feeding on scraps offered by strangers, \\
unconcerned for sensual pleasures?” 

%
\end{verse}

Then\marginnote{3.1} some time later Venerable Nanda stayed in the wilderness, ate only almsfood, wore rag robes, and lived without concern for sensual pleasures. 

%
\section*{{\suttatitleacronym SN 21.9}{\suttatitletranslation With Tissa }{\suttatitleroot Tissasutta}}
\addcontentsline{toc}{section}{\tocacronym{SN 21.9} \toctranslation{With Tissa } \tocroot{Tissasutta}}
\markboth{With Tissa }{Tissasutta}
\extramarks{SN 21.9}{SN 21.9}

At\marginnote{1.1} \textsanskrit{Sāvatthī}. 

Then\marginnote{1.2} Venerable Tissa—the Buddha’s cousin on his father’s side—went to the Buddha, bowed, and sat down to one side. He was miserable and sad, with tears flowing. Then the Buddha said to him: 

“Tissa,\marginnote{1.3} why are you sitting there so miserable and sad, with tears flowing?” 

“Sir,\marginnote{1.4} it’s because the mendicants beset me on all sides with sneering and jeering.” 

“That’s\marginnote{1.5} because you admonish others, but don’t accept admonition yourself. As a gentleman who has gone forth out of faith from the lay life to homelessness, it’s not appropriate for you to admonish others without accepting admonition yourself. It’s appropriate for you to admonish others and accept admonition yourself.” 

That\marginnote{2.1} is what the Buddha said. Then the Holy One, the Teacher, went on to say: 

\begin{verse}%
“Why\marginnote{3.1} are you angry? Don’t be angry! \\
It’s better to not be angry, Tissa. \\
For this spiritual life is lived \\
in order to remove anger, conceit, and denigration.” 

%
\end{verse}

%
\section*{{\suttatitleacronym SN 21.10}{\suttatitletranslation A Mendicant Named Senior }{\suttatitleroot Theranāmakasutta}}
\addcontentsline{toc}{section}{\tocacronym{SN 21.10} \toctranslation{A Mendicant Named Senior } \tocroot{Theranāmakasutta}}
\markboth{A Mendicant Named Senior }{Theranāmakasutta}
\extramarks{SN 21.10}{SN 21.10}

At\marginnote{1.1} one time the Buddha was staying near \textsanskrit{Rājagaha}, in the Bamboo Grove, the squirrels’ feeding ground. 

Now\marginnote{1.2} at that time there was a certain mendicant named Senior. He lived alone and praised living alone. He entered the village for alms alone, returned alone, sat in private alone, and focussed on walking mindfully alone. 

Then\marginnote{1.4} several mendicants went up to the Buddha, bowed, sat down to one side, and said to him, “Sir, there’s a certain mendicant named Senior who lives alone and praises living alone.” 

So\marginnote{2.1} the Buddha addressed one of the monks, “Please, monk, in my name tell the mendicant Senior that the teacher summons him.” 

“Yes,\marginnote{2.4} sir,” that monk replied. He went to Venerable Senior and said to him, “Reverend Senior, the teacher summons you.” 

“Yes,\marginnote{2.6} reverend,” that monk replied. He went to the Buddha, bowed, and sat down to one side. The Buddha said to him: 

“Is\marginnote{2.7} it really true, Senior, that you live alone and praise living alone?” 

“Yes,\marginnote{2.8} sir.” 

“But\marginnote{2.9} in what way do you live alone and praise living alone?” 

“Well,\marginnote{2.10} sir, I enter the village for alms alone, return alone, sit in private alone, and focus on walking mindfully alone. That’s how I live alone and praise living alone.” 

“That\marginnote{3.1} is a kind of living alone, I don’t deny it. But as to how living alone is fulfilled in detail, listen and apply your mind well, I will speak.” 

“Yes,\marginnote{3.4} sir,” he replied. 

“And\marginnote{3.5} how, Senior, is living alone fulfilled in detail? It’s when what’s in the past is given up, what’s in the future is relinquished, and desire and greed for present incarnations is eliminated. That’s how living alone is fulfilled in detail.” 

That\marginnote{4.1} is what the Buddha said. Then the Holy One, the Teacher, went on to say: 

\begin{verse}%
“The\marginnote{5.1} champion, all-knower, so very intelligent, \\
is unsullied in the midst of all things. \\
He’s given up all, freed in the ending of craving: \\
I declare that man to be one who lives alone.” 

%
\end{verse}

%
\section*{{\suttatitleacronym SN 21.11}{\suttatitletranslation With Mahākappina }{\suttatitleroot Mahākappinasutta}}
\addcontentsline{toc}{section}{\tocacronym{SN 21.11} \toctranslation{With Mahākappina } \tocroot{Mahākappinasutta}}
\markboth{With Mahākappina }{Mahākappinasutta}
\extramarks{SN 21.11}{SN 21.11}

At\marginnote{1.1} \textsanskrit{Sāvatthī}. 

Then\marginnote{1.2} Venerable \textsanskrit{Mahākappina} went to see the Buddha. 

The\marginnote{1.3} Buddha saw him coming off in the distance, and addressed the mendicants: “Mendicants, do you see that monk coming—white, thin, with a pointy nose?” 

“Yes,\marginnote{1.6} sir.” 

“That\marginnote{1.7} mendicant is very mighty and powerful. It’s not easy to find an attainment that he has not already attained. And he has realized the supreme end of the spiritual path in this very life. He lives having achieved with his own insight the goal for which gentlemen rightly go forth from the lay life to homelessness.” 

That\marginnote{2.1} is what the Buddha said. Then the Holy One, the Teacher, went on to say: 

\begin{verse}%
“The\marginnote{3.1} aristocrat is best among people \\
who take clan as the standard. \\
But one accomplished in knowledge and conduct \\
is best among gods and humans. 

The\marginnote{4.1} sun blazes by day, \\
the moon glows at night, \\
the aristocrat shines in armor, \\
and the brahmin shines in absorption. \\
But all day and all night, \\
the Buddha shines with glory.” 

%
\end{verse}

%
\section*{{\suttatitleacronym SN 21.12}{\suttatitletranslation Companions }{\suttatitleroot Sahāyakasutta}}
\addcontentsline{toc}{section}{\tocacronym{SN 21.12} \toctranslation{Companions } \tocroot{Sahāyakasutta}}
\markboth{Companions }{Sahāyakasutta}
\extramarks{SN 21.12}{SN 21.12}

At\marginnote{1.1} \textsanskrit{Sāvatthī}. 

Then\marginnote{1.2} two mendicants who were companions, protégés of Venerable \textsanskrit{Mahākappina}, went to see the Buddha. 

The\marginnote{1.3} Buddha saw them coming off in the distance, and addressed the mendicants: “Mendicants, do you see those monks coming who are companions, protégés of Venerable \textsanskrit{Mahākappina}?” 

“Yes,\marginnote{1.6} sir.” 

“Those\marginnote{1.7} mendicants are very mighty and powerful. It’s not easy to find an attainment that they have not already attained. And they’ve realized the supreme end of the spiritual path in this very life. They live having achieved with their own insight the goal for which gentlemen rightly go forth from the lay life to homelessness.” 

That\marginnote{2.1} is what the Buddha said. Then the Holy One, the Teacher, went on to say: 

\begin{verse}%
“These\marginnote{3.1} companion mendicants \\
have been together for a long time. \\
The true teaching has brought them together, \\
the teaching proclaimed by the Buddha. 

They’ve\marginnote{4.1} been well trained by Kappina \\
in the teaching proclaimed by the Noble One. \\
They bear their final body, \\
having vanquished \textsanskrit{Māra} and his mount.” 

%
\end{verse}

\scendkanda{The Linked Discourses on monks are complete. }

\scendbook{The Book of Causality is finished. }

%
\backmatter%
\chapter*{Colophon}
\addcontentsline{toc}{chapter}{Colophon}
\markboth{Colophon}{Colophon}

\section*{The Translator}

Bhikkhu Sujato was born as Anthony Aidan Best on 4/11/1966 in Perth, Western Australia. He grew up in the pleasant suburbs of Mt Lawley and Attadale alongside his sister Nicola, who was the good child. His mother, Margaret Lorraine Huntsman née Pinder, said “he’ll either be a priest or a poet”, while his father, Anthony Thomas Best, advised him to “never do anything for money”. He attended Aquinas College, a Catholic school, where he decided to become an atheist. At the University of WA he studied philosophy, aiming to learn what he wanted to do with his life. Finding that what he wanted to do was play guitar, he dropped out. His main band was named Martha’s Vineyard, which achieved modest success in the indie circuit. 

A seemingly random encounter with a roadside joey took him to Thailand, where he entered his first meditation retreat at Wat Ram Poeng, Chieng Mai in 1992. Feeling the call to the Buddha’s path, he took full ordination in Wat Pa Nanachat in 1994, where his teachers were Ajahn Pasanno and Ajahn Jayasaro. In 1997 he returned to Perth to study with Ajahn Brahm at Bodhinyana Monastery. 

He spent several years practicing in seclusion in Malaysia and Thailand before establishing Santi Forest Monastery in Bundanoon, NSW, in 2003. There he was instrumental in supporting the establishment of the Theravada bhikkhuni order in Australia and advocating for women’s rights. He continues to teach in Australia and globally, with a special concern for the moral implications of climate change and other forms of environmental destruction. He has published a series of books of original and groundbreaking research on early Buddhism. 

In 2005 he founded SuttaCentral together with Rod Bucknell and John Kelly. In 2015, seeing the need for a complete, accurate, plain English translation of the Pali texts, he undertook the task, spending nearly three years in isolation on the isle of Qi Mei off the coast of the nation of Taiwan. He completed the four main \textsanskrit{Nikāyas} in 2018, and the early books of the Khuddaka \textsanskrit{Nikāya} were complete by 2021. All this work is dedicated to the public domain and is entirely free of copyright encumbrance. 

In 2019 he returned to Sydney where he established Lokanta Vihara (The Monastery at the End of the World). 

\section*{Creation Process}

Primary source was the digital \textsanskrit{Mahāsaṅgīti} edition of the Pali \textsanskrit{Tipiṭaka}. Translated from the Pali, with reference to several English translations, especially those of Bhikkhu Bodhi.

\section*{The Translation}

This translation was part of a project to translate the four Pali \textsanskrit{Nikāyas} with the following aims: plain, approachable English; consistent terminology; accurate rendition of the Pali; free of copyright. It was made during 2016–2018 while Bhikkhu Sujato was staying in Qimei, Taiwan.

\section*{About SuttaCentral}

SuttaCentral publishes early Buddhist texts. Since 2005 we have provided root texts in Pali, Chinese, Sanskrit, Tibetan, and other languages, parallels between these texts, and translations in many modern languages. Building on the work of generations of scholars, we offer our contribution freely.

SuttaCentral is driven by volunteer contributions, and in addition we employ professional developers. We offer a sponsorship program for high quality translations from the original languages. Financial support for SuttaCentral is handled by the SuttaCentral Development Trust, a charitable trust registered in Australia.

\section*{About Bilara}

“Bilara” means “cat” in Pali, and it is the name of our Computer Assisted Translation (CAT) software. Bilara is a web app that enables translators to translate early Buddhist texts into their own language. These translations are published on SuttaCentral with the root text and translation side by side.

\section*{About SuttaCentral Editions}

The SuttaCentral Editions project makes high quality books from selected Bilara translations. These are published in formats including HTML, EPUB, PDF, and print.

You are welcome to print any of our Editions.

%
\end{document}