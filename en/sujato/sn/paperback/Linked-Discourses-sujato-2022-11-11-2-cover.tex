\documentclass[coverheight=9in,coverwidth=6in,spinewidth=0.7018918918918919in,pagecolor=backgroundcolor,bleedwidth=3mm,marklength=0in,12pt]{bookcover}%
%
%
\letnamebookcoverpart{bg spine bottom}{bg spine}[,,,7.6in]

\letnamebookcoverpart{bg spine top}{bg spine}[,,,-7.6in]

\newbookcovercomponenttype{center rotate}{
\vfill
\centering
\rotatebox[origin=c]{-90}{#1}
\vfill}

\usepackage{setspace}
\usepackage[12pt]{moresize}
\usepackage{microtype}
\usepackage{fontspec}
\usepackage[code=Code39]{makebarcode}
\usepackage{contour}
\usepackage{graphicx}
\usepackage{polyglossia}
\usepackage{tabulary}
\usepackage{varwidth}
\usepackage{array}
\setlength{\tabcolsep}{1em}

\setdefaultlanguage[]{english}
\setotherlanguage[script=Latin]{sanskrit}

\newcommand*{\z}[1]{\contourlength{2pt}\contour{backgroundcolor}{#1}}%

\newcommand*{\spinevolumenumber}[1]{\small\centering\lowercase{\color{secondarytextcolor}{\scshape{#1}}}}%

\newcommand*{\spinevolumeacronym}[1]{\footnotesize\color{secondarytextcolor}{#1}}%

\newcommand*{\spinetranslationtitle}[1]{\setstretch{.7}\centering\lowercase{\scshape\LARGE{#1}}}

\newcommand*{\spinecreatorname}[1]{\setstretch{.7}\centering\lowercase{\scshape\LARGE{#1}}}

\newcommand*{\fronttranslationtitle}[1]{\flushleft\setstretch{3.5}\Lightfont{\fontsize{60pt}{80pt}\selectfont{#1}}}%

\newcommand*{\fronttranslationsubtitle}[1]{\flushleft\normalfont\setstretch{2}{\Huge{\color{secondarytextcolor}{#1}}}\vspace{0em}}%

\newcommand*{\frontbyline}[1]{\flushright\setstretch{1}{\textit{\Large{\color{secondarytextcolor}{#1}}}}\vspace{-.75em}}%

\newcommand*{\frontcreatorname}[1]{\flushright\setstretch{1}{\Huge{#1}}}%

\newcommand*{\frontvolumenumber}[1]{\flushleft{\lowercase{\scshape\Large{\color{secondarytextcolor}{#1}}}}\vspace{-.2em}}%

\newcommand*{\frontvolumeacronym}[1]{\flushleft{\huge{#1}\vspace{0em}}}%

\newcommand*{\frontvolumetranslationtitle}[1]{\vspace{-.4em}\flushleft{\setstretch{.2}{\large{#1}}\vspace{-.4em}}}%

\newcommand*{\frontvolumeroottitle}[1]{\flushleft{\setstretch{.2}{\large\itshape{\color{secondarytextcolor}{#1}}}}}%

\newcommand*{\backpublicationblurb}[1]{{#1}}%

\newcommand*{\backvolumeblurb}[1]{\quad{#1}\bigskip}%

\newcommand*{\backpublisherblurb}[1]{\begin{center}\begin{minipage}[t][]{3in}{\begin{center}{\textit{\small{\color{secondarytextcolor}{#1}}}}\end{center}}\end{minipage}\end{center}}%

\newcommand*{\backpublishername}[1]{\Noligaturecaptionfont{\textsc{#1}}}%

\newcommand*{\backisbn}[1]{\fcolorbox{backgroundcolor}{backgroundcolor}{\barcode{#1}}}%


\setmainfont[Numbers=OldStyle]{Arno Pro}

\newfontfamily\Lightfont[]{Arno Pro Light Display}

\newfontfamily\Noligaturecaptionfont[Renderer=Basic]{Arno Pro Caption}


% use a small amount of tracking on small caps
\SetTracking[ spacing = {25*,166, } ]{ encoding = *, shape = sc }{ 50 }

% hang quotes (requires microtype)
\microtypesetup{
  protrusion = true,
  expansion  = true,
  tracking   = true,
  factor     = 1000,
  patch      = all,
  final
}

% Custom protrusion rules to allow hanging punctuation
\SetProtrusion
{ encoding = *}
{
% char   right left
  {-} = {    , 500 },
  % Double Quotes
  \textquotedblleft
      = {1000,     },
  \textquotedblright
      = {    , 1000},
  \quotedblbase
      = {1000,     },
  % Single Quotes
  \textquoteleft
      = {1000,     },
  \textquoteright
      = {    , 1000},
  \quotesinglbase
      = {1000,     }
}

%GRID%
%\usepackage{pagegrid}
%\pagegridsetup{top-left, step=.5in}
%GRID%

\setstretch{1.05}

% individual
% red colors

\providecolor{textcolor}{HTML}{FCF1ED}

\providecolor{secondarytextcolor}{HTML}{DCBFB5}

\providecolor{backgroundcolor}{HTML}{AF2C00}

\color{textcolor}

% images. position of cover image can be adjusted if necessary

\newcommand*{\backlogo}[1]{\raisebox{-.5ex}{\includegraphics[height=1em]{/app/sutta_publisher/images/sclogored.png}}}

\newcommand*{\spinelogo}[1]{\centering{\includegraphics[height=1.5em]{/app/sutta_publisher/images/sclogored.png}}}

\newcommand*{\coverimage}[1]{\vspace*{3in}\hspace*{2in}{\includegraphics[height=8in]{/app/sutta_publisher/images/red2.png}}}%
%
\begin{document}%
\pagestyle{empty}%
\normalsize%
\begin{bookcover}

% spine

\bookcovercomponent{center}{bg spine top}{
\begin{minipage}[t][][c]{.9\spinewidth}

\spinevolumenumber{Vol. 2}

\spinevolumeacronym{The Group of Linked Discourses Beginning With Causation}

\end{minipage}
}

\bookcovercomponent{center rotate}{spine}[0in,0in,0in,0in]{

\begin{tabulary}{7in}{V{4in}V{3in}}
\spinetranslationtitle{Linked Discourses} & \spinecreatorname{Bhikkhu Sujato}
\end{tabulary}

}

\bookcovercomponent{center}{bg spine bottom}{

\spinelogo{}
}

% front cover

\bookcovercomponent{normal}{bg front}{
{\coverimage{}}
}

\bookcovercomponent{normal}{front}[.95in,.92in,1in,.75in]{

\fronttranslationtitle{\z{Linked} \z{Discourses}}

\fronttranslationsubtitle{\z{A} \z{plain} \z{translation} \z{of} \z{the} \z{\textsanskrit{Saṁyutta}} \z{\textsanskrit{Nikāya}}}

\hfil{\color{secondarytextcolor}{\rule{4in}{1pt}}}\hfil

\frontbyline{\z{translated} \z{by}}
\frontcreatorname{\z{Bhikkhu} \z{Sujato}}

\vfill

\setstretch{1.05}

\frontvolumenumber{Volume \z{2}}
\frontvolumeacronym{\z{SN} \z{12–21}}
\frontvolumetranslationtitle{\z{The} \z{Group} \z{of} \z{Linked} \z{Discourses} \z{Beginning} \z{With} \z{Causation}}
\frontvolumeroottitle{\z{\textsanskrit{Nidānavaggasaṁyutta}}}
}

% Text on the back cover
\bookcovercomponent{normal}{back}[1in,.85in,1in,1in]{

\backpublicationblurb{The “Linked” or “Connected” Discourses (\textsanskrit{Saṁyutta} \textsanskrit{Nikāya}) is a collection of over a thousand short discourses organized by either a theme of Dhamma or the person who is speaking. It is the primary source work for fundamental themes such the five aggregates, dependent origination, the noble eightfold path, mindfulness meditation, and the four noble truths. The radical significance of these teachings is unpacked through systematic analysis leavened with parables, anecdotes, and similes.}

\backvolumeblurb{“The Group of Linked Discourses Beginning With Causation” is named after the first and longest section, which deals with the core Buddhist teaching of Dependent Origination. Dependent origination presents a series of conditional links laying bare how suffering originates and how it ends. Integrating psychological and existential aspects of suffering, it shows how when bound by attachment we make choices that bind us to transmigrating into future lives, thus explaining how rebirth takes place without having to invoke metaphysical concepts such as a “soul”.The remaining nine sections deal with miscellaneous secondary themes, some organized by subject, others by person.}

\backpublisherblurb{The wisdom of the Buddha has been preserved in a vast ocean of ancient texts. SuttaCentral offers fresh translations of these texts in the world’s languages. Setting aside the boundaries of language and tradition, we let the Buddha speak for himself.}

\vfill

\begin{minipage}[b][][b]{4in}

\begin{minipage}[c][][c]{2in}
\backlogo{} \backpublishername{SuttaCentral}
\end{minipage}\begin{minipage}[c][][c]{2in}
\backisbn{978-1-76132-088-0}
\end{minipage}

\end{minipage}

}

\end{bookcover}%
\end{document}