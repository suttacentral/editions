\documentclass[12pt,openany]{book}%
\usepackage{lastpage}%
%
\usepackage[inner=1in, outer=1in, top=.7in, bottom=1in, papersize={6in,9in}, headheight=13pt]{geometry}
\usepackage{polyglossia}
\usepackage[12pt]{moresize}
\usepackage{soul}%
\usepackage{microtype}
\usepackage{tocbasic}
\usepackage{realscripts}
\usepackage{epigraph}%
\usepackage{setspace}%
\usepackage{sectsty}
\usepackage{fontspec}
\usepackage{marginnote}
\usepackage[bottom]{footmisc}
\usepackage{enumitem}
\usepackage{fancyhdr}
\usepackage{extramarks}
\usepackage{graphicx}
\usepackage{verse}
\usepackage{relsize}
\usepackage{etoolbox}
\usepackage[a-3u]{pdfx}

\hypersetup{
colorlinks=true,
urlcolor=black,
linkcolor=black,
citecolor=black
}

% use a small amount of tracking on small caps
\SetTracking[ spacing = {25*,166, } ]{ encoding = *, shape = sc }{ 25 }

% add a blank page
\newcommand{\blankpage}{
\newpage
\thispagestyle{empty}
\mbox{}
\newpage
}

% define languages
\setdefaultlanguage[]{english}
\setotherlanguage[script=Latin]{sanskrit}

%\usepackage{pagegrid}
%\pagegridsetup{top-left, step=.25in}

% define fonts
% use if arno sanskrit is unavailable
%\setmainfont{Gentium Plus}
%\newfontfamily\Semiboldsubheadfont[]{Gentium Plus}
%\newfontfamily\Semiboldnormalfont[]{Gentium Plus}
%\newfontfamily\Lightfont[]{Gentium Plus}
%\newfontfamily\Marginalfont[]{Gentium Plus}
%\newfontfamily\Allsmallcapsfont[RawFeature=+c2sc]{Gentium Plus}
%\newfontfamily\Noligaturefont[Renderer=Basic]{Gentium Plus}
%\newfontfamily\Noligaturecaptionfont[Renderer=Basic]{Gentium Plus}
%\newfontfamily\Fleuronfont[Ornament=1]{Gentium Plus}

% use if arno sanskrit is available. display is applied to \chapter and \part, subhead to \section and \subsection. When specifying semibold, the italic must be defined.
\setmainfont[Numbers=OldStyle]{Arno Pro}
\newfontfamily\Semibolddisplayfont[BoldItalicFont = Arno Pro Semibold Italic Display]{Arno Pro Semibold Display} %
\newfontfamily\Semiboldsubheadfont[BoldItalicFont = Arno Pro Semibold Italic Subhead]{Arno Pro Semibold Subhead}
\newfontfamily\Semiboldnormalfont[BoldItalicFont = Arno Pro Semibold Italic]{Arno Pro Semibold}
\newfontfamily\Marginalfont[RawFeature=+subs]{Arno Pro Regular}
\newfontfamily\Allsmallcapsfont[RawFeature=+c2sc]{Arno Pro}
\newfontfamily\Noligaturefont[Renderer=Basic]{Arno Pro}
\newfontfamily\Noligaturecaptionfont[Renderer=Basic]{Arno Pro Caption}

% chinese fonts
\newfontfamily\cjk{Noto Serif TC}
\newcommand*{\langlzh}[1]{\cjk{#1}\normalfont}%

% logo
\newfontfamily\Logofont{sclogo.ttf}
\newcommand*{\sclogo}[1]{\large\Logofont{#1}}

% use subscript numerals for margin notes
\renewcommand*{\marginfont}{\Marginalfont}

% ensure margin notes have consistent vertical alignment
\renewcommand*{\marginnotevadjust}{-.17em}

% use compact lists
\setitemize{noitemsep,leftmargin=1em}
\setenumerate{noitemsep,leftmargin=1em}
\setdescription{noitemsep, style=unboxed, leftmargin=0em}

% style ToC
\DeclareTOCStyleEntries[
  raggedentrytext,
  linefill=\hfill,
  pagenumberwidth=.5in,
  pagenumberformat=\normalfont,
  entryformat=\normalfont
]{tocline}{chapter,section}


  \setlength\topsep{0pt}%
  \setlength\parskip{0pt}%

% define new \centerpars command for use in ToC. This ensures centering, proper wrapping, and no page break after
\def\startcenter{%
  \par
  \begingroup
  \leftskip=0pt plus 1fil
  \rightskip=\leftskip
  \parindent=0pt
  \parfillskip=0pt
}
\def\stopcenter{%
  \par
  \endgroup
}
\long\def\centerpars#1{\startcenter#1\stopcenter}

% redefine part, so that it adds a toc entry without page number
\let\oldcontentsline\contentsline
\newcommand{\nopagecontentsline}[3]{\oldcontentsline{#1}{#2}{}}

    \makeatletter
\renewcommand*\l@part[2]{%
  \ifnum \c@tocdepth >-2\relax
    \addpenalty{-\@highpenalty}%
    \addvspace{0em \@plus\p@}%
    \setlength\@tempdima{3em}%
    \begingroup
      \parindent \z@ \rightskip \@pnumwidth
      \parfillskip -\@pnumwidth
      {\leavevmode
       \setstretch{.85}\large\scshape\centerpars{#1}\vspace*{-1em}\llap{#2}}\par
       \nobreak
         \global\@nobreaktrue
         \everypar{\global\@nobreakfalse\everypar{}}%
    \endgroup
  \fi}
\makeatother

\makeatletter
\def\@pnumwidth{2em}
\makeatother

% define new sectioning command, which is only used in volumes where the pannasa is found in some parts but not others, especially in an and sn

\newcommand*{\pannasa}[1]{\clearpage\thispagestyle{empty}\begin{center}\vspace*{14em}\setstretch{.85}\huge\itshape\scshape\MakeLowercase{#1}\end{center}}

    \makeatletter
\newcommand*\l@pannasa[2]{%
  \ifnum \c@tocdepth >-2\relax
    \addpenalty{-\@highpenalty}%
    \addvspace{.5em \@plus\p@}%
    \setlength\@tempdima{3em}%
    \begingroup
      \parindent \z@ \rightskip \@pnumwidth
      \parfillskip -\@pnumwidth
      {\leavevmode
       \setstretch{.85}\large\itshape\scshape\lowercase{\centerpars{#1}}\vspace*{-1em}\llap{#2}}\par
       \nobreak
         \global\@nobreaktrue
         \everypar{\global\@nobreakfalse\everypar{}}%
    \endgroup
  \fi}
\makeatother

% don't put page number on first page of toc (relies on etoolbox)
\patchcmd{\chapter}{plain}{empty}{}{}

% global line height
\setstretch{1.05}

% allow linebreak after em-dash
\catcode`\—=13
\protected\def—{\unskip\textemdash\allowbreak}

% style headings with secsty. chapter and section are defined per-edition
\partfont{\setstretch{.85}\normalfont\centering\textsc}
\subsectionfont{\setstretch{.85}\Semiboldsubheadfont}%
\subsubsectionfont{\setstretch{.85}\Semiboldnormalfont}

% style elements of suttatitle
\newcommand*{\suttatitleacronym}[1]{\smaller[2]{#1}\vspace*{.3em}}
\newcommand*{\suttatitletranslation}[1]{\linebreak{#1}}
\newcommand*{\suttatitleroot}[1]{\linebreak\smaller[2]\itshape{#1}}

\DeclareTOCStyleEntries[
  indent=3.3em,
  dynindent,
  beforeskip=.2em plus -2pt minus -1pt,
]{tocline}{section}

\DeclareTOCStyleEntries[
  indent=0em,
  dynindent,
  beforeskip=.4em plus -2pt minus -1pt,
]{tocline}{chapter}

\newcommand*{\tocacronym}[1]{\hspace*{-3.3em}{#1}\quad}
\newcommand*{\toctranslation}[1]{#1}
\newcommand*{\tocroot}[1]{(\textit{#1})}
\newcommand*{\tocchapterline}[1]{\bfseries\itshape{#1}}


% redefine paragraph and subparagraph headings to not be inline
\makeatletter
% Change the style of paragraph headings %
\renewcommand\paragraph{\@startsection{paragraph}{4}{\z@}%
            {-2.5ex\@plus -1ex \@minus -.25ex}%
            {1.25ex \@plus .25ex}%
            {\noindent\Semiboldnormalfont\normalsize}}

% Change the style of subparagraph headings %
\renewcommand\subparagraph{\@startsection{subparagraph}{5}{\z@}%
            {-2.5ex\@plus -1ex \@minus -.25ex}%
            {1.25ex \@plus .25ex}%
            {\noindent\Semiboldnormalfont\small}}
\makeatother

% use etoolbox to suppress page numbers on \part
\patchcmd{\part}{\thispagestyle{plain}}{\thispagestyle{empty}}
  {}{\errmessage{Cannot patch \string\part}}

% and to reduce margins on quotation
\patchcmd{\quotation}{\rightmargin}{\leftmargin 1.2em \rightmargin}{}{}
\AtBeginEnvironment{quotation}{\small}

% titlepage
\newcommand*{\titlepageTranslationTitle}[1]{{\begin{center}\begin{large}{#1}\end{large}\end{center}}}
\newcommand*{\titlepageCreatorName}[1]{{\begin{center}\begin{normalsize}{#1}\end{normalsize}\end{center}}}

% halftitlepage
\newcommand*{\halftitlepageTranslationTitle}[1]{\setstretch{2.5}{\begin{Huge}\uppercase{\so{#1}}\end{Huge}}}
\newcommand*{\halftitlepageTranslationSubtitle}[1]{\setstretch{1.2}{\begin{large}{#1}\end{large}}}
\newcommand*{\halftitlepageFleuron}[1]{{\begin{large}\Fleuronfont{{#1}}\end{large}}}
\newcommand*{\halftitlepageByline}[1]{{\begin{normalsize}\textit{{#1}}\end{normalsize}}}
\newcommand*{\halftitlepageCreatorName}[1]{{\begin{LARGE}{\textsc{#1}}\end{LARGE}}}
\newcommand*{\halftitlepageVolumeNumber}[1]{{\begin{normalsize}{\Allsmallcapsfont{\textsc{#1}}}\end{normalsize}}}
\newcommand*{\halftitlepageVolumeAcronym}[1]{{\begin{normalsize}{#1}\end{normalsize}}}
\newcommand*{\halftitlepageVolumeTranslationTitle}[1]{{\begin{Large}{\textsc{#1}}\end{Large}}}
\newcommand*{\halftitlepageVolumeRootTitle}[1]{{\begin{normalsize}{\Allsmallcapsfont{\textsc{\itshape #1}}}\end{normalsize}}}
\newcommand*{\halftitlepagePublisher}[1]{{\begin{large}{\Noligaturecaptionfont\textsc{#1}}\end{large}}}

% epigraph
\renewcommand{\epigraphflush}{center}
\renewcommand*{\epigraphwidth}{.85\textwidth}
\newcommand*{\epigraphTranslatedTitle}[1]{\vspace*{.5em}\footnotesize\textsc{#1}\\}%
\newcommand*{\epigraphRootTitle}[1]{\footnotesize\textit{#1}\\}%
\newcommand*{\epigraphReference}[1]{\footnotesize{#1}}%

% custom commands for html styling classes
\newcommand*{\scnamo}[1]{\begin{center}\textit{#1}\end{center}}
\newcommand*{\scendsection}[1]{\begin{center}\textit{#1}\end{center}}
\newcommand*{\scendsutta}[1]{\begin{center}\textit{#1}\end{center}}
\newcommand*{\scendbook}[1]{\begin{center}\uppercase{#1}\end{center}}
\newcommand*{\scendkanda}[1]{\begin{center}\textbf{#1}\end{center}}
\newcommand*{\scend}[1]{\begin{center}\textit{#1}\end{center}}
\newcommand*{\scuddanaintro}[1]{\textit{#1}}
\newcommand*{\scendvagga}[1]{\begin{center}\textbf{#1}\end{center}}
\newcommand*{\scrule}[1]{\textbf{#1}}
\newcommand*{\scadd}[1]{\textit{#1}}
\newcommand*{\scevam}[1]{\textsc{#1}}
\newcommand*{\scspeaker}[1]{\hspace{2em}\textit{#1}}
\newcommand*{\scbyline}[1]{\begin{flushright}\textit{#1}\end{flushright}\bigskip}

% custom command for thematic break = hr
\newcommand*{\thematicbreak}{\begin{center}\rule[.5ex]{6em}{.4pt}\begin{normalsize}\quad\Fleuronfont{•}\quad\end{normalsize}\rule[.5ex]{6em}{.4pt}\end{center}}

% manage and style page header and footer. "fancy" has header and footer, "plain" has footer only

\pagestyle{fancy}
\fancyhf{}
\fancyfoot[RE,LO]{\thepage}
\fancyfoot[LE,RO]{\footnotesize\lastleftxmark}
\fancyhead[CE]{\setstretch{.85}\Noligaturefont\MakeLowercase{\textsc{\firstrightmark}}}
\fancyhead[CO]{\setstretch{.85}\Noligaturefont\MakeLowercase{\textsc{\firstleftmark}}}
\renewcommand{\headrulewidth}{0pt}
\fancypagestyle{plain}{ %
\fancyhf{} % remove everything
\fancyfoot[RE,LO]{\thepage}
\fancyfoot[LE,RO]{\footnotesize\lastleftxmark}
\renewcommand{\headrulewidth}{0pt}
\renewcommand{\footrulewidth}{0pt}}

% style footnotes
\setlength{\skip\footins}{1em}

\makeatletter
\newcommand{\@makefntextcustom}[1]{%
    \parindent 0em%
    \thefootnote.\enskip #1%
}
\renewcommand{\@makefntext}[1]{\@makefntextcustom{#1}}
\makeatother

% hang quotes (requires microtype)
\microtypesetup{
  protrusion = true,
  expansion  = true,
  tracking   = true,
  factor     = 1000,
  patch      = all,
  final
}

% Custom protrusion rules to allow hanging punctuation
\SetProtrusion
{ encoding = *}
{
% char   right left
  {-} = {    , 500 },
  % Double Quotes
  \textquotedblleft
      = {1000,     },
  \textquotedblright
      = {    , 1000},
  \quotedblbase
      = {1000,     },
  % Single Quotes
  \textquoteleft
      = {1000,     },
  \textquoteright
      = {    , 1000},
  \quotesinglbase
      = {1000,     }
}

% make latex use actual font em for parindent, not Computer Modern Roman
\AtBeginDocument{\setlength{\parindent}{1em}}%
%

% Default values; a bit sloppier than normal
\tolerance 1414
\hbadness 1414
\emergencystretch 1.5em
\hfuzz 0.3pt
\clubpenalty = 10000
\widowpenalty = 10000
\displaywidowpenalty = 10000
\hfuzz \vfuzz
 \raggedbottom%

\title{Linked Discourses}
\author{Bhikkhu Sujato}
\date{}%
% define a different fleuron for each edition
\newfontfamily\Fleuronfont[Ornament=40]{Arno Pro}

% Define heading styles per edition for chapter and section. Suttatitle can be either of these, depending on the volume. 

\let\oldfrontmatter\frontmatter
\renewcommand{\frontmatter}{%
\chapterfont{\setstretch{.85}\normalfont\centering}%
\sectionfont{\setstretch{.85}\Semiboldsubheadfont}%
\oldfrontmatter}

\let\oldmainmatter\mainmatter
\renewcommand{\mainmatter}{%
\chapterfont{\setstretch{.85}\normalfont\centering}%
\sectionfont{\setstretch{.85}\normalfont\centering}%
\oldmainmatter}

\let\oldbackmatter\backmatter
\renewcommand{\backmatter}{%
\chapterfont{\setstretch{.85}\normalfont\centering}%
\sectionfont{\setstretch{.85}\Semiboldsubheadfont}%
\oldbackmatter}
%
%
\begin{document}%
\normalsize%
\frontmatter%
\setlength{\parindent}{0cm}

\pagestyle{empty}

\maketitle

\blankpage%
\begin{center}

\vspace*{2.2em}

\halftitlepageTranslationTitle{Linked Discourses}

\vspace*{1em}

\halftitlepageTranslationSubtitle{A plain translation of the Saṁyutta Nikāya}

\vspace*{2em}

\halftitlepageFleuron{•}

\vspace*{2em}

\halftitlepageByline{translated and introduced by}

\vspace*{.5em}

\halftitlepageCreatorName{Bhikkhu Sujato}

\vspace*{4em}

\halftitlepageVolumeNumber{Volume 3}

\smallskip

\halftitlepageVolumeAcronym{SN 22–34}

\smallskip

\halftitlepageVolumeTranslationTitle{The Group of Linked Discourses Beginning With the Aggregates}

\smallskip

\halftitlepageVolumeRootTitle{Khandhavaggasaṁyutta}

\vspace*{\fill}

\sclogo{0}
 \halftitlepagePublisher{SuttaCentral}

\end{center}

\newpage
%
\setstretch{1.05}

\begin{footnotesize}

\textit{Linked Discourses} is a translation of the Saṁyuttanikāya by Bhikkhu Sujato.

\medskip

Creative Commons Zero (CC0)

To the extent possible under law, Bhikkhu Sujato has waived all copyright and related or neighboring rights to \textit{Linked Discourses}.

\medskip

This work is published from Australia.

\begin{center}
\textit{This translation is an expression of an ancient spiritual text that has been passed down by the Buddhist tradition for the benefit of all sentient beings. It is dedicated to the public domain via Creative Commons Zero (CC0). You are encouraged to copy, reproduce, adapt, alter, or otherwise make use of this translation. The translator respectfully requests that any use be in accordance with the values and principles of the Buddhist community.}
\end{center}

\medskip

\begin{description}
    \item[Web publication date] 2018
    \item[This edition] 2022-11-17 08:29:33
    \item[Publication type] paperback
    \item[Edition] ed5
    \item[Number of volumes] 5
    \item[Publication ISBN] 978-1-76132-086-6
    \item[Publication URL] https://suttacentral.net/editions/sn/en/sujato
    \item[Source URL] https://github.com/suttacentral/bilara-data/tree/published/translation/en/sujato/sutta/sn
    \item[Publication number] scpub4
\end{description}

\medskip

Published by SuttaCentral

\medskip

\textit{SuttaCentral,\\
c/o Alwis \& Alwis Pty Ltd\\
Kaurna Country,\\
Suite 12,\\
198 Greenhill Road,\\
Eastwood,\\
SA 5063,\\
Australia}

\end{footnotesize}

\newpage

\setlength{\parindent}{1.5em}%%
\tableofcontents
\newpage
\pagestyle{fancy}
%
\mainmatter%
\pagestyle{fancy}%
%
%
\addtocontents{toc}{\let\protect\contentsline\protect\nopagecontentsline}
\part*{Linked Discourses on the Aggregates }
\addcontentsline{toc}{part}{Linked Discourses on the Aggregates }
\markboth{}{}
\addtocontents{toc}{\let\protect\contentsline\protect\oldcontentsline}

%
\addtocontents{toc}{\let\protect\contentsline\protect\nopagecontentsline}
\pannasa{The First Fifty }
\addcontentsline{toc}{pannasa}{The First Fifty }
\markboth{}{}
\addtocontents{toc}{\let\protect\contentsline\protect\oldcontentsline}

%
\addtocontents{toc}{\let\protect\contentsline\protect\nopagecontentsline}
\chapter*{The Chapter on Nakula’s Father }
\addcontentsline{toc}{chapter}{\tocchapterline{The Chapter on Nakula’s Father }}
\addtocontents{toc}{\let\protect\contentsline\protect\oldcontentsline}

%
\section*{{\suttatitleacronym SN 22.1}{\suttatitletranslation Nakula’s Father }{\suttatitleroot Nakulapitusutta}}
\addcontentsline{toc}{section}{\tocacronym{SN 22.1} \toctranslation{Nakula’s Father } \tocroot{Nakulapitusutta}}
\markboth{Nakula’s Father }{Nakulapitusutta}
\extramarks{SN 22.1}{SN 22.1}

\scevam{So\marginnote{1.1} I have heard. }At one time the Buddha was staying in the land of the Bhaggas on Crocodile Hill, in the deer park at \textsanskrit{Bhesakaḷā}’s Wood. 

Then\marginnote{1.3} the householder Nakula’s father went up to the Buddha, bowed, sat down to one side, and said to the Buddha: 

“Sir,\marginnote{2.1} I’m an old man, elderly and senior. I’m advanced in years and have reached the final stage of life. My body is ailing and I’m constantly unwell. I hardly ever get to see the esteemed mendicants. May the Buddha please advise me and instruct me. It will be for my lasting welfare and happiness.” 

“That’s\marginnote{3.1} so true, householder! That’s so true, householder! For this body is ailing, trapped in its shell. If anyone dragging around this body claimed to be healthy even for a minute, what is that but foolishness? 

So\marginnote{3.4} you should train like this: ‘Though my body is ailing, my mind will be healthy.’ That’s how you should train.” 

And\marginnote{4.1} then the householder Nakula’s father approved and agreed with what the Buddha said. He got up from his seat, bowed, and respectfully circled the Buddha, keeping him on his right. Then he went up to Venerable \textsanskrit{Sāriputta}, bowed, and sat down to one side. \textsanskrit{Sāriputta} said to him: 

“Householder,\marginnote{4.2} your faculties are so very clear, and your complexion is pure and bright. Did you get to hear a Dhamma talk in the Buddha’s presence today?” 

“What\marginnote{5.1} else, sir, could it possibly be?\footnote{This idiom is unusual, if not unique, and a translation should try to convery something of the dramatic quality of it. The force of it seems to be, “What else could possibly have had this effect on me?”, which is not well captured by BB’s “Why not”, and better by Thanissaro’s “How could it be otherwise”. } Just now the Buddha anointed me with the deathless ambrosia of a Dhamma talk.”\footnote{Reinforcing the unique and dramatic quality of the previous line, amatena abhisitto is another unique usage. It is one of the few, if not only, places in the EBTs where the Rg Vedic sense of “ambrosia” for amata is prominent. } 

“But\marginnote{5.3} what kind of ambrosial Dhamma talk has the Buddha anointed you with?” 

So\marginnote{5.4} Nakula’s father told \textsanskrit{Sāriputta} all that had happened, and said, “That’s the ambrosial Dhamma talk that the Buddha anointed me with.” 

“But\marginnote{7.1} didn’t you feel the need to ask the Buddha the further question: ‘Sir, how do you define someone ailing in body and ailing in mind, and someone ailing in body and healthy in mind’?” 

“Sir,\marginnote{7.3} we would travel a long way to learn the meaning of this statement in the presence of Venerable \textsanskrit{Sāriputta}. May Venerable \textsanskrit{Sāriputta} himself please clarify the meaning of this.” 

“Well\marginnote{8.1} then, householder, listen and pay close attention, I will speak.” 

“Yes,\marginnote{8.2} sir,” replied Nakula’s father. \textsanskrit{Sāriputta} said this: 

“And\marginnote{9.1} how is a person ailing in body and ailing in mind? It’s when an unlearned ordinary person has not seen the noble ones, and is neither skilled nor trained in the qualities of a noble one. They’ve not seen good persons, and are neither skilled nor trained in the qualities of a good person. They regard form as self, self as having form, form in self, or self in form. They’re obsessed with the thought: ‘I am form, form is mine!’ But that form of theirs decays and perishes, which gives rise to sorrow, lamentation, pain, sadness, and distress. 

They\marginnote{10.1} regard feeling as self, self as having feeling, feeling in self, or self in feeling. They’re obsessed with the thought: ‘I am feeling, feeling is mine!’ But that feeling of theirs decays and perishes, which gives rise to sorrow, lamentation, pain, sadness, and distress. 

They\marginnote{11.1} regard perception as self, self as having perception, perception in self, or self in perception. They’re obsessed with the thought: ‘I am perception, perception is mine!’ But that perception of theirs decays and perishes, which gives rise to sorrow, lamentation, pain, sadness, and distress. 

They\marginnote{12.1} regard choices as self, self as having choices, choices in self, or self in choices. They’re obsessed with the thought: ‘I am choices, choices are mine!’ But those choices of theirs decay and perish, which gives rise to sorrow, lamentation, pain, sadness, and distress. 

They\marginnote{13.1} regard consciousness as self, self as having consciousness, consciousness in self, or self in consciousness. They’re obsessed with the thought: ‘I am consciousness, consciousness is mine!’ But that consciousness of theirs decays and perishes, which gives rise to sorrow, lamentation, pain, sadness, and distress. 

That’s\marginnote{13.5} how a person is ailing in body and ailing in mind. 

And\marginnote{14.1} how is a person ailing in body and healthy in mind? It’s when a learned noble disciple has seen the noble ones, and is skilled and trained in the teaching of the noble ones. They’ve seen good persons, and are skilled and trained in the teaching of the good persons. They don’t regard form as self, self as having form, form in self, or self in form. They’re not obsessed with the thought: ‘I am form, form is mine!’ So when that form of theirs decays and perishes, it doesn’t give rise to sorrow, lamentation, pain, sadness, and distress. 

They\marginnote{15.1} don’t regard feeling as self, self as having feeling, feeling in self, or self in feeling. They’re not obsessed with the thought: ‘I am feeling, feeling is mine!’ So when that feeling of theirs decays and perishes, it doesn’t give rise to sorrow, lamentation, pain, sadness, and distress. 

They\marginnote{16.1} don’t regard perception as self, self as having perception, perception in self, or self in perception. They’re not obsessed with the thought: ‘I am perception, perception is mine!’ So when that perception of theirs decays and perishes, it doesn’t give rise to sorrow, lamentation, pain, sadness, and distress. 

They\marginnote{17.1} don’t regard choices as self, self as having choices, choices in self, or self in choices. They’re not obsessed with the thought: ‘I am choices, choices are mine!’ So when those choices of theirs decay and perish, it doesn’t give rise to sorrow, lamentation, pain, sadness, and distress. 

They\marginnote{18.1} don’t regard consciousness as self, self as having consciousness, consciousness in self, or self in consciousness. They’re not obsessed with the thought: ‘I am consciousness, consciousness is mine!’ So when that consciousness of theirs decays and perishes, it doesn’t give rise to sorrow, lamentation, pain, sadness, and distress. 

That’s\marginnote{18.5} how a person is ailing in body and healthy in mind.” 

That’s\marginnote{19.1} what Venerable \textsanskrit{Sāriputta} said. Satisfied, Nakula’s father was happy with what \textsanskrit{Sāriputta} said. 

%
\section*{{\suttatitleacronym SN 22.2}{\suttatitletranslation At Devadaha }{\suttatitleroot Devadahasutta}}
\addcontentsline{toc}{section}{\tocacronym{SN 22.2} \toctranslation{At Devadaha } \tocroot{Devadahasutta}}
\markboth{At Devadaha }{Devadahasutta}
\extramarks{SN 22.2}{SN 22.2}

\scevam{So\marginnote{1.1} I have heard. }At one time the Buddha was staying in the land of the Sakyans, where they have a town named Devadaha. 

Then\marginnote{1.3} several mendicants who were heading for the west went up to the Buddha, bowed, sat down to one side, and said to him, “Sir, we wish to go to a western land to take up residence there.” 

“But\marginnote{2.1} mendicants, have you consulted with \textsanskrit{Sāriputta}?” 

“No,\marginnote{2.2} sir, we haven’t.” 

“You\marginnote{2.3} should consult with \textsanskrit{Sāriputta}. He’s astute, and supports his spiritual companions, the mendicants.” 

“Yes,\marginnote{2.5} sir,” they replied. 

Now\marginnote{3.1} at that time Venerable \textsanskrit{Sāriputta} was meditating not far from the Buddha in a clump of golden shower trees.\footnote{The identification of elagala with cassia tora, found in BB, PTS dict, and ultimately probably derived from MW, is unsatisfactory. It is a weed, growing 50cm or so high, hardly a fitting place for meditation, and completely contradicting the commentary. The alternative, Cassia alata or Senna alata is better, since it grows 3-4 M tall, with a glorious golden flower. However it’s native to Mexico. The Chinese parallels don’t seem to offer any assistance. SA 108 has 坐一堅固樹下, where 堅固 might stand for \textsanskrit{gāḍha}, and thus the compound \textsanskrit{ekagāḍha}. But this gets us no further in understanding the sanskrit. Moreoever 堅固 may also stand for \textsanskrit{sāla} or \textsanskrit{sāla}-\textsanskrit{vṛkṣa}, so it could just mean standing under “a” sal tree, where 一 stands for annatara. Sticking with the Cassia family, I’ve gone with cassia fistula, which would at lest be a nice place to meditate! } And then those mendicants approved and agreed with what the Buddha said. They got up from their seat, bowed, and respectfully circled the Buddha, keeping him on their right. Then they went up to Venerable \textsanskrit{Sāriputta}, and exchanged greetings with him. 

When\marginnote{3.3} the greetings and polite conversation were over, they sat down to one side and said to him, “Reverend \textsanskrit{Sāriputta}, we wish to go to a western land to take up residence there. We have consulted with the Teacher.” 

“Reverends,\marginnote{4.1} there are those who question a mendicant who has gone abroad—astute aristocrats, brahmins, householders, and ascetics—for astute people are inquisitive: ‘But what does the venerables’ Teacher teach? What does he explain?’ I trust the venerables have properly heard, learned, attended, and remembered the teachings, and penetrated them with wisdom. That way, when answering you will repeat what the Buddha has said and not misrepresent him with an untruth. You will explain in line with the teaching, with no legitimate grounds for rebuke and criticism.” 

“Reverend,\marginnote{5.1} we would travel a long way to learn the meaning of this statement in the presence of Venerable \textsanskrit{Sāriputta}. May Venerable \textsanskrit{Sāriputta} himself please clarify the meaning of this.” 

“Well\marginnote{5.3} then, reverends, listen and pay close attention, I will speak.” 

“Yes,\marginnote{5.4} reverend,” they replied. \textsanskrit{Sāriputta} said this: 

“Reverends,\marginnote{6.1} there are those who question a mendicant who has gone abroad—astute aristocrats, brahmins, householders, and ascetics—for astute people are inquisitive: ‘But what does the venerables’ Teacher teach? What does he explain?’ When questioned like this, reverends, you should answer: ‘Reverend, our Teacher explained the removal of desire and lust.’ 

When\marginnote{7.1} you answer like this, such astute people may inquire further: ‘But regarding what does the venerables’ teacher explain the removal of desire and lust?’ When questioned like this, reverends, you should answer: ‘Our teacher explains the removal of desire and lust for form, feeling, perception, choices, and consciousness.’ 

When\marginnote{8.1} you answer like this, such astute people may inquire further: ‘But what drawback has he seen that he teaches the removal of desire and lust for form, feeling, perception, choices, and consciousness?’ When questioned like this, reverends, you should answer: ‘If you’re not free of greed, desire, fondness, thirst, passion, and craving for form, when that form decays and perishes it gives rise to sorrow, lamentation, pain, sadness, and distress. If you’re not free of greed, desire, fondness, thirst, passion, and craving for feeling … perception … choices … consciousness, when that consciousness decays and perishes it gives rise to sorrow, lamentation, pain, sadness, and distress. This is the drawback our Teacher has seen that he teaches the removal of desire and lust for form, feeling, perception, choices, and consciousness.’ 

When\marginnote{9.1} you answer like this, such astute people may inquire further: ‘But what benefit has he seen that he teaches the removal of desire and lust for form, feeling, perception, choices, and consciousness?’ When questioned like this, reverends, you should answer: ‘If you are rid of greed, desire, fondness, thirst, passion, and craving for form, when that form decays and perishes it doesn’t give rise to sorrow, lamentation, pain, sadness, and distress. If you are rid of greed, desire, fondness, thirst, passion, and craving for feeling … perception … choices … consciousness, when that consciousness decays and perishes it doesn’t give rise to sorrow, lamentation, pain, sadness, and distress. This is the benefit our Teacher has seen that he teaches the removal of desire and lust for form, feeling, perception, choices, and consciousness.’ 

If\marginnote{10.1} those who acquired and kept unskillful qualities were to live happily in the present life, free of anguish, distress, and fever; and if, when their body breaks up, after death, they could expect to go to a good place, the Buddha would not praise giving up unskillful qualities. But since those who acquire and keep unskillful qualities live unhappily in the present life, full of anguish, distress, and fever; and since, when their body breaks up, after death, they can expect to go to a bad place, the Buddha praises giving up unskillful qualities. 

If\marginnote{11.1} those who embraced and kept skillful qualities were to live unhappily in the present life, full of anguish, distress, and fever; and if, when their body breaks up, after death, they could expect to go to a bad place, the Buddha would not praise embracing skillful qualities. But since those who embrace and keep skillful qualities live happily in the present life, free of anguish, distress, and fever; and since, when their body breaks up, after death, they can expect to go to a good place, the Buddha praises embracing skillful qualities.” 

This\marginnote{12.1} is what Venerable \textsanskrit{Sāriputta} said. Satisfied, the mendicants were happy with what \textsanskrit{Sāriputta} said. 

%
\section*{{\suttatitleacronym SN 22.3}{\suttatitletranslation With Hāliddikāni }{\suttatitleroot Hāliddikānisutta}}
\addcontentsline{toc}{section}{\tocacronym{SN 22.3} \toctranslation{With Hāliddikāni } \tocroot{Hāliddikānisutta}}
\markboth{With Hāliddikāni }{Hāliddikānisutta}
\extramarks{SN 22.3}{SN 22.3}

\scevam{So\marginnote{1.1} I have heard. }At one time Venerable \textsanskrit{Mahākaccāna} was staying in the land of the Avantis near Kuraraghara on Steep Mountain. 

Then\marginnote{1.3} the householder \textsanskrit{Hāliddikāni} went up to Venerable \textsanskrit{Mahākaccāna}, bowed, sat down to one side, and said to him, “Sir, this was said by the Buddha in the Chapter of the Eights, in ‘The Questions of \textsanskrit{Māgandiya}’: 

\begin{verse}%
‘After\marginnote{2.1} leaving shelter to migrate unsettled,\footnote{The line and it’s metaphorical interpretation work on two levels. The oka and niketa are the same basic meaning, but I have translated in line with the underlying etymology: oka from “place of comfort, shelter”, niketa as “heaping, settlement”. Sari means “wander, roam, drift”. Here it has a literal meaning, a homeless wanderer. But later one is said to be a “okasari” “house-wanderer”. How does this make sense? The root -sari is the same as the root for samsara, and so an okasari is one who wanders in samsara. Compare the standard simile for the dibbacakkhu, where a person walks out of one home down the street, and into another. Cf the modern idiom “couch-surfer”. I use the same term as i use for samsara to bring out this connection. } \\
a sage doesn’t get close to anyone in town. \\
Rid of sensual pleasures, expecting nothing, \\
they wouldn’t get in arguments with people.’ 

%
\end{verse}

How\marginnote{3.1} should we see the detailed meaning of the Buddha’s brief statement?” 

“Householder,\marginnote{4.1} the form element is a shelter for consciousness. One whose consciousness is shackled to greed for the form element is called a migrant going from shelter to shelter. The feeling element is a shelter for consciousness. One whose consciousness is attached to greed for the feeling element is called a migrant going from shelter to shelter. The perception element is a shelter for consciousness. One whose consciousness is attached to greed for the perception element is called a migrant going from shelter to shelter. The choices element is a shelter for consciousness. One whose consciousness is attached to greed for the choices element is called a migrant going from shelter to shelter. That’s how one is a migrant going from shelter to shelter. 

And\marginnote{5.1} how is one a migrant with no shelter? The Realized One has given up any desire, greed, relishing, and craving for the form element; any attraction, grasping, mental fixation, insistence, and underlying tendencies. He has cut it off at the root, made it like a palm stump, obliterated it, so it’s unable to arise in the future. That’s why the Realized One is called a migrant with no shelter. The Realized One has given up any desire, greed, relishing, and craving for the feeling element … the perception element … the choices element … the consciousness element; any attraction, grasping, mental fixation, insistence, and underlying tendencies. He has cut it off at the root, made it like a palm stump, obliterated it, so it’s unable to arise in the future. That’s why the Realized One is called a migrant with no shelter. That’s how one is a migrant with no shelter. 

And\marginnote{6.1} how is one a migrant going from settlement to settlement? Being attached to migrating from settlement to settlement in pursuit of sights, one is called a migrant going from settlement to settlement. Being attached to migrating from settlement to settlement in pursuit of sounds … smells … tastes … touches … thoughts, one is called a migrant going from settlement to settlement. That’s how one is a migrant going from settlement to settlement. 

And\marginnote{7.1} how is one an unsettled migrant? The Realized One has given up attachment to migrating from settlement to settlement in pursuit of sights. He has cut it off at the root, made it like a palm stump, obliterated it, so it’s unable to arise in the future. That’s why the Realized One is called an unsettled migrant. The Realized One has given up attachment to migrating from settlement to settlement in pursuit of sounds … smells … tastes … touches … thoughts. He has cut it off at the root, made it like a palm stump, obliterated it, so it’s unable to arise in the future. That’s why the Realized One is called an unsettled migrant. That’s how one is an unsettled migrant. 

And\marginnote{8.1} how does one get close to people in town? It’s when someone mixes closely with laypeople, sharing their joys and sorrows—happy when they’re happy and sad when they’re sad—and getting involved in their business. That’s how one gets close to people in town. 

And\marginnote{9.1} how does one not get close to people in town? It’s when a mendicant doesn’t mix closely with laypeople, not sharing their joys and sorrows—not happy when they’re happy or sad when they’re sad—and not getting involved in their business. That’s how one doesn’t get close to people in town. 

And\marginnote{10.1} how is one not rid of sensual pleasures? It’s when someone isn’t rid of greed, desire, fondness, thirst, passion, and craving for sensual pleasures. That’s how one is not rid of sensual pleasures. 

And\marginnote{11.1} how is one rid of sensual pleasures? It’s when someone is rid of greed, desire, fondness, thirst, passion, and craving for sensual pleasures. That’s how one is rid of sensual pleasures. 

And\marginnote{12.1} how does one have expectations? It’s when someone thinks: ‘In the future, may I be of such form, such feeling, such perception, such choices, and such consciousness!’ That’s how one has expectations. 

And\marginnote{13.1} how does one expect nothing? It’s when someone doesn’t think: ‘In the future, may I be of such form, such feeling, such perception, such choices, and such consciousness!’ That’s how one expects nothing. 

And\marginnote{14.1} how does one argue with people? It’s when someone takes part in this sort of discussion: ‘You don’t understand this teaching and training. I understand this teaching and training. What, you understand this teaching and training? You’re practicing wrong. I’m practicing right. You said last what you should have said first. You said first what you should have said last. I stay on topic, you don’t. What you’ve thought so much about has been disproved. Your doctrine is refuted. Go on, save your doctrine! You’re trapped; get yourself out of this—if you can!’\footnote{Sahita as a quality of dhamma talks is found at AN 4.139 and AN 8.16. The basic meaning is “unified, together”. It is the same as the Vedic “samhita”. To me, “consistent” is more of a logical property, whereas what I think this means is “to the point, on topic.” Also note that PTS dict for this says reading avicinna is to be preferred. Under vicinna it says adhicinna is preferred. Handy. } That’s how one argues with people. 

And\marginnote{15.1} how does one not argue with people? It’s when a mendicant doesn’t take part in this sort of discussion: ‘You don’t understand this teaching and training … get yourself out of this—if you can!’ That’s how one doesn’t argue with people. 

So,\marginnote{16.1} householder, that’s how to understand the detailed meaning of what the Buddha said in brief in the Chapter of the Eights, in ‘The Questions of \textsanskrit{Māgandiya}’: 

\begin{verse}%
‘After\marginnote{17.1} leaving shelter to migrate unsettled, \\
a sage doesn’t get close to anyone in town. \\
Rid of sensual pleasures, expecting nothing, \\
they wouldn’t get in arguments with people.’” 

%
\end{verse}

%
\section*{{\suttatitleacronym SN 22.4}{\suttatitletranslation Hāliddikāni (2nd) }{\suttatitleroot Dutiyahāliddikānisutta}}
\addcontentsline{toc}{section}{\tocacronym{SN 22.4} \toctranslation{Hāliddikāni (2nd) } \tocroot{Dutiyahāliddikānisutta}}
\markboth{Hāliddikāni (2nd) }{Dutiyahāliddikānisutta}
\extramarks{SN 22.4}{SN 22.4}

\scevam{So\marginnote{1.1} I have heard. }At one time Venerable \textsanskrit{Mahākaccāna} was staying in the land of the Avantis near Kuraraghara on Steep Mountain. 

Then\marginnote{1.3} the householder \textsanskrit{Hāliddikāni} went up to Venerable \textsanskrit{Mahākaccāna} … and asked him, “Sir, this was said by the Buddha in ‘The Questions of Sakka’: ‘Those ascetics and brahmins who are freed due to the ending of craving have reached the ultimate goal, the ultimate sanctuary, the ultimate spiritual life, the ultimate end, and are best among gods and humans.’ 

How\marginnote{2.1} should we see the detailed meaning of the Buddha’s brief statement?” 

“Householder,\marginnote{3.1} consider any desire, greed, relishing, and craving for the form element; any attraction, grasping, mental fixation, insistence, and underlying tendencies. With the ending, fading away, cessation, giving away, and letting go of that, the mind is said to be ‘well freed’. 

Consider\marginnote{4.1} any desire, greed, relishing, and craving for the feeling element … the perception element … the choices element … the consciousness element; any attraction, grasping, mental fixation, insistence, and underlying tendencies. With the ending, fading away, cessation, giving away, and letting go of that, the mind is said to be ‘well freed’. 

So,\marginnote{5.1} householder, that’s how to understand the detailed meaning of what the Buddha said in brief in ‘The Questions of Sakka’: ‘Those ascetics and brahmins who are freed due to the ending of craving have reached the ultimate goal, the ultimate sanctuary, the ultimate spiritual life, the ultimate end, and are best among gods and humans.’” 

%
\section*{{\suttatitleacronym SN 22.5}{\suttatitletranslation Development of Immersion }{\suttatitleroot Samādhisutta}}
\addcontentsline{toc}{section}{\tocacronym{SN 22.5} \toctranslation{Development of Immersion } \tocroot{Samādhisutta}}
\markboth{Development of Immersion }{Samādhisutta}
\extramarks{SN 22.5}{SN 22.5}

\scevam{So\marginnote{1.1} I have heard. }At \textsanskrit{Sāvatthī}. 

“Mendicants,\marginnote{1.4} develop immersion. A mendicant who has immersion truly understands. What do they truly understand? The origin and ending of form, feeling, perception, choices, and consciousness. 

And\marginnote{2.1} what is the origin of form, feeling, perception, choices, and consciousness? It’s when a mendicant approves, welcomes, and keeps clinging. 

What\marginnote{3.1} do they approve, welcome, and keep clinging to? They approve, welcome, and keep clinging to form. This gives rise to relishing. Relishing forms is grasping. Their grasping is a condition for continued existence. Continued existence is a condition for rebirth. Rebirth is a condition for old age and death, sorrow, lamentation, pain, sadness, and distress to come to be. That is how this entire mass of suffering originates. 

They\marginnote{4.1} approve, welcome, and keep clinging to feeling … perception … choices … consciousness. This gives rise to relishing. Relishing consciousness is grasping. Their grasping is a condition for continued existence. Continued existence is a condition for rebirth. Rebirth is a condition that gives rise to old age and death, sorrow, lamentation, pain, sadness, and distress. That is how this entire mass of suffering originates. 

This\marginnote{5.1} is the origin of form, feeling, perception, choices, and consciousness. 

And\marginnote{6.1} what is the ending of form, feeling, perception, choices, and consciousness? 

It’s\marginnote{7.1} when a mendicant doesn’t approve, welcome, or keep clinging. 

What\marginnote{8.1} don’t they approve, welcome, or keep clinging to? They don’t approve, welcome, or keep clinging to form. As a result, relishing of form ceases. When that relishing ceases, grasping ceases. When grasping ceases, continued existence ceases. … That is how this entire mass of suffering ceases. 

They\marginnote{9.1} don’t approve, welcome, or keep clinging to feeling … perception … choices … consciousness. As a result, relishing of consciousness ceases. When that relishing ceases, grasping ceases. … That is how this entire mass of suffering ceases. 

This\marginnote{12.1} is the ending of form, feeling, perception, choices, and consciousness.” 

%
\section*{{\suttatitleacronym SN 22.6}{\suttatitletranslation Retreat }{\suttatitleroot Paṭisallāṇasutta}}
\addcontentsline{toc}{section}{\tocacronym{SN 22.6} \toctranslation{Retreat } \tocroot{Paṭisallāṇasutta}}
\markboth{Retreat }{Paṭisallāṇasutta}
\extramarks{SN 22.6}{SN 22.6}

At\marginnote{1.1} \textsanskrit{Sāvatthī}. 

“Mendicants,\marginnote{1.2} meditate in retreat. A mendicant in retreat truly understands. What do they truly understand? The origin and ending of form, feeling, perception, choices, and consciousness. …” 

(Expand\marginnote{1.6} in detail as in the previous discourse.) 

%
\section*{{\suttatitleacronym SN 22.7}{\suttatitletranslation Anxiety Because of Grasping }{\suttatitleroot Upādāparitassanāsutta}}
\addcontentsline{toc}{section}{\tocacronym{SN 22.7} \toctranslation{Anxiety Because of Grasping } \tocroot{Upādāparitassanāsutta}}
\markboth{Anxiety Because of Grasping }{Upādāparitassanāsutta}
\extramarks{SN 22.7}{SN 22.7}

At\marginnote{1.1} \textsanskrit{Sāvatthī}. 

“Mendicants,\marginnote{1.2} I will teach you how grasping leads to anxiety, and how not grasping leads to freedom from anxiety. Listen and pay close attention, I will speak.” 

“Yes,\marginnote{1.4} sir,” they replied. The Buddha said this: 

“And\marginnote{2.1} how does grasping lead to anxiety? It’s when an unlearned ordinary person has not seen the noble ones, and is neither skilled nor trained in the teaching of the noble ones. They’ve not seen good persons, and are neither skilled nor trained in the teaching of the good persons. They regard form as self, self as having form, form in self, or self in form. But that form of theirs decays and perishes, and consciousness latches on to the perishing of form. Anxieties occupy their mind, born of latching on to the perishing of form, and originating in accordance with natural principles.\footnote{See https://discourse.suttacentral.net/t/parallels-and-translation-of-sn-22-7-help/2926 } So they become frightened, worried, concerned, and anxious because of grasping. 

They\marginnote{3.1} regard feeling as self … 

They\marginnote{4.1} regard perception as self … 

They\marginnote{4.2} regard choices as self … 

They\marginnote{5.1} regard consciousness as self, self as having consciousness, consciousness in self, or self in consciousness. But that consciousness of theirs decays and perishes, and consciousness latches on to the perishing of consciousness. Anxieties occupy their mind, born of latching on to the perishing of consciousness, and originating in accordance with natural principles. So they become frightened, worried, concerned, and anxious because of grasping. That’s how grasping leads to anxiety. 

And\marginnote{6.1} how does not grasping lead to freedom from anxiety? It’s when a learned noble disciple has seen the noble ones, and is skilled and trained in the teaching of the noble ones. They’ve seen good persons, and are skilled and trained in the teaching of the good persons. They don’t regard form as self, self as having form, form in self, or self in form. When that form of theirs decays and perishes, consciousness doesn’t latch on to the perishing of form. Anxieties—born of latching on to the perishing of form and originating in accordance with natural principles—don’t occupy their mind. So they don’t become frightened, worried, concerned, or anxious because of grasping. 

They\marginnote{7.1} don’t regard feeling as self … 

They\marginnote{8.1} don’t regard perception as self … 

They\marginnote{8.2} don’t regard choices as self … 

They\marginnote{9.1} don’t regard consciousness as self … When that consciousness of theirs decays and perishes, consciousness doesn’t latch on to the perishing of consciousness. Anxieties—born of latching on to the perishing of consciousness and originating in accordance with natural principles—don’t occupy their mind. So they don’t become frightened, worried, concerned, or anxious because of grasping. That’s how not grasping leads to freedom from anxiety.” 

%
\section*{{\suttatitleacronym SN 22.8}{\suttatitletranslation Anxiety Because of Grasping (2nd) }{\suttatitleroot Dutiyaupādāparitassanāsutta}}
\addcontentsline{toc}{section}{\tocacronym{SN 22.8} \toctranslation{Anxiety Because of Grasping (2nd) } \tocroot{Dutiyaupādāparitassanāsutta}}
\markboth{Anxiety Because of Grasping (2nd) }{Dutiyaupādāparitassanāsutta}
\extramarks{SN 22.8}{SN 22.8}

At\marginnote{1.1} \textsanskrit{Sāvatthī}. 

“Mendicants,\marginnote{1.2} I will teach you how grasping leads to anxiety, and how not grasping leads to freedom from anxiety. Listen and pay close attention, I will speak. And how does grasping lead to anxiety? It’s when an unlearned ordinary person regards form like this: ‘This is mine, I am this, this is my self.’ But that form of theirs decays and perishes, which gives rise to sorrow, lamentation, pain, sadness, and distress. They regard feeling … perception … choices … consciousness like this: ‘This is mine, I am this, this is my self.’ But that consciousness of theirs decays and perishes, which gives rise to sorrow, lamentation, pain, sadness, and distress. That’s how grasping leads to anxiety. 

And\marginnote{2.1} how does not grasping lead to freedom from anxiety? It’s when a learned noble disciple regards form like this: ‘This is not mine, I am not this, this is not my self.’ When that form of theirs decays and perishes, it doesn’t give rise to sorrow, lamentation, pain, sadness, and distress. They regard feeling … perception … choices … consciousness like this: ‘This is not mine, I am not this, this is not my self.’ When that consciousness of theirs decays and perishes, it doesn’t give rise to sorrow, lamentation, pain, sadness, and distress. That’s how not grasping leads to freedom from anxiety.” 

%
\section*{{\suttatitleacronym SN 22.9}{\suttatitletranslation Impermanence in the Three Times }{\suttatitleroot Kālattayaaniccasutta}}
\addcontentsline{toc}{section}{\tocacronym{SN 22.9} \toctranslation{Impermanence in the Three Times } \tocroot{Kālattayaaniccasutta}}
\markboth{Impermanence in the Three Times }{Kālattayaaniccasutta}
\extramarks{SN 22.9}{SN 22.9}

At\marginnote{1.1} \textsanskrit{Sāvatthī}. 

“Mendicants,\marginnote{1.2} form of the past and future is impermanent, let alone the present. 

Seeing\marginnote{1.4} this, a learned noble disciple doesn’t worry about past form, doesn’t look forward to enjoying future form,\footnote{Note that in similar contexts, anapekkha and abhinandati are regularly oriented towards the past and future respectively. Nanda and it variants straddle the meanings of “joy “and” crave”, in fact are standard synonyms of tanha, etc. Here it means “has an desire to experience the joys of …” for which “look forward to” is the closest English idiom. } and they practice for disillusionment, dispassion, and cessation regarding present form. 

Feeling\marginnote{1.7} … 

Perception\marginnote{1.8} … 

Choices\marginnote{1.9} … 

Consciousness\marginnote{1.11} of the past and future is impermanent, let alone the present. 

Seeing\marginnote{1.16} this, a learned noble disciple doesn’t worry about past consciousness, doesn’t look forward to enjoying future consciousness, and they practice for disillusionment, dispassion, and cessation regarding present consciousness.” 

%
\section*{{\suttatitleacronym SN 22.10}{\suttatitletranslation Suffering in the Three Times }{\suttatitleroot Kālattayadukkhasutta}}
\addcontentsline{toc}{section}{\tocacronym{SN 22.10} \toctranslation{Suffering in the Three Times } \tocroot{Kālattayadukkhasutta}}
\markboth{Suffering in the Three Times }{Kālattayadukkhasutta}
\extramarks{SN 22.10}{SN 22.10}

At\marginnote{1.1} \textsanskrit{Sāvatthī}. 

“Mendicants,\marginnote{1.2} form of the past and future is suffering, let alone the present. 

Seeing\marginnote{1.4} this, a learned noble disciple doesn’t worry about past form, doesn’t look forward to enjoying future form, and they practice for disillusionment, dispassion, and cessation regarding present form. 

Feeling\marginnote{1.7} … 

Perception\marginnote{1.8} … 

Choices\marginnote{1.9} … 

Consciousness\marginnote{1.10} of the past and future is suffering, let alone the present. 

Seeing\marginnote{1.12} this, a learned noble disciple doesn’t worry about past consciousness, doesn’t look forward to enjoying future consciousness, and they practice for disillusionment, dispassion, and cessation regarding present consciousness.” 

%
\section*{{\suttatitleacronym SN 22.11}{\suttatitletranslation Not-Self in the Three Times }{\suttatitleroot Kālattayaanattasutta}}
\addcontentsline{toc}{section}{\tocacronym{SN 22.11} \toctranslation{Not-Self in the Three Times } \tocroot{Kālattayaanattasutta}}
\markboth{Not-Self in the Three Times }{Kālattayaanattasutta}
\extramarks{SN 22.11}{SN 22.11}

At\marginnote{1.1} \textsanskrit{Sāvatthī}. 

“Mendicants,\marginnote{1.2} form of the past and future is not-self, let alone the present. 

Seeing\marginnote{1.4} this, a learned noble disciple doesn’t worry about past form, doesn’t look forward to enjoying future form, and they practice for disillusionment, dispassion, and cessation regarding present form. 

Feeling\marginnote{1.7} … 

Perception\marginnote{1.8} … 

Choices\marginnote{1.9} … 

Consciousness\marginnote{1.10} of the past and future is not-self, let alone the present. 

Seeing\marginnote{1.12} this, a learned noble disciple doesn’t worry about past consciousness, doesn’t look forward to enjoying future consciousness, and they practice for the disillusionment, dispassion, and cessation regarding present consciousness.” 

%
\addtocontents{toc}{\let\protect\contentsline\protect\nopagecontentsline}
\chapter*{The Chapter on Impermanence }
\addcontentsline{toc}{chapter}{\tocchapterline{The Chapter on Impermanence }}
\addtocontents{toc}{\let\protect\contentsline\protect\oldcontentsline}

%
\section*{{\suttatitleacronym SN 22.12}{\suttatitletranslation Impermanence }{\suttatitleroot Aniccasutta}}
\addcontentsline{toc}{section}{\tocacronym{SN 22.12} \toctranslation{Impermanence } \tocroot{Aniccasutta}}
\markboth{Impermanence }{Aniccasutta}
\extramarks{SN 22.12}{SN 22.12}

\scevam{So\marginnote{1.1} I have heard. }At \textsanskrit{Sāvatthī}. 

“Mendicants,\marginnote{1.4} form, feeling, perception, choices, and consciousness are impermanent. 

Seeing\marginnote{1.5} this, a learned noble disciple grows disillusioned with form, feeling, perception, choices, and consciousness. Being disillusioned, desire fades away. When desire fades away they’re freed. When they’re freed, they know they’re freed. 

They\marginnote{1.7} understand: ‘Rebirth is ended, the spiritual journey has been completed, what had to be done has been done, there is no return to any state of existence.’” 

%
\section*{{\suttatitleacronym SN 22.13}{\suttatitletranslation Suffering }{\suttatitleroot Dukkhasutta}}
\addcontentsline{toc}{section}{\tocacronym{SN 22.13} \toctranslation{Suffering } \tocroot{Dukkhasutta}}
\markboth{Suffering }{Dukkhasutta}
\extramarks{SN 22.13}{SN 22.13}

At\marginnote{1.1} \textsanskrit{Sāvatthī}. 

“Mendicants,\marginnote{1.2} form, feeling, perception, choices, and consciousness are suffering. 

Seeing\marginnote{1.3} this … They understand: ‘… there is no return to any state of existence.’” 

%
\section*{{\suttatitleacronym SN 22.14}{\suttatitletranslation Not-Self }{\suttatitleroot Anattasutta}}
\addcontentsline{toc}{section}{\tocacronym{SN 22.14} \toctranslation{Not-Self } \tocroot{Anattasutta}}
\markboth{Not-Self }{Anattasutta}
\extramarks{SN 22.14}{SN 22.14}

At\marginnote{1.1} \textsanskrit{Sāvatthī}. 

“Mendicants,\marginnote{1.2} form, feeling, perception, choices, and consciousness are not-self. 

Seeing\marginnote{1.3} this, a learned noble disciple grows disillusioned with form, feeling, perception, choices, and consciousness. Being disillusioned, desire fades away. When desire fades away they’re freed. When they’re freed, they know they’re freed. 

They\marginnote{1.5} understand: ‘Rebirth is ended, the spiritual journey has been completed, what had to be done has been done, there is no return to any state of existence.’” 

%
\section*{{\suttatitleacronym SN 22.15}{\suttatitletranslation That Which is Impermanent }{\suttatitleroot Yadaniccasutta}}
\addcontentsline{toc}{section}{\tocacronym{SN 22.15} \toctranslation{That Which is Impermanent } \tocroot{Yadaniccasutta}}
\markboth{That Which is Impermanent }{Yadaniccasutta}
\extramarks{SN 22.15}{SN 22.15}

At\marginnote{1.1} \textsanskrit{Sāvatthī}. 

“Mendicants,\marginnote{1.2} form is impermanent. What’s impermanent is suffering. What’s suffering is not-self. And what’s not-self should be truly seen with right understanding like this: ‘This is not mine, I am not this, this is not my self.’ 

Feeling\marginnote{1.6} is impermanent … 

Perception\marginnote{1.10} is impermanent … 

Choices\marginnote{1.11} are impermanent … 

Consciousness\marginnote{1.12} is impermanent. What’s impermanent is suffering. What’s suffering is not-self. And what’s not-self should be truly seen with right understanding like this: ‘This is not mine, I am not this, this is not my self.’ 

Seeing\marginnote{1.16} this … They understand: ‘… there is no return to any state of existence.’” 

%
\section*{{\suttatitleacronym SN 22.16}{\suttatitletranslation That Which is Suffering }{\suttatitleroot Yaṁdukkhasutta}}
\addcontentsline{toc}{section}{\tocacronym{SN 22.16} \toctranslation{That Which is Suffering } \tocroot{Yaṁdukkhasutta}}
\markboth{That Which is Suffering }{Yaṁdukkhasutta}
\extramarks{SN 22.16}{SN 22.16}

At\marginnote{1.1} \textsanskrit{Sāvatthī}. 

“Mendicants,\marginnote{1.2} form is suffering. What’s suffering is not-self. And what’s not-self should be truly seen with right understanding like this: ‘This is not mine, I am not this, this is not my self.’ 

Feeling\marginnote{1.5} is suffering … 

Perception\marginnote{1.6} is suffering … 

Choices\marginnote{1.7} are suffering … 

Consciousness\marginnote{1.8} is suffering. What’s suffering is not-self. And what’s not-self should be truly seen with right understanding like this: ‘This is not mine, I am not this, this is not my self.’ 

Seeing\marginnote{1.11} this … They understand: ‘… there is no return to any state of existence.’” 

%
\section*{{\suttatitleacronym SN 22.17}{\suttatitletranslation That Which is Not-Self }{\suttatitleroot Yadanattāsutta}}
\addcontentsline{toc}{section}{\tocacronym{SN 22.17} \toctranslation{That Which is Not-Self } \tocroot{Yadanattāsutta}}
\markboth{That Which is Not-Self }{Yadanattāsutta}
\extramarks{SN 22.17}{SN 22.17}

At\marginnote{1.1} \textsanskrit{Sāvatthī}. 

“Mendicants,\marginnote{1.2} form is not-self. And what’s not-self should be truly seen with right understanding like this: ‘This is not mine, I am not this, this is not my self.’ 

Feeling\marginnote{1.4} is not-self … 

Perception\marginnote{1.5} is not-self … 

Choices\marginnote{1.6} are not-self … 

Consciousness\marginnote{1.7} is not-self. And what’s not-self should be truly seen with right understanding like this: ‘This is not mine, I am not this, this is not my self.’ 

Seeing\marginnote{1.9} this … They understand: ‘… there is no return to any state of existence.’” 

%
\section*{{\suttatitleacronym SN 22.18}{\suttatitletranslation Impermanence With Its Cause }{\suttatitleroot Sahetuaniccasutta}}
\addcontentsline{toc}{section}{\tocacronym{SN 22.18} \toctranslation{Impermanence With Its Cause } \tocroot{Sahetuaniccasutta}}
\markboth{Impermanence With Its Cause }{Sahetuaniccasutta}
\extramarks{SN 22.18}{SN 22.18}

At\marginnote{1.1} \textsanskrit{Sāvatthī}. 

“Mendicants,\marginnote{1.2} form is impermanent. The cause and reason that gives rise to form is also impermanent. Since form is produced by what is impermanent, how could it be permanent? 

Feeling\marginnote{1.5} is impermanent … 

Perception\marginnote{1.8} is impermanent … 

Choices\marginnote{1.9} are impermanent … 

Consciousness\marginnote{1.12} is impermanent. The cause and reason that gives rise to consciousness is also impermanent. Since consciousness is produced by what is impermanent, how could it be permanent? 

Seeing\marginnote{1.15} this … They understand: ‘… there is no return to any state of existence.’” 

%
\section*{{\suttatitleacronym SN 22.19}{\suttatitletranslation Suffering With Its Cause }{\suttatitleroot Sahetudukkhasutta}}
\addcontentsline{toc}{section}{\tocacronym{SN 22.19} \toctranslation{Suffering With Its Cause } \tocroot{Sahetudukkhasutta}}
\markboth{Suffering With Its Cause }{Sahetudukkhasutta}
\extramarks{SN 22.19}{SN 22.19}

At\marginnote{1.1} \textsanskrit{Sāvatthī}. 

“Mendicants,\marginnote{1.2} form is suffering. The cause and reason that gives rise to form is also suffering. Since form is produced by what is suffering, how could it be happiness? 

Feeling\marginnote{1.5} is suffering … 

Perception\marginnote{1.6} is suffering … 

Choices\marginnote{1.7} are suffering … 

Consciousness\marginnote{1.8} is suffering. The cause and reason that gives rise to consciousness is also suffering. Since consciousness is produced by what is suffering, how could it be happiness? 

Seeing\marginnote{1.11} this … They understand: ‘… there is no return to any state of existence.’” 

%
\section*{{\suttatitleacronym SN 22.20}{\suttatitletranslation Not-Self With Its Cause }{\suttatitleroot Sahetuanattasutta}}
\addcontentsline{toc}{section}{\tocacronym{SN 22.20} \toctranslation{Not-Self With Its Cause } \tocroot{Sahetuanattasutta}}
\markboth{Not-Self With Its Cause }{Sahetuanattasutta}
\extramarks{SN 22.20}{SN 22.20}

At\marginnote{1.1} \textsanskrit{Sāvatthī}. 

“Mendicants,\marginnote{1.2} form is not-self. The cause and reason that gives rise to form is also not-self. Since form is produced by what is not-self, how could it be self? 

Feeling\marginnote{1.5} is not-self … 

Perception\marginnote{1.6} is not-self … 

Choices\marginnote{1.7} are not-self … 

Consciousness\marginnote{1.8} is not-self. The cause and reason that gives rise to consciousness is also not-self. Since consciousness is produced by what is not-self, how could it be self? 

Seeing\marginnote{1.11} this … They understand: ‘… there is no return to any state of existence.’” 

%
\section*{{\suttatitleacronym SN 22.21}{\suttatitletranslation With Ānanda }{\suttatitleroot Ānandasutta}}
\addcontentsline{toc}{section}{\tocacronym{SN 22.21} \toctranslation{With Ānanda } \tocroot{Ānandasutta}}
\markboth{With Ānanda }{Ānandasutta}
\extramarks{SN 22.21}{SN 22.21}

At\marginnote{1.1} \textsanskrit{Sāvatthī}. 

Then\marginnote{1.2} Venerable Ānanda went up to the Buddha, bowed, sat down to one side, and said to the Buddha: 

“Sir,\marginnote{1.3} they speak of ‘cessation’. The cessation of what things does this refer to?” 

“Ānanda,\marginnote{1.5} form is impermanent, conditioned, dependently originated, liable to end, vanish, fade away, and cease. Its cessation is what ‘cessation’ refers to. 

Feeling\marginnote{1.7} … 

Perception\marginnote{1.9} … 

Choices\marginnote{1.10} … 

Consciousness\marginnote{1.12} is impermanent, conditioned, dependently originated, liable to end, vanish, fade away, and cease. Its cessation is what ‘cessation’ refers to. 

When\marginnote{1.14} they speak of ‘cessation’, its the cessation of these things that this refers to.” 

%
\addtocontents{toc}{\let\protect\contentsline\protect\nopagecontentsline}
\chapter*{The Chapter on the Burden }
\addcontentsline{toc}{chapter}{\tocchapterline{The Chapter on the Burden }}
\addtocontents{toc}{\let\protect\contentsline\protect\oldcontentsline}

%
\section*{{\suttatitleacronym SN 22.22}{\suttatitletranslation The Burden }{\suttatitleroot Bhārasutta}}
\addcontentsline{toc}{section}{\tocacronym{SN 22.22} \toctranslation{The Burden } \tocroot{Bhārasutta}}
\markboth{The Burden }{Bhārasutta}
\extramarks{SN 22.22}{SN 22.22}

At\marginnote{1.1} \textsanskrit{Sāvatthī}. 

“Mendicants,\marginnote{1.2} I will teach you the burden, the bearer of the burden, the picking up of the burden, and the putting down of the burden. Listen … 

And\marginnote{1.4} what is the burden? The five grasping aggregates, it should be said. What five? The grasping aggregates of form, feeling, perception, choices, and consciousness. This is called the burden. 

And\marginnote{2.1} who is the bearer of the burden? The person, it should be said; the venerable of such and such name and clan. This is called the bearer of the burden. 

And\marginnote{3.1} what is the picking up of the burden? It’s the craving that leads to future lives, mixed up with relishing and greed, chasing pleasure in various realms. That is,\footnote{As usual, I prefer to translate tatra tatra and related idioms as distributive. } craving for sensual pleasures, craving to continue existence, and craving to end existence. This is called the picking up of the burden. 

And\marginnote{4.1} what is the putting down of the burden? It’s the fading away and cessation of that very same craving with nothing left over; giving it away, letting it go, releasing it, and not adhering to it. This is called the putting down of the burden.” 

That\marginnote{5.1} is what the Buddha said. Then the Holy One, the Teacher, went on to say: 

\begin{verse}%
“The\marginnote{6.1} five aggregates are indeed burdens, \\
and the person is the bearer of the burden. \\
Picking up the burden is suffering in the world, \\
and putting the burden down is happiness. 

When\marginnote{7.1} the heavy burden is put down \\
without picking up another, \\
and having plucked out craving, root and all, \\
you’re hungerless, extinguished.” 

%
\end{verse}

%
\section*{{\suttatitleacronym SN 22.23}{\suttatitletranslation Complete Understanding }{\suttatitleroot Pariññasutta}}
\addcontentsline{toc}{section}{\tocacronym{SN 22.23} \toctranslation{Complete Understanding } \tocroot{Pariññasutta}}
\markboth{Complete Understanding }{Pariññasutta}
\extramarks{SN 22.23}{SN 22.23}

At\marginnote{1.1} \textsanskrit{Sāvatthī}. 

“Mendicants,\marginnote{1.2} I will teach you the things that should be completely understood, and complete understanding. Listen … 

And\marginnote{1.4} what things should be completely understood? Form, feeling, perception, choices, and consciousness. These are called the things that should be completely understood. 

And\marginnote{1.7} what is complete understanding? The ending of greed, hate, and delusion. This is called complete understanding.” 

%
\section*{{\suttatitleacronym SN 22.24}{\suttatitletranslation Directly Knowing }{\suttatitleroot Abhijānasutta}}
\addcontentsline{toc}{section}{\tocacronym{SN 22.24} \toctranslation{Directly Knowing } \tocroot{Abhijānasutta}}
\markboth{Directly Knowing }{Abhijānasutta}
\extramarks{SN 22.24}{SN 22.24}

At\marginnote{1.1} \textsanskrit{Sāvatthī}. 

“Mendicants,\marginnote{1.2} without directly knowing and completely understanding form, without dispassion for it and giving it up, you can’t end suffering. 

Without\marginnote{1.3} directly knowing and completely understanding feeling … perception … choices … consciousness, without dispassion for it and giving it up, you can’t end suffering. 

By\marginnote{1.7} directly knowing and completely understanding form, having dispassion for it and giving it up, you can end suffering. 

By\marginnote{1.8} directly knowing and completely understanding feeling … perception … choices … consciousness, having dispassion for it and giving it up, you can end suffering.” 

%
\section*{{\suttatitleacronym SN 22.25}{\suttatitletranslation Desire and Greed }{\suttatitleroot Chandarāgasutta}}
\addcontentsline{toc}{section}{\tocacronym{SN 22.25} \toctranslation{Desire and Greed } \tocroot{Chandarāgasutta}}
\markboth{Desire and Greed }{Chandarāgasutta}
\extramarks{SN 22.25}{SN 22.25}

At\marginnote{1.1} \textsanskrit{Sāvatthī}. 

“Mendicants,\marginnote{1.2} give up desire and greed for form. Thus that form will be given up, cut off at the root, made like a palm stump, obliterated, and unable to arise in the future. 

Give\marginnote{1.4} up desire and greed for feeling … perception … choices … consciousness. Thus that consciousness will be given up, cut off at the root, made like a palm stump, obliterated, and unable to arise in the future.” 

%
\section*{{\suttatitleacronym SN 22.26}{\suttatitletranslation Gratification }{\suttatitleroot Assādasutta}}
\addcontentsline{toc}{section}{\tocacronym{SN 22.26} \toctranslation{Gratification } \tocroot{Assādasutta}}
\markboth{Gratification }{Assādasutta}
\extramarks{SN 22.26}{SN 22.26}

At\marginnote{1.1} \textsanskrit{Sāvatthī}. 

“Mendicants,\marginnote{1.2} before my awakening—when I was still unawakened but intent on awakening—I thought: ‘What’s the gratification, the drawback, and the escape when it comes to form … feeling … perception … choices … and consciousness?’ 

Then\marginnote{1.8} it occurred to me: ‘The pleasure and happiness that arise from form: this is its gratification. That form is impermanent, suffering, and perishable: this is its drawback. Removing and giving up desire and greed for form: this is its escape. The pleasure and happiness that arise from feeling … perception … choices … consciousness: this is its gratification. That consciousness is impermanent, suffering, and perishable: this is its drawback. Removing and giving up desire and greed for consciousness: this is its escape.’ 

As\marginnote{2.1} long as I didn’t truly understand these five grasping aggregates’ gratification, drawback, and escape in this way for what they are, I didn’t announce my supreme perfect awakening in this world with its gods, \textsanskrit{Māras}, and \textsanskrit{Brahmās}, this population with its ascetics and brahmins, its gods and humans. 

But\marginnote{2.2} when I did truly understand these five grasping aggregates’ gratification, drawback, and escape in this way for what they are, I announced my supreme perfect awakening in this world with its gods, \textsanskrit{Māras}, and \textsanskrit{Brahmās}, this population with its ascetics and brahmins, its gods and humans. 

Knowledge\marginnote{2.4} and vision arose in me: ‘My freedom is unshakable; this is my last rebirth; now there’ll be no more future lives.’” 

%
\section*{{\suttatitleacronym SN 22.27}{\suttatitletranslation Gratification (2nd) }{\suttatitleroot Dutiyaassādasutta}}
\addcontentsline{toc}{section}{\tocacronym{SN 22.27} \toctranslation{Gratification (2nd) } \tocroot{Dutiyaassādasutta}}
\markboth{Gratification (2nd) }{Dutiyaassādasutta}
\extramarks{SN 22.27}{SN 22.27}

At\marginnote{1.1} \textsanskrit{Sāvatthī}. 

“Mendicants,\marginnote{1.2} I went in search of form’s gratification, and I found it. I’ve seen clearly with wisdom the full extent of form’s gratification. I went in search of form’s drawback, and I found it. I’ve seen clearly with wisdom the full extent of form’s drawback. I went in search of form’s escape, and I found it. I’ve seen clearly with wisdom the full extent of form’s escape. 

I\marginnote{1.11} went in search of the gratification of feeling … perception … choices … and consciousness, and I found it. I’ve seen clearly with wisdom the full extent of consciousness’s gratification. I went in search of consciousness’s drawback, and I found it. I’ve seen clearly with wisdom the full extent of consciousness’s drawback. I went in search of consciousness’s escape, and I found it. I’ve seen clearly with wisdom the full extent of consciousness’s escape. 

As\marginnote{1.23} long as I didn’t truly understand these five grasping aggregates’ gratification, drawback, and escape for what they are, I didn’t announce my supreme perfect awakening … But when I did truly understand these five grasping aggregates’ gratification, drawback, and escape for what they are, I announced my supreme perfect awakening … 

Knowledge\marginnote{1.25} and vision arose in me: ‘My freedom is unshakable; this is my last rebirth; now there’ll be no more future lives.’” 

%
\section*{{\suttatitleacronym SN 22.28}{\suttatitletranslation Gratification (3rd) }{\suttatitleroot Tatiyaassādasutta}}
\addcontentsline{toc}{section}{\tocacronym{SN 22.28} \toctranslation{Gratification (3rd) } \tocroot{Tatiyaassādasutta}}
\markboth{Gratification (3rd) }{Tatiyaassādasutta}
\extramarks{SN 22.28}{SN 22.28}

At\marginnote{1.1} \textsanskrit{Sāvatthī}. 

“Mendicants,\marginnote{1.2} if there were no gratification in form, sentient beings wouldn’t be aroused by it. But since there is gratification in form, sentient beings do love it. If form had no drawback, sentient beings wouldn’t grow disillusioned with it. But since form has a drawback, sentient beings do grow disillusioned with it. If there were no escape from form, sentient beings wouldn’t escape from it. But since there is an escape from form, sentient beings do escape from it. 

If\marginnote{1.8} there were no gratification in feeling … perception … choices … consciousness, sentient beings wouldn’t be aroused by it. But since there is gratification in consciousness, sentient beings do love it. If consciousness had no drawback, sentient beings wouldn’t grow disillusioned with it. But since consciousness has a drawback, sentient beings do grow disillusioned with it. If there were no escape from consciousness, sentient beings wouldn’t escape from it. But since there is an escape from consciousness, sentient beings do escape from it. 

As\marginnote{2.1} long as sentient beings don’t truly understand these five grasping aggregates’ gratification, drawback, and escape for what they are, they haven’t escaped from this world—with its gods, \textsanskrit{Māras}, and \textsanskrit{Brahmās}, this population with its ascetics and brahmins, its gods and humans—and they don’t live detached, liberated, with a mind free of limits. 

But\marginnote{2.3} when sentient beings truly understand these five grasping aggregates’ gratification, drawback, and escape for what they are, they’ve escaped from this world—with its gods, \textsanskrit{Māras}, and \textsanskrit{Brahmās}, this population with its ascetics and brahmins, its gods and humans—and they live detached, liberated, with a mind free of limits.” 

%
\section*{{\suttatitleacronym SN 22.29}{\suttatitletranslation Taking Pleasure }{\suttatitleroot Abhinandanasutta}}
\addcontentsline{toc}{section}{\tocacronym{SN 22.29} \toctranslation{Taking Pleasure } \tocroot{Abhinandanasutta}}
\markboth{Taking Pleasure }{Abhinandanasutta}
\extramarks{SN 22.29}{SN 22.29}

At\marginnote{1.1} \textsanskrit{Sāvatthī}. 

“Mendicants,\marginnote{1.2} if you take pleasure in form, you take pleasure in suffering. If you take pleasure in suffering, I say you’re not exempt from suffering. 

If\marginnote{1.4} you take pleasure in feeling … perception … choices … consciousness, you take pleasure in suffering. If you take pleasure in suffering, I say you’re not exempt from suffering. 

If\marginnote{1.9} you don’t take pleasure in form, you don’t take pleasure in suffering. If you don’t take pleasure in suffering, I say you’re exempt from suffering. 

If\marginnote{1.11} you don’t take pleasure in feeling … perception … choices … consciousness, you don’t take pleasure in suffering. If you don’t take pleasure in suffering, I say you’re exempt from suffering.” 

%
\section*{{\suttatitleacronym SN 22.30}{\suttatitletranslation Arising }{\suttatitleroot Uppādasutta}}
\addcontentsline{toc}{section}{\tocacronym{SN 22.30} \toctranslation{Arising } \tocroot{Uppādasutta}}
\markboth{Arising }{Uppādasutta}
\extramarks{SN 22.30}{SN 22.30}

At\marginnote{1.1} \textsanskrit{Sāvatthī}. 

“Mendicants,\marginnote{1.2} the arising, continuation, rebirth, and manifestation of form is the arising of suffering, the continuation of diseases, and the manifestation of old age and death. 

The\marginnote{1.3} arising, continuation, rebirth, and manifestation of feeling … perception … choices … consciousness is the arising of suffering, the continuation of diseases, and the manifestation of old age and death. 

The\marginnote{1.7} cessation, settling, and ending of form is the cessation of suffering, the settling of diseases, and the ending of old age and death. 

The\marginnote{1.8} cessation, settling, and ending of feeling … perception … choices … consciousness is the cessation of suffering, the settling of diseases, and the ending of old age and death.” 

%
\section*{{\suttatitleacronym SN 22.31}{\suttatitletranslation The Root of Misery }{\suttatitleroot Aghamūlasutta}}
\addcontentsline{toc}{section}{\tocacronym{SN 22.31} \toctranslation{The Root of Misery } \tocroot{Aghamūlasutta}}
\markboth{The Root of Misery }{Aghamūlasutta}
\extramarks{SN 22.31}{SN 22.31}

At\marginnote{1.1} \textsanskrit{Sāvatthī}. 

“Mendicants,\marginnote{1.2} I will teach you misery and the root of misery. Listen … 

And\marginnote{1.4} what is misery? Form, feeling, perception, choices, and consciousness are misery. This is called misery. 

And\marginnote{1.7} what is the root of misery? It’s the craving that leads to future lives, mixed up with relishing and greed, chasing pleasure in various realms. That is, craving for sensual pleasures, craving to continue existence, and craving to end existence. This is called the root of misery.” 

%
\section*{{\suttatitleacronym SN 22.32}{\suttatitletranslation The Breakable }{\suttatitleroot Pabhaṅgusutta}}
\addcontentsline{toc}{section}{\tocacronym{SN 22.32} \toctranslation{The Breakable } \tocroot{Pabhaṅgusutta}}
\markboth{The Breakable }{Pabhaṅgusutta}
\extramarks{SN 22.32}{SN 22.32}

At\marginnote{1.1} \textsanskrit{Sāvatthī}. 

“Mendicants,\marginnote{1.2} I will teach you the breakable and the unbreakable. Listen … 

And\marginnote{1.4} what is the breakable? What is the unbreakable? Form is breakable, but its cessation, settling, and ending is unbreakable. 

Feeling\marginnote{1.7} … perception … choices … consciousness is breakable, but its cessation, settling, and ending is unbreakable.” 

%
\addtocontents{toc}{\let\protect\contentsline\protect\nopagecontentsline}
\chapter*{The Chapter on Not Yours }
\addcontentsline{toc}{chapter}{\tocchapterline{The Chapter on Not Yours }}
\addtocontents{toc}{\let\protect\contentsline\protect\oldcontentsline}

%
\section*{{\suttatitleacronym SN 22.33}{\suttatitletranslation It’s Not Yours }{\suttatitleroot Natumhākasutta}}
\addcontentsline{toc}{section}{\tocacronym{SN 22.33} \toctranslation{It’s Not Yours } \tocroot{Natumhākasutta}}
\markboth{It’s Not Yours }{Natumhākasutta}
\extramarks{SN 22.33}{SN 22.33}

At\marginnote{1.1} \textsanskrit{Sāvatthī}. 

“Mendicants,\marginnote{1.2} give up what’s not yours. Giving it up will be for your welfare and happiness. And what isn’t yours? Form isn’t yours: give it up. Giving it up will be for your welfare and happiness. 

Feeling\marginnote{1.7} … 

Perception\marginnote{1.9} … 

Choices\marginnote{1.10} … 

Consciousness\marginnote{1.12} isn’t yours: give it up. Giving it up will be for your welfare and happiness. 

Suppose\marginnote{2.1} a person was to carry off the grass, sticks, branches, and leaves in this Jeta’s Grove, or burn them, or do what they want with them. Would you think: ‘This person is carrying us off, burning us, or doing what they want with us’?” 

“No,\marginnote{2.4} sir. Why is that? Because that’s neither self nor belonging to self.” 

“In\marginnote{2.7} the same way, mendicants, form isn’t yours: give it up. Giving it up will be for your welfare and happiness. 

Feeling\marginnote{2.9} … 

Perception\marginnote{2.11} … 

Choices\marginnote{2.12} … 

Consciousness\marginnote{2.13} isn’t yours: give it up. Giving it up will be for your welfare and happiness.” 

%
\section*{{\suttatitleacronym SN 22.34}{\suttatitletranslation It’s Not Yours (2nd) }{\suttatitleroot Dutiyanatumhākasutta}}
\addcontentsline{toc}{section}{\tocacronym{SN 22.34} \toctranslation{It’s Not Yours (2nd) } \tocroot{Dutiyanatumhākasutta}}
\markboth{It’s Not Yours (2nd) }{Dutiyanatumhākasutta}
\extramarks{SN 22.34}{SN 22.34}

At\marginnote{1.1} \textsanskrit{Sāvatthī}. 

“Mendicants,\marginnote{1.2} give up what’s not yours. Giving it up will be for your welfare and happiness. And what isn’t yours? 

Form\marginnote{1.5} isn’t yours: give it up. Giving it up will be for your welfare and happiness. 

Feeling\marginnote{1.7} … 

Perception\marginnote{1.8} … 

Choices\marginnote{1.9} … 

Consciousness\marginnote{1.10} isn’t yours: give it up. Giving it up will be for your welfare and happiness. 

Give\marginnote{1.12} up what’s not yours. Giving it up will be for your welfare and happiness.” 

%
\section*{{\suttatitleacronym SN 22.35}{\suttatitletranslation A Mendicant }{\suttatitleroot Aññatarabhikkhusutta}}
\addcontentsline{toc}{section}{\tocacronym{SN 22.35} \toctranslation{A Mendicant } \tocroot{Aññatarabhikkhusutta}}
\markboth{A Mendicant }{Aññatarabhikkhusutta}
\extramarks{SN 22.35}{SN 22.35}

At\marginnote{1.1} \textsanskrit{Sāvatthī}. 

Then\marginnote{1.2} a mendicant went up to the Buddha, bowed, sat down to one side, and said to him, “Sir, may the Buddha please teach me Dhamma in brief. When I’ve heard it, I’ll live alone, withdrawn, diligent, keen, and resolute.” 

“Mendicant,\marginnote{1.5} you’re defined by what you have an underlying tendency for. You’re not defined by what you have no underlying tendency for.” 

“Understood,\marginnote{1.7} Blessed One! Understood, Holy One!” 

“But\marginnote{2.1} how do you see the detailed meaning of my brief statement?” 

“If\marginnote{2.2} you have an underlying tendency for form, you’re defined by that. If you have an underlying tendency for feeling … perception … choices … consciousness, you’re defined by that. 

If\marginnote{2.7} you have no underlying tendency for form, you’re not defined by that. If you have no underlying tendency for feeling … perception … choices … consciousness, you’re not defined by that. 

That’s\marginnote{2.12} how I understand the detailed meaning of the Buddha’s brief statement.” 

“Good,\marginnote{3.1} good, mendicant! It’s good that you understand the detailed meaning of what I’ve said in brief like this. 

If\marginnote{3.3} you have an underlying tendency for form, you’re defined by that. If you have an underlying tendency for feeling … perception … choices … consciousness, you’re defined by that. 

If\marginnote{3.8} you have no underlying tendency for form, you’re not defined by that. If you have no underlying tendency for feeling … perception … choices … consciousness, you’re not defined by that. 

This\marginnote{3.13} is how to understand the detailed meaning of what I said in brief.” 

And\marginnote{4.1} then that mendicant approved and agreed with what the Buddha said. He got up from his seat, bowed, and respectfully circled the Buddha, keeping him on his right, before leaving. 

Then\marginnote{5.1} that mendicant, living alone, withdrawn, diligent, keen, and resolute, soon realized the supreme end of the spiritual path in this very life. He lived having achieved with his own insight the goal for which gentlemen rightly go forth from the lay life to homelessness. 

He\marginnote{5.2} understood: “Rebirth is ended; the spiritual journey has been completed; what had to be done has been done; there is no return to any state of existence.” And that mendicant became one of the perfected. 

%
\section*{{\suttatitleacronym SN 22.36}{\suttatitletranslation A Mendicant (2nd) }{\suttatitleroot Dutiyaaññatarabhikkhusutta}}
\addcontentsline{toc}{section}{\tocacronym{SN 22.36} \toctranslation{A Mendicant (2nd) } \tocroot{Dutiyaaññatarabhikkhusutta}}
\markboth{A Mendicant (2nd) }{Dutiyaaññatarabhikkhusutta}
\extramarks{SN 22.36}{SN 22.36}

At\marginnote{1.1} \textsanskrit{Sāvatthī}. 

Then\marginnote{1.2} a mendicant went up to the Buddha … and asked him, “Sir, may the Buddha please teach me Dhamma in brief. When I’ve heard it, I’ll live alone, withdrawn, diligent, keen, and resolute.” 

“Mendicant,\marginnote{1.4} you’re measured against what you have an underlying tendency for, and you’re defined by what you’re measured against. You’re not measured against what you have no underlying tendency for, and you’re not defined by what you’re not measured against.” 

“Understood,\marginnote{1.8} Blessed One! Understood, Holy One!” 

“But\marginnote{2.1} how do you see the detailed meaning of my brief statement?” 

“If\marginnote{2.2} you have an underlying tendency for form, you’re measured against that, and you’re defined by what you’re measured against. If you have an underlying tendency for feeling … perception … choices … consciousness, you’re measured against that, and you’re defined by what you’re measured against. 

If\marginnote{2.9} you have no underlying tendency for form, you’re not measured against that, and you’re not defined by what you’re not measured against. If you have no underlying tendency for feeling … perception … choices … consciousness, you’re not measured against that, and you’re not defined by what you’re not measured against. 

That’s\marginnote{2.16} how I understand the detailed meaning of the Buddha’s brief statement.” 

“Good,\marginnote{3.1} good, mendicant! It’s good that you understand the detailed meaning of what I’ve said in brief like this. 

If\marginnote{3.3} you have an underlying tendency for form, you’re measured against that, and you’re defined by what you’re measured against. If you have an underlying tendency for feeling … perception … choices … consciousness, you’re measured against that, and you’re defined by what you’re measured against. 

If\marginnote{3.10} you have no underlying tendency for form, you’re not measured against that, and you’re not defined by what you’re not measured against. If you have no underlying tendency for feeling … perception … choices … consciousness, you’re not measured against that, and you’re not defined by what you’re not measured against. 

This\marginnote{3.17} is how to understand the detailed meaning of what I said in brief.” … 

And\marginnote{3.18} that mendicant became one of the perfected. 

%
\section*{{\suttatitleacronym SN 22.37}{\suttatitletranslation With Ānanda }{\suttatitleroot Ānandasutta}}
\addcontentsline{toc}{section}{\tocacronym{SN 22.37} \toctranslation{With Ānanda } \tocroot{Ānandasutta}}
\markboth{With Ānanda }{Ānandasutta}
\extramarks{SN 22.37}{SN 22.37}

At\marginnote{1.1} \textsanskrit{Sāvatthī}. 

And\marginnote{1.2} then Venerable Ānanda … sitting to one side, the Buddha said to him: 

“Ānanda,\marginnote{2.1} suppose they were to ask you: ‘Reverend Ānanda, what are the things for which arising is evident, vanishing is evident, and change while persisting is evident?’ How would you answer?” 

“Sir,\marginnote{2.4} suppose they were to ask me: ‘What are the things for which arising is evident, vanishing is evident, and change while persisting is evident?’ I’d answer like this: 

‘Reverend,\marginnote{2.7} the arising of form is evident, its vanishing is evident, and change while persisting is evident. The arising of feeling … perception … choices … consciousness is evident, its vanishing is evident, and change while persisting is evident. These are the things for which arising is evident, vanishing is evident, and change while persisting is evident.’ 

That’s\marginnote{2.13} how I’d answer such a question.” 

“Good,\marginnote{3.1} good, Ānanda. The arising of form is evident, its vanishing is evident, and change while persisting is evident. The arising of feeling … perception … choices … consciousness is evident, its vanishing is evident, and change while persisting is evident. These are the things for which arising is evident, vanishing is evident, and change while persisting is evident. 

That’s\marginnote{3.8} how you should answer such a question.” 

%
\section*{{\suttatitleacronym SN 22.38}{\suttatitletranslation With Ānanda (2nd) }{\suttatitleroot Dutiyaānandasutta}}
\addcontentsline{toc}{section}{\tocacronym{SN 22.38} \toctranslation{With Ānanda (2nd) } \tocroot{Dutiyaānandasutta}}
\markboth{With Ānanda (2nd) }{Dutiyaānandasutta}
\extramarks{SN 22.38}{SN 22.38}

At\marginnote{1.1} \textsanskrit{Sāvatthī}. 

Sitting\marginnote{1.2} to one side, the Buddha said to Ānanda: 

“Ānanda,\marginnote{2.1} suppose they were to ask you: ‘Reverend Ānanda, what are the things for which arising, vanishing, and change while persisting were evident? What are the things for which arising, vanishing, and change while persisting will be evident? What are the things for which arising, vanishing, and change while persisting are evident?’ How would you answer?” 

“Sir,\marginnote{2.6} suppose they were to ask me: ‘Reverend Ānanda, what are the things for which arising, vanishing, and change while persisting were evident? What are the things for which arising, vanishing, and change while persisting will be evident? What are the things for which arising, vanishing, and change while persisting are evident?’ I’d answer like this: 

‘Whatever\marginnote{2.11} form has passed, ceased, and perished, its arising, vanishing, and change while persisting were evident. Whatever feeling … perception … choices … consciousness has passed, ceased, and perished, its arising, vanishing, and change while persisting were evident. These the things for which arising, vanishing, and change while persisting were evident. 

Whatever\marginnote{3.1} form is not yet born, and has not yet appeared, its arising, vanishing, and change while persisting will be evident. Whatever feeling … perception … choices … consciousness is not yet born, and has not yet appeared, its arising, vanishing, and change while persisting will be evident. These are the things for which arising, vanishing, and change while persisting will be evident. 

Whatever\marginnote{4.1} form has been born, and has appeared, its arising, vanishing, and change while persisting is evident. Whatever feeling … perception … choices … consciousness has been born, and has appeared, its arising, vanishing, and change while persisting are evident. These are the things for which arising is evident, vanishing is evident, and change while persisting is evident.’ That’s how I’d answer such a question.” 

“Good,\marginnote{5.1} good, Ānanda. Whatever form has passed, ceased, and perished, its arising, vanishing, and change while persisting were evident. Whatever feeling … perception … choices … consciousness has passed, ceased, and perished, its arising, vanishing, and change while persisting were evident. These the things for which arising, vanishing, and change while persisting were evident. 

Whatever\marginnote{6.1} form is not yet born, and has not yet appeared, its arising, vanishing, and change while persisting will be evident. Whatever feeling … perception … choices … consciousness is not yet born, and has not yet appeared, its arising, vanishing, and change while persisting will be evident. These are the things for which arising, vanishing, and change while persisting will be evident. 

Whatever\marginnote{7.1} form has been born, and has appeared, its arising, vanishing, and change while persisting are evident. Whatever feeling … perception … choices … consciousness has been born, and has appeared, its arising, vanishing, and change while persisting are evident. These are the things for which arising is evident, vanishing is evident, and change while persisting is evident. 

That’s\marginnote{7.9} how you should answer such a question.” 

%
\section*{{\suttatitleacronym SN 22.39}{\suttatitletranslation In Line With the Teachings }{\suttatitleroot Anudhammasutta}}
\addcontentsline{toc}{section}{\tocacronym{SN 22.39} \toctranslation{In Line With the Teachings } \tocroot{Anudhammasutta}}
\markboth{In Line With the Teachings }{Anudhammasutta}
\extramarks{SN 22.39}{SN 22.39}

At\marginnote{1.1} \textsanskrit{Sāvatthī}. 

“Mendicants,\marginnote{1.2} when a mendicant is practicing in line with the teachings, this is what’s in line with the teachings. 

They\marginnote{1.3} should live full of disillusionment for form, feeling, perception, choices, and consciousness. Living in this way, they completely understand form, feeling, perception, choices, and consciousness. Completely understanding form, feeling, perception, choices, and consciousness, they’re freed from these things. They’re freed from rebirth, old age, and death, from sorrow, lamentation, pain, sadness, and distress. They’re freed from suffering, I say.” 

%
\section*{{\suttatitleacronym SN 22.40}{\suttatitletranslation In Line with the Teachings (2nd) }{\suttatitleroot Dutiyaanudhammasutta}}
\addcontentsline{toc}{section}{\tocacronym{SN 22.40} \toctranslation{In Line with the Teachings (2nd) } \tocroot{Dutiyaanudhammasutta}}
\markboth{In Line with the Teachings (2nd) }{Dutiyaanudhammasutta}
\extramarks{SN 22.40}{SN 22.40}

At\marginnote{1.1} \textsanskrit{Sāvatthī}. 

“Mendicants,\marginnote{1.2} when a mendicant is practicing in line with the teachings, this is what’s in line with the teachings. They should live observing impermanence in form, feeling, perception, choices, and consciousness. … They’re freed from suffering, I say.” 

%
\section*{{\suttatitleacronym SN 22.41}{\suttatitletranslation In Line with the Teachings (3rd) }{\suttatitleroot Tatiyaanudhammasutta}}
\addcontentsline{toc}{section}{\tocacronym{SN 22.41} \toctranslation{In Line with the Teachings (3rd) } \tocroot{Tatiyaanudhammasutta}}
\markboth{In Line with the Teachings (3rd) }{Tatiyaanudhammasutta}
\extramarks{SN 22.41}{SN 22.41}

At\marginnote{1.1} \textsanskrit{Sāvatthī}. 

“Mendicants,\marginnote{1.2} when a mendicant is practicing in line with the teachings, this is what’s in line with the teachings. They should live observing suffering in form, feeling, perception, choices, and consciousness. … They’re freed from suffering, I say.” 

%
\section*{{\suttatitleacronym SN 22.42}{\suttatitletranslation In Line with the Teachings (4th) }{\suttatitleroot Catutthaanudhammasutta}}
\addcontentsline{toc}{section}{\tocacronym{SN 22.42} \toctranslation{In Line with the Teachings (4th) } \tocroot{Catutthaanudhammasutta}}
\markboth{In Line with the Teachings (4th) }{Catutthaanudhammasutta}
\extramarks{SN 22.42}{SN 22.42}

At\marginnote{1.1} \textsanskrit{Sāvatthī}. 

“Mendicants,\marginnote{1.2} when a mendicant is practicing in line with the teachings, this is what’s in line with the teachings. They should live observing not-self in form, feeling, perception, choices, and consciousness. … 

\scendsection{They’re freed from suffering, I say.” }

%
\addtocontents{toc}{\let\protect\contentsline\protect\nopagecontentsline}
\chapter*{The Chapter on Be Your Own Island }
\addcontentsline{toc}{chapter}{\tocchapterline{The Chapter on Be Your Own Island }}
\addtocontents{toc}{\let\protect\contentsline\protect\oldcontentsline}

%
\section*{{\suttatitleacronym SN 22.43}{\suttatitletranslation Be Your Own Island }{\suttatitleroot Attadīpasutta}}
\addcontentsline{toc}{section}{\tocacronym{SN 22.43} \toctranslation{Be Your Own Island } \tocroot{Attadīpasutta}}
\markboth{Be Your Own Island }{Attadīpasutta}
\extramarks{SN 22.43}{SN 22.43}

At\marginnote{1.1} \textsanskrit{Sāvatthī}. 

“Mendicants,\marginnote{1.2} be your own island, your own refuge, with no other refuge. Let the teaching be your island and your refuge, with no other refuge. 

When\marginnote{1.3} you live like this, you should examine the cause: ‘From what are sorrow, lamentation, pain, sadness, and distress born and produced?’ 

And,\marginnote{2.1} mendicants, from what are sorrow, lamentation, pain, sadness, and distress born and produced? It’s when an unlearned ordinary person has not seen the noble ones, and is neither skilled nor trained in the teaching of the noble ones. They’ve not seen good persons, and are neither skilled nor trained in the teaching of the good persons. They regard form as self, self as having form, form in self, or self in form. But that form of theirs decays and perishes, which gives rise to sorrow, lamentation, pain, sadness, and distress. 

They\marginnote{2.6} regard feeling as self … 

They\marginnote{2.9} regard perception as self … 

They\marginnote{2.10} regard choices as self … 

They\marginnote{2.11} regard consciousness as self, self as having consciousness, consciousness in self, or self in consciousness. But that consciousness of theirs decays and perishes, which gives rise to sorrow, lamentation, pain, sadness, and distress. 

Sorrow,\marginnote{3.1} lamentation, pain, sadness, and distress are given up when you understand the impermanence of form—its perishing, fading away, and cessation—and you truly see with right understanding that all form, whether past or present, is impermanent, suffering, and perishable. When these things are given up there’s no anxiety. Without anxiety you live happily. A mendicant who lives happily is said to be extinguished in that respect. 

Sorrow,\marginnote{3.3} lamentation, pain, sadness, and distress are given up when you understand the impermanence of feeling … 

perception\marginnote{3.5} … 

choices\marginnote{3.6} … 

consciousness—its\marginnote{3.8} perishing, fading away, and cessation—and you truly see with right understanding that all consciousness, whether past or present, is impermanent, suffering, and perishable. When these things are given up there’s no anxiety. Without anxiety you live happily. A mendicant who lives happily is said to be extinguished in that respect.” 

%
\section*{{\suttatitleacronym SN 22.44}{\suttatitletranslation Practice }{\suttatitleroot Paṭipadāsutta}}
\addcontentsline{toc}{section}{\tocacronym{SN 22.44} \toctranslation{Practice } \tocroot{Paṭipadāsutta}}
\markboth{Practice }{Paṭipadāsutta}
\extramarks{SN 22.44}{SN 22.44}

At\marginnote{1.1} \textsanskrit{Sāvatthī}. 

“Mendicants,\marginnote{1.2} I will teach you the practice that leads to the origin of identity and the practice that leads to the cessation of identity. Listen … 

And\marginnote{1.4} what is the practice that leads to the origin of identity? It’s when an unlearned ordinary person has not seen the noble ones, and is neither skilled nor trained in the teaching of the noble ones. They’ve not seen good persons, and are neither skilled nor trained in the teaching of the good persons. 

They\marginnote{1.6} regard form as self, self as having form, form in self, or self in form. 

They\marginnote{1.7} regard feeling as self … 

They\marginnote{1.8} regard perception as self … 

They\marginnote{1.9} regard choices as self … 

They\marginnote{1.10} regard consciousness as self, self as having consciousness, consciousness in self, or self in consciousness. 

This\marginnote{1.11} is called the practice that leads to the origin of identity.\footnote{Punctuation in MS is misleading. } And that’s why it’s called a way of regarding things that leads to the origin of suffering. 

And\marginnote{2.1} what is the practice that leads to the cessation of identity? It’s when a learned noble disciple has seen the noble ones, and is skilled and trained in the teaching of the noble ones. They’ve seen good persons, and are skilled and trained in the teaching of the good persons. 

They\marginnote{2.3} don’t regard form as self, self as having form, form in self, or self in form. 

They\marginnote{2.4} don’t regard feeling as self … 

They\marginnote{2.5} don’t regard perception as self … 

They\marginnote{2.6} don’t regard choices as self … 

They\marginnote{2.7} don’t regard consciousness as self, self as having consciousness, consciousness in self, or self in consciousness. 

This\marginnote{2.8} is called the practice that leads to the cessation of identity. And that’s why it’s called a way of regarding things that leads to the cessation of suffering.” 

%
\section*{{\suttatitleacronym SN 22.45}{\suttatitletranslation Impermanence }{\suttatitleroot Aniccasutta}}
\addcontentsline{toc}{section}{\tocacronym{SN 22.45} \toctranslation{Impermanence } \tocroot{Aniccasutta}}
\markboth{Impermanence }{Aniccasutta}
\extramarks{SN 22.45}{SN 22.45}

At\marginnote{1.1} \textsanskrit{Sāvatthī}. 

“Mendicants,\marginnote{1.2} form is impermanent. What’s impermanent is suffering. What’s suffering is not-self. And what’s not-self should be truly seen with right understanding like this: ‘This is not mine, I am not this, this is not my self.’ Seeing truly with right understanding like this, the mind becomes dispassionate and freed from defilements by not grasping. 

Feeling\marginnote{1.7} is impermanent … 

Perception\marginnote{1.8} … 

Choices\marginnote{1.9} … 

Consciousness\marginnote{1.10} is impermanent. What’s impermanent is suffering. What’s suffering is not-self. And what’s not-self should be truly seen with right understanding like this: ‘This is not mine, I am not this, this is not my self.’ Seeing truly with right understanding like this, the mind becomes dispassionate and freed from defilements by not grasping. 

If\marginnote{1.15} a mendicant’s mind is dispassionate towards the form element, the feeling element, the perception element, the choices element, and the consciousness element, it’s freed from defilements by not grasping. 

Being\marginnote{1.19} free, it’s stable. Being stable, it’s content. Being content, they’re not anxious. Not being anxious, they personally become extinguished. 

They\marginnote{1.20} understand: ‘Rebirth is ended, the spiritual journey has been completed, what had to be done has been done, there is no return to any state of existence.’” 

%
\section*{{\suttatitleacronym SN 22.46}{\suttatitletranslation Impermanence (2nd) }{\suttatitleroot Dutiyaaniccasutta}}
\addcontentsline{toc}{section}{\tocacronym{SN 22.46} \toctranslation{Impermanence (2nd) } \tocroot{Dutiyaaniccasutta}}
\markboth{Impermanence (2nd) }{Dutiyaaniccasutta}
\extramarks{SN 22.46}{SN 22.46}

At\marginnote{1.1} \textsanskrit{Sāvatthī}. 

“Mendicants,\marginnote{1.2} form is impermanent. What’s impermanent is suffering. What’s suffering is not-self. And what’s not-self should be truly seen with right understanding like this: ‘This is not mine, I am not this, this is not my self.’ 

Feeling\marginnote{1.6} is impermanent … 

Perception\marginnote{1.7} is impermanent … 

Choices\marginnote{1.8} are impermanent … 

Consciousness\marginnote{1.9} is impermanent. What’s impermanent is suffering. What’s suffering is not-self. And what’s not-self should be truly seen with right understanding like this: ‘This is not mine, I am not this, this is not my self.’ 

Seeing\marginnote{2.1} truly with right understanding like this, they have no theories about the past. Not having theories about the past, they have no theories about the future. Not having theories about the future, they don’t obstinately stick to them. Not misapprehending, the mind becomes dispassionate towards form, feeling, perception, choices, and consciousness; it’s freed from defilements by not grasping. 

Being\marginnote{2.9} free, it’s stable. Being stable, it’s content. Being content, they’re not anxious. Not being anxious, they personally become extinguished. 

They\marginnote{2.10} understand: ‘Rebirth is ended, the spiritual journey has been completed, what had to be done has been done, there is no return to any state of existence.’” 

%
\section*{{\suttatitleacronym SN 22.47}{\suttatitletranslation Ways of Regarding }{\suttatitleroot Samanupassanāsutta}}
\addcontentsline{toc}{section}{\tocacronym{SN 22.47} \toctranslation{Ways of Regarding } \tocroot{Samanupassanāsutta}}
\markboth{Ways of Regarding }{Samanupassanāsutta}
\extramarks{SN 22.47}{SN 22.47}

At\marginnote{1.1} \textsanskrit{Sāvatthī}. 

“Mendicants,\marginnote{1.2} whatever ascetics and brahmins regard various kinds of things as self, all regard the five grasping aggregates, or one of them. 

What\marginnote{1.3} five? It’s when an unlearned ordinary person has not seen the noble ones, and is neither skilled nor trained in the teaching of the noble ones. They’ve not seen good persons, and are neither skilled nor trained in the teaching of the good persons. 

They\marginnote{1.5} regard form as self, self as having form, form in self, or self in form. They regard feeling … perception … choices … consciousness as self, self as having consciousness, consciousness in self, or self in consciousness. 

So\marginnote{2.1} they’re not rid of this way of regarding things and the conceit ‘I am’.\footnote{See BB’s note. } As long as they’re not rid of the conceit ‘I am’, the five faculties are conceived—the eye, ear, nose, tongue, and body. The mind, thoughts, and the element of ignorance are all present. Struck by feelings born of contact with ignorance, an unlearned ordinary person thinks ‘I am’, ‘I am this’, ‘I will be’, ‘I will not be’, ‘I will have form’, ‘I will be formless’, ‘I will be percipient’, ‘I will not be percipient’, ‘I will be neither percipient nor non-percipient’. 

The\marginnote{3.1} five faculties stay right where they are. But a learned noble disciple gives up ignorance about them and gives rise to knowledge. With the fading away of ignorance and the arising of knowledge, they don’t think ‘I am’, ‘I am this’, ‘I will be’, ‘I will not be’, ‘I will have form’, ‘I will be formless’, ‘I will be percipient’, ‘I will be non-percipient’, ‘I will be neither percipient nor non-percipient’.” 

%
\section*{{\suttatitleacronym SN 22.48}{\suttatitletranslation Aggregates }{\suttatitleroot Khandhasutta}}
\addcontentsline{toc}{section}{\tocacronym{SN 22.48} \toctranslation{Aggregates } \tocroot{Khandhasutta}}
\markboth{Aggregates }{Khandhasutta}
\extramarks{SN 22.48}{SN 22.48}

At\marginnote{1.1} \textsanskrit{Sāvatthī}. 

“Mendicants,\marginnote{1.2} I will teach you the five aggregates and the five grasping aggregates. Listen … 

And\marginnote{1.4} what are the five aggregates? 

Any\marginnote{1.5} kind of form at all—past, future, or present; internal or external; coarse or fine; inferior or superior; far or near: this is called the aggregate of form. 

Any\marginnote{1.6} kind of feeling at all … 

Any\marginnote{1.7} kind of perception at all … 

Any\marginnote{1.8} kind of choices at all … 

Any\marginnote{1.9} kind of consciousness at all—past, future, or present; internal or external; coarse or fine; inferior or superior; far or near: this is called the aggregate of consciousness. 

These\marginnote{1.10} are called the five aggregates. 

And\marginnote{2.1} what are the five grasping aggregates? 

Any\marginnote{2.2} kind of form at all—past, future, or present; internal or external; coarse or fine; inferior or superior; far or near, which is accompanied by defilements and is prone to being grasped: this is called the aggregate of form connected with grasping. 

Any\marginnote{2.3} kind of feeling at all … 

Any\marginnote{2.4} kind of perception at all … 

Any\marginnote{2.5} kind of choices at all … 

Any\marginnote{2.6} kind of consciousness at all—past, future, or present; internal or external; coarse or fine; inferior or superior; far or near, which is accompanied by defilements and is prone to being grasped: this is called the aggregate of consciousness connected with grasping. 

These\marginnote{2.7} are called the five grasping aggregates.” 

%
\section*{{\suttatitleacronym SN 22.49}{\suttatitletranslation With Soṇa }{\suttatitleroot Soṇasutta}}
\addcontentsline{toc}{section}{\tocacronym{SN 22.49} \toctranslation{With Soṇa } \tocroot{Soṇasutta}}
\markboth{With Soṇa }{Soṇasutta}
\extramarks{SN 22.49}{SN 22.49}

\scevam{So\marginnote{1.1} I have heard. }At one time the Buddha was staying near \textsanskrit{Rājagaha}, in the Bamboo Grove, the squirrels’ feeding ground. 

Then\marginnote{1.3} the householder \textsanskrit{Soṇa} went up to the Buddha … The Buddha said to him: 

“\textsanskrit{Soṇa},\marginnote{2.1} there are ascetics and brahmins who—based on form, which is impermanent, suffering, and perishable—regard themselves thus: ‘I’m better’, or ‘I’m equal’, or ‘I’m worse’. What is that but a failure to see truly? Based on feeling … perception … choices … consciousness, which is impermanent, suffering, and perishable, they regard themselves thus: ‘I’m better’, or ‘I’m equal’, or ‘I’m worse’. What is that but a failure to see truly? 

There\marginnote{3.1} are ascetics and brahmins who—based on form, which is impermanent, suffering, and perishable—don’t regard themselves thus: ‘I’m better’, or ‘I’m equal’, or ‘I’m worse’. What is that but seeing truly? Based on feeling … perception … choices … consciousness, which is impermanent, suffering, and perishable, they don’t regard themselves thus: ‘I’m better’, or ‘I’m equal’, or ‘I’m worse’. What is that but seeing truly? 

What\marginnote{4.1} do you think, \textsanskrit{Soṇa}? Is form permanent or impermanent?” 

“Impermanent,\marginnote{4.3} sir.” 

“But\marginnote{4.4} if it’s impermanent, is it suffering or happiness?” 

“Suffering,\marginnote{4.5} sir.” 

“But\marginnote{4.6} if it’s impermanent, suffering, and perishable, is it fit to be regarded thus: ‘This is mine, I am this, this is my self’?” 

“No,\marginnote{4.8} sir.” 

“Is\marginnote{4.9} feeling … perception … choices … consciousness permanent or impermanent?” 

“Impermanent,\marginnote{4.14} sir.” 

“But\marginnote{4.15} if it’s impermanent, is it suffering or happiness?” 

“Suffering,\marginnote{4.16} sir.” 

“But\marginnote{4.17} if it’s impermanent, suffering, and perishable, is it fit to be regarded thus: ‘This is mine, I am this, this is my self’?” 

“No,\marginnote{4.19} sir.” 

“So,\marginnote{5.1} \textsanskrit{Soṇa}, you should truly see any kind of form at all—past, future, or present; internal or external; coarse or fine; inferior or superior; far or near: \emph{all} form—with right understanding: ‘This is not mine, I am not this, this is not my self.’ 

You\marginnote{6.1} should truly see any kind of feeling … perception … choices … consciousness at all—past, future, or present; internal or external; coarse or fine; inferior or superior; far or near: \emph{all} consciousness—with right understanding: ‘This is not mine, I am not this, this is not my self.’ 

Seeing\marginnote{7.1} this, a learned noble disciple grows disillusioned with form, feeling, perception, choices, and consciousness. Being disillusioned, desire fades away. When desire fades away they’re freed. When they’re freed, they know they’re freed. 

They\marginnote{7.3} understand: ‘Rebirth is ended, the spiritual journey has been completed, what had to be done has been done, there is no return to any state of existence.’” 

%
\section*{{\suttatitleacronym SN 22.50}{\suttatitletranslation With Soṇa (2nd) }{\suttatitleroot Dutiyasoṇasutta}}
\addcontentsline{toc}{section}{\tocacronym{SN 22.50} \toctranslation{With Soṇa (2nd) } \tocroot{Dutiyasoṇasutta}}
\markboth{With Soṇa (2nd) }{Dutiyasoṇasutta}
\extramarks{SN 22.50}{SN 22.50}

\scevam{So\marginnote{1.1} I have heard. }At one time the Buddha was staying near \textsanskrit{Rājagaha}, in the Bamboo Grove, the squirrels’ feeding ground. 

Then\marginnote{1.3} the householder \textsanskrit{Soṇa} went up to the Buddha, bowed, and sat down to one side. The Buddha said to him: 

“\textsanskrit{Soṇa},\marginnote{2.1} there are ascetics and brahmins who don’t understand form, its origin, its cessation, and the practice that leads to its cessation. They don’t understand feeling … perception … choices … consciousness, its origin, its cessation, and the practice that leads to its cessation. I don’t regard them as true ascetics and brahmins. Those venerables don’t realize the goal of life as an ascetic or brahmin, and don’t live having realized it with their own insight. 

There\marginnote{3.1} are ascetics and brahmins who do understand form, its origin, its cessation, and the practice that leads to its cessation. They do understand feeling … perception … choices … consciousness, its origin, its cessation, and the practice that leads to its cessation. I regard them as true ascetics and brahmins. Those venerables realize the goal of life as an ascetic or brahmin, and live having realized it with their own insight.” 

%
\section*{{\suttatitleacronym SN 22.51}{\suttatitletranslation The End of Relishing }{\suttatitleroot Nandikkhayasutta}}
\addcontentsline{toc}{section}{\tocacronym{SN 22.51} \toctranslation{The End of Relishing } \tocroot{Nandikkhayasutta}}
\markboth{The End of Relishing }{Nandikkhayasutta}
\extramarks{SN 22.51}{SN 22.51}

At\marginnote{1.1} \textsanskrit{Sāvatthī}. 

“Mendicants,\marginnote{1.2} form really is impermanent. A mendicant sees that it is impermanent: that’s their right view. Seeing rightly, they grow disillusioned. When relishing ends, greed ends. When greed ends, relishing ends. When relishing and greed end, the mind is freed, and is said to be well freed. 

Feeling\marginnote{1.6} … 

Perception\marginnote{1.10} … 

Choices\marginnote{1.11} … 

Consciousness\marginnote{1.15} really is impermanent. A mendicant sees that it is impermanent: that’s their right view. Seeing rightly, they grow disillusioned. When relishing ends, greed ends. When greed ends, relishing ends. When relishing and greed end, the mind is freed, and is said to be well freed.” 

%
\section*{{\suttatitleacronym SN 22.52}{\suttatitletranslation The End of Relishing (2nd) }{\suttatitleroot Dutiyanandikkhayasutta}}
\addcontentsline{toc}{section}{\tocacronym{SN 22.52} \toctranslation{The End of Relishing (2nd) } \tocroot{Dutiyanandikkhayasutta}}
\markboth{The End of Relishing (2nd) }{Dutiyanandikkhayasutta}
\extramarks{SN 22.52}{SN 22.52}

At\marginnote{1.1} \textsanskrit{Sāvatthī}. 

“Mendicants,\marginnote{1.2} properly attend to form. Truly see the impermanence of form. When a mendicant does this, they grow disillusioned with form. When relishing ends, greed ends. When greed ends, relishing ends. When relishing and greed end, the mind is freed, and is said to be well freed. 

Properly\marginnote{1.6} attend to feeling … 

perception\marginnote{1.10} … 

choices\marginnote{1.11} … 

consciousness.\marginnote{1.15} Truly see the impermanence of consciousness. When a mendicant does this, they grow disillusioned with consciousness. When relishing ends, greed ends. When greed ends, relishing ends. When relishing and greed end, the mind is freed, and is said to be well freed.” 

%
\addtocontents{toc}{\let\protect\contentsline\protect\nopagecontentsline}
\pannasa{The Middle Fifty }
\addcontentsline{toc}{pannasa}{The Middle Fifty }
\markboth{}{}
\addtocontents{toc}{\let\protect\contentsline\protect\oldcontentsline}

%
\addtocontents{toc}{\let\protect\contentsline\protect\nopagecontentsline}
\chapter*{The Chapter on Involvement }
\addcontentsline{toc}{chapter}{\tocchapterline{The Chapter on Involvement }}
\addtocontents{toc}{\let\protect\contentsline\protect\oldcontentsline}

%
\section*{{\suttatitleacronym SN 22.53}{\suttatitletranslation Involvement }{\suttatitleroot Upayasutta}}
\addcontentsline{toc}{section}{\tocacronym{SN 22.53} \toctranslation{Involvement } \tocroot{Upayasutta}}
\markboth{Involvement }{Upayasutta}
\extramarks{SN 22.53}{SN 22.53}

At\marginnote{1.1} \textsanskrit{Sāvatthī}. 

“Mendicants,\marginnote{1.2} if you’re involved, you’re not free. If you’re not involved, you’re free. 

As\marginnote{1.3} long as consciousness remains, it would remain involved with form, supported by form, founded on form. And with a sprinkle of relishing, it would grow, increase, and mature.\footnote{“Stands” is over literal. The point here is to differentiate between the consciousness that “remains” or “lasts”, continuing in samsara because of its supporting condition, or that which does not last. Thiti in its various forms, of course, frequently has this sense. } 

Or\marginnote{1.4} consciousness would remain involved with feeling … 

Or\marginnote{1.5} consciousness would remain involved with perception … 

Or\marginnote{1.6} as long as consciousness remains, it would remain involved with choices, supported by choices, grounded on choices. And with a sprinkle of relishing, it would grow, increase, and mature. 

Mendicants,\marginnote{2.1} suppose you say: ‘Apart from form, feeling, perception, and choices, I will describe the coming and going of consciousness, its passing away and reappearing, its growth, increase, and maturity.’ That is not possible. 

If\marginnote{3.1} a mendicant has given up greed for the form element, the support is cut off, and there is no foundation for consciousness. 

If\marginnote{3.3} a mendicant has given up greed for the feeling element … 

perception\marginnote{3.4} element … 

choices\marginnote{3.5} element … 

consciousness\marginnote{3.6} element, the support is cut off, and there is no foundation for consciousness. Since that consciousness does not become established and does not grow, with no power to regenerate, it is freed. 

Being\marginnote{3.9} free, it’s stable. Being stable, it’s content. Being content, they’re not anxious. Not being anxious, they personally become extinguished. 

They\marginnote{3.10} understand: ‘Rebirth is ended, the spiritual journey has been completed, what had to be done has been done, there is no return to any state of existence.’” 

%
\section*{{\suttatitleacronym SN 22.54}{\suttatitletranslation A Seed }{\suttatitleroot Bījasutta}}
\addcontentsline{toc}{section}{\tocacronym{SN 22.54} \toctranslation{A Seed } \tocroot{Bījasutta}}
\markboth{A Seed }{Bījasutta}
\extramarks{SN 22.54}{SN 22.54}

At\marginnote{1.1} \textsanskrit{Sāvatthī}. 

“Mendicants,\marginnote{1.2} there are five kinds of plants propagated from seeds.\footnote{Bijajata are not “kinds of seeds “but” plants grown from seeds”. https://suttacentral.net/pi/pi-tv-bu-vb-pc11 \textsanskrit{Bhūtagāmo} \textsanskrit{nāma} \textsanskrit{pañca} \textsanskrit{bījajātāni}. Similarly, mulabija is not “root-seeds “but” plants grown from roots”, eg ginger, turmeric, etc. BB refers to the Vinaya passage that makes this clear. However he follows the comm for SN, which says \textsanskrit{bījajātānīti} \textsanskrit{bījāni}. This contradicts the Vinaya comm, which says: \textsanskrit{bījehi} \textsanskrit{jātāni} \textsanskrit{bījajātāni}; \textsanskrit{rukkhādīnametaṁ} \textsanskrit{adhivacanaṁ} } What five? Plants propagated from roots, stems, cuttings, or joints; and those from regular seeds are the fifth. 

Suppose\marginnote{1.5} these five kinds of plants propagated from seeds were intact, unspoiled, not weather-damaged, fertile, and well-kept. But there’s no soil or water.\footnote{sukhasayita doesn’t means “well-planted” as per BB; for how could it be planted if there is no soil? It means “well-laid by”, i.e. well kept or well preserved, as BB has in AN 3.34 } Then would these five kinds of plants propagated from seeds reach growth, increase, and maturity?” 

“No,\marginnote{1.7} sir.” 

“Suppose\marginnote{1.8} these five kinds of plants propagated from seeds were intact, unspoiled, not weather-damaged, fertile, and well-kept. And there is soil and water. Then would these five kinds of plants propagated from seeds reach growth, increase, and maturity?” 

“Yes,\marginnote{1.10} sir.” 

“The\marginnote{1.11} four grounds of consciousness should be seen as like the earth element. Relishing and greed should be seen as like the water element. Consciousness with its fuel should be seen as like the five kinds of plants propagated from seeds. 

As\marginnote{2.1} long as consciousness remains, it would remain involved with form, supported by form, grounded on form. And with a sprinkle of relishing, it would grow, increase, and mature. 

Or\marginnote{2.2} consciousness would remain involved with feeling … 

Or\marginnote{2.3} consciousness would remain involved with perception … 

Or\marginnote{2.4} as long as consciousness remains, it would remain involved with choices, supported by choices, grounded on choices. And with a sprinkle of relishing, it would grow, increase, and mature. 

Mendicants,\marginnote{3.1} suppose you say: ‘Apart from form, feeling, perception, and choices, I will describe the coming and going of consciousness, its passing away and reappearing, its growth, increase, and maturity.’ That is not possible. 

If\marginnote{4.1} a mendicant has given up greed for the form element, the support is cut off, and there is no foundation for consciousness. 

If\marginnote{4.3} a mendicant has given up greed for the feeling element … 

perception\marginnote{4.4} element … 

choices\marginnote{4.5} element … 

consciousness\marginnote{4.6} element, the support is cut off, and there is no foundation for consciousness. Since that consciousness does not become established and does not grow, with no power to regenerate, it is freed. 

Being\marginnote{4.9} free, it’s stable. Being stable, it’s content. Being content, they’re not anxious. Not being anxious, they personally become extinguished. 

They\marginnote{4.10} understand: ‘Rebirth is ended … there is no return to any state of existence.’” 

%
\section*{{\suttatitleacronym SN 22.55}{\suttatitletranslation An Inspired Saying }{\suttatitleroot Udānasutta}}
\addcontentsline{toc}{section}{\tocacronym{SN 22.55} \toctranslation{An Inspired Saying } \tocroot{Udānasutta}}
\markboth{An Inspired Saying }{Udānasutta}
\extramarks{SN 22.55}{SN 22.55}

At\marginnote{1.1} \textsanskrit{Sāvatthī}. 

There\marginnote{1.2} the Buddha expressed this heartfelt sentiment: “‘It might not be, and it might not be mine. It will not be, and it will not be mine.’\footnote{See BB’s note. } A mendicant who makes such a resolution can cut off the five lower fetters.” 

When\marginnote{1.5} he said this, one of the mendicants asked the Buddha, “But sir, how can a mendicant who makes such a resolution cut off the five lower fetters?” 

“Mendicant,\marginnote{2.1} take an unlearned ordinary person who has not seen the noble ones, and is neither skilled nor trained in their teaching. They’ve not seen good persons, and are neither skilled nor trained in their teaching. 

They\marginnote{2.2} regard form as self, self as having form, form in self, or self in form. They regard feeling … perception … choices … consciousness as self, self as having consciousness, consciousness in self, or self in consciousness. 

They\marginnote{3.1} don’t truly understand form—which is impermanent—as impermanent. They don’t truly understand feeling … perception … choices … consciousness—which is impermanent—as impermanent. 

They\marginnote{4.1} don’t truly understand form—which is suffering—as suffering. They don’t truly understand feeling … perception … choices … consciousness—which is suffering—as suffering. 

They\marginnote{5.1} don’t truly understand form—which is not-self—as not-self. They don’t truly understand feeling … perception … choices … consciousness—which is not-self—as not-self. 

They\marginnote{6.1} don’t truly understand form—which is conditioned—as conditioned. They don’t truly understand feeling … perception … choices … consciousness—which is conditioned—as conditioned. 

They\marginnote{6.6} don’t truly understand that form will disappear. They don’t truly understand that feeling … perception … choices … consciousness will disappear. 

But\marginnote{7.1} a learned noble disciple has seen the noble ones, and is skilled and trained in the teaching of the noble ones. They’ve seen good persons, and are skilled and trained in the teaching of the good persons. They don’t regard form as self … They don’t regard feeling … perception … choices … consciousness as self. 

They\marginnote{8.1} truly understand form—which is impermanent—as impermanent. They truly understand feeling … perception … choices … consciousness—which is impermanent—as impermanent. 

They\marginnote{8.6} truly understand form … feeling … perception … choices … consciousness—which is suffering—as suffering. 

They\marginnote{8.7} truly understand form … feeling … perception … choices … consciousness—which is not-self—as not-self. 

They\marginnote{8.8} truly understand form … feeling … perception … choices … consciousness—which is conditioned—as conditioned. 

They\marginnote{8.9} truly understand that form will disappear. They truly understand that feeling … perception … choices … consciousness will disappear. 

It’s\marginnote{9.1} because of the disappearance of form, feeling, perception, choices, and consciousness that a mendicant who makes such a resolution—‘It might not be, and it might not be mine. It will not be, and it will not be mine’—can cut off the five lower fetters.” 

“Sir,\marginnote{9.4} a mendicant who makes such a resolution can cut off the five lower fetters.\footnote{Note that both punctuation and text incorrectly end the direct speech here. } 

But\marginnote{10.1} how are they to know and see in order to end the defilements in the present life?” 

“Mendicant,\marginnote{10.2} an unlearned ordinary person worries about things that aren’t a worry. For an unlearned ordinary person worries: ‘It might not be, and it might not be mine. It will not be, and it will not be mine.’ 

A\marginnote{11.1} learned noble disciple doesn’t worry about things that aren’t a worry. For a learned noble disciple doesn’t worry: ‘It might not be, and it might not be mine. It will not be, and it will not be mine.’ 

As\marginnote{11.4} long as consciousness remains, it would remain involved with form, supported by form, founded on form. And with a sprinkle of relishing, it would grow, increase, and mature. 

Or\marginnote{11.5} consciousness would remain involved with feeling … 

Or\marginnote{11.6} consciousness would remain involved with perception … 

Or\marginnote{11.7} consciousness would remain involved with choices, supported by choices, grounded on choices. And with a sprinkle of relishing, it would grow, increase, and mature. 

Suppose,\marginnote{12.1} mendicant, you were to say: ‘Apart from form, feeling, perception, and choices, I will describe the coming and going of consciousness, its passing away and reappearing, its growth, increase, and maturity.’ That is not possible. 

If\marginnote{13.1} a mendicant has given up greed for the form element, the support is cut off, and there is no foundation for consciousness. 

If\marginnote{13.2} a mendicant has given up greed for the feeling element … 

perception\marginnote{13.3} element … 

choices\marginnote{13.4} element … 

consciousness\marginnote{13.5} element, the support is cut off, and there is no foundation for consciousness. Since that consciousness does not become established and does not grow, with no power to regenerate, it is freed. 

Being\marginnote{13.7} free, it’s stable. Being stable, it’s content. Being content, they’re not anxious. Not being anxious, they personally become extinguished. 

They\marginnote{13.8} understand: ‘Rebirth is ended … there is no return to any state of existence.’ 

The\marginnote{13.9} ending of the defilements is for one who knows and sees this.” 

%
\section*{{\suttatitleacronym SN 22.56}{\suttatitletranslation Perspectives }{\suttatitleroot Upādānaparipavattasutta}}
\addcontentsline{toc}{section}{\tocacronym{SN 22.56} \toctranslation{Perspectives } \tocroot{Upādānaparipavattasutta}}
\markboth{Perspectives }{Upādānaparipavattasutta}
\extramarks{SN 22.56}{SN 22.56}

At\marginnote{1.1} \textsanskrit{Sāvatthī}. 

“Mendicants,\marginnote{1.2} there are these five grasping aggregates. What five? The grasping aggregates of form, feeling, perception, choices, and consciousness. 

As\marginnote{1.5} long as I didn’t truly understand these five grasping aggregates from four perspectives, I didn’t announce my supreme perfect awakening in this world with its gods, \textsanskrit{Māras}, and \textsanskrit{Brahmās}, this population with its ascetics and brahmins, its gods and humans. 

But\marginnote{1.6} when I did truly understand these five grasping aggregates from four perspectives, I announced my supreme perfect awakening in this world with its gods, \textsanskrit{Māras}, and \textsanskrit{Brahmās}, this population with its ascetics and brahmins, its gods and humans. 

And\marginnote{2.1} how are there four perspectives? I directly knew form, its origin, its cessation, and the practice that leads to its cessation. I directly knew feeling … perception … choices … consciousness, its origin, its cessation, and the practice that leads to its cessation. 

And\marginnote{3.1} what is form? The four primary elements, and form derived from the four primary elements. This is called form. Form originates from food. When food ceases, form ceases. The practice that leads to the cessation of form is simply this noble eightfold path, that is: right view, right thought, right speech, right action, right livelihood, right effort, right mindfulness, and right immersion. 

Whatever\marginnote{4.1} ascetics and brahmins have directly known form in this way—and its origin, its cessation, and the practice that leads to its cessation—and are practicing for disillusionment, dispassion, and cessation regarding form: they are practicing well. Those who practice well have a firm footing in this teaching and training. 

Those\marginnote{5.1} ascetics and brahmins who have directly known form in this way—and its origin, its cessation, and the practice that leads to its cessation—and due to disillusionment, dispassion, and cessation regarding form, are freed by not grasping: they are well freed. Those who are well freed are consummate ones. For consummate ones, there is no cycle of rebirths to be found. 

And\marginnote{6.1} what is feeling? There are these six classes of feeling: feeling born of contact through the eye, ear, nose, tongue, body, and mind. This is called feeling. Feeling originates from contact. When contact ceases, feeling ceases. The practice that leads to the cessation of feelings is simply this noble eightfold path … 

And\marginnote{9.1} what is perception? There are these six classes of perception: perceptions of sights, sounds, smells, tastes, touches, and thoughts. This is called perception. Perception originates from contact. When contact ceases, perception ceases. The practice that leads to the cessation of perceptions is simply this noble eightfold path … 

And\marginnote{10.1} what are choices? There are these six classes of intention: intention regarding sights, sounds, smells, tastes, touches, and thoughts. These are called choices. Choices originate from contact. When contact ceases, choices cease. The practice that leads to the cessation of choices is simply this noble eightfold path … 

And\marginnote{13.1} what is consciousness? There are these six classes of consciousness: eye, ear, nose, tongue, body, and mind consciousness. This is called consciousness. Consciousness originates from name and form. When name and form cease, consciousness ceases. The practice that leads to the cessation of consciousness is simply this noble eightfold path, that is: right view, right thought, right speech, right action, right livelihood, right effort, right mindfulness, and right immersion. 

Whatever\marginnote{14.1} ascetics and brahmins have directly known consciousness in this way—and its origin, its cessation, and the practice that leads to its cessation—and are practicing for disillusionment, dispassion, and cessation regarding consciousness: they are practicing well. Those who practice well have a firm footing in this teaching and training. 

Those\marginnote{15.1} ascetics and brahmins who have directly known consciousness in this way—and its origin, its cessation, and the practice that leads to its cessation—and due to disillusionment, dispassion, and cessation regarding consciousness, are freed by not grasping: they are well freed. Those who are well freed are consummate ones. For consummate ones, there is no cycle of rebirths to be found.” 

%
\section*{{\suttatitleacronym SN 22.57}{\suttatitletranslation Seven Cases }{\suttatitleroot Sattaṭṭhānasutta}}
\addcontentsline{toc}{section}{\tocacronym{SN 22.57} \toctranslation{Seven Cases } \tocroot{Sattaṭṭhānasutta}}
\markboth{Seven Cases }{Sattaṭṭhānasutta}
\extramarks{SN 22.57}{SN 22.57}

At\marginnote{1.1} \textsanskrit{Sāvatthī}. 

“Mendicants,\marginnote{1.2} in this teaching and training a mendicant who is skilled in seven cases and who examines in three ways is called consummate, accomplished, a supreme person. 

And\marginnote{1.3} how is a mendicant skilled in seven cases? It’s when a mendicant understands form, its origin, its cessation, and the practice that leads to its cessation. They understand form’s gratification, drawback, and escape. They understand feeling … perception … choices … consciousness, its origin, its cessation, and the practice that leads to its cessation. They understand consciousness’s gratification, drawback, and escape. 

And\marginnote{2.1} what is form? The four primary elements, and form derived from the four primary elements. This is called form. Form originates from food. When food ceases, form ceases. The practice that leads to the cessation of form is simply this noble eightfold path, that is: right view, right thought, right speech, right action, right livelihood, right effort, right mindfulness, and right immersion. 

The\marginnote{3.1} pleasure and happiness that arise from form: this is its gratification. That form is impermanent, suffering, and perishable: this is its drawback. Removing and giving up desire and greed for form: this is its escape. 

Those\marginnote{4.1} ascetics and brahmins who have directly known form in this way—and its origin, its cessation, and the practice that leads to its cessation; its gratification, drawback, and escape—and are practicing for disillusionment, dispassion, and cessation regarding form: they are practicing well. Those who practice well have a firm footing in this teaching and training. 

Those\marginnote{5.1} ascetics and brahmins who have directly known form in this way—and its origin, its cessation, and the practice that leads to its cessation; its gratification, drawback, and escape—and due to disillusionment, dispassion, and cessation regarding form, are freed by not grasping: they are well freed. Those who are well freed are consummate ones. For consummate ones, there is no cycle of rebirths to be found. 

And\marginnote{6.1} what is feeling? There are these six classes of feeling: feeling born of eye contact … feeling born of mind contact. This is called feeling. Feeling originates from contact. When contact ceases, feeling ceases. The practice that leads to the cessation of feelings is simply this noble eightfold path, that is: right view, right thought, right speech, right action, right livelihood, right effort, right mindfulness, and right immersion. 

The\marginnote{7.1} pleasure and happiness that arise from feeling: this is its gratification. That feeling is impermanent, suffering, and perishable: this is its drawback. Removing and giving up desire and greed for feeling: this is its escape. … 

And\marginnote{10.1} what is perception? There are these six classes of perception: perceptions of sights, sounds, smells, tastes, touches, and thoughts. This is called perception. Perception originates from contact. When contact ceases, perception ceases. The practice that leads to the cessation of perceptions is simply this noble eightfold path … 

And\marginnote{11.1} what are choices? There are these six classes of intention: intention regarding sights … intention regarding thoughts. These are called choices. Choices originate from contact. When contact ceases, choices cease. The practice that leads to the cessation of choices is simply this noble eightfold path … 

And\marginnote{14.1} what is consciousness? There are these six classes of consciousness: eye, ear, nose, tongue, body, and mind consciousness. This is called consciousness. Consciousness originates from name and form. When name and form cease, consciousness ceases. The practice that leads to the cessation of consciousness is simply this noble eightfold path, that is: right view, right thought, right speech, right action, right livelihood, right effort, right mindfulness, and right immersion. 

The\marginnote{15.1} pleasure and happiness that arise from consciousness: this is its gratification. That consciousness is impermanent, suffering, and perishable: this is its drawback. Removing and giving up desire and greed for consciousness: this is its escape. 

Those\marginnote{16.1} ascetics and brahmins who have directly known consciousness in this way—and its origin, its cessation, and the practice that leads to its cessation; its gratification, drawback, and escape—and are practicing for disillusionment, dispassion, and cessation regarding consciousness: they are practicing well. Those who practice well have a firm footing in this teaching and training. 

Those\marginnote{17.1} ascetics and brahmins who have directly known consciousness in this way—and its origin, its cessation, and the practice that leads to its cessation; its gratification, drawback, and escape—and due to disillusionment, dispassion, and cessation regarding consciousness, are freed by not grasping: they are well freed. Those who are well freed are consummate ones. For consummate ones, there is no cycle of rebirths to be found. That’s how a mendicant is skilled in seven cases. 

And\marginnote{18.1} how does a mendicant examine in three ways? It’s when a mendicant examines the elements, sense fields, and dependent origination. That’s how a mendicant examines in three ways. 

In\marginnote{18.4} this teaching and training, a mendicant who is skilled in seven cases and who examines in three ways is called consummate, accomplished, a supreme person.” 

%
\section*{{\suttatitleacronym SN 22.58}{\suttatitletranslation The Fully Awakened Buddha }{\suttatitleroot Sammāsambuddhasutta}}
\addcontentsline{toc}{section}{\tocacronym{SN 22.58} \toctranslation{The Fully Awakened Buddha } \tocroot{Sammāsambuddhasutta}}
\markboth{The Fully Awakened Buddha }{Sammāsambuddhasutta}
\extramarks{SN 22.58}{SN 22.58}

At\marginnote{1.1} \textsanskrit{Sāvatthī}. 

“Mendicants,\marginnote{1.2} a Realized One, a perfected one, a fully awakened Buddha is freed by not grasping, due to disillusionment, dispassion, and cessation regarding form. They’re called a fully awakened Buddha. A mendicant freed by wisdom is also freed by not grasping, due to disillusionment, dispassion, and cessation regarding form. They’re called a mendicant freed by wisdom. 

A\marginnote{2.1} Realized One, a perfected one, a fully awakened Buddha is freed by not grasping, due to disillusionment, dispassion, and cessation regarding feeling … 

perception\marginnote{3.1} … 

choices\marginnote{3.2} … 

consciousness.\marginnote{3.3} They’re called a fully awakened Buddha. A mendicant freed by wisdom is also freed by not grasping, due to disillusionment, dispassion, and cessation regarding consciousness. They’re called a mendicant freed by wisdom. 

What,\marginnote{4.1} then, is the difference between a Realized One, a perfected one, a fully awakened Buddha, and a mendicant freed by wisdom?” 

“Our\marginnote{4.2} teachings are rooted in the Buddha. He is our guide and our refuge. Sir, may the Buddha himself please clarify the meaning of this. The mendicants will listen and remember it.” 

“Well\marginnote{4.3} then, mendicants, listen and pay close attention, I will speak.” 

“Yes,\marginnote{4.4} sir,” they replied. The Buddha said this: 

“The\marginnote{5.1} Realized One, the perfected one, the fully awakened Buddha gave rise to the unarisen path, gave birth to the unborn path, and explained the unexplained path. He is the knower of the path, the discoverer of the path, the expert on the path. And now the disciples live following the path; they acquire it later. 

This\marginnote{5.3} is the difference between a Realized One, a perfected one, a fully awakened Buddha, and a mendicant freed by wisdom.” 

%
\section*{{\suttatitleacronym SN 22.59}{\suttatitletranslation The Characteristic of Not-Self }{\suttatitleroot Anattalakkhaṇasutta}}
\addcontentsline{toc}{section}{\tocacronym{SN 22.59} \toctranslation{The Characteristic of Not-Self } \tocroot{Anattalakkhaṇasutta}}
\markboth{The Characteristic of Not-Self }{Anattalakkhaṇasutta}
\extramarks{SN 22.59}{SN 22.59}

At\marginnote{1.1} one time the Buddha was staying near Benares, in the deer park at Isipatana. There the Buddha addressed the group of five mendicants: 

“Mendicants!”\marginnote{1.3} 

“Venerable\marginnote{1.4} sir,” they replied. The Buddha said this: 

“Mendicants,\marginnote{2.1} form is not-self. For if form were self, it wouldn’t lead to affliction. And you could compel form: ‘May my form be like this! May it not be like that!’ But because form is not-self, it leads to affliction. And you can’t compel form: ‘May my form be like this! May it not be like that!’ 

Feeling\marginnote{3.1} is not-self … 

Perception\marginnote{4.1} is not-self … 

Choices\marginnote{4.2} are not-self … 

Consciousness\marginnote{5.1} is not-self. For if consciousness were self, it wouldn’t lead to affliction. And you could compel consciousness: ‘May my consciousness be like this! May it not be like that!’ But because consciousness is not-self, it leads to affliction. And you can’t compel consciousness: ‘May my consciousness be like this! May it not be like that!’ 

What\marginnote{6.1} do you think, mendicants? Is form permanent or impermanent?” 

“Impermanent,\marginnote{6.3} sir.” 

“But\marginnote{6.4} if it’s impermanent, is it suffering or happiness?” 

“Suffering,\marginnote{6.5} sir.” 

“But\marginnote{6.6} if it’s impermanent, suffering, and perishable, is it fit to be regarded thus: ‘This is mine, I am this, this is my self’?” 

“No,\marginnote{6.8} sir.” 

“Is\marginnote{6.9} feeling permanent or impermanent?” … 

“Is\marginnote{6.10} perception permanent or impermanent?” … 

“Are\marginnote{6.11} choices permanent or impermanent?” … 

“Is\marginnote{6.12} consciousness permanent or impermanent?” 

“Impermanent,\marginnote{6.13} sir.” 

“But\marginnote{6.14} if it’s impermanent, is it suffering or happiness?” 

“Suffering,\marginnote{6.15} sir.” 

“But\marginnote{6.16} if it’s impermanent, suffering, and perishable, is it fit to be regarded thus: ‘This is mine, I am this, this is my self’?” 

“No,\marginnote{6.18} sir.” 

“So\marginnote{7.1} you should truly see any kind of form at all—past, future, or present; internal or external; coarse or fine; inferior or superior; far or near: \emph{all} form—with right understanding: ‘This is not mine, I am not this, this is not my self.’ 

Any\marginnote{8.1} kind of feeling at all … 

Any\marginnote{9.1} kind of perception at all … 

Any\marginnote{9.2} kind of choices at all … 

You\marginnote{10.1} should truly see any kind of consciousness at all—past, future, or present; internal or external; coarse or fine; inferior or superior; far or near: \emph{all} consciousness—with right understanding: ‘This is not mine, I am not this, this is not my self.’ 

Seeing\marginnote{11.1} this, a learned noble disciple grows disillusioned with form, feeling, perception, choices, and consciousness. Being disillusioned, desire fades away. When desire fades away they’re freed. When they’re freed, they know they’re freed. 

They\marginnote{11.3} understand: ‘Rebirth is ended, the spiritual journey has been completed, what had to be done has been done, there is no return to any state of existence.’” 

That\marginnote{12.1} is what the Buddha said. Satisfied, the group of five mendicants were happy with what the Buddha said. And while this discourse was being spoken, the minds of the group of five mendicants were freed from defilements by not grasping. 

%
\section*{{\suttatitleacronym SN 22.60}{\suttatitletranslation With Mahāli }{\suttatitleroot Mahālisutta}}
\addcontentsline{toc}{section}{\tocacronym{SN 22.60} \toctranslation{With Mahāli } \tocroot{Mahālisutta}}
\markboth{With Mahāli }{Mahālisutta}
\extramarks{SN 22.60}{SN 22.60}

\scevam{So\marginnote{1.1} I have heard. }At one time the Buddha was staying near \textsanskrit{Vesālī}, at the Great Wood, in the hall with the peaked roof. 

Then\marginnote{1.3} \textsanskrit{Mahāli} the Licchavi went up to the Buddha … and said to him: 

“Sir,\marginnote{2.1} \textsanskrit{Pūraṇa} Kassapa says this: ‘There is no cause or reason for the corruption of sentient beings. Sentient beings are corrupted without cause or reason. There’s no cause or reason for the purification of sentient beings. Sentient beings are purified without cause or reason.’ What does the Buddha say about this?” 

“\textsanskrit{Mahāli},\marginnote{3.1} there is a cause and reason for the corruption of sentient beings. Sentient beings are corrupted with cause and reason. There is a cause and reason for the purification of sentient beings. Sentient beings are purified with cause and reason.” 

“But\marginnote{4.1} sir, what is the cause and reason for the corruption of sentient beings? How are sentient beings corrupted with cause and reason?” 

“\textsanskrit{Mahāli},\marginnote{5.1} if form were exclusively painful—soaked and steeped in pain and not steeped in pleasure—sentient beings wouldn’t be aroused by it. But because form is pleasurable—soaked and steeped in pleasure and not steeped in pain—sentient beings do lust after it. Since they lust after it, they’re caught up in it, and so they become corrupted. This is a cause and reason for the corruption of sentient beings. This is how sentient beings are corrupted with cause and reason. 

If\marginnote{6.1} feeling … 

perception\marginnote{7.1} … 

choices\marginnote{7.2} … 

consciousness\marginnote{8.1} were exclusively painful—soaked and steeped in pain and not steeped in pleasure—sentient beings wouldn’t be aroused by it. But because consciousness is pleasurable—soaked and steeped in pleasure and not steeped in pain—sentient beings do lust after it. Since they lust after it, they’re caught up in it, and so they become corrupted. This is a cause and reason for the corruption of sentient beings. This is how sentient beings are corrupted with cause and reason.” 

“But\marginnote{9.1} sir, what is the cause and reason for the purification of sentient beings? How are sentient beings purified with cause and reason?” 

“\textsanskrit{Mahāli},\marginnote{9.3} if form were exclusively pleasurable—soaked and steeped in pleasure and not steeped in pain—sentient beings wouldn’t grow disillusioned with it. But because form is painful—soaked and steeped in pain and not steeped in pleasure—sentient beings do grow disillusioned with it. Being disillusioned, desire fades away. When desire fades away they are purified. This is a cause and reason for the purification of sentient beings. This is how sentient beings are purified with cause and reason. 

If\marginnote{10.1} feeling … 

perception\marginnote{10.2} … 

choices\marginnote{10.3} … 

consciousness\marginnote{10.4} were exclusively pleasurable—soaked and steeped in pleasure and not steeped in pain—sentient beings wouldn’t grow disillusioned with it. But because consciousness is painful—soaked and steeped in pain and not steeped in pleasure—sentient beings do grow disillusioned with it. Being disillusioned, desire fades away. When desire fades away they are purified. This is a cause and reason for the purification of sentient beings. This is how sentient beings are purified with cause and reason.” 

%
\section*{{\suttatitleacronym SN 22.61}{\suttatitletranslation Burning }{\suttatitleroot Ādittasutta}}
\addcontentsline{toc}{section}{\tocacronym{SN 22.61} \toctranslation{Burning } \tocroot{Ādittasutta}}
\markboth{Burning }{Ādittasutta}
\extramarks{SN 22.61}{SN 22.61}

At\marginnote{1.1} \textsanskrit{Sāvatthī}. 

“Mendicants,\marginnote{1.2} form, feeling, perception, choices, and consciousness are burning. 

Seeing\marginnote{1.3} this, a learned noble disciple grows disillusioned with form, feeling, perception, choices, and consciousness. Being disillusioned, desire fades away. When desire fades away they’re freed. When they’re freed, they know they’re freed. 

They\marginnote{1.5} understand: ‘Rebirth is ended, the spiritual journey has been completed, what had to be done has been done, there is no return to any state of existence.’” 

%
\section*{{\suttatitleacronym SN 22.62}{\suttatitletranslation The Scope of Language }{\suttatitleroot Niruttipathasutta}}
\addcontentsline{toc}{section}{\tocacronym{SN 22.62} \toctranslation{The Scope of Language } \tocroot{Niruttipathasutta}}
\markboth{The Scope of Language }{Niruttipathasutta}
\extramarks{SN 22.62}{SN 22.62}

At\marginnote{1.1} \textsanskrit{Sāvatthī}. 

“Mendicants,\marginnote{1.2} there are these three scopes of language, terminology, and descriptions. They’re uncorrupted, as they have been since the beginning. They’re not being corrupted now, nor will they be. Sensible ascetics and brahmins don’t look down on them.\footnote{Contra BB, the “patha” are not the five aggregates (there’s three of them!) but the three ways of using language, i.e. the three tenses. } What three? When form has passed, ceased, and perished, its designation, label, and description is ‘was’. It’s not ‘is’ or ‘will be’. 

When\marginnote{2.1} feeling … 

perception\marginnote{3.1} … 

choices\marginnote{3.2} … 

consciousness\marginnote{4.1} has passed, ceased, and perished, its designation, label, and description is ‘was’. It’s not ‘is’ or ‘will be’. 

When\marginnote{5.1} form is not yet born, and has not yet appeared, its designation, label, and description is ‘will be’. It’s not ‘is’ or ‘was’. 

When\marginnote{6.1} feeling … 

perception\marginnote{7.1} … 

choices\marginnote{7.2} … 

consciousness\marginnote{8.1} is not yet born, and has not yet appeared, its designation, label, and description is ‘will be’. It’s not ‘is’ or ‘was’. 

When\marginnote{9.1} form has been born, and has appeared, its designation, label, and description is ‘is’. It’s not ‘was’ or ‘will be’. 

When\marginnote{10.1} feeling … 

perception\marginnote{11.1} … 

choices\marginnote{11.2} … 

consciousness\marginnote{12.1} has been born, and has appeared, its designation, label, and description is ‘is’. It’s not ‘was’ or ‘will be’. 

These\marginnote{13.1} are the three scopes of language, terminology, and descriptions. They’re uncorrupted, as they have been since the beginning. They’re not being corrupted now, nor will they be. Sensible ascetics and brahmins don’t look down on them. 

Even\marginnote{13.2} those wanderers of the past, Vassa and \textsanskrit{Bhañña} of \textsanskrit{Ukkalā}, who taught the doctrines of no-cause, inaction, and nihilism, didn’t imagine that these three scopes of language should be criticized or rejected. Why is that? For fear of blame, attack, and condemnation.” 

%
\addtocontents{toc}{\let\protect\contentsline\protect\nopagecontentsline}
\chapter*{The Chapter on the Perfected Ones }
\addcontentsline{toc}{chapter}{\tocchapterline{The Chapter on the Perfected Ones }}
\addtocontents{toc}{\let\protect\contentsline\protect\oldcontentsline}

%
\section*{{\suttatitleacronym SN 22.63}{\suttatitletranslation When You Grasp }{\suttatitleroot Upādiyamānasutta}}
\addcontentsline{toc}{section}{\tocacronym{SN 22.63} \toctranslation{When You Grasp } \tocroot{Upādiyamānasutta}}
\markboth{When You Grasp }{Upādiyamānasutta}
\extramarks{SN 22.63}{SN 22.63}

\scevam{So\marginnote{1.1} I have heard. }At one time the Buddha was staying near \textsanskrit{Sāvatthī} in Jeta’s Grove, \textsanskrit{Anāthapiṇḍika}’s monastery. 

Then\marginnote{1.3} a mendicant went up to the Buddha, bowed, sat down to one side, and said to him, “Sir, may the Buddha please teach me Dhamma in brief. When I’ve heard it, I’ll live alone, withdrawn, diligent, keen, and resolute.” 

“When\marginnote{1.5} you grasp, mendicant, you’re bound by \textsanskrit{Māra}. Not grasping, you’re free from the Wicked One.” 

“Understood,\marginnote{1.7} Blessed One! Understood, Holy One!” 

“But\marginnote{2.1} how do you see the detailed meaning of my brief statement?” 

“Sir,\marginnote{2.2} when you grasp form you’re bound by \textsanskrit{Māra}. Not grasping, you’re free from the Wicked One. When you grasp feeling … perception … choices … consciousness, you’re bound by \textsanskrit{Māra}. Not grasping, you’re free from the Wicked One. 

That’s\marginnote{2.10} how I understand the detailed meaning of the Buddha’s brief statement.” 

“Good,\marginnote{3.1} good, mendicant! It’s good that you understand the detailed meaning of what I’ve said in brief like this. 

When\marginnote{3.3} you grasp form you’re bound by \textsanskrit{Māra}. Not grasping, you’re free from the Wicked One. When you grasp feeling … perception … choices … consciousness, you’re bound by \textsanskrit{Māra}. Not grasping, you’re free from the Wicked One. 

This\marginnote{3.10} is how to understand the detailed meaning of what I said in brief.” 

And\marginnote{4.1} then that mendicant approved and agreed with what the Buddha said. He got up from his seat, bowed, and respectfully circled the Buddha, keeping him on his right, before leaving. 

Then\marginnote{4.2} that mendicant, living alone, withdrawn, diligent, keen, and resolute, soon realized the supreme end of the spiritual path in this very life. He lived having achieved with his own insight the goal for which gentlemen rightly go forth from the lay life to homelessness. 

He\marginnote{4.3} understood: “Rebirth is ended; the spiritual journey has been completed; what had to be done has been done; there is no return to any state of existence.” And that mendicant became one of the perfected. 

%
\section*{{\suttatitleacronym SN 22.64}{\suttatitletranslation When You Identify }{\suttatitleroot Maññamānasutta}}
\addcontentsline{toc}{section}{\tocacronym{SN 22.64} \toctranslation{When You Identify } \tocroot{Maññamānasutta}}
\markboth{When You Identify }{Maññamānasutta}
\extramarks{SN 22.64}{SN 22.64}

At\marginnote{1.1} \textsanskrit{Sāvatthī}. 

Then\marginnote{1.2} a mendicant went up to the Buddha … and asked him, “Sir, may the Buddha please teach me Dhamma in brief. When I’ve heard it, I’ll live alone, withdrawn, diligent, keen, and resolute.” 

“When\marginnote{1.4} you identify, mendicant, you’re bound by \textsanskrit{Māra}.\footnote{\textsanskrit{Maññati} in such contexts means “think about in terms of a self”. Nyanamoli/Bodhi try to capture this with “conceive”, punning on “conceptualize/conceit”. But the pun is highly obscure without a detailed knowledge of the issues. Using “identify” I am trying to make the subtext clearer. } Not identifying, you’re free from the Wicked One.” 

“Understood,\marginnote{1.6} Blessed One! Understood, Holy One!” 

“But\marginnote{2.1} how do you see the detailed meaning of my brief statement?” 

“Sir,\marginnote{2.2} when you identify with form you’re bound by \textsanskrit{Māra}. Not identifying, you’re free from the Wicked One. When you identify with feeling … perception … choices … consciousness, you’re bound by \textsanskrit{Māra}. Not identifying, you’re free from the Wicked One. 

That’s\marginnote{2.9} how I understand the detailed meaning of the Buddha’s brief statement.” 

“Good,\marginnote{3.1} good, mendicant! It’s good that you understand the detailed meaning of what I’ve said in brief like this. 

When\marginnote{3.3} you identify with form you’re bound by \textsanskrit{Māra}. Not identifying, you’re free from the Wicked One. When you identify with feeling … perception … choices … consciousness, you’re bound by \textsanskrit{Māra}. Not identifying, you’re free from the Wicked One. 

This\marginnote{3.10} is how to understand the detailed meaning of what I said in brief.” … 

And\marginnote{3.11} that mendicant became one of the perfected. 

%
\section*{{\suttatitleacronym SN 22.65}{\suttatitletranslation When You Take Pleasure }{\suttatitleroot Abhinandamānasutta}}
\addcontentsline{toc}{section}{\tocacronym{SN 22.65} \toctranslation{When You Take Pleasure } \tocroot{Abhinandamānasutta}}
\markboth{When You Take Pleasure }{Abhinandamānasutta}
\extramarks{SN 22.65}{SN 22.65}

At\marginnote{1.1} \textsanskrit{Sāvatthī}. 

Then\marginnote{1.2} a mendicant went up to the Buddha … and asked him, “Sir, may the Buddha please teach me Dhamma in brief. When I’ve heard it, I’ll live alone, withdrawn, diligent, keen, and resolute.” 

“When\marginnote{1.4} you take pleasure, mendicant, you’re bound by \textsanskrit{Māra}. Not taking pleasure, you’re free from the Wicked One.” 

“Understood,\marginnote{1.6} Blessed One! Understood, Holy One!” 

“But\marginnote{2.1} how do you see the detailed meaning of my brief statement?” 

“Sir,\marginnote{2.2} when you take pleasure in form you’re bound by \textsanskrit{Māra}. Not taking pleasure, you’re free from the Wicked One. When you take pleasure in feeling … perception … choices … consciousness you’re bound by \textsanskrit{Māra}. Not taking pleasure, you’re free from the Wicked One. 

That’s\marginnote{2.9} how I understand the detailed meaning of the Buddha’s brief statement.” 

“Good,\marginnote{3.1} good, mendicant! It’s good that you understand the detailed meaning of what I’ve said in brief like this. 

When\marginnote{3.3} you take pleasure in form you’re bound by \textsanskrit{Māra}. Not taking pleasure, you’re free from the Wicked One. When you take pleasure in feeling … perception … choices … consciousness you’re bound by \textsanskrit{Māra}. Not taking pleasure, you’re free from the Wicked One. 

This\marginnote{3.10} is how to understand the detailed meaning of what I said in brief.” … 

And\marginnote{3.11} that mendicant became one of the perfected. 

%
\section*{{\suttatitleacronym SN 22.66}{\suttatitletranslation Impermanence }{\suttatitleroot Aniccasutta}}
\addcontentsline{toc}{section}{\tocacronym{SN 22.66} \toctranslation{Impermanence } \tocroot{Aniccasutta}}
\markboth{Impermanence }{Aniccasutta}
\extramarks{SN 22.66}{SN 22.66}

At\marginnote{1.1} \textsanskrit{Sāvatthī}. 

Then\marginnote{1.2} a mendicant went up to the Buddha … and asked him, “Sir, may the Buddha please teach me Dhamma in brief. When I’ve heard it, I’ll live alone, withdrawn, diligent, keen, and resolute.” 

“Mendicant,\marginnote{1.4} give up desire for anything that’s impermanent.” 

“Understood,\marginnote{1.5} Blessed One! Understood, Holy One!” 

“But\marginnote{2.1} how do you see the detailed meaning of my brief statement?” 

“Sir,\marginnote{2.2} form is impermanent; I should give up desire for it. 

Feeling\marginnote{2.3} … 

Perception\marginnote{2.4} … 

Choices\marginnote{2.5} … 

Consciousness\marginnote{2.6} is impermanent; I should give up desire for it. 

That’s\marginnote{2.7} how I understand the detailed meaning of the Buddha’s brief statement.” 

“Good,\marginnote{3.1} good, mendicant! It’s good that you understand the detailed meaning of what I’ve said in brief like this. 

Form\marginnote{3.3} is impermanent; you should give up desire for it. 

Feeling\marginnote{3.4} … 

Perception\marginnote{3.5} … 

Choices\marginnote{3.6} … 

Consciousness\marginnote{3.7} is impermanent; you should give up desire for it. 

This\marginnote{3.8} is how to understand the detailed meaning of what I said in brief.” … 

And\marginnote{3.9} that mendicant became one of the perfected. 

%
\section*{{\suttatitleacronym SN 22.67}{\suttatitletranslation Suffering }{\suttatitleroot Dukkhasutta}}
\addcontentsline{toc}{section}{\tocacronym{SN 22.67} \toctranslation{Suffering } \tocroot{Dukkhasutta}}
\markboth{Suffering }{Dukkhasutta}
\extramarks{SN 22.67}{SN 22.67}

At\marginnote{1.1} \textsanskrit{Sāvatthī}. 

Then\marginnote{1.2} a mendicant went up to the Buddha … and asked him, “Sir, may the Buddha please teach me Dhamma in brief. When I’ve heard it, I’ll live alone, withdrawn, diligent, keen, and resolute.” 

“Mendicant,\marginnote{1.4} give up desire for anything that’s suffering.” 

“Understood,\marginnote{1.5} Blessed One! Understood, Holy One!” 

“But\marginnote{2.1} how do you see the detailed meaning of my brief statement?” 

“Sir,\marginnote{2.2} form is suffering; I should give up desire for it. 

Feeling\marginnote{2.3} … 

Perception\marginnote{2.4} … 

Choices\marginnote{2.5} … 

Consciousness\marginnote{2.6} is suffering; I should give up desire for it. 

That’s\marginnote{2.7} how I understand the detailed meaning of the Buddha’s brief statement.” 

“Good,\marginnote{3.1} good, mendicant! It’s good that you understand the detailed meaning of what I’ve said in brief like this. 

Form\marginnote{3.3} is suffering; you should give up desire for it. 

Feeling\marginnote{3.4} … 

Perception\marginnote{3.5} … 

Choices\marginnote{3.6} … 

Consciousness\marginnote{3.7} is suffering; you should give up desire for it. 

This\marginnote{3.8} is how to understand the detailed meaning of what I said in brief.” … 

And\marginnote{3.9} that mendicant became one of the perfected. 

%
\section*{{\suttatitleacronym SN 22.68}{\suttatitletranslation Not-Self }{\suttatitleroot Anattasutta}}
\addcontentsline{toc}{section}{\tocacronym{SN 22.68} \toctranslation{Not-Self } \tocroot{Anattasutta}}
\markboth{Not-Self }{Anattasutta}
\extramarks{SN 22.68}{SN 22.68}

At\marginnote{1.1} \textsanskrit{Sāvatthī}. 

Then\marginnote{1.2} a mendicant went up to the Buddha … and asked him, “Sir, may the Buddha please teach me Dhamma in brief. When I’ve heard it, I’ll live alone, withdrawn, diligent, keen, and resolute.” 

“Mendicant,\marginnote{1.4} give up desire for what is not-self.” 

“Understood,\marginnote{1.5} Blessed One! Understood, Holy One!” 

“But\marginnote{2.1} how do you see the detailed meaning of my brief statement?” 

“Sir,\marginnote{2.2} form is not-self; I should give up desire for it. 

Feeling\marginnote{2.3} … 

Perception\marginnote{2.4} … 

Choices\marginnote{2.5} … 

Consciousness\marginnote{2.6} is not-self; I should give up desire for it. 

That’s\marginnote{2.7} how I understand the detailed meaning of the Buddha’s brief statement.” 

“Good,\marginnote{3.1} good, mendicant! It’s good that you understand the detailed meaning of what I’ve said in brief like this. 

Form\marginnote{3.3} is not-self; you should give up desire for it. 

Feeling\marginnote{3.4} … 

Perception\marginnote{3.5} … 

Choices\marginnote{3.6} … 

Consciousness\marginnote{3.7} is not-self; you should give up desire for it. 

This\marginnote{3.8} is how to understand the detailed meaning of what I said in brief.” … 

And\marginnote{3.9} that mendicant became one of the perfected. 

%
\section*{{\suttatitleacronym SN 22.69}{\suttatitletranslation Not Belonging to Self }{\suttatitleroot Anattaniyasutta}}
\addcontentsline{toc}{section}{\tocacronym{SN 22.69} \toctranslation{Not Belonging to Self } \tocroot{Anattaniyasutta}}
\markboth{Not Belonging to Self }{Anattaniyasutta}
\extramarks{SN 22.69}{SN 22.69}

At\marginnote{1.1} \textsanskrit{Sāvatthī}. 

Then\marginnote{1.2} a mendicant went up to the Buddha … and asked him, “Sir, may the Buddha please teach me Dhamma in brief. When I’ve heard it, I’ll live alone, withdrawn, diligent, keen, and resolute.” 

“Mendicant,\marginnote{1.4} give up desire for anything that doesn’t belong to self.” 

“Understood,\marginnote{1.5} Blessed One! Understood, Holy One!” 

“But\marginnote{2.1} how do you see the detailed meaning of my brief statement?” 

“Sir,\marginnote{2.2} form doesn’t belong to self; I should give up desire for it. 

Feeling\marginnote{2.3} … 

Perception\marginnote{2.4} … 

Choices\marginnote{2.5} … 

Consciousness\marginnote{2.6} doesn’t belong to self; I should give up desire for it. 

That’s\marginnote{2.7} how I understand the detailed meaning of the Buddha’s brief statement.” 

“Good,\marginnote{3.1} good, mendicant! It’s good that you understand the detailed meaning of what I’ve said in brief like this. 

Form\marginnote{3.3} doesn’t belong to self; you should give up desire for it. 

Feeling\marginnote{3.4} … 

Perception\marginnote{3.5} … 

Choices\marginnote{3.6} … 

Consciousness\marginnote{3.7} doesn’t belong to self; you should give up desire for it. 

This\marginnote{3.8} is how to understand the detailed meaning of what I said in brief.” … 

And\marginnote{3.9} that mendicant became one of the perfected. 

%
\section*{{\suttatitleacronym SN 22.70}{\suttatitletranslation Definitely Arousing }{\suttatitleroot Rajanīyasaṇṭhitasutta}}
\addcontentsline{toc}{section}{\tocacronym{SN 22.70} \toctranslation{Definitely Arousing } \tocroot{Rajanīyasaṇṭhitasutta}}
\markboth{Definitely Arousing }{Rajanīyasaṇṭhitasutta}
\extramarks{SN 22.70}{SN 22.70}

At\marginnote{1.1} \textsanskrit{Sāvatthī}. 

Then\marginnote{1.2} a mendicant went up to the Buddha … and asked him, “Sir, may the Buddha please teach me Dhamma in brief. When I’ve heard it, I’ll live alone, withdrawn, diligent, keen, and resolute.” 

“Mendicant,\marginnote{1.4} give up desire for anything that’s stuck in what’s arousing.”\footnote{Not sure why BB has “appears” here. Santhita is fairly rare as a suffix, but it clearly means “established in, firmly fixed in”. Could BB have misread it as sandittha? } 

“Understood,\marginnote{1.5} Blessed One! Understood, Holy One!” 

“But\marginnote{2.1} how do you see the detailed meaning of my brief statement?” 

“Sir,\marginnote{2.2} form is stuck in what’s arousing; I should give up desire for it. 

Feeling\marginnote{2.3} … 

Perception\marginnote{2.4} … 

Choices\marginnote{2.5} … 

Consciousness\marginnote{2.6} is stuck in what’s arousing; I should give up desire for it. 

That’s\marginnote{2.7} how I understand the detailed meaning of the Buddha’s brief statement.” 

“Good,\marginnote{3.1} good, mendicant! It’s good that you understand the detailed meaning of what I’ve said in brief like this. 

Form\marginnote{3.3} is stuck in what’s arousing; you should give up desire for it. 

Feeling\marginnote{3.4} … 

Perception\marginnote{3.5} … 

Choices\marginnote{3.6} … 

Consciousness\marginnote{3.7} is stuck in what’s arousing; you should give up desire for it. 

This\marginnote{3.8} is how to understand the detailed meaning of what I said in brief.” … 

And\marginnote{3.9} that mendicant became one of the perfected. 

%
\section*{{\suttatitleacronym SN 22.71}{\suttatitletranslation With Rādha }{\suttatitleroot Rādhasutta}}
\addcontentsline{toc}{section}{\tocacronym{SN 22.71} \toctranslation{With Rādha } \tocroot{Rādhasutta}}
\markboth{With Rādha }{Rādhasutta}
\extramarks{SN 22.71}{SN 22.71}

At\marginnote{1.1} \textsanskrit{Sāvatthī}. 

Then\marginnote{1.2} Venerable \textsanskrit{Rādha} went up to the Buddha … and asked him, “Sir, how does one know and see so that there’s no ego, possessiveness, or underlying tendency to conceit for this conscious body and all external stimuli?” 

“\textsanskrit{Rādha},\marginnote{1.4} one truly sees any kind of form at all—past, future, or present; internal or external; coarse or fine; inferior or superior; far or near: \emph{all} form—with right understanding: ‘This is not mine, I am not this, this is not my self.’ 

One\marginnote{1.5} truly sees any kind of feeling … perception … choices … consciousness at all—past, future, or present; internal or external; coarse or fine; inferior or superior; far or near: \emph{all} consciousness—with right understanding: ‘This is not mine, I am not this, this is not my self.’ 

That’s\marginnote{1.9} how to know and see so that there’s no ego, possessiveness, or underlying tendency to conceit for this conscious body and all external stimuli.” … 

And\marginnote{1.10} Venerable \textsanskrit{Rādha} became one of the perfected. 

%
\section*{{\suttatitleacronym SN 22.72}{\suttatitletranslation With Surādha }{\suttatitleroot Surādhasutta}}
\addcontentsline{toc}{section}{\tocacronym{SN 22.72} \toctranslation{With Surādha } \tocroot{Surādhasutta}}
\markboth{With Surādha }{Surādhasutta}
\extramarks{SN 22.72}{SN 22.72}

At\marginnote{1.1} \textsanskrit{Sāvatthī}. 

Then\marginnote{1.2} Venerable \textsanskrit{Surādha} said to the Buddha: 

“Sir,\marginnote{1.3} how does one know and see so that the mind is rid of ego, possessiveness, and conceit for this conscious body and all external stimuli; and going beyond discrimination, it’s peaceful and well freed?” 

“\textsanskrit{Surādha},\marginnote{1.4} one is freed by not grasping having truly seen any kind of form at all—past, future, or present; internal or external; coarse or fine; inferior or superior; far or near: \emph{all} form—with right understanding: ‘This is not mine, I am not this, this is not my self.’ 

One\marginnote{1.5} is freed by not grasping having truly seen any kind of feeling … perception … choices … consciousness at all—past, future, or present; internal or external; coarse or fine; inferior or superior; far or near: \emph{all} consciousness—with right understanding: ‘This is not mine, I am not this, this is not my self.’ \footnote{Handling of abbreviations here is corrupted. Not in PTS ed; this is found only in burmese edition. } 

That’s\marginnote{1.12} how to know and see so that the mind is rid of ego, possessiveness, and conceit for this conscious body and all external stimuli; and going beyond discrimination, it’s peaceful and well freed.” … 

And\marginnote{1.13} Venerable \textsanskrit{Surādha} became one of the perfected. 

%
\addtocontents{toc}{\let\protect\contentsline\protect\nopagecontentsline}
\chapter*{The Chapter on Itchy }
\addcontentsline{toc}{chapter}{\tocchapterline{The Chapter on Itchy }}
\addtocontents{toc}{\let\protect\contentsline\protect\oldcontentsline}

%
\section*{{\suttatitleacronym SN 22.73}{\suttatitletranslation Gratification }{\suttatitleroot Assādasutta}}
\addcontentsline{toc}{section}{\tocacronym{SN 22.73} \toctranslation{Gratification } \tocroot{Assādasutta}}
\markboth{Gratification }{Assādasutta}
\extramarks{SN 22.73}{SN 22.73}

At\marginnote{1.1} \textsanskrit{Sāvatthī}. 

“Mendicants,\marginnote{1.2} an unlearned ordinary person doesn’t truly understand the gratification, the drawback, and the escape when it comes to form, feeling, perception, choices, and consciousness. A learned noble disciple does truly understand the gratification, the drawback, and the escape when it comes to form, feeling, perception, choices, and consciousness.” 

%
\section*{{\suttatitleacronym SN 22.74}{\suttatitletranslation Origin }{\suttatitleroot Samudayasutta}}
\addcontentsline{toc}{section}{\tocacronym{SN 22.74} \toctranslation{Origin } \tocroot{Samudayasutta}}
\markboth{Origin }{Samudayasutta}
\extramarks{SN 22.74}{SN 22.74}

At\marginnote{1.1} \textsanskrit{Sāvatthī}. 

“Mendicants,\marginnote{1.2} an unlearned ordinary person doesn’t truly understand the origin, the ending, the gratification, the drawback, and the escape when it comes to form, feeling, perception, choices, and consciousness. A learned noble disciple does truly understand the origin, the ending, the gratification, the drawback, and the escape when it comes to form, feeling, perception, choices, and consciousness.” 

%
\section*{{\suttatitleacronym SN 22.75}{\suttatitletranslation Origin (2nd) }{\suttatitleroot Dutiyasamudayasutta}}
\addcontentsline{toc}{section}{\tocacronym{SN 22.75} \toctranslation{Origin (2nd) } \tocroot{Dutiyasamudayasutta}}
\markboth{Origin (2nd) }{Dutiyasamudayasutta}
\extramarks{SN 22.75}{SN 22.75}

At\marginnote{1.1} \textsanskrit{Sāvatthī}. 

“Mendicants,\marginnote{1.2} a learned noble disciple truly understands the origin, the ending, the gratification, the drawback, and the escape when it comes to form, feeling, perception, choices, and consciousness.” 

%
\section*{{\suttatitleacronym SN 22.76}{\suttatitletranslation The Perfected Ones }{\suttatitleroot Arahantasutta}}
\addcontentsline{toc}{section}{\tocacronym{SN 22.76} \toctranslation{The Perfected Ones } \tocroot{Arahantasutta}}
\markboth{The Perfected Ones }{Arahantasutta}
\extramarks{SN 22.76}{SN 22.76}

At\marginnote{1.1} \textsanskrit{Sāvatthī}. 

“Mendicants,\marginnote{1.2} form is impermanent. What’s impermanent is suffering. What’s suffering is not-self. And what’s not-self should be truly seen with right understanding like this: ‘This is not mine, I am not this, this is not my self.’ 

Feeling\marginnote{1.6} … 

Perception\marginnote{1.7} … 

Choices\marginnote{1.8} … 

Consciousness\marginnote{1.9} is impermanent. What’s impermanent is suffering. What’s suffering is not-self. And what’s not-self should be truly seen with right understanding like this: ‘This is not mine, I am not this, this is not my self.’ 

Seeing\marginnote{2.1} this, a learned noble disciple grows disillusioned with form, feeling, perception, choices, and consciousness. Being disillusioned, desire fades away. When desire fades away they’re freed. When they’re freed, they know they’re freed. 

They\marginnote{2.3} understand: ‘Rebirth is ended, the spiritual journey has been completed, what had to be done has been done, there is no return to any state of existence.’ As far as there are abodes of sentient beings, even up until the pinnacle of existence, the perfected ones are the foremost and the best.” 

That\marginnote{3.1} is what the Buddha said. Then the Holy One, the Teacher, went on to say: 

\begin{verse}%
“Oh!\marginnote{4.1} How happy are the perfected ones! \\
Craving is not found in them, \\
the conceit ‘I am’ is cut off, \\
and the net of delusion is shattered. 

They’ve\marginnote{5.1} attained imperturbability, \\
their minds are unclouded, \\
nothing in the world clings to them, \\
they’ve become holy, undefiled. 

Completely\marginnote{6.1} understanding the five aggregates, \\
their domain is the seven good qualities. \\
Those good people are praiseworthy, \\
the Buddha’s rightful children. 

Endowed\marginnote{7.1} with the seven gems, \\
and trained in the three trainings, \\
the great heroes live on, \\
with fear and dread given up. 

Endowed\marginnote{8.1} with ten factors, \\
those giants have immersion. \\
These are the best in the world, \\
craving is not found in them. 

The\marginnote{9.1} master’s knowledge has arisen: \\
‘This bag of bones is my last.’ \\
They are independent of others \\
in the core of the spiritual path. 

Unwavering\marginnote{10.1} in the face of discrimination, \\
they’re freed from future lives. \\
They’ve reached the level of the tamed, \\
in the world, they’re the winners. 

Above,\marginnote{11.1} below, all round, \\
relishing is not found in them. \\
They roar their lion’s roar: \\
‘The awakened are supreme in the world!’” 

%
\end{verse}

%
\section*{{\suttatitleacronym SN 22.77}{\suttatitletranslation The Perfected Ones (2nd) }{\suttatitleroot Dutiyaarahantasutta}}
\addcontentsline{toc}{section}{\tocacronym{SN 22.77} \toctranslation{The Perfected Ones (2nd) } \tocroot{Dutiyaarahantasutta}}
\markboth{The Perfected Ones (2nd) }{Dutiyaarahantasutta}
\extramarks{SN 22.77}{SN 22.77}

At\marginnote{1.1} \textsanskrit{Sāvatthī}. 

“Mendicants,\marginnote{1.2} form is impermanent. What’s impermanent is suffering. What’s suffering is not-self. And what’s not-self should be truly seen with right understanding like this: ‘This is not mine, I am not this, this is not my self.’ 

Seeing\marginnote{2.1} this, a learned noble disciple grows disillusioned with form, feeling, perception, choices, and consciousness. Being disillusioned, desire fades away. When desire fades away they’re freed. When they’re freed, they know they’re freed. 

They\marginnote{2.3} understand: ‘Rebirth is ended, the spiritual journey has been completed, what had to be done has been done, there is no return to any state of existence.’ 

As\marginnote{2.4} far as there are abodes of sentient beings, even up until the pinnacle of existence, the perfected ones are the foremost and the best.” 

%
\section*{{\suttatitleacronym SN 22.78}{\suttatitletranslation The Lion }{\suttatitleroot Sīhasutta}}
\addcontentsline{toc}{section}{\tocacronym{SN 22.78} \toctranslation{The Lion } \tocroot{Sīhasutta}}
\markboth{The Lion }{Sīhasutta}
\extramarks{SN 22.78}{SN 22.78}

At\marginnote{1.1} \textsanskrit{Sāvatthī}. 

“Mendicants,\marginnote{1.2} towards evening the lion, king of beasts, emerges from his den, yawns, looks all around the four quarters, and roars his lion’s roar three times. Then he sets out on the hunt. And whatever animals hear the roar of the lion, king of beasts, are typically filled with fear, awe, and terror. They return to their lairs, be they in a hole, the water, or a wood; and the birds take to the air. Even the royal elephants, bound with strong harness in the villages, towns, and capital cities, break apart their bonds, and urinate and defecate in terror as they flee here and there. That’s how powerful is the lion, king of beasts, among animals, how illustrious and mighty. 

In\marginnote{2.1} the same way, when a Realized One arises in the world—perfected, a fully awakened Buddha, accomplished in knowledge and conduct, holy, knower of the world, supreme guide for those who wish to train, teacher of gods and humans, awakened, blessed—he teaches the Dhamma: ‘Such is form, such is the origin of form, such is the ending of form. Such is feeling … Such is perception … Such are choices … Such is consciousness, such is the origin of consciousness, such is the ending of consciousness.’ 

Now,\marginnote{2.7} there are gods who are long-lived, beautiful, and very happy, living for ages in their divine palaces. When they hear this teaching by the Realized One, they’re typically filled with fear, awe, and terror. ‘Oh no! It turns out we’re impermanent, though we thought we were permanent! It turns out we don’t last, though we thought we were everlasting! It turns out we’re short-lived, though we thought we were eternal! It turns out that we’re impermanent, not lasting, short-lived, and included within identity.’ That’s how powerful is the Realized One in the world with its gods, how illustrious and mighty.” 

That\marginnote{2.13} is what the Buddha said. Then the Holy One, the Teacher, went on to say: 

\begin{verse}%
“The\marginnote{3.1} Buddha, the teacher without a peer \\
in all the world with its gods, \\
rolls forth the Wheel of Dhamma \\
from his own insight: 

identity,\marginnote{4.1} its cessation, \\
the origin of identity, \\
and the noble eightfold path \\
that leads to the stilling of suffering. 

And\marginnote{5.1} then the long-lived gods, \\
so beautiful and glorious, \\
are afraid and full of terror, \\
like the other beasts when they hear a lion. 

‘We\marginnote{6.1} haven’t transcended identity! \\
It turns out we’re impermanent!’ \\
So they say when they hear the word \\
of the perfected one, free and poised.” 

%
\end{verse}

%
\section*{{\suttatitleacronym SN 22.79}{\suttatitletranslation Itchy }{\suttatitleroot Khajjanīyasutta}}
\addcontentsline{toc}{section}{\tocacronym{SN 22.79} \toctranslation{Itchy } \tocroot{Khajjanīyasutta}}
\markboth{Itchy }{Khajjanīyasutta}
\extramarks{SN 22.79}{SN 22.79}

At\marginnote{1.1} \textsanskrit{Sāvatthī}. 

“Mendicants,\marginnote{1.2} whatever ascetics and brahmins recollect many kinds of past lives, all recollect the five grasping aggregates, or one of them. What five? ‘I had such form in the past.’ Recollecting thus, it’s only form that they recollect. ‘I had such feeling … perception … choices … consciousness in the past.’ Recollecting thus, it’s only consciousness that they recollect. 

And\marginnote{2.1} why do you call it form? It’s deformed; that’s why it’s called ‘form’. Deformed by what? Deformed by cold, heat, hunger, and thirst, and deformed by the touch of flies, mosquitoes, wind, sun, and reptiles. It’s deformed; that’s why it’s called ‘form’. 

And\marginnote{3.1} why do you call it feeling? It feels; that’s why it’s called ‘feeling’. And what does it feel? It feels pleasure, pain, and neutral. It feels; that’s why it’s called ‘feeling’. 

And\marginnote{4.1} why do you call it perception? It perceives; that’s why it’s called ‘perception’. And what does it perceive? It perceives blue, yellow, red, and white. It perceives; that’s why it’s called ‘perception’. 

And\marginnote{5.1} why do you call them choices? Choices produce conditioned phenomena; that’s why they’re called ‘choices’.\footnote{Good luck translating this. } And what are the conditioned phenomena that they produce? Form is a conditioned phenomenon; choices are what make it into form. Feeling is a conditioned phenomenon; choices are what make it into feeling. Perception is a conditioned phenomenon; choices are what make it into perception. Choices are conditioned phenomena; choices are what make them into choices. Consciousness is a conditioned phenomenon; choices are what make it into consciousness. Choices produce conditioned phenomena; that’s why they’re called ‘choices’. 

And\marginnote{6.1} why do you call it consciousness? It cognizes; that’s why it’s called ‘consciousness’. And what does it cognize? It cognizes sour, bitter, pungent, sweet, hot, mild, salty, and bland.\footnote{kharika is to the root ksayati, to burn, thus = “hot”, not “sharp” (BB) or “alkaline” (Thanissaro) } It cognizes; that’s why it’s called ‘consciousness’. 

A\marginnote{7.1} learned noble disciple reflects on this: ‘Currently I’m itched by form.\footnote{Khajja is from the root to eat. But it also has the idiomatic meaning “to itch” https://suttacentral.net/define/khajjati The English word “itch” is likewise derived from “eat”. In this case, I’m not clear on what it means to say these things “devour” you. The comm explains it as “itch” (see BB’s note) and for once I’m inclined to agree with them. At least it gives a straightforward meaning. It’s parallel with “tormented”. } In the past I was also itched by form just like now. If I were to look forward to enjoying form in the future, I’d be itched by form in the future just as I am today.’ Reflecting like this they don’t worry about past form, they don’t look forward to enjoying future form, and they practice for disillusionment, dispassion, and cessation regarding present form. 

‘Currently\marginnote{8.1} I’m itched by feeling … 

perception\marginnote{9.1} … 

choices\marginnote{9.2} … 

consciousness.\marginnote{10.1} In the past I was also itched by consciousness just like now. If I were to look forward to enjoying consciousness in the future, I’d be itched by consciousness in the future just as I am today.’ Reflecting like this they don’t worry about past consciousness, they don’t look forward to enjoying future consciousness, and they practice for disillusionment, dispassion, and cessation regarding present consciousness. 

What\marginnote{11.1} do you think, mendicants? Is form permanent or impermanent?” 

“Impermanent,\marginnote{11.3} sir.” 

“But\marginnote{11.4} if it’s impermanent, is it suffering or happiness?” 

“Suffering,\marginnote{11.5} sir.” 

“But\marginnote{11.6} if it’s impermanent, suffering, and perishable, is it fit to be regarded thus: ‘This is mine, I am this, this is my self’?” 

“No,\marginnote{11.8} sir.” 

“Is\marginnote{11.9} feeling … perception … choices … consciousness permanent or impermanent?” 

“Impermanent,\marginnote{11.13} sir.” 

“But\marginnote{11.14} if it’s impermanent, is it suffering or happiness?” 

“Suffering,\marginnote{11.15} sir.” 

“But\marginnote{11.16} if it’s impermanent, suffering, and perishable, is it fit to be regarded thus: ‘This is mine, I am this, this is my self’?” 

“No,\marginnote{11.18} sir.” 

“So\marginnote{11.19} you should truly see any kind of form at all—past, future, or present; internal or external; coarse or fine; inferior or superior; far or near: \emph{all} form—with right understanding: ‘This is not mine, I am not this, this is not my self.’ You should truly see any kind of feeling … perception … choices … consciousness at all—past, future, or present; internal or external; coarse or fine; inferior or superior; far or near: \emph{all} consciousness—with right understanding: ‘This is not mine, I am not this, this is not my self.’ 

This\marginnote{12.1} is called a noble disciple who gets rid of things and doesn’t accumulate them; who gives things up and doesn’t grasp at them; who discards things and doesn’t amass them; who dissipates things and doesn’t get clouded by them.\footnote{The idiom is dispersing a cloud or steam or smoke, as opposed to creating a cloud of smoke, etc. } 

And\marginnote{12.5} what things do they get rid of and not accumulate? They get rid of form and don’t accumulate it. They get rid of feeling … perception … choices … consciousness and don’t accumulate it. 

And\marginnote{12.11} what things do they give up and not grasp? They give up form and don’t grasp it. They give up feeling … perception … choices … consciousness and don’t grasp it. 

And\marginnote{12.17} what things do they discard and not amass? They discard form and don’t amass it. They discard feeling … perception … choices … consciousness and don’t amass it. 

And\marginnote{12.23} what things do they dissipate and not get clouded by? They dissipate form and don’t get clouded by it. They dissipate feeling … perception … choices … consciousness and don’t get clouded by it. 

Seeing\marginnote{13.1} this, a learned noble disciple grows disillusioned with form, feeling, perception, choices, and consciousness. Being disillusioned, desire fades away. When desire fades away they’re freed. When they’re freed, they know they’re freed. 

They\marginnote{13.3} understand: ‘Rebirth is ended, the spiritual journey has been completed, what had to be done has been done, there is no return to any state of existence.’ 

This\marginnote{14.1} is called a mendicant who neither gets rid of things nor accumulates them, but remains after getting rid of them. They neither give things up nor grasp them, but remain after giving them up. They neither discard things nor amass them, but remain after discarding them. They neither dissipate things nor get clouded by them, but remain after dissipating them. 

And\marginnote{14.2} what things do they neither get rid of nor accumulate, but remain after getting rid of them?\footnote{MS punctuation in this passage is wrong, and I have corrected it in the Pali. } They neither get rid of nor accumulate form, but remain after getting rid of it. They neither get rid of nor accumulate feeling … perception … choices … consciousness, but remain after getting rid of it. 

And\marginnote{14.8} what things do they neither give up nor grasp, but remain after giving them up? They neither give up nor grasp form, but remain after giving it up. They neither give up nor grasp feeling … perception … choices … consciousness, but remain after giving it up. 

And\marginnote{14.14} what things do they neither discard nor amass, but remain after discarding them? They neither discard nor amass form, but remain after discarding it. They neither discard nor amass feeling … perception … choices … consciousness, but remain after discarding it. 

And\marginnote{14.20} what things do they neither dissipate nor get clouded by, but remain after dissipating them? They neither dissipate nor get clouded by form, but remain after dissipating it. They neither dissipate nor get clouded by feeling … perception … choices … consciousness, but remain after dissipating it. 

When\marginnote{14.26} a mendicant’s mind is freed like this, the gods together with Indra, \textsanskrit{Brahmā}, and \textsanskrit{Pajāpati} worship them from afar: 

\begin{verse}%
‘Homage\marginnote{15.1} to you, O thoroughbred! \\
Homage to you, supreme among men! \\
We don’t understand \\
the basis of your absorption.’” 

%
\end{verse}

%
\section*{{\suttatitleacronym SN 22.80}{\suttatitletranslation Beggars }{\suttatitleroot Piṇḍolyasutta}}
\addcontentsline{toc}{section}{\tocacronym{SN 22.80} \toctranslation{Beggars } \tocroot{Piṇḍolyasutta}}
\markboth{Beggars }{Piṇḍolyasutta}
\extramarks{SN 22.80}{SN 22.80}

At\marginnote{1.1} one time the Buddha was staying in the land of the Sakyans, near Kapilavatthu in the Banyan Tree Monastery. 

Then\marginnote{1.2} the Buddha, having dismissed the mendicant \textsanskrit{Saṅgha} for some reason, robed up in the morning and, taking his bowl and robe, entered Kapilavatthu for alms. He wandered for alms in Kapilavatthu. After the meal, on his return from almsround, he went to the Great Wood, plunged deep into it, and sat at the root of a young wood apple tree for the day’s meditation. 

Then\marginnote{2.1} as he was in private retreat this thought came to his mind, “I’ve sent the mendicant \textsanskrit{Saṅgha} away. But there are mendicants here who are junior, recently gone forth, newly come to this teaching and training. Not seeing me they may change and fall apart. If a young calf doesn’t see its mother it may change and fall apart. … Or if young seedlings don’t get water they may change and fall apart. In the same way, there are mendicants here who are junior, recently gone forth, newly come to this teaching and training. Not seeing me they may change and fall apart. Why don’t I support the mendicant \textsanskrit{Saṅgha} now as I did in the past?” 

Then\marginnote{3.1} \textsanskrit{Brahmā} Sahampati knew what the Buddha was thinking. As easily as a strong person would extend or contract their arm, he vanished from the \textsanskrit{Brahmā} realm and reappeared in front of the Buddha. He arranged his robe over one shoulder, raised his joined palms toward the Buddha, and said: “That’s so true, Blessed One! That’s so true, Holy One! The Buddha has sent the mendicant \textsanskrit{Saṅgha} away. But there are mendicants who are junior, recently gone forth, newly come to this teaching and training. … May the Buddha be happy with the mendicant \textsanskrit{Saṅgha}! May the Buddha welcome the mendicant \textsanskrit{Saṅgha}! May the Buddha support the mendicant \textsanskrit{Saṅgha} now as he did in the past!” 

The\marginnote{4.1} Buddha consented in silence. Then \textsanskrit{Brahmā} Sahampati, knowing that the Buddha had consented, bowed, and respectfully circled the Buddha, keeping him on his right, before vanishing right there. 

Then\marginnote{5.1} in the late afternoon, the Buddha came out of retreat and went to the Banyan Tree Monastery, where he sat on the seat spread out. Then he used his psychic power to will that the mendicants would come to him timidly, alone or in pairs. Those mendicants approached the Buddha timidly, bowed, and sat down to one side. The Buddha said to them: 

“Mendicants,\marginnote{6.1} this relying on alms is an extreme lifestyle.\footnote{Not sure about BB’s “lowest” for anta here. } The world curses you: ‘You beggar, walking bowl in hand!’\footnote{I found BB’s note here confusing. For the record, \textsanskrit{kapāla} is not found in the text, but in the commentary. } Yet earnest gentlemen take it up for a good reason.\footnote{The meaning of atthavasika is clarified at https://suttacentral.net/pi/pi-tv-pvr14. A mendicant who is approaching the sangha with some litigation should do so, maong a long list of reasons: atthavasikena \textsanskrit{bhavitabbaṁ} no parisakappikena. To paraphrase, this seems to mean “looking for a genuinely good outcome for all concerned, not trying to get excused by the assembly”. I choose “earnest” as an English term with a similar sense. } Not because they’ve been forced to by kings or bandits, or because they’re in debt or threatened, or to earn a living. But because they’re swamped by rebirth, old age, and death; by sorrow, lamentation, pain, sadness, and distress. They’re swamped by suffering, mired in suffering. And they think, ‘Hopefully I can find an end to this entire mass of suffering.’ 

That’s\marginnote{7.1} how this gentleman has gone forth. Yet they covet sensual pleasures; they’re infatuated, full of ill will and malicious intent. They are unmindful, lacking situational awareness and immersion, with straying mind and undisciplined faculties. Suppose there was a firebrand for lighting a funeral pyre, burning at both ends, and smeared with dung in the middle. It couldn’t be used as timber either in the village or the wilderness. I say that person is just like this. They’ve missed out on the pleasures of the lay life, and haven’t fulfilled the goal of the ascetic life. 

There\marginnote{8.1} are these three unskillful thoughts. Sensual, malicious, and cruel thoughts. And where do these three unskillful thoughts cease without anything left over? In those who meditate with their mind firmly established in the four kinds of mindfulness meditation; or who develop signless immersion. Just this much is quite enough motivation to develop signless immersion. When signless immersion is developed and cultivated it is very fruitful and beneficial. 

There\marginnote{9.1} are these two views. Views favoring continued existence and views favoring ending existence. A learned noble disciple reflects on this: ‘Is there anything in the world that I could grasp without fault?’ They understand: ‘There’s nothing in the world that I could grasp without fault. For in grasping I would grasp only at form, feeling, perception, choices, or consciousness. That grasping of mine would be a condition for continued existence. Continued existence is a condition for rebirth. Rebirth is a condition for old age and death, sorrow, lamentation, pain, sadness, and distress to come to be. That is how this entire mass of suffering originates. 

What\marginnote{10.1} do you think, mendicants? Is form permanent or impermanent?” 

“Impermanent,\marginnote{10.3} sir.” 

“But\marginnote{10.4} if it’s impermanent, is it suffering or happiness?” 

“Suffering,\marginnote{10.5} sir.” 

“But\marginnote{10.6} if it’s impermanent, suffering, and perishable, is it fit to be regarded thus: ‘This is mine, I am this, this is my self’?” 

“No,\marginnote{10.8} sir.” 

“Is\marginnote{10.9} feeling … perception … choices … consciousness permanent or impermanent?” … 

“So\marginnote{10.13} you should truly see … 

Seeing\marginnote{10.14} this … They understand: ‘… there is no return to any state of existence.’” 

%
\section*{{\suttatitleacronym SN 22.81}{\suttatitletranslation At Pārileyya }{\suttatitleroot Pālileyyasutta}}
\addcontentsline{toc}{section}{\tocacronym{SN 22.81} \toctranslation{At Pārileyya } \tocroot{Pālileyyasutta}}
\markboth{At Pārileyya }{Pālileyyasutta}
\extramarks{SN 22.81}{SN 22.81}

At\marginnote{1.1} one time the Buddha was staying near Kosambi, in Ghosita’s Monastery. 

Then\marginnote{1.2} the Buddha robed up in the morning and, taking his bowl and robe, entered Kosambi for alms. After the meal, on his return from almsround, he set his lodgings in order himself. Taking his bowl and robe, without informing his attendants or taking leave of the mendicant \textsanskrit{Saṅgha}, he set out to go wandering alone, with no companion. 

Then,\marginnote{2.1} not long after the Buddha had left, one of the mendicants went to Venerable Ānanda and told him what had happened. Ānanda said, “Reverend, when the Buddha leaves like this it means he wants to stay alone. At this time no-one should follow him.” 

Then\marginnote{3.1} the Buddha, traveling stage by stage, arrived at \textsanskrit{Pārileyya}, where he stayed at the root of a sacred sal tree. Then several mendicants went up to Venerable Ānanda and exchanged greetings with him. 

When\marginnote{3.4} the greetings and polite conversation were over, they sat down to one side and said to him, “Reverend, it’s been a long time since we’ve heard a Dhamma talk from the Buddha. We wish to hear a Dhamma talk from the Buddha.” 

Then\marginnote{4.1} Venerable Ānanda together with those mendicants went to \textsanskrit{Pārileyya} to see the Buddha. They bowed and sat down to one side, and the Buddha educated, encouraged, fired up, and inspired them with a Dhamma talk. 

Now\marginnote{4.3} at that time one of the monks had the thought, “How do you know and see in order to end the defilements in the present life?” 

Then\marginnote{4.5} the Buddha, knowing what that monk was thinking, addressed the mendicants: 

“Mendicants,\marginnote{4.6} I’ve taught the Dhamma analytically. I’ve analytically taught the four kinds of mindfulness meditation, the four right efforts, the four bases of psychic power, the five faculties, the five powers, the seven awakening factors, and the noble eightfold path. That’s how I’ve taught the Dhamma analytically. Though I’ve taught the Dhamma analytically, still a certain mendicant present here has this thought:\footnote{Contra BB, I read this as a locative absolute. The context, and the panidhekaccassa, suggest the sense of “though, despite”. It seems a bit harsh, but I guess the Buddha was in a bad mood. Unless we take this sense, the structure of the Buddha’s answer is hard to account for. } ‘How do you know and see in order to end the defilements in the present life?’ 

And\marginnote{5.1} how, mendicants, do you know and see in order to end the defilements in the present life? Take an unlearned ordinary person who has not seen the noble ones, and is neither skilled nor trained in the teaching of the noble ones. They’ve not seen good persons, and are neither skilled nor trained in the teaching of the good persons. 

They\marginnote{5.3} regard form as self. But that regarding is just a conditioned phenomenon. And what’s the source, origin, birthplace, and inception of that conditioned phenomenon? When an unlearned ordinary person is struck by feelings born of contact with ignorance, craving arises. That conditioned phenomenon is born from that. So that conditioned phenomenon is impermanent, conditioned, and dependently originated. And that craving, that feeling, that contact,\footnote{The syntax here suggests that vedana and phassa are to be taken together. But the later passages show that this is just an artifact of abbreviation. } and that ignorance are also impermanent, conditioned, and dependently originated. That’s how you should know and see in order to end the defilements in the present life. 

Perhaps\marginnote{6.1} they don’t regard form as self, but they still regard self as possessing form. But that regarding is just a conditioned phenomenon. … 

Perhaps\marginnote{7.1} they don’t regard form as self, or self as possessing form, but they still regard form in self. But that regarding is just a conditioned phenomenon. … 

Perhaps\marginnote{8.1} they don’t regard form as self, or self as possessing form, or form in self, but they still regard self in form. But that regarding is just a conditioned phenomenon. … 

Perhaps\marginnote{9.1} they don’t regard form as self, or self as possessing form, or form in self, or self in form. But they regard feeling as self … perception as self … choices as self … consciousness as self … But that regarding is just a conditioned phenomenon. And what’s the source of that conditioned phenomenon? When an unlearned ordinary person is struck by feelings born of contact with ignorance, craving arises. That conditioned phenomenon is born from that. So that conditioned phenomenon is impermanent, conditioned, and dependently originated. And that craving, that feeling, that contact, and that ignorance are also impermanent, conditioned, and dependently originated. That’s how you should know and see in order to end the defilements in the present life. 

Perhaps\marginnote{10.1} they don’t regard form or feeling or perception or choices or consciousness as self. Still, they have such a view: ‘The self and the cosmos are one and the same. After passing away I will be permanent, everlasting, eternal, and imperishable.’\footnote{This kind of double demonstrative construction is an emphatic identity. This is in fact the Upanishadic thesis. } But that eternalist view is just a conditioned phenomenon. And what’s the source of that conditioned phenomenon? … That’s how you should know and see in order to end the defilements in the present life. 

Perhaps\marginnote{11.1} they don’t regard form or feeling or perception or choices or consciousness as self. Nor do they have such a view: ‘The self and the cosmos are one and the same. After passing away I will be permanent, everlasting, eternal, and imperishable.’ Still, they have such a view: ‘I might not be, and it might not be mine. I will not be, and it will not be mine.’ But that annihilationist view is just a conditioned phenomenon. And what’s the source of that conditioned phenomenon? … That’s how you should know and see in order to end the defilements in the present life. 

Perhaps\marginnote{12.1} they don’t regard form or feeling or perception or choices or consciousness as self. Nor do they have such a view: ‘The self and the cosmos are one and the same. After passing away I will be permanent, everlasting, eternal, and imperishable.’ Nor do they have such a view: ‘I might not be, and it might not be mine. I will not be, and it will not be mine.’ Still, they have doubts and uncertainties. They’re undecided about the true teaching. That doubt and uncertainty, the indecision about the true teaching, is just a conditioned phenomenon. And what’s the source of that conditioned phenomenon? When an unlearned ordinary person is struck by feelings born of contact with ignorance, craving arises. That conditioned phenomenon is born from that. So that conditioned phenomenon is impermanent, conditioned, and dependently originated. And that craving, that feeling, that contact, and that ignorance are also impermanent, conditioned, and dependently originated. That’s how you should know and see in order to end the defilements in the present life.” 

%
\section*{{\suttatitleacronym SN 22.82}{\suttatitletranslation A Full Moon Night }{\suttatitleroot Puṇṇamasutta}}
\addcontentsline{toc}{section}{\tocacronym{SN 22.82} \toctranslation{A Full Moon Night } \tocroot{Puṇṇamasutta}}
\markboth{A Full Moon Night }{Puṇṇamasutta}
\extramarks{SN 22.82}{SN 22.82}

At\marginnote{1.1} one time the Buddha was staying near \textsanskrit{Sāvatthī} in the Eastern Monastery, the stilt longhouse of \textsanskrit{Migāra}’s mother, together with a large \textsanskrit{Saṅgha} of mendicants. Now, at that time it was the sabbath—the full moon on the fifteenth day—and the Buddha was sitting in the open surrounded by the \textsanskrit{Saṅgha} of monks. 

Then\marginnote{2.1} one of the mendicants got up from their seat, arranged their robe over one shoulder, raised their joined palms toward the Buddha, and said: 

“Sir,\marginnote{2.2} I’d like to ask the Buddha about a certain point, if you’d take the time to answer.” 

“Well\marginnote{2.3} then, mendicant, take your own seat and ask what you wish.” 

“Yes,\marginnote{2.4} sir,” replied that mendicant. He took his seat and said to the Buddha: 

“Sir,\marginnote{2.5} are these the five grasping aggregates, that is: form, feeling, perception, choices, and consciousness?” 

“Yes,\marginnote{3.1} they are,” replied the Buddha. 

Saying\marginnote{3.3} “Good, sir”, that mendicant approved and agreed with what the Buddha said. Then he asked another question: 

“But\marginnote{4.1} sir, what is the root of these five grasping aggregates?” 

“These\marginnote{4.2} five grasping aggregates are rooted in desire.” …\footnote{It looks like a sadhu passage has been elided here. Indeed, in the parallel at MN 109, this sadhu passage is omitted, and so are the ones following, while here, only this one is omitted. } “But sir, is that grasping the exact same thing as the five grasping aggregates? Or is grasping one thing and the five grasping aggregates another?” 

“Neither.\marginnote{4.4} Rather, the desire and greed for them is the grasping there.” 

Saying\marginnote{4.5} “Good, sir”, that mendicant asked another question: 

“But\marginnote{5.1} sir, can there be different kinds of desire and greed for the five grasping aggregates?” 

“There\marginnote{5.2} can,” said the Buddha. 

“It’s\marginnote{5.3} when someone thinks: ‘In the future, may I be of such form, such feeling, such perception, such choices, or such consciousness!’ That’s how there can be different kinds of desire and greed for the five grasping aggregates.”\footnote{MS question mark here is incorrect. } 

Saying\marginnote{5.6} “Good, sir”, that mendicant asked another question: 

“Sir,\marginnote{6.1} what is the scope of the term ‘aggregates’ as applied to the aggregates?”\footnote{BB’s “in what way” is not quite right here. kittavata means “to what extent” and is used to define or delineate. The question is whether there’s anything outside the khandhas. } 

“Any\marginnote{6.2} kind of form at all—past, future, or present; internal or external; coarse or fine; inferior or superior; far or near: this is called the aggregate of form. Any kind of feeling at all … Any kind of perception at all … Any kind of choices at all … Any kind of consciousness at all—past, future, or present; internal or external; coarse or fine; inferior or superior; far or near: this is called the aggregate of consciousness. That’s the scope of the term ‘aggregates’ as applied to the aggregates.” 

Saying\marginnote{6.8} “Good, sir”, that mendicant asked another question: 

“What\marginnote{7.1} is the cause, sir, what is the reason why the aggregate of form is found? What is the cause, what is the reason why the aggregate of feeling … perception … choices … consciousness is found?” 

“The\marginnote{7.6} four primary elements are the reason why the aggregate of form is found. Contact is the reason why the aggregates of feeling, perception, and choices are found. Name and form are the reasons why the aggregate of consciousness is found.” 

Saying\marginnote{7.11} “Good, sir”, that mendicant asked another question: 

“Sir,\marginnote{8.1} how does identity view come about?” 

“It’s\marginnote{8.2} because an unlearned ordinary person has not seen the noble ones, and is neither skilled nor trained in the teaching of the noble ones. They’ve not seen good persons, and are neither skilled nor trained in the teaching of the good persons. They regard form as self, self as having form, form in self, or self in form. They regard feeling … perception … choices … consciousness as self, self as having consciousness, consciousness in self, or self in consciousness. That’s how identity view comes about.” 

Saying\marginnote{8.9} “Good, sir”, that mendicant … asked another question: 

“But\marginnote{9.1} sir, how does identity view not come about?” 

It’s\marginnote{9.2} because a learned noble disciple has seen the noble ones, and is skilled and trained in the teaching of the noble ones. They’ve seen good persons, and are skilled and trained in the teaching of the good persons. They don’t regard form as self, self as having form, form in self, or self in form. They don’t regard feeling … perception … choices … consciousness as self, self as having consciousness, consciousness in self, or self in consciousness. That’s how identity view does not come about.” 

Saying\marginnote{9.9} “Good, sir”, that mendicant … asked another question: 

“Sir,\marginnote{10.1} what’s the gratification, the drawback, and the escape when it comes to form, feeling, perception, choices, and consciousness?” 

“The\marginnote{10.6} pleasure and happiness that arise from form: this is its gratification. That form is impermanent, suffering, and perishable: this is its drawback. Removing and giving up desire and greed for form: this is its escape. The pleasure and happiness that arise from feeling … perception … choices … consciousness: this is its gratification. That consciousness is impermanent, suffering, and perishable: this is its drawback. Removing and giving up desire and greed for consciousness: this is its escape.” 

Saying\marginnote{10.15} “Good, sir”, that mendicant approved and agreed with what the Buddha said. Then he asked another question: 

“Sir,\marginnote{11.1} how does one know and see so that there’s no ego, possessiveness, or underlying tendency to conceit for this conscious body and all external stimuli?” 

“One\marginnote{11.2} truly sees any kind of form at all—past, future, or present; internal or external; coarse or fine; inferior or superior; far or near: \emph{all} form—with right understanding: ‘This is not mine, I am not this, this is not my self.’ They truly see any kind of feeling … perception … choices … consciousness at all—past, future, or present; internal or external; coarse or fine; inferior or superior; far or near, \emph{all} consciousness—with right understanding: ‘This is not mine, I am not this, this is not my self.’ That’s how to know and see so that there’s no ego, possessiveness, or underlying tendency to conceit for this conscious body and all external stimuli.” 

Now\marginnote{12.1} at that time one of the mendicants had the thought: 

“So\marginnote{12.2} it seems, good sir, that form, feeling, perception, choices, and consciousness are not-self.\footnote{Here and in the parallel passage at MN 108, BB omits the bho, without comment. Thanissaro does likewise. However it appears in both cases, without variants. In fact the idiom iti kira is invariably followed by a vocative, so it can’t be a mistake. Horner translates it in MN 108, with the note: “This looks like a case where a monk, in thought, applies bho to himself. Or else he is thinking (as translated at K.S. iii. 88) “so then you say.” ” } Then what self will the deeds done by not-self affect?” 

Then\marginnote{12.4} the Buddha, knowing what that monk was thinking, addressed the mendicants: 

“It’s\marginnote{13.1} possible that some foolish person here—unknowing and ignorant, their mind dominated by craving—thinks they can overstep the teacher’s instructions. They think: ‘So it seems, good sir, that form, feeling, perception, choices, and consciousness are not-self. Then what self will the deeds done by not-self affect?’ Now, mendicants, you have been educated by me in questioning with regards to all these things in all such cases.\footnote{Once again, BB translates tatra tatra as “here and there”, though it is barely intelligible. Clearly here, as usual, a strongly distributive sense is required. } 

What\marginnote{14.1} do you think, mendicants? Is form permanent or impermanent?” 

“Impermanent,\marginnote{14.3} sir.” 

“Is\marginnote{14.4} feeling … perception … choices … consciousness permanent or impermanent?” 

“Impermanent,\marginnote{14.8} sir.” 

“But\marginnote{14.9} if it’s impermanent, is it suffering or happiness?” 

“Suffering,\marginnote{14.10} sir.” 

“But\marginnote{14.11} if it’s impermanent, suffering, and perishable, is it fit to be regarded thus: ‘This is mine, I am this, this is my self’?” 

“No,\marginnote{14.13} sir.” 

“So\marginnote{14.14} you should truly see … Seeing this … They understand: ‘… there is no return to any state of existence.’” 

\begin{verse}%
“Two\marginnote{15.1} on the aggregates; exactly the same; and can there be;\footnote{This is very unusual, a sutta-uddana. } \\
on the term; and on the cause; \\
two questions on identity; \\
gratification; and that with consciousness: \\
these are the ten questions \\
the mendicant came to ask.” 

%
\end{verse}

%
\addtocontents{toc}{\let\protect\contentsline\protect\nopagecontentsline}
\chapter*{The Chapter on Senior Mendicants }
\addcontentsline{toc}{chapter}{\tocchapterline{The Chapter on Senior Mendicants }}
\addtocontents{toc}{\let\protect\contentsline\protect\oldcontentsline}

%
\section*{{\suttatitleacronym SN 22.83}{\suttatitletranslation With Ānanda }{\suttatitleroot Ānandasutta}}
\addcontentsline{toc}{section}{\tocacronym{SN 22.83} \toctranslation{With Ānanda } \tocroot{Ānandasutta}}
\markboth{With Ānanda }{Ānandasutta}
\extramarks{SN 22.83}{SN 22.83}

At\marginnote{1.1} \textsanskrit{Sāvatthī}. 

There\marginnote{1.2} Ānanda addressed the mendicants: “Reverends, mendicants!” 

“Reverend,”\marginnote{1.4} they replied. Ānanda said this: 

“Reverends,\marginnote{2.1} the venerable named \textsanskrit{Puṇṇa} son of \textsanskrit{Mantāṇī} was very helpful to me when I was just ordained. He gave me this advice: ‘Reverend Ānanda, the notion “I am” occurs because of grasping, not by not grasping.\footnote{Take care of the double nested quotes. } Grasping what? The notion “I am” occurs because of grasping form, feeling, perception, choices, and consciousness, not by not grasping. 

Suppose\marginnote{3.1} there was a woman or man who was young, youthful, and fond of adornments, and they check their own reflection in a clean bright mirror or a clear bowl of water. They’d look because of grasping, not by not grasping. In the same way, the notion “I am” occurs because of grasping form, feeling, perception, choices, and consciousness, not by not grasping. 

What\marginnote{4.1} do you think, Reverend Ānanda? Is form permanent or impermanent?’ 

‘Impermanent,\marginnote{4.3} reverend.’\footnote{Note here the extreme equality in terms of address: even a newly ordained monk speaks like this to their teacher. } 

‘Is\marginnote{4.4} feeling … perception … choices … consciousness permanent or impermanent?’ 

‘Impermanent,\marginnote{4.8} reverend.’ … 

‘So\marginnote{4.9} you should truly see … Seeing this … They understand: “… there is no return to any state of existence.”’ 

Reverends,\marginnote{4.12} the venerable named \textsanskrit{Puṇṇa} son of \textsanskrit{Mantāṇī} was very helpful to me when I was just ordained. He gave me this advice. And now that I’ve heard this teaching from Venerable \textsanskrit{Puṇṇa} son of \textsanskrit{Mantāṇī}, I’ve comprehended the teaching.” 

%
\section*{{\suttatitleacronym SN 22.84}{\suttatitletranslation With Tissa }{\suttatitleroot Tissasutta}}
\addcontentsline{toc}{section}{\tocacronym{SN 22.84} \toctranslation{With Tissa } \tocroot{Tissasutta}}
\markboth{With Tissa }{Tissasutta}
\extramarks{SN 22.84}{SN 22.84}

At\marginnote{1.1} \textsanskrit{Sāvatthī}. 

Now\marginnote{1.2} at that time Venerable Tissa, the Buddha’s paternal cousin, informed several mendicants: 

“Reverends,\marginnote{1.3} my body feels like it’s drugged. I’m disorientated, the teachings don’t spring to mind, and dullness and drowsiness fill my mind. I lead the spiritual life dissatisfied, and have doubts about the teachings.” 

Then\marginnote{2.1} several mendicants went up to the Buddha, bowed, sat down to one side, and told him what had happened. 

So\marginnote{3.1} the Buddha addressed a certain monk, “Please, monk, in my name tell the mendicant Tissa that the Teacher summons him.”\footnote{PTS has the usual \textsanskrit{satthā} \textsanskrit{taṁ}, \textsanskrit{āvuso} \textsanskrit{sāriputta}, \textsanskrit{āmanteti}, which is presumably omitted by mistake in MS. } 

“Yes,\marginnote{3.3} sir,” that monk replied. He went to Tissa and said to him, “Reverend Tissa, the teacher summons you.” 

“Yes,\marginnote{3.5} reverend,” Tissa replied. He went to the Buddha, bowed, and sat down to one side. The Buddha said to him: 

“Is\marginnote{3.6} it really true, Tissa, that you informed several mendicants that your body feels like it’s drugged … and you have doubts about the teachings?” 

“Yes,\marginnote{3.8} sir.” 

“What\marginnote{3.9} do you think, Tissa? If you’re not rid of greed, desire, fondness, thirst, passion, and craving for form, when that form decays and perishes, will it give rise to sorrow, lamentation, pain, sadness, and distress?” 

“Yes,\marginnote{3.12} sir.” 

“Good,\marginnote{4.1} good, Tissa!\footnote{The Buddha’s incredible skill in teaching … } That’s how it is, Tissa, when you’re not rid of greed for form. 

If\marginnote{4.4} you’re not rid of greed for feeling … perception … choices … consciousness, when that consciousness decays and perishes, will it give rise to sorrow, lamentation, pain, sadness, and distress?”\footnote{An obvious typo here in anigata } 

“Yes,\marginnote{5.4} sir.” 

“Good,\marginnote{6.1} good, Tissa! That’s how it is, Tissa, when you’re not rid of greed for consciousness. 

What\marginnote{6.4} do you think, Tissa? If you are rid of greed, desire, fondness, thirst, passion, and craving for form, when that form decays and perishes, will it give rise to sorrow, lamentation, pain, sadness, and distress?” 

“No,\marginnote{6.6} sir.” 

“Good,\marginnote{7.1} good, Tissa! That’s how it is, Tissa, when you are rid of greed for form … feeling … perception … choices … consciousness. 

What\marginnote{8.4} do you think, Tissa? Is form permanent or impermanent?” 

“Impermanent,\marginnote{8.6} sir.” 

“Is\marginnote{8.7} feeling … perception … choices … consciousness permanent or impermanent?” 

“Impermanent,\marginnote{8.11} sir.” 

“So\marginnote{8.12} you should truly see … Seeing this … They understand: ‘… there is no return to any state of existence.’ 

Suppose,\marginnote{9.1} Tissa, there were two people. One was not skilled in the path, the other was. The one not skilled in the path would question the one skilled in the path, who would reply: ‘Come, good man, this is the path. Go down it a little, and you’ll see a fork in the road. Ignore the left, and take the right-hand path.\footnote{Interesting use of muncati and ganhati. } Go a little further, and you’ll see a dark forest grove. Go a little further, and you’ll see an expanse of low-lying marshes. Go a little further, and you’ll see a large, steep cliff. Go a little further, and you’ll see level, cleared parkland.’ 

I’ve\marginnote{10.1} made up this simile to make a point. And this is what it means. 

‘A\marginnote{10.3} person who is not skilled in the path’ is a term for an ordinary unlearned person. 

‘A\marginnote{10.4} person who is skilled in the path’ is a term for the Realized One, the perfected one, the fully awakened Buddha. 

‘A\marginnote{10.5} fork in the road’ is a term for doubt. 

‘The\marginnote{10.6} left-hand path’ is a term for the wrong eightfold path, that is, wrong view … wrong immersion. 

‘The\marginnote{10.8} right-hand path’ is a term for the noble eightfold path, that is, right view … right immersion. 

‘A\marginnote{10.10} dark forest grove’ is a term for ignorance. 

‘An\marginnote{10.11} expanse of low-lying marshes’ is a term for sensual pleasures. 

‘A\marginnote{10.12} large, steep cliff’ is a term for anger and distress. 

‘Level,\marginnote{10.13} cleared parkland’ is a term for extinguishment. 

Rejoice,\marginnote{10.14} Tissa, rejoice! I’m here to advise you, to support you, and to teach you.”\footnote{Wow, that’s a major variant. It looks very much like a commentarial gloss. } 

That\marginnote{11.1} is what the Buddha said. Satisfied, Venerable Tissa was happy with what the Buddha said. 

%
\section*{{\suttatitleacronym SN 22.85}{\suttatitletranslation With Yamaka }{\suttatitleroot Yamakasutta}}
\addcontentsline{toc}{section}{\tocacronym{SN 22.85} \toctranslation{With Yamaka } \tocroot{Yamakasutta}}
\markboth{With Yamaka }{Yamakasutta}
\extramarks{SN 22.85}{SN 22.85}

At\marginnote{1.1} one time Venerable \textsanskrit{Sāriputta} was staying near \textsanskrit{Sāvatthī} in Jeta’s Grove, \textsanskrit{Anāthapiṇḍika}’s monastery. 

Now\marginnote{1.2} at that time a mendicant called Yamaka had the following harmful misconception: “As I understand the Buddha’s teaching, a mendicant who has ended the defilements is annihilated and destroyed when their body breaks up, and doesn’t exist after death.” 

Several\marginnote{2.1} mendicants heard about this. They went to Yamaka and exchanged greetings with him. 

When\marginnote{2.4} the greetings and polite conversation were over, they sat down to one side and said to him, “Is it really true, Reverend Yamaka, that you have such a harmful misconception: ‘As I understand the Buddha’s teaching, a mendicant who has ended the defilements is annihilated and destroyed when their body breaks up, and doesn’t exist after death’?” 

“Yes,\marginnote{3.3} reverends, that’s how I understand the Buddha’s teaching.” 

“Don’t\marginnote{4.1} say that, Yamaka! Don’t misrepresent the Buddha, for misrepresentation of the Buddha is not good. And the Buddha would not say that.” But even though admonished by those mendicants, Yamaka obstinately stuck to that misconception and insisted on stating it. 

When\marginnote{5.1} those mendicants were unable to dissuade Yamaka from that misconception, they got up from their seats and went to see Venerable \textsanskrit{Sāriputta}. They told him what had happened, and said, “May Venerable \textsanskrit{Sāriputta} please go to the mendicant Yamaka out of compassion.” \textsanskrit{Sāriputta} consented in silence. 

Then\marginnote{5.6} in the late afternoon, Venerable \textsanskrit{Sāriputta} came out of retreat, went to Venerable Yamaka and exchanged greetings with him. Seated to one side he said to Yamaka: 

“Is\marginnote{6.1} it really true, Reverend Yamaka, that you have such a harmful misconception: ‘As I understand the Buddha’s teaching, a mendicant who has ended the defilements is annihilated and destroyed when their body breaks up, and doesn’t exist after death’?” 

“Yes,\marginnote{6.3} reverend, that’s how I understand the Buddha’s teaching.” 

“What\marginnote{7.1} do you think, Yamaka? Is form permanent or impermanent?” 

“Impermanent,\marginnote{7.3} reverend.” 

“Is\marginnote{7.4} feeling … perception … choices … consciousness permanent or impermanent?” 

“Impermanent,\marginnote{7.8} reverend.” 

“So\marginnote{7.9} you should truly see … Seeing this … They understand: ‘… there is no return to any state of existence.’ 

What\marginnote{8.1} do you think, Reverend Yamaka? Do you regard the Realized One as form?” 

“No,\marginnote{8.3} reverend.” 

“Do\marginnote{8.4} you regard the Realized One as feeling … perception … choices … consciousness?” 

“No,\marginnote{8.9} reverend.” 

“What\marginnote{9.1} do you think, Reverend Yamaka? Do you regard the Realized One as in form?” 

“No,\marginnote{9.3} reverend.” 

“Or\marginnote{9.4} do you regard the Realized One as distinct from form?” 

“No,\marginnote{9.5} reverend.” 

“Do\marginnote{9.6} you regard the Realized One as in feeling … or distinct from feeling … as in perception … or distinct from perception … as in choices … or distinct from choices … as in consciousness?” 

“No,\marginnote{9.13} reverend.” 

“Or\marginnote{9.14} do you regard the Realized One as distinct from consciousness?” 

“No,\marginnote{9.15} reverend.” 

“What\marginnote{10.1} do you think, Yamaka?\footnote{This passage is unclear, see discussion at https://discourse.suttacentral.net/t/a-problematic-reading-in-the-yamaka-sutta/2944 } Do you regard the Realized One as possessing form, feeling, perception, choices, and consciousness?” 

“No,\marginnote{10.3} reverend.” 

“What\marginnote{11.1} do you think, Yamaka? Do you regard the Realized One as one who is without form, feeling, perception, choices, and consciousness?” 

“No,\marginnote{11.3} reverend.” 

“In\marginnote{11.4} that case, Reverend Yamaka, since you don’t acknowledge the Realized One as a genuine fact in the present life, is it appropriate to declare: ‘As I understand the Buddha’s teaching, a mendicant who has ended the defilements is annihilated and destroyed when their body breaks up, and doesn’t exist after death.’?” 

“Reverend\marginnote{12.1} \textsanskrit{Sāriputta}, in my ignorance, I used to have that misconception. But now that I’ve heard the teaching from Venerable \textsanskrit{Sāriputta} I’ve given up that misconception, and I’ve comprehended the teaching.” 

“Reverend\marginnote{13.1} Yamaka, suppose they were to ask you: ‘When their body breaks up, after death, what happens to a perfected one, who has ended the defilements?’ How would you answer?” 

“Reverend,\marginnote{13.4} if they were to ask this, I’d answer like this: ‘Reverend, form is impermanent. What’s impermanent is suffering. What’s suffering has ceased and ended. 

Feeling\marginnote{13.10} … perception … choices … consciousness is impermanent. What’s impermanent is suffering. What’s suffering has ceased and ended.’ That’s how I’d answer such a question.” 

“Good,\marginnote{14.1} good, Reverend Yamaka! Well then, I shall give you a simile to make the meaning even clearer. Suppose there was a householder or householder’s son who was rich, with a lot of money and great wealth, and a bodyguard for protection. Then along comes a person who wants to harm, injure, and threaten him, and take his life. They’d think: ‘This householder or householder’s son is rich, with a lot of money and great wealth, and a bodyguard for protection. It won’t be easy to take his life by force. Why don’t I get close to him, then take his life?’ So he goes up to that householder or householder’s son and says: ‘Sir, I would serve you.’ Then they would serve that householder or householder’s son. They’d get up before him and go to bed after him, and be obliging, behaving nicely and speaking politely. The householder or householder’s son would consider them as a friend and companion, and come to trust them. But when that person realizes that they’ve gained the trust of the householder or householder’s son, then, when they know he’s alone, they’d take his life with a sharp knife. 

What\marginnote{15.1} do you think, Yamaka? When that person went to the householder or householder’s son and offered to serve him, weren’t they a killer then, though he didn’t know that this was his killer? And when they got up before him and went to bed after him, being obliging, behaving nicely and speaking politely, weren’t they a killer then, though he didn’t know that this was his killer? And when, knowing he was alone, they took his life with a sharp knife, weren’t they a killer then, though he didn’t know that this was his killer?” 

“Yes,\marginnote{15.12} reverend.” 

“In\marginnote{15.13} the same way, an unlearned ordinary person has not seen the noble ones, and is neither skilled nor trained in the teaching of the noble ones. They’ve not seen good persons, and are neither skilled nor trained in the teaching of the good persons. They regard form as self, self as having form, form in self, or self in form. They regard feeling … perception … choices … They regard consciousness as self, self as having consciousness, consciousness in self, or self in consciousness. 

They\marginnote{16.1} don’t truly understand form—which is impermanent—as impermanent. They don’t truly understand feeling … perception … choices … consciousness—which is impermanent—as impermanent. 

They\marginnote{17.1} don’t truly understand form—which is suffering—as suffering. They don’t truly understand feeling … perception … choices … consciousness—which is suffering—as suffering. 

They\marginnote{18.1} don’t truly understand form—which is not-self—as not-self. They don’t truly understand feeling … perception … choices … consciousness—which is not-self—as not-self. 

They\marginnote{19.1} don’t truly understand form—which is conditioned—as conditioned. They don’t truly understand feeling … perception … choices … consciousness—which is conditioned—as conditioned. 

They\marginnote{20.1} don’t truly understand form—which is a killer—as a killer. They don’t truly understand feeling … perception … choices … consciousness—which is a killer—as a killer. 

They’re\marginnote{21.1} attracted to form, grasp it, and commit to the notion that it is ‘my self’. They’re attracted to feeling … perception … choices … consciousness, grasp it, and commit to the notion that it is ‘my self’. And when you’ve gotten involved with and grasped these five grasping aggregates, they lead to your lasting harm and suffering. 

A\marginnote{22.1} learned noble disciple has seen the noble ones, and is skilled and trained in the teaching of the noble ones. They’ve seen good persons, and are skilled and trained in the teaching of the good persons. They don’t regard form as self, self as having form, form in self, or self in form. They don’t regard feeling … perception … choices … consciousness as self, self as having consciousness, consciousness in self, or self in consciousness. 

They\marginnote{23.1} truly understand form—which is impermanent—as impermanent. They truly understand feeling … perception … choices … consciousness—which is impermanent—as impermanent. 

They\marginnote{24.1} truly understand form—which is suffering—as suffering. They truly understand feeling … perception … choices … consciousness—which is suffering—as suffering. 

They\marginnote{25.1} truly understand form—which is not-self—as not-self. They truly understand feeling … perception … choices … consciousness—which is not-self—as not-self. 

They\marginnote{26.1} truly understand form—which is conditioned—as conditioned. They truly understand feeling … perception … choices … consciousness—which is conditioned—as conditioned. 

They\marginnote{27.1} truly understand form—which is a killer—as a killer. They truly understand feeling … perception … choices … consciousness—which is a killer—as a killer. 

Not\marginnote{28.1} being attracted to form, they don’t grasp it, and commit to the notion that it is ‘my self’. Not being attracted to feeling … perception … choices … consciousness, they don't grasp it, and commit to the notion that it is ‘my self’. And when you’re not attracted to and don’t grasp these five grasping aggregates, they lead to your lasting welfare and happiness.” 

“Reverend\marginnote{28.7} \textsanskrit{Sāriputta}, this is how it is when you have such venerables as spiritual companions to advise and instruct you out of kindness and compassion. And after hearing this teaching by Venerable \textsanskrit{Sāriputta}, my mind is freed from the defilements by not grasping.” 

%
\section*{{\suttatitleacronym SN 22.86}{\suttatitletranslation With Anurādha }{\suttatitleroot Anurādhasutta}}
\addcontentsline{toc}{section}{\tocacronym{SN 22.86} \toctranslation{With Anurādha } \tocroot{Anurādhasutta}}
\markboth{With Anurādha }{Anurādhasutta}
\extramarks{SN 22.86}{SN 22.86}

At\marginnote{1.1} one time the Buddha was staying near \textsanskrit{Vesālī}, at the Great Wood, in the hall with the peaked roof. Now at that time Venerable \textsanskrit{Anurādha} was staying not far from the Buddha in a wilderness hut. Then several wanderers who follow other paths went up to Venerable \textsanskrit{Anurādha} and exchanged greetings with him. When the greetings and polite conversation were over, they sat down to one side and said to him: 

“Reverend\marginnote{1.5} \textsanskrit{Anurādha}, when a Realized One is describing a Realized One—a supreme person, highest of people, who has reached the highest point—they describe them in these four ways: After death, a Realized One exists, or doesn’t exist, or both exists and doesn’t exist, or neither exists nor doesn’t exist.” 

When\marginnote{2.1} they said this, Venerable \textsanskrit{Anurādha} said to those wanderers: 

“Reverends,\marginnote{2.2} when a Realized One is describing a Realized One—a supreme person, highest of people, who has reached the highest point—they describe them other than these four ways: After death, a Realized One exists, or doesn’t exist, or both exists and doesn’t exist, or neither exists nor doesn’t exist.” When he said this, the wanderers said to him: 

“This\marginnote{2.5} mendicant must be junior, recently gone forth, or else a foolish, incompetent senior mendicant.” Then, after rebuking Venerable \textsanskrit{Anurādha} by calling him “junior” and “foolish”, the wanderers got up from their seats and left. 

Soon\marginnote{3.1} after they had left, \textsanskrit{Anurādha} thought: 

“If\marginnote{3.2} those wanderers were to inquire further, how should I answer them so as to repeat what the Buddha has said, and not misrepresent him with an untruth? How should I explain in line with his teaching, so that there would be no legitimate grounds for rebuke and criticism?” 

Then\marginnote{4.1} Venerable \textsanskrit{Anurādha} went up to the Buddha, bowed, sat down to one side, and told him all that had happened. 

“What\marginnote{7.1} do you think, \textsanskrit{Anurādha}? Is form permanent or impermanent?” 

“Impermanent,\marginnote{7.3} sir.” 

“But\marginnote{7.4} if it’s impermanent, is it suffering or happiness?” 

“Suffering,\marginnote{7.5} sir.” 

“But\marginnote{7.6} if it’s impermanent, suffering, and perishable, is it fit to be regarded thus: ‘This is mine, I am this, this is my self’?” 

“No,\marginnote{7.8} sir.” 

“Is\marginnote{7.9} feeling … perception … choices … consciousness permanent or impermanent?” 

“Impermanent,\marginnote{7.13} sir.” … “So you should truly see … Seeing this … They understand: ‘… there is no return to any state of existence.’ 

What\marginnote{8.1} do you think, \textsanskrit{Anurādha}? Do you regard the Realized One as form?” 

“No,\marginnote{8.3} sir.” 

“Do\marginnote{8.4} you regard the Realized One as feeling … perception … choices … consciousness?” 

“No,\marginnote{8.8} sir.” 

“What\marginnote{9.1} do you think, \textsanskrit{Anurādha}? Do you regard the Realized One as in form?” 

“No,\marginnote{9.3} sir.” 

“Or\marginnote{9.4} do you regard the Realized One as distinct from form?” 

“No,\marginnote{9.5} sir.” 

“Do\marginnote{9.6} you regard the Realized One as in feeling … or distinct from feeling … as in perception … or distinct from perception … as in choices … or distinct from choices … as in consciousness … or as distinct from consciousness?” 

“No,\marginnote{9.14} sir.” 

“What\marginnote{10.1} do you think, \textsanskrit{Anurādha}?\footnote{This passage is unclear, see discussion at https://discourse.suttacentral.net/t/a-problematic-reading-in-the-yamaka-sutta/2944 } Do you regard the Realized One as possessing form, feeling, perception, choices, and consciousness?” 

“No,\marginnote{10.3} sir.” 

“What\marginnote{11.1} do you think, \textsanskrit{Anurādha}? Do you regard the Realized One as one who is without form, feeling, perception, choices, and consciousness?” 

“No,\marginnote{11.3} sir.” 

“In\marginnote{12.1} that case, \textsanskrit{Anurādha}, since you don’t acknowledge the Realized One as a genuine fact in the present life, is it appropriate to declare: ‘Reverends, when a Realized One is describing a Realized One—a supreme person, highest of people, who has reached the highest point—they describe them other than these four ways: After death, a Realized One exists, or doesn’t exist, or both exists and doesn’t exist, or neither exists nor doesn’t exist’?” 

“No,\marginnote{12.4} sir.” 

“Good,\marginnote{13.1} good, \textsanskrit{Anurādha}! In the past, as today, I describe suffering and the cessation of suffering.” 

%
\section*{{\suttatitleacronym SN 22.87}{\suttatitletranslation With Vakkali }{\suttatitleroot Vakkalisutta}}
\addcontentsline{toc}{section}{\tocacronym{SN 22.87} \toctranslation{With Vakkali } \tocroot{Vakkalisutta}}
\markboth{With Vakkali }{Vakkalisutta}
\extramarks{SN 22.87}{SN 22.87}

At\marginnote{1.1} one time the Buddha was staying near \textsanskrit{Rājagaha}, in the Bamboo Grove, the squirrels’ feeding ground. 

Now\marginnote{1.2} at that time Venerable Vakkali was staying in a potter’s shed, and he was sick, suffering, gravely ill. 

Then\marginnote{1.3} he addressed his carers, “Please, reverends, go to the Buddha, and in my name bow with your head to his feet. Say to him: ‘Sir, the mendicant Vakkali is sick, suffering, and gravely ill. He bows with his head to your feet.’ And then say: ‘Sir, please go to the mendicant Vakkali out of compassion.’” 

“Yes,\marginnote{1.8} reverend,” those monks replied. They did as he asked. The Buddha consented in silence. 

Then\marginnote{2.1} the Buddha robed up and, taking his bowl and robes, went to Venerable Vakkali. Venerable Vakkali saw the Buddha coming off in the distance and tried to rise on his cot. 

But\marginnote{2.3} the Buddha said to him, “It’s all right, Vakkali, don’t get up. There are some seats spread out, I will sit there.” 

He\marginnote{2.6} sat on the seat spread out and said to Vakkali, “I hope you’re keeping well, Vakkali; I hope you’re alright. And I hope the pain is fading, not growing, that its fading is evident, not its growing.” 

“Sir,\marginnote{2.8} I’m not all right, I’m not getting by. My pain is terrible and growing, not fading; its growing is apparent, not its fading.” 

“I\marginnote{2.9} hope you don’t have any remorse or regret?” 

“Indeed,\marginnote{2.10} sir, I have no little remorse and regret.” 

“I\marginnote{2.11} hope you have no reason to blame yourself when it comes to ethical conduct?” 

“No\marginnote{2.12} sir, I have no reason to blame myself when it comes to ethical conduct.” 

“In\marginnote{2.13} that case, Vakkali, why do you have remorse and regret?” 

“For\marginnote{2.14} a long time I’ve wanted to go and see the Buddha, but I was physically too weak.” 

“Enough,\marginnote{3.1} Vakkali! Why would you want to see this rotten body? One who sees the teaching sees me. One who sees me sees the teaching. Seeing the teaching, you see me. Seeing me, you see the teaching. 

What\marginnote{4.1} do you think, Vakkali? Is form permanent or impermanent?” 

“Impermanent,\marginnote{4.3} sir.” 

“But\marginnote{4.4} if it’s impermanent, is it suffering or happiness?” 

“Suffering,\marginnote{4.5} sir.” 

“But\marginnote{4.6} if it’s impermanent, suffering, and perishable, is it fit to be regarded thus: ‘This is mine, I am this, this is my self’?” 

“No,\marginnote{4.8} sir.” 

“Is\marginnote{4.9} feeling … perception … choices … consciousness permanent or impermanent?” 

“Impermanent,\marginnote{4.13} sir.” … 

“So\marginnote{4.16} you should truly see … Seeing this … They understand: ‘… there is no return to any state of existence.’” 

And\marginnote{5.1} then, after giving Venerable Vakkali this advice, the Buddha got up from his seat and went to the Vulture’s Peak Mountain. 

Then\marginnote{5.2} Venerable Vakkali addressed his carers, “Come on, reverends, lift my cot and take me to the Black Rock on the slopes of Isigili. It’s unthinkable for one like me to die in an inhabited area!” 

“Yes,\marginnote{5.5} reverend,” replied those mendicants, and did as he asked. Meanwhile, the Buddha spent the rest of that night and day on Vulture’s Peak Mountain. 

Then,\marginnote{5.7} late at night, two glorious deities, lighting up the entire Vulture’s Peak, went up to the Buddha, bowed, and stood to one side. 

One\marginnote{5.8} deity said to him, “Sir, the mendicant Vakkali is intent on liberation!” 

And\marginnote{5.10} another deity said to him, “He’ll definitely be well-freed!” 

This\marginnote{5.12} is what those deities said. Then they bowed and respectfully circled the Buddha, keeping him on their right side, before vanishing right there. 

Then,\marginnote{6.1} when the night had passed, the Buddha addressed the mendicants: 

“Come,\marginnote{6.2} mendicants, go to the mendicant Vakkali and tell him: 

‘Vakkali,\marginnote{7.1} hear the word of the Buddha and two deities. 

Late\marginnote{7.2} last night, two glorious deities, lighting up the entire Vulture’s Peak, went up to the Buddha, bowed, and stood to one side. 

One\marginnote{7.3} deity said to him, “Sir, the mendicant Vakkali is intent on liberation!”\footnote{MS lack quotes, but here it is double-nested. } 

The\marginnote{7.5} other deity said to him, “He’ll definitely be well-freed!” 

And\marginnote{7.7} the Buddha said, “Do not fear, Vakkali, do not fear! Your death will not be a bad one; your passing will not be a bad one.”’” 

“Yes,\marginnote{7.11} sir,” those monks replied. They went to Vakkali and said to him: 

“Vakkali,\marginnote{7.12} hear the word of the Buddha and two deities.” 

Then\marginnote{8.1} Vakkali addressed his carers, “Please, reverends, help me off my cot. It’s unthinkable for one like me to listen to the Buddha’s instructions sitting on a high seat.” 

“Yes,\marginnote{8.4} reverend,” replied those mendicants, and helped him off his cot. They repeated what the Buddha had said. 

Vakkali\marginnote{8.14} said, “Well then, reverends, in my name bow with your head at the Buddha’s feet. Say to him: ‘Sir, the mendicant Vakkali is sick, suffering, gravely ill. He bows with his head to your feet.’ And then say: ‘Form is impermanent. I have no doubt of that. I’m certain that what is impermanent is suffering. And I’m certain that I have no desire or greed or fondness for what is impermanent, suffering, and perishable. 

Feeling\marginnote{8.22} is impermanent … 

Perception\marginnote{8.26} is impermanent … 

Choices\marginnote{8.27} are impermanent … 

Consciousness\marginnote{8.31} is impermanent. I have no doubt of that. I’m certain that what is impermanent is suffering. And I’m certain that I have no desire or greed or fondness for what is impermanent, suffering, and perishable.’” 

“Yes,\marginnote{8.35} reverend,” those monks replied, and left. And then, not long after those mendicants had left, Venerable Vakkali slit his wrists. 

Then\marginnote{9.1} those mendicants went up to the Buddha and told him Vakkali’s message. 

Then\marginnote{10.1} the Buddha said to the mendicants, “Come, mendicants, let’s go to the Black Rock on the slopes of Isigili, where Vakkali, the gentleman, slit his wrists.” 

“Yes,\marginnote{10.4} sir,” they replied. 

Then\marginnote{10.5} the Buddha together with several mendicants went to the Black Rock on the slopes of Isigili. The Buddha saw Vakkali off in the distance lying on his cot, having cast off the aggregates.\footnote{See discussion https://discourse.suttacentral.net/t/turning-shoulders-and-casting-off-aggregates/2949 } 

Now\marginnote{11.1} at that time a cloud of black smoke was moving east, west, north, south, above, below, and in-between. 

The\marginnote{11.2} Buddha said to the mendicants, “Mendicants, do you see that cloud of black smoke moving east, west, north, south, above, below, and in-between?” 

“Yes,\marginnote{11.4} sir.” 

“That’s\marginnote{11.5} \textsanskrit{Māra} the Wicked searching for Vakkali’s consciousness, wondering: ‘Where is Vakkali’s consciousness established?’ But since his consciousness is not established, Vakkali is extinguished.” 

%
\section*{{\suttatitleacronym SN 22.88}{\suttatitletranslation With Assaji }{\suttatitleroot Assajisutta}}
\addcontentsline{toc}{section}{\tocacronym{SN 22.88} \toctranslation{With Assaji } \tocroot{Assajisutta}}
\markboth{With Assaji }{Assajisutta}
\extramarks{SN 22.88}{SN 22.88}

At\marginnote{1.1} one time the Buddha was staying near \textsanskrit{Rājagaha}, in the Bamboo Grove, the squirrels’ feeding ground. Now at that time Venerable Assaji was staying in a monastery built by a Kassapa, and he was sick, suffering, gravely ill. Then he addressed his carers: 

“Please,\marginnote{1.4} reverends, go to the Buddha, and in my name bow with your head to his feet. Say to him: ‘Sir, the mendicant Assaji is sick, suffering, gravely ill. He bows with his head to your feet.’ And then say: ‘Sir, please go to the mendicant Assaji out of compassion.’” 

“Yes,\marginnote{1.9} reverend,” those monks replied. They did as he asked. The Buddha consented in silence. 

Then\marginnote{2.1} in the late afternoon, the Buddha came out of retreat and went to Venerable Assaji. Venerable Assaji saw the Buddha coming off in the distance, and tried to rise on his cot. 

But\marginnote{2.4} the Buddha said to him, “It’s all right, Assaji, don’t get up. There are some seats spread out, I will sit there.” 

He\marginnote{2.7} sat on the seat spread out and said, “I hope you’re keeping well, Assaji; I hope you’re alright. And I hope the pain is fading, not growing, that its fading is evident, not its growing.” 

“Sir,\marginnote{3.1} I’m not all right, I’m not getting by. My pain is terrible and growing, not fading, its growing is evident, not its fading.” 

“I\marginnote{3.2} hope you don’t have any remorse or regret?” 

“Indeed,\marginnote{3.3} sir, I have no little remorse and regret.” 

“I\marginnote{3.4} hope you have no reason to blame yourself when it comes to ethical conduct?” 

“No\marginnote{3.5} sir, I have no reason to blame myself when it comes to ethical conduct.” 

“In\marginnote{3.6} that case, Assaji, why do you have remorse and regret?” 

“Sir,\marginnote{3.7} before my time of illness I meditated having completely stilled the physical process. But now I can’t get immersion. Since I can’t get immersion, I think: ‘May I not decline!’” 

“Assaji,\marginnote{3.10} there are ascetics and brahmins for whom \textsanskrit{samādhi} is the essence, equating immersion with the ascetic life. They think: ‘May we not decline!’ 

What\marginnote{4.1} do you think, Assaji? Is form permanent or impermanent?” 

“Impermanent,\marginnote{4.3} sir.” … 

“Is\marginnote{4.4} consciousness permanent or impermanent?” … 

“So\marginnote{4.5} you should truly see … Seeing this … They understand: ‘… there is no return to any state of existence.’ 

If\marginnote{4.8} they feel a pleasant feeling, they understand that it’s impermanent, that they’re not attached to it, and that they don’t relish it. If they feel a painful feeling, they understand that it’s impermanent, that they’re not attached to it, and that they don’t relish it. If they feel a neutral feeling, they understand that it’s impermanent, that they’re not attached to it, and that they don’t relish it. If they feel a pleasant feeling, they feel it detached. If they feel a painful feeling, they feel it detached. If they feel a neutral feeling, they feel it detached. Feeling the end of the body approaching, they understand: ‘I feel the end of the body approaching.’ Feeling the end of life approaching, they understand: ‘I feel the end of life approaching.’ They understand: ‘When my body breaks up and my life is over, everything that’s felt, being no longer relished, will become cool right here.’ 

Suppose\marginnote{5.1} an oil lamp depended on oil and a wick to burn. As the oil and the wick are used up, it would be extinguished due to lack of fuel. 

In\marginnote{5.3} the same way, feeling the end of the body approaching, they understand: ‘I feel the end of the body approaching.’ Feeling the end of life approaching, they understand: ‘I feel the end of life approaching.’ They understand: ‘When my body breaks up and my life is over, everything that’s felt, being no longer relished, will become cool right here.’” 

%
\section*{{\suttatitleacronym SN 22.89}{\suttatitletranslation With Khemaka }{\suttatitleroot Khemakasutta}}
\addcontentsline{toc}{section}{\tocacronym{SN 22.89} \toctranslation{With Khemaka } \tocroot{Khemakasutta}}
\markboth{With Khemaka }{Khemakasutta}
\extramarks{SN 22.89}{SN 22.89}

At\marginnote{1.1} one time several senior mendicants were staying near Kosambi, in Ghosita’s Monastery. Now at that time Venerable Khemaka was staying in the Jujube Tree Monastery, and he was sick, suffering, gravely ill. 

In\marginnote{1.3} the late afternoon those senior mendicants came out of retreat and addressed Venerable \textsanskrit{Dāsaka}, “Please, Reverend \textsanskrit{Dāsaka}, go to the mendicant Khemaka and say to him: ‘Reverend Khemaka, the seniors hope you’re keeping well; they hope you’re alright. They hope that your pain is fading, not growing, that its fading is evident, not its growing.’” 

“Yes,\marginnote{1.7} reverends,” replied \textsanskrit{Dāsaka}. He went to Khemaka and said to him: 

“Reverend\marginnote{1.8} Khemaka, the seniors hope you’re keeping well; they hope you’re alright. They hope that your pain is fading, not growing, that its fading is evident, not its growing.” 

“Reverend,\marginnote{1.10} I’m not keeping well, I’m not alright. My pain is terrible and growing, not fading; its growing is evident, not its fading.” 

Then\marginnote{2.1} \textsanskrit{Dāsaka} went to those seniors and told them what had happened. They said, 

“Please,\marginnote{2.4} Reverend \textsanskrit{Dāsaka}, go to the mendicant Khemaka and say to him: ‘Reverend Khemaka, the seniors say that these five grasping aggregates have been taught by the Buddha, that is: the grasping aggregates of form, feeling, perception, choices, and consciousness. Do you regard anything among these five grasping aggregates as self or as belonging to self?’” 

“Yes,\marginnote{3.1} reverends,” replied \textsanskrit{Dāsaka}. He relayed the message to Khemaka, who replied: 

“These\marginnote{3.6} five grasping aggregates have been taught by the Buddha, that is: the grasping aggregates of form, feeling, perception, choices, and consciousness. I do not regard anything among these five grasping aggregates as self or as belonging to self.” 

Then\marginnote{4.1} \textsanskrit{Dāsaka} went to those seniors and told them what had happened. They said: 

“Please,\marginnote{4.6} Reverend \textsanskrit{Dāsaka}, go to the mendicant Khemaka and say to him: ‘Reverend Khemaka, the seniors say that these five grasping aggregates have been taught by the Buddha, that is: the grasping aggregates of form, feeling, perception, choices, and consciousness. If, as it seems, Venerable Khemaka does not regard anything among these five grasping aggregates as self or as belonging to self, then he is a perfected one, with defilements ended.’” 

“Yes,\marginnote{5.1} reverends,” replied \textsanskrit{Dāsaka}. He relayed the message to Khemaka, who replied: 

“These\marginnote{5.6} five grasping aggregates have been taught by the Buddha, that is: the grasping aggregates of form, feeling, perception, choices, and consciousness. I do not regard anything among these five grasping aggregates as self or as belonging to self, yet I am not a perfected one, with defilements ended. For when it comes to the five grasping aggregates I’m not rid of the conceit ‘I am’. But I don’t regard anything as ‘I am this’.” 

Then\marginnote{6.1} \textsanskrit{Dāsaka} went to those seniors and told them what had happened. They said: 

“Please,\marginnote{7.1} Reverend \textsanskrit{Dāsaka}, go to the mendicant Khemaka and say to him: ‘Reverend Khemaka, the seniors ask, when you say “I am”, what is it that you’re talking about? Is it form or apart from form? Is it feeling … perception … choices … consciousness, or apart from consciousness? When you say “I am”, what is it that you’re talking about?” 

“Yes,\marginnote{8.1} reverends,” replied \textsanskrit{Dāsaka}. He relayed the message to Khemaka, who replied: 

“Enough,\marginnote{8.10} Reverend \textsanskrit{Dāsaka}! What’s the point in running back and forth? Bring my staff, I’ll go to see the senior mendicants myself.” 

Then\marginnote{9.1} Venerable Khemaka, leaning on a staff, went to those senior mendicants and exchanged greetings with them. When the greetings and polite conversation were over, he sat down to one side. They said to him: 

“Reverend\marginnote{9.3} Khemaka, when you say ‘I am’, what is it that you’re talking about? Is it form or apart from form? Is it feeling … perception … choices … consciousness, or apart from consciousness? When you say ‘I am’, what is it that you’re talking about?” 

“Reverends,\marginnote{9.10} I don’t say ‘I am’ with reference to form, or apart from form. I don’t say ‘I am’ with reference to feeling … perception … choices … consciousness, or apart from consciousness. For when it comes to the five grasping aggregates I’m not rid of the conceit ‘I am’. But I don’t regard anything as ‘I am this’. 

It’s\marginnote{10.1} like the scent of a blue water lily, or a pink or white lotus. Would it be right to say that the scent belongs to the petals or the stalk or the pistil?” 

“No,\marginnote{10.3} reverend.” 

“Then,\marginnote{10.4} reverends, how should it be said?” 

“It\marginnote{10.5} would be right to say that the scent belongs to the flower.” 

“In\marginnote{10.6} the same way, reverends, I don’t say ‘I am’ with reference to form, or apart from form. I don’t say ‘I am’ with reference to feeling … perception … choices … consciousness, or apart from consciousness. For when it comes to the five grasping aggregates I’m not rid of the conceit ‘I am’. But I don’t regard anything as ‘I am this’. 

Although\marginnote{11.1} a noble disciple has given up the five lower fetters, they still have a lingering residue of the conceit ‘I am’, the desire ‘I am’, and the underlying tendency ‘I am’ which has not been eradicated. After some time they meditate observing rise and fall in the five grasping aggregates. ‘Such is form, such is the origin of form, such is the ending of form. Such is feeling … Such is perception … Such are choices … Such is consciousness, such is the origin of consciousness, such is the ending of consciousness.’ As they do so, that lingering residue is eradicated. 

Suppose\marginnote{12.1} there was a cloth that was dirty and soiled, so the owners give it to a launderer. The launderer kneads it thoroughly with salt, lye, and cow dung, and rinses it in clear water. Although that cloth is clean and bright, it still has a lingering scent of salt, lye, or cow dung that had not been eradicated. The launderer returns it to its owners, who store it in a fragrant casket. And that lingering scent would be eradicated. 

In\marginnote{12.6} the same way, although a noble disciple has given up the five lower fetters, they still have a lingering residue of the conceit ‘I am’, the desire ‘I am’, and the underlying tendency ‘I am’ which has not been eradicated. After some time they meditate observing rise and fall in the five grasping aggregates. ‘Such is form, such is the origin of form, such is the ending of form. Such is feeling … Such is perception … Such are choices … Such is consciousness, such is the origin of consciousness, such is the ending of consciousness.’ As they do so, that lingering residue is eradicated.” 

When\marginnote{13.1} he said this, the senior mendicants said to Venerable Khemaka, “We didn’t want to trouble Venerable Khemaka with our questions. But you’re capable of explaining, teaching, asserting, establishing, clarifying, analyzing, and revealing the Buddha’s instructions in detail. And that’s just what you’ve done.” 

That’s\marginnote{14.1} what Venerable Khemaka said. Satisfied, the senior mendicants were happy with what Venerable Khemaka said. And while this discourse was being spoken, the minds of sixty senior mendicants and of Venerable Khemaka were freed from defilements by not grasping. 

%
\section*{{\suttatitleacronym SN 22.90}{\suttatitletranslation With Channa }{\suttatitleroot Channasutta}}
\addcontentsline{toc}{section}{\tocacronym{SN 22.90} \toctranslation{With Channa } \tocroot{Channasutta}}
\markboth{With Channa }{Channasutta}
\extramarks{SN 22.90}{SN 22.90}

At\marginnote{1.1} one time several senior mendicants were staying near Benares, in the deer park at Isipatana. 

Then\marginnote{1.2} in the late afternoon, Venerable Channa came out of retreat. Taking a key, he went from dwelling to dwelling, going up to the senior mendicants and saying, “May the venerable senior mendicants advise me and instruct me! May they give me a Dhamma talk so that I can see the teaching!” 

When\marginnote{2.1} he said this, the senior mendicants said to Venerable Channa: 

“Reverend\marginnote{2.2} Channa, form, feeling, perception, choices, and consciousness are impermanent. Form, feeling, perception, choices, and consciousness are not-self. All conditions are impermanent. All things are not-self.” 

Then\marginnote{3.1} Venerable Channa thought, “I too think in this way. … And yet my mind isn’t eager, confident, settled, and decided about the stilling of all activities, the letting go of all attachments, the ending of craving, fading away, cessation, extinguishment. Anxiety and grasping arise. And the mind reverts to thinking: ‘So then who exactly is my self?’ But that doesn’t happen for someone who sees the teaching. Who can teach me the Dhamma so that I can see the teaching?” 

Then\marginnote{4.1} Venerable Channa thought, “The Venerable Ānanda is staying near Kosambi, in Ghosita’s Monastery. He’s praised by the Buddha and esteemed by his sensible spiritual companions. He’s quite capable of teaching me the Dhamma so that I can see the teaching. Since I have so much trust in Venerable Ānanda, why don’t I go to see him?” 

Then\marginnote{4.5} Channa set his lodgings in order and, taking his bowl and robe, set out for Kosambi. He went to see Ānanda in Ghosita’s Monastery, exchanged greetings with him, and told him what had happened. Then he said, 

“May\marginnote{7.1} Venerable Ānanda advise me and instruct me! May he give me a Dhamma talk so that I can see the teaching!” 

“I’m\marginnote{8.1} already delighted with Venerable Channa. Hopefully you’ve opened yourself up and cut through your emotional barrenness.\footnote{Reading khila rather than \textsanskrit{khīla}. I think this passage has a specific Vinaya meaning: he has confessed and healed the breach in the sangha. It’s difficult to convey that sense. } Listen well, Channa. You are capable of understanding the teaching.” 

Then\marginnote{8.4} right away Channa was filled with lofty rapture and joy, “It seems I’m capable of understanding the teaching!” 

“Reverend\marginnote{9.1} Channa, I heard and learned in the presence of the Buddha his advice to the mendicant \textsanskrit{Kaccānagotta}: 

‘\textsanskrit{Kaccāna},\marginnote{9.2} this world mostly relies on the dual notions of existence and non-existence.\footnote{MS omits quote marks. Don’t forget to adjust the nesting level. } 

But\marginnote{9.3} when you truly see the origin of the world with right understanding, you won’t have the notion of non-existence regarding the world. And when you truly see the cessation of the world with right understanding, you won’t have the notion of existence regarding the world. 

The\marginnote{9.5} world is for the most part shackled by attraction, grasping, and insisting. 

But\marginnote{9.6} if—when it comes to this attraction, grasping, mental fixation, insistence, and underlying tendency—you don’t get attracted, grasp, and commit to the notion “my self”, you’ll have no doubt or uncertainty that what arises is just suffering arising, and what ceases is just suffering ceasing. Your knowledge about this is independent of others. 

This\marginnote{9.9} is how right view is defined. 

“All\marginnote{9.10} exists”: this is one extreme. 

“All\marginnote{9.11} does not exist”: this is the second extreme. 

Avoiding\marginnote{9.12} these two extremes, the Realized One teaches by the middle way: 

“Ignorance\marginnote{9.13} is a condition for choices. Choices are a condition for consciousness. … That is how this entire mass of suffering originates. 

When\marginnote{9.16} ignorance fades away and ceases with nothing left over, choices cease. … That is how this entire mass of suffering ceases.”’” 

“Reverend\marginnote{10.1} Ānanda, this is how it is when you have such venerables as spiritual companions to advise and instruct you out of kindness and compassion. And now that I’ve heard this teaching from Venerable Ānanda, I’ve comprehended the teaching.” 

%
\section*{{\suttatitleacronym SN 22.91}{\suttatitletranslation Rāhula }{\suttatitleroot Rāhulasutta}}
\addcontentsline{toc}{section}{\tocacronym{SN 22.91} \toctranslation{Rāhula } \tocroot{Rāhulasutta}}
\markboth{Rāhula }{Rāhulasutta}
\extramarks{SN 22.91}{SN 22.91}

At\marginnote{1.1} \textsanskrit{Sāvatthī}. 

Then\marginnote{1.2} Venerable \textsanskrit{Rāhula} went up to the Buddha, bowed, sat down to one side, and said to him: 

“Sir,\marginnote{1.3} how does one know and see so that there’s no ego, possessiveness, or underlying tendency to conceit for this conscious body and all external stimuli?” 

“\textsanskrit{Rāhula},\marginnote{2.1} one truly sees any kind of form at all—past, future, or present; internal or external; coarse or fine; inferior or superior; far or near: \emph{all} form—with right understanding: ‘This is not mine, I am not this, this is not my self.’ 

One\marginnote{2.2} truly sees any kind of feeling … perception … choices … consciousness at all—past, future, or present; internal or external; coarse or fine; inferior or superior; far or near: \emph{all} consciousness—with right understanding: ‘This is not mine, I am not this, this is not my self.’ 

That’s\marginnote{2.7} how to know and see so that there’s no ego, possessiveness, or underlying tendency to conceit for this conscious body and all external stimuli.” 

%
\section*{{\suttatitleacronym SN 22.92}{\suttatitletranslation Rāhula (2nd) }{\suttatitleroot Dutiyarāhulasutta}}
\addcontentsline{toc}{section}{\tocacronym{SN 22.92} \toctranslation{Rāhula (2nd) } \tocroot{Dutiyarāhulasutta}}
\markboth{Rāhula (2nd) }{Dutiyarāhulasutta}
\extramarks{SN 22.92}{SN 22.92}

At\marginnote{1.1} \textsanskrit{Sāvatthī}. 

Seated\marginnote{1.2} to one side, \textsanskrit{Rāhula} said to the Buddha: 

“Sir,\marginnote{1.3} how does one know and see so that the mind is rid of ego, possessiveness, and conceit for this conscious body and all external stimuli; and going beyond discrimination, it’s peaceful and well freed?” 

“\textsanskrit{Rāhula},\marginnote{2.1} when one truly sees any kind of form at all—past, future, or present; internal or external; coarse or fine; inferior or superior; far or near: \emph{all} form—with right understanding: ‘This is not mine, I am not this, this is not my self,’ one is freed by not grasping. 

One\marginnote{2.2} truly sees any kind of feeling … perception … choices … When one truly sees any kind of consciousness at all—past, future, or present; internal or external; coarse or fine; inferior or superior; far or near: \emph{all} consciousness—with right understanding: ‘This is not mine, I am not this, this is not my self,’ one is freed by not grasping. 

That’s\marginnote{2.6} how to know and see so that the mind is rid of ego, possessiveness, and conceit for this conscious body and all external stimuli; and going beyond discrimination, it’s peaceful and well freed.” 

%
\addtocontents{toc}{\let\protect\contentsline\protect\nopagecontentsline}
\chapter*{The Chapter on Flowers }
\addcontentsline{toc}{chapter}{\tocchapterline{The Chapter on Flowers }}
\addtocontents{toc}{\let\protect\contentsline\protect\oldcontentsline}

%
\section*{{\suttatitleacronym SN 22.93}{\suttatitletranslation A River }{\suttatitleroot Nadīsutta}}
\addcontentsline{toc}{section}{\tocacronym{SN 22.93} \toctranslation{A River } \tocroot{Nadīsutta}}
\markboth{A River }{Nadīsutta}
\extramarks{SN 22.93}{SN 22.93}

At\marginnote{1.1} \textsanskrit{Sāvatthī}. 

“Suppose,\marginnote{1.2} mendicants, there was a mountain river that flowed swiftly, going far, carrying all before it. If wild sugarcane, kusa grass, reeds, vetiver, or trees grew on either bank, they’d overhang the river. And if a person who was being swept along by the current grabbed the wild sugarcane, kusa grass, reeds, vetiver, or trees, it’d break off, and they’d come to ruin because of that. 

In\marginnote{1.3} the same way, an unlearned ordinary person has not seen the noble ones, and is neither skilled nor trained in the teaching of the noble ones. They’ve not seen good persons, and are neither skilled nor trained in the teaching of the good persons. 

They\marginnote{1.4} regard form as self, self as having form, form in self, or self in form. But their form breaks off, and they come to ruin because of that. They regard feeling … perception … choices … consciousness as self, self as having consciousness, consciousness in self, or self in consciousness. But their consciousness breaks off, and they come to ruin because of that. 

What\marginnote{1.13} do you think, mendicants? Is form permanent or impermanent?” 

“Impermanent,\marginnote{1.15} sir.” … 

“Is\marginnote{1.17} feeling … perception … choices … consciousness permanent or impermanent?” 

“Impermanent,\marginnote{1.21} sir.” 

“So\marginnote{1.22} you should truly see … Seeing this … They understand: ‘… there is no return to any state of existence.’” 

%
\section*{{\suttatitleacronym SN 22.94}{\suttatitletranslation Flowers }{\suttatitleroot Pupphasutta}}
\addcontentsline{toc}{section}{\tocacronym{SN 22.94} \toctranslation{Flowers } \tocroot{Pupphasutta}}
\markboth{Flowers }{Pupphasutta}
\extramarks{SN 22.94}{SN 22.94}

At\marginnote{1.1} \textsanskrit{Sāvatthī}. 

“Mendicants,\marginnote{1.2} I don’t argue with the world; it’s the world that argues with me. When your speech is in line with the teaching you don’t argue with anyone in the world. What the astute agree on as not existing, I too say does not exist. What the astute agree on as existing, I too say exists. 

And\marginnote{2.1} what do the astute agree on as not existing, which I too say does not exist? Form that is permanent, everlasting, eternal, and imperishable. 

Feeling\marginnote{2.3} … 

Perception\marginnote{2.4} … 

Choices\marginnote{2.5} … 

Consciousness\marginnote{2.6} that is permanent, everlasting, eternal, and imperishable. This is what the astute agree on as not existing, which I too say does not exist. 

And\marginnote{3.1} what do the astute agree on as existing, which I too say exists? Form that is impermanent, suffering, and perishable. 

Feeling\marginnote{3.3} … Perception … Choices … 

Consciousness\marginnote{3.4} that is impermanent, suffering, and perishable. This is what the astute agree on as existing, which I too say exists. 

There\marginnote{4.1} is a temporal phenomenon in the world that the Realized One understands and comprehends.\footnote{“Temporal” plays on the dual meanings of “worldly” and “temporary”. } Then he explains, teaches, asserts, establishes, clarifies, analyzes, and reveals it. 

And\marginnote{5.1} what is that temporal phenomenon in the world? Form is a temporal phenomenon in the world that the Realized One understands and comprehends. Then he explains, teaches, asserts, establishes, clarifies, analyzes, and reveals it. 

This\marginnote{6.1} being so, what can I do about a foolish ordinary person, blind and sightless, who does not know or see? 

Feeling\marginnote{6.2} … 

Perception\marginnote{6.3} … 

Choices\marginnote{6.4} … 

Consciousness\marginnote{6.5} is a temporal phenomenon in the world that the Realized One understands and comprehends. Then he explains, teaches, asserts, establishes, clarifies, analyzes, and reveals it. 

This\marginnote{7.1} being so, what can I do about a foolish ordinary person, blind and sightless, who does not know or see? 

Suppose\marginnote{8.1} there was a blue water lily, or a pink or white lotus. Though it sprouted and grew in the water, it would rise up above the water and stand with no water clinging to it. In the same way, though I was born and grew up in the world, I live having mastered the world, unsullied by the world.” 

%
\section*{{\suttatitleacronym SN 22.95}{\suttatitletranslation A Lump of Foam }{\suttatitleroot Pheṇapiṇḍūpamasutta}}
\addcontentsline{toc}{section}{\tocacronym{SN 22.95} \toctranslation{A Lump of Foam } \tocroot{Pheṇapiṇḍūpamasutta}}
\markboth{A Lump of Foam }{Pheṇapiṇḍūpamasutta}
\extramarks{SN 22.95}{SN 22.95}

At\marginnote{1.1} one time the Buddha was staying near \textsanskrit{Ayojjhā} on the bank of the Ganges river. There the Buddha addressed the mendicants: 

“Mendicants,\marginnote{2.1} suppose this Ganges river was carrying along a big lump of foam. And a person with good eyesight would see it and contemplate it, examining it carefully. And it would appear to them as completely void, hollow, and insubstantial. For what substance could there be in a lump of foam? 

In\marginnote{2.2} the same way, a mendicant sees and contemplates any kind of form at all—past, future, or present; internal or external; coarse or fine; inferior or superior; near or far—examining it carefully. And it appears to them as completely void, hollow, and insubstantial. For what substance could there be in form? 

Suppose\marginnote{3.1} it was the time of autumn, when the rain was falling heavily, and a bubble on the water forms and pops right away. And a person with good eyesight would see it and contemplate it, examining it carefully. And it would appear to them as completely void, hollow, and insubstantial. For what substance could there be in a water bubble? 

In\marginnote{3.2} the same way, a mendicant sees and contemplates any kind of feeling at all … examining it carefully. And it appears to them as completely void, hollow, and insubstantial. For what substance could there be in feeling? 

Suppose\marginnote{4.1} that in the last month of summer, at noon, a shimmering mirage appears. And a person with good eyesight would see it and contemplate it, examining it carefully. And it would appear to them as completely void, hollow, and insubstantial. For what substance could there be in a mirage? 

In\marginnote{4.2} the same way, a mendicant sees and contemplates any kind of perception at all … examining it carefully. And it appears to them as completely void, hollow, and insubstantial. For what substance could there be in perception? 

Suppose\marginnote{5.1} there was a person in need of heartwood. Wandering in search of heartwood, they’d take a sharp axe and enter a forest. There they’d see a big banana tree, straight and young and grown free of defects. They’d cut it down at the base, cut off the top, and unroll the coiled sheaths. But they wouldn’t even find sapwood, much less heartwood. And a person with good eyesight would see it and contemplate it, examining it carefully. And it would appear to them as completely void, hollow, and insubstantial. For what substance could there be in a banana tree?\footnote{Reading akukkuccajata. Here is my note on the corresponding passage at AN 9.196: BB has “without a fruit-bud core” for \textsanskrit{akukkuccakajāta}. It’s not supported by comm or CPD. Evidently he’s following the comm for SN 22.95, where it refers to a banana tree. But I don’t even know what structure “fruit-bud core” refers to. There are multiple readings in the various contexts, but given that it follows ujum navam, I think it’s safe to assume that it’s meant to be the same word. I think it’s also safe to assume it doesn’t refer to any specific botanical structure, as these are very different trees. Most likely the term is a pseudo-synonym of uju and nava, and this fits the etymology of akukkucca, “not-ill-made”. } 

In\marginnote{5.2} the same way, a mendicant sees and contemplates any kind of choices at all … examining them carefully. And they appear to them as completely void, hollow, and insubstantial. For what substance could there be in choices? 

Suppose\marginnote{6.1} a magician or their apprentice was to perform a magic trick at the crossroads. And a person with good eyesight would see it and contemplate it, examining it carefully. And it would appear to them as completely void, hollow, and insubstantial. For what substance could there be in a magic trick? 

In\marginnote{6.2} the same way, a mendicant sees and contemplates any kind of consciousness at all—past, future, or present; internal or external; coarse or fine; inferior or superior; near or far—examining it carefully. And it appears to them as completely void, hollow, and insubstantial. For what substance could there be in consciousness? 

Seeing\marginnote{7.1} this, a learned noble disciple grows disillusioned with form, feeling, perception, choices, and consciousness. Being disillusioned, desire fades away. When desire fades away they’re freed. When they’re freed, they know they’re freed. They understand: ‘… there is no return to any state of existence.’” 

That\marginnote{8.1} is what the Buddha said. Then the Holy One, the Teacher, went on to say: 

\begin{verse}%
“Form\marginnote{9.1} is like a lump of foam; \\
feeling is like a bubble; \\
perception seems like a mirage; \\
choices like a banana tree; \\
and consciousness like a magic trick: \\
so taught the kinsman of the Sun. 

However\marginnote{10.1} you contemplate them, \\
examining them carefully, \\
they’re void and hollow \\
when you look at them closely. 

Concerning\marginnote{11.1} this body, \\
he of vast wisdom has taught \\
that when three things are given up, \\
you’ll see this form discarded. 

Vitality,\marginnote{12.1} warmth, and consciousness: \\
when they leave the body, \\
it lies there tossed aside, \\
food for others, mindless. 

Such\marginnote{13.1} is this process, \\
this illusion, cooed over by fools. \\
It’s said to be a killer, \\
for no substance is found here. 

An\marginnote{14.1} energetic mendicant \\
should examine the aggregates like this, \\
with situational awareness and mindfulness \\
whether by day or by night. 

They\marginnote{15.1} should give up all fetters, \\
and make a refuge for themselves. \\
They should live as though their head was on fire, \\
aspiring to the imperishable state.” 

%
\end{verse}

%
\section*{{\suttatitleacronym SN 22.96}{\suttatitletranslation A Lump of Cow Dung }{\suttatitleroot Gomayapiṇḍasutta}}
\addcontentsline{toc}{section}{\tocacronym{SN 22.96} \toctranslation{A Lump of Cow Dung } \tocroot{Gomayapiṇḍasutta}}
\markboth{A Lump of Cow Dung }{Gomayapiṇḍasutta}
\extramarks{SN 22.96}{SN 22.96}

At\marginnote{1.1} \textsanskrit{Sāvatthī}. 

Seated\marginnote{1.2} to one side, that mendicant said to the Buddha: 

“Sir,\marginnote{1.3} is there any form at all that’s permanent, everlasting, eternal, imperishable, and will last forever and ever? Is there any feeling … perception … choices … consciousness at all that’s permanent, everlasting, eternal, imperishable, and will last forever and ever?” 

“Mendicant,\marginnote{1.8} there is no form at all that’s permanent, everlasting, eternal, imperishable, and will last forever and ever. There’s no feeling … perception … choices … consciousness at all that’s permanent, everlasting, eternal, imperishable, and will last forever and ever.” 

Then\marginnote{2.1} the Buddha, picking up a lump of cow dung, said to the mendicants: 

“There’s\marginnote{2.2} not even this much of any incarnation that’s permanent, everlasting, eternal, imperishable, and will last forever and ever. If there were, this living of the spiritual life for the complete ending of suffering would not be found. But since there isn’t, this living of the spiritual life for the complete ending of suffering is found. 

Once\marginnote{3.1} upon a time I was an anointed aristocratic king. I had 84,000 cities, with the capital \textsanskrit{Kusāvatī} the foremost. I had 84,000 palaces, with the palace named Principle the foremost. I had 84,000 chambers, with the great foyer the foremost. I had 84,000 couches made of ivory or heartwood or gold or silver, spread with woolen covers—shag-piled or embroidered with flowers—and spread with a fine deer hide, with a canopy above and red pillows at both ends.\footnote{PTS has rupiya, with a note that it is absent in Burmese edition. below our text has rupiya, so this must be an oversight. } I had 84,000 bull elephants with gold adornments and banners, covered with gold netting, with the royal bull elephant named Sabbath the foremost. I had 84,000 horses with gold adornments and banners, covered with gold netting, with the royal steed named Thundercloud the foremost. I had 84,000 chariots with gold adornments and banners, covered with gold netting, with the chariot named Triumph the foremost. I had 84,000 jewels, with the jewel-treasure the foremost. I had 84,000 women, with Queen \textsanskrit{Subhaddā} the foremost. I had 84,000 aristocrat vassals, with the counselor-treasure the foremost. I had 84,000 milk cows with silken reins and bronze pails.\footnote{Re. \textsanskrit{dukūlasandhanāni} with vll. as noted here, BB at AN 9.20 has has “jute tethers”, noting “Mp does not provide a gloss and PED does not offer a useful definition under any of those readings. But in PED \textsanskrit{sandāna} is defined as “cord, tether, fetter.””. At SN 22.96 he has “tethers of fine jute”. However, jute is a coarse cloth and can hardly be meant here. Skt dictionaries consistently say “Woven silk, silk-garment, a very fine garment in general”, which fits with our context. The only commentarial gloss I can find for this is that on SN 22.96, which says “\textsanskrit{Dukūlasandānānīti} \textsanskrit{dukūlasantharāni}”. But this yields the sense, “cover or mat”, which may be right, but is less than convincing. At MN 98, a reference missed by PTS dict, we have \textsanskrit{sandāna} as part of the equipment of a horse that is cut by one who is freed. There I render as “reins”, and this seems apt here too. } I had 8,400,000,000 fine cloths of linen, cotton, silk, and wool. I had 84,000 servings of food, which were presented to me as offerings in the morning and evening. 

Of\marginnote{4.1} those 84,000 cities, I only stayed in one, the capital \textsanskrit{Kusāvatī}. Of those 84,000 mansions, I only dwelt in one, the Palace of Principle. Of those 84,000 chambers, I only dwelt in the great foyer. Of those 84,000 couches, I only used one, made of ivory or heartwood or gold or silver. Of those 84,000 bull elephants, I only rode one, the royal bull elephant named Sabbath. Of those 84,000 horses, I only rode one, the royal horse named Thundercloud. Of those 84,000 chariots, I only rode one, the chariot named Triumph. Of those 84,000 women, I was only served by one, a maiden of the aristocratic or merchant classes.\footnote{See https://discourse.suttacentral.net/t/is-there-such-a-word-as-velamika/2952 } Of those 8,400,000,000 cloths, I only wore one pair, made of fine linen, cotton, silk, and wool. Of those 84,000 servings of food, I only had one, eating at most a serving of rice and suitable sauce. 

And\marginnote{4.11} so all those conditioned phenomena have passed, ceased, and perished. So impermanent are conditions, so unstable are conditions, so unreliable are conditions. This is quite enough for you to become disillusioned, dispassionate, and freed regarding all conditions.” 

%
\section*{{\suttatitleacronym SN 22.97}{\suttatitletranslation A Fingernail }{\suttatitleroot Nakhasikhāsutta}}
\addcontentsline{toc}{section}{\tocacronym{SN 22.97} \toctranslation{A Fingernail } \tocroot{Nakhasikhāsutta}}
\markboth{A Fingernail }{Nakhasikhāsutta}
\extramarks{SN 22.97}{SN 22.97}

At\marginnote{1.1} \textsanskrit{Sāvatthī}. 

Seated\marginnote{1.2} to one side, that mendicant said to the Buddha: 

“Sir,\marginnote{1.3} is there any form at all that’s permanent, everlasting, eternal, imperishable, and will last forever and ever? Is there any feeling … perception … choices … consciousness at all that’s permanent, everlasting, eternal, imperishable, and will last forever and ever?” 

“Mendicant,\marginnote{1.8} there is no form at all that’s permanent, everlasting, eternal, imperishable, and will last forever and ever. There’s no feeling … perception … choices … consciousness at all that’s permanent, everlasting, eternal, imperishable, and will last forever and ever.” 

Then\marginnote{2.1} the Buddha, picking up a little bit of dirt under his fingernail, addressed that mendicant: 

“There’s\marginnote{2.2} not even this much of any form that’s permanent, everlasting, eternal, imperishable, and will last forever and ever. If there were, this living of the spiritual life for the complete ending of suffering would not be found. But since there isn’t, this living of the spiritual life for the complete ending of suffering is found. 

There’s\marginnote{3.1} not even this much of any feeling … 

perception\marginnote{4.1} … 

choices\marginnote{4.2} … 

consciousness\marginnote{5.1} that’s permanent, everlasting, eternal, imperishable, and will last forever and ever. If there were, this living of the spiritual life for the complete ending of suffering would not be found. But since there isn’t, this living of the spiritual life for the complete ending of suffering is found. 

What\marginnote{6.1} do you think, mendicant? Is form permanent or impermanent?” 

“Impermanent,\marginnote{6.3} sir.” 

“Is\marginnote{6.4} feeling … perception … choices … consciousness permanent or impermanent?” 

“Impermanent,\marginnote{6.8} sir.” … 

“So\marginnote{6.9} you should truly see … Seeing this … They understand: ‘… there is no return to any state of existence.’” 

%
\section*{{\suttatitleacronym SN 22.98}{\suttatitletranslation Plain Version }{\suttatitleroot Suddhikasutta}}
\addcontentsline{toc}{section}{\tocacronym{SN 22.98} \toctranslation{Plain Version } \tocroot{Suddhikasutta}}
\markboth{Plain Version }{Suddhikasutta}
\extramarks{SN 22.98}{SN 22.98}

At\marginnote{1.1} \textsanskrit{Sāvatthī}. 

Seated\marginnote{1.2} to one side, that mendicant said to the Buddha: 

“Sir,\marginnote{1.3} is there any form at all that’s permanent, everlasting, eternal, imperishable, and will last forever and ever? Is there any feeling … perception … choices … consciousness at all that’s permanent, everlasting, eternal, imperishable, and will last forever and ever?” 

“Mendicant,\marginnote{1.8} there is no form at all that’s permanent, everlasting, eternal, imperishable, and will last forever and ever. There’s no feeling … perception … choices … consciousness at all that’s permanent, everlasting, eternal, imperishable, and will last forever and ever.” 

%
\section*{{\suttatitleacronym SN 22.99}{\suttatitletranslation A Leash }{\suttatitleroot Gaddulabaddhasutta}}
\addcontentsline{toc}{section}{\tocacronym{SN 22.99} \toctranslation{A Leash } \tocroot{Gaddulabaddhasutta}}
\markboth{A Leash }{Gaddulabaddhasutta}
\extramarks{SN 22.99}{SN 22.99}

At\marginnote{1.1} \textsanskrit{Sāvatthī}. 

“Mendicants,\marginnote{1.2} transmigration has no known beginning. No first point is found of sentient beings roaming and transmigrating, shrouded by ignorance and fettered by craving. 

There\marginnote{1.4} comes a time when the ocean dries up and evaporates and is no more. But still, I say, there is no making an end of suffering for sentient beings roaming and transmigrating, shrouded by ignorance and fettered by craving. 

There\marginnote{1.6} comes a time when Sineru the king of mountains is burned up and destroyed, and is no more. But still, I say, there is no making an end of suffering for sentient beings roaming and transmigrating, shrouded by ignorance and fettered by craving. 

There\marginnote{1.8} comes a time when the great earth is burned up and destroyed, and is no more. But still, I say, there is no making an end of suffering for sentient beings roaming and transmigrating, shrouded by ignorance and fettered by craving. 

Suppose\marginnote{2.1} a dog on a leash was tethered to a strong post or pillar. It would just keep running and circling around that post or pillar.\footnote{\textsanskrit{sā} (= Skt \textsanskrit{śvan}) is a short form for “dog”. Masculine! } 

In\marginnote{2.2} the same way, take an unlearned ordinary person who has not seen the noble ones, and is neither skilled nor trained in their teaching. They’ve not seen good persons, and are neither skilled nor trained in their teaching. They regard form … feeling … perception … choices … consciousness as self, self as having consciousness, consciousness in self, or self in consciousness. They just keep running and circling around form, feeling, perception, choices, and consciousness. Doing so, they’re not freed from form, feeling, perception, choices, and consciousness. They’re not freed from rebirth, old age, and death, from sorrow, lamentation, pain, sadness, and distress. They’re not freed from suffering, I say. 

A\marginnote{3.1} learned noble disciple has seen the noble ones, and is skilled and trained in the teaching of the noble ones. They’ve seen good persons, and are skilled and trained in the teaching of the good persons. They don’t regard form … feeling … perception … choices … or consciousness as self, self as having consciousness, consciousness in self, or self in consciousness. They don’t keep running and circling around form, feeling, perception, choices, and consciousness. By not doing so, they’re freed from form, feeling, perception, choices, and consciousness. They’re freed from rebirth, old age, and death, from sorrow, lamentation, pain, sadness, and distress. They’re freed from suffering, I say.” 

%
\section*{{\suttatitleacronym SN 22.100}{\suttatitletranslation A Leash (2nd) }{\suttatitleroot Dutiyagaddulabaddhasutta}}
\addcontentsline{toc}{section}{\tocacronym{SN 22.100} \toctranslation{A Leash (2nd) } \tocroot{Dutiyagaddulabaddhasutta}}
\markboth{A Leash (2nd) }{Dutiyagaddulabaddhasutta}
\extramarks{SN 22.100}{SN 22.100}

At\marginnote{1.1} \textsanskrit{Sāvatthī}. 

“Mendicants,\marginnote{1.2} transmigration has no known beginning. No first point is found of sentient beings roaming and transmigrating, shrouded by ignorance and fettered by craving. Suppose a dog on a leash was tethered to a strong post or pillar. Whether walking, standing, sitting, or lying down, it stays right beside that post or pillar. 

In\marginnote{1.5} the same way, an unlearned ordinary person regards form like this: ‘This is mine, I am this, this is my self.’ They regard feeling … perception … choices … consciousness like this: ‘This is mine, I am this, this is my self.’ When walking, they walk right beside the five grasping aggregates. When standing … sitting … lying down, they lie down right beside the five grasping aggregates. 

So\marginnote{1.14} you should regularly check your own mind: ‘For a long time this mind has been corrupted by greed, hate, and delusion.’ Sentient beings are corrupted because the mind is corrupted. Sentient beings are purified because the mind is purified. 

Mendicants,\marginnote{2.1} have you seen the picture called ‘Conduct’?”\footnote{See BB’s note for the name of the picture. The commentarial explanation is hardly plausible. Surely the picture must, like very many illustrations in the ancient world, be a pedagogic device, where various kinds of conduct are depicted, perhaps with their results. } 

“Yes,\marginnote{2.2} sir.” 

“That\marginnote{2.3} picture was elaborated by the mind, but the mind is even more elaborate than that.\footnote{It is of course impossible to capture in English the pun here between citta = mind and citta = colorful. } 

So\marginnote{2.4} you should regularly check your own mind: ‘For a long time this mind has been corrupted by greed, hate, and delusion.’ Sentient beings are corrupted because the mind is corrupted. Sentient beings are purified because the mind is purified. 

I\marginnote{3.1} don’t see any other order of beings as elaborate as the animal realm.\footnote{MS punctuation is wrong, I have corrected it. } The creatures in the animal realm were elaborated by the mind, but the mind is even more elaborate than that. 

So\marginnote{3.3} you should regularly check your own mind: ‘For a long time this mind has been corrupted by greed, hate, and delusion.’ Sentient beings are corrupted because the mind is corrupted. Sentient beings are purified because the mind is purified. 

Suppose\marginnote{4.1} an artist or painter had some dye, red lac, turmeric, indigo, or rose madder. And on a polished plank or a wall or a canvas they’d create the image of a woman or a man, complete in all its various parts. 

In\marginnote{4.2} the same way, when an unlearned ordinary person creates a future life, all they create is form, feeling, perception, choices, and consciousness. 

What\marginnote{4.3} do you think, mendicants? Is form permanent or impermanent?” 

“Impermanent,\marginnote{4.5} sir.” 

“Is\marginnote{4.6} feeling … perception … choices … consciousness permanent or impermanent?” … 

“So\marginnote{4.10} you should truly see … Seeing this … They understand: ‘… there is no return to any state of existence.’” 

%
\section*{{\suttatitleacronym SN 22.101}{\suttatitletranslation The Adze }{\suttatitleroot Vāsijaṭasutta}}
\addcontentsline{toc}{section}{\tocacronym{SN 22.101} \toctranslation{The Adze } \tocroot{Vāsijaṭasutta}}
\markboth{The Adze }{Vāsijaṭasutta}
\extramarks{SN 22.101}{SN 22.101}

At\marginnote{1.1} \textsanskrit{Sāvatthī}. 

“Mendicants,\marginnote{1.2} I say that the ending of defilements is for one who knows and sees, not for one who does not know or see. For one who knows and sees what? ‘Such is form, such is the origin of form, such is the ending of form. Such is feeling … Such is perception … Such are choices … Such is consciousness, such is the origin of consciousness, such is the ending of consciousness.’ The ending of the defilements is for one who knows and sees this. 

When\marginnote{2.1} a mendicant is not committed to development, they might wish: ‘If only my mind was freed from the defilements by not grasping!’ Even so, their mind is not freed from defilements by not grasping. Why is that? You should say: ‘It’s because they’re undeveloped.’ Undeveloped in what? Undeveloped in the four kinds of mindfulness meditation, the four right efforts, the four bases of psychic power, the five faculties, the five powers, the seven awakening factors, and the noble eightfold path.\footnote{Note this passage is identical with AN 7.71, except “undeveloped” is absent there. } 

Suppose\marginnote{3.1} there was a chicken with eight or ten or twelve eggs. But she had not properly sat on them to keep them warm and incubated. That chicken might wish: ‘If only my chicks could break out of the eggshell with their claws and beak and hatch safely!’ But they can’t break out and hatch safely. Why is that? Because that chicken with eight or ten or twelve eggs has not properly sat on them to keep them warm and incubated. 

In\marginnote{3.9} the same way, when a mendicant is not committed to development, they might wish: ‘If only my mind was freed from the defilements by not grasping!’ Even so, their mind is not freed from defilements by not grasping. Why is that? You should say: ‘It’s because they’re undeveloped.’ Undeveloped in what? Undeveloped in the four kinds of mindfulness meditation, the four right efforts, the four bases of psychic power, the five faculties, the five powers, the seven awakening factors, and the noble eightfold path. 

When\marginnote{4.1} a mendicant is committed to development, they might not wish: ‘If only my mind was freed from the defilements by not grasping!’ Even so, their mind is freed from defilements by not grasping. Why is that? You should say: ‘It’s because they are developed.’ Developed in what? Developed in the four kinds of mindfulness meditation, the four right efforts, the four bases of psychic power, the five faculties, the five powers, the seven awakening factors, and the noble eightfold path. 

Suppose\marginnote{5.1} there was a chicken with eight or ten or twelve eggs. And she properly sat on them to keep them warm and incubated. That chicken might not wish: ‘If only my chicks could break out of the eggshell with their claws and beak and hatch safely!’ But still they can break out and hatch safely. Why is that? Because that chicken with eight or ten or twelve eggs properly sat on them to keep them warm and incubated. 

In\marginnote{5.9} the same way, when a mendicant is committed to development, they might not wish: ‘If only my mind was freed from the defilements by not grasping!’ Even so, their mind is freed from defilements by not grasping. Why is that? You should say: ‘It’s because they are developed.’ Developed in what? Developed in the four kinds of mindfulness meditation, the four right efforts, the four bases of psychic power, the five faculties, the five powers, the seven awakening factors, and the noble eightfold path. 

Suppose\marginnote{6.1} a carpenter or their apprentice sees the marks of his fingers and thumb on the handle of his adze. They don’t know how much of the handle was worn away today, how much yesterday, and how much previously. They just know what has been worn away. 

In\marginnote{6.5} the same way, when a mendicant is committed to development, they don’t know how much of the defilements were worn away today, how much yesterday, and how much previously. They just know what has been worn away. Suppose there was a sea-faring ship bound together with ropes. For six months they deteriorated in the water. Then in the cold season it was hauled up on dry land, where the ropes were weathered by wind and sun. When the clouds soaked it with rain, the ropes would readily collapse and rot away.\footnote{The reading vassamassani is probably a mistake. Other manuscripts have cha \textsanskrit{māsani}, and MS has that at SN 45.158, AN 7.71 too. } In the same way, when a mendicant is committed to development their fetters readily collapse and rot away.” 

%
\section*{{\suttatitleacronym SN 22.102}{\suttatitletranslation The Perception of Impermanence }{\suttatitleroot Aniccasaññāsutta}}
\addcontentsline{toc}{section}{\tocacronym{SN 22.102} \toctranslation{The Perception of Impermanence } \tocroot{Aniccasaññāsutta}}
\markboth{The Perception of Impermanence }{Aniccasaññāsutta}
\extramarks{SN 22.102}{SN 22.102}

At\marginnote{1.1} \textsanskrit{Sāvatthī}. 

“Mendicants,\marginnote{1.2} when the perception of impermanence is developed and cultivated it eliminates all desire for sensual pleasures, for rebirth in the realm of luminous form, and for rebirth in a future life. It eliminates all ignorance and eradicates all conceit ‘I am’.\footnote{The appearance here of both ruparaga and bhavaraga is unusual, perhaps unique. Normally we find ruparaga and aruparaga. } 

In\marginnote{2.1} the autumn, a farmer ploughing with a large plough shears through all the root networks.\footnote{Not sure why BB has “ploughshare” here. The ploughshare is the blade, which is \textsanskrit{nangalaphāla}. Nangala must mean the whole plough, which can be carried on a shoulder as in SN 4.19. Also, rootlets is a mistake. Rootlets are the young sprouts, but the word here is \textsanskrit{mūlasantānakāni}, where santanaka means “continuity”, i.e. “network”. } In the same way, when the perception of impermanence is developed … it eradicates all conceit ‘I am’. 

A\marginnote{3.1} reed-cutter, having cut the reeds, grabs them at the top and shakes them down, shakes them about, and shakes them off.\footnote{See my note on AN 6.53. } In the same way, when the perception of impermanence is developed … it eradicates all conceit ‘I am’. 

When\marginnote{4.1} the stalk of a bunch of mangoes is cut, all the mangoes attached to the stalk will follow along. In the same way, when the perception of impermanence is developed … it eradicates all conceit ‘I am’. 

The\marginnote{5.1} rafters of a bungalow all lean to the peak, slope to the peak, and meet at the peak, so the peak is said to be the topmost of them all. In the same way, when the perception of impermanence is developed … it eradicates all conceit ‘I am’. 

Of\marginnote{6.1} all kinds of fragrant root, spikenard is said to be the best. In the same way, when the perception of impermanence is developed … it eradicates all conceit ‘I am’. 

Of\marginnote{7.1} all kinds of fragrant heartwood, red sandalwood is said to be the best. In the same way, when the perception of impermanence is developed … it eradicates all conceit ‘I am’. 

Of\marginnote{8.1} all kinds of fragrant flower, jasmine is said to be the best. In the same way, when the perception of impermanence is developed … it eradicates all conceit ‘I am’. 

All\marginnote{9.1} lesser kings are vassals of a wheel-turning monarch, so the wheel-turning monarch is said to be the foremost of them all. In the same way, when the perception of impermanence is developed … it eradicates all conceit ‘I am’. 

The\marginnote{10.1} radiance of all the stars is not worth a sixteenth part of the moon’s radiance, so the moon’s radiance is said to be the best of them all. In the same way, when the perception of impermanence is developed … it eradicates all conceit ‘I am’. 

After\marginnote{11.1} the rainy season the sky is clear and cloudless. And when the sun rises, it dispels all the darkness from the sky as it shines and glows and radiates. In the same way, when the perception of impermanence is developed and cultivated it eliminates all desire for sensual pleasures, for rebirth in the realm of luminous form, and for rebirth in a future life. It eliminates all ignorance and eradicates all conceit ‘I am’. 

And\marginnote{12.1} how is the perception of impermanence developed and cultivated so that … it eradicates all conceit ‘I am’? ‘Such is form, such is the origin of form, such is the ending of form. Such is feeling … Such is perception … Such are choices … Such is consciousness, such is the origin of consciousness, such is the ending of consciousness.’ 

That’s\marginnote{12.7} how the perception of impermanence is developed and cultivated so that it eliminates all desire for sensual pleasures, for rebirth in the realm of luminous form, and for rebirth in a future life. That’s how it eliminates all ignorance and eradicates all conceit ‘I am’.” 

%
\addtocontents{toc}{\let\protect\contentsline\protect\nopagecontentsline}
\pannasa{The Final Fifty }
\addcontentsline{toc}{pannasa}{The Final Fifty }
\markboth{}{}
\addtocontents{toc}{\let\protect\contentsline\protect\oldcontentsline}

%
\addtocontents{toc}{\let\protect\contentsline\protect\nopagecontentsline}
\chapter*{The Chapter on Sides }
\addcontentsline{toc}{chapter}{\tocchapterline{The Chapter on Sides }}
\addtocontents{toc}{\let\protect\contentsline\protect\oldcontentsline}

%
\section*{{\suttatitleacronym SN 22.103}{\suttatitletranslation Sides }{\suttatitleroot Antasutta}}
\addcontentsline{toc}{section}{\tocacronym{SN 22.103} \toctranslation{Sides } \tocroot{Antasutta}}
\markboth{Sides }{Antasutta}
\extramarks{SN 22.103}{SN 22.103}

At\marginnote{1.1} \textsanskrit{Sāvatthī}. 

“Mendicants,\marginnote{1.2} there are these four sides. What four? The side of identity, the side of the origin of identity, the side of the cessation of identity, and the side of the practice that leads to the cessation of identity. And what is the side of identity? It should be said: the five grasping aggregates. What five? That is, the grasping aggregates of form, feeling, perception, choices, and consciousness.\footnote{This seems clumsy here. } This is called the side of identity. 

And\marginnote{2.1} what is the side of the origin of identity? It’s the craving that leads to future lives, mixed up with relishing and greed, chasing pleasure in various realms. That is, craving for sensual pleasures, craving to continue existence, and craving to end existence. This is called the side of the origin of identity. 

And\marginnote{3.1} what is the side of the cessation of identity? It’s the fading away and cessation of that very same craving with nothing left over; giving it away, letting it go, releasing it, and not adhering to it. This is called the side of the cessation of identity. 

And\marginnote{4.1} what is the side of the practice that leads to the cessation of identity? It is simply this noble eightfold path, that is: right view, right thought, right speech, right action, right livelihood, right effort, right mindfulness, and right immersion. This is called the side of the practice that leads to the cessation of identity. These are the four sides.” 

%
\section*{{\suttatitleacronym SN 22.104}{\suttatitletranslation Suffering }{\suttatitleroot Dukkhasutta}}
\addcontentsline{toc}{section}{\tocacronym{SN 22.104} \toctranslation{Suffering } \tocroot{Dukkhasutta}}
\markboth{Suffering }{Dukkhasutta}
\extramarks{SN 22.104}{SN 22.104}

At\marginnote{1.1} \textsanskrit{Sāvatthī}. 

“Mendicants,\marginnote{1.2} I will teach you suffering, the origin of suffering, the cessation of suffering, and the practice that leads to the cessation of suffering. Listen … 

And\marginnote{1.4} what is suffering? It should be said: the five grasping aggregates. What five? That is, the grasping aggregates of form, feeling, perception, choices, and consciousness. This is called suffering. 

And\marginnote{1.9} what is the origin of suffering? It’s the craving that leads to future lives, mixed up with relishing and greed, chasing pleasure in various realms. That is, craving for sensual pleasures, craving to continue existence, and craving to end existence. This is called the origin of suffering. 

And\marginnote{1.12} what is the cessation of suffering? It’s the fading away and cessation of that very same craving with nothing left over; giving it away, letting it go, releasing it, and not adhering to it. This is called the cessation of suffering. 

And\marginnote{1.15} what is the practice that leads to the cessation of suffering? It is simply this noble eightfold path, that is: right view, right thought, right speech, right action, right livelihood, right effort, right mindfulness, and right immersion. This is called the practice that leads to the cessation of suffering.” 

%
\section*{{\suttatitleacronym SN 22.105}{\suttatitletranslation Identity }{\suttatitleroot Sakkāyasutta}}
\addcontentsline{toc}{section}{\tocacronym{SN 22.105} \toctranslation{Identity } \tocroot{Sakkāyasutta}}
\markboth{Identity }{Sakkāyasutta}
\extramarks{SN 22.105}{SN 22.105}

At\marginnote{1.1} \textsanskrit{Sāvatthī}. 

“Mendicants,\marginnote{1.2} I will teach you identity, the origin of identity, the cessation of identity, and the practice that leads to the cessation of identity. Listen … 

And\marginnote{1.4} what is identity? It should be said: the five grasping aggregates. What five? That is, the grasping aggregates of form, feeling, perception, choices, and consciousness. This is called identity. 

And\marginnote{1.9} what is the origin of identity? It’s the craving that leads to future lives, mixed up with relishing and greed, chasing pleasure in various realms. That is, craving for sensual pleasures, craving to continue existence, and craving to end existence. This is called the origin of identity. 

And\marginnote{1.12} what is the cessation of identity? It’s the fading away and cessation of that very same craving with nothing left over; giving it away, letting it go, releasing it, and not clinging to it. This is called the cessation of identity. 

And\marginnote{1.15} what is the practice that leads to the cessation of identity? It is simply this noble eightfold path, that is: right view, right thought, right speech, right action, right livelihood, right effort, right mindfulness, and right immersion. This is called the practice that leads to the cessation of identity.” 

%
\section*{{\suttatitleacronym SN 22.106}{\suttatitletranslation Should Be Completely Understood }{\suttatitleroot Pariññeyyasutta}}
\addcontentsline{toc}{section}{\tocacronym{SN 22.106} \toctranslation{Should Be Completely Understood } \tocroot{Pariññeyyasutta}}
\markboth{Should Be Completely Understood }{Pariññeyyasutta}
\extramarks{SN 22.106}{SN 22.106}

At\marginnote{1.1} \textsanskrit{Sāvatthī}. 

“Mendicants,\marginnote{1.2} I will teach you the things that should be completely understood, complete understanding, and the person who has completely understood. Listen … 

And\marginnote{1.4} what things should be completely understood? Form, feeling, perception, choices, and consciousness. These are called the things that should be completely understood. 

And\marginnote{1.11} what is complete understanding? The ending of greed, hate, and delusion.\footnote{It is from this definition that I prefer “complete “to” full”. Complete has a suggestion of “to the very end” which is slightly different to the connotations of full. } This is called complete understanding. 

And\marginnote{1.14} what is the person who has completely understood? It should be said: a perfected one, the venerable of such and such name and clan. This is called the person who has completely understood.” 

%
\section*{{\suttatitleacronym SN 22.107}{\suttatitletranslation Ascetics (1st) }{\suttatitleroot Samaṇasutta}}
\addcontentsline{toc}{section}{\tocacronym{SN 22.107} \toctranslation{Ascetics (1st) } \tocroot{Samaṇasutta}}
\markboth{Ascetics (1st) }{Samaṇasutta}
\extramarks{SN 22.107}{SN 22.107}

At\marginnote{1.1} \textsanskrit{Sāvatthī}. 

“Mendicants,\marginnote{1.2} there are these five grasping aggregates. What five? That is, the grasping aggregates of form, feeling, perception, choices, and consciousness. 

There\marginnote{1.5} are ascetics and brahmins who don’t truly understand these five grasping aggregates’ gratification, drawback, and escape … 

There\marginnote{1.6} are ascetics and brahmins who do truly understand …”\footnote{Pali text is incorrectly punctuated here, there should be a … } 

%
\section*{{\suttatitleacronym SN 22.108}{\suttatitletranslation Ascetics (2nd) }{\suttatitleroot Dutiyasamaṇasutta}}
\addcontentsline{toc}{section}{\tocacronym{SN 22.108} \toctranslation{Ascetics (2nd) } \tocroot{Dutiyasamaṇasutta}}
\markboth{Ascetics (2nd) }{Dutiyasamaṇasutta}
\extramarks{SN 22.108}{SN 22.108}

At\marginnote{1.1} \textsanskrit{Sāvatthī}. 

“Mendicants,\marginnote{1.2} there are these five grasping aggregates. What five? That is, the grasping aggregates of form, feeling, perception, choices, and consciousness. 

There\marginnote{1.5} are ascetics and brahmins who don’t truly understand these five grasping aggregates’ origin, ending, gratification, drawback, and escape … Those venerables don’t realize the goal of life as an ascetic or brahmin … 

There\marginnote{1.6} are ascetics and brahmins who do truly understand … Those venerables realize the goal of life as an ascetic or brahmin, and live having realized it with their own insight.” 

%
\section*{{\suttatitleacronym SN 22.109}{\suttatitletranslation A Stream-Enterer }{\suttatitleroot Sotāpannasutta}}
\addcontentsline{toc}{section}{\tocacronym{SN 22.109} \toctranslation{A Stream-Enterer } \tocroot{Sotāpannasutta}}
\markboth{A Stream-Enterer }{Sotāpannasutta}
\extramarks{SN 22.109}{SN 22.109}

At\marginnote{1.1} \textsanskrit{Sāvatthī}. 

“Mendicants,\marginnote{1.2} there are these five grasping aggregates. What five? That is, the grasping aggregates of form, feeling, perception, choices, and consciousness. A noble disciple comes to truly understand these five grasping aggregates’ origin, ending, gratification, drawback, and escape. Such a noble disciple is called a stream-enterer, not liable to be reborn in the underworld, bound for awakening.” 

%
\section*{{\suttatitleacronym SN 22.110}{\suttatitletranslation A Perfected One }{\suttatitleroot Arahantasutta}}
\addcontentsline{toc}{section}{\tocacronym{SN 22.110} \toctranslation{A Perfected One } \tocroot{Arahantasutta}}
\markboth{A Perfected One }{Arahantasutta}
\extramarks{SN 22.110}{SN 22.110}

At\marginnote{1.1} \textsanskrit{Sāvatthī}. 

“Mendicants,\marginnote{1.2} there are these five grasping aggregates. What five? That is, the grasping aggregates of form, feeling, perception, choices, and consciousness. A mendicant comes to be freed by not grasping after truly understanding these five grasping aggregates’ origin, ending, gratification, drawback, and escape. Such a mendicant is called a perfected one, with defilements ended, who has completed the spiritual journey, done what had to be done, laid down the burden, achieved their own true goal, utterly ended the fetters of rebirth, and is rightly freed through enlightenment.” 

%
\section*{{\suttatitleacronym SN 22.111}{\suttatitletranslation Giving Up Desire }{\suttatitleroot Chandappahānasutta}}
\addcontentsline{toc}{section}{\tocacronym{SN 22.111} \toctranslation{Giving Up Desire } \tocroot{Chandappahānasutta}}
\markboth{Giving Up Desire }{Chandappahānasutta}
\extramarks{SN 22.111}{SN 22.111}

At\marginnote{1.1} \textsanskrit{Sāvatthī}. 

“Mendicants,\marginnote{1.2} you should give up any desire, greed, relishing, and craving for form. Thus that form will be given up, cut off at the root, made like a palm stump, obliterated, and unable to arise in the future. 

You\marginnote{1.4} should give up any desire, greed, relishing, and craving for feeling … perception … choices … consciousness. Thus that consciousness will be given up, cut off at the root, made like a palm stump, obliterated, and unable to arise in the future.” 

%
\section*{{\suttatitleacronym SN 22.112}{\suttatitletranslation Giving Up Desire (2nd) }{\suttatitleroot Dutiyachandappahānasutta}}
\addcontentsline{toc}{section}{\tocacronym{SN 22.112} \toctranslation{Giving Up Desire (2nd) } \tocroot{Dutiyachandappahānasutta}}
\markboth{Giving Up Desire (2nd) }{Dutiyachandappahānasutta}
\extramarks{SN 22.112}{SN 22.112}

At\marginnote{1.1} \textsanskrit{Sāvatthī}. 

“Mendicants,\marginnote{1.2} you should give up any desire, greed, relishing, and craving for form; and any attraction, grasping, mental fixation, insistence, and underlying tendencies. Thus that form will be given up, cut off at the root, made like a palm stump, obliterated, and unable to arise in the future. 

You\marginnote{1.4} should give up any desire, greed, relishing, and craving for feeling … perception … choices … consciousness; and any attraction, grasping, mental fixation, insistence, and underlying tendencies. Thus that consciousness will be given up, cut off at the root, made like a palm stump, obliterated, and unable to arise in the future.” 

%
\addtocontents{toc}{\let\protect\contentsline\protect\nopagecontentsline}
\chapter*{The Chapter on a Dhamma Speaker }
\addcontentsline{toc}{chapter}{\tocchapterline{The Chapter on a Dhamma Speaker }}
\addtocontents{toc}{\let\protect\contentsline\protect\oldcontentsline}

%
\section*{{\suttatitleacronym SN 22.113}{\suttatitletranslation Ignorance }{\suttatitleroot Avijjāsutta}}
\addcontentsline{toc}{section}{\tocacronym{SN 22.113} \toctranslation{Ignorance } \tocroot{Avijjāsutta}}
\markboth{Ignorance }{Avijjāsutta}
\extramarks{SN 22.113}{SN 22.113}

At\marginnote{1.1} \textsanskrit{Sāvatthī}. 

Then\marginnote{1.2} a mendicant went up to the Buddha and said to him: 

“Sir,\marginnote{1.4} they speak of this thing called ‘ignorance’. What is ignorance? And how is an ignorant person defined?” 

“Mendicant,\marginnote{1.7} it’s when an unlearned ordinary person doesn’t understand form, its origin, its cessation, and the practice that leads to its cessation. They don’t understand feeling … perception … choices … consciousness, its origin, its cessation, and the practice that leads to its cessation. 

This\marginnote{1.12} is called ignorance. And this is how an ignorant person is defined.” 

%
\section*{{\suttatitleacronym SN 22.114}{\suttatitletranslation Knowledge }{\suttatitleroot Vijjāsutta}}
\addcontentsline{toc}{section}{\tocacronym{SN 22.114} \toctranslation{Knowledge } \tocroot{Vijjāsutta}}
\markboth{Knowledge }{Vijjāsutta}
\extramarks{SN 22.114}{SN 22.114}

At\marginnote{1.1} \textsanskrit{Sāvatthī}. 

Seated\marginnote{1.2} to one side, that mendicant said to the Buddha: 

“Sir,\marginnote{1.3} they speak of this thing called ‘knowledge’. What is knowledge? And how is a knowledgeable person defined?” 

“Mendicant,\marginnote{1.6} it’s when a learned noble disciple understands form, its origin, its cessation, and the practice that leads to its cessation. They understand feeling … perception … choices … consciousness, its origin, its cessation, and the practice that leads to its cessation. 

This\marginnote{1.11} is called knowledge. And this is how a knowledgeable person is defined.” 

%
\section*{{\suttatitleacronym SN 22.115}{\suttatitletranslation A Dhamma speaker }{\suttatitleroot Dhammakathikasutta}}
\addcontentsline{toc}{section}{\tocacronym{SN 22.115} \toctranslation{A Dhamma speaker } \tocroot{Dhammakathikasutta}}
\markboth{A Dhamma speaker }{Dhammakathikasutta}
\extramarks{SN 22.115}{SN 22.115}

At\marginnote{1.1} \textsanskrit{Sāvatthī}. 

Seated\marginnote{1.2} to one side, that mendicant said to the Buddha: 

“Sir,\marginnote{1.3} they speak of a ‘Dhamma speaker’. How is a Dhamma speaker defined?” 

“Mendicant,\marginnote{1.5} if a mendicant teaches Dhamma for disillusionment, dispassion, and cessation regarding form, they’re qualified to be called a ‘mendicant who speaks on Dhamma’. 

If\marginnote{1.6} they practice for disillusionment, dispassion, and cessation regarding form, they’re qualified to be called a ‘mendicant who practices in line with the teaching’. 

If\marginnote{1.7} they’re freed by not grasping by disillusionment, dispassion, and cessation regarding form, they’re qualified to be called a ‘mendicant who has attained extinguishment in this very life’. 

If\marginnote{1.8} a mendicant teaches Dhamma for disillusionment with feeling … perception … choices … consciousness, for its fading away and cessation, they’re qualified to be called a ‘mendicant who speaks on Dhamma’. 

If\marginnote{1.12} they practice for disillusionment, dispassion, and cessation regarding consciousness, they’re qualified to be called a ‘mendicant who practices in line with the teaching’. 

If\marginnote{1.13} they’re freed by not grasping by disillusionment, dispassion, and cessation regarding consciousness, they’re qualified to be called a ‘mendicant who has attained extinguishment in this very life’.” 

%
\section*{{\suttatitleacronym SN 22.116}{\suttatitletranslation A Dhamma speaker (2nd) }{\suttatitleroot Dutiyadhammakathikasutta}}
\addcontentsline{toc}{section}{\tocacronym{SN 22.116} \toctranslation{A Dhamma speaker (2nd) } \tocroot{Dutiyadhammakathikasutta}}
\markboth{A Dhamma speaker (2nd) }{Dutiyadhammakathikasutta}
\extramarks{SN 22.116}{SN 22.116}

At\marginnote{1.1} \textsanskrit{Sāvatthī}. 

Seated\marginnote{1.2} to one side, that mendicant said to the Buddha: 

“Sir,\marginnote{1.3} they speak of a ‘Dhamma speaker’. How is a Dhamma speaker defined? How is a mendicant who practices in line with the teaching defined? And how is a mendicant who has attained extinguishment in this very life defined?” 

“Mendicant,\marginnote{1.6} if a mendicant teaches Dhamma for disillusionment, dispassion, and cessation regarding form, they’re qualified to be called a ‘mendicant who speaks on Dhamma’. 

If\marginnote{1.7} they practice for disillusionment, dispassion, and cessation regarding form, they’re qualified to be called a ‘mendicant who practices in line with the teaching’. 

If\marginnote{1.8} they’re freed by not grasping by disillusionment, dispassion, and cessation regarding form, they’re qualified to be called a ‘mendicant who has attained extinguishment in this very life’. 

If\marginnote{1.9} a mendicant teaches Dhamma for disillusionment with feeling … perception … choices … consciousness, for its fading away and cessation, they’re qualified to be called a ‘mendicant who speaks on Dhamma’. 

If\marginnote{1.13} they practice for disillusionment, dispassion, and cessation regarding consciousness, they’re qualified to be called a ‘mendicant who practices in line with the teaching’. 

If\marginnote{1.14} they’re freed by not grasping by disillusionment, dispassion, and cessation regarding consciousness, they’re qualified to be called a ‘mendicant who has attained extinguishment in this very life’.” 

%
\section*{{\suttatitleacronym SN 22.117}{\suttatitletranslation Shackles }{\suttatitleroot Bandhanasutta}}
\addcontentsline{toc}{section}{\tocacronym{SN 22.117} \toctranslation{Shackles } \tocroot{Bandhanasutta}}
\markboth{Shackles }{Bandhanasutta}
\extramarks{SN 22.117}{SN 22.117}

At\marginnote{1.1} \textsanskrit{Sāvatthī}. 

“Mendicants,\marginnote{1.2} take an unlearned ordinary person who has not seen the noble ones, and is neither skilled nor trained in the teaching of the noble ones. They’ve not seen good persons, and are neither skilled nor trained in the teaching of the good persons. They regard form as self, self as having form, form in self, or self in form. They’re called an unlearned ordinary person who is bound to form, inside and out. They see neither the near shore nor the far shore. They’re born in bonds and die in bonds, and in bonds they go from this world to the next.\footnote{Reading \textsanskrit{jāyati}, contra BB. The parallel at SA 74 supports jayati: 以縛生,以縛死 } 

They\marginnote{1.5} regard feeling … perception … choices … consciousness as self. They’re called an unlearned ordinary person who is bound to consciousness, inside and out. They see neither the near shore nor the far shore. They’re born in bonds and die in bonds, and in bonds they go from this world to the next. 

A\marginnote{2.1} learned noble disciple has seen the noble ones, and is skilled and trained in the teaching of the noble ones. They’ve seen good persons, and are skilled and trained in the teaching of the good persons. They don’t regard form as self, self as having form, form in self, or self in form. They’re called a learned noble disciple who is not bound to form, inside or out. They see the near shore and the far shore. They’re exempt from suffering, I say. 

They\marginnote{2.4} don’t regard feeling … perception … choices … consciousness as self. They’re called a learned noble disciple who is not bound to consciousness, inside or out. They see the near shore and the far shore. They’re exempt from suffering, I say.” 

%
\section*{{\suttatitleacronym SN 22.118}{\suttatitletranslation Questioning }{\suttatitleroot Paripucchitasutta}}
\addcontentsline{toc}{section}{\tocacronym{SN 22.118} \toctranslation{Questioning } \tocroot{Paripucchitasutta}}
\markboth{Questioning }{Paripucchitasutta}
\extramarks{SN 22.118}{SN 22.118}

At\marginnote{1.1} \textsanskrit{Sāvatthī}. 

“What\marginnote{1.2} do you think, mendicants? Do you regard form like this: ‘This is mine, I am this, this is my self’?” 

“No,\marginnote{1.4} sir.” 

“Good,\marginnote{1.5} mendicants! Form should be truly seen with right understanding like this: ‘This is not mine, I am not this, this is not my self.’ Do you regard feeling … perception … choices … consciousness like this: ‘This is mine, I am this, this is my self’?” 

“No,\marginnote{1.11} sir.” 

“Good,\marginnote{1.12} mendicants! Consciousness should be truly seen with right understanding like this: ‘This is not mine, I am not this, this is not my self.’ 

Seeing\marginnote{1.14} this … They understand: ‘… there is no return to any state of existence.’” 

%
\section*{{\suttatitleacronym SN 22.119}{\suttatitletranslation Questioning (2nd) }{\suttatitleroot Dutiyaparipucchitasutta}}
\addcontentsline{toc}{section}{\tocacronym{SN 22.119} \toctranslation{Questioning (2nd) } \tocroot{Dutiyaparipucchitasutta}}
\markboth{Questioning (2nd) }{Dutiyaparipucchitasutta}
\extramarks{SN 22.119}{SN 22.119}

At\marginnote{1.1} \textsanskrit{Sāvatthī}. 

“What\marginnote{1.2} do you think, mendicants? Do you regard form like this: ‘This is not mine, I am not this, this is not my self’?” 

“Yes,\marginnote{1.4} sir.” 

“Good,\marginnote{1.5} mendicants! Form should be truly seen with right understanding like this: ‘This is not mine, I am not this, this is not my self.’ Do you regard feeling … perception … choices … consciousness like this: ‘This is not mine, I am not this, this is not my self’?” 

“Yes,\marginnote{1.11} sir.” 

“Good,\marginnote{1.12} mendicants! Consciousness should be truly seen with right understanding like this: ‘This is not mine, I am not this, this is not my self.’ 

Seeing\marginnote{1.14} this … They understand: ‘… there is no return to any state of existence.’” 

%
\section*{{\suttatitleacronym SN 22.120}{\suttatitletranslation Things Prone To Being Fettered }{\suttatitleroot Saṁyojaniyasutta}}
\addcontentsline{toc}{section}{\tocacronym{SN 22.120} \toctranslation{Things Prone To Being Fettered } \tocroot{Saṁyojaniyasutta}}
\markboth{Things Prone To Being Fettered }{Saṁyojaniyasutta}
\extramarks{SN 22.120}{SN 22.120}

At\marginnote{1.1} \textsanskrit{Sāvatthī}. 

“Mendicants,\marginnote{1.2} I will teach you the things that are prone to being fettered, and the fetter. Listen … 

What\marginnote{1.4} are the things that are prone to being fettered? And what is the fetter? 

Form\marginnote{1.5} is something that’s prone to being fettered. The desire and greed for it is the fetter. 

Feeling\marginnote{1.7} … 

Perception\marginnote{1.8} … 

Choices\marginnote{1.9} … 

Consciousness\marginnote{1.10} is something that’s prone to being fettered. The desire and greed for it is the fetter. 

These\marginnote{1.12} are called the things that are prone to being fettered, and this is the fetter.” 

%
\section*{{\suttatitleacronym SN 22.121}{\suttatitletranslation Things Prone To Being Grasped }{\suttatitleroot Upādāniyasutta}}
\addcontentsline{toc}{section}{\tocacronym{SN 22.121} \toctranslation{Things Prone To Being Grasped } \tocroot{Upādāniyasutta}}
\markboth{Things Prone To Being Grasped }{Upādāniyasutta}
\extramarks{SN 22.121}{SN 22.121}

At\marginnote{1.1} \textsanskrit{Sāvatthī}. 

“Mendicants,\marginnote{1.2} I will teach you the things that are prone to being grasped, and the grasping. Listen … 

What\marginnote{1.4} are the things that are prone to being grasped? And what is the grasping? 

Form\marginnote{1.5} is something that’s prone to being grasped. The desire and greed for it is the grasping. 

Feeling\marginnote{1.7} … 

Perception\marginnote{1.8} … 

Choices\marginnote{1.9} … 

Consciousness\marginnote{1.10} is something that’s prone to being grasped. The desire and greed for it is the grasping. 

These\marginnote{1.12} are called the things that are prone to being grasped, and this is the grasping.” 

%
\section*{{\suttatitleacronym SN 22.122}{\suttatitletranslation An Ethical Mendicant }{\suttatitleroot Sīlavantasutta}}
\addcontentsline{toc}{section}{\tocacronym{SN 22.122} \toctranslation{An Ethical Mendicant } \tocroot{Sīlavantasutta}}
\markboth{An Ethical Mendicant }{Sīlavantasutta}
\extramarks{SN 22.122}{SN 22.122}

At\marginnote{1.1} one time Venerable \textsanskrit{Sāriputta} and Venerable \textsanskrit{Mahākoṭṭhita} were staying near Benares, in the deer park at Isipatana. Then in the late afternoon, Venerable \textsanskrit{Mahākoṭṭhita} came out of retreat, went to Venerable \textsanskrit{Sāriputta}, and said: 

“Reverend\marginnote{1.3} \textsanskrit{Sāriputta}, what things should an ethical mendicant properly attend to?” 

“Reverend\marginnote{1.4} \textsanskrit{Koṭṭhita}, an ethical mendicant should properly attend to the five grasping aggregates as impermanent, as suffering, as diseased, as a boil, as a dart, as misery, as an affliction, as alien, as falling apart, as empty, as not-self. What five? That is, the grasping aggregates of form, feeling, perception, choices, and consciousness. An ethical mendicant should properly attend to these five grasping aggregates as impermanent, as suffering, as diseased, as a boil, as a dart, as misery, as an affliction, as alien, as falling apart, as empty, as not-self. It’s possible that an ethical mendicant who properly attends to the five grasping aggregates will realize the fruit of stream-entry.” 

“But\marginnote{2.1} Reverend \textsanskrit{Sāriputta}, what things should a mendicant stream-enterer properly attend to?” 

“A\marginnote{2.2} mendicant stream-enterer should also properly attend to these five grasping aggregates as impermanent … as not-self. It’s possible that a mendicant stream-enterer who properly attends to the five grasping aggregates will realize the fruit of once-return.” 

“But\marginnote{3.1} Reverend \textsanskrit{Sāriputta}, what things should a mendicant once-returner properly attend to?” 

“A\marginnote{3.2} mendicant once-returner should also properly attend to these five grasping aggregates as impermanent … as not-self. It’s possible that a mendicant once-returner who properly attends to the five grasping aggregates will realize the fruit of non-return.” 

“But\marginnote{4.1} Reverend \textsanskrit{Sāriputta}, what things should a mendicant non-returner properly attend to?” 

“A\marginnote{4.2} mendicant non-returner should also properly attend to these five grasping aggregates as impermanent … as not-self. It’s possible that a mendicant non-returner who properly attends to the five grasping aggregates will realize perfection.” 

“But\marginnote{5.1} Reverend \textsanskrit{Sāriputta}, what things should a perfected one properly attend to?” 

“Reverend\marginnote{5.2} \textsanskrit{Koṭṭhita}, a perfected one should also properly attend to the five grasping aggregates as impermanent, as suffering, as diseased, as a boil, as a dart, as misery, as an affliction, as alien, as falling apart, as empty, as not-self. A perfected one has nothing more to do, and nothing that needs improvement. Still, these things, when developed and cultivated, lead to blissful meditation in the present life, and also to mindfulness and situational awareness.” 

%
\section*{{\suttatitleacronym SN 22.123}{\suttatitletranslation Educated }{\suttatitleroot Sutavantasutta}}
\addcontentsline{toc}{section}{\tocacronym{SN 22.123} \toctranslation{Educated } \tocroot{Sutavantasutta}}
\markboth{Educated }{Sutavantasutta}
\extramarks{SN 22.123}{SN 22.123}

At\marginnote{1.1} one time Venerable \textsanskrit{Sāriputta} and Venerable \textsanskrit{Mahākoṭṭhita} were staying near Benares, in the deer park at Isipatana. Then in the late afternoon, Venerable \textsanskrit{Mahākoṭṭhita} came out of retreat, went to Venerable \textsanskrit{Sāriputta}, bowed, sat down to one side, and said: 

“Reverend\marginnote{2.1} \textsanskrit{Sāriputta}, what things should a learned mendicantt properly attend to?” 

“An\marginnote{2.2} educated mendicant should properly attend to these five grasping aggregates as impermanent … as not-self. What five? That is, the grasping aggregates of form, feeling, perception, choices, and consciousness. An educated mendicant should properly attend to these five grasping aggregates as impermanent … as not-self. It’s possible that a learned mendicantt who properly attends to the five grasping aggregates will realize the fruit of stream-entry.” 

“But\marginnote{3.1} Reverend \textsanskrit{Sāriputta}, what things should a mendicant stream-enterer properly attend to?” 

“A\marginnote{3.2} mendicant stream-enterer should also properly attend to these five grasping aggregates as impermanent … as not-self. It’s possible that a mendicant stream-enterer who properly attends to the five grasping aggregates will realize the fruit of once-return.” … “It’s possible that a mendicant once-returner who properly attends to the five grasping aggregates will realize the fruit of non-return.” … “It’s possible that a mendicant non-returner who regards the five grasping aggregates in this way will realize the fruit of perfection.” 

“But\marginnote{4.1} Reverend \textsanskrit{Sāriputta}, what things should a perfected one properly attend to?” 

“Reverend\marginnote{4.2} \textsanskrit{Koṭṭhita}, a perfected one should properly attend to the five grasping aggregates as impermanent, as suffering, as diseased, as a boil, as a dart, as misery, as an affliction, as alien, as falling apart, as empty, as not-self. A perfected one has nothing more to do, and nothing that needs improvement. Still, these things, when developed and cultivated, lead to blissful meditation in the present life, and also to mindfulness and situational awareness.” 

%
\section*{{\suttatitleacronym SN 22.124}{\suttatitletranslation With Kappa }{\suttatitleroot Kappasutta}}
\addcontentsline{toc}{section}{\tocacronym{SN 22.124} \toctranslation{With Kappa } \tocroot{Kappasutta}}
\markboth{With Kappa }{Kappasutta}
\extramarks{SN 22.124}{SN 22.124}

At\marginnote{1.1} \textsanskrit{Sāvatthī}. 

Then\marginnote{1.2} Venerable Kappa went up to the Buddha, bowed, sat down to one side, and said to him: 

“Sir,\marginnote{1.3} how does one know and see so that there’s no ego, possessiveness, or underlying tendency to conceit for this conscious body and all external stimuli?” 

“Kappa,\marginnote{2.1} one truly sees any kind of form at all—past, future, or present; internal or external; coarse or fine; inferior or superior; far or near: \emph{all} form—with right understanding: ‘This is not mine, I am not this, this is not my self.’ 

One\marginnote{2.2} truly sees any kind of feeling … perception … choices … consciousness at all—past, future, or present; internal or external; coarse or fine; inferior or superior; far or near: \emph{all} consciousness—with right understanding: ‘This is not mine, I am not this, this is not my self.’ 

That’s\marginnote{2.6} how to know and see so that there’s no ego, possessiveness, or underlying tendency to conceit for this conscious body and all external stimuli.” 

%
\section*{{\suttatitleacronym SN 22.125}{\suttatitletranslation With Kappa (2nd) }{\suttatitleroot Dutiyakappasutta}}
\addcontentsline{toc}{section}{\tocacronym{SN 22.125} \toctranslation{With Kappa (2nd) } \tocroot{Dutiyakappasutta}}
\markboth{With Kappa (2nd) }{Dutiyakappasutta}
\extramarks{SN 22.125}{SN 22.125}

At\marginnote{1.1} \textsanskrit{Sāvatthī}. 

Seated\marginnote{1.2} to one side, Venerable Kappa said to the Buddha: 

“Sir,\marginnote{1.3} how does one know and see so that the mind is rid of ego, possessiveness, and conceit for this conscious body and all external stimuli; and going beyond discrimination, it’s peaceful and well freed?” 

“Kappa,\marginnote{2.1} one is freed by not grasping having truly seen any kind of form at all—past, future, or present; internal or external; coarse or fine; inferior or superior; far or near: \emph{all} form—with right understanding: ‘This is not mine, I am not this, this is not my self.’ 

One\marginnote{2.2} is freed by not grasping having truly seen any kind of feeling … perception … choices … consciousness at all—past, future, or present; internal or external; coarse or fine; inferior or superior; far or near: \emph{all} consciousness—with right understanding: ‘This is not mine, I am not this, this is not my self.’ 

That’s\marginnote{2.6} how to know and see so that the mind is rid of ego, possessiveness, and conceit for this conscious body and all external stimuli; and going beyond discrimination, it’s peaceful and well freed.” 

%
\addtocontents{toc}{\let\protect\contentsline\protect\nopagecontentsline}
\chapter*{The Chapter on Ignorance }
\addcontentsline{toc}{chapter}{\tocchapterline{The Chapter on Ignorance }}
\addtocontents{toc}{\let\protect\contentsline\protect\oldcontentsline}

%
\section*{{\suttatitleacronym SN 22.126}{\suttatitletranslation Liable To Originate }{\suttatitleroot Samudayadhammasutta}}
\addcontentsline{toc}{section}{\tocacronym{SN 22.126} \toctranslation{Liable To Originate } \tocroot{Samudayadhammasutta}}
\markboth{Liable To Originate }{Samudayadhammasutta}
\extramarks{SN 22.126}{SN 22.126}

At\marginnote{1.1} \textsanskrit{Sāvatthī}. 

Then\marginnote{1.2} a mendicant went up to the Buddha, bowed, sat down to one side, and said to him: 

“Sir,\marginnote{1.3} they speak of this thing called ‘ignorance’. What is ignorance? And how is an ignorant person defined?” 

“Mendicant,\marginnote{2.1} it’s when an unlearned ordinary person doesn’t truly understand form, which is liable to originate, as form which is liable to originate. They don’t truly understand form, which is liable to vanish, as form which is liable to vanish. They don’t truly understand form, which is liable to originate and vanish, as form which is liable to originate and vanish. 

They\marginnote{2.4} don’t truly understand feeling … perception … choices … consciousness, which is liable to originate, as consciousness which is liable to originate. They don’t truly understand consciousness, which is liable to vanish, as consciousness which is liable to vanish. They don’t truly understand consciousness, which is liable to originate and vanish, as consciousness which is liable to originate and vanish. 

This\marginnote{2.14} is called ignorance. And this is how an ignorant person is defined.” 

When\marginnote{3.1} he said this, the mendicant said to the Buddha: 

“Sir,\marginnote{3.2} they speak of this thing called ‘knowledge’. What is knowledge? And how is a knowledgeable person defined?” 

“Mendicant,\marginnote{4.1} it’s when a learned noble disciple truly understands form, which is liable to originate, as form which is liable to originate. They truly understand form, which is liable to vanish, as form which is liable to vanish. They truly understand form, which is liable to originate and vanish, as form which is liable to originate and vanish. 

They\marginnote{4.4} truly understand feeling … perception … choices … consciousness, which is liable to originate, as consciousness which is liable to originate. They truly understand consciousness, which is liable to vanish, as consciousness which is liable to vanish. They truly understand consciousness, which is liable to originate and vanish, as consciousness which is liable to originate and vanish. 

This\marginnote{4.14} is called knowledge. And this is how a knowledgeable person is defined.” 

%
\section*{{\suttatitleacronym SN 22.127}{\suttatitletranslation Liable To Originate (2nd) }{\suttatitleroot Dutiyasamudayadhammasutta}}
\addcontentsline{toc}{section}{\tocacronym{SN 22.127} \toctranslation{Liable To Originate (2nd) } \tocroot{Dutiyasamudayadhammasutta}}
\markboth{Liable To Originate (2nd) }{Dutiyasamudayadhammasutta}
\extramarks{SN 22.127}{SN 22.127}

At\marginnote{1.1} one time Venerable \textsanskrit{Sāriputta} and Venerable \textsanskrit{Mahākoṭṭhita} were staying near Benares, in the deer park at Isipatana. … 

\textsanskrit{Mahākoṭṭhita}\marginnote{1.2} said to \textsanskrit{Sāriputta}: 

“Reverend\marginnote{1.3} \textsanskrit{Sāriputta}, they speak of this thing called ‘ignorance’. What is ignorance? And how is an ignorant person defined?” 

“Reverend,\marginnote{2.1} it’s when an unlearned ordinary person doesn’t truly understand form, which is liable to originate … liable to vanish … liable to originate and vanish, as form which is liable to originate and vanish. 

They\marginnote{2.2} don’t truly understand feeling … perception … choices … consciousness, which is liable to originate … liable to vanish … liable to originate and vanish, as consciousness which is liable to originate and vanish. 

This\marginnote{2.6} is called ignorance. And this is how an ignorant person is defined.” 

%
\section*{{\suttatitleacronym SN 22.128}{\suttatitletranslation Liable To Originate (3rd) }{\suttatitleroot Tatiyasamudayadhammasutta}}
\addcontentsline{toc}{section}{\tocacronym{SN 22.128} \toctranslation{Liable To Originate (3rd) } \tocroot{Tatiyasamudayadhammasutta}}
\markboth{Liable To Originate (3rd) }{Tatiyasamudayadhammasutta}
\extramarks{SN 22.128}{SN 22.128}

At\marginnote{1.1} one time Venerable \textsanskrit{Sāriputta} and Venerable \textsanskrit{Mahākoṭṭhita} were staying near Benares, in the deer park at Isipatana. … 

\textsanskrit{Mahākoṭṭhita}\marginnote{1.2} said to \textsanskrit{Sāriputta}: 

“Reverend\marginnote{1.3} \textsanskrit{Sāriputta}, they speak of this thing called ‘knowledge’. What is knowledge? And how is a knowledgeable person defined?” 

“Reverend,\marginnote{2.1} it’s when a learned noble disciple truly understands form, which is liable to originate … liable to vanish … liable to originate and vanish, as form which is liable to originate and vanish. 

They\marginnote{2.2} truly understand feeling … perception … choices … consciousness, which is liable to originate … liable to vanish … liable to originate and vanish, as consciousness which is liable to originate and vanish. 

This\marginnote{2.6} is called knowledge. And this is how a knowledgeable person is defined.” 

%
\section*{{\suttatitleacronym SN 22.129}{\suttatitletranslation Gratification }{\suttatitleroot Assādasutta}}
\addcontentsline{toc}{section}{\tocacronym{SN 22.129} \toctranslation{Gratification } \tocroot{Assādasutta}}
\markboth{Gratification }{Assādasutta}
\extramarks{SN 22.129}{SN 22.129}

At\marginnote{1.1} Benares. \textsanskrit{Mahākoṭṭhita} said to \textsanskrit{Sāriputta}: 

“Reverend\marginnote{1.3} \textsanskrit{Sāriputta}, they speak of this thing called ‘ignorance’. What is ignorance? And how is an ignorant person defined?” 

“Reverend,\marginnote{2.1} an unlearned ordinary person doesn’t truly understand the gratification, the drawback, and the escape when it comes to form, feeling, perception, choices, and consciousness. 

This\marginnote{2.6} is called ignorance. And this is how an ignorant person is defined.” 

%
\section*{{\suttatitleacronym SN 22.130}{\suttatitletranslation Gratification (2nd) }{\suttatitleroot Dutiyaassādasutta}}
\addcontentsline{toc}{section}{\tocacronym{SN 22.130} \toctranslation{Gratification (2nd) } \tocroot{Dutiyaassādasutta}}
\markboth{Gratification (2nd) }{Dutiyaassādasutta}
\extramarks{SN 22.130}{SN 22.130}

At\marginnote{1.1} Benares. 

“Reverend\marginnote{1.2} \textsanskrit{Sāriputta}, they speak of this thing called ‘knowledge’. What is knowledge? And how is a knowledgeable person defined?” 

“Reverend,\marginnote{2.1} a learned noble disciple truly understands the gratification, the drawback, and the escape when it comes to form, feeling, perception, choices, and consciousness. 

This\marginnote{2.6} is called knowledge. And this is how a knowledgeable person is defined.” 

%
\section*{{\suttatitleacronym SN 22.131}{\suttatitletranslation Origin }{\suttatitleroot Samudayasutta}}
\addcontentsline{toc}{section}{\tocacronym{SN 22.131} \toctranslation{Origin } \tocroot{Samudayasutta}}
\markboth{Origin }{Samudayasutta}
\extramarks{SN 22.131}{SN 22.131}

At\marginnote{1.1} Benares. 

“Reverend\marginnote{1.2} \textsanskrit{Sāriputta}, they speak of this thing called ‘ignorance’. What is ignorance? And how is an ignorant person defined?” 

“Reverend,\marginnote{2.1} an unlearned ordinary person doesn’t truly understand the origin, the ending, the gratification, the drawback, and the escape when it comes to form, feeling, perception, choices, and consciousness. 

This\marginnote{2.6} is called ignorance. And this is how an ignorant person is defined.” 

%
\section*{{\suttatitleacronym SN 22.132}{\suttatitletranslation Origin (2nd) }{\suttatitleroot Dutiyasamudayasutta}}
\addcontentsline{toc}{section}{\tocacronym{SN 22.132} \toctranslation{Origin (2nd) } \tocroot{Dutiyasamudayasutta}}
\markboth{Origin (2nd) }{Dutiyasamudayasutta}
\extramarks{SN 22.132}{SN 22.132}

At\marginnote{1.1} Benares. \textsanskrit{Mahākoṭṭhita} said to \textsanskrit{Sāriputta}: 

“Reverend\marginnote{1.3} \textsanskrit{Sāriputta}, they speak of this thing called ‘knowledge’. What is knowledge? And how is a knowledgeable person defined?” 

“Reverend,\marginnote{2.1} a learned noble disciple truly understands the origin, the ending, the gratification, the drawback, and the escape when it comes to form, feeling, perception, choices, and consciousness. 

This\marginnote{2.6} is called knowledge. And this is how a knowledgeable person is defined.” 

%
\section*{{\suttatitleacronym SN 22.133}{\suttatitletranslation With Koṭṭhita }{\suttatitleroot Koṭṭhikasutta}}
\addcontentsline{toc}{section}{\tocacronym{SN 22.133} \toctranslation{With Koṭṭhita } \tocroot{Koṭṭhikasutta}}
\markboth{With Koṭṭhita }{Koṭṭhikasutta}
\extramarks{SN 22.133}{SN 22.133}

At\marginnote{1.1} Benares. \textsanskrit{Sāriputta} said to \textsanskrit{Mahākoṭṭhita}: 

“Reverend\marginnote{1.4} \textsanskrit{Koṭṭhita}, they speak of this thing called ‘ignorance’. What is ignorance? And how is an ignorant person defined?” 

“Reverend,\marginnote{2.1} an unlearned ordinary person doesn’t truly understand the gratification, the drawback, and the escape when it comes to form, feeling, perception, choices, and consciousness. 

This\marginnote{2.6} is called ignorance. And this is how an ignorant person is defined.” 

When\marginnote{3.1} he said this, Venerable \textsanskrit{Sāriputta} said to him: 

“Reverend\marginnote{3.2} \textsanskrit{Koṭṭhita}, they speak of this thing called ‘knowledge’. What is knowledge? And how is a knowledgeable person defined?” 

“Reverend,\marginnote{4.1} a learned noble disciple truly understands the gratification, the drawback, and the escape when it comes to form, feeling, perception, choices, and consciousness. 

This\marginnote{4.6} is called knowledge. And this is how a knowledgeable person is defined.” 

%
\section*{{\suttatitleacronym SN 22.134}{\suttatitletranslation With Koṭṭhita (2nd) }{\suttatitleroot Dutiyakoṭṭhikasutta}}
\addcontentsline{toc}{section}{\tocacronym{SN 22.134} \toctranslation{With Koṭṭhita (2nd) } \tocroot{Dutiyakoṭṭhikasutta}}
\markboth{With Koṭṭhita (2nd) }{Dutiyakoṭṭhikasutta}
\extramarks{SN 22.134}{SN 22.134}

At\marginnote{1.1} Benares. 

“Reverend\marginnote{1.2} \textsanskrit{Koṭṭhita}, they speak of this thing called ‘ignorance’. What is ignorance? And how is an ignorant person defined?” 

“Reverend,\marginnote{2.1} an unlearned ordinary person doesn’t truly understand the origin, the ending, the gratification, the drawback, and the escape when it comes to form, feeling, perception, choices, and consciousness. 

This\marginnote{2.6} is called ignorance. And this is how an ignorant person is defined.” 

When\marginnote{3.1} he said this, Venerable \textsanskrit{Sāriputta} said to him: 

“Reverend\marginnote{3.2} \textsanskrit{Koṭṭhita}, they speak of this thing called ‘knowledge’. What is knowledge? And how is a knowledgeable person defined?” 

“Reverend,\marginnote{4.1} a learned noble disciple truly understands the origin, the ending, the gratification, the drawback, and the escape when it comes to form, feeling, perception, choices, and consciousness. 

This\marginnote{4.6} is called knowledge. And this is how a knowledgeable person is defined.” 

%
\section*{{\suttatitleacronym SN 22.135}{\suttatitletranslation With Koṭṭhita (3rd) }{\suttatitleroot Tatiyakoṭṭhikasutta}}
\addcontentsline{toc}{section}{\tocacronym{SN 22.135} \toctranslation{With Koṭṭhita (3rd) } \tocroot{Tatiyakoṭṭhikasutta}}
\markboth{With Koṭṭhita (3rd) }{Tatiyakoṭṭhikasutta}
\extramarks{SN 22.135}{SN 22.135}

The\marginnote{1.1} same setting. \textsanskrit{Sāriputta} said to \textsanskrit{Mahākoṭṭhita}: 

“Reverend\marginnote{1.3} \textsanskrit{Koṭṭhita}, they speak of this thing called ‘ignorance’. What is ignorance? And how is an ignorant person defined?” 

“Reverend,\marginnote{2.1} it’s when an unlearned ordinary person doesn’t understand form, its origin, its cessation, and the practice that leads to its cessation. They don’t understand feeling … perception … choices … consciousness, its origin, its cessation, and the practice that leads to its cessation. 

This\marginnote{2.6} is called ignorance. And this is how an ignorant person is defined.” 

When\marginnote{3.1} he said this, Venerable \textsanskrit{Sāriputta} said to him: 

“Reverend\marginnote{3.2} \textsanskrit{Koṭṭhita}, they speak of this thing called ‘knowledge’. What is knowledge? And how is a knowledgeable person defined?” 

“Reverend,\marginnote{4.1} it’s when a learned noble disciple understands form, its origin, its cessation, and the practice that leads to its cessation. They understand feeling … perception … choices … consciousness, its origin, its cessation, and the practice that leads to its cessation. 

This\marginnote{4.6} is called knowledge. And this is how a knowledgeable person is defined.” 

%
\addtocontents{toc}{\let\protect\contentsline\protect\nopagecontentsline}
\chapter*{The Chapter on Burning Chaff }
\addcontentsline{toc}{chapter}{\tocchapterline{The Chapter on Burning Chaff }}
\addtocontents{toc}{\let\protect\contentsline\protect\oldcontentsline}

%
\section*{{\suttatitleacronym SN 22.136}{\suttatitletranslation Burning Chaff }{\suttatitleroot Kukkuḷasutta}}
\addcontentsline{toc}{section}{\tocacronym{SN 22.136} \toctranslation{Burning Chaff } \tocroot{Kukkuḷasutta}}
\markboth{Burning Chaff }{Kukkuḷasutta}
\extramarks{SN 22.136}{SN 22.136}

At\marginnote{1.1} \textsanskrit{Sāvatthī}. 

“Mendicants,\marginnote{1.2} form, feeling, perception, choices, and consciousness are burning chaff.\footnote{Kukkula is a rare term, and there is little early evidence for its meaning. In Pali it appears here and in SN 10.5 where it is parallel to a hell. Elsewhere too it is the name of a hell, so perhaps we should simply use “hellish”. In Skt it has the sense of “chaff, burning chaff”, but this too seems thinly attested. Nevertheless, the sight of burning off chaff of rice and other crops was probably a common one, so I accept this. See https://static.squarespace.com/static/4f334481cb12c1acadc57623/5177e444e4b0244b5f673736/5177e4a0e4b0244b5f67cd57/1266225113983/1000w/4-1.jpg I can’t find any support for BB’s “hot embers”, though. } 

Seeing\marginnote{1.3} this, a learned noble disciple grows disillusioned with form, feeling, perception, choices, and consciousness. Being disillusioned, desire fades away. When desire fades away they’re freed. When they’re freed, they know they’re freed. 

They\marginnote{1.5} understand: ‘Rebirth is ended, the spiritual journey has been completed, what had to be done has been done, there is no return to any state of existence.’” 

%
\section*{{\suttatitleacronym SN 22.137}{\suttatitletranslation Impermanence }{\suttatitleroot Aniccasutta}}
\addcontentsline{toc}{section}{\tocacronym{SN 22.137} \toctranslation{Impermanence } \tocroot{Aniccasutta}}
\markboth{Impermanence }{Aniccasutta}
\extramarks{SN 22.137}{SN 22.137}

At\marginnote{1.1} \textsanskrit{Sāvatthī}. 

“Mendicants,\marginnote{1.2} you should give up desire for what is impermanent. 

And\marginnote{1.3} what is impermanent? Form is impermanent; you should give up desire for it. 

Feeling\marginnote{1.5} … 

Perception\marginnote{1.6} … 

Choices\marginnote{1.7} … 

Consciousness\marginnote{1.8} is impermanent; you should give up desire for it. 

You\marginnote{1.9} should give up desire for what is impermanent.” 

%
\section*{{\suttatitleacronym SN 22.138}{\suttatitletranslation Impermanence (2nd) }{\suttatitleroot Dutiyaaniccasutta}}
\addcontentsline{toc}{section}{\tocacronym{SN 22.138} \toctranslation{Impermanence (2nd) } \tocroot{Dutiyaaniccasutta}}
\markboth{Impermanence (2nd) }{Dutiyaaniccasutta}
\extramarks{SN 22.138}{SN 22.138}

At\marginnote{1.1} \textsanskrit{Sāvatthī}. 

“Mendicants,\marginnote{1.2} you should give up greed for what is impermanent. And what is impermanent? 

Form\marginnote{1.4} is impermanent; you should give up greed for it. 

Feeling\marginnote{1.5} … 

Perception\marginnote{1.6} … 

Choices\marginnote{1.7} … 

Consciousness\marginnote{1.8} is impermanent; you should give up greed for it. 

You\marginnote{1.9} should give up greed for what is impermanent.” 

%
\section*{{\suttatitleacronym SN 22.139}{\suttatitletranslation Impermanence (3rd) }{\suttatitleroot Tatiyaaniccasutta}}
\addcontentsline{toc}{section}{\tocacronym{SN 22.139} \toctranslation{Impermanence (3rd) } \tocroot{Tatiyaaniccasutta}}
\markboth{Impermanence (3rd) }{Tatiyaaniccasutta}
\extramarks{SN 22.139}{SN 22.139}

At\marginnote{1.1} \textsanskrit{Sāvatthī}. 

“Mendicants,\marginnote{1.2} you should give up desire and greed for what is impermanent. And what is impermanent? 

Form\marginnote{1.4} is impermanent; you should give up desire and greed for it. 

Feeling\marginnote{1.5} … 

Perception\marginnote{1.6} … 

Choices\marginnote{1.7} … 

Consciousness\marginnote{1.8} is impermanent; you should give up desire and greed for it. 

You\marginnote{1.9} should give up desire and greed for what is impermanent.” 

%
\section*{{\suttatitleacronym SN 22.140}{\suttatitletranslation Suffering }{\suttatitleroot Dukkhasutta}}
\addcontentsline{toc}{section}{\tocacronym{SN 22.140} \toctranslation{Suffering } \tocroot{Dukkhasutta}}
\markboth{Suffering }{Dukkhasutta}
\extramarks{SN 22.140}{SN 22.140}

At\marginnote{1.1} \textsanskrit{Sāvatthī}. 

“Mendicants,\marginnote{1.2} you should give up desire for what is suffering. …” 

%
\section*{{\suttatitleacronym SN 22.141}{\suttatitletranslation Suffering (2nd) }{\suttatitleroot Dutiyadukkhasutta}}
\addcontentsline{toc}{section}{\tocacronym{SN 22.141} \toctranslation{Suffering (2nd) } \tocroot{Dutiyadukkhasutta}}
\markboth{Suffering (2nd) }{Dutiyadukkhasutta}
\extramarks{SN 22.141}{SN 22.141}

At\marginnote{1.1} \textsanskrit{Sāvatthī}. 

“Mendicants,\marginnote{1.2} you should give up greed for what is suffering. …” 

%
\section*{{\suttatitleacronym SN 22.142}{\suttatitletranslation Suffering (3rd) }{\suttatitleroot Tatiyadukkhasutta}}
\addcontentsline{toc}{section}{\tocacronym{SN 22.142} \toctranslation{Suffering (3rd) } \tocroot{Tatiyadukkhasutta}}
\markboth{Suffering (3rd) }{Tatiyadukkhasutta}
\extramarks{SN 22.142}{SN 22.142}

At\marginnote{1.1} \textsanskrit{Sāvatthī}. 

“Mendicants,\marginnote{1.2} you should give up desire and greed for what is suffering. …” 

%
\section*{{\suttatitleacronym SN 22.143}{\suttatitletranslation Not-Self }{\suttatitleroot Anattasutta}}
\addcontentsline{toc}{section}{\tocacronym{SN 22.143} \toctranslation{Not-Self } \tocroot{Anattasutta}}
\markboth{Not-Self }{Anattasutta}
\extramarks{SN 22.143}{SN 22.143}

At\marginnote{1.1} \textsanskrit{Sāvatthī}. 

“Mendicants,\marginnote{1.2} you should give up desire for what is not-self. …” 

%
\section*{{\suttatitleacronym SN 22.144}{\suttatitletranslation Not-Self (2nd) }{\suttatitleroot Dutiyaanattasutta}}
\addcontentsline{toc}{section}{\tocacronym{SN 22.144} \toctranslation{Not-Self (2nd) } \tocroot{Dutiyaanattasutta}}
\markboth{Not-Self (2nd) }{Dutiyaanattasutta}
\extramarks{SN 22.144}{SN 22.144}

At\marginnote{1.1} \textsanskrit{Sāvatthī}. 

“Mendicants,\marginnote{1.2} you should give up greed for what is not-self. …” 

%
\section*{{\suttatitleacronym SN 22.145}{\suttatitletranslation Not-Self (3rd) }{\suttatitleroot Tatiyaanattasutta}}
\addcontentsline{toc}{section}{\tocacronym{SN 22.145} \toctranslation{Not-Self (3rd) } \tocroot{Tatiyaanattasutta}}
\markboth{Not-Self (3rd) }{Tatiyaanattasutta}
\extramarks{SN 22.145}{SN 22.145}

At\marginnote{1.1} \textsanskrit{Sāvatthī}. 

“Mendicants,\marginnote{1.2} you should give up desire and greed for what is not-self. …” 

%
\section*{{\suttatitleacronym SN 22.146}{\suttatitletranslation Full of Disillusionment }{\suttatitleroot Nibbidābahulasutta}}
\addcontentsline{toc}{section}{\tocacronym{SN 22.146} \toctranslation{Full of Disillusionment } \tocroot{Nibbidābahulasutta}}
\markboth{Full of Disillusionment }{Nibbidābahulasutta}
\extramarks{SN 22.146}{SN 22.146}

At\marginnote{1.1} \textsanskrit{Sāvatthī}. 

“Mendicants,\marginnote{1.2} when a gentleman has gone forth out of faith, this is what’s in line with the teachings. They should live full of disillusionment for form, feeling, perception, choices, and consciousness. Living in this way, they completely understand form, feeling, perception, choices, and consciousness. Completely understanding form, feeling, perception, choices, and consciousness, they’re freed from these things. They’re freed from rebirth, old age, and death, from sorrow, lamentation, pain, sadness, and distress. They’re freed from suffering, I say.” 

%
\section*{{\suttatitleacronym SN 22.147}{\suttatitletranslation Observing Impermanence }{\suttatitleroot Aniccānupassīsutta}}
\addcontentsline{toc}{section}{\tocacronym{SN 22.147} \toctranslation{Observing Impermanence } \tocroot{Aniccānupassīsutta}}
\markboth{Observing Impermanence }{Aniccānupassīsutta}
\extramarks{SN 22.147}{SN 22.147}

At\marginnote{1.1} \textsanskrit{Sāvatthī}. 

“Mendicants,\marginnote{1.2} when a gentleman has gone forth out of faith, this is what’s in line with the teachings. They should live observing impermanence in form, feeling, perception, choices, and consciousness. … They’re freed from suffering, I say.” 

%
\section*{{\suttatitleacronym SN 22.148}{\suttatitletranslation Observing Suffering }{\suttatitleroot Dukkhānupassīsutta}}
\addcontentsline{toc}{section}{\tocacronym{SN 22.148} \toctranslation{Observing Suffering } \tocroot{Dukkhānupassīsutta}}
\markboth{Observing Suffering }{Dukkhānupassīsutta}
\extramarks{SN 22.148}{SN 22.148}

At\marginnote{1.1} \textsanskrit{Sāvatthī}. 

“Mendicants,\marginnote{1.2} when a gentleman has gone forth out of faith, this is what’s in line with the teachings. They should live observing suffering in form, feeling, perception, choices, and consciousness. … They’re freed from suffering, I say.” 

%
\section*{{\suttatitleacronym SN 22.149}{\suttatitletranslation Observing Not-Self }{\suttatitleroot Anattānupassīsutta}}
\addcontentsline{toc}{section}{\tocacronym{SN 22.149} \toctranslation{Observing Not-Self } \tocroot{Anattānupassīsutta}}
\markboth{Observing Not-Self }{Anattānupassīsutta}
\extramarks{SN 22.149}{SN 22.149}

At\marginnote{1.1} \textsanskrit{Sāvatthī}. 

“Mendicants,\marginnote{1.2} when a gentleman has gone forth out of faith, this is what’s in line with the teachings. They should live observing not-self in form, feeling, perception, choices, and consciousness. … They’re freed from suffering, I say.” 

%
\addtocontents{toc}{\let\protect\contentsline\protect\nopagecontentsline}
\chapter*{The Chapter on Views }
\addcontentsline{toc}{chapter}{\tocchapterline{The Chapter on Views }}
\addtocontents{toc}{\let\protect\contentsline\protect\oldcontentsline}

%
\section*{{\suttatitleacronym SN 22.150}{\suttatitletranslation In Oneself }{\suttatitleroot Ajjhattasutta}}
\addcontentsline{toc}{section}{\tocacronym{SN 22.150} \toctranslation{In Oneself } \tocroot{Ajjhattasutta}}
\markboth{In Oneself }{Ajjhattasutta}
\extramarks{SN 22.150}{SN 22.150}

At\marginnote{1.1} \textsanskrit{Sāvatthī}. 

“Mendicants,\marginnote{1.2} when what exists, because of grasping what, do pleasure and pain arise in oneself?”\footnote{See BB’s notes here and at SN 22.83 on the dual meaning of upadaya. In addition to the reasons he gives, I think it’s useful to use “grasping” in these contexts, as it helps illuminate the sense of upadanakkhandha. } 

“Our\marginnote{1.3} teachings are rooted in the Buddha. …”\footnote{missing quotes in MS } 

“When\marginnote{1.4} form exists, because of grasping form, pleasure and pain arise in oneself. When feeling … perception … choices … consciousness exists, because of grasping consciousness, pleasure and pain arise in oneself. 

What\marginnote{1.9} do you think, mendicants? Is form permanent or impermanent?” 

“Impermanent,\marginnote{1.11} sir.” 

“But\marginnote{1.12} if it’s impermanent, is it suffering or happiness?” 

“Suffering,\marginnote{1.13} sir.” 

“But\marginnote{1.14} by not grasping what’s impermanent, suffering, and perishable, would pleasure and pain arise in oneself?” 

“No,\marginnote{1.15} sir.” 

“Is\marginnote{1.16} feeling … perception … choices … consciousness permanent or impermanent?” 

“Impermanent,\marginnote{1.20} sir.” 

“But\marginnote{1.21} if it’s impermanent, is it suffering or happiness?” 

“Suffering,\marginnote{1.22} sir.” 

“But\marginnote{1.23} by not grasping what’s impermanent, suffering, and perishable, would pleasure and pain arise in oneself?” 

“No,\marginnote{1.24} sir.” 

“Seeing\marginnote{1.25} this … They understand: ‘… there is no return to any state of existence.’” 

%
\section*{{\suttatitleacronym SN 22.151}{\suttatitletranslation This Is Mine }{\suttatitleroot Etaṁmamasutta}}
\addcontentsline{toc}{section}{\tocacronym{SN 22.151} \toctranslation{This Is Mine } \tocroot{Etaṁmamasutta}}
\markboth{This Is Mine }{Etaṁmamasutta}
\extramarks{SN 22.151}{SN 22.151}

At\marginnote{1.1} \textsanskrit{Sāvatthī}. 

“Mendicants,\marginnote{1.2} when what exists, because of grasping what and insisting on what, does someone regard things like this: ‘This is mine, I am this, this is my self’?” 

“Our\marginnote{1.4} teachings are rooted in the Buddha. …” 

“When\marginnote{1.5} form exists, because of grasping form and insisting on form … When consciousness exists, because of grasping consciousness and insisting on consciousness, someone regards it like this: ‘This is mine, I am this, this is my self.’ 

What\marginnote{1.8} do you think, mendicants? Is form permanent or impermanent?” 

“Impermanent,\marginnote{1.10} sir.” … 

“But\marginnote{1.11} by not grasping what’s impermanent, suffering, and perishable, would you regard it like this: ‘This is mine, I am this, this is my self’?” 

“No,\marginnote{1.12} sir.” 

“Is\marginnote{1.13} feeling … perception … choices … consciousness permanent or impermanent?” 

“Impermanent,\marginnote{1.17} sir.” … 

“But\marginnote{1.18} by not grasping what’s impermanent, suffering, and perishable, would you regard it like this: ‘This is mine, I am this, this is my self’?” 

“No,\marginnote{1.19} sir.” 

“Seeing\marginnote{1.20} this … They understand: ‘… there is no return to any state of existence.’” 

%
\section*{{\suttatitleacronym SN 22.152}{\suttatitletranslation This Is My Self }{\suttatitleroot Soattāsutta}}
\addcontentsline{toc}{section}{\tocacronym{SN 22.152} \toctranslation{This Is My Self } \tocroot{Soattāsutta}}
\markboth{This Is My Self }{Soattāsutta}
\extramarks{SN 22.152}{SN 22.152}

At\marginnote{1.1} \textsanskrit{Sāvatthī}. 

“Mendicants,\marginnote{1.2} when what exists, because of grasping what and insisting on what, does the view arise: ‘The self and the cosmos are one and the same. After passing away I will be permanent, everlasting, eternal, and imperishable’?” 

“Our\marginnote{1.4} teachings are rooted in the Buddha. …” 

“When\marginnote{1.5} form exists, because of grasping form and insisting on form, the view arises: ‘The self and the cosmos are one and the same. After passing away I will be permanent, everlasting, eternal, and imperishable.’ When feeling … perception … choices … consciousness exists, because of grasping consciousness and insisting on consciousness, the view arises: ‘The self and the cosmos are one and the same. After passing away I will be permanent, everlasting, eternal, and imperishable.’ 

What\marginnote{2.1} do you think, mendicants? Is form permanent or impermanent?” 

“Impermanent,\marginnote{2.3} sir.” 

“But\marginnote{2.4} if it’s impermanent, is it suffering or happiness?” 

“Suffering,\marginnote{2.5} sir.” 

“But\marginnote{2.6} by not grasping what’s impermanent, suffering, and perishable, would the view arise: ‘The self and the cosmos are one and the same. After passing away I will be permanent, everlasting, eternal, and imperishable’?” 

“No,\marginnote{2.8} sir.” 

“Is\marginnote{2.9} feeling … perception … choices … consciousness permanent or impermanent?” 

“Impermanent,\marginnote{2.13} sir.” 

“But\marginnote{2.14} if it’s impermanent, is it suffering or happiness?” 

“Suffering,\marginnote{2.15} sir.” 

“But\marginnote{2.16} by not grasping what’s impermanent, suffering, and perishable, would the view arise: ‘The self and the cosmos are one and the same. After passing away I will be permanent, everlasting, eternal, and imperishable’?” 

“No,\marginnote{2.18} sir.” 

“Seeing\marginnote{2.19} this … They understand: ‘… there is no return to any state of existence.’” 

%
\section*{{\suttatitleacronym SN 22.153}{\suttatitletranslation It Might Not Be Mine }{\suttatitleroot Nocamesiyāsutta}}
\addcontentsline{toc}{section}{\tocacronym{SN 22.153} \toctranslation{It Might Not Be Mine } \tocroot{Nocamesiyāsutta}}
\markboth{It Might Not Be Mine }{Nocamesiyāsutta}
\extramarks{SN 22.153}{SN 22.153}

At\marginnote{1.1} \textsanskrit{Sāvatthī}. 

“Mendicants,\marginnote{1.2} when what exists, because of grasping what and insisting on what, does the view arise: ‘I might not be, and it might not be mine. I will not be, and it will not be mine’?” 

“Our\marginnote{1.4} teachings are rooted in the Buddha. …” 

“When\marginnote{1.5} form exists, because of grasping form and insisting on form, the view arises: ‘I might not be, and it might not be mine. I will not be, and it will not be mine.’ When feeling … perception … choices … consciousness exists, because of grasping consciousness and insisting on consciousness, the view arises: ‘I might not be, and it might not be mine. I will not be, and it will not be mine.’ 

What\marginnote{1.12} do you think, mendicants? Is form permanent or impermanent?” 

“Impermanent,\marginnote{1.14} sir.” 

“But\marginnote{1.15} if it’s impermanent, is it suffering or happiness?” 

“Suffering,\marginnote{1.16} sir.” 

“But\marginnote{1.17} by not grasping what’s impermanent, suffering, and perishable, would the view arise: ‘I might not be, and it might not be mine. I will not be, and it will not be mine’?” 

“No,\marginnote{1.19} sir.” 

“Is\marginnote{1.20} feeling … perception … choices … consciousness permanent or impermanent?” 

“Impermanent,\marginnote{1.24} sir.” 

“But\marginnote{1.25} if it’s impermanent, is it suffering or happiness?” 

“Suffering,\marginnote{1.26} sir.” 

“But\marginnote{1.27} by not grasping what’s impermanent, suffering, and perishable, would the view arise: ‘I might not be, and it might not be mine. I will not be, and it will not be mine’?” 

“No,\marginnote{1.29} sir.” 

“Seeing\marginnote{1.30} this … They understand: ‘… there is no return to any state of existence.’” 

%
\section*{{\suttatitleacronym SN 22.154}{\suttatitletranslation Wrong View }{\suttatitleroot Micchādiṭṭhisutta}}
\addcontentsline{toc}{section}{\tocacronym{SN 22.154} \toctranslation{Wrong View } \tocroot{Micchādiṭṭhisutta}}
\markboth{Wrong View }{Micchādiṭṭhisutta}
\extramarks{SN 22.154}{SN 22.154}

At\marginnote{1.1} \textsanskrit{Sāvatthī}. 

“Mendicants,\marginnote{1.2} when what exists, because of grasping what and insisting on what, does wrong view arise?” 

“Our\marginnote{1.3} teachings are rooted in the Buddha. …” 

“When\marginnote{1.4} form exists, because of grasping form and insisting on form, wrong view arises. When feeling … perception … choices … consciousness exists, because of grasping consciousness and insisting on consciousness, wrong view arises. 

What\marginnote{1.9} do you think, mendicants? Is form permanent or impermanent?” 

“Impermanent,\marginnote{1.11} sir.” … 

“But\marginnote{1.12} by not grasping what’s impermanent, suffering, and perishable, would wrong view arise?” 

“No,\marginnote{1.14} sir.” 

“Is\marginnote{1.15} feeling … perception … choices … consciousness permanent or impermanent?” 

“Impermanent,\marginnote{1.19} sir.” 

“But\marginnote{1.20} if it’s impermanent, is it suffering or happiness?” 

“Suffering,\marginnote{1.21} sir.” 

“But\marginnote{1.22} by not grasping what’s impermanent, suffering, and perishable, would wrong view arise?” 

“No,\marginnote{1.23} sir.” 

“Seeing\marginnote{1.24} this … They understand: ‘… there is no return to any state of existence.’” 

%
\section*{{\suttatitleacronym SN 22.155}{\suttatitletranslation Identity View }{\suttatitleroot Sakkāyadiṭṭhisutta}}
\addcontentsline{toc}{section}{\tocacronym{SN 22.155} \toctranslation{Identity View } \tocroot{Sakkāyadiṭṭhisutta}}
\markboth{Identity View }{Sakkāyadiṭṭhisutta}
\extramarks{SN 22.155}{SN 22.155}

At\marginnote{1.1} \textsanskrit{Sāvatthī}. 

“Mendicants,\marginnote{1.2} when what exists, because of grasping what and insisting on what, does identity view arise?” 

“Our\marginnote{1.3} teachings are rooted in the Buddha. …” 

“When\marginnote{1.4} form exists, because of grasping form and insisting on form, identity view arises. When feeling … perception … choices … consciousness exists, because of grasping consciousness and insisting on consciousness, identity view arises. 

What\marginnote{1.9} do you think, mendicants? Is form permanent or impermanent?” 

“Impermanent,\marginnote{1.11} sir.” … 

“But\marginnote{1.12} by not grasping what’s impermanent, suffering, and perishable, would identity view arise?” 

“No,\marginnote{1.14} sir.” 

“Is\marginnote{1.15} feeling … perception … choices … consciousness permanent or impermanent?” 

“Impermanent,\marginnote{1.19} sir.” … 

“But\marginnote{1.20} by not grasping what’s impermanent, suffering, and perishable, would identity view arise?” 

“No,\marginnote{1.22} sir.” 

“Seeing\marginnote{1.23} this … They understand: ‘… there is no return to any state of existence.’” 

%
\section*{{\suttatitleacronym SN 22.156}{\suttatitletranslation View of Self }{\suttatitleroot Attānudiṭṭhisutta}}
\addcontentsline{toc}{section}{\tocacronym{SN 22.156} \toctranslation{View of Self } \tocroot{Attānudiṭṭhisutta}}
\markboth{View of Self }{Attānudiṭṭhisutta}
\extramarks{SN 22.156}{SN 22.156}

At\marginnote{1.1} \textsanskrit{Sāvatthī}. 

“Mendicants,\marginnote{1.2} when what exists, because of grasping what and insisting on what, does view of self arise?” 

“Our\marginnote{1.3} teachings are rooted in the Buddha. …” 

“When\marginnote{1.4} form exists, because of grasping form and insisting on form, view of self arises. When feeling … perception … choices … consciousness exists, because of grasping consciousness and insisting on consciousness, view of self arises. 

What\marginnote{1.9} do you think, mendicants? Is form permanent or impermanent?” 

“Impermanent,\marginnote{1.11} sir.” … 

“But\marginnote{1.12} by not grasping what’s impermanent, suffering, and perishable, would view of self arise?” 

“No,\marginnote{1.14} sir.” 

“Is\marginnote{1.15} feeling … perception … choices … consciousness permanent or impermanent?” 

“Impermanent,\marginnote{1.19} sir.” … 

“But\marginnote{1.20} by not grasping what’s impermanent, suffering, and perishable, would view of self arise?” 

“No,\marginnote{1.22} sir.” 

“Seeing\marginnote{1.23} this … They understand: ‘… there is no return to any state of existence.’” 

%
\section*{{\suttatitleacronym SN 22.157}{\suttatitletranslation Insistence }{\suttatitleroot Abhinivesasutta}}
\addcontentsline{toc}{section}{\tocacronym{SN 22.157} \toctranslation{Insistence } \tocroot{Abhinivesasutta}}
\markboth{Insistence }{Abhinivesasutta}
\extramarks{SN 22.157}{SN 22.157}

At\marginnote{1.1} \textsanskrit{Sāvatthī}. 

“Mendicants,\marginnote{1.2} when what exists, because of grasping what and insisting on what, do fetters, insistence, and shackles arise?” 

“Our\marginnote{1.3} teachings are rooted in the Buddha. …” 

“When\marginnote{1.4} form exists, because of grasping form and insisting on form, fetters, insistence, and shackles arise. When feeling … perception … choices … consciousness exists, because of grasping consciousness and insisting on consciousness, fetters, insistence, and shackles arise. 

What\marginnote{1.9} do you think, mendicants? Is form permanent or impermanent?” 

“Impermanent,\marginnote{1.11} sir.” … 

“But\marginnote{1.12} by not grasping what’s impermanent, suffering, and perishable, would fetters, insistence, and shackles arise?” 

“No,\marginnote{1.14} sir.” … 

“Seeing\marginnote{1.15} this … They understand: ‘… there is no return to any state of existence.’” 

%
\section*{{\suttatitleacronym SN 22.158}{\suttatitletranslation Insistence (2nd) }{\suttatitleroot Dutiyaabhinivesasutta}}
\addcontentsline{toc}{section}{\tocacronym{SN 22.158} \toctranslation{Insistence (2nd) } \tocroot{Dutiyaabhinivesasutta}}
\markboth{Insistence (2nd) }{Dutiyaabhinivesasutta}
\extramarks{SN 22.158}{SN 22.158}

At\marginnote{1.1} \textsanskrit{Sāvatthī}. 

“Mendicants,\marginnote{1.2} when what exists, because of grasping what and insisting on what, do fetters, insistence, shackles, and attachments arise?” 

“Our\marginnote{1.3} teachings are rooted in the Buddha. …” 

“When\marginnote{1.4} form exists, because of grasping form and insisting on form, fetters, insistence, shackles, and attachments arise. When feeling … perception … choices … consciousness exists, because of grasping consciousness and insisting on consciousness, fetters, insistence, shackles, and attachments arise. 

What\marginnote{1.9} do you think, mendicants? Is form permanent or impermanent?” 

“Impermanent,\marginnote{1.11} sir.” … 

“But\marginnote{1.12} by not grasping what’s impermanent, suffering, and perishable, would fetters, insistence, shackles, and attachments arise?” 

“No,\marginnote{1.14} sir.” 

“Seeing\marginnote{1.15} this … They understand: ‘… there is no return to any state of existence.’” 

%
\section*{{\suttatitleacronym SN 22.159}{\suttatitletranslation With Ānanda }{\suttatitleroot Ānandasutta}}
\addcontentsline{toc}{section}{\tocacronym{SN 22.159} \toctranslation{With Ānanda } \tocroot{Ānandasutta}}
\markboth{With Ānanda }{Ānandasutta}
\extramarks{SN 22.159}{SN 22.159}

At\marginnote{1.1} \textsanskrit{Sāvatthī}. 

Then\marginnote{1.2} Venerable Ānanda went up to the Buddha, bowed, sat down to one side, and said to him: 

“Sir,\marginnote{1.3} may the Buddha please teach me Dhamma in brief. When I’ve heard it, I’ll live alone, withdrawn, diligent, keen, and resolute.” 

“What\marginnote{2.1} do you think, Ānanda? Is form permanent or impermanent?” 

“Impermanent,\marginnote{2.3} sir.” 

“But\marginnote{2.4} if it’s impermanent, is it suffering or happiness?” 

“Suffering,\marginnote{2.5} sir.” 

“But\marginnote{2.6} if it’s impermanent, suffering, and liable to wear out, is it fit to be regarded thus: ‘This is mine, I am this, this is my self’?” 

“No,\marginnote{2.8} sir.” 

“Is\marginnote{2.9} feeling … perception … choices … consciousness permanent or impermanent?” 

“Impermanent,\marginnote{2.13} sir.” 

“But\marginnote{2.14} if it’s impermanent, is it suffering or happiness?” 

“Suffering,\marginnote{2.15} sir.” 

“But\marginnote{2.16} if it’s impermanent, suffering, and liable to wear out, is it fit to be regarded thus: ‘This is mine, I am this, this is my self’?” 

“No,\marginnote{2.18} sir.” … 

“Seeing\marginnote{2.19} this … They understand: ‘… there is no return to any state of existence.’” 

\scendsutta{The Linked Discourses on the aggregates are complete. }

%
\addtocontents{toc}{\let\protect\contentsline\protect\nopagecontentsline}
\part*{Linked Discourses with Rādha }
\addcontentsline{toc}{part}{Linked Discourses with Rādha }
\markboth{}{}
\addtocontents{toc}{\let\protect\contentsline\protect\oldcontentsline}

%
\addtocontents{toc}{\let\protect\contentsline\protect\nopagecontentsline}
\chapter*{First Chapter About Māra }
\addcontentsline{toc}{chapter}{\tocchapterline{First Chapter About Māra }}
\addtocontents{toc}{\let\protect\contentsline\protect\oldcontentsline}

%
\section*{{\suttatitleacronym SN 23.1}{\suttatitletranslation About Māra }{\suttatitleroot Mārasutta}}
\addcontentsline{toc}{section}{\tocacronym{SN 23.1} \toctranslation{About Māra } \tocroot{Mārasutta}}
\markboth{About Māra }{Mārasutta}
\extramarks{SN 23.1}{SN 23.1}

At\marginnote{1.1} \textsanskrit{Sāvatthī}. 

Then\marginnote{1.2} Venerable \textsanskrit{Rādha} went up to the Buddha, bowed, sat down to one side, and said to him: 

“Sir,\marginnote{2.1} they speak of this thing called ‘\textsanskrit{Māra}’. How is \textsanskrit{Māra} defined?” 

“When\marginnote{2.3} there is form, \textsanskrit{Rādha}, there may be \textsanskrit{Māra}, or the murderer, or the murdered. So you should see form as \textsanskrit{Māra}, the murderer, the murdered, the diseased, the boil, the dart, the misery, the miserable. Those who see it like this see rightly. When there is feeling … perception … choices … consciousness, there may be \textsanskrit{Māra}, or the murderer, or the murdered. So you should see consciousness as \textsanskrit{Māra}, the murderer, the murdered, the diseased, the boil, the dart, the misery, the miserable. Those who see it like this see rightly.” 

“But\marginnote{3.1} sir, what’s the purpose of seeing rightly?” 

“Disillusionment\marginnote{3.2} is the purpose of seeing rightly.” 

“But\marginnote{3.3} what’s the purpose of disillusionment?” 

“Dispassion\marginnote{3.4} is the purpose of disillusionment.” 

“But\marginnote{3.5} what’s the purpose of dispassion?” 

“Freedom\marginnote{3.6} is the purpose of dispassion.” 

“But\marginnote{3.7} what’s the purpose of freedom?” 

“Extinguishment\marginnote{3.8} is the purpose of freedom.” 

“But\marginnote{3.9} sir, what is the purpose of extinguishment?” 

“Your\marginnote{3.10} question goes too far, \textsanskrit{Rādha}. You couldn’t figure out the limit of questions. For extinguishment is the culmination, destination, and end of the spiritual life.” 

%
\section*{{\suttatitleacronym SN 23.2}{\suttatitletranslation Sentient Beings }{\suttatitleroot Sattasutta}}
\addcontentsline{toc}{section}{\tocacronym{SN 23.2} \toctranslation{Sentient Beings } \tocroot{Sattasutta}}
\markboth{Sentient Beings }{Sattasutta}
\extramarks{SN 23.2}{SN 23.2}

At\marginnote{1.1} \textsanskrit{Sāvatthī}. 

Seated\marginnote{1.2} to one side, Venerable \textsanskrit{Rādha} said to the Buddha: 

“Sir,\marginnote{1.3} they speak of this thing called a ‘sentient being’. How is a sentient being defined?” 

“\textsanskrit{Rādha},\marginnote{1.5} when you cling, strongly cling, to desire, greed, relishing, and craving for form, then a being is spoken of. When you cling, strongly cling, to desire, greed, relishing, and craving for feeling … perception … choices … consciousness, then a being is spoken of. 

Suppose\marginnote{2.1} some boys or girls were playing with sandcastles. As long as they’re not rid of greed, desire, fondness, thirst, passion, and craving for those sandcastles, they cherish them, fancy them, treasure them, and treat them as their own. But when they are rid of greed, desire, fondness, thirst, passion, and craving for those sandcastles, they scatter, destroy, and demolish them with their hands and feet, making them unplayable. 

In\marginnote{2.4} the same way, you should scatter, destroy, and demolish form, making it unplayable. And you should practice for the ending of craving. You should scatter, destroy, and demolish feeling … perception … choices … consciousness, making it unplayable. And you should practice for the ending of craving. For the ending of craving is extinguishment.” 

%
\section*{{\suttatitleacronym SN 23.3}{\suttatitletranslation The Conduit To Rebirth }{\suttatitleroot Bhavanettisutta}}
\addcontentsline{toc}{section}{\tocacronym{SN 23.3} \toctranslation{The Conduit To Rebirth } \tocroot{Bhavanettisutta}}
\markboth{The Conduit To Rebirth }{Bhavanettisutta}
\extramarks{SN 23.3}{SN 23.3}

At\marginnote{1.1} \textsanskrit{Sāvatthī}. 

Seated\marginnote{1.2} to one side, Venerable \textsanskrit{Rādha} said to the Buddha: 

“Sir,\marginnote{1.3} they speak of this thing called ‘the cessation of the conduit to rebirth’. What is the conduit to rebirth? And what is the cessation of the conduit to rebirth?” 

“\textsanskrit{Rādha},\marginnote{1.5} any desire, greed, relishing, and craving for form; and any attraction, grasping, mental fixation, insistence, and underlying tendencies—this is called the conduit to rebirth. Their cessation is the cessation of the conduit to rebirth. 

Any\marginnote{1.8} desire, greed, relishing, and craving for feeling … perception … choices … consciousness; and any attraction, grasping, mental fixation, insistence, and underlying tendencies—this is called the conduit to rebirth. Their cessation is the cessation of the conduit to rebirth.” 

%
\section*{{\suttatitleacronym SN 23.4}{\suttatitletranslation Should Be Completely Understood }{\suttatitleroot Pariññeyyasutta}}
\addcontentsline{toc}{section}{\tocacronym{SN 23.4} \toctranslation{Should Be Completely Understood } \tocroot{Pariññeyyasutta}}
\markboth{Should Be Completely Understood }{Pariññeyyasutta}
\extramarks{SN 23.4}{SN 23.4}

At\marginnote{1.1} \textsanskrit{Sāvatthī}. 

Then\marginnote{1.2} Venerable \textsanskrit{Rādha} went up to the Buddha, bowed, and sat down to one side. The Buddha said to him: 

“\textsanskrit{Rādha},\marginnote{2.1} I will teach you the things that should be completely understood, complete understanding, and the person who has completely understood. Listen and pay close attention, I will speak.” 

“Yes,\marginnote{2.3} sir,” \textsanskrit{Rādha} replied. The Buddha said this: 

“And\marginnote{2.5} what things should be completely understood? Form, feeling, perception, choices, and consciousness. These are called the things that should be completely understood. 

And\marginnote{2.8} what is complete understanding? The ending of greed, hate, and delusion. This is called complete understanding. 

And\marginnote{2.11} what is the person who has completely understood? It should be said: a perfected one, the venerable of such and such name and clan. This is called the person who has completely understood.” 

%
\section*{{\suttatitleacronym SN 23.5}{\suttatitletranslation Ascetics and Brahmins }{\suttatitleroot Samaṇasutta}}
\addcontentsline{toc}{section}{\tocacronym{SN 23.5} \toctranslation{Ascetics and Brahmins } \tocroot{Samaṇasutta}}
\markboth{Ascetics and Brahmins }{Samaṇasutta}
\extramarks{SN 23.5}{SN 23.5}

At\marginnote{1.1} \textsanskrit{Sāvatthī}. 

When\marginnote{1.2} Venerable \textsanskrit{Rādha} was seated to one side, the Buddha said to him: 

“\textsanskrit{Rādha},\marginnote{1.3} there are these five grasping aggregates. What five? The grasping aggregates of form, feeling, perception, choices, and consciousness. 

There\marginnote{1.6} are ascetics and brahmins who don’t truly understand these five grasping aggregates’ gratification, drawback, and escape. I don’t regard them as true ascetics and brahmins. Those venerables don’t realize the goal of life as an ascetic or brahmin, and don’t live having realized it with their own insight. 

There\marginnote{1.8} are ascetics and brahmins who do truly understand these five grasping aggregates’ gratification, drawback, and escape. I regard them as true ascetics and brahmins. Those venerables realize the goal of life as an ascetic or brahmin, and live having realized it with their own insight.” 

%
\section*{{\suttatitleacronym SN 23.6}{\suttatitletranslation Ascetics and Brahmins (2nd) }{\suttatitleroot Dutiyasamaṇasutta}}
\addcontentsline{toc}{section}{\tocacronym{SN 23.6} \toctranslation{Ascetics and Brahmins (2nd) } \tocroot{Dutiyasamaṇasutta}}
\markboth{Ascetics and Brahmins (2nd) }{Dutiyasamaṇasutta}
\extramarks{SN 23.6}{SN 23.6}

At\marginnote{1.1} \textsanskrit{Sāvatthī}. 

When\marginnote{1.2} Venerable \textsanskrit{Rādha} was seated to one side, the Buddha said to him: 

“\textsanskrit{Rādha},\marginnote{1.3} there are these five grasping aggregates. What five? The grasping aggregates of form, feeling, perception, choices, and consciousness. 

There\marginnote{1.6} are ascetics and brahmins who don’t truly understand these five grasping aggregates’ origin, ending, gratification, drawback, and escape … Those venerables don’t realize the goal of life as an ascetic or brahmin … 

There\marginnote{1.7} are ascetics and brahmins who do truly understand … Those venerables realize the goal of life as an ascetic or brahmin, and live having realized it with their own insight.” 

%
\section*{{\suttatitleacronym SN 23.7}{\suttatitletranslation A Stream-Enterer }{\suttatitleroot Sotāpannasutta}}
\addcontentsline{toc}{section}{\tocacronym{SN 23.7} \toctranslation{A Stream-Enterer } \tocroot{Sotāpannasutta}}
\markboth{A Stream-Enterer }{Sotāpannasutta}
\extramarks{SN 23.7}{SN 23.7}

At\marginnote{1.1} \textsanskrit{Sāvatthī}. 

When\marginnote{1.2} Venerable \textsanskrit{Rādha} was seated to one side, the Buddha said to him: 

“\textsanskrit{Rādha},\marginnote{1.3} there are these five grasping aggregates. What five? The grasping aggregates of form, feeling, perception, choices, and consciousness. When a noble disciple truly understands these five grasping aggregates’ origin, ending, gratification, drawback, and escape, they’re called a noble disciple who is a stream-enterer, not liable to be reborn in the underworld, bound for awakening.” 

%
\section*{{\suttatitleacronym SN 23.8}{\suttatitletranslation A Perfected One }{\suttatitleroot Arahantasutta}}
\addcontentsline{toc}{section}{\tocacronym{SN 23.8} \toctranslation{A Perfected One } \tocroot{Arahantasutta}}
\markboth{A Perfected One }{Arahantasutta}
\extramarks{SN 23.8}{SN 23.8}

At\marginnote{1.1} \textsanskrit{Sāvatthī}. 

When\marginnote{1.2} Venerable \textsanskrit{Rādha} was seated to one side, the Buddha said to him: 

“\textsanskrit{Rādha},\marginnote{1.3} there are these five grasping aggregates. What five? The grasping aggregates of form, feeling, perception, choices, and consciousness. A mendicant comes to be freed by not grasping after truly understanding these five grasping aggregates’ origin, ending, gratification, drawback, and escape. Such a mendicant is called a perfected one, with defilements ended, who has completed the spiritual journey, done what had to be done, laid down the burden, achieved their own true goal, utterly ended the fetters of rebirth, and is rightly freed through enlightenment.” 

%
\section*{{\suttatitleacronym SN 23.9}{\suttatitletranslation Desire and Greed }{\suttatitleroot Chandarāgasutta}}
\addcontentsline{toc}{section}{\tocacronym{SN 23.9} \toctranslation{Desire and Greed } \tocroot{Chandarāgasutta}}
\markboth{Desire and Greed }{Chandarāgasutta}
\extramarks{SN 23.9}{SN 23.9}

At\marginnote{1.1} \textsanskrit{Sāvatthī}. 

When\marginnote{1.2} Venerable \textsanskrit{Rādha} was seated to one side, the Buddha said to him: 

“\textsanskrit{Rādha},\marginnote{1.3} you should give up any desire, greed, relishing, and craving for form. Thus that form will be given up, cut off at the root, made like a palm stump, obliterated, and unable to arise in the future. 

You\marginnote{1.5} should give up any desire, greed, relishing, and craving for feeling … perception … choices … consciousness. Thus that consciousness will be given up, cut off at the root, made like a palm stump, obliterated, and unable to arise in the future.” 

%
\section*{{\suttatitleacronym SN 23.10}{\suttatitletranslation Desire and Greed (2nd) }{\suttatitleroot Dutiyachandarāgasutta}}
\addcontentsline{toc}{section}{\tocacronym{SN 23.10} \toctranslation{Desire and Greed (2nd) } \tocroot{Dutiyachandarāgasutta}}
\markboth{Desire and Greed (2nd) }{Dutiyachandarāgasutta}
\extramarks{SN 23.10}{SN 23.10}

At\marginnote{1.1} \textsanskrit{Sāvatthī}. 

When\marginnote{1.2} Venerable \textsanskrit{Rādha} was seated to one side, the Buddha said to him: 

“\textsanskrit{Rādha},\marginnote{1.3} you should give up any desire, greed, relishing, and craving for form; and any attraction, grasping, mental fixation, insistence, and underlying tendencies. Thus that form will be given up, cut off at the root, made like a palm stump, obliterated, and unable to arise in the future. 

You\marginnote{1.5} should give up any desire, greed, relishing, and craving for feeling … perception … choices … consciousness; and any attraction, grasping, mental fixation, insistence, and underlying tendencies. Thus that consciousness will be given up, cut off at the root, made like a palm stump, obliterated, and unable to arise in the future.” 

%
\addtocontents{toc}{\let\protect\contentsline\protect\nopagecontentsline}
\chapter*{Second Chapter About Māra }
\addcontentsline{toc}{chapter}{\tocchapterline{Second Chapter About Māra }}
\addtocontents{toc}{\let\protect\contentsline\protect\oldcontentsline}

%
\section*{{\suttatitleacronym SN 23.11}{\suttatitletranslation About Māra }{\suttatitleroot Mārasutta}}
\addcontentsline{toc}{section}{\tocacronym{SN 23.11} \toctranslation{About Māra } \tocroot{Mārasutta}}
\markboth{About Māra }{Mārasutta}
\extramarks{SN 23.11}{SN 23.11}

At\marginnote{1.1} \textsanskrit{Sāvatthī}. 

Seated\marginnote{1.2} to one side, Venerable \textsanskrit{Rādha} said to the Buddha: 

“Sir,\marginnote{1.3} they speak of this thing called ‘\textsanskrit{Māra}’. How is \textsanskrit{Māra} defined?” 

“\textsanskrit{Rādha},\marginnote{1.5} form is \textsanskrit{Māra}, feeling is \textsanskrit{Māra}, perception is \textsanskrit{Māra}, choices are \textsanskrit{Māra}, consciousness is \textsanskrit{Māra}. 

Seeing\marginnote{1.6} this, a learned noble disciple grows disillusioned with form, feeling, perception, choices, and consciousness. Being disillusioned, desire fades away. When desire fades away they’re freed. When they’re freed, they know they’re freed. 

They\marginnote{1.8} understand: ‘Rebirth is ended, the spiritual journey has been completed, what had to be done has been done, there is no return to any state of existence.’” 

%
\section*{{\suttatitleacronym SN 23.12}{\suttatitletranslation A Māra-like Nature }{\suttatitleroot Māradhammasutta}}
\addcontentsline{toc}{section}{\tocacronym{SN 23.12} \toctranslation{A Māra-like Nature } \tocroot{Māradhammasutta}}
\markboth{A Māra-like Nature }{Māradhammasutta}
\extramarks{SN 23.12}{SN 23.12}

At\marginnote{1.1} \textsanskrit{Sāvatthī}. 

Seated\marginnote{1.2} to one side, Venerable \textsanskrit{Rādha} said to the Buddha: 

“Sir,\marginnote{1.3} they speak of this thing called ‘\textsanskrit{Māra}-like nature’. What is a \textsanskrit{Māra}-like nature?” 

“\textsanskrit{Rādha},\marginnote{1.5} form has a \textsanskrit{Māra}-like nature. Feeling, perception, choices, and consciousness have a \textsanskrit{Māra}-like nature. 

Seeing\marginnote{1.6} this … They understand: ‘… there is no return to any state of existence.’” 

%
\section*{{\suttatitleacronym SN 23.13}{\suttatitletranslation Impermanence }{\suttatitleroot Aniccasutta}}
\addcontentsline{toc}{section}{\tocacronym{SN 23.13} \toctranslation{Impermanence } \tocroot{Aniccasutta}}
\markboth{Impermanence }{Aniccasutta}
\extramarks{SN 23.13}{SN 23.13}

At\marginnote{1.1} \textsanskrit{Sāvatthī}. 

Seated\marginnote{1.2} to one side, Venerable \textsanskrit{Rādha} said to the Buddha: 

“Sir,\marginnote{1.3} they speak of this thing called ‘impermanence’. What is impermanence?” 

“\textsanskrit{Rādha},\marginnote{1.5} form, feeling, perception, choices, and consciousness are impermanent. 

Seeing\marginnote{1.6} this … They understand: ‘… there is no return to any state of existence.’” 

%
\section*{{\suttatitleacronym SN 23.14}{\suttatitletranslation Naturally Impermanent }{\suttatitleroot Aniccadhammasutta}}
\addcontentsline{toc}{section}{\tocacronym{SN 23.14} \toctranslation{Naturally Impermanent } \tocroot{Aniccadhammasutta}}
\markboth{Naturally Impermanent }{Aniccadhammasutta}
\extramarks{SN 23.14}{SN 23.14}

At\marginnote{1.1} \textsanskrit{Sāvatthī}. 

Seated\marginnote{1.2} to one side, Venerable \textsanskrit{Rādha} said to the Buddha: 

“Sir,\marginnote{1.3} they speak of this thing called ‘naturally impermanent’. What is naturally impermanent? 

“\textsanskrit{Rādha},\marginnote{1.5} form, feeling, perception, choices, and consciousness are naturally impermanent. 

Seeing\marginnote{1.6} this … They understand: ‘… there is no return to any state of existence.’” 

%
\section*{{\suttatitleacronym SN 23.15}{\suttatitletranslation Suffering }{\suttatitleroot Dukkhasutta}}
\addcontentsline{toc}{section}{\tocacronym{SN 23.15} \toctranslation{Suffering } \tocroot{Dukkhasutta}}
\markboth{Suffering }{Dukkhasutta}
\extramarks{SN 23.15}{SN 23.15}

At\marginnote{1.1} \textsanskrit{Sāvatthī}. 

Seated\marginnote{1.2} to one side, Venerable \textsanskrit{Rādha} said to the Buddha: 

“Sir,\marginnote{1.3} they speak of this thing called ‘suffering’. What is suffering?” 

“\textsanskrit{Rādha},\marginnote{1.5} form, feeling, perception, choices, and consciousness are suffering. 

Seeing\marginnote{1.6} this … They understand: ‘… there is no return to any state of existence.’” 

%
\section*{{\suttatitleacronym SN 23.16}{\suttatitletranslation Entailing Suffering }{\suttatitleroot Dukkhadhammasutta}}
\addcontentsline{toc}{section}{\tocacronym{SN 23.16} \toctranslation{Entailing Suffering } \tocroot{Dukkhadhammasutta}}
\markboth{Entailing Suffering }{Dukkhadhammasutta}
\extramarks{SN 23.16}{SN 23.16}

At\marginnote{1.1} \textsanskrit{Sāvatthī}. 

Seated\marginnote{1.2} to one side, Venerable \textsanskrit{Rādha} said to the Buddha: 

“Sir,\marginnote{1.3} they speak of ‘things that entail suffering’. What are the things that entail suffering?” 

“\textsanskrit{Rādha},\marginnote{1.5} form, feeling, perception, choices, and consciousness are things that entail suffering. 

Seeing\marginnote{1.6} this … They understand: ‘… there is no return to any state of existence.’” 

%
\section*{{\suttatitleacronym SN 23.17}{\suttatitletranslation Not-Self }{\suttatitleroot Anattasutta}}
\addcontentsline{toc}{section}{\tocacronym{SN 23.17} \toctranslation{Not-Self } \tocroot{Anattasutta}}
\markboth{Not-Self }{Anattasutta}
\extramarks{SN 23.17}{SN 23.17}

At\marginnote{1.1} \textsanskrit{Sāvatthī}. 

Seated\marginnote{1.2} to one side, Venerable \textsanskrit{Rādha} said to the Buddha: 

“Sir,\marginnote{1.3} they speak of this thing called ‘not-self’. What is not-self?” 

“\textsanskrit{Rādha},\marginnote{1.5} form, feeling, perception, choices, and consciousness are not-self. 

Seeing\marginnote{1.6} this … They understand: ‘… there is no return to any state of existence.’” 

%
\section*{{\suttatitleacronym SN 23.18}{\suttatitletranslation Naturally Not-Self }{\suttatitleroot Anattadhammasutta}}
\addcontentsline{toc}{section}{\tocacronym{SN 23.18} \toctranslation{Naturally Not-Self } \tocroot{Anattadhammasutta}}
\markboth{Naturally Not-Self }{Anattadhammasutta}
\extramarks{SN 23.18}{SN 23.18}

At\marginnote{1.1} \textsanskrit{Sāvatthī}. 

Seated\marginnote{1.2} to one side, Venerable \textsanskrit{Rādha} said to the Buddha: 

“Sir,\marginnote{1.3} they speak of this thing called ‘naturally not-self’. What is naturally not-self?” 

“\textsanskrit{Rādha},\marginnote{1.5} form, feeling, perception, choices, and consciousness are naturally not-self. 

Seeing\marginnote{1.6} this … They understand: ‘… there is no return to any state of existence.’” 

%
\section*{{\suttatitleacronym SN 23.19}{\suttatitletranslation Liable To End }{\suttatitleroot Khayadhammasutta}}
\addcontentsline{toc}{section}{\tocacronym{SN 23.19} \toctranslation{Liable To End } \tocroot{Khayadhammasutta}}
\markboth{Liable To End }{Khayadhammasutta}
\extramarks{SN 23.19}{SN 23.19}

At\marginnote{1.1} \textsanskrit{Sāvatthī}. 

Seated\marginnote{1.2} to one side, Venerable \textsanskrit{Rādha} said to the Buddha: 

“Sir,\marginnote{1.3} they speak of things being ‘liable to end’. What is liable to end?” 

“\textsanskrit{Rādha},\marginnote{1.5} form, feeling, perception, choices, and consciousness are liable to end. 

Seeing\marginnote{1.6} this … They understand: ‘… there is no return to any state of existence.’” 

%
\section*{{\suttatitleacronym SN 23.20}{\suttatitletranslation Liable To Vanish }{\suttatitleroot Vayadhammasutta}}
\addcontentsline{toc}{section}{\tocacronym{SN 23.20} \toctranslation{Liable To Vanish } \tocroot{Vayadhammasutta}}
\markboth{Liable To Vanish }{Vayadhammasutta}
\extramarks{SN 23.20}{SN 23.20}

At\marginnote{1.1} \textsanskrit{Sāvatthī}. 

Seated\marginnote{1.2} to one side, Venerable \textsanskrit{Rādha} said to the Buddha: 

“Sir,\marginnote{1.3} they speak of things being ‘liable to vanish’. What is liable to vanish?” 

“\textsanskrit{Rādha},\marginnote{1.5} form, feeling, perception, choices, and consciousness are liable to vanish. 

Seeing\marginnote{1.6} this … They understand: ‘… there is no return to any state of existence.’” 

%
\section*{{\suttatitleacronym SN 23.21}{\suttatitletranslation Liable To Originate }{\suttatitleroot Samudayadhammasutta}}
\addcontentsline{toc}{section}{\tocacronym{SN 23.21} \toctranslation{Liable To Originate } \tocroot{Samudayadhammasutta}}
\markboth{Liable To Originate }{Samudayadhammasutta}
\extramarks{SN 23.21}{SN 23.21}

At\marginnote{1.1} \textsanskrit{Sāvatthī}. 

Seated\marginnote{1.2} to one side, Venerable \textsanskrit{Rādha} said to the Buddha: 

“Sir,\marginnote{1.3} they speak of things being ‘liable to originate’. What is liable to originate?” 

“\textsanskrit{Rādha},\marginnote{1.5} form, feeling, perception, choices, and consciousness are liable to originate. 

Seeing\marginnote{1.6} this … They understand: ‘… there is no return to any state of existence.’” 

%
\section*{{\suttatitleacronym SN 23.22}{\suttatitletranslation Liable To Cease }{\suttatitleroot Nirodhadhammasutta}}
\addcontentsline{toc}{section}{\tocacronym{SN 23.22} \toctranslation{Liable To Cease } \tocroot{Nirodhadhammasutta}}
\markboth{Liable To Cease }{Nirodhadhammasutta}
\extramarks{SN 23.22}{SN 23.22}

At\marginnote{1.1} \textsanskrit{Sāvatthī}. 

Seated\marginnote{1.2} to one side, Venerable \textsanskrit{Rādha} said to the Buddha: 

“Sir,\marginnote{1.3} they speak of things being ‘liable to cease’. What is liable to cease?” 

“\textsanskrit{Rādha},\marginnote{1.5} form, feeling, perception, choices, and consciousness are liable to cease. 

Seeing\marginnote{1.6} this … They understand: ‘… there is no return to any state of existence.’” 

%
\addtocontents{toc}{\let\protect\contentsline\protect\nopagecontentsline}
\chapter*{The Chapter on Aspiration }
\addcontentsline{toc}{chapter}{\tocchapterline{The Chapter on Aspiration }}
\addtocontents{toc}{\let\protect\contentsline\protect\oldcontentsline}

%
\section*{{\suttatitleacronym SN 23.23–33}{\suttatitletranslation Eleven Discourses on Māra, Etc. }{\suttatitleroot Mārādisuttaekādasaka}}
\addcontentsline{toc}{section}{\tocacronym{SN 23.23–33} \toctranslation{Eleven Discourses on Māra, Etc. } \tocroot{Mārādisuttaekādasaka}}
\markboth{Eleven Discourses on Māra, Etc. }{Mārādisuttaekādasaka}
\extramarks{SN 23.23–33}{SN 23.23–33}

At\marginnote{1.1} \textsanskrit{Sāvatthī}. 

Seated\marginnote{1.2} to one side, Venerable \textsanskrit{Rādha} said to the Buddha: 

“Sir,\marginnote{1.3} may the Buddha please teach me Dhamma in brief. When I’ve heard it, I’ll live alone, withdrawn, diligent, keen, and resolute.” 

“\textsanskrit{Rādha},\marginnote{2.1} you should give up any desire, any greed, any desire and greed for whatever is \textsanskrit{Māra}. And what is \textsanskrit{Māra}? Form is \textsanskrit{Māra}. You should give up any desire, any greed, any desire and greed for it. 

Feeling\marginnote{2.4} … 

Perception\marginnote{2.5} … 

Choices\marginnote{2.6} … 

Consciousness\marginnote{2.7} is \textsanskrit{Māra}. You should give up any desire, any greed, any desire and greed for it. You should give up any desire, any greed, any desire and greed for whatever is \textsanskrit{Māra}.” 

“You\marginnote{1.1} should give up any desire, any greed, any desire and greed for whatever is of \textsanskrit{Māra}-like nature …” 

“…\marginnote{1.1} impermanent …” 

“…\marginnote{1.1} naturally impermanent …” 

“…\marginnote{1.1} suffering …” 

“…\marginnote{1.1} things that entail suffering …” 

“…\marginnote{1.1} not-self …” 

“…\marginnote{1.1} naturally not-self …” 

“…\marginnote{1.1} liable to end …” 

“…\marginnote{1.1} liable to vanish …” 

“…\marginnote{1.1} liable to originate …” 

%
\section*{{\suttatitleacronym SN 23.34}{\suttatitletranslation Liable To Cease }{\suttatitleroot Nirodhadhammasutta}}
\addcontentsline{toc}{section}{\tocacronym{SN 23.34} \toctranslation{Liable To Cease } \tocroot{Nirodhadhammasutta}}
\markboth{Liable To Cease }{Nirodhadhammasutta}
\extramarks{SN 23.34}{SN 23.34}

At\marginnote{1.1} \textsanskrit{Sāvatthī}. 

Venerable\marginnote{1.2} \textsanskrit{Rādha} said to the Buddha: 

“Sir,\marginnote{1.3} may the Buddha please teach me Dhamma in brief. When I’ve heard it, I’ll live alone, withdrawn, diligent, keen, and resolute.” 

“\textsanskrit{Rādha},\marginnote{2.1} you should give up any desire, any greed, any desire and greed for whatever is liable to cease. And what is liable to cease? Form is liable to cease. You should give up any desire, any greed, any desire and greed for it. 

Feeling\marginnote{2.4} … 

Perception\marginnote{2.5} … 

Choices\marginnote{2.6} … 

Consciousness\marginnote{2.7} is liable to cease. You should give up any desire, any greed, any desire and greed for it. You should give up any desire, any greed, any desire and greed for whatever is liable to cease.” 

%
\addtocontents{toc}{\let\protect\contentsline\protect\nopagecontentsline}
\chapter*{The Chapter on Sitting Close }
\addcontentsline{toc}{chapter}{\tocchapterline{The Chapter on Sitting Close }}
\addtocontents{toc}{\let\protect\contentsline\protect\oldcontentsline}

%
\section*{{\suttatitleacronym SN 23.35–45}{\suttatitletranslation Eleven Discourses on Māra, Etc. }{\suttatitleroot Mārādisuttaekādasaka}}
\addcontentsline{toc}{section}{\tocacronym{SN 23.35–45} \toctranslation{Eleven Discourses on Māra, Etc. } \tocroot{Mārādisuttaekādasaka}}
\markboth{Eleven Discourses on Māra, Etc. }{Mārādisuttaekādasaka}
\extramarks{SN 23.35–45}{SN 23.35–45}

At\marginnote{1.1} \textsanskrit{Sāvatthī}. 

When\marginnote{1.2} Venerable \textsanskrit{Rādha} was seated to one side, the Buddha said to him: 

“\textsanskrit{Rādha},\marginnote{1.3} you should give up any desire, any greed, any desire and greed for whatever is \textsanskrit{Māra}. And what is \textsanskrit{Māra}? Form is \textsanskrit{Māra}. You should give up any desire, any greed, any desire and greed for it. … 

Consciousness\marginnote{1.6} is \textsanskrit{Māra}. You should give up any desire, any greed, any desire and greed for it. You should give up any desire, any greed, any desire and greed for whatever is \textsanskrit{Māra}.” 

“You\marginnote{1.1} should give up any desire, any greed, any desire and greed for whatever is of \textsanskrit{Māra}-like nature …” 

“…\marginnote{1.1} impermanent …” 

“…\marginnote{1.1} naturally impermanent …” 

“…\marginnote{1.1} suffering …” 

“…\marginnote{1.1} things that entail suffering …” 

“…\marginnote{1.1} not-self …” 

“…\marginnote{1.1} naturally not-self …” 

“…\marginnote{1.1} liable to end …” 

“…\marginnote{1.1} liable to vanish …” 

“…\marginnote{1.1} liable to originate …” 

%
\section*{{\suttatitleacronym SN 23.46}{\suttatitletranslation Liable To Cease }{\suttatitleroot Nirodhadhammasutta}}
\addcontentsline{toc}{section}{\tocacronym{SN 23.46} \toctranslation{Liable To Cease } \tocroot{Nirodhadhammasutta}}
\markboth{Liable To Cease }{Nirodhadhammasutta}
\extramarks{SN 23.46}{SN 23.46}

At\marginnote{1.1} \textsanskrit{Sāvatthī}. 

When\marginnote{1.2} Venerable \textsanskrit{Rādha} was seated to one side, the Buddha said to him: 

“\textsanskrit{Rādha},\marginnote{1.3} you should give up any desire, any greed, any desire and greed for whatever is liable to cease. And what is liable to cease? Form is liable to cease. You should give up any desire, any greed, any desire and greed for it. 

Feeling\marginnote{1.6} … 

Perception\marginnote{1.7} … 

Choices\marginnote{1.8} … 

Consciousness\marginnote{1.9} is liable to cease. You should give up any desire, any greed, any desire and greed for it. You should give up any desire, any greed, any desire and greed for whatever is liable to cease.” 

\scendsutta{The Linked Discourses with \textsanskrit{Rādha} are complete. }

%
\addtocontents{toc}{\let\protect\contentsline\protect\nopagecontentsline}
\part*{Linked Discourses on Views }
\addcontentsline{toc}{part}{Linked Discourses on Views }
\markboth{}{}
\addtocontents{toc}{\let\protect\contentsline\protect\oldcontentsline}

%
\addtocontents{toc}{\let\protect\contentsline\protect\nopagecontentsline}
\chapter*{The Chapter on Stream-Entry }
\addcontentsline{toc}{chapter}{\tocchapterline{The Chapter on Stream-Entry }}
\addtocontents{toc}{\let\protect\contentsline\protect\oldcontentsline}

%
\section*{{\suttatitleacronym SN 24.1}{\suttatitletranslation Winds }{\suttatitleroot Vātasutta}}
\addcontentsline{toc}{section}{\tocacronym{SN 24.1} \toctranslation{Winds } \tocroot{Vātasutta}}
\markboth{Winds }{Vātasutta}
\extramarks{SN 24.1}{SN 24.1}

At\marginnote{1.1} one time the Buddha was staying near \textsanskrit{Sāvatthī} in Jeta’s Grove. The Buddha said this: 

“Mendicants,\marginnote{1.3} when what exists, because of grasping what and insisting on what, does the view arise: ‘Winds don’t blow; rivers don’t flow; pregnant women don’t give birth; the moon and stars neither rise nor set, but stand firm like a pillar.’?” 

“Our\marginnote{2.1} teachings are rooted in the Buddha. He is our guide and our refuge. Sir, may the Buddha himself please clarify the meaning of this. The mendicants will listen and remember it.” 

“Well\marginnote{2.2} then, mendicants, listen and pay close attention, I will speak.” 

“Yes,\marginnote{2.3} sir,” they replied. The Buddha said this: 

“When\marginnote{3.1} form exists, because of grasping form and insisting on form, the view arises: ‘Winds don’t blow; rivers don’t flow; pregnant women don’t give birth; the moon and stars neither rise nor set, but stand firm like a pillar.’ When feeling … perception … choices … consciousness exists, because of grasping consciousness and insisting on consciousness, the view arises: ‘Winds don’t blow; rivers don’t flow; pregnant women don’t give birth; the moon and stars neither rise nor set, but stand firm like a pillar.’ 

What\marginnote{3.8} do you think, mendicants? Is form permanent or impermanent?” 

“Impermanent,\marginnote{3.10} sir.” 

“But\marginnote{3.11} if it’s impermanent, is it suffering or happiness?” 

“Suffering,\marginnote{3.12} sir.” 

“But\marginnote{3.13} by not grasping what’s impermanent, suffering, and perishable, would the view arise: ‘Winds don’t blow; rivers don’t flow; pregnant women don’t give birth; the moon and stars neither rise nor set, but stand firm like a pillar’?” 

“No,\marginnote{3.15} sir.” 

“Is\marginnote{4.1} feeling … perception … choices … consciousness permanent or impermanent?” 

“Impermanent,\marginnote{4.5} sir.” 

“But\marginnote{4.6} if it’s impermanent, is it suffering or happiness?” 

“Suffering,\marginnote{4.7} sir.” 

“But\marginnote{4.8} by not grasping what’s impermanent, suffering, and perishable, would the view arise: ‘Winds don’t blow; rivers don’t flow; pregnant women don’t give birth; the moon and stars neither rise nor set, but stand firm like a pillar’?” 

“No,\marginnote{4.10} sir.” 

“That\marginnote{4.11} which is seen, heard, thought, known, attained, sought, and explored by the mind: is that permanent or impermanent?” 

“Impermanent,\marginnote{4.12} sir.” 

“But\marginnote{4.13} if it’s impermanent, is it suffering or happiness?” 

“Suffering,\marginnote{4.14} sir.” 

“But\marginnote{4.15} by not grasping what’s impermanent, suffering, and perishable, would the view arise: ‘Winds don’t blow; rivers don’t flow; pregnant women don’t give birth; the moon and stars neither rise nor set, but stand firm like a pillar’?” 

“No,\marginnote{4.17} sir.” 

“When\marginnote{5.1} a noble disciple has given up doubt in these six cases, and has given up doubt in suffering, its origin, its cessation, and the practice that leads to its cessation, they’re called a noble disciple who is a stream-enterer, not liable to be reborn in the underworld, bound for awakening.” 

%
\section*{{\suttatitleacronym SN 24.2}{\suttatitletranslation This Is Mine }{\suttatitleroot Etaṁmamasutta}}
\addcontentsline{toc}{section}{\tocacronym{SN 24.2} \toctranslation{This Is Mine } \tocroot{Etaṁmamasutta}}
\markboth{This Is Mine }{Etaṁmamasutta}
\extramarks{SN 24.2}{SN 24.2}

At\marginnote{1.1} \textsanskrit{Sāvatthī}. 

“Mendicants,\marginnote{1.2} when what exists, because of grasping what and insisting on what, does the view arise: ‘This is mine, I am this, this is my self’?” 

“Our\marginnote{1.4} teachings are rooted in the Buddha. …” 

“When\marginnote{1.5} form exists, because of grasping form and insisting on form, the view arises: ‘This is mine, I am this, this is my self.’ When feeling … perception … choices … consciousness exists, because of grasping consciousness and insisting on consciousness, the view arises: ‘This is mine, I am this, this is my self.’ 

What\marginnote{2.1} do you think, mendicants? Is form permanent or impermanent?” 

“Impermanent,\marginnote{2.3} sir.” … 

“Is\marginnote{2.4} feeling … perception … choices … consciousness permanent or impermanent?” 

“Impermanent,\marginnote{2.8} sir.” … 

“That\marginnote{2.12} which is seen, heard, thought, known, attained, sought, and explored by the mind: is that permanent or impermanent?” 

“Impermanent,\marginnote{2.13} sir.” 

“But\marginnote{2.14} if it’s impermanent, is it suffering or happiness?” 

“Suffering,\marginnote{2.15} sir.” 

“But\marginnote{2.16} by not grasping what’s impermanent, suffering, and perishable, would the view arise: ‘This is mine, I am this, this is my self’?” 

“No,\marginnote{2.18} sir.” 

“When\marginnote{3.1} a noble disciple has given up doubt in these six cases, and has given up doubt in suffering, its origin, its cessation, and the practice that leads to its cessation, they’re called a noble disciple who is a stream-enterer, not liable to be reborn in the underworld, bound for awakening.” 

%
\section*{{\suttatitleacronym SN 24.3}{\suttatitletranslation This Is My Self }{\suttatitleroot Soattāsutta}}
\addcontentsline{toc}{section}{\tocacronym{SN 24.3} \toctranslation{This Is My Self } \tocroot{Soattāsutta}}
\markboth{This Is My Self }{Soattāsutta}
\extramarks{SN 24.3}{SN 24.3}

At\marginnote{1.1} \textsanskrit{Sāvatthī}. 

“Mendicants,\marginnote{1.2} when what exists, because of grasping what and insisting on what, does the view arise: ‘The self and the cosmos are one and the same. After passing away I will be permanent, everlasting, eternal, and imperishable’?” 

“Our\marginnote{1.4} teachings are rooted in the Buddha. …” 

“When\marginnote{2.1} form exists, because of grasping form and insisting on form, the view arises: ‘The self and the cosmos are one and the same. After passing away I will be permanent, everlasting, eternal, and imperishable.’ When feeling … perception … choices … consciousness exists, because of grasping consciousness and insisting on consciousness, the view arises: ‘The self and the cosmos are one and the same. After passing away I will be permanent, everlasting, eternal, and imperishable.’ 

What\marginnote{3.1} do you think, mendicants? Is form permanent or impermanent?” 

“Impermanent,\marginnote{3.3} sir.” … 

“Is\marginnote{3.8} feeling … perception … choices … consciousness permanent or impermanent?” 

“Impermanent,\marginnote{3.12} sir.” … 

“That\marginnote{3.17} which is seen, heard, thought, known, attained, sought, and explored by the mind: is that permanent or impermanent?” 

“Impermanent,\marginnote{3.18} sir.” … 

“But\marginnote{3.19} by not grasping what’s impermanent, suffering, and perishable, would such a view arise?” 

“No,\marginnote{3.21} sir.” 

“When\marginnote{4.1} a noble disciple has given up doubt in these six cases, and has given up doubt in suffering, its origin, its cessation, and the practice that leads to its cessation, they’re called a noble disciple who is a stream-enterer, not liable to be reborn in the underworld, bound for awakening.” 

%
\section*{{\suttatitleacronym SN 24.4}{\suttatitletranslation It Might Not Be Mine }{\suttatitleroot Nocamesiyāsutta}}
\addcontentsline{toc}{section}{\tocacronym{SN 24.4} \toctranslation{It Might Not Be Mine } \tocroot{Nocamesiyāsutta}}
\markboth{It Might Not Be Mine }{Nocamesiyāsutta}
\extramarks{SN 24.4}{SN 24.4}

At\marginnote{1.1} \textsanskrit{Sāvatthī}. 

“Mendicants,\marginnote{1.2} when what exists, because of grasping what and insisting on what, does the view arise: ‘I might not be, and it might not be mine. I will not be, and it will not be mine’?” 

“Our\marginnote{1.4} teachings are rooted in the Buddha. …” 

“When\marginnote{2.1} form exists, because of grasping form and insisting on form, the view arises: ‘I might not be, and it might not be mine. I will not be, and it will not be mine.’ When feeling … perception … choices … consciousness exists, because of grasping consciousness and insisting on consciousness, the view arises: ‘I might not be, and it might not be mine. I will not be, and it will not be mine.’ 

What\marginnote{3.1} do you think, mendicants? Is form permanent or impermanent?” 

“Impermanent,\marginnote{3.3} sir.” … 

“Is\marginnote{3.7} feeling … perception … choices … consciousness permanent or impermanent?” 

“Impermanent,\marginnote{3.11} sir.” … 

“That\marginnote{3.15} which is seen, heard, thought, known, attained, sought, and explored by the mind: is that permanent or impermanent?” 

“Impermanent,\marginnote{3.16} sir.” … 

“But\marginnote{3.17} by not grasping what’s impermanent, suffering, and perishable, would such a view arise?” 

“No,\marginnote{3.19} sir.” 

“When\marginnote{4.1} a noble disciple has given up doubt in these six cases, and has given up doubt in suffering, its origin, its cessation, and the practice that leads to its cessation, they’re called a noble disciple who is a stream-enterer, not liable to be reborn in the underworld, bound for awakening.” 

%
\section*{{\suttatitleacronym SN 24.5}{\suttatitletranslation There’s No Meaning in Giving }{\suttatitleroot Natthidinnasutta}}
\addcontentsline{toc}{section}{\tocacronym{SN 24.5} \toctranslation{There’s No Meaning in Giving } \tocroot{Natthidinnasutta}}
\markboth{There’s No Meaning in Giving }{Natthidinnasutta}
\extramarks{SN 24.5}{SN 24.5}

At\marginnote{1.1} \textsanskrit{Sāvatthī}. 

“Mendicants,\marginnote{1.2} when what exists, because of grasping what and insisting on what, does the view arise: ‘There’s no meaning in giving, sacrifice, or offerings. There’s no fruit or result of good and bad deeds. There’s no afterlife. There’s no such thing as mother and father, or beings that are reborn spontaneously. And there’s no ascetic or brahmin who is well attained and practiced, and who describes the afterlife after realizing it with their own insight. This person is made up of the four primary elements. When they die, the earth in their body merges and coalesces with the main mass of earth. The water in their body merges and coalesces with the main mass of water. The fire in their body merges and coalesces with the main mass of fire. The air in their body merges and coalesces with the main mass of air. The faculties are transferred to space. Four men with a bier carry away the corpse. Their footprints show the way to the cemetery. The bones become bleached. Offerings dedicated to the gods end in ashes. Giving is a doctrine of morons. When anyone affirms a positive teaching it’s just baseless, false nonsense. Both the foolish and the astute are annihilated and destroyed when their body breaks up, and don’t exist after death’?” 

“Our\marginnote{1.13} teachings are rooted in the Buddha. …” 

“When\marginnote{2.1} form exists, because of grasping form and insisting on form, the view arises: ‘There’s no meaning in giving, sacrifice, or offerings. … Both the foolish and the astute are annihilated and destroyed when their body breaks up, and don’t exist after death.’ When feeling … perception … choices … consciousness exists, because of grasping consciousness and insisting on consciousness, the view arises: ‘There’s no meaning in giving, sacrifice, or offerings. … Both the foolish and the astute are annihilated and destroyed when their body breaks up, and don’t exist after death.’ 

What\marginnote{3.1} do you think, mendicants? Is form permanent or impermanent?” 

“Impermanent,\marginnote{3.3} sir.” … 

“Is\marginnote{3.8} feeling … perception … choices … consciousness permanent or impermanent?” 

“Impermanent,\marginnote{3.12} sir.” … 

“That\marginnote{3.17} which is seen, heard, thought, known, attained, sought, and explored by the mind: is that permanent or impermanent?” 

“Impermanent,\marginnote{3.18} sir.” … 

“But\marginnote{3.19} by not grasping what’s impermanent, suffering, and perishable, would such a view arise?” 

“No,\marginnote{3.23} sir.” 

“When\marginnote{4.1} a noble disciple has given up doubt in these six cases, and has given up doubt in suffering, its origin, its cessation, and the practice that leads to its cessation, they’re called a noble disciple who is a stream-enterer, not liable to be reborn in the underworld, bound for awakening.” 

%
\section*{{\suttatitleacronym SN 24.6}{\suttatitletranslation Acting }{\suttatitleroot Karotosutta}}
\addcontentsline{toc}{section}{\tocacronym{SN 24.6} \toctranslation{Acting } \tocroot{Karotosutta}}
\markboth{Acting }{Karotosutta}
\extramarks{SN 24.6}{SN 24.6}

At\marginnote{1.1} \textsanskrit{Sāvatthī}. 

“Mendicants,\marginnote{1.2} when what exists, because of grasping what and insisting on what, does the view arise: ‘The one who acts does nothing wrong when they punish, mutilate, torture, aggrieve, oppress, intimidate, or when they encourage others to do the same. Nothing bad is done when they kill, steal, break into houses, plunder wealth, steal from isolated buildings, commit highway robbery, commit adultery, and lie. If you were to reduce all the living creatures of this earth to one heap and mass of flesh with a razor-edged chakram, no evil comes of that, and no outcome of evil. If you were to go along the south bank of the Ganges killing, mutilating, and torturing, and encouraging others to do the same, no evil comes of that, and no outcome of evil. If you were to go along the north bank of the Ganges giving and sacrificing and encouraging others to do the same, no merit comes of that, and no outcome of merit. In giving, self-control, restraint, and truthfulness there is no merit or outcome of merit’?” 

“Our\marginnote{1.8} teachings are rooted in the Buddha. …” 

“When\marginnote{2.1} form exists, because of grasping form and insisting on form, the view arises: ‘The one who acts does nothing wrong … there is no merit or outcome of merit.’ When feeling … perception … choices … consciousness exists, because of grasping consciousness and insisting on consciousness, the view arises: ‘The one who acts does nothing wrong … there is no merit or outcome of merit.’ 

What\marginnote{3.1} do you think, mendicants? Is form permanent or impermanent?” 

“Impermanent,\marginnote{3.3} sir.” … 

“Is\marginnote{3.8} feeling … perception … choices … consciousness permanent or impermanent?” 

“Impermanent,\marginnote{3.12} sir.” … 

“That\marginnote{3.17} which is seen, heard, thought, known, attained, sought, and explored by the mind: is that permanent or impermanent?” 

“Impermanent,\marginnote{3.18} sir.” … 

“But\marginnote{3.19} by not grasping what’s impermanent, suffering, and perishable, would such a view arise?” 

“No,\marginnote{3.22} sir.” 

“When\marginnote{4.1} a noble disciple has given up doubt in these six cases, and has given up doubt in suffering, its origin, its cessation, and the practice that leads to its cessation, they’re called a noble disciple who is a stream-enterer, not liable to be reborn in the underworld, bound for awakening.” 

%
\section*{{\suttatitleacronym SN 24.7}{\suttatitletranslation Cause }{\suttatitleroot Hetusutta}}
\addcontentsline{toc}{section}{\tocacronym{SN 24.7} \toctranslation{Cause } \tocroot{Hetusutta}}
\markboth{Cause }{Hetusutta}
\extramarks{SN 24.7}{SN 24.7}

At\marginnote{1.1} \textsanskrit{Sāvatthī}. 

“Mendicants,\marginnote{1.2} when what exists, because of grasping what and insisting on what, does the view arise: ‘There is no cause or reason for the corruption of sentient beings. Sentient beings are corrupted without cause or reason. There’s no cause or reason for the purification of sentient beings. Sentient beings are purified without cause or reason. There is no power, no energy, no human strength or vigor. All sentient beings, all living creatures, all beings, all souls lack control, power, and energy. Molded by destiny, circumstance, and nature, they experience pleasure and pain in the six classes of rebirth’?” 

“Our\marginnote{1.9} teachings are rooted in the Buddha. …” 

“When\marginnote{2.1} form exists, because of grasping form and insisting on form, the view arises: ‘There is no cause or reason … they experience pleasure and pain in the six classes of rebirth.’ When feeling … perception … choices … consciousness exists, because of grasping consciousness and insisting on consciousness, the view arises: ‘There is no cause or reason … they experience pleasure and pain in the six classes of rebirth.’ 

What\marginnote{3.1} do you think, mendicants? Is form permanent or impermanent?” 

“Impermanent,\marginnote{3.3} sir.” … 

“Is\marginnote{3.8} feeling … perception … choices … consciousness permanent or impermanent?” 

“Impermanent,\marginnote{3.12} sir.” … 

“That\marginnote{3.17} which is seen, heard, thought, known, attained, sought, and explored by the mind: is that permanent or impermanent?” 

“Impermanent,\marginnote{3.18} sir.” … 

“But\marginnote{3.19} by not grasping what’s impermanent, suffering, and perishable, would such a view arise?” 

“No,\marginnote{3.22} sir.” 

“When\marginnote{4.1} a noble disciple has given up doubt in these six cases, and has given up doubt in suffering, its origin, its cessation, and the practice that leads to its cessation, they’re called a noble disciple who is a stream-enterer, not liable to be reborn in the underworld, bound for awakening.” 

%
\section*{{\suttatitleacronym SN 24.8}{\suttatitletranslation The Extensive View }{\suttatitleroot Mahādiṭṭhisutta}}
\addcontentsline{toc}{section}{\tocacronym{SN 24.8} \toctranslation{The Extensive View } \tocroot{Mahādiṭṭhisutta}}
\markboth{The Extensive View }{Mahādiṭṭhisutta}
\extramarks{SN 24.8}{SN 24.8}

At\marginnote{1.1} \textsanskrit{Sāvatthī}. 

“Mendicants,\marginnote{1.2} when what exists, because of grasping what and insisting on what, does the view arise: ‘There are these seven substances that are not made, not derived, not created, without a creator, barren, steady as a mountain peak, standing firm like a pillar. They don’t move or deteriorate or obstruct each other. They’re unable to cause pleasure, pain, or neutral feeling to each other. What seven? The substances of earth, water, fire, air; pleasure, pain, and the soul is the seventh. These seven substances are not made, not derived, not created, without a creator, barren, steady as a mountain peak, standing firm like a pillar. They don’t move or deteriorate or obstruct each other. They’re unable to cause pleasure, pain, or neutral feeling to each other. If you chop off someone’s head with a sharp sword, you don’t take anyone’s life. The sword simply passes through the gap between the seven substances. There are 1.4 million main wombs, and 6,000, and 600. There are 500 deeds, and five, and three. There are deeds and half-deeds. There are 62 paths, 62 sub-eons, six classes of rebirth, and eight stages in a person’s life. There are 4,900 \textsanskrit{Ājīvaka} ascetics, 4,900 wanderers, and 4,900 naked ascetics. There are 2,000 faculties, 3,000 hells, and 36 realms of dust. There are seven percipient embryos, seven non-percipient embryos, and seven embryos without attachments. There are seven gods, seven humans, and seven goblins. There are seven lakes, seven winds, seven cliffs, and 700 cliffs. There are seven dreams and 700 dreams. There are 8.4 million great eons through which the foolish and the astute transmigrate before making an end of suffering. And here there is no such thing as this: “By this precept or observance or mortification or spiritual life I shall force unripened deeds to bear their fruit, or eliminate old deeds by experiencing their results little by little”—for that cannot be. Pleasure and pain are allotted. Transmigration lasts only for a limited period, so there’s no increase or decrease, no getting better or worse. It’s like how, when you toss a ball of string, it rolls away unraveling. In the same way, after transmigrating the foolish and the astute will make an end of suffering’?” 

“Our\marginnote{1.16} teachings are rooted in the Buddha. …” 

“When\marginnote{2.1} form exists, because of grasping form and insisting on form, the view arises: ‘There are these seven substances that are not made … the foolish and the astute will make an end of suffering.’ When feeling … perception … choices … consciousness exists, because of grasping consciousness and insisting on consciousness, the view arises: ‘There are these seven substances that are not made … the foolish and the astute will make an end of suffering.’ 

What\marginnote{3.1} do you think, mendicants? Is form permanent or impermanent?” 

“Impermanent,\marginnote{3.3} sir.” … 

“That\marginnote{3.8} which is seen, heard, thought, known, attained, sought, and explored by the mind: is that permanent or impermanent?” 

“Impermanent,\marginnote{3.9} sir.” … 

“But\marginnote{3.10} by not grasping what’s impermanent, suffering, and perishable, would such a view arise?” 

“No,\marginnote{3.13} sir.” 

“When\marginnote{4.1} a noble disciple has given up doubt in these six cases, and has given up doubt in suffering, its origin, its cessation, and the practice that leads to its cessation, they’re called a noble disciple who is a stream-enterer, not liable to be reborn in the underworld, bound for awakening.” 

%
\section*{{\suttatitleacronym SN 24.9}{\suttatitletranslation The Cosmos is Eternal }{\suttatitleroot Sassatadiṭṭhisutta}}
\addcontentsline{toc}{section}{\tocacronym{SN 24.9} \toctranslation{The Cosmos is Eternal } \tocroot{Sassatadiṭṭhisutta}}
\markboth{The Cosmos is Eternal }{Sassatadiṭṭhisutta}
\extramarks{SN 24.9}{SN 24.9}

At\marginnote{1.1} \textsanskrit{Sāvatthī}. 

“Mendicants,\marginnote{1.2} when what exists, because of grasping what and insisting on what, does the view arise: ‘The cosmos is eternal’?” 

“Our\marginnote{1.4} teachings are rooted in the Buddha. …” 

“When\marginnote{2.1} form exists, because of grasping form and insisting on form, the view arises: ‘The cosmos is eternal.’ When feeling … perception … choices … consciousness exists, because of grasping consciousness and insisting on consciousness, the view arises: ‘The cosmos is eternal.’ 

What\marginnote{3.1} do you think, mendicants? Is form permanent or impermanent?” 

“Impermanent,\marginnote{3.3} sir.” … 

“But\marginnote{3.19} by not grasping what’s impermanent, suffering, and perishable, would such a view arise?” 

“No,\marginnote{3.21} sir.” 

“When\marginnote{4.1} a noble disciple has given up doubt in these six cases, and has given up doubt in suffering, its origin, its cessation, and the practice that leads to its cessation, they’re called a noble disciple who is a stream-enterer, not liable to be reborn in the underworld, bound for awakening.” 

%
\section*{{\suttatitleacronym SN 24.10}{\suttatitletranslation The Cosmos Is Not Eternal }{\suttatitleroot Asassatadiṭṭhisutta}}
\addcontentsline{toc}{section}{\tocacronym{SN 24.10} \toctranslation{The Cosmos Is Not Eternal } \tocroot{Asassatadiṭṭhisutta}}
\markboth{The Cosmos Is Not Eternal }{Asassatadiṭṭhisutta}
\extramarks{SN 24.10}{SN 24.10}

At\marginnote{1.1} \textsanskrit{Sāvatthī}. 

“Mendicants,\marginnote{1.2} when what exists, because of grasping what and insisting on what, does the view arise: ‘The cosmos is not eternal’?” 

“Our\marginnote{1.4} teachings are rooted in the Buddha. …” 

“When\marginnote{2.1} form exists …” … 

“But\marginnote{2.9} by not grasping what’s impermanent, suffering, and perishable, would such a view arise?” 

“No,\marginnote{2.11} sir.” 

“When\marginnote{3.1} a noble disciple has given up doubt in these six cases, and has given up doubt in suffering, its origin, its cessation, and the practice that leads to its cessation, they’re called a noble disciple who is a stream-enterer, not liable to be reborn in the underworld, bound for awakening.” 

%
\section*{{\suttatitleacronym SN 24.11}{\suttatitletranslation The World Is Finite }{\suttatitleroot Antavāsutta}}
\addcontentsline{toc}{section}{\tocacronym{SN 24.11} \toctranslation{The World Is Finite } \tocroot{Antavāsutta}}
\markboth{The World Is Finite }{Antavāsutta}
\extramarks{SN 24.11}{SN 24.11}

At\marginnote{1.1} \textsanskrit{Sāvatthī}. 

“Mendicants,\marginnote{1.2} when what exists, because of grasping what and insisting on what, does the view arise: ‘The world is finite’?” … 

%
\section*{{\suttatitleacronym SN 24.12}{\suttatitletranslation The World Is Infinite }{\suttatitleroot Anantavāsutta}}
\addcontentsline{toc}{section}{\tocacronym{SN 24.12} \toctranslation{The World Is Infinite } \tocroot{Anantavāsutta}}
\markboth{The World Is Infinite }{Anantavāsutta}
\extramarks{SN 24.12}{SN 24.12}

At\marginnote{1.1} \textsanskrit{Sāvatthī}. 

“Mendicants,\marginnote{1.2} when what exists, because of grasping what and insisting on what, does the view arise: ‘The world is infinite’?” … 

%
\section*{{\suttatitleacronym SN 24.13}{\suttatitletranslation The Soul and the Body Are Identical }{\suttatitleroot Taṁjīvaṁtaṁsarīraṁsutta}}
\addcontentsline{toc}{section}{\tocacronym{SN 24.13} \toctranslation{The Soul and the Body Are Identical } \tocroot{Taṁjīvaṁtaṁsarīraṁsutta}}
\markboth{The Soul and the Body Are Identical }{Taṁjīvaṁtaṁsarīraṁsutta}
\extramarks{SN 24.13}{SN 24.13}

At\marginnote{1.1} \textsanskrit{Sāvatthī}. 

“Mendicants,\marginnote{1.2} when what exists, because of grasping what and insisting on what, does the view arise: ‘The soul and the body are identical’?” … 

%
\section*{{\suttatitleacronym SN 24.14}{\suttatitletranslation The Soul and the Body Are Different Things }{\suttatitleroot Aññaṁjīvaṁaññaṁsarīraṁsutta}}
\addcontentsline{toc}{section}{\tocacronym{SN 24.14} \toctranslation{The Soul and the Body Are Different Things } \tocroot{Aññaṁjīvaṁaññaṁsarīraṁsutta}}
\markboth{The Soul and the Body Are Different Things }{Aññaṁjīvaṁaññaṁsarīraṁsutta}
\extramarks{SN 24.14}{SN 24.14}

At\marginnote{1.1} \textsanskrit{Sāvatthī}. 

“Mendicants,\marginnote{1.2} when what exists, because of grasping what and insisting on what, does the view arise: ‘The soul and the body are different things’?” … 

%
\section*{{\suttatitleacronym SN 24.15}{\suttatitletranslation A Realized One Exists }{\suttatitleroot Hotitathāgatosutta}}
\addcontentsline{toc}{section}{\tocacronym{SN 24.15} \toctranslation{A Realized One Exists } \tocroot{Hotitathāgatosutta}}
\markboth{A Realized One Exists }{Hotitathāgatosutta}
\extramarks{SN 24.15}{SN 24.15}

At\marginnote{1.1} \textsanskrit{Sāvatthī}. 

“Mendicants,\marginnote{1.2} when what exists, because of grasping what and insisting on what, does the view arise: ‘A Realized One exists after death’?” … 

%
\section*{{\suttatitleacronym SN 24.16}{\suttatitletranslation A Realized One Doesn’t Exist }{\suttatitleroot Nahotitathāgatosutta}}
\addcontentsline{toc}{section}{\tocacronym{SN 24.16} \toctranslation{A Realized One Doesn’t Exist } \tocroot{Nahotitathāgatosutta}}
\markboth{A Realized One Doesn’t Exist }{Nahotitathāgatosutta}
\extramarks{SN 24.16}{SN 24.16}

At\marginnote{1.1} \textsanskrit{Sāvatthī}. 

“Mendicants,\marginnote{1.2} when what exists, because of grasping what and insisting on what, does the view arise: ‘A Realized One doesn’t exist after death’?” … 

%
\section*{{\suttatitleacronym SN 24.17}{\suttatitletranslation A Realized One Both Exists and Doesn’t Exist }{\suttatitleroot Hoticanacahotitathāgatosutta}}
\addcontentsline{toc}{section}{\tocacronym{SN 24.17} \toctranslation{A Realized One Both Exists and Doesn’t Exist } \tocroot{Hoticanacahotitathāgatosutta}}
\markboth{A Realized One Both Exists and Doesn’t Exist }{Hoticanacahotitathāgatosutta}
\extramarks{SN 24.17}{SN 24.17}

At\marginnote{1.1} \textsanskrit{Sāvatthī}. 

“Mendicants,\marginnote{1.2} when what exists, because of grasping what and insisting on what, does the view arise: ‘A Realized One both exists and doesn’t exist after death’?” … 

%
\section*{{\suttatitleacronym SN 24.18}{\suttatitletranslation A Realized One Neither Exists Nor Doesn’t Exist }{\suttatitleroot Nevahotinanahotitathāgatosutta}}
\addcontentsline{toc}{section}{\tocacronym{SN 24.18} \toctranslation{A Realized One Neither Exists Nor Doesn’t Exist } \tocroot{Nevahotinanahotitathāgatosutta}}
\markboth{A Realized One Neither Exists Nor Doesn’t Exist }{Nevahotinanahotitathāgatosutta}
\extramarks{SN 24.18}{SN 24.18}

At\marginnote{1.1} \textsanskrit{Sāvatthī}. 

“Mendicants,\marginnote{1.2} when what exists, because of grasping what and insisting on what, does the view arise: ‘A Realized One neither exists nor doesn’t exist after death’?” 

“Our\marginnote{1.4} teachings are rooted in the Buddha. …” 

“When\marginnote{2.1} form exists, because of grasping form and insisting on form, the view arises: ‘A Realized One neither exists nor doesn’t exist after death’ … 

What\marginnote{3.1} do you think, mendicants? Is form permanent or impermanent?” 

“Impermanent,\marginnote{3.3} sir.” … “But by not grasping what’s impermanent, suffering, and perishable, would the view arise: ‘A Realized One neither exists nor doesn’t exist after death’?” 

“No,\marginnote{3.6} sir.” 

“That\marginnote{3.7} which is seen, heard, thought, known, attained, sought, and explored by the mind: is that permanent or impermanent?” 

“Impermanent,\marginnote{3.8} sir.” 

“But\marginnote{3.9} if it’s impermanent, is it suffering or happiness?” 

“Suffering,\marginnote{3.10} sir.” 

“But\marginnote{3.11} by not grasping what’s impermanent, suffering, and perishable, would the view arise: ‘A Realized One neither exists nor doesn’t exist after death’?” 

“No,\marginnote{3.13} sir.” 

“When\marginnote{4.1} a noble disciple has given up doubt in these six cases, and has given up doubt in suffering, its origin, its cessation, and the practice that leads to its cessation, they’re called a noble disciple who is a stream-enterer, not liable to be reborn in the underworld, bound for awakening.” 

%
\addtocontents{toc}{\let\protect\contentsline\protect\nopagecontentsline}
\chapter*{The Chapter on the Second Round }
\addcontentsline{toc}{chapter}{\tocchapterline{The Chapter on the Second Round }}
\addtocontents{toc}{\let\protect\contentsline\protect\oldcontentsline}

%
\section*{{\suttatitleacronym SN 24.19}{\suttatitletranslation Winds }{\suttatitleroot Vātasutta}}
\addcontentsline{toc}{section}{\tocacronym{SN 24.19} \toctranslation{Winds } \tocroot{Vātasutta}}
\markboth{Winds }{Vātasutta}
\extramarks{SN 24.19}{SN 24.19}

At\marginnote{1.1} \textsanskrit{Sāvatthī}. 

“Mendicants,\marginnote{1.2} when what exists, because of grasping what and insisting on what, does the view arise: ‘Winds don’t blow; rivers don’t flow; pregnant women don’t give birth; the moon and stars neither rise nor set, but stand firm like a pillar’?” 

“Our\marginnote{1.4} teachings are rooted in the Buddha. …” 

“When\marginnote{2.1} form exists, because of grasping form and insisting on form, the view arises: ‘Winds don’t blow; rivers don’t flow; pregnant women don’t give birth; the moon and stars neither rise nor set, but stand firm like a pillar.’ When feeling … perception … choices … consciousness exists, because of grasping consciousness and insisting on consciousness, the view arises: ‘Winds don’t blow; rivers don’t flow; pregnant women don’t give birth; the moon and stars neither rise nor set, but stand firm like a pillar.’ 

What\marginnote{3.1} do you think, mendicants? Is form permanent or impermanent?” 

“Impermanent,\marginnote{3.3} sir.” … 

“But\marginnote{3.4} by not grasping what’s impermanent, suffering, and perishable, would the view arise: ‘Winds don’t blow; rivers don’t flow; pregnant women don’t give birth; the moon and stars neither rise nor set, but stand firm like a pillar’?” 

“No,\marginnote{3.6} sir.” 

“And\marginnote{3.7} so, when suffering exists, because of grasping suffering and insisting on suffering, the view arises: ‘Winds don’t blow; rivers don’t flow; pregnant women don’t give birth; the moon and stars neither rise nor set, but stand firm like a pillar.’ Is feeling … perception … choices … consciousness permanent or impermanent?” 

“Impermanent,\marginnote{3.13} sir.” … 

“But\marginnote{3.14} by not grasping what’s impermanent, suffering, and perishable, would such a view arise?” 

“No,\marginnote{3.16} sir.” 

“And\marginnote{3.17} so, when suffering exists, because of grasping suffering and insisting on suffering, the view arises: ‘Winds don’t blow; rivers don’t flow; pregnant women don’t give birth; the moon and stars neither rise nor set, but stand firm like a pillar.’” 

%
\section*{{\suttatitleacronym SN 24.20–35}{\suttatitletranslation This Is Mine, Etc. }{\suttatitleroot Etaṁmamādisutta}}
\addcontentsline{toc}{section}{\tocacronym{SN 24.20–35} \toctranslation{This Is Mine, Etc. } \tocroot{Etaṁmamādisutta}}
\markboth{This Is Mine, Etc. }{Etaṁmamādisutta}
\extramarks{SN 24.20–35}{SN 24.20–35}

(These\marginnote{1.1} should be expanded in the same way as discourses 2 through 17 of the previous chapter.) 

%
\section*{{\suttatitleacronym SN 24.36}{\suttatitletranslation Neither Exists Nor Doesn’t Exist }{\suttatitleroot Nevahotinanahotisutta}}
\addcontentsline{toc}{section}{\tocacronym{SN 24.36} \toctranslation{Neither Exists Nor Doesn’t Exist } \tocroot{Nevahotinanahotisutta}}
\markboth{Neither Exists Nor Doesn’t Exist }{Nevahotinanahotisutta}
\extramarks{SN 24.36}{SN 24.36}

At\marginnote{1.1} \textsanskrit{Sāvatthī}. 

“Mendicants,\marginnote{1.2} when what exists, because of grasping what and insisting on what, does the view arise: ‘A Realized One neither exists nor doesn’t exist after death’?” 

“Our\marginnote{1.4} teachings are rooted in the Buddha. …” 

“When\marginnote{2.1} form exists, because of grasping form and insisting on form, the view arises: ‘A Realized One neither exists nor doesn’t exist after death.’ When feeling … perception … choices … consciousness exists, because of grasping consciousness and insisting on consciousness, the view arises: ‘a Realized One neither exists nor doesn’t exist after death.’ 

What\marginnote{3.1} do you think, mendicants? Is form permanent or impermanent?” 

“Impermanent,\marginnote{3.3} sir.” … 

“And\marginnote{3.7} so, when suffering exists, because of grasping suffering and insisting on suffering, the view arises: ‘A Realized One neither exists nor doesn’t exist after death.’” … 

“Is\marginnote{3.9} feeling … perception … choices … consciousness permanent or impermanent?” 

“Impermanent,\marginnote{3.13} sir.” … 

“But\marginnote{3.14} by not grasping what’s impermanent, suffering, and perishable, would such a view arise?” 

“No,\marginnote{3.16} sir.” 

“And\marginnote{3.17} so, when suffering exists, because of grasping suffering and insisting on suffering, the view arises: ‘A Realized One neither exists nor doesn’t exist after death.’” 

%
\section*{{\suttatitleacronym SN 24.37}{\suttatitletranslation The Self Has Form }{\suttatitleroot Rūpīattāsutta}}
\addcontentsline{toc}{section}{\tocacronym{SN 24.37} \toctranslation{The Self Has Form } \tocroot{Rūpīattāsutta}}
\markboth{The Self Has Form }{Rūpīattāsutta}
\extramarks{SN 24.37}{SN 24.37}

At\marginnote{1.1} \textsanskrit{Sāvatthī}. 

“Mendicants,\marginnote{1.2} when what exists, because of grasping what and insisting on what, does the view arise: ‘The self has form and is well after death’?” … 

%
\section*{{\suttatitleacronym SN 24.38}{\suttatitletranslation The Self Is Formless }{\suttatitleroot Arūpīattāsutta}}
\addcontentsline{toc}{section}{\tocacronym{SN 24.38} \toctranslation{The Self Is Formless } \tocroot{Arūpīattāsutta}}
\markboth{The Self Is Formless }{Arūpīattāsutta}
\extramarks{SN 24.38}{SN 24.38}

At\marginnote{1.1} \textsanskrit{Sāvatthī}. 

“Mendicants,\marginnote{1.2} when what exists, because of grasping what and insisting on what, does the view arise: ‘The self is formless and is well after death’?” … 

%
\section*{{\suttatitleacronym SN 24.39}{\suttatitletranslation The Self Has Form and Is Formless }{\suttatitleroot Rūpīcaarūpīcaattāsutta}}
\addcontentsline{toc}{section}{\tocacronym{SN 24.39} \toctranslation{The Self Has Form and Is Formless } \tocroot{Rūpīcaarūpīcaattāsutta}}
\markboth{The Self Has Form and Is Formless }{Rūpīcaarūpīcaattāsutta}
\extramarks{SN 24.39}{SN 24.39}

At\marginnote{1.1} \textsanskrit{Sāvatthī}. 

“‘The\marginnote{1.2} self has form and is formless, and is well after death’?” … 

%
\section*{{\suttatitleacronym SN 24.40}{\suttatitletranslation The Self Neither Has Form Nor Is Formless }{\suttatitleroot Nevarūpīnārūpīattāsutta}}
\addcontentsline{toc}{section}{\tocacronym{SN 24.40} \toctranslation{The Self Neither Has Form Nor Is Formless } \tocroot{Nevarūpīnārūpīattāsutta}}
\markboth{The Self Neither Has Form Nor Is Formless }{Nevarūpīnārūpīattāsutta}
\extramarks{SN 24.40}{SN 24.40}

“‘The\marginnote{1.1} self neither has form nor is formless, and is well after death’?” … 

%
\section*{{\suttatitleacronym SN 24.41}{\suttatitletranslation The Self Is Perfectly Happy }{\suttatitleroot Ekantasukhīsutta}}
\addcontentsline{toc}{section}{\tocacronym{SN 24.41} \toctranslation{The Self Is Perfectly Happy } \tocroot{Ekantasukhīsutta}}
\markboth{The Self Is Perfectly Happy }{Ekantasukhīsutta}
\extramarks{SN 24.41}{SN 24.41}

“‘The\marginnote{1.1} self is perfectly happy, and is well after death’?” … 

%
\section*{{\suttatitleacronym SN 24.42}{\suttatitletranslation Exclusively Suffering }{\suttatitleroot Ekantadukkhīsutta}}
\addcontentsline{toc}{section}{\tocacronym{SN 24.42} \toctranslation{Exclusively Suffering } \tocroot{Ekantadukkhīsutta}}
\markboth{Exclusively Suffering }{Ekantadukkhīsutta}
\extramarks{SN 24.42}{SN 24.42}

“‘The\marginnote{1.1} self is exclusively suffering, and is well after death’?” … 

%
\section*{{\suttatitleacronym SN 24.43}{\suttatitletranslation The Self Is Happy and Suffering }{\suttatitleroot Sukhadukkhīsutta}}
\addcontentsline{toc}{section}{\tocacronym{SN 24.43} \toctranslation{The Self Is Happy and Suffering } \tocroot{Sukhadukkhīsutta}}
\markboth{The Self Is Happy and Suffering }{Sukhadukkhīsutta}
\extramarks{SN 24.43}{SN 24.43}

“‘The\marginnote{1.1} self is happy and suffering, and is well after death’?” … 

%
\section*{{\suttatitleacronym SN 24.44}{\suttatitletranslation The Self Is Neither Happy Nor Suffering }{\suttatitleroot Adukkhamasukhīsutta}}
\addcontentsline{toc}{section}{\tocacronym{SN 24.44} \toctranslation{The Self Is Neither Happy Nor Suffering } \tocroot{Adukkhamasukhīsutta}}
\markboth{The Self Is Neither Happy Nor Suffering }{Adukkhamasukhīsutta}
\extramarks{SN 24.44}{SN 24.44}

“‘The\marginnote{1.1} self is neither happy nor suffering, and is well after death’?” … 

%
\addtocontents{toc}{\let\protect\contentsline\protect\nopagecontentsline}
\chapter*{The Chapter on the Third Round }
\addcontentsline{toc}{chapter}{\tocchapterline{The Chapter on the Third Round }}
\addtocontents{toc}{\let\protect\contentsline\protect\oldcontentsline}

%
\section*{{\suttatitleacronym SN 24.45}{\suttatitletranslation Winds }{\suttatitleroot Navātasutta}}
\addcontentsline{toc}{section}{\tocacronym{SN 24.45} \toctranslation{Winds } \tocroot{Navātasutta}}
\markboth{Winds }{Navātasutta}
\extramarks{SN 24.45}{SN 24.45}

At\marginnote{1.1} \textsanskrit{Sāvatthī}. 

“Mendicants,\marginnote{1.2} when what exists, because of grasping what and insisting on what, does the view arise: ‘Winds don’t blow; rivers don’t flow; pregnant women don’t give birth; the moon and stars neither rise nor set, but stand firm like a pillar’?” 

“Our\marginnote{1.4} teachings are rooted in the Buddha. …” 

“When\marginnote{2.1} form exists, because of grasping form and insisting on form, the view arises: ‘Winds don’t blow; rivers don’t flow; pregnant women don’t give birth; the moon and stars neither rise nor set, but stand firm like a pillar.’ When feeling … perception … choices … consciousness exists, because of grasping consciousness and insisting on consciousness, the view arises: ‘Winds don’t blow; rivers don’t flow; pregnant women don’t give birth; the moon and stars neither rise nor set, but stand firm like a pillar.’ 

What\marginnote{3.1} do you think, mendicants? Is form permanent or impermanent?” 

“Impermanent,\marginnote{3.3} sir.” … 

“But\marginnote{3.4} by not grasping what’s impermanent, suffering, and perishable, would the view arise: ‘Winds don’t blow; rivers don’t flow; pregnant women don’t give birth; the moon and stars neither rise nor set, but stand firm like a pillar’?” 

“No,\marginnote{3.6} sir.” 

“And\marginnote{3.7} so, what’s impermanent is suffering. When this exists, grasping at this, the view arises: ‘Winds don’t blow; rivers don’t flow; pregnant women don’t give birth; the moon and stars neither rise nor set, but stand firm like a pillar.’ Is feeling … perception … choices … consciousness permanent or impermanent?” 

“Impermanent,\marginnote{3.14} sir.” … 

“And\marginnote{3.18} so, what’s impermanent is suffering. When this exists, grasping at this, the view arises: ‘Winds don’t blow; rivers don’t flow; pregnant women don’t give birth; the moon and stars neither rise nor set, but stand firm like a pillar.’” 

%
\section*{{\suttatitleacronym SN 24.46–69}{\suttatitletranslation This Is Mine, etc. }{\suttatitleroot Etaṁmamādisutta}}
\addcontentsline{toc}{section}{\tocacronym{SN 24.46–69} \toctranslation{This Is Mine, etc. } \tocroot{Etaṁmamādisutta}}
\markboth{This Is Mine, etc. }{Etaṁmamādisutta}
\extramarks{SN 24.46–69}{SN 24.46–69}

(To\marginnote{1.1} be completed in the same way as discourses 20 through 43 of the second chapter.) 

%
\section*{{\suttatitleacronym SN 24.70}{\suttatitletranslation The Self Is Neither Happy Nor Suffering }{\suttatitleroot Adukkhamasukhīsutta}}
\addcontentsline{toc}{section}{\tocacronym{SN 24.70} \toctranslation{The Self Is Neither Happy Nor Suffering } \tocroot{Adukkhamasukhīsutta}}
\markboth{The Self Is Neither Happy Nor Suffering }{Adukkhamasukhīsutta}
\extramarks{SN 24.70}{SN 24.70}

At\marginnote{1.1} \textsanskrit{Sāvatthī}. 

“Mendicants,\marginnote{1.2} when what exists, because of grasping what and insisting on what, does the view arise: ‘The self is neither happy nor suffering, and is well after death’?” 

“Our\marginnote{1.4} teachings are rooted in the Buddha. …” 

“When\marginnote{2.1} form exists, because of grasping form and insisting on form, the view arises: ‘The self is neither happy nor suffering, and is well after death.’ When feeling … perception … choices … consciousness exists, because of grasping consciousness and insisting on consciousness, the view arises: ‘The self is neither happy nor suffering, and is well after death.’ 

What\marginnote{3.1} do you think, mendicants? Is form permanent or impermanent?” 

“Impermanent,\marginnote{3.3} sir.” … 

“And\marginnote{3.7} so, what’s impermanent is suffering. When this exists, grasping at this, the view arises: ‘The self is neither happy nor suffering, and is well after death.’ Is feeling … perception … choices … consciousness permanent or impermanent?” 

“Impermanent,\marginnote{3.14} sir.” … 

“But\marginnote{3.15} by not grasping what’s impermanent, suffering, and perishable, would such a view arise?” 

“No,\marginnote{3.17} sir.” 

“And\marginnote{3.18} so, what’s impermanent is suffering. When this exists, grasping at this, the view arises: ‘The self is neither happy nor suffering, and is well after death.’” 

%
\addtocontents{toc}{\let\protect\contentsline\protect\nopagecontentsline}
\chapter*{The Chapter on the Fourth Round }
\addcontentsline{toc}{chapter}{\tocchapterline{The Chapter on the Fourth Round }}
\addtocontents{toc}{\let\protect\contentsline\protect\oldcontentsline}

%
\section*{{\suttatitleacronym SN 24.71}{\suttatitletranslation Winds }{\suttatitleroot Navātasutta}}
\addcontentsline{toc}{section}{\tocacronym{SN 24.71} \toctranslation{Winds } \tocroot{Navātasutta}}
\markboth{Winds }{Navātasutta}
\extramarks{SN 24.71}{SN 24.71}

At\marginnote{1.1} \textsanskrit{Sāvatthī}. 

“Mendicants,\marginnote{1.2} when what exists, because of grasping what and insisting on what, does the view arise: ‘Winds don’t blow; rivers don’t flow; pregnant women don’t give birth; the moon and stars neither rise nor set, but stand firm like a pillar’?” 

“Our\marginnote{1.4} teachings are rooted in the Buddha. …” 

“When\marginnote{2.1} form exists, because of grasping form and insisting on form, the view arises: ‘Winds don’t blow; rivers don’t flow; pregnant women don’t give birth; the moon and stars neither rise nor set, but stand firm like a pillar.’ When feeling … perception … choices … consciousness exists, because of grasping consciousness and insisting on consciousness, the view arises: ‘Winds don’t blow; rivers don’t flow; pregnant women don’t give birth; the moon and stars neither rise nor set, but stand firm like a pillar.’ What do you think, mendicants? Is form permanent or impermanent?” 

“Impermanent,\marginnote{2.10} sir.” 

“But\marginnote{2.11} if it’s impermanent, is it suffering or happiness?” 

“Suffering,\marginnote{2.12} sir.” 

“But\marginnote{2.13} if it’s impermanent, suffering, and liable to wear out, is it fit to be regarded thus: ‘This is mine, I am this, this is my self’?” 

“No,\marginnote{2.15} sir.” 

“Is\marginnote{2.16} feeling … perception … choices … consciousness permanent or impermanent?” 

“Impermanent,\marginnote{2.20} sir.” 

“But\marginnote{2.21} if it’s impermanent, is it suffering or happiness?” 

“Suffering,\marginnote{2.22} sir.” 

“But\marginnote{2.23} if it’s impermanent, suffering, and liable to wear out, is it fit to be regarded thus: ‘This is mine, I am this, this is my self’?” 

“No,\marginnote{2.25} sir.” 

“So\marginnote{3.1} you should truly see any kind of form at all—past, future, or present; internal or external; coarse or fine; inferior or superior; far or near: \emph{all} form—with right understanding: ‘This is not mine, I am not this, this is not my self.’ You should truly see any kind of feeling … perception … choices … consciousness at all—past, future, or present; internal or external; coarse or fine; inferior or superior; far or near: \emph{all} consciousness—with right understanding: ‘This is not mine, I am not this, this is not my self.’ 

Seeing\marginnote{4.1} this … They understand: ‘… there is no return to any state of existence.’” 

%
\section*{{\suttatitleacronym SN 24.72–95}{\suttatitletranslation This Is Mine, Etc. }{\suttatitleroot Etaṁmamādisutta}}
\addcontentsline{toc}{section}{\tocacronym{SN 24.72–95} \toctranslation{This Is Mine, Etc. } \tocroot{Etaṁmamādisutta}}
\markboth{This Is Mine, Etc. }{Etaṁmamādisutta}
\extramarks{SN 24.72–95}{SN 24.72–95}

(To\marginnote{1.1} be completed in the same way as the 24 discourses of the second chapter.) 

%
\section*{{\suttatitleacronym SN 24.96}{\suttatitletranslation The Self Is Neither Happy Nor Suffering }{\suttatitleroot Adukkhamasukhīsutta}}
\addcontentsline{toc}{section}{\tocacronym{SN 24.96} \toctranslation{The Self Is Neither Happy Nor Suffering } \tocroot{Adukkhamasukhīsutta}}
\markboth{The Self Is Neither Happy Nor Suffering }{Adukkhamasukhīsutta}
\extramarks{SN 24.96}{SN 24.96}

At\marginnote{1.1} \textsanskrit{Sāvatthī}. 

“Mendicants,\marginnote{1.2} when what exists, because of grasping what and insisting on what, does the view arise: ‘The self is neither happy nor suffering, and is well after death’?” 

“Our\marginnote{1.4} teachings are rooted in the Buddha. …” 

“When\marginnote{2.1} form exists, because of grasping form and insisting on form, the view arises: ‘The self is neither happy nor suffering, and is well after death.’ When feeling … perception … choices … consciousness exists, because of grasping consciousness and insisting on consciousness, the view arises: ‘The self is neither happy nor suffering, and is well after death.’ 

What\marginnote{3.1} do you think, mendicants? Is form permanent or impermanent?” 

“Impermanent,\marginnote{3.3} sir.” 

“But\marginnote{3.4} if it’s impermanent, is it suffering or happiness?” 

“Suffering,\marginnote{3.5} sir.” 

“But\marginnote{3.6} if it’s impermanent, suffering, and liable to wear out, is it fit to be regarded thus: ‘This is mine, I am this, this is my self’?” 

“No,\marginnote{3.8} sir.” 

“Is\marginnote{3.9} feeling … perception … choices … consciousness permanent or impermanent?” 

“Impermanent,\marginnote{3.13} sir.” 

“But\marginnote{3.14} if it’s impermanent, is it suffering or happiness?” 

“Suffering,\marginnote{3.15} sir.” 

“But\marginnote{3.16} if it’s impermanent, suffering, and liable to wear out, is it fit to be regarded thus: ‘This is mine, I am this, this is my self’?” 

“No,\marginnote{3.18} sir.” 

“So\marginnote{4.1} you should truly see any kind of form at all—past, future, or present; internal or external; coarse or fine; inferior or superior; far or near: \emph{all} form—with right understanding: ‘This is not mine, I am not this, this is not my self.’ You should truly see any kind of feeling … perception … choices … consciousness at all—past, future, or present; internal or external; coarse or fine; inferior or superior; far or near: \emph{all} consciousness—with right understanding: ‘This is not mine, I am not this, this is not my self.’ 

Seeing\marginnote{5.1} this, a learned noble disciple grows disillusioned with form, feeling, perception, choices, and consciousness. Being disillusioned, desire fades away. When desire fades away they’re freed. When they’re freed, they know they’re freed. 

They\marginnote{5.3} understand: ‘Rebirth is ended, the spiritual journey has been completed, what had to be done has been done, there is no return to any state of existence.’” 

\scendsutta{The Linked Discourses on views are complete. }

%
\addtocontents{toc}{\let\protect\contentsline\protect\nopagecontentsline}
\part*{Linked Discourses on Arrival at the Truth }
\addcontentsline{toc}{part}{Linked Discourses on Arrival at the Truth }
\markboth{}{}
\addtocontents{toc}{\let\protect\contentsline\protect\oldcontentsline}

%
\addtocontents{toc}{\let\protect\contentsline\protect\nopagecontentsline}
\chapter*{The Chapter on the Eye }
\addcontentsline{toc}{chapter}{\tocchapterline{The Chapter on the Eye }}
\addtocontents{toc}{\let\protect\contentsline\protect\oldcontentsline}

%
\section*{{\suttatitleacronym SN 25.1}{\suttatitletranslation The Eye }{\suttatitleroot Cakkhusutta}}
\addcontentsline{toc}{section}{\tocacronym{SN 25.1} \toctranslation{The Eye } \tocroot{Cakkhusutta}}
\markboth{The Eye }{Cakkhusutta}
\extramarks{SN 25.1}{SN 25.1}

At\marginnote{1.1} \textsanskrit{Sāvatthī}. 

“Mendicants,\marginnote{1.2} the eye is impermanent, decaying, and perishing. The ear, nose, tongue, body, and mind are impermanent, decaying, and perishing. 

Someone\marginnote{1.8} who has faith and confidence in these principles is called a follower by faith. They’ve arrived at inevitability regarding the right path, they’ve arrived at the level of the good person, and they’ve transcended the level of the bad person. They can’t do any deed which would make them be reborn in hell, the animal realm, or the ghost realm. They can’t die without realizing the fruit of stream-entry. 

Someone\marginnote{2.1} who accepts these principles after considering them with a degree of wisdom is called a follower of the teachings. They’ve arrived at inevitability regarding the right path, they’ve arrived at the level of the good person, and they’ve transcended the level of the bad person. They can’t do any deed which would make them be reborn in hell, the animal realm, or the ghost realm. They can’t die without realizing the fruit of stream-entry. 

Someone\marginnote{2.4} who understands and sees these principles is called a stream-enterer, not liable to be reborn in the underworld, bound for awakening.” 

%
\section*{{\suttatitleacronym SN 25.2}{\suttatitletranslation Sights }{\suttatitleroot Rūpasutta}}
\addcontentsline{toc}{section}{\tocacronym{SN 25.2} \toctranslation{Sights } \tocroot{Rūpasutta}}
\markboth{Sights }{Rūpasutta}
\extramarks{SN 25.2}{SN 25.2}

At\marginnote{1.1} \textsanskrit{Sāvatthī}. 

“Mendicants,\marginnote{1.2} sights are impermanent, decaying, and perishing. Sounds, smells, tastes, touches, and thoughts are impermanent, decaying, and perishing. 

Someone\marginnote{1.8} who has faith and confidence in these principles is called a follower by faith. They’ve arrived at inevitability regarding the right path, they’ve arrived at the level of the good person, and they’ve transcended the level of the bad person. They can’t do any deed which would make them be reborn in hell, the animal realm, or the ghost realm. They can’t die without realizing the fruit of stream-entry. 

Someone\marginnote{2.1} who accepts these principles after considering them with a degree of wisdom is called a follower of the teachings. They’ve arrived at inevitability regarding the right path, they’ve arrived at the level of the good person, and they’ve transcended the level of the bad person. They can’t do any deed which would make them be reborn in hell, the animal realm, or the ghost realm. They can’t die without realizing the fruit of stream-entry. 

Someone\marginnote{2.4} who understands and sees these principles is called a stream-enterer, not liable to be reborn in the underworld, bound for awakening.” 

%
\section*{{\suttatitleacronym SN 25.3}{\suttatitletranslation Consciousness }{\suttatitleroot Viññāṇasutta}}
\addcontentsline{toc}{section}{\tocacronym{SN 25.3} \toctranslation{Consciousness } \tocroot{Viññāṇasutta}}
\markboth{Consciousness }{Viññāṇasutta}
\extramarks{SN 25.3}{SN 25.3}

At\marginnote{1.1} \textsanskrit{Sāvatthī}. 

“Mendicants,\marginnote{1.2} eye consciousness is impermanent, decaying, and perishing. Ear consciousness, nose consciousness, tongue consciousness, body consciousness, and mind consciousness are impermanent, decaying, and perishing. 

Someone\marginnote{1.8} who has faith and confidence in these principles is called a follower by faith. …” 

%
\section*{{\suttatitleacronym SN 25.4}{\suttatitletranslation Contact }{\suttatitleroot Samphassasutta}}
\addcontentsline{toc}{section}{\tocacronym{SN 25.4} \toctranslation{Contact } \tocroot{Samphassasutta}}
\markboth{Contact }{Samphassasutta}
\extramarks{SN 25.4}{SN 25.4}

At\marginnote{1.1} \textsanskrit{Sāvatthī}. 

“Mendicants,\marginnote{1.2} eye contact is impermanent, decaying, and perishing. Ear contact, nose contact, tongue contact, body contact, and mind contact are impermanent, decaying, and perishing. 

Someone\marginnote{1.8} who has faith and confidence in these principles is called a follower by faith. …” 

%
\section*{{\suttatitleacronym SN 25.5}{\suttatitletranslation Feeling }{\suttatitleroot Samphassajasutta}}
\addcontentsline{toc}{section}{\tocacronym{SN 25.5} \toctranslation{Feeling } \tocroot{Samphassajasutta}}
\markboth{Feeling }{Samphassajasutta}
\extramarks{SN 25.5}{SN 25.5}

At\marginnote{1.1} \textsanskrit{Sāvatthī}. 

“Mendicants,\marginnote{1.2} feeling born of eye contact is impermanent, decaying, and perishing. Feeling born of ear contact, feeling born of nose contact, feeling born of tongue contact, feeling born of body contact, and feeling born of mind contact are impermanent, decaying, and perishing. 

Someone\marginnote{1.8} who has faith and confidence in these principles is called a follower by faith. …” 

%
\section*{{\suttatitleacronym SN 25.6}{\suttatitletranslation Perception }{\suttatitleroot Rūpasaññāsutta}}
\addcontentsline{toc}{section}{\tocacronym{SN 25.6} \toctranslation{Perception } \tocroot{Rūpasaññāsutta}}
\markboth{Perception }{Rūpasaññāsutta}
\extramarks{SN 25.6}{SN 25.6}

At\marginnote{1.1} \textsanskrit{Sāvatthī}. 

“Mendicants,\marginnote{1.2} perception of sights is impermanent, decaying, and perishing. Perception of sounds, perception of smells, perception of tastes, perception of touches, and perception of thoughts are impermanent, decaying, and perishing. 

Someone\marginnote{1.8} who has faith and confidence in these principles is called a follower by faith. …” 

%
\section*{{\suttatitleacronym SN 25.7}{\suttatitletranslation Intention }{\suttatitleroot Rūpasañcetanāsutta}}
\addcontentsline{toc}{section}{\tocacronym{SN 25.7} \toctranslation{Intention } \tocroot{Rūpasañcetanāsutta}}
\markboth{Intention }{Rūpasañcetanāsutta}
\extramarks{SN 25.7}{SN 25.7}

At\marginnote{1.1} \textsanskrit{Sāvatthī}. 

“Mendicants,\marginnote{1.2} intention regarding sights is impermanent, decaying, and perishing. Intention regarding sounds, intention regarding smells, intention regarding tastes, intention regarding touches, and intentions regarding thoughts are impermanent, decaying, and perishing. 

Someone\marginnote{1.8} who has faith and confidence in these principles is called a follower by faith. …” 

%
\section*{{\suttatitleacronym SN 25.8}{\suttatitletranslation Craving For Sights }{\suttatitleroot Rūpataṇhāsutta}}
\addcontentsline{toc}{section}{\tocacronym{SN 25.8} \toctranslation{Craving For Sights } \tocroot{Rūpataṇhāsutta}}
\markboth{Craving For Sights }{Rūpataṇhāsutta}
\extramarks{SN 25.8}{SN 25.8}

At\marginnote{1.1} \textsanskrit{Sāvatthī}. 

“Mendicants,\marginnote{1.2} craving for sights is impermanent, decaying, and perishing. Craving for sounds, craving for smells, craving for tastes, craving for touches, and craving for thoughts are impermanent, decaying, and perishing. 

Someone\marginnote{1.8} who has faith and confidence in these principles is called a follower by faith. …” 

%
\section*{{\suttatitleacronym SN 25.9}{\suttatitletranslation Elements }{\suttatitleroot Pathavīdhātusutta}}
\addcontentsline{toc}{section}{\tocacronym{SN 25.9} \toctranslation{Elements } \tocroot{Pathavīdhātusutta}}
\markboth{Elements }{Pathavīdhātusutta}
\extramarks{SN 25.9}{SN 25.9}

At\marginnote{1.1} \textsanskrit{Sāvatthī}. 

“Mendicants,\marginnote{1.2} the earth element is impermanent, decaying, and perishing. The water element, the fire element, the air element, the space element, and the consciousness element are impermanent, decaying, and perishing. 

Someone\marginnote{1.8} who has faith and confidence in these principles is called a follower by faith. …” 

%
\section*{{\suttatitleacronym SN 25.10}{\suttatitletranslation The Aggregates }{\suttatitleroot Khandhasutta}}
\addcontentsline{toc}{section}{\tocacronym{SN 25.10} \toctranslation{The Aggregates } \tocroot{Khandhasutta}}
\markboth{The Aggregates }{Khandhasutta}
\extramarks{SN 25.10}{SN 25.10}

At\marginnote{1.1} \textsanskrit{Sāvatthī}. 

“Mendicants,\marginnote{1.2} form is impermanent, decaying, and perishing. Feeling, perception, choices, and consciousness are impermanent, decaying, and perishing. 

Someone\marginnote{1.7} who has faith and confidence in these principles is called a follower by faith. They’ve arrived at inevitability regarding the right path, they’ve arrived at the level of the good person, and they’ve transcended the level of the bad person. They can’t do any deed which would make them be reborn in hell, the animal realm, or the ghost realm. They can’t die without realizing the fruit of stream-entry. 

Someone\marginnote{2.1} who accepts these principles after considering them with a degree of wisdom is called a follower of the teachings. They’ve arrived at inevitability regarding the right path, they’ve arrived at the level of the good person, and they’ve transcended the level of the bad person. They can’t do any deed which would make them be reborn in hell, the animal realm, or the ghost realm. They can’t die without realizing the fruit of stream-entry. 

Someone\marginnote{2.4} who understands and sees these principles is called a stream-enterer, not liable to be reborn in the underworld, bound for awakening.” 

\scendsutta{The Linked Discourses on arrival are complete. }

%
\addtocontents{toc}{\let\protect\contentsline\protect\nopagecontentsline}
\part*{Linked Discourses on Arising }
\addcontentsline{toc}{part}{Linked Discourses on Arising }
\markboth{}{}
\addtocontents{toc}{\let\protect\contentsline\protect\oldcontentsline}

%
\addtocontents{toc}{\let\protect\contentsline\protect\nopagecontentsline}
\chapter*{The Chapter on Arising }
\addcontentsline{toc}{chapter}{\tocchapterline{The Chapter on Arising }}
\addtocontents{toc}{\let\protect\contentsline\protect\oldcontentsline}

%
\section*{{\suttatitleacronym SN 26.1}{\suttatitletranslation The Eye }{\suttatitleroot Cakkhusutta}}
\addcontentsline{toc}{section}{\tocacronym{SN 26.1} \toctranslation{The Eye } \tocroot{Cakkhusutta}}
\markboth{The Eye }{Cakkhusutta}
\extramarks{SN 26.1}{SN 26.1}

At\marginnote{1.1} \textsanskrit{Sāvatthī}. 

“Mendicants,\marginnote{1.2} the arising, continuation, rebirth, and manifestation of the eye is the arising of suffering, the continuation of diseases, and the manifestation of old age and death. The arising, continuation, rebirth, and manifestation of the ear, nose, tongue, body, and mind is the arising of suffering, the continuation of diseases, and the manifestation of old age and death. The cessation, settling, and ending of the eye is the cessation of suffering, the settling of diseases, and the ending of old age and death. The cessation, settling, and ending of the ear, nose, tongue, body, and mind is the cessation of suffering, the settling of diseases, and the ending of old age and death.” 

%
\section*{{\suttatitleacronym SN 26.2}{\suttatitletranslation Sights }{\suttatitleroot Rūpasutta}}
\addcontentsline{toc}{section}{\tocacronym{SN 26.2} \toctranslation{Sights } \tocroot{Rūpasutta}}
\markboth{Sights }{Rūpasutta}
\extramarks{SN 26.2}{SN 26.2}

At\marginnote{1.1} \textsanskrit{Sāvatthī}. 

“Mendicants,\marginnote{1.2} the arising, continuation, rebirth, and manifestation of sights is the arising of suffering, the continuation of diseases, and the manifestation of old age and death. The arising, continuation, rebirth, and manifestation of sounds, smells, tastes, touches, and thoughts is the arising of suffering, the continuation of diseases, and the manifestation of old age and death. The cessation, settling, and ending of sights, sounds, smells, tastes, touches, and thoughts is the cessation of suffering, the settling of diseases, and the ending of old age and death.” 

%
\section*{{\suttatitleacronym SN 26.3}{\suttatitletranslation Consciousness }{\suttatitleroot Viññāṇasutta}}
\addcontentsline{toc}{section}{\tocacronym{SN 26.3} \toctranslation{Consciousness } \tocroot{Viññāṇasutta}}
\markboth{Consciousness }{Viññāṇasutta}
\extramarks{SN 26.3}{SN 26.3}

At\marginnote{1.1} \textsanskrit{Sāvatthī}. 

“Mendicants,\marginnote{1.2} the arising of eye consciousness … mind consciousness … is the manifestation of old age and death. The cessation of eye consciousness … mind consciousness … is the ending of old age and death.” 

%
\section*{{\suttatitleacronym SN 26.4}{\suttatitletranslation Contact }{\suttatitleroot Samphassasutta}}
\addcontentsline{toc}{section}{\tocacronym{SN 26.4} \toctranslation{Contact } \tocroot{Samphassasutta}}
\markboth{Contact }{Samphassasutta}
\extramarks{SN 26.4}{SN 26.4}

At\marginnote{1.1} \textsanskrit{Sāvatthī}. 

“Mendicants,\marginnote{1.2} the arising of eye contact … mind contact … is the manifestation of old age and death. The cessation of eye contact … mind contact … is the ending of old age and death.” 

%
\section*{{\suttatitleacronym SN 26.5}{\suttatitletranslation Feeling }{\suttatitleroot Samphassajasutta}}
\addcontentsline{toc}{section}{\tocacronym{SN 26.5} \toctranslation{Feeling } \tocroot{Samphassajasutta}}
\markboth{Feeling }{Samphassajasutta}
\extramarks{SN 26.5}{SN 26.5}

At\marginnote{1.1} \textsanskrit{Sāvatthī}. 

“Mendicants,\marginnote{1.2} the arising of feeling born of eye contact … 

the\marginnote{2.1} arising of feeling born of mind contact … is the manifestation of old age and death. The cessation of feeling born of eye contact … the cessation of feeling born of mind contact … is the ending of old age and death.” 

%
\section*{{\suttatitleacronym SN 26.6}{\suttatitletranslation Perception }{\suttatitleroot Saññāsutta}}
\addcontentsline{toc}{section}{\tocacronym{SN 26.6} \toctranslation{Perception } \tocroot{Saññāsutta}}
\markboth{Perception }{Saññāsutta}
\extramarks{SN 26.6}{SN 26.6}

At\marginnote{1.1} \textsanskrit{Sāvatthī}. 

“Mendicants,\marginnote{1.2} the arising of perception of sights … perception of thoughts … is the manifestation of old age and death. The cessation of perception of sights … perception of thoughts … is the ending of old age and death.” 

%
\section*{{\suttatitleacronym SN 26.7}{\suttatitletranslation Intention }{\suttatitleroot Sañcetanāsutta}}
\addcontentsline{toc}{section}{\tocacronym{SN 26.7} \toctranslation{Intention } \tocroot{Sañcetanāsutta}}
\markboth{Intention }{Sañcetanāsutta}
\extramarks{SN 26.7}{SN 26.7}

At\marginnote{1.1} \textsanskrit{Sāvatthī}. 

“Mendicants,\marginnote{1.2} the arising of intentions regarding sights … intentions regarding thoughts … is the manifestation of old age and death. The cessation of intentions regarding sights … intentions regarding thoughts … is the ending of old age and death.” 

%
\section*{{\suttatitleacronym SN 26.8}{\suttatitletranslation Craving }{\suttatitleroot Taṇhāsutta}}
\addcontentsline{toc}{section}{\tocacronym{SN 26.8} \toctranslation{Craving } \tocroot{Taṇhāsutta}}
\markboth{Craving }{Taṇhāsutta}
\extramarks{SN 26.8}{SN 26.8}

At\marginnote{1.1} \textsanskrit{Sāvatthī}. 

“Mendicants,\marginnote{1.2} the arising of craving for sights … craving for thoughts … is the manifestation of old age and death. The cessation of craving for sights … craving for thoughts … is the ending of old age and death.” 

%
\section*{{\suttatitleacronym SN 26.9}{\suttatitletranslation Elements }{\suttatitleroot Dhātusutta}}
\addcontentsline{toc}{section}{\tocacronym{SN 26.9} \toctranslation{Elements } \tocroot{Dhātusutta}}
\markboth{Elements }{Dhātusutta}
\extramarks{SN 26.9}{SN 26.9}

At\marginnote{1.1} \textsanskrit{Sāvatthī}. 

“Mendicants,\marginnote{1.2} the arising, continuation, rebirth, and manifestation of the earth element, the water element, the fire element, the air element, the space element, and the consciousness element is the arising of suffering, the continuation of diseases, and the manifestation of old age and death. The cessation of the earth element, the water element, the fire element, the air element, the space element, and the consciousness element is the cessation of suffering, the settling of diseases, and the ending of old age and death.” 

%
\section*{{\suttatitleacronym SN 26.10}{\suttatitletranslation The Aggregates }{\suttatitleroot Khandhasutta}}
\addcontentsline{toc}{section}{\tocacronym{SN 26.10} \toctranslation{The Aggregates } \tocroot{Khandhasutta}}
\markboth{The Aggregates }{Khandhasutta}
\extramarks{SN 26.10}{SN 26.10}

At\marginnote{1.1} \textsanskrit{Sāvatthī}. 

“Mendicants,\marginnote{1.2} the arising, continuation, rebirth, and manifestation of form is the arising of suffering, the continuation of diseases, and the manifestation of old age and death. The arising, continuation, rebirth, and manifestation of feeling, perception, choices, and consciousness is the arising of suffering, the continuation of diseases, and the manifestation of old age and death. The cessation, settling, and ending of form is the cessation of suffering, the settling of diseases, and the ending of old age and death. The cessation, settling, and ending of feeling, perception, choices, and consciousness is the cessation of suffering, the settling of diseases, and the ending of old age and death.” 

\scendsutta{The Linked Discourses on arising are complete. }

%
\addtocontents{toc}{\let\protect\contentsline\protect\nopagecontentsline}
\part*{Linked Discourses on Corruptions }
\addcontentsline{toc}{part}{Linked Discourses on Corruptions }
\markboth{}{}
\addtocontents{toc}{\let\protect\contentsline\protect\oldcontentsline}

%
\addtocontents{toc}{\let\protect\contentsline\protect\nopagecontentsline}
\chapter*{The Chapter on Corruptions }
\addcontentsline{toc}{chapter}{\tocchapterline{The Chapter on Corruptions }}
\addtocontents{toc}{\let\protect\contentsline\protect\oldcontentsline}

%
\section*{{\suttatitleacronym SN 27.1}{\suttatitletranslation The Eye }{\suttatitleroot Cakkhusutta}}
\addcontentsline{toc}{section}{\tocacronym{SN 27.1} \toctranslation{The Eye } \tocroot{Cakkhusutta}}
\markboth{The Eye }{Cakkhusutta}
\extramarks{SN 27.1}{SN 27.1}

At\marginnote{1.1} \textsanskrit{Sāvatthī}. 

“Mendicants,\marginnote{1.2} desire and greed for the eye, ear, nose, tongue, body, or mind is a corruption of the mind. When a mendicant has given up mental corruption in these six cases, their mind inclines to renunciation. A mind imbued with renunciation is declared to be capable of directly knowing anything that can be realized.” 

%
\section*{{\suttatitleacronym SN 27.2}{\suttatitletranslation Sights }{\suttatitleroot Rūpasutta}}
\addcontentsline{toc}{section}{\tocacronym{SN 27.2} \toctranslation{Sights } \tocroot{Rūpasutta}}
\markboth{Sights }{Rūpasutta}
\extramarks{SN 27.2}{SN 27.2}

At\marginnote{1.1} \textsanskrit{Sāvatthī}. 

“Mendicants,\marginnote{1.2} desire and greed for sights, sounds, smells, tastes, touches, or thoughts is a corruption of the mind. When a mendicant has given up mental corruption in these six cases, their mind inclines to renunciation. A mind imbued with renunciation is declared to be capable of directly knowing anything that can be realized.” 

%
\section*{{\suttatitleacronym SN 27.3}{\suttatitletranslation Consciousness }{\suttatitleroot Viññāṇasutta}}
\addcontentsline{toc}{section}{\tocacronym{SN 27.3} \toctranslation{Consciousness } \tocroot{Viññāṇasutta}}
\markboth{Consciousness }{Viññāṇasutta}
\extramarks{SN 27.3}{SN 27.3}

At\marginnote{1.1} \textsanskrit{Sāvatthī}. 

“Mendicants,\marginnote{1.2} desire and greed for eye consciousness, ear consciousness, nose consciousness, tongue consciousness, body consciousness, or mind consciousness is a corruption of the mind. When a mendicant has given up mental corruption in these six cases, their mind inclines to renunciation. A mind imbued with renunciation is declared to be capable of directly knowing anything that can be realized.” 

%
\section*{{\suttatitleacronym SN 27.4}{\suttatitletranslation Contact }{\suttatitleroot Samphassasutta}}
\addcontentsline{toc}{section}{\tocacronym{SN 27.4} \toctranslation{Contact } \tocroot{Samphassasutta}}
\markboth{Contact }{Samphassasutta}
\extramarks{SN 27.4}{SN 27.4}

At\marginnote{1.1} \textsanskrit{Sāvatthī}. 

“Mendicants,\marginnote{1.2} desire and greed for eye contact, ear contact, nose contact, tongue contact, body contact, or mind contact is a corruption of the mind. …” 

%
\section*{{\suttatitleacronym SN 27.5}{\suttatitletranslation Feeling }{\suttatitleroot Samphassajasutta}}
\addcontentsline{toc}{section}{\tocacronym{SN 27.5} \toctranslation{Feeling } \tocroot{Samphassajasutta}}
\markboth{Feeling }{Samphassajasutta}
\extramarks{SN 27.5}{SN 27.5}

At\marginnote{1.1} \textsanskrit{Sāvatthī}. 

“Mendicants,\marginnote{1.2} desire and greed for feeling born of eye contact, feeling born of ear contact, feeling born of nose contact, feeling born of tongue contact, feeling born of body contact, or feeling born of mind contact is a defilement of the mind. …” 

%
\section*{{\suttatitleacronym SN 27.6}{\suttatitletranslation Perception }{\suttatitleroot Saññāsutta}}
\addcontentsline{toc}{section}{\tocacronym{SN 27.6} \toctranslation{Perception } \tocroot{Saññāsutta}}
\markboth{Perception }{Saññāsutta}
\extramarks{SN 27.6}{SN 27.6}

At\marginnote{1.1} \textsanskrit{Sāvatthī}. 

“Mendicants,\marginnote{1.2} desire and greed for perception of sights, perception of sounds, perception of smells, perception of tastes, perception of touches, or perception of thoughts is a corruption of the mind. …” 

%
\section*{{\suttatitleacronym SN 27.7}{\suttatitletranslation Intention }{\suttatitleroot Sañcetanāsutta}}
\addcontentsline{toc}{section}{\tocacronym{SN 27.7} \toctranslation{Intention } \tocroot{Sañcetanāsutta}}
\markboth{Intention }{Sañcetanāsutta}
\extramarks{SN 27.7}{SN 27.7}

At\marginnote{1.1} \textsanskrit{Sāvatthī}. 

“Mendicants,\marginnote{1.2} desire and greed for intention regarding sights, intention regarding sounds, intention regarding smells, intention regarding tastes, intention regarding touches, or intention regarding thoughts is a corruption of the mind. …” 

%
\section*{{\suttatitleacronym SN 27.8}{\suttatitletranslation Craving }{\suttatitleroot Taṇhāsutta}}
\addcontentsline{toc}{section}{\tocacronym{SN 27.8} \toctranslation{Craving } \tocroot{Taṇhāsutta}}
\markboth{Craving }{Taṇhāsutta}
\extramarks{SN 27.8}{SN 27.8}

At\marginnote{1.1} \textsanskrit{Sāvatthī}. 

“Mendicants,\marginnote{1.2} desire and greed for craving for sights, craving for sounds, craving for smells, craving for tastes, craving for touches, or craving for thoughts is a corruption of the mind. …” 

%
\section*{{\suttatitleacronym SN 27.9}{\suttatitletranslation Elements }{\suttatitleroot Dhātusutta}}
\addcontentsline{toc}{section}{\tocacronym{SN 27.9} \toctranslation{Elements } \tocroot{Dhātusutta}}
\markboth{Elements }{Dhātusutta}
\extramarks{SN 27.9}{SN 27.9}

At\marginnote{1.1} \textsanskrit{Sāvatthī}. 

“Mendicants,\marginnote{1.2} desire and greed for the earth element, the water element, the fire element, the air element, the space element, or the consciousness element is a corruption of the mind. …” 

%
\section*{{\suttatitleacronym SN 27.10}{\suttatitletranslation The Aggregates }{\suttatitleroot Khandhasutta}}
\addcontentsline{toc}{section}{\tocacronym{SN 27.10} \toctranslation{The Aggregates } \tocroot{Khandhasutta}}
\markboth{The Aggregates }{Khandhasutta}
\extramarks{SN 27.10}{SN 27.10}

At\marginnote{1.1} \textsanskrit{Sāvatthī}. 

“Mendicants,\marginnote{1.2} desire and greed for form, feeling, perception, choices, or consciousness is a corruption of the mind. When a mendicant has given up mental corruption in these five cases, their mind inclines to renunciation. A mind imbued with renunciation is declared to be capable of directly knowing anything that can be realized.” 

\scendsutta{The Linked Discourses on corruptions are complete. }

%
\addtocontents{toc}{\let\protect\contentsline\protect\nopagecontentsline}
\part*{Linked Discourses with Sāriputta }
\addcontentsline{toc}{part}{Linked Discourses with Sāriputta }
\markboth{}{}
\addtocontents{toc}{\let\protect\contentsline\protect\oldcontentsline}

%
\addtocontents{toc}{\let\protect\contentsline\protect\nopagecontentsline}
\chapter*{The Chapter with Sāriputta }
\addcontentsline{toc}{chapter}{\tocchapterline{The Chapter with Sāriputta }}
\addtocontents{toc}{\let\protect\contentsline\protect\oldcontentsline}

%
\section*{{\suttatitleacronym SN 28.1}{\suttatitletranslation Born of Seclusion }{\suttatitleroot Vivekajasutta}}
\addcontentsline{toc}{section}{\tocacronym{SN 28.1} \toctranslation{Born of Seclusion } \tocroot{Vivekajasutta}}
\markboth{Born of Seclusion }{Vivekajasutta}
\extramarks{SN 28.1}{SN 28.1}

At\marginnote{1.1} one time Venerable \textsanskrit{Sāriputta} was staying near \textsanskrit{Sāvatthī} in Jeta’s Grove, \textsanskrit{Anāthapiṇḍika}’s monastery. Then Venerable \textsanskrit{Sāriputta} robed up in the morning and, taking his bowl and robe, entered \textsanskrit{Sāvatthī} for alms. He wandered for alms in \textsanskrit{Sāvatthī}. After the meal, on his return from almsround, he went to the Dark Forest, plunged deep into it, and sat at the root of a tree for the day’s meditation. 

Then\marginnote{2.1} in the late afternoon, \textsanskrit{Sāriputta} came out of retreat and went to Jeta’s Grove, \textsanskrit{Anāthapiṇḍika}’s monastery. Venerable Ānanda saw him coming off in the distance, and said to him: 

“Reverend\marginnote{2.4} \textsanskrit{Sāriputta}, your faculties are so very clear, and your complexion is pure and bright. What meditation were you practicing today?” 

“Reverend,\marginnote{3.1} quite secluded from sensual pleasures, secluded from unskillful qualities, I entered and remained in the first absorption, which has the rapture and bliss born of seclusion, while placing the mind and keeping it connected. But it didn’t occur to me: ‘I am entering the first absorption’ or ‘I have entered the first absorption’ or ‘I am emerging from the first absorption’.” 

“That\marginnote{3.4} must be because Venerable \textsanskrit{Sāriputta} has long ago totally eradicated ego, possessiveness, and the underlying tendency to conceit. That’s why it didn’t occur to you: ‘I am entering the first absorption’ or ‘I have entered the first absorption’ or ‘I am emerging from the first absorption’.” 

%
\section*{{\suttatitleacronym SN 28.2}{\suttatitletranslation Without Placing the Mind }{\suttatitleroot Avitakkasutta}}
\addcontentsline{toc}{section}{\tocacronym{SN 28.2} \toctranslation{Without Placing the Mind } \tocroot{Avitakkasutta}}
\markboth{Without Placing the Mind }{Avitakkasutta}
\extramarks{SN 28.2}{SN 28.2}

At\marginnote{1.1} \textsanskrit{Sāvatthī}. 

Venerable\marginnote{1.2} Ānanda saw Venerable \textsanskrit{Sāriputta} and said to him: 

“Reverend\marginnote{1.3} \textsanskrit{Sāriputta}, your faculties are so very clear, and your complexion is pure and bright. What meditation were you practicing today?” 

“Reverend,\marginnote{2.1} as the placing of the mind and keeping it connected were stilled, I entered and remained in the second absorption, which has the rapture and bliss born of immersion, with internal clarity and confidence, and unified mind, without placing the mind and keeping it connected. But it didn’t occur to me: ‘I am entering the second absorption’ or ‘I have entered the second absorption’ or ‘I am emerging from the second absorption’.” 

“That\marginnote{2.4} must be because Venerable \textsanskrit{Sāriputta} has long ago totally eradicated ego, possessiveness, and the underlying tendency to conceit. That’s why it didn’t occur to you: ‘I am entering the second absorption’ or ‘I have entered the second absorption’ or ‘I am emerging from the second absorption’.” 

%
\section*{{\suttatitleacronym SN 28.3}{\suttatitletranslation Rapture }{\suttatitleroot Pītisutta}}
\addcontentsline{toc}{section}{\tocacronym{SN 28.3} \toctranslation{Rapture } \tocroot{Pītisutta}}
\markboth{Rapture }{Pītisutta}
\extramarks{SN 28.3}{SN 28.3}

At\marginnote{1.1} \textsanskrit{Sāvatthī}. 

Venerable\marginnote{1.2} Ānanda saw Venerable \textsanskrit{Sāriputta} and said to him: 

“Reverend\marginnote{1.3} \textsanskrit{Sāriputta}, your faculties are so very clear, and your complexion is pure and bright. What meditation were you practicing today?” 

“Reverend,\marginnote{2.1} with the fading away of rapture, I entered and remained in the third absorption, where I meditated with equanimity, mindful and aware, personally experiencing the bliss of which the noble ones declare, ‘Equanimous and mindful, one meditates in bliss.’ But it didn’t occur to me: ‘I am entering the third absorption’ or ‘I have entered the third absorption’ or ‘I am emerging from the third absorption’.” 

“That\marginnote{2.4} must be because Venerable \textsanskrit{Sāriputta} has long ago totally eradicated ego, possessiveness, and the underlying tendency to conceit. That’s why it didn’t occur to you: ‘I am entering the third absorption’ or ‘I have entered the third absorption’ or ‘I am emerging from the third absorption’.” 

%
\section*{{\suttatitleacronym SN 28.4}{\suttatitletranslation Equanimity }{\suttatitleroot Upekkhāsutta}}
\addcontentsline{toc}{section}{\tocacronym{SN 28.4} \toctranslation{Equanimity } \tocroot{Upekkhāsutta}}
\markboth{Equanimity }{Upekkhāsutta}
\extramarks{SN 28.4}{SN 28.4}

At\marginnote{1.1} \textsanskrit{Sāvatthī}. 

Venerable\marginnote{1.2} Ānanda saw Venerable \textsanskrit{Sāriputta} and said to him: 

“Reverend\marginnote{1.3} \textsanskrit{Sāriputta}, your faculties are so very clear, and your complexion is pure and bright. What meditation were you practicing today?” 

“Reverend,\marginnote{2.1} with the giving up of pleasure and pain, and the ending of former happiness and sadness, I entered and remained in the fourth absorption, without pleasure or pain, with pure equanimity and mindfulness. But it didn’t occur to me: ‘I am entering the fourth absorption’ or ‘I have entered the fourth absorption’ or ‘I am emerging from the fourth absorption’.” 

“That\marginnote{2.4} must be because Venerable \textsanskrit{Sāriputta} has long ago totally eradicated ego, possessiveness, and the underlying tendency to conceit. That’s why it didn’t occur to you: ‘I am entering the fourth absorption’ or ‘I have entered the fourth absorption’ or ‘I am emerging from the fourth absorption’.” 

%
\section*{{\suttatitleacronym SN 28.5}{\suttatitletranslation The Dimension of Infinite Space }{\suttatitleroot Ākāsānañcāyatanasutta}}
\addcontentsline{toc}{section}{\tocacronym{SN 28.5} \toctranslation{The Dimension of Infinite Space } \tocroot{Ākāsānañcāyatanasutta}}
\markboth{The Dimension of Infinite Space }{Ākāsānañcāyatanasutta}
\extramarks{SN 28.5}{SN 28.5}

At\marginnote{1.1} \textsanskrit{Sāvatthī}. 

Venerable\marginnote{1.2} Ānanda saw Venerable \textsanskrit{Sāriputta} … 

“Reverend,\marginnote{1.3} going totally beyond perceptions of form, with the ending of perceptions of impingement, not focusing on perceptions of diversity, aware that ‘space is infinite’, I entered and remained in the dimension of infinite space. …” … 

%
\section*{{\suttatitleacronym SN 28.6}{\suttatitletranslation The Dimension of Infinite Consciousness }{\suttatitleroot Viññāṇañcāyatanasutta}}
\addcontentsline{toc}{section}{\tocacronym{SN 28.6} \toctranslation{The Dimension of Infinite Consciousness } \tocroot{Viññāṇañcāyatanasutta}}
\markboth{The Dimension of Infinite Consciousness }{Viññāṇañcāyatanasutta}
\extramarks{SN 28.6}{SN 28.6}

At\marginnote{1.1} \textsanskrit{Sāvatthī}. 

Venerable\marginnote{1.2} Ānanda saw Venerable \textsanskrit{Sāriputta} … 

“Reverend,\marginnote{1.3} going totally beyond the dimension of infinite space, aware that ‘consciousness is infinite’, I entered and remained in the dimension of infinite consciousness. …” … 

%
\section*{{\suttatitleacronym SN 28.7}{\suttatitletranslation The Dimension of Nothingness }{\suttatitleroot Ākiñcaññāyatanasutta}}
\addcontentsline{toc}{section}{\tocacronym{SN 28.7} \toctranslation{The Dimension of Nothingness } \tocroot{Ākiñcaññāyatanasutta}}
\markboth{The Dimension of Nothingness }{Ākiñcaññāyatanasutta}
\extramarks{SN 28.7}{SN 28.7}

At\marginnote{1.1} \textsanskrit{Sāvatthī}. 

Venerable\marginnote{1.2} Ānanda saw Venerable \textsanskrit{Sāriputta} … 

“Reverend,\marginnote{1.3} going totally beyond the dimension of infinite consciousness, aware that ‘there is nothing at all’, I entered and remained in the dimension of nothingness. …” … 

%
\section*{{\suttatitleacronym SN 28.8}{\suttatitletranslation The Dimension of Neither Perception Nor Non-Perception }{\suttatitleroot Nevasaññānāsaññāyatanasutta}}
\addcontentsline{toc}{section}{\tocacronym{SN 28.8} \toctranslation{The Dimension of Neither Perception Nor Non-Perception } \tocroot{Nevasaññānāsaññāyatanasutta}}
\markboth{The Dimension of Neither Perception Nor Non-Perception }{Nevasaññānāsaññāyatanasutta}
\extramarks{SN 28.8}{SN 28.8}

At\marginnote{1.1} \textsanskrit{Sāvatthī}. 

Venerable\marginnote{1.2} Ānanda saw Venerable \textsanskrit{Sāriputta} … 

“Reverend,\marginnote{1.3} going totally beyond the dimension of nothingness, I entered and remained in the dimension of neither perception nor non-perception. …” … 

%
\section*{{\suttatitleacronym SN 28.9}{\suttatitletranslation The Attainment of Cessation }{\suttatitleroot Nirodhasamāpattisutta}}
\addcontentsline{toc}{section}{\tocacronym{SN 28.9} \toctranslation{The Attainment of Cessation } \tocroot{Nirodhasamāpattisutta}}
\markboth{The Attainment of Cessation }{Nirodhasamāpattisutta}
\extramarks{SN 28.9}{SN 28.9}

At\marginnote{1.1} \textsanskrit{Sāvatthī}. 

Venerable\marginnote{1.2} Ānanda saw Venerable \textsanskrit{Sāriputta} … 

“Reverend,\marginnote{1.3} going totally beyond the dimension of neither perception nor non-perception, I entered and remained in the cessation of perception and feeling. But it didn’t occur to me: ‘I am entering the cessation of perception and feeling’ or ‘I have entered the cessation of perception and feeling’ or ‘I am emerging from the cessation of perception and feeling’.” 

“That\marginnote{1.6} must be because Venerable \textsanskrit{Sāriputta} has long ago totally eradicated ego, possessiveness, and the underlying tendency to conceit. That’s why it didn’t occur to you: ‘I am entering the cessation of perception and feeling’ or ‘I have entered the cessation of perception and feeling’ or ‘I am emerging from the cessation of perception and feeling’.” 

%
\section*{{\suttatitleacronym SN 28.10}{\suttatitletranslation With Sucimukhī }{\suttatitleroot Sucimukhīsutta}}
\addcontentsline{toc}{section}{\tocacronym{SN 28.10} \toctranslation{With Sucimukhī } \tocroot{Sucimukhīsutta}}
\markboth{With Sucimukhī }{Sucimukhīsutta}
\extramarks{SN 28.10}{SN 28.10}

At\marginnote{1.1} one time Venerable \textsanskrit{Sāriputta} was staying near \textsanskrit{Rājagaha}, in the Bamboo Grove, the squirrels’ feeding ground. Then he robed up in the morning and, taking his bowl and robe, entered \textsanskrit{Rājagaha} for alms. After wandering indiscriminately for almsfood in \textsanskrit{Rājagaha}, he ate his almsfood by a wall. 

Then\marginnote{1.4} the wanderer \textsanskrit{Sucimukhī} went up to Venerable \textsanskrit{Sāriputta} and said to him: 

“Ascetic,\marginnote{2.1} do you eat facing downwards?” 

“No,\marginnote{2.2} sister.” 

“Well\marginnote{2.3} then, do you eat facing upwards?” 

“No,\marginnote{2.4} sister.” 

“Well\marginnote{2.5} then, do you eat facing the cardinal directions?” 

“No,\marginnote{2.6} sister.” 

“Well\marginnote{2.7} then, do you eat facing the intermediate directions?” 

“No,\marginnote{2.8} sister.” 

“When\marginnote{3.1} asked if you eat facing all these directions, you answer ‘no, sister’. How exactly do you eat, ascetic?” 

“Sister,\marginnote{4.2} those ascetics and brahmins who earn a living by geomancy—an unworthy branch of knowledge, a wrong livelihood—are said to eat facing downwards. 

Those\marginnote{4.3} ascetics and brahmins who earn a living by astrology—an unworthy branch of knowledge, a wrong livelihood—are said to eat facing upwards. 

Those\marginnote{4.4} ascetics and brahmins who earn a living by running errands and messages—a wrong livelihood—are said to eat facing the cardinal directions. 

Those\marginnote{4.5} ascetics and brahmins who earn a living by palmistry—an unworthy branch of knowledge, a wrong livelihood—are said to eat facing the intermediate directions. 

I\marginnote{5.1} don’t earn a living by any of these means. I seek alms in a principled manner, and I eat it in a principled manner.” 

Then\marginnote{6.1} \textsanskrit{Sucimukhī} the wanderer went around \textsanskrit{Rājagaha} from street to street and from square to square, and announced: “The Sakyan ascetics eat food in a principled manner! The Sakyan ascetics eat food blamelessly! Give almsfood to the Sakyan ascetics!” 

\scendsutta{The Linked Discourses on \textsanskrit{Sāriputta} are complete. }

%
\addtocontents{toc}{\let\protect\contentsline\protect\nopagecontentsline}
\part*{Linked Discourses on Dragons }
\addcontentsline{toc}{part}{Linked Discourses on Dragons }
\markboth{}{}
\addtocontents{toc}{\let\protect\contentsline\protect\oldcontentsline}

%
\addtocontents{toc}{\let\protect\contentsline\protect\nopagecontentsline}
\chapter*{The Chapter on Dragons }
\addcontentsline{toc}{chapter}{\tocchapterline{The Chapter on Dragons }}
\addtocontents{toc}{\let\protect\contentsline\protect\oldcontentsline}

%
\section*{{\suttatitleacronym SN 29.1}{\suttatitletranslation Plain Version }{\suttatitleroot Suddhikasutta}}
\addcontentsline{toc}{section}{\tocacronym{SN 29.1} \toctranslation{Plain Version } \tocroot{Suddhikasutta}}
\markboth{Plain Version }{Suddhikasutta}
\extramarks{SN 29.1}{SN 29.1}

At\marginnote{1.1} \textsanskrit{Sāvatthī}. 

“Mendicants,\marginnote{1.2} dragons reproduce in these four ways. What four? Dragons are born from eggs, from a womb, from moisture, or spontaneously. These are the four ways that dragons reproduce.” 

%
\section*{{\suttatitleacronym SN 29.2}{\suttatitletranslation Better }{\suttatitleroot Paṇītatarasutta}}
\addcontentsline{toc}{section}{\tocacronym{SN 29.2} \toctranslation{Better } \tocroot{Paṇītatarasutta}}
\markboth{Better }{Paṇītatarasutta}
\extramarks{SN 29.2}{SN 29.2}

At\marginnote{1.1} \textsanskrit{Sāvatthī}. 

“Mendicants,\marginnote{1.2} dragons reproduce in these four ways. What four? Dragons are born from an egg, from a womb, from moisture, or spontaneously. Of these, dragons born from a womb, from moisture, or spontaneously are better than those born from an egg. Dragons born from moisture or spontaneously are better than those born from an egg or from a womb. Dragons born spontaneously are better than those born from an egg, from a womb, or from moisture. These are the four ways that dragons reproduce.” 

%
\section*{{\suttatitleacronym SN 29.3}{\suttatitletranslation Sabbath }{\suttatitleroot Uposathasutta}}
\addcontentsline{toc}{section}{\tocacronym{SN 29.3} \toctranslation{Sabbath } \tocroot{Uposathasutta}}
\markboth{Sabbath }{Uposathasutta}
\extramarks{SN 29.3}{SN 29.3}

At\marginnote{1.1} one time the Buddha was staying near \textsanskrit{Sāvatthī} in Jeta’s Grove, \textsanskrit{Anāthapiṇḍika}’s monastery. Then a mendicant went up to the Buddha, sat down to one side, and said to him: 

“Sir,\marginnote{1.3} what is the cause, what is the reason why some egg-born dragons keep the sabbath, having transformed their bodies?” 

“Mendicant,\marginnote{2.1} it’s when some egg-born dragons think: ‘In the past we did both kinds of deeds by body, speech, and mind. When the body broke up, after death, we were reborn in the company of the egg-born dragons. If today we do good things by body, speech, and mind, when the body breaks up, after death, we may be reborn in a good place, a heavenly realm. Come, let us do good things by way of body, speech, and mind.’ This is the cause, this is the reason why some egg-born dragons keep the sabbath, having transformed their bodies.” 

%
\section*{{\suttatitleacronym SN 29.4}{\suttatitletranslation Sabbath (2nd) }{\suttatitleroot Dutiyauposathasutta}}
\addcontentsline{toc}{section}{\tocacronym{SN 29.4} \toctranslation{Sabbath (2nd) } \tocroot{Dutiyauposathasutta}}
\markboth{Sabbath (2nd) }{Dutiyauposathasutta}
\extramarks{SN 29.4}{SN 29.4}

At\marginnote{1.1} \textsanskrit{Sāvatthī}. 

Then\marginnote{1.2} a mendicant went up to the Buddha … and asked him, “Sir, what is the cause, what is the reason why some womb-born dragons keep the sabbath, having transformed their bodies?” 

(All\marginnote{1.4} should be told in full.) 

%
\section*{{\suttatitleacronym SN 29.5}{\suttatitletranslation Sabbath (3rd) }{\suttatitleroot Tatiyauposathasutta}}
\addcontentsline{toc}{section}{\tocacronym{SN 29.5} \toctranslation{Sabbath (3rd) } \tocroot{Tatiyauposathasutta}}
\markboth{Sabbath (3rd) }{Tatiyauposathasutta}
\extramarks{SN 29.5}{SN 29.5}

At\marginnote{1.1} \textsanskrit{Sāvatthī}. 

Seated\marginnote{1.2} to one side, that mendicant said to the Buddha: 

“Sir,\marginnote{1.3} what is the cause, what is the reason why some moisture-born dragons keep the sabbath, having transformed their bodies?” 

(All\marginnote{1.4} should be told in full.) 

%
\section*{{\suttatitleacronym SN 29.6}{\suttatitletranslation Sabbath (4th) }{\suttatitleroot Catutthauposathasutta}}
\addcontentsline{toc}{section}{\tocacronym{SN 29.6} \toctranslation{Sabbath (4th) } \tocroot{Catutthauposathasutta}}
\markboth{Sabbath (4th) }{Catutthauposathasutta}
\extramarks{SN 29.6}{SN 29.6}

At\marginnote{1.1} \textsanskrit{Sāvatthī}. 

Seated\marginnote{1.2} to one side, that mendicant said to the Buddha: 

“Sir,\marginnote{1.3} what is the cause, what is the reason why some spontaneously-born dragons keep the sabbath, having transformed their bodies?” 

(All\marginnote{2.1} should be told in full.) 

%
\section*{{\suttatitleacronym SN 29.7}{\suttatitletranslation They’ve Heard }{\suttatitleroot Sutasutta}}
\addcontentsline{toc}{section}{\tocacronym{SN 29.7} \toctranslation{They’ve Heard } \tocroot{Sutasutta}}
\markboth{They’ve Heard }{Sutasutta}
\extramarks{SN 29.7}{SN 29.7}

At\marginnote{1.1} \textsanskrit{Sāvatthī}. 

Seated\marginnote{1.2} to one side, that mendicant said to the Buddha: 

“Sir,\marginnote{1.3} what is the cause, what is the reason why someone, when their body breaks up, after death, is reborn in the company of the egg-born dragons?” 

“Mendicant,\marginnote{2.1} it’s when someone does both kinds of deeds by body, speech, and mind. And they’ve heard: ‘The egg-born dragons are long-lived, beautiful, and very happy.’ They think: ‘If only, when my body breaks up, after death, I would be reborn in the company of the egg-born dragons!’ When their body breaks up, after death, they’re reborn in the company of the egg-born dragons. This is the cause, this is the reason why someone, when their body breaks up, after death, is reborn in the company of the egg-born dragons.” 

%
\section*{{\suttatitleacronym SN 29.8}{\suttatitletranslation They’ve Heard (2nd) }{\suttatitleroot Dutiyasutasutta}}
\addcontentsline{toc}{section}{\tocacronym{SN 29.8} \toctranslation{They’ve Heard (2nd) } \tocroot{Dutiyasutasutta}}
\markboth{They’ve Heard (2nd) }{Dutiyasutasutta}
\extramarks{SN 29.8}{SN 29.8}

At\marginnote{1.1} \textsanskrit{Sāvatthī}. 

Seated\marginnote{1.2} to one side, that mendicant said to the Buddha: 

“Sir,\marginnote{1.3} what is the cause, what is the reason why someone, when their body breaks up, after death, is reborn in the company of the womb-born dragons?” 

(All\marginnote{1.4} should be told in full.) 

%
\section*{{\suttatitleacronym SN 29.9}{\suttatitletranslation They’ve Heard (3rd) }{\suttatitleroot Tatiyasutasutta}}
\addcontentsline{toc}{section}{\tocacronym{SN 29.9} \toctranslation{They’ve Heard (3rd) } \tocroot{Tatiyasutasutta}}
\markboth{They’ve Heard (3rd) }{Tatiyasutasutta}
\extramarks{SN 29.9}{SN 29.9}

At\marginnote{1.1} \textsanskrit{Sāvatthī}. 

Seated\marginnote{1.2} to one side, that mendicant said to the Buddha: 

“Sir,\marginnote{1.3} what is the cause, what is the reason why someone, when their body breaks up, after death, is reborn in the company of the moisture-born dragons?” 

(All\marginnote{1.4} should be told in full.) 

%
\section*{{\suttatitleacronym SN 29.10}{\suttatitletranslation They’ve Heard (4th) }{\suttatitleroot Catutthasutasutta}}
\addcontentsline{toc}{section}{\tocacronym{SN 29.10} \toctranslation{They’ve Heard (4th) } \tocroot{Catutthasutasutta}}
\markboth{They’ve Heard (4th) }{Catutthasutasutta}
\extramarks{SN 29.10}{SN 29.10}

At\marginnote{1.1} \textsanskrit{Sāvatthī}. 

Seated\marginnote{1.2} to one side, that mendicant said to the Buddha: 

“Sir,\marginnote{1.3} what is the cause, what is the reason why someone, when their body breaks up, after death, is reborn in the company of the spontaneously-born dragons?” 

(All\marginnote{2.1} should be told in full.) 

%
\section*{{\suttatitleacronym SN 29.11–20}{\suttatitletranslation Ten Discourses On How Giving Helps to Become Egg-Born }{\suttatitleroot Aṇḍajadānūpakārasuttadasaka}}
\addcontentsline{toc}{section}{\tocacronym{SN 29.11–20} \toctranslation{Ten Discourses On How Giving Helps to Become Egg-Born } \tocroot{Aṇḍajadānūpakārasuttadasaka}}
\markboth{Ten Discourses On How Giving Helps to Become Egg-Born }{Aṇḍajadānūpakārasuttadasaka}
\extramarks{SN 29.11–20}{SN 29.11–20}

Seated\marginnote{1.1} to one side, that mendicant said to the Buddha: 

“Sir,\marginnote{1.2} what is the cause, what is the reason why someone, when their body breaks up, after death, is reborn in the company of the egg-born dragons?” 

“Mendicant,\marginnote{2.1} it’s when someone does both kinds of deeds by body, speech, and mind. And they’ve heard: ‘The egg-born dragons are long-lived, beautiful, and very happy.’ They think: ‘If only, when my body breaks up, after death, I would be reborn in the company of the egg-born dragons!’ They give food … drink … clothing … a vehicle … a garland … fragrance … makeup … a bed … a house … a lamp. When their body breaks up, after death, they’re reborn in the company of the egg-born dragons. This is the cause, this is the reason why someone, when their body breaks up, after death, is reborn in the company of the egg-born dragons.” 

%
\section*{{\suttatitleacronym SN 29.21–50}{\suttatitletranslation Thirty Discourses On How Giving Helps to Become Womb-Born, Etc. }{\suttatitleroot Jalābujādidānūpakārasuttattiṁsaka}}
\addcontentsline{toc}{section}{\tocacronym{SN 29.21–50} \toctranslation{Thirty Discourses On How Giving Helps to Become Womb-Born, Etc. } \tocroot{Jalābujādidānūpakārasuttattiṁsaka}}
\markboth{Thirty Discourses On How Giving Helps to Become Womb-Born, Etc. }{Jalābujādidānūpakārasuttattiṁsaka}
\extramarks{SN 29.21–50}{SN 29.21–50}

At\marginnote{1.1} \textsanskrit{Sāvatthī}. 

Seated\marginnote{1.2} to one side, that mendicant said to the Buddha: 

“Sir,\marginnote{1.3} what is the cause, what is the reason why someone, when their body breaks up, after death, is reborn in the company of the womb-born dragons … moisture-born dragons … spontaneously-born dragons?” 

“Mendicant,\marginnote{2.1} it’s when someone does both kinds of deeds by body, speech, and mind. And they’ve heard: ‘The spontaneously-born dragons are long-lived, beautiful, and very happy.’ They think: ‘If only, when my body breaks up, after death, I would be reborn in the company of the spontaneously-born dragons!’ They give food … drink … a lamp. When their body breaks up, after death, they’re reborn in the company of the spontaneously-born dragons. This is the cause, this is the reason why someone, when their body breaks up, after death, is reborn in the company of the spontaneously-born dragons.” 

(Each\marginnote{3.1} set of ten discourses of this series should be treated in the same way.) 

\scendsutta{The Linked Discourses on dragons are complete. }

%
\addtocontents{toc}{\let\protect\contentsline\protect\nopagecontentsline}
\part*{Linked Discourses on Phoenixes }
\addcontentsline{toc}{part}{Linked Discourses on Phoenixes }
\markboth{}{}
\addtocontents{toc}{\let\protect\contentsline\protect\oldcontentsline}

%
\addtocontents{toc}{\let\protect\contentsline\protect\nopagecontentsline}
\chapter*{The Chapter on Phoenixes }
\addcontentsline{toc}{chapter}{\tocchapterline{The Chapter on Phoenixes }}
\addtocontents{toc}{\let\protect\contentsline\protect\oldcontentsline}

%
\section*{{\suttatitleacronym SN 30.1}{\suttatitletranslation Plain Version }{\suttatitleroot Suddhikasutta}}
\addcontentsline{toc}{section}{\tocacronym{SN 30.1} \toctranslation{Plain Version } \tocroot{Suddhikasutta}}
\markboth{Plain Version }{Suddhikasutta}
\extramarks{SN 30.1}{SN 30.1}

At\marginnote{1.1} \textsanskrit{Sāvatthī}. 

“Mendicants,\marginnote{1.2} phoenixes reproduce in these four ways. What four? Phoenixes are born from eggs, from a womb, from moisture, or spontaneously. These are the four ways that phoenixes reproduce.” 

%
\section*{{\suttatitleacronym SN 30.2}{\suttatitletranslation They Carry Off }{\suttatitleroot Harantisutta}}
\addcontentsline{toc}{section}{\tocacronym{SN 30.2} \toctranslation{They Carry Off } \tocroot{Harantisutta}}
\markboth{They Carry Off }{Harantisutta}
\extramarks{SN 30.2}{SN 30.2}

At\marginnote{1.1} \textsanskrit{Sāvatthī}. 

“Mendicants,\marginnote{1.2} phoenixes reproduce in these four ways. What four? Phoenixes are born from eggs, from a womb, from moisture, or spontaneously. These are the four ways that phoenixes reproduce. Of these, phoenixes born from an egg can only carry off dragons born from an egg, not those born from a womb, from moisture, or spontaneously. Phoenixes born from a womb can carry off dragons born from an egg or from a womb, but not those born from moisture or spontaneously. Phoenixes born from moisture can carry off dragons born from an egg, from a womb, or from moisture, but not those born spontaneously. Phoenixes born spontaneously can carry off dragons born from an egg, from a womb, from moisture, or spontaneously. These are the four ways that phoenixes reproduce.” 

%
\section*{{\suttatitleacronym SN 30.3}{\suttatitletranslation Both Kinds of Deeds }{\suttatitleroot Dvayakārīsutta}}
\addcontentsline{toc}{section}{\tocacronym{SN 30.3} \toctranslation{Both Kinds of Deeds } \tocroot{Dvayakārīsutta}}
\markboth{Both Kinds of Deeds }{Dvayakārīsutta}
\extramarks{SN 30.3}{SN 30.3}

At\marginnote{1.1} \textsanskrit{Sāvatthī}. 

Then\marginnote{1.2} a mendicant went up to the Buddha, bowed, sat down to one side, and said to him: 

“Sir,\marginnote{1.3} what is the cause, what is the reason why someone, when their body breaks up, after death, is reborn in the company of the egg-born phoenixes?” 

“Mendicant,\marginnote{1.4} it’s when someone does both kinds of deeds by body, speech, and mind. And they’ve heard: ‘The egg-born phoenixes are long-lived, beautiful, and very happy.’ They think: ‘If only, when my body breaks up, after death, I would be reborn in the company of the egg-born phoenixes!’ When their body breaks up, after death, they’re reborn in the company of the egg-born phoenixes. This is the cause, this is the reason why someone, when their body breaks up, after death, is reborn in the company of the egg-born phoenixes.” 

%
\section*{{\suttatitleacronym SN 30.4–6}{\suttatitletranslation Both Kinds of Deeds (2nd–4th) }{\suttatitleroot Dutiyādidvayakārīsuttattika}}
\addcontentsline{toc}{section}{\tocacronym{SN 30.4–6} \toctranslation{Both Kinds of Deeds (2nd–4th) } \tocroot{Dutiyādidvayakārīsuttattika}}
\markboth{Both Kinds of Deeds (2nd–4th) }{Dutiyādidvayakārīsuttattika}
\extramarks{SN 30.4–6}{SN 30.4–6}

At\marginnote{1.1} \textsanskrit{Sāvatthī}. 

Seated\marginnote{1.2} to one side, that mendicant said to the Buddha: 

“Sir,\marginnote{1.3} what is the cause, what is the reason why someone, when their body breaks up, after death, is reborn in the company of the womb-born phoenixes … moisture-born phoenixes … or spontaneously-born phoenixes?” 

(All\marginnote{1.6} should be told in full.) 

%
\section*{{\suttatitleacronym SN 30.7–16}{\suttatitletranslation Ten Discourses On How Giving Helps to Become Egg-Born }{\suttatitleroot Aṇḍajadānūpakārasuttadasaka}}
\addcontentsline{toc}{section}{\tocacronym{SN 30.7–16} \toctranslation{Ten Discourses On How Giving Helps to Become Egg-Born } \tocroot{Aṇḍajadānūpakārasuttadasaka}}
\markboth{Ten Discourses On How Giving Helps to Become Egg-Born }{Aṇḍajadānūpakārasuttadasaka}
\extramarks{SN 30.7–16}{SN 30.7–16}

At\marginnote{1.1} \textsanskrit{Sāvatthī}. 

Seated\marginnote{1.2} to one side, that mendicant said to the Buddha: 

“Sir,\marginnote{1.3} what is the cause, what is the reason why someone, when their body breaks up, after death, is reborn in the company of the egg-born phoenixes?” 

“Mendicant,\marginnote{1.4} it’s when someone does both kinds of deeds by body, speech, and mind. And they’ve heard: ‘The egg-born phoenixes are long-lived, beautiful, and very happy.’ They think: ‘If only, when my body breaks up, after death, I would be reborn in the company of the egg-born phoenixes!’ They give food … drink … clothing … a vehicle … a garland … fragrance … makeup … a bed … a house … a lamp. When their body breaks up, after death, they’re reborn in the company of the egg-born phoenixes. This is the cause, this is the reason why someone, when their body breaks up, after death, is reborn in the company of the egg-born phoenixes.” 

%
\section*{{\suttatitleacronym SN 30.17–46}{\suttatitletranslation How Giving Helps to Become Womb-Born, Etc. }{\suttatitleroot Jalābujadānūpakārasutta}}
\addcontentsline{toc}{section}{\tocacronym{SN 30.17–46} \toctranslation{How Giving Helps to Become Womb-Born, Etc. } \tocroot{Jalābujadānūpakārasutta}}
\markboth{How Giving Helps to Become Womb-Born, Etc. }{Jalābujadānūpakārasutta}
\extramarks{SN 30.17–46}{SN 30.17–46}

At\marginnote{1.1} \textsanskrit{Sāvatthī}. 

Seated\marginnote{1.2} to one side, that mendicant said to the Buddha: 

“Sir,\marginnote{1.3} what is the cause, what is the reason why someone, when their body breaks up, after death, is reborn in the company of the womb-born phoenixes … moisture-born phoenixes … or spontaneously-born phoenixes?” 

(All\marginnote{1.6} should be told in full.) 

\scendsutta{The Linked Discourses on phoenixes are complete. }

%
\addtocontents{toc}{\let\protect\contentsline\protect\nopagecontentsline}
\part*{Linked Discourses on Fairies }
\addcontentsline{toc}{part}{Linked Discourses on Fairies }
\markboth{}{}
\addtocontents{toc}{\let\protect\contentsline\protect\oldcontentsline}

%
\addtocontents{toc}{\let\protect\contentsline\protect\nopagecontentsline}
\chapter*{The Chapter on Fairies }
\addcontentsline{toc}{chapter}{\tocchapterline{The Chapter on Fairies }}
\addtocontents{toc}{\let\protect\contentsline\protect\oldcontentsline}

%
\section*{{\suttatitleacronym SN 31.1}{\suttatitletranslation Plain Version }{\suttatitleroot Suddhikasutta}}
\addcontentsline{toc}{section}{\tocacronym{SN 31.1} \toctranslation{Plain Version } \tocroot{Suddhikasutta}}
\markboth{Plain Version }{Suddhikasutta}
\extramarks{SN 31.1}{SN 31.1}

At\marginnote{1.1} one time the Buddha was staying near \textsanskrit{Sāvatthī} in Jeta’s Grove, \textsanskrit{Anāthapiṇḍika}’s monastery. … The Buddha said this: 

“Mendicants,\marginnote{1.3} I will teach you about the gods of fairykind. Listen … 

And\marginnote{1.5} what are the gods of fairykind? There are gods who live in fragrant roots, fragrant heartwood, fragrant softwood, fragrant bark, fragrant shoots, fragrant leaves, fragrant flowers, fragrant fruit, fragrant sap, and fragrant scents. These are called the gods of fairykind.” 

%
\section*{{\suttatitleacronym SN 31.2}{\suttatitletranslation Good Conduct }{\suttatitleroot Sucaritasutta}}
\addcontentsline{toc}{section}{\tocacronym{SN 31.2} \toctranslation{Good Conduct } \tocroot{Sucaritasutta}}
\markboth{Good Conduct }{Sucaritasutta}
\extramarks{SN 31.2}{SN 31.2}

At\marginnote{1.1} \textsanskrit{Sāvatthī}. 

Seated\marginnote{1.2} to one side, that mendicant said to the Buddha: 

“Sir,\marginnote{1.3} what is the cause, what is the reason why someone, when their body breaks up, after death, is reborn in the company of the gods of fairykind?” 

“Mendicant,\marginnote{1.4} it’s when someone does good things by way of body, speech, and mind. And they’ve heard: ‘The gods of fairykind are long-lived, beautiful, and very happy.’ They think: ‘If only, when my body breaks up, after death, I would be reborn in the company of the gods of fairykind!’ When their body breaks up, after death, they’re reborn in the company of the gods of fairykind. This is the cause, this is the reason why someone, when their body breaks up, after death, is reborn in the company of the gods of fairykind.” 

%
\section*{{\suttatitleacronym SN 31.3}{\suttatitletranslation A Giver of Fragrant Roots }{\suttatitleroot Mūlagandhadātāsutta}}
\addcontentsline{toc}{section}{\tocacronym{SN 31.3} \toctranslation{A Giver of Fragrant Roots } \tocroot{Mūlagandhadātāsutta}}
\markboth{A Giver of Fragrant Roots }{Mūlagandhadātāsutta}
\extramarks{SN 31.3}{SN 31.3}

At\marginnote{1.1} \textsanskrit{Sāvatthī}. 

Seated\marginnote{1.2} to one side, that mendicant said to the Buddha: 

“Sir,\marginnote{1.3} what is the cause, what is the reason why someone, when their body breaks up, after death, is reborn in the company of the gods of fairykind who live in fragrant roots?” 

“Mendicant,\marginnote{1.4} it’s when someone does good things by way of body, speech, and mind. And they’ve heard: ‘The gods of fairykind who live in fragrant roots are long-lived, beautiful, and very happy.’ They think: ‘If only, when my body breaks up, after death, I would be reborn in the company of the gods of fairykind who live in fragrant roots!’ They give gifts of fragrant roots. When their body breaks up, after death, they’re reborn in the company of the gods of fairykind who live in fragrant roots. This is the cause, this is the reason why someone, when their body breaks up, after death, is reborn in the company of the gods of fairykind who live in fragrant roots.” 

%
\section*{{\suttatitleacronym SN 31.4–12}{\suttatitletranslation Nine Discourses On Givers of Fragrant Heartwood, Etc. }{\suttatitleroot Sāragandhādidātāsuttanavaka}}
\addcontentsline{toc}{section}{\tocacronym{SN 31.4–12} \toctranslation{Nine Discourses On Givers of Fragrant Heartwood, Etc. } \tocroot{Sāragandhādidātāsuttanavaka}}
\markboth{Nine Discourses On Givers of Fragrant Heartwood, Etc. }{Sāragandhādidātāsuttanavaka}
\extramarks{SN 31.4–12}{SN 31.4–12}

At\marginnote{1.1} \textsanskrit{Sāvatthī}. 

Seated\marginnote{1.2} to one side, that mendicant said to the Buddha: 

“Sir,\marginnote{1.3} what is the cause, what is the reason why someone, when their body breaks up, after death, is reborn in the company of the gods of fairykind who live in fragrant heartwood … softwood … bark … sprouts … leaves … flowers … fruit … sap … fragrant scents?” 

“Mendicant,\marginnote{1.12} it’s when someone does good things by way of body, speech, and mind. And they’ve heard: ‘The gods of fairykind who live in fragrant heartwood … fragrant scents are long-lived, beautiful, and very happy.’ They think: ‘If only, when my body breaks up, after death, I would be reborn in the company of the gods of fairykind who live in fragrant heartwood … fragrant scents!’ They give gifts of fragrant heartwood … fragrant scents. When their body breaks up, after death, they’re reborn in the company of the gods of fairykind who live in fragrant scents. This is the cause, this is the reason why someone, when their body breaks up, after death, is reborn in the company of the gods of fairykind who live on fragrant scents.” 

%
\section*{{\suttatitleacronym SN 31.13–22}{\suttatitletranslation Ten Discourses On How Giving Helps to Become a Fragrant Root Fairy }{\suttatitleroot Mūlagandhadānūpakārasuttadasaka}}
\addcontentsline{toc}{section}{\tocacronym{SN 31.13–22} \toctranslation{Ten Discourses On How Giving Helps to Become a Fragrant Root Fairy } \tocroot{Mūlagandhadānūpakārasuttadasaka}}
\markboth{Ten Discourses On How Giving Helps to Become a Fragrant Root Fairy }{Mūlagandhadānūpakārasuttadasaka}
\extramarks{SN 31.13–22}{SN 31.13–22}

At\marginnote{1.1} \textsanskrit{Sāvatthī}. 

Seated\marginnote{1.2} to one side, that mendicant said to the Buddha: 

“Sir,\marginnote{1.3} what is the cause, what is the reason why someone, when their body breaks up, after death, is reborn in the company of the gods of fairykind who live in fragrant roots?” 

“Mendicant,\marginnote{1.4} it’s when someone does good things by way of body, speech, and mind. And they’ve heard: ‘The gods of fairykind who live in fragrant roots are long-lived, beautiful, and very happy.’ They think: ‘If only, when my body breaks up, after death, I would be reborn in the company of the gods of fairykind who live in fragrant roots!’ They give food … drink … clothing … a vehicle … a garland … fragrance … makeup … a bed … a house … a lamp. When their body breaks up, after death, they’re reborn in the company of the gods of fairykind who live in fragrant roots. This is the cause, this is the reason why someone, when their body breaks up, after death, is reborn in the company of the gods of fairykind who live on fragrant scents.” 

%
\section*{{\suttatitleacronym SN 31.23–112}{\suttatitletranslation Ninety Discourses On How Giving Helps to Become a Fragrant Heartwood Fairy }{\suttatitleroot Sāragandhādidānūpakārasuttanavutika}}
\addcontentsline{toc}{section}{\tocacronym{SN 31.23–112} \toctranslation{Ninety Discourses On How Giving Helps to Become a Fragrant Heartwood Fairy } \tocroot{Sāragandhādidānūpakārasuttanavutika}}
\markboth{Ninety Discourses On How Giving Helps to Become a Fragrant Heartwood Fairy }{Sāragandhādidānūpakārasuttanavutika}
\extramarks{SN 31.23–112}{SN 31.23–112}

At\marginnote{1.1} \textsanskrit{Sāvatthī}. 

Seated\marginnote{1.2} to one side, that mendicant said to the Buddha: 

“Sir,\marginnote{1.3} what is the cause, what is the reason why someone, when their body breaks up, after death, is reborn in the company of the gods of fairykind who live in fragrant heartwood … softwood … bark … sprouts … leaves … flowers … fruit … sap … fragrant scents?” 

“Mendicant,\marginnote{1.12} it’s when someone does good things by way of body, speech, and mind. And they’ve heard: ‘The gods of fairykind who live in fragrant scents are long-lived, beautiful, and very happy.’ They think: ‘If only, when my body breaks up, after death, I would be reborn in the company of the gods of fairykind who live in fragrant scents!’ They give food … drink … clothing … a vehicle … a garland … fragrance … makeup … a bed … a house … a lamp. When their body breaks up, after death, they’re reborn in the company of the gods of fairykind who live in fragrant scents. This is the cause, this is the reason why someone, when their body breaks up, after death, is reborn in the company of the gods of fairykind who live on fragrant scents.” 

\scendsutta{The Linked Discourses on fairykind are completed. }

%
\addtocontents{toc}{\let\protect\contentsline\protect\nopagecontentsline}
\part*{Linked Discourses on Cloud Gods }
\addcontentsline{toc}{part}{Linked Discourses on Cloud Gods }
\markboth{}{}
\addtocontents{toc}{\let\protect\contentsline\protect\oldcontentsline}

%
\addtocontents{toc}{\let\protect\contentsline\protect\nopagecontentsline}
\chapter*{The Chapter on Gods of the Clouds }
\addcontentsline{toc}{chapter}{\tocchapterline{The Chapter on Gods of the Clouds }}
\addtocontents{toc}{\let\protect\contentsline\protect\oldcontentsline}

%
\section*{{\suttatitleacronym SN 32.1}{\suttatitletranslation Plain Version }{\suttatitleroot Suddhikasutta}}
\addcontentsline{toc}{section}{\tocacronym{SN 32.1} \toctranslation{Plain Version } \tocroot{Suddhikasutta}}
\markboth{Plain Version }{Suddhikasutta}
\extramarks{SN 32.1}{SN 32.1}

At\marginnote{1.1} \textsanskrit{Sāvatthī}. 

“Mendicants,\marginnote{1.2} I will teach you about the gods of the clouds. Listen … 

And\marginnote{1.4} what are the gods of the clouds? There are gods of the cool clouds, warm clouds, thunder clouds, windy clouds, and rainy clouds. These are called the gods of the clouds.” 

%
\section*{{\suttatitleacronym SN 32.2}{\suttatitletranslation Good Conduct }{\suttatitleroot Sucaritasutta}}
\addcontentsline{toc}{section}{\tocacronym{SN 32.2} \toctranslation{Good Conduct } \tocroot{Sucaritasutta}}
\markboth{Good Conduct }{Sucaritasutta}
\extramarks{SN 32.2}{SN 32.2}

At\marginnote{1.1} \textsanskrit{Sāvatthī}. 

Seated\marginnote{1.2} to one side, that mendicant said to the Buddha: 

“Sir,\marginnote{1.3} what is the cause, what is the reason why someone, when their body breaks up, after death, is reborn in the company of the gods of the clouds?” 

“Mendicant,\marginnote{1.4} it’s when someone does good things by way of body, speech, and mind. And they’ve heard: ‘The gods of the clouds are long-lived, beautiful, and very happy.’ They think: ‘If only, when my body breaks up, after death, I would be reborn in the company of the gods of the clouds!’ When their body breaks up, after death, they’re reborn in the company of the gods of the clouds. This is the cause, this is the reason why someone, when their body breaks up, after death, is reborn in the company of the gods of the clouds.” 

%
\section*{{\suttatitleacronym SN 32.3–12}{\suttatitletranslation Ten Discourses On How Giving Helps to Become a Cool Cloud God }{\suttatitleroot Sītavalāhakadānūpakārasuttadasaka}}
\addcontentsline{toc}{section}{\tocacronym{SN 32.3–12} \toctranslation{Ten Discourses On How Giving Helps to Become a Cool Cloud God } \tocroot{Sītavalāhakadānūpakārasuttadasaka}}
\markboth{Ten Discourses On How Giving Helps to Become a Cool Cloud God }{Sītavalāhakadānūpakārasuttadasaka}
\extramarks{SN 32.3–12}{SN 32.3–12}

At\marginnote{1.1} \textsanskrit{Sāvatthī}. 

Seated\marginnote{1.2} to one side, that mendicant said to the Buddha: 

“Sir,\marginnote{1.3} what is the cause, what is the reason why someone, when their body breaks up, after death, is reborn in the company of the gods of cool clouds?” 

“Mendicant,\marginnote{1.4} it’s when someone does good things by way of body, speech, and mind. And they’ve heard: ‘The gods of cool clouds are long-lived, beautiful, and very happy.’ They think: ‘If only, when my body breaks up, after death, I would be reborn in the company of the gods of the cool clouds!’ They give food … a lamp. When their body breaks up, after death, they’re reborn in the company of the gods of cool clouds. This is the cause, this is the reason why someone, when their body breaks up, after death, is reborn in the company of the gods of cool clouds.” 

%
\section*{{\suttatitleacronym SN 32.13–52}{\suttatitletranslation How Giving Helps to Become a Warm Cloud God, Etc. }{\suttatitleroot Uṇhavalāhakadānūpakārasutta}}
\addcontentsline{toc}{section}{\tocacronym{SN 32.13–52} \toctranslation{How Giving Helps to Become a Warm Cloud God, Etc. } \tocroot{Uṇhavalāhakadānūpakārasutta}}
\markboth{How Giving Helps to Become a Warm Cloud God, Etc. }{Uṇhavalāhakadānūpakārasutta}
\extramarks{SN 32.13–52}{SN 32.13–52}

At\marginnote{1.1} \textsanskrit{Sāvatthī}. 

Seated\marginnote{1.2} to one side, that mendicant said to the Buddha: 

“Sir,\marginnote{1.3} what is the cause, what is the reason why someone, when their body breaks up, after death, is reborn in the company of the gods of warm clouds … thunder clouds … windy clouds … rainy clouds?” 

“Mendicant,\marginnote{1.7} it’s when someone does good things by way of body, speech, and mind. And they’ve heard: ‘The gods of rainy clouds are long-lived, beautiful, and very happy.’ They think: ‘If only, when my body breaks up, after death, I would be reborn in the company of the gods of rainy clouds!’ They give food … a lamp. When their body breaks up, after death, they’re reborn in the company of the gods of rainy clouds. This is the cause, this is the reason why someone, when their body breaks up, after death, is reborn in the company of the gods of rainy clouds.” 

%
\section*{{\suttatitleacronym SN 32.53}{\suttatitletranslation Gods of the Cool Clouds }{\suttatitleroot Sītavalāhakasutta}}
\addcontentsline{toc}{section}{\tocacronym{SN 32.53} \toctranslation{Gods of the Cool Clouds } \tocroot{Sītavalāhakasutta}}
\markboth{Gods of the Cool Clouds }{Sītavalāhakasutta}
\extramarks{SN 32.53}{SN 32.53}

At\marginnote{1.1} \textsanskrit{Sāvatthī}. 

Seated\marginnote{1.2} to one side, that mendicant said to the Buddha: 

“Sir,\marginnote{1.3} what is the cause, what is the reason why sometimes it becomes cool?” 

“Mendicant,\marginnote{1.4} there are what are called gods of the cool clouds. Sometimes they think: ‘Why don’t we revel in our own kind of enjoyment?’ Then, in accordance with their wish, it becomes cool. This is the cause, this is the reason why sometimes it becomes cool.” 

%
\section*{{\suttatitleacronym SN 32.54}{\suttatitletranslation Gods of the Warm Clouds }{\suttatitleroot Uṇhavalāhakasutta}}
\addcontentsline{toc}{section}{\tocacronym{SN 32.54} \toctranslation{Gods of the Warm Clouds } \tocroot{Uṇhavalāhakasutta}}
\markboth{Gods of the Warm Clouds }{Uṇhavalāhakasutta}
\extramarks{SN 32.54}{SN 32.54}

At\marginnote{1.1} \textsanskrit{Sāvatthī}. 

Seated\marginnote{1.2} to one side, that mendicant said to the Buddha: 

“Sir,\marginnote{1.3} what is the cause, what is the reason why sometimes it becomes warm?” 

“Mendicant,\marginnote{1.4} there are what are called gods of the warm clouds. Sometimes they think: ‘Why don’t we revel in our own kind of enjoyment?’ Then, in accordance with their wish, it becomes warm. This is the cause, this is the reason why sometimes it becomes warm.” 

%
\section*{{\suttatitleacronym SN 32.55}{\suttatitletranslation Gods of the Storm Clouds }{\suttatitleroot Abbhavalāhakasutta}}
\addcontentsline{toc}{section}{\tocacronym{SN 32.55} \toctranslation{Gods of the Storm Clouds } \tocroot{Abbhavalāhakasutta}}
\markboth{Gods of the Storm Clouds }{Abbhavalāhakasutta}
\extramarks{SN 32.55}{SN 32.55}

At\marginnote{1.1} \textsanskrit{Sāvatthī}. 

Seated\marginnote{1.2} to one side, that mendicant said to the Buddha: 

“Sir,\marginnote{1.3} what is the cause, what is the reason why sometimes it becomes stormy?” 

“Mendicant,\marginnote{1.4} there are what are called gods of the storm clouds. Sometimes they think: ‘Why don’t we revel in our own kind of enjoyment?’ Then, in accordance with their wish, it becomes stormy. This is the cause, this is the reason why sometimes it becomes stormy.” 

%
\section*{{\suttatitleacronym SN 32.56}{\suttatitletranslation Gods of the Windy Clouds }{\suttatitleroot Vātavalāhakasutta}}
\addcontentsline{toc}{section}{\tocacronym{SN 32.56} \toctranslation{Gods of the Windy Clouds } \tocroot{Vātavalāhakasutta}}
\markboth{Gods of the Windy Clouds }{Vātavalāhakasutta}
\extramarks{SN 32.56}{SN 32.56}

At\marginnote{1.1} \textsanskrit{Sāvatthī}. 

Seated\marginnote{1.2} to one side, that mendicant said to the Buddha: 

“Sir,\marginnote{1.3} what is the cause, what is the reason why sometimes it becomes windy?” 

“Mendicant,\marginnote{1.4} there are what are called gods of the windy clouds. Sometimes they think: ‘Why don’t we revel in our own kind of enjoyment?’ Then, in accordance with their wish, it becomes windy. This is the cause, this is the reason why sometimes it becomes windy.” 

%
\section*{{\suttatitleacronym SN 32.57}{\suttatitletranslation Gods of the Rainy Clouds }{\suttatitleroot Vassavalāhakasutta}}
\addcontentsline{toc}{section}{\tocacronym{SN 32.57} \toctranslation{Gods of the Rainy Clouds } \tocroot{Vassavalāhakasutta}}
\markboth{Gods of the Rainy Clouds }{Vassavalāhakasutta}
\extramarks{SN 32.57}{SN 32.57}

At\marginnote{1.1} \textsanskrit{Sāvatthī}. 

Seated\marginnote{1.2} to one side, that mendicant said to the Buddha: 

“Sir,\marginnote{1.3} what is the cause, what is the reason why sometimes it rains?” 

“Mendicant,\marginnote{1.4} there are what are called gods of the rainy clouds. Sometimes they think: ‘Why don’t we revel in our own kind of enjoyment?’ Then, in accordance with their wish, it becomes rainy. This is the cause, this is the reason why sometimes it rains.” 

\scendsutta{The Linked Discourses on gods of the clouds are complete. }

%
\addtocontents{toc}{\let\protect\contentsline\protect\nopagecontentsline}
\part*{Linked Discourses with Vacchagotta }
\addcontentsline{toc}{part}{Linked Discourses with Vacchagotta }
\markboth{}{}
\addtocontents{toc}{\let\protect\contentsline\protect\oldcontentsline}

%
\addtocontents{toc}{\let\protect\contentsline\protect\nopagecontentsline}
\chapter*{The Chapter with Vacchagotta }
\addcontentsline{toc}{chapter}{\tocchapterline{The Chapter with Vacchagotta }}
\addtocontents{toc}{\let\protect\contentsline\protect\oldcontentsline}

%
\section*{{\suttatitleacronym SN 33.1}{\suttatitletranslation Not Knowing Form }{\suttatitleroot Rūpaaññāṇasutta}}
\addcontentsline{toc}{section}{\tocacronym{SN 33.1} \toctranslation{Not Knowing Form } \tocroot{Rūpaaññāṇasutta}}
\markboth{Not Knowing Form }{Rūpaaññāṇasutta}
\extramarks{SN 33.1}{SN 33.1}

At\marginnote{1.1} one time the Buddha was staying near \textsanskrit{Sāvatthī} in Jeta’s Grove, \textsanskrit{Anāthapiṇḍika}’s monastery. Then the wanderer Vacchagotta went up to the Buddha and exchanged greetings with him. When the greetings and polite conversation were over, he sat down to one side and said to the Buddha: 

“What\marginnote{1.4} is the cause, Master Gotama, what is the reason why these various misconceptions arise in the world? That is: the cosmos is eternal, or not eternal, or finite, or infinite; the soul and the body are the same thing, or they are different things; after death, a Realized One exists, or doesn’t exist, or both exists and doesn’t exist, or neither exists nor doesn’t exist.” 

“Vaccha,\marginnote{1.6} it is because of not knowing form, its origin, its cessation, and the practice that leads to its cessation that these various misconceptions arise in the world. This is the cause, this is the reason.” 

%
\section*{{\suttatitleacronym SN 33.2}{\suttatitletranslation Not Knowing Feeling }{\suttatitleroot Vedanāaññāṇasutta}}
\addcontentsline{toc}{section}{\tocacronym{SN 33.2} \toctranslation{Not Knowing Feeling } \tocroot{Vedanāaññāṇasutta}}
\markboth{Not Knowing Feeling }{Vedanāaññāṇasutta}
\extramarks{SN 33.2}{SN 33.2}

At\marginnote{1.1} \textsanskrit{Sāvatthī}. 

Then\marginnote{1.2} the wanderer Vacchagotta said to the Buddha: 

“What\marginnote{1.3} is the cause, Master Gotama, what is the reason why these various misconceptions arise in the world? That is: the cosmos is eternal, or not eternal … after death, a Realized One neither exists nor doesn’t exist.” 

“Vaccha,\marginnote{1.6} it is because of not knowing feeling, its origin, its cessation, and the practice that leads to its cessation that these various misconceptions arise in the world. This is the cause, this is the reason.” 

%
\section*{{\suttatitleacronym SN 33.3}{\suttatitletranslation Not Knowing Perception }{\suttatitleroot Saññāaññāṇasutta}}
\addcontentsline{toc}{section}{\tocacronym{SN 33.3} \toctranslation{Not Knowing Perception } \tocroot{Saññāaññāṇasutta}}
\markboth{Not Knowing Perception }{Saññāaññāṇasutta}
\extramarks{SN 33.3}{SN 33.3}

At\marginnote{1.1} \textsanskrit{Sāvatthī}. 

Then\marginnote{1.2} the wanderer Vacchagotta said to the Buddha: 

“What\marginnote{1.3} is the cause, Master Gotama, what is the reason why these various misconceptions arise in the world? …” 

“Vaccha,\marginnote{1.6} it is because of not knowing perception, its origin, its cessation, and the practice that leads to its cessation …” 

%
\section*{{\suttatitleacronym SN 33.4}{\suttatitletranslation Not Knowing Choices }{\suttatitleroot Saṅkhāraaññāṇasutta}}
\addcontentsline{toc}{section}{\tocacronym{SN 33.4} \toctranslation{Not Knowing Choices } \tocroot{Saṅkhāraaññāṇasutta}}
\markboth{Not Knowing Choices }{Saṅkhāraaññāṇasutta}
\extramarks{SN 33.4}{SN 33.4}

At\marginnote{1.1} \textsanskrit{Sāvatthī}. 

Then\marginnote{1.2} the wanderer Vacchagotta said to the Buddha: 

“What\marginnote{1.3} is the cause, Master Gotama, what is the reason why these various misconceptions arise in the world? …” 

“Vaccha,\marginnote{1.6} it is because of not knowing choices, their origin, their cessation, and the practice that leads to their cessation …” 

%
\section*{{\suttatitleacronym SN 33.5}{\suttatitletranslation Not Knowing Consciousness }{\suttatitleroot Viññāṇaaññāṇasutta}}
\addcontentsline{toc}{section}{\tocacronym{SN 33.5} \toctranslation{Not Knowing Consciousness } \tocroot{Viññāṇaaññāṇasutta}}
\markboth{Not Knowing Consciousness }{Viññāṇaaññāṇasutta}
\extramarks{SN 33.5}{SN 33.5}

At\marginnote{1.1} \textsanskrit{Sāvatthī}. 

Then\marginnote{1.2} the wanderer Vacchagotta said to the Buddha: 

“What\marginnote{1.3} is the cause, Master Gotama, what is the reason why these various misconceptions arise in the world? …” 

“Vaccha,\marginnote{1.6} it is because of not knowing consciousness, its origin, its cessation, and the practice that leads to its cessation …” 

%
\section*{{\suttatitleacronym SN 33.6–10}{\suttatitletranslation Five Discourses on Not Seeing Form, Etc. }{\suttatitleroot Rūpaadassanādisuttapañcaka}}
\addcontentsline{toc}{section}{\tocacronym{SN 33.6–10} \toctranslation{Five Discourses on Not Seeing Form, Etc. } \tocroot{Rūpaadassanādisuttapañcaka}}
\markboth{Five Discourses on Not Seeing Form, Etc. }{Rūpaadassanādisuttapañcaka}
\extramarks{SN 33.6–10}{SN 33.6–10}

At\marginnote{1.1} \textsanskrit{Sāvatthī}. 

Then\marginnote{1.2} the wanderer Vacchagotta said to the Buddha: 

“What\marginnote{1.3} is the cause, Master Gotama, what is the reason why these various misconceptions arise in the world? …” 

“Vaccha,\marginnote{1.6} it is because of not seeing form … feeling … perception … choices … consciousness, its origin, its cessation, and the practice that leads to its cessation …” 

%
\section*{{\suttatitleacronym SN 33.11–15}{\suttatitletranslation Five Discourses on Not Comprehending Form, Etc. }{\suttatitleroot Rūpaanabhisamayādisuttapañcaka}}
\addcontentsline{toc}{section}{\tocacronym{SN 33.11–15} \toctranslation{Five Discourses on Not Comprehending Form, Etc. } \tocroot{Rūpaanabhisamayādisuttapañcaka}}
\markboth{Five Discourses on Not Comprehending Form, Etc. }{Rūpaanabhisamayādisuttapañcaka}
\extramarks{SN 33.11–15}{SN 33.11–15}

At\marginnote{1.1} \textsanskrit{Sāvatthī}. 

“Vaccha,\marginnote{1.2} it is because of not comprehending form … 

feeling\marginnote{1.1} … 

perception\marginnote{1.1} … 

choices\marginnote{1.1} … 

consciousness\marginnote{1.1} …” 

%
\section*{{\suttatitleacronym SN 33.16–20}{\suttatitletranslation Five Discourses on Not Understanding Form, Etc. }{\suttatitleroot Rūpaananubodhādisuttapañcaka}}
\addcontentsline{toc}{section}{\tocacronym{SN 33.16–20} \toctranslation{Five Discourses on Not Understanding Form, Etc. } \tocroot{Rūpaananubodhādisuttapañcaka}}
\markboth{Five Discourses on Not Understanding Form, Etc. }{Rūpaananubodhādisuttapañcaka}
\extramarks{SN 33.16–20}{SN 33.16–20}

At\marginnote{1.1} \textsanskrit{Sāvatthī}. 

“Vaccha,\marginnote{1.2} it is because of not understanding form … 

feeling\marginnote{1.1} … 

perception\marginnote{1.1} … 

choices\marginnote{1.1} … 

consciousness\marginnote{1.1} …” 

%
\section*{{\suttatitleacronym SN 33.21–25}{\suttatitletranslation Five Discourses on Not Penetrating Form, Etc. }{\suttatitleroot Rūpaappaṭivedhādisuttapañcaka}}
\addcontentsline{toc}{section}{\tocacronym{SN 33.21–25} \toctranslation{Five Discourses on Not Penetrating Form, Etc. } \tocroot{Rūpaappaṭivedhādisuttapañcaka}}
\markboth{Five Discourses on Not Penetrating Form, Etc. }{Rūpaappaṭivedhādisuttapañcaka}
\extramarks{SN 33.21–25}{SN 33.21–25}

At\marginnote{1.1} \textsanskrit{Sāvatthī}. 

“Vaccha,\marginnote{1.2} it is because of not penetrating form …” 

%
\section*{{\suttatitleacronym SN 33.26–30}{\suttatitletranslation Five Discourses on Not Distinguishing Form, Etc. }{\suttatitleroot Rūpaasallakkhaṇādisuttapañcaka}}
\addcontentsline{toc}{section}{\tocacronym{SN 33.26–30} \toctranslation{Five Discourses on Not Distinguishing Form, Etc. } \tocroot{Rūpaasallakkhaṇādisuttapañcaka}}
\markboth{Five Discourses on Not Distinguishing Form, Etc. }{Rūpaasallakkhaṇādisuttapañcaka}
\extramarks{SN 33.26–30}{SN 33.26–30}

At\marginnote{1.1} \textsanskrit{Sāvatthī}. 

“Vaccha,\marginnote{1.2} it is because of not distinguishing form …” 

%
\section*{{\suttatitleacronym SN 33.31–35}{\suttatitletranslation Five Discourses on Not Detecting Form, Etc. }{\suttatitleroot Rūpaanupalakkhaṇādisuttapañcaka}}
\addcontentsline{toc}{section}{\tocacronym{SN 33.31–35} \toctranslation{Five Discourses on Not Detecting Form, Etc. } \tocroot{Rūpaanupalakkhaṇādisuttapañcaka}}
\markboth{Five Discourses on Not Detecting Form, Etc. }{Rūpaanupalakkhaṇādisuttapañcaka}
\extramarks{SN 33.31–35}{SN 33.31–35}

At\marginnote{1.1} \textsanskrit{Sāvatthī}. 

“Vaccha,\marginnote{1.2} it is because of not detecting form …” 

%
\section*{{\suttatitleacronym SN 33.36–40}{\suttatitletranslation Five Discourses on Not Differentiating Form, Etc. }{\suttatitleroot Rūpaappaccupalakkhaṇādisuttapañcaka}}
\addcontentsline{toc}{section}{\tocacronym{SN 33.36–40} \toctranslation{Five Discourses on Not Differentiating Form, Etc. } \tocroot{Rūpaappaccupalakkhaṇādisuttapañcaka}}
\markboth{Five Discourses on Not Differentiating Form, Etc. }{Rūpaappaccupalakkhaṇādisuttapañcaka}
\extramarks{SN 33.36–40}{SN 33.36–40}

At\marginnote{1.1} \textsanskrit{Sāvatthī}. 

“Vaccha,\marginnote{1.2} it is because of not differentiating form …” 

%
\section*{{\suttatitleacronym SN 33.41–45}{\suttatitletranslation Five Discourses on Not Examining Form, Etc. }{\suttatitleroot Rūpaasamapekkhaṇādisuttapañcaka}}
\addcontentsline{toc}{section}{\tocacronym{SN 33.41–45} \toctranslation{Five Discourses on Not Examining Form, Etc. } \tocroot{Rūpaasamapekkhaṇādisuttapañcaka}}
\markboth{Five Discourses on Not Examining Form, Etc. }{Rūpaasamapekkhaṇādisuttapañcaka}
\extramarks{SN 33.41–45}{SN 33.41–45}

At\marginnote{1.1} \textsanskrit{Sāvatthī}. 

“Vaccha,\marginnote{1.2} it is because of not examining form …” 

%
\section*{{\suttatitleacronym SN 33.46–50}{\suttatitletranslation Five Discourses on Not Scrutinizing Form, Etc. }{\suttatitleroot Rūpaappaccupekkhaṇādisuttapañcaka}}
\addcontentsline{toc}{section}{\tocacronym{SN 33.46–50} \toctranslation{Five Discourses on Not Scrutinizing Form, Etc. } \tocroot{Rūpaappaccupekkhaṇādisuttapañcaka}}
\markboth{Five Discourses on Not Scrutinizing Form, Etc. }{Rūpaappaccupekkhaṇādisuttapañcaka}
\extramarks{SN 33.46–50}{SN 33.46–50}

At\marginnote{1.1} \textsanskrit{Sāvatthī}. 

“Vaccha,\marginnote{1.2} it is because of not scrutinizing form …” 

%
\section*{{\suttatitleacronym SN 33.51–54}{\suttatitletranslation Four Discourses on Not Directly Experiencing Form, Etc. }{\suttatitleroot Rūpaappaccakkhakammādisuttacatukka}}
\addcontentsline{toc}{section}{\tocacronym{SN 33.51–54} \toctranslation{Four Discourses on Not Directly Experiencing Form, Etc. } \tocroot{Rūpaappaccakkhakammādisuttacatukka}}
\markboth{Four Discourses on Not Directly Experiencing Form, Etc. }{Rūpaappaccakkhakammādisuttacatukka}
\extramarks{SN 33.51–54}{SN 33.51–54}

At\marginnote{1.1} \textsanskrit{Sāvatthī}. 

Then\marginnote{1.2} the wanderer Vacchagotta went up to the Buddha and exchanged greetings with him. When the greetings and polite conversation were over, he sat down to one side, and said to the Buddha: 

“What\marginnote{1.4} is the cause, Master Gotama, what is the reason why these various misconceptions arise in the world? …” 

“Vaccha,\marginnote{1.7} it is because of not directly experiencing form … 

feeling\marginnote{1.1} … 

perception\marginnote{1.1} … 

choices\marginnote{1.1} …” 

%
\section*{{\suttatitleacronym SN 33.55}{\suttatitletranslation Not Directly Experiencing Consciousness }{\suttatitleroot Viññāṇaappaccakkhakammasutta}}
\addcontentsline{toc}{section}{\tocacronym{SN 33.55} \toctranslation{Not Directly Experiencing Consciousness } \tocroot{Viññāṇaappaccakkhakammasutta}}
\markboth{Not Directly Experiencing Consciousness }{Viññāṇaappaccakkhakammasutta}
\extramarks{SN 33.55}{SN 33.55}

At\marginnote{1.1} \textsanskrit{Sāvatthī}. 

“Vaccha,\marginnote{1.2} it is because of not directly experiencing consciousness, its origin, its cessation, and the practice that leads to its cessation that these various misconceptions arise in the world. This is the cause, this is the reason.” 

\scendsutta{The Linked Discourses with Vacchagotta are completed. }

%
\addtocontents{toc}{\let\protect\contentsline\protect\nopagecontentsline}
\part*{Linked Discourses on Absorption }
\addcontentsline{toc}{part}{Linked Discourses on Absorption }
\markboth{}{}
\addtocontents{toc}{\let\protect\contentsline\protect\oldcontentsline}

%
\addtocontents{toc}{\let\protect\contentsline\protect\nopagecontentsline}
\chapter*{The Chapter on Absorption }
\addcontentsline{toc}{chapter}{\tocchapterline{The Chapter on Absorption }}
\addtocontents{toc}{\let\protect\contentsline\protect\oldcontentsline}

%
\section*{{\suttatitleacronym SN 34.1}{\suttatitletranslation Entering Immersion }{\suttatitleroot Samādhimūlakasamāpattisutta}}
\addcontentsline{toc}{section}{\tocacronym{SN 34.1} \toctranslation{Entering Immersion } \tocroot{Samādhimūlakasamāpattisutta}}
\markboth{Entering Immersion }{Samādhimūlakasamāpattisutta}
\extramarks{SN 34.1}{SN 34.1}

At\marginnote{1.1} \textsanskrit{Sāvatthī}. 

“Mendicants,\marginnote{1.2} there are these four meditators. What four? 

One\marginnote{1.4} meditator is skilled in immersion but not in entering it. 

One\marginnote{1.5} meditator is not skilled in immersion but is skilled in entering it. 

One\marginnote{1.6} meditator is skilled neither in immersion nor in entering it. 

One\marginnote{1.7} meditator is skilled both in immersion and in entering it. 

Of\marginnote{1.8} these, the meditator skilled in immersion and in entering it is the foremost, best, chief, highest, and finest of the four. 

From\marginnote{1.9} a cow comes milk, from milk comes curds, from curds come butter, from butter comes ghee, and from ghee comes cream of ghee. And the cream of ghee is said to be the best of these. 

In\marginnote{1.10} the same way, the meditator skilled in immersion and entering it is the foremost, best, leading, highest, and finest of the four.” 

%
\section*{{\suttatitleacronym SN 34.2}{\suttatitletranslation Remaining in Immersion }{\suttatitleroot Samādhimūlakaṭhitisutta}}
\addcontentsline{toc}{section}{\tocacronym{SN 34.2} \toctranslation{Remaining in Immersion } \tocroot{Samādhimūlakaṭhitisutta}}
\markboth{Remaining in Immersion }{Samādhimūlakaṭhitisutta}
\extramarks{SN 34.2}{SN 34.2}

At\marginnote{1.1} \textsanskrit{Sāvatthī}. 

“Mendicants,\marginnote{1.2} there are these four meditators. What four? 

One\marginnote{1.4} meditator is skilled in immersion but not in remaining in it. 

One\marginnote{1.5} meditator is skilled in remaining in immersion but is not skilled in immersion. 

One\marginnote{1.6} meditator is skilled neither in immersion nor in remaining in it. 

One\marginnote{1.7} meditator is skilled both in immersion and in remaining in it. 

Of\marginnote{1.8} these, the meditator skilled in immersion and in remaining in it is the foremost, best, leading, highest, and finest of the four. 

From\marginnote{1.9} a cow comes milk, from milk comes curds, from curds come butter, from butter comes ghee, and from ghee comes cream of ghee. And the cream of ghee is said to be the best of these. 

In\marginnote{1.10} the same way, the meditator skilled in immersion and remaining in it is the foremost, best, leading, highest, and finest of the four.” 

%
\section*{{\suttatitleacronym SN 34.3}{\suttatitletranslation Emerging From Immersion }{\suttatitleroot Samādhimūlakavuṭṭhānasutta}}
\addcontentsline{toc}{section}{\tocacronym{SN 34.3} \toctranslation{Emerging From Immersion } \tocroot{Samādhimūlakavuṭṭhānasutta}}
\markboth{Emerging From Immersion }{Samādhimūlakavuṭṭhānasutta}
\extramarks{SN 34.3}{SN 34.3}

At\marginnote{1.1} \textsanskrit{Sāvatthī}. 

“Mendicants,\marginnote{1.2} there are these four meditators. What four? 

One\marginnote{1.4} meditator is skilled in immersion but not in emerging from it. …” 

%
\section*{{\suttatitleacronym SN 34.4}{\suttatitletranslation Gladdening for Immersion }{\suttatitleroot Samādhimūlakakallitasutta}}
\addcontentsline{toc}{section}{\tocacronym{SN 34.4} \toctranslation{Gladdening for Immersion } \tocroot{Samādhimūlakakallitasutta}}
\markboth{Gladdening for Immersion }{Samādhimūlakakallitasutta}
\extramarks{SN 34.4}{SN 34.4}

At\marginnote{1.1} \textsanskrit{Sāvatthī}. 

“Mendicants,\marginnote{1.2} there are these four meditators. What four? 

One\marginnote{1.4} meditator is skilled in immersion but not in gladdening the mind for immersion. …” 

%
\section*{{\suttatitleacronym SN 34.5}{\suttatitletranslation Supports For Immersion }{\suttatitleroot Samādhimūlakaārammaṇasutta}}
\addcontentsline{toc}{section}{\tocacronym{SN 34.5} \toctranslation{Supports For Immersion } \tocroot{Samādhimūlakaārammaṇasutta}}
\markboth{Supports For Immersion }{Samādhimūlakaārammaṇasutta}
\extramarks{SN 34.5}{SN 34.5}

At\marginnote{1.1} \textsanskrit{Sāvatthī}. 

“Mendicants,\marginnote{1.2} there are these four meditators. What four? 

One\marginnote{1.4} meditator is skilled in immersion but not in the supports for immersion. …” 

%
\section*{{\suttatitleacronym SN 34.6}{\suttatitletranslation Meditation Subjects For Immersion }{\suttatitleroot Samādhimūlakagocarasutta}}
\addcontentsline{toc}{section}{\tocacronym{SN 34.6} \toctranslation{Meditation Subjects For Immersion } \tocroot{Samādhimūlakagocarasutta}}
\markboth{Meditation Subjects For Immersion }{Samādhimūlakagocarasutta}
\extramarks{SN 34.6}{SN 34.6}

At\marginnote{1.1} \textsanskrit{Sāvatthī}. 

“Mendicants,\marginnote{1.2} there are these four meditators. What four? 

One\marginnote{1.4} meditator is skilled in immersion but not in the meditation subjects for immersion. …” 

%
\section*{{\suttatitleacronym SN 34.7}{\suttatitletranslation Projecting the Mind Purified by Immersion }{\suttatitleroot Samādhimūlakaabhinīhārasutta}}
\addcontentsline{toc}{section}{\tocacronym{SN 34.7} \toctranslation{Projecting the Mind Purified by Immersion } \tocroot{Samādhimūlakaabhinīhārasutta}}
\markboth{Projecting the Mind Purified by Immersion }{Samādhimūlakaabhinīhārasutta}
\extramarks{SN 34.7}{SN 34.7}

At\marginnote{1.1} \textsanskrit{Sāvatthī}. 

“Mendicants,\marginnote{1.2} there are these four meditators. What four? 

One\marginnote{1.4} meditator is skilled in immersion but not in projecting the mind purified by immersion. …” 

%
\section*{{\suttatitleacronym SN 34.8}{\suttatitletranslation Carefulness in Immersion }{\suttatitleroot Samādhimūlakasakkaccakārīsutta}}
\addcontentsline{toc}{section}{\tocacronym{SN 34.8} \toctranslation{Carefulness in Immersion } \tocroot{Samādhimūlakasakkaccakārīsutta}}
\markboth{Carefulness in Immersion }{Samādhimūlakasakkaccakārīsutta}
\extramarks{SN 34.8}{SN 34.8}

At\marginnote{1.1} \textsanskrit{Sāvatthī}. 

“Mendicants,\marginnote{1.2} there are these four meditators. What four? 

One\marginnote{1.4} meditator is skilled in immersion but not in practicing carefully for it. …” 

%
\section*{{\suttatitleacronym SN 34.9}{\suttatitletranslation Persistence in Immersion }{\suttatitleroot Samādhimūlakasātaccakārīsutta}}
\addcontentsline{toc}{section}{\tocacronym{SN 34.9} \toctranslation{Persistence in Immersion } \tocroot{Samādhimūlakasātaccakārīsutta}}
\markboth{Persistence in Immersion }{Samādhimūlakasātaccakārīsutta}
\extramarks{SN 34.9}{SN 34.9}

At\marginnote{1.1} \textsanskrit{Sāvatthī}. 

“Mendicants,\marginnote{1.2} there are these four meditators. What four? 

One\marginnote{1.4} meditator is skilled in immersion but not in practicing persistently for it. …” 

%
\section*{{\suttatitleacronym SN 34.10}{\suttatitletranslation Conducive to Immersion }{\suttatitleroot Samādhimūlakasappāyakārīsutta}}
\addcontentsline{toc}{section}{\tocacronym{SN 34.10} \toctranslation{Conducive to Immersion } \tocroot{Samādhimūlakasappāyakārīsutta}}
\markboth{Conducive to Immersion }{Samādhimūlakasappāyakārīsutta}
\extramarks{SN 34.10}{SN 34.10}

At\marginnote{1.1} \textsanskrit{Sāvatthī}. 

“Mendicants,\marginnote{1.2} there are these four meditators. What four? 

One\marginnote{1.4} meditator is skilled in immersion but not in doing what’s conducive to it. …” 

%
\section*{{\suttatitleacronym SN 34.11}{\suttatitletranslation Entering and Remaining }{\suttatitleroot Samāpattimūlakaṭhitisutta}}
\addcontentsline{toc}{section}{\tocacronym{SN 34.11} \toctranslation{Entering and Remaining } \tocroot{Samāpattimūlakaṭhitisutta}}
\markboth{Entering and Remaining }{Samāpattimūlakaṭhitisutta}
\extramarks{SN 34.11}{SN 34.11}

At\marginnote{1.1} \textsanskrit{Sāvatthī}. 

“Mendicants,\marginnote{1.2} there are these four meditators. What four? 

One\marginnote{1.4} meditator is skilled in entering immersion but not in remaining in it. …” 

%
\section*{{\suttatitleacronym SN 34.12}{\suttatitletranslation Entering and Emerging }{\suttatitleroot Samāpattimūlakavuṭṭhānasutta}}
\addcontentsline{toc}{section}{\tocacronym{SN 34.12} \toctranslation{Entering and Emerging } \tocroot{Samāpattimūlakavuṭṭhānasutta}}
\markboth{Entering and Emerging }{Samāpattimūlakavuṭṭhānasutta}
\extramarks{SN 34.12}{SN 34.12}

At\marginnote{1.1} \textsanskrit{Sāvatthī}. 

“Mendicants,\marginnote{1.2} there are these four meditators. What four? 

One\marginnote{1.4} meditator is skilled in entering immersion but not in emerging from it. …” 

%
\section*{{\suttatitleacronym SN 34.13}{\suttatitletranslation Entering and Gladdening }{\suttatitleroot Samāpattimūlakakallitasutta}}
\addcontentsline{toc}{section}{\tocacronym{SN 34.13} \toctranslation{Entering and Gladdening } \tocroot{Samāpattimūlakakallitasutta}}
\markboth{Entering and Gladdening }{Samāpattimūlakakallitasutta}
\extramarks{SN 34.13}{SN 34.13}

At\marginnote{1.1} \textsanskrit{Sāvatthī}. 

“Mendicants,\marginnote{1.2} there are these four meditators. What four? 

One\marginnote{1.4} meditator is skilled in entering immersion but not in gladdening the mind for immersion. …” 

%
\section*{{\suttatitleacronym SN 34.14}{\suttatitletranslation Entering and Supports }{\suttatitleroot Samāpattimūlakaārammaṇasutta}}
\addcontentsline{toc}{section}{\tocacronym{SN 34.14} \toctranslation{Entering and Supports } \tocroot{Samāpattimūlakaārammaṇasutta}}
\markboth{Entering and Supports }{Samāpattimūlakaārammaṇasutta}
\extramarks{SN 34.14}{SN 34.14}

At\marginnote{1.1} \textsanskrit{Sāvatthī}. 

“Mendicants,\marginnote{1.2} there are these four meditators. What four? 

One\marginnote{1.4} meditator is skilled in entering immersion but not in the supports for it. …” 

%
\section*{{\suttatitleacronym SN 34.15}{\suttatitletranslation Entering and Meditation Subjects }{\suttatitleroot Samāpattimūlakagocarasutta}}
\addcontentsline{toc}{section}{\tocacronym{SN 34.15} \toctranslation{Entering and Meditation Subjects } \tocroot{Samāpattimūlakagocarasutta}}
\markboth{Entering and Meditation Subjects }{Samāpattimūlakagocarasutta}
\extramarks{SN 34.15}{SN 34.15}

At\marginnote{1.1} \textsanskrit{Sāvatthī}. 

“Mendicants,\marginnote{1.2} there are these four meditators. What four? 

One\marginnote{1.4} meditator is skilled in entering immersion but not in the mindfulness meditation subjects for immersion. …” 

%
\section*{{\suttatitleacronym SN 34.16}{\suttatitletranslation Entering and Projecting }{\suttatitleroot Samāpattimūlakaabhinīhārasutta}}
\addcontentsline{toc}{section}{\tocacronym{SN 34.16} \toctranslation{Entering and Projecting } \tocroot{Samāpattimūlakaabhinīhārasutta}}
\markboth{Entering and Projecting }{Samāpattimūlakaabhinīhārasutta}
\extramarks{SN 34.16}{SN 34.16}

At\marginnote{1.1} \textsanskrit{Sāvatthī}. 

“Mendicants,\marginnote{1.2} there are these four meditators. What four? 

One\marginnote{1.4} meditator is skilled in entering immersion but not in projecting the mind purified by immersion. …” 

%
\section*{{\suttatitleacronym SN 34.17}{\suttatitletranslation Entering and Carefulness }{\suttatitleroot Samāpattimūlakasakkaccasutta}}
\addcontentsline{toc}{section}{\tocacronym{SN 34.17} \toctranslation{Entering and Carefulness } \tocroot{Samāpattimūlakasakkaccasutta}}
\markboth{Entering and Carefulness }{Samāpattimūlakasakkaccasutta}
\extramarks{SN 34.17}{SN 34.17}

At\marginnote{1.1} \textsanskrit{Sāvatthī}. 

“Mendicants,\marginnote{1.2} there are these four meditators. What four? 

One\marginnote{1.4} meditator is skilled in entering immersion but not in practicing carefully for it. …” 

%
\section*{{\suttatitleacronym SN 34.18}{\suttatitletranslation Entering and Persistence }{\suttatitleroot Samāpattimūlakasātaccasutta}}
\addcontentsline{toc}{section}{\tocacronym{SN 34.18} \toctranslation{Entering and Persistence } \tocroot{Samāpattimūlakasātaccasutta}}
\markboth{Entering and Persistence }{Samāpattimūlakasātaccasutta}
\extramarks{SN 34.18}{SN 34.18}

At\marginnote{1.1} \textsanskrit{Sāvatthī}. 

“Mendicants,\marginnote{1.2} there are these four meditators. What four? 

One\marginnote{1.4} meditator is skilled in entering immersion but not in practicing persistently for it. …” 

%
\section*{{\suttatitleacronym SN 34.19}{\suttatitletranslation Entering and What’s Conducive }{\suttatitleroot Samāpattimūlakasappāyakārīsutta}}
\addcontentsline{toc}{section}{\tocacronym{SN 34.19} \toctranslation{Entering and What’s Conducive } \tocroot{Samāpattimūlakasappāyakārīsutta}}
\markboth{Entering and What’s Conducive }{Samāpattimūlakasappāyakārīsutta}
\extramarks{SN 34.19}{SN 34.19}

At\marginnote{1.1} \textsanskrit{Sāvatthī}. 

“Mendicants,\marginnote{1.2} there are these four meditators. What four? 

One\marginnote{1.4} meditator is skilled in entering immersion but not in doing what’s conducive to it. …” 

%
\section*{{\suttatitleacronym SN 34.20–27}{\suttatitletranslation Eight on Remaining and Emergence, Etc. }{\suttatitleroot Ṭhitimūlakavuṭṭhānasuttādiaṭṭhaka}}
\addcontentsline{toc}{section}{\tocacronym{SN 34.20–27} \toctranslation{Eight on Remaining and Emergence, Etc. } \tocroot{Ṭhitimūlakavuṭṭhānasuttādiaṭṭhaka}}
\markboth{Eight on Remaining and Emergence, Etc. }{Ṭhitimūlakavuṭṭhānasuttādiaṭṭhaka}
\extramarks{SN 34.20–27}{SN 34.20–27}

At\marginnote{1.1} \textsanskrit{Sāvatthī}. 

“Mendicants,\marginnote{1.2} there are these four meditators. What four? 

One\marginnote{1.4} meditator is skilled in remaining in immersion but not in emerging from it. …” 

\scendsection{(These eight discourses should be expanded in line with the previous set.) }

%
\section*{{\suttatitleacronym SN 34.28–34}{\suttatitletranslation Seven on Emergence and Gladdening, Etc. }{\suttatitleroot Vuṭṭhānamūlakakallitasuttādisattaka}}
\addcontentsline{toc}{section}{\tocacronym{SN 34.28–34} \toctranslation{Seven on Emergence and Gladdening, Etc. } \tocroot{Vuṭṭhānamūlakakallitasuttādisattaka}}
\markboth{Seven on Emergence and Gladdening, Etc. }{Vuṭṭhānamūlakakallitasuttādisattaka}
\extramarks{SN 34.28–34}{SN 34.28–34}

At\marginnote{1.1} \textsanskrit{Sāvatthī}. 

“Mendicants,\marginnote{1.2} there are these four meditators. What four? 

One\marginnote{1.4} meditator is skilled in emerging from immersion but not in gladdening the mind for immersion. …” 

\scendsection{(These seven discourses should be expanded in line with the previous set.) }

%
\section*{{\suttatitleacronym SN 34.35–40}{\suttatitletranslation Six on Gladdening and Support, Etc. }{\suttatitleroot Kallitamūlakaārammaṇasuttādichakka}}
\addcontentsline{toc}{section}{\tocacronym{SN 34.35–40} \toctranslation{Six on Gladdening and Support, Etc. } \tocroot{Kallitamūlakaārammaṇasuttādichakka}}
\markboth{Six on Gladdening and Support, Etc. }{Kallitamūlakaārammaṇasuttādichakka}
\extramarks{SN 34.35–40}{SN 34.35–40}

At\marginnote{1.1} \textsanskrit{Sāvatthī}. 

“One\marginnote{1.2} meditator is skilled in gladdening the mind for immersion but not in the supports for immersion. …” 

\scendsection{(These six discourses should be expanded in line with the previous set.) }

%
\section*{{\suttatitleacronym SN 34.41–45}{\suttatitletranslation Five on Support and Subjects, Etc. }{\suttatitleroot Ārammaṇamūlakagocarasuttādipañcaka}}
\addcontentsline{toc}{section}{\tocacronym{SN 34.41–45} \toctranslation{Five on Support and Subjects, Etc. } \tocroot{Ārammaṇamūlakagocarasuttādipañcaka}}
\markboth{Five on Support and Subjects, Etc. }{Ārammaṇamūlakagocarasuttādipañcaka}
\extramarks{SN 34.41–45}{SN 34.41–45}

At\marginnote{1.1} \textsanskrit{Sāvatthī}. 

“One\marginnote{1.2} meditator is skilled in the supports for immersion but not in the mindfulness meditation subjects for immersion. …” 

\scendsection{(These five discourses should be expanded in line with the previous set.) }

%
\section*{{\suttatitleacronym SN 34.46–49}{\suttatitletranslation Four on Subjects and Projection, Etc. }{\suttatitleroot Gocaramūlakaabhinīhārasuttādicatukka}}
\addcontentsline{toc}{section}{\tocacronym{SN 34.46–49} \toctranslation{Four on Subjects and Projection, Etc. } \tocroot{Gocaramūlakaabhinīhārasuttādicatukka}}
\markboth{Four on Subjects and Projection, Etc. }{Gocaramūlakaabhinīhārasuttādicatukka}
\extramarks{SN 34.46–49}{SN 34.46–49}

At\marginnote{1.1} \textsanskrit{Sāvatthī}. 

“One\marginnote{1.2} meditator is skilled in the mindfulness meditation subjects for immersion but not in projecting the mind purified by immersion. …” 

(These\marginnote{2.1} four discourses should be expanded in line with the previous set.) 

%
\section*{{\suttatitleacronym SN 34.50–52}{\suttatitletranslation Three on Projection and Carefulness }{\suttatitleroot Abhinīhāramūlakasakkaccasuttāditika}}
\addcontentsline{toc}{section}{\tocacronym{SN 34.50–52} \toctranslation{Three on Projection and Carefulness } \tocroot{Abhinīhāramūlakasakkaccasuttāditika}}
\markboth{Three on Projection and Carefulness }{Abhinīhāramūlakasakkaccasuttāditika}
\extramarks{SN 34.50–52}{SN 34.50–52}

At\marginnote{1.1} \textsanskrit{Sāvatthī}. 

“One\marginnote{1.2} meditator is skilled in projecting the mind purified by immersion but not in practicing carefully for it. …” 

(These\marginnote{2.1} three discourses should be expanded in line with the previous set.) 

%
\section*{{\suttatitleacronym SN 34.53–54}{\suttatitletranslation Two on Carefulness and Persistence }{\suttatitleroot Sakkaccamūlakasātaccakārīsuttadukādi}}
\addcontentsline{toc}{section}{\tocacronym{SN 34.53–54} \toctranslation{Two on Carefulness and Persistence } \tocroot{Sakkaccamūlakasātaccakārīsuttadukādi}}
\markboth{Two on Carefulness and Persistence }{Sakkaccamūlakasātaccakārīsuttadukādi}
\extramarks{SN 34.53–54}{SN 34.53–54}

At\marginnote{1.1} \textsanskrit{Sāvatthī}. 

“One\marginnote{1.2} meditator is skilled in practicing carefully for immersion but not in practicing persistently for it. …” 

(These\marginnote{2.1} two discourses should be expanded in line with the previous set.) 

%
\section*{{\suttatitleacronym SN 34.55}{\suttatitletranslation Persistence and What’s Conducive }{\suttatitleroot Sātaccamūlakasappāyakārīsutta}}
\addcontentsline{toc}{section}{\tocacronym{SN 34.55} \toctranslation{Persistence and What’s Conducive } \tocroot{Sātaccamūlakasappāyakārīsutta}}
\markboth{Persistence and What’s Conducive }{Sātaccamūlakasappāyakārīsutta}
\extramarks{SN 34.55}{SN 34.55}

At\marginnote{1.1} \textsanskrit{Sāvatthī}. 

“Mendicants,\marginnote{1.2} there are these four meditators. What four? 

One\marginnote{1.4} meditator is skilled in practicing persistently for immersion but not in doing what’s conducive to it. 

One\marginnote{1.5} meditator is skilled in doing what’s conducive to immersion but not in practicing persistently for it. 

One\marginnote{1.6} meditator is skilled neither in practicing persistently for immersion nor in doing what’s conducive to it. 

One\marginnote{1.7} meditator is skilled both in practicing persistently for immersion and in doing what’s conducive to it. 

Of\marginnote{1.8} these, the meditator skilled both in practicing persistently for immersion and in doing what’s conducive to it is the foremost, best, leading, highest, and finest of the four. 

From\marginnote{1.9} a cow comes milk, from milk comes curds, from curds come butter, from butter comes ghee, and from ghee comes cream of ghee. And the cream of ghee is said to be the best of these. 

In\marginnote{1.10} the same way, the meditator skilled both in practicing persistently for immersion and in doing what’s conducive to it is the foremost, best, leading, highest, and finest of the four.” 

That\marginnote{1.11} is what the Buddha said. Satisfied, the mendicants were happy with what the Buddha said. 

\scendsutta{The Linked Discourses on Absorption are complete. }

\scendbook{The Book of the Aggregates is finished. }

%
\backmatter%
\chapter*{Colophon}
\addcontentsline{toc}{chapter}{Colophon}
\markboth{Colophon}{Colophon}

\section*{The Translator}

Bhikkhu Sujato was born as Anthony Aidan Best on 4/11/1966 in Perth, Western Australia. He grew up in the pleasant suburbs of Mt Lawley and Attadale alongside his sister Nicola, who was the good child. His mother, Margaret Lorraine Huntsman née Pinder, said “he’ll either be a priest or a poet”, while his father, Anthony Thomas Best, advised him to “never do anything for money”. He attended Aquinas College, a Catholic school, where he decided to become an atheist. At the University of WA he studied philosophy, aiming to learn what he wanted to do with his life. Finding that what he wanted to do was play guitar, he dropped out. His main band was named Martha’s Vineyard, which achieved modest success in the indie circuit. 

A seemingly random encounter with a roadside joey took him to Thailand, where he entered his first meditation retreat at Wat Ram Poeng, Chieng Mai in 1992. Feeling the call to the Buddha’s path, he took full ordination in Wat Pa Nanachat in 1994, where his teachers were Ajahn Pasanno and Ajahn Jayasaro. In 1997 he returned to Perth to study with Ajahn Brahm at Bodhinyana Monastery. 

He spent several years practicing in seclusion in Malaysia and Thailand before establishing Santi Forest Monastery in Bundanoon, NSW, in 2003. There he was instrumental in supporting the establishment of the Theravada bhikkhuni order in Australia and advocating for women’s rights. He continues to teach in Australia and globally, with a special concern for the moral implications of climate change and other forms of environmental destruction. He has published a series of books of original and groundbreaking research on early Buddhism. 

In 2005 he founded SuttaCentral together with Rod Bucknell and John Kelly. In 2015, seeing the need for a complete, accurate, plain English translation of the Pali texts, he undertook the task, spending nearly three years in isolation on the isle of Qi Mei off the coast of the nation of Taiwan. He completed the four main \textsanskrit{Nikāyas} in 2018, and the early books of the Khuddaka \textsanskrit{Nikāya} were complete by 2021. All this work is dedicated to the public domain and is entirely free of copyright encumbrance. 

In 2019 he returned to Sydney where he established Lokanta Vihara (The Monastery at the End of the World). 

\section*{Creation Process}

Primary source was the digital \textsanskrit{Mahāsaṅgīti} edition of the Pali \textsanskrit{Tipiṭaka}. Translated from the Pali, with reference to several English translations, especially those of Bhikkhu Bodhi.

\section*{The Translation}

This translation was part of a project to translate the four Pali \textsanskrit{Nikāyas} with the following aims: plain, approachable English; consistent terminology; accurate rendition of the Pali; free of copyright. It was made during 2016–2018 while Bhikkhu Sujato was staying in Qimei, Taiwan.

\section*{About SuttaCentral}

SuttaCentral publishes early Buddhist texts. Since 2005 we have provided root texts in Pali, Chinese, Sanskrit, Tibetan, and other languages, parallels between these texts, and translations in many modern languages. We build on the work of generations of scholars, and offer our contribution freely.

SuttaCentral is driven by volunteer contributions, and in addition we employ professional developers. We offer a sponsorship program for high quality translations from the original languages. Financial support for SuttaCentral is handled by the SuttaCentral Development Trust, a charitable trust registered in Australia.

\section*{About Bilara}

“Bilara” means “cat” in Pali, and it is the name of our Computer Assisted Translation (CAT) software. Bilara is a web app that enables translators to translate early Buddhist texts into their own language. These translations are published on SuttaCentral with the root text and translation side by side.

\section*{About SuttaCentral Editions}

The SuttaCentral Editions project makes high quality books from selected Bilara translations. These are published in formats including HTML, EPUB, PDF, and print.

If you want to print any of our Editions, please let us know and we will help prepare a file to your specifications.

%
\end{document}