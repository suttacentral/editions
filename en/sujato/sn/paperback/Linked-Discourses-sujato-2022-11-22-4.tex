\documentclass[12pt,openany]{book}%
\usepackage{lastpage}%
%
\usepackage[inner=1in, outer=1in, top=.7in, bottom=1in, papersize={6in,9in}, headheight=13pt]{geometry}
\usepackage{polyglossia}
\usepackage[12pt]{moresize}
\usepackage{soul}%
\usepackage{microtype}
\usepackage{tocbasic}
\usepackage{realscripts}
\usepackage{epigraph}%
\usepackage{setspace}%
\usepackage{sectsty}
\usepackage{fontspec}
\usepackage{marginnote}
\usepackage[bottom]{footmisc}
\usepackage{enumitem}
\usepackage{fancyhdr}
\usepackage{extramarks}
\usepackage{graphicx}
\usepackage{verse}
\usepackage{relsize}
\usepackage{etoolbox}
\usepackage[a-3u]{pdfx}

\hypersetup{
colorlinks=true,
urlcolor=black,
linkcolor=black,
citecolor=black
}

% use a small amount of tracking on small caps
\SetTracking[ spacing = {25*,166, } ]{ encoding = *, shape = sc }{ 25 }

% add a blank page
\newcommand{\blankpage}{
\newpage
\thispagestyle{empty}
\mbox{}
\newpage
}

% define languages
\setdefaultlanguage[]{english}
\setotherlanguage[script=Latin]{sanskrit}

%\usepackage{pagegrid}
%\pagegridsetup{top-left, step=.25in}

% define fonts
% use if arno sanskrit is unavailable
%\setmainfont{Gentium Plus}
%\newfontfamily\Semiboldsubheadfont[]{Gentium Plus}
%\newfontfamily\Semiboldnormalfont[]{Gentium Plus}
%\newfontfamily\Lightfont[]{Gentium Plus}
%\newfontfamily\Marginalfont[]{Gentium Plus}
%\newfontfamily\Allsmallcapsfont[RawFeature=+c2sc]{Gentium Plus}
%\newfontfamily\Noligaturefont[Renderer=Basic]{Gentium Plus}
%\newfontfamily\Noligaturecaptionfont[Renderer=Basic]{Gentium Plus}
%\newfontfamily\Fleuronfont[Ornament=1]{Gentium Plus}

% use if arno sanskrit is available. display is applied to \chapter and \part, subhead to \section and \subsection. When specifying semibold, the italic must be defined.
\setmainfont[Numbers=OldStyle]{Arno Pro}
\newfontfamily\Semibolddisplayfont[BoldItalicFont = Arno Pro Semibold Italic Display]{Arno Pro Semibold Display} %
\newfontfamily\Semiboldsubheadfont[BoldItalicFont = Arno Pro Semibold Italic Subhead]{Arno Pro Semibold Subhead}
\newfontfamily\Semiboldnormalfont[BoldItalicFont = Arno Pro Semibold Italic]{Arno Pro Semibold}
\newfontfamily\Marginalfont[RawFeature=+subs]{Arno Pro Regular}
\newfontfamily\Allsmallcapsfont[RawFeature=+c2sc]{Arno Pro}
\newfontfamily\Noligaturefont[Renderer=Basic]{Arno Pro}
\newfontfamily\Noligaturecaptionfont[Renderer=Basic]{Arno Pro Caption}

% chinese fonts
\newfontfamily\cjk{Noto Serif TC}
\newcommand*{\langlzh}[1]{\cjk{#1}\normalfont}%

% logo
\newfontfamily\Logofont{sclogo.ttf}
\newcommand*{\sclogo}[1]{\large\Logofont{#1}}

% use subscript numerals for margin notes
\renewcommand*{\marginfont}{\Marginalfont}

% ensure margin notes have consistent vertical alignment
\renewcommand*{\marginnotevadjust}{-.17em}

% use compact lists
\setitemize{noitemsep,leftmargin=1em}
\setenumerate{noitemsep,leftmargin=1em}
\setdescription{noitemsep, style=unboxed, leftmargin=0em}

% style ToC
\DeclareTOCStyleEntries[
  raggedentrytext,
  linefill=\hfill,
  pagenumberwidth=.5in,
  pagenumberformat=\normalfont,
  entryformat=\normalfont
]{tocline}{chapter,section}


  \setlength\topsep{0pt}%
  \setlength\parskip{0pt}%

% define new \centerpars command for use in ToC. This ensures centering, proper wrapping, and no page break after
\def\startcenter{%
  \par
  \begingroup
  \leftskip=0pt plus 1fil
  \rightskip=\leftskip
  \parindent=0pt
  \parfillskip=0pt
}
\def\stopcenter{%
  \par
  \endgroup
}
\long\def\centerpars#1{\startcenter#1\stopcenter}

% redefine part, so that it adds a toc entry without page number
\let\oldcontentsline\contentsline
\newcommand{\nopagecontentsline}[3]{\oldcontentsline{#1}{#2}{}}

    \makeatletter
\renewcommand*\l@part[2]{%
  \ifnum \c@tocdepth >-2\relax
    \addpenalty{-\@highpenalty}%
    \addvspace{0em \@plus\p@}%
    \setlength\@tempdima{3em}%
    \begingroup
      \parindent \z@ \rightskip \@pnumwidth
      \parfillskip -\@pnumwidth
      {\leavevmode
       \setstretch{.85}\large\scshape\centerpars{#1}\vspace*{-1em}\llap{#2}}\par
       \nobreak
         \global\@nobreaktrue
         \everypar{\global\@nobreakfalse\everypar{}}%
    \endgroup
  \fi}
\makeatother

\makeatletter
\def\@pnumwidth{2em}
\makeatother

% define new sectioning command, which is only used in volumes where the pannasa is found in some parts but not others, especially in an and sn

\newcommand*{\pannasa}[1]{\clearpage\thispagestyle{empty}\begin{center}\vspace*{14em}\setstretch{.85}\huge\itshape\scshape\MakeLowercase{#1}\end{center}}

    \makeatletter
\newcommand*\l@pannasa[2]{%
  \ifnum \c@tocdepth >-2\relax
    \addpenalty{-\@highpenalty}%
    \addvspace{.5em \@plus\p@}%
    \setlength\@tempdima{3em}%
    \begingroup
      \parindent \z@ \rightskip \@pnumwidth
      \parfillskip -\@pnumwidth
      {\leavevmode
       \setstretch{.85}\large\itshape\scshape\lowercase{\centerpars{#1}}\vspace*{-1em}\llap{#2}}\par
       \nobreak
         \global\@nobreaktrue
         \everypar{\global\@nobreakfalse\everypar{}}%
    \endgroup
  \fi}
\makeatother

% don't put page number on first page of toc (relies on etoolbox)
\patchcmd{\chapter}{plain}{empty}{}{}

% global line height
\setstretch{1.05}

% allow linebreak after em-dash
\catcode`\—=13
\protected\def—{\unskip\textemdash\allowbreak}

% style headings with secsty. chapter and section are defined per-edition
\partfont{\setstretch{.85}\normalfont\centering\textsc}
\subsectionfont{\setstretch{.85}\Semiboldsubheadfont}%
\subsubsectionfont{\setstretch{.85}\Semiboldnormalfont}

% style elements of suttatitle
\newcommand*{\suttatitleacronym}[1]{\smaller[2]{#1}\vspace*{.3em}}
\newcommand*{\suttatitletranslation}[1]{\linebreak{#1}}
\newcommand*{\suttatitleroot}[1]{\linebreak\smaller[2]\itshape{#1}}

\DeclareTOCStyleEntries[
  indent=3.3em,
  dynindent,
  beforeskip=.2em plus -2pt minus -1pt,
]{tocline}{section}

\DeclareTOCStyleEntries[
  indent=0em,
  dynindent,
  beforeskip=.4em plus -2pt minus -1pt,
]{tocline}{chapter}

\newcommand*{\tocacronym}[1]{\hspace*{-3.3em}{#1}\quad}
\newcommand*{\toctranslation}[1]{#1}
\newcommand*{\tocroot}[1]{(\textit{#1})}
\newcommand*{\tocchapterline}[1]{\bfseries\itshape{#1}}


% redefine paragraph and subparagraph headings to not be inline
\makeatletter
% Change the style of paragraph headings %
\renewcommand\paragraph{\@startsection{paragraph}{4}{\z@}%
            {-2.5ex\@plus -1ex \@minus -.25ex}%
            {1.25ex \@plus .25ex}%
            {\noindent\Semiboldnormalfont\normalsize}}

% Change the style of subparagraph headings %
\renewcommand\subparagraph{\@startsection{subparagraph}{5}{\z@}%
            {-2.5ex\@plus -1ex \@minus -.25ex}%
            {1.25ex \@plus .25ex}%
            {\noindent\Semiboldnormalfont\small}}
\makeatother

% use etoolbox to suppress page numbers on \part
\patchcmd{\part}{\thispagestyle{plain}}{\thispagestyle{empty}}
  {}{\errmessage{Cannot patch \string\part}}

% and to reduce margins on quotation
\patchcmd{\quotation}{\rightmargin}{\leftmargin 1.2em \rightmargin}{}{}
\AtBeginEnvironment{quotation}{\small}

% titlepage
\newcommand*{\titlepageTranslationTitle}[1]{{\begin{center}\begin{large}{#1}\end{large}\end{center}}}
\newcommand*{\titlepageCreatorName}[1]{{\begin{center}\begin{normalsize}{#1}\end{normalsize}\end{center}}}

% halftitlepage
\newcommand*{\halftitlepageTranslationTitle}[1]{\setstretch{2.5}{\begin{Huge}\uppercase{\so{#1}}\end{Huge}}}
\newcommand*{\halftitlepageTranslationSubtitle}[1]{\setstretch{1.2}{\begin{large}{#1}\end{large}}}
\newcommand*{\halftitlepageFleuron}[1]{{\begin{large}\Fleuronfont{{#1}}\end{large}}}
\newcommand*{\halftitlepageByline}[1]{{\begin{normalsize}\textit{{#1}}\end{normalsize}}}
\newcommand*{\halftitlepageCreatorName}[1]{{\begin{LARGE}{\textsc{#1}}\end{LARGE}}}
\newcommand*{\halftitlepageVolumeNumber}[1]{{\begin{normalsize}{\Allsmallcapsfont{\textsc{#1}}}\end{normalsize}}}
\newcommand*{\halftitlepageVolumeAcronym}[1]{{\begin{normalsize}{#1}\end{normalsize}}}
\newcommand*{\halftitlepageVolumeTranslationTitle}[1]{{\begin{Large}{\textsc{#1}}\end{Large}}}
\newcommand*{\halftitlepageVolumeRootTitle}[1]{{\begin{normalsize}{\Allsmallcapsfont{\textsc{\itshape #1}}}\end{normalsize}}}
\newcommand*{\halftitlepagePublisher}[1]{{\begin{large}{\Noligaturecaptionfont\textsc{#1}}\end{large}}}

% epigraph
\renewcommand{\epigraphflush}{center}
\renewcommand*{\epigraphwidth}{.85\textwidth}
\newcommand*{\epigraphTranslatedTitle}[1]{\vspace*{.5em}\footnotesize\textsc{#1}\\}%
\newcommand*{\epigraphRootTitle}[1]{\footnotesize\textit{#1}\\}%
\newcommand*{\epigraphReference}[1]{\footnotesize{#1}}%

% custom commands for html styling classes
\newcommand*{\scnamo}[1]{\begin{center}\textit{#1}\end{center}}
\newcommand*{\scendsection}[1]{\begin{center}\textit{#1}\end{center}}
\newcommand*{\scendsutta}[1]{\begin{center}\textit{#1}\end{center}}
\newcommand*{\scendbook}[1]{\begin{center}\uppercase{#1}\end{center}}
\newcommand*{\scendkanda}[1]{\begin{center}\textbf{#1}\end{center}}
\newcommand*{\scend}[1]{\begin{center}\textit{#1}\end{center}}
\newcommand*{\scuddanaintro}[1]{\textit{#1}}
\newcommand*{\scendvagga}[1]{\begin{center}\textbf{#1}\end{center}}
\newcommand*{\scrule}[1]{\textbf{#1}}
\newcommand*{\scadd}[1]{\textit{#1}}
\newcommand*{\scevam}[1]{\textsc{#1}}
\newcommand*{\scspeaker}[1]{\hspace{2em}\textit{#1}}
\newcommand*{\scbyline}[1]{\begin{flushright}\textit{#1}\end{flushright}\bigskip}

% custom command for thematic break = hr
\newcommand*{\thematicbreak}{\begin{center}\rule[.5ex]{6em}{.4pt}\begin{normalsize}\quad\Fleuronfont{•}\quad\end{normalsize}\rule[.5ex]{6em}{.4pt}\end{center}}

% manage and style page header and footer. "fancy" has header and footer, "plain" has footer only

\pagestyle{fancy}
\fancyhf{}
\fancyfoot[RE,LO]{\thepage}
\fancyfoot[LE,RO]{\footnotesize\lastleftxmark}
\fancyhead[CE]{\setstretch{.85}\Noligaturefont\MakeLowercase{\textsc{\firstrightmark}}}
\fancyhead[CO]{\setstretch{.85}\Noligaturefont\MakeLowercase{\textsc{\firstleftmark}}}
\renewcommand{\headrulewidth}{0pt}
\fancypagestyle{plain}{ %
\fancyhf{} % remove everything
\fancyfoot[RE,LO]{\thepage}
\fancyfoot[LE,RO]{\footnotesize\lastleftxmark}
\renewcommand{\headrulewidth}{0pt}
\renewcommand{\footrulewidth}{0pt}}

% style footnotes
\setlength{\skip\footins}{1em}

\makeatletter
\newcommand{\@makefntextcustom}[1]{%
    \parindent 0em%
    \thefootnote.\enskip #1%
}
\renewcommand{\@makefntext}[1]{\@makefntextcustom{#1}}
\makeatother

% hang quotes (requires microtype)
\microtypesetup{
  protrusion = true,
  expansion  = true,
  tracking   = true,
  factor     = 1000,
  patch      = all,
  final
}

% Custom protrusion rules to allow hanging punctuation
\SetProtrusion
{ encoding = *}
{
% char   right left
  {-} = {    , 500 },
  % Double Quotes
  \textquotedblleft
      = {1000,     },
  \textquotedblright
      = {    , 1000},
  \quotedblbase
      = {1000,     },
  % Single Quotes
  \textquoteleft
      = {1000,     },
  \textquoteright
      = {    , 1000},
  \quotesinglbase
      = {1000,     }
}

% make latex use actual font em for parindent, not Computer Modern Roman
\AtBeginDocument{\setlength{\parindent}{1em}}%
%

% Default values; a bit sloppier than normal
\tolerance 1414
\hbadness 1414
\emergencystretch 1.5em
\hfuzz 0.3pt
\clubpenalty = 10000
\widowpenalty = 10000
\displaywidowpenalty = 10000
\hfuzz \vfuzz
 \raggedbottom%

\title{Linked Discourses}
\author{Bhikkhu Sujato}
\date{}%
% define a different fleuron for each edition
\newfontfamily\Fleuronfont[Ornament=40]{Arno Pro}

% Define heading styles per edition for chapter and section. Suttatitle can be either of these, depending on the volume. 

\let\oldfrontmatter\frontmatter
\renewcommand{\frontmatter}{%
\chapterfont{\setstretch{.85}\normalfont\centering}%
\sectionfont{\setstretch{.85}\Semiboldsubheadfont}%
\oldfrontmatter}

\let\oldmainmatter\mainmatter
\renewcommand{\mainmatter}{%
\chapterfont{\setstretch{.85}\normalfont\centering}%
\sectionfont{\setstretch{.85}\normalfont\centering}%
\oldmainmatter}

\let\oldbackmatter\backmatter
\renewcommand{\backmatter}{%
\chapterfont{\setstretch{.85}\normalfont\centering}%
\sectionfont{\setstretch{.85}\Semiboldsubheadfont}%
\oldbackmatter}
%
%
\begin{document}%
\normalsize%
\frontmatter%
\setlength{\parindent}{0cm}

\pagestyle{empty}

\maketitle

\blankpage%
\begin{center}

\vspace*{2.2em}

\halftitlepageTranslationTitle{Linked Discourses}

\vspace*{1em}

\halftitlepageTranslationSubtitle{A plain translation of the Saṁyutta Nikāya}

\vspace*{2em}

\halftitlepageFleuron{•}

\vspace*{2em}

\halftitlepageByline{translated and introduced by}

\vspace*{.5em}

\halftitlepageCreatorName{Bhikkhu Sujato}

\vspace*{4em}

\halftitlepageVolumeNumber{Volume 4}

\smallskip

\halftitlepageVolumeAcronym{SN 35–44}

\smallskip

\halftitlepageVolumeTranslationTitle{The Group of Linked Discourses Beginning With the Six Sense Fields }

\smallskip

\halftitlepageVolumeRootTitle{Saḷāyatanavaggasaṁyutta}

\vspace*{\fill}

\sclogo{0}
 \halftitlepagePublisher{SuttaCentral}

\end{center}

\newpage
%
\setstretch{1.05}

\begin{footnotesize}

\textit{Linked Discourses} is a translation of the Saṁyuttanikāya by Bhikkhu Sujato.

\medskip

Creative Commons Zero (CC0)

To the extent possible under law, Bhikkhu Sujato has waived all copyright and related or neighboring rights to \textit{Linked Discourses}.

\medskip

This work is published from Australia.

\begin{center}
\textit{This translation is an expression of an ancient spiritual text that has been passed down by the Buddhist tradition for the benefit of all sentient beings. It is dedicated to the public domain via Creative Commons Zero (CC0). You are encouraged to copy, reproduce, adapt, alter, or otherwise make use of this translation. The translator respectfully requests that any use be in accordance with the values and principles of the Buddhist community.}
\end{center}

\medskip

\begin{description}
    \item[Web publication date] 2018
    \item[This edition] 2022-11-22 08:17:58
    \item[Publication type] paperback
    \item[Edition] ed5
    \item[Number of volumes] 5
    \item[Publication ISBN] 978-1-76132-086-6
    \item[Publication URL] https://suttacentral.net/editions/sn/en/sujato
    \item[Source URL] https://github.com/suttacentral/bilara-data/tree/published/translation/en/sujato/sutta/sn
    \item[Publication number] scpub4
\end{description}

\medskip

Published by SuttaCentral

\medskip

\textit{SuttaCentral,\\
c/o Alwis \& Alwis Pty Ltd\\
Kaurna Country,\\
Suite 12,\\
198 Greenhill Road,\\
Eastwood,\\
SA 5063,\\
Australia}

\end{footnotesize}

\newpage

\setlength{\parindent}{1.5em}%%
\tableofcontents
\newpage
\pagestyle{fancy}
%
\mainmatter%
\pagestyle{fancy}%
%
%
\addtocontents{toc}{\let\protect\contentsline\protect\nopagecontentsline}
\part*{Linked Discourses on the Six Sense Fields }
\addcontentsline{toc}{part}{Linked Discourses on the Six Sense Fields }
\markboth{}{}
\addtocontents{toc}{\let\protect\contentsline\protect\oldcontentsline}

%
\addtocontents{toc}{\let\protect\contentsline\protect\nopagecontentsline}
\pannasa{The First Fifty }
\addcontentsline{toc}{pannasa}{The First Fifty }
\markboth{}{}
\addtocontents{toc}{\let\protect\contentsline\protect\oldcontentsline}

%
\addtocontents{toc}{\let\protect\contentsline\protect\nopagecontentsline}
\chapter*{The Chapter on  Impermanence }
\addcontentsline{toc}{chapter}{\tocchapterline{The Chapter on  Impermanence }}
\addtocontents{toc}{\let\protect\contentsline\protect\oldcontentsline}

%
\section*{{\suttatitleacronym SN 35.1}{\suttatitletranslation The Interior as Impermanent }{\suttatitleroot Ajjhattāniccasutta}}
\addcontentsline{toc}{section}{\tocacronym{SN 35.1} \toctranslation{The Interior as Impermanent } \tocroot{Ajjhattāniccasutta}}
\markboth{The Interior as Impermanent }{Ajjhattāniccasutta}
\extramarks{SN 35.1}{SN 35.1}

So\marginnote{1.1} I have heard. At one time the Buddha was staying near \textsanskrit{Sāvatthī} in Jeta’s Grove, \textsanskrit{Anāthapiṇḍika}’s monastery. There the Buddha addressed the mendicants, “Mendicants!” 

“Venerable\marginnote{1.5} sir,” they replied. The Buddha said this: 

“Mendicants,\marginnote{2.1} the eye is impermanent. What’s impermanent is suffering. What’s suffering is not-self. And what’s not-self should be truly seen with right understanding like this: ‘This is not mine, I am not this, this is not my self.’ 

The\marginnote{2.5} ear is impermanent. … 

The\marginnote{2.7} nose is impermanent. … 

The\marginnote{2.9} tongue is impermanent. … 

The\marginnote{2.13} body is impermanent. … 

The\marginnote{2.15} mind is impermanent. What’s impermanent is suffering. What’s suffering is not-self. And what’s not-self should be truly seen with right understanding like this: ‘This is not mine, I am not this, this is not my self.’ 

Seeing\marginnote{2.19} this, a learned noble disciple grows disillusioned with the eye, ear, nose, tongue, body, and mind. Being disillusioned, desire fades away. When desire fades away they’re freed. When they’re freed, they know they’re freed. 

They\marginnote{2.21} understand: ‘Rebirth is ended, the spiritual journey has been completed, what had to be done has been done, there is no return to any state of existence.’” 

%
\section*{{\suttatitleacronym SN 35.2}{\suttatitletranslation The Interior as Suffering }{\suttatitleroot Ajjhattadukkhasutta}}
\addcontentsline{toc}{section}{\tocacronym{SN 35.2} \toctranslation{The Interior as Suffering } \tocroot{Ajjhattadukkhasutta}}
\markboth{The Interior as Suffering }{Ajjhattadukkhasutta}
\extramarks{SN 35.2}{SN 35.2}

“Mendicants,\marginnote{1.1} the eye is suffering. What’s suffering is not-self. And what’s not-self should be truly seen with right understanding like this: ‘This is not mine, I am not this, this is not my self.’ 

The\marginnote{1.4} ear, nose, tongue, body, and mind are suffering. What’s suffering is not-self. And what’s not-self should be truly seen with right understanding like this: ‘This is not mine, I am not this, this is not my self.’ 

Seeing\marginnote{1.11} this … They understand: ‘… there is no return to any state of existence.’” 

%
\section*{{\suttatitleacronym SN 35.3}{\suttatitletranslation The Interior as Not-Self }{\suttatitleroot Ajjhattānattasutta}}
\addcontentsline{toc}{section}{\tocacronym{SN 35.3} \toctranslation{The Interior as Not-Self } \tocroot{Ajjhattānattasutta}}
\markboth{The Interior as Not-Self }{Ajjhattānattasutta}
\extramarks{SN 35.3}{SN 35.3}

“Mendicants,\marginnote{1.1} the eye is not-self. And what’s not-self should be truly seen with right understanding like this: ‘This is not mine, I am not this, this is not my self.’ 

The\marginnote{1.3} ear, nose, tongue, body, and mind are not-self. And what’s not-self should be truly seen with right understanding like this: ‘This is not mine, I am not this, this is not my self.’ 

Seeing\marginnote{1.9} this … They understand: ‘… there is no return to any state of existence.’” 

%
\section*{{\suttatitleacronym SN 35.4}{\suttatitletranslation The Exterior as Impermanent }{\suttatitleroot Bāhirāniccasutta}}
\addcontentsline{toc}{section}{\tocacronym{SN 35.4} \toctranslation{The Exterior as Impermanent } \tocroot{Bāhirāniccasutta}}
\markboth{The Exterior as Impermanent }{Bāhirāniccasutta}
\extramarks{SN 35.4}{SN 35.4}

“Mendicants,\marginnote{1.1} sights are impermanent. What’s impermanent is suffering. What’s suffering is not-self. And what’s not-self should be truly seen with right understanding like this: ‘This is not mine, I am not this, this is not my self.’ 

Sounds,\marginnote{1.5} smells, tastes, touches, and thoughts are impermanent. What’s impermanent is suffering. What’s suffering is not-self. And what’s not-self should be truly seen with right understanding like this: ‘This is not mine, I am not this, this is not my self.’ 

Seeing\marginnote{1.13} this, a learned noble disciple grows disillusioned with sights, sounds, smells, tastes, touches, and thoughts. Being disillusioned, desire fades away. When desire fades away they’re freed. When they’re freed, they know they’re freed. 

They\marginnote{1.15} understand: ‘Rebirth is ended, the spiritual journey has been completed, what had to be done has been done, there is no return to any state of existence.’” 

%
\section*{{\suttatitleacronym SN 35.5}{\suttatitletranslation The Exterior as Suffering }{\suttatitleroot Bāhiradukkhasutta}}
\addcontentsline{toc}{section}{\tocacronym{SN 35.5} \toctranslation{The Exterior as Suffering } \tocroot{Bāhiradukkhasutta}}
\markboth{The Exterior as Suffering }{Bāhiradukkhasutta}
\extramarks{SN 35.5}{SN 35.5}

“Mendicants,\marginnote{1.1} sights are suffering. What’s suffering is not-self. And what’s not-self should be truly seen with right understanding like this: ‘This is not mine, I am not this, this is not my self.’ …” 

%
\section*{{\suttatitleacronym SN 35.6}{\suttatitletranslation The Exterior as Not-Self }{\suttatitleroot Bāhirānattasutta}}
\addcontentsline{toc}{section}{\tocacronym{SN 35.6} \toctranslation{The Exterior as Not-Self } \tocroot{Bāhirānattasutta}}
\markboth{The Exterior as Not-Self }{Bāhirānattasutta}
\extramarks{SN 35.6}{SN 35.6}

“Mendicants,\marginnote{1.1} sights are not-self. And what’s not-self should be truly seen with right understanding like this: ‘This is not mine, I am not this, this is not my self.’ …” 

%
\section*{{\suttatitleacronym SN 35.7}{\suttatitletranslation The Interior as Impermanent in the Three Times }{\suttatitleroot Ajjhattāniccātītānāgatasutta}}
\addcontentsline{toc}{section}{\tocacronym{SN 35.7} \toctranslation{The Interior as Impermanent in the Three Times } \tocroot{Ajjhattāniccātītānāgatasutta}}
\markboth{The Interior as Impermanent in the Three Times }{Ajjhattāniccātītānāgatasutta}
\extramarks{SN 35.7}{SN 35.7}

“Mendicants,\marginnote{1.1} the eye of the past and future is impermanent, let alone the present. 

Seeing\marginnote{1.3} this, a learned noble disciple doesn’t worry about the eye of the past, they don’t look forward to enjoying the eye in the future, and they practice for disillusionment, dispassion, and cessation regarding the eye in the present. 

The\marginnote{1.6} ear … nose … tongue … body … mind of the past and future is impermanent, let alone the present. 

Seeing\marginnote{1.16} this, a learned noble disciple doesn’t worry about the mind of the past, they don’t look forward to enjoying the mind in the future, and they practice for disillusionment, dispassion, and cessation regarding the mind in the present.” 

%
\section*{{\suttatitleacronym SN 35.8}{\suttatitletranslation The Interior as Suffering in the Three Times }{\suttatitleroot Ajjhattadukkhātītānāgatasutta}}
\addcontentsline{toc}{section}{\tocacronym{SN 35.8} \toctranslation{The Interior as Suffering in the Three Times } \tocroot{Ajjhattadukkhātītānāgatasutta}}
\markboth{The Interior as Suffering in the Three Times }{Ajjhattadukkhātītānāgatasutta}
\extramarks{SN 35.8}{SN 35.8}

“Mendicants,\marginnote{1.1} the eye of the past and future is suffering, let alone the present. 

Seeing\marginnote{1.3} this, a learned noble disciple doesn’t worry about the eye of the past, they don’t look forward to enjoying the eye in the future, and they practice for disillusionment, dispassion, and cessation regarding the eye in the present. …” 

%
\section*{{\suttatitleacronym SN 35.9}{\suttatitletranslation The Interior as Not-Self in the Three Times }{\suttatitleroot Ajjhattānattātītānāgatasutta}}
\addcontentsline{toc}{section}{\tocacronym{SN 35.9} \toctranslation{The Interior as Not-Self in the Three Times } \tocroot{Ajjhattānattātītānāgatasutta}}
\markboth{The Interior as Not-Self in the Three Times }{Ajjhattānattātītānāgatasutta}
\extramarks{SN 35.9}{SN 35.9}

“Mendicants,\marginnote{1.1} the eye of the past and future is not-self, let alone the present. 

Seeing\marginnote{1.3} this, a learned noble disciple doesn’t worry about the eye of the past, they don’t look forward to enjoying the eye in the future, and they practice for disillusionment, dispassion, and cessation regarding the eye in the present. …” 

%
\section*{{\suttatitleacronym SN 35.10}{\suttatitletranslation The Exterior as Impermanent in the Three Times }{\suttatitleroot Bāhirāniccātītānāgatasutta}}
\addcontentsline{toc}{section}{\tocacronym{SN 35.10} \toctranslation{The Exterior as Impermanent in the Three Times } \tocroot{Bāhirāniccātītānāgatasutta}}
\markboth{The Exterior as Impermanent in the Three Times }{Bāhirāniccātītānāgatasutta}
\extramarks{SN 35.10}{SN 35.10}

“Mendicants,\marginnote{1.1} sights of the past and future are impermanent, let alone the present. …” 

%
\section*{{\suttatitleacronym SN 35.11}{\suttatitletranslation The Exterior as Suffering in the Three Times }{\suttatitleroot Bāhiradukkhātītānāgatasutta}}
\addcontentsline{toc}{section}{\tocacronym{SN 35.11} \toctranslation{The Exterior as Suffering in the Three Times } \tocroot{Bāhiradukkhātītānāgatasutta}}
\markboth{The Exterior as Suffering in the Three Times }{Bāhiradukkhātītānāgatasutta}
\extramarks{SN 35.11}{SN 35.11}

“Mendicants,\marginnote{1.1} sights of the past and future are suffering, let alone the present. …” 

%
\section*{{\suttatitleacronym SN 35.12}{\suttatitletranslation The Exterior as Not-Self in the Three Times }{\suttatitleroot Bāhirānattātītānāgatasutta}}
\addcontentsline{toc}{section}{\tocacronym{SN 35.12} \toctranslation{The Exterior as Not-Self in the Three Times } \tocroot{Bāhirānattātītānāgatasutta}}
\markboth{The Exterior as Not-Self in the Three Times }{Bāhirānattātītānāgatasutta}
\extramarks{SN 35.12}{SN 35.12}

“Mendicants,\marginnote{1.1} sights of the past and future are not-self, let alone the present. …” 

%
\addtocontents{toc}{\let\protect\contentsline\protect\nopagecontentsline}
\chapter*{The Chapter on Pairs }
\addcontentsline{toc}{chapter}{\tocchapterline{The Chapter on Pairs }}
\addtocontents{toc}{\let\protect\contentsline\protect\oldcontentsline}

%
\section*{{\suttatitleacronym SN 35.13}{\suttatitletranslation Before My Awakening (Interior) }{\suttatitleroot Paṭhamapubbesambodhasutta}}
\addcontentsline{toc}{section}{\tocacronym{SN 35.13} \toctranslation{Before My Awakening (Interior) } \tocroot{Paṭhamapubbesambodhasutta}}
\markboth{Before My Awakening (Interior) }{Paṭhamapubbesambodhasutta}
\extramarks{SN 35.13}{SN 35.13}

At\marginnote{1.1} \textsanskrit{Sāvatthī}. 

“Mendicants,\marginnote{1.2} before my awakening—when I was still unawakened but intent on awakening—I thought: ‘What’s the gratification, the drawback, and the escape when it comes to the eye … ear … nose … tongue … body … and mind?’ 

Then\marginnote{1.9} it occurred to me: ‘The pleasure and happiness that arise from the eye: this is its gratification. That the eye is impermanent, suffering, and perishable: this is its drawback. Removing and giving up desire and greed for the eye: this is its escape. 

The\marginnote{1.13} pleasure and happiness that arise from the ear … nose … tongue … body … mind: this is its gratification. That the mind is impermanent, suffering, and perishable: this is its drawback. Removing and giving up desire and greed for the mind: this is its escape.’ 

As\marginnote{2.1} long as I didn’t truly understand these six interior sense fields’ gratification, drawback, and escape in this way for what they are, I didn’t announce my supreme perfect awakening in this world with its gods, \textsanskrit{Māras}, and \textsanskrit{Brahmās}, this population with its ascetics and brahmins, its gods and humans. 

But\marginnote{2.2} when I did truly understand these six interior sense fields’ gratification, drawback, and escape in this way for what they are, I announced my supreme perfect awakening in this world with its gods, \textsanskrit{Māras}, and \textsanskrit{Brahmās}, this population with its ascetics and brahmins, its gods and humans. 

Knowledge\marginnote{2.3} and vision arose in me: ‘My freedom is unshakable; this is my last rebirth; now there’ll be no more future lives.’” 

%
\section*{{\suttatitleacronym SN 35.14}{\suttatitletranslation Before My Awakening (Exterior) }{\suttatitleroot Dutiyapubbesambodhasutta}}
\addcontentsline{toc}{section}{\tocacronym{SN 35.14} \toctranslation{Before My Awakening (Exterior) } \tocroot{Dutiyapubbesambodhasutta}}
\markboth{Before My Awakening (Exterior) }{Dutiyapubbesambodhasutta}
\extramarks{SN 35.14}{SN 35.14}

“Mendicants,\marginnote{1.1} before my awakening—when I was still unawakened but intent on awakening—I thought: ‘What’s the gratification, the drawback, and the escape when it comes to sights … sounds … smells … tastes … touches … and thoughts?’ …” 

%
\section*{{\suttatitleacronym SN 35.15}{\suttatitletranslation In Search of Gratification (Interior) }{\suttatitleroot Paṭhamaassādapariyesanasutta}}
\addcontentsline{toc}{section}{\tocacronym{SN 35.15} \toctranslation{In Search of Gratification (Interior) } \tocroot{Paṭhamaassādapariyesanasutta}}
\markboth{In Search of Gratification (Interior) }{Paṭhamaassādapariyesanasutta}
\extramarks{SN 35.15}{SN 35.15}

“Mendicants,\marginnote{1.1} I went in search of the eye’s gratification, and I found it. I’ve seen clearly with wisdom the full extent of the eye’s gratification. I went in search of the eye’s drawback, and I found it. I’ve seen clearly with wisdom the full extent of the eye’s drawback. I went in search of escape from the eye, and I found it. I’ve seen clearly with wisdom the full extent of escape from the eye. 

I\marginnote{1.10} went in search of the ear’s … nose’s … tongue’s … body’s … mind’s gratification, and I found it. I’ve seen clearly with wisdom the full extent of the mind’s gratification. I went in search of the mind’s drawback, and I found it. I’ve seen clearly with wisdom the full extent of the mind’s drawback. I went in search of escape from the mind, and I found it. I’ve seen clearly with wisdom the full extent of escape from the mind. 

As\marginnote{2.1} long as I didn’t truly understand these six interior sense fields’ gratification, drawback, and escape for what they are, I didn’t announce my supreme perfect awakening … 

But\marginnote{2.2} when I did truly understand … 

Knowledge\marginnote{2.3} and vision arose in me: ‘My freedom is unshakable; this is my last rebirth; now there’ll be no more future lives.’” 

%
\section*{{\suttatitleacronym SN 35.16}{\suttatitletranslation In Search of Gratification (Exterior) }{\suttatitleroot Dutiyaassādapariyesanasutta}}
\addcontentsline{toc}{section}{\tocacronym{SN 35.16} \toctranslation{In Search of Gratification (Exterior) } \tocroot{Dutiyaassādapariyesanasutta}}
\markboth{In Search of Gratification (Exterior) }{Dutiyaassādapariyesanasutta}
\extramarks{SN 35.16}{SN 35.16}

“Mendicants,\marginnote{1.1} I went in search of the gratification of sights, and I found it. …” 

%
\section*{{\suttatitleacronym SN 35.17}{\suttatitletranslation If There Were No Gratification (Interior) }{\suttatitleroot Paṭhamanoceassādasutta}}
\addcontentsline{toc}{section}{\tocacronym{SN 35.17} \toctranslation{If There Were No Gratification (Interior) } \tocroot{Paṭhamanoceassādasutta}}
\markboth{If There Were No Gratification (Interior) }{Paṭhamanoceassādasutta}
\extramarks{SN 35.17}{SN 35.17}

“Mendicants,\marginnote{1.1} if there were no gratification in the eye, sentient beings wouldn’t be aroused by it. But since there is gratification in the eye, sentient beings do love it. If the eye had no drawback, sentient beings wouldn’t grow disillusioned with it. But since the eye has a drawback, sentient beings do grow disillusioned with it. If there were no escape from the eye, sentient beings wouldn’t escape from it. But since there is an escape from the eye, sentient beings do escape from it. 

If\marginnote{1.7} there were no gratification in the ear … nose … tongue … body … mind, sentient beings wouldn’t be aroused by it. But since there is gratification in the mind, sentient beings do love it. If the mind had no drawback, sentient beings wouldn’t grow disillusioned with it. But since the mind has a drawback, sentient beings do grow disillusioned with it. If there were no escape from the mind, sentient beings wouldn’t escape from it. But since there is an escape from the mind, sentient beings do escape from it. 

As\marginnote{2.1} long as sentient beings don’t truly understand these six interior sense fields’ gratification, drawback, and escape for what they are, they haven’t escaped from this world—with its gods, \textsanskrit{Māras}, and \textsanskrit{Brahmās}, this population with its ascetics and brahmins, its gods and humans—and they don’t live detached, liberated, with a mind free of limits. 

But\marginnote{2.2} when sentient beings truly understand these six interior sense fields’ gratification, drawback, and escape for what they are, they’ve escaped from this world—with its gods, \textsanskrit{Māras}, and \textsanskrit{Brahmās}, this population with its ascetics and brahmins, its gods and humans—and they live detached, liberated, with a mind free of limits.” 

%
\section*{{\suttatitleacronym SN 35.18}{\suttatitletranslation If There Were No Gratification (Exterior) }{\suttatitleroot Dutiyanoceassādasutta}}
\addcontentsline{toc}{section}{\tocacronym{SN 35.18} \toctranslation{If There Were No Gratification (Exterior) } \tocroot{Dutiyanoceassādasutta}}
\markboth{If There Were No Gratification (Exterior) }{Dutiyanoceassādasutta}
\extramarks{SN 35.18}{SN 35.18}

“Mendicants,\marginnote{1.1} if there were no gratification in sights, sentient beings wouldn’t be aroused by them. …” 

%
\section*{{\suttatitleacronym SN 35.19}{\suttatitletranslation Taking Pleasure (Interior) }{\suttatitleroot Paṭhamābhinandasutta}}
\addcontentsline{toc}{section}{\tocacronym{SN 35.19} \toctranslation{Taking Pleasure (Interior) } \tocroot{Paṭhamābhinandasutta}}
\markboth{Taking Pleasure (Interior) }{Paṭhamābhinandasutta}
\extramarks{SN 35.19}{SN 35.19}

“Mendicants,\marginnote{1.1} if you take pleasure in the eye, you take pleasure in suffering. If you take pleasure in suffering, I say you’re not exempt from suffering. 

If\marginnote{1.3} you take pleasure in the ear … nose … tongue … body … mind, you take pleasure in suffering. If you take pleasure in suffering, I say you’re not exempt from suffering. 

If\marginnote{2.1} you don’t take pleasure in the eye, you don’t take pleasure in suffering. If you don’t take pleasure in suffering, I say you’re exempt from suffering. 

If\marginnote{2.3} you don’t take pleasure in the ear … nose … tongue … body … mind, you don’t take pleasure in suffering. If you don’t take pleasure in suffering, I say you’re exempt from suffering.” 

%
\section*{{\suttatitleacronym SN 35.20}{\suttatitletranslation Taking Pleasure (Exterior) }{\suttatitleroot Dutiyābhinandasutta}}
\addcontentsline{toc}{section}{\tocacronym{SN 35.20} \toctranslation{Taking Pleasure (Exterior) } \tocroot{Dutiyābhinandasutta}}
\markboth{Taking Pleasure (Exterior) }{Dutiyābhinandasutta}
\extramarks{SN 35.20}{SN 35.20}

“Mendicants,\marginnote{1.1} if you take pleasure in sights, you take pleasure in suffering. If you take pleasure in suffering, I say you’re not exempt from suffering. …” 

%
\section*{{\suttatitleacronym SN 35.21}{\suttatitletranslation The Arising of Suffering (Interior) }{\suttatitleroot Paṭhamadukkhuppādasutta}}
\addcontentsline{toc}{section}{\tocacronym{SN 35.21} \toctranslation{The Arising of Suffering (Interior) } \tocroot{Paṭhamadukkhuppādasutta}}
\markboth{The Arising of Suffering (Interior) }{Paṭhamadukkhuppādasutta}
\extramarks{SN 35.21}{SN 35.21}

“Mendicants,\marginnote{1.1} the arising, continuation, rebirth, and manifestation of the eye is the arising of suffering, the continuation of diseases, and the manifestation of old age and death. The arising, continuation, rebirth, and manifestation of the ear … nose … tongue … body … and mind is the arising of suffering, the continuation of diseases, and the manifestation of old age and death. 

The\marginnote{2.1} cessation, settling, and ending of the eye is the cessation of suffering, the settling of diseases, and the ending of old age and death. The cessation, settling, and ending of the ear, nose, tongue, body, and mind is the cessation of suffering, the settling of diseases, and the ending of old age and death.” 

%
\section*{{\suttatitleacronym SN 35.22}{\suttatitletranslation The Arising of Suffering (Exterior) }{\suttatitleroot Dutiyadukkhuppādasutta}}
\addcontentsline{toc}{section}{\tocacronym{SN 35.22} \toctranslation{The Arising of Suffering (Exterior) } \tocroot{Dutiyadukkhuppādasutta}}
\markboth{The Arising of Suffering (Exterior) }{Dutiyadukkhuppādasutta}
\extramarks{SN 35.22}{SN 35.22}

“Mendicants,\marginnote{1.1} the arising, continuation, rebirth, and manifestation of sights is the arising of suffering, the continuation of diseases, and the manifestation of old age and death. The arising, continuation, rebirth, and manifestation of sounds, smells, tastes, touches, and thoughts is the arising of suffering, the continuation of diseases, and the manifestation of old age and death. 

The\marginnote{2.1} cessation, settling, and ending of sights, sounds, smells, tastes, touches, and thoughts is the cessation of suffering, the settling of diseases, and the ending of old age and death.” 

%
\addtocontents{toc}{\let\protect\contentsline\protect\nopagecontentsline}
\chapter*{The Chapter on the All }
\addcontentsline{toc}{chapter}{\tocchapterline{The Chapter on the All }}
\addtocontents{toc}{\let\protect\contentsline\protect\oldcontentsline}

%
\section*{{\suttatitleacronym SN 35.23}{\suttatitletranslation All }{\suttatitleroot Sabbasutta}}
\addcontentsline{toc}{section}{\tocacronym{SN 35.23} \toctranslation{All } \tocroot{Sabbasutta}}
\markboth{All }{Sabbasutta}
\extramarks{SN 35.23}{SN 35.23}

At\marginnote{1.1} \textsanskrit{Sāvatthī}. 

“Mendicants,\marginnote{1.2} I will teach you the all. Listen … 

And\marginnote{1.4} what is the all? It’s just the eye and sights, the ear and sounds, the nose and smells, the tongue and tastes, the body and touches, and the mind and thoughts. This is called the all. 

Mendicants,\marginnote{1.7} suppose someone was to say: ‘I’ll reject this all and describe another all.’ They’d have no grounds for that, they’d be stumped by questions, and, in addition, they’d get frustrated. Why is that? Because they’re out of their element.” 

%
\section*{{\suttatitleacronym SN 35.24}{\suttatitletranslation Giving Up }{\suttatitleroot Pahānasutta}}
\addcontentsline{toc}{section}{\tocacronym{SN 35.24} \toctranslation{Giving Up } \tocroot{Pahānasutta}}
\markboth{Giving Up }{Pahānasutta}
\extramarks{SN 35.24}{SN 35.24}

“Mendicants,\marginnote{1.1} I will teach you the principle for giving up the all. Listen … 

And\marginnote{1.3} what is the principle for giving up the all? The eye should be given up. Sights should be given up. Eye consciousness should be given up. Eye contact should be given up. The painful, pleasant, or neutral feeling that arises conditioned by eye contact should also be given up. 

The\marginnote{1.5} ear … nose … tongue … body … mind should be given up. Thoughts should be given up. Mind consciousness should be given up. Mind contact should be given up. The painful, pleasant, or neutral feeling that arises conditioned by mind contact should be given up. 

This\marginnote{1.9} is the principle for giving up the all.” 

%
\section*{{\suttatitleacronym SN 35.25}{\suttatitletranslation Giving Up By Direct Knowledge and Complete Understanding }{\suttatitleroot Abhiññāpariññāpahānasutta}}
\addcontentsline{toc}{section}{\tocacronym{SN 35.25} \toctranslation{Giving Up By Direct Knowledge and Complete Understanding } \tocroot{Abhiññāpariññāpahānasutta}}
\markboth{Giving Up By Direct Knowledge and Complete Understanding }{Abhiññāpariññāpahānasutta}
\extramarks{SN 35.25}{SN 35.25}

“Mendicants,\marginnote{1.1} I will teach you the principle for giving up the all by direct knowledge and complete understanding. Listen … 

And\marginnote{1.3} what is the principle for giving up the all by direct knowledge and complete understanding? The eye should be given up by direct knowledge and complete understanding. Sights should be given up by direct knowledge and complete understanding. Eye consciousness should be given up by direct knowledge and complete understanding. Eye contact should be given up by direct knowledge and complete understanding. The painful, pleasant, or neutral feeling that arises conditioned by eye contact should be given up by direct knowledge and complete understanding. 

The\marginnote{1.5} ear … nose … tongue … body … mind should be given up by direct knowledge and complete understanding. Thoughts should be given up by direct knowledge and complete understanding. Mind consciousness should be given up by direct knowledge and complete understanding. Mind contact should be given up by direct knowledge and complete understanding. The painful, pleasant, or neutral feeling that arises conditioned by mind contact should be given up by direct knowledge and complete understanding. 

This\marginnote{1.9} is the principle for giving up the all by direct knowledge and complete understanding.” 

%
\section*{{\suttatitleacronym SN 35.26}{\suttatitletranslation Without Completely Understanding (1st) }{\suttatitleroot Paṭhamaaparijānanasutta}}
\addcontentsline{toc}{section}{\tocacronym{SN 35.26} \toctranslation{Without Completely Understanding (1st) } \tocroot{Paṭhamaaparijānanasutta}}
\markboth{Without Completely Understanding (1st) }{Paṭhamaaparijānanasutta}
\extramarks{SN 35.26}{SN 35.26}

“Mendicants,\marginnote{1.1} without directly knowing and completely understanding the all, without dispassion for it and giving it up, you can’t end suffering. And what is the all, without directly knowing and completely understanding which, without dispassion for it and giving it up, you can’t end suffering? 

Without\marginnote{1.3} directly knowing and completely understanding the eye, without dispassion for it and giving it up, you can’t end suffering. Without directly knowing and completely understanding sights … eye consciousness … eye contact … painful, pleasant, or neutral feeling that arises conditioned by eye contact, without dispassion for it and giving it up, you can’t end suffering. 

Without\marginnote{1.8} directly knowing and completely understanding the ear … the nose … the tongue … the body … the mind, without dispassion for it and giving it up, you can’t end suffering. Without directly knowing and completely understanding thoughts … mind consciousness … mind contact … painful, pleasant, or neutral feeling that arises conditioned by mind contact, without dispassion for it and giving it up, you can’t end suffering. 

This\marginnote{1.19} is the all, without directly knowing and completely understanding which, without dispassion for it and giving it up, you can’t end suffering. 

By\marginnote{2.1} directly knowing and completely understanding the all, having dispassion for it and giving it up, you can end suffering. And what is the all, directly knowing and completely understanding which, having dispassion for it and giving it up, you can end suffering? 

By\marginnote{2.3} directly knowing and completely understanding the eye … the ear … the nose … the tongue … the body … the mind, having dispassion for it and giving it up, you can end suffering. By directly knowing and completely understanding thoughts … mind consciousness … mind contact … painful, pleasant, or neutral feeling that arises conditioned by mind contact, having dispassion for it and giving it up, you can end suffering. 

This\marginnote{2.19} is the all, directly knowing and completely understanding which, having dispassion for it and giving it up, you can end suffering.” 

%
\section*{{\suttatitleacronym SN 35.27}{\suttatitletranslation Without Completely Understanding (2nd) }{\suttatitleroot Dutiyaaparijānanasutta}}
\addcontentsline{toc}{section}{\tocacronym{SN 35.27} \toctranslation{Without Completely Understanding (2nd) } \tocroot{Dutiyaaparijānanasutta}}
\markboth{Without Completely Understanding (2nd) }{Dutiyaaparijānanasutta}
\extramarks{SN 35.27}{SN 35.27}

“Mendicants,\marginnote{1.1} without directly knowing and completely understanding the all, without dispassion for it and giving it up, you can’t end suffering. And what is the all, without directly knowing and completely understanding which, without dispassion for it and giving it up, you can’t end suffering? 

The\marginnote{1.3} eye, sights, eye consciousness, and things known by eye consciousness. 

The\marginnote{1.4} ear … nose … tongue … body … 

The\marginnote{1.6} mind, thoughts, mind consciousness, and things known by mind consciousness. 

This\marginnote{1.7} is the all, without directly knowing and completely understanding which, without dispassion for it and giving it up, you can’t end suffering. 

By\marginnote{2.1} directly knowing and completely understanding the all, having dispassion for it and giving it up, you can end suffering. And what is the all, directly knowing and completely understanding which, having dispassion for it and giving it up, you can end suffering? 

The\marginnote{2.3} eye, sights, eye consciousness, and things known by eye consciousness. 

The\marginnote{2.4} ear … nose … tongue … body … 

The\marginnote{2.6} mind, thoughts, mind consciousness, and things known by mind consciousness. 

This\marginnote{2.7} is the all, directly knowing and completely understanding which, having dispassion for it and giving it up, you can end suffering.” 

%
\section*{{\suttatitleacronym SN 35.28}{\suttatitletranslation Burning }{\suttatitleroot Ādittasutta}}
\addcontentsline{toc}{section}{\tocacronym{SN 35.28} \toctranslation{Burning } \tocroot{Ādittasutta}}
\markboth{Burning }{Ādittasutta}
\extramarks{SN 35.28}{SN 35.28}

At\marginnote{1.1} one time the Buddha was staying near \textsanskrit{Gayā} on \textsanskrit{Gayā} Head together with a thousand mendicants. There the Buddha addressed the mendicants: 

“Mendicants,\marginnote{1.3} all is burning. And what is the all that is burning? 

The\marginnote{1.5} eye is burning. Sights are burning. Eye consciousness is burning. Eye contact is burning. The painful, pleasant, or neutral feeling that arises conditioned by eye contact is also burning. Burning with what? Burning with the fires of greed, hate, and delusion. Burning with rebirth, old age, and death, with sorrow, lamentation, pain, sadness, and distress. 

The\marginnote{1.8} ear … nose … tongue … body … 

The\marginnote{1.11} mind is burning. Thoughts are burning. Mind consciousness is burning. Mind contact is burning. The painful, pleasant, or neutral feeling that arises conditioned by mind contact is also burning. Burning with what? Burning with the fires of greed, hate, and delusion. Burning with rebirth, old age, and death, with sorrow, lamentation, pain, sadness, and distress, I say. 

Seeing\marginnote{1.14} this, a learned noble disciple grows disillusioned with the eye, sights, eye consciousness, and eye contact. And they grow disillusioned with the painful, pleasant, or neutral feeling that arises conditioned by eye contact. 

They\marginnote{1.15} grow disillusioned with the ear … nose … tongue … body … mind … painful, pleasant, or neutral feeling that arises conditioned by mind contact. 

Being\marginnote{1.16} disillusioned, desire fades away. When desire fades away they’re freed. When they’re freed, they know they’re freed. 

They\marginnote{1.17} understand: ‘Rebirth is ended, the spiritual journey has been completed, what had to be done has been done, there is no return to any state of existence.’” 

That\marginnote{2.1} is what the Buddha said. Satisfied, the mendicants were happy with what the Buddha said. And while this discourse was being spoken, the minds of the thousand mendicants were freed from defilements by not grasping. 

%
\section*{{\suttatitleacronym SN 35.29}{\suttatitletranslation Oppressed }{\suttatitleroot Addhabhūtasutta}}
\addcontentsline{toc}{section}{\tocacronym{SN 35.29} \toctranslation{Oppressed } \tocroot{Addhabhūtasutta}}
\markboth{Oppressed }{Addhabhūtasutta}
\extramarks{SN 35.29}{SN 35.29}

\scevam{So\marginnote{1.1} I have heard. }At one time the Buddha was staying near \textsanskrit{Rājagaha}, in the Bamboo Grove, the squirrels’ feeding ground. There the Buddha addressed the mendicants: 

“Mendicants,\marginnote{1.4} all is oppressed. And what is the all that is oppressed? 

The\marginnote{1.6} eye is oppressed. Sights are oppressed. Eye consciousness is oppressed. Eye contact is oppressed. The painful, pleasant, or neutral feeling that arises conditioned by eye contact is also oppressed. Oppressed by what? Oppressed by the fires of greed, hate, and delusion. Oppressed by rebirth, old age, and death, by sorrow, lamentation, pain, sadness, and distress, I say. 

The\marginnote{1.9} ear … nose … tongue … body … mind is oppressed. Thoughts are oppressed. Mind consciousness is oppressed. Mind contact is oppressed. The painful, pleasant, or neutral feeling that arises conditioned by mind contact is also oppressed. Oppressed by what? Oppressed by greed, hate, and delusion. Oppressed by rebirth, old age, and death, by sorrow, lamentation, pain, sadness, and distress, I say. 

Seeing\marginnote{1.16} this, a learned noble disciple grows disillusioned with the eye, sights, eye consciousness, and eye contact. And they grow disillusioned with the painful, pleasant, or neutral feeling that arises conditioned by eye contact. 

They\marginnote{1.17} grow disillusioned with the ear … nose … tongue … body … mind … painful, pleasant, or neutral feeling that arises conditioned by mind contact. Being disillusioned, desire fades away. When desire fades away they’re freed. When they’re freed, they know they’re freed. 

They\marginnote{1.19} understand: ‘Rebirth is ended, the spiritual journey has been completed, what had to be done has been done, there is no return to any state of existence.’” 

%
\section*{{\suttatitleacronym SN 35.30}{\suttatitletranslation The Practice Fit for Uprooting }{\suttatitleroot Samugghātasāruppasutta}}
\addcontentsline{toc}{section}{\tocacronym{SN 35.30} \toctranslation{The Practice Fit for Uprooting } \tocroot{Samugghātasāruppasutta}}
\markboth{The Practice Fit for Uprooting }{Samugghātasāruppasutta}
\extramarks{SN 35.30}{SN 35.30}

“Mendicants,\marginnote{1.1} I will teach you the practice fit for uprooting all identifying. Listen and pay close attention, I will speak. … 

And\marginnote{1.3} what is the practice fit for uprooting all identifying? 

It’s\marginnote{1.4} when a mendicant does not identify with the eye, does not identify regarding the eye, does not identify as the eye, and does not identify ‘the eye is mine.’ They don’t identify with sights, they don’t identify regarding sights, they don’t identify as sights, and they don’t identify ‘sights are mine.’ They don’t identify with eye consciousness … eye contact … They don’t identify with the pleasant, painful, or neutral feeling that arises conditioned by eye contact. They don’t identify regarding that, they don’t identify as that, and they don’t identify ‘that is mine.’ 

They\marginnote{1.9} don’t identify with the ear … nose … tongue … body … mind … They don’t identify with the pleasant, painful, or neutral feeling that arises conditioned by mind contact. They don’t identify regarding that, they don’t identify as that, and they don’t identify ‘that is mine.’ 

They\marginnote{1.19} don’t identify with all, they don’t identify regarding all, they don’t identify as all, and they don’t identify ‘all is mine.’ Not identifying, they don’t grasp at anything in the world. Not grasping, they’re not anxious. Not being anxious, they personally become extinguished. 

They\marginnote{1.22} understand: ‘Rebirth is ended, the spiritual journey has been completed, what had to be done has been done, there is no return to any state of existence.’ 

This\marginnote{1.23} is the practice fit for uprooting all identifying.” 

%
\section*{{\suttatitleacronym SN 35.31}{\suttatitletranslation The Practice Conducive to Uprooting (1st) }{\suttatitleroot Paṭhamasamugghātasappāyasutta}}
\addcontentsline{toc}{section}{\tocacronym{SN 35.31} \toctranslation{The Practice Conducive to Uprooting (1st) } \tocroot{Paṭhamasamugghātasappāyasutta}}
\markboth{The Practice Conducive to Uprooting (1st) }{Paṭhamasamugghātasappāyasutta}
\extramarks{SN 35.31}{SN 35.31}

“Mendicants,\marginnote{1.1} I will teach you the practice that’s conducive to uprooting all identifying. Listen … 

And\marginnote{1.3} what is the practice that’s conducive to uprooting all identifying? It’s when a mendicant does not identify with the eye, does not identify in the eye, does not identify from the eye, and does not identify: ‘The eye is mine.’ They don’t identify with sights … eye consciousness … eye contact. And they don’t identify with the pleasant, painful, or neutral feeling that arises conditioned by eye contact. They don’t identify in that, they don’t identify from that, and they don’t identify: ‘That is mine.’ 

For\marginnote{1.7} whatever you identify with, whatever you identify in, whatever you identify as, and whatever you identify as ‘mine’: that becomes something else. The world is attached to being, taking pleasure only in being, yet it becomes something else. 

They\marginnote{1.9} don’t identify with the ear … nose … tongue … body … mind … They don’t identify with the pleasant, painful, or neutral feeling that arises conditioned by mind contact. They don’t identify in that, they don’t identify from that, and they don’t identify: ‘That is mine.’ 

For\marginnote{1.19} whatever you identify with, whatever you identify in, whatever you identify as, and whatever you identify as ‘mine’: that becomes something else. The world is attached to being, taking pleasure only in being, yet it becomes something else. 

As\marginnote{1.21} far as the aggregates, elements, and sense fields extend, they don’t identify with that, they don’t identify in that, they don’t identify from that, and they don’t identify: ‘That is mine.’ Not identifying, they don’t grasp at anything in the world. Not grasping, they’re not anxious. Not being anxious, they personally become extinguished. 

They\marginnote{1.24} understand: ‘Rebirth is ended, the spiritual journey has been completed, what had to be done has been done, there is no return to any state of existence.’ 

This\marginnote{1.25} is the practice that’s conducive to uprooting all identifying.” 

%
\section*{{\suttatitleacronym SN 35.32}{\suttatitletranslation The Practice Conducive to Uprooting (2nd) }{\suttatitleroot Dutiyasamugghātasappāyasutta}}
\addcontentsline{toc}{section}{\tocacronym{SN 35.32} \toctranslation{The Practice Conducive to Uprooting (2nd) } \tocroot{Dutiyasamugghātasappāyasutta}}
\markboth{The Practice Conducive to Uprooting (2nd) }{Dutiyasamugghātasappāyasutta}
\extramarks{SN 35.32}{SN 35.32}

“Mendicants,\marginnote{1.1} I will teach you the practice that’s conducive to uprooting all identifying. Listen … 

And\marginnote{1.3} what is the practice that’s conducive to uprooting all identifying? 

What\marginnote{2.1} do you think, mendicants? Is the eye permanent or impermanent?” 

“Impermanent,\marginnote{3.1} sir.” 

“But\marginnote{4.1} if it’s impermanent, is it suffering or happiness?” 

“Suffering,\marginnote{5.1} sir.” 

“But\marginnote{6.1} if it’s impermanent, suffering, and liable to wear out, is it fit to be regarded thus: ‘This is mine, I am this, this is my self’?” 

“No,\marginnote{7.1} sir.” 

“Are\marginnote{8.1} sights … eye consciousness … eye contact … 

The\marginnote{10.1} pleasant, painful, or neutral feeling that arises conditioned by eye contact: is that permanent or impermanent?” 

“Impermanent,\marginnote{11.1} sir.” 

“But\marginnote{12.1} if it’s impermanent, is it suffering or happiness?” 

“Suffering,\marginnote{13.1} sir.” 

“But\marginnote{14.1} if it’s impermanent, suffering, and liable to wear out, is it fit to be regarded thus: ‘This is mine, I am this, this is my self’?” 

“No,\marginnote{15.1} sir.” … 

“Is\marginnote{16.1} the ear … nose … tongue … 

body\marginnote{20.1} … mind … 

The\marginnote{22.1} pleasant, painful, or neutral feeling that arises conditioned by mind contact: is that permanent or impermanent?” 

“Impermanent,\marginnote{23.1} sir.” 

“But\marginnote{24.1} if it’s impermanent, is it suffering or happiness?” 

“Suffering,\marginnote{25.1} sir.” 

“But\marginnote{26.1} if it’s impermanent, suffering, and liable to wear out, is it fit to be regarded thus: ‘This is mine, I am this, this is my self’?” 

“No,\marginnote{27.1} sir.” 

“Seeing\marginnote{28.1} this, a learned noble disciple grows disillusioned with the eye, sights, eye consciousness, and eye contact. And they grow disillusioned with the painful, pleasant, or neutral feeling that arises conditioned by eye contact. 

They\marginnote{28.3} grow disillusioned with the ear … nose … tongue … body … They grow disillusioned with the mind, thoughts, mind consciousness, and mind contact. And they grow disillusioned with the painful, pleasant, or neutral feeling that arises conditioned by mind contact. 

Being\marginnote{28.6} disillusioned, desire fades away. When desire fades away they’re freed. When they’re freed, they know they’re freed. 

They\marginnote{28.7} understand: ‘Rebirth is ended, the spiritual journey has been completed, what had to be done has been done, there is no return to any state of existence.’ This is the practice that’s conducive to uprooting all identifying.” 

%
\addtocontents{toc}{\let\protect\contentsline\protect\nopagecontentsline}
\chapter*{The Chapter on Liable to Be Reborn }
\addcontentsline{toc}{chapter}{\tocchapterline{The Chapter on Liable to Be Reborn }}
\addtocontents{toc}{\let\protect\contentsline\protect\oldcontentsline}

%
\section*{{\suttatitleacronym SN 35.33–42}{\suttatitletranslation Ten on Liable to Be Reborn, Etc. }{\suttatitleroot Jātidhammāsutta}}
\addcontentsline{toc}{section}{\tocacronym{SN 35.33–42} \toctranslation{Ten on Liable to Be Reborn, Etc. } \tocroot{Jātidhammāsutta}}
\markboth{Ten on Liable to Be Reborn, Etc. }{Jātidhammāsutta}
\extramarks{SN 35.33–42}{SN 35.33–42}

At\marginnote{1.1} \textsanskrit{Sāvatthī}. 

“Mendicants,\marginnote{1.3} all is liable to be reborn. And what is the all that is liable to be reborn? The eye, sights, eye consciousness, and eye contact are liable to be reborn. And the pleasant, painful, or neutral feeling that arises conditioned by eye contact is also liable to be reborn. 

The\marginnote{1.10} ear … nose … tongue … body … The mind, thoughts, mind consciousness, and mind contact are liable to be reborn. And the pleasant, painful, or neutral feeling that arises conditioned by mind contact is also liable to be reborn. 

Seeing\marginnote{1.18} this a learned noble disciple grows disillusioned … They understand: ‘… there is no return to any state of existence.’” 

“Mendicants,\marginnote{1.1} all is liable to grow old. …” 

“Mendicants,\marginnote{1.1} all is liable to fall sick. …” 

“Mendicants,\marginnote{1.1} all is liable to die. …” 

“Mendicants,\marginnote{1.1} all is liable to sorrow. …” 

“Mendicants,\marginnote{1.1} all is liable to be corrupted. …” 

“Mendicants,\marginnote{1.1} all is liable to end. …” 

“Mendicants,\marginnote{1.1} all is liable to vanish. …” 

“Mendicants,\marginnote{1.1} all is liable to originate. …” 

“Mendicants,\marginnote{1.1} all is liable to cease. …” 

%
\addtocontents{toc}{\let\protect\contentsline\protect\nopagecontentsline}
\chapter*{The Chapter on All is Impermanent }
\addcontentsline{toc}{chapter}{\tocchapterline{The Chapter on All is Impermanent }}
\addtocontents{toc}{\let\protect\contentsline\protect\oldcontentsline}

%
\section*{{\suttatitleacronym SN 35.43–51}{\suttatitletranslation Nine on Impermanence, Etc. }{\suttatitleroot Aniccādisuttanavaka}}
\addcontentsline{toc}{section}{\tocacronym{SN 35.43–51} \toctranslation{Nine on Impermanence, Etc. } \tocroot{Aniccādisuttanavaka}}
\markboth{Nine on Impermanence, Etc. }{Aniccādisuttanavaka}
\extramarks{SN 35.43–51}{SN 35.43–51}

At\marginnote{1.1} \textsanskrit{Sāvatthī}. 

“Mendicants,\marginnote{1.3} all is impermanent. And what is the all that is impermanent? The eye, sights, eye consciousness, and eye contact are impermanent. And the pleasant, painful, or neutral feeling that arises conditioned by eye contact is also impermanent. 

The\marginnote{1.7} ear … nose … tongue … body … The mind, thoughts, mind consciousness, and mind contact are impermanent. The painful, pleasant, or neutral feeling that arises conditioned by mind contact is also impermanent. 

Seeing\marginnote{1.12} this, a learned noble disciple grows disillusioned … 

They\marginnote{1.17} understand: ‘Rebirth is ended, the spiritual journey has been completed, what had to be done has been done, there is no return to any state of existence.’” 

“Mendicants,\marginnote{1.1} all is suffering. …” 

“Mendicants,\marginnote{1.1} all is not-self. …” 

“Mendicants,\marginnote{1.1} all is to be directly known. …” 

“Mendicants,\marginnote{1.1} all is to be completely understood. …” 

“Mendicants,\marginnote{1.1} all is to be given up. …” 

“Mendicants,\marginnote{1.1} all is to be realized. …” 

“Mendicants,\marginnote{1.1} all is to be directly known and completely understood. …” 

“Mendicants,\marginnote{1.1} all is troubled. …” 

%
\section*{{\suttatitleacronym SN 35.52}{\suttatitletranslation Disturbed }{\suttatitleroot Upassaṭṭhasutta}}
\addcontentsline{toc}{section}{\tocacronym{SN 35.52} \toctranslation{Disturbed } \tocroot{Upassaṭṭhasutta}}
\markboth{Disturbed }{Upassaṭṭhasutta}
\extramarks{SN 35.52}{SN 35.52}

“Mendicants,\marginnote{1.1} all is disturbed. And what is the all that is disturbed? The eye, sights, eye consciousness, and eye contact are disturbed. And the pleasant, painful, or neutral feeling that arises conditioned by eye contact is also disturbed. 

The\marginnote{1.5} ear … nose … tongue … body … 

The\marginnote{1.8} mind, thoughts, mind consciousness, and mind contact are disturbed. And the pleasant, painful, or neutral feeling that arises conditioned by mind contact is also disturbed. 

Seeing\marginnote{1.10} this, a learned noble disciple grows disillusioned … 

They\marginnote{1.15} understand: ‘Rebirth is ended, the spiritual journey has been completed, what had to be done has been done, there is no return to any state of existence.’” 

%
\addtocontents{toc}{\let\protect\contentsline\protect\nopagecontentsline}
\pannasa{The Second Fifty }
\addcontentsline{toc}{pannasa}{The Second Fifty }
\markboth{}{}
\addtocontents{toc}{\let\protect\contentsline\protect\oldcontentsline}

%
\addtocontents{toc}{\let\protect\contentsline\protect\nopagecontentsline}
\chapter*{The Chapter on Ignorance }
\addcontentsline{toc}{chapter}{\tocchapterline{The Chapter on Ignorance }}
\addtocontents{toc}{\let\protect\contentsline\protect\oldcontentsline}

%
\section*{{\suttatitleacronym SN 35.53}{\suttatitletranslation Giving Up Ignorance }{\suttatitleroot Avijjāpahānasutta}}
\addcontentsline{toc}{section}{\tocacronym{SN 35.53} \toctranslation{Giving Up Ignorance } \tocroot{Avijjāpahānasutta}}
\markboth{Giving Up Ignorance }{Avijjāpahānasutta}
\extramarks{SN 35.53}{SN 35.53}

At\marginnote{1.1} \textsanskrit{Sāvatthī}. 

Then\marginnote{1.2} a mendicant went up to the Buddha, bowed, sat down to one side, and said to him: 

“Sir,\marginnote{1.3} how does one know and see so as to give up ignorance and give rise to knowledge?” 

“Mendicant,\marginnote{2.1} knowing and seeing the eye, sights, eye consciousness, and eye contact as impermanent, ignorance is given up and knowledge arises. And also knowing and seeing the pleasant, painful, or neutral feeling that arises conditioned by eye contact as impermanent, ignorance is given up and knowledge arises. 

Knowing\marginnote{2.6} and seeing the ear … nose … tongue … body … 

Knowing\marginnote{2.10} and seeing the mind, thoughts, mind consciousness, and mind contact as impermanent, ignorance is given up and knowledge arises. And also knowing and seeing the pleasant, painful, or neutral feeling that arises conditioned by mind contact as impermanent, ignorance is given up and knowledge arises. 

That’s\marginnote{2.15} how to know and see so as to give up ignorance and give rise to knowledge.” 

%
\section*{{\suttatitleacronym SN 35.54}{\suttatitletranslation Giving Up Fetters }{\suttatitleroot Saṁyojanappahānasutta}}
\addcontentsline{toc}{section}{\tocacronym{SN 35.54} \toctranslation{Giving Up Fetters } \tocroot{Saṁyojanappahānasutta}}
\markboth{Giving Up Fetters }{Saṁyojanappahānasutta}
\extramarks{SN 35.54}{SN 35.54}

“Sir,\marginnote{1.1} how does one know and see so that the fetters are given up?” 

“Mendicant,\marginnote{1.2} knowing and seeing the eye as impermanent, the fetters are given up …” 

%
\section*{{\suttatitleacronym SN 35.55}{\suttatitletranslation Uprooting the Fetters }{\suttatitleroot Saṁyojanasamugghātasutta}}
\addcontentsline{toc}{section}{\tocacronym{SN 35.55} \toctranslation{Uprooting the Fetters } \tocroot{Saṁyojanasamugghātasutta}}
\markboth{Uprooting the Fetters }{Saṁyojanasamugghātasutta}
\extramarks{SN 35.55}{SN 35.55}

“Sir,\marginnote{1.1} how does one know and see so that the fetters are uprooted?” 

“Mendicant,\marginnote{1.2} knowing and seeing the eye as not-self, the fetters are uprooted …” 

%
\section*{{\suttatitleacronym SN 35.56}{\suttatitletranslation Giving Up Defilements }{\suttatitleroot Āsavapahānasutta}}
\addcontentsline{toc}{section}{\tocacronym{SN 35.56} \toctranslation{Giving Up Defilements } \tocroot{Āsavapahānasutta}}
\markboth{Giving Up Defilements }{Āsavapahānasutta}
\extramarks{SN 35.56}{SN 35.56}

“Sir,\marginnote{1.1} how does one know and see so that the defilements are given up?” … 

%
\section*{{\suttatitleacronym SN 35.57}{\suttatitletranslation Uprooting Defilements }{\suttatitleroot Āsavasamugghātasutta}}
\addcontentsline{toc}{section}{\tocacronym{SN 35.57} \toctranslation{Uprooting Defilements } \tocroot{Āsavasamugghātasutta}}
\markboth{Uprooting Defilements }{Āsavasamugghātasutta}
\extramarks{SN 35.57}{SN 35.57}

“Sir,\marginnote{1.1} how does one know and see so that the defilements are uprooted?” … 

%
\section*{{\suttatitleacronym SN 35.58}{\suttatitletranslation Giving Up Tendencies }{\suttatitleroot Anusayapahānasutta}}
\addcontentsline{toc}{section}{\tocacronym{SN 35.58} \toctranslation{Giving Up Tendencies } \tocroot{Anusayapahānasutta}}
\markboth{Giving Up Tendencies }{Anusayapahānasutta}
\extramarks{SN 35.58}{SN 35.58}

“Sir,\marginnote{1.1} how does one know and see so that the underlying tendencies are given up?” … 

%
\section*{{\suttatitleacronym SN 35.59}{\suttatitletranslation Uprooting Tendencies }{\suttatitleroot Anusayasamugghātasutta}}
\addcontentsline{toc}{section}{\tocacronym{SN 35.59} \toctranslation{Uprooting Tendencies } \tocroot{Anusayasamugghātasutta}}
\markboth{Uprooting Tendencies }{Anusayasamugghātasutta}
\extramarks{SN 35.59}{SN 35.59}

“Sir,\marginnote{1.1} how does one know and see so that the underlying tendencies are uprooted?” … 

%
\section*{{\suttatitleacronym SN 35.60}{\suttatitletranslation The Complete Understanding of All Grasping }{\suttatitleroot Sabbupādānapariññāsutta}}
\addcontentsline{toc}{section}{\tocacronym{SN 35.60} \toctranslation{The Complete Understanding of All Grasping } \tocroot{Sabbupādānapariññāsutta}}
\markboth{The Complete Understanding of All Grasping }{Sabbupādānapariññāsutta}
\extramarks{SN 35.60}{SN 35.60}

“Mendicants,\marginnote{1.1} I will teach you the principle for the complete understanding of all grasping. Listen … 

And\marginnote{1.3} what is the principle for the complete understanding of all grasping? 

Eye\marginnote{1.4} consciousness arises dependent on the eye and sights. The meeting of the three is contact. Contact is a condition for feeling. 

Seeing\marginnote{1.6} this, a learned noble disciple grows disillusioned with the eye, sights, eye consciousness, eye contact, and feeling. Being disillusioned, desire fades away. When desire fades away they’re freed. When they are released, they understand: ‘I have completely understood grasping.’ 

Ear\marginnote{1.8} consciousness arises dependent on the ear and sounds. … 

Nose\marginnote{1.9} consciousness arises dependent on the nose and smells. … 

Tongue\marginnote{1.10} consciousness arises dependent on the tongue and tastes. … 

Body\marginnote{1.11} consciousness arises dependent on the body and touches. … 

Mind\marginnote{1.12} consciousness arises dependent on the mind and thoughts. The meeting of the three is contact. Contact is a condition for feeling. 

Seeing\marginnote{1.14} this, a learned noble disciple grows disillusioned with the mind, thoughts, mind consciousness, mind contact, and feeling. Being disillusioned, desire fades away. When desire fades away they’re freed. When they are released, they understand: ‘I have completely understood grasping.’ 

This\marginnote{1.16} is the principle for the complete understanding of all grasping.” 

%
\section*{{\suttatitleacronym SN 35.61}{\suttatitletranslation The Depletion of All Fuel (1st) }{\suttatitleroot Paṭhamasabbupādānapariyādānasutta}}
\addcontentsline{toc}{section}{\tocacronym{SN 35.61} \toctranslation{The Depletion of All Fuel (1st) } \tocroot{Paṭhamasabbupādānapariyādānasutta}}
\markboth{The Depletion of All Fuel (1st) }{Paṭhamasabbupādānapariyādānasutta}
\extramarks{SN 35.61}{SN 35.61}

“Mendicants,\marginnote{1.1} I will teach you the principle for depleting all fuel. Listen … 

And\marginnote{1.3} what is the principle for depleting all fuel? 

Eye\marginnote{1.4} consciousness arises dependent on the eye and sights. The meeting of the three is contact. Contact is a condition for feeling. 

Seeing\marginnote{1.6} this, a learned noble disciple grows disillusioned with the eye, sights, eye consciousness, eye contact, and feeling. Being disillusioned, desire fades away. When desire fades away they’re freed. When they are released, they understand: ‘I have completely depleted grasping.’ 

Ear\marginnote{1.8} … nose … tongue … body … 

Mind\marginnote{1.9} consciousness arises dependent on the mind and thoughts. The meeting of the three is contact. Contact is a condition for feeling. 

Seeing\marginnote{1.11} this, a learned noble disciple grows disillusioned with the mind, thoughts, mind consciousness, mind contact, and feeling. Being disillusioned, desire fades away. When desire fades away they’re freed. When they are released, they understand: ‘I have completely depleted grasping.’ 

This\marginnote{1.13} is the principle for depleting all fuel.” 

%
\section*{{\suttatitleacronym SN 35.62}{\suttatitletranslation The Depletion of All Fuel (2nd) }{\suttatitleroot Dutiyasabbupādānapariyādānasutta}}
\addcontentsline{toc}{section}{\tocacronym{SN 35.62} \toctranslation{The Depletion of All Fuel (2nd) } \tocroot{Dutiyasabbupādānapariyādānasutta}}
\markboth{The Depletion of All Fuel (2nd) }{Dutiyasabbupādānapariyādānasutta}
\extramarks{SN 35.62}{SN 35.62}

“Mendicants,\marginnote{1.1} I will teach you the principle for depleting all fuel. Listen … 

And\marginnote{1.3} what is the principle for depleting all fuel? 

What\marginnote{2.1} do you think, mendicants? Is the eye permanent or impermanent?” 

“Impermanent,\marginnote{3.1} sir.” 

“But\marginnote{4.1} if it’s impermanent, is it suffering or happiness?” 

“Suffering,\marginnote{5.1} sir.” 

“But\marginnote{6.1} if it’s impermanent, suffering, and liable to wear out, is it fit to be regarded thus: ‘This is mine, I am this, this is my self’?” 

“No,\marginnote{7.1} sir.” 

“Sights\marginnote{8.1} … eye consciousness … eye contact … 

The\marginnote{12.1} pleasant, painful, or neutral feeling that arises conditioned by eye contact: is that permanent or impermanent?” 

“Impermanent,\marginnote{13.1} sir.” … 

“Ear\marginnote{14.1} … nose … tongue … body … mind … thoughts … mind consciousness … mind contact … The pleasant, painful, or neutral feeling that arises conditioned by mind contact: is that permanent or impermanent?” 

“Impermanent,\marginnote{15.1} sir.” 

“But\marginnote{16.1} if it’s impermanent, is it suffering or happiness?” 

“Suffering,\marginnote{17.1} sir.” 

“But\marginnote{18.1} if it’s impermanent, suffering, and liable to wear out, is it fit to be regarded thus: ‘This is mine, I am this, this is my self’?” 

“No,\marginnote{19.1} sir.” 

“Seeing\marginnote{20.1} this, a learned noble disciple grows disillusioned with the eye, sights, eye consciousness, and eye contact. And they grow disillusioned with the painful, pleasant, or neutral feeling that arises conditioned by eye contact. 

They\marginnote{20.3} grow disillusioned with the ear … nose … tongue … body … 

They\marginnote{20.4} grow disillusioned with the mind, thoughts, mind consciousness, and mind contact. And they grow disillusioned with the painful, pleasant, or neutral feeling that arises conditioned by mind contact. Being disillusioned, desire fades away. When desire fades away they’re freed. When they’re freed, they know they’re freed. 

They\marginnote{20.7} understand: ‘Rebirth is ended, the spiritual journey has been completed, what had to be done has been done, there is no return to any state of existence.’ 

This\marginnote{20.8} is the principle for depleting all fuel.” 

%
\addtocontents{toc}{\let\protect\contentsline\protect\nopagecontentsline}
\chapter*{The Chapter with Migajāla }
\addcontentsline{toc}{chapter}{\tocchapterline{The Chapter with Migajāla }}
\addtocontents{toc}{\let\protect\contentsline\protect\oldcontentsline}

%
\section*{{\suttatitleacronym SN 35.63}{\suttatitletranslation With Migajāla (1st) }{\suttatitleroot Paṭhamamigajālasutta}}
\addcontentsline{toc}{section}{\tocacronym{SN 35.63} \toctranslation{With Migajāla (1st) } \tocroot{Paṭhamamigajālasutta}}
\markboth{With Migajāla (1st) }{Paṭhamamigajālasutta}
\extramarks{SN 35.63}{SN 35.63}

At\marginnote{1.1} \textsanskrit{Sāvatthī}. 

Then\marginnote{1.2} Venerable \textsanskrit{Migajāla} went up to the Buddha … and said to him: 

“Sir,\marginnote{1.4} they speak of one who lives alone. How is one who lives alone defined? And how is living with a partner defined?” 

“\textsanskrit{Migajāla},\marginnote{2.1} there are sights known by the eye that are likable, desirable, agreeable, pleasant, sensual, and arousing. If a mendicant approves, welcomes, and keeps clinging to them, this gives rise to relishing. When there’s relishing there’s lust. When there’s lust there is a fetter. A mendicant who is fettered by relishing is said to live with a partner. 

There\marginnote{2.7} are sounds … smells … tastes … touches … 

There\marginnote{2.8} are thoughts known by the mind that are likable, desirable, agreeable, pleasant, sensual, and arousing. If a mendicant approves, welcomes, and keeps clinging to them, this gives rise to relishing. When there’s relishing there’s lust. When there’s lust there is a fetter. A mendicant who is fettered by relishing is said to live with a partner. 

A\marginnote{2.14} mendicant who lives like this is said to live with a partner, even if they frequent remote lodgings in the wilderness and the forest that are quiet and still, far from the madding crowd, remote from human settlements, and fit for retreat. Why is that? For craving is their partner, and they haven’t given it up. That’s why they’re said to live with a partner. 

There\marginnote{3.1} are sights known by the eye that are likable, desirable, agreeable, pleasant, sensual, and arousing. If a mendicant doesn’t approve, welcome, and keep clinging to them, relishing ceases. When there’s no relishing there’s no lust. When there’s no lust there’s no fetter. A mendicant who is not fettered by relishing is said to live alone. 

There\marginnote{3.7} are sounds … smells … tastes … touches … 

There\marginnote{3.8} are thoughts known by the mind that are likable, desirable, agreeable, pleasant, sensual, and arousing. If a mendicant doesn’t approve, welcome, and keep clinging to them, relishing ceases. When there’s no relishing there’s no lust. When there’s no lust there’s no fetter. 

A\marginnote{3.13} mendicant who is not fettered by relishing is said to live alone. A mendicant who lives like this is said to live alone, even if they live within a village crowded by monks, nuns, laymen, and laywomen; by rulers and their ministers, and teachers of other paths and their disciples. Why is that? For craving is their partner, and they have given it up. That’s why they’re said to live alone.” 

%
\section*{{\suttatitleacronym SN 35.64}{\suttatitletranslation With Migajāla (2nd) }{\suttatitleroot Dutiyamigajālasutta}}
\addcontentsline{toc}{section}{\tocacronym{SN 35.64} \toctranslation{With Migajāla (2nd) } \tocroot{Dutiyamigajālasutta}}
\markboth{With Migajāla (2nd) }{Dutiyamigajālasutta}
\extramarks{SN 35.64}{SN 35.64}

Then\marginnote{1.1} Venerable \textsanskrit{Migajāla} went up to the Buddha … and said to him: 

“Sir,\marginnote{1.3} may the Buddha please teach me Dhamma in brief. When I’ve heard it, I’ll live alone, withdrawn, diligent, keen, and resolute.” 

“\textsanskrit{Migajāla},\marginnote{2.1} there are sights known by the eye that are likable, desirable, agreeable, pleasant, sensual, and arousing. If a mendicant approves, welcomes, and keep clinging to them, this gives rise to relishing. Relishing is the origin of suffering, I say. 

There\marginnote{2.5} are sounds … smells … tastes … touches … thoughts known by the mind that are likable, desirable, agreeable, pleasant, sensual, and arousing. If a mendicant approves, welcomes, and keeps clinging to them, this gives rise to relishing. Relishing is the origin of suffering, I say. 

There\marginnote{3.1} are sights known by the eye that are likable, desirable, agreeable, pleasant, sensual, and arousing. If a mendicant doesn’t approve, welcome, and keep clinging to them, relishing ceases. When relishing ceases, suffering ceases, I say. 

There\marginnote{3.4} are sounds … smells … tastes … touches … thoughts known by the mind that are likable, desirable, agreeable, pleasant, sensual, and arousing. If a mendicant doesn’t approve, welcome, and keep clinging to them, relishing ceases. When relishing ceases, suffering ceases, I say.” 

And\marginnote{4.1} then Venerable \textsanskrit{Migajāla} approved and agreed with what the Buddha said. He got up from his seat, bowed, and respectfully circled the Buddha, keeping him on his right, before leaving. 

Then\marginnote{4.2} \textsanskrit{Migajāla}, living alone, withdrawn, diligent, keen, and resolute, soon realized the supreme end of the spiritual path in this very life. He lived having achieved with his own insight the goal for which gentlemen rightly go forth from the lay life to homelessness. 

He\marginnote{4.3} understood: “Rebirth is ended; the spiritual journey has been completed; what had to be done has been done; there is no return to any state of existence.” And \textsanskrit{Migajāla} became one of the perfected. 

%
\section*{{\suttatitleacronym SN 35.65}{\suttatitletranslation Samiddhi’s Question About Māra }{\suttatitleroot Paṭhamasamiddhimārapañhāsutta}}
\addcontentsline{toc}{section}{\tocacronym{SN 35.65} \toctranslation{Samiddhi’s Question About Māra } \tocroot{Paṭhamasamiddhimārapañhāsutta}}
\markboth{Samiddhi’s Question About Māra }{Paṭhamasamiddhimārapañhāsutta}
\extramarks{SN 35.65}{SN 35.65}

At\marginnote{1.1} one time the Buddha was staying near \textsanskrit{Rājagaha}, in the Bamboo Grove, the squirrels’ feeding ground. Then Venerable Samiddhi went up to the Buddha … and said to him: 

“Sir,\marginnote{1.4} they speak of this thing called ‘\textsanskrit{Māra}’. How do we define \textsanskrit{Māra} or what is known as \textsanskrit{Māra}?” 

“Samiddhi,\marginnote{2.1} where there is the eye, sights, eye consciousness, and phenomena to be known by eye consciousness, there is \textsanskrit{Māra} or what is known as \textsanskrit{Māra}. 

Where\marginnote{2.2} there is the ear, sounds, ear consciousness, and phenomena to be known by ear consciousness, there is \textsanskrit{Māra} or what is known as \textsanskrit{Māra}. 

Where\marginnote{2.3} there is the nose, smells, nose consciousness, and phenomena to be known by nose consciousness, there is \textsanskrit{Māra} or what is known as \textsanskrit{Māra}. 

Where\marginnote{2.4} there is the tongue, tastes, tongue consciousness, and phenomena to be known by tongue consciousness, there is \textsanskrit{Māra} or what is known as \textsanskrit{Māra}. 

Where\marginnote{2.5} there is the body, touches, body consciousness, and phenomena to be known by body consciousness, there is \textsanskrit{Māra} or what is known as \textsanskrit{Māra}. 

Where\marginnote{2.6} there is the mind, thoughts, mind consciousness, and phenomena to be known by mind consciousness, there is \textsanskrit{Māra} or what is known as \textsanskrit{Māra}. 

Where\marginnote{3.1} there is no eye, no sights, no eye consciousness, and no phenomena to be known by eye consciousness, there is no \textsanskrit{Māra} or what is known as \textsanskrit{Māra}. 

Where\marginnote{3.2} there is no ear … no nose … no tongue … no body … 

Where\marginnote{3.6} there is no mind, no thoughts, no mind consciousness, and no phenomena to be known by mind consciousness, there is no \textsanskrit{Māra} or what is known as \textsanskrit{Māra}.” 

%
\section*{{\suttatitleacronym SN 35.66}{\suttatitletranslation Samiddhi’s Question About a Sentient Being }{\suttatitleroot Samiddhisattapañhāsutta}}
\addcontentsline{toc}{section}{\tocacronym{SN 35.66} \toctranslation{Samiddhi’s Question About a Sentient Being } \tocroot{Samiddhisattapañhāsutta}}
\markboth{Samiddhi’s Question About a Sentient Being }{Samiddhisattapañhāsutta}
\extramarks{SN 35.66}{SN 35.66}

“Sir,\marginnote{1.1} they speak of this thing called a ‘sentient being’. How do we define a sentient being or what is known as a sentient being?” … 

%
\section*{{\suttatitleacronym SN 35.67}{\suttatitletranslation Samiddhi’s Question About Suffering }{\suttatitleroot Samiddhidukkhapañhāsutta}}
\addcontentsline{toc}{section}{\tocacronym{SN 35.67} \toctranslation{Samiddhi’s Question About Suffering } \tocroot{Samiddhidukkhapañhāsutta}}
\markboth{Samiddhi’s Question About Suffering }{Samiddhidukkhapañhāsutta}
\extramarks{SN 35.67}{SN 35.67}

“Sir,\marginnote{1.1} they speak of this thing called ‘suffering’. How do we define suffering or what is known as suffering?” … 

%
\section*{{\suttatitleacronym SN 35.68}{\suttatitletranslation Samiddhi’s Question About the World }{\suttatitleroot Samiddhilokapañhāsutta}}
\addcontentsline{toc}{section}{\tocacronym{SN 35.68} \toctranslation{Samiddhi’s Question About the World } \tocroot{Samiddhilokapañhāsutta}}
\markboth{Samiddhi’s Question About the World }{Samiddhilokapañhāsutta}
\extramarks{SN 35.68}{SN 35.68}

“Sir,\marginnote{1.1} they speak of this thing called ‘the world’. How do we define the world or what is known as the world?” 

“Samiddhi,\marginnote{1.3} where there is the eye, sights, eye consciousness, and phenomena to be known by eye consciousness, there is the world or what is known as the world. Where there is the ear … nose … tongue … body … Where there is the mind, thoughts, mind consciousness, and phenomena to be known by mind consciousness, there is the world or what is known as the world. 

Where\marginnote{2.1} there is no eye, no sights, no eye consciousness, and no phenomena to be known by eye consciousness, there is no world or what is known as the world. Where there is no ear … nose … tongue … body … Where there is no mind, no thoughts, no mind consciousness, and no phenomena to be known by mind consciousness, there is no world or what is known as the world.” 

%
\section*{{\suttatitleacronym SN 35.69}{\suttatitletranslation Upasena and the Viper }{\suttatitleroot Upasenaāsīvisasutta}}
\addcontentsline{toc}{section}{\tocacronym{SN 35.69} \toctranslation{Upasena and the Viper } \tocroot{Upasenaāsīvisasutta}}
\markboth{Upasena and the Viper }{Upasenaāsīvisasutta}
\extramarks{SN 35.69}{SN 35.69}

At\marginnote{1.1} one time the venerables \textsanskrit{Sāriputta} and Upasena were staying near \textsanskrit{Rājagaha} in the Cool Grove, under the Snake’s Hood Grotto. Now at that time a viper fell on Upasena’s body, and he addressed the mendicants, “Come, reverends, lift this body onto a cot and carry it outside before it’s scattered right here like a handful of chaff.” 

When\marginnote{2.1} he said this, \textsanskrit{Sāriputta} said to him, “But we don’t see any impairment in your body or deterioration of your faculties. Yet you say: ‘Come, reverends, lift this body onto a cot and carry it outside before it’s scattered right here like a handful of chaff.’” 

“Reverend\marginnote{2.7} \textsanskrit{Sāriputta}, there may be an impairment in body or deterioration of faculties for someone who thinks: ‘I am the eye’ or ‘the eye is mine.’ Or ‘I am the ear … nose … tongue … body …’ Or ‘I am the mind’ or ‘the mind is mine.’ 

But\marginnote{2.12} I don’t think like that. So why would there be an impairment in my body or deterioration of my faculties?” 

“That\marginnote{3.1} must be because Venerable Upasena has long ago totally eradicated ego, possessiveness, and the underlying tendency to conceit. 

That’s\marginnote{3.2} why it doesn’t occur to you: ‘I am the eye’ or ‘the eye is mine.’ Or ‘I am the ear … nose … tongue … body …’ Or ‘I am the mind’ or ‘the mind is mine.’” 

Then\marginnote{3.6} those mendicants lifted Upasena’s body onto a cot and carried it outside. And his body was scattered right there like a handful of chaff. 

%
\section*{{\suttatitleacronym SN 35.70}{\suttatitletranslation Upavāṇa on What is Visible in This Very Life }{\suttatitleroot Upavāṇasandiṭṭhikasutta}}
\addcontentsline{toc}{section}{\tocacronym{SN 35.70} \toctranslation{Upavāṇa on What is Visible in This Very Life } \tocroot{Upavāṇasandiṭṭhikasutta}}
\markboth{Upavāṇa on What is Visible in This Very Life }{Upavāṇasandiṭṭhikasutta}
\extramarks{SN 35.70}{SN 35.70}

Then\marginnote{1.1} Venerable \textsanskrit{Upavāṇa} went up to the Buddha … and said to him: 

“Sir,\marginnote{1.3} they speak of ‘a teaching visible in this very life’. In what way is the teaching visible in this very life, immediately effective, inviting inspection, relevant, so that sensible people can know it for themselves?” 

“\textsanskrit{Upavāṇa},\marginnote{2.1} take a mendicant who sees a sight with their eyes. They experience both the sight and the desire for the sight. There is desire for sights in them, and they understand that. Since this is so, this is how the teaching is visible in this very life, immediately effective, inviting inspection, relevant, so that sensible people can know it for themselves. 

Next,\marginnote{3.1} take a mendicant who hears … smells … tastes … touches … 

Next,\marginnote{4.1} take a mendicant who knows a thought with their mind. They experience both the thought and the desire for the thought. There is desire for thoughts in them, and they understand that. Since this is so, this is how the teaching is visible in this very life, immediately effective, inviting inspection, relevant, so that sensible people can know it for themselves. 

Take\marginnote{5.1} a mendicant who sees a sight with their eyes. They experience the sight but no desire for the sight. There is no desire for sights in them, and they understand that. Since this is so, this is how the teaching is visible in this very life, immediately effective, inviting inspection, relevant, so that sensible people can know it for themselves. 

Next,\marginnote{6.1} take a mendicant who hears … smells … tastes … touches … 

Next,\marginnote{7.1} take a mendicant who knows a thought with their mind. They experience the thought but no desire for the thought. There is no desire for thoughts in them, and they understand that. Since this is so, this is how the teaching is visible in this very life, immediately effective, inviting inspection, relevant, so that sensible people can know it for themselves.” 

%
\section*{{\suttatitleacronym SN 35.71}{\suttatitletranslation Six Fields of Contact (1st) }{\suttatitleroot Paṭhamachaphassāyatanasutta}}
\addcontentsline{toc}{section}{\tocacronym{SN 35.71} \toctranslation{Six Fields of Contact (1st) } \tocroot{Paṭhamachaphassāyatanasutta}}
\markboth{Six Fields of Contact (1st) }{Paṭhamachaphassāyatanasutta}
\extramarks{SN 35.71}{SN 35.71}

“Mendicants,\marginnote{1.1} anyone who doesn’t truly understand the six fields of contact’s origin, ending, gratification, drawback, and escape has not completed the spiritual journey and is far from this teaching and training.” 

When\marginnote{2.1} he said this, one of the mendicants said to the Buddha, “Here, sir, I’m lost. For I don’t truly understand the six fields of contact’s origin, ending, gratification, drawback, and escape.” 

“What\marginnote{3.1} do you think, mendicant? Do you regard the eye like this: ‘This is mine, I am this, this is my self’?” 

“No,\marginnote{4.1} sir.” 

“Good,\marginnote{5.1} mendicant! And regarding the eye, you will truly see clearly with right wisdom that: ‘This is not mine, I am not this, this is not my self.’ Just this is the end of suffering. 

Do\marginnote{5.3} you regard the ear … nose … tongue … body … 

Do\marginnote{7.1} you regard the mind like this: ‘This is mine, I am this, this is my self’?” 

“No,\marginnote{8.1} sir.” 

“Good,\marginnote{9.1} mendicant! And regarding the mind, you will truly see clearly with right wisdom that: ‘This is not mine, I am not this, this is not my self.’ Just this is the end of suffering.” 

%
\section*{{\suttatitleacronym SN 35.72}{\suttatitletranslation Six Fields of Contact (2nd) }{\suttatitleroot Dutiyachaphassāyatanasutta}}
\addcontentsline{toc}{section}{\tocacronym{SN 35.72} \toctranslation{Six Fields of Contact (2nd) } \tocroot{Dutiyachaphassāyatanasutta}}
\markboth{Six Fields of Contact (2nd) }{Dutiyachaphassāyatanasutta}
\extramarks{SN 35.72}{SN 35.72}

“Mendicants,\marginnote{1.1} anyone who doesn’t truly understand the six fields of contact’s origin, ending, gratification, drawback, and escape has not completed the spiritual journey and is far from this teaching and training.” 

When\marginnote{2.1} he said this, one of the mendicants said to the Buddha, “Here, sir, I’m lost, truly lost. For I don’t truly understand the six fields of contact’s origin, ending, gratification, drawback, and escape.” 

“What\marginnote{3.1} do you think, mendicant? Do you regard the eye like this: ‘This is not mine, I am not this, this is not my self’?” 

“Yes,\marginnote{4.1} sir.” 

“Good,\marginnote{5.1} mendicant! And regarding the eye, you will truly see clearly with right wisdom that: ‘This is not mine, I am not this, this is not my self.’ In this way you will give up the first field of contact, so that there are no more future lives. 

Do\marginnote{6.1} you regard the ear … nose … tongue … body … 

Do\marginnote{9.1} you regard the mind like this: ‘This is not mine, I am not this, this is not my self’?” 

“Yes,\marginnote{10.1} sir.” 

“Good,\marginnote{11.1} mendicant! And regarding the mind, you will truly see clearly with right wisdom that: ‘This is not mine, I am not this, this is not my self.’ In this way you will give up the sixth field of contact, so that there are no more future lives.” 

%
\section*{{\suttatitleacronym SN 35.73}{\suttatitletranslation Six Fields of Contact (3rd) }{\suttatitleroot Tatiyachaphassāyatanasutta}}
\addcontentsline{toc}{section}{\tocacronym{SN 35.73} \toctranslation{Six Fields of Contact (3rd) } \tocroot{Tatiyachaphassāyatanasutta}}
\markboth{Six Fields of Contact (3rd) }{Tatiyachaphassāyatanasutta}
\extramarks{SN 35.73}{SN 35.73}

“Mendicants,\marginnote{1.1} anyone who doesn’t truly understand the six fields of contact’s origin, ending, gratification, drawback, and escape has not completed the spiritual journey and is far from this teaching and training.” 

When\marginnote{2.1} he said this, one of the mendicants said to the Buddha, “Here, sir, I’m lost, truly lost. For I don’t truly understand the six fields of contact’s origin, ending, gratification, drawback, and escape.” 

“What\marginnote{3.1} do you think, mendicant? Is the eye permanent or impermanent?” 

“Impermanent,\marginnote{4.1} sir.” 

“But\marginnote{5.1} if it’s impermanent, is it suffering or happiness?” 

“Suffering,\marginnote{6.1} sir.” 

“But\marginnote{7.1} if it’s impermanent, suffering, and liable to wear out, is it fit to be regarded thus: ‘This is mine, I am this, this is my self’?” 

“No,\marginnote{8.1} sir.” 

“Is\marginnote{9.1} the ear … nose … tongue … body … mind permanent or impermanent?” 

“Impermanent,\marginnote{10.1} sir.” 

“But\marginnote{11.1} if it’s impermanent, is it suffering or happiness?” 

“Suffering,\marginnote{12.1} sir.” 

“But\marginnote{13.1} if it’s impermanent, suffering, and liable to wear out, is it fit to be regarded thus: ‘This is mine, I am this, this is my self’?” 

“No,\marginnote{14.1} sir.” 

“Seeing\marginnote{15.1} this, a learned noble disciple grows disillusioned with the eye, ear, nose, tongue, body, and mind. Being disillusioned, desire fades away. When desire fades away they’re freed. When they’re freed, they know they’re freed. 

They\marginnote{15.3} understand: ‘Rebirth is ended, the spiritual journey has been completed, what had to be done has been done, there is no return to any state of existence.’” 

%
\addtocontents{toc}{\let\protect\contentsline\protect\nopagecontentsline}
\chapter*{The Chapter on Sick }
\addcontentsline{toc}{chapter}{\tocchapterline{The Chapter on Sick }}
\addtocontents{toc}{\let\protect\contentsline\protect\oldcontentsline}

%
\section*{{\suttatitleacronym SN 35.74}{\suttatitletranslation Sick (1st) }{\suttatitleroot Paṭhamagilānasutta}}
\addcontentsline{toc}{section}{\tocacronym{SN 35.74} \toctranslation{Sick (1st) } \tocroot{Paṭhamagilānasutta}}
\markboth{Sick (1st) }{Paṭhamagilānasutta}
\extramarks{SN 35.74}{SN 35.74}

At\marginnote{1.1} \textsanskrit{Sāvatthī}. 

Then\marginnote{1.2} a mendicant went up to the Buddha, and said to him, “Sir, in such and such a monastery there’s a mendicant who is junior and not well-known. He’s sick, suffering, gravely ill. Please go to him out of compassion.” 

When\marginnote{2.1} the Buddha heard that the mendicant was junior and ill, understanding that he was not well-known, he went to him. That mendicant saw the Buddha coming off in the distance and tried to rise on his cot. 

The\marginnote{2.4} Buddha said to that monk, “It’s all right, mendicant, don’t get up. There are some seats spread out, I will sit there.” 

He\marginnote{2.7} sat on the seat spread out and said to the mendicant, “I hope you’re keeping well, mendicant; I hope you’re alright. I hope that your pain is fading, not growing, that its fading is evident, not its growing.” 

“Sir,\marginnote{3.1} I’m not keeping well, I’m not alright. The pain is terrible and growing, not fading; its growing is evident, not its fading.” 

“I\marginnote{4.1} hope you don’t have any remorse or regret?” 

“Indeed,\marginnote{5.1} sir, I have no little remorse and regret.” 

“I\marginnote{6.1} hope you have no reason to blame yourself when it comes to ethical conduct?” 

“No\marginnote{7.1} sir, I have no reason to blame myself when it comes to ethical conduct.” 

“In\marginnote{8.1} that case, mendicant, why do you have remorse and regret?” 

“Because\marginnote{9.1} I understand that the Buddha has not taught the Dhamma merely for the sake of ethical purity.” 

“If\marginnote{10.1} that is so, what exactly do you understand to be the purpose of teaching the Dhamma?” 

“I\marginnote{11.1} understand that the Buddha has taught the Dhamma for the purpose of the fading away of greed.” 

“Good,\marginnote{12.1} good, mendicant! It’s good that you understand that I’ve taught the Dhamma for the purpose of the fading away of greed. For that is indeed the purpose. What do you think, mendicant? Is the eye permanent or impermanent?” 

“Impermanent,\marginnote{13.1} sir.” … 

“Is\marginnote{14.1} the ear … nose … tongue … body … mind permanent or impermanent?” 

“Impermanent,\marginnote{15.1} sir.” 

“But\marginnote{16.1} if it’s impermanent, is it suffering or happiness?” 

“Suffering,\marginnote{17.1} sir.” 

“But\marginnote{18.1} if it’s impermanent, suffering, and liable to wear out, is it fit to be regarded thus: ‘This is mine, I am this, this is my self’?” 

“No,\marginnote{19.1} sir.” 

“Seeing\marginnote{20.1} this, a learned noble disciple grows disillusioned with the eye, ear, nose, tongue, body, and mind. Being disillusioned, desire fades away. When desire fades away they’re freed. When they’re freed, they know they’re freed. 

They\marginnote{20.3} understand: ‘Rebirth is ended … there is no return to any state of existence.’” 

That\marginnote{21.1} is what the Buddha said. Satisfied, that mendicant was happy with what the Buddha said. And while this discourse was being spoken, the stainless, immaculate vision of the Dhamma arose in that mendicant: 

“Everything\marginnote{21.4} that has a beginning has an end.” 

%
\section*{{\suttatitleacronym SN 35.75}{\suttatitletranslation Sick (2nd) }{\suttatitleroot Dutiyagilānasutta}}
\addcontentsline{toc}{section}{\tocacronym{SN 35.75} \toctranslation{Sick (2nd) } \tocroot{Dutiyagilānasutta}}
\markboth{Sick (2nd) }{Dutiyagilānasutta}
\extramarks{SN 35.75}{SN 35.75}

Then\marginnote{1.1} a mendicant went up to the Buddha … and asked him, “Sir, in such and such a monastery there’s a mendicant who is junior and not well-known. He’s sick, suffering, gravely ill. Please go to him out of compassion.” 

When\marginnote{2.1} the Buddha heard that the mendicant was junior and ill, understanding that he was not well-known, he went to him. That mendicant saw the Buddha coming off in the distance and tried to rise on his cot. 

Then\marginnote{2.4} the Buddha said to that monk, “It’s all right, mendicant, don’t get up. There are some seats spread out, I will sit there.” 

He\marginnote{2.7} sat on the seat spread out and said to the mendicant, “I hope you’re keeping well, mendicant; I hope you’re alright. I hope that your pain is fading, not growing, that its fading is evident, not its growing.” 

“Sir,\marginnote{3.1} I’m not keeping well, I’m not alright. … 

I\marginnote{3.2} have no reason to blame myself when it comes to ethical conduct.” 

“In\marginnote{4.1} that case, mendicant, why do you have remorse and regret?” 

“Because\marginnote{5.1} I understand that the Buddha has not taught the Dhamma merely for the sake of ethical purity.” 

“If\marginnote{6.1} that is so, what exactly do you understand to be the purpose of teaching the Dhamma?” 

“I\marginnote{7.1} understand that the Buddha has taught the Dhamma for the purpose of complete extinguishment by not grasping.” 

“Good,\marginnote{8.1} good, mendicant! It’s good that you understand that I’ve taught the Dhamma for the purpose of complete extinguishment by not grasping. For that is indeed the purpose. 

What\marginnote{9.1} do you think, mendicant? Is the eye permanent or impermanent?” 

“Impermanent,\marginnote{10.1} sir.” … 

“Is\marginnote{11.1} the ear … nose … tongue … body … mind … mind consciousness … mind contact … The pleasant, painful, or neutral feeling that arises conditioned by mind contact: is that permanent or impermanent?” 

“Impermanent,\marginnote{12.1} sir.” 

“But\marginnote{13.1} if it’s impermanent, is it suffering or happiness?” 

“Suffering,\marginnote{14.1} sir.” 

“But\marginnote{15.1} if it’s impermanent, suffering, and liable to wear out, is it fit to be regarded thus: ‘This is mine, I am this, this is my self’?” 

“No,\marginnote{16.1} sir.” 

“Seeing\marginnote{17.1} this, a learned noble disciple grows disillusioned with the eye … ear … nose … tongue … body … mind … mind consciousness … mind contact … They grow disillusioned with the painful, pleasant, or neutral feeling that arises conditioned by mind contact. Being disillusioned, desire fades away. When desire fades away they’re freed. When they’re freed, they know they’re freed. 

They\marginnote{17.7} understand: ‘Rebirth is ended, the spiritual journey has been completed, what had to be done has been done, there is no return to any state of existence.’” 

That\marginnote{18.1} is what the Buddha said. Satisfied, that mendicant was happy with what the Buddha said. And while this discourse was being spoken, the mind of that mendicant was freed from defilements by not grasping. 

%
\section*{{\suttatitleacronym SN 35.76}{\suttatitletranslation With Rādha on Impermanence }{\suttatitleroot Rādhaaniccasutta}}
\addcontentsline{toc}{section}{\tocacronym{SN 35.76} \toctranslation{With Rādha on Impermanence } \tocroot{Rādhaaniccasutta}}
\markboth{With Rādha on Impermanence }{Rādhaaniccasutta}
\extramarks{SN 35.76}{SN 35.76}

The\marginnote{1.1} Venerable \textsanskrit{Rādha} went up to the Buddha … and said to him, “Sir, may the Buddha please teach me Dhamma in brief. When I’ve heard it, I’ll live alone, withdrawn, diligent, keen, and resolute.” 

“\textsanskrit{Rādha},\marginnote{1.4} you should give up desire for what is impermanent. And what is impermanent? The eye, sights, eye consciousness, and eye contact are impermanent. And the pleasant, painful, or neutral feeling that arises conditioned by eye contact is also impermanent. You should give up desire for it. 

The\marginnote{1.9} ear … nose … tongue … body … The mind, thoughts, mind consciousness, and mind contact are impermanent. And the pleasant, painful, or neutral feeling that arises conditioned by mind contact is also impermanent. You should give up desire for it. 

You\marginnote{1.16} should give up desire for what is impermanent.” 

%
\section*{{\suttatitleacronym SN 35.77}{\suttatitletranslation With Rādha on Suffering }{\suttatitleroot Rādhadukkhasutta}}
\addcontentsline{toc}{section}{\tocacronym{SN 35.77} \toctranslation{With Rādha on Suffering } \tocroot{Rādhadukkhasutta}}
\markboth{With Rādha on Suffering }{Rādhadukkhasutta}
\extramarks{SN 35.77}{SN 35.77}

“\textsanskrit{Rādha},\marginnote{1.1} you should give up desire for what is suffering. …” 

%
\section*{{\suttatitleacronym SN 35.78}{\suttatitletranslation With Rādha on Not-Self }{\suttatitleroot Rādhaanattasutta}}
\addcontentsline{toc}{section}{\tocacronym{SN 35.78} \toctranslation{With Rādha on Not-Self } \tocroot{Rādhaanattasutta}}
\markboth{With Rādha on Not-Self }{Rādhaanattasutta}
\extramarks{SN 35.78}{SN 35.78}

“\textsanskrit{Rādha},\marginnote{1.1} you should give up desire for what is not-self. …” 

%
\section*{{\suttatitleacronym SN 35.79}{\suttatitletranslation Giving Up Ignorance (1st) }{\suttatitleroot Paṭhamaavijjāpahānasutta}}
\addcontentsline{toc}{section}{\tocacronym{SN 35.79} \toctranslation{Giving Up Ignorance (1st) } \tocroot{Paṭhamaavijjāpahānasutta}}
\markboth{Giving Up Ignorance (1st) }{Paṭhamaavijjāpahānasutta}
\extramarks{SN 35.79}{SN 35.79}

Then\marginnote{1.1} a mendicant went up to the Buddha … and said to him: 

“Sir,\marginnote{1.3} is there one thing such that by giving it up a mendicant gives up ignorance and gives rise to knowledge?” 

“There\marginnote{2.1} is, mendicant.” 

“But\marginnote{3.1} what is that one thing?” 

“Ignorance\marginnote{4.1} is one thing such that by giving it up a mendicant gives up ignorance and gives rise to knowledge.” 

“But\marginnote{5.1} how does a mendicant know and see so as to give up ignorance and give rise to knowledge?” 

“When\marginnote{6.1} a mendicant knows and sees the eye, sights, eye consciousness, and eye contact as impermanent, ignorance is given up and knowledge arises. And also knowing and seeing the pleasant, painful, or neutral feeling that arises conditioned by eye contact as impermanent, ignorance is given up and knowledge arises. … 

Knowing\marginnote{6.3} and seeing the mind, thoughts, mind consciousness, and mind contact as impermanent, ignorance is given up and knowledge arises. And also knowing and seeing the pleasant, painful, or neutral feeling that arises conditioned by mind contact as impermanent, ignorance is given up and knowledge arises. 

That’s\marginnote{6.5} how a mendicant knows and sees so as to give up ignorance and give rise to knowledge.” 

%
\section*{{\suttatitleacronym SN 35.80}{\suttatitletranslation Giving Up Ignorance (2nd) }{\suttatitleroot Dutiyaavijjāpahānasutta}}
\addcontentsline{toc}{section}{\tocacronym{SN 35.80} \toctranslation{Giving Up Ignorance (2nd) } \tocroot{Dutiyaavijjāpahānasutta}}
\markboth{Giving Up Ignorance (2nd) }{Dutiyaavijjāpahānasutta}
\extramarks{SN 35.80}{SN 35.80}

Then\marginnote{1.1} a mendicant went up to the Buddha … and asked him, “Sir, is there one thing such that by giving it up a mendicant gives up ignorance and gives rise to knowledge?” 

“There\marginnote{2.1} is, mendicant.” 

“But\marginnote{3.1} what is that one thing?” 

“Ignorance\marginnote{4.1} is one thing such that by giving it up a mendicant gives up ignorance and gives rise to knowledge.” 

“But\marginnote{5.1} how does a mendicant know and see so as to give up ignorance and give rise to knowledge?” 

“It’s\marginnote{6.1} when a mendicant has heard: ‘Nothing is worth insisting on.’ When a mendicant has heard that nothing is worth insisting on, they directly know all things. Directly knowing all things, they completely understand all things. Completely understanding all things, they see all signs as other. They see the eye, sights, eye consciousness, and eye contact as other. And they also see the pleasant, painful, or neutral feeling that arises conditioned by eye contact as other. … 

They\marginnote{6.9} see the mind, thoughts, mind consciousness, and mind contact as other. And they also see the pleasant, painful, or neutral feeling that arises conditioned by mind contact as other. That’s how a mendicant knows and sees so as to give up ignorance and give rise to knowledge.” 

%
\section*{{\suttatitleacronym SN 35.81}{\suttatitletranslation Several Mendicants }{\suttatitleroot Sambahulabhikkhusutta}}
\addcontentsline{toc}{section}{\tocacronym{SN 35.81} \toctranslation{Several Mendicants } \tocroot{Sambahulabhikkhusutta}}
\markboth{Several Mendicants }{Sambahulabhikkhusutta}
\extramarks{SN 35.81}{SN 35.81}

Then\marginnote{1.1} several mendicants went up to the Buddha … and asked him, “Sir, sometimes wanderers who follow other paths ask us: ‘Reverends, what’s the purpose of leading the spiritual life under the ascetic Gotama?’ We answer them like this: ‘The purpose of leading the spiritual life under the Buddha is to completely understand suffering.’ 

Answering\marginnote{1.6} this way, we trust that we repeat what the Buddha has said, and don’t misrepresent him with an untruth. We trust our explanation is in line with the teaching, and that there are no legitimate grounds for rebuke or criticism.” 

“Indeed,\marginnote{2.1} in answering this way you repeat what I’ve said, and don’t misrepresent me with an untruth. Your explanation is in line with the teaching, and there are no legitimate grounds for rebuke or criticism. For the purpose of leading the spiritual life under me is to completely understand suffering. 

If\marginnote{2.3} wanderers who follow other paths were to ask you: ‘Reverends, what is that suffering?’ You should answer them: ‘Reverends, the eye is suffering. The purpose of leading the spiritual life under the Buddha is to completely understand this. Sights … Eye consciousness … Eye contact … The pleasant, painful, or neutral feeling that arises conditioned by eye contact is also suffering. The purpose of leading the spiritual life under the Buddha is to completely understand this. 

Ear\marginnote{2.10} … Nose … Tongue … Body … Mind … The pleasant, painful, or neutral feeling that arises conditioned by mind contact is also suffering. The purpose of leading the spiritual life under the Buddha is to completely understand this. 

This\marginnote{2.13} is that suffering. The purpose of leading the spiritual life under the Buddha is to completely understand this.’ When questioned by wanderers who follow other paths, that’s how you should answer them.” 

%
\section*{{\suttatitleacronym SN 35.82}{\suttatitletranslation A Question On the World }{\suttatitleroot Lokapañhāsutta}}
\addcontentsline{toc}{section}{\tocacronym{SN 35.82} \toctranslation{A Question On the World } \tocroot{Lokapañhāsutta}}
\markboth{A Question On the World }{Lokapañhāsutta}
\extramarks{SN 35.82}{SN 35.82}

Then\marginnote{1.1} a mendicant went up to the Buddha … and said to him: 

“Sir,\marginnote{2.1} they speak of this thing called ‘the world’. How is the world defined?” 

“It\marginnote{2.3} wears away, mendicant, that’s why it’s called ‘the world’. And what is wearing away? The eye is wearing away. Sights … eye consciousness … eye contact is wearing away. The painful, pleasant, or neutral feeling that arises conditioned by eye contact is also wearing away. 

The\marginnote{2.7} ear … nose … tongue … body … The mind … thoughts … mind consciousness … mind contact is wearing away. The painful, pleasant, or neutral feeling that arises conditioned by mind contact is also wearing away. 

It\marginnote{2.9} wears away, mendicant, that’s why it’s called ‘the world’.” 

%
\section*{{\suttatitleacronym SN 35.83}{\suttatitletranslation Phagguna’s Question }{\suttatitleroot Phaggunapañhāsutta}}
\addcontentsline{toc}{section}{\tocacronym{SN 35.83} \toctranslation{Phagguna’s Question } \tocroot{Phaggunapañhāsutta}}
\markboth{Phagguna’s Question }{Phaggunapañhāsutta}
\extramarks{SN 35.83}{SN 35.83}

And\marginnote{1.1} then Venerable Phagguna went up to the Buddha … and said to him: 

“Sir,\marginnote{2.1} suppose someone were to describe the Buddhas of the past who have become completely extinguished, cut off proliferation, cut off the track, finished off the cycle, and transcended suffering. Does the eye exist by which they could be described? 

Does\marginnote{2.2} the ear … nose … tongue … body exist …? Does the mind exist by which they could be described?” 

“Phagguna,\marginnote{3.1} suppose someone were to describe the Buddhas of the past who have become completely extinguished, cut off proliferation, cut off the track, finished off the cycle, and transcended suffering. The eye does not exist by which they could be described. 

The\marginnote{3.2} ear … nose … tongue … body does not exist … The mind does not exist by which they could be described.” 

%
\addtocontents{toc}{\let\protect\contentsline\protect\nopagecontentsline}
\chapter*{The Chapter with Channa }
\addcontentsline{toc}{chapter}{\tocchapterline{The Chapter with Channa }}
\addtocontents{toc}{\let\protect\contentsline\protect\oldcontentsline}

%
\section*{{\suttatitleacronym SN 35.84}{\suttatitletranslation Liable to Wear Out }{\suttatitleroot Palokadhammasutta}}
\addcontentsline{toc}{section}{\tocacronym{SN 35.84} \toctranslation{Liable to Wear Out } \tocroot{Palokadhammasutta}}
\markboth{Liable to Wear Out }{Palokadhammasutta}
\extramarks{SN 35.84}{SN 35.84}

At\marginnote{1.1} \textsanskrit{Sāvatthī}. 

Then\marginnote{1.2} Venerable Ānanda went up to the Buddha, bowed, sat down to one side, and said to him: 

“Sir,\marginnote{2.1} they speak of this thing called ‘the world’. How is the world defined?” 

“Ānanda,\marginnote{2.3} that which is liable to wear out is called the world in the training of the Noble One. And what is liable to wear out? The eye is liable to wear out. Sights … eye consciousness … eye contact is liable to wear out. The painful, pleasant, or neutral feeling that arises conditioned by eye contact is also liable to wear out. 

The\marginnote{2.6} ear … nose … tongue … body … The mind … thoughts … mind consciousness … mind contact is liable to wear out. The painful, pleasant, or neutral feeling that arises conditioned by mind contact is also liable to wear out. 

That\marginnote{2.8} which is liable to wear out is called the world in the training of the Noble One.” 

%
\section*{{\suttatitleacronym SN 35.85}{\suttatitletranslation The World is Empty }{\suttatitleroot Suññatalokasutta}}
\addcontentsline{toc}{section}{\tocacronym{SN 35.85} \toctranslation{The World is Empty } \tocroot{Suññatalokasutta}}
\markboth{The World is Empty }{Suññatalokasutta}
\extramarks{SN 35.85}{SN 35.85}

And\marginnote{1.1} then Venerable Ānanda … said to the Buddha: 

“Sir,\marginnote{1.2} they say that ‘the world is empty’. What does the saying ‘the world is empty’ refer to?” 

“Ānanda,\marginnote{1.4} they say that ‘the world is empty’ because it’s empty of self or what belongs to self. And what is empty of self or what belongs to self? The eye, sights, eye consciousness, and eye contact are empty of self or what belongs to self. … 

The\marginnote{1.8} pleasant, painful, or neutral feeling that arises conditioned by mind contact is also empty of self or what belongs to self. They say that ‘the world is empty’ because it’s empty of self or what belongs to self.” 

%
\section*{{\suttatitleacronym SN 35.86}{\suttatitletranslation A Teaching In Brief }{\suttatitleroot Saṁkhittadhammasutta}}
\addcontentsline{toc}{section}{\tocacronym{SN 35.86} \toctranslation{A Teaching In Brief } \tocroot{Saṁkhittadhammasutta}}
\markboth{A Teaching In Brief }{Saṁkhittadhammasutta}
\extramarks{SN 35.86}{SN 35.86}

Seated\marginnote{1.1} to one side, Venerable Ānanda said to the Buddha: 

“Sir,\marginnote{1.2} may the Buddha please teach me Dhamma in brief. When I’ve heard it, I’ll live alone, withdrawn, diligent, keen, and resolute.” 

“What\marginnote{2.1} do you think, Ānanda? Is the eye permanent or impermanent?” 

“Impermanent,\marginnote{3.1} sir.” 

“But\marginnote{4.1} if it’s impermanent, is it suffering or happiness?” 

“Suffering,\marginnote{5.1} sir.” 

“But\marginnote{6.1} if it’s impermanent, suffering, and liable to wear out, is it fit to be regarded thus: ‘This is mine, I am this, this is my self’?” 

“No,\marginnote{7.1} sir.” 

“Are\marginnote{8.1} sights … eye consciousness … eye contact … 

The\marginnote{10.2} pleasant, painful, or neutral feeling that arises conditioned by eye contact: is that permanent or impermanent?” 

“Impermanent,\marginnote{11.1} sir.” 

“But\marginnote{12.1} if it’s impermanent, is it suffering or happiness?” 

“Suffering,\marginnote{13.1} sir.” 

“But\marginnote{14.1} if it’s impermanent, suffering, and liable to wear out, is it fit to be regarded thus: ‘This is mine, I am this, this is my self’?” 

“No,\marginnote{15.1} sir.” … 

“Is\marginnote{16.1} the ear … nose … tongue … body … mind … 

The\marginnote{18.1} pleasant, painful, or neutral feeling that arises conditioned by mind contact: is that permanent or impermanent?” 

“Impermanent,\marginnote{19.1} sir.” 

“But\marginnote{20.1} if it’s impermanent, is it suffering or happiness?” 

“Suffering,\marginnote{21.1} sir.” 

“But\marginnote{22.1} if it’s impermanent, suffering, and liable to wear out, is it fit to be regarded thus: ‘This is mine, I am this, this is my self’?” 

“No,\marginnote{23.1} sir.” 

“Seeing\marginnote{24.1} this, a learned noble disciple grows disillusioned with the eye, sights, eye consciousness, and eye contact. And they grow disillusioned with the painful, pleasant, or neutral feeling that arises conditioned by eye contact. 

They\marginnote{24.2} grow disillusioned with the ear … nose … tongue … body … mind … painful, pleasant, or neutral feeling that arises conditioned by mind contact. 

Being\marginnote{24.3} disillusioned, desire fades away. When desire fades away they’re freed. When they’re freed, they know they’re freed. 

They\marginnote{24.4} understand: ‘Rebirth is ended, the spiritual journey has been completed, what had to be done has been done, there is no return to any state of existence.’” 

%
\section*{{\suttatitleacronym SN 35.87}{\suttatitletranslation With Channa }{\suttatitleroot Channasutta}}
\addcontentsline{toc}{section}{\tocacronym{SN 35.87} \toctranslation{With Channa } \tocroot{Channasutta}}
\markboth{With Channa }{Channasutta}
\extramarks{SN 35.87}{SN 35.87}

At\marginnote{1.1} one time the Buddha was staying near \textsanskrit{Rājagaha}, in the Bamboo Grove, the squirrels’ feeding ground. 

Now\marginnote{1.2} at that time the venerables \textsanskrit{Sāriputta}, \textsanskrit{Mahācunda}, and Channa were staying on the Vulture’s Peak Mountain. Now at that time Venerable Channa was sick, suffering, gravely ill. 

Then\marginnote{1.4} in the late afternoon, Venerable \textsanskrit{Sāriputta} came out of retreat, went to Venerable \textsanskrit{Mahācunda} and said to him, “Come, Reverend Cunda, let’s go to see Venerable Channa and ask about his illness.” 

“Yes,\marginnote{1.6} reverend,” replied \textsanskrit{Mahācunda}. 

And\marginnote{2.1} then \textsanskrit{Sāriputta} and \textsanskrit{Mahācunda} went to see Channa and sat down on the seats spread out. \textsanskrit{Sāriputta} said to Channa: “I hope you’re keeping well, Reverend Channa; I hope you’re alright. I hope that your pain is fading, not growing, that its fading is evident, not its growing.” 

“Reverend\marginnote{3.1} \textsanskrit{Sāriputta}, I’m not keeping well, I’m not alright. The pain is terrible and growing, not fading; its growing is evident, not its fading. The winds piercing my head are so severe, it feels like a strong man drilling into my head with a sharp point. The pain in my head is so severe, it feels like a strong man tightening a tough leather strap around my head. The winds slicing my belly are so severe, like a deft butcher or their apprentice were slicing open a cows’s belly open with a meat cleaver. The burning in my body is so severe, it feels like two strong men grabbing a weaker man by the arms to burn and scorch him on a pit of glowing coals. I’m not keeping well, I’m not alright. The pain is terrible and growing, not fading; its growing is evident, not its fading. 

Reverend\marginnote{3.10} \textsanskrit{Sāriputta}, I will slit my wrists. I don’t wish to live.” 

“Please\marginnote{4.1} don’t slit your wrists! Venerable Channa, keep going! We want you to keep going. 

If\marginnote{4.3} you don’t have any suitable food, we’ll find it for you. If you don’t have suitable medicine, we’ll find it for you. If you don’t have a capable carer, we’ll find one for you. 

Please\marginnote{4.6} don’t slit your wrists! Venerable Channa, keep going! We want you to keep going.” 

“Reverend\marginnote{5.1} \textsanskrit{Sāriputta}, it’s not that I don’t have suitable food; I do have suitable food. It’s not that I don’t have suitable medicine; I do have suitable medicine. It’s not that I don’t have a capable carer; I do have a capable carer. 

Moreover,\marginnote{5.7} for a long time now I have served the Teacher with love, not without love. For it is proper for a disciple to serve the Teacher with love, not without love. You should remember this: ‘The mendicant Channa will slit his wrists blamelessly.’” 

“I’d\marginnote{6.1} like to ask Venerable Channa about a certain point, if you’d take the time to answer.” 

“Ask,\marginnote{6.2} Reverend \textsanskrit{Sāriputta}. When I’ve heard it I’ll know.” 

“Reverend\marginnote{7.1} Channa, do you regard the eye, eye consciousness, and things knowable by eye consciousness in this way: ‘This is mine, I am this, this is my self’? 

Do\marginnote{7.2} you regard the ear … nose … tongue … body … mind, mind consciousness, and things knowable by mind consciousness in this way: ‘This is mine, I am this, this is my self’?” 

“Reverend\marginnote{8.1} \textsanskrit{Sāriputta}, I regard the eye, eye consciousness, and things knowable by eye consciousness in this way: ‘This is not mine, I am not this, this is not my self.’ 

I\marginnote{8.2} regard the ear … nose … tongue … body … mind, mind consciousness, and things knowable by mind consciousness in this way: ‘This is not mine, I am not this, this is not my self’.” 

“Reverend\marginnote{9.1} Channa, what have you seen, what have you directly known in these things that you regard them in this way: ‘This is not mine, I am not this, this is not my self’?” 

“Reverend\marginnote{10.1} \textsanskrit{Sāriputta}, after seeing cessation, after directly knowing cessation in these things I regard them in this way: ‘This is not mine, I am not this, this is not my self’.” 

When\marginnote{11.1} he said this, Venerable \textsanskrit{Mahācunda} said to Venerable Channa, “So, Reverend Channa, you should pay close attention to this instruction of the Buddha whenever you can: 

‘For\marginnote{11.3} the dependent there is agitation. For the independent there’s no agitation. When there’s no agitation there is tranquility. When there’s tranquility there’s no inclination. When there’s no inclination, there’s no coming and going. When there’s no coming and going, there’s no passing away and reappearing. When there’s no passing away and reappearing, there’s no this world or world beyond or between the two. Just this is the end of suffering.’” 

And\marginnote{12.1} when the venerables \textsanskrit{Sāriputta} and \textsanskrit{Mahācunda} had given Venerable Channa this advice they got up from their seat and left. Not long after those venerables had left, Venerable Channa slit his wrists. 

Then\marginnote{13.1} \textsanskrit{Sāriputta} went up to the Buddha, bowed, sat down to one side, and said to him, “Sir, Venerable Channa has slit his wrists. Where has he been reborn in his next life?” 

“\textsanskrit{Sāriputta},\marginnote{13.4} didn’t the mendicant Channa declare his blamelessness to you personally?” 

“Sir,\marginnote{13.5} there is a Vajjian village named Pubbavijjhana where Channa had families with whom he was friendly, intimate, and familiar.” 

“The\marginnote{13.7} mendicant Channa did indeed have such families, \textsanskrit{Sāriputta}. But this is not enough for me to call someone ‘blameworthy’. When someone lays down this body and takes up another body, I call them ‘blameworthy’. But the mendicant Channa did no such thing. 

You\marginnote{13.11} should remember this: ‘The mendicant Channa slit his wrists blamelessly.’” 

%
\section*{{\suttatitleacronym SN 35.88}{\suttatitletranslation With Puṇṇa }{\suttatitleroot Puṇṇasutta}}
\addcontentsline{toc}{section}{\tocacronym{SN 35.88} \toctranslation{With Puṇṇa } \tocroot{Puṇṇasutta}}
\markboth{With Puṇṇa }{Puṇṇasutta}
\extramarks{SN 35.88}{SN 35.88}

And\marginnote{1.1} then Venerable \textsanskrit{Puṇṇa} went up to the Buddha, bowed, sat down to one side, and said to him: 

“Sir,\marginnote{1.2} may the Buddha please teach me Dhamma in brief. When I’ve heard it, I’ll live alone, withdrawn, diligent, keen, and resolute.” 

“\textsanskrit{Puṇṇa},\marginnote{2.1} there are sights known by the eye that are likable, desirable, agreeable, pleasant, sensual, and arousing. If a mendicant approves, welcomes, and keeps clinging to them, this gives rise to relishing. Relishing is the origin of suffering, I say. 

There\marginnote{2.5} are sounds … smells … tastes … touches … There are thoughts known by the mind that are likable, desirable, agreeable, pleasant, sensual, and arousing. If a mendicant approves, welcomes, and keeps clinging to them, this gives rise to relishing. Relishing is the origin of suffering, I say. 

There\marginnote{3.1} are sights known by the eye that are likable, desirable, agreeable, pleasant, sensual, and arousing. If a mendicant doesn’t approve, welcome, and keep clinging to them, relishing ceases. When relishing ceases, suffering ceases, I say. … 

There\marginnote{3.4} are thoughts known by the mind that are likable, desirable, agreeable, pleasant, sensual, and arousing. If a mendicant doesn’t approve, welcome, and keep clinging to them, relishing ceases. When relishing ceases, suffering ceases, I say. 

\textsanskrit{Puṇṇa},\marginnote{4.1} now that I’ve given you this brief advice, what country will you live in?” 

“Sir,\marginnote{4.2} there’s a country called \textsanskrit{Sunāparanta}; I will live there.” 

“The\marginnote{5.1} people of \textsanskrit{Sunāparanta} are wild and rough, \textsanskrit{Puṇṇa}. If they abuse and insult you, what will you think of them?” 

“If\marginnote{6.1} they abuse and insult me, I will think: ‘These people of \textsanskrit{Sunāparanta} are gracious, truly gracious, since they don’t hit me with their fists.’ That’s what I’ll think, Blessed One. That’s what I’ll think, Holy One.” 

“But\marginnote{7.1} if they do hit you with their fists, what will you think of them then?” 

“If\marginnote{8.1} they hit me with their fists, I’ll think: ‘These people of \textsanskrit{Sunāparanta} are gracious, truly gracious, since they don’t throw stones at me.’ That’s what I’ll think, Blessed One. That’s what I’ll think, Holy One.” 

“But\marginnote{9.1} if they do throw stones at you, what will you think of them then?” 

“If\marginnote{10.1} they throw stones at me, I’ll think: ‘These people of \textsanskrit{Sunāparanta} are gracious, truly gracious, since they don’t beat me with a club.’ That’s what I’ll think, Blessed One. That’s what I’ll think, Holy One.” 

“But\marginnote{11.1} if they do beat you with a club, what will you think of them then?” 

“If\marginnote{12.1} they beat me with a club, I’ll think: ‘These people of \textsanskrit{Sunāparanta} are gracious, truly gracious, since they don’t stab me with a knife.’ That’s what I’ll think, Blessed One. That’s what I’ll think, Holy One.” 

“But\marginnote{13.1} if they do stab you with a knife, what will you think of them then?” 

“If\marginnote{14.1} they stab me with a knife, I’ll think: ‘These people of \textsanskrit{Sunāparanta} are gracious, truly gracious, since they don’t take my life with a sharp knife.’ That’s what I’ll think, Blessed One. That’s what I’ll think, Holy One.” 

“But\marginnote{15.1} if they do take your life with a sharp knife, what will you think of them then?” 

“If\marginnote{16.1} they take my life with a sharp knife, I’ll think: ‘There are disciples of the Buddha who looked for someone to assist with slitting their wrists because they were horrified, repelled, and disgusted with the body and with life. And I have found this without looking!’ That’s what I’ll think, Blessed One. That’s what I’ll think, Holy One.” 

“Good,\marginnote{17.1} good \textsanskrit{Puṇṇa}! Having such self-control and peacefulness, you will be quite capable of living in \textsanskrit{Sunāparanta}. Now, \textsanskrit{Puṇṇa}, go at your convenience.” 

And\marginnote{18.1} then \textsanskrit{Puṇṇa} welcomed and agreed with the Buddha’s words. He got up from his seat, bowed, and respectfully circled the Buddha, keeping him on his right. Then he set his lodgings in order and, taking his bowl and robe, set out for \textsanskrit{Sunāparanta}. 

Traveling\marginnote{18.2} stage by stage, he arrived at \textsanskrit{Sunāparanta}, and stayed there. Within that rainy season he confirmed around five hundred male and five hundred female lay followers. And within that same rainy season he realized the three knowledges. And within that same rainy season he became completely extinguished. 

Then\marginnote{19.1} several mendicants went up to the Buddha … and asked him, “Sir, the gentleman named \textsanskrit{Puṇṇa}, who was advised in brief by the Buddha, has passed away. Where has he been reborn in his next life?” 

“Mendicants,\marginnote{20.1} \textsanskrit{Puṇṇa} was astute. He practiced in line with the teachings, and did not trouble me about the teachings. \textsanskrit{Puṇṇa} has become completely extinguished.” 

%
\section*{{\suttatitleacronym SN 35.89}{\suttatitletranslation With Bāhiya }{\suttatitleroot Bāhiyasutta}}
\addcontentsline{toc}{section}{\tocacronym{SN 35.89} \toctranslation{With Bāhiya } \tocroot{Bāhiyasutta}}
\markboth{With Bāhiya }{Bāhiyasutta}
\extramarks{SN 35.89}{SN 35.89}

Then\marginnote{1.1} Venerable \textsanskrit{Bāhiya} went up to the Buddha, bowed, sat down to one side, and said to him: 

“Sir,\marginnote{1.2} may the Buddha please teach me Dhamma in brief. When I’ve heard it, I’ll live alone, withdrawn, diligent, keen, and resolute.” 

“What\marginnote{2.1} do you think, \textsanskrit{Bāhiya}? Is the eye permanent or impermanent?” 

“Impermanent,\marginnote{3.1} sir.” 

“But\marginnote{4.1} if it’s impermanent, is it suffering or happiness?” 

“Suffering,\marginnote{5.1} sir.” 

“But\marginnote{6.1} if it’s impermanent, suffering, and liable to wear out, is it fit to be regarded thus: ‘This is mine, I am this, this is my self’?” 

“No,\marginnote{7.1} sir.” 

“Are\marginnote{8.1} sights … eye consciousness … eye contact … 

The\marginnote{9.4} pleasant, painful, or neutral feeling that arises conditioned by mind contact: is that permanent or impermanent?” 

“Impermanent,\marginnote{10.1} sir.” 

“But\marginnote{11.1} if it’s impermanent, is it suffering or happiness?” 

“Suffering,\marginnote{12.1} sir.” 

“But\marginnote{13.1} if it’s impermanent, suffering, and liable to wear out, is it fit to be regarded thus: ‘This is mine, I am this, this is my self’?” 

“No,\marginnote{14.1} sir.” 

“Seeing\marginnote{15.1} this, a learned noble disciple grows disillusioned with the eye, sights, eye consciousness, and eye contact. And they grow disillusioned with the painful, pleasant, or neutral feeling that arises conditioned by eye contact. 

They\marginnote{15.2} grow disillusioned with the ear … nose … tongue … body … mind … painful, pleasant, or neutral feeling that arises conditioned by mind contact. 

Being\marginnote{15.3} disillusioned, desire fades away. When desire fades away they’re freed. When they’re freed, they know they’re freed. 

They\marginnote{15.4} understand: ‘Rebirth is ended, the spiritual journey has been completed, what had to be done has been done, there is no return to any state of existence.’” 

And\marginnote{16.1} then Venerable \textsanskrit{Bāhiya} approved and agreed with what the Buddha said. He got up from his seat, bowed, and respectfully circled the Buddha, keeping him on his right, before leaving. 

Then\marginnote{16.2} \textsanskrit{Bāhiya}, living alone, withdrawn, diligent, keen, and resolute, soon realized the supreme end of the spiritual path in this very life. He lived having achieved with his own insight the goal for which gentlemen rightly go forth from the lay life to homelessness. 

He\marginnote{16.3} understood: “Rebirth is ended; the spiritual journey has been completed; what had to be done has been done; there is no return to any state of existence.” And Venerable \textsanskrit{Bāhiya} became one of the perfected. 

%
\section*{{\suttatitleacronym SN 35.90}{\suttatitletranslation Turbulence (1st) }{\suttatitleroot Paṭhamaejāsutta}}
\addcontentsline{toc}{section}{\tocacronym{SN 35.90} \toctranslation{Turbulence (1st) } \tocroot{Paṭhamaejāsutta}}
\markboth{Turbulence (1st) }{Paṭhamaejāsutta}
\extramarks{SN 35.90}{SN 35.90}

“Mendicants,\marginnote{1.1} turbulence is a disease, a boil, a dart. That’s why the Realized One lives unperturbed, with dart drawn out. 

Now,\marginnote{1.3} a mendicant might wish: ‘May I live unperturbed, with dart drawn out.’ 

So\marginnote{1.4} let them not identify with the eye, let them not identify regarding the eye, let them not identify as the eye, let them not identify ‘the eye is mine.’ Let them not identify sights … eye consciousness … eye contact … Let them not identify with the pleasant, painful, or neutral feeling that arises conditioned by eye contact. Let them not identify regarding that, let them not identify as that, and let them not identify ‘that is mine.’ 

Let\marginnote{2.1} them not identify the ear … nose … tongue … body … mind … thoughts … mind consciousness … mind contact … Let them not identify with the pleasant, painful, or neutral feeling that arises conditioned by mind contact. Let them not identify regarding that, let them not identify as that, and let them not identify ‘that is mine.’ 

Let\marginnote{3.7} them not identify with all, let them not identify regarding all, let them not identify as all, let them not identify ‘all is mine’. 

Not\marginnote{4.1} identifying, they don’t grasp at anything in the world. Not grasping, they’re not anxious. Not being anxious, they personally become extinguished. 

They\marginnote{4.3} understand: ‘Rebirth is ended, the spiritual journey has been completed, what had to be done has been done, there is no return to any state of existence.’” 

%
\section*{{\suttatitleacronym SN 35.91}{\suttatitletranslation Turbulence (2nd) }{\suttatitleroot Dutiyaejāsutta}}
\addcontentsline{toc}{section}{\tocacronym{SN 35.91} \toctranslation{Turbulence (2nd) } \tocroot{Dutiyaejāsutta}}
\markboth{Turbulence (2nd) }{Dutiyaejāsutta}
\extramarks{SN 35.91}{SN 35.91}

“Mendicants,\marginnote{1.1} turbulence is a disease, a boil, a dart. That’s why the Realized One lives unperturbed, with dart drawn out. 

Now,\marginnote{1.3} a mendicant might wish: ‘May I live unperturbed, with dart drawn out.’ 

So\marginnote{1.4} let them not identify with the eye, let them not identify in the eye, let them not identify from the eye, let them not identify: ‘The eye is mine.’ Let them not identify with sights … eye consciousness … eye contact … Let them not identify with the pleasant, painful, or neutral feeling that arises conditioned by eye contact. Let them not identify in that, let them not identify from that, and let them not identify: ‘That is mine.’ For whatever you identify with, whatever you identify in, whatever you identify as, and whatever you identify to be ‘mine’: that becomes something else. The world is attached to being, taking pleasure only in being, yet it becomes something else. 

Let\marginnote{2.1} them not identify with the ear … nose … tongue … body … 

Let\marginnote{3.1} them not identify with the mind … mind consciousness … mind contact … Let them not identify with the pleasant, painful, or neutral feeling that arises conditioned by mind contact. Let them not identify in that, let them not identify as that, and let them not identify: ‘That is mine.’ For whatever you identify with, whatever you identify in, whatever you identify as, and whatever you identify to be ‘mine’: that becomes something else. The world is attached to being, taking pleasure only in being, yet it becomes something else. 

As\marginnote{4.1} far as the aggregates, elements, and sense fields extend, they don’t identify with that, they don’t identify in that, they don’t identify as that, and they don’t identify: ‘That is mine.’ 

Not\marginnote{4.2} identifying, they don’t grasp at anything in the world. Not grasping, they’re not anxious. Not being anxious, they personally become extinguished. 

They\marginnote{4.4} understand: ‘Rebirth is ended, the spiritual journey has been completed, what had to be done has been done, there is no return to any state of existence.’” 

%
\section*{{\suttatitleacronym SN 35.92}{\suttatitletranslation A Duality (1st) }{\suttatitleroot Paṭhamadvayasutta}}
\addcontentsline{toc}{section}{\tocacronym{SN 35.92} \toctranslation{A Duality (1st) } \tocroot{Paṭhamadvayasutta}}
\markboth{A Duality (1st) }{Paṭhamadvayasutta}
\extramarks{SN 35.92}{SN 35.92}

“Mendicants,\marginnote{1.1} I will teach you a duality. Listen … 

And\marginnote{1.3} what is a duality? It’s just the eye and sights, the ear and sounds, the nose and smells, the tongue and tastes, the body and touches, and the mind and thoughts. This is called a duality. 

Mendicants,\marginnote{2.1} suppose someone was to say: ‘I’ll reject this duality and describe another duality.’ They’d have no grounds for that, they’d be stumped by questions, and, in addition, they’d get frustrated. Why is that? Because they’re out of their element.” 

%
\section*{{\suttatitleacronym SN 35.93}{\suttatitletranslation A Duality (2nd) }{\suttatitleroot Dutiyadvayasutta}}
\addcontentsline{toc}{section}{\tocacronym{SN 35.93} \toctranslation{A Duality (2nd) } \tocroot{Dutiyadvayasutta}}
\markboth{A Duality (2nd) }{Dutiyadvayasutta}
\extramarks{SN 35.93}{SN 35.93}

“Mendicants,\marginnote{1.1} consciousness exists dependent on a duality. And what is that duality? 

Eye\marginnote{1.3} consciousness arises dependent on the eye and sights. The eye is impermanent, decaying, and perishing. Sights are impermanent, decaying, and perishing. So this duality is tottering and toppling; it’s impermanent, decaying, and perishing. Eye consciousness is impermanent, decaying, and perishing. And the causes and reasons that give rise to eye consciousness are also impermanent, decaying, and perishing. But since eye consciousness has arisen dependent on conditions that are impermanent, how could it be permanent? 

The\marginnote{1.10} meeting, coming together, and joining together of these three things is called eye contact. Eye contact is also impermanent, decaying, and perishing. And the causes and reasons that give rise to eye contact are also impermanent, decaying, and perishing. But since eye contact has arisen dependent on conditions that are impermanent, how could it be permanent? 

Contacted,\marginnote{1.14} one feels, intends, and perceives. So these things too are tottering and toppling; they’re impermanent, decaying, and perishing. 

Ear\marginnote{2.1} consciousness … Nose consciousness … Tongue consciousness … Body consciousness … 

Mind\marginnote{3.1} consciousness arises dependent on the mind and thoughts. The mind is impermanent, decaying, and perishing. Thoughts are impermanent, decaying, and perishing. So this duality is tottering and toppling; it’s impermanent, decaying, and perishing. Mind consciousness is impermanent, decaying, and perishing. And the causes and reasons that give rise to mind consciousness are also impermanent, decaying, and perishing. But since mind consciousness has arisen dependent on conditions that are impermanent, how could it be permanent? 

The\marginnote{3.8} meeting, coming together, and joining together of these three things is called mind contact. Mind contact is also impermanent, decaying, and perishing. And the causes and reasons that give rise to mind contact are also impermanent, decaying, and perishing. But since mind contact has arisen dependent on conditions that are impermanent, how could it be permanent? 

Contacted,\marginnote{3.12} one feels, intends, and perceives. So these things too are tottering and toppling; they’re impermanent, decaying, and perishing. 

This\marginnote{3.14} is how consciousness exists dependent on a duality.” 

%
\addtocontents{toc}{\let\protect\contentsline\protect\nopagecontentsline}
\chapter*{The Chapter on the Sixes }
\addcontentsline{toc}{chapter}{\tocchapterline{The Chapter on the Sixes }}
\addtocontents{toc}{\let\protect\contentsline\protect\oldcontentsline}

%
\section*{{\suttatitleacronym SN 35.94}{\suttatitletranslation Untamed, Unguarded }{\suttatitleroot Adantaaguttasutta}}
\addcontentsline{toc}{section}{\tocacronym{SN 35.94} \toctranslation{Untamed, Unguarded } \tocroot{Adantaaguttasutta}}
\markboth{Untamed, Unguarded }{Adantaaguttasutta}
\extramarks{SN 35.94}{SN 35.94}

At\marginnote{1.1} \textsanskrit{Sāvatthī}. 

“Mendicants,\marginnote{1.2} these six fields of contact bring suffering when they’re untamed, unguarded, unprotected, and unrestrained. What six? 

The\marginnote{1.4} field of eye contact brings suffering when it’s untamed, unguarded, unprotected, and unrestrained. 

The\marginnote{1.5} field of ear contact … nose contact … tongue contact … body contact … 

The\marginnote{1.6} field of mind contact brings suffering when it’s untamed, unguarded, unprotected, and unrestrained. 

These\marginnote{1.7} six fields of contact bring suffering when they’re untamed, unguarded, unprotected, and unrestrained. 

These\marginnote{2.1} six fields of contact bring happiness when they’re well tamed, well guarded, well protected, and well restrained. What six? 

The\marginnote{2.3} field of eye contact brings happiness when it’s well tamed, well guarded, well protected, and well restrained. 

The\marginnote{2.4} field of ear contact … nose contact … tongue contact … body contact … 

The\marginnote{2.5} field of mind contact brings happiness when it’s well tamed, well guarded, well protected, and well restrained. 

These\marginnote{2.6} six fields of contact bring happiness when they’re well tamed, well guarded, well protected, and well restrained.” 

That\marginnote{2.7} is what the Buddha said. Then the Holy One, the Teacher, went on to say: 

\begin{verse}%
“Mendicants,\marginnote{3.1} it’s just the six fields of contact \\
that lead the unrestrained to suffering. \\
Those who understand how to restrain them \\
live with faith as partner, uncorrupted. 

When\marginnote{4.1} you’ve seen pleasant sights \\
and unpleasant ones, too, \\
get rid of all manner of desire for the pleasant, \\
without hating what you don’t like. 

When\marginnote{5.1} you’ve heard sounds both liked and disliked, \\
don’t fall under the thrall of sounds you like, \\
get rid of hate for the unliked, \\
and don’t hurt your mind by thinking of what you don’t like. 

When\marginnote{6.1} you’ve smelled a pleasant, fragrant scent, \\
and one that’s foul and unpleasant, \\
get rid of repulsion for the unpleasant, \\
while not yielding to desire for the pleasant. 

When\marginnote{7.1} you’ve enjoyed a sweet, delicious taste, \\
and sometimes those that are bitter, \\
don’t be attached to enjoying sweet tastes, \\
and don’t despise the bitter. 

Don’t\marginnote{8.1} be intoxicated by a pleasant touch, \\
and don’t tremble at a painful touch. \\
Look with equanimity at the duality of pleasant and painful contacts, \\
without favoring or opposing anything. 

People\marginnote{9.1} generally let their perceptions proliferate; \\
perceiving and proliferating, they are attracted. \\
When you’ve expelled all thoughts of the lay life, \\
wander intent on renunciation. 

When\marginnote{10.1} the mind is well developed like this regarding the six, \\
it doesn’t waver at contacts at all. \\
Mendicants, those who have mastered greed and hate \\
go beyond birth and death.” 

%
\end{verse}

%
\section*{{\suttatitleacronym SN 35.95}{\suttatitletranslation Māluṅkyaputta }{\suttatitleroot Mālukyaputtasutta}}
\addcontentsline{toc}{section}{\tocacronym{SN 35.95} \toctranslation{Māluṅkyaputta } \tocroot{Mālukyaputtasutta}}
\markboth{Māluṅkyaputta }{Mālukyaputtasutta}
\extramarks{SN 35.95}{SN 35.95}

Then\marginnote{1.1} Venerable \textsanskrit{Māluṅkyaputta} went up to the Buddha … and asked him, “Sir, may the Buddha please teach me Dhamma in brief. When I’ve heard it, I’ll live alone, withdrawn, diligent, keen, and resolute.” 

“Well\marginnote{2.1} now, \textsanskrit{Māluṅkyaputta}, what are we to say to the young monks, when even an old man like you, elderly and senior, advanced in years, having reached the final stage of life, asks for brief advice?” 

“Sir,\marginnote{3.1} even though I’m an old man, elderly and senior, may the Buddha please teach me Dhamma in brief! May the Holy one please teach me in brief! Hopefully I can understand the meaning of what the Buddha says. Hopefully I can be an heir of the Buddha’s teaching!” 

“What\marginnote{4.1} do you think, \textsanskrit{Māluṅkyaputta}? Do you have any desire or greed or fondness for sights known by the eye that you haven’t seen, you’ve never seen before, you don’t see, and you don’t think would be seen?” 

“No,\marginnote{4.3} sir.” 

“Do\marginnote{5.1} you have any desire or greed or affection for sounds known by the ear … 

smells\marginnote{6.1} known by the nose … 

tastes\marginnote{7.1} known by the tongue … 

touches\marginnote{8.1} known by the body … 

thoughts\marginnote{9.1} known by the mind that you haven’t known, you’ve never known before, you don’t know, and you don’t think would be known?” 

“No,\marginnote{9.2} sir.” 

“In\marginnote{10.1} that case, when it comes to things that are to be seen, heard, thought, and known: in the seen will be merely the seen; in the heard will be merely the heard; in the thought will be merely the thought; in the known will be merely the known. When this is the case, you won’t be ‘by that’. When you’re not ‘by that’, you won’t be ‘in that’. When you’re not ‘in that’, you won’t be in this world or the world beyond or in between the two. Just this is the end of suffering.” 

“This\marginnote{11.1} is how I understand the detailed meaning of the Buddha’s brief statement: 

\begin{verse}%
‘When\marginnote{12.1} you see a sight, mindfulness is lost \\
as attention latches on a pleasant feature. \\
Experiencing it with a mind full of desire, \\
you keep clinging to it. 

Many\marginnote{13.1} feelings grow \\
arising from sights. \\
The mind is damaged \\
by covetousness and cruelty. \\
Heaping up suffering like this, \\
you’re said to be far from extinguishment. 

When\marginnote{14.1} you hear a sound, mindfulness is lost \\
as attention latches on a pleasant feature. \\
Experiencing it with a mind full of desire, \\
you keep clinging to it. 

Many\marginnote{15.1} feelings grow \\
arising from sounds. \\
The mind is damaged \\
by covetousness and cruelty. \\
Heaping up suffering like this, \\
you’re said to be far from extinguishment. 

When\marginnote{16.1} you smell an odor, mindfulness is lost \\
as attention latches on a pleasant feature. \\
Experiencing it with a mind full of desire, \\
you keep clinging to it. 

Many\marginnote{17.1} feelings grow \\
arising from smells. \\
The mind is damaged \\
by covetousness and cruelty. \\
Heaping up suffering like this, \\
you’re said to be far from extinguishment. 

When\marginnote{18.1} you enjoy a taste, mindfulness is lost \\
as attention latches on a pleasant feature. \\
Experiencing it with a mind full of desire, \\
you keep clinging to it. 

Many\marginnote{19.1} feelings grow \\
arising from tastes. \\
The mind is damaged \\
by covetousness and cruelty. \\
Heaping up suffering like this, \\
you’re said to be far from extinguishment. 

When\marginnote{20.1} you sense a touch, mindfulness is lost \\
as attention latches on a pleasant feature. \\
Experiencing it with a mind full of desire, \\
you keep clinging to it. 

Many\marginnote{21.1} feelings grow \\
arising from touches. \\
The mind is damaged \\
by covetousness and cruelty. \\
Heaping up suffering like this, \\
you’re said to be far from extinguishment. 

When\marginnote{22.1} you know a thought, mindfulness is lost \\
as attention latches on a pleasant feature. \\
Experiencing it with a mind full of desire, \\
you keep clinging to it. 

Many\marginnote{23.1} feelings grow \\
arising from thoughts. \\
The mind is damaged \\
by covetousness and cruelty. \\
Heaping up suffering like this, \\
you’re said to be far from extinguishment. 

When\marginnote{24.1} you see a sight with mindfulness, \\
there’s no desire for sights. \\
Experiencing it with a mind free of desire, \\
you don’t keep clinging to it. 

Even\marginnote{25.1} as you see a sight \\
and get familiar with how it feels, \\
you wear away, you don’t heap up: \\
that’s how to live mindfully. \\
Eroding suffering like this, \\
you’re said to be in the presence of extinguishment. 

When\marginnote{26.1} you hear a sound with mindfulness, \\
there’s no desire for sounds. \\
Experiencing it with a mind free of desire, \\
you don’t keep clinging to it. 

Even\marginnote{27.1} as you hear a sound \\
and get familiar with how it feels, \\
you wear away, you don’t heap up: \\
that’s how to live mindfully. \\
Eroding suffering like this, \\
you’re said to be in the presence of extinguishment. 

When\marginnote{28.1} you smell an odor with mindfulness, \\
there’s no desire for odors. \\
Experiencing it with a mind free of desire, \\
you don’t keep clinging to it. 

Even\marginnote{29.1} as you smell an odor \\
and get familiar with how it feels, \\
you wear away, you don’t heap up: \\
that’s how to live mindfully. \\
Eroding suffering like this, \\
you’re said to be in the presence of extinguishment. 

Enjoying\marginnote{30.1} a taste with mindfulness, \\
there’s no desire for tastes. \\
Experiencing it with a mind free of desire, \\
you don’t keep clinging to it. 

Even\marginnote{31.1} as you savor a taste \\
and get familiar with how it feels, \\
you wear away, you don’t heap up: \\
that’s how to live mindfully. \\
Eroding suffering like this, \\
you’re said to be in the presence of extinguishment. 

When\marginnote{32.1} you sense a touch with mindfulness, \\
there’s no desire for touches. \\
Experiencing it with a mind free of desire, \\
you don’t keep clinging to it. 

Even\marginnote{33.1} as you sense a touch \\
and get familiar with how it feels, \\
you wear away, you don’t heap up: \\
that’s how to live mindfully. \\
Eroding suffering like this, \\
you’re said to be in the presence of extinguishment. 

When\marginnote{34.1} you know a thought with mindfulness, \\
there’s no desire for thoughts. \\
Experiencing it with a mind free of desire, \\
you don’t keep clinging to it. 

Even\marginnote{35.1} as you know a thought \\
and get familiar with how it feels, \\
you wear away, you don’t heap up: \\
that’s how to live mindfully. \\
Eroding suffering like this, \\
you’re said to be in the presence of extinguishment.’ 

%
\end{verse}

That’s\marginnote{36.1} how I understand the detailed meaning of the Buddha’s brief statement.” 

“Good,\marginnote{36.2} good, \textsanskrit{Māluṅkyaputta}! It’s good that you understand the detailed meaning of what I’ve said in brief like this. 

\begin{verse}%
(The\marginnote{37.1} Buddha repeats the verses in full.) 

%
\end{verse}

This\marginnote{41.1} is how to understand the detailed meaning of what I said in brief.” 

And\marginnote{42.1} then Venerable \textsanskrit{Māluṅkyaputta} approved and agreed with what the Buddha said. He got up from his seat, bowed, and respectfully circled the Buddha, keeping him on his right, before leaving. Then \textsanskrit{Māluṅkyaputta}, living alone, withdrawn, diligent, keen, and resolute, soon realized the supreme end of the spiritual path in this very life. He lived having achieved with his own insight the goal for which gentlemen rightly go forth from the lay life to homelessness. 

He\marginnote{42.3} understood: “Rebirth is ended; the spiritual journey has been completed; what had to be done has been done; there is no return to any state of existence.” And Venerable \textsanskrit{Māluṅkyaputta} became one of the perfected. 

%
\section*{{\suttatitleacronym SN 35.96}{\suttatitletranslation Liable to Decline }{\suttatitleroot Parihānadhammasutta}}
\addcontentsline{toc}{section}{\tocacronym{SN 35.96} \toctranslation{Liable to Decline } \tocroot{Parihānadhammasutta}}
\markboth{Liable to Decline }{Parihānadhammasutta}
\extramarks{SN 35.96}{SN 35.96}

“Mendicants,\marginnote{1.1} I will teach you who is liable to decline, who is not liable to decline, and the six fields of mastery. Listen … 

And\marginnote{1.3} how is someone liable to decline? When a mendicant sees a sight with the eye, bad, unskillful phenomena arise: memories and thoughts prone to fetters. Suppose that mendicant tolerates them and doesn’t give them up, get rid of them, eliminate them, and obliterate them. They should understand: ‘My skillful qualities are declining. For this is what the Buddha calls decline.’ 

Furthermore,\marginnote{2.1} when a mendicant hears a sound … smells an odor … tastes a flavor … feels a touch … knows a thought with the mind, bad, unskillful phenomena arise: memories and thoughts prone to fetters. If that mendicant tolerates them and doesn’t give them up, get rid of them, eliminate them, and obliterate them, they should understand: ‘My skillful qualities are declining. For this is what the Buddha calls decline.’ That’s how someone is liable to decline. 

And\marginnote{3.1} how is someone not liable to decline? When a mendicant sees a sight with the eye, bad, unskillful phenomena arise: memories and thoughts prone to fetters. Suppose that mendicant doesn’t tolerate them but gives them up, gets rid of them, eliminates them, and obliterates them. They should understand: ‘My skillful qualities are not declining. For this is what the Buddha calls non-decline.’ 

Furthermore,\marginnote{4.1} when a mendicant hears a sound … smells an odor … tastes a flavor … feels a touch … knows a thought with the mind, bad, unskillful phenomena arise: memories and thoughts prone to fetters. Suppose that mendicant doesn’t tolerate them but gives them up, gets rid of them, eliminates them, and obliterates them. They should understand: ‘My skillful qualities are not declining. For this is what the Buddha calls non-decline.’ That’s how someone is not liable to decline. 

And\marginnote{5.1} what are the six fields of mastery? When a mendicant sees a sight with the eye, bad, unskillful phenomena don’t arise: memories and thoughts prone to fetters. They should understand: ‘This sense field has been mastered. For this is what the Buddha calls a field of mastery.’ … Furthermore, when a mendicant knows a thought with the mind, bad, unskillful phenomena don’t arise: memories and thoughts prone to fetters. They should understand: ‘This sense field has been mastered. For this is what the Buddha calls a field of mastery.’ These are the six fields of mastery.” 

%
\section*{{\suttatitleacronym SN 35.97}{\suttatitletranslation One Who Lives Negligently }{\suttatitleroot Pamādavihārīsutta}}
\addcontentsline{toc}{section}{\tocacronym{SN 35.97} \toctranslation{One Who Lives Negligently } \tocroot{Pamādavihārīsutta}}
\markboth{One Who Lives Negligently }{Pamādavihārīsutta}
\extramarks{SN 35.97}{SN 35.97}

“Mendicants,\marginnote{1.1} I will teach you who lives negligently and who lives diligently. Listen … 

And\marginnote{1.3} how does someone live negligently? 

When\marginnote{1.4} you live with the eye faculty unrestrained, your mind becomes polluted when it comes to sights known by the eye. When the mind is polluted, there’s no joy. When there’s no joy, there’s no rapture. When there’s no rapture, there’s no tranquility. When there’s no tranquility, there’s suffering. When one is suffering, the mind does not become immersed in \textsanskrit{samādhi}. When the mind is not immersed in \textsanskrit{samādhi}, principles do not become clear. Because principles have not become clear, you’re considered to live negligently. 

When\marginnote{1.12} you live with the ear … nose … tongue … body … mind faculty unrestrained, your mind becomes polluted when it comes to thoughts known by the mind. When the mind is polluted, there’s no joy. When there’s no joy, there’s no rapture. When there’s no rapture, there’s no tranquility. When there’s no tranquility, there’s suffering. When one is suffering, the mind does not become immersed in \textsanskrit{samādhi}. When the mind is not immersed in \textsanskrit{samādhi}, principles do not become clear. Because principles have not become clear, you’re considered to live negligently. 

That’s\marginnote{1.23} how someone lives negligently. 

And\marginnote{2.1} how does someone live diligently? 

When\marginnote{2.2} you live with the eye faculty restrained, your mind doesn’t become polluted when it comes to sights known by the eye. When the mind isn’t polluted, joy springs up. Being joyful, rapture springs up. When the mind is full of rapture, the body becomes tranquil. When the body is tranquil, one feels bliss. And when blissful, the mind becomes immersed in \textsanskrit{samādhi}. When the mind is immersed in \textsanskrit{samādhi}, principles become clear. Because principles have become clear, you’re considered to live diligently. 

When\marginnote{2.10} you live with the ear … nose … tongue … body … mind faculty restrained, your mind doesn’t become polluted when it comes to thoughts known by the mind. When the mind isn’t polluted, joy springs up. Being joyful, rapture springs up. When the mind is full of rapture, the body becomes tranquil. When the body is tranquil, one feels bliss. And when blissful, the mind becomes immersed in \textsanskrit{samādhi}. When the mind is immersed in \textsanskrit{samādhi}, principles become clear. Because principles have become clear, you’re considered to live diligently. 

That’s\marginnote{2.20} how someone lives diligently.” 

%
\section*{{\suttatitleacronym SN 35.98}{\suttatitletranslation Restraint }{\suttatitleroot Saṁvarasutta}}
\addcontentsline{toc}{section}{\tocacronym{SN 35.98} \toctranslation{Restraint } \tocroot{Saṁvarasutta}}
\markboth{Restraint }{Saṁvarasutta}
\extramarks{SN 35.98}{SN 35.98}

“Mendicants,\marginnote{1.1} I will teach you who is restrained and who is unrestrained. Listen … 

And\marginnote{1.3} how is someone unrestrained? 

There\marginnote{1.4} are sights known by the eye that are likable, desirable, agreeable, pleasant, sensual, and arousing. If a mendicant approves, welcomes, and keeps clinging to them, they should understand: ‘My skillful qualities are declining. For this is what the Buddha calls decline.’ 

There\marginnote{1.8} are sounds … smells … tastes … touches … thoughts known by the mind that are likable, desirable, agreeable, pleasant, sensual, and arousing. If a mendicant approves, welcomes, and keeps clinging to them, they should understand: ‘My skillful qualities are declining. For this is what the Buddha calls decline.’ 

This\marginnote{1.13} is how someone is unrestrained. 

And\marginnote{2.1} how is someone restrained? 

There\marginnote{2.2} are sights known by the eye that are likable, desirable, agreeable, pleasant, sensual, and arousing. If a mendicant doesn’t approve, welcome, and keep clinging to them, they should understand: ‘My skillful qualities are not declining. For this is what the Buddha calls non-decline.’ 

There\marginnote{2.6} are sounds … smells … tastes … touches … thoughts known by the mind that are likable, desirable, agreeable, pleasant, sensual, and arousing. If a mendicant doesn’t approve, welcome, and keep clinging to them, they should understand: ‘My skillful qualities are not declining. For this is what the Buddha calls non-decline.’ 

This\marginnote{2.11} is how someone is restrained.” 

%
\section*{{\suttatitleacronym SN 35.99}{\suttatitletranslation Immersion }{\suttatitleroot Samādhisutta}}
\addcontentsline{toc}{section}{\tocacronym{SN 35.99} \toctranslation{Immersion } \tocroot{Samādhisutta}}
\markboth{Immersion }{Samādhisutta}
\extramarks{SN 35.99}{SN 35.99}

“Mendicants,\marginnote{1.1} develop immersion. A mendicant who has immersion truly understands. What do they truly understand? 

They\marginnote{1.4} truly understand that the eye is impermanent. They truly understand that sights … eye consciousness … eye contact … the pleasant, painful, or neutral feeling that arises conditioned by eye contact is impermanent. … 

They\marginnote{1.9} truly understand that the mind is impermanent. They truly understand that thoughts … mind consciousness … mind contact … the pleasant, painful, or neutral feeling that arises conditioned by mind contact is impermanent. 

Mendicants,\marginnote{1.14} develop immersion. A mendicant who has immersion truly understands.” 

%
\section*{{\suttatitleacronym SN 35.100}{\suttatitletranslation Retreat }{\suttatitleroot Paṭisallānasutta}}
\addcontentsline{toc}{section}{\tocacronym{SN 35.100} \toctranslation{Retreat } \tocroot{Paṭisallānasutta}}
\markboth{Retreat }{Paṭisallānasutta}
\extramarks{SN 35.100}{SN 35.100}

“Mendicants,\marginnote{1.1} meditate in retreat. A mendicant in retreat truly understands. What do they truly understand? 

They\marginnote{1.4} truly understand that the eye is impermanent. They truly understand that sights … eye consciousness … eye contact … the pleasant, painful, or neutral feeling that arises conditioned by mind contact is impermanent. 

Mendicants,\marginnote{1.9} meditate in retreat. A mendicant in retreat truly understands.” 

%
\section*{{\suttatitleacronym SN 35.101}{\suttatitletranslation It’s Not Yours (1st) }{\suttatitleroot Paṭhamanatumhākasutta}}
\addcontentsline{toc}{section}{\tocacronym{SN 35.101} \toctranslation{It’s Not Yours (1st) } \tocroot{Paṭhamanatumhākasutta}}
\markboth{It’s Not Yours (1st) }{Paṭhamanatumhākasutta}
\extramarks{SN 35.101}{SN 35.101}

“Mendicants,\marginnote{1.1} give up what’s not yours. Giving it up will be for your welfare and happiness. And what isn’t yours? 

The\marginnote{1.4} eye isn’t yours: give it up. Giving it up will be for your welfare and happiness. Sights … Eye consciousness … Eye contact … The pleasant, painful, or neutral feeling that arises conditioned by eye contact isn’t yours: give it up. Giving it up will be for your welfare and happiness. 

The\marginnote{2.1} ear … nose … tongue … body … 

The\marginnote{3.1} mind isn’t yours: give it up. Giving it up will be for your welfare and happiness. Thoughts … Mind consciousness … Mind contact … The pleasant, painful, or neutral feeling that arises conditioned by mind contact isn’t yours: give it up. Giving it up will be for your welfare and happiness. 

Suppose\marginnote{4.1} a person was to carry off the grass, sticks, branches, and leaves in this Jeta’s Grove, or burn them, or do what they want with them. Would you think: ‘This person is carrying us off, burning us, or doing what they want with us’?” 

“No,\marginnote{5.1} sir. Why is that? Because that’s neither self nor belonging to self.” 

“In\marginnote{8.1} the same way, the eye isn’t yours: give it up. Giving it up will be for your welfare and happiness. … 

The\marginnote{8.7} pleasant, painful, or neutral feeling that arises conditioned by mind contact isn’t yours: give it up. Giving it up will be for your welfare and happiness.” 

%
\section*{{\suttatitleacronym SN 35.102}{\suttatitletranslation It’s Not Yours (2nd) }{\suttatitleroot Dutiyanatumhākasutta}}
\addcontentsline{toc}{section}{\tocacronym{SN 35.102} \toctranslation{It’s Not Yours (2nd) } \tocroot{Dutiyanatumhākasutta}}
\markboth{It’s Not Yours (2nd) }{Dutiyanatumhākasutta}
\extramarks{SN 35.102}{SN 35.102}

“Mendicants,\marginnote{1.1} give up what’s not yours. Giving it up will be for your welfare and happiness. And what isn’t yours? 

The\marginnote{1.4} eye isn’t yours: give it up. Giving it up will be for your welfare and happiness. Sights … Eye consciousness … Eye contact … 

The\marginnote{1.16} pleasant, painful, or neutral feeling that arises conditioned by mind contact isn’t yours: give it up. Giving it up will be for your welfare and happiness. 

Give\marginnote{1.19} up what’s not yours. Giving it up will be for your welfare and happiness.” 

%
\section*{{\suttatitleacronym SN 35.103}{\suttatitletranslation About Uddaka }{\suttatitleroot Udakasutta}}
\addcontentsline{toc}{section}{\tocacronym{SN 35.103} \toctranslation{About Uddaka } \tocroot{Udakasutta}}
\markboth{About Uddaka }{Udakasutta}
\extramarks{SN 35.103}{SN 35.103}

“Mendicants,\marginnote{1.1} Uddaka, son of \textsanskrit{Rāma}, used to say: ‘This for sure is the knowledge master! This for sure is the conqueror of all! This for sure is the boil’s root dug out, never dug out before!’ 

Even\marginnote{1.3} though Uddaka, son of \textsanskrit{Rāma}, was no knowledge master, he said ‘I’m a knowledge master.’ Though he was no conqueror of all, he said ‘I’m conqueror of all.’ And though the boil’s root was not dug out, he said ‘I’ve dug out the boil’s root.’ 

Here’s\marginnote{1.4} how a mendicant would rightly say: ‘This for sure is the knowledge master! This for sure is the conqueror of all! This for sure is the boil’s root dug out, never dug out before!’ 

And\marginnote{2.1} how is someone a knowledge master? It’s when a mendicant truly understands the six fields of contact’s origin, ending, gratification, drawback, and escape. That’s how a mendicant is a knowledge master. 

And\marginnote{3.1} how is a mendicant a conqueror of all? It’s when a mendicant comes to be freed by not grasping after truly understanding the six fields of contact’s origin, ending, gratification, drawback, and escape. That’s how a mendicant is a conqueror of all. 

And\marginnote{4.1} how has a mendicant dug out the boil’s root, which was never dug out before? ‘Boil’ is a term for this body made up of the four primary elements, produced by mother and father, built up from rice and porridge, liable to impermanence, to wearing away and erosion, to breaking up and destruction. ‘Boil’s root’ is a term for craving. It’s when a mendicant has given up craving, cut it off at the root, made it like a palm stump, obliterated it, so it’s unable to arise in the future. That’s how a mendicant has dug out the boil’s root, which was never dug out before. 

Uddaka,\marginnote{5.1} son of \textsanskrit{Rāma}, used to say: ‘This for sure is the knowledge master! This for sure is the conqueror of all! This for sure is the boil’s root dug out, never dug out before!’ 

Even\marginnote{5.3} though Uddaka, son of \textsanskrit{Rāma}, was no knowledge master, he said ‘I’m a knowledge master.’ Though he was no conqueror of all, he said ‘I’m conqueror of all.’ And though the boil’s root was not dug out, he said ‘I’ve dug out the boil’s root.’ 

But\marginnote{5.5} that’s how a mendicant would rightly say: ‘This for sure is the knowledge master! This for sure is the conqueror of all! This for sure is the boil’s root dug out, never dug out before!’” 

%
\addtocontents{toc}{\let\protect\contentsline\protect\nopagecontentsline}
\pannasa{The Third Fifty }
\addcontentsline{toc}{pannasa}{The Third Fifty }
\markboth{}{}
\addtocontents{toc}{\let\protect\contentsline\protect\oldcontentsline}

%
\addtocontents{toc}{\let\protect\contentsline\protect\nopagecontentsline}
\chapter*{The Chapter on Sanctuary }
\addcontentsline{toc}{chapter}{\tocchapterline{The Chapter on Sanctuary }}
\addtocontents{toc}{\let\protect\contentsline\protect\oldcontentsline}

%
\section*{{\suttatitleacronym SN 35.104}{\suttatitletranslation Sanctuary }{\suttatitleroot Yogakkhemisutta}}
\addcontentsline{toc}{section}{\tocacronym{SN 35.104} \toctranslation{Sanctuary } \tocroot{Yogakkhemisutta}}
\markboth{Sanctuary }{Yogakkhemisutta}
\extramarks{SN 35.104}{SN 35.104}

At\marginnote{1.1} \textsanskrit{Sāvatthī}. 

“Mendicants,\marginnote{1.2} I will teach you an exposition of the teaching, an explanation of one who has reached sanctuary. Listen … 

And\marginnote{1.4} what is an exposition of the teaching, an explanation of one who has reached sanctuary? 

There\marginnote{1.5} are sights known by the eye that are likable, desirable, agreeable, pleasant, sensual, and arousing. The Realized One has given these up, cut them off at the root, made them like a palm stump, and obliterated them, so they are unable to arise in the future. He teaches meditation for giving them up. That’s why the Realized One is called one who has reached sanctuary. … 

There\marginnote{1.8} are thoughts known by the mind that are likable, desirable, agreeable, pleasant, sensual, and arousing. The Realized One has given these up, cut them off at the root, made them like a palm stump, and obliterated them, so they are unable to arise in the future. He teaches meditation for giving them up. That’s why the Realized One is called one who has reached sanctuary. 

This\marginnote{1.11} is an exposition of the teaching, an explanation of one who has reached sanctuary.” 

%
\section*{{\suttatitleacronym SN 35.105}{\suttatitletranslation Because of Grasping }{\suttatitleroot Upādāyasutta}}
\addcontentsline{toc}{section}{\tocacronym{SN 35.105} \toctranslation{Because of Grasping } \tocroot{Upādāyasutta}}
\markboth{Because of Grasping }{Upādāyasutta}
\extramarks{SN 35.105}{SN 35.105}

“Mendicants,\marginnote{1.1} when what exists, because of grasping what, do pleasure and pain arise in oneself?” 

“Our\marginnote{2.1} teachings are rooted in the Buddha. …” 

“Mendicants,\marginnote{3.1} when there’s an eye, because of grasping the eye, pleasure and pain arise in oneself. … When there’s a mind, because of grasping the mind, pleasure and pain arise in oneself. 

What\marginnote{3.3} do you think, mendicants? Is the eye permanent or impermanent?” 

“Impermanent,\marginnote{4.1} sir.” 

“But\marginnote{5.1} if it’s impermanent, is it suffering or happiness?” 

“Suffering,\marginnote{6.1} sir.” 

“But\marginnote{7.1} by not grasping what’s impermanent, suffering, and perishable, would pleasure and pain arise in oneself?” 

“No,\marginnote{8.1} sir.” … 

“Is\marginnote{9.1} the ear … nose … tongue … body … mind permanent or impermanent?” 

“Impermanent,\marginnote{16.1} sir.” 

“But\marginnote{17.1} if it’s impermanent, is it suffering or happiness?” 

“Suffering,\marginnote{18.1} sir.” 

“But\marginnote{19.1} by not grasping what’s impermanent, suffering, and perishable, would pleasure and pain arise in oneself?” 

“No,\marginnote{20.1} sir.” 

“Seeing\marginnote{21.1} this, a learned noble disciple grows disillusioned with the eye, ear, nose, tongue, body, and mind. Being disillusioned, desire fades away. When desire fades away they’re freed. When they’re freed, they know they’re freed. 

They\marginnote{21.3} understand: ‘Rebirth is ended, the spiritual journey has been completed, what had to be done has been done, there is no return to any state of existence.’” 

%
\section*{{\suttatitleacronym SN 35.106}{\suttatitletranslation The Origin of Suffering }{\suttatitleroot Dukkhasamudayasutta}}
\addcontentsline{toc}{section}{\tocacronym{SN 35.106} \toctranslation{The Origin of Suffering } \tocroot{Dukkhasamudayasutta}}
\markboth{The Origin of Suffering }{Dukkhasamudayasutta}
\extramarks{SN 35.106}{SN 35.106}

“Mendicants,\marginnote{1.1} I will teach you the origin and ending of suffering. Listen … 

And\marginnote{1.3} what, mendicants, is the origin of suffering? Eye consciousness arises dependent on the eye and sights. The meeting of the three is contact. Contact is a condition for feeling. Feeling is a condition for craving. This is the origin of suffering … 

Mind\marginnote{1.11} consciousness arises dependent on the mind and thoughts. The meeting of the three is contact. Contact is a condition for feeling. Feeling is a condition for craving. This is the origin of suffering. 

And\marginnote{2.1} what is the ending of suffering? Eye consciousness arises dependent on the eye and sights. The meeting of the three is contact. Contact is a condition for feeling. Feeling is a condition for craving. When that craving fades away and ceases with nothing left over, grasping ceases. When grasping ceases, continued existence ceases. When continued existence ceases, rebirth ceases. When rebirth ceases, old age and death, sorrow, lamentation, pain, sadness, and distress cease. That is how this entire mass of suffering ceases. This is the ending of suffering. … 

Mind\marginnote{2.12} consciousness arises dependent on the mind and thoughts. The meeting of the three is contact. Contact is a condition for feeling. Feeling is a condition for craving. When that craving fades away and ceases with nothing left over, grasping ceases. When grasping ceases, continued existence ceases. When continued existence ceases, rebirth ceases. When rebirth ceases, old age and death, sorrow, lamentation, pain, sadness, and distress cease. That is how this entire mass of suffering ceases. This is the ending of suffering.” 

%
\section*{{\suttatitleacronym SN 35.107}{\suttatitletranslation The Origin of the World }{\suttatitleroot Lokasamudayasutta}}
\addcontentsline{toc}{section}{\tocacronym{SN 35.107} \toctranslation{The Origin of the World } \tocroot{Lokasamudayasutta}}
\markboth{The Origin of the World }{Lokasamudayasutta}
\extramarks{SN 35.107}{SN 35.107}

“Mendicants,\marginnote{1.1} I will teach you the origin and ending of the world. Listen … 

And\marginnote{1.3} what, mendicants, is the origin of the world? Eye consciousness arises dependent on the eye and sights. The meeting of the three is contact. Contact is a condition for feeling. Feeling is a condition for craving. Craving is a condition for grasping. Grasping is a condition for continued existence. Continued existence is a condition for rebirth. Rebirth is a condition for old age and death, sorrow, lamentation, pain, sadness, and distress to come to be. This is the origin of the world. … 

Mind\marginnote{1.13} consciousness arises dependent on the mind and thoughts. The meeting of the three is contact. Contact is a condition for feeling. Feeling is a condition for craving. Craving is a condition for grasping. Grasping is a condition for continued existence. Continued existence is a condition for rebirth. Rebirth is a condition for old age and death, sorrow, lamentation, pain, sadness, and distress to come to be. This is the origin of the world. 

And\marginnote{2.1} what is the ending of the world? Eye consciousness arises dependent on the eye and sights. The meeting of the three is contact. Contact is a condition for feeling. Feeling is a condition for craving. When that craving fades away and ceases with nothing left over, grasping ceases. When grasping ceases, continued existence ceases. When continued existence ceases, rebirth ceases. When rebirth ceases, old age and death, sorrow, lamentation, pain, sadness, and distress cease. That is how this entire mass of suffering ceases. This is the ending of the world. … 

Mind\marginnote{2.12} consciousness arises dependent on the mind and thoughts. The meeting of the three is contact. Contact is a condition for feeling. Feeling is a condition for craving. When that craving fades away and ceases with nothing left over, grasping ceases. When grasping ceases … That is how this entire mass of suffering ceases. This is the ending of the world.” 

%
\section*{{\suttatitleacronym SN 35.108}{\suttatitletranslation I’m Better }{\suttatitleroot Seyyohamasmisutta}}
\addcontentsline{toc}{section}{\tocacronym{SN 35.108} \toctranslation{I’m Better } \tocroot{Seyyohamasmisutta}}
\markboth{I’m Better }{Seyyohamasmisutta}
\extramarks{SN 35.108}{SN 35.108}

“Mendicants,\marginnote{1.1} when what exists, because of grasping what and insisting on what, do people think ‘I’m better’ or ‘I’m equal’ or ‘I’m worse’?” 

“Our\marginnote{2.1} teachings are rooted in the Buddha. …” 

“When\marginnote{3.1} there’s an eye, because of grasping the eye and insisting on the eye, people think ‘I’m better’ or ‘I’m equal’ or ‘I’m worse’. … 

When\marginnote{3.3} there’s a mind, because of grasping the mind and insisting on the mind, people think ‘I’m better’ or ‘I’m equal’ or ‘I’m worse’. 

What\marginnote{3.4} do you think, mendicants? Is the eye permanent or impermanent?” 

“Impermanent,\marginnote{4.1} sir.” 

“But\marginnote{5.1} if it’s impermanent, is it suffering or happiness?” 

“Suffering,\marginnote{6.1} sir.” 

“But\marginnote{7.1} by not grasping what’s impermanent, suffering, and perishable, would people think ‘I’m better’ or ‘I’m equal’ or ‘I’m worse’?” 

“No,\marginnote{8.1} sir.” … 

“Is\marginnote{10.1} the mind permanent or impermanent?” 

“Impermanent,\marginnote{11.1} sir.” 

“But\marginnote{12.1} if it’s impermanent, is it suffering or happiness?” 

“Suffering,\marginnote{13.1} sir.” 

“But\marginnote{14.1} by not grasping what’s impermanent, suffering, and perishable, would people think ‘I’m better’ or ‘I’m equal’ or ‘I’m worse’?” 

“No,\marginnote{15.1} sir.” 

“Seeing\marginnote{16.1} this, a learned noble disciple grows disillusioned with the eye, ear, nose, tongue, body, and mind. Being disillusioned, desire fades away. When desire fades away they’re freed. When they’re freed, they know they’re freed. 

They\marginnote{16.3} understand: ‘Rebirth is ended, the spiritual journey has been completed, what had to be done has been done, there is no return to any state of existence.’” 

%
\section*{{\suttatitleacronym SN 35.109}{\suttatitletranslation Things Prone to Being Fettered }{\suttatitleroot Saṁyojaniyasutta}}
\addcontentsline{toc}{section}{\tocacronym{SN 35.109} \toctranslation{Things Prone to Being Fettered } \tocroot{Saṁyojaniyasutta}}
\markboth{Things Prone to Being Fettered }{Saṁyojaniyasutta}
\extramarks{SN 35.109}{SN 35.109}

“Mendicants,\marginnote{1.1} I will teach you the things that are prone to being fettered, and the fetter. Listen … 

What\marginnote{1.3} are the things that are prone to being fettered? And what is the fetter? The eye is something that’s prone to being fettered. The desire and greed for it is the fetter. 

The\marginnote{1.6} ear … nose … tongue … body … mind is something that’s prone to being fettered. The desire and greed for it is the fetter. These are called the things that are prone to being fettered, and this is the fetter.” 

%
\section*{{\suttatitleacronym SN 35.110}{\suttatitletranslation Things Prone to Being Grasped }{\suttatitleroot Upādāniyasutta}}
\addcontentsline{toc}{section}{\tocacronym{SN 35.110} \toctranslation{Things Prone to Being Grasped } \tocroot{Upādāniyasutta}}
\markboth{Things Prone to Being Grasped }{Upādāniyasutta}
\extramarks{SN 35.110}{SN 35.110}

“Mendicants,\marginnote{1.1} I will teach you the things that are prone to being grasped, and the grasping. Listen … 

What\marginnote{1.3} are the things that are prone to being grasped? And what is the grasping? The eye is something that’s prone to being grasped. The desire and greed for it is the grasping. 

The\marginnote{1.6} ear … nose … tongue … body … mind is something that’s prone to being grasped. The desire and greed for it is the grasping. These are called the things that are prone to being grasped, and this is the grasping.” 

%
\section*{{\suttatitleacronym SN 35.111}{\suttatitletranslation Complete Understanding of the Interior }{\suttatitleroot Ajjhattikāyatanaparijānanasutta}}
\addcontentsline{toc}{section}{\tocacronym{SN 35.111} \toctranslation{Complete Understanding of the Interior } \tocroot{Ajjhattikāyatanaparijānanasutta}}
\markboth{Complete Understanding of the Interior }{Ajjhattikāyatanaparijānanasutta}
\extramarks{SN 35.111}{SN 35.111}

“Mendicants,\marginnote{1.1} without directly knowing and completely understanding the eye, without dispassion for it and giving it up, you can’t end suffering. 

Without\marginnote{1.2} directly knowing and completely understanding the ear … nose … tongue … body … mind, without dispassion for it and giving it up, you can’t end suffering. 

By\marginnote{1.7} directly knowing and completely understanding the eye, having dispassion for it and giving it up, you can end suffering. 

By\marginnote{1.8} directly knowing and completely understanding the ear … nose … tongue … body … mind, having dispassion for it and giving it up, you can end suffering.” 

%
\section*{{\suttatitleacronym SN 35.112}{\suttatitletranslation Complete Understanding of the Exterior }{\suttatitleroot Bāhirāyatanaparijānanasutta}}
\addcontentsline{toc}{section}{\tocacronym{SN 35.112} \toctranslation{Complete Understanding of the Exterior } \tocroot{Bāhirāyatanaparijānanasutta}}
\markboth{Complete Understanding of the Exterior }{Bāhirāyatanaparijānanasutta}
\extramarks{SN 35.112}{SN 35.112}

“Mendicants,\marginnote{1.1} without directly knowing and completely understanding sights … sounds … smells … tastes … touches … thoughts, without dispassion for them and giving them up, you can’t end suffering. 

By\marginnote{1.7} directly knowing and completely understanding sights … sounds … smells … tastes … touches … thoughts, having dispassion for them and giving them up, you can end suffering.” 

%
\section*{{\suttatitleacronym SN 35.113}{\suttatitletranslation Listening In }{\suttatitleroot Upassutisutta}}
\addcontentsline{toc}{section}{\tocacronym{SN 35.113} \toctranslation{Listening In } \tocroot{Upassutisutta}}
\markboth{Listening In }{Upassutisutta}
\extramarks{SN 35.113}{SN 35.113}

At\marginnote{1.1} one time the Buddha was staying at \textsanskrit{Nādika} in the brick house. Then while the Buddha was in private retreat he spoke this exposition of the teaching: 

“Eye\marginnote{1.3} consciousness arises dependent on the eye and sights. The meeting of the three is contact. Contact is a condition for feeling. Feeling is a condition for craving. Craving is a condition for grasping. Grasping is a condition for continued existence. Continued existence is a condition for rebirth. Rebirth is a condition for old age and death, sorrow, lamentation, pain, sadness, and distress to come to be. That is how this entire mass of suffering originates. 

Ear\marginnote{1.11} … nose … tongue … body … 

Mind\marginnote{1.12} consciousness arises dependent on the mind and thoughts. The meeting of the three is contact. Contact is a condition for feeling. Feeling is a condition for craving. Craving is a condition for grasping. Grasping is a condition for continued existence. Continued existence is a condition for rebirth. Rebirth is a condition for old age and death, sorrow, lamentation, pain, sadness, and distress to come to be. That is how this entire mass of suffering originates. 

Eye\marginnote{2.1} consciousness arises dependent on the eye and sights. The meeting of the three is contact. Contact is a condition for feeling. Feeling is a condition for craving. When that craving fades away and ceases with nothing left over, grasping ceases. When grasping ceases, continued existence ceases. When continued existence ceases, rebirth ceases. When rebirth ceases, old age and death, sorrow, lamentation, pain, sadness, and distress cease. That is how this entire mass of suffering ceases. 

Ear\marginnote{2.9} … nose … tongue … body … 

Mind\marginnote{2.10} consciousness arises dependent on the mind and thoughts. The meeting of the three is contact. Contact is a condition for feeling. Feeling is a condition for craving. When that craving fades away and ceases with nothing left over, grasping ceases. When grasping ceases … That is how this entire mass of suffering ceases.” 

Now\marginnote{3.1} at that time a certain monk was standing listening in on the Buddha. The Buddha saw him and said, “Monk, did you hear that exposition of the teaching?” 

“Yes,\marginnote{3.5} sir.” 

“Learn\marginnote{3.6} that exposition of the teaching, memorize it, and remember it. That exposition of the teaching is beneficial and relates to the fundamentals of the spiritual life.” 

%
\addtocontents{toc}{\let\protect\contentsline\protect\nopagecontentsline}
\chapter*{The Chapter on the World and the Kinds of Sensual Stimulation }
\addcontentsline{toc}{chapter}{\tocchapterline{The Chapter on the World and the Kinds of Sensual Stimulation }}
\addtocontents{toc}{\let\protect\contentsline\protect\oldcontentsline}

%
\section*{{\suttatitleacronym SN 35.114}{\suttatitletranslation Māra’s Snare (1st) }{\suttatitleroot Paṭhamamārapāsasutta}}
\addcontentsline{toc}{section}{\tocacronym{SN 35.114} \toctranslation{Māra’s Snare (1st) } \tocroot{Paṭhamamārapāsasutta}}
\markboth{Māra’s Snare (1st) }{Paṭhamamārapāsasutta}
\extramarks{SN 35.114}{SN 35.114}

“Mendicants,\marginnote{1.1} there are sights known by the eye that are likable, desirable, agreeable, pleasant, sensual, and arousing. If a mendicant approves, welcomes, and keeps clinging to them, they’re called a mendicant trapped in \textsanskrit{Māra}’s lair, fallen under \textsanskrit{Māra}’s sway, and caught in \textsanskrit{Māra}’s snare. They’re bound by \textsanskrit{Māra}’s bonds, and the Wicked One can do with them what he wants. 

There\marginnote{2.1} are sounds … smells … tastes … touches … 

There\marginnote{3.1} are thoughts known by the mind that are likable, desirable, agreeable, pleasant, sensual, and arousing. If a mendicant approves, welcomes, and keep clinging to them, they’re called a mendicant trapped in \textsanskrit{Māra}’s lair, fallen under \textsanskrit{Māra}’s sway, and caught in \textsanskrit{Māra}’s snare. They’re bound by \textsanskrit{Māra}, and the Wicked One can do with them what he wants. 

There\marginnote{4.1} are sights known by the eye that are likable, desirable, agreeable, pleasant, sensual, and arousing. If a mendicant doesn’t approve, welcome, and keep clinging to them, they’re called a mendicant not trapped in \textsanskrit{Māra}’s lair, not fallen under \textsanskrit{Māra}’s sway, and released from \textsanskrit{Māra}’s snare. They’re free from \textsanskrit{Māra}’s bonds, and the Wicked One cannot do with them what he wants. 

There\marginnote{5.1} are sounds … smells … tastes … touches … 

There\marginnote{6.1} are thoughts known by the mind that are likable, desirable, agreeable, pleasant, sensual, and arousing. If a mendicant doesn’t approve, welcome, and keep clinging to them, they’re called a mendicant not trapped in \textsanskrit{Māra}’s lair, not fallen under \textsanskrit{Māra}’s sway, and released from \textsanskrit{Māra}’s snare. They’re free from \textsanskrit{Māra}’s bonds, and the Wicked One cannot do with them what he wants.” 

%
\section*{{\suttatitleacronym SN 35.115}{\suttatitletranslation Māra’s Snare (2nd) }{\suttatitleroot Dutiyamārapāsasutta}}
\addcontentsline{toc}{section}{\tocacronym{SN 35.115} \toctranslation{Māra’s Snare (2nd) } \tocroot{Dutiyamārapāsasutta}}
\markboth{Māra’s Snare (2nd) }{Dutiyamārapāsasutta}
\extramarks{SN 35.115}{SN 35.115}

“Mendicants,\marginnote{1.1} there are sights known by the eye that are likable, desirable, agreeable, pleasant, sensual, and arousing. If a mendicant approves, welcomes, and keeps clinging to them, they’re called a mendicant who is bound when it comes to sights known by the eye. They’re trapped in \textsanskrit{Māra}’s lair, fallen under \textsanskrit{Māra}’s sway, and caught in \textsanskrit{Māra}’s snare. They’re bound by \textsanskrit{Māra}’s bonds, and the Wicked One can do with them what he wants. 

There\marginnote{2.1} are sounds … smells … tastes … touches … thoughts known by the mind that are likable, desirable, agreeable, pleasant, sensual, and arousing. If a mendicant approves, welcomes, and keeps clinging to them, they’re called a mendicant who is bound when it comes to thoughts known by the mind. They’re trapped in \textsanskrit{Māra}’s lair, fallen under \textsanskrit{Māra}’s sway, and caught in \textsanskrit{Māra}’s snare. They’re bound by \textsanskrit{Māra}’s bonds, and the Wicked One can do with them what he wants. 

There\marginnote{3.1} are sights known by the eye that are likable, desirable, agreeable, pleasant, sensual, and arousing. If a mendicant doesn’t approve, welcome, and keep clinging to them, they’re called a mendicant not trapped in \textsanskrit{Māra}’s lair, not fallen under \textsanskrit{Māra}’s sway, and released from \textsanskrit{Māra}’s snare. They’re free from \textsanskrit{Māra}’s bonds, and the Wicked One cannot do with them what he wants. 

There\marginnote{4.1} are sounds … smells … tastes … touches … thoughts known by the mind that are likable, desirable, agreeable, pleasant, sensual, and arousing. If a mendicant doesn’t approve, welcome, and keep clinging to them, they’re called a mendicant not trapped in \textsanskrit{Māra}’s lair, not fallen under \textsanskrit{Māra}’s sway, and released from \textsanskrit{Māra}’s snare. They’re free from \textsanskrit{Māra}’s bonds, and the Wicked One cannot do with them what he wants.” 

%
\section*{{\suttatitleacronym SN 35.116}{\suttatitletranslation Traveling to the End of the World }{\suttatitleroot Lokantagamanasutta}}
\addcontentsline{toc}{section}{\tocacronym{SN 35.116} \toctranslation{Traveling to the End of the World } \tocroot{Lokantagamanasutta}}
\markboth{Traveling to the End of the World }{Lokantagamanasutta}
\extramarks{SN 35.116}{SN 35.116}

“Mendicants,\marginnote{1.1} I say it’s not possible to know or see or reach the end of the world by traveling. But I also say there’s no making an end of suffering without reaching the end of the world.” 

When\marginnote{1.3} he had spoken, the Blessed One got up from his seat and entered his dwelling. 

Soon\marginnote{1.4} after the Buddha left, those mendicants considered, “The Buddha gave this brief passage for recitation, then entered his dwelling without explaining the meaning in detail. … Who can explain in detail the meaning of this brief passage for recitation given by the Buddha?” 

Then\marginnote{2.1} those mendicants thought, “This Venerable Ānanda is praised by the Buddha and esteemed by his sensible spiritual companions. He is capable of explaining in detail the meaning of this brief passage for recitation given by the Buddha. Let’s go to him, and ask him about this matter.” 

Then\marginnote{3.1} those mendicants went to Ānanda and exchanged greetings with him. When the greetings and polite conversation were over, they sat down to one side. They told him what had happened, and said, “May Venerable Ānanda please explain this.” 

“Reverends,\marginnote{5.1} suppose there was a person in need of heartwood. And while wandering in search of heartwood he’d come across a large tree standing with heartwood. But he’d pass over the roots and trunk, imagining that the heartwood should be sought in the branches and leaves. 

Such\marginnote{5.2} is the consequence for the venerables. Though you were face to face with the Buddha, you overlooked him, imagining that you should ask me about this matter. For he is the Buddha, who knows and sees. He is vision, he is knowledge, he is the truth, he is supreme. He is the teacher, the proclaimer, the elucidator of meaning, the bestower of the deathless, the lord of truth, the Realized One. That was the time to approach the Buddha and ask about this matter. You should have remembered it in line with the Buddha’s answer.” 

“Certainly\marginnote{6.1} he is the Buddha, who knows and sees. He is vision, he is knowledge, he is the truth, he is supreme. He is the teacher, the proclaimer, the elucidator of meaning, the bestower of the deathless, the lord of truth, the Realized One. That was the time to approach the Buddha and ask about this matter. We should have remembered it in line with the Buddha’s answer. 

Still,\marginnote{6.5} Venerable Ānanda is praised by the Buddha and esteemed by his sensible spiritual companions. You are capable of explaining in detail the meaning of this brief passage for recitation given by the Buddha. Please explain this, if it’s no trouble.” 

“Then\marginnote{7.1} listen and pay close attention, I will speak.” 

“Yes,\marginnote{7.2} reverend,” they replied. Ānanda said this: 

“Reverends,\marginnote{8.1} the Buddha gave this brief passage for recitation, then entered his dwelling without explaining the meaning in detail: 

‘Mendicants,\marginnote{8.2} I say it’s not possible to know or see or reach the end of the world by traveling. But I also say there’s no making an end of suffering without reaching the end of the world.’ 

This\marginnote{8.4} is how I understand the detailed meaning of this passage for recitation. 

Whatever\marginnote{8.5} in the world through which you perceive the world and conceive the world is called the world in the training of the Noble One. And through what in the world do you perceive the world and conceive the world? 

Through\marginnote{8.8} the eye in the world you perceive the world and conceive the world. Through the ear … nose … tongue … body … mind in the world you perceive the world and conceive the world. 

Whatever\marginnote{8.14} in the world through which you perceive the world and conceive the world is called the world in the training of the Noble One. 

When\marginnote{8.16} the Buddha gave this brief passage for recitation, then entered his dwelling without explaining the meaning in detail: 

‘Mendicants,\marginnote{8.17} I say it’s not possible to know or see or reach the end of the world by traveling. But I also say there’s no making an end of suffering without reaching the end of the world.’ 

That\marginnote{8.19} is how I understand the detailed meaning of this summary. 

If\marginnote{8.20} you wish, you may go to the Buddha and ask him about this. You should remember it in line with the Buddha’s answer.” 

“Yes,\marginnote{9.1} reverend,” replied those mendicants. Then they rose from their seats and went to the Buddha, bowed, sat down to one side, and told him what had happened. 

Then\marginnote{10.1} they said, “And Ānanda explained the meaning to us in this manner, with these words and phrases.” 

“Mendicants,\marginnote{11.1} Ānanda is astute, he has great wisdom. If you came to me and asked this question, I would answer it in exactly the same way as Ānanda. That is what it means, and that’s how you should remember it.” 

%
\section*{{\suttatitleacronym SN 35.117}{\suttatitletranslation The Kinds of Sensual Stimulation }{\suttatitleroot Kāmaguṇasutta}}
\addcontentsline{toc}{section}{\tocacronym{SN 35.117} \toctranslation{The Kinds of Sensual Stimulation } \tocroot{Kāmaguṇasutta}}
\markboth{The Kinds of Sensual Stimulation }{Kāmaguṇasutta}
\extramarks{SN 35.117}{SN 35.117}

“Mendicants,\marginnote{1.1} before my awakening—when I was still unawakened but intent on awakening—I thought: ‘My mind might often stray towards the five kinds of sensual stimulation that I formerly experienced—which have passed, ceased, and perished—or to those in the present, or in the future a little.’ 

Then\marginnote{1.3} it occurred to me: ‘In my own way I should practice diligence, mindfulness, and protecting the mind regarding the five kinds of sensual stimulation that I formerly experienced—which have passed, ceased, and perished.’ 

So,\marginnote{1.5} mendicants, your minds might also often stray towards the five kinds of sensual stimulation that you formerly experienced—which have passed, ceased, and perished—or to those in the present, or in the future a little. So in your own way you should practice diligence, mindfulness, and protecting the mind regarding the five kinds of sensual stimulation that you formerly experienced—which have passed, ceased, and perished. 

So\marginnote{1.7} you should understand that dimension where the eye ceases and perception of sights fades away. You should understand that dimension where the ear … nose … tongue … body … mind ceases and perception of thoughts fades away.” 

When\marginnote{1.10} he had spoken, the Blessed One got up from his seat and entered his dwelling. 

Soon\marginnote{2.1} after the Buddha left, those mendicants considered, “The Buddha gave this brief summary, then entered his dwelling without explaining the meaning in detail. … Who can explain in detail the meaning of this brief summary given by the Buddha?” 

Then\marginnote{3.1} those mendicants thought, “This Venerable Ānanda is praised by the Buddha and esteemed by his sensible spiritual companions. He is capable of explaining in detail the meaning of this brief summary given by the Buddha. Let’s go to him, and ask him about this matter.” 

Then\marginnote{4.1} those mendicants went to Ānanda, and exchanged greetings with him. When the greetings and polite conversation were over, they sat down to one side. They told him what had happened, and said, “May Venerable Ānanda please explain this.” 

“Reverends,\marginnote{6.1} suppose there was a person in need of heartwood. …” 

“Please\marginnote{6.2} explain this, if it’s no trouble.” 

“Then\marginnote{7.1} listen and pay close attention, I will speak.” 

“Yes,\marginnote{7.2} reverend,” they replied. Ānanda said this: 

“Reverends,\marginnote{8.1} the Buddha gave this brief summary, then entered his dwelling without explaining the meaning in detail: 

‘So\marginnote{8.2} you should understand that dimension where the eye ceases and perception of sights fades away. You should understand that dimension where the ear … nose … tongue … body … mind ceases and perception of thoughts fades away.’ 

And\marginnote{8.4} this is how I understand the detailed meaning of this summary. 

The\marginnote{8.5} Buddha was referring to the cessation of the six sense fields when he said: ‘So you should understand that dimension where the eye ceases and perception of sights fades away. You should understand that dimension where the ear … nose … tongue … body … mind ceases and perception of thoughts fades away.’ 

The\marginnote{8.8} Buddha gave this brief summary, then entered his dwelling without explaining the meaning in detail. And this is how I understand the detailed meaning of this summary. 

If\marginnote{8.12} you wish, you may go to the Buddha and ask him about this. You should remember it in line with the Buddha’s answer.” 

“Yes,\marginnote{9.1} reverend,” replied those mendicants. Then they rose from their seats and went to the Buddha, bowed, sat down to one side, and told him what had happened. 

Then\marginnote{10.1} they said, “And Ānanda explained the meaning to us in this manner, with these words and phrases.” 

“Mendicants,\marginnote{11.1} Ānanda is astute, he has great wisdom. If you came to me and asked this question, I would answer it in exactly the same way as Ānanda. That is what it means, and that’s how you should remember it.” 

%
\section*{{\suttatitleacronym SN 35.118}{\suttatitletranslation The Question of Sakka }{\suttatitleroot Sakkapañhasutta}}
\addcontentsline{toc}{section}{\tocacronym{SN 35.118} \toctranslation{The Question of Sakka } \tocroot{Sakkapañhasutta}}
\markboth{The Question of Sakka }{Sakkapañhasutta}
\extramarks{SN 35.118}{SN 35.118}

At\marginnote{1.1} one time the Buddha was staying near \textsanskrit{Rājagaha}, on the Vulture’s Peak Mountain. And then Sakka, lord of gods, went up to the Buddha, bowed, stood to one side, and said to him: 

“What\marginnote{1.3} is the cause, sir, what is the reason why some sentient beings aren’t fully extinguished in the present life? What is the cause, what is the reason why some sentient beings are fully extinguished in the present life?” 

“Lord\marginnote{2.1} of gods, there are sights known by the eye that are likable, desirable, agreeable, pleasant, sensual, and arousing. If a mendicant approves, welcomes, and keeps clinging to them, their consciousness relies on that and grasps it. A mendicant with grasping does not become extinguished. 

There\marginnote{3.1} are sounds … smells … tastes … touches … thoughts known by the mind that are likable, desirable, agreeable, pleasant, sensual, and arousing. If a mendicant approves, welcomes, and keeps clinging to them, their consciousness relies on that and grasps it. A mendicant with grasping does not become extinguished. That’s the cause, that’s the reason why some sentient beings aren’t fully extinguished in the present life. 

There\marginnote{4.1} are sights known by the eye that are likable, desirable, agreeable, pleasant, sensual, and arousing. If a mendicant doesn’t approve, welcome, and keep clinging to them, their consciousness doesn’t rely on that and grasp it. A mendicant free of grasping becomes extinguished. 

There\marginnote{5.1} are sounds … smells … tastes … touches … thoughts known by the mind that are likable, desirable, agreeable, pleasant, sensual, and arousing. If a mendicant doesn’t approve, welcome, and keep clinging to them, their consciousness doesn’t rely on that and grasp it. A mendicant free of grasping becomes extinguished. That’s the cause, that’s the reason why some sentient beings are fully extinguished in the present life.” 

%
\section*{{\suttatitleacronym SN 35.119}{\suttatitletranslation The Question of Pañcasikha }{\suttatitleroot Pañcasikhasutta}}
\addcontentsline{toc}{section}{\tocacronym{SN 35.119} \toctranslation{The Question of Pañcasikha } \tocroot{Pañcasikhasutta}}
\markboth{The Question of Pañcasikha }{Pañcasikhasutta}
\extramarks{SN 35.119}{SN 35.119}

At\marginnote{1.1} one time the Buddha was staying near \textsanskrit{Rājagaha}, on the Vulture’s Peak Mountain. And then the fairy \textsanskrit{Pañcasikha} went up to the Buddha, bowed, stood to one side, and said to him: 

“What\marginnote{1.3} is the cause, sir, what is the reason why some sentient beings aren’t fully extinguished in the present life? What is the cause, sir, what is the reason why some sentient beings are fully extinguished in the present life?” 

“\textsanskrit{Pañcasikha},\marginnote{1.5} there are sights known by the eye … 

thoughts\marginnote{1.6} known by the mind that are likable, desirable, agreeable, pleasant, sensual, and arousing. If a mendicant approves, welcomes, and keeps clinging to them, their consciousness relies on that and grasps it. A mendicant with grasping does not become extinguished. That’s the cause, that’s the reason why some sentient beings aren’t fully extinguished in the present life. 

There\marginnote{2.1} are sights known by the eye … 

thoughts\marginnote{2.2} known by the mind that are likable, desirable, agreeable, pleasant, sensual, and arousing. If a mendicant doesn’t approve, welcome, and keep clinging to them, their consciousness doesn’t rely on that and grasp it. A mendicant free of grasping becomes extinguished. That’s the cause, that’s the reason why some sentient beings are fully extinguished in the present life.” 

%
\section*{{\suttatitleacronym SN 35.120}{\suttatitletranslation Sāriputta and the Pupil }{\suttatitleroot Sāriputtasaddhivihārikasutta}}
\addcontentsline{toc}{section}{\tocacronym{SN 35.120} \toctranslation{Sāriputta and the Pupil } \tocroot{Sāriputtasaddhivihārikasutta}}
\markboth{Sāriputta and the Pupil }{Sāriputtasaddhivihārikasutta}
\extramarks{SN 35.120}{SN 35.120}

At\marginnote{1.1} one time Venerable \textsanskrit{Sāriputta} was staying near \textsanskrit{Sāvatthī} in Jeta’s Grove, \textsanskrit{Anāthapiṇḍika}’s monastery. Then a certain mendicant went up to Venerable \textsanskrit{Sāriputta}, and exchanged greetings with him. 

When\marginnote{1.3} the greetings and polite conversation were over, he sat down to one side, and said to him, “Reverend \textsanskrit{Sāriputta}, a mendicant pupil of mine has resigned the training and returned to a lesser life.” 

“That’s\marginnote{2.1} how it is, reverend, when someone doesn’t guard the sense doors, eats too much, and is not committed to wakefulness. It’s not possible for such a mendicant to maintain the full and pure spiritual life for the rest of their life. But it is possible for a mendicant to maintain the full and pure spiritual life for the rest of their life if they guard the sense doors, eat in moderation, and are committed to wakefulness. 

And\marginnote{3.1} how does someone guard the sense doors? When a mendicant sees a sight with the eyes, they don’t get caught up in the features and details. If the faculty of sight were left unrestrained, bad unskillful qualities of desire and aversion would become overwhelming. For this reason, they practice restraint, protecting the faculty of sight, and achieving its restraint. When they hear a sound with their ears … When they smell an odor with their nose … When they taste a flavor with their tongue … When they feel a touch with their body … When they know a thought with their mind, they don’t get caught up in the features and details. If the faculty of mind were left unrestrained, bad unskillful qualities of desire and aversion would become overwhelming. For this reason, they practice restraint, protecting the faculty of mind, and achieving its restraint. That’s how someone guards the sense doors. 

And\marginnote{4.1} how does someone eat in moderation? It’s when a mendicant reflects properly on the food that they eat: ‘Not for fun, indulgence, adornment, or decoration, but only to sustain this body, to avoid harm, and to support spiritual practice. In this way, I shall put an end to old discomfort and not give rise to new discomfort, and I will live blamelessly and at ease.’ That’s how someone eats in moderation. 

And\marginnote{5.1} how is someone committed to wakefulness? It’s when a mendicant practices walking and sitting meditation by day, purifying their mind from obstacles. In the evening, they continue to practice walking and sitting meditation. In the middle of the night, they lie down in the lion’s posture—on the right side, placing one foot on top of the other—mindful and aware, and focused on the time of getting up. In the last part of the night, they get up and continue to practice walking and sitting meditation, purifying their mind from obstacles. That’s how someone is committed to wakefulness. 

So\marginnote{5.7} you should train like this: ‘We will guard the sense doors, eat in moderation, and be committed to wakefulness.’ That’s how you should train.” 

%
\section*{{\suttatitleacronym SN 35.121}{\suttatitletranslation Advice to Rāhula }{\suttatitleroot Rāhulovādasutta}}
\addcontentsline{toc}{section}{\tocacronym{SN 35.121} \toctranslation{Advice to Rāhula } \tocroot{Rāhulovādasutta}}
\markboth{Advice to Rāhula }{Rāhulovādasutta}
\extramarks{SN 35.121}{SN 35.121}

At\marginnote{1.1} one time the Buddha was staying near \textsanskrit{Sāvatthī} in Jeta’s Grove, \textsanskrit{Anāthapiṇḍika}’s monastery. 

Then\marginnote{1.2} as he was in private retreat this thought came to his mind, “The qualities that ripen in freedom have ripened in \textsanskrit{Rāhula}. Why don’t I lead him further to the ending of defilements?” 

Then\marginnote{1.5} the Buddha robed up in the morning and, taking his bowl and robe, wandered for alms in \textsanskrit{Sāvatthī}. After the meal, on his return from almsround, he addressed Venerable \textsanskrit{Rāhula}, “\textsanskrit{Rāhula}, get your sitting cloth. Let’s go to the Dark Forest for the day’s meditation.” 

“Yes,\marginnote{1.9} sir,” replied \textsanskrit{Rāhula}. Taking his sitting cloth he followed behind the Buddha. 

Now\marginnote{2.1} at that time many thousands of deities followed the Buddha, thinking, “Today the Buddha will lead \textsanskrit{Rāhula} further to the ending of defilements!” 

Then\marginnote{2.3} the Buddha plunged deep into the Dark Forest and sat at the root of a tree on the seat spread out. \textsanskrit{Rāhula} bowed to the Buddha and sat down to one side. The Buddha said to him: 

“What\marginnote{3.1} do you think, \textsanskrit{Rāhula}? Is the eye permanent or impermanent?” 

“Impermanent,\marginnote{4.1} sir.” 

“But\marginnote{5.1} if it’s impermanent, is it suffering or happiness?” 

“Suffering,\marginnote{6.1} sir.” 

“But\marginnote{7.1} if it’s impermanent, suffering, and liable to wear out, is it fit to be regarded thus: ‘This is mine, I am this, this is my self’?” 

“No,\marginnote{8.1} sir.” 

“Are\marginnote{9.1} sights … eye consciousness … eye contact permanent or impermanent?” 

“Impermanent,\marginnote{14.1} sir.” … 

“Anything\marginnote{15.1} included in feeling, perception, choices, and consciousness that arises conditioned by eye contact: is that permanent or impermanent?” 

“Impermanent,\marginnote{16.1} sir.” 

“But\marginnote{17.1} if it’s impermanent, is it suffering or happiness?” 

“Suffering,\marginnote{18.1} sir.” 

“But\marginnote{19.1} if it’s impermanent, suffering, and liable to wear out, is it fit to be regarded thus: ‘This is mine, I am this, this is my self’?” 

“No,\marginnote{20.1} sir.” 

“Is\marginnote{21.1} the ear … nose … tongue … body … mind permanent or impermanent?” 

“Impermanent,\marginnote{34.1} sir.” 

“But\marginnote{35.1} if it’s impermanent, is it suffering or happiness?” 

“Suffering,\marginnote{36.1} sir.” 

“But\marginnote{37.1} if it’s impermanent, suffering, and liable to wear out, is it fit to be regarded thus: ‘This is mine, I am this, this is my self’?” 

“No,\marginnote{38.1} sir.” 

“Are\marginnote{39.1} thoughts … mind consciousness … mind contact permanent or impermanent?” 

“Impermanent,\marginnote{44.1} sir.” … 

“Anything\marginnote{45.1} included in feeling, perception, choices, and consciousness that arises conditioned by mind contact: is that permanent or impermanent?” 

“Impermanent,\marginnote{46.1} sir.” 

“But\marginnote{47.1} if it’s impermanent, is it suffering or happiness?” 

“Suffering,\marginnote{48.1} sir.” 

“But\marginnote{49.1} if it’s impermanent, suffering, and liable to wear out, is it fit to be regarded thus: ‘This is mine, I am this, this is my self’?” 

“No,\marginnote{50.1} sir.” 

“Seeing\marginnote{51.1} this, a learned noble disciple grows disillusioned with the eye, sights, eye consciousness, and eye contact. And they become disillusioned with anything included in feeling, perception, choices, and consciousness that arises conditioned by eye contact. 

They\marginnote{51.2} grow disillusioned with the ear … nose … tongue … body … 

They\marginnote{52.1} grow disillusioned with the mind, thoughts, mind consciousness, and mind contact. And they grow disillusioned with anything included in feeling, perception, choices, and consciousness that arises conditioned by mind contact. 

Being\marginnote{52.2} disillusioned, desire fades away. When desire fades away they’re freed. When they’re freed, they know they’re freed. 

They\marginnote{52.3} understand: ‘Rebirth is ended, the spiritual journey has been completed, what had to be done has been done, there is no return to any state of existence.’” 

That\marginnote{53.1} is what the Buddha said. Satisfied, Venerable \textsanskrit{Rāhula} was happy with what the Buddha said. And while this discourse was being spoken, \textsanskrit{Rāhula}’s mind was freed from defilements by not grasping. 

And\marginnote{53.4} the stainless, immaculate vision of the Dhamma arose in those thousands of deities: 

“Everything\marginnote{53.5} that has a beginning has an end.” 

%
\section*{{\suttatitleacronym SN 35.122}{\suttatitletranslation Things Prone to Being Fettered }{\suttatitleroot Saṁyojaniyadhammasutta}}
\addcontentsline{toc}{section}{\tocacronym{SN 35.122} \toctranslation{Things Prone to Being Fettered } \tocroot{Saṁyojaniyadhammasutta}}
\markboth{Things Prone to Being Fettered }{Saṁyojaniyadhammasutta}
\extramarks{SN 35.122}{SN 35.122}

“Mendicants,\marginnote{1.1} I will teach you the things that are prone to being fettered, and the fetter. Listen … 

What\marginnote{1.3} are the things that are prone to being fettered? And what is the fetter? There are sights known by the eye that are likable, desirable, agreeable, pleasant, sensual, and arousing. These are called the things that are prone to being fettered. The desire and greed for them is the fetter. 

There\marginnote{1.7} are sounds … smells … tastes … touches … thoughts known by the mind that are likable, desirable, agreeable, pleasant, sensual, and arousing. These are called the things that are prone to being fettered. The desire and greed for them is the fetter.” 

%
\section*{{\suttatitleacronym SN 35.123}{\suttatitletranslation Things Prone to Being Grasped }{\suttatitleroot Upādāniyadhammasutta}}
\addcontentsline{toc}{section}{\tocacronym{SN 35.123} \toctranslation{Things Prone to Being Grasped } \tocroot{Upādāniyadhammasutta}}
\markboth{Things Prone to Being Grasped }{Upādāniyadhammasutta}
\extramarks{SN 35.123}{SN 35.123}

“Mendicants,\marginnote{1.1} I will teach you the things that are prone to being grasped, and the grasping. Listen … 

What\marginnote{1.3} are the things that are prone to being grasped? And what is the grasping? There are sights known by the eye that are likable, desirable, agreeable, pleasant, sensual, and arousing. These are called the things that are prone to being grasped. The desire and greed for them is the grasping. 

There\marginnote{1.7} are sounds … smells … tastes … touches … thoughts known by the mind that are likable, desirable, agreeable, pleasant, sensual, and arousing. These are called the things that are prone to being grasped. The desire and greed for them is the grasping.” 

%
\addtocontents{toc}{\let\protect\contentsline\protect\nopagecontentsline}
\chapter*{The Chapter on Householders }
\addcontentsline{toc}{chapter}{\tocchapterline{The Chapter on Householders }}
\addtocontents{toc}{\let\protect\contentsline\protect\oldcontentsline}

%
\section*{{\suttatitleacronym SN 35.124}{\suttatitletranslation At Vesālī }{\suttatitleroot Vesālīsutta}}
\addcontentsline{toc}{section}{\tocacronym{SN 35.124} \toctranslation{At Vesālī } \tocroot{Vesālīsutta}}
\markboth{At Vesālī }{Vesālīsutta}
\extramarks{SN 35.124}{SN 35.124}

At\marginnote{1.1} one time the Buddha was staying near \textsanskrit{Vesālī}, at the Great Wood, in the hall with the peaked roof. Then the householder Ugga of \textsanskrit{Vesālī} went up to the Buddha, sat down to one side, and said to him: 

“What\marginnote{1.3} is the cause, sir, what is the reason why some sentient beings aren’t fully extinguished in the present life? What is the cause, sir, what is the reason why some sentient beings are fully extinguished in the present life?” 

“Householder,\marginnote{2.1} there are sights known by the eye that are likable, desirable, agreeable, pleasant, sensual, and arousing. If a mendicant approves, welcomes, and keeps clinging to them, their consciousness relies on that and grasps it. A mendicant with grasping does not become extinguished. 

There\marginnote{2.4} are sounds … smells … tastes … touches … thoughts known by the mind that are likable, desirable, agreeable, pleasant, sensual, and arousing. If a mendicant approves, welcomes, and keeps clinging to them, their consciousness relies on that and grasps it. A mendicant with grasping does not become extinguished. 

That’s\marginnote{2.8} the cause, that’s the reason why some sentient beings aren’t fully extinguished in the present life. 

There\marginnote{3.1} are sights known by the eye that are likable, desirable, agreeable, pleasant, sensual, and arousing. If a mendicant doesn’t approve, welcome, and keep clinging to them, their consciousness doesn’t rely on that and grasp it. A mendicant free of grasping becomes extinguished. 

There\marginnote{3.4} are sounds … smells … tastes … touches … thoughts known by the mind that are likable, desirable, agreeable, pleasant, sensual, and arousing. If a mendicant doesn’t approve, welcome, and keep clinging to them, their consciousness doesn’t rely on that and grasp it. A mendicant free of grasping becomes extinguished. 

That’s\marginnote{3.8} the cause, that’s the reason why some sentient beings are fully extinguished in the present life.” 

%
\section*{{\suttatitleacronym SN 35.125}{\suttatitletranslation In the Land of the Vajjis }{\suttatitleroot Vajjīsutta}}
\addcontentsline{toc}{section}{\tocacronym{SN 35.125} \toctranslation{In the Land of the Vajjis } \tocroot{Vajjīsutta}}
\markboth{In the Land of the Vajjis }{Vajjīsutta}
\extramarks{SN 35.125}{SN 35.125}

At\marginnote{1.1} one time the Buddha was staying in the land of the Vajjis at the village of Hatthi. Then the householder Ugga of Hatthi went up to the Buddha, sat down to one side, and said to him: 

“What\marginnote{1.3} is the cause, sir, what is the reason why some sentient beings aren’t fully extinguished in the present life? What is the cause, sir, what is the reason why some sentient beings are fully extinguished in the present life?” … 

(This\marginnote{1.5} should be told in full as in the previous discourse.) 

%
\section*{{\suttatitleacronym SN 35.126}{\suttatitletranslation At Nālandā }{\suttatitleroot Nāḷandasutta}}
\addcontentsline{toc}{section}{\tocacronym{SN 35.126} \toctranslation{At Nālandā } \tocroot{Nāḷandasutta}}
\markboth{At Nālandā }{Nāḷandasutta}
\extramarks{SN 35.126}{SN 35.126}

At\marginnote{1.1} one time the Buddha was staying near \textsanskrit{Nālandā} in \textsanskrit{Pāvārika}’s mango grove. 

Then\marginnote{1.2} the householder \textsanskrit{Upāli} went up to the Buddha … and asked him, “What is the cause, sir, what is the reason why some sentient beings aren’t fully extinguished in the present life? What is the cause, sir, what is the reason why some sentient beings are fully extinguished in the present life?” … 

(This\marginnote{1.5} should be told in full as in SN 35.124.) 

%
\section*{{\suttatitleacronym SN 35.127}{\suttatitletranslation With Bhāradvāja }{\suttatitleroot Bhāradvājasutta}}
\addcontentsline{toc}{section}{\tocacronym{SN 35.127} \toctranslation{With Bhāradvāja } \tocroot{Bhāradvājasutta}}
\markboth{With Bhāradvāja }{Bhāradvājasutta}
\extramarks{SN 35.127}{SN 35.127}

At\marginnote{1.1} one time Venerable \textsanskrit{Bhāradvāja} the Alms-gatherer was staying near Kosambi, in Ghosita’s Monastery. Then King Udena went up to \textsanskrit{Bhāradvāja} the Alms-gatherer and exchanged greetings with him. When the greetings and polite conversation were over, he sat down to one side, and said to him: 

“Master\marginnote{1.4} \textsanskrit{Bhāradvāja}, there are these young monks who are youthful, black-haired, blessed with youth, in the prime of life; and they’ve never played around with sensual pleasures. What is the cause, what is the reason why they practice the full and pure spiritual life as long as they live, maintaining it for a long time?” 

“Great\marginnote{1.5} king, this has been stated by the Blessed One, who knows and sees, the perfected one, the fully awakened Buddha: 

‘Please,\marginnote{1.6} monks, think of women your mother’s age as your mother. Think of women your sister’s age as your sister. And think of women your daughter’s age as your daughter.’ 

This\marginnote{1.7} is a cause, great king, this is a reason why these young monks practice the full and pure spiritual life as long as they live, maintaining it for a long time.” 

“But\marginnote{2.1} Master \textsanskrit{Bhāradvāja}, the mind is wanton. Sometimes thoughts of desire come up even for women your mother’s age, your sister’s age, or your daughter’s age. Is there another cause, another reason why these young monks live the full and pure spiritual life for their entire life?” 

“Great\marginnote{3.1} king, this has been stated by the Blessed One, who knows and sees, the perfected one, the fully awakened Buddha: 

‘Please,\marginnote{3.2} monks, examine your own body up from the soles of the feet and down from the tips of the hairs, wrapped in skin and full of many kinds of filth. In this body there is head hair, body hair, nails, teeth, skin, flesh, sinews, bones, bone marrow, kidneys, heart, liver, diaphragm, spleen, lungs, intestines, mesentery, undigested food, feces, bile, phlegm, pus, blood, sweat, fat, tears, grease, saliva, snot, synovial fluid, urine.’ 

This\marginnote{3.4} is also a cause, great king, this is a reason why these young monks live the full and pure spiritual life for their entire life, maintaining it for a long time.” 

“This\marginnote{3.5} is easy to do for those mendicants who have developed their physical endurance, ethics, mind, and wisdom. But it’s hard to do for those mendicants who have not developed their physical endurance, ethics, mind, and wisdom. Sometimes I plan to focus on something as ugly, but only its beauty comes to mind. Is there another cause, another reason why these young monks live the full and pure spiritual life for their entire life?” 

“Great\marginnote{4.1} king, this has been stated by the Blessed One, who knows and sees, the perfected one, the fully awakened Buddha: 

‘Please,\marginnote{4.2} monks, live with sense doors guarded. When you see a sight with your eyes, don’t get caught up in the features and details. If the faculty of sight were left unrestrained, bad unskillful qualities of desire and aversion would become overwhelming. For this reason, practice restraint, protect the faculty of sight, and achieve its restraint. When you hear a sound with your ears … When you smell an odor with your nose … When you taste a flavor with your tongue … When you feel a touch with your body … When you know a thought with your mind, don’t get caught up in the features and details. If the faculty of mind were left unrestrained, bad unskillful qualities of desire and aversion would become overwhelming. For this reason, practice restraint, protect the faculty of mind, and achieve its restraint.’ 

This\marginnote{4.11} is also a cause, great king, this is a reason why these young monks practice the full and pure spiritual life as long as they live, maintaining it for a long time.” 

“It’s\marginnote{5.1} incredible, Master \textsanskrit{Bhāradvāja}, it’s amazing! How well this was said by the Buddha! This is the real cause, this is the reason why these young monks practice the full and pure spiritual life as long as they live, maintaining it for a long time. 

For\marginnote{5.4} sometimes I too enter the harem with unprotected body, speech, mind, mindfulness, and sense faculties. At those times powerful thoughts of desire get the better of me. But sometimes I enter the harem with protected body, speech, mind, mindfulness, and sense faculties. At those times such thoughts of desire don’t get the better of me. 

Excellent,\marginnote{5.6} Master \textsanskrit{Bhāradvāja}! Excellent! As if he were righting the overturned, or revealing the hidden, or pointing out the path to the lost, or lighting a lamp in the dark so people with good eyes can see what’s there, Master \textsanskrit{Bhāradvāja} has made the teaching clear in many ways. I go for refuge to the Buddha, to the teaching, and to the mendicant \textsanskrit{Saṅgha}. From this day forth, may Master \textsanskrit{Bhāradvāja} remember me as a lay follower who has gone for refuge for life.” 

%
\section*{{\suttatitleacronym SN 35.128}{\suttatitletranslation With Soṇa }{\suttatitleroot Soṇasutta}}
\addcontentsline{toc}{section}{\tocacronym{SN 35.128} \toctranslation{With Soṇa } \tocroot{Soṇasutta}}
\markboth{With Soṇa }{Soṇasutta}
\extramarks{SN 35.128}{SN 35.128}

At\marginnote{1.1} one time the Buddha was staying near \textsanskrit{Rājagaha}, in the Bamboo Grove, the squirrels’ feeding ground. 

Then\marginnote{1.2} the householder \textsanskrit{Soṇa} went up to the Buddha, bowed, sat down to one side, and said to him: 

“What\marginnote{1.3} is the cause, sir, what is the reason why some sentient beings aren’t fully extinguished in the present life? What is the cause, sir, what is the reason why some sentient beings are fully extinguished in the present life?” … 

(This\marginnote{1.5} should be told in full as in SN 35.118.) 

%
\section*{{\suttatitleacronym SN 35.129}{\suttatitletranslation With Ghosita }{\suttatitleroot Ghositasutta}}
\addcontentsline{toc}{section}{\tocacronym{SN 35.129} \toctranslation{With Ghosita } \tocroot{Ghositasutta}}
\markboth{With Ghosita }{Ghositasutta}
\extramarks{SN 35.129}{SN 35.129}

At\marginnote{1.1} one time Venerable Ānanda was staying near Kosambi, in Ghosita’s Monastery. Then the householder Ghosita went up to Venerable Ānanda, and said to him: 

“Sir,\marginnote{1.3} Ānanda, they speak of ‘the diversity of elements’. In what way did the Buddha speak of the diversity of elements?” 

“Householder,\marginnote{1.5} the eye element is found, as are agreeable sights, and eye consciousness. Pleasant feeling arises dependent on a contact to be experienced as pleasant. The eye element is found, as are disagreeable sights, and eye consciousness. Painful feeling arises dependent on a contact to be experienced as painful. The eye element is found, as are sights that are a basis for equanimity, and eye consciousness. Neutral feeling arises dependent on a contact to be experienced as neutral. 

The\marginnote{1.11} ear … nose … tongue … body … mind element is found, as are agreeable thoughts, and mind consciousness. Pleasant feeling arises dependent on a contact to be experienced as pleasant. The mind element is found, as are disagreeable thoughts, and mind consciousness. Painful feeling arises dependent on a contact to be experienced as painful. The mind element is found, as are thoughts that are a basis for equanimity, and mind consciousness. Neutral feeling arises dependent on a contact to be experienced as neutral. 

This\marginnote{1.23} is how the Buddha spoke of the diversity of elements.” 

%
\section*{{\suttatitleacronym SN 35.130}{\suttatitletranslation With Hāliddikāni }{\suttatitleroot Hāliddikānisutta}}
\addcontentsline{toc}{section}{\tocacronym{SN 35.130} \toctranslation{With Hāliddikāni } \tocroot{Hāliddikānisutta}}
\markboth{With Hāliddikāni }{Hāliddikānisutta}
\extramarks{SN 35.130}{SN 35.130}

At\marginnote{1.1} one time Venerable \textsanskrit{Mahākaccāna} was staying in the land of the Avantis near Kuraraghara on Steep Mountain. 

Then\marginnote{1.2} the householder \textsanskrit{Hāliddikāni} went up to Venerable \textsanskrit{Mahākaccāna} … and asked him, “Sir, this was said by the Buddha: ‘Diversity of elements gives rise to diversity of contacts, and diversity of contacts gives rise to diversity of feelings.’ How does diversity of elements give rise to diversity of contacts, and diversity of contacts gives rise to diversity of feelings?” 

“Householder,\marginnote{1.6} it’s when a mendicant sees a sight and understands it to be agreeable. There is eye consciousness; and pleasant feeling arises dependent on a contact to be experienced as pleasant. Then they see a sight and understand it to be disagreeable. There is eye consciousness; and painful feeling arises dependent on a contact to be experienced as painful. Then they see a sight and understand it to be a basis for equanimity. There is eye consciousness; and neutral feeling arises dependent on a contact to be experienced as neutral. 

Furthermore,\marginnote{2.1} a mendicant hears a sound with the ear … smells an odor with the nose … tastes a flavor with the tongue … feels a touch with the body … knows a thought with the mind and understands it to be agreeable. There is mind consciousness; and pleasant feeling arises dependent on a contact to be experienced as pleasant. Then they know a thought and understand it to be disagreeable. There is mind consciousness; and painful feeling arises dependent on a contact to be experienced as painful. Then they know a thought and understand it to be a basis for equanimity. Neutral feeling arises dependent on a contact to be experienced as neutral. 

That’s\marginnote{2.11} how diversity of elements gives rise to diversity of contacts, and diversity of contacts gives rise to diversity of feelings.” 

%
\section*{{\suttatitleacronym SN 35.131}{\suttatitletranslation Nakula’s Father }{\suttatitleroot Nakulapitusutta}}
\addcontentsline{toc}{section}{\tocacronym{SN 35.131} \toctranslation{Nakula’s Father } \tocroot{Nakulapitusutta}}
\markboth{Nakula’s Father }{Nakulapitusutta}
\extramarks{SN 35.131}{SN 35.131}

At\marginnote{1.1} one time the Buddha was staying in the land of the Bhaggas on Crocodile Hill, in the deer park at \textsanskrit{Bhesakaḷā}’s Wood. 

Then\marginnote{1.2} the householder Nakula’s father went up to the Buddha … and asked him, “What is the cause, sir, what is the reason why some sentient beings aren’t fully extinguished in the present life? What is the cause, sir, what is the reason why some sentient beings are fully extinguished in the present life?” … 

(This\marginnote{1.5} should be told in full as in SN 35.118.) 

%
\section*{{\suttatitleacronym SN 35.132}{\suttatitletranslation With Lohicca }{\suttatitleroot Lohiccasutta}}
\addcontentsline{toc}{section}{\tocacronym{SN 35.132} \toctranslation{With Lohicca } \tocroot{Lohiccasutta}}
\markboth{With Lohicca }{Lohiccasutta}
\extramarks{SN 35.132}{SN 35.132}

At\marginnote{1.1} one time Venerable \textsanskrit{Mahākaccāna} was staying in the land of the Avantis in a wilderness hut near \textsanskrit{Makkarakaṭa}. 

Then\marginnote{1.2} several youths, students of the brahmin Lohicca, approached \textsanskrit{Mahākaccāna}’s wilderness hut while collecting firewood. They walked and wandered all around the hut, making a dreadful racket and all kinds of jeers: “These shavelings, fake ascetics, riffraff, black spawn from the feet of our kinsman, the Lord! They’re honored, respected, revered, venerated, and esteemed by those who pretend to inherit Vedic culture.” 

And\marginnote{1.4} then \textsanskrit{Mahākaccāna} left his dwelling and said to those brahmin students, “Students, stop being so noisy. I will speak to you on the teaching.” 

When\marginnote{1.7} this was said, the students fell silent. Then \textsanskrit{Mahākaccāna} recited these verses for them. 

\begin{verse}%
“The\marginnote{2.1} brahmins of old championed ethics, \\
and remembered the ancient traditions. \\
Their sense doors were guarded, well protected, \\
and they had mastered anger. 

Those\marginnote{3.1} brahmins who remembered the ancient traditions \\
enjoyed virtue and absorption. 

But\marginnote{4.1} these have lost their way. Claiming to recite, \\
they live out of balance, judging everyone by their clan. \\
Mastered by anger, they take up many arms, \\
attacking both the strong and the weak. 

All\marginnote{5.1} is vain for someone who doesn’t guard the sense doors, \\
like the wealth a person finds in a dream. \\
Fasting, sleeping on bare ground, \\
bathing at dawn, the three Vedas, 

rough\marginnote{6.1} hides, dreadlocks, and dirt, \\
hymns, precepts and observances, and self-mortification, \\
those fake bent staffs, \\
and rinsing with water. \\
These emblems of the brahmins \\
are only used to generate profits. 

A\marginnote{7.1} mind that’s serene, \\
clear and undisturbed, \\
kind to all creatures: \\
that’s the path to attainment of \textsanskrit{Brahmā}!” 

%
\end{verse}

Then\marginnote{8.1} those students, offended and upset, went to the brahmin Lohicca and said to him, “Please, master, you should know this. The ascetic \textsanskrit{Mahākaccāna} condemns and rejects outright the hymns of the brahmins!” 

When\marginnote{8.3} they said this, Lohicca was offended and upset. Then he thought, “But it wouldn’t be appropriate for me to abuse or insult the ascetic \textsanskrit{Mahākaccāna} solely because of what I’ve heard from these students. Why don’t I go and ask him about it?” 

Then\marginnote{9.1} the brahmin Lohicca together with those students went to Venerable \textsanskrit{Mahākaccāna} and exchanged greetings with him. 

When\marginnote{9.2} the greetings and polite conversation were over, he sat down to one side and said to him, “Master \textsanskrit{Kaccāna}, did several young students of mine come by here collecting firewood?” 

“They\marginnote{9.4} did, brahmin.” 

“But\marginnote{9.5} did you have some discussion with them?” 

“I\marginnote{9.6} did.” 

“But\marginnote{9.7} what kind of discussion did you have with them?” 

“This\marginnote{9.8} is the discussion I had with these students.” 

\begin{verse}%
(\textsanskrit{Mahākaccāna}\marginnote{10.1} repeats the verses.) 

%
\end{verse}

“Master\marginnote{12.1} \textsanskrit{Kaccāna} spoke of someone who doesn’t guard the sense doors. How do you define someone who doesn’t guard the sense doors?” 

“Brahmin,\marginnote{12.3} take someone who sees a sight with their eyes. If it’s pleasant they hold on to it, but if it’s unpleasant they dislike it. They live with mindfulness of the body unestablished and their heart restricted. And they don’t truly understand the freedom of heart and freedom by wisdom where those arisen bad, unskillful qualities cease without anything left over. 

When\marginnote{12.5} they hear a sound with their ears … 

When\marginnote{12.6} they smell an odor with their nose … 

When\marginnote{12.7} they taste a flavor with their tongue … 

When\marginnote{12.8} they feel a touch with their body … 

When\marginnote{12.9} they know a thought with their mind, if it’s pleasant they hold on to it, but if it’s unpleasant they dislike it. They live with mindfulness of the body unestablished and a limited heart. And they don’t truly understand the freedom of heart and freedom by wisdom where those arisen bad, unskillful qualities cease without anything left over. 

That’s\marginnote{12.11} how someone doesn’t guard the sense doors.” 

“It’s\marginnote{12.12} incredible, Master \textsanskrit{Kaccāna}, it’s amazing! How accurately you’ve explained someone whose sense doors are unguarded! 

You\marginnote{13.1} also spoke of someone who does guard the sense doors. How do you define someone who does guard the sense doors?” 

“Brahmin,\marginnote{13.3} take a mendicant who sees a sight with their eyes. If it’s pleasant they don’t hold on to it, and if it’s unpleasant they don’t dislike it. They live with mindfulness of the body established and a limitless heart. And they truly understand the freedom of heart and freedom by wisdom where those arisen bad, unskillful qualities cease without anything left over. 

When\marginnote{13.5} they hear a sound with their ears … 

When\marginnote{13.6} they smell an odor with their nose … 

When\marginnote{13.7} they taste a flavor with their tongue … 

When\marginnote{13.8} they feel a touch with their body … 

When\marginnote{13.9} they know a thought with their mind, if it’s pleasant they don’t hold on to it, and if it’s unpleasant they don’t dislike it. They live with mindfulness of the body established and a limitless heart. And they truly understand the freedom of heart and freedom by wisdom where those arisen bad, unskillful qualities cease without anything left over. 

That’s\marginnote{13.11} how someone guards the sense doors.” 

“It’s\marginnote{14.1} incredible, Master \textsanskrit{Kaccāna}, it’s amazing! How accurately you’ve explained someone whose sense doors are guarded! Excellent, Master \textsanskrit{Kaccāna}! Excellent! As if he were righting the overturned, or revealing the hidden, or pointing out the path to the lost, or lighting a lamp in the dark so people with good eyes can see what’s there, Master \textsanskrit{Kaccāna} has made the teaching clear in many ways. I go for refuge to the Buddha, to the teaching, and to the mendicant \textsanskrit{Saṅgha}. From this day forth, may Master \textsanskrit{Kaccāna} remember me as a lay follower who has gone for refuge for life. 

Please\marginnote{14.7} come to my family just as you go to the families of the lay followers in \textsanskrit{Makkarakaṭa}. The brahmin boys and girls there will bow to you, rise in your presence, and give you a seat and water. That will be for their lasting welfare and happiness.” 

%
\section*{{\suttatitleacronym SN 35.133}{\suttatitletranslation Verahaccāni }{\suttatitleroot Verahaccānisutta}}
\addcontentsline{toc}{section}{\tocacronym{SN 35.133} \toctranslation{Verahaccāni } \tocroot{Verahaccānisutta}}
\markboth{Verahaccāni }{Verahaccānisutta}
\extramarks{SN 35.133}{SN 35.133}

At\marginnote{1.1} one time Venerable \textsanskrit{Udāyī} was staying near \textsanskrit{Kāmaṇḍā} in the brahmin Todeyya’s mango grove. 

Then\marginnote{1.2} a boy who was a student of the brahmin lady of the \textsanskrit{Verahaccāni} clan went up to \textsanskrit{Udāyī} and exchanged greetings with him. When the greetings and polite conversation were over, he sat down to one side. \textsanskrit{Udāyī} educated, encouraged, fired up, and inspired that student with a Dhamma talk. 

Then\marginnote{1.5} that student went to the brahmin lady of the \textsanskrit{Verahaccāni} clan and said to her, “Please, madam, you should know this. The ascetic \textsanskrit{Udāyī} teaches Dhamma that’s good in the beginning, good in the middle, and good in the end, meaningful and well-phrased. And he reveals a spiritual practice that’s entirely full and pure.” 

“Then,\marginnote{2.1} student, invite him in my name for tomorrow’s meal.” 

“Yes,\marginnote{2.2} madam,” he replied. He went to \textsanskrit{Udāyī} and said, “Sir, may Master \textsanskrit{Udāyī} please accept an offering of tomorrow’s meal from my teacher’s wife, the brahmin lady of the \textsanskrit{Verahaccāni} clan.” \textsanskrit{Udāyī} consented in silence. 

Then\marginnote{2.5} when the night had passed, \textsanskrit{Udāyī} robed up in the morning and, taking his bowl and robe, went to the brahmin lady’s home, and sat down on the seat spread out. Then the brahmin lady served and satisfied \textsanskrit{Udāyī} with her own hands with a variety of delicious foods. 

When\marginnote{2.7} \textsanskrit{Udāyī} had eaten and washed his hand and bowl, she put on a pair of shoes, sat on a high seat, covered her head, and said to him, “Ascetic, preach the Dhamma.” 

“There\marginnote{2.9} will be an occasion for that, sister,” he replied, then got up from his seat and left. 

For\marginnote{3.1} a second time that student went to Venerable \textsanskrit{Udāyī} … 

And\marginnote{3.4} for a second time that student went to the brahmin lady of the \textsanskrit{Verahaccāni} clan … 

She\marginnote{4.1} said to him, “You keep praising the ascetic \textsanskrit{Udāyī} like this. But when I asked him to preach the Dhamma he just said that there would be an occasion for that, and then he got up and left.” 

“Madam,\marginnote{4.3} that’s because you put on a pair of shoes, sat on a high seat, and covered your head before inviting him to teach. For the masters respect the teaching.” 

“Then,\marginnote{4.6} student, invite him in my name for tomorrow’s meal.” 

“Yes,\marginnote{4.7} madam,” he replied. … 

Then\marginnote{5.1} the brahmin lady served and satisfied \textsanskrit{Udāyī} with her own hands with a variety of delicious foods. 

When\marginnote{5.3} \textsanskrit{Udāyī} had eaten and washed his hand and bowl, she took off her shoes, sat on a low seat, uncovered her head, and said to him, “Sir, when what exists do the perfected ones declare that there is pleasure and pain? When what doesn’t exist do the perfected ones not declare that there is pleasure and pain?” 

“Sister,\marginnote{6.1} when there’s an eye, the perfected ones declare that there is pleasure and pain. When there’s no eye, the perfected ones don’t declare that there is pleasure and pain. When there’s an ear … nose … tongue … body … mind, the perfected ones declare that there is pleasure and pain. When there’s no mind, the perfected ones don’t declare that there is pleasure and pain.” 

When\marginnote{7.1} he said this, the brahmin lady said to \textsanskrit{Udāyī}, “Excellent, sir! Excellent! As if he were righting the overturned, or revealing the hidden, or pointing out the path to the lost, or lighting a lamp in the dark so people with good eyes can see what’s there, Venerable \textsanskrit{Udāyī} has made the teaching clear in many ways. I go for refuge to the Buddha, to the teaching, and to the mendicant \textsanskrit{Saṅgha}. From this day forth, may Venerable \textsanskrit{Udāyī} remember me as a lay follower who has gone for refuge for life.” 

%
\addtocontents{toc}{\let\protect\contentsline\protect\nopagecontentsline}
\chapter*{The Chapter at Devadaha }
\addcontentsline{toc}{chapter}{\tocchapterline{The Chapter at Devadaha }}
\addtocontents{toc}{\let\protect\contentsline\protect\oldcontentsline}

%
\section*{{\suttatitleacronym SN 35.134}{\suttatitletranslation At Devadaha }{\suttatitleroot Devadahasutta}}
\addcontentsline{toc}{section}{\tocacronym{SN 35.134} \toctranslation{At Devadaha } \tocroot{Devadahasutta}}
\markboth{At Devadaha }{Devadahasutta}
\extramarks{SN 35.134}{SN 35.134}

At\marginnote{1.1} one time the Buddha was staying in the land of the Sakyans, near the Sakyan town named Devadaha. There the Buddha addressed the mendicants: 

“When\marginnote{1.3} it comes to the six fields of contact, mendicants, I don’t say that all mendicants have work to do with diligence, nor do I say that none of them have work to do with diligence. 

I\marginnote{1.4} say that, when it comes to the six fields of contact, mendicants don’t have work to do with diligence if they are perfected, with defilements ended, having completed the spiritual journey, done what had to be done, laid down the burden, achieved their own goal, utterly ended the fetters of rebirth, and become rightly freed through enlightenment. Why is that? 

They’ve\marginnote{1.6} done their work with diligence, and are incapable of negligence. 

I\marginnote{1.7} say that, when it comes to the six fields of contact, mendicants do have work to do with diligence if they are trainees, who haven’t achieved their heart’s desire, but live aspiring to the supreme sanctuary. Why is that? 

There\marginnote{1.9} are sights known by the eye that are pleasant and also those that are unpleasant. Though experiencing them again and again they don’t occupy the mind. Their energy is roused up and unflagging, their mindfulness is established and lucid, their body is tranquil and undisturbed, and their mind is immersed in \textsanskrit{samādhi}. Seeing this fruit of diligence, I say that those mendicants have work to do with diligence when it comes to the six fields of contact. … 

There\marginnote{1.13} are thoughts known by the mind that are pleasant and also those that are unpleasant. Though experiencing them again and again they don’t occupy the mind. Their energy is roused up and unflagging, their mindfulness is established and lucid, their body is tranquil and undisturbed, and their mind is immersed in \textsanskrit{samādhi}. Seeing this fruit of diligence, I say that those mendicants have work to do with diligence when it comes to the six fields of contact.” 

%
\section*{{\suttatitleacronym SN 35.135}{\suttatitletranslation Opportunity }{\suttatitleroot Khaṇasutta}}
\addcontentsline{toc}{section}{\tocacronym{SN 35.135} \toctranslation{Opportunity } \tocroot{Khaṇasutta}}
\markboth{Opportunity }{Khaṇasutta}
\extramarks{SN 35.135}{SN 35.135}

“You’re\marginnote{1.1} fortunate, mendicants, so very fortunate, to have the opportunity to lead the spiritual life. I’ve seen the hell called ‘the six fields of contact’. There, whatever sight you see with your eye is unlikable, not likable; undesirable, not desirable; unpleasant, not pleasant. 

Whatever\marginnote{1.5} sound you hear … 

Whatever\marginnote{1.6} odor you smell … 

Whatever\marginnote{1.7} flavor you taste … 

Whatever\marginnote{1.8} touch you feel … 

Whatever\marginnote{1.9} thought you know with your mind is unlikable, not likable; undesirable, not desirable; unpleasant, not pleasant. You’re fortunate, mendicants, so very fortunate, to have the opportunity to lead the spiritual life. I’ve seen the heaven called ‘the six fields of contact’. There, whatever sight you see with your eye is likable, not unlikable; desirable, not undesirable; pleasant, not unpleasant. 

Whatever\marginnote{1.14} sound … odor … flavor … touch … 

Whatever\marginnote{1.15} thought you know with your mind is likable, not unlikable; desirable, not undesirable; pleasant, not unpleasant. You’re fortunate, mendicants, so very fortunate, to have the opportunity to lead the spiritual life.” 

%
\section*{{\suttatitleacronym SN 35.136}{\suttatitletranslation Liking Sights (1st) }{\suttatitleroot Paṭhamarūpārāmasutta}}
\addcontentsline{toc}{section}{\tocacronym{SN 35.136} \toctranslation{Liking Sights (1st) } \tocroot{Paṭhamarūpārāmasutta}}
\markboth{Liking Sights (1st) }{Paṭhamarūpārāmasutta}
\extramarks{SN 35.136}{SN 35.136}

“Mendicants,\marginnote{1.1} gods and humans like sights, they love them and enjoy them. But when sights perish, fade away, and cease, gods and humans live in suffering. 

Gods\marginnote{1.3} and humans like sounds … smells … tastes … touches … thoughts, they love them and enjoy them. But when thoughts perish, fade away, and cease, gods and humans live in suffering. 

The\marginnote{1.10} Realized One has truly understood the origin, ending, gratification, drawback, and escape of sights, so he doesn’t like, love, or enjoy them. When sights perish, fade away, and cease, the Realized One lives happily. 

The\marginnote{1.12} Realized One has truly understood the origin, ending, gratification, drawback, and escape of sounds … smells … tastes … touches … thoughts, so he doesn’t like, love, or enjoy them. When thoughts perish, fade away, and cease, the Realized One lives happily.” 

That\marginnote{1.18} is what the Buddha said. Then the Holy One, the Teacher, went on to say: 

\begin{verse}%
“Sights,\marginnote{2.1} sounds, tastes, smells, \\
touches and thoughts, the lot of them—\\
they’re likable, desirable, and pleasurable \\
as long as you can say that they exist. 

For\marginnote{3.1} all the world with its gods, \\
this is what they agree is happiness. \\
And where they cease \\
is agreed on as suffering for them. 

The\marginnote{4.1} noble ones have seen that happiness \\
is the cessation of identity. \\
This insight by those who see \\
contradicts the whole world. 

What\marginnote{5.1} others say is happiness \\
the noble ones say is suffering. \\
What others say is suffering \\
the noble ones know as happiness. 

See,\marginnote{6.1} this teaching is hard to understand, \\
it confuses the ignorant. \\
There is darkness for the shrouded; \\
blackness for those who don’t see. 

But\marginnote{7.1} the good are open; \\
like light for those who see. \\
Though close, they do not understand, \\
those fools inexpert in the teaching. 

They’re\marginnote{8.1} mired in desire to be reborn, \\
flowing along the stream of lives, \\
mired in \textsanskrit{Māra}’s sway: \\
this teaching isn’t easy for them to understand. 

Who,\marginnote{9.1} apart from the noble ones, \\
is qualified to understand this state? \\
Having rightly understood this state, \\
the undefiled become fully extinguished.” 

%
\end{verse}

%
\section*{{\suttatitleacronym SN 35.137}{\suttatitletranslation Liking Sights (2nd) }{\suttatitleroot Dutiyarūpārāmasutta}}
\addcontentsline{toc}{section}{\tocacronym{SN 35.137} \toctranslation{Liking Sights (2nd) } \tocroot{Dutiyarūpārāmasutta}}
\markboth{Liking Sights (2nd) }{Dutiyarūpārāmasutta}
\extramarks{SN 35.137}{SN 35.137}

“Mendicants,\marginnote{1.1} gods and humans like sights, they love them and enjoy them. But when sights perish, fade away, and cease, gods and humans live in suffering. … 

The\marginnote{1.9} Realized One has truly understood the origin, ending, gratification, drawback, and escape of sights, so he doesn’t like, love, or enjoy them. When sights perish, fade away, and cease, the Realized One lives happily. …” 

%
\section*{{\suttatitleacronym SN 35.138}{\suttatitletranslation Not Yours (1st) }{\suttatitleroot Paṭhamanatumhākasutta}}
\addcontentsline{toc}{section}{\tocacronym{SN 35.138} \toctranslation{Not Yours (1st) } \tocroot{Paṭhamanatumhākasutta}}
\markboth{Not Yours (1st) }{Paṭhamanatumhākasutta}
\extramarks{SN 35.138}{SN 35.138}

“Mendicants,\marginnote{1.1} give up what’s not yours. Giving it up will be for your welfare and happiness. And what isn’t yours? The eye isn’t yours: give it up. Giving it up will be for your welfare and happiness. 

The\marginnote{1.6} ear … nose … tongue … body … mind isn’t yours: give it up. Giving it up will be for your welfare and happiness. 

Suppose\marginnote{1.10} a person was to carry off the grass, sticks, branches, and leaves in this Jeta’s Grove, or burn them, or do what they want with them. Would you think: ‘This person is carrying us off, burning us, or doing what they want with us’?” 

“No,\marginnote{1.12} sir. Why is that? Because that’s neither self nor belonging to self.” 

“In\marginnote{1.15} the same way, the eye isn’t yours: give it up. Giving it up will be for your welfare and happiness. 

The\marginnote{1.17} ear … nose … tongue … body … mind isn’t yours: give it up. Giving it up will be for your welfare and happiness.” 

%
\section*{{\suttatitleacronym SN 35.139}{\suttatitletranslation Not Yours (2nd) }{\suttatitleroot Dutiyanatumhākasutta}}
\addcontentsline{toc}{section}{\tocacronym{SN 35.139} \toctranslation{Not Yours (2nd) } \tocroot{Dutiyanatumhākasutta}}
\markboth{Not Yours (2nd) }{Dutiyanatumhākasutta}
\extramarks{SN 35.139}{SN 35.139}

“Mendicants,\marginnote{1.1} give up what’s not yours. Giving it up will be for your welfare and happiness. And what isn’t yours? Sights aren’t yours: give them up. Giving them up will be for your welfare and happiness. 

Sounds\marginnote{1.6} … smells … tastes … touches … thoughts aren’t yours: give them up. Giving it up will be for your welfare and happiness. 

Suppose\marginnote{1.12} a person was to carry off the grass, sticks, branches, and leaves in this Jeta’s Grove … 

In\marginnote{1.13} the same way, sights aren’t yours: give them up. Giving them up will be for your welfare and happiness. …” 

%
\section*{{\suttatitleacronym SN 35.140}{\suttatitletranslation Interior and Cause Are Impermanent }{\suttatitleroot Ajjhattaaniccahetusutta}}
\addcontentsline{toc}{section}{\tocacronym{SN 35.140} \toctranslation{Interior and Cause Are Impermanent } \tocroot{Ajjhattaaniccahetusutta}}
\markboth{Interior and Cause Are Impermanent }{Ajjhattaaniccahetusutta}
\extramarks{SN 35.140}{SN 35.140}

“Mendicants,\marginnote{1.1} the eye is impermanent. The cause and reason that gives rise to the eye is also impermanent. Since the eye is produced by what is impermanent, how could it be permanent? 

The\marginnote{1.4} ear … nose … tongue … body … mind is impermanent. The cause and reason that gives rise to the mind is also impermanent. Since the mind is produced by what is impermanent, how could it be permanent? 

Seeing\marginnote{1.10} this, a learned noble disciple grows disillusioned with the eye, ear, nose, tongue, body, and mind. Being disillusioned, desire fades away. When desire fades away they’re freed. When they’re freed, they know they’re freed. 

They\marginnote{1.12} understand: ‘Rebirth is ended, the spiritual journey has been completed, what had to be done has been done, there is no return to any state of existence.’” 

%
\section*{{\suttatitleacronym SN 35.141}{\suttatitletranslation Interior and Cause Are Suffering }{\suttatitleroot Ajjhattadukkhahetusutta}}
\addcontentsline{toc}{section}{\tocacronym{SN 35.141} \toctranslation{Interior and Cause Are Suffering } \tocroot{Ajjhattadukkhahetusutta}}
\markboth{Interior and Cause Are Suffering }{Ajjhattadukkhahetusutta}
\extramarks{SN 35.141}{SN 35.141}

“Mendicants,\marginnote{1.1} the eye is suffering. The cause and reason that gives rise to the eye is also suffering. Since the eye is produced by what is suffering, how could it be happiness? 

The\marginnote{1.4} ear … nose … tongue … body … mind is suffering. The cause and reason that gives rise to the mind is also suffering. Since the mind is produced by what is suffering, how could it be happiness? 

Seeing\marginnote{1.10} this … They understand: ‘… there is no return to any state of existence.’” 

%
\section*{{\suttatitleacronym SN 35.142}{\suttatitletranslation Interior and Cause Are Not-Self }{\suttatitleroot Ajjhattānattahetusutta}}
\addcontentsline{toc}{section}{\tocacronym{SN 35.142} \toctranslation{Interior and Cause Are Not-Self } \tocroot{Ajjhattānattahetusutta}}
\markboth{Interior and Cause Are Not-Self }{Ajjhattānattahetusutta}
\extramarks{SN 35.142}{SN 35.142}

“Mendicants,\marginnote{1.1} the eye is not-self. The cause and reason that gives rise to the eye is also not-self. Since the eye is produced by what is not-self, how could it be self? 

The\marginnote{1.4} ear … nose … tongue … body … mind is not-self. The cause and reason that gives rise to the mind is also not-self. Since the mind is produced by what is not-self, how could it be self? 

Seeing\marginnote{1.10} this … They understand: ‘… there is no return to any state of existence.’” 

%
\section*{{\suttatitleacronym SN 35.143}{\suttatitletranslation Exterior and Cause Are Impermanent }{\suttatitleroot Bāhirāniccahetusutta}}
\addcontentsline{toc}{section}{\tocacronym{SN 35.143} \toctranslation{Exterior and Cause Are Impermanent } \tocroot{Bāhirāniccahetusutta}}
\markboth{Exterior and Cause Are Impermanent }{Bāhirāniccahetusutta}
\extramarks{SN 35.143}{SN 35.143}

“Mendicants,\marginnote{1.1} sights are impermanent. The cause and reason that gives rise to sights is also impermanent. Since sights are produced by what is impermanent, how could they be permanent? 

Sounds\marginnote{1.4} … 

Smells\marginnote{1.5} … 

Tastes\marginnote{1.6} … 

Touches\marginnote{1.7} … 

Thoughts\marginnote{1.8} are impermanent. The cause and reason that gives rise to thoughts is also impermanent. Since thoughts are produced by what is impermanent, how could they be permanent? 

Seeing\marginnote{1.11} this … They understand: ‘… there is no return to any state of existence.’” 

%
\section*{{\suttatitleacronym SN 35.144}{\suttatitletranslation Exterior and Cause Are Suffering }{\suttatitleroot Bāhiradukkhahetusutta}}
\addcontentsline{toc}{section}{\tocacronym{SN 35.144} \toctranslation{Exterior and Cause Are Suffering } \tocroot{Bāhiradukkhahetusutta}}
\markboth{Exterior and Cause Are Suffering }{Bāhiradukkhahetusutta}
\extramarks{SN 35.144}{SN 35.144}

“Mendicants,\marginnote{1.1} sights are suffering. The cause and reason that gives rise to sights is also suffering. Since sights are produced by what is suffering, how could they be happiness? 

Sounds\marginnote{1.4} … 

Smells\marginnote{1.5} … 

Tastes\marginnote{1.6} … 

Touches\marginnote{1.7} … 

Thoughts\marginnote{1.8} are suffering. The cause and reason that gives rise to thoughts is also suffering. Since thoughts are produced by what is suffering, how could they be happiness? 

Seeing\marginnote{1.11} this … They understand: ‘… there is no return to any state of existence.’” 

%
\section*{{\suttatitleacronym SN 35.145}{\suttatitletranslation Exterior and Cause Are Not-Self }{\suttatitleroot Bāhirānattahetusutta}}
\addcontentsline{toc}{section}{\tocacronym{SN 35.145} \toctranslation{Exterior and Cause Are Not-Self } \tocroot{Bāhirānattahetusutta}}
\markboth{Exterior and Cause Are Not-Self }{Bāhirānattahetusutta}
\extramarks{SN 35.145}{SN 35.145}

“Mendicants,\marginnote{1.1} sights are not-self. The cause and reason that gives rise to sights is also not-self. Since sights are produced by what is not-self, how could they be self? 

Sounds\marginnote{1.4} … 

Smells\marginnote{1.5} … 

Tastes\marginnote{1.6} … 

Touches\marginnote{1.7} … 

Thoughts\marginnote{1.8} are not-self. The cause and reason that gives rise to thoughts is also not-self. Since thoughts are produced by what is not-self, how could they be self? 

Seeing\marginnote{1.11} this … Being disillusioned, desire fades away. When desire fades away they’re freed. When they’re freed, they know they’re freed. 

They\marginnote{1.13} understand: ‘Rebirth is ended, the spiritual journey has been completed, what had to be done has been done, there is no return to any state of existence.’” 

%
\addtocontents{toc}{\let\protect\contentsline\protect\nopagecontentsline}
\chapter*{The Chapter on the Old and the New }
\addcontentsline{toc}{chapter}{\tocchapterline{The Chapter on the Old and the New }}
\addtocontents{toc}{\let\protect\contentsline\protect\oldcontentsline}

%
\section*{{\suttatitleacronym SN 35.146}{\suttatitletranslation The Cessation of Action }{\suttatitleroot Kammanirodhasutta}}
\addcontentsline{toc}{section}{\tocacronym{SN 35.146} \toctranslation{The Cessation of Action } \tocroot{Kammanirodhasutta}}
\markboth{The Cessation of Action }{Kammanirodhasutta}
\extramarks{SN 35.146}{SN 35.146}

“Mendicants,\marginnote{1.1} I will teach you old action, new action, the cessation of action, and the practice that leads to the cessation of action. Listen and pay close attention, I will speak. … 

And\marginnote{1.3} what is old action? 

The\marginnote{1.4} eye is old action. It should be seen as produced by choices and intentions, as something to be felt. 

The\marginnote{1.5} ear … nose … tongue … body … mind is old action. It should be seen as produced by choices and intentions, as something to be felt. 

This\marginnote{1.7} is called old action. 

And\marginnote{1.8} what is new action? 

The\marginnote{1.9} deeds you currently perform by way of body, speech, and mind. 

This\marginnote{1.10} is called new action. 

And\marginnote{1.11} what is the cessation of action? 

When\marginnote{1.12} you experience freedom due to the cessation of deeds by body, speech, and mind. 

This\marginnote{1.13} is called the cessation of action. 

And\marginnote{1.14} what’s the practice that leads to the cessation of action? 

It\marginnote{1.15} is simply this noble eightfold path, that is: right view, right thought, right speech, right action, right livelihood, right effort, right mindfulness, and right immersion. 

This\marginnote{1.17} is called the practice that leads to the cessation of action. 

So,\marginnote{1.18} mendicants, I’ve taught you old action, new action, the cessation of action, and the practice that leads to the cessation of action. 

Out\marginnote{1.19} of compassion, I’ve done what a teacher should do who wants what’s best for their disciples. Here are these roots of trees, and here are these empty huts. Practice absorption, mendicants! Don’t be negligent! Don’t regret it later! This is my instruction to you.” 

%
\section*{{\suttatitleacronym SN 35.147}{\suttatitletranslation The Impermanent as Conducive to Extinguishment }{\suttatitleroot Aniccanibbānasappāyasutta}}
\addcontentsline{toc}{section}{\tocacronym{SN 35.147} \toctranslation{The Impermanent as Conducive to Extinguishment } \tocroot{Aniccanibbānasappāyasutta}}
\markboth{The Impermanent as Conducive to Extinguishment }{Aniccanibbānasappāyasutta}
\extramarks{SN 35.147}{SN 35.147}

“Mendicants,\marginnote{1.1} I will teach you a practice that’s conducive to extinguishment. Listen … 

And\marginnote{1.3} what is that practice that’s conducive to extinguishment? 

It’s\marginnote{1.4} when a mendicant sees that the eye, sights, eye consciousness, and eye contact are impermanent. And they see that the painful, pleasant, or neutral feeling that arises conditioned by eye contact is also impermanent. 

They\marginnote{1.5} see that the ear … nose … tongue … body … mind, thoughts, mind-consciousness, and mind contact are impermanent. And they see that the painful, pleasant, or neutral feeling that arises conditioned by mind contact is also impermanent. 

This\marginnote{1.7} is that practice that’s conducive to extinguishment.” 

%
\section*{{\suttatitleacronym SN 35.148}{\suttatitletranslation The Suffering as Conducive to Extinguishment }{\suttatitleroot Dukkhanibbānasappāyasutta}}
\addcontentsline{toc}{section}{\tocacronym{SN 35.148} \toctranslation{The Suffering as Conducive to Extinguishment } \tocroot{Dukkhanibbānasappāyasutta}}
\markboth{The Suffering as Conducive to Extinguishment }{Dukkhanibbānasappāyasutta}
\extramarks{SN 35.148}{SN 35.148}

“Mendicants,\marginnote{1.1} I will teach you a practice that’s conducive to extinguishment. Listen … 

And\marginnote{1.3} what is that practice that’s conducive to extinguishment? 

It’s\marginnote{1.4} when a mendicant sees that the eye, sights, eye consciousness, and eye contact are suffering. And they see that the painful, pleasant, or neutral feeling that arises conditioned by eye contact is also suffering. 

They\marginnote{1.5} see that the ear … nose … tongue … body … mind, thoughts, mind-consciousness, and mind contact are suffering. And they see that the painful, pleasant, or neutral feeling that arises conditioned by mind contact is also suffering. 

This\marginnote{1.7} is that practice that’s conducive to extinguishment.” 

%
\section*{{\suttatitleacronym SN 35.149}{\suttatitletranslation Not-Self as Conducive to Extinguishment }{\suttatitleroot Anattanibbānasappāyasutta}}
\addcontentsline{toc}{section}{\tocacronym{SN 35.149} \toctranslation{Not-Self as Conducive to Extinguishment } \tocroot{Anattanibbānasappāyasutta}}
\markboth{Not-Self as Conducive to Extinguishment }{Anattanibbānasappāyasutta}
\extramarks{SN 35.149}{SN 35.149}

“Mendicants,\marginnote{1.1} I will teach you a practice that’s conducive to extinguishment. Listen … 

And\marginnote{1.3} what is that practice that’s conducive to extinguishment? 

It’s\marginnote{1.4} when a mendicant sees that the eye, sights, eye consciousness, and eye contact are not-self. And they see that the painful, pleasant, or neutral feeling that arises conditioned by eye contact is also not-self. 

They\marginnote{1.5} see that the ear … nose … tongue … body … mind, thoughts, mind-consciousness, and mind contact are not-self. And they see that the painful, pleasant, or neutral feeling that arises conditioned by mind contact is also not-self. 

This\marginnote{1.6} is that practice that’s conducive to extinguishment.” 

%
\section*{{\suttatitleacronym SN 35.150}{\suttatitletranslation A Practice Conducive to Extinguishment }{\suttatitleroot Nibbānasappāyapaṭipadāsutta}}
\addcontentsline{toc}{section}{\tocacronym{SN 35.150} \toctranslation{A Practice Conducive to Extinguishment } \tocroot{Nibbānasappāyapaṭipadāsutta}}
\markboth{A Practice Conducive to Extinguishment }{Nibbānasappāyapaṭipadāsutta}
\extramarks{SN 35.150}{SN 35.150}

“Mendicants,\marginnote{1.1} I will teach you a practice that’s conducive to extinguishment. Listen … 

And\marginnote{1.3} what is that practice that’s conducive to extinguishment? 

What\marginnote{1.4} do you think, mendicants? Is the eye permanent or impermanent?” 

“Impermanent,\marginnote{2.1} sir.” 

“But\marginnote{3.1} if it’s impermanent, is it suffering or happiness?” 

“Suffering,\marginnote{4.1} sir.” 

“But\marginnote{5.1} if it’s impermanent, suffering, and liable to wear out, is it fit to be regarded thus: ‘This is mine, I am this, this is my self’?” 

“No,\marginnote{6.1} sir.” 

“Are\marginnote{7.1} sights … eye consciousness … eye contact … 

The\marginnote{9.3} pleasant, painful, or neutral feeling that arises conditioned by mind contact: is that permanent or impermanent?” 

“Impermanent,\marginnote{10.1} sir.” 

“But\marginnote{11.1} if it’s impermanent, is it suffering or happiness?” 

“Suffering,\marginnote{12.1} sir.” 

“But\marginnote{13.1} if it’s impermanent, suffering, and liable to wear out, is it fit to be regarded thus: ‘This is mine, I am this, this is my self’?” 

“No,\marginnote{14.1} sir.” 

Seeing\marginnote{15.1} this, a learned noble disciple grows disillusioned with the eye, sights, eye consciousness, and eye contact. And they grow disillusioned with the painful, pleasant, or neutral feeling that arises conditioned by eye contact. 

They\marginnote{15.2} grow disillusioned with the ear … nose … tongue … body … mind … painful, pleasant, or neutral feeling that arises conditioned by mind contact. Being disillusioned, desire fades away. When desire fades away they’re freed. … 

They\marginnote{15.4} understand: ‘… there is no return to any state of existence.’ 

This\marginnote{15.5} is that practice that’s conducive to extinguishment.” 

%
\section*{{\suttatitleacronym SN 35.151}{\suttatitletranslation A Student }{\suttatitleroot Antevāsikasutta}}
\addcontentsline{toc}{section}{\tocacronym{SN 35.151} \toctranslation{A Student } \tocroot{Antevāsikasutta}}
\markboth{A Student }{Antevāsikasutta}
\extramarks{SN 35.151}{SN 35.151}

“Mendicants,\marginnote{1.1} this spiritual life is lived without a resident student and without a teaching master. A mendicant who lives with a resident student and a teaching master lives in suffering and discomfort. A mendicant who lives without a resident student and a teaching master lives in happiness and comfort. 

And\marginnote{1.4} how does a mendicant who lives with a resident student and a teaching master live in suffering and discomfort? 

When\marginnote{1.5} a mendicant sees a sight with the eye, bad, unskillful phenomena arise: memories and thoughts prone to fetters. Those qualities reside within. Since they have bad unskillful qualities residing within, they’re said to have a resident student. Those qualities master them. Since they’re mastered by bad unskillful qualities, they’re said to have a teaching master. 

Furthermore,\marginnote{2.1} when a mendicant hears … smells … tastes … touches … knows a thought with the mind, bad, unskillful phenomena arise: memories and thoughts prone to fetters. Those qualities reside within. Since they have bad unskillful qualities residing within, they’re said to have a resident student. Those qualities master them. Since they’re mastered by bad unskillful qualities, they’re said to have a teaching master. That’s how a mendicant who lives with a resident student and a teaching master lives in suffering and discomfort. 

And\marginnote{4.1} how does a mendicant who lives without a resident student and a teaching master live in happiness and comfort? 

When\marginnote{4.2} a mendicant sees a sight with the eye, bad, unskillful phenomena don’t arise: memories and thoughts prone to fetters. Those qualities don’t reside within. Since they don’t have bad unskillful qualities residing within, they’re said to not have a resident student. Those qualities don’t master them. Since they’re not mastered by bad unskillful qualities, they’re said to not have a teaching master. 

Furthermore,\marginnote{5.1} when a mendicant hears … smells … tastes … touches … knows a thought with the mind, bad, unskillful phenomena don’t arise: memories and thoughts prone to fetters. Those qualities don’t reside within. Since they don’t have bad unskillful qualities residing within, they’re said to not have a resident student. Those qualities don’t master them. Since they’re not mastered by bad unskillful qualities, they’re said to not have a teaching master. 

That’s\marginnote{6.6} how a mendicant who lives without a resident student and a teaching master lives in happiness and comfort. 

This\marginnote{6.7} spiritual life is lived without a resident student and without a teaching master. A mendicant who lives with a resident student and a teaching master lives in suffering and discomfort. A mendicant who lives without a resident student and a teaching master lives in happiness and comfort.” 

%
\section*{{\suttatitleacronym SN 35.152}{\suttatitletranslation What’s the Purpose of the Spiritual Life? }{\suttatitleroot Kimatthiyabrahmacariyasutta}}
\addcontentsline{toc}{section}{\tocacronym{SN 35.152} \toctranslation{What’s the Purpose of the Spiritual Life? } \tocroot{Kimatthiyabrahmacariyasutta}}
\markboth{What’s the Purpose of the Spiritual Life? }{Kimatthiyabrahmacariyasutta}
\extramarks{SN 35.152}{SN 35.152}

“Mendicants,\marginnote{1.1} if wanderers who follow another path were to ask you: ‘Reverends, what’s the purpose of leading the spiritual life under the ascetic Gotama?’ 

You\marginnote{1.3} should answer them: ‘The purpose of leading the spiritual life under the Buddha is to completely understand suffering.’ 

If\marginnote{1.5} wanderers who follow other paths were to ask you: ‘Reverends, what is that suffering?’ 

You\marginnote{1.7} should answer them: ‘The eye is suffering. The purpose of leading the spiritual life under the Buddha is to completely understand this. Sights … Eye consciousness … Eye contact … The pleasant, painful, or neutral feeling that arises conditioned by eye contact is also suffering. The purpose of leading the spiritual life under the Buddha is to completely understand this. 

The\marginnote{2.11} ear … nose … tongue … body … mind … The pleasant, painful, or neutral feeling that arises conditioned by mind contact is also suffering. The purpose of living the spiritual life under the Buddha is to completely understand this. This is that suffering. The purpose of leading the spiritual life under the Buddha is to completely understand this.’ 

When\marginnote{2.18} questioned by wanderers who follow other paths, that’s how you should answer them.” 

%
\section*{{\suttatitleacronym SN 35.153}{\suttatitletranslation Is There a Method? }{\suttatitleroot Atthinukhopariyāyasutta}}
\addcontentsline{toc}{section}{\tocacronym{SN 35.153} \toctranslation{Is There a Method? } \tocroot{Atthinukhopariyāyasutta}}
\markboth{Is There a Method? }{Atthinukhopariyāyasutta}
\extramarks{SN 35.153}{SN 35.153}

“Mendicants,\marginnote{1.1} is there a method—apart from faith, preference, oral tradition, reasoned contemplation, or acceptance of a view after consideration—that a mendicant can rely on to declare their enlightenment? That is: ‘I understand: “Rebirth is ended, the spiritual journey has been completed, what had to be done has been done, there is no return to any state of existence.”’” 

“Our\marginnote{1.3} teachings are rooted in the Buddha. He is our guide and our refuge. Sir, may the Buddha himself please clarify the meaning of this. The mendicants will listen and remember it.” 

“Well\marginnote{1.4} then, mendicants, listen and pay close attention, I will speak.” 

“Yes,\marginnote{1.5} sir,” they replied. The Buddha said this: 

“There\marginnote{1.7} is a method—apart from faith, preference, oral tradition, reasoned contemplation, or acceptance of a view after consideration—that a mendicant can rely on to declare their enlightenment. That is: ‘I understand: “Rebirth is ended, the spiritual journey has been completed, what had to be done has been done, there is no return to any state of existence.”’ 

And\marginnote{2.1} what is that method? Take a mendicant who sees a sight with the eye. When they have greed, hate, and delusion in them, they understand ‘I have greed, hate, and delusion in me.’ When they don’t have greed, hate, and delusion in them, they understand ‘I don’t have greed, hate, and delusion in me.’ Since this is so, are these things understood by faith, preference, oral tradition, reasoned contemplation, or acceptance of a view after consideration?” 

“No,\marginnote{2.5} sir.” 

“Aren’t\marginnote{2.6} they understood by seeing them with wisdom?” 

“Yes,\marginnote{2.7} sir.” 

“This\marginnote{2.8} is a method—apart from faith, preference, oral tradition, reasoned contemplation, or acceptance of a view after consideration—that a mendicant can rely on to declare their enlightenment. That is: ‘I understand: “Rebirth is ended, the spiritual journey has been completed, what had to be done has been done, there is no return to any state of existence.”’ 

Furthermore,\marginnote{3.1} a mendicant hears a sound … smells an odor … tastes a flavor … feels a touch … knows a thought with the mind. When they have greed, hate, and delusion in them, they understand ‘I have greed, hate, and delusion in me.’ When they don’t have greed, hate, and delusion in them, they understand ‘I don’t have greed, hate, and delusion in me.’ Since this is so, are these things understood by faith, preference, oral tradition, reasoned contemplation, or acceptance of a view after consideration?” 

“No,\marginnote{4.4} sir.” 

“Aren’t\marginnote{4.5} they understood by seeing them with wisdom?” 

“Yes,\marginnote{4.6} sir.” 

“This\marginnote{4.7} too is a method—apart from faith, preference, oral tradition, reasoned contemplation, or acceptance of a view after consideration—that a mendicant can rely on to declare their enlightenment. That is: ‘I understand: “Rebirth is ended, the spiritual journey has been completed, what had to be done has been done, there is no return to any state of existence.”’” 

%
\section*{{\suttatitleacronym SN 35.154}{\suttatitletranslation Endowed With Faculties }{\suttatitleroot Indriyasampannasutta}}
\addcontentsline{toc}{section}{\tocacronym{SN 35.154} \toctranslation{Endowed With Faculties } \tocroot{Indriyasampannasutta}}
\markboth{Endowed With Faculties }{Indriyasampannasutta}
\extramarks{SN 35.154}{SN 35.154}

Then\marginnote{1.1} a mendicant went up to the Buddha … and asked him, “Sir, they speak of someone who is ‘accomplished regarding the faculties’. How is someone accomplished regarding the faculties defined?” 

“Mendicant,\marginnote{2.1} if someone meditates observing rise and fall in the eye faculty, they grow disillusioned with the eye faculty. 

If\marginnote{2.2} they meditate observing rise and fall in the ear faculty … nose faculty … tongue faculty … body faculty … mind faculty, they grow disillusioned with the mind faculty. 

Being\marginnote{2.4} disillusioned, desire fades away. … When they’re freed, they know they’re freed. 

They\marginnote{2.5} understand: ‘Rebirth is ended, the spiritual journey has been completed, what had to be done has been done, there is no return to any state of existence.’ 

This\marginnote{2.6} is how someone who is accomplished regarding the faculties is defined.” 

%
\section*{{\suttatitleacronym SN 35.155}{\suttatitletranslation A Dhamma Speaker }{\suttatitleroot Dhammakathikapucchasutta}}
\addcontentsline{toc}{section}{\tocacronym{SN 35.155} \toctranslation{A Dhamma Speaker } \tocroot{Dhammakathikapucchasutta}}
\markboth{A Dhamma Speaker }{Dhammakathikapucchasutta}
\extramarks{SN 35.155}{SN 35.155}

Then\marginnote{1.1} a mendicant went up to the Buddha … and asked him, “Sir, they speak of a ‘Dhamma speaker’. How is a Dhamma speaker defined?” 

“If\marginnote{2.1} a mendicant teaches Dhamma for disillusionment, dispassion, and cessation regarding the eye, they’re qualified to be called a ‘mendicant who speaks on Dhamma’. 

If\marginnote{2.2} they practice for disillusionment, dispassion, and cessation regarding the eye, they’re qualified to be called a ‘mendicant who practices in line with the teaching’. 

If\marginnote{2.3} they’re freed by not grasping by disillusionment, dispassion, and cessation regarding the eye, they’re qualified to be called a ‘mendicant who has attained extinguishment in this very life’. 

If\marginnote{2.4} a mendicant teaches Dhamma for disillusionment with the ear … nose … tongue … body … mind, for its fading away and cessation, they’re qualified to be called a ‘mendicant who speaks on Dhamma’. 

If\marginnote{2.6} they practice for disillusionment, dispassion, and cessation regarding the mind, they’re qualified to be called a ‘mendicant who practices in line with the teaching’. 

If\marginnote{2.7} they’re freed by not grasping by disillusionment, dispassion, and cessation regarding the mind, they’re qualified to be called a ‘mendicant who has attained extinguishment in this very life’.” 

%
\addtocontents{toc}{\let\protect\contentsline\protect\nopagecontentsline}
\pannasa{The Fourth Fifty }
\addcontentsline{toc}{pannasa}{The Fourth Fifty }
\markboth{}{}
\addtocontents{toc}{\let\protect\contentsline\protect\oldcontentsline}

%
\addtocontents{toc}{\let\protect\contentsline\protect\nopagecontentsline}
\chapter*{The Chapter on the End of Relishing }
\addcontentsline{toc}{chapter}{\tocchapterline{The Chapter on the End of Relishing }}
\addtocontents{toc}{\let\protect\contentsline\protect\oldcontentsline}

%
\section*{{\suttatitleacronym SN 35.156}{\suttatitletranslation The Interior and the End of Relishing }{\suttatitleroot Ajjhattanandikkhayasutta}}
\addcontentsline{toc}{section}{\tocacronym{SN 35.156} \toctranslation{The Interior and the End of Relishing } \tocroot{Ajjhattanandikkhayasutta}}
\markboth{The Interior and the End of Relishing }{Ajjhattanandikkhayasutta}
\extramarks{SN 35.156}{SN 35.156}

“Mendicants,\marginnote{1.1} the eye really is impermanent. A mendicant sees that it is impermanent: that’s their right view. Seeing rightly, they grow disillusioned. When relishing ends, greed ends. When greed ends, relishing ends. When relishing and greed end, the mind is said to be well freed. 

The\marginnote{1.6} ear … nose … tongue … body … mind really is impermanent. A mendicant sees that it is impermanent: that’s their right view. Seeing rightly, they grow disillusioned. When relishing ends, greed ends. When greed ends, relishing ends. When relishing and greed end, the mind is said to be well freed.” 

%
\section*{{\suttatitleacronym SN 35.157}{\suttatitletranslation The Exterior and the End of Relishing }{\suttatitleroot Bāhiranandikkhayasutta}}
\addcontentsline{toc}{section}{\tocacronym{SN 35.157} \toctranslation{The Exterior and the End of Relishing } \tocroot{Bāhiranandikkhayasutta}}
\markboth{The Exterior and the End of Relishing }{Bāhiranandikkhayasutta}
\extramarks{SN 35.157}{SN 35.157}

“Mendicants,\marginnote{1.1} sights really are impermanent. A mendicant sees that they are impermanent: that’s their right view. Seeing rightly, they grow disillusioned. When relishing ends, greed ends. When greed ends, relishing ends. When relishing and greed end, the mind is said to be well freed. 

Sounds\marginnote{1.6} … Smells … Tastes … Touches … Thoughts really are impermanent. A mendicant sees that they are impermanent: that’s their right view. Seeing rightly, they grow disillusioned. When relishing ends, greed ends. When greed ends, relishing ends. When relishing and greed end, the mind is said to be well freed.” 

%
\section*{{\suttatitleacronym SN 35.158}{\suttatitletranslation Focus, the Interior, and the End of Relishing }{\suttatitleroot Ajjhattaaniccanandikkhayasutta}}
\addcontentsline{toc}{section}{\tocacronym{SN 35.158} \toctranslation{Focus, the Interior, and the End of Relishing } \tocroot{Ajjhattaaniccanandikkhayasutta}}
\markboth{Focus, the Interior, and the End of Relishing }{Ajjhattaaniccanandikkhayasutta}
\extramarks{SN 35.158}{SN 35.158}

“Mendicants,\marginnote{1.1} properly attend to the eye. Truly see the impermanence of the eye. When a mendicant does this, they grow disillusioned with the eye. When relishing ends, greed ends. When greed ends, relishing ends. When relishing and greed end, the mind is said to be well freed. 

Properly\marginnote{1.6} attend to the ear … nose … tongue … body … mind. Truly see the impermanence of the mind. When a mendicant does this, they grow disillusioned with the mind. When relishing ends, greed ends. When greed ends, relishing ends. When relishing and greed end, the mind is said to be well freed.” 

%
\section*{{\suttatitleacronym SN 35.159}{\suttatitletranslation Focus, the Exterior, and the End of Relishing }{\suttatitleroot Bāhiraaniccanandikkhayasutta}}
\addcontentsline{toc}{section}{\tocacronym{SN 35.159} \toctranslation{Focus, the Exterior, and the End of Relishing } \tocroot{Bāhiraaniccanandikkhayasutta}}
\markboth{Focus, the Exterior, and the End of Relishing }{Bāhiraaniccanandikkhayasutta}
\extramarks{SN 35.159}{SN 35.159}

“Mendicants,\marginnote{1.1} properly attend to sights. Truly see the impermanence of sights. When a mendicant does this, they grow disillusioned with sights. When relishing ends, greed ends. When greed ends, relishing ends. When relishing and greed end, the mind is said to be well freed. 

Properly\marginnote{1.6} attend to sounds … smells … tastes … touches … thoughts. Truly see the impermanence of thoughts. When a mendicant does this, they grow disillusioned with thoughts. When relishing ends, greed ends. When greed ends, relishing ends. When relishing and greed end, the mind is said to be well freed.” 

%
\section*{{\suttatitleacronym SN 35.160}{\suttatitletranslation On Immersion at Jīvaka’s Mango Grove }{\suttatitleroot Jīvakambavanasamādhisutta}}
\addcontentsline{toc}{section}{\tocacronym{SN 35.160} \toctranslation{On Immersion at Jīvaka’s Mango Grove } \tocroot{Jīvakambavanasamādhisutta}}
\markboth{On Immersion at Jīvaka’s Mango Grove }{Jīvakambavanasamādhisutta}
\extramarks{SN 35.160}{SN 35.160}

At\marginnote{1.1} one time the Buddha was staying near \textsanskrit{Rājagaha} in \textsanskrit{Jīvaka}’s Mango Grove. There the Buddha addressed the mendicants: 

“Mendicants,\marginnote{1.4} develop immersion. For a mendicant with immersion, things become truly clear. And what becomes truly clear? 

It\marginnote{1.7} becomes truly clear that the eye, sights, eye consciousness, and eye contact are impermanent. And it also becomes truly clear that the painful, pleasant, or neutral feeling that arises conditioned by eye contact is impermanent. 

It\marginnote{1.8} becomes truly clear that the ear … nose … tongue … body … mind, thoughts, mind consciousness, and mind contact are impermanent. And it also becomes truly clear that the painful, pleasant, or neutral feeling that arises conditioned by mind contact is impermanent. 

Mendicants,\marginnote{1.11} develop immersion. For a mendicant with immersion, things become truly clear.” 

%
\section*{{\suttatitleacronym SN 35.161}{\suttatitletranslation On Retreat at Jīvaka’s Mango Grove }{\suttatitleroot Jīvakambavanapaṭisallānasutta}}
\addcontentsline{toc}{section}{\tocacronym{SN 35.161} \toctranslation{On Retreat at Jīvaka’s Mango Grove } \tocroot{Jīvakambavanapaṭisallānasutta}}
\markboth{On Retreat at Jīvaka’s Mango Grove }{Jīvakambavanapaṭisallānasutta}
\extramarks{SN 35.161}{SN 35.161}

At\marginnote{1.1} one time the Buddha was staying near \textsanskrit{Rājagaha} in \textsanskrit{Jīvaka}’s Mango Grove. There the Buddha addressed the mendicants: 

“Mendicants,\marginnote{1.3} meditate in retreat. For a mendicant who meditates in retreat, things become truly clear. And what becomes truly clear? 

It\marginnote{1.6} becomes truly clear that the eye, sights, eye consciousness, and eye contact are impermanent. And it also becomes truly clear that the painful, pleasant, or neutral feeling that arises conditioned by eye contact is impermanent. … 

It\marginnote{1.7} becomes truly clear that the mind, thoughts, mind consciousness, and mind contact are impermanent. And it also becomes truly clear that the painful, pleasant, or neutral feeling that arises conditioned by mind contact is impermanent. 

Mendicants,\marginnote{1.11} meditate in retreat. For a mendicant who meditates in retreat, things become truly clear.” 

%
\section*{{\suttatitleacronym SN 35.162}{\suttatitletranslation With Koṭṭhita on Impermanence }{\suttatitleroot Koṭṭhikaaniccasutta}}
\addcontentsline{toc}{section}{\tocacronym{SN 35.162} \toctranslation{With Koṭṭhita on Impermanence } \tocroot{Koṭṭhikaaniccasutta}}
\markboth{With Koṭṭhita on Impermanence }{Koṭṭhikaaniccasutta}
\extramarks{SN 35.162}{SN 35.162}

Then\marginnote{1.1} Venerable \textsanskrit{Mahākoṭṭhita} went up to the Buddha … and asked him, “Sir, may the Buddha please teach me Dhamma in brief. When I’ve heard it, I’ll live alone, withdrawn, diligent, keen, and resolute.” 

“\textsanskrit{Koṭṭhita},\marginnote{2.1} you should give up desire for what is impermanent. And what is impermanent? The eye, sights, eye consciousness, and eye contact are impermanent: you should give up desire for them. The pleasant, painful, or neutral feeling that arises conditioned by eye contact is also impermanent: you should give up desire for it. 

The\marginnote{2.8} ear … nose … tongue … body … The mind, thoughts, mind consciousness, and mind contact are impermanent: you should give up desire for them. The pleasant, painful, or neutral feeling that arises conditioned by mind contact is also impermanent: you should give up desire for it. 

\textsanskrit{Koṭṭhita},\marginnote{2.18} you should give up desire for what is impermanent.” 

%
\section*{{\suttatitleacronym SN 35.163}{\suttatitletranslation With Koṭṭhita on Suffering }{\suttatitleroot Koṭṭhikadukkhasutta}}
\addcontentsline{toc}{section}{\tocacronym{SN 35.163} \toctranslation{With Koṭṭhita on Suffering } \tocroot{Koṭṭhikadukkhasutta}}
\markboth{With Koṭṭhita on Suffering }{Koṭṭhikadukkhasutta}
\extramarks{SN 35.163}{SN 35.163}

Then\marginnote{1.1} Venerable \textsanskrit{Mahākoṭṭhita} … said to the Buddha: 

“Sir,\marginnote{1.2} may the Buddha please teach me Dhamma in brief. When I’ve heard it, I’ll live alone, withdrawn, diligent, keen, and resolute.” 

“\textsanskrit{Koṭṭhita},\marginnote{1.3} you should give up desire for what is suffering. And what is suffering? 

The\marginnote{1.5} eye, sights, eye consciousness, and eye contact are suffering: you should give up desire for them. The pleasant, painful, or neutral feeling that arises conditioned by eye contact is also suffering; you should give up desire for it. 

The\marginnote{1.10} ear … nose … tongue … body … The mind, thoughts, mind consciousness, and mind contact are suffering: you should give up desire for them. The pleasant, painful, or neutral feeling that arises conditioned by mind contact is also suffering: you should give up desire for it. 

\textsanskrit{Koṭṭhita},\marginnote{1.16} you should give up desire for what is suffering.” 

%
\section*{{\suttatitleacronym SN 35.164}{\suttatitletranslation With Koṭṭhita on Not-Self }{\suttatitleroot Koṭṭhikaanattasutta}}
\addcontentsline{toc}{section}{\tocacronym{SN 35.164} \toctranslation{With Koṭṭhita on Not-Self } \tocroot{Koṭṭhikaanattasutta}}
\markboth{With Koṭṭhita on Not-Self }{Koṭṭhikaanattasutta}
\extramarks{SN 35.164}{SN 35.164}

“\textsanskrit{Koṭṭhita},\marginnote{1.3} you should give up desire for what is not-self. And what is not-self? 

The\marginnote{1.5} eye, sights, eye consciousness, and eye contact are not-self: you should give up desire for them. The pleasant, painful, or neutral feeling that arises conditioned by eye contact is also not-self: You should give up desire for it. 

The\marginnote{1.10} ear … nose … tongue … body … The mind, thoughts, mind consciousness, and mind contact … The pleasant, painful, or neutral feeling that arises conditioned by mind contact is also not-self: you should give up desire for it. 

\textsanskrit{Koṭṭhita},\marginnote{1.16} you should give up desire for what is not-self.” 

%
\section*{{\suttatitleacronym SN 35.165}{\suttatitletranslation Giving Up Wrong View }{\suttatitleroot Micchādiṭṭhipahānasutta}}
\addcontentsline{toc}{section}{\tocacronym{SN 35.165} \toctranslation{Giving Up Wrong View } \tocroot{Micchādiṭṭhipahānasutta}}
\markboth{Giving Up Wrong View }{Micchādiṭṭhipahānasutta}
\extramarks{SN 35.165}{SN 35.165}

Then\marginnote{1.1} a mendicant went up to the Buddha … and said to him: 

“Sir,\marginnote{1.3} how does one know and see so that wrong view is given up?” 

“Mendicant,\marginnote{2.1} knowing and seeing the eye, sights, eye consciousness, and eye contact as impermanent, wrong view is given up. … 

And\marginnote{2.5} also knowing and seeing the pleasant, painful, or neutral feeling that arises conditioned by mind contact as impermanent, wrong view is given up. 

This\marginnote{2.6} is how to know and see so that wrong view is given up.” 

%
\section*{{\suttatitleacronym SN 35.166}{\suttatitletranslation Giving Up Identity View }{\suttatitleroot Sakkāyadiṭṭhipahānasutta}}
\addcontentsline{toc}{section}{\tocacronym{SN 35.166} \toctranslation{Giving Up Identity View } \tocroot{Sakkāyadiṭṭhipahānasutta}}
\markboth{Giving Up Identity View }{Sakkāyadiṭṭhipahānasutta}
\extramarks{SN 35.166}{SN 35.166}

Then\marginnote{1.1} a mendicant went up to the Buddha … and said to him: 

“Sir,\marginnote{1.3} how does one know and see so that identity view is given up?” 

“Mendicant,\marginnote{1.4} knowing and seeing the eye, sights, eye consciousness, and eye contact as suffering, identity view is given up. … 

And\marginnote{1.8} also knowing and seeing the pleasant, painful, or neutral feeling that arises conditioned by mind contact as suffering, identity view is given up. 

This\marginnote{1.9} is how to know and see so that identity view is given up.” 

%
\section*{{\suttatitleacronym SN 35.167}{\suttatitletranslation Giving Up View of Self }{\suttatitleroot Attānudiṭṭhipahānasutta}}
\addcontentsline{toc}{section}{\tocacronym{SN 35.167} \toctranslation{Giving Up View of Self } \tocroot{Attānudiṭṭhipahānasutta}}
\markboth{Giving Up View of Self }{Attānudiṭṭhipahānasutta}
\extramarks{SN 35.167}{SN 35.167}

Then\marginnote{1.1} a mendicant went up to the Buddha … and said to him: 

“Sir,\marginnote{1.3} how does one know and see so that view of self is given up?” 

“Mendicant,\marginnote{1.4} knowing and seeing the eye, sights, eye consciousness, and eye contact as not-self, view of self is given up. … 

And\marginnote{1.10} also knowing and seeing the pleasant, painful, or neutral feeling that arises conditioned by mind contact as not-self, view of self is given up.” 

%
\addtocontents{toc}{\let\protect\contentsline\protect\nopagecontentsline}
\chapter*{The Chapter on Sixty Abbreviated Texts }
\addcontentsline{toc}{chapter}{\tocchapterline{The Chapter on Sixty Abbreviated Texts }}
\addtocontents{toc}{\let\protect\contentsline\protect\oldcontentsline}

%
\section*{{\suttatitleacronym SN 35.168}{\suttatitletranslation Desire for the Impermanent Interior }{\suttatitleroot Ajjhattaaniccachandasutta}}
\addcontentsline{toc}{section}{\tocacronym{SN 35.168} \toctranslation{Desire for the Impermanent Interior } \tocroot{Ajjhattaaniccachandasutta}}
\markboth{Desire for the Impermanent Interior }{Ajjhattaaniccachandasutta}
\extramarks{SN 35.168}{SN 35.168}

“Mendicants,\marginnote{1.1} you should give up desire for what is impermanent. And what is impermanent? 

The\marginnote{1.3} eye, ear, nose, tongue, body, and mind are impermanent: you should give up desire for them. You should give up desire for what is impermanent.” 

%
\section*{{\suttatitleacronym SN 35.169}{\suttatitletranslation Greed for the Impermanent Interior }{\suttatitleroot Ajjhattaaniccarāgasutta}}
\addcontentsline{toc}{section}{\tocacronym{SN 35.169} \toctranslation{Greed for the Impermanent Interior } \tocroot{Ajjhattaaniccarāgasutta}}
\markboth{Greed for the Impermanent Interior }{Ajjhattaaniccarāgasutta}
\extramarks{SN 35.169}{SN 35.169}

“Mendicants,\marginnote{1.1} you should give up greed for what is impermanent. And what is impermanent? 

The\marginnote{1.3} eye, ear, nose, tongue, body, and mind are impermanent …” 

%
\section*{{\suttatitleacronym SN 35.170}{\suttatitletranslation Desire and Greed for the Impermanent Interior }{\suttatitleroot Ajjhattaaniccachandarāgasutta}}
\addcontentsline{toc}{section}{\tocacronym{SN 35.170} \toctranslation{Desire and Greed for the Impermanent Interior } \tocroot{Ajjhattaaniccachandarāgasutta}}
\markboth{Desire and Greed for the Impermanent Interior }{Ajjhattaaniccachandarāgasutta}
\extramarks{SN 35.170}{SN 35.170}

“Mendicants,\marginnote{1.1} you should give up desire and greed for what is impermanent. And what is impermanent? 

The\marginnote{1.3} eye, ear, nose, tongue, body, and mind are impermanent …” 

%
\section*{{\suttatitleacronym SN 35.171–173}{\suttatitletranslation Desire, Etc. for the Suffering Interior }{\suttatitleroot Dukkhachandādisutta}}
\addcontentsline{toc}{section}{\tocacronym{SN 35.171–173} \toctranslation{Desire, Etc. for the Suffering Interior } \tocroot{Dukkhachandādisutta}}
\markboth{Desire, Etc. for the Suffering Interior }{Dukkhachandādisutta}
\extramarks{SN 35.171–173}{SN 35.171–173}

“Mendicants,\marginnote{1.1} you should give up desire … greed … desire and greed for what is suffering. And what is suffering? 

The\marginnote{1.3} eye, ear, nose, tongue, body, and mind are suffering …” 

%
\section*{{\suttatitleacronym SN 35.174–176}{\suttatitletranslation Desire, Etc. for the Not-Self Interior }{\suttatitleroot Anattachandādisutta}}
\addcontentsline{toc}{section}{\tocacronym{SN 35.174–176} \toctranslation{Desire, Etc. for the Not-Self Interior } \tocroot{Anattachandādisutta}}
\markboth{Desire, Etc. for the Not-Self Interior }{Anattachandādisutta}
\extramarks{SN 35.174–176}{SN 35.174–176}

“Mendicants,\marginnote{1.1} you should give up desire … greed … desire and greed for what is not-self. And what is not-self? 

The\marginnote{1.3} eye, ear, nose, tongue, body, and mind are not-self …” 

%
\section*{{\suttatitleacronym SN 35.177–179}{\suttatitletranslation Desire, Etc. for the Impermanent Exterior }{\suttatitleroot Bāhirāniccachandādisutta}}
\addcontentsline{toc}{section}{\tocacronym{SN 35.177–179} \toctranslation{Desire, Etc. for the Impermanent Exterior } \tocroot{Bāhirāniccachandādisutta}}
\markboth{Desire, Etc. for the Impermanent Exterior }{Bāhirāniccachandādisutta}
\extramarks{SN 35.177–179}{SN 35.177–179}

“Mendicants,\marginnote{1.1} you should give up desire … greed … desire and greed for what is impermanent. And what is impermanent? 

Sights,\marginnote{1.3} sounds, smells, tastes, touches, and thoughts are impermanent …” 

%
\section*{{\suttatitleacronym SN 35.180–182}{\suttatitletranslation Desire, Etc. for the Suffering Exterior }{\suttatitleroot Bāhiradukkhachandādisutta}}
\addcontentsline{toc}{section}{\tocacronym{SN 35.180–182} \toctranslation{Desire, Etc. for the Suffering Exterior } \tocroot{Bāhiradukkhachandādisutta}}
\markboth{Desire, Etc. for the Suffering Exterior }{Bāhiradukkhachandādisutta}
\extramarks{SN 35.180–182}{SN 35.180–182}

“Mendicants,\marginnote{1.1} you should give up desire … greed … desire and greed for what is suffering. And what is suffering? 

Sights,\marginnote{1.3} sounds, smells, tastes, touches, and thoughts are suffering …” 

%
\section*{{\suttatitleacronym SN 35.183–185}{\suttatitletranslation Desire, Etc. for the Not-Self Exterior }{\suttatitleroot Bāhirānattachandādisutta}}
\addcontentsline{toc}{section}{\tocacronym{SN 35.183–185} \toctranslation{Desire, Etc. for the Not-Self Exterior } \tocroot{Bāhirānattachandādisutta}}
\markboth{Desire, Etc. for the Not-Self Exterior }{Bāhirānattachandādisutta}
\extramarks{SN 35.183–185}{SN 35.183–185}

“Mendicants,\marginnote{1.1} you should give up desire … greed … desire and greed for what is not-self. And what is not-self? 

Sights,\marginnote{1.3} sounds, smells, tastes, touches, and thoughts are not-self …” 

%
\section*{{\suttatitleacronym SN 35.186}{\suttatitletranslation The Interior Was Impermanent in the Past }{\suttatitleroot Ajjhattātītāniccasutta}}
\addcontentsline{toc}{section}{\tocacronym{SN 35.186} \toctranslation{The Interior Was Impermanent in the Past } \tocroot{Ajjhattātītāniccasutta}}
\markboth{The Interior Was Impermanent in the Past }{Ajjhattātītāniccasutta}
\extramarks{SN 35.186}{SN 35.186}

“Mendicants,\marginnote{1.1} in the past the eye, ear, nose, tongue, body, and mind were impermanent. 

Seeing\marginnote{1.2} this, a learned noble disciple grows disillusioned with the eye, ear, nose, tongue, body, and mind. Being disillusioned, desire fades away. When desire fades away they’re freed. When they’re freed, they know they’re freed. 

They\marginnote{1.4} understand: ‘Rebirth is ended, the spiritual journey has been completed, what had to be done has been done, there is no return to any state of existence.’” 

%
\section*{{\suttatitleacronym SN 35.187}{\suttatitletranslation The Interior Will Be Impermanent in the Future }{\suttatitleroot Ajjhattānāgatāniccasutta}}
\addcontentsline{toc}{section}{\tocacronym{SN 35.187} \toctranslation{The Interior Will Be Impermanent in the Future } \tocroot{Ajjhattānāgatāniccasutta}}
\markboth{The Interior Will Be Impermanent in the Future }{Ajjhattānāgatāniccasutta}
\extramarks{SN 35.187}{SN 35.187}

“Mendicants,\marginnote{1.1} in the future the eye, ear, nose, tongue, body, and mind will be impermanent …” 

%
\section*{{\suttatitleacronym SN 35.188}{\suttatitletranslation The Interior Is Impermanent in the Present }{\suttatitleroot Ajjhattapaccuppannāniccasutta}}
\addcontentsline{toc}{section}{\tocacronym{SN 35.188} \toctranslation{The Interior Is Impermanent in the Present } \tocroot{Ajjhattapaccuppannāniccasutta}}
\markboth{The Interior Is Impermanent in the Present }{Ajjhattapaccuppannāniccasutta}
\extramarks{SN 35.188}{SN 35.188}

“Mendicants,\marginnote{1.1} in the present the eye, ear, nose, tongue, body, and mind are impermanent …” 

%
\section*{{\suttatitleacronym SN 35.189–191}{\suttatitletranslation The Interior as Suffering in the Three Times }{\suttatitleroot Ajjhattātītādidukkhasutta}}
\addcontentsline{toc}{section}{\tocacronym{SN 35.189–191} \toctranslation{The Interior as Suffering in the Three Times } \tocroot{Ajjhattātītādidukkhasutta}}
\markboth{The Interior as Suffering in the Three Times }{Ajjhattātītādidukkhasutta}
\extramarks{SN 35.189–191}{SN 35.189–191}

“Mendicants,\marginnote{1.1} in the past … future … present the eye, ear, nose, tongue, body, and mind are suffering …” 

%
\section*{{\suttatitleacronym SN 35.192–194}{\suttatitletranslation The Interior as Not-Self in the Three Times }{\suttatitleroot Ajjhattātītādianattasutta}}
\addcontentsline{toc}{section}{\tocacronym{SN 35.192–194} \toctranslation{The Interior as Not-Self in the Three Times } \tocroot{Ajjhattātītādianattasutta}}
\markboth{The Interior as Not-Self in the Three Times }{Ajjhattātītādianattasutta}
\extramarks{SN 35.192–194}{SN 35.192–194}

“Mendicants,\marginnote{1.1} in the past … future … present the eye, ear, nose, tongue, body, and mind are not-self …” 

%
\section*{{\suttatitleacronym SN 35.195–197}{\suttatitletranslation The Exterior as Impermanent in the Three Times }{\suttatitleroot Bāhirātītādianiccasutta}}
\addcontentsline{toc}{section}{\tocacronym{SN 35.195–197} \toctranslation{The Exterior as Impermanent in the Three Times } \tocroot{Bāhirātītādianiccasutta}}
\markboth{The Exterior as Impermanent in the Three Times }{Bāhirātītādianiccasutta}
\extramarks{SN 35.195–197}{SN 35.195–197}

“Mendicants,\marginnote{1.1} in the past … future … present sights, sounds, smells, tastes, touches, and thoughts are impermanent …” 

%
\section*{{\suttatitleacronym SN 35.198–200}{\suttatitletranslation The Exterior as Suffering in the Three Times }{\suttatitleroot Bāhirātītādidukkhasutta}}
\addcontentsline{toc}{section}{\tocacronym{SN 35.198–200} \toctranslation{The Exterior as Suffering in the Three Times } \tocroot{Bāhirātītādidukkhasutta}}
\markboth{The Exterior as Suffering in the Three Times }{Bāhirātītādidukkhasutta}
\extramarks{SN 35.198–200}{SN 35.198–200}

“Mendicants,\marginnote{1.1} in the past … future … present sights, sounds, smells, tastes, touches, and thoughts are suffering …” 

%
\section*{{\suttatitleacronym SN 35.201–203}{\suttatitletranslation The Exterior as Not-Self in the Three Times }{\suttatitleroot Bāhirātītādianattasutta}}
\addcontentsline{toc}{section}{\tocacronym{SN 35.201–203} \toctranslation{The Exterior as Not-Self in the Three Times } \tocroot{Bāhirātītādianattasutta}}
\markboth{The Exterior as Not-Self in the Three Times }{Bāhirātītādianattasutta}
\extramarks{SN 35.201–203}{SN 35.201–203}

“Mendicants,\marginnote{1.1} in the past … future … present sights, sounds, smells, tastes, touches, and thoughts are not-self …” 

%
\section*{{\suttatitleacronym SN 35.204}{\suttatitletranslation The Interior and What’s Impermanent in the Past }{\suttatitleroot Ajjhattātītayadaniccasutta}}
\addcontentsline{toc}{section}{\tocacronym{SN 35.204} \toctranslation{The Interior and What’s Impermanent in the Past } \tocroot{Ajjhattātītayadaniccasutta}}
\markboth{The Interior and What’s Impermanent in the Past }{Ajjhattātītayadaniccasutta}
\extramarks{SN 35.204}{SN 35.204}

“Mendicants,\marginnote{1.1} in the past the eye, ear, nose, tongue, body, and mind were impermanent. What’s impermanent is suffering. What’s suffering is not-self. And what’s not-self should be truly seen with right understanding like this: ‘This is not mine, I am not this, this is not my self.’ …” 

%
\section*{{\suttatitleacronym SN 35.205}{\suttatitletranslation The Interior and What’s Impermanent in the Future }{\suttatitleroot Ajjhattānāgatayadaniccasutta}}
\addcontentsline{toc}{section}{\tocacronym{SN 35.205} \toctranslation{The Interior and What’s Impermanent in the Future } \tocroot{Ajjhattānāgatayadaniccasutta}}
\markboth{The Interior and What’s Impermanent in the Future }{Ajjhattānāgatayadaniccasutta}
\extramarks{SN 35.205}{SN 35.205}

“Mendicants,\marginnote{1.1} in the future the eye, ear, nose, tongue, body, and mind will be impermanent. What’s impermanent is suffering …” 

%
\section*{{\suttatitleacronym SN 35.206}{\suttatitletranslation The Interior and What’s Impermanent in the Present }{\suttatitleroot Ajjhattapaccuppannayadaniccasutta}}
\addcontentsline{toc}{section}{\tocacronym{SN 35.206} \toctranslation{The Interior and What’s Impermanent in the Present } \tocroot{Ajjhattapaccuppannayadaniccasutta}}
\markboth{The Interior and What’s Impermanent in the Present }{Ajjhattapaccuppannayadaniccasutta}
\extramarks{SN 35.206}{SN 35.206}

“Mendicants,\marginnote{1.1} in the present the eye, ear, nose, tongue, body, and mind are impermanent. What’s impermanent is suffering. …” 

%
\section*{{\suttatitleacronym SN 35.207–209}{\suttatitletranslation The Interior and What’s Suffering in the Three Times }{\suttatitleroot Ajjhattātītādiyaṁdukkhasutta}}
\addcontentsline{toc}{section}{\tocacronym{SN 35.207–209} \toctranslation{The Interior and What’s Suffering in the Three Times } \tocroot{Ajjhattātītādiyaṁdukkhasutta}}
\markboth{The Interior and What’s Suffering in the Three Times }{Ajjhattātītādiyaṁdukkhasutta}
\extramarks{SN 35.207–209}{SN 35.207–209}

“Mendicants,\marginnote{1.1} in the past … future … present the eye, ear, nose, tongue, body, and mind are suffering. What’s suffering is not-self …” 

%
\section*{{\suttatitleacronym SN 35.210–212}{\suttatitletranslation The Interior and What’s Not-Self in the Three Times }{\suttatitleroot Ajjhattātītādiyadanattasutta}}
\addcontentsline{toc}{section}{\tocacronym{SN 35.210–212} \toctranslation{The Interior and What’s Not-Self in the Three Times } \tocroot{Ajjhattātītādiyadanattasutta}}
\markboth{The Interior and What’s Not-Self in the Three Times }{Ajjhattātītādiyadanattasutta}
\extramarks{SN 35.210–212}{SN 35.210–212}

“Mendicants,\marginnote{1.1} in the past … future … present the eye, ear, nose, tongue, body, and mind are not-self. And what’s not-self should be truly seen with right understanding like this: ‘This is not mine, I am not this, this is not my self.’ …” 

%
\section*{{\suttatitleacronym SN 35.213–215}{\suttatitletranslation The Exterior and What’s Impermanent in the Three Times }{\suttatitleroot Bāhirātītādiyadaniccasutta}}
\addcontentsline{toc}{section}{\tocacronym{SN 35.213–215} \toctranslation{The Exterior and What’s Impermanent in the Three Times } \tocroot{Bāhirātītādiyadaniccasutta}}
\markboth{The Exterior and What’s Impermanent in the Three Times }{Bāhirātītādiyadaniccasutta}
\extramarks{SN 35.213–215}{SN 35.213–215}

“Mendicants,\marginnote{1.1} in the past … future … present sights, sounds, smells, tastes, touches, and thoughts are impermanent. What’s impermanent is suffering …” 

%
\section*{{\suttatitleacronym SN 35.216–218}{\suttatitletranslation The Exterior and What’s Suffering in the Three Times }{\suttatitleroot Bāhirātītādiyaṁdukkhasutta}}
\addcontentsline{toc}{section}{\tocacronym{SN 35.216–218} \toctranslation{The Exterior and What’s Suffering in the Three Times } \tocroot{Bāhirātītādiyaṁdukkhasutta}}
\markboth{The Exterior and What’s Suffering in the Three Times }{Bāhirātītādiyaṁdukkhasutta}
\extramarks{SN 35.216–218}{SN 35.216–218}

“Mendicants,\marginnote{1.1} in the past … future … present sights, sounds, smells, tastes, touches, and thoughts are suffering. What’s suffering is not-self …” 

%
\section*{{\suttatitleacronym SN 35.219–221}{\suttatitletranslation The Exterior and What’s Not-Self in the Three Times }{\suttatitleroot Bāhirātītādiyadanattasutta}}
\addcontentsline{toc}{section}{\tocacronym{SN 35.219–221} \toctranslation{The Exterior and What’s Not-Self in the Three Times } \tocroot{Bāhirātītādiyadanattasutta}}
\markboth{The Exterior and What’s Not-Self in the Three Times }{Bāhirātītādiyadanattasutta}
\extramarks{SN 35.219–221}{SN 35.219–221}

“Mendicants,\marginnote{1.1} in the past … future … present sights, sounds, smells, tastes, touches, and thoughts are not-self. And what’s not-self should be truly seen with right understanding like this: ‘This is not mine, I am not this, this is not my self.’ …” 

%
\section*{{\suttatitleacronym SN 35.222}{\suttatitletranslation The Interior as Impermanent }{\suttatitleroot Ajjhattāyatanaaniccasutta}}
\addcontentsline{toc}{section}{\tocacronym{SN 35.222} \toctranslation{The Interior as Impermanent } \tocroot{Ajjhattāyatanaaniccasutta}}
\markboth{The Interior as Impermanent }{Ajjhattāyatanaaniccasutta}
\extramarks{SN 35.222}{SN 35.222}

“Mendicants,\marginnote{1.1} the eye, ear, nose, tongue, body, and mind are impermanent. 

Seeing\marginnote{1.2} this … They understand: ‘… there is no return to any state of existence.’” 

%
\section*{{\suttatitleacronym SN 35.223}{\suttatitletranslation The Interior as Suffering }{\suttatitleroot Ajjhattāyatanadukkhasutta}}
\addcontentsline{toc}{section}{\tocacronym{SN 35.223} \toctranslation{The Interior as Suffering } \tocroot{Ajjhattāyatanadukkhasutta}}
\markboth{The Interior as Suffering }{Ajjhattāyatanadukkhasutta}
\extramarks{SN 35.223}{SN 35.223}

“Mendicants,\marginnote{1.1} the eye, ear, nose, tongue, body, and mind are suffering. …” 

Seeing\marginnote{1.2} this … They understand: ‘… there is no return to any state of existence.’” 

%
\section*{{\suttatitleacronym SN 35.224}{\suttatitletranslation The Interior as Not-Self }{\suttatitleroot Ajjhattāyatanaanattasutta}}
\addcontentsline{toc}{section}{\tocacronym{SN 35.224} \toctranslation{The Interior as Not-Self } \tocroot{Ajjhattāyatanaanattasutta}}
\markboth{The Interior as Not-Self }{Ajjhattāyatanaanattasutta}
\extramarks{SN 35.224}{SN 35.224}

“Mendicants,\marginnote{1.1} the eye, ear, nose, tongue, body, and mind are not-self. 

Seeing\marginnote{1.2} this … They understand: ‘… there is no return to any state of existence.’” 

%
\section*{{\suttatitleacronym SN 35.225}{\suttatitletranslation The Exterior as Impermanent }{\suttatitleroot Bāhirāyatanaaniccasutta}}
\addcontentsline{toc}{section}{\tocacronym{SN 35.225} \toctranslation{The Exterior as Impermanent } \tocroot{Bāhirāyatanaaniccasutta}}
\markboth{The Exterior as Impermanent }{Bāhirāyatanaaniccasutta}
\extramarks{SN 35.225}{SN 35.225}

“Mendicants,\marginnote{1.1} sights, sounds, smells, tastes, touches, and thoughts are impermanent. 

Seeing\marginnote{1.2} this … They understand: ‘… there is no return to any state of existence.’” 

%
\section*{{\suttatitleacronym SN 35.226}{\suttatitletranslation The Exterior as Suffering }{\suttatitleroot Bāhirāyatanadukkhasutta}}
\addcontentsline{toc}{section}{\tocacronym{SN 35.226} \toctranslation{The Exterior as Suffering } \tocroot{Bāhirāyatanadukkhasutta}}
\markboth{The Exterior as Suffering }{Bāhirāyatanadukkhasutta}
\extramarks{SN 35.226}{SN 35.226}

“Mendicants,\marginnote{1.1} sights, sounds, smells, tastes, touches, and thoughts are suffering. 

Seeing\marginnote{1.2} this … They understand: ‘… there is no return to any state of existence.’” 

%
\section*{{\suttatitleacronym SN 35.227}{\suttatitletranslation The Exterior as Not-Self }{\suttatitleroot Bāhirāyatanaanattasutta}}
\addcontentsline{toc}{section}{\tocacronym{SN 35.227} \toctranslation{The Exterior as Not-Self } \tocroot{Bāhirāyatanaanattasutta}}
\markboth{The Exterior as Not-Self }{Bāhirāyatanaanattasutta}
\extramarks{SN 35.227}{SN 35.227}

“Mendicants,\marginnote{1.1} sights, sounds, smells, tastes, touches, and thoughts are not-self. 

Seeing\marginnote{1.2} this … They understand: ‘… there is no return to any state of existence.’” 

%
\addtocontents{toc}{\let\protect\contentsline\protect\nopagecontentsline}
\chapter*{The Chapter on the Ocean }
\addcontentsline{toc}{chapter}{\tocchapterline{The Chapter on the Ocean }}
\addtocontents{toc}{\let\protect\contentsline\protect\oldcontentsline}

%
\section*{{\suttatitleacronym SN 35.228}{\suttatitletranslation The Ocean (1st) }{\suttatitleroot Paṭhamasamuddasutta}}
\addcontentsline{toc}{section}{\tocacronym{SN 35.228} \toctranslation{The Ocean (1st) } \tocroot{Paṭhamasamuddasutta}}
\markboth{The Ocean (1st) }{Paṭhamasamuddasutta}
\extramarks{SN 35.228}{SN 35.228}

“Mendicants,\marginnote{1.1} an unlearned ordinary person speaks of the ocean. But that’s not the ocean in the training of the Noble One. That’s just a large body of water, a large sea of water. For a person, the eye is an ocean, and its currents are made of sights. 

Someone\marginnote{2.1} who can withstand those currents is said to have crossed over the ocean of the eye, with its waves and whirlpools, its saltwater crocodiles and monsters. Crossed over, the brahmin stands on the far shore. 

For\marginnote{2.3} a person, the ear … nose … tongue … body … mind is an ocean, and its currents are made of thoughts. Someone who can withstand those currents is said to have crossed over the ocean of the mind, with its waves and whirlpools, its saltwater crocodiles and monsters. Crossed over, the brahmin stands on the far shore.” 

That\marginnote{2.11} is what the Buddha said. Then the Holy One, the Teacher, went on to say: 

\begin{verse}%
“A\marginnote{3.1} knowledge master who’s crossed the ocean so hard to cross, \\
with its saltwater crocodiles and monsters, its waves, whirlpools, and dangers; \\
they’ve completed the spiritual journey, and gone to the end of the world, \\
they’re called ‘one who has gone beyond’.” 

%
\end{verse}

%
\section*{{\suttatitleacronym SN 35.229}{\suttatitletranslation The Ocean (2nd) }{\suttatitleroot Dutiyasamuddasutta}}
\addcontentsline{toc}{section}{\tocacronym{SN 35.229} \toctranslation{The Ocean (2nd) } \tocroot{Dutiyasamuddasutta}}
\markboth{The Ocean (2nd) }{Dutiyasamuddasutta}
\extramarks{SN 35.229}{SN 35.229}

“Mendicants,\marginnote{1.1} an unlearned ordinary person speaks of the ocean. But that’s not the ocean in the training of the Noble One. That’s just a large body of water, a large sea of water. 

There\marginnote{1.4} are sights known by the eye that are likable, desirable, agreeable, pleasant, sensual, and arousing. This is called the ocean in the training of the Noble One. And it’s here that this world—with its gods, \textsanskrit{Māras} and \textsanskrit{Brahmās}, this population with its ascetics and brahmins, gods and humans—is for the most part sunk. It’s become tangled like string, knotted like a ball of thread, and matted like rushes and reeds, and it doesn’t escape the places of loss, the bad places, the underworld, transmigration. 

There\marginnote{2.1} are sounds … smells … tastes … touches … thoughts known by the mind that are likable, desirable, agreeable, pleasant, sensual, and arousing. This is called the ocean in the training of the Noble One. And it’s here that this world—with its gods, \textsanskrit{Māras} and \textsanskrit{Brahmās}, this population with its ascetics and brahmins, gods and humans—is for the most part sunk. It’s become tangled like string, knotted like a ball of thread, and matted like rushes and reeds, and it doesn’t escape the places of loss, the bad places, the underworld, transmigration. 

\begin{verse}%
Those\marginnote{3.1} in whom greed, hate, and ignorance \\
have faded away; \\
have crossed the ocean so hard to cross, \\
with its saltwater crocodiles and monsters, its waves and dangers. 

They’ve\marginnote{4.1} escaped their chains, given up death, and have no attachments. \\
They’ve given up suffering, so there are no more future lives. \\
They’ve come to an end, and cannot be measured; \\
and they’ve confounded the King of Death, I say.” 

%
\end{verse}

%
\section*{{\suttatitleacronym SN 35.230}{\suttatitletranslation The Simile of the Fisherman }{\suttatitleroot Bāḷisikopamasutta}}
\addcontentsline{toc}{section}{\tocacronym{SN 35.230} \toctranslation{The Simile of the Fisherman } \tocroot{Bāḷisikopamasutta}}
\markboth{The Simile of the Fisherman }{Bāḷisikopamasutta}
\extramarks{SN 35.230}{SN 35.230}

“Mendicants,\marginnote{1.1} suppose a fisherman was to cast a baited hook into a deep lake. Seeing the bait, a fish would swallow it. And so the fish that swallowed the hook would meet with tragedy and disaster, and the fisherman can do what he wants with it. 

In\marginnote{2.1} the same way, there are these six hooks in the world that mean tragedy and slaughter for living creatures. What six? 

There\marginnote{2.3} are sights known by the eye that are likable, desirable, agreeable, pleasant, sensual, and arousing. If a mendicant approves, welcomes, and keeps clinging to them, they’re called a mendicant who has swallowed \textsanskrit{Māra}’s hook. They’ve met with tragedy and disaster, and the Wicked One can do with them what he wants. 

There\marginnote{2.6} are sounds … smells … tastes … touches … thoughts known by the mind that are likable, desirable, agreeable, pleasant, sensual, and arousing. If a mendicant approves, welcomes, and keeps clinging to them, they’re called a mendicant who has swallowed \textsanskrit{Māra}’s hook. They’ve met with tragedy and disaster, and the Wicked One can do with them what he wants. 

There\marginnote{4.1} are sights known by the eye that are likable, desirable, agreeable, pleasant, sensual, and arousing. If a mendicant doesn’t approve, welcome, and keep clinging to them, they’re called a mendicant who hasn’t swallowed \textsanskrit{Māra}’s hook. They’ve broken the hook, destroyed it. They haven’t met with tragedy and disaster, and the Wicked One cannot do with them what he wants. 

There\marginnote{5.1} are sounds … smells … tastes … touches … thoughts known by the mind that are likable, desirable, agreeable, pleasant, sensual, and arousing. If a mendicant doesn’t approve, welcome, and keep clinging to them, they’re called a mendicant who hasn’t swallowed \textsanskrit{Māra}’s hook. They’ve broken the hook, destroyed it. They haven’t met with tragedy and disaster, and the Wicked One cannot do with them what he wants.” 

%
\section*{{\suttatitleacronym SN 35.231}{\suttatitletranslation The Simile of the Latex-Producing Tree }{\suttatitleroot Khīrarukkhopamasutta}}
\addcontentsline{toc}{section}{\tocacronym{SN 35.231} \toctranslation{The Simile of the Latex-Producing Tree } \tocroot{Khīrarukkhopamasutta}}
\markboth{The Simile of the Latex-Producing Tree }{Khīrarukkhopamasutta}
\extramarks{SN 35.231}{SN 35.231}

“Mendicants,\marginnote{1.1} take any monk or nun who, when it comes to sights known by the eye, still has greed, hate, and delusion, and has not given them up. If even trivial sights come into their range of vision they overcome their mind, let alone those that are compelling. Why is that? Because they still have greed, hate, and delusion, and have not given them up. 

When\marginnote{2.1} it comes to sounds … smells … tastes … touches … thoughts known by the mind, they still have greed, hate, and delusion, and have not given them up. If even trivial thoughts come into the range of the mind they overcome their mind, let alone those that are compelling. Why is that? Because they still have greed, hate, and delusion, and have not given them up. 

Suppose\marginnote{4.1} there was a latex-producing tree—such as a bodhi, a banyan, a wavy leaf fig, or a cluster fig—that’s a tender young sapling. If a man were to chop it here and there with a sharp axe, would latex come out?” 

“Yes,\marginnote{4.3} sir.” 

Why\marginnote{4.4} is that? Because it still has latex.” 

“In\marginnote{5.1} the same way, take any monk or nun who, when it comes to sights known by the eye, still has greed, hate, and delusion, and has not given them up. If even trivial sights come into their range of vision they overcome their mind, let alone those that are compelling. Why is that? Because they still have greed, hate, and delusion, and have not given them up. 

When\marginnote{6.1} it comes to sounds … smells … tastes … touches … thoughts known by the mind, they still have greed, hate, and delusion, and have not given them up. If even trivial thoughts come into the range of the mind they overcome their mind, let alone those that are compelling. Why is that? Because they still have greed, hate, and delusion, and have not given them up. 

Take\marginnote{8.1} any monk or nun who, when it comes to sights known by the eye, has no greed, hate, and delusion left, and has given them up. If even compelling sights come into their range of vision they don’t overcome their mind, let alone those that are trivial. Why is that? Because they have no greed, hate, and delusion left, and have given them up. 

When\marginnote{9.1} it comes to sounds … smells … tastes … touches … thoughts known by the mind, they have no greed, hate, and delusion left, and have given them up. If even compelling thoughts come into the range of the mind they don’t overcome their mind, let alone those that are trivial. Why is that? Because they have no greed, hate, and delusion left, and have given them up. 

Suppose\marginnote{9.4} there was a latex-producing tree—such as a bodhi, a banyan, a wavy leaf fig, or a cluster fig—that’s dried up, withered, and decrepit. If a man were to chop it here and there with a sharp axe, would latex come out?” 

“No,\marginnote{9.6} sir. Why is that? Because it has no latex left.” 

“In\marginnote{10.1} the same way, take any monk or nun who, when it comes to sights known by the eye, has no greed, hate, and delusion left, and has given them up. If even compelling sights come into their range of vision they don’t overcome their mind, let alone those that are trivial. Why is that? Because they have no greed, hate, and delusion left, and have given them up. 

When\marginnote{11.1} it comes to sounds … smells … tastes … touches … thoughts known by the mind, they have no greed, hate, and delusion left, and have given them up. If even compelling thoughts come into the range of the mind they don’t overcome their mind, let alone those that are trivial. Why is that? Because they have no greed, hate, and delusion left, and have given them up.” 

%
\section*{{\suttatitleacronym SN 35.232}{\suttatitletranslation With Koṭṭhita }{\suttatitleroot Koṭṭhikasutta}}
\addcontentsline{toc}{section}{\tocacronym{SN 35.232} \toctranslation{With Koṭṭhita } \tocroot{Koṭṭhikasutta}}
\markboth{With Koṭṭhita }{Koṭṭhikasutta}
\extramarks{SN 35.232}{SN 35.232}

At\marginnote{1.1} one time Venerable \textsanskrit{Sāriputta} and Venerable \textsanskrit{Mahākoṭṭhita} were staying near Benares, in the deer park at Isipatana. Then in the late afternoon, Venerable \textsanskrit{Mahākoṭṭhita} came out of retreat, went to Venerable \textsanskrit{Sāriputta}, and exchanged greetings with him. When the greetings and polite conversation were over, he sat down to one side and said to \textsanskrit{Sāriputta}: 

“Reverend\marginnote{2.1} \textsanskrit{Sāriputta}, which is it? Is the eye the fetter of sights, or are sights the fetter of the eye? Is the ear … nose … tongue … body … mind the fetter of thoughts, or are thoughts the fetter of the mind?” 

“Reverend\marginnote{3.1} \textsanskrit{Koṭṭhita}, the eye is not the fetter of sights, nor are sights the fetter of the eye. The fetter there is the desire and greed that arises from the pair of them. The ear … nose … tongue … body … mind is not the fetter of thoughts, nor are thoughts the fetter of the mind. The fetter there is the desire and greed that arises from the pair of them. 

Suppose\marginnote{4.1} there was a black ox and a white ox yoked by a single harness or yoke. Would it be right to say that the black ox is the yoke of the white ox, or the white ox is the yoke of the black ox?” 

“No,\marginnote{4.4} reverend. The black ox is not the yoke of the white ox, nor is the white ox the yoke of the black ox. The yoke there is the single harness or yoke that they’re yoked by.” 

“In\marginnote{5.1} the same way, the eye is not the fetter of sights, nor are sights the fetter of the eye. The fetter there is the desire and greed that arises from the pair of them. The ear … nose … tongue … body … mind is not the fetter of thoughts, nor are thoughts the fetter of the mind. The fetter there is the desire and greed that arises from the pair of them. 

If\marginnote{6.1} the eye were the fetter of sights, or if sights were the fetter of the eye, this living of the spiritual life for the complete ending of suffering would not be found. However, since this is not the case, but the fetter there is the desire and greed that arises from the pair of them, this living of the spiritual life for the complete ending of suffering is found. 

If\marginnote{7.1} the ear … nose … tongue … body … mind were the fetter of thoughts, or if thoughts were the fetter of the mind, this living of the spiritual life for the complete ending of suffering would not be found. However, since this is not the case, but the fetter there is the desire and greed that arises from the pair of them, this living of the spiritual life for the complete ending of suffering is found. 

This\marginnote{9.1} too is a way to understand how this is so. 

The\marginnote{10.1} Buddha has an eye with which he sees a sight. But he has no desire and greed, for his mind is well freed. The Buddha has an ear … nose … tongue … The Buddha has a body with which he senses touch. But he has no desire and greed, for his mind is well freed. The Buddha has a mind 

with\marginnote{10.22} which he knows thought. But he has no desire and greed, for his mind is well freed. 

This\marginnote{11.1} too is a way to understand how the eye is not the fetter of sights, nor are sights the fetter of the eye. The fetter there is the desire and greed that arises from the pair of them. The ear … nose … tongue … body … mind is not the fetter of thoughts, nor are thoughts the fetter of the mind. The fetter there is the desire and greed that arises from the pair of them.” 

%
\section*{{\suttatitleacronym SN 35.233}{\suttatitletranslation With Kāmabhū }{\suttatitleroot Kāmabhūsutta}}
\addcontentsline{toc}{section}{\tocacronym{SN 35.233} \toctranslation{With Kāmabhū } \tocroot{Kāmabhūsutta}}
\markboth{With Kāmabhū }{Kāmabhūsutta}
\extramarks{SN 35.233}{SN 35.233}

At\marginnote{1.1} one time the venerables Ānanda and \textsanskrit{Kāmabhū} were staying near Kosambi, in Ghosita’s Monastery. 

Then\marginnote{1.2} in the late afternoon, Venerable \textsanskrit{Kāmabhū} came out of retreat, went to Venerable Ānanda, and exchanged greetings with him. When the greetings and polite conversation were over, he sat down to one side and said to Ānanda: 

“Reverend\marginnote{2.1} Ānanda, which is it? Is the eye the fetter of sights, or are sights the fetter of the eye? Is the ear … nose … tongue … body … mind the fetter of thoughts, or are thoughts the fetter of the mind?” 

“Reverend\marginnote{3.1} \textsanskrit{Kāmabhū}, the eye is not the fetter of sights, nor are sights the fetter of the eye. The fetter there is the desire and greed that arises from the pair of them. The ear … nose … tongue … body … mind is not the fetter of thoughts, nor are thoughts the fetter of the mind. The fetter there is the desire and greed that arises from the pair of them. 

Suppose\marginnote{4.1} there was a black ox and a white ox yoked by a single harness or yoke. Would it be right to say that the black ox is the yoke of the white ox, or the white ox is the yoke of the black ox?” 

“No,\marginnote{4.4} reverend. The black ox is not the yoke of the white ox, nor is the white ox the yoke of the black ox. The yoke there is the single harness or yoke that they’re yoked by.” 

“In\marginnote{4.7} the same way, the eye is not the fetter of sights, nor are sights the fetter of the eye. The ear … nose … tongue … body … mind is not the fetter of thoughts, nor are thoughts the fetter of the mind. The fetter there is the desire and greed that arises from the pair of them.” 

%
\section*{{\suttatitleacronym SN 35.234}{\suttatitletranslation With Udāyī }{\suttatitleroot Udāyīsutta}}
\addcontentsline{toc}{section}{\tocacronym{SN 35.234} \toctranslation{With Udāyī } \tocroot{Udāyīsutta}}
\markboth{With Udāyī }{Udāyīsutta}
\extramarks{SN 35.234}{SN 35.234}

At\marginnote{1.1} one time the venerables Ānanda and \textsanskrit{Udāyī} were staying near Kosambi, in Ghosita’s Monastery. 

Then\marginnote{1.2} in the late afternoon, Venerable \textsanskrit{Udāyī} came out of retreat, went to Venerable Ānanda, and exchanged greetings with him. When the greetings and polite conversation were over, he sat down to one side and said to Ānanda: 

“Reverend\marginnote{2.1} Ānanda, the Buddha has explained, opened, and illuminated in many ways how this body is not-self. Is it possible to explain consciousness in the same way? To teach, assert, establish, clarify, analyze, and reveal how consciousness is not-self?” 

“It\marginnote{3.1} is possible, Reverend \textsanskrit{Udāyī}. 

Does\marginnote{4.1} eye consciousness arise dependent on the eye and sights?” 

“Yes,\marginnote{4.2} reverend.” 

“If\marginnote{4.3} the cause and reason that gives rise to eye consciousness were to totally and utterly cease without anything left over, would eye consciousness still be found?” 

“No,\marginnote{4.4} reverend.” 

“In\marginnote{4.5} this way, too, it can be understood how consciousness is not-self. 

Does\marginnote{5.1} ear … nose … tongue … body … mind consciousness arise dependent on the mind and thoughts?” 

“Yes,\marginnote{6.2} reverend.” 

“If\marginnote{6.3} the cause and reason that gives rise to mind consciousness were to totally and utterly cease without anything left over, would mind consciousness still be found?” 

“No,\marginnote{6.4} reverend.” 

“In\marginnote{6.5} this way, too, it can be understood how consciousness is not-self. 

Suppose\marginnote{7.1} there was a person in need of heartwood. Wandering in search of heartwood, they’d take a sharp axe and enter a forest. There they’d see a big banana tree, straight and young and grown free of defects. They’d cut it down at the base, cut off the root, cut off the top, and unroll the coiled sheaths. But they wouldn’t even find sapwood, much less heartwood. 

In\marginnote{7.2} the same way, a mendicant sees these six fields of contact as neither self nor belonging to self. So seeing, they don’t grasp anything in the world. Not grasping, they’re not anxious. Not being anxious, they personally become extinguished. 

They\marginnote{7.5} understand: ‘Rebirth is ended, the spiritual journey has been completed, what had to be done has been done, there is no return to any state of existence.’” 

%
\section*{{\suttatitleacronym SN 35.235}{\suttatitletranslation The Exposition on Burning }{\suttatitleroot Ādittapariyāyasutta}}
\addcontentsline{toc}{section}{\tocacronym{SN 35.235} \toctranslation{The Exposition on Burning } \tocroot{Ādittapariyāyasutta}}
\markboth{The Exposition on Burning }{Ādittapariyāyasutta}
\extramarks{SN 35.235}{SN 35.235}

“Mendicants,\marginnote{1.1} I will teach you an exposition of the teaching on burning. Listen … 

And\marginnote{1.3} what is the exposition of the teaching on burning? 

You’d\marginnote{1.4} be better off mutilating your eye faculty with a red-hot iron nail, burning, blazing and glowing, than getting caught up in the features by way of the details in sights known by the eye. For if you die at a time when your consciousness is still tied to gratification in the features or details, it’s possible you’ll go to one of two destinations: hell or the animal realm. I speak having seen this drawback. 

You’d\marginnote{2.1} be better off mutilating your ear faculty with a sharp iron spike … 

You’d\marginnote{3.1} be better off mutilating your nose faculty with a sharp nail cutter … 

You’d\marginnote{4.1} be better off mutilating your tongue faculty with a sharp razor … 

You’d\marginnote{5.1} be better off mutilating your body faculty with a sharp spear, burning, blazing and glowing, than getting caught up in the features by way of the details in touches known by the body. For if you die at a time when your consciousness is still tied to gratification in the features or details, it’s possible you’ll go to one of two destinations: hell or the animal realm. I speak having seen this drawback. 

You’d\marginnote{6.1} be better off sleeping. For I say that sleep is useless, fruitless, and unconsciousness for the living. But while you’re asleep you won’t fall under the sway of such thoughts that would make you create a schism in the \textsanskrit{Saṅgha}. I speak having seen this drawback. 

A\marginnote{7.1} noble disciple reflects on this: ‘Forget mutilating the eye faculty with a red-hot iron nail, burning, blazing and glowing! I’d better focus on the fact that the eye, sights, eye consciousness, and eye contact are impermanent. And the painful, pleasant, or neutral feeling that arises conditioned by eye contact is also impermanent. 

Forget\marginnote{8.1} mutilating the ear faculty with a sharp iron spike, burning, blazing and glowing! I’d better focus on the fact that the ear, sounds, ear consciousness, and ear contact are impermanent. And the painful, pleasant, or neutral feeling that arises conditioned by ear contact is also impermanent. 

Forget\marginnote{9.1} mutilating the nose faculty with a sharp nail cutter, burning, blazing and glowing! I’d better focus on the fact that the nose, smells, nose consciousness, and nose contact are impermanent. And the painful, pleasant, or neutral feeling that arises conditioned by nose contact is also impermanent. 

Forget\marginnote{10.1} mutilating the tongue faculty with a sharp razor, burning, blazing and glowing! I’d better focus on the fact that the tongue, tastes, tongue consciousness, and tongue contact are impermanent. And the painful, pleasant, or neutral feeling that arises conditioned by tongue contact is also impermanent. 

Forget\marginnote{11.1} mutilating the body faculty with a sharp spear, burning, blazing and glowing! I’d better focus on the fact that the body, touches, body consciousness, and body contact are impermanent. And the painful, pleasant, or neutral feeling that arises conditioned by body contact is also impermanent. 

Forget\marginnote{12.1} sleeping! I’d better focus on the fact that the mind, thoughts, mind consciousness, and mind contact are impermanent. And the painful, pleasant, or neutral feeling that arises conditioned by mind contact is also impermanent.’ 

Seeing\marginnote{13.1} this, a learned noble disciple grows disillusioned with the eye, sights, eye consciousness, and eye contact. And they become disillusioned with the painful, pleasant, or neutral feeling that arises conditioned by eye contact. They grow disillusioned with the ear … nose … tongue … body … mind … painful, pleasant, or neutral feeling that arises conditioned by mind contact. 

Being\marginnote{13.3} disillusioned, desire fades away. When desire fades away they’re freed. When they’re freed, they know they’re freed. 

They\marginnote{13.4} understand: ‘Rebirth is ended, the spiritual journey has been completed, what had to be done has been done, there is no return to any state of existence.’ 

This\marginnote{13.5} is the exposition of the teaching on burning.” 

%
\section*{{\suttatitleacronym SN 35.236}{\suttatitletranslation The Simile of Hands and Feet (1st) }{\suttatitleroot Paṭhamahatthapādopamasutta}}
\addcontentsline{toc}{section}{\tocacronym{SN 35.236} \toctranslation{The Simile of Hands and Feet (1st) } \tocroot{Paṭhamahatthapādopamasutta}}
\markboth{The Simile of Hands and Feet (1st) }{Paṭhamahatthapādopamasutta}
\extramarks{SN 35.236}{SN 35.236}

“Mendicants,\marginnote{1.1} when there are hands, picking up and putting down are found. When there are feet, coming and going are found. When there are joints, contracting and extending are found. When there’s a belly, hunger and thirst are found. 

In\marginnote{1.5} the same way, when there’s an eye, pleasure and pain arise internally conditioned by eye contact. When there’s an ear … nose … tongue … body … mind, pleasure and pain arise internally conditioned by mind contact. 

When\marginnote{2.1} there are no hands, picking up and putting down aren’t found. When there are no feet, coming and going aren’t found. When there are no joints, contracting and extending aren’t found. When there’s no belly, hunger and thirst aren’t found. 

In\marginnote{2.5} the same way, when there’s no eye, pleasure and pain don’t arise internally conditioned by eye contact. When there’s no ear … nose … tongue … body … mind, pleasure and pain don’t arise internally conditioned by mind contact.” 

%
\section*{{\suttatitleacronym SN 35.237}{\suttatitletranslation The Simile of Hands and Feet (2nd) }{\suttatitleroot Dutiyahatthapādopamasutta}}
\addcontentsline{toc}{section}{\tocacronym{SN 35.237} \toctranslation{The Simile of Hands and Feet (2nd) } \tocroot{Dutiyahatthapādopamasutta}}
\markboth{The Simile of Hands and Feet (2nd) }{Dutiyahatthapādopamasutta}
\extramarks{SN 35.237}{SN 35.237}

“Mendicants,\marginnote{1.1} when there are hands, there’s picking up and putting down. When there are feet, there’s coming and going. When there are joints, there’s contracting and extending. When there’s a belly, there’s hunger and thirst. 

In\marginnote{1.5} the same way, when there’s an eye, pleasure and pain arise internally conditioned by eye contact. When there’s an ear … nose … tongue … body … mind, pleasure and pain arise internally conditioned by mind contact. 

When\marginnote{2.1} there are no hands, there’s no picking up and putting down. When there are no feet, there’s no coming and going. When there are no joints, there’s no contracting and extending. When there’s no belly, there’s no hunger and thirst. 

In\marginnote{2.5} the same way, when there’s no eye, pleasure and pain don’t arise internally conditioned by eye contact. When there’s no ear … nose … tongue … body … mind, pleasure and pain don’t arise internally conditioned by mind contact.” 

%
\addtocontents{toc}{\let\protect\contentsline\protect\nopagecontentsline}
\chapter*{The Chapter on the Simile of the Vipers }
\addcontentsline{toc}{chapter}{\tocchapterline{The Chapter on the Simile of the Vipers }}
\addtocontents{toc}{\let\protect\contentsline\protect\oldcontentsline}

%
\section*{{\suttatitleacronym SN 35.238}{\suttatitletranslation The Simile of the Vipers }{\suttatitleroot Āsīvisopamasutta}}
\addcontentsline{toc}{section}{\tocacronym{SN 35.238} \toctranslation{The Simile of the Vipers } \tocroot{Āsīvisopamasutta}}
\markboth{The Simile of the Vipers }{Āsīvisopamasutta}
\extramarks{SN 35.238}{SN 35.238}

“Mendicants,\marginnote{1.1} suppose there were four lethal poisonous vipers. Then a person would come along who wants to live and doesn’t want to die, who wants to be happy and recoils from pain. 

They’d\marginnote{1.3} say to him, ‘Mister, here are four lethal poisonous vipers. They must be periodically picked up, washed, fed, and put to sleep. But when one or other of these four poisonous vipers gets angry with you, you’ll meet with death or deadly pain. So then, mister, do what has to be done.’ 

Then\marginnote{2.1} that man, terrified of those four poisonous vipers, would flee this way or that. 

They’d\marginnote{2.2} say to him, ‘Mister, there are five deadly enemies chasing you, thinking: “When we catch sight of him, we’ll murder him right there!” So then, mister, do what has to be done.’ 

Then\marginnote{3.1} that man, terrified of those four poisonous vipers and those five deadly enemies, would flee this way or that. 

They’d\marginnote{3.2} say to him, ‘Mister, there’s a sixth hidden killer chasing you with a drawn sword, thinking: “When I catch sight of him, I’ll chop off his head right there!” So then, mister, do what has to be done.’ 

Then\marginnote{4.1} that man, terrified of those four poisonous vipers and those five deadly enemies and the hidden killer, would flee this way or that. 

He’d\marginnote{4.2} see an empty village. But whatever house he enters is vacant, deserted, and empty. And whatever vessel he touches is vacant, hollow, and empty. 

They’d\marginnote{4.5} say to him, ‘Mister, there are bandits who raid villages, and they’re striking now. So then, mister, do what has to be done.’ 

Then\marginnote{5.1} that man, terrified of those four poisonous vipers and those five deadly enemies and the hidden killer and the bandits, would flee this way or that. 

He’d\marginnote{5.2} see a large deluge, whose near shore is dubious and perilous, while the far shore is a sanctuary free of peril. But there’s no ferryboat or bridge for crossing over. 

Then\marginnote{5.4} that man thought, ‘Why don’t I gather grass, sticks, branches, and leaves and make a raft? Riding on the raft, and paddling with my hands and feet, I can safely reach the far shore.’ 

And\marginnote{6.1} so that man did exactly that. Having crossed over and gone beyond, the brahmin stands on the far shore. 

I’ve\marginnote{7.1} made up this simile to make a point. And this is the point. 

‘Four\marginnote{7.3} lethal poisonous vipers’ is a term for the four primary elements: the elements of earth, water, fire, and air. 

‘Five\marginnote{8.1} deadly enemies’ is a term for the five grasping aggregates, that is: form, feeling, perception, choices, and consciousness. 

‘The\marginnote{9.1} sixth hidden killer with a drawn sword’ is a term for relishing and greed. 

‘Empty\marginnote{10.1} village’ is a term for the six interior sense fields. If an astute, competent, clever person investigates this in relation to the eye, it appears vacant, hollow, and empty. If an astute, competent, clever person investigates this in relation to the ear … nose … tongue … body … mind, it appears vacant, hollow, and empty. 

‘Bandits\marginnote{11.1} who raid villages’ is a term for the six exterior sense fields. The eye is struck by both agreeable and disagreeable sights. The ear … nose … tongue … body … mind is struck by both agreeable and disagreeable thoughts. 

‘Large\marginnote{12.1} deluge’ is a term for the four floods: the floods of sensual pleasures, desire to be reborn, views, and ignorance. 

‘The\marginnote{13.1} near shore that’s dubious and perilous’ is a term for identity. 

‘The\marginnote{14.1} far shore, a sanctuary free of peril’ is a term for extinguishment. 

‘The\marginnote{15.1} raft’ is a term for the noble eightfold path, that is: right view, right thought, right speech, right action, right livelihood, right effort, right mindfulness, and right immersion. 

‘Paddling\marginnote{16.1} with hands and feet’ is a term for being energetic. 

‘Crossed\marginnote{17.1} over, gone beyond, the brahmin stands on the shore’ is a term for a perfected one.” 

%
\section*{{\suttatitleacronym SN 35.239}{\suttatitletranslation The Simile of the Chariot }{\suttatitleroot Rathopamasutta}}
\addcontentsline{toc}{section}{\tocacronym{SN 35.239} \toctranslation{The Simile of the Chariot } \tocroot{Rathopamasutta}}
\markboth{The Simile of the Chariot }{Rathopamasutta}
\extramarks{SN 35.239}{SN 35.239}

“Mendicants,\marginnote{1.1} when a mendicant has three qualities they’re full of joy and happiness in the present life, and they have laid the groundwork for ending the defilements. What three? 

They\marginnote{1.3} guard the sense doors, eat in moderation, and are committed to wakefulness. 

And\marginnote{2.1} how does a mendicant guard the sense doors? 

When\marginnote{2.2} a mendicant sees a sight with their eyes, they don’t get caught up in the features and details. If the faculty of sight were left unrestrained, bad unskillful qualities of desire and aversion would become overwhelming. For this reason, they practice restraint, protecting the faculty of sight, and achieving its restraint. 

When\marginnote{2.4} they hear a sound with their ears … 

When\marginnote{2.5} they smell an odor with their nose … 

When\marginnote{2.6} they taste a flavor with their tongue … 

When\marginnote{2.7} they feel a touch with their body … 

When\marginnote{2.8} they know a thought with their mind, they don’t get caught up in the features and details. If the faculty of mind were left unrestrained, bad unskillful qualities of desire and aversion would become overwhelming. For this reason, they practice restraint, protecting the faculty of mind, and achieving its restraint. 

Suppose\marginnote{2.10} a chariot stood harnessed to thoroughbreds at a level crossroads, with a goad ready. Then a deft horse trainer, a master charioteer, might mount the chariot, taking the reins in his right hand and goad in the left. He’d drive out and back wherever he wishes, whenever he wishes. 

In\marginnote{2.11} the same way, a mendicant trains to protect, control, tame, and pacify these six senses. 

That’s\marginnote{2.12} how a mendicant guards the sense doors. 

And\marginnote{3.1} how does a mendicant eat in moderation? 

It’s\marginnote{3.2} when a mendicant reflects properly on the food that they eat: ‘Not for fun, indulgence, adornment, or decoration, but only to sustain this body, to avoid harm, and to support spiritual practice. In this way, I shall put an end to old discomfort and not give rise to new discomfort, and I will live blamelessly and at ease.’ 

It’s\marginnote{3.4} like a person who puts ointment on a wound only so that it can heal; or who oils an axle only so that it can carry a load. 

In\marginnote{3.5} the same way, a mendicant reflects properly on the food that they eat: ‘Not for fun, indulgence, adornment, or decoration, but only to sustain this body, to avoid harm, and to support spiritual practice. In this way, I shall put an end to old discomfort and not give rise to new discomfort, and I will live blamelessly and at ease.’ 

That’s\marginnote{3.7} how a mendicant eats in moderation. 

And\marginnote{4.1} how is a mendicant committed to wakefulness? 

It’s\marginnote{4.2} when a mendicant practices walking and sitting meditation by day, purifying their mind from obstacles. In the evening, they continue to practice walking and sitting meditation. In the middle of the night, they lie down in the lion’s posture—on the right side, placing one foot on top of the other—mindful and aware, and focused on the time of getting up. In the last part of the night, they get up and continue to practice walking and sitting meditation, purifying their mind from obstacles. 

This\marginnote{4.6} is how a mendicant is committed to wakefulness. 

When\marginnote{4.7} a mendicant has these three qualities they’re full of joy and happiness in the present life, and they have laid the groundwork for ending the defilements.” 

%
\section*{{\suttatitleacronym SN 35.240}{\suttatitletranslation The Simile of the Tortoise }{\suttatitleroot Kummopamasutta}}
\addcontentsline{toc}{section}{\tocacronym{SN 35.240} \toctranslation{The Simile of the Tortoise } \tocroot{Kummopamasutta}}
\markboth{The Simile of the Tortoise }{Kummopamasutta}
\extramarks{SN 35.240}{SN 35.240}

“Once\marginnote{1.1} upon a time, mendicants, a tortoise was grazing along the bank of a river in the afternoon. At the same time, a jackal was also hunting along the river bank. The tortoise saw the jackal off in the distance hunting, so it drew its limbs and neck inside its shell, and kept passive and silent. 

But\marginnote{1.5} the jackal also saw the tortoise off in the distance grazing. So it went up to the tortoise and waiting nearby, thinking, ‘When that tortoise sticks one or other of its limbs or neck out from its shell, I’ll grab it right there, rip it out, and eat it!’ 

But\marginnote{1.8} when that tortoise didn’t stick one or other of its limbs or neck out from its shell, the jackal left disappointed, since it couldn’t find a vulnerability. 

In\marginnote{2.1} the same way, \textsanskrit{Māra} the Wicked is always waiting nearby, thinking: ‘Hopefully I can find a vulnerability in the eye, ear, nose, tongue, body, or mind.’ That’s why you should live with sense doors guarded. 

When\marginnote{2.6} you see a sight with your eyes, don’t get caught up in the features and details. If the faculty of sight were left unrestrained, bad unskillful qualities of desire and aversion would become overwhelming. For this reason, practice restraint, protecting the faculty of sight, and achieving its restraint. 

When\marginnote{2.8} you hear a sound with your ears … 

When\marginnote{2.9} you smell an odor with your nose … 

When\marginnote{2.10} you taste a flavor with your tongue … 

When\marginnote{2.11} you feel a touch with your body … 

When\marginnote{2.12} you know a thought with your mind, don’t get caught up in the features and details. If the faculty of mind were left unrestrained, bad unskillful qualities of desire and aversion would become overwhelming. For this reason, practice restraint, protecting the faculty of mind, and achieving its restraint. 

When\marginnote{2.14} you live with your sense doors restrained, \textsanskrit{Māra} will leave you disappointed, since he can’t find a vulnerability, just like the jackal left the tortoise. 

\begin{verse}%
A\marginnote{3.1} mendicant should collect their thoughts \\
as a tortoise draws its limbs into its shell. \\
Independent, not disturbing others, \\
someone who’s extinguished wouldn’t blame anyone.” 

%
\end{verse}

%
\section*{{\suttatitleacronym SN 35.241}{\suttatitletranslation The Simile of the Tree Trunk (1st) }{\suttatitleroot Paṭhamadārukkhandhopamasutta}}
\addcontentsline{toc}{section}{\tocacronym{SN 35.241} \toctranslation{The Simile of the Tree Trunk (1st) } \tocroot{Paṭhamadārukkhandhopamasutta}}
\markboth{The Simile of the Tree Trunk (1st) }{Paṭhamadārukkhandhopamasutta}
\extramarks{SN 35.241}{SN 35.241}

At\marginnote{1.1} one time the Buddha was staying near Kosambi on the bank of the Ganges river. 

Seeing\marginnote{1.2} a large tree trunk being carried along by the current, he addressed the mendicants, “Mendicants, do you see that large tree trunk being carried along by the current of the Ganges river?” 

“Yes,\marginnote{1.5} sir.” 

“Mendicants,\marginnote{1.6} assume that that tree trunk doesn’t collide with the near shore or the far shore, or sink in the middle, or get stranded on high ground. And assume that it doesn’t get taken by humans or non-humans or caught up in a whirlpool, and that it doesn’t rot away. In that case, that tree trunk will slant, slope, and incline towards the ocean. Why is that? Because the current of the Ganges river slants, slopes, and inclines towards the ocean. 

In\marginnote{2.1} the same way, assume that you don’t collide with the near shore or the far shore, or sink in the middle, or get stranded on high ground. And assume that you don’t get taken by humans or non-humans or caught up in a whirlpool, and that you don’t rot away. In that case, you will slant, slope, and incline towards extinguishment. Why is that? Because right view slants, slopes, and inclines towards extinguishment.” 

When\marginnote{2.6} he said this, one of the mendicants asked the Buddha: 

“But\marginnote{2.7} sir, what’s the near shore and what’s the far shore? What’s sinking in the middle? What’s getting stranded on high ground? What’s getting taken by humans or non-humans? What’s getting caught up in a whirlpool? And what’s rotting away?” 

“‘The\marginnote{3.1} near shore’, mendicant, is a term for the six interior sense fields. 

‘The\marginnote{3.2} far shore’ is a term for the six exterior sense fields. 

‘Sinking\marginnote{3.3} in the middle’ is a term for greed and relishing. 

‘Stranded\marginnote{3.4} on high ground’ is a term for the conceit ‘I am’. 

And\marginnote{4.1} what’s getting taken by humans? It’s when someone mixes closely with laypeople, sharing their joys and sorrows—happy when they’re happy and sad when they’re sad—and getting involved in their business. That’s called getting taken by humans. 

And\marginnote{5.1} what’s getting taken by non-humans? It’s when someone leads the spiritual life wishing to be reborn in one of the orders of gods: ‘By this precept or observance or mortification or spiritual life, may I become one of the gods!’ That’s called getting taken by non-humans. ‘Caught up in a whirlpool’ is a term for the five kinds of sensual stimulation. 

And\marginnote{6.1} what’s rotting away? It’s when some person is unethical, of bad qualities, filthy, with suspicious behavior, underhand, no true ascetic or spiritual practitioner—though claiming to be one—rotten inside, corrupt, and depraved. This is called ‘rotting away’.” 

Now\marginnote{7.1} at that time Nanda the cowherd was sitting not far from the Buddha. Then he said to the Buddha: 

“I\marginnote{7.3} won’t collide with the near shore or the far shore, or sink in the middle, or get stranded on high ground. And I won’t get taken by humans or non-humans or caught up in a whirlpool, and I won’t rot away. Sir, may I receive the going forth, the ordination in the Buddha’s presence?” 

“Well\marginnote{7.5} then, Nanda, return the cows to their owners.” 

“Sir,\marginnote{7.6} the cows will go back by themselves, since they love their calves.” 

“Still,\marginnote{7.7} Nanda, you should return them to their owners.” 

Then\marginnote{7.8} Nanda, after returning the cows to their owners, went up to the Buddha and said to him, “Sir, I have returned the cows to their owners. May I receive the going forth, the ordination in the Buddha’s presence?” 

And\marginnote{7.11} the cowherd Nanda received the going forth, the ordination in the Buddha’s presence. Not long after his ordination, Venerable Nanda became one of the perfected. 

%
\section*{{\suttatitleacronym SN 35.242}{\suttatitletranslation The Simile of the Tree Trunk (2nd) }{\suttatitleroot Dutiyadārukkhandhopamasutta}}
\addcontentsline{toc}{section}{\tocacronym{SN 35.242} \toctranslation{The Simile of the Tree Trunk (2nd) } \tocroot{Dutiyadārukkhandhopamasutta}}
\markboth{The Simile of the Tree Trunk (2nd) }{Dutiyadārukkhandhopamasutta}
\extramarks{SN 35.242}{SN 35.242}

At\marginnote{1.1} one time the Buddha was staying near \textsanskrit{Kimibilā} on the bank of the Ganges river. Seeing a large tree trunk being carried along by the current, he addressed the mendicants: “Mendicants, do you see that large tree trunk being carried along by the current of the Ganges river?” 

“Yes,\marginnote{1.5} sir,” they replied. … When this was said, Venerable Kimbila said to the Buddha: 

“But\marginnote{1.7} sir, what’s the near shore and what’s the far shore? What’s sinking in the middle? What’s getting stranded on high ground? What’s getting taken by humans or non-humans? What’s getting caught up in a whirlpool? And what’s rotting away?” … 

“And\marginnote{1.8} what, Kimbila, is rotting away? It’s when a mendicant has committed the kind of corrupt offense from which there is no rehabilitation. This is called ‘rotting away’.” 

%
\section*{{\suttatitleacronym SN 35.243}{\suttatitletranslation The Explanation on the Corrupt }{\suttatitleroot Avassutapariyāyasutta}}
\addcontentsline{toc}{section}{\tocacronym{SN 35.243} \toctranslation{The Explanation on the Corrupt } \tocroot{Avassutapariyāyasutta}}
\markboth{The Explanation on the Corrupt }{Avassutapariyāyasutta}
\extramarks{SN 35.243}{SN 35.243}

At\marginnote{1.1} one time the Buddha was staying in the land of the Sakyans, near Kapilavatthu in the Banyan Tree Monastery. Now at that time a new town hall had recently been constructed for the Sakyans of Kapilavatthu. It had not yet been occupied by an ascetic or brahmin or any person at all. 

Then\marginnote{1.3} the Sakyans of Kapilavatthu went up to the Buddha, bowed, sat down to one side, and said to him, “Sir, a new town hall has recently been constructed for the Sakyans of Kapilavatthu. It has not yet been occupied by an ascetic or brahmin or any person at all. May the Buddha be the first to use it, and only then will the Sakyans of Kapilavatthu use it. That would be for the lasting welfare and happiness of the Sakyans of Kapilavatthu.” The Buddha consented in silence. 

Knowing\marginnote{2.1} that the Buddha had consented, the Sakyans got up from their seat, bowed, and respectfully circled the Buddha, keeping him on their right. Then they went to the new town hall, where they spread carpets all over, prepared seats, set up a water jar, and placed a lamp. Then they went back to the Buddha and told him of their preparations, saying, “Please, sir, come at your convenience.” 

Then\marginnote{2.4} the Buddha robed up and, taking his bowl and robe, went to the new town hall together with the \textsanskrit{Saṅgha} of mendicants. Having washed his feet he entered the town hall and sat against the central column facing east. The \textsanskrit{Saṅgha} of mendicants also washed their feet, entered the town hall, and sat against the west wall facing east, with the Buddha right in front of them. The Sakyans of Kapilavatthu also washed their feet, entered the town hall, and sat against the east wall facing west, with the Buddha right in front of them. The Buddha spent most of the night educating, encouraging, firing up, and inspiring the Sakyans with a Dhamma talk. Then he dismissed them, saying, “The night is getting late, Gotamas. Please go at your convenience.” 

“Yes,\marginnote{2.10} sir,” replied the Sakyans. They got up from their seat, bowed, and respectfully circled the Buddha, keeping him on their right, before leaving. 

And\marginnote{3.1} then, soon after the Sakyans had left, the Buddha addressed Venerable \textsanskrit{Mahāmoggallāna}, “\textsanskrit{Moggallāna}, the \textsanskrit{Saṅgha} of mendicants is rid of dullness and drowsiness. Give them some Dhamma talk as you feel inspired. My back is sore, I’ll stretch it.” 

“Yes,\marginnote{3.6} sir,” \textsanskrit{Mahāmoggallāna} replied. And then the Buddha spread out his outer robe folded in four and laid down in the lion’s posture—on the right side, placing one foot on top of the other—mindful and aware, and focused on the time of getting up. 

There\marginnote{3.8} Venerable \textsanskrit{Mahāmoggallāna} addressed the mendicants: “Reverends, mendicants!” 

“Reverend,”\marginnote{3.10} they replied. Venerable \textsanskrit{Mahāmoggallāna} said this: 

“I\marginnote{3.12} will teach you the explanation of the corrupt and the uncorrupted. Listen and pay close attention, I will speak.” 

“Yes,\marginnote{3.14} reverend,” they replied. Venerable \textsanskrit{Mahāmoggallāna} said this: 

“And\marginnote{4.1} how is someone corrupt? 

Take\marginnote{4.2} a mendicant who sees a sight with the eye. If it’s pleasant they hold on to it, but if it’s unpleasant they dislike it. They live with mindfulness of the body unestablished and their heart restricted. And they don’t truly understand the freedom of heart and freedom by wisdom where those arisen bad, unskillful qualities cease without anything left over. 

They\marginnote{4.4} hear a sound … smell an odor … taste a flavor … feel a touch … know a thought with the mind. If it’s pleasant they hold on to it, but if it’s unpleasant they dislike it. They live with mindfulness of the body unestablished and a limited heart. And they don’t truly understand the freedom of heart and freedom by wisdom where those arisen bad, unskillful qualities cease without anything left over. 

This\marginnote{4.7} is called a mendicant who is corrupt when it comes to sights known by the eye, sounds … smells … tastes … touches … thoughts known by the mind. 

When\marginnote{4.10} a mendicant lives like this, if \textsanskrit{Māra} comes at them through the eye he finds a vulnerability and gets hold of them. If \textsanskrit{Māra} comes at them through the ear … nose … tongue … body … mind he finds a vulnerability and gets hold of them. 

Suppose\marginnote{5.1} there was a house made of reeds or straw that was dried up, withered, and decrepit. If a person came to it with a burning grass torch from the east, west, north, south, below, above, or from anywhere, the fire would find a vulnerability, it would get a foothold. 

In\marginnote{5.9} the same way, when a mendicant lives like this, if \textsanskrit{Māra} comes at them through the eye he finds a vulnerability and gets hold of them. If \textsanskrit{Māra} comes at them through the ear … nose … tongue … body … mind he finds a vulnerability and gets hold of them. 

When\marginnote{5.12} a mendicant lives like this, they’re mastered by sights, sounds, smells, tastes, touches, and thoughts, they don’t master these things. 

This\marginnote{5.18} is called a mendicant who has been mastered by sights, sounds, smells, tastes, touches, and thoughts. They’re mastered, not a master. Bad, unskillful qualities have mastered them, which are corrupting, leading to future lives, hurtful, and resulting in suffering and future rebirth, old age, and death. 

That’s\marginnote{5.19} how someone is corrupt. 

And\marginnote{6.1} how is someone uncorrupted? 

Take\marginnote{6.2} a mendicant who sees a sight with the eye. If it’s pleasant they don’t hold on to it, and if it’s unpleasant they don’t dislike it. They live with mindfulness of the body established and a limitless heart. And they truly understand the freedom of heart and freedom by wisdom where those arisen bad, unskillful qualities cease without anything left over. 

They\marginnote{6.4} hear a sound … smell an odor … taste a flavor … feel a touch … know a thought with the mind. If it’s pleasant they don’t hold on to it, and if it’s unpleasant they don’t dislike it. They live with mindfulness of the body established and a limitless heart. And they truly understand the freedom of heart and freedom by wisdom where those arisen bad, unskillful qualities cease without anything left over. 

This\marginnote{6.7} is called a mendicant who is uncorrupted when it comes to sights known by the eye, sounds … smells … tastes … touches … thoughts known by the mind. 

When\marginnote{6.9} a mendicant lives like this, if \textsanskrit{Māra} comes at them through the eye he doesn’t find a vulnerability or get hold of them. If \textsanskrit{Māra} comes at them through the ear … nose … tongue … body … mind he doesn’t find a vulnerability or get hold of them. 

Suppose\marginnote{7.1} there was a bungalow or hall made of thick clay with its plaster still wet. If a person came to it with a burning grass torch from the east, west, north, south, below, above, or from anywhere, the fire wouldn’t find a vulnerability, it would get no foothold. 

In\marginnote{7.9} the same way, when a mendicant lives like this, if \textsanskrit{Māra} comes at them through the eye he doesn’t find a vulnerability or get hold of them. If \textsanskrit{Māra} comes at them through the ear … nose … tongue … body … mind he doesn’t find a vulnerability or get hold of them. 

When\marginnote{7.11} a mendicant lives like this, they master sights, sounds, smells, tastes, touches, and thoughts, they’re not mastered by these things. 

This\marginnote{7.17} is called a mendicant who has mastered sights, sounds, smells, tastes, touches, and thoughts. They’re a master, not mastered. Bad, unskillful qualities have been mastered by them, which are corrupting, leading to future lives, hurtful, and resulting in suffering and future rebirth, old age, and death. 

That’s\marginnote{7.18} how someone is uncorrupted.” 

Then\marginnote{8.1} the Buddha got up and said to Venerable \textsanskrit{Mahāmoggallāna}: 

“Good,\marginnote{8.2} good, \textsanskrit{Moggallāna}! It’s good that you’ve taught this explanation of the corrupt and the uncorrupted.” 

This\marginnote{9.1} is what Venerable \textsanskrit{Mahāmoggallāna} said, and the teacher approved. Satisfied, the mendicants were happy with what \textsanskrit{Mahāmoggallāna} said. 

%
\section*{{\suttatitleacronym SN 35.244}{\suttatitletranslation Entailing Suffering }{\suttatitleroot Dukkhadhammasutta}}
\addcontentsline{toc}{section}{\tocacronym{SN 35.244} \toctranslation{Entailing Suffering } \tocroot{Dukkhadhammasutta}}
\markboth{Entailing Suffering }{Dukkhadhammasutta}
\extramarks{SN 35.244}{SN 35.244}

“Mendicants,\marginnote{1.1} when a mendicant truly understands the origin and ending of all things that entail suffering, then they’ve seen sensual pleasures in such a way that they have no underlying tendency for desire, affection, infatuation, and passion for sensual pleasures. And they’ve awakened to a way of conduct and a way of living such that, when they live in that way, bad, unskillful qualities of desire and grief don’t overwhelm them. 

And\marginnote{2.1} how does a mendicant truly understand the origin and ending of all things that entail suffering? 

‘Such\marginnote{2.2} is form, such is the origin of form, such is the ending of form. Such is feeling … perception … choices … consciousness, such is the origin of consciousness, such is the ending of consciousness.’ 

That’s\marginnote{2.7} how a mendicant truly understands the origin and ending of all things that entail suffering. 

And\marginnote{3.1} how has a mendicant seen sensual pleasures in such a way that they have no underlying tendency for desire, affection, infatuation, and passion for sensual pleasures? 

Suppose\marginnote{3.2} there was a pit of glowing coals deeper than a man’s height, filled with glowing coals that neither flamed nor smoked. Then a person would come along who wants to live and doesn’t want to die, who wants to be happy and recoils from pain. Then two strong men grab would grab each arm and drag them towards the pit of glowing coals. They’d writhe and struggle to and fro. Why is that? For that person knows, ‘If I fall in that pit of glowing coals, that will result in my death or deadly pain.’ 

In\marginnote{3.5} the same way, when a mendicant has seen sensual pleasures as like a pit of glowing coals, they have no underlying tendency for desire, affection, infatuation, and passion for sensual pleasures. 

And\marginnote{4.1} how has a mendicant awakened to a way of conduct and a way of living such that, when they live in that way, bad, unskillful qualities of desire and grief don’t overwhelm them? 

Suppose\marginnote{4.2} a person was to enter a thicket full of thorns. They’d have thorns in front and behind, to the left and right, below and above. So they’d go forward mindfully and come back mindfully, thinking, ‘May I not get any thorns!’ 

In\marginnote{4.3} the same way, whatever in the world seems nice and pleasant is called a thorn in the training of the Noble One. When they understand what a thorn is, they should understand restraint and lack of restraint. 

And\marginnote{5.1} how is someone unrestrained? 

Take\marginnote{5.2} a mendicant who sees a sight with the eye. If it’s pleasant they hold on to it, but if it’s unpleasant they dislike it. They live with mindfulness of the body unestablished and their heart restricted. And they don’t truly understand the freedom of heart and freedom by wisdom where those arisen bad, unskillful qualities cease without anything left over. 

They\marginnote{5.4} hear a sound … smell an odor … taste a flavor … feel a touch … know a thought with the mind. If it’s pleasant they hold on to it, but if it’s unpleasant they dislike it. They live with mindfulness of the body unestablished and a limited heart. And they don’t truly understand the freedom of heart and freedom by wisdom where those arisen bad, unskillful qualities cease without anything left over. 

This\marginnote{5.7} is how someone is unrestrained. 

And\marginnote{6.1} how is someone restrained? 

Take\marginnote{6.2} a mendicant who sees a sight with the eye. If it’s pleasant they don’t hold on to it, and if it’s unpleasant they don’t dislike it. They live with mindfulness of the body established and a limitless heart. And they truly understand the freedom of heart and freedom by wisdom where those arisen bad, unskillful qualities cease without anything left over. 

They\marginnote{6.4} hear a sound … smell an odor … taste a flavor … feel a touch … know a thought with the mind. If it’s pleasant they don’t hold on to it, and if it’s unpleasant they don’t dislike it. They live with mindfulness of the body established and a limitless heart. And they truly understand the freedom of heart and freedom by wisdom where those arisen bad, unskillful qualities cease without anything left over. 

This\marginnote{6.7} is how someone is restrained. 

Though\marginnote{7.1} that mendicant conducts themselves and lives in this way, every so often they might lose mindfulness, and bad, unskillful memories and thoughts prone to fetters arise. If this happens, their mindfulness is slow to come up, but they quickly give them up, get rid of, eliminate, and obliterate those thoughts. 

Suppose\marginnote{8.1} there was an iron cauldron that had been heated all day, and a person let two or three drops of water fall onto it. The drops would be slow to fall, but they’d quickly dry up and evaporate. 

In\marginnote{8.2} the same way, though that mendicant conducts themselves and lives in this way, every so often they might lose mindfulness, and bad, unskillful memories and thoughts prone to fetters arise. If this happens, their mindfulness is slow to come up, but they quickly give them up, get rid of, eliminate, and obliterate those thoughts. 

This\marginnote{8.3} is how a mendicant has awakened to a way of conduct and a way of living such that, when they live in that way, bad, unskillful qualities of desire and grief don’t overwhelm them. 

While\marginnote{8.4} that mendicant conducts themselves in this way and lives in this way, it may be that rulers or their ministers, friends or colleagues, relatives or family would invite them to accept wealth, saying, ‘Please, mister, why let these ocher robes torment you? Why follow the practice of shaving your head and carrying an alms bowl? Come, return to a lesser life, enjoy wealth, and make merit!’ But it’s simply impossible for a mendicant who conducts themselves in this way and lives in this way to resign the training and return to a lesser life. 

Suppose\marginnote{9.1} that, although the Ganges river slants, slopes, and inclines to the east, a large crowd were to come along with a spade and basket, saying: ‘We’ll make this Ganges river slant, slope, and incline to the west!’ 

What\marginnote{9.2} do you think, mendicants? Would they still succeed?” 

“No,\marginnote{9.4} sir. Why is that? The Ganges river slants, slopes, and inclines to the east. It’s not easy to make it slant, slope, and incline to the west. That large crowd will eventually get weary and frustrated.” 

“In\marginnote{9.9} the same way, while that mendicant conducts themselves in this way and lives in this way, it may be that rulers or their ministers, friends or colleagues, relatives or family should invite them to accept wealth, saying, ‘Please, mister, why let these ocher robes torment you? Why follow the practice of shaving your head and carrying an alms bowl? Come, return to a lesser life, enjoy wealth, and make merit!’ But it’s simply impossible for a mendicant who conducts themselves in this way and lives in this way to resign the training and return to a lesser life. 

Why\marginnote{9.14} is that? Because for a long time that mendicant’s mind has slanted, sloped, and inclined to seclusion. So it’s impossible for them to return to a lesser life.” 

%
\section*{{\suttatitleacronym SN 35.245}{\suttatitletranslation The Simile of the Parrot Tree }{\suttatitleroot Kiṁsukopamasutta}}
\addcontentsline{toc}{section}{\tocacronym{SN 35.245} \toctranslation{The Simile of the Parrot Tree } \tocroot{Kiṁsukopamasutta}}
\markboth{The Simile of the Parrot Tree }{Kiṁsukopamasutta}
\extramarks{SN 35.245}{SN 35.245}

Then\marginnote{1.1} one mendicant went up to another mendicant and asked, “Reverend, at what point is a mendicant’s vision well purified?” 

“When\marginnote{1.3} a mendicant truly understands the origin and ending of the six fields of contact, at that point their vision is well purified.” 

Not\marginnote{2.1} content with that answer, that mendicant went up to a series of other mendicants and received the following answers: 

“When\marginnote{2.3} a mendicant truly understands the origin and ending of the five grasping aggregates, at that point their vision is well purified.” 

“When\marginnote{3.3} a mendicant truly understands the origin and ending of the four primary elements, at that point their vision is well purified.” 

“When\marginnote{4.3} a mendicant truly understands that everything that has a beginning has an end, at that point their vision is well purified.” 

Not\marginnote{5.1} content with any of those answers, that mendicant went up to the Buddha and told him what had happened. Then he asked, “Sir, at what point is a mendicant’s vision well purified?” 

“Mendicant,\marginnote{6.1} suppose a person had never seen a parrot tree. They’d go up to someone who had seen a parrot tree and ask them, ‘Mister, what’s a parrot tree like?’ 

They’d\marginnote{6.4} say, ‘A parrot tree is blackish, like a charred stump.’ Now, at that time a parrot tree may well have been just as that person saw it. 

Not\marginnote{7.1} content with that answer, that person would go up to a series of other people and receive the following answers: ‘A parrot tree is reddish, like a lump of meat.’ ‘A parrot tree has flaking bark and burst pods, like an acacia.’ ‘A parrot tree has luxuriant, shady foliage, like a banyan.’ Now, at each of those times a parrot tree may well have been just as those people saw them. 

In\marginnote{7.16} the same way, those good people each answered according to what they were focused on when their vision was well purified. 

Suppose\marginnote{8.1} there was a king’s frontier citadel with fortified embankments, ramparts, and arches, and six gates. And it has a gatekeeper who is astute, competent, and clever. He keeps strangers out and lets known people in. 

A\marginnote{8.3} swift pair of messengers would arrive from the east and say to the gatekeeper, ‘Mister, where is the lord of the city?’ 

They’d\marginnote{8.5} say, ‘There he is, sirs, seated at the central square.’ 

Then\marginnote{8.7} that swift pair of messengers would deliver a message of truth to the lord of the city and depart the way they came. 

A\marginnote{8.8} swift pair of messengers would come from the west … north … south … deliver a message of truth to the lord of the city and depart the way they came. 

I’ve\marginnote{9.1} made up this simile to make a point. And this is the point. 

‘City’\marginnote{9.3} is a term for this body made up of the four primary elements, produced by mother and father, built up from rice and porridge, liable to impermanence, to wearing away and erosion, to breaking up and destruction. 

‘Six\marginnote{9.4} gates’ is a term for the six interior sense fields. 

‘Gatekeeper’\marginnote{9.5} is a term for mindfulness. 

‘A\marginnote{9.6} swift pair of messengers’ is a term for serenity and discernment. 

‘The\marginnote{9.7} lord of the city’ is a term for consciousness. 

‘The\marginnote{9.8} central square’ is a term for the four primary elements: the elements of earth, water, fire, and air. 

‘A\marginnote{9.10} message of truth’ is a term for extinguishment. 

‘The\marginnote{9.11} way they came’ is a term for the noble eightfold path, that is, right view, right thought, right speech, right action, right livelihood, right effort, right mindfulness, and right immersion.” 

%
\section*{{\suttatitleacronym SN 35.246}{\suttatitletranslation The Simile of the Harp }{\suttatitleroot Vīṇopamasutta}}
\addcontentsline{toc}{section}{\tocacronym{SN 35.246} \toctranslation{The Simile of the Harp } \tocroot{Vīṇopamasutta}}
\markboth{The Simile of the Harp }{Vīṇopamasutta}
\extramarks{SN 35.246}{SN 35.246}

“Mendicants,\marginnote{1.1} any monk or nun who has desire or greed or hate or delusion or repulsion come up for sights known by the eye should shield their mind from them: ‘This path is dangerous and perilous, thorny and tangled; it’s a wrong turn, a bad path, a harmful way. This path is frequented by bad people, not by good people. It’s not worthy of you.’ The mind should be shielded from this when it comes to sights known by the eye. 

Any\marginnote{1.6} monk or nun who has desire or greed or hate or delusion or repulsion come up for sounds … smells … tastes … touches … thoughts known by the mind should shield their mind against them: ‘This path is dangerous and perilous, thorny and tangled; it’s a wrong turn, a bad path, a harmful way. This path is frequented by bad people, not by good people. It’s not worthy of you.’ The mind should be shielded from this when it comes to thoughts known by the mind. 

Suppose\marginnote{2.1} the crops have ripened, but the caretaker is negligent. If an ox fond of crops invades the crops they’d indulge themselves as much as they like. 

In\marginnote{2.3} the same way, when an unlearned ordinary person doesn’t exercise restraint when it comes to the six fields of contact, they indulge themselves in the five kinds of sensual stimulation as much as they like. 

Suppose\marginnote{3.1} the crops have ripened, and the caretaker is diligent. If an ox fond of crops invades the crops the caretaker would grab them firmly by the muzzle. Then they’d grab them above the hump and hold them fast there. Then they’d give them a good thrashing before driving them away. For a second time, and even a third time, the same thing might happen. As a result, no matter how long they stand or sit in a village or wilderness, that ox fond of crops would never invade that crop again, remembering the beating they got earlier. 

In\marginnote{3.14} the same way, when a mendicant’s mind is subdued, well subdued when it comes to the six fields of contact, becomes stilled internally; it settles, unifies, and becomes immersed in \textsanskrit{samādhi}. 

Suppose\marginnote{4.1} a king or their minister had never heard the sound of an arched harp. When he first hears the sound, he’d say, ‘My man, what is making this sound, so arousing, sensuous, intoxicating, infatuating, and captivating?’ 

They’d\marginnote{4.5} say to him, ‘That, sir, is an arched harp.’ 

He’d\marginnote{4.7} say, ‘Go, my man, fetch me that arched harp.’ 

So\marginnote{4.9} they’d fetch it and say, ‘This, sir, is that arched harp.’ 

He’d\marginnote{4.12} say, ‘I’ve had enough of that arched harp! Just fetch me the sound.’ 

They’d\marginnote{4.14} say, ‘Sir, this arched harp is made of many components assembled together, which make a sound when they’re played. That is, it depends on the body, the skin, the neck, the head, the strings, the plectrum, and a person to play it properly. That’s how an arched harp is made of many components assembled together, which make a sound when they’re played.’ 

But\marginnote{4.19} he’d split that harp into ten pieces or a hundred pieces, then splinter it up. He’d burn the splinters with fire, and reduce them to ashes. Then he’d sweep away the ashes in a strong wind, or float them away down a swift stream. 

Then\marginnote{4.22} he’d say, ‘It seems that there’s nothing to this thing called an arched harp or whatever’s called an arched harp! But people waste their time with it, negligent and heedless!’ 

In\marginnote{4.24} the same way, a mendicant searches for form, feeling, perception, choices, and consciousness anywhere they might be reborn. As they search in this way, their thoughts of ‘I’ or ‘mine’ or ‘I am’ are no more.” 

%
\section*{{\suttatitleacronym SN 35.247}{\suttatitletranslation The Simile of Six Animals }{\suttatitleroot Chappāṇakopamasutta}}
\addcontentsline{toc}{section}{\tocacronym{SN 35.247} \toctranslation{The Simile of Six Animals } \tocroot{Chappāṇakopamasutta}}
\markboth{The Simile of Six Animals }{Chappāṇakopamasutta}
\extramarks{SN 35.247}{SN 35.247}

“Mendicants,\marginnote{1.1} suppose a person with wounded and festering limbs was to enter a thicket of thorny reeds. The kusa thorns would pierce their feet, and the reed leaves would scratch their limbs. And that would cause that person to experience even more pain and distress. 

In\marginnote{1.4} the same way, some mendicant goes to a village or a wilderness and gets scolded, ‘This venerable, acting like this, behaving like this, is a filthy village thorn.’ Understanding that they’re a thorn, they should understand restraint and lack of restraint. 

And\marginnote{2.1} how is someone unrestrained? 

Take\marginnote{2.2} a mendicant who sees a sight with their eyes. If it’s pleasant they hold on to it, but if it’s unpleasant they dislike it. They live with mindfulness of the body unestablished and their heart restricted. And they don’t truly understand the freedom of heart and freedom by wisdom where those arisen bad, unskillful qualities cease without anything left over. 

When\marginnote{2.4} they hear a sound with their ears … 

When\marginnote{2.5} they smell an odor with their nose … 

When\marginnote{2.6} they taste a flavor with their tongue … 

When\marginnote{2.7} they feel a touch with their body … 

When\marginnote{2.8} they know a thought with their mind, if it’s pleasant they hold on to it, but if it’s unpleasant they dislike it. They live with mindfulness of the body unestablished and a limited heart. And they don’t truly understand the freedom of heart and freedom by wisdom where those arisen bad, unskillful qualities cease without anything left over. 

Suppose\marginnote{3.1} a person was to catch six animals, with diverse territories and feeding grounds, and tie them up with a strong rope. They’d catch a snake, a crocodile, a bird, a dog, a jackal, and a monkey, tie each up with a strong rope, then tie a knot in the middle and let them loose. 

Then\marginnote{3.9} those six animals with diverse domains and territories would each pull towards their own domain and territory. The snake would pull one way, thinking ‘I’m going into an anthill!’ The crocodile would pull another way, thinking ‘I’m going into the water!’ The bird would pull another way, thinking ‘I’m flying into the sky!’ The dog would pull another way, thinking ‘I’m going into the village!’ The jackal would pull another way, thinking ‘I’m going into the charnel ground!’ The monkey would pull another way, thinking ‘I’m going into the jungle!’ When those six animals became exhausted and worn out, the strongest of them would get their way, and they’d all have to submit to their control. 

In\marginnote{3.12} the same way, when a mendicant has not developed or cultivated mindfulness of the body, their eye pulls towards pleasant sights, but is put off by unpleasant sights. Their ear … nose … tongue … body … mind pulls towards pleasant thoughts, but is put off by unpleasant thoughts. 

This\marginnote{3.14} is how someone is unrestrained. 

And\marginnote{4.1} how is someone restrained? 

Take\marginnote{4.2} a mendicant who sees a sight with their eyes. If it’s pleasant they don’t hold on to it, and if it’s unpleasant they don’t dislike it. They live with mindfulness of the body established and a limitless heart. And they truly understand the freedom of heart and freedom by wisdom where those arisen bad, unskillful qualities cease without anything left over. 

They\marginnote{4.4} hear a sound … smell an odor … taste a flavor … feel a touch … know a thought with their mind. If it’s pleasant they don’t hold on to it, and if it’s unpleasant they don’t dislike it. They live with mindfulness of the body established and a limitless heart. And they truly understand the freedom of heart and freedom by wisdom where those arisen bad, unskillful qualities cease without anything left over. 

Suppose\marginnote{5.1} a person was to catch six animals, with diverse territories and feeding grounds, and tie them up with a strong rope. They’d catch a snake, a crocodile, a bird, a dog, a jackal, and a monkey, tie each up with a strong rope, then tether them to a strong post or pillar. 

Then\marginnote{5.9} those six animals with diverse domains and territories would each pull towards their own domain and territory. The snake would pull one way, thinking ‘I’m going into an anthill!’ The crocodile would pull another way, thinking ‘I’m going into the water!’ The bird would pull another way, thinking ‘I’m flying into the sky!’ The dog would pull another way, thinking ‘I’m going into the village!’ The jackal would pull another way, thinking ‘I’m going into the charnel ground!’ The monkey would pull another way, thinking ‘I’m going into the jungle!’ When those six animals became exhausted and worn out, they’d stand or sit or lie down right by that post or pillar. 

In\marginnote{5.12} the same way, when a mendicant has developed and cultivated mindfulness of the body, their eye doesn’t pull towards pleasant sights, and isn’t put off by unpleasant sights. Their ear … nose … tongue … body … mind doesn’t pull towards pleasant thoughts, and isn’t put off by unpleasant thoughts. This is how someone is restrained. 

‘A\marginnote{6.1} strong post or pillar’ is a term for mindfulness of the body. 

So\marginnote{6.2} you should train like this: ‘We will develop mindfulness of the body. We’ll cultivate it, make it our vehicle and our basis, keep it up, consolidate it, and properly implement it.’ That’s how you should train.” 

%
\section*{{\suttatitleacronym SN 35.248}{\suttatitletranslation The Sheaf of Barley }{\suttatitleroot Yavakalāpisutta}}
\addcontentsline{toc}{section}{\tocacronym{SN 35.248} \toctranslation{The Sheaf of Barley } \tocroot{Yavakalāpisutta}}
\markboth{The Sheaf of Barley }{Yavakalāpisutta}
\extramarks{SN 35.248}{SN 35.248}

“Mendicants,\marginnote{1.1} suppose a sheaf of barley was placed at a crossroads. Then six people would come along carrying flails, and started threshing the sheaf of barley. So that sheaf of barley would be thoroughly threshed by those six flails. Then a seventh person would come along carrying a flail, and they’d give the sheaf of barley a seventh threshing. So that sheaf of barley would be even more thoroughly threshed by that seventh flail. 

In\marginnote{1.8} the same way, an unlearned ordinary person is struck in the eye by both pleasant and unpleasant sights. They’re struck in the ear … nose … tongue … body … mind by both pleasant and unpleasant thoughts. And if that unlearned ordinary person has intentions regarding rebirth into a new state of existence in the future, that foolish person is even more thoroughly struck, like that sheaf of barley threshed by the seventh person. 

Once\marginnote{2.1} upon a time, a battle was fought between the gods and the demons. Then Vepacitti, lord of demons, addressed the demons, ‘My good sirs, if the demons defeat the gods in this battle, bind Sakka, the lord of gods, by his limbs and neck and bring him to my presence in the citadel of the demons.’ 

Meanwhile,\marginnote{2.4} Sakka, lord of gods, addressed the gods of the Thirty-Three, ‘My good sirs, if the gods defeat the demons in this battle, bind Vepacitti by his limbs and neck and bring him to my presence in the Sudhamma hall of the gods.’ 

In\marginnote{2.6} that battle the gods won and the demons lost. So the gods of the Thirty-Three bound Vepacitti by his limbs and neck and brought him to Sakka’s presence in the Sudhamma hall of the gods. 

And\marginnote{2.8} there Vepacitti remained bound by his limbs and neck. That is, until he thought, ‘It’s the gods who are principled, while the demons are unprincipled. Now I belong right here in the castle of the gods.’ Then he found himself freed from the bonds on his limbs and neck. He entertained himself, supplied and provided with the five kinds of heavenly sensual stimulation. 

But\marginnote{2.12} when he thought, ‘It’s the demons who are principled, while the gods are unprincipled. Now I will go over there to the citadel of the demons,’ he found himself bound by his limbs and neck, and the five kinds of heavenly sensual stimulation disappeared. 

That’s\marginnote{2.15} how subtly Vepacitti was bound. But the bonds of \textsanskrit{Māra} are even more subtle than that. When you identify, you’re bound by \textsanskrit{Māra}. Not identifying, you’re free from the Wicked One. 

These\marginnote{3.1} are all forms of identifying: ‘I am’, ‘I am this’, ‘I will be’, ‘I will not be’, ‘I will have form’, ‘I will be formless’, ‘I will be percipient’, ‘I will be non-percipient’, ‘I will be neither percipient nor non-percipient.’ Conceit is a disease, a boil, a dart. So mendicants, you should train yourselves like this: ‘We will live with a heart that does not identify.’ 

These\marginnote{4.1} are all disturbances: ‘I am’, ‘I am this’, ‘I will be’, ‘I will not be’, ‘I will have form’, ‘I will be formless’, ‘I will be percipient’, ‘I will be non-percipient’, ‘I will be neither percipient nor non-percipient.’ Disturbances are a disease, a boil, a dart. So mendicants, you should train yourselves like this: ‘We will live with a heart free of disturbances.’ 

These\marginnote{5.1} are all tremblings: ‘I am’, ‘I am this’, ‘I will be’, ‘I will not be’, ‘I will have form’, ‘I will be formless’, ‘I will be percipient’, ‘I will be non-percipient’, ‘I will be neither percipient nor non-percipient.’ Trembling is a disease, a boil, a dart. So mendicants, you should train yourselves like this: ‘We will live with a heart free of tremblings.’ 

These\marginnote{6.1} are all proliferations: ‘I am’, ‘I am this’, ‘I will be’, ‘I will not be’, ‘I will have form’, ‘I will be formless’, ‘I will be percipient’, ‘I will be non-percipient’, ‘I will be neither percipient nor non-percipient.’ Proliferation is a disease, a boil, a dart. So mendicants, you should train yourselves like this: ‘We will live with a heart free of proliferation.’ 

These\marginnote{7.1} are all conceits: ‘I am’, ‘I am this’, ‘I will be’, ‘I will not be’, ‘I will have form’, ‘I will be formless’, ‘I will be percipient’, ‘I will be non-percipient’, ‘I will be neither percipient nor non-percipient.’ Conceit is a disease, a boil, a dart. So mendicants, you should train yourselves like this: ‘We will live with a heart that has struck down conceit.’” 

\scendsutta{The Linked Discourses on the six sense fields are complete. }

%
\addtocontents{toc}{\let\protect\contentsline\protect\nopagecontentsline}
\part*{Linked Discourses on Feelings }
\addcontentsline{toc}{part}{Linked Discourses on Feelings }
\markboth{}{}
\addtocontents{toc}{\let\protect\contentsline\protect\oldcontentsline}

%
\addtocontents{toc}{\let\protect\contentsline\protect\nopagecontentsline}
\chapter*{The Chapter with Verses }
\addcontentsline{toc}{chapter}{\tocchapterline{The Chapter with Verses }}
\addtocontents{toc}{\let\protect\contentsline\protect\oldcontentsline}

%
\section*{{\suttatitleacronym SN 36.1}{\suttatitletranslation Immersion }{\suttatitleroot Samādhisutta}}
\addcontentsline{toc}{section}{\tocacronym{SN 36.1} \toctranslation{Immersion } \tocroot{Samādhisutta}}
\markboth{Immersion }{Samādhisutta}
\extramarks{SN 36.1}{SN 36.1}

“Mendicants,\marginnote{1.1} there are these three feelings. What three? 

Pleasant,\marginnote{1.3} painful, and neutral feeling. These are the three feelings. 

\begin{verse}%
Stilled,\marginnote{2.1} aware, \\
a mindful disciple of the Buddha \\
understands feelings, \\
the cause of feelings, 

where\marginnote{3.1} they cease, \\
and the path that leads to their ending. \\
With the ending of feelings, a mendicant \\
is hungerless, extinguished.” 

%
\end{verse}

%
\section*{{\suttatitleacronym SN 36.2}{\suttatitletranslation Pleasure }{\suttatitleroot Sukhasutta}}
\addcontentsline{toc}{section}{\tocacronym{SN 36.2} \toctranslation{Pleasure } \tocroot{Sukhasutta}}
\markboth{Pleasure }{Sukhasutta}
\extramarks{SN 36.2}{SN 36.2}

“Mendicants,\marginnote{1.1} there are these three feelings. What three? 

Pleasant,\marginnote{1.3} painful, and neutral feeling. These are the three feelings. 

\begin{verse}%
Whatever\marginnote{2.1} is felt \\
internally and externally—\\
whether pleasure or pain \\
as well as what’s neutral—

having\marginnote{3.1} known this as suffering, \\
deceptive, falling apart, \\
one sees them vanish as they’re experienced again and again: \\
that’s how to be free of desire for them.” 

%
\end{verse}

%
\section*{{\suttatitleacronym SN 36.3}{\suttatitletranslation Giving Up }{\suttatitleroot Pahānasutta}}
\addcontentsline{toc}{section}{\tocacronym{SN 36.3} \toctranslation{Giving Up } \tocroot{Pahānasutta}}
\markboth{Giving Up }{Pahānasutta}
\extramarks{SN 36.3}{SN 36.3}

“Mendicants,\marginnote{1.1} there are these three feelings. What three? 

Pleasant,\marginnote{1.3} painful, and neutral feeling. 

The\marginnote{1.4} underlying tendency to greed should be given up when it comes to pleasant feeling. The underlying tendency to repulsion should be given up when it comes to painful feeling. The underlying tendency to ignorance should be given up when it comes to neutral feeling. 

When\marginnote{1.5} a mendicant has given up these underlying tendencies, they’re called a mendicant without underlying tendencies, who sees rightly, has cut off craving, untied the fetters, and by rightly comprehending conceit has made an end of suffering. 

\begin{verse}%
When\marginnote{2.1} you feel pleasure \\
without understanding feeling, \\
the underlying tendency to greed is there, \\
if you don’t see the escape. 

When\marginnote{3.1} you feel pain \\
without understanding feeling, \\
the underlying tendency to repulsion is there, \\
if you don’t see the escape. 

As\marginnote{4.1} for that peaceful, neutral feeling: \\
he of vast wisdom has taught \\
that if you relish it, \\
you’re still not released from suffering. 

But\marginnote{5.1} when a mendicant is keen, \\
not neglecting situational awareness, \\
that astute person \\
understands all feelings. 

Completely\marginnote{6.1} understanding feelings, \\
they’re without defilements in this very life. \\
That knowledge master is firm in principle; \\
when their body breaks up, they can’t be reckoned.” 

%
\end{verse}

%
\section*{{\suttatitleacronym SN 36.4}{\suttatitletranslation The Abyss }{\suttatitleroot Pātālasutta}}
\addcontentsline{toc}{section}{\tocacronym{SN 36.4} \toctranslation{The Abyss } \tocroot{Pātālasutta}}
\markboth{The Abyss }{Pātālasutta}
\extramarks{SN 36.4}{SN 36.4}

“Mendicants,\marginnote{1.1} when an unlearned ordinary person says that there’s a hellish abyss under the ocean, they’re speaking of something that doesn’t exist. 

‘Hellish\marginnote{1.5} abyss’ is a term for painful physical feelings. 

When\marginnote{1.6} an unlearned ordinary person experiences painful physical feelings they sorrow and wail and lament, beating their breast and falling into confusion. They’re called an unlearned ordinary person who hasn’t stood up in the hellish abyss and has gained no footing. 

When\marginnote{1.8} a learned noble disciple experiences painful physical feelings they don’t sorrow or wail or lament, beating their breast and falling into confusion. They’re called a learned noble disciple who has stood up in the hellish abyss and gained a footing. 

\begin{verse}%
If\marginnote{2.1} you can’t abide \\
those painful physical feelings \\
that arise and sap your vitality; \\
if you tremble at their touch, 

weeping\marginnote{3.1} and wailing, \\
a weakling lacking strength—\\
you won’t stand up in the hellish abyss \\
and gain a footing. 

If\marginnote{4.1} you can endure \\
those painful physical feelings \\
that arise and sap your vitality; \\
if you don’t tremble at their touch—\\
you stand up in the hellish abyss \\
and gain a footing.” 

%
\end{verse}

%
\section*{{\suttatitleacronym SN 36.5}{\suttatitletranslation Should Be Seen }{\suttatitleroot Daṭṭhabbasutta}}
\addcontentsline{toc}{section}{\tocacronym{SN 36.5} \toctranslation{Should Be Seen } \tocroot{Daṭṭhabbasutta}}
\markboth{Should Be Seen }{Daṭṭhabbasutta}
\extramarks{SN 36.5}{SN 36.5}

“Mendicants,\marginnote{1.1} there are these three feelings. What three? 

Pleasant,\marginnote{1.3} painful, and neutral feeling. 

Pleasant\marginnote{1.4} feeling should be seen as suffering. Painful feeling should be seen as a dart. Neutral feeling should be seen as impermanent. 

When\marginnote{1.5} a mendicant has seen these three feelings in this way, they’re called a mendicant who has cut off craving, untied the fetters, and by rightly comprehending conceit has made an end of suffering. 

\begin{verse}%
A\marginnote{2.1} mendicant who sees pleasure as pain, \\
and suffering as a dart, \\
and that peaceful, neutral feeling \\
as impermanent 

sees\marginnote{3.1} rightly; \\
they completely understand feelings. \\
Completely understanding feelings, \\
they’re without defilements in this very life. \\
That knowledge master is firm in principle; \\
when their body breaks up, they can’t be reckoned.” 

%
\end{verse}

%
\section*{{\suttatitleacronym SN 36.6}{\suttatitletranslation An Arrow }{\suttatitleroot Sallasutta}}
\addcontentsline{toc}{section}{\tocacronym{SN 36.6} \toctranslation{An Arrow } \tocroot{Sallasutta}}
\markboth{An Arrow }{Sallasutta}
\extramarks{SN 36.6}{SN 36.6}

“Mendicants,\marginnote{1.1} an unlearned ordinary person feels pleasant, painful, and neutral feelings. A learned noble disciple also feels pleasant, painful, and neutral feelings. What, then, is the difference between an ordinary unlearned person and a learned noble disciple?” 

“Our\marginnote{1.4} teachings are rooted in the Buddha. …” 

“When\marginnote{1.5} an unlearned ordinary person experiences painful physical feelings they sorrow and wail and lament, beating their breast and falling into confusion. They experience two feelings: physical and mental. 

It’s\marginnote{1.8} like a person who is struck with an arrow, only to be struck with a second arrow. That person experiences the feeling of two arrows. 

In\marginnote{1.11} the same way, when an unlearned ordinary person experiences painful physical feelings they sorrow and wail and lament, beating their breast and falling into confusion. They experience two feelings: physical and mental. 

When\marginnote{1.14} they’re touched by painful feeling, they resist it. The underlying tendency for repulsion towards painful feeling underlies that. 

When\marginnote{1.16} touched by painful feeling they look forward to enjoying sensual pleasures. Why is that? Because an unlearned ordinary person doesn’t understand any escape from painful feeling apart from sensual pleasures. Since they look forward to enjoying sensual pleasures, the underlying tendency to greed for pleasant feeling underlies that. 

They\marginnote{1.20} don’t truly understand feelings’ origin, ending, gratification, drawback, and escape. The underlying tendency to ignorance about neutral feeling underlies that. 

If\marginnote{1.22} they feel a pleasant feeling, they feel it attached. If they feel a painful feeling, they feel it attached. If they feel a neutral feeling, they feel it attached. 

They’re\marginnote{1.25} called an unlearned ordinary person who is attached to rebirth, old age, and death, to sorrow, lamentation, pain, sadness, and distress, I say. 

When\marginnote{2.1} a learned noble disciple experiences painful physical feelings they don’t sorrow or wail or lament, beating their breast and falling into confusion. They experience one feeling: physical, not mental. 

It’s\marginnote{3.1} like a person who is struck with an arrow, but was not struck with a second arrow. That person would experience the feeling of one arrow. 

In\marginnote{3.4} the same way, when a learned noble disciple experiences painful physical feelings they don’t sorrow or wail or lament, beating their breast and falling into confusion. They experience one feeling: physical, not mental. 

When\marginnote{3.7} they’re touched by painful feeling, they don’t resist it. There’s no underlying tendency for repulsion towards painful feeling underlying that. 

When\marginnote{3.9} touched by painful feeling they don’t look forward to enjoying sensual pleasures. Why is that? Because a learned noble disciple understands an escape from painful feeling apart from sensual pleasures. Since they don’t look forward to enjoying sensual pleasures, there’s no underlying tendency to greed for pleasant feeling underlying that. 

They\marginnote{3.13} truly understand feelings’ origin, ending, gratification, drawback, and escape. There’s no underlying tendency to ignorance about neutral feeling underlying that. 

If\marginnote{3.15} they feel a pleasant feeling, they feel it detached. If they feel a painful feeling, they feel it detached. If they feel a neutral feeling, they feel it detached. 

They’re\marginnote{3.18} called a learned noble disciple who is detached from rebirth, old age, and death, from sorrow, lamentation, pain, sadness, and distress, I say. 

This\marginnote{3.19} is the difference between a learned noble disciple and an unlearned ordinary person. 

\begin{verse}%
A\marginnote{4.1} wise and learned person isn’t affected \\
by feelings of pleasure and pain. \\
This is the great difference in skill \\
between the wise and the ordinary. 

A\marginnote{5.1} learned person who has assessed the teaching \\
discerns this world and the next. \\
Desirable things don’t disturb their mind, \\
nor are they repelled by the undesirable. 

Both\marginnote{6.1} favoring and opposing \\
are cleared and ended, they are no more. \\
Knowing the stainless, sorrowless state, \\
they who have gone beyond rebirth understand rightly.” 

%
\end{verse}

%
\section*{{\suttatitleacronym SN 36.7}{\suttatitletranslation The Infirmary (1st) }{\suttatitleroot Paṭhamagelaññasutta}}
\addcontentsline{toc}{section}{\tocacronym{SN 36.7} \toctranslation{The Infirmary (1st) } \tocroot{Paṭhamagelaññasutta}}
\markboth{The Infirmary (1st) }{Paṭhamagelaññasutta}
\extramarks{SN 36.7}{SN 36.7}

At\marginnote{1.1} one time the Buddha was staying near \textsanskrit{Vesālī}, at the Great Wood, in the hall with the peaked roof. 

Then\marginnote{1.2} in the late afternoon, the Buddha came out of retreat and went to the infirmary, where he sat down on the seat spread out, and addressed the mendicants: 

“Mendicants,\marginnote{2.1} a mendicant should await their time mindful and aware. This is my instruction to you. 

And\marginnote{3.1} how is a mendicant mindful? It’s when a mendicant meditates by observing an aspect of the body—keen, aware, and mindful, rid of desire and aversion for the world. They meditate observing an aspect of feelings … They meditate observing an aspect of the mind … They meditate observing an aspect of principles—keen, aware, and mindful, rid of desire and aversion for the world. That’s how a mendicant is mindful. 

And\marginnote{4.1} how is a mendicant aware? It’s when a mendicant acts with situational awareness when going out and coming back; when looking ahead and aside; when bending and extending the limbs; when bearing the outer robe, bowl and robes; when eating, drinking, chewing, and tasting; when urinating and defecating; when walking, standing, sitting, sleeping, waking, speaking, and keeping silent. That’s how a mendicant acts with situational awareness. A mendicant should await their time mindful and aware. This is my instruction to you. 

While\marginnote{5.1} a mendicant is meditating like this—mindful, aware, diligent, keen, and resolute—if pleasant feelings arise, they understand: ‘A pleasant feeling has arisen in me. That’s dependent, not independent. Dependent on what? Dependent on my own body. But this body is impermanent, conditioned, dependently originated. So how could a pleasant feeling be permanent, since it has arisen dependent on a body that is impermanent, conditioned, and dependently originated?’ They meditate observing impermanence, vanishing, dispassion, cessation, and letting go in the body and pleasant feeling. As they do so, they give up the underlying tendency for greed for the body and pleasant feeling. 

While\marginnote{6.1} a mendicant is meditating like this—mindful, aware, diligent, keen, and resolute—if painful feelings arise, they understand: ‘A painful feeling has arisen in me. That’s dependent, not independent. Dependent on what? Dependent on my own body. But this body is impermanent, conditioned, dependently originated. So how could a painful feeling be permanent, since it has arisen dependent on a body that is impermanent, conditioned, and dependently originated?’ They meditate observing impermanence, vanishing, dispassion, cessation, and letting go in the body and painful feeling. As they do so, they give up the underlying tendency for repulsion towards the body and painful feeling. 

While\marginnote{7.1} a mendicant is meditating like this—mindful, aware, diligent, keen, and resolute—if neutral feelings arise, they understand: ‘A neutral feeling has arisen in me. That’s dependent, not independent. Dependent on what? Dependent on my own body. But this body is impermanent, conditioned, dependently originated. So how could a neutral feeling be permanent, since it has arisen dependent on a body that is impermanent, conditioned, and dependently originated?’ They meditate observing impermanence, vanishing, dispassion, cessation, and letting go in the body and neutral feeling. As they do so, they give up the underlying tendency for ignorance towards the body and neutral feeling. 

If\marginnote{8.1} they feel a pleasant feeling, they understand that it’s impermanent, that they’re not attached to it, and that they don’t take pleasure in it. If they feel a painful feeling, they understand that it’s impermanent, that they’re not attached to it, and that they don’t take pleasure in it. If they feel a neutral feeling, they understand that it’s impermanent, that they’re not attached to it, and that they don’t take pleasure in it. 

If\marginnote{8.4} they feel a pleasant feeling, they feel it detached. If they feel a painful feeling, they feel it detached. If they feel a neutral feeling, they feel it detached. 

Feeling\marginnote{8.7} the end of the body approaching, they understand: ‘I feel the end of the body approaching.’ Feeling the end of life approaching, they understand: ‘I feel the end of life approaching.’ They understand: ‘When my body breaks up and my life has come to an end, everything that’s felt, since I no longer take pleasure in it, will become cool right here.’ 

Suppose\marginnote{9.1} an oil lamp depended on oil and a wick to burn. As the oil and the wick are used up, it would be extinguished due to lack of fuel. 

In\marginnote{9.3} the same way, feeling the end of the body approaching, a mendicant understands: ‘I feel the end of the body approaching.’ Feeling the end of life approaching, a mendicant understands: ‘I feel the end of life approaching.’ They understand: ‘When my body breaks up and my life is over, everything that’s felt, since I no longer take pleasure in it, will become cool right here.’” 

%
\section*{{\suttatitleacronym SN 36.8}{\suttatitletranslation The Infirmary (2nd) }{\suttatitleroot Dutiyagelaññasutta}}
\addcontentsline{toc}{section}{\tocacronym{SN 36.8} \toctranslation{The Infirmary (2nd) } \tocroot{Dutiyagelaññasutta}}
\markboth{The Infirmary (2nd) }{Dutiyagelaññasutta}
\extramarks{SN 36.8}{SN 36.8}

At\marginnote{1.1} one time the Buddha was staying near \textsanskrit{Vesālī}, at the Great Wood, in the hall with the peaked roof. 

Then\marginnote{1.2} in the late afternoon, the Buddha came out of retreat and went to the infirmary, where he sat down on the seat spread out, and addressed the mendicants: 

“Mendicants,\marginnote{2.1} a mendicant should await their time mindful and aware. This is my instruction to you. 

And\marginnote{3.1} how is a mendicant mindful? It’s when a mendicant meditates by observing an aspect of the body—keen, aware, and mindful, rid of desire and aversion for the world. They meditate observing an aspect of feelings … They meditate observing an aspect of the mind … They meditate observing an aspect of principles—keen, aware, and mindful, rid of desire and aversion for the world. That’s how a mendicant is mindful. 

And\marginnote{4.1} how is a mendicant aware? It’s when a mendicant acts with situational awareness when going out and coming back; when looking ahead and aside; when bending and extending the limbs; when bearing the outer robe, bowl and robes; when eating, drinking, chewing, and tasting; when urinating and defecating; when walking, standing, sitting, sleeping, waking, speaking, and keeping silent. That’s how a mendicant is aware. 

A\marginnote{4.4} mendicant should await their time mindful and aware. This is my instruction to you. 

While\marginnote{5.1} a mendicant is meditating like this—mindful, aware, diligent, keen, and resolute—if pleasant feelings arise, they understand: ‘A pleasant feeling has arisen in me. That’s dependent, not independent. Dependent on what? Dependent on this very contact. But this contact is impermanent, conditioned, dependently originated. So how could a pleasant feeling be permanent, since it has arisen dependent on contact that is impermanent, conditioned, and dependently originated?’ They meditate observing impermanence, vanishing, dispassion, cessation, and letting go in contact and pleasant feeling. As they do so, they give up the underlying tendency for greed for contact and pleasant feeling. 

While\marginnote{6.1} a mendicant is meditating like this—mindful, aware, diligent, keen, and resolute—if painful feelings arise … if neutral feelings arise, they understand: ‘A neutral feeling has arisen in me. That’s dependent, not independent. Dependent on what? Dependent on this very contact. 

(Expand\marginnote{6.9} in detail as in the previous discourse.) 

They\marginnote{6.10} understand: ‘When my body breaks up and my life is over, everything that’s felt, since I no longer take pleasure in it, will become cool right here.’ 

Suppose\marginnote{7.1} an oil lamp depended on oil and a wick to burn. As the oil and the wick are used up, it would be extinguished due to lack of fuel. 

In\marginnote{7.3} the same way, feeling the end of the body approaching, a mendicant understands: ‘I feel the end of the body approaching.’ Feeling the end of life approaching, they understand: ‘I feel the end of life approaching.’ They understand: ‘When my body breaks up and my life is over, everything that’s felt, since I no longer take pleasure in it, will become cool right here.’” 

%
\section*{{\suttatitleacronym SN 36.9}{\suttatitletranslation Impermanent }{\suttatitleroot Aniccasutta}}
\addcontentsline{toc}{section}{\tocacronym{SN 36.9} \toctranslation{Impermanent } \tocroot{Aniccasutta}}
\markboth{Impermanent }{Aniccasutta}
\extramarks{SN 36.9}{SN 36.9}

“Mendicants,\marginnote{1.1} these three feelings are impermanent, conditioned, dependently originated, liable to end, vanish, fade away, and cease. What three? 

Pleasant,\marginnote{1.3} painful, and neutral feeling. These are the three feelings that are impermanent, conditioned, dependently originated, liable to end, vanish, fade away, and cease.” 

%
\section*{{\suttatitleacronym SN 36.10}{\suttatitletranslation Rooted in Contact }{\suttatitleroot Phassamūlakasutta}}
\addcontentsline{toc}{section}{\tocacronym{SN 36.10} \toctranslation{Rooted in Contact } \tocroot{Phassamūlakasutta}}
\markboth{Rooted in Contact }{Phassamūlakasutta}
\extramarks{SN 36.10}{SN 36.10}

“Mendicants,\marginnote{1.1} these three feelings are born, rooted, sourced, and conditioned by contact. What three? 

Pleasant,\marginnote{1.3} painful, and neutral feeling. 

Pleasant\marginnote{1.4} feeling arises dependent on a contact to be experienced as pleasant. With the cessation of that contact to be experienced as pleasant, the corresponding pleasant feeling ceases and stops. Painful feeling arises dependent on a contact to be experienced as painful. With the cessation of that contact to be experienced as painful, the corresponding painful feeling ceases and stops. Neutral feeling arises dependent on a contact to be experienced as neutral. With the cessation of that contact to be experienced as neutral, the corresponding neutral feeling ceases and stops. 

When\marginnote{1.10} you rub two sticks together, heat is generated and fire is produced. But when you part the sticks and lay them aside, any corresponding heat ceases and stops. 

In\marginnote{1.11} the same way, these three feelings are born, rooted, sourced, and conditioned by contact. The appropriate feeling arises dependent on the corresponding contact. When the corresponding contact ceases, the appropriate feeling ceases.” 

%
\addtocontents{toc}{\let\protect\contentsline\protect\nopagecontentsline}
\chapter*{The Chapter on In Private }
\addcontentsline{toc}{chapter}{\tocchapterline{The Chapter on In Private }}
\addtocontents{toc}{\let\protect\contentsline\protect\oldcontentsline}

%
\section*{{\suttatitleacronym SN 36.11}{\suttatitletranslation In Private }{\suttatitleroot Rahogatasutta}}
\addcontentsline{toc}{section}{\tocacronym{SN 36.11} \toctranslation{In Private } \tocroot{Rahogatasutta}}
\markboth{In Private }{Rahogatasutta}
\extramarks{SN 36.11}{SN 36.11}

Then\marginnote{1.1} a mendicant went up to the Buddha, bowed, sat down to one side, and said to him: 

“Just\marginnote{1.2} now, sir, as I was in private retreat this thought came to mind. The Buddha has spoken of three feelings. Pleasant, painful, and neutral feeling. These are the three feelings the Buddha has spoken of. 

But\marginnote{1.6} the Buddha has also said: ‘Suffering includes whatever is felt.’ What was the Buddha referring to when he said this?” 

“Good,\marginnote{2.1} good, mendicant! I have spoken of these three feelings. Pleasant, painful, and neutral feeling. These are the three feelings I have spoken of. 

But\marginnote{2.5} I have also said: ‘Suffering includes whatever is felt.’ 

When\marginnote{2.7} I said this I was referring to the impermanence of conditions, to the fact that conditions are liable to end, vanish, fade away, cease, and perish. 

But\marginnote{2.15} I have also explained the progressive cessation of conditions. For someone who has attained the first absorption, speech has ceased. For someone who has attained the second absorption, the placing of the mind and keeping it connected have ceased. For someone who has attained the third absorption, rapture has ceased. For someone who has attained the fourth absorption, breathing has ceased. For someone who has attained the dimension of infinite space, the perception of form has ceased. For someone who has attained the dimension of infinite consciousness, the perception of the dimension of infinite space has ceased. For someone who has attained the dimension of nothingness, the perception of the dimension of infinite consciousness has ceased. For someone who has attained the dimension of neither perception nor non-perception, the perception of the dimension of nothingness has ceased. For someone who has attained the cessation of perception and feeling, perception and feeling have ceased. For a mendicant who has ended the defilements, greed, hate, and delusion have ceased. 

And\marginnote{2.26} I have also explained the progressive stilling of conditions. For someone who has attained the first absorption, speech has stilled. For someone who has attained the second absorption, the placing of the mind and keeping it connected have stilled. … For someone who has attained the cessation of perception and feeling, perception and feeling have stilled. For a mendicant who has ended the defilements, greed, hate, and delusion have stilled. 

There\marginnote{2.31} are these six levels of tranquility. For someone who has attained the first absorption, speech has been tranquilized. For someone who has attained the second absorption, the placing of the mind and keeping it connected have been tranquilized. For someone who has attained the third absorption, rapture has been tranquilized. For someone who has attained the fourth absorption, breathing has been tranquilized. For someone who has attained the cessation of perception and feeling, perception and feeling have been tranquilized. For a mendicant who has ended the defilements, greed, hate, and delusion have been tranquilized.” 

%
\section*{{\suttatitleacronym SN 36.12}{\suttatitletranslation In the Sky (1st) }{\suttatitleroot Paṭhamaākāsasutta}}
\addcontentsline{toc}{section}{\tocacronym{SN 36.12} \toctranslation{In the Sky (1st) } \tocroot{Paṭhamaākāsasutta}}
\markboth{In the Sky (1st) }{Paṭhamaākāsasutta}
\extramarks{SN 36.12}{SN 36.12}

“Mendicants,\marginnote{1.1} various winds blow in the sky. Winds blow from the east, the west, the north, and the south. There are winds that are dusty and dustless, cool and warm, weak and strong. 

In\marginnote{1.3} the same way, various feelings arise in this body: pleasant, painful, and neutral feelings. 

\begin{verse}%
There\marginnote{2.1} are many and various \\
winds that blow in the sky. \\
From the east they come, also the west, \\
the north, and then the south. 

They\marginnote{3.1} are dusty and dustless, \\
cool and sometimes warm, \\
strong and weak; \\
these are the different breezes that blow. 

So\marginnote{4.1} too, in this body \\
feelings arise, \\
pleasant and painful, \\
and those that are neutral. 

But\marginnote{5.1} when a mendicant is keen, \\
not neglecting situational awareness, \\
that astute person \\
understands all feelings. 

Completely\marginnote{6.1} understanding feelings, \\
they’re without defilements in this very life. \\
That knowledge master is firm in principle; \\
when their body breaks up, they can’t be reckoned.” 

%
\end{verse}

%
\section*{{\suttatitleacronym SN 36.13}{\suttatitletranslation In the Sky (2nd) }{\suttatitleroot Dutiyaākāsasutta}}
\addcontentsline{toc}{section}{\tocacronym{SN 36.13} \toctranslation{In the Sky (2nd) } \tocroot{Dutiyaākāsasutta}}
\markboth{In the Sky (2nd) }{Dutiyaākāsasutta}
\extramarks{SN 36.13}{SN 36.13}

“Mendicants,\marginnote{1.1} various winds blow in the sky. Winds blow from the east, the west, the north, and the south. There are winds that are dusty and dustless, cool and warm, weak and strong. 

In\marginnote{1.3} the same way, various feelings arise in this body: pleasant, painful, and neutral feelings.” 

%
\section*{{\suttatitleacronym SN 36.14}{\suttatitletranslation A Guest House }{\suttatitleroot Agārasutta}}
\addcontentsline{toc}{section}{\tocacronym{SN 36.14} \toctranslation{A Guest House } \tocroot{Agārasutta}}
\markboth{A Guest House }{Agārasutta}
\extramarks{SN 36.14}{SN 36.14}

“Mendicants,\marginnote{1.1} suppose there was a guest house. Lodgers come from the east, west, north, and south. Aristocrats, brahmins, merchants, and workers all stay there. 

In\marginnote{1.2} the same way, various feelings arise in this body: pleasant, painful, and neutral feelings. Also material pleasant, painful, and neutral feelings arise. Also spiritual pleasant, painful, and neutral feelings arise.” 

%
\section*{{\suttatitleacronym SN 36.15}{\suttatitletranslation With Ānanda (1st) }{\suttatitleroot Paṭhamaānandasutta}}
\addcontentsline{toc}{section}{\tocacronym{SN 36.15} \toctranslation{With Ānanda (1st) } \tocroot{Paṭhamaānandasutta}}
\markboth{With Ānanda (1st) }{Paṭhamaānandasutta}
\extramarks{SN 36.15}{SN 36.15}

Then\marginnote{1.1} Venerable Ānanda went up to the Buddha … sat down to one side, and said to him: 

“Sir,\marginnote{1.2} what is feeling? What’s the origin of feeling? What’s the cessation of feeling? What’s the practice that leads to the cessation of feeling? And what is feeling’s gratification, drawback, and escape?” 

“Ānanda,\marginnote{1.4} there are these three feelings: pleasant, painful, and neutral. These are called feeling. 

Feeling\marginnote{1.7} originates from contact. When contact ceases, feeling ceases. 

The\marginnote{1.9} practice that leads to the cessation of feelings is simply this noble eightfold path, that is: right view, right thought, right speech, right action, right livelihood, right effort, right mindfulness, and right immersion. 

The\marginnote{1.11} pleasure and happiness that arise from feeling: this is its gratification. 

That\marginnote{1.12} feeling is impermanent, suffering, and perishable: this is its drawback. 

Removing\marginnote{1.13} and giving up desire and greed for feeling: this is its escape. 

But\marginnote{1.14} I have also explained the progressive cessation of conditions. For someone who has attained the first absorption, speech has ceased. … For someone who has attained the cessation of perception and feeling, perception and feeling have ceased. For a mendicant who has ended the defilements, greed, hate, and delusion have ceased. 

And\marginnote{1.18} I have also explained the progressive stilling of conditions. For someone who has attained the first absorption, speech has stilled. … For someone who has attained the cessation of perception and feeling, perception and feeling have stilled. For a mendicant who has ended the defilements, greed, hate, and delusion have stilled. 

And\marginnote{1.22} I have also explained the progressive tranquilizing of conditions. For someone who has attained the first absorption, speech has been tranquilized. … For someone who has attained the dimension of infinite space, the perception of form has been tranquilized. For someone who has attained the dimension of infinite consciousness, the perception of the dimension of infinite space has been tranquilized. For someone who has attained the dimension of nothingness, the perception of the dimension of infinite consciousness has been tranquilized. For someone who has attained the dimension of neither perception nor non-perception, the perception of the dimension of nothingness has been tranquilized. For someone who has attained the cessation of perception and feeling, perception and feeling have been tranquilized. For a mendicant who has ended the defilements, greed, hate, and delusion have been tranquilized.” 

%
\section*{{\suttatitleacronym SN 36.16}{\suttatitletranslation With Ānanda (2nd) }{\suttatitleroot Dutiyaānandasutta}}
\addcontentsline{toc}{section}{\tocacronym{SN 36.16} \toctranslation{With Ānanda (2nd) } \tocroot{Dutiyaānandasutta}}
\markboth{With Ānanda (2nd) }{Dutiyaānandasutta}
\extramarks{SN 36.16}{SN 36.16}

Then\marginnote{1.1} Venerable Ānanda went up to the Buddha, bowed, and sat down to one side. The Buddha said to him, “Ānanda, what is feeling? What’s the origin of feeling? What’s the cessation of feeling? What’s the practice that leads to the cessation of feeling? And what is feeling’s gratification, drawback, and escape?” 

“Our\marginnote{1.4} teachings are rooted in the Buddha. He is our guide and our refuge. Sir, may the Buddha himself please clarify the meaning of this. The mendicants will listen and remember it.” 

“Well\marginnote{1.5} then, Ānanda, listen and pay close attention, I will speak.” 

“Yes,\marginnote{1.6} sir,” Ānanda replied. The Buddha said this: 

“Ānanda,\marginnote{1.8} there are these three feelings: pleasant, painful, and neutral. These are called feeling. … 

For\marginnote{1.12} a mendicant who has ended the defilements, greed, hate, and delusion have been tranquilized.” 

%
\section*{{\suttatitleacronym SN 36.17}{\suttatitletranslation With Several Mendicants (1st) }{\suttatitleroot Paṭhamasambahulasutta}}
\addcontentsline{toc}{section}{\tocacronym{SN 36.17} \toctranslation{With Several Mendicants (1st) } \tocroot{Paṭhamasambahulasutta}}
\markboth{With Several Mendicants (1st) }{Paṭhamasambahulasutta}
\extramarks{SN 36.17}{SN 36.17}

Then\marginnote{1.1} several mendicants went up to the Buddha, bowed, sat down to one side, and said to him: 

“Sir,\marginnote{1.2} what is feeling? What’s the origin of feeling? What’s the cessation of feeling? What’s the practice that leads to the cessation of feeling? And what is feeling’s gratification, drawback, and escape?” 

“Mendicants,\marginnote{1.4} there are these three feelings: pleasant, painful, and neutral. These are called feeling. 

Feeling\marginnote{1.7} originates from contact. When contact ceases, feeling ceases. 

The\marginnote{1.9} practice that leads to the cessation of feelings is simply this noble eightfold path, that is: right view, right thought, right speech, right action, right livelihood, right effort, right mindfulness, and right immersion. 

The\marginnote{1.11} pleasure and happiness that arise from feeling: this is its gratification. That feeling is impermanent, suffering, and perishable: this is its drawback. Removing and giving up desire and greed for feeling: this is its escape. 

But\marginnote{2.1} I have also explained the progressive cessation of conditions. … 

For\marginnote{2.13} a mendicant who has ended the defilements, greed, hate, and delusion have been tranquilized.” 

%
\section*{{\suttatitleacronym SN 36.18}{\suttatitletranslation With Several Mendicants (2nd) }{\suttatitleroot Dutiyasambahulasutta}}
\addcontentsline{toc}{section}{\tocacronym{SN 36.18} \toctranslation{With Several Mendicants (2nd) } \tocroot{Dutiyasambahulasutta}}
\markboth{With Several Mendicants (2nd) }{Dutiyasambahulasutta}
\extramarks{SN 36.18}{SN 36.18}

Then\marginnote{1.1} several mendicants went up to the Buddha … The Buddha said to them: 

“Mendicants,\marginnote{1.3} what is feeling? What’s the origin of feeling? What’s the cessation of feeling? What’s the practice that leads to the cessation of feeling? And what is feeling’s gratification, drawback, and escape?” 

“Our\marginnote{1.5} teachings are rooted in the Buddha. …” 

“Mendicants,\marginnote{1.6} there are these three feelings: pleasant, painful, and neutral. These are called feeling. …” 

(This\marginnote{1.9} should be told in full as in the previous discourse.) 

%
\section*{{\suttatitleacronym SN 36.19}{\suttatitletranslation With Pañcakaṅga }{\suttatitleroot Pañcakaṅgasutta}}
\addcontentsline{toc}{section}{\tocacronym{SN 36.19} \toctranslation{With Pañcakaṅga } \tocroot{Pañcakaṅgasutta}}
\markboth{With Pañcakaṅga }{Pañcakaṅgasutta}
\extramarks{SN 36.19}{SN 36.19}

Then\marginnote{1.1} the master builder \textsanskrit{Pañcakaṅga} went up to Venerable \textsanskrit{Udāyī}, bowed, sat down to one side, and asked him, “Sir, how many feelings has the Buddha spoken of?” 

“Master\marginnote{1.3} builder, the Buddha has spoken of three feelings: pleasant, painful, and neutral. The Buddha has spoken of these three feelings.” 

When\marginnote{1.6} he said this, \textsanskrit{Pañcakaṅga} said to \textsanskrit{Udāyī}, “Sir, \textsanskrit{Udāyī}, the Buddha hasn’t spoken of three feelings. He’s spoken of two feelings: pleasant and painful. The Buddha said that neutral feeling is included as a peaceful and subtle kind of pleasure.” 

For\marginnote{2.1} a second time, \textsanskrit{Udāyī} said to him, “The Buddha hasn’t spoken of two feelings, he’s spoken of three.” 

For\marginnote{2.6} a second time, \textsanskrit{Pañcakaṅga} said to \textsanskrit{Udāyī}, “The Buddha hasn’t spoken of three feelings, he’s spoken of two.” 

And\marginnote{3.1} for a third time, \textsanskrit{Udāyī} said to him, “The Buddha hasn’t spoken of two feelings, he’s spoken of three.” 

And\marginnote{3.6} for a third time, \textsanskrit{Pañcakaṅga} said to \textsanskrit{Udāyī}, “The Buddha hasn’t spoken of three feelings, he’s spoken of two.” 

But\marginnote{3.11} neither was able to persuade the other. 

Venerable\marginnote{3.12} Ānanda heard this discussion between \textsanskrit{Udāyī} and \textsanskrit{Pañcakaṅga}. He went to the Buddha, bowed, sat down to one side, and informed the Buddha of all they had discussed. 

“Ānanda,\marginnote{5.1} the explanation by the mendicant \textsanskrit{Udāyī}, which the master builder \textsanskrit{Pañcakaṅga} didn’t agree with, was quite correct. But the explanation by \textsanskrit{Pañcakaṅga}, which \textsanskrit{Udāyī} didn’t agree with, was also quite correct. 

In\marginnote{5.3} one explanation I’ve spoken of two feelings. In another explanation I’ve spoken of three feelings, or five, six, eighteen, thirty-six, or a hundred and eight feelings. 

I’ve\marginnote{5.10} explained the teaching in all these different ways. This being so, you can expect that those who don’t concede, approve, or agree with what has been well spoken will argue, quarrel, and dispute, continually wounding each other with barbed words. 

I’ve\marginnote{5.12} explained the teaching in all these different ways. This being so, you can expect that those who do concede, approve, or agree with what has been well spoken will live in harmony, appreciating each other, without quarreling, blending like milk and water, and regarding each other with kindly eyes. 

There\marginnote{6.1} are these five kinds of sensual stimulation. What five? Sights known by the eye that are likable, desirable, agreeable, pleasant, sensual, and arousing. … Touches known by the body that are likable, desirable, agreeable, pleasant, sensual, and arousing. These are the five kinds of sensual stimulation. The pleasure and happiness that arise from these five kinds of sensual stimulation is called sensual pleasure. There are those who would say that this is the highest pleasure and happiness that sentient beings experience. But I don’t acknowledge that. Why is that? Because there is another pleasure that is finer than that. 

And\marginnote{7.1} what is that pleasure? It’s when a mendicant, quite secluded from sensual pleasures, secluded from unskillful qualities, enters and remains in the first absorption, which has the rapture and bliss born of seclusion, while placing the mind and keeping it connected. This is a pleasure that is finer than that. There are those who would say that this is the highest pleasure and happiness that sentient beings experience. But I don’t acknowledge that. Why is that? Because there is another pleasure that is finer than that. 

And\marginnote{8.1} what is that pleasure? It’s when, as the placing of the mind and keeping it connected are stilled, a mendicant enters and remains in the second absorption, which has the rapture and bliss born of immersion, with internal clarity and confidence, and unified mind, without placing the mind and keeping it connected. This is a pleasure that is finer than that. There are those who would say that this is the highest pleasure and happiness that sentient beings experience. But I don’t acknowledge that. Why is that? Because there is another pleasure that is finer than that. 

And\marginnote{9.1} what is that pleasure? It’s when, with the fading away of rapture, a mendicant enters and remains in the third absorption, where they meditate with equanimity, mindful and aware, personally experiencing the bliss of which the noble ones declare, ‘Equanimous and mindful, one meditates in bliss.’ This is a pleasure that is finer than that. There are those who would say that this is the highest pleasure and happiness that sentient beings experience. But I don’t acknowledge that. Why is that? Because there is another pleasure that is finer than that. 

And\marginnote{10.1} what is that pleasure? It’s when, giving up pleasure and pain, and ending former happiness and sadness, a mendicant enters and remains in the fourth absorption, without pleasure or pain, with pure equanimity and mindfulness. This is a pleasure that is finer than that. There are those who would say that this is the highest pleasure and happiness that sentient beings experience. But I don’t acknowledge that. Why is that? Because there is another pleasure that is finer than that. 

And\marginnote{11.1} what is that pleasure? It’s when a mendicant—going totally beyond perceptions of form, with the ending of perceptions of impingement, not focusing on perceptions of diversity—aware that ‘space is infinite’, enters and remains in the dimension of infinite space. This is a pleasure that is finer than that. There are those who would say that this is the highest pleasure and happiness that sentient beings experience. But I don’t acknowledge that. Why is that? Because there is another pleasure that is finer than that. 

And\marginnote{12.1} what is that pleasure? It’s when a mendicant, going totally beyond the dimension of infinite space, aware that ‘consciousness is infinite’, enters and remains in the dimension of infinite consciousness. This is a pleasure that is finer than that. There are those who would say that this is the highest pleasure and happiness that sentient beings experience. But I don’t acknowledge that. Why is that? Because there is another pleasure that is finer than that. 

And\marginnote{13.1} what is that pleasure? It’s when a mendicant, going totally beyond the dimension of infinite consciousness, aware that ‘there is nothing at all’, enters and remains in the dimension of nothingness. This is a pleasure that is finer than that. There are those who would say that this is the highest pleasure and happiness that sentient beings experience. But I don’t acknowledge that. Why is that? Because there is another pleasure that is finer than that. 

And\marginnote{14.1} what is that pleasure? It’s when a mendicant, going totally beyond the dimension of nothingness, enters and remains in the dimension of neither perception nor non-perception. This is a pleasure that is finer than that. There are those who would say that this is the highest pleasure and happiness that sentient beings experience. But I don’t acknowledge that. Why is that? Because there is another pleasure that is finer than that. 

And\marginnote{15.1} what is that pleasure? It’s when a mendicant, going totally beyond the dimension of neither perception nor non-perception, enters and remains in the cessation of perception and feeling. This is a pleasure that is finer than that. 

It’s\marginnote{16.1} possible that wanderers who follow other paths might say: ‘The ascetic Gotama spoke of the cessation of perception and feeling, and he includes it in happiness. What’s up with that?’ 

When\marginnote{16.4} wanderers who follow other paths say this, you should say to them: ‘Reverends, when the Buddha describes what’s included in happiness, he’s not just referring to pleasant feeling. The Realized One describes pleasure as included in happiness wherever it’s found, and in whatever context.’” 

%
\section*{{\suttatitleacronym SN 36.20}{\suttatitletranslation A Mendicant }{\suttatitleroot Bhikkhusutta}}
\addcontentsline{toc}{section}{\tocacronym{SN 36.20} \toctranslation{A Mendicant } \tocroot{Bhikkhusutta}}
\markboth{A Mendicant }{Bhikkhusutta}
\extramarks{SN 36.20}{SN 36.20}

“Mendicants,\marginnote{1.1} in one explanation I’ve spoken of two feelings. In another explanation I’ve spoken of three feelings, or five, six, eighteen, thirty-six, or a hundred and eight feelings. 

I’ve\marginnote{1.2} taught the Dhamma with all these explanations. This being so, you can expect that those who don’t concede, approve, or agree with what has been well spoken will argue, quarrel, and dispute, continually wounding each other with barbed words. 

I’ve\marginnote{1.4} taught the Dhamma with all these explanations. This being so, you can expect that those who do concede, approve, or agree with what has been well spoken will live in harmony, appreciating each other, without quarreling, blending like milk and water, and regarding each other with kindly eyes. 

There\marginnote{2.1} are these five kinds of sensual stimulation. … 

It’s\marginnote{2.2} possible that wanderers who follow other paths might say: ‘The ascetic Gotama spoke of the cessation of perception and feeling, and he includes it in happiness. What’s up with that?’ 

Mendicants,\marginnote{2.5} when wanderers who follow other paths say this, you should say to them: ‘Reverends, when the Buddha describes what’s included in happiness, he’s not just referring to pleasant feeling. The Realized One describes pleasure as included in happiness wherever it’s found, and in whatever context.’” 

%
\addtocontents{toc}{\let\protect\contentsline\protect\nopagecontentsline}
\chapter*{The Chapter on the Explanation of the Hundred and Eight }
\addcontentsline{toc}{chapter}{\tocchapterline{The Chapter on the Explanation of the Hundred and Eight }}
\addtocontents{toc}{\let\protect\contentsline\protect\oldcontentsline}

%
\section*{{\suttatitleacronym SN 36.21}{\suttatitletranslation With Sīvaka }{\suttatitleroot Sīvakasutta}}
\addcontentsline{toc}{section}{\tocacronym{SN 36.21} \toctranslation{With Sīvaka } \tocroot{Sīvakasutta}}
\markboth{With Sīvaka }{Sīvakasutta}
\extramarks{SN 36.21}{SN 36.21}

At\marginnote{1.1} one time the Buddha was staying near \textsanskrit{Rājagaha}, in the Bamboo Grove, the squirrels’ feeding ground. 

Then\marginnote{1.2} the wanderer \textsanskrit{Moḷiyasīvaka} went up to the Buddha and exchanged greetings with him. When the greetings and polite conversation were over, he sat down to one side and said to the Buddha: 

“Master\marginnote{1.4} Gotama, there are some ascetics and brahmins who have this doctrine and view: ‘Everything this individual experiences—pleasurable, painful, or neutral—is because of past deeds.’ What does Master Gotama say about this?” 

“\textsanskrit{Sīvaka},\marginnote{2.1} some feelings stem from bile disorders. You can know this from your own personal experience, and it is generally agreed to be true. Since this is so, the ascetics and brahmins whose view is that everything an individual experiences is because of past deeds go beyond personal experience and beyond what is generally agreed to be true. So those ascetics and brahmins are wrong, I say. 

Some\marginnote{3.1} feelings stem from phlegm disorders … wind disorders … their conjunction … change in weather … not taking care of yourself … overexertion … Some feelings are the result of past deeds. You can know this from your own personal experience, and it is generally agreed to be true. Since this is so, the ascetics and brahmins whose view is that everything an individual experiences is because of past deeds go beyond personal experience and beyond what is generally agreed to be true. So those ascetics and brahmins are wrong, I say.” 

When\marginnote{4.1} he said this, the wanderer \textsanskrit{Moḷiyasīvaka} said to the Buddha, “Excellent, Master Gotama! Excellent! … From this day forth, may Master Gotama remember me as a lay follower who has gone for refuge for life.” 

\begin{verse}%
“Bile,\marginnote{5.1} phlegm, and wind, \\
their conjunction, and the weather, \\
not taking care of yourself, overexertion, \\
and the result of deeds is the eighth.” 

%
\end{verse}

%
\section*{{\suttatitleacronym SN 36.22}{\suttatitletranslation The Explanation of the Hundred and Eight }{\suttatitleroot Aṭṭhasatasutta}}
\addcontentsline{toc}{section}{\tocacronym{SN 36.22} \toctranslation{The Explanation of the Hundred and Eight } \tocroot{Aṭṭhasatasutta}}
\markboth{The Explanation of the Hundred and Eight }{Aṭṭhasatasutta}
\extramarks{SN 36.22}{SN 36.22}

“Mendicants,\marginnote{1.1} I will teach you an exposition of the teaching on the hundred and eight. Listen … 

And\marginnote{1.3} what is the exposition of the teaching on the hundred and eight? Mendicants, in one explanation I’ve spoken of two feelings. In another explanation I’ve spoken of three feelings, or five, six, eighteen, thirty-six, or a hundred and eight feelings. 

And\marginnote{2.1} what are the two feelings? Physical and mental. These are called the two feelings. 

And\marginnote{2.4} what are the three feelings? Pleasant, painful, and neutral feelings. … 

And\marginnote{2.7} what are the five feelings? The faculties of pleasure, pain, happiness, sadness, and equanimity. … 

And\marginnote{2.10} what are the six feelings? Feeling born of eye contact … ear contact … nose contact … tongue contact … body contact … mind contact. … 

And\marginnote{2.14} what are the eighteen feelings? There are six preoccupations with happiness, six preoccupations with sadness, and six preoccupations with equanimity. … 

And\marginnote{2.17} what are the thirty-six feelings? Six kinds of lay happiness and six kinds of renunciate happiness. Six kinds of lay sadness and six kinds of renunciate sadness. Six kinds of lay equanimity and six kinds of renunciate equanimity. … 

And\marginnote{2.20} what are the hundred and eight feelings? Thirty six feelings in the past, future, and present. These are called the hundred and eight feelings. 

This\marginnote{2.23} is the exposition of the teaching on the hundred and eight.” 

%
\section*{{\suttatitleacronym SN 36.23}{\suttatitletranslation With a Mendicant }{\suttatitleroot Aññatarabhikkhusutta}}
\addcontentsline{toc}{section}{\tocacronym{SN 36.23} \toctranslation{With a Mendicant } \tocroot{Aññatarabhikkhusutta}}
\markboth{With a Mendicant }{Aññatarabhikkhusutta}
\extramarks{SN 36.23}{SN 36.23}

Then\marginnote{1.1} a mendicant went up to the Buddha, bowed, sat down to one side, and said to him: 

“Sir,\marginnote{1.2} what is feeling? What’s the origin of feeling? What’s the practice that leads to the origin of feeling? What’s the cessation of feeling? What’s the practice that leads to the cessation of feeling? And what is feeling’s gratification, drawback, and escape?” 

“Mendicant,\marginnote{2.1} there are these three feelings: pleasant, painful, and neutral. These are called feeling. 

Feeling\marginnote{2.4} originates from contact. Craving is the practice that leads to the origin of feeling. 

When\marginnote{2.6} contact ceases, feeling ceases. The practice that leads to the cessation of feelings is simply this noble eightfold path, that is: right view, right thought, right speech, right action, right livelihood, right effort, right mindfulness, and right immersion. 

The\marginnote{2.9} pleasure and happiness that arise from feeling: this is its gratification. 

That\marginnote{2.10} feeling is impermanent, suffering, and perishable: this is its drawback. 

Removing\marginnote{2.11} and giving up desire and greed for feeling: this is its escape.” 

%
\section*{{\suttatitleacronym SN 36.24}{\suttatitletranslation Before }{\suttatitleroot Pubbasutta}}
\addcontentsline{toc}{section}{\tocacronym{SN 36.24} \toctranslation{Before } \tocroot{Pubbasutta}}
\markboth{Before }{Pubbasutta}
\extramarks{SN 36.24}{SN 36.24}

“Mendicants,\marginnote{1.1} before my awakening—when I was still unawakened but intent on awakening—I thought: ‘What is feeling? What’s the origin of feeling? What’s the practice that leads to the origin of feeling? What’s the cessation of feeling? What’s the practice that leads to the cessation of feeling? And what is feeling’s gratification, drawback, and escape?’ 

Then\marginnote{1.4} it occurred to me: ‘There are these three feelings: pleasant, painful, and neutral. These are called feeling. Feeling originates from contact. Craving is the practice that leads to the origin of feeling … Removing and giving up desire and greed for feeling: this is its escape.’” 

%
\section*{{\suttatitleacronym SN 36.25}{\suttatitletranslation Knowledge }{\suttatitleroot Ñāṇasutta}}
\addcontentsline{toc}{section}{\tocacronym{SN 36.25} \toctranslation{Knowledge } \tocroot{Ñāṇasutta}}
\markboth{Knowledge }{Ñāṇasutta}
\extramarks{SN 36.25}{SN 36.25}

“‘These\marginnote{1.1} are the feelings.’ Such was the vision, knowledge, wisdom, realization, and light that arose in me regarding teachings not learned before from another. 

‘This\marginnote{1.2} is the origin of feeling.’ … 

‘This\marginnote{1.3} is the practice that leads to the origin of feeling.’ … 

‘This\marginnote{1.4} is the cessation of feeling.’ … 

‘This\marginnote{1.5} is the practice that leads to the cessation of feeling.’ … 

‘This\marginnote{1.6} is the gratification of feeling.’ … 

‘This\marginnote{1.7} is the drawback of feeling.’ … 

‘This\marginnote{1.8} is the escape from feeling.’ Such was the vision, knowledge, wisdom, realization, and light that arose in me regarding teachings not learned before from another.” 

%
\section*{{\suttatitleacronym SN 36.26}{\suttatitletranslation With Several Mendicants }{\suttatitleroot Sambahulabhikkhusutta}}
\addcontentsline{toc}{section}{\tocacronym{SN 36.26} \toctranslation{With Several Mendicants } \tocroot{Sambahulabhikkhusutta}}
\markboth{With Several Mendicants }{Sambahulabhikkhusutta}
\extramarks{SN 36.26}{SN 36.26}

Then\marginnote{1.1} several mendicants went up to the Buddha, bowed, sat down to one side, and said to him: 

“Sir,\marginnote{1.2} what is feeling? What’s the origin of feeling? What’s the practice that leads to the origin of feeling? What’s the cessation of feeling? What’s the practice that leads to the cessation of feeling? And what is feeling’s gratification, drawback, and escape?” 

“Mendicants,\marginnote{1.5} there are these three feelings. pleasant, painful, and neutral. These are called feeling. 

Feeling\marginnote{1.8} originates from contact. Craving is the practice that leads to the origin of feeling. 

When\marginnote{1.10} contact ceases, feeling ceases. … 

Removing\marginnote{1.11} and giving up desire and greed for feeling: this is its escape.” 

%
\section*{{\suttatitleacronym SN 36.27}{\suttatitletranslation Ascetics and Brahmins (1st) }{\suttatitleroot Paṭhamasamaṇabrāhmaṇasutta}}
\addcontentsline{toc}{section}{\tocacronym{SN 36.27} \toctranslation{Ascetics and Brahmins (1st) } \tocroot{Paṭhamasamaṇabrāhmaṇasutta}}
\markboth{Ascetics and Brahmins (1st) }{Paṭhamasamaṇabrāhmaṇasutta}
\extramarks{SN 36.27}{SN 36.27}

“Mendicants,\marginnote{1.1} there are these three feelings. What three? Pleasant, painful, and neutral feeling. 

There\marginnote{1.4} are ascetics and brahmins who don’t truly understand these three feelings’ gratification, drawback, and escape. I don’t regard them as true ascetics and brahmins. Those venerables don’t realize the goal of life as an ascetic or brahmin, and don’t live having realized it with their own insight. 

There\marginnote{1.6} are ascetics and brahmins who do truly understand these three feelings’ gratification, drawback, and escape. I regard them as true ascetics and brahmins. Those venerables realize the goal of life as an ascetic or brahmin, and live having realized it with their own insight.” 

%
\section*{{\suttatitleacronym SN 36.28}{\suttatitletranslation Ascetics and Brahmins (2nd) }{\suttatitleroot Dutiyasamaṇabrāhmaṇasutta}}
\addcontentsline{toc}{section}{\tocacronym{SN 36.28} \toctranslation{Ascetics and Brahmins (2nd) } \tocroot{Dutiyasamaṇabrāhmaṇasutta}}
\markboth{Ascetics and Brahmins (2nd) }{Dutiyasamaṇabrāhmaṇasutta}
\extramarks{SN 36.28}{SN 36.28}

“Mendicants,\marginnote{1.1} there are these three feelings. What three? Pleasant, painful, and neutral feeling. 

There\marginnote{1.4} are ascetics and brahmins who don’t truly understand these three feelings’ origin, ending, gratification, drawback, and escape. … 

There\marginnote{1.5} are ascetics and brahmins who do truly understand …” 

%
\section*{{\suttatitleacronym SN 36.29}{\suttatitletranslation Ascetics and Brahmins (3rd) }{\suttatitleroot Tatiyasamaṇabrāhmaṇasutta}}
\addcontentsline{toc}{section}{\tocacronym{SN 36.29} \toctranslation{Ascetics and Brahmins (3rd) } \tocroot{Tatiyasamaṇabrāhmaṇasutta}}
\markboth{Ascetics and Brahmins (3rd) }{Tatiyasamaṇabrāhmaṇasutta}
\extramarks{SN 36.29}{SN 36.29}

“Mendicants,\marginnote{1.1} there are ascetics and brahmins who don’t understand feeling, its origin, its cessation, and the practice that leads to its cessation. … 

There\marginnote{1.2} are ascetics and brahmins who do understand …” 

%
\section*{{\suttatitleacronym SN 36.30}{\suttatitletranslation Plain Version }{\suttatitleroot Suddhikasutta}}
\addcontentsline{toc}{section}{\tocacronym{SN 36.30} \toctranslation{Plain Version } \tocroot{Suddhikasutta}}
\markboth{Plain Version }{Suddhikasutta}
\extramarks{SN 36.30}{SN 36.30}

“Mendicants,\marginnote{1.1} there are these three feelings. What three? Pleasant, painful, and neutral feelings. These are the three feelings.” 

%
\section*{{\suttatitleacronym SN 36.31}{\suttatitletranslation Spiritual }{\suttatitleroot Nirāmisasutta}}
\addcontentsline{toc}{section}{\tocacronym{SN 36.31} \toctranslation{Spiritual } \tocroot{Nirāmisasutta}}
\markboth{Spiritual }{Nirāmisasutta}
\extramarks{SN 36.31}{SN 36.31}

“Mendicants,\marginnote{1.1} there is material rapture, spiritual rapture, and even more spiritual rapture. 

There\marginnote{1.2} is material pleasure, spiritual pleasure, and even more spiritual pleasure. 

There\marginnote{1.3} is material equanimity, spiritual equanimity, and even more spiritual equanimity. 

There\marginnote{1.4} is material liberation, spiritual liberation, and even more spiritual liberation. 

And\marginnote{1.5} what is material rapture? There are these five kinds of sensual stimulation. What five? Sights known by the eye that are likable, desirable, agreeable, pleasant, sensual, and arousing. Sounds … Smells … Tastes … Touches known by the body that are likable, desirable, agreeable, pleasant, sensual, and arousing. These are the five kinds of sensual stimulation. The rapture that arises from these five kinds of sensual stimulation is called material rapture. 

And\marginnote{2.1} what is spiritual rapture? It’s when a mendicant, quite secluded from sensual pleasures, secluded from unskillful qualities, enters and remains in the first absorption, which has the rapture and bliss born of seclusion, while placing the mind and keeping it connected. As the placing of the mind and keeping it connected are stilled, they enter and remain in the second absorption, which has the rapture and bliss born of immersion, with internal clarity and confidence, and unified mind, without placing the mind and keeping it connected. This is called spiritual rapture. 

And\marginnote{3.1} what is even more spiritual rapture? When a mendicant who has ended the defilements reviews their mind free from greed, hate, and delusion, rapture arises. This is called even more spiritual rapture. 

And\marginnote{4.1} what is material pleasure? Mendicants, there are these five kinds of sensual stimulation. What five? Sights known by the eye that are likable, desirable, agreeable, pleasant, sensual, and arousing. Sounds … Smells … Tastes … Touches known by the body that are likable, desirable, agreeable, pleasant, sensual, and arousing. These are the five kinds of sensual stimulation. The pleasure and happiness that arise from these five kinds of sensual stimulation is called material pleasure. 

And\marginnote{5.1} what is spiritual pleasure? It’s when a mendicant, quite secluded from sensual pleasures, secluded from unskillful qualities, enters and remains in the first absorption, which has the rapture and bliss born of seclusion, while placing the mind and keeping it connected. As the placing of the mind and keeping it connected are stilled, they enter and remain in the second absorption, which has the rapture and bliss born of immersion, with internal clarity and confidence, and unified mind, without placing the mind and keeping it connected. And with the fading away of rapture, they enter and remain in the third absorption, where they meditate with equanimity, mindful and aware, personally experiencing the bliss of which the noble ones declare, ‘Equanimous and mindful, one meditates in bliss.’ This is called spiritual pleasure. 

And\marginnote{6.1} what is even more spiritual pleasure? When a mendicant who has ended the defilements reviews their mind free from greed, hate, and delusion, pleasure and happiness arises. This is called even more spiritual pleasure. 

And\marginnote{7.1} what is material equanimity? There are these five kinds of sensual stimulation. What five? Sights known by the eye that are likable, desirable, agreeable, pleasant, sensual, and arousing. Sounds … Smells … Tastes … Touches known by the body that are likable, desirable, agreeable, pleasant, sensual, and arousing. These are the five kinds of sensual stimulation. The equanimity that arises from these five kinds of sensual stimulation is called material equanimity. 

And\marginnote{8.1} what is spiritual equanimity? It’s when, giving up pleasure and pain, and ending former happiness and sadness, a mendicant enters and remains in the fourth absorption, without pleasure or pain, with pure equanimity and mindfulness. This is called spiritual equanimity. 

And\marginnote{9.1} what is even more spiritual equanimity? When a mendicant who has ended the defilements reviews their mind free from greed, hate, and delusion, equanimity arises. This is called even more spiritual equanimity. 

And\marginnote{10.1} what is material liberation? Liberation connected with form is material. 

And\marginnote{11.1} what is spiritual liberation? Liberation connected with the formless is spiritual. 

And\marginnote{12.1} what is even more spiritual liberation? When a mendicant who has ended the defilements reviews their mind free from greed, hate, and delusion, liberation arises. This is called even more spiritual liberation.” 

\scendsutta{The Linked Discourses on feeling are complete. }

%
\addtocontents{toc}{\let\protect\contentsline\protect\nopagecontentsline}
\part*{Linked Discourses on Females }
\addcontentsline{toc}{part}{Linked Discourses on Females }
\markboth{}{}
\addtocontents{toc}{\let\protect\contentsline\protect\oldcontentsline}

%
\addtocontents{toc}{\let\protect\contentsline\protect\nopagecontentsline}
\chapter*{First Chapter of Abbreviated Texts }
\addcontentsline{toc}{chapter}{\tocchapterline{First Chapter of Abbreviated Texts }}
\addtocontents{toc}{\let\protect\contentsline\protect\oldcontentsline}

%
\section*{{\suttatitleacronym SN 37.1}{\suttatitletranslation A Female }{\suttatitleroot Mātugāmasutta}}
\addcontentsline{toc}{section}{\tocacronym{SN 37.1} \toctranslation{A Female } \tocroot{Mātugāmasutta}}
\markboth{A Female }{Mātugāmasutta}
\extramarks{SN 37.1}{SN 37.1}

“Mendicants,\marginnote{1.1} when a female has five factors she is extremely undesirable to a man. What five? She’s not attractive, wealthy, or ethical; she’s idle, and she doesn’t beget children. When a female has these five factors she is extremely undesirable to a man. 

When\marginnote{1.5} a female has five factors she is extremely desirable to a man. What five? She’s attractive, wealthy, and ethical; she’s skillful and tireless, and she begets children. When a female has these five factors she is extremely desirable to a man.” 

%
\section*{{\suttatitleacronym SN 37.2}{\suttatitletranslation A Man }{\suttatitleroot Purisasutta}}
\addcontentsline{toc}{section}{\tocacronym{SN 37.2} \toctranslation{A Man } \tocroot{Purisasutta}}
\markboth{A Man }{Purisasutta}
\extramarks{SN 37.2}{SN 37.2}

“Mendicants,\marginnote{1.1} when a man has five factors he is extremely undesirable to a female. What five? He’s not attractive, wealthy, or ethical; he’s idle, and he doesn’t beget children. When a man has these five factors he is extremely undesirable to a female. 

When\marginnote{1.5} a man has five factors he is extremely desirable to a female. What five? He’s attractive, wealthy, and ethical; he’s skillful and tireless, and he begets children. When a man has these five factors he is extremely desirable to a female.” 

%
\section*{{\suttatitleacronym SN 37.3}{\suttatitletranslation Particular Suffering }{\suttatitleroot Āveṇikadukkhasutta}}
\addcontentsline{toc}{section}{\tocacronym{SN 37.3} \toctranslation{Particular Suffering } \tocroot{Āveṇikadukkhasutta}}
\markboth{Particular Suffering }{Āveṇikadukkhasutta}
\extramarks{SN 37.3}{SN 37.3}

“Mendicants,\marginnote{1.1} there are these five kinds of suffering that particularly apply to females. They’re undergone by females and not by men. What five? 

Firstly,\marginnote{1.3} a female, while still young, goes to live with her husband’s family and is separated from her relatives. This is the first kind of suffering that particularly applies to females. 

Furthermore,\marginnote{1.5} a female undergoes the menstrual cycle. This is the second kind of suffering that particularly applies to females. 

Furthermore,\marginnote{1.7} a female undergoes pregnancy. This is the third kind of suffering that particularly applies to females. 

Furthermore,\marginnote{1.9} a female gives birth. This is the fourth kind of suffering that particularly applies to females. 

Furthermore,\marginnote{1.11} a female provides services for a man. This is the fifth kind of suffering that particularly applies to females. 

These\marginnote{1.13} are the five kinds of suffering that particularly apply to females. They’re undergone by females and not by men.” 

%
\section*{{\suttatitleacronym SN 37.4}{\suttatitletranslation Three Qualities }{\suttatitleroot Tīhidhammehisutta}}
\addcontentsline{toc}{section}{\tocacronym{SN 37.4} \toctranslation{Three Qualities } \tocroot{Tīhidhammehisutta}}
\markboth{Three Qualities }{Tīhidhammehisutta}
\extramarks{SN 37.4}{SN 37.4}

“Mendicants,\marginnote{1.1} when females have three qualities, when their body breaks up, after death, they are mostly reborn in a place of loss, a bad place, the underworld, hell. What three? 

A\marginnote{1.3} female lives at home with a heart full of the stain of stinginess in the morning, jealousy in the afternoon, and sexual desire in the evening. 

When\marginnote{1.6} females have these three qualities, when their body breaks up, after death, they are mostly reborn in a place of loss, a bad place, the underworld, hell.” 

%
\section*{{\suttatitleacronym SN 37.5}{\suttatitletranslation Irritable }{\suttatitleroot Kodhanasutta}}
\addcontentsline{toc}{section}{\tocacronym{SN 37.5} \toctranslation{Irritable } \tocroot{Kodhanasutta}}
\markboth{Irritable }{Kodhanasutta}
\extramarks{SN 37.5}{SN 37.5}

Then\marginnote{1.1} Venerable Anuruddha went up to the Buddha, sat down to one side, and said to him: 

“Sometimes,\marginnote{1.2} sir, with my clairvoyance that’s purified and superhuman, I see that a female—when her body breaks up, after death—is reborn in a place of loss, a bad place, the underworld, hell. How many qualities do females have so that they’re reborn in a place of loss, a bad place, the underworld, hell?” 

“Anuruddha,\marginnote{2.1} when females have five qualities, when their body breaks up, after death, they are reborn in a place of loss, a bad place, the underworld, hell. What five? 

They’re\marginnote{2.3} faithless, shameless, imprudent, irritable, and witless. 

When\marginnote{2.4} females have these five qualities, when their body breaks up, after death, they are reborn in a place of loss, a bad place, the underworld, hell.” 

%
\section*{{\suttatitleacronym SN 37.6}{\suttatitletranslation Hostility }{\suttatitleroot Upanāhīsutta}}
\addcontentsline{toc}{section}{\tocacronym{SN 37.6} \toctranslation{Hostility } \tocroot{Upanāhīsutta}}
\markboth{Hostility }{Upanāhīsutta}
\extramarks{SN 37.6}{SN 37.6}

“…\marginnote{1.1} They’re faithless, shameless, imprudent, hostile, and witless. …” 

%
\section*{{\suttatitleacronym SN 37.7}{\suttatitletranslation Jealous }{\suttatitleroot Issukīsutta}}
\addcontentsline{toc}{section}{\tocacronym{SN 37.7} \toctranslation{Jealous } \tocroot{Issukīsutta}}
\markboth{Jealous }{Issukīsutta}
\extramarks{SN 37.7}{SN 37.7}

“…\marginnote{1.1} They’re faithless, shameless, imprudent, jealous, and witless. …” 

%
\section*{{\suttatitleacronym SN 37.8}{\suttatitletranslation Stingy }{\suttatitleroot Maccharīsutta}}
\addcontentsline{toc}{section}{\tocacronym{SN 37.8} \toctranslation{Stingy } \tocroot{Maccharīsutta}}
\markboth{Stingy }{Maccharīsutta}
\extramarks{SN 37.8}{SN 37.8}

“…\marginnote{1.1} They’re faithless, shameless, imprudent, stingy, and witless. …” 

%
\section*{{\suttatitleacronym SN 37.9}{\suttatitletranslation Adultery }{\suttatitleroot Aticārīsutta}}
\addcontentsline{toc}{section}{\tocacronym{SN 37.9} \toctranslation{Adultery } \tocroot{Aticārīsutta}}
\markboth{Adultery }{Aticārīsutta}
\extramarks{SN 37.9}{SN 37.9}

“…\marginnote{1.1} They’re faithless, shameless, imprudent, adulterous, and witless. …” 

%
\section*{{\suttatitleacronym SN 37.10}{\suttatitletranslation Unethical }{\suttatitleroot Dussīlasutta}}
\addcontentsline{toc}{section}{\tocacronym{SN 37.10} \toctranslation{Unethical } \tocroot{Dussīlasutta}}
\markboth{Unethical }{Dussīlasutta}
\extramarks{SN 37.10}{SN 37.10}

“…\marginnote{1.1} They’re faithless, shameless, imprudent, unethical, and witless. …” 

%
\section*{{\suttatitleacronym SN 37.11}{\suttatitletranslation Unlearned }{\suttatitleroot Appassutasutta}}
\addcontentsline{toc}{section}{\tocacronym{SN 37.11} \toctranslation{Unlearned } \tocroot{Appassutasutta}}
\markboth{Unlearned }{Appassutasutta}
\extramarks{SN 37.11}{SN 37.11}

“…\marginnote{1.1} They’re faithless, shameless, imprudent, unlearned, and witless. …” 

%
\section*{{\suttatitleacronym SN 37.12}{\suttatitletranslation Lazy }{\suttatitleroot Kusītasutta}}
\addcontentsline{toc}{section}{\tocacronym{SN 37.12} \toctranslation{Lazy } \tocroot{Kusītasutta}}
\markboth{Lazy }{Kusītasutta}
\extramarks{SN 37.12}{SN 37.12}

“…\marginnote{1.1} They’re faithless, shameless, imprudent, lazy, and witless. …” 

%
\section*{{\suttatitleacronym SN 37.13}{\suttatitletranslation Unmindful }{\suttatitleroot Muṭṭhassatisutta}}
\addcontentsline{toc}{section}{\tocacronym{SN 37.13} \toctranslation{Unmindful } \tocroot{Muṭṭhassatisutta}}
\markboth{Unmindful }{Muṭṭhassatisutta}
\extramarks{SN 37.13}{SN 37.13}

“…\marginnote{1.1} They’re faithless, shameless, imprudent, unmindful, and witless. …” 

%
\section*{{\suttatitleacronym SN 37.14}{\suttatitletranslation Five Threats }{\suttatitleroot Pañcaverasutta}}
\addcontentsline{toc}{section}{\tocacronym{SN 37.14} \toctranslation{Five Threats } \tocroot{Pañcaverasutta}}
\markboth{Five Threats }{Pañcaverasutta}
\extramarks{SN 37.14}{SN 37.14}

“Anuruddha,\marginnote{1.1} when females have five qualities, when their body breaks up, after death, they are reborn in a place of loss, a bad place, the underworld, hell. What five? They kill living creatures, steal, commit sexual misconduct, lie, and consume alcoholic drinks that cause negligence. When females have these five qualities, when their body breaks up, after death, they are reborn in a place of loss, a bad place, the underworld, hell.” 

%
\addtocontents{toc}{\let\protect\contentsline\protect\nopagecontentsline}
\chapter*{Second Chapter of Abbreviated Texts }
\addcontentsline{toc}{chapter}{\tocchapterline{Second Chapter of Abbreviated Texts }}
\addtocontents{toc}{\let\protect\contentsline\protect\oldcontentsline}

%
\section*{{\suttatitleacronym SN 37.15}{\suttatitletranslation Loving }{\suttatitleroot Akkodhanasutta}}
\addcontentsline{toc}{section}{\tocacronym{SN 37.15} \toctranslation{Loving } \tocroot{Akkodhanasutta}}
\markboth{Loving }{Akkodhanasutta}
\extramarks{SN 37.15}{SN 37.15}

Then\marginnote{1.1} Venerable Anuruddha went up to the Buddha … and asked him, “Sometimes, sir, with my clairvoyance that’s purified and superhuman, I see that a female—when her body breaks up, after death—is reborn in a good place, a heavenly realm. How many qualities do females have so that they’re reborn in a good place, a heavenly realm?” 

“Anuruddha,\marginnote{2.1} when females have five qualities, when their body breaks up, after death, they are reborn in a good place, a heavenly realm. What five? They’re faithful, conscientious, prudent, loving, and wise. When females have these five qualities, when their body breaks up, after death, they are reborn in a good place, a heavenly realm.” 

%
\section*{{\suttatitleacronym SN 37.16}{\suttatitletranslation Free of Hostility }{\suttatitleroot Anupanāhīsutta}}
\addcontentsline{toc}{section}{\tocacronym{SN 37.16} \toctranslation{Free of Hostility } \tocroot{Anupanāhīsutta}}
\markboth{Free of Hostility }{Anupanāhīsutta}
\extramarks{SN 37.16}{SN 37.16}

“…\marginnote{1.1} They’re faithful, conscientious, prudent, free of hostility, and wise. …” 

%
\section*{{\suttatitleacronym SN 37.17}{\suttatitletranslation Free of Jealousy }{\suttatitleroot Anissukīsutta}}
\addcontentsline{toc}{section}{\tocacronym{SN 37.17} \toctranslation{Free of Jealousy } \tocroot{Anissukīsutta}}
\markboth{Free of Jealousy }{Anissukīsutta}
\extramarks{SN 37.17}{SN 37.17}

“…\marginnote{1.1} They’re faithful, conscientious, prudent, free of jealousy, and wise. …” 

%
\section*{{\suttatitleacronym SN 37.18}{\suttatitletranslation Free of Stinginess }{\suttatitleroot Amaccharīsutta}}
\addcontentsline{toc}{section}{\tocacronym{SN 37.18} \toctranslation{Free of Stinginess } \tocroot{Amaccharīsutta}}
\markboth{Free of Stinginess }{Amaccharīsutta}
\extramarks{SN 37.18}{SN 37.18}

“…\marginnote{1.1} They’re faithful, conscientious, prudent, free of stinginess, and wise. …” 

%
\section*{{\suttatitleacronym SN 37.19}{\suttatitletranslation Not Adulterous }{\suttatitleroot Anaticārīsutta}}
\addcontentsline{toc}{section}{\tocacronym{SN 37.19} \toctranslation{Not Adulterous } \tocroot{Anaticārīsutta}}
\markboth{Not Adulterous }{Anaticārīsutta}
\extramarks{SN 37.19}{SN 37.19}

“…\marginnote{1.1} They’re faithful, conscientious, prudent, not adulterous, and wise. …” 

%
\section*{{\suttatitleacronym SN 37.20}{\suttatitletranslation Ethical }{\suttatitleroot Susīlasutta}}
\addcontentsline{toc}{section}{\tocacronym{SN 37.20} \toctranslation{Ethical } \tocroot{Susīlasutta}}
\markboth{Ethical }{Susīlasutta}
\extramarks{SN 37.20}{SN 37.20}

“…\marginnote{1.1} They’re faithful, conscientious, prudent, ethical, and wise. …” 

%
\section*{{\suttatitleacronym SN 37.21}{\suttatitletranslation Educated }{\suttatitleroot Bahussutasutta}}
\addcontentsline{toc}{section}{\tocacronym{SN 37.21} \toctranslation{Educated } \tocroot{Bahussutasutta}}
\markboth{Educated }{Bahussutasutta}
\extramarks{SN 37.21}{SN 37.21}

“…\marginnote{1.1} They’re faithful, conscientious, prudent, educated, and wise. …” 

%
\section*{{\suttatitleacronym SN 37.22}{\suttatitletranslation Energetic }{\suttatitleroot Āraddhavīriyasutta}}
\addcontentsline{toc}{section}{\tocacronym{SN 37.22} \toctranslation{Energetic } \tocroot{Āraddhavīriyasutta}}
\markboth{Energetic }{Āraddhavīriyasutta}
\extramarks{SN 37.22}{SN 37.22}

“…\marginnote{1.1} They’re faithful, conscientious, prudent, energetic, and wise. …” 

%
\section*{{\suttatitleacronym SN 37.23}{\suttatitletranslation Mindful }{\suttatitleroot Upaṭṭhitassatisutta}}
\addcontentsline{toc}{section}{\tocacronym{SN 37.23} \toctranslation{Mindful } \tocroot{Upaṭṭhitassatisutta}}
\markboth{Mindful }{Upaṭṭhitassatisutta}
\extramarks{SN 37.23}{SN 37.23}

“…\marginnote{1.1} They’re faithful, conscientious, prudent, mindful, and wise. …” 

%
\section*{{\suttatitleacronym SN 37.24}{\suttatitletranslation Five Precepts }{\suttatitleroot Pañcasīlasutta}}
\addcontentsline{toc}{section}{\tocacronym{SN 37.24} \toctranslation{Five Precepts } \tocroot{Pañcasīlasutta}}
\markboth{Five Precepts }{Pañcasīlasutta}
\extramarks{SN 37.24}{SN 37.24}

“Anuruddha,\marginnote{1.1} when females have five qualities, when their body breaks up, after death, they are reborn in a good place, a heavenly realm. What five? They don’t kill living creatures, steal, commit sexual misconduct, lie, or consume alcoholic drinks that cause negligence. When females have these five qualities, when their body breaks up, after death, they are reborn in a good place, a heavenly realm.” 

%
\addtocontents{toc}{\let\protect\contentsline\protect\nopagecontentsline}
\chapter*{The Chapter on Fools }
\addcontentsline{toc}{chapter}{\tocchapterline{The Chapter on Fools }}
\addtocontents{toc}{\let\protect\contentsline\protect\oldcontentsline}

%
\section*{{\suttatitleacronym SN 37.25}{\suttatitletranslation Assured }{\suttatitleroot Visāradasutta}}
\addcontentsline{toc}{section}{\tocacronym{SN 37.25} \toctranslation{Assured } \tocroot{Visāradasutta}}
\markboth{Assured }{Visāradasutta}
\extramarks{SN 37.25}{SN 37.25}

“Mendicants,\marginnote{1.1} there are these five powers of a female. What five? Attractiveness, wealth, relatives, children, and ethical behavior. These are the five powers of a female. A female living at home with these five qualities is self-assured.” 

%
\section*{{\suttatitleacronym SN 37.26}{\suttatitletranslation Under Her Thumb }{\suttatitleroot Pasayhasutta}}
\addcontentsline{toc}{section}{\tocacronym{SN 37.26} \toctranslation{Under Her Thumb } \tocroot{Pasayhasutta}}
\markboth{Under Her Thumb }{Pasayhasutta}
\extramarks{SN 37.26}{SN 37.26}

“Mendicants,\marginnote{1.1} there are these five powers of a female. What five? Attractiveness, wealth, relatives, children, and ethical behavior. These are the five powers of a female. A female living at home with these five powers has her husband under her thumb.” 

%
\section*{{\suttatitleacronym SN 37.27}{\suttatitletranslation Mastered }{\suttatitleroot Abhibhuyyasutta}}
\addcontentsline{toc}{section}{\tocacronym{SN 37.27} \toctranslation{Mastered } \tocroot{Abhibhuyyasutta}}
\markboth{Mastered }{Abhibhuyyasutta}
\extramarks{SN 37.27}{SN 37.27}

“Mendicants,\marginnote{1.1} there are these five powers of a female. What five? Attractiveness, wealth, relatives, children, and ethical behavior. These are the five powers of a female. A female living at home with these five powers has her husband under her mastery.” 

%
\section*{{\suttatitleacronym SN 37.28}{\suttatitletranslation One }{\suttatitleroot Ekasutta}}
\addcontentsline{toc}{section}{\tocacronym{SN 37.28} \toctranslation{One } \tocroot{Ekasutta}}
\markboth{One }{Ekasutta}
\extramarks{SN 37.28}{SN 37.28}

“Mendicants,\marginnote{1.1} when a man has one power he has a female under his mastery. What one power? The power of authority. Mastered by this, a female’s powers of attractiveness, wealth, relatives, children, and ethical behavior do not avail her.” 

%
\section*{{\suttatitleacronym SN 37.29}{\suttatitletranslation In That Respect }{\suttatitleroot Aṅgasutta}}
\addcontentsline{toc}{section}{\tocacronym{SN 37.29} \toctranslation{In That Respect } \tocroot{Aṅgasutta}}
\markboth{In That Respect }{Aṅgasutta}
\extramarks{SN 37.29}{SN 37.29}

“Mendicants,\marginnote{1.1} there are these five powers of a female. What five? 

Attractiveness,\marginnote{1.3} wealth, relatives, children, and ethical behavior. 

A\marginnote{1.4} female who has the power of attractiveness but not the power of wealth is incomplete in that respect. But when she has the power of attractiveness and the power of wealth she’s complete in that respect. 

A\marginnote{1.8} female who has the powers of attractiveness and wealth, but not the power of relatives is incomplete in that respect. But when she has the powers of attractiveness, wealth, and relatives she’s complete in that respect. 

A\marginnote{1.12} female who has the powers of attractiveness, wealth, and relatives, but not the power of children is incomplete in that respect. But when she has the powers of attractiveness, wealth, relatives, and children she’s complete in that respect. 

A\marginnote{1.16} female who has the powers of attractiveness, wealth, relatives, and children, but not the power of ethical behavior is incomplete in that respect. But when she has the powers of attractiveness, wealth, relatives, children, and ethical behavior she’s complete in that respect. 

These\marginnote{1.20} are the five powers of a female.” 

%
\section*{{\suttatitleacronym SN 37.30}{\suttatitletranslation They Send Her Away }{\suttatitleroot Nāsentisutta}}
\addcontentsline{toc}{section}{\tocacronym{SN 37.30} \toctranslation{They Send Her Away } \tocroot{Nāsentisutta}}
\markboth{They Send Her Away }{Nāsentisutta}
\extramarks{SN 37.30}{SN 37.30}

“Mendicants,\marginnote{1.1} there are these five powers of a female. What five? 

Attractiveness,\marginnote{1.3} wealth, relatives, children, and ethical behavior. 

If\marginnote{1.4} a female has the power of attractiveness but not that of ethical behavior, the family will send her away, they won’t accommodate her. 

If\marginnote{1.5} a female has the powers of attractiveness and wealth but not that of ethical behavior, the family will send her away, they won’t accommodate her. 

If\marginnote{1.6} a female has the powers of attractiveness, wealth, and relatives, but not that of ethical behavior, the family will send her away, they won’t accommodate her. 

If\marginnote{1.7} a female has the powers of attractiveness, wealth, relatives, and children, but not that of ethical behavior, the family will send her away, they won’t accommodate her. 

If\marginnote{1.8} a female has the power of ethical behavior but not that of attractiveness, the family will accommodate her, they won’t send her away. 

If\marginnote{1.9} a female has the power of ethical behavior but not that of wealth, the family will accommodate her, they won’t send her away. 

If\marginnote{1.10} a female has the power of ethical behavior but not that of relatives, the family will accommodate her, they won’t send her away. 

If\marginnote{1.11} a female has the power of ethical behavior but not that of children, the family will accommodate her, they won’t send her away. 

These\marginnote{1.12} are the five powers of a female.” 

%
\section*{{\suttatitleacronym SN 37.31}{\suttatitletranslation Cause }{\suttatitleroot Hetusutta}}
\addcontentsline{toc}{section}{\tocacronym{SN 37.31} \toctranslation{Cause } \tocroot{Hetusutta}}
\markboth{Cause }{Hetusutta}
\extramarks{SN 37.31}{SN 37.31}

“Mendicants,\marginnote{1.1} there are these five powers of a female. What five? Attractiveness, wealth, relatives, children, and ethical behavior. 

It\marginnote{1.4} is not because of the powers of attractiveness, wealth, relatives, or children that females, when their body breaks up, after death, are reborn in a good place, a heavenly realm. It is because of the power of ethical behavior that females, when their body breaks up, after death, are reborn in a good place, a heavenly realm. 

These\marginnote{1.6} are the five powers of a female.” 

%
\section*{{\suttatitleacronym SN 37.32}{\suttatitletranslation Things }{\suttatitleroot Ṭhānasutta}}
\addcontentsline{toc}{section}{\tocacronym{SN 37.32} \toctranslation{Things } \tocroot{Ṭhānasutta}}
\markboth{Things }{Ṭhānasutta}
\extramarks{SN 37.32}{SN 37.32}

“Mendicants,\marginnote{1.1} there are five things that are hard to get for females who have not made merit. What five? 

‘May\marginnote{1.3} I be born into a suitable family!’ This is the first thing. 

‘Having\marginnote{1.5} been born in a suitable family, may I marry into a suitable family!’ This is the second thing. 

‘Having\marginnote{1.7} been born in a suitable family and married into a suitable family, may I live at home without a co-wife!’ This is the third thing. 

‘Having\marginnote{1.9} been born in a suitable family, and married into a suitable family, and living at home without a co-wife, may I have children!’ This is the fourth thing. 

‘Having\marginnote{1.11} been born in a suitable family, and married into a suitable family, and living at home without a co-wife, and having had children, may I master my husband!’ This is the fifth thing. 

These\marginnote{1.13} are the five things that are hard to get for females who have not made merit. 

There\marginnote{2.1} are five things that are easy to get for females who have made merit. What five? 

‘May\marginnote{2.3} I be born into a suitable family!’ This is the first thing. 

‘Having\marginnote{2.5} been born into a suitable family, may I marry into a suitable family!’ This is the second thing. 

‘Having\marginnote{2.7} been born into a suitable family and married into a suitable family, may I live at home without a co-wife!’ This is the third thing. 

‘Having\marginnote{2.9} been born into a suitable family, and married into a suitable family, and living at home without a co-wife, may I have children!’ This is the fourth thing. 

‘Having\marginnote{2.11} been born into a suitable family, and married into a suitable family, and living at home without a co-wife, and having had children, may I master my husband!’ This is the fifth thing. 

These\marginnote{2.13} are the five things that are easy to get for females who have made merit.” 

%
\section*{{\suttatitleacronym SN 37.33}{\suttatitletranslation Living With Self-Assurance }{\suttatitleroot Pañcasīlavisāradasutta}}
\addcontentsline{toc}{section}{\tocacronym{SN 37.33} \toctranslation{Living With Self-Assurance } \tocroot{Pañcasīlavisāradasutta}}
\markboth{Living With Self-Assurance }{Pañcasīlavisāradasutta}
\extramarks{SN 37.33}{SN 37.33}

“Mendicants,\marginnote{1.1} a female living at home with five qualities is self-assured. What five? She doesn’t kill living creatures, steal, commit sexual misconduct, lie, or consume alcoholic drinks that cause negligence. A female living at home with these five qualities is self-assured.” 

%
\section*{{\suttatitleacronym SN 37.34}{\suttatitletranslation Growth }{\suttatitleroot Vaḍḍhīsutta}}
\addcontentsline{toc}{section}{\tocacronym{SN 37.34} \toctranslation{Growth } \tocroot{Vaḍḍhīsutta}}
\markboth{Growth }{Vaḍḍhīsutta}
\extramarks{SN 37.34}{SN 37.34}

“Mendicants,\marginnote{1.1} a female noble disciple who grows in five ways grows nobly, taking on what is essential and excellent in this life. What five? She grows in faith, ethics, learning, generosity, and wisdom. A female noble disciple who grows in these five ways grows nobly, taking on what is essential and excellent in this life. 

\begin{verse}%
When\marginnote{2.1} she grows in faith and ethics, \\
wisdom, and both generosity and learning—\\
a virtuous laywoman such as she \\
takes on what is essential for herself in this life.” 

%
\end{verse}

\scendsutta{The Linked Discourses on females are complete. }

%
\addtocontents{toc}{\let\protect\contentsline\protect\nopagecontentsline}
\part*{Linked Discourses with Jambukhādaka }
\addcontentsline{toc}{part}{Linked Discourses with Jambukhādaka }
\markboth{}{}
\addtocontents{toc}{\let\protect\contentsline\protect\oldcontentsline}

%
\addtocontents{toc}{\let\protect\contentsline\protect\nopagecontentsline}
\chapter*{The Chapter with Jambukhādaka }
\addcontentsline{toc}{chapter}{\tocchapterline{The Chapter with Jambukhādaka }}
\addtocontents{toc}{\let\protect\contentsline\protect\oldcontentsline}

%
\section*{{\suttatitleacronym SN 38.1}{\suttatitletranslation A Question About Extinguishment }{\suttatitleroot Nibbānapañhāsutta}}
\addcontentsline{toc}{section}{\tocacronym{SN 38.1} \toctranslation{A Question About Extinguishment } \tocroot{Nibbānapañhāsutta}}
\markboth{A Question About Extinguishment }{Nibbānapañhāsutta}
\extramarks{SN 38.1}{SN 38.1}

At\marginnote{1.1} one time Venerable \textsanskrit{Sāriputta} was staying in the land of the Magadhans near the little village of \textsanskrit{Nālaka}. Then the wanderer \textsanskrit{Jambukhādaka} went up to Venerable \textsanskrit{Sāriputta} and exchanged greetings with him. When the greetings and polite conversation were over, he sat down to one side and said to \textsanskrit{Sāriputta}: 

“Reverend\marginnote{2.1} \textsanskrit{Sāriputta}, they speak of this thing called ‘extinguishment’. What is extinguishment?” 

“Reverend,\marginnote{2.3} the ending of greed, hate, and delusion is called extinguishment.” 

“But,\marginnote{2.5} reverend, is there a path and a practice for realizing this extinguishment?” 

“There\marginnote{2.6} is, reverend.” 

“Well,\marginnote{2.7} what is it?” 

“It\marginnote{2.8} is simply this noble eightfold path, that is: right view, right thought, right speech, right action, right livelihood, right effort, right mindfulness, and right immersion. This is the path, the practice, for realizing this extinguishment.” 

“Reverend,\marginnote{2.11} this is a fine path, a fine practice, for realizing this extinguishment. Just this much is enough to be diligent.” 

%
\section*{{\suttatitleacronym SN 38.2}{\suttatitletranslation A Question About Perfection }{\suttatitleroot Arahattapañhāsutta}}
\addcontentsline{toc}{section}{\tocacronym{SN 38.2} \toctranslation{A Question About Perfection } \tocroot{Arahattapañhāsutta}}
\markboth{A Question About Perfection }{Arahattapañhāsutta}
\extramarks{SN 38.2}{SN 38.2}

“Reverend\marginnote{1.1} \textsanskrit{Sāriputta}, they speak of this thing called ‘perfection’. What is perfection?” 

“Reverend,\marginnote{1.3} the ending of greed, hate, and delusion is called perfection.” 

“But,\marginnote{1.5} reverend, is there a path and a practice for realizing this perfection?” 

“There\marginnote{1.6} is, reverend.” 

“Well,\marginnote{1.7} what is it?” 

“It\marginnote{1.8} is simply this noble eightfold path, that is: right view, right thought, right speech, right action, right livelihood, right effort, right mindfulness, and right immersion. This is the path, the practice, for realizing this perfection.” 

“Reverend,\marginnote{1.11} this is a fine path, a fine practice, for realizing this perfection. Just this much is enough to be diligent.” 

%
\section*{{\suttatitleacronym SN 38.3}{\suttatitletranslation Principled Speech }{\suttatitleroot Dhammavādīpañhāsutta}}
\addcontentsline{toc}{section}{\tocacronym{SN 38.3} \toctranslation{Principled Speech } \tocroot{Dhammavādīpañhāsutta}}
\markboth{Principled Speech }{Dhammavādīpañhāsutta}
\extramarks{SN 38.3}{SN 38.3}

“Reverend\marginnote{1.1} \textsanskrit{Sāriputta}, who in the world have principled speech? Who in the world practice well? Who are the Holy Ones in the world?” 

“Reverend,\marginnote{1.2} those who teach principles for giving up greed, hate, and delusion have principled speech in the world. Those who practice for giving up greed, hate, and delusion are practicing well in the world. Those who have given up greed, hate, and delusion—so they’re cut off at the root, made like a palm stump, obliterated, and unable to arise in the future—are Holy Ones in the world.” 

“But,\marginnote{2.1} reverend, is there a path and a practice for giving up that greed, hate, and delusion?” 

“There\marginnote{2.2} is, reverend.” 

“Well,\marginnote{2.3} what is it?” 

“It\marginnote{2.4} is simply this noble eightfold path, that is: right view, right thought, right speech, right action, right livelihood, right effort, right mindfulness, and right immersion. This is the path, this is the practice for giving up that greed, hate, and delusion.” 

“This\marginnote{2.7} is a fine path, a fine practice, for giving up greed, hate, and delusion. Just this much is enough to be diligent.” 

%
\section*{{\suttatitleacronym SN 38.4}{\suttatitletranslation What’s the Purpose }{\suttatitleroot Kimatthiyasutta}}
\addcontentsline{toc}{section}{\tocacronym{SN 38.4} \toctranslation{What’s the Purpose } \tocroot{Kimatthiyasutta}}
\markboth{What’s the Purpose }{Kimatthiyasutta}
\extramarks{SN 38.4}{SN 38.4}

“Reverend\marginnote{1.1} \textsanskrit{Sāriputta}, what’s the purpose of leading the spiritual life under the ascetic Gotama?” 

“The\marginnote{1.2} purpose of leading the spiritual life under the Buddha is to completely understand suffering.” 

“But,\marginnote{1.3} reverend, is there a path and a practice for completely understanding this suffering?” 

“There\marginnote{1.4} is.” … 

%
\section*{{\suttatitleacronym SN 38.5}{\suttatitletranslation Solace }{\suttatitleroot Assāsappattasutta}}
\addcontentsline{toc}{section}{\tocacronym{SN 38.5} \toctranslation{Solace } \tocroot{Assāsappattasutta}}
\markboth{Solace }{Assāsappattasutta}
\extramarks{SN 38.5}{SN 38.5}

“Reverend\marginnote{1.1} \textsanskrit{Sāriputta}, they speak of this thing called ‘gaining solace’. At what point do you gain solace?” 

“When\marginnote{1.3} a mendicant truly understands the six fields of contact’s origin, ending, gratification, drawback, and escape, at that point they’ve gained solace.” 

“But,\marginnote{1.4} reverend, is there a path and a practice for realizing this solace?” 

“There\marginnote{1.5} is.” … 

%
\section*{{\suttatitleacronym SN 38.6}{\suttatitletranslation Ultimate Solace }{\suttatitleroot Paramassāsappattasutta}}
\addcontentsline{toc}{section}{\tocacronym{SN 38.6} \toctranslation{Ultimate Solace } \tocroot{Paramassāsappattasutta}}
\markboth{Ultimate Solace }{Paramassāsappattasutta}
\extramarks{SN 38.6}{SN 38.6}

“Reverend\marginnote{1.1} \textsanskrit{Sāriputta}, they speak of this thing called ‘gaining ultimate solace’. At what point do you gain ultimate solace?” 

“When\marginnote{1.3} a mendicant is freed by not grasping after truly understanding the six fields of contact’s origin, ending, gratification, drawback, and escape, at that point they’ve gained ultimate solace.” 

“But,\marginnote{1.4} reverend, is there a path and a practice for realizing this ultimate solace?” 

“There\marginnote{1.5} is.” … 

%
\section*{{\suttatitleacronym SN 38.7}{\suttatitletranslation A Question About Feeling }{\suttatitleroot Vedanāpañhāsutta}}
\addcontentsline{toc}{section}{\tocacronym{SN 38.7} \toctranslation{A Question About Feeling } \tocroot{Vedanāpañhāsutta}}
\markboth{A Question About Feeling }{Vedanāpañhāsutta}
\extramarks{SN 38.7}{SN 38.7}

“Reverend\marginnote{1.1} \textsanskrit{Sāriputta}, they speak of this thing called ‘feeling’. What is feeling?” 

“Reverend,\marginnote{1.3} there are three feelings. What three? Pleasant, painful, and neutral feeling. These are the three feelings.” 

“But\marginnote{1.7} reverend, is there a path and a practice for completely understanding these three feelings?” 

“There\marginnote{1.8} is.” … 

%
\section*{{\suttatitleacronym SN 38.8}{\suttatitletranslation A Question About Defilements }{\suttatitleroot Āsavapañhāsutta}}
\addcontentsline{toc}{section}{\tocacronym{SN 38.8} \toctranslation{A Question About Defilements } \tocroot{Āsavapañhāsutta}}
\markboth{A Question About Defilements }{Āsavapañhāsutta}
\extramarks{SN 38.8}{SN 38.8}

“Reverend\marginnote{1.1} \textsanskrit{Sāriputta}, they speak of this thing called ‘defilement’. What is defilement?” 

“Reverend,\marginnote{1.3} there are three defilements. The defilements of sensuality, desire to be reborn, and ignorance. These are the three defilements.” 

“But,\marginnote{1.6} reverend, is there a path and a practice for completely understanding these three defilements?” 

“There\marginnote{1.7} is.” … 

%
\section*{{\suttatitleacronym SN 38.9}{\suttatitletranslation A Question About Ignorance }{\suttatitleroot Avijjāpañhāsutta}}
\addcontentsline{toc}{section}{\tocacronym{SN 38.9} \toctranslation{A Question About Ignorance } \tocroot{Avijjāpañhāsutta}}
\markboth{A Question About Ignorance }{Avijjāpañhāsutta}
\extramarks{SN 38.9}{SN 38.9}

“Reverend\marginnote{1.1} \textsanskrit{Sāriputta}, they speak of this thing called ‘ignorance’. What is ignorance?” 

“Not\marginnote{1.3} knowing about suffering, the origin of suffering, the cessation of suffering, and the practice that leads to the cessation of suffering. This is called ignorance.” 

“But,\marginnote{1.5} reverend, is there a path and a practice for giving up that ignorance?” 

“There\marginnote{1.6} is.” … 

%
\section*{{\suttatitleacronym SN 38.10}{\suttatitletranslation A Question About Craving }{\suttatitleroot Taṇhāpañhāsutta}}
\addcontentsline{toc}{section}{\tocacronym{SN 38.10} \toctranslation{A Question About Craving } \tocroot{Taṇhāpañhāsutta}}
\markboth{A Question About Craving }{Taṇhāpañhāsutta}
\extramarks{SN 38.10}{SN 38.10}

“Reverend\marginnote{1.1} \textsanskrit{Sāriputta}, they speak of this thing called ‘craving’. What is craving?” 

“Reverend,\marginnote{1.3} there are these three cravings. Craving for sensual pleasures, craving to continue existence, and craving to end existence. These are the three cravings.” 

“But,\marginnote{1.6} reverend, is there a path and a practice for completely understanding these cravings?” 

“There\marginnote{1.7} is.” … 

%
\section*{{\suttatitleacronym SN 38.11}{\suttatitletranslation A Question About Floods }{\suttatitleroot Oghapañhāsutta}}
\addcontentsline{toc}{section}{\tocacronym{SN 38.11} \toctranslation{A Question About Floods } \tocroot{Oghapañhāsutta}}
\markboth{A Question About Floods }{Oghapañhāsutta}
\extramarks{SN 38.11}{SN 38.11}

“Reverend\marginnote{1.1} \textsanskrit{Sāriputta}, they speak of this thing called ‘a flood’. What is a flood?” 

“Reverend,\marginnote{1.3} there are these four floods. The floods of sensuality, desire to be reborn, views, and ignorance. These are the four floods.” 

“But,\marginnote{1.6} reverend, is there a path and a practice for completely understanding these floods?” 

“There\marginnote{1.7} is.” … 

%
\section*{{\suttatitleacronym SN 38.12}{\suttatitletranslation A Question About Grasping }{\suttatitleroot Upādānapañhāsutta}}
\addcontentsline{toc}{section}{\tocacronym{SN 38.12} \toctranslation{A Question About Grasping } \tocroot{Upādānapañhāsutta}}
\markboth{A Question About Grasping }{Upādānapañhāsutta}
\extramarks{SN 38.12}{SN 38.12}

“Reverend\marginnote{1.1} \textsanskrit{Sāriputta}, they speak of this thing called ‘grasping’. What is grasping?” 

“Reverend,\marginnote{1.3} there are these four kinds of grasping. Grasping at sensual pleasures, views, precepts and observances, and theories of a self. These are the four kinds of grasping.” 

“But,\marginnote{1.6} reverend, is there a path and a practice for completely understanding these four kinds of grasping?” 

“There\marginnote{1.7} is.” … 

%
\section*{{\suttatitleacronym SN 38.13}{\suttatitletranslation A Question About States of Existence }{\suttatitleroot Bhavapañhāsutta}}
\addcontentsline{toc}{section}{\tocacronym{SN 38.13} \toctranslation{A Question About States of Existence } \tocroot{Bhavapañhāsutta}}
\markboth{A Question About States of Existence }{Bhavapañhāsutta}
\extramarks{SN 38.13}{SN 38.13}

“Reverend\marginnote{1.1} \textsanskrit{Sāriputta}, they speak of these things called ‘states of existence’. What are states of existence?” 

“Reverend,\marginnote{1.3} there are these three states of existence. Existence in the sensual realm, the realm of luminous form, and the formless realm. These are the three states of existence.” 

“But,\marginnote{1.6} reverend, is there a path and a practice for completely understanding these three states of existence?” 

“There\marginnote{1.7} is.” … 

%
\section*{{\suttatitleacronym SN 38.14}{\suttatitletranslation A Question About Suffering }{\suttatitleroot Dukkhapañhāsutta}}
\addcontentsline{toc}{section}{\tocacronym{SN 38.14} \toctranslation{A Question About Suffering } \tocroot{Dukkhapañhāsutta}}
\markboth{A Question About Suffering }{Dukkhapañhāsutta}
\extramarks{SN 38.14}{SN 38.14}

“Reverend\marginnote{1.1} \textsanskrit{Sāriputta}, they speak of this thing called ‘suffering’. What is suffering?” 

“Reverend,\marginnote{1.3} there are these three forms of suffering. The suffering inherent in painful feeling; the suffering inherent in conditions; and the suffering inherent in perishing. These are the three forms of suffering.” 

“But,\marginnote{1.6} reverend, is there a path and a practice for completely understanding these three forms of suffering?” 

“There\marginnote{1.7} is.” … 

%
\section*{{\suttatitleacronym SN 38.15}{\suttatitletranslation A Question About Identity }{\suttatitleroot Sakkāyapañhāsutta}}
\addcontentsline{toc}{section}{\tocacronym{SN 38.15} \toctranslation{A Question About Identity } \tocroot{Sakkāyapañhāsutta}}
\markboth{A Question About Identity }{Sakkāyapañhāsutta}
\extramarks{SN 38.15}{SN 38.15}

“Reverend\marginnote{1.1} \textsanskrit{Sāriputta}, they speak of this thing called ‘identity’. What is identity?” 

“Reverend,\marginnote{1.3} the Buddha said that these five grasping aggregates are identity. That is, form, feeling, perception, choices, and consciousness. The Buddha said that these five grasping aggregates are identity.” 

“But,\marginnote{1.6} reverend, is there a path and a practice for completely understanding this identity?” 

“There\marginnote{1.7} is.” … 

%
\section*{{\suttatitleacronym SN 38.16}{\suttatitletranslation A Question About What’s Hard to Do }{\suttatitleroot Dukkarapañhāsutta}}
\addcontentsline{toc}{section}{\tocacronym{SN 38.16} \toctranslation{A Question About What’s Hard to Do } \tocroot{Dukkarapañhāsutta}}
\markboth{A Question About What’s Hard to Do }{Dukkarapañhāsutta}
\extramarks{SN 38.16}{SN 38.16}

“Reverend\marginnote{1.1} \textsanskrit{Sāriputta}, in this teaching and training, what is hard to do?” 

“Going\marginnote{1.2} forth, reverend, is hard to do in this teaching and training.” 

“But\marginnote{1.3} what’s hard to do for someone who has gone forth?” 

“When\marginnote{1.4} you’ve gone forth it’s hard to be satisfied.” 

“But\marginnote{1.5} what’s hard to do for someone who is satisfied?” 

“When\marginnote{1.6} you’re satisfied, it’s hard to practice in line with the teaching.” 

“But\marginnote{1.7} if a mendicant practices in line with the teaching, will it take them long to become a perfected one?” 

“Not\marginnote{1.8} long, reverend.” 

\scendsutta{The Linked Discourses with \textsanskrit{Jambukhādaka} are complete. }

%
\addtocontents{toc}{\let\protect\contentsline\protect\nopagecontentsline}
\part*{Linked Discourses with Sāmaṇḍaka }
\addcontentsline{toc}{part}{Linked Discourses with Sāmaṇḍaka }
\markboth{}{}
\addtocontents{toc}{\let\protect\contentsline\protect\oldcontentsline}

%
\addtocontents{toc}{\let\protect\contentsline\protect\nopagecontentsline}
\chapter*{The Chapter with Sāmaṇḍaka }
\addcontentsline{toc}{chapter}{\tocchapterline{The Chapter with Sāmaṇḍaka }}
\addtocontents{toc}{\let\protect\contentsline\protect\oldcontentsline}

%
\section*{{\suttatitleacronym SN 39.1–15}{\suttatitletranslation With Sāmaṇḍaka on Extinguishment }{\suttatitleroot Sāmaṇḍakasutta}}
\addcontentsline{toc}{section}{\tocacronym{SN 39.1–15} \toctranslation{With Sāmaṇḍaka on Extinguishment } \tocroot{Sāmaṇḍakasutta}}
\markboth{With Sāmaṇḍaka on Extinguishment }{Sāmaṇḍakasutta}
\extramarks{SN 39.1–15}{SN 39.1–15}

At\marginnote{1.1} one time Venerable \textsanskrit{Sāriputta} was staying in the land of the Vajjis near \textsanskrit{Ukkacelā} on the bank of the Ganges river. Then the wanderer \textsanskrit{Sāmaṇḍaka} went up to Venerable \textsanskrit{Sāriputta} and exchanged greetings with him. When the greetings and polite conversation were over, he sat down to one side and said to \textsanskrit{Sāriputta}: 

“Reverend\marginnote{2.1} \textsanskrit{Sāriputta}, they speak of this thing called ‘extinguishment’. What is extinguishment?” 

“Reverend,\marginnote{2.3} the ending of greed, hate, and delusion is called extinguishment.” 

“But,\marginnote{2.5} reverend, is there a path and a practice for realizing this extinguishment?” 

“There\marginnote{2.6} is, reverend.” 

“Well,\marginnote{3.1} what is it?” 

“It\marginnote{3.2} is simply this noble eightfold path, that is: right view, right thought, right speech, right action, right livelihood, right effort, right mindfulness, and right immersion. This is the path, the practice, for realizing this extinguishment.” 

“Reverend,\marginnote{3.5} this is a fine path, a fine practice, for realizing this extinguishment. Just this much is enough to be diligent.” 

(These\marginnote{4.1} should be expanded in the same way as the Linked Discourses with \textsanskrit{Jambukhādaka}.) 

%
\section*{{\suttatitleacronym SN 39.16}{\suttatitletranslation Hard to Do }{\suttatitleroot Dukkarasutta}}
\addcontentsline{toc}{section}{\tocacronym{SN 39.16} \toctranslation{Hard to Do } \tocroot{Dukkarasutta}}
\markboth{Hard to Do }{Dukkarasutta}
\extramarks{SN 39.16}{SN 39.16}

“Reverend\marginnote{1.1} \textsanskrit{Sāriputta}, in this teaching and training, what is hard to do?” 

“Going\marginnote{1.2} forth, reverend, is hard to do in this teaching and training.” 

“But\marginnote{1.3} what’s hard to do for someone who has gone forth?” 

“When\marginnote{1.4} you’ve gone forth it’s hard to be satisfied.” 

“But\marginnote{1.5} what’s hard to do for someone who is satisfied?” 

“When\marginnote{1.6} you’re satisfied, it’s hard to practice in line with the teaching.” 

“But\marginnote{1.7} if a mendicant practices in line with the teaching, will it take them long to become a perfected one?” 

“Not\marginnote{1.8} long, reverend.” 

\scendsutta{The Linked Discourses with \textsanskrit{Sāmaṇḍaka} are complete. }

%
\addtocontents{toc}{\let\protect\contentsline\protect\nopagecontentsline}
\part*{Linked Discourses with Moggallāna }
\addcontentsline{toc}{part}{Linked Discourses with Moggallāna }
\markboth{}{}
\addtocontents{toc}{\let\protect\contentsline\protect\oldcontentsline}

%
\addtocontents{toc}{\let\protect\contentsline\protect\nopagecontentsline}
\chapter*{The Chapter with Moggallāna }
\addcontentsline{toc}{chapter}{\tocchapterline{The Chapter with Moggallāna }}
\addtocontents{toc}{\let\protect\contentsline\protect\oldcontentsline}

%
\section*{{\suttatitleacronym SN 40.1}{\suttatitletranslation A Question About the First Absorption }{\suttatitleroot Paṭhamajhānapañhāsutta}}
\addcontentsline{toc}{section}{\tocacronym{SN 40.1} \toctranslation{A Question About the First Absorption } \tocroot{Paṭhamajhānapañhāsutta}}
\markboth{A Question About the First Absorption }{Paṭhamajhānapañhāsutta}
\extramarks{SN 40.1}{SN 40.1}

At\marginnote{1.1} one time Venerable \textsanskrit{Mahāmoggallāna} was staying near \textsanskrit{Sāvatthī} in Jeta’s Grove, \textsanskrit{Anāthapiṇḍika}’s monastery. There Venerable \textsanskrit{Mahāmoggallāna} addressed the mendicants: “Reverends, mendicants!” 

“Reverend,”\marginnote{1.4} they replied. Venerable \textsanskrit{Mahāmoggallāna} said this: 

“Just\marginnote{2.1} now, reverends, as I was in private retreat this thought came to mind: ‘They speak of this thing called the “first absorption”. What is the first absorption?’ It occurred to me: ‘It’s when a mendicant, quite secluded from sensual pleasures, secluded from unskillful qualities, enters and remains in the first absorption, which has the rapture and bliss born of seclusion, while placing the mind and keeping it connected. This is called the first absorption.’ 

And\marginnote{2.7} so … I was entering and remaining in the first absorption. While I was in that meditation, perceptions and attentions accompanied by sensual pleasures beset me. 

Then\marginnote{3.1} the Buddha came up to me with his psychic power and said, ‘\textsanskrit{Moggallāna}, \textsanskrit{Moggallāna}! Don’t neglect the first absorption, brahmin! Settle your mind in the first absorption; unify your mind and immerse it in the first absorption.’ 

And\marginnote{3.4} so, after some time … I entered and remained in the first absorption. 

So\marginnote{3.5} if anyone should be rightly called a disciple who attained to great direct knowledge with help from the Teacher, it’s me.” 

%
\section*{{\suttatitleacronym SN 40.2}{\suttatitletranslation A Question About the Second Absorption }{\suttatitleroot Dutiyajhānapañhāsutta}}
\addcontentsline{toc}{section}{\tocacronym{SN 40.2} \toctranslation{A Question About the Second Absorption } \tocroot{Dutiyajhānapañhāsutta}}
\markboth{A Question About the Second Absorption }{Dutiyajhānapañhāsutta}
\extramarks{SN 40.2}{SN 40.2}

“They\marginnote{1.1} speak of this thing called the 'second absorption'. What is the second absorption? It occurred to me: ‘As the placing of the mind and keeping it connected are stilled, a mendicant enters and remains in the second absorption, which has the rapture and bliss born of immersion, with internal clarity and confidence, and unified mind, without placing the mind and keeping it connected. This is called the second absorption.’ 

And\marginnote{1.6} so … I was entering and remaining in the second absorption. While I was in that meditation, perceptions and attentions accompanied by placing the mind beset me. 

Then\marginnote{2.1} the Buddha came up to me with his psychic power and said, ‘\textsanskrit{Moggallāna}, \textsanskrit{Moggallāna}! Don’t neglect the second absorption, brahmin! Settle your mind in the second absorption; unify your mind and immerse it in the second absorption.’ 

And\marginnote{2.4} so, after some time … I entered and remained in the second absorption. 

So\marginnote{2.5} if anyone should be rightly called a disciple who attained to great direct knowledge with help from the Teacher, it’s me.” 

%
\section*{{\suttatitleacronym SN 40.3}{\suttatitletranslation A Question About the Third Absorption }{\suttatitleroot Tatiyajhānapañhāsutta}}
\addcontentsline{toc}{section}{\tocacronym{SN 40.3} \toctranslation{A Question About the Third Absorption } \tocroot{Tatiyajhānapañhāsutta}}
\markboth{A Question About the Third Absorption }{Tatiyajhānapañhāsutta}
\extramarks{SN 40.3}{SN 40.3}

“They\marginnote{1.1} speak of this thing called the ‘third absorption’. What is the third absorption? It occurred to me: ‘With the fading away of rapture, a mendicant enters and remains in the third absorption, where they meditate with equanimity, mindful and aware, personally experiencing the bliss of which the noble ones declare, “Equanimous and mindful, one meditates in bliss.” This is called the third absorption.’ 

And\marginnote{1.6} so … I was entering and remaining in the third absorption. While I was in that meditation, perceptions and attentions accompanied by rapture beset me. 

Then\marginnote{2.1} the Buddha came up to me with his psychic power and said, ‘\textsanskrit{Moggallāna}, \textsanskrit{Moggallāna}! Don’t neglect the third absorption, brahmin! Settle your mind in the third absorption; unify your mind and immerse it in the third absorption.’ 

And\marginnote{2.4} so, after some time … I entered and remained in the third absorption. So if anyone should be rightly called a disciple who attained to great direct knowledge with help from the Teacher, it’s me.” 

%
\section*{{\suttatitleacronym SN 40.4}{\suttatitletranslation A Question About the Fourth Absorption }{\suttatitleroot Catutthajhānapañhāsutta}}
\addcontentsline{toc}{section}{\tocacronym{SN 40.4} \toctranslation{A Question About the Fourth Absorption } \tocroot{Catutthajhānapañhāsutta}}
\markboth{A Question About the Fourth Absorption }{Catutthajhānapañhāsutta}
\extramarks{SN 40.4}{SN 40.4}

“They\marginnote{1.1} speak of this thing called the ‘fourth absorption’. What is the fourth absorption? It occurred to me: ‘It’s when, giving up pleasure and pain, and ending former happiness and sadness, a mendicant enters and remains in the fourth absorption, without pleasure or pain, with pure equanimity and mindfulness. This is called the fourth absorption.’ 

And\marginnote{1.6} so … I was entering and remaining in the fourth absorption. While I was in that meditation, perceptions and attentions accompanied by pleasure beset me. 

Then\marginnote{2.1} the Buddha came up to me with his psychic power and said, ‘\textsanskrit{Moggallāna}, \textsanskrit{Moggallāna}! Don’t neglect the fourth absorption, brahmin! Settle your mind in the fourth absorption; unify your mind and immerse it in the fourth absorption.’ 

And\marginnote{2.4} so, after some time … I entered and remained in the fourth absorption. 

So\marginnote{2.5} if anyone should be rightly called a disciple who attained to great direct knowledge with help from the Teacher, it’s me.” 

%
\section*{{\suttatitleacronym SN 40.5}{\suttatitletranslation A Question About the Dimension of Infinite Space }{\suttatitleroot Ākāsānañcāyatanapañhāsutta}}
\addcontentsline{toc}{section}{\tocacronym{SN 40.5} \toctranslation{A Question About the Dimension of Infinite Space } \tocroot{Ākāsānañcāyatanapañhāsutta}}
\markboth{A Question About the Dimension of Infinite Space }{Ākāsānañcāyatanapañhāsutta}
\extramarks{SN 40.5}{SN 40.5}

“They\marginnote{1.1} speak of this thing called the ‘dimension of infinite space’. What is the dimension of infinite space? It occurred to me: ‘It’s when a mendicant—going totally beyond perceptions of form, with the ending of perceptions of impingement, not focusing on perceptions of diversity—aware that “space is infinite”, enters and remains in the dimension of infinite space. This is called the dimension of infinite space.’ 

And\marginnote{1.6} so … I was entering and remaining in the dimension of infinite space. While I was in that meditation, perceptions and attentions accompanied by forms beset me. 

Then\marginnote{2.1} the Buddha came up to me with his psychic power and said, ‘\textsanskrit{Moggallāna}, \textsanskrit{Moggallāna}! Don’t neglect the dimension of infinite space, brahmin! Settle your mind in the dimension of infinite space; unify your mind and immerse it in the dimension of infinite space.’ 

And\marginnote{2.4} so, after some time … I entered and remained in the dimension of infinite space. 

So\marginnote{2.5} if anyone should be rightly called a disciple who attained to great direct knowledge with help from the Teacher, it’s me.” 

%
\section*{{\suttatitleacronym SN 40.6}{\suttatitletranslation A Question About the Dimension of Infinite Consciousness }{\suttatitleroot Viññāṇañcāyatanapañhāsutta}}
\addcontentsline{toc}{section}{\tocacronym{SN 40.6} \toctranslation{A Question About the Dimension of Infinite Consciousness } \tocroot{Viññāṇañcāyatanapañhāsutta}}
\markboth{A Question About the Dimension of Infinite Consciousness }{Viññāṇañcāyatanapañhāsutta}
\extramarks{SN 40.6}{SN 40.6}

“They\marginnote{1.1} speak of this thing called the ‘dimension of infinite consciousness’. What is the dimension of infinite consciousness? It occurred to me: ‘It’s when a mendicant, going totally beyond the dimension of infinite space, aware that “consciousness is infinite”, enters and remains in the dimension of infinite consciousness. This is called the dimension of infinite consciousness.’ 

And\marginnote{1.6} so … I was entering and remaining in the dimension of infinite consciousness. While I was in that meditation, perceptions and attentions accompanied by the dimension of infinite space beset me. 

Then\marginnote{2.1} the Buddha came up to me with his psychic power and said, ‘\textsanskrit{Moggallāna}, \textsanskrit{Moggallāna}! Don’t neglect the dimension of infinite consciousness, brahmin! Settle your mind in the dimension of infinite consciousness; unify your mind and immerse it in the dimension of infinite consciousness.’ 

And\marginnote{2.4} so, after some time … I entered and remained in the dimension of infinite consciousness. 

So\marginnote{2.5} if anyone should be rightly called a disciple who attained to great direct knowledge with help from the Teacher, it’s me.” 

%
\section*{{\suttatitleacronym SN 40.7}{\suttatitletranslation A Question About the Dimension of Nothingness }{\suttatitleroot Ākiñcaññāyatanapañhāsutta}}
\addcontentsline{toc}{section}{\tocacronym{SN 40.7} \toctranslation{A Question About the Dimension of Nothingness } \tocroot{Ākiñcaññāyatanapañhāsutta}}
\markboth{A Question About the Dimension of Nothingness }{Ākiñcaññāyatanapañhāsutta}
\extramarks{SN 40.7}{SN 40.7}

“They\marginnote{1.1} speak of this thing called the ‘dimension of nothingness’. What is the dimension of nothingness? It occurred to me: ‘It’s when a mendicant, going totally beyond the dimension of infinite consciousness, aware that “there is nothing at all”, enters and remains in the dimension of nothingness. This is called the dimension of nothingness.’ 

And\marginnote{1.6} so … I was entering and remaining in the dimension of nothingness. While I was in that meditation, perceptions and attentions accompanied by the dimension of infinite consciousness beset me. 

Then\marginnote{2.1} the Buddha came up to me with his psychic power and said, ‘\textsanskrit{Moggallāna}, \textsanskrit{Moggallāna}! Don’t neglect the dimension of nothingness, brahmin! Settle your mind in the dimension of nothingness; unify your mind and immerse it in the dimension of nothingness.’ 

And\marginnote{2.4} so, after some time … I entered and remained in the dimension of nothingness. 

So\marginnote{2.5} if anyone should be rightly called a disciple who attained to great direct knowledge with help from the Teacher, it’s me.” 

%
\section*{{\suttatitleacronym SN 40.8}{\suttatitletranslation A Question About the Dimension of Neither Perception Nor Non-Perception }{\suttatitleroot Nevasaññānāsaññāyatanapañhāsutta}}
\addcontentsline{toc}{section}{\tocacronym{SN 40.8} \toctranslation{A Question About the Dimension of Neither Perception Nor Non-Perception } \tocroot{Nevasaññānāsaññāyatanapañhāsutta}}
\markboth{A Question About the Dimension of Neither Perception Nor Non-Perception }{Nevasaññānāsaññāyatanapañhāsutta}
\extramarks{SN 40.8}{SN 40.8}

“They\marginnote{1.1} speak of this thing called the ‘dimension of neither perception nor non-perception’. What is the dimension of neither perception nor non-perception? It occurred to me: ‘It’s when a mendicant, going totally beyond the dimension of nothingness, enters and remains in the dimension of neither perception nor non-perception. This is called the dimension of neither perception nor non-perception.’ 

And\marginnote{1.6} so … I was entering and remaining in the dimension of neither perception nor non-perception. While I was in that meditation, perceptions and attentions accompanied by the dimension of nothingness beset me. 

Then\marginnote{2.1} the Buddha came up to me with his psychic power and said, ‘\textsanskrit{Moggallāna}, \textsanskrit{Moggallāna}! Don’t neglect the dimension of neither perception nor non-perception, brahmin! Settle your mind in the dimension of neither perception nor non-perception; unify your mind and immerse it in the dimension of neither perception nor non-perception.’ 

And\marginnote{2.4} so, after some time … I entered and remained in the dimension of neither perception nor non-perception. 

So\marginnote{2.5} if anyone should be rightly called a disciple who attained to great direct knowledge with help from the Teacher, it’s me.” 

%
\section*{{\suttatitleacronym SN 40.9}{\suttatitletranslation A Question About the Signless }{\suttatitleroot Animittapañhāsutta}}
\addcontentsline{toc}{section}{\tocacronym{SN 40.9} \toctranslation{A Question About the Signless } \tocroot{Animittapañhāsutta}}
\markboth{A Question About the Signless }{Animittapañhāsutta}
\extramarks{SN 40.9}{SN 40.9}

“They\marginnote{1.1} speak of this thing called the ‘signless immersion of the heart’. What is the signless immersion of the heart? It occurred to me: ‘It’s when a mendicant, not focusing on any signs, enters and remains in the signless immersion of the heart. This is called the signless immersion of the heart.’ 

And\marginnote{1.6} so … I was entering and remaining in the signless immersion of the heart. While I was in that meditation, my consciousness followed after signs. 

Then\marginnote{2.1} the Buddha came up to me with his psychic power and said, ‘\textsanskrit{Moggallāna}, \textsanskrit{Moggallāna}! Don’t neglect the signless immersion of the heart, brahmin! Settle your mind in the signless immersion of the heart; unify your mind and immerse it in the signless immersion of the heart.’ 

And\marginnote{2.4} so, after some time … I entered and remained in the signless immersion of the heart. 

So\marginnote{2.5} if anyone should be rightly called a disciple who attained to great direct knowledge with help from the Teacher, it’s me.” 

%
\section*{{\suttatitleacronym SN 40.10}{\suttatitletranslation With Sakka }{\suttatitleroot Sakkasutta}}
\addcontentsline{toc}{section}{\tocacronym{SN 40.10} \toctranslation{With Sakka } \tocroot{Sakkasutta}}
\markboth{With Sakka }{Sakkasutta}
\extramarks{SN 40.10}{SN 40.10}

And\marginnote{1.1} then Venerable \textsanskrit{Mahāmoggallāna}, as easily as a strong person would extend or contract their arm, vanished from Jeta’s Grove and reappeared among the gods of the Thirty-Three. Then Sakka, lord of gods, with five hundred deities came up to \textsanskrit{Mahāmoggallāna}, bowed, and stood to one side. \textsanskrit{Mahāmoggallāna} said to him: 

“Lord\marginnote{2.1} of gods, it’s good to go for refuge to the Buddha. It’s the reason why some sentient beings, when their body breaks up, after death, are reborn in a good place, a heavenly realm. It’s good to go for refuge to the teaching. It’s the reason why some sentient beings, when their body breaks up, after death, are reborn in a good place, a heavenly realm. It’s good to go for refuge to the \textsanskrit{Saṅgha}. It’s the reason why some sentient beings, when their body breaks up, after death, are reborn in a good place, a heavenly realm.” 

“My\marginnote{3.1} good \textsanskrit{Moggallāna}, it’s good to go for refuge to the Buddha … the teaching … the \textsanskrit{Saṅgha}. It’s the reason why some sentient beings, when their body breaks up, after death, are reborn in a good place, a heavenly realm.” 

Then\marginnote{4.1} Sakka, lord of gods, with six hundred deities … 

Then\marginnote{4.2} Sakka, lord of gods, with seven hundred deities … 

Then\marginnote{4.3} Sakka, lord of gods, with eight hundred deities … 

Then\marginnote{4.4} Sakka, lord of gods, with eighty thousand deities … 

Then\marginnote{7.1} Sakka, lord of gods, with five hundred deities came up to \textsanskrit{Mahāmoggallāna}, bowed, and stood to one side. \textsanskrit{Mahāmoggallāna} said to him: 

“Lord\marginnote{8.1} of gods, it’s good to have experiential confidence in the Buddha: ‘That Blessed One is perfected, a fully awakened Buddha, accomplished in knowledge and conduct, holy, knower of the world, supreme guide for those who wish to train, teacher of gods and humans, awakened, blessed.’ It’s the reason why some sentient beings, when their body breaks up, after death, are reborn in a good place, a heavenly realm. 

It’s\marginnote{9.1} good to have experiential confidence in the teaching: ‘The teaching is well explained by the Buddha—visible in this very life, immediately effective, inviting inspection, relevant, so that sensible people can know it for themselves.’ It’s the reason why some sentient beings, when their body breaks up, after death, are reborn in a good place, a heavenly realm. 

It’s\marginnote{10.1} good to have experiential confidence in the \textsanskrit{Saṅgha}: ‘The \textsanskrit{Saṅgha} of the Buddha’s disciples is practicing the way that’s good, direct, methodical, and proper. It consists of the four pairs, the eight individuals. This is the \textsanskrit{Saṅgha} of the Buddha’s disciples that is worthy of offerings dedicated to the gods, worthy of hospitality, worthy of a religious donation, worthy of greeting with joined palms, and is the supreme field of merit for the world.’ It’s the reason why some sentient beings, when their body breaks up, after death, are reborn in a good place, a heavenly realm. 

It’s\marginnote{11.1} good to have the ethical conduct that’s loved by the noble ones, unbroken, impeccable, spotless, and unmarred, liberating, praised by sensible people, not mistaken, and leading to immersion. It’s the reason why some sentient beings, when their body breaks up, after death, are reborn in a good place, a heavenly realm.” 

“My\marginnote{12.1} good \textsanskrit{Moggallāna}, it’s good to have experiential confidence in the Buddha … the teaching … the \textsanskrit{Saṅgha} … and to have the ethical conduct that’s loved by the noble ones … It’s the reason why some sentient beings, when their body breaks up, after death, are reborn in a good place, a heavenly realm.” 

Then\marginnote{16.1} Sakka, lord of gods, with six hundred deities … 

Then\marginnote{16.2} Sakka, lord of gods, with seven hundred deities … 

Then\marginnote{16.3} Sakka, lord of gods, with eight hundred deities … 

Then\marginnote{16.4} Sakka, lord of gods, with eighty thousand deities … 

Then\marginnote{25.1} Sakka, lord of gods, with five hundred deities came up to \textsanskrit{Mahāmoggallāna}, bowed, and stood to one side. \textsanskrit{Mahāmoggallāna} said to him: 

“Lord\marginnote{26.1} of gods, it’s good to go for refuge to the Buddha. It’s the reason why some sentient beings, when their body breaks up, after death, are reborn in a good place, a heavenly realm. They surpass other gods in ten respects: divine life span, beauty, happiness, glory, sovereignty, sights, sounds, smells, tastes, and touches. 

It’s\marginnote{27.1} good to go for refuge to the teaching … 

It’s\marginnote{28.1} good to go for refuge to the \textsanskrit{Saṅgha}. It’s the reason why some sentient beings, when their body breaks up, after death, are reborn in a good place, a heavenly realm. They surpass other gods in ten respects: divine life span, beauty, happiness, glory, sovereignty, sights, sounds, smells, tastes, and touches.” 

“My\marginnote{29.1} good \textsanskrit{Moggallāna}, it’s good to go for refuge to the Buddha …” 

Then\marginnote{32.1} Sakka, lord of gods, with six hundred deities … 

Then\marginnote{32.2} Sakka, lord of gods, with seven hundred deities … 

Then\marginnote{32.3} Sakka, lord of gods, with eight hundred deities … 

Then\marginnote{32.4} Sakka, lord of gods, with eighty thousand deities … 

Then\marginnote{37.1} Sakka, lord of gods, with five hundred deities came up to \textsanskrit{Mahāmoggallāna}, bowed, and stood to one side. \textsanskrit{Mahāmoggallāna} said to him: 

“Lord\marginnote{38.1} of gods, it’s good to have experiential confidence in the Buddha: ‘That Blessed One is perfected, a fully awakened Buddha … teacher of gods and humans, awakened, blessed.’ It’s the reason why some sentient beings, when their body breaks up, after death, are reborn in a good place, a heavenly realm. They surpass other gods in ten respects: divine life span, beauty, happiness, glory, sovereignty, sights, sounds, smells, tastes, and touches. 

It’s\marginnote{39.1} good to have experiential confidence in the teaching … 

It’s\marginnote{40.1} good to have experiential confidence in the \textsanskrit{Saṅgha} … 

It’s\marginnote{41.1} good to have the ethical conduct that’s loved by the noble ones …” 

“My\marginnote{42.1} good \textsanskrit{Moggallāna}, it’s good to have experiential confidence in the Buddha …” 

Then\marginnote{46.1} Sakka, lord of gods, with six hundred deities … 

Then\marginnote{46.2} Sakka, lord of gods, with seven hundred deities … 

Then\marginnote{46.3} Sakka, lord of gods, with eight hundred deities … 

Then\marginnote{46.4} Sakka, lord of gods, with eighty thousand deities … 

%
\section*{{\suttatitleacronym SN 40.11}{\suttatitletranslation With Candana, Etc. }{\suttatitleroot Candanasutta}}
\addcontentsline{toc}{section}{\tocacronym{SN 40.11} \toctranslation{With Candana, Etc. } \tocroot{Candanasutta}}
\markboth{With Candana, Etc. }{Candanasutta}
\extramarks{SN 40.11}{SN 40.11}

Then\marginnote{1.1} the god Candana … 

the\marginnote{2.1} god \textsanskrit{Suyāma} … 

the\marginnote{3.1} god Santusita … 

the\marginnote{4.1} god Sunimmita … 

the\marginnote{5.1} god Vasavatti … 

(These\marginnote{6.1} abbreviated texts should be expanded as in the Discourse With Sakka.) 

\scendsutta{The Linked Discourses on \textsanskrit{Moggallāna} are complete. }

%
\addtocontents{toc}{\let\protect\contentsline\protect\nopagecontentsline}
\part*{Linked Discourses with Citta the Householder }
\addcontentsline{toc}{part}{Linked Discourses with Citta the Householder }
\markboth{}{}
\addtocontents{toc}{\let\protect\contentsline\protect\oldcontentsline}

%
\addtocontents{toc}{\let\protect\contentsline\protect\nopagecontentsline}
\chapter*{The Chapter with Citta }
\addcontentsline{toc}{chapter}{\tocchapterline{The Chapter with Citta }}
\addtocontents{toc}{\let\protect\contentsline\protect\oldcontentsline}

%
\section*{{\suttatitleacronym SN 41.1}{\suttatitletranslation The Fetter }{\suttatitleroot Saṁyojanasutta}}
\addcontentsline{toc}{section}{\tocacronym{SN 41.1} \toctranslation{The Fetter } \tocroot{Saṁyojanasutta}}
\markboth{The Fetter }{Saṁyojanasutta}
\extramarks{SN 41.1}{SN 41.1}

At\marginnote{1.1} one time several senior mendicants were staying near \textsanskrit{Macchikāsaṇḍa} in the Wild Mango Grove. Now at that time, after the meal, on return from almsround, several senior mendicants sat together in the pavilion and this discussion came up among them: 

“Reverends,\marginnote{1.3} the ‘fetter’ and the ‘things prone to being fettered’: do these things differ in both meaning and phrasing? Or do they mean the same thing, and differ only in the phrasing?” 

Some\marginnote{1.4} senior mendicants answered like this: “Reverends, the ‘fetter’ and the ‘things prone to being fettered’: these things differ in both meaning and phrasing.” 

But\marginnote{1.6} some senior mendicants answered like this: “Reverends, the ‘fetter’ and the ‘things prone to being fettered’ mean the same thing; they differ only in the phrasing.” 

Now\marginnote{2.1} at that time the householder Citta had arrived at Migapathaka on some business. He heard about what those senior mendicants were discussing. 

So\marginnote{3.1} he went up to them, bowed, sat down to one side, and said to them, “Sirs, I heard that you were discussing whether the ‘fetter’ and the ‘things prone to being fettered’ differ in both meaning and phrasing, or whether they mean the same thing, and differ only in the phrasing.” 

“That’s\marginnote{3.7} right, householder.” 

“Sirs,\marginnote{4.1} the ‘fetter’ and the ‘things prone to being fettered’: these things differ in both meaning and phrasing. 

Well\marginnote{4.2} then, sirs, I shall give you a simile. For by means of a simile some sensible people understand the meaning of what is said. 

Suppose\marginnote{4.4} there was a black ox and a white ox yoked by a single harness or yoke. Would it be right to say that the black ox is the yoke of the white ox, or the white ox is the yoke of the black ox?” 

“No,\marginnote{4.7} householder. The black ox is not the yoke of the white ox, nor is the white ox the yoke of the black ox. The yoke there is the single harness or yoke that they’re yoked by.” 

“In\marginnote{4.10} the same way, the eye is not the fetter of sights, nor are sights the fetter of the eye. The fetter there is the desire and greed that arises from the pair of them. 

The\marginnote{4.12} ear … nose … tongue … body … mind is not the fetter of thoughts, nor are thoughts the fetter of the mind. The fetter there is the desire and greed that arises from the pair of them.” 

“You’re\marginnote{4.19} fortunate, householder, so very fortunate, to traverse the Buddha’s deep teachings with the eye of wisdom.” 

%
\section*{{\suttatitleacronym SN 41.2}{\suttatitletranslation Isidatta (1st) }{\suttatitleroot Paṭhamaisidattasutta}}
\addcontentsline{toc}{section}{\tocacronym{SN 41.2} \toctranslation{Isidatta (1st) } \tocroot{Paṭhamaisidattasutta}}
\markboth{Isidatta (1st) }{Paṭhamaisidattasutta}
\extramarks{SN 41.2}{SN 41.2}

At\marginnote{1.1} one time several senior mendicants were staying near \textsanskrit{Macchikāsaṇḍa} in the Wild Mango Grove. 

Then\marginnote{1.2} Citta the householder went up to them, bowed, sat down to one side, and said to them, “Sirs, may the senior mendicants please accept my offering of tomorrow’s meal.” 

They\marginnote{1.4} consented in silence. Knowing that the senior mendicants had consented, Citta got up from his seat, bowed, and respectfully circled them, keeping them on his right, before leaving. 

Then\marginnote{1.6} when the night had passed, the senior mendicants robed up in the morning and, taking their bowls and robes, went to Citta’s home, and sat down on the seats spread out. 

Then\marginnote{2.1} Citta went up to them, bowed, sat down to one side, and asked the senior venerable, “Sir, they speak of ‘the diversity of elements’. In what way did the Buddha speak of the diversity of elements?” 

When\marginnote{2.4} he said this, the senior venerable kept silent. 

For\marginnote{2.5} a second time … 

And\marginnote{2.9} for a third time, Citta asked him, “Sir, they speak of ‘the diversity of elements’. In what way did the Buddha speak of the diversity of elements?” 

And\marginnote{2.12} a second time and a third time the senior venerable kept silent. 

Now\marginnote{3.1} at that time Venerable Isidatta was the most junior mendicant in that \textsanskrit{Saṅgha}. He said to the senior venerable, “Sir, may I answer Citta’s question?” 

“Answer\marginnote{3.4} it, Reverend Isidatta.” 

“Householder,\marginnote{3.5} is this your question: ‘They speak of “the diversity of elements”. In what way did the Buddha speak of the diversity of elements?’” 

“Yes,\marginnote{3.8} sir.” 

“This\marginnote{3.9} is the diversity of elements spoken of by the Buddha. 

The\marginnote{3.10} eye element, the sights element, the eye consciousness element … 

The\marginnote{3.11} mind element, the thought element, the mind consciousness element. 

This\marginnote{3.12} is how the Buddha spoke of the diversity of elements.” 

Then\marginnote{4.1} Citta, having approved and agreed with what Isidatta said, served and satisfied the senior mendicants with his own hands with a variety of delicious foods. When the senior mendicants had eaten and washed their hands and bowls, they got up from their seats and left. 

Then\marginnote{4.3} the senior venerable said to Venerable Isidatta, “Isidatta, it’s good that you felt inspired to answer that question, because I didn’t. So when a similar question comes up, you should also answer it as you feel inspired.” 

%
\section*{{\suttatitleacronym SN 41.3}{\suttatitletranslation With Isidatta (2nd) }{\suttatitleroot Dutiyaisidattasutta}}
\addcontentsline{toc}{section}{\tocacronym{SN 41.3} \toctranslation{With Isidatta (2nd) } \tocroot{Dutiyaisidattasutta}}
\markboth{With Isidatta (2nd) }{Dutiyaisidattasutta}
\extramarks{SN 41.3}{SN 41.3}

At\marginnote{1.1} one time several senior mendicants were staying near \textsanskrit{Macchikāsaṇḍa} in the Wild Mango Grove. 

Then\marginnote{1.2} Citta the householder went up to them, bowed, sat down to one side, and said to them, “Sirs, may the senior mendicants please accept my offering of tomorrow’s meal.” 

They\marginnote{1.4} consented in silence. Then, knowing that the senior mendicants had consented, Citta got up from his seat, bowed, and respectfully circled them, keeping them on his right, before leaving. 

Then\marginnote{1.6} when the night had passed, the senior mendicants robed up in the morning and, taking their bowls and robes, went to Citta’s home, and sat down on the seats spread out. 

So\marginnote{2.1} he went up to them, bowed, sat down to one side, and said to the senior venerable: 

“Sir,\marginnote{2.2} there are many different views that arise in the world. For example: the cosmos is eternal, or not eternal, or finite, or infinite; the soul and the body are the same thing, or they are different things; after death, a Realized One exists, or doesn’t exist, or both exists and doesn’t exist, or neither exists nor doesn’t exist. And also the sixty-two misconceptions spoken of in the Prime Net Discourse. When what exists do these views come to be? When what doesn’t exist do these views not come to be?” 

When\marginnote{3.1} he said this, the senior venerable kept silent. 

For\marginnote{3.2} a second time … 

And\marginnote{3.3} for a third time, Citta said to him: 

“Sir,\marginnote{3.4} there are many different views that arise in the world. … When what exists do these views come to be? When what doesn’t exist do these views not come to be?” 

And\marginnote{3.8} a second time and a third time the senior venerable kept silent. 

Now\marginnote{4.1} at that time Venerable Isidatta was the most junior mendicant in that \textsanskrit{Saṅgha}. He said to the senior venerable, “Sir, may I answer Citta’s question?” 

“Answer\marginnote{4.4} it, Reverend Isidatta.” 

“Householder,\marginnote{4.5} is this your question: ‘There are many different views that arise in the world … When what exists do these views come to be? When what doesn’t exist do these views not come to be?’’ “Yes, sir.” 

“Householder,\marginnote{4.10} there are many different views that arise in the world. For example: the cosmos is eternal, or not eternal, or finite, or infinite; the soul and the body are the same thing, or they are different things; after death, a Realized One exists, or doesn’t exist, or both exists and doesn’t exist, or neither exists nor doesn’t exist. And also the sixty-two misconceptions spoken of in the Prime Net Discourse. 

These\marginnote{4.13} views come to be when identity view exists. When identity view does not exist they do not come to be.” 

“But\marginnote{5.1} sir, how does identity view come about?” 

“It’s\marginnote{5.2} when an unlearned ordinary person has not seen the noble ones, and is neither skilled nor trained in the teaching of the noble ones. They’ve not seen good persons, and are neither skilled nor trained in the teaching of the good persons. 

They\marginnote{5.3} regard form as self, self as having form, form in self, or self in form. They regard feeling … perception … choices … consciousness as self, self as having consciousness, consciousness in self, or self in consciousness. 

That’s\marginnote{5.8} how identity view comes about.” 

“But\marginnote{6.1} sir, how does identity view not come about?” 

“It’s\marginnote{6.2} when a learned noble disciple has seen the noble ones, and is skilled and trained in the teaching of the noble ones. They’ve seen good persons, and are skilled and trained in the teaching of the good persons. 

They\marginnote{6.3} don’t regard form as self, self as having form, form in self, or self in form. They don’t regard feeling … perception … choices … consciousness as self, self as having consciousness, consciousness in self, or self in consciousness. 

That’s\marginnote{6.8} how identity view does not come about.” 

“Sir,\marginnote{7.1} where has Venerable Isidatta come from?” 

“I\marginnote{7.2} come from Avanti, householder.” 

“Sir,\marginnote{7.3} there’s a friend of mine called Isidatta who I’ve never met. He’s gone forth from a good family in Avanti. Have you met him?” 

“Yes,\marginnote{7.5} householder.” 

“Sir,\marginnote{7.6} where is that venerable now staying?” When he said this, Isidatta kept silent. 

“Sir,\marginnote{7.8} are you that Isidatta?” 

“Yes,\marginnote{7.9} householder.” 

“Sir,\marginnote{7.10} I hope Venerable Isidatta is happy here in \textsanskrit{Macchikāsaṇḍa}, for the Wild Mango Grove is lovely. I’ll make sure that Venerable Isidatta is provided with robes, almsfood, lodgings, and medicines and supplies for the sick.” 

“That’s\marginnote{7.13} nice of you to say, householder.” 

Then\marginnote{8.1} Citta, having approved and agreed with what Isidatta said, served and satisfied the senior mendicants with his own hands with a variety of delicious foods. When the senior mendicants had eaten and washed their hands and bowls, they got up from their seats and left. 

Then\marginnote{8.3} the senior venerable said to Venerable Isidatta, “Isidatta, it’s good that you felt inspired to answer that question, because I didn’t. So when a similar question comes up, you should also answer it as you feel inspired.” 

But\marginnote{8.7} Isidatta set his lodgings in order and, taking his bowl and robe, left \textsanskrit{Macchikasaṇḍa}, never to return. 

%
\section*{{\suttatitleacronym SN 41.4}{\suttatitletranslation Mahaka’s Demonstration }{\suttatitleroot Mahakapāṭihāriyasutta}}
\addcontentsline{toc}{section}{\tocacronym{SN 41.4} \toctranslation{Mahaka’s Demonstration } \tocroot{Mahakapāṭihāriyasutta}}
\markboth{Mahaka’s Demonstration }{Mahakapāṭihāriyasutta}
\extramarks{SN 41.4}{SN 41.4}

At\marginnote{1.1} one time several senior mendicants were staying near \textsanskrit{Macchikāsaṇḍa} in the Wild Mango Grove. 

Then\marginnote{1.2} Citta the householder went up to them, bowed, sat down to one side, and said to them, “Sirs, may the senior mendicants please accept my offering of tomorrow’s meal in my barn.” 

They\marginnote{1.4} consented in silence. Then, knowing that the senior mendicants had consented, Citta got up from his seat, bowed, and respectfully circled them, keeping them on his right, before leaving. 

Then\marginnote{1.6} when the night had passed, the senior mendicants robed up in the morning and, taking their bowls and robes, went to Citta’s barn, and sat down on the seats spread out. 

Then\marginnote{2.1} Citta served and satisfied the senior mendicants with his own hands with delicious milk-rice made with ghee. When the senior mendicants had eaten and washed their hands and bowls, they got up from their seats and left. Citta instructed that the remainder of the food be distributed, then followed behind the senior mendicants. 

Now\marginnote{2.4} at that time the heat was sweltering. And those senior mendicants walked along as if their bodies were melting, as happens after a meal. 

Now\marginnote{3.1} at that time Venerable Mahaka was the most junior mendicant in that \textsanskrit{Saṅgha}. Then Venerable Mahaka said to the senior venerable, “Wouldn’t it be nice, sir, if a cool wind blew, a cloud canopy formed, and a gentle rain drizzled down?” 

“It\marginnote{4.1} would indeed be nice, Reverend Mahaka.” Then Mahaka used his psychic power to will that a cool wind would blow, a cloud canopy would form, and a gentle rain would drizzle down. 

Then\marginnote{4.3} Citta thought, “The most junior mendicant in this \textsanskrit{Saṅgha} has such psychic power!” 

When\marginnote{4.5} they reached the monastery, Mahaka said to the senior venerable, “Sir, is that sufficient?” 

“That’s\marginnote{4.7} sufficient, Reverend Mahaka, you’ve done enough and offered enough.” Then the senior mendicants entered their dwellings, and Mahaka entered his own dwelling. 

Then\marginnote{5.1} Citta went up to Mahaka, bowed, sat down to one side, and said to him, “Sir, please show me a superhuman demonstration of psychic power.” 

“Well,\marginnote{5.3} then, householder, place your upper robe on the porch and spread a handful of grass on it.” 

“Yes,\marginnote{5.4} sir,” replied Citta, and did as he was asked. 

Mahaka\marginnote{5.5} entered his dwelling and latched the door. Then he used his psychic power to will that a flame shoot out through the keyhole and the chink in the door, and it burned up the grass but not the upper robe. Then Citta shook out his upper robe and stood to one side, shocked and awestruck. 

Mahaka\marginnote{5.7} left his dwelling and said to Citta, “Is that sufficient, householder?” 

“That’s\marginnote{6.1} sufficient, sir, you’ve done enough and offered enough. I hope Venerable Mahaka is happy here in \textsanskrit{Macchikāsaṇḍa}, for the Wild Mango Grove is lovely. I’ll make sure that Venerable Mahaka is provided with robes, almsfood, lodgings, and medicines and supplies for the sick.” 

“That’s\marginnote{6.7} nice of you to say, householder.” 

But\marginnote{6.8} Mahaka set his lodgings in order and, taking his bowl and robe, left \textsanskrit{Macchikasaṇḍa}, never to return. 

%
\section*{{\suttatitleacronym SN 41.5}{\suttatitletranslation With Kāmabhū (1st) }{\suttatitleroot Paṭhamakāmabhūsutta}}
\addcontentsline{toc}{section}{\tocacronym{SN 41.5} \toctranslation{With Kāmabhū (1st) } \tocroot{Paṭhamakāmabhūsutta}}
\markboth{With Kāmabhū (1st) }{Paṭhamakāmabhūsutta}
\extramarks{SN 41.5}{SN 41.5}

At\marginnote{1.1} one time Venerable \textsanskrit{Kāmabhū} was staying near \textsanskrit{Macchikāsaṇḍa} in the Wild Mango Grove. 

Then\marginnote{1.2} Citta the householder went up to Venerable \textsanskrit{Kāmabhū}, bowed, and sat down to one side. \textsanskrit{Kāmabhū} said to him, “Householder, there is this saying: 

\begin{verse}%
‘With\marginnote{3.1} flawless wheel and white canopy, \\
the one-spoke chariot rolls on. \\
See it come, untroubled, \\
with stream cut, unbound.’ 

%
\end{verse}

How\marginnote{4.1} should we see the detailed meaning of this brief statement?” 

“Sir,\marginnote{4.2} was this spoken by the Buddha?” 

“Yes,\marginnote{4.3} householder.” 

“Well\marginnote{4.4} then, sir, please wait a moment while I consider the meaning of this.” Then after a short silence Citta said to \textsanskrit{Kāmabhū}: 

“Sir,\marginnote{5.1} ‘flawless wheel’ is a term for ethics. 

‘White\marginnote{5.2} canopy’ is a term for freedom. 

‘One\marginnote{5.3} spoke’ is a term for mindfulness. 

‘Rolls\marginnote{5.4} on’ is a term for going forward and coming back. 

‘Chariot’\marginnote{5.5} is a term for this body made up of the four primary elements, produced by mother and father, built up from rice and porridge, liable to impermanence, to wearing away and erosion, to breaking up and destruction. 

Greed,\marginnote{5.6} hate, and delusion are troubles. A mendicant who has ended the defilements has given these up, cut them off at the root, made them like a palm stump, and obliterated them, so they are unable to arise in the future. That’s why a mendicant who has ended the defilements is called ‘untroubled’. 

‘The\marginnote{5.9} one who comes’ is a term for the perfected one. 

‘Stream’\marginnote{5.10} is a term for craving. A mendicant who has ended the defilements has given this up, cut it off at the root, made it like a palm stump, and obliterated it, so it’s unable to arise in the future. That’s why a mendicant who has ended the defilements is said to have ‘cut the stream’. 

Greed,\marginnote{5.13} hate, and delusion are bonds. A mendicant who has ended the defilements has given these up, cut them off at the root, made them like a palm stump, and obliterated them, so they are unable to arise in the future. That’s why a mendicant who has ended the defilements is called ‘unbound’. 

So,\marginnote{5.16} sir, that’s how I understand the detailed meaning of what the Buddha said in brief: 

\begin{verse}%
‘With\marginnote{6.1} flawless wheel and white canopy, \\
the one-spoke chariot rolls on. \\
See it come, untroubled, \\
with stream cut, unbound.’” 

%
\end{verse}

“You’re\marginnote{7.2} fortunate, householder, so very fortunate, to traverse the Buddha’s deep teachings with the eye of wisdom.” 

%
\section*{{\suttatitleacronym SN 41.6}{\suttatitletranslation With Kāmabhū (2nd) }{\suttatitleroot Dutiyakāmabhūsutta}}
\addcontentsline{toc}{section}{\tocacronym{SN 41.6} \toctranslation{With Kāmabhū (2nd) } \tocroot{Dutiyakāmabhūsutta}}
\markboth{With Kāmabhū (2nd) }{Dutiyakāmabhūsutta}
\extramarks{SN 41.6}{SN 41.6}

At\marginnote{1.1} one time Venerable \textsanskrit{Kāmabhū} was staying near \textsanskrit{Macchikāsaṇḍa} in the Wild Mango Grove. Then Citta the householder went up to Venerable \textsanskrit{Kāmabhū}, sat down to one side, and said to him: 

“Sir,\marginnote{1.3} how many processes are there?” 

“Householder,\marginnote{1.4} there are three processes. Physical, verbal, and mental processes.” 

Saying\marginnote{1.6} “Good, sir,” Citta approved and agreed with what \textsanskrit{Kāmabhū} said. Then he asked another question: 

“But\marginnote{1.7} sir, what is the physical process? What’s the verbal process? What’s the mental process?” 

“Breathing\marginnote{1.8} is a physical process. Placing the mind and keeping it connected are verbal processes. Perception and feeling are mental processes.” 

Saying\marginnote{2.1} “Good, sir,” he asked another question: 

“But\marginnote{2.2} sir, why is breathing a physical process? Why are placing the mind and keeping it connected verbal processes? Why are perception and feeling mental processes?” 

“Breathing\marginnote{2.3} is physical. It’s tied up with the body, that’s why breathing is a physical process. First you place the mind and keep it connected, then you break into speech. That’s why placing the mind and keeping it connected are verbal processes. Perception and feeling are mental. They’re tied up with the mind, that’s why perception and feeling are mental processes.” 

Saying\marginnote{3.1} “Good, sir,” he asked another question: 

“But\marginnote{3.2} sir, how does someone attain the cessation of perception and feeling?” 

“A\marginnote{3.3} mendicant who is entering such an attainment does not think: ‘I will enter the cessation of perception and feeling’ or ‘I am entering the cessation of perception and feeling’ or ‘I have entered the cessation of perception and feeling.’ Rather, their mind has been previously developed so as to lead to such a state.” 

Saying\marginnote{4.1} “Good, sir,” he asked another question: 

“But\marginnote{4.2} sir, which cease first for a mendicant who is entering the cessation of perception and feeling: physical, verbal, or mental processes?” 

“Verbal\marginnote{4.3} processes cease first, then physical, then mental.” 

Saying\marginnote{5.1} “Good, sir,” he asked another question: 

“What’s\marginnote{5.2} the difference between someone who has passed away and a mendicant who has attained the cessation of perception and feeling?” 

“When\marginnote{5.3} someone dies, their physical, verbal, and mental processes have ceased and stilled; their vitality is spent; their warmth is dissipated; and their faculties have disintegrated. When a mendicant has attained the cessation of perception and feeling, their physical, verbal, and mental processes have ceased and stilled. But their vitality is not spent; their warmth is not dissipated; and their faculties are very clear. That’s the difference between someone who has passed away and a mendicant who has attained the cessation of perception and feeling.” 

Saying\marginnote{6.1} “Good, sir,” he asked another question: 

“But\marginnote{6.2} sir, how does someone emerge from the cessation of perception and feeling?” 

“A\marginnote{6.3} mendicant who is emerging from such an attainment does not think: ‘I will emerge from the cessation of perception and feeling’ or ‘I am emerging from the cessation of perception and feeling’ or ‘I have emerged from the cessation of perception and feeling.’ Rather, their mind has been previously developed so as to lead to such a state.” 

Saying\marginnote{7.1} “Good, sir,” he asked another question: 

“But\marginnote{7.2} sir, which arise first for a mendicant who is emerging from the cessation of perception and feeling: physical, verbal, or mental processes?” 

“Mental\marginnote{7.3} processes arise first, then physical, then verbal.” 

Saying\marginnote{8.1} “Good, sir,” he asked another question: 

“But\marginnote{8.2} sir, when a mendicant has emerged from the attainment of the cessation of perception and feeling, how many kinds of contact do they experience?” 

“They\marginnote{8.3} experience three kinds of contact: emptiness, signless, and undirected contacts.” 

Saying\marginnote{9.1} “Good, sir,” he asked another question: 

“But\marginnote{9.2} sir, when a mendicant has emerged from the attainment of the cessation of perception and feeling, what does their mind slant, slope, and incline to?” 

“Their\marginnote{9.3} mind slants, slopes, and inclines to seclusion.” 

Saying\marginnote{10.1} “Good, sir,” Citta approved and agreed with what \textsanskrit{Kāmabhū} said. Then he asked another question: 

“But\marginnote{10.2} sir, how many things are helpful for attaining the cessation of perception and feeling?” 

“Well,\marginnote{10.3} householder, you’ve finally asked what you should have asked first! Nevertheless, I will answer you. Two things are helpful for attaining the cessation of perception and feeling: serenity and discernment.” 

%
\section*{{\suttatitleacronym SN 41.7}{\suttatitletranslation With Godatta }{\suttatitleroot Godattasutta}}
\addcontentsline{toc}{section}{\tocacronym{SN 41.7} \toctranslation{With Godatta } \tocroot{Godattasutta}}
\markboth{With Godatta }{Godattasutta}
\extramarks{SN 41.7}{SN 41.7}

At\marginnote{1.1} one time Venerable Godatta was staying near \textsanskrit{Macchikāsaṇḍa} in the Wild Mango Grove. Then Citta the householder went up to Venerable Godatta, bowed, and sat down to one side. Godatta said to him: 

“Householder,\marginnote{1.3} the limitless release of the heart, and the release of the heart through nothingness, and the release of the heart through emptiness, and the signless release of the heart: do these things differ in both meaning and phrasing? Or do they mean the same thing, and differ only in the phrasing?” 

“Sir,\marginnote{1.4} there is a way in which these things differ in both meaning and phrasing. But there’s also a way in which they mean the same thing, and differ only in the phrasing. 

And\marginnote{2.1} what’s the way in which these things differ in both meaning and phrasing? 

It’s\marginnote{2.2} when a mendicant meditates spreading a heart full of love to one direction, and to the second, and to the third, and to the fourth. In the same way above, below, across, everywhere, all around, they spread a heart full of love to the whole world—abundant, expansive, limitless, free of enmity and ill will. They meditate spreading a heart full of compassion … They meditate spreading a heart full of rejoicing … They meditate spreading a heart full of equanimity to one direction, and to the second, and to the third, and to the fourth. In the same way above, below, across, everywhere, all around, they spread a heart full of equanimity to the whole world—abundant, expansive, limitless, free of enmity and ill will. This is called the limitless release of the heart. 

And\marginnote{3.1} what is the release of the heart through nothingness? It’s when a mendicant, going totally beyond the dimension of infinite consciousness, aware that ‘there is nothing at all’, enters and remains in the dimension of nothingness. This is called the release of the heart through nothingness. 

And\marginnote{4.1} what is the release of the heart through emptiness? It’s when a mendicant has gone to a wilderness, or to the root of a tree, or to an empty hut, and reflects like this: ‘This is empty of a self or what belongs to a self.’ This is called the release of the heart through emptiness. 

And\marginnote{5.1} what is the signless heart’s release? It’s when a mendicant, not focusing on any signs, enters and remains in the signless immersion of the heart. This is called the signless release of the heart. 

This\marginnote{5.4} is the way in which these things differ in both meaning and phrasing. 

And\marginnote{6.1} what’s the way in which they mean the same thing, and differ only in the phrasing? 

Greed,\marginnote{6.2} hate, and delusion are makers of limits. A mendicant who has ended the defilements has given these up, cut them off at the root, made them like a palm stump, and obliterated them, so they are unable to arise in the future. The unshakable release of the heart is said to be the best kind of limitless release of the heart. That unshakable release of the heart is empty of greed, hate, and delusion. 

Greed\marginnote{6.6} is something, hate is something, and delusion is something. A mendicant who has ended the defilements has given these up, cut them off at the root, made them like a palm stump, and obliterated them, so they are unable to arise in the future. The unshakable release of the heart is said to be the best kind of release of the heart through nothingness. That unshakable release of the heart is empty of greed, hate, and delusion. 

Greed,\marginnote{6.10} hate, and delusion are makers of signs. A mendicant who has ended the defilements has given these up, cut them off at the root, made them like a palm stump, and obliterated them, so they are unable to arise in the future. The unshakable release of the heart is said to be the best kind of signless release of the heart. That unshakable release of the heart is empty of greed, hate, and delusion. 

This\marginnote{6.14} is the way in which they mean the same thing, and differ only in the phrasing.” 

“You’re\marginnote{6.15} fortunate, householder, so very fortunate, to traverse the Buddha’s deep teachings with the eye of wisdom.” 

%
\section*{{\suttatitleacronym SN 41.8}{\suttatitletranslation Nigaṇṭha Nāṭaputta }{\suttatitleroot Nigaṇṭhanāṭaputtasutta}}
\addcontentsline{toc}{section}{\tocacronym{SN 41.8} \toctranslation{Nigaṇṭha Nāṭaputta } \tocroot{Nigaṇṭhanāṭaputtasutta}}
\markboth{Nigaṇṭha Nāṭaputta }{Nigaṇṭhanāṭaputtasutta}
\extramarks{SN 41.8}{SN 41.8}

Now\marginnote{1.1} at that time \textsanskrit{Nigaṇṭha} \textsanskrit{Nāṭaputta} had arrived at \textsanskrit{Macchikāsaṇḍa} together with a large assembly of Jain ascetics. 

Citta\marginnote{1.2} the householder heard that they had arrived. Together with several lay followers, he went up to \textsanskrit{Nigaṇṭha} \textsanskrit{Nātaputta} and exchanged greetings with him. 

When\marginnote{1.4} the greetings and polite conversation were over, he sat down to one side. \textsanskrit{Nigaṇṭha} \textsanskrit{Nātaputta} said to him, “Householder, do you have faith in the ascetic Gotama’s claim that there is a state of immersion without placing the mind and keeping it connected; that there is the cessation of placing the mind and keeping it connected?” 

“Sir,\marginnote{2.1} in this case I don’t rely on faith in the Buddha’s claim that there is a state of immersion without placing the mind and keeping it connected; that there is the cessation of placing the mind and keeping it connected.” 

When\marginnote{2.3} he said this, \textsanskrit{Nigaṇṭha} \textsanskrit{Nātaputta} looked up at his assembly and said, “See, good sirs, how straightforward this householder Citta is! He’s not devious or deceitful at all. To imagine that you can stop placing the mind and keeping it connected would be like imagining that you can catch the wind in a net, or dam the Ganges river with your own hand.” 

“What\marginnote{3.1} do you think, sir? Which is better—knowledge or faith?” 

“Knowledge\marginnote{3.3} is definitely better than faith, householder.” 

“Well\marginnote{3.4} sir, whenever I want, quite secluded from sensual pleasures, secluded from unskillful qualities, I enter and remain in the first absorption, which has the rapture and bliss born of seclusion, while placing the mind and keeping it connected. And whenever I want, as the placing of the mind and keeping it connected are stilled … I enter and remain in the second absorption. And whenever I want, with the fading away of rapture … I enter and remain in the third absorption. And whenever I want, giving up pleasure and pain … I enter and remain in the fourth absorption. 

And\marginnote{3.8} so, sir, since I know and see like this, why should I rely on faith in another ascetic or brahmin who claims that there is a state of immersion without placing the mind and keeping it connected; that there is the cessation of placing the mind and keeping it connected?” 

When\marginnote{4.1} he said this, \textsanskrit{Nigaṇṭha} \textsanskrit{Nātaputta} looked askance at his own assembly and said, “See, good sirs, how crooked this householder Citta is! He’s so devious and deceitful!” 

“Sir,\marginnote{5.1} just now I understood you to say: ‘See, good sirs, how straightforward this householder Citta is! He’s not devious or deceitful at all.’ But then I understood you to say: ‘See, good sirs, how crooked this householder Citta is! He’s so devious and deceitful!’ If your first statement is true, the second is wrong. If your first statement is wrong, the second is true. 

And\marginnote{5.7} also, sir, these ten legitimate questions are relevant. When you understand what they mean, then, together with your assembly of Jain ascetics, you can rebut me. ‘One thing: question, summary, and answer. Two … three … four … five … six … seven … eight … nine … ten things: question, summary, and answer.’” 

Then\marginnote{5.19} Citta got up from his seat and left without asking \textsanskrit{Nigaṇṭha} \textsanskrit{Nātaputta} these ten legitimate questions. 

%
\section*{{\suttatitleacronym SN 41.9}{\suttatitletranslation With Kassapa, the Naked Ascetic }{\suttatitleroot Acelakassapasutta}}
\addcontentsline{toc}{section}{\tocacronym{SN 41.9} \toctranslation{With Kassapa, the Naked Ascetic } \tocroot{Acelakassapasutta}}
\markboth{With Kassapa, the Naked Ascetic }{Acelakassapasutta}
\extramarks{SN 41.9}{SN 41.9}

Now\marginnote{1.1} at that time the naked ascetic Kassapa, who in lay life was an old friend of Citta, had arrived at \textsanskrit{Macchikāsaṇḍa}. 

Citta\marginnote{1.2} the householder heard that he had arrived. So he went up to him, and they exchanged greetings. 

When\marginnote{1.4} the greetings and polite conversation were over, he sat down to one side and said to the naked ascetic Kassapa, “Sir, Kassapa, how long has it been since you went forth?” 

“It’s\marginnote{1.6} been thirty years, householder.” 

“But\marginnote{1.7} sir, in these thirty years have you achieved any superhuman distinction in knowledge and vision worthy of the noble ones, a meditation at ease?” 

“I\marginnote{1.8} have no such achievement, householder, only nakedness, baldness, and pokes in the buttocks.” 

Citta\marginnote{1.9} said to him, “It’s incredible, it’s amazing, how well explained the teaching is. For in thirty years you have achieved no superhuman distinction in knowledge and vision worthy of the noble ones, no meditation at ease, only nakedness, baldness, and pokes in the buttocks.” 

“But\marginnote{2.1} householder, how long have you been a lay follower?” 

“It’s\marginnote{2.2} been thirty years, sir.” 

“But\marginnote{2.3} householder, in these thirty years have you achieved any superhuman distinction in knowledge and vision worthy of the noble ones, a meditation at ease?” 

“How,\marginnote{2.4} sir, could I not? For whenever I want, quite secluded from sensual pleasures, secluded from unskillful qualities, I enter and remain in the first absorption, which has the rapture and bliss born of seclusion, while placing the mind and keeping it connected. And whenever I want, as the placing of the mind and keeping it connected are stilled … I enter and remain in the second absorption. And whenever I want, with the fading away of rapture … I enter and remain in the third absorption. And whenever I want, giving up pleasure and pain … I enter and remain in the fourth absorption. 

If\marginnote{2.9} I pass away before the Buddha, it wouldn’t be surprising if the Buddha declares of me: ‘The householder Citta is bound by no fetter that might return him to this world.’” 

When\marginnote{2.11} this was said, Kassapa said to Citta, “It’s incredible, it’s amazing, how well explained the teaching is. For a white-clothed layperson can achieve such a superhuman distinction in knowledge and vision worthy of the noble ones, a meditation at ease. Sir, may I receive the going forth, the ordination in the Buddha’s presence?” 

Then\marginnote{3.1} Citta the householder took the naked ascetic Kassapa to see the senior mendicants, and said to them: 

“Sirs,\marginnote{3.2} this is the naked ascetic Kassapa, who in lay life was an old friend of mine. May the senior monks give him the going forth, the ordination. I’ll make sure that he’s provided with robes, almsfood, lodgings, and medicines and supplies for the sick.” 

And\marginnote{3.5} the naked ascetic Kassapa received the going forth, the ordination in this teaching and training. Not long after his ordination, Venerable Kassapa, living alone, withdrawn, diligent, keen, and resolute, soon realized the supreme end of the spiritual path in this very life. He lived having achieved with his own insight the goal for which gentlemen rightly go forth from the lay life to homelessness. 

He\marginnote{3.7} understood: “Rebirth is ended; the spiritual journey has been completed; what had to be done has been done; there is no return to any state of existence.” And Venerable Kassapa became one of the perfected. 

%
\section*{{\suttatitleacronym SN 41.10}{\suttatitletranslation Seeing the Sick }{\suttatitleroot Gilānadassanasutta}}
\addcontentsline{toc}{section}{\tocacronym{SN 41.10} \toctranslation{Seeing the Sick } \tocroot{Gilānadassanasutta}}
\markboth{Seeing the Sick }{Gilānadassanasutta}
\extramarks{SN 41.10}{SN 41.10}

Now\marginnote{1.1} at that time the householder Citta was sick, suffering, gravely ill. 

Then\marginnote{1.2} several deities of the parks, forests, trees, and those who haunt the herbs, grass, and big trees came together and said to Citta, “Householder, make a wish to become a wheel-turning monarch in the future!” 

When\marginnote{2.1} they said this, Citta said to them, “That too is impermanent! That too will pass! That too will be left behind!” 

When\marginnote{2.3} he said this, his friends and colleagues, relatives and family members said, “Be mindful, master! Don’t babble.” 

“What\marginnote{2.5} have I said that makes you say that?” 

“It’s\marginnote{2.7} because you said: ‘That too is impermanent! That too will pass! That too will be left behind!’” 

“Oh,\marginnote{2.9} well, that’s because the deities of the parks, forests, trees, and those who haunt the herbs, grass, and big trees said to me: ‘Householder, make a wish to become a wheel-turning monarch in the future!’ So I said to them: ‘That too is impermanent! That too will pass! That too will be left behind!’” 

“But\marginnote{2.13} what reason do they see for saying that to you?” 

“They\marginnote{2.15} think: ‘This householder Citta is ethical, of good character. If he makes a wish to become a wheel-turning monarch in the future, his heart’s wish will succeed because of the purity of his ethics. And then as a proper, principled king he will provide proper spirit-offerings.’ That’s the reason they see for saying to me: ‘Householder, make a wish to become a wheel-turning monarch in the future!’ So I said to them: ‘That too is impermanent! That too will pass! That too will be left behind!’” 

“Then,\marginnote{3.1} master, advise us!” 

“So\marginnote{3.2} you should train like this: 

We\marginnote{3.3} will have experiential confidence in the Buddha: ‘That Blessed One is perfected, a fully awakened Buddha, accomplished in knowledge and conduct, holy, knower of the world, supreme guide for those who wish to train, teacher of gods and humans, awakened, blessed.’ 

We\marginnote{3.5} will have experiential confidence in the teaching: ‘The teaching is well explained by the Buddha—visible in this very life, immediately effective, inviting inspection, relevant, so that sensible people can know it for themselves.’ 

We\marginnote{3.7} will have experiential confidence in the \textsanskrit{Saṅgha}: ‘The \textsanskrit{Saṅgha} of the Buddha’s disciples is practicing the way that’s good, direct, methodical, and proper. It consists of the four pairs, the eight individuals. This is the \textsanskrit{Saṅgha} of the Buddha’s disciples that is worthy of offerings dedicated to the gods, worthy of hospitality, worthy of a religious donation, worthy of greeting with joined palms, and is the supreme field of merit for the world.’ 

And\marginnote{3.9} we will share without reservation all the gifts available to give in our family with those who are ethical and of good character.” 

Then,\marginnote{3.10} after Citta had encouraged his friends and colleagues, relatives and family members in the Buddha, the teaching, the \textsanskrit{Saṅgha}, and generosity, he passed away. 

\scendsutta{The Linked Discourses on Citta the Householder are complete. }

%
\addtocontents{toc}{\let\protect\contentsline\protect\nopagecontentsline}
\part*{Linked Discourses with Chiefs }
\addcontentsline{toc}{part}{Linked Discourses with Chiefs }
\markboth{}{}
\addtocontents{toc}{\let\protect\contentsline\protect\oldcontentsline}

%
\addtocontents{toc}{\let\protect\contentsline\protect\nopagecontentsline}
\chapter*{The Chapter on Chiefs }
\addcontentsline{toc}{chapter}{\tocchapterline{The Chapter on Chiefs }}
\addtocontents{toc}{\let\protect\contentsline\protect\oldcontentsline}

%
\section*{{\suttatitleacronym SN 42.1}{\suttatitletranslation Vicious }{\suttatitleroot Caṇḍasutta}}
\addcontentsline{toc}{section}{\tocacronym{SN 42.1} \toctranslation{Vicious } \tocroot{Caṇḍasutta}}
\markboth{Vicious }{Caṇḍasutta}
\extramarks{SN 42.1}{SN 42.1}

At\marginnote{1.1} \textsanskrit{Sāvatthī}. 

Then\marginnote{1.2} the chief named Fury went up to the Buddha, bowed, sat down to one side, and said to him: 

“What\marginnote{1.3} is the cause, sir, what is the reason why some people are regarded as furious, while others are regarded as sweet-natured?” 

“Take\marginnote{1.5} someone who hasn’t given up greed. So they get annoyed by others, and they show it. They’re regarded as furious. They haven’t given up hate. So they get annoyed by others, and they show it. They’re regarded as furious. They haven’t given up delusion. So they get annoyed by others, and they show it. They’re regarded as furious. This is the cause, this is the reason why some people are regarded as furious. 

But\marginnote{2.1} take someone who has given up greed. So they don’t get annoyed by others, and don’t show it. They’re regarded as sweet-natured. They’ve given up hate. So they don’t get annoyed by others, and don’t show it. They’re regarded as sweet-natured. They’ve given up delusion. So they don’t get annoyed by others, and don’t show it. They’re regarded as sweet-natured. This is the cause, this is the reason why some people are regarded as sweet-natured.” 

When\marginnote{3.1} he said this, the chief named Fury said to the Buddha, “Excellent, sir! Excellent! As if he were righting the overturned, or revealing the hidden, or pointing out the path to the lost, or lighting a lamp in the dark so people with good eyes can see what’s there, the Buddha has made the teaching clear in many ways. I go for refuge to the Buddha, to the teaching, and to the mendicant \textsanskrit{Saṅgha}. From this day forth, may the Buddha remember me as a lay follower who has gone for refuge for life.” 

%
\section*{{\suttatitleacronym SN 42.2}{\suttatitletranslation With Tāḷapuṭa }{\suttatitleroot Tālapuṭasutta}}
\addcontentsline{toc}{section}{\tocacronym{SN 42.2} \toctranslation{With Tāḷapuṭa } \tocroot{Tālapuṭasutta}}
\markboth{With Tāḷapuṭa }{Tālapuṭasutta}
\extramarks{SN 42.2}{SN 42.2}

At\marginnote{1.1} one time the Buddha was staying near \textsanskrit{Rājagaha}, in the Bamboo Grove, the squirrels’ feeding ground. Then \textsanskrit{Tāḷapuṭa} the dancing master came up to the Buddha, bowed, sat down to one side, and said to the Buddha: 

“Sir,\marginnote{1.3} I have heard that the dancers of the past who were teachers of teachers said: ‘Suppose a dancer entertains and amuses people on a stage or at a festival with truth and lies. When their body breaks up, after death, they’re reborn in the company of laughing gods.’ What does the Buddha say about this?” 

“Enough,\marginnote{1.6} chief, let it be. Don’t ask me that.” 

For\marginnote{1.7} a second time … 

And\marginnote{1.12} for a third time \textsanskrit{Tāḷapuṭa} said to the Buddha: 

“Sir,\marginnote{1.13} I have heard that the dancers of the past who were teachers of teachers said: ‘Suppose a dancer entertains and amuses people on a stage or at a festival with truth and lies. When their body breaks up, after death, they’re reborn in the company of laughing gods.’ What does the Buddha say about this?” 

“Clearly,\marginnote{2.1} chief, I’m not getting through to you when I say: ‘Enough, chief, let it be. Don’t ask me that.’ Nevertheless, I will answer you. 

When\marginnote{2.5} sentient beings are still not free of greed, and are still bound by greed, a dancer in a stage or festival presents them with even more arousing things. When sentient beings are still not free of hate, and are still bound by hate, a dancer in a stage or festival presents them with even more hateful things. When sentient beings are still not free of delusion, and are still bound by delusion, a dancer in a stage or festival presents them with even more delusory things. And so, being heedless and negligent themselves, they’ve encouraged others to be heedless and negligent. When their body breaks up, after death, they’re reborn in the hell called ‘Laughter’. 

But\marginnote{2.12} if you have such a view: ‘Suppose a dancer entertains and amuses people on a stage or at a festival with truth and lies. When their body breaks up, after death, they’re reborn in the company of laughing gods.’ This is your wrong view. An individual with wrong view is reborn in one of two places, I say: hell or the animal realm.” 

When\marginnote{3.1} he said this, \textsanskrit{Tāḷapuṭa} cried and burst out in tears. 

“This\marginnote{3.2} is what I didn’t get through to you when I said: ‘Enough, chief, let it be. Don’t ask me that.’” 

“Sir,\marginnote{3.4} I’m not crying because of what the Buddha said. But sir, for a long time I’ve been cheated, tricked, and deceived by the dancers of the past who were teachers of teachers, who said: ‘Suppose a dancer entertains and amuses people on a stage or at a festival with truth and lies. When their body breaks up, after death, they’re reborn in the company of laughing gods.’ 

Excellent,\marginnote{3.7} sir! Excellent! As if he were righting the overturned, or revealing the hidden, or pointing out the path to the lost, or lighting a lamp in the dark so people with good eyes can see what’s there, the Buddha has made the teaching clear in many ways. I go for refuge to the Buddha, to the teaching, and to the mendicant \textsanskrit{Saṅgha}. Sir, may I receive the going forth, the ordination in the Buddha’s presence?” 

And\marginnote{3.11} the dancing master \textsanskrit{Tāḷapuṭa} received the going forth, the ordination in the Buddha’s presence. Not long after his ordination, Venerable \textsanskrit{Tāḷapuṭa} became one of the perfected. 

%
\section*{{\suttatitleacronym SN 42.3}{\suttatitletranslation A Warrior }{\suttatitleroot Yodhājīvasutta}}
\addcontentsline{toc}{section}{\tocacronym{SN 42.3} \toctranslation{A Warrior } \tocroot{Yodhājīvasutta}}
\markboth{A Warrior }{Yodhājīvasutta}
\extramarks{SN 42.3}{SN 42.3}

Then\marginnote{1.1} Dustin the warrior chief went up to the Buddha, bowed, sat down to one side, and said to him: 

“Sir,\marginnote{1.2} I have heard that the warriors of the past who were teachers of teachers said: ‘Suppose a warrior, while striving and struggling in battle, is killed and finished off by his foes. When his body breaks up, after death, he’s reborn in the company of the gods of the fallen.’ What does the Buddha say about this?” 

“Enough,\marginnote{1.5} chief, let it be. Don’t ask me that.” 

For\marginnote{1.6} a second time … 

And\marginnote{1.7} for a third time the warrior chief said to the Buddha: 

“Sir,\marginnote{1.8} I have heard that the warriors of the past who were teachers of teachers said: ‘Suppose a warrior, while striving and struggling in battle, is killed and finished off by his foes. When his body breaks up, after death, he’s reborn in the company of the gods of the fallen.’ What does the Buddha say about this?” 

“Clearly,\marginnote{2.1} chief, I’m not getting through to you when I say: ‘Enough, chief, let it be. Don’t ask me that.’ Nevertheless, I will answer you. 

When\marginnote{2.4} a warrior strives and struggles in battle, their mind is already low, degraded, and misdirected as they think: ‘May these sentient beings be killed, slaughtered, slain, destroyed, or annihilated!’ His foes kill him and finish him off, and when his body breaks up, after death, he’s reborn in the hell called ‘The Fallen’. 

But\marginnote{2.8} if you have such a view: ‘Suppose a warrior, while striving and struggling in battle, is killed and finished off by his foes. When his body breaks up, after death, he’s reborn in the company of the gods of the fallen.’ This is your wrong view. An individual with wrong view is reborn in one of two places, I say: hell or the animal realm.” 

When\marginnote{3.1} he said this, Dustin the warrior chief cried and burst out in tears. 

“This\marginnote{3.2} is what I didn’t get through to you when I said: ‘Enough, chief, let it be. Don’t ask me that.’” 

“Sir,\marginnote{3.4} I’m not crying because of what the Buddha said. But sir, for a long time I’ve been cheated, tricked, and deceived by the warriors of the past who were teachers of teachers, who said: ‘Suppose a warrior, while striving and struggling in battle, is killed and finished off by his foes. When his body breaks up, after death, he’s reborn in the company of the gods of the fallen.’ 

Excellent,\marginnote{3.7} sir! Excellent! … From this day forth, may the Buddha remember me as a lay follower who has gone for refuge for life.” 

%
\section*{{\suttatitleacronym SN 42.4}{\suttatitletranslation An Elephant Warrior }{\suttatitleroot Hatthārohasutta}}
\addcontentsline{toc}{section}{\tocacronym{SN 42.4} \toctranslation{An Elephant Warrior } \tocroot{Hatthārohasutta}}
\markboth{An Elephant Warrior }{Hatthārohasutta}
\extramarks{SN 42.4}{SN 42.4}

Then\marginnote{1.1} an elephant warrior chief went up to the Buddha … “From this day forth, may the Buddha remember me as a lay follower who has gone for refuge for life.” 

%
\section*{{\suttatitleacronym SN 42.5}{\suttatitletranslation A Cavalryman }{\suttatitleroot Assārohasutta}}
\addcontentsline{toc}{section}{\tocacronym{SN 42.5} \toctranslation{A Cavalryman } \tocroot{Assārohasutta}}
\markboth{A Cavalryman }{Assārohasutta}
\extramarks{SN 42.5}{SN 42.5}

Then\marginnote{1.1} a cavalry chief went up to the Buddha … 

“From\marginnote{4.1} this day forth, may the Buddha remember me as a lay follower who has gone for refuge for life.” 

%
\section*{{\suttatitleacronym SN 42.6}{\suttatitletranslation With Asibandhaka’s Son }{\suttatitleroot Asibandhakaputtasutta}}
\addcontentsline{toc}{section}{\tocacronym{SN 42.6} \toctranslation{With Asibandhaka’s Son } \tocroot{Asibandhakaputtasutta}}
\markboth{With Asibandhaka’s Son }{Asibandhakaputtasutta}
\extramarks{SN 42.6}{SN 42.6}

At\marginnote{1.1} one time the Buddha was staying near \textsanskrit{Nālandā} in \textsanskrit{Pāvārika}’s mango grove. 

Then\marginnote{1.2} Asibandhaka’s son the chief went up to the Buddha, bowed, sat down to one side, and said to him: 

“Sir,\marginnote{2.1} there are western brahmins draped with moss who carry pitchers, immerse themselves in water, and serve the sacred flame. When someone has passed away, they truly lift them up, raise them up, and guide them along to heaven. But what about the Blessed One, the perfected one, the fully awakened Buddha: is he able to ensure that the whole world will be reborn in a good place, a heavenly realm when their body breaks up, after death?” 

“Well\marginnote{3.1} then, chief, I’ll ask you about this in return, and you can answer as you like. 

What\marginnote{4.1} do you think, chief? Take a person who kills living creatures, steals, and commits sexual misconduct. They use speech that’s false, divisive, harsh, or nonsensical. And they’re covetous, malicious, and have wrong view. And a large crowd comes together to offer up prayers and praise, circumambulating them with joined palms and saying: ‘When this person’s body breaks up, after death, may they be reborn in a good place, a heavenly realm!’ What do you think, chief? Would that person be reborn in heaven because of their prayers?” 

“No,\marginnote{4.7} sir.” 

“Chief,\marginnote{5.1} suppose a person were to throw a broad rock into a deep lake. And a large crowd was to come together to offer up prayers and praise, circumambulating it with joined palms, and saying: ‘Rise, good rock! Float, good rock! Float to shore, good rock!’ What do you think, chief? Would that broad rock rise up or float because of their prayers?” 

“No,\marginnote{5.6} sir.” 

“In\marginnote{5.7} the same way, take a person who kills living creatures, steals, and commits sexual misconduct. They use speech that’s false, divisive, harsh, or nonsensical. And they’re covetous, malicious, and have wrong view. Even though a large crowd comes together to offer up prayers and praise … when their body breaks up, after death, they’re reborn in a place of loss, a bad place, the underworld, hell. 

What\marginnote{6.1} do you think, chief? Take a person who doesn’t kill living creatures, steal, or commit sexual misconduct. They don’t use speech that’s false, divisive, harsh, or nonsensical. And they’re contented, kind-hearted, and have right view. And a large crowd comes together to offer up prayers and praise, circumambulating them with joined palms and saying: ‘When this person’s body breaks up, after death, may they be reborn in a place of loss, a bad place, the underworld, hell!’ What do you think, chief? Would that person be reborn in hell because of their prayers?” 

“No,\marginnote{6.7} sir.” 

“Chief,\marginnote{7.1} suppose a person were to sink a pot of ghee or oil into a deep lake and break it open. Its shards and chips would sink down, while the ghee or oil in it would rise up. And a large crowd was to come together to offer up prayers and praise, circumambulating it with joined palms and saying: ‘Sink, good ghee or oil! Descend, good ghee or oil! Go down, good ghee or oil!” What do you think, chief? Would that ghee or oil sink and descend because of their prayers?” 

“No,\marginnote{7.6} sir.” 

“In\marginnote{8.1} the same way, take a person who doesn’t kill living creatures, steal, or commit sexual misconduct. They don’t use speech that’s false, divisive, harsh, or nonsensical. And they’re contented, kind-hearted, and have right view. Even though a large crowd comes together to offer up prayers and praise … when their body breaks up, after death, they’re reborn in a good place, a heavenly realm.” 

When\marginnote{8.4} he said this, Asibandhaka’s son the chief said to the Buddha, “Excellent, sir! … From this day forth, may the Buddha remember me as a lay follower who has gone for refuge for life.” 

%
\section*{{\suttatitleacronym SN 42.7}{\suttatitletranslation The Simile of the Field }{\suttatitleroot Khettūpamasutta}}
\addcontentsline{toc}{section}{\tocacronym{SN 42.7} \toctranslation{The Simile of the Field } \tocroot{Khettūpamasutta}}
\markboth{The Simile of the Field }{Khettūpamasutta}
\extramarks{SN 42.7}{SN 42.7}

At\marginnote{1.1} one time the Buddha was staying near \textsanskrit{Nālandā} in \textsanskrit{Pāvārika}’s mango grove. Then Asibandhaka’s son the chief went up to the Buddha, bowed, sat down to one side, and said to him: 

“Sir,\marginnote{1.3} doesn’t the Buddha live full of compassion for all living beings?” 

“Yes,\marginnote{1.4} chief.” 

“Well,\marginnote{1.5} sir, why exactly do you teach some people thoroughly and others less thoroughly?” 

“Well\marginnote{1.6} then, chief, I’ll ask you about this in return, and you can answer as you like. What do you think? Suppose a farmer has three fields: one’s good, one’s average, and one’s poor—bad ground of sand and salt. What do you think? When that farmer wants to plant seeds, where would he plant them first: the good field, the average one, or the poor one?” 

“Sir,\marginnote{1.9} he’d plant them first in the good field, then the average, then he may or may not plant seed in the poor field. Why is that? Because at least it can be fodder for the cattle.” 

“To\marginnote{2.1} me, the monks and nuns are like the good field. I teach them the Dhamma that’s good in the beginning, good in the middle, and good in the end, meaningful and well-phrased. And I reveal a spiritual practice that’s entirely full and pure. Why is that? Because they live with me as their island, protection, shelter, and refuge. 

To\marginnote{2.5} me, the laymen and laywomen are like the average field. I also teach them the Dhamma that’s good in the beginning, good in the middle, and good in the end, meaningful and well-phrased. And I reveal a spiritual practice that’s entirely full and pure. Why is that? Because they live with me as their island, protection, shelter, and refuge. 

To\marginnote{2.9} me, the ascetics, brahmins, and wanderers who follow other paths are like the poor field, the bad ground of sand and salt. I also teach them the Dhamma that’s good in the beginning, good in the middle, and good in the end, meaningful and well-phrased. And I reveal a spiritual practice that’s entirely full and pure. Why is that? Hopefully they might understand even a single sentence, which would be for their lasting welfare and happiness. 

Suppose\marginnote{3.1} a person had three water jars: one that’s uncracked and nonporous; one that’s uncracked but porous; and one that’s cracked and porous. What do you think? When that person wants to store water, where would they store it first: in the jar that’s uncracked and nonporous, the one that’s uncracked but porous, or the one that’s cracked and porous?” 

“Sir,\marginnote{3.3} they’d store water first in the jar that’s uncracked and nonporous, then the one that’s uncracked but porous, then they may or may not store water in the one that’s cracked and porous. Why is that? Because at least it can be used for washing the dishes.” 

“To\marginnote{4.1} me, the monks and nuns are like the water jar that’s uncracked and nonporous. I teach them the Dhamma that’s good in the beginning, good in the middle, and good in the end, meaningful and well-phrased. And I reveal a spiritual practice that’s entirely full and pure. Why is that? Because they live with me as their island, protection, shelter, and refuge. 

To\marginnote{4.5} me, the laymen and laywomen are like the water jar that’s uncracked but porous. I teach them the Dhamma that’s good in the beginning, good in the middle, and good in the end, meaningful and well-phrased. And I reveal a spiritual practice that’s entirely full and pure. Why is that? Because they live with me as their island, protection, shelter, and refuge. 

To\marginnote{4.9} me, the ascetics, brahmins, and wanderers who follow other paths are like the water jar that’s cracked and porous. I also teach them the Dhamma that’s good in the beginning, good in the middle, and good in the end, meaningful and well-phrased. And I reveal a spiritual practice that’s entirely full and pure. Why is that? Hopefully they might understand even a single sentence, which would be for their lasting welfare and happiness.” 

When\marginnote{5.1} he said this, Asibandhaka’s son the chief said to the Buddha, “Excellent, sir! Excellent! … From this day forth, may the Buddha remember me as a lay follower who has gone for refuge for life.” 

%
\section*{{\suttatitleacronym SN 42.8}{\suttatitletranslation A Horn Blower }{\suttatitleroot Saṅkhadhamasutta}}
\addcontentsline{toc}{section}{\tocacronym{SN 42.8} \toctranslation{A Horn Blower } \tocroot{Saṅkhadhamasutta}}
\markboth{A Horn Blower }{Saṅkhadhamasutta}
\extramarks{SN 42.8}{SN 42.8}

At\marginnote{1.1} one time the Buddha was staying near \textsanskrit{Nālandā} in \textsanskrit{Pāvārika}’s mango grove. 

Then\marginnote{1.2} Asibandhaka’s son the chief, who was a disciple of the Jains, went up to the Buddha, and sat down to one side. The Buddha said to him, “Chief, how does \textsanskrit{Nigaṇṭha} \textsanskrit{Nātaputta} teach his disciples?” 

“Sir,\marginnote{1.4} this is how \textsanskrit{Nigaṇṭha} \textsanskrit{Nātaputta} teaches his disciples: ‘Everyone who kills a living creature, steal, commits sexual misconduct, or lies goes to a place of loss, to hell. You’re led on by what you usually live by.’ This is how \textsanskrit{Nigaṇṭha} \textsanskrit{Nātaputta} teaches his disciples.” 

“‘You’re\marginnote{1.8} led on by what you usually live by’: if this were true, then, according to what \textsanskrit{Nigaṇṭha} \textsanskrit{Nātaputta} says, no-one would go to a place of loss, to hell. 

What\marginnote{2.1} do you think, chief? Take a person who kills living creatures. If we compare periods of time during the day and night, which is more frequent: the occasions when they’re killing or when they’re not killing?” 

“The\marginnote{2.3} occasions when they’re killing are less frequent, while the occasions when they’re not killing are more frequent.” 

“‘You’re\marginnote{2.4} led on by what you usually live by’: if this were true, then, according to what \textsanskrit{Nigaṇṭha} \textsanskrit{Nātaputta} says, no-one would go to a place of loss, to hell. 

What\marginnote{3.1} do you think, chief? Take a person who steals … 

Take\marginnote{4.1} a person who commits sexual misconduct … 

Take\marginnote{5.1} a person who lies. If we compare periods of time during the day and night, which is more frequent: the occasions when they’re lying or when they’re not lying?” 

“The\marginnote{5.2} occasions when they’re lying are less frequent, while the occasions when they’re not lying are more frequent.” 

“‘You’re\marginnote{5.3} led on by what you usually live by’: if this were true, then, according to what \textsanskrit{Nigaṇṭha} \textsanskrit{Nātaputta} says, no-one would go to a place of loss, to hell. 

Take\marginnote{6.1} some teacher who has this doctrine and view: ‘Everyone who kills a living creature, steals, commits sexual misconduct, or lies goes to a place of loss, to hell.’ And there’s a disciple who is devoted to that teacher. They think: ‘My teacher has this doctrine and view: ‘Everyone who kills a living creature, steals, commits sexual misconduct, or lies goes to a place of loss, to hell.’ But I’ve killed living creatures … stolen … committed sexual misconduct … or lied. They get the view: ‘I too am going to a place of loss, to hell.’ Unless they give up that speech and thought, and let go of that view, they will be cast down to hell. 

But\marginnote{7.1} consider when a Realized One arises in the world, perfected, a fully awakened Buddha, accomplished in knowledge and conduct, holy, knower of the world, supreme guide for those who wish to train, teacher of gods and humans, awakened, blessed. In many ways he criticizes and denounces killing living creatures, saying: ‘Stop killing living creatures!’ He criticizes and denounces stealing … sexual misconduct … lying, saying: ‘Stop lying!’ And there’s a disciple who is devoted to that teacher. Then they reflect: ‘In many ways the Buddha criticizes and denounces killing living creatures, saying: “Stop killing living creatures!” But I have killed living creatures to a certain extent. That’s not right, it’s not good, and I feel remorseful because of it. But I can’t undo what I have done.’ Reflecting like this, they give up killing living creatures, and in future they don’t kill living creatures. That’s how to give up this bad deed and get past it. 

‘In\marginnote{8.1} many ways the Buddha criticizes and denounces stealing … 

‘In\marginnote{9.1} many ways the Buddha criticizes and denounces sexual misconduct … 

‘In\marginnote{10.1} many ways the Buddha criticizes and denounces lying, saying: “Stop lying!” But I have lied to a certain extent. That’s not right, it’s not good, and I feel remorseful because of it. But I can’t undo what I have done.’ Reflecting like this, they give up lying, and in future they refrain from lying. That’s how to give up this bad deed and get past it. 

They\marginnote{11.1} give up killing living creatures. They give up stealing. They give up sexual misconduct. They give up lying. They give up divisive speech. They give up harsh speech. They give up talking nonsense. They give up covetousness. They give up ill will and malevolence. They give up wrong view and have right view. 

That\marginnote{12.1} noble disciple is rid of desire, rid of ill will, unconfused, aware, and mindful. They meditate spreading a heart full of love to one direction, and to the second, and to the third, and to the fourth. In the same way above, below, across, everywhere, all around, they spread a heart full of love to the whole world—abundant, expansive, limitless, free of enmity and ill will. Suppose there was a powerful horn blower. They’d easily make themselves heard in the four quarters. In the same way, when the heart’s release by love has been developed and cultivated like this, any limited deeds they’ve done don’t remain or persist there. 

Then\marginnote{13.1} that noble disciple is rid of desire, rid of ill will, unconfused, aware, and mindful. They meditate spreading a heart full of compassion … They meditate spreading a heart full of rejoicing … They meditate spreading a heart full of equanimity to one direction, and to the second, and to the third, and to the fourth. In the same way above, below, across, everywhere, all around, they spread a heart full of equanimity to the whole world—abundant, expansive, limitless, free of enmity and ill will. Suppose there was a powerful horn blower. They’d easily make themselves heard in the four quarters. In the same way, when the heart’s release by equanimity has been developed and cultivated like this, any limited deeds they’ve done don’t remain or persist there.” 

When\marginnote{14.1} he said this, Asibandhaka’s son the chief said to the Buddha, “Excellent, sir! Excellent! … From this day forth, may the Buddha remember me as a lay follower who has gone for refuge for life.” 

%
\section*{{\suttatitleacronym SN 42.9}{\suttatitletranslation Families }{\suttatitleroot Kulasutta}}
\addcontentsline{toc}{section}{\tocacronym{SN 42.9} \toctranslation{Families } \tocroot{Kulasutta}}
\markboth{Families }{Kulasutta}
\extramarks{SN 42.9}{SN 42.9}

At\marginnote{1.1} one time the Buddha was wandering in the land of the Kosalans together with a large \textsanskrit{Saṅgha} of mendicants when he arrived at \textsanskrit{Nāḷandā}. There he stayed near \textsanskrit{Nālandā} in \textsanskrit{Pāvārika}’s mango grove. 

Now\marginnote{2.1} that was a time of famine and scarcity in \textsanskrit{Nāḷandā}, with blighted crops turned to straw. At that time \textsanskrit{Nigaṇṭha} \textsanskrit{Nāṭaputta} was residing at \textsanskrit{Nāḷandā} together with a large assembly of Jain ascetics. Then Asibandhaka’s son the chief, who was a disciple of the Jains, went up to \textsanskrit{Nigaṇṭha} \textsanskrit{Nāṭaputta}, bowed, and sat down to one side. \textsanskrit{Nigaṇṭha} \textsanskrit{Nāṭaputta} said to him: 

“Come,\marginnote{2.4} chief, refute the ascetic Gotama’s doctrine. Then you will get a good reputation: ‘Asibandhaka’s son the chief refuted the doctrine of the ascetic Gotama, so mighty and powerful!’” 

“But\marginnote{3.1} sir, how am I to do this?” 

“Here,\marginnote{3.2} brahmin, go to the ascetic Gotama and say to him: ‘Sir, don’t you in many ways praise kindness, protection, and compassion for families?’ When he’s asked this, if he answers: ‘Indeed I do, chief,’ say this to him: ‘So what exactly are you doing, wandering together with this large \textsanskrit{Saṅgha} of mendicants during a time of famine and scarcity, with blighted crops turned to straw? The Buddha is practicing to annihilate, collapse, and ruin families!’ When you put this dilemma to him, the Buddha won’t be able to either spit it out or swallow it down.” 

“Yes,\marginnote{3.9} sir,” replied Asibandhaka’s son. He got up from his seat, bowed, and respectfully circled \textsanskrit{Nigaṇṭha} \textsanskrit{Nāṭaputta}, keeping him on his right. Then he went to the Buddha, bowed, sat down to one side, and said to him: 

“Sir,\marginnote{4.1} don’t you in many ways praise kindness, protection, and compassion for families?” 

“Indeed\marginnote{4.2} I do, chief.” 

“So\marginnote{4.3} what exactly are you doing, wandering together with this large \textsanskrit{Saṅgha} of mendicants during a time of famine and scarcity, with blighted crops turned to straw? The Buddha is practicing to annihilate, collapse, and ruin families!” 

“Well,\marginnote{4.5} chief, I recollect ninety eons back but I’m not aware of any family that’s been ruined merely by offering some cooked almsfood. Rather, rich, affluent, and wealthy families—with lots of gold and silver, lots of property and assets, and lots of money and grain—all acquired their wealth because of generosity, truth, and restraint. 

Chief,\marginnote{4.7} there are eight causes and reasons for the ruin of families. Their ruin stems from rulers, bandits, fire, or flood. Or their savings vanish. Or their business fails due to not applying themselves to work. Or a wastrel is born into the family who squanders and fritters away their wealth. And impermanence is the eighth. These are the eight causes and reasons for the ruin of families. 

Given\marginnote{4.10} that these eight reasons are found, suppose someone says this: ‘The Buddha is practicing to annihilate, collapse, and ruin families!’ Unless they give up that speech and thought, and let go of that view, they will be cast down to hell.” 

When\marginnote{5.1} he said this, Asibandhaka’s son the chief said to the Buddha, “Excellent, sir! Excellent! … From this day forth, may the Buddha remember me as a lay follower who has gone for refuge for life.” 

%
\section*{{\suttatitleacronym SN 42.10}{\suttatitletranslation With Maṇicūḷaka }{\suttatitleroot Maṇicūḷakasutta}}
\addcontentsline{toc}{section}{\tocacronym{SN 42.10} \toctranslation{With Maṇicūḷaka } \tocroot{Maṇicūḷakasutta}}
\markboth{With Maṇicūḷaka }{Maṇicūḷakasutta}
\extramarks{SN 42.10}{SN 42.10}

At\marginnote{1.1} one time the Buddha was staying near \textsanskrit{Rājagaha}, in the Bamboo Grove, the squirrels’ feeding ground. Now at that time while the king’s retinue was sitting together in the royal compound this discussion came up among them, “Gold and money are proper for Sakyan ascetics. They accept and receive gold and money.” 

Now\marginnote{2.1} at that time the chief \textsanskrit{Maṇicūḷaka} was sitting in that assembly. He said to that retinue, “Good sirs, don’t say that. Gold and money are not proper for Sakyan ascetics. They neither accept nor receive gold and money. They have set aside gems and gold, and rejected gold and money.” He was able to persuade that assembly. 

Then\marginnote{2.6} \textsanskrit{Maṇicūḷaka} went up to the Buddha, bowed, sat down to one side, and told him what had happened. He then said, “Answering this way, I trust that I repeat what the Buddha has said, and don’t misrepresent him with an untruth. I trust my explanation is in line with the teaching, and that there are no legitimate grounds for rebuke or criticism.” 

“Indeed,\marginnote{3.1} in answering this way you repeat what I’ve said, and don’t misrepresent me with an untruth. Your explanation is in line with the teaching, and there are no legitimate grounds for rebuke or criticism. 

Gold\marginnote{3.2} and money are not proper for Sakyan ascetics. They neither accept nor receive gold and money. They have set aside gems and gold, and rejected gold and money. 

If\marginnote{3.3} gold and money were proper for them, then the five kinds of sensual stimulation would also be proper. And if the five kinds of sensual stimulation are proper for them, you should definitely regard them as not having the qualities of an ascetic or a follower of the Sakyan. 

Rather,\marginnote{3.5} chief, I say this: Straw may be looked for by one needing straw; wood may be looked for by one needing wood; a cart may be looked for by one needing a cart; a workman may be looked for by one needing a workman. But I say that there is no way they can accept or look for gold and money.” 

%
\section*{{\suttatitleacronym SN 42.11}{\suttatitletranslation With Bhadraka }{\suttatitleroot Bhadrakasutta}}
\addcontentsline{toc}{section}{\tocacronym{SN 42.11} \toctranslation{With Bhadraka } \tocroot{Bhadrakasutta}}
\markboth{With Bhadraka }{Bhadrakasutta}
\extramarks{SN 42.11}{SN 42.11}

At\marginnote{1.1} one time the Buddha was staying in the land of the Mallas, near the Mallian town called Uruvelakappa. Then Bhadraka the village chief went up to the Buddha, bowed, sat down to one side, and said to him: 

“Please,\marginnote{1.3} sir, teach me the origin and cessation of suffering.” 

“Chief,\marginnote{1.4} if I were to teach you about the origin and ending of suffering in the past, saying ‘this is how it was in the past,’ you might have doubts or uncertainties about that. If I were to teach you about the origin and ending of suffering in the future, saying ‘this is how it will be in the future,’ you might have doubts or uncertainties about that. Rather, chief, I will teach you about the origin and ending of suffering as I am sitting right here and you are sitting right there. Listen and pay close attention, I will speak.” 

“Yes,\marginnote{1.10} sir,” Bhadraka replied. The Buddha said this: 

“What\marginnote{2.1} do you think, chief? Are there any people here in Uruvelakappa who, if they were executed, imprisoned, fined, or condemned, it would cause you sorrow, lamentation, pain, sadness, and distress?” 

“There\marginnote{2.3} are, sir.” 

“But\marginnote{2.4} are there any people here in Uruvelakappa who, if they were executed, imprisoned, fined, or condemned, it would not cause you sorrow, lamentation, pain, sadness, and distress?” 

“There\marginnote{2.5} are, sir.” 

“What’s\marginnote{2.6} the cause, chief, what’s the reason why, if this was to happen to some people it could cause you sorrow, while if it happens to others it does not?” 

“The\marginnote{2.7} people regarding whom this would give rise to sorrow are those I desire and love. The people regarding whom this would not give rise to sorrow are those I don’t desire and love.” 

“With\marginnote{2.9} this present phenomenon that is seen, known, immediate, attained, and fathomed, you may infer to the past and future: ‘All the suffering that arose in the past was rooted and sourced in desire. For desire is the root of suffering. All the suffering that will arise in the future will be rooted and sourced in desire. For desire is the root of suffering.’” 

“It’s\marginnote{2.14} incredible, sir, it’s amazing! How well said this was by the Buddha! ‘All the suffering that arises is rooted and sourced in desire. For desire is the root of suffering.’ 

I\marginnote{2.18} have a boy called \textsanskrit{Ciravāsi}, who resides in a house away from here. I rise early and send someone, saying: ‘Go, my man, and check on my boy \textsanskrit{Ciravāsi}.’ Until they get back I worry: ‘I hope nothing’s wrong with \textsanskrit{Ciravāsi}!’” 

“What\marginnote{3.1} do you think, chief? If \textsanskrit{Ciravāsi} was executed, imprisoned, fined, or condemned, would it cause you sorrow, lamentation, pain, sadness, and distress?” 

“How\marginnote{3.3} could it not, sir?” 

“This\marginnote{3.4} too is a way to understand: ‘All the suffering that arises is rooted and sourced in desire. For desire is the root of suffering.’ 

What\marginnote{4.1} do you think, chief? Before you’d seen or heard of \textsanskrit{Ciravāsi}’s mother, did you have any desire or love or fondness for her?” 

“No,\marginnote{4.3} sir.” 

“Then\marginnote{4.4} was it because you saw or heard of her that you had desire or love or fondness for her?” 

“Yes,\marginnote{4.6} sir.” 

“What\marginnote{5.1} do you think, chief? If \textsanskrit{Ciravāsi}’s mother was executed, imprisoned, fined, or condemned, would it cause you sorrow, lamentation, pain, sadness, and distress?” 

“How\marginnote{5.3} could it not, sir?” 

“This\marginnote{5.4} too is a way to understand: ‘All the suffering that arises is rooted and sourced in desire. For desire is the root of suffering.’” 

%
\section*{{\suttatitleacronym SN 42.12}{\suttatitletranslation With Rāsiya }{\suttatitleroot Rāsiyasutta}}
\addcontentsline{toc}{section}{\tocacronym{SN 42.12} \toctranslation{With Rāsiya } \tocroot{Rāsiyasutta}}
\markboth{With Rāsiya }{Rāsiyasutta}
\extramarks{SN 42.12}{SN 42.12}

Then\marginnote{1.1} \textsanskrit{Rāsiya} the chief went up to the Buddha, bowed, sat down to one side, and said to him: 

“Sir,\marginnote{1.2} I have heard this: ‘The ascetic Gotama criticizes all forms of mortification. He categorically condemns and denounces those self-mortifiers who live rough.’ Do those who say this repeat what the Buddha has said, and not misrepresent him with an untruth? Is their explanation in line with the teaching? Are there any legitimate grounds for rebuke and criticism?” 

“Chief,\marginnote{1.4} those who say this do not repeat what I have said. They misrepresent me with what is false, hollow, and untrue. 

These\marginnote{2.1} two extremes should not be cultivated by one who has gone forth. Indulgence in sensual pleasures, which is low, crude, ordinary, ignoble, and pointless. And indulgence in self-mortification, which is painful, ignoble, and pointless. 

Avoiding\marginnote{2.3} these two extremes, the Realized One woke up by understanding the middle way of practice, which gives vision and knowledge, and leads to peace, direct knowledge, awakening, and extinguishment. 

And\marginnote{2.4} what is that middle way of practice? It is simply this noble eightfold path, that is: right view, right thought, right speech, right action, right livelihood, right effort, right mindfulness, and right immersion. 

This,\marginnote{2.7} chief, is the middle way of practice, woken up to by the Realized One, which gives vision and knowledge, and leads to peace, direct knowledge, awakening, and extinguishment. 

There\marginnote{3.1} are these three kinds of pleasure seekers in the world. What three? Take a pleasure seeker who seeks wealth using illegitimate, coercive means, and who doesn’t make themselves happy and pleased, or share it and make merit. Next, a pleasure seeker seeks wealth using illegitimate, coercive means. They make themselves happy and pleased, but don’t share it and make merit. Next, a pleasure seeker seeks wealth using illegitimate, coercive means. They make themselves happy and pleased, and they share it and make merit. 

Next,\marginnote{4.1} a pleasure seeker seeks wealth using means both legitimate and illegitimate, and coercive and non-coercive. They don’t make themselves happy and pleased, or share it and make merit. Next, a pleasure seeker seeks wealth using means both legitimate and illegitimate, and coercive and non-coercive. They don’t make themselves happy and pleased, or share it and make merit. Next, a pleasure seeker seeks wealth using means both legitimate and illegitimate, and coercive and non-coercive. They make themselves happy and pleased, and they share it and make merit. 

Next,\marginnote{5.1} a pleasure seeker seeks wealth using legitimate, non-coercive means. They don’t make themselves happy and pleased, or share it and make merit. Next, a pleasure seeker seeks wealth using legitimate, non-coercive means. They make themselves happy and pleased, but don’t share it and make merit. Next, a pleasure seeker seeks wealth using legitimate, non-coercive means. They make themselves happy and pleased, and they share it and make merit. They enjoy that wealth tied, infatuated, attached, blind to the drawbacks, and not understanding the escape. Next, a pleasure seeker seeks wealth using legitimate, non-coercive means. They make themselves happy and pleased, and they share it and make merit. And they enjoy that wealth untied, uninfatuated, unattached, seeing the drawbacks, and understanding the escape. 

Now,\marginnote{6.1} consider the pleasure seeker who seeks wealth using illegitimate, coercive means, and who doesn’t make themselves happy and pleased, or share it and make merit. They may be criticized on three grounds. What three? They seek wealth using illegitimate, coercive means. This is the first ground for criticism. They don’t make themselves happy and pleased. This is the second ground for criticism. They don’t share it and make merit. This is the third ground for criticism. This pleasure seeker may be criticized on these three grounds. 

Now,\marginnote{7.1} consider the pleasure seeker who seeks wealth using illegitimate, coercive means, and who makes themselves happy and pleased, but doesn’t share it and make merit. This pleasure seeker may be criticized on two grounds, and praised on one. What are the two grounds for criticism? They seek wealth using illegitimate, coercive means. This is the first ground for criticism. They don’t share it and make merit. This is the second ground for criticism. What is the one ground for praise? They make themselves happy and pleased. This is the one ground for praise. This pleasure seeker may be criticized on these two grounds, and praised on this one. 

Now,\marginnote{8.1} consider the pleasure seeker who seeks wealth using illegitimate, coercive means, and who makes themselves happy and pleased, and shares it and makes merit. This pleasure seeker may be criticized on one ground, and praised on two. What is the one ground for criticism? They seek wealth using illegitimate, coercive means. This is the one ground for criticism. What are the two grounds for praise? They make themselves happy and pleased. This is the first ground for praise. They share it and make merit. This is the second ground for praise. This pleasure seeker may be criticized on this one ground, and praised on these two. 

Now,\marginnote{9.1} consider the pleasure seeker who seeks wealth using means both legitimate and illegitimate, and coercive and non-coercive, and who doesn’t make themselves happy and pleased, or share it and make merit. They may be praised on one ground, and criticized on three. What is the one ground for praise? They seek wealth using legitimate, non-coercive means. This is the one ground for praise. What are the three grounds for criticism? They seek wealth using illegitimate, coercive means. This is the first ground for criticism. They don’t make themselves happy and pleased. This is the second ground for criticism. They don’t share it and make merit. This is the third ground for criticism. This pleasure seeker may be praised on this one ground, and criticized on these three. 

Now,\marginnote{10.1} consider the pleasure seeker who seeks wealth using means both legitimate and illegitimate, and coercive and non-coercive, and makes themselves happy and pleased, but doesn’t share it and make merit. They may be praised on two grounds, and criticized on two. What are the two grounds for praise? They seek wealth using legitimate, non-coercive means. This is the first ground for praise. They make themselves happy and pleased. This is the second ground for praise. What are the two grounds for criticism? They seek wealth using illegitimate, coercive means. This is the first ground for criticism. They don’t share it and make merit. This is the second ground for criticism. This pleasure seeker may be praised on these two grounds, and criticized on these two. 

Now,\marginnote{11.1} consider the pleasure seeker who seeks wealth using means both legitimate and illegitimate, and coercive and non-coercive, and who makes themselves happy and pleased, and shares it and makes merit. They may be praised on three grounds, and criticized on one. What are the three grounds for praise? They seek wealth using legitimate, non-coercive means. This is the first ground for praise. They make themselves happy and pleased. This is the second ground for praise. They share it and make merit. This is the third ground for praise. What is the one ground for criticism? They seek wealth using illegitimate, coercive means. This is the one ground for criticism. This pleasure seeker may be praised on these three grounds, and criticized on this one. 

Now,\marginnote{12.1} consider the pleasure seeker who seeks wealth using legitimate, non-coercive means, but who doesn’t make themselves happy and pleased, or share it and make merit. They may be praised on one ground, and criticized on two. What is the one ground for praise? They seek wealth using legitimate, non-coercive means. This is the one ground for praise. What are the two grounds for criticism? They don’t make themselves happy and pleased. This is the first ground for criticism. They don’t share it and make merit. This is the second ground for criticism. This pleasure seeker may be praised on this one ground, and criticized on these two. 

Now,\marginnote{13.1} consider the pleasure seeker who seeks wealth using legitimate, non-coercive means, and who makes themselves happy and pleased, but doesn’t share it and make merit. This pleasure seeker may be praised on two grounds, and criticized on one. What are the two grounds for praise? They seek wealth using legitimate, non-coercive means. This is the first ground for praise. They make themselves happy and pleased. This is the second ground for praise. What is the one ground for criticism? They don’t share it and make merit. This is the one ground for criticism. This pleasure seeker may be praised on these two grounds, and criticized on this one. 

Now,\marginnote{14.1} consider the pleasure seeker who seeks wealth using legitimate, non-coercive means, and who makes themselves happy and pleased, and shares it and makes merit. But they enjoy that wealth tied, infatuated, attached, blind to the drawbacks, and not understanding the escape. They may be praised on three grounds and criticized on one. What are the three grounds for praise? They seek wealth using legitimate, non-coercive means. This is the first ground for praise. They make themselves happy and pleased. This is the second ground for praise. They share it and make merit. This is the third ground for praise. What is the one ground for criticism? They enjoy that wealth tied, infatuated, attached, blind to the drawbacks, and not understanding the escape. This is the one ground for criticism. This pleasure seeker may be praised on these three grounds, and criticized on this one. 

Now,\marginnote{15.1} consider the pleasure seeker who seeks wealth using legitimate, non-coercive means, and who makes themselves happy and pleased, and shares it and makes merit. And they enjoy that wealth untied, uninfatuated, unattached, seeing the drawbacks, and understanding the escape. This pleasure seeker may be praised on four grounds. What are the four grounds for praise? They seek wealth using legitimate, non-coercive means. This is the first ground for praise. They make themselves happy and pleased. This is the second ground for praise. They share it and make merit. This is the third ground for praise. They enjoy that wealth untied, uninfatuated, unattached, seeing the drawbacks, and understanding the escape. This is the fourth ground for praise. This pleasure seeker may be praised on these four grounds. 

These\marginnote{16.1} three self-mortifiers who live rough are found in the world. What three? 

Take\marginnote{16.3} a self-mortifier who has gone forth from the lay life to homelessness, thinking: ‘Hopefully I will achieve a skillful quality! Hopefully I will realize a superhuman distinction in knowledge and vision worthy of the noble ones!’ They mortify and torment themselves. But they don’t achieve any skillful quality, or realize any superhuman distinction in knowledge and vision worthy of the noble ones. 

Take\marginnote{17.1} another self-mortifier who has gone forth from the lay life to homelessness, thinking: ‘Hopefully I will achieve a skillful quality! Hopefully I will realize a superhuman distinction in knowledge and vision worthy of the noble ones!’ They mortify and torment themselves. And they achieve a skillful quality, but don’t realize any superhuman distinction in knowledge and vision worthy of the noble ones. 

Take\marginnote{18.1} another self-mortifier who has gone forth from the lay life to homelessness, thinking: ‘Hopefully I will achieve a skillful quality! Hopefully I will realize a superhuman distinction in knowledge and vision worthy of the noble ones!’ They mortify and torment themselves. And they achieve a skillful quality, and they realize a superhuman distinction in knowledge and vision worthy of the noble ones. 

In\marginnote{19.1} this case, the first self-mortifier may be criticized on three grounds. What three? They mortify and torment themselves. This is the first ground for criticism. They don’t achieve a skillful quality. This is the second ground for criticism. They don’t realize a superhuman distinction in knowledge and vision worthy of the noble ones. This is the third ground for criticism. This self-mortifier may be criticized on these three grounds. 

In\marginnote{20.1} this case, the second self-mortifier may be criticized on two grounds, and praised on one. What are the two grounds for criticism? They mortify and torment themselves. This is the first ground for criticism. They don’t realize a superhuman distinction in knowledge and vision worthy of the noble ones. This is the second ground for criticism. What is the one ground for praise? They achieve a skillful quality. This is the one ground for praise. This self-mortifier may be criticized on these two grounds, and praised on one. 

In\marginnote{21.1} this case, the third self-mortifier may be criticized on one ground, and praised on two. What is the one ground for criticism? They mortify and torment themselves. This is the one ground for criticism. What are the two grounds for praise? They achieve a skillful quality. This is the first ground for praise. They realize a superhuman distinction in knowledge and vision worthy of the noble ones. This is the second ground for praise. This self-mortifier may be criticized on this one ground, and praised on two. 

There\marginnote{22.1} are these three kinds of wearing away that are visible in this very life, immediately effective, inviting inspection, relevant, so that sensible people can know them for themselves. What three? 

A\marginnote{22.3} greedy person, because of greed, intends to hurt themselves, hurt others, and hurt both. When they’ve given up greed they don’t have such intentions. This wearing away is visible in this very life, immediately effective, inviting inspection, relevant, so that sensible people can know it for themselves. 

A\marginnote{22.6} hateful person, because of hate, intends to hurt themselves, hurt others, and hurt both. When they’ve given up hate they don’t have such intentions. This wearing away is visible in this very life, immediately effective, inviting inspection, relevant, so that sensible people can know it for themselves. 

A\marginnote{22.9} deluded person, because of delusion, intends to hurt themselves, hurt others, and hurt both. When they’ve given up delusion they don’t have such intentions. This wearing away is visible in this very life, immediately effective, inviting inspection, relevant, so that sensible people can know it for themselves. 

These\marginnote{22.12} are the three kinds of wearing away that are visible in this very life, immediately effective, inviting inspection, relevant, so that sensible people can know them for themselves.” 

When\marginnote{23.1} he said this, \textsanskrit{Rāsiya} the chief said to the Buddha, “Excellent, sir! Excellent! … From this day forth, may the Buddha remember me as a lay follower who has gone for refuge for life.” 

%
\section*{{\suttatitleacronym SN 42.13}{\suttatitletranslation With Pāṭaliya }{\suttatitleroot Pāṭaliyasutta}}
\addcontentsline{toc}{section}{\tocacronym{SN 42.13} \toctranslation{With Pāṭaliya } \tocroot{Pāṭaliyasutta}}
\markboth{With Pāṭaliya }{Pāṭaliyasutta}
\extramarks{SN 42.13}{SN 42.13}

At\marginnote{1.1} one time the Buddha was staying in the land of the Koliyans, where they have a town called Uttara. Then \textsanskrit{Pāṭaliya} the chief went up to the Buddha, bowed, sat down to one side, and said to him: 

“Sir,\marginnote{1.3} I have heard this: ‘The ascetic Gotama knows magic.’ Do those who say this repeat what the Buddha has said, and not misrepresent him with an untruth? Is their explanation in line with the teaching? Are there any legitimate grounds for rebuke and criticism? For we don’t want to misrepresent the Blessed One.” 

“Chief,\marginnote{2.3} those who say this repeat what I have said, and don’t misrepresent me with an untruth. Their explanation is in line with the teaching, and there are no legitimate grounds for rebuke and criticism.” 

“Sir,\marginnote{2.4} we didn’t believe that what those ascetics and brahmins said was really true. But it seems the ascetic Gotama is a magician!” 

“Chief,\marginnote{2.6} does someone who says ‘I know magic’ also say ‘I am a magician’?” 

“That’s\marginnote{2.7} right, Blessed One! That’s right, Holy One!” 

“Well\marginnote{2.8} then, brahmin, I’ll ask you about this in return, and you can answer as you like. 

What\marginnote{3.1} do you think, chief? Do you know the Koliyan officers with drooping headdresses?” 

“I\marginnote{3.3} know them, sir.” 

“And\marginnote{3.4} what’s their job?” 

“To\marginnote{3.5} put a stop to bandits and to deliver messages for the Koliyans.” 

“What\marginnote{3.6} do you think, chief? Are the Koliyan officers with drooping headdresses moral or immoral?” 

“I\marginnote{3.8} know that they’re immoral, of bad character, sir. They are among those in the world who are immoral and of bad character.” 

“Would\marginnote{3.9} it be right to say that \textsanskrit{Pāṭaliya} knows the Kolyian officers with drooping headdresses who are immoral, of bad character, so he too must be immoral and of bad character.” 

“No,\marginnote{3.11} sir. I’m quite different from the Koliyan officers with drooping headdresses, we have quite different characters.” 

“So\marginnote{3.13} if you can know those officers of bad character while you are not of bad character, why can’t the Realized One know magic, without being a magician? 

I\marginnote{3.14} understand magic and its result. And I understand how magicians practice so that when their body breaks up, after death, they’re reborn in a place of loss, a bad place, the underworld, hell. 

I\marginnote{4.1} understand killing living creatures and its result. And I understand how those who kill living creatures practice so that when their body breaks up, after death, they’re reborn in a place of loss, a bad place, the underworld, hell. I understand stealing … sexual misconduct … lying … divisive speech … harsh speech … talking nonsense … covetousness … ill will … wrong view and its result. And I understand how those who have wrong view practice so that when their body breaks up, after death, they’re reborn in a place of loss, a bad place, the underworld, hell. 

There\marginnote{5.1} are some ascetics and brahmins who have this doctrine and view: ‘Everyone who kills living creatures experiences pain and sadness in the present life. Everyone who steals … commits sexual misconduct … lies experiences pain and sadness in the present life.’ 

But\marginnote{6.1} you can see someone, garlanded and adorned, nicely bathed and anointed, hair and beard dressed, taking his pleasure with women as if he were a king. You might ask someone: ‘Mister, what did that man do?’ And they’d reply: ‘Mister, that man attacked the king’s enemy and killed them. The king was delighted and gave him this reward. That’s why he’s garlanded and adorned, nicely bathed and anointed, hair and beard dressed, taking his pleasure with women as if he were a king.’ 

And\marginnote{7.1} you can see someone else, his arms tied tightly behind his back with a strong rope. His head is shaven and he’s marched from street to street and from square to square to the beating of a harsh drum. Then he’s taken out the south gate and there, to the south of the city, they chop off his head. You might ask someone: ‘Mister, what did that man do?’ And they’d reply: ‘Mister, that man is an enemy of the king who has murdered a man or a woman. That’s why the rulers arrested him and inflicted such punishment.’ 

What\marginnote{8.1} do you think, chief? Have you seen or heard of such a thing?” 

“Sir,\marginnote{8.3} we have seen it and heard of it, and we will hear of it again.” 

“Since\marginnote{8.4} this is so, the ascetics and brahmins whose view is that everyone who kills living creatures experiences pain and sadness in the present life: are they right or wrong?” 

“They’re\marginnote{8.6} wrong, sir.” 

“But\marginnote{8.7} those who speak hollow, false nonsense: are they moral or immoral?” 

“Immoral,\marginnote{8.8} sir.” 

“And\marginnote{8.9} are those who are immoral, of bad character practicing wrongly or rightly?” 

“They’re\marginnote{8.10} practicing wrongly, sir.” 

“And\marginnote{8.11} do those who are practicing wrongly have wrong view or right view?” 

“They\marginnote{8.12} have wrong view, sir.” 

“But\marginnote{8.13} is it appropriate to have confidence in those of wrong view?” 

“No,\marginnote{8.14} sir.” 

“You\marginnote{9.1} can see someone, garlanded and adorned … ‘Mister, that man attacked the king’s enemy and took their valuables. The king was delighted and gave him this reward. …’ … 

And\marginnote{10.1} you can see someone else, his arms tied tightly behind his back … ‘Mister, that man took something from a village or wilderness, with the intention to commit theft. That’s why the rulers arrested him and inflicted such punishment.’ What do you think, chief? Have you seen or heard of such a thing?” 

“Sir,\marginnote{10.8} we have seen it and heard of it, and we will hear of it again.” 

“Since\marginnote{10.9} this is so, the ascetics and brahmins whose view is that everyone who steals experiences pain and sadness in the present life: are they right or wrong? … Is it appropriate to have confidence in them?” 

“No,\marginnote{10.12} sir.” 

“You\marginnote{11.1} can see someone, garlanded and adorned … ‘Mister, that man had sexual relations with the wives of an enemy king. The king was delighted and gave him this reward. …’ … 

And\marginnote{12.1} you can see someone else, his arms tied tightly behind his back … ‘Mister, that man had sexual relations with the women and maidens of good families. That’s why the rulers arrested him and inflicted such punishment.’ What do you think, chief? Have you seen or heard of such a thing?” 

“Sir,\marginnote{12.9} we have seen it and heard of it, and we will hear of it again.” 

“Since\marginnote{12.10} this is so, the ascetics and brahmins whose view is that everyone who commits sexual misconduct experiences pain and sadness in the present life: are they right or wrong? … Is it appropriate to have confidence in them?” 

“No,\marginnote{12.13} sir.” 

“And\marginnote{13.1} you can see someone, garlanded and adorned … ‘Mister, that man amused the king with lies. The king was delighted and gave him this reward. …’ … 

And\marginnote{14.1} you can see someone else, his arms tied tightly behind his back … ‘Mister, that man has ruined a householder or householder’s child by lying. That’s why the rulers arrested him and inflicted such punishment.’ What do you think, chief? Have you seen or heard of such a thing?” 

“Sir,\marginnote{14.9} we have seen it and heard of it, and we will hear of it again.” 

“Since\marginnote{14.10} this is so, the ascetics and brahmins whose view is that everyone who lies experiences pain and sadness in the present life: are they right or wrong?” 

“They’re\marginnote{14.12} wrong, sir.” 

“But\marginnote{14.13} those who speak hollow, false nonsense: are they moral or immoral?” 

“Immoral,\marginnote{14.14} sir.” 

“And\marginnote{14.15} are those who are immoral, of bad character practicing wrongly or rightly?” 

“They’re\marginnote{14.16} practicing wrongly, sir.” 

“And\marginnote{14.17} do those who are practicing wrongly have wrong view or right view?” 

“They\marginnote{14.18} have wrong view, sir.” 

“But\marginnote{14.19} is it appropriate to have confidence in those of wrong view?” 

“No,\marginnote{14.20} sir. 

It’s\marginnote{15.1} incredible, sir, it’s amazing! I have a guest house, where there are cots, seats, water pots, and oil lamps. Whenever an ascetic or brahmin comes to stay, I share what I have as best I can. Once it so happened, sir, that four teachers of different views and opinions came to stay at my guest house. 

One\marginnote{16.1} teacher had this doctrine and view: ‘There’s no meaning in giving, sacrifice, or offerings. There’s no fruit or result of good and bad deeds. There’s no afterlife. There’s no such thing as mother and father, or beings that are reborn spontaneously. And there’s no ascetic or brahmin who is well attained and practiced, and who describes the afterlife after realizing it with their own insight.’ 

One\marginnote{17.1} teacher had this doctrine and view: ‘There is meaning in giving, sacrifice, and offerings. There are fruits and results of good and bad deeds. There is an afterlife. There are such things as mother and father, and beings that are reborn spontaneously. And there are ascetics and brahmins who are well attained and practiced, and who describe the afterlife after realizing it with their own insight.’ 

One\marginnote{18.1} teacher had this doctrine and view: ‘The one who acts does nothing wrong when they punish, mutilate, torture, aggrieve, oppress, intimidate, or when they encourage others to do the same. Nothing bad is done when they kill, steal, break into houses, plunder wealth, steal from isolated buildings, commit highway robbery, commit adultery, and lie. If you were to reduce all the living creatures of this earth to one heap and mass of flesh with a razor-edged chakram, no evil comes of that, and no outcome of evil. If you were to go along the south bank of the Ganges killing, mutilating, and torturing, and encouraging others to do the same, no evil comes of that, and no outcome of evil. If you were to go along the north bank of the Ganges giving and sacrificing and encouraging others to do the same, no merit comes of that, and no outcome of merit. In giving, self-control, restraint, and truthfulness there is no merit or outcome of merit.’ 

One\marginnote{19.1} teacher had this doctrine and view: ‘The one who acts does a bad deed when they punish, mutilate, torture, aggrieve, oppress, intimidate, or when they encourage others to do the same. A bad deed is done when they kill, steal, break into houses, plunder wealth, steal from isolated buildings, commit highway robbery, commit adultery, and lie. If you were to reduce all the living creatures of this earth to one heap and mass of flesh with a razor-edged chakram, evil comes of that, and an outcome of evil. If you were to go along the south bank of the Ganges killing, mutilating, and torturing, and encouraging others to do the same, evil comes of that, and an outcome of evil. If you were to go along the north bank of the Ganges giving and sacrificing and encouraging others to do the same, merit comes of that, and an outcome of merit. In giving, self-control, restraint, and truthfulness there is merit and outcome of merit.’ 

I\marginnote{20.1} had doubt and uncertainty about that: ‘I wonder who of these respected ascetics and brahmins speaks the truth, and who speaks falsehood?’” 

“Chief,\marginnote{21.1} no wonder you’re doubting and uncertain. Doubt has come up in you about an uncertain matter.” 

“I\marginnote{21.3} am quite confident that the Buddha is capable of teaching me so that I can give up this state of uncertainty.” 

“Chief,\marginnote{22.1} there is immersion based on understanding of principle. If you gain such mental immersion, you can give up that cause of uncertainty. 

And\marginnote{22.3} what is immersion based on understanding of principle? It’s when a noble disciple has given up killing living creatures, stealing, sexual misconduct, lying, divisive speech, harsh speech, talking nonsense, covetousness, ill will, and wrong view. 

Then\marginnote{23.1} that noble disciple is rid of desire, rid of ill will, unconfused, aware, and mindful. They meditate spreading a heart full of love to one direction, and to the second, and to the third, and to the fourth. In the same way above, below, across, everywhere, all around, they spread a heart full of love to the whole world—abundant, expansive, limitless, free of enmity and ill will. 

They\marginnote{23.2} reflect thus: ‘That teacher who had this doctrine and view: “There’s no meaning in giving, sacrifice, or offerings. There’s no fruit or result of good and bad deeds. There’s no afterlife. There’s no such thing as mother and father, or beings that are reborn spontaneously. And there’s no ascetic or brahmin who is well attained and practiced, and who describes the afterlife after realizing it with their own insight.” If what this good teacher says is true, it’s a safe bet for me to not hurt any creature firm or frail. I win on both counts, since I’m restrained in body, speech, and mind, and when my body breaks up, after death, I’ll be reborn in a good place, a heavenly realm.’ 

Joy\marginnote{23.7} springs up in them. Being joyful, rapture springs up. When the mind is full of rapture, the body becomes tranquil. When the body is tranquil, they feel bliss. And when blissful, the mind becomes immersed in \textsanskrit{samādhi}. This is that immersion based on understanding of principle. If you gain such mental immersion, you can give up that state of uncertainty. 

Then\marginnote{24.1} that noble disciple is rid of desire, rid of ill will, unconfused, aware, and mindful. They meditate spreading a heart full of love to one direction, and to the second, and to the third, and to the fourth. In the same way above, below, across, everywhere, all around, they spread a heart full of love to the whole world—abundant, expansive, limitless, free of enmity and ill will. 

They\marginnote{24.2} reflect thus: ‘That teacher who had this doctrine and view: “There is meaning in giving, sacrifice, and offerings. There are fruits and results of good and bad deeds. There is an afterlife. There are such things as mother and father, and beings that are reborn spontaneously. And there are ascetics and brahmins who are well attained and practiced, and who describe the afterlife after realizing it with their own insight.” If what this good teacher says is true, it’s a safe bet for me to not hurt any creature firm or frail. I win on both counts, since I’m restrained in body, speech, and mind, and when my body breaks up, after death, I’ll be reborn in a good place, a heavenly realm.’ 

Joy\marginnote{24.7} springs up in them. Being joyful, rapture springs up. When the mind is full of rapture, the body becomes tranquil. When the body is tranquil, they feel bliss. And when blissful, the mind becomes immersed in \textsanskrit{samādhi}. This is that immersion based on understanding of principle. If you gain such mental immersion, you can give up that state of uncertainty. 

Then\marginnote{25.1} that noble disciple is rid of desire, rid of ill will, unconfused, aware, and mindful. They meditate spreading a heart full of love to one direction, and to the second, and to the third, and to the fourth. In the same way above, below, across, everywhere, all around, they spread a heart full of love to the whole world—abundant, expansive, limitless, free of enmity and ill will. 

They\marginnote{25.2} reflect thus: ‘That teacher who had this doctrine and view: “The one who acts does nothing wrong when they punish, mutilate, torture, aggrieve, oppress, intimidate, or when they encourage others to do the same. Nothing bad is done when they kill, steal, break into houses, plunder wealth, steal from isolated buildings, commit highway robbery, commit adultery, and lie. If you were to reduce all the living creatures of this earth to one heap and mass of flesh with a razor-edged chakram, no evil comes of that, and no outcome of evil. If you were to go along the south bank of the Ganges killing, mutilating, and torturing, and encouraging others to do the same, no evil comes of that, and no outcome of evil. If you were to go along the north bank of the Ganges giving and sacrificing and encouraging others to do the same, no merit comes of that, and no outcome of merit. In giving, self-control, restraint, and truthfulness there is no merit or outcome of merit.” If what this good teacher says is true, it’s a safe bet for me to not hurt any creature firm or frail. I win on both counts, since I’m restrained in body, speech, and mind, and when my body breaks up, after death, I’ll be reborn in a good place, a heavenly realm.’ 

Joy\marginnote{25.11} springs up in them. Being joyful, rapture springs up. When the mind is full of rapture, the body becomes tranquil. When the body is tranquil, they feel bliss. And when blissful, the mind becomes immersed in \textsanskrit{samādhi}. This is that immersion based on understanding of principle. If you gain such mental immersion, you can give up that state of uncertainty. 

Then\marginnote{26.1} that noble disciple is rid of desire, rid of ill will, unconfused, aware, and mindful. They meditate spreading a heart full of love to one direction, and to the second, and to the third, and to the fourth. In the same way above, below, across, everywhere, all around, they spread a heart full of love to the whole world—abundant, expansive, limitless, free of enmity and ill will. 

They\marginnote{26.2} reflect thus: ‘That teacher who had this doctrine and view: “The one who acts does a bad deed when they punish, mutilate, torture, aggrieve, oppress, intimidate, or when they encourage others to do the same. A bad deed is done when they kill, steal, break into houses, plunder wealth, steal from isolated buildings, commit highway robbery, commit adultery, and lie. If you were to reduce all the living creatures of this earth to one heap and mass of flesh with a razor-edged chakram, evil comes of that, and an outcome of evil. If you were to go along the south bank of the Ganges killing, mutilating, and torturing, and encouraging others to do the same, evil comes of that, and an outcome of evil. If you were to go along the north bank of the Ganges giving and sacrificing and encouraging others to do the same, merit comes of that, and an outcome of merit. In giving, self-control, restraint, and truthfulness there is merit and outcome of merit.” If what this good teacher says is true, it’s a safe bet for me to not hurt any creature firm or frail. I win on both counts, since I’m restrained in body, speech, and mind, and when my body breaks up, after death, I’ll be reborn in a good place, a heavenly realm.’ 

Joy\marginnote{26.11} springs up in them. Being joyful, rapture springs up. When the mind is full of rapture, the body becomes tranquil. When the body is tranquil, they feel bliss. And when blissful, the mind becomes immersed in \textsanskrit{samādhi}. This is that immersion based on understanding of principle. If you gain such mental immersion, you can give up that state of uncertainty. 

Then\marginnote{27.1} that noble disciple is rid of desire, rid of ill will, unconfused, aware, and mindful. They meditate spreading a heart full of compassion … rejoicing … equanimity to one direction, and to the second, and to the third, and to the fourth. In the same way above, below, across, everywhere, all around, they spread a heart full of equanimity to the whole world—abundant, expansive, limitless, free of enmity and ill will. 

They\marginnote{31.1} reflect thus: ‘If what this good teacher says is true, it’s a safe bet for me to not hurt any creature firm or frail. I win on both counts, since I’m restrained in body, speech, and mind, and when my body breaks up, after death, I’ll be reborn in a good place, a heavenly realm.’ 

Joy\marginnote{31.11} springs up in them. Being joyful, rapture springs up. When the mind is full of rapture, the body becomes tranquil. When the body is tranquil, they feel bliss. And when blissful, the mind becomes immersed in \textsanskrit{samādhi}. This is that immersion based on understanding of principle. If you gain such mental immersion, you can give up that state of uncertainty.” 

When\marginnote{32.1} he said this, \textsanskrit{Pāṭaḷiya} the chief said to the Buddha, “Excellent, sir! Excellent! … From this day forth, may Master Gotama remember me as a lay follower who has gone for refuge for life.” 

\scendsutta{The Linked Discourses on chiefs are complete. }

%
\addtocontents{toc}{\let\protect\contentsline\protect\nopagecontentsline}
\part*{Linked Discourses on the Unconditioned }
\addcontentsline{toc}{part}{Linked Discourses on the Unconditioned }
\markboth{}{}
\addtocontents{toc}{\let\protect\contentsline\protect\oldcontentsline}

%
\addtocontents{toc}{\let\protect\contentsline\protect\nopagecontentsline}
\chapter*{Chapter One }
\addcontentsline{toc}{chapter}{\tocchapterline{Chapter One }}
\addtocontents{toc}{\let\protect\contentsline\protect\oldcontentsline}

%
\section*{{\suttatitleacronym SN 43.1}{\suttatitletranslation Mindfulness of the Body }{\suttatitleroot Kāyagatāsatisutta}}
\addcontentsline{toc}{section}{\tocacronym{SN 43.1} \toctranslation{Mindfulness of the Body } \tocroot{Kāyagatāsatisutta}}
\markboth{Mindfulness of the Body }{Kāyagatāsatisutta}
\extramarks{SN 43.1}{SN 43.1}

At\marginnote{1.1} \textsanskrit{Sāvatthī}. 

“Mendicants,\marginnote{1.2} I will teach you the unconditioned and the path that leads to the unconditioned. Listen … 

And\marginnote{1.4} what is the unconditioned? The ending of greed, hate, and delusion. This is called the unconditioned. And what is the path that leads to the unconditioned? Mindfulness of the body. This is called the path that leads to the unconditioned. 

So,\marginnote{2.1} mendicants, I’ve taught you the unconditioned and the path that leads to the unconditioned. Out of compassion, I’ve done what a teacher should do who wants what’s best for their disciples. Here are these roots of trees, and here are these empty huts. Practice absorption, mendicants! Don’t be negligent! Don’t regret it later! This is my instruction to you.” 

%
\section*{{\suttatitleacronym SN 43.2}{\suttatitletranslation Serenity and Discernment }{\suttatitleroot Samathavipassanāsutta}}
\addcontentsline{toc}{section}{\tocacronym{SN 43.2} \toctranslation{Serenity and Discernment } \tocroot{Samathavipassanāsutta}}
\markboth{Serenity and Discernment }{Samathavipassanāsutta}
\extramarks{SN 43.2}{SN 43.2}

“Mendicants,\marginnote{1.1} I will teach you the unconditioned and the path that leads to the unconditioned. Listen … 

And\marginnote{1.3} what is the unconditioned? The ending of greed, hate, and delusion. This is called the unconditioned. And what is the path that leads to the unconditioned? Serenity and discernment. This is called the path that leads to the unconditioned. …” 

%
\section*{{\suttatitleacronym SN 43.3}{\suttatitletranslation Placing the Mind and Keeping it Connected }{\suttatitleroot Savitakkasavicārasutta}}
\addcontentsline{toc}{section}{\tocacronym{SN 43.3} \toctranslation{Placing the Mind and Keeping it Connected } \tocroot{Savitakkasavicārasutta}}
\markboth{Placing the Mind and Keeping it Connected }{Savitakkasavicārasutta}
\extramarks{SN 43.3}{SN 43.3}

“And\marginnote{1.1} what is the path that leads to the unconditioned? Immersion with placing the mind and keeping it connected. Immersion without placing the mind, but just keeping it connected. Immersion without placing the mind or keeping it connected. …” 

%
\section*{{\suttatitleacronym SN 43.4}{\suttatitletranslation Emptiness Immersion }{\suttatitleroot Suññatasamādhisutta}}
\addcontentsline{toc}{section}{\tocacronym{SN 43.4} \toctranslation{Emptiness Immersion } \tocroot{Suññatasamādhisutta}}
\markboth{Emptiness Immersion }{Suññatasamādhisutta}
\extramarks{SN 43.4}{SN 43.4}

“And\marginnote{1.1} what is the path that leads to the unconditioned? Emptiness immersion; signless immersion; undirected immersion. …” 

%
\section*{{\suttatitleacronym SN 43.5}{\suttatitletranslation Mindfulness Meditation }{\suttatitleroot Satipaṭṭhānasutta}}
\addcontentsline{toc}{section}{\tocacronym{SN 43.5} \toctranslation{Mindfulness Meditation } \tocroot{Satipaṭṭhānasutta}}
\markboth{Mindfulness Meditation }{Satipaṭṭhānasutta}
\extramarks{SN 43.5}{SN 43.5}

“And\marginnote{1.1} what is the path that leads to the unconditioned? The four kinds of mindfulness meditation. …” 

%
\section*{{\suttatitleacronym SN 43.6}{\suttatitletranslation Right Efforts }{\suttatitleroot Sammappadhānasutta}}
\addcontentsline{toc}{section}{\tocacronym{SN 43.6} \toctranslation{Right Efforts } \tocroot{Sammappadhānasutta}}
\markboth{Right Efforts }{Sammappadhānasutta}
\extramarks{SN 43.6}{SN 43.6}

“And\marginnote{1.1} what is the path that leads to the unconditioned? The four right efforts. …” 

%
\section*{{\suttatitleacronym SN 43.7}{\suttatitletranslation Bases of Psychic Power }{\suttatitleroot Iddhipādasutta}}
\addcontentsline{toc}{section}{\tocacronym{SN 43.7} \toctranslation{Bases of Psychic Power } \tocroot{Iddhipādasutta}}
\markboth{Bases of Psychic Power }{Iddhipādasutta}
\extramarks{SN 43.7}{SN 43.7}

“And\marginnote{1.1} what is the path that leads to the unconditioned? The four bases of psychic power. …” 

%
\section*{{\suttatitleacronym SN 43.8}{\suttatitletranslation Faculties }{\suttatitleroot Indriyasutta}}
\addcontentsline{toc}{section}{\tocacronym{SN 43.8} \toctranslation{Faculties } \tocroot{Indriyasutta}}
\markboth{Faculties }{Indriyasutta}
\extramarks{SN 43.8}{SN 43.8}

“And\marginnote{1.1} what is the path that leads to the unconditioned? The five faculties. …” 

%
\section*{{\suttatitleacronym SN 43.9}{\suttatitletranslation Powers }{\suttatitleroot Balasutta}}
\addcontentsline{toc}{section}{\tocacronym{SN 43.9} \toctranslation{Powers } \tocroot{Balasutta}}
\markboth{Powers }{Balasutta}
\extramarks{SN 43.9}{SN 43.9}

“And\marginnote{1.1} what is the path that leads to the unconditioned? The five powers. …” 

%
\section*{{\suttatitleacronym SN 43.10}{\suttatitletranslation Awakening Factors }{\suttatitleroot Bojjhaṅgasutta}}
\addcontentsline{toc}{section}{\tocacronym{SN 43.10} \toctranslation{Awakening Factors } \tocroot{Bojjhaṅgasutta}}
\markboth{Awakening Factors }{Bojjhaṅgasutta}
\extramarks{SN 43.10}{SN 43.10}

“And\marginnote{1.1} what is the path that leads to the unconditioned? The seven awakening factors. …” 

%
\section*{{\suttatitleacronym SN 43.11}{\suttatitletranslation The Path }{\suttatitleroot Maggaṅgasutta}}
\addcontentsline{toc}{section}{\tocacronym{SN 43.11} \toctranslation{The Path } \tocroot{Maggaṅgasutta}}
\markboth{The Path }{Maggaṅgasutta}
\extramarks{SN 43.11}{SN 43.11}

“And\marginnote{1.1} what is the path that leads to the unconditioned? The noble eightfold path. This is called the path that leads to the unconditioned. 

So,\marginnote{1.4} mendicants, I’ve taught you the unconditioned and the path that leads to the unconditioned. Out of compassion, I’ve done what a teacher should do who wants what’s best for their disciples. Here are these roots of trees, and here are these empty huts. Practice absorption, mendicants! Don’t be negligent! Don’t regret it later! This is my instruction to you.” 

%
\addtocontents{toc}{\let\protect\contentsline\protect\nopagecontentsline}
\chapter*{Chapter Two }
\addcontentsline{toc}{chapter}{\tocchapterline{Chapter Two }}
\addtocontents{toc}{\let\protect\contentsline\protect\oldcontentsline}

%
\section*{{\suttatitleacronym SN 43.12}{\suttatitletranslation The Unconditioned }{\suttatitleroot Asaṅkhatasutta}}
\addcontentsline{toc}{section}{\tocacronym{SN 43.12} \toctranslation{The Unconditioned } \tocroot{Asaṅkhatasutta}}
\markboth{The Unconditioned }{Asaṅkhatasutta}
\extramarks{SN 43.12}{SN 43.12}

“Mendicants,\marginnote{1.1} I will teach you the unconditioned and the path that leads to the unconditioned. Listen … 

And\marginnote{1.3} what is the unconditioned? The ending of greed, hate, and delusion. This is called the unconditioned. 

And\marginnote{1.6} what is the path that leads to the unconditioned? Serenity. This is called the path that leads to the unconditioned. 

So,\marginnote{1.9} mendicants, I’ve taught you the unconditioned and the path that leads to the unconditioned. 

Out\marginnote{1.10} of compassion, I’ve done what a teacher should do who wants what’s best for their disciples. Here are these roots of trees, and here are these empty huts. Practice absorption, mendicants! Don’t be negligent! Don’t regret it later! This is my instruction to you.” 

“Mendicants,\marginnote{2.1} I will teach you the unconditioned and the path that leads to the unconditioned. Listen … 

And\marginnote{2.3} what is the unconditioned? The ending of greed, hate, and delusion. This is called the unconditioned. 

And\marginnote{2.6} what is the path that leads to the unconditioned? Discernment. This is called the path that leads to the unconditioned. …” 

“And\marginnote{3.1} what is the path that leads to the unconditioned? 

Immersion\marginnote{3.2} with placing the mind and keeping it connected. … Immersion without placing the mind, but just keeping it connected. … Immersion without placing the mind or keeping it connected. … 

Emptiness\marginnote{4.1} immersion. … Signless immersion. … Undirected immersion. … 

A\marginnote{5.1} mendicant meditates by observing an aspect of the body—keen, aware, and mindful, rid of desire and aversion for the world. … A mendicant meditates by observing an aspect of feelings … A mendicant meditates by observing an aspect of the mind … A mendicant meditates by observing an aspect of principles … 

A\marginnote{6.1} mendicant generates enthusiasm, tries, makes an effort, exerts the mind, and strives so that bad, unskillful qualities don’t arise. … A mendicant generates enthusiasm, tries, makes an effort, exerts the mind, and strives so that bad, unskillful qualities are given up. … A mendicant generates enthusiasm, tries, makes an effort, exerts the mind, and strives so that skillful qualities arise. … A mendicant generates enthusiasm, tries, makes an effort, exerts the mind, and strives so that skillful qualities that have arisen remain, are not lost, but increase, mature, and are fulfilled by development. 

A\marginnote{7.1} mendicant develops the basis of psychic power that has immersion due to enthusiasm, and active effort. … A mendicant develops the basis of psychic power that has immersion due to energy … immersion due to mental development … immersion due to inquiry, and active effort. … 

A\marginnote{8.1} mendicant develops the faculty of faith, which relies on seclusion, fading away, and cessation, and ripens as letting go. … A mendicant develops the faculty of energy … mindfulness … immersion … wisdom, which relies on seclusion, fading away, and cessation, and ripens as letting go. … 

A\marginnote{9.1} mendicant develops the power of faith … energy … mindfulness … immersion … wisdom, which relies on seclusion, fading away, and cessation, and ripens as letting go. … 

A\marginnote{10.1} mendicant develops the awakening factor of mindfulness … investigation of principles … energy … rapture … tranquility … immersion … equanimity, which relies on seclusion, fading away, and cessation, and ripens as letting go. 

A\marginnote{11.1} mendicant develops right view … right thought … right speech … right action … right livelihood … right effort … right mindfulness … right immersion, which relies on seclusion, fading away, and cessation, and ripens as letting go. 

This\marginnote{11.16} is called the path that leads to the unconditioned. 

So,\marginnote{11.17} mendicants, I’ve taught you the unconditioned and the path that leads to the unconditioned. 

Out\marginnote{11.18} of compassion, I’ve done what a teacher should do who wants what’s best for their disciples. Here are these roots of trees, and here are these empty huts. Practice absorption, mendicants! Don’t be negligent! Don’t regret it later! This is my instruction to you.” 

%
\section*{{\suttatitleacronym SN 43.13}{\suttatitletranslation Uninclined }{\suttatitleroot Anatasutta}}
\addcontentsline{toc}{section}{\tocacronym{SN 43.13} \toctranslation{Uninclined } \tocroot{Anatasutta}}
\markboth{Uninclined }{Anatasutta}
\extramarks{SN 43.13}{SN 43.13}

“Mendicants,\marginnote{1.1} I will teach you the uninclined …” 

(This\marginnote{1.4} should be expanded in detail as with the unconditioned in the previous chapter.) 

%
\section*{{\suttatitleacronym SN 43.14–43}{\suttatitletranslation Undefiled, Etc. }{\suttatitleroot Anāsavādisutta}}
\addcontentsline{toc}{section}{\tocacronym{SN 43.14–43} \toctranslation{Undefiled, Etc. } \tocroot{Anāsavādisutta}}
\markboth{Undefiled, Etc. }{Anāsavādisutta}
\extramarks{SN 43.14–43}{SN 43.14–43}

“Mendicants,\marginnote{1.1} I will teach you the undefiled … 

the\marginnote{1.1} truth … 

the\marginnote{1.1} far shore … 

the\marginnote{1.1} subtle … 

the\marginnote{1.1} very hard to see … 

the\marginnote{1.1} unaging … 

the\marginnote{1.1} constant … 

the\marginnote{1.1} not falling apart … 

the\marginnote{1.1} invisible … 

the\marginnote{1.1} unproliferated … 

the\marginnote{1.1} peaceful … 

the\marginnote{1.1} deathless … 

the\marginnote{1.1} sublime … 

the\marginnote{1.1} state of grace … 

the\marginnote{1.1} sanctuary … 

the\marginnote{1.1} ending of craving … 

the\marginnote{1.1} incredible … 

the\marginnote{1.1} amazing … 

the\marginnote{1.1} untroubled … 

the\marginnote{1.1} not liable to trouble … 

extinguishment\marginnote{1.1} … 

the\marginnote{1.1} unafflicted … 

dispassion\marginnote{1.1} … 

purity\marginnote{1.1} … 

freedom\marginnote{1.1} … 

not\marginnote{1.1} adhering … 

the\marginnote{1.1} island … 

the\marginnote{1.1} protection … 

the\marginnote{1.1} shelter … 

the\marginnote{1.1} refuge …” 

%
\section*{{\suttatitleacronym SN 43.44}{\suttatitletranslation The Haven }{\suttatitleroot Parāyanasutta}}
\addcontentsline{toc}{section}{\tocacronym{SN 43.44} \toctranslation{The Haven } \tocroot{Parāyanasutta}}
\markboth{The Haven }{Parāyanasutta}
\extramarks{SN 43.44}{SN 43.44}

“Mendicants,\marginnote{1.1} I will teach you the haven and the path that leads to the haven. Listen … 

And\marginnote{1.3} what is the haven? The ending of greed, hate, and delusion. This is called the haven. 

And\marginnote{1.6} what is the path that leads to the haven? Mindfulness of the body. This is called the path that leads to the haven. 

So,\marginnote{1.9} mendicants, I’ve taught you the haven and the path that leads to the haven. Out of compassion, I’ve done what a teacher should do who wants what’s best for their disciples. Here are these roots of trees, and here are these empty huts. Practice absorption, mendicants! Don’t be negligent! Don’t regret it later! This is my instruction to you.” 

(This\marginnote{1.12} should be expanded as with the unconditioned.) 

\scendsutta{The Linked Discourses on the unconditioned are complete. }

%
\addtocontents{toc}{\let\protect\contentsline\protect\nopagecontentsline}
\part*{Linked Discourses on the Undeclared }
\addcontentsline{toc}{part}{Linked Discourses on the Undeclared }
\markboth{}{}
\addtocontents{toc}{\let\protect\contentsline\protect\oldcontentsline}

%
\addtocontents{toc}{\let\protect\contentsline\protect\nopagecontentsline}
\chapter*{The Chapter on the Undeclared Points }
\addcontentsline{toc}{chapter}{\tocchapterline{The Chapter on the Undeclared Points }}
\addtocontents{toc}{\let\protect\contentsline\protect\oldcontentsline}

%
\section*{{\suttatitleacronym SN 44.1}{\suttatitletranslation With Khemā }{\suttatitleroot Khemāsutta}}
\addcontentsline{toc}{section}{\tocacronym{SN 44.1} \toctranslation{With Khemā } \tocroot{Khemāsutta}}
\markboth{With Khemā }{Khemāsutta}
\extramarks{SN 44.1}{SN 44.1}

At\marginnote{1.1} one time the Buddha was staying near \textsanskrit{Sāvatthī} in Jeta’s Grove, \textsanskrit{Anāthapiṇḍika}’s monastery. 

Now\marginnote{1.2} at that time the nun \textsanskrit{Khemā} was wandering in the land of the Kosalans between \textsanskrit{Sāvatthī} and \textsanskrit{Sāketa} when she took up residence in \textsanskrit{Toraṇavatthu}. Then King Pasenadi was traveling from \textsanskrit{Sāketa} to \textsanskrit{Sāvatthī}, and he too stayed in \textsanskrit{Toraṇavatthu} for a single night. 

Then\marginnote{1.4} King Pasenadi addressed a man, “Please, mister, check if there’s a suitable ascetic or brahmin in \textsanskrit{Toraṇavatthu} to whom I can pay homage.” 

“Yes,\marginnote{2.1} Your Majesty,” replied that man. He searched all over \textsanskrit{Toraṇavatthu}, but he couldn’t see a suitable ascetic or brahmin for the king to pay homage to. 

But\marginnote{2.2} he saw that the nun \textsanskrit{Khemā} was staying there, so he went to the king and said to him, “Your Majesty, there’s no ascetic or brahmin in \textsanskrit{Toraṇavatthu} for the king to pay homage to. But there is the nun \textsanskrit{Khemā}, who’s a disciple of the Blessed One, the perfected one, the fully awakened Buddha. She has a good reputation as being astute, competent, clever, learned, a brilliant speaker, and eloquent. Your Majesty may pay homage to her.” 

Then\marginnote{4.1} King Pasenadi of Kosala went up to the nun \textsanskrit{Khemā}, bowed, sat down to one side, and said to her: 

“Ma’am,\marginnote{4.2} does a Realized One exist after death?” 

“Great\marginnote{4.3} king, this has not been declared by the Buddha.” 

“Well\marginnote{4.5} then, does a Realized One not exist after death?” 

“This\marginnote{4.6} too has not been declared by the Buddha.” 

“Well\marginnote{4.8} then, does a Realized One both exist and not exist after death?” 

“This\marginnote{4.9} has not been declared by the Buddha.” 

“Well\marginnote{4.11} then, does a Realized One neither exist nor not exist after death?” 

“This\marginnote{4.12} too has not been declared by the Buddha.” 

“Ma’am,\marginnote{5.1} when asked these questions, you say that this has not been declared by the Buddha. What’s the cause, what’s the reason why this has not been declared by the Buddha?” 

“Well\marginnote{6.1} then, great king, I’ll ask you about this in return, and you can answer as you like. What do you think, great king? 

Is\marginnote{6.3} there any accountant or finger-tallier or reckoner who can count the grains of sand in the Ganges, that is, how many grains of sand there are, how many hundreds or thousands or hundreds of thousands of grains of sand?” 

“No,\marginnote{6.5} ma’am.” 

“Is\marginnote{6.6} there any accountant or finger-tallier or reckoner who can count the water in the ocean, that is, how many gallons of water there are, how many hundreds or thousands or hundreds of thousands of gallons of water?” 

“No,\marginnote{6.8} ma’am. Why is that? Because the ocean is deep, immeasurable, and hard to fathom.” 

“In\marginnote{6.11} the same way, great king, any form by which a Realized One might be described has been cut off at the root, made like a palm stump, obliterated, and unable to arise in the future. A Realized One is freed from reckoning in terms of form. They’re deep, immeasurable, and hard to fathom, like the ocean. To say that after death, a Realized One exists, or doesn’t exist, or both exists and doesn’t exist, or neither exists nor doesn’t exist: none of these apply. 

Any\marginnote{7.1} feeling … perception … choices … consciousness by which a Realized One might be described has been cut off at the root, made like a palm stump, obliterated, and unable to arise in the future. A Realized One is freed from reckoning in terms of consciousness. They’re deep, immeasurable, and hard to fathom, like the ocean. To say that after death, a Realized One exists, or doesn’t exist, or both exists and doesn’t exist, or neither exists nor doesn’t exist: none of these apply.” 

Then\marginnote{9.5} King Pasenadi approved and agreed with what the nun \textsanskrit{Khemā} said. Then he got up from his seat, bowed, and respectfully circled her, keeping her on his right, before leaving. 

Then\marginnote{10.1} on a later occasion King Pasenadi of Kosala went up to the Buddha, bowed, and sat down to one side. He asked the Buddha exactly the same questions he had asked the nun \textsanskrit{Khemā}, and received the same answers. 

He\marginnote{13.1} said, “It’s incredible, sir, it’s amazing! How the meaning and the phrasing of the teacher and the disciple fit together and agree without contradiction when it comes to the chief matter! This one time I went to the nun \textsanskrit{Khemā} and asked her about this matter. And she explained it to me with these words and phrases, just like the Buddha. It’s incredible, sir, it’s amazing! How the meaning and the phrasing of the teacher and the disciple fit together and agree without contradiction when it comes to the chief matter! 

Well,\marginnote{13.7} now, sir, I must go. I have many duties, and much to do.” 

“Please,\marginnote{13.9} great king, go at your convenience.” 

Then\marginnote{13.10} King Pasenadi approved and agreed with what the Buddha said. Then he got up from his seat, bowed, and respectfully circled him, keeping him on his right, before leaving. 

%
\section*{{\suttatitleacronym SN 44.2}{\suttatitletranslation With Anurādha }{\suttatitleroot Anurādhasutta}}
\addcontentsline{toc}{section}{\tocacronym{SN 44.2} \toctranslation{With Anurādha } \tocroot{Anurādhasutta}}
\markboth{With Anurādha }{Anurādhasutta}
\extramarks{SN 44.2}{SN 44.2}

At\marginnote{1.1} one time the Buddha was staying near \textsanskrit{Vesālī}, at the Great Wood, in the hall with the peaked roof. 

Now\marginnote{1.2} at that time Venerable \textsanskrit{Anurādha} was staying not far from the Buddha in a wilderness hut. Then several wanderers who follow other paths went up to Venerable \textsanskrit{Anurādha} and exchanged greetings with him. 

When\marginnote{1.4} the greetings and polite conversation were over, they sat down to one side and said to him: 

“Reverend\marginnote{1.5} \textsanskrit{Anurādha}, when a Realized One is describing a Realized One—a supreme person, highest of people, who has reached the highest point—they describe them in these four ways: After death, a Realized One exists, or doesn’t exist, or both exists and doesn’t exist, or neither exists nor doesn’t exist.” 

“Reverends,\marginnote{1.7} when a Realized One is describing a Realized One—a supreme person, highest of people, who has reached the highest point—they describe them other than these four ways: After death, a Realized One exists, or doesn’t exist, or both exists and doesn’t exist, or neither exists nor doesn’t exist.” 

When\marginnote{1.9} he said this, the wanderers said to him, “This mendicant must be junior, recently gone forth, or else a foolish, incompetent senior mendicant.” Then, after rebuking Venerable \textsanskrit{Anurādha} by calling him “junior” and “foolish”, the wanderers got up from their seat and left. 

Soon\marginnote{2.1} after they had left, \textsanskrit{Anurādha} thought, “If those wanderers were to inquire further, how should I answer them so as to repeat what the Buddha has said, and not misrepresent him with an untruth? How should I explain in line with his teaching, so that there would be no legitimate grounds for rebuke and criticism?” 

Then\marginnote{2.4} Venerable \textsanskrit{Anurādha} went up to the Buddha, bowed, sat down to one side, and told him what had happened. 

“What\marginnote{3.1} do you think, \textsanskrit{Anurādha}? Is form permanent or impermanent?” 

“Impermanent,\marginnote{4.1} sir.” 

“But\marginnote{5.1} if it’s impermanent, is it suffering or happiness?” 

“Suffering,\marginnote{6.1} sir.” 

“But\marginnote{7.1} if it’s impermanent, suffering, and liable to wear out, is it fit to be regarded thus: ‘This is mine, I am this, this is my self’?” 

“No,\marginnote{8.1} sir.” 

“Is\marginnote{9.1} feeling … perception … choices … consciousness permanent or impermanent?” 

“Impermanent,\marginnote{10.1} sir.” 

“But\marginnote{11.1} if it’s impermanent, is it suffering or happiness?” 

“Suffering,\marginnote{12.1} sir.” 

“But\marginnote{13.1} if it’s impermanent, suffering, and liable to wear out, is it fit to be regarded thus: ‘This is mine, I am this, this is my self’?” 

“No,\marginnote{14.1} sir.” 

“So,\marginnote{15.1} \textsanskrit{Anurādha}, you should truly see any kind of form at all—past, future, or present; internal or external; coarse or fine; inferior or superior; far or near: \emph{all} form—with right understanding: ‘This is not mine, I am not this, this is not my self.’ Any kind of feeling … perception … choices … consciousness at all—past, future, or present; internal or external; coarse or fine; inferior or superior; far or near: \emph{all} consciousness—with right understanding: ‘This is not mine, I am not this, this is not my self.’ 

Seeing\marginnote{15.6} this, a learned noble disciple grows disillusioned with form, feeling, perception, choices, and consciousness. Being disillusioned, desire fades away. When desire fades away they’re freed. When they’re freed, they know they’re freed. 

They\marginnote{15.8} understand: ‘Rebirth is ended, the spiritual journey has been completed, what had to be done has been done, there is no return to any state of existence.’ 

What\marginnote{16.1} do you think, \textsanskrit{Anurādha}? Do you regard the Realized One as form?” 

“No,\marginnote{16.3} sir.” 

“Do\marginnote{16.4} you regard the Realized One as feeling … perception … choices … consciousness?” 

“No,\marginnote{16.11} sir.” 

“What\marginnote{16.12} do you think, \textsanskrit{Anurādha}? Do you regard the Realized One as in form?” 

“No,\marginnote{16.14} sir.” 

“Or\marginnote{16.15} do you regard the Realized One as distinct from form?” 

“No,\marginnote{16.16} sir.” 

“Do\marginnote{16.17} you regard the Realized One as in feeling … or distinct from feeling … as in perception … or distinct from perception … as in choices … or distinct from choices … as in consciousness … or as distinct from consciousness?” 

“No,\marginnote{16.26} sir.” 

“What\marginnote{17.1} do you think, \textsanskrit{Anurādha}? Do you regard the Realized One as possessing form, feeling, perception, choices, and consciousness?” 

“No,\marginnote{17.3} sir.” 

“What\marginnote{17.4} do you think, \textsanskrit{Anurādha}? Do you regard the Realized One as one who is without form, feeling, perception, choices, and consciousness?” 

“No,\marginnote{17.6} sir.” 

“In\marginnote{17.7} that case, \textsanskrit{Anurādha}, since you don’t acknowledge the Realized One as a genuine fact in the present life, is it appropriate to declare: ‘Reverends, when a Realized One is describing a Realized One—a supreme person, highest of people, who has reached the highest point—they describe them other than these four ways: After death, a Realized One exists, or doesn’t exist, or both exists and doesn’t exist, or neither exists nor doesn’t exist’?” 

“No,\marginnote{17.11} sir.” 

“Good,\marginnote{17.12} good, \textsanskrit{Anurādha}! In the past, as today, what I describe is suffering and the cessation of suffering.” 

%
\section*{{\suttatitleacronym SN 44.3}{\suttatitletranslation With Sāriputta and Koṭṭhita (1st) }{\suttatitleroot Paṭhamasāriputtakoṭṭhikasutta}}
\addcontentsline{toc}{section}{\tocacronym{SN 44.3} \toctranslation{With Sāriputta and Koṭṭhita (1st) } \tocroot{Paṭhamasāriputtakoṭṭhikasutta}}
\markboth{With Sāriputta and Koṭṭhita (1st) }{Paṭhamasāriputtakoṭṭhikasutta}
\extramarks{SN 44.3}{SN 44.3}

At\marginnote{1.1} one time Venerable \textsanskrit{Sāriputta} and Venerable \textsanskrit{Mahākoṭṭhita} were staying near Benares, in the deer park at Isipatana. 

Then\marginnote{1.2} in the late afternoon, Venerable \textsanskrit{Mahākoṭṭhita} came out of retreat, went to Venerable \textsanskrit{Sāriputta}, and exchanged greetings with him. When the greetings and polite conversation were over, he sat down to one side, and said to \textsanskrit{Sāriputta}: 

“Reverend\marginnote{2.1} \textsanskrit{Sāriputta}, does a Realized One exist after death?” 

“Reverend,\marginnote{2.2} this has not been declared by the Buddha.” 

“Well\marginnote{2.4} then, does a Realized One not exist after death? … 

Does\marginnote{2.7} a Realized One both exist and not exist after death? … 

Does\marginnote{2.10} a Realized One neither exist nor not exist after death?” 

“This\marginnote{2.11} too has not been declared by the Buddha.” 

“Reverend,\marginnote{3.1} when asked these questions, you say that they have not been declared by the Buddha. What’s the cause, what’s the reason why they have not been declared by the Buddha?” 

“Reverend,\marginnote{4.1} ‘does a Realized One exist after death?’ is included in form. ‘Does a Realized One not exist after death?’ is included in form. ‘Does a Realized One both exist and not exist after death?’ is included in form. ‘Does a Realized One neither exist nor not exist after death?’ is included in form. 

‘Does\marginnote{4.5} a Realized One exist after death?’ is included in feeling … perception … choices … consciousness. ‘Does a Realized One not exist after death?’ is included in consciousness. ‘Does a Realized One both exist and not exist after death?’ is included in consciousness. ‘Does a Realized One neither exist nor not exist after death?’ is included in consciousness. 

This\marginnote{4.21} is the cause, this is the reason why this has not been declared by the Buddha.” 

%
\section*{{\suttatitleacronym SN 44.4}{\suttatitletranslation With Sāriputta and Koṭṭhita (2nd) }{\suttatitleroot Dutiyasāriputtakoṭṭhikasutta}}
\addcontentsline{toc}{section}{\tocacronym{SN 44.4} \toctranslation{With Sāriputta and Koṭṭhita (2nd) } \tocroot{Dutiyasāriputtakoṭṭhikasutta}}
\markboth{With Sāriputta and Koṭṭhita (2nd) }{Dutiyasāriputtakoṭṭhikasutta}
\extramarks{SN 44.4}{SN 44.4}

At\marginnote{1.1} one time Venerable \textsanskrit{Sāriputta} and Venerable \textsanskrit{Mahākoṭṭhita} were staying near Benares, in the deer park at Isipatana. … 

(The\marginnote{1.2} same down as far as:) “What’s the cause, reverend, what’s the reason why this has not been declared by the Buddha?” 

“Reverend,\marginnote{1.4} not truly knowing and seeing form, its origin, its cessation, and the practice that leads to its cessation, one thinks ‘a Realized One exists after death’ or ‘a Realized One doesn’t exist after death’ or ‘a Realized One both exists and doesn’t exist after death’ or ‘a Realized One neither exists nor doesn’t exist after death.’ 

Not\marginnote{1.8} truly knowing or seeing feeling … perception … choices … consciousness, its origin, its cessation, and the practice that leads to its cessation, one thinks ‘a Realized One exists after death’ or ‘a Realized One doesn’t exist after death’ or ‘a Realized One both exists and doesn’t exist after death’ or ‘A Realized One neither exists nor doesn’t exist after death.’ 

Truly\marginnote{2.1} knowing and seeing form … feeling … perception … choices … consciousness, its origin, its cessation, and the practice that leads to its cessation, one doesn’t think ‘a Realized One exists after death’ or ‘a Realized One doesn’t exist after death’ or ‘a Realized One both exists and doesn’t exist after death’ or ‘a Realized One neither exists nor doesn’t exist after death.’ 

This\marginnote{2.10} is the cause, this is the reason why this has not been declared by the Buddha.” 

%
\section*{{\suttatitleacronym SN 44.5}{\suttatitletranslation With Sāriputta and Koṭṭhita (3rd) }{\suttatitleroot Tatiyasāriputtakoṭṭhikasutta}}
\addcontentsline{toc}{section}{\tocacronym{SN 44.5} \toctranslation{With Sāriputta and Koṭṭhita (3rd) } \tocroot{Tatiyasāriputtakoṭṭhikasutta}}
\markboth{With Sāriputta and Koṭṭhita (3rd) }{Tatiyasāriputtakoṭṭhikasutta}
\extramarks{SN 44.5}{SN 44.5}

At\marginnote{1.1} one time Venerable \textsanskrit{Sāriputta} and Venerable \textsanskrit{Mahākoṭṭhita} were staying near Benares, in the deer park at Isipatana. … 

(The\marginnote{1.2} same down as far as:) “What’s the cause, reverend, what’s the reason why this has not been declared by the Buddha?” 

“Reverend,\marginnote{1.4} if you’re not rid of greed, desire, fondness, thirst, passion, and craving for form … feeling … perception … choices … consciousness, you think ‘a Realized One exists after death’ … ‘a Realized One neither exists nor doesn’t exist after death.’ 

If\marginnote{1.11} you are rid of greed for form … feeling … perception … choices … consciousness, you don’t think ‘a Realized One exists after death’ … ‘a Realized One neither exists nor doesn’t exist after death.’ 

This\marginnote{1.17} is the cause, this is the reason why this has not been declared by the Buddha.” 

%
\section*{{\suttatitleacronym SN 44.6}{\suttatitletranslation With Sāriputta and Koṭṭhita (4th) }{\suttatitleroot Catutthasāriputtakoṭṭhikasutta}}
\addcontentsline{toc}{section}{\tocacronym{SN 44.6} \toctranslation{With Sāriputta and Koṭṭhita (4th) } \tocroot{Catutthasāriputtakoṭṭhikasutta}}
\markboth{With Sāriputta and Koṭṭhita (4th) }{Catutthasāriputtakoṭṭhikasutta}
\extramarks{SN 44.6}{SN 44.6}

At\marginnote{1.1} one time Venerable \textsanskrit{Sāriputta} and Venerable \textsanskrit{Mahākoṭṭhita} were staying near Benares, in the deer park at Isipatana. 

Then\marginnote{1.2} in the late afternoon, Venerable \textsanskrit{Sāriputta} came out of retreat, went to Venerable \textsanskrit{Mahākoṭṭhita}, and they greeted each other. When the greetings and polite conversation were over, he sat down to one side and said to \textsanskrit{Mahākoṭṭhita}: 

“Reverend\marginnote{1.4} \textsanskrit{Koṭṭhita}, does a Realized One exist after death?” … 

“Reverend,\marginnote{1.5} when asked these questions, you say that this has not been declared by the Buddha. What’s the cause, what’s the reason why this has not been declared by the Buddha?” 

“Reverend,\marginnote{2.1} if you like, love, and enjoy form, and don’t truly see the cessation of form, you think ‘a Realized One exists after death’ or ‘a Realized One doesn’t exist after death’ or ‘a Realized One both exists and doesn’t exist after death’ or ‘a Realized One neither exists nor doesn’t exist after death.’ 

If\marginnote{2.5} you like, love, and enjoy feeling … perception … choices … consciousness, and don’t truly see the cessation of consciousness, you think ‘a Realized One exists after death’ … ‘a Realized One neither exists nor doesn’t exist after death.’ 

If\marginnote{3.1} you don’t like, love, and enjoy form … feeling … perception … choices … consciousness, and you truly see the cessation of consciousness, you don’t think ‘a Realized One exists after death’ … ‘a Realized One neither exists nor doesn’t exist after death.’ 

This\marginnote{3.8} is the cause, this is the reason why this has not been declared by the Buddha.” 

“But\marginnote{4.1} reverend, could there be another way of explaining why this was not declared by the Buddha?” 

“There\marginnote{4.2} could, reverend. If you like, love, and enjoy existence, and don’t truly see the cessation of continued existence, you think ‘a Realized One exists after death’ … ‘a Realized One neither exists nor doesn’t exist after death.’ If you don’t like, love, and enjoy existence, and you truly see the cessation of continued existence, you don’t think ‘a Realized One exists after death’ … ‘a Realized One neither exists nor doesn’t exist after death.’ This too is a way of explaining why this was not declared by the Buddha.” 

“But\marginnote{5.1} reverend, could there be another way of explaining why this was not declared by the Buddha?” 

“There\marginnote{5.2} could, reverend. 

If\marginnote{5.3} you like, love, and enjoy grasping, and don’t truly see the cessation of grasping, you think ‘a Realized One exists after death’ … ‘a Realized One neither exists nor doesn’t exist after death.’ 

If\marginnote{5.5} you don’t like, love, and enjoy grasping, and you truly see the cessation of grasping, you don’t think ‘a Realized One exists after death’ … ‘a Realized One neither exists nor doesn’t exist after death.’ 

This\marginnote{5.7} too is a way of explaining why this was not declared by the Buddha.” 

“But\marginnote{6.1} reverend, could there be another way of explaining why this was not declared by the Buddha?” 

“There\marginnote{6.2} could, reverend. 

If\marginnote{6.3} you like, love, and enjoy craving, and don’t truly see the cessation of craving, you think ‘a Realized One exists after death’ … ‘a Realized One neither exists nor doesn’t exist after death.’ 

If\marginnote{6.5} you don’t like, love, and enjoy craving, and you truly see the cessation of craving, you don’t think ‘a Realized One exists after death’ … ‘a Realized One neither exists nor doesn’t exist after death.’ 

This\marginnote{6.7} too is a way of explaining why this was not declared by the Buddha.” 

“But\marginnote{7.1} reverend, could there be another way of explaining why this was not declared by the Buddha?” 

“Seriously,\marginnote{7.2} reverend, what more could you want? For one who is freed due to the ending of craving, there is no cycle of rebirths to be found.” 

%
\section*{{\suttatitleacronym SN 44.7}{\suttatitletranslation With Moggallāna }{\suttatitleroot Moggallānasutta}}
\addcontentsline{toc}{section}{\tocacronym{SN 44.7} \toctranslation{With Moggallāna } \tocroot{Moggallānasutta}}
\markboth{With Moggallāna }{Moggallānasutta}
\extramarks{SN 44.7}{SN 44.7}

Then\marginnote{1.1} the wanderer Vacchagotta went up to Venerable \textsanskrit{Mahāmoggallāna}, and exchanged greetings with him. When the greetings and polite conversation were over, he sat down to one side, and said to \textsanskrit{Mahāmoggallāna}: 

“Master\marginnote{2.1} \textsanskrit{Moggallāna}, is this right: ‘the cosmos is eternal’?” 

“Vaccha,\marginnote{2.2} this has not been declared by the Buddha.” 

“Then\marginnote{2.4} is this right: ‘the cosmos is not eternal’ … ‘the world is finite’ … ‘the world is infinite’ … ‘the soul and the body are identical’ … ‘the soul and the body are different things’ … ‘a Realized One exists after death’ … ‘a Realized One doesn’t survive after death’ … ‘a Realized One both exists and doesn’t exist after death’ … ‘a Realized One neither exists nor doesn’t exist after death’?” 

“This\marginnote{2.29} too has not been declared by the Buddha.” 

“What’s\marginnote{3.1} the cause, Master \textsanskrit{Moggallāna}, what’s the reason why, when the wanderers who follow other paths are asked these questions, they declare one of these to be true? And what’s the reason why, when the ascetic Gotama is asked these questions, he does not declare one of these to be true?” 

“Vaccha,\marginnote{4.1} the wanderers who follow other paths regard the eye like this: ‘This is mine, I am this, this is my self.’ They regard the ear … nose … tongue … body … mind like this: ‘This is mine, I am this, this is my self.’ 

That’s\marginnote{4.4} why, when asked, they declare one of those answers to be true. 

The\marginnote{4.7} Realized One, the perfected one, the fully awakened Buddha regards the eye like this: ‘This is not mine, I am not this, this is not my self.’ He regards the ear … nose … tongue … body … mind like this: ‘This is not mine, I am not this, this is not my self.’ 

That’s\marginnote{4.10} why, when asked, he does not declare one of those answers to be true.” 

Then\marginnote{5.1} the wanderer Vacchagotta got up from his seat and went to the Buddha and exchanged greetings with him. When the greetings and polite conversation were over, he sat down to one side. He asked the Buddha the same questions, and received the same answers. 

He\marginnote{8.1} said, “It’s incredible, Master Gotama, it’s amazing! How the meaning and the phrasing of the teacher and the disciple fit together and agree without contradiction when it comes to the chief matter! Just now I went to the ascetic \textsanskrit{Mahāmoggallāna} and asked him about this matter. And he explained it to me with these words and phrases, just like Master Gotama. It’s incredible, Master Gotama, it’s amazing! How the meaning and the phrasing of the teacher and the disciple fit together and agree without contradiction when it comes to the chief matter!” 

%
\section*{{\suttatitleacronym SN 44.8}{\suttatitletranslation With Vacchagotta }{\suttatitleroot Vacchagottasutta}}
\addcontentsline{toc}{section}{\tocacronym{SN 44.8} \toctranslation{With Vacchagotta } \tocroot{Vacchagottasutta}}
\markboth{With Vacchagotta }{Vacchagottasutta}
\extramarks{SN 44.8}{SN 44.8}

Then\marginnote{1.1} the wanderer Vacchagotta went up to the Buddha and exchanged greetings with him. When the greetings and polite conversation were over, he sat down to one side, and said to the Buddha: 

“Master\marginnote{1.3} Gotama, is this right: ‘the cosmos is eternal’?” 

“This\marginnote{1.4} has not been declared by me, Vaccha.” … 

“Then\marginnote{1.5} is this right: ‘a Realized One neither exists nor doesn’t exist after death’?” 

“This\marginnote{1.6} too has not been declared by me.” 

“What’s\marginnote{2.1} the cause, Master Gotama, what’s the reason why the wanderers who follow other paths answer these questions when asked? And what’s the cause, what’s the reason why Master Gotama doesn’t answer these questions when asked?” 

“Vaccha,\marginnote{3.1} the wanderers who follow other paths regard form as self, self as having form, form in self, or self in form. They regard feeling … perception … choices … consciousness as self, self as having consciousness, consciousness in self, or self in consciousness. 

That’s\marginnote{3.6} why they answer these questions when asked. 

The\marginnote{3.9} Realized One doesn’t regard form as self, self as having form, form in self, or self in form. He doesn’t regard feeling … perception … choices … consciousness as self, self as having consciousness, consciousness in self, or self in consciousness. 

That’s\marginnote{3.14} why he doesn’t answer these questions when asked.” 

Then\marginnote{4.1} the wanderer Vacchagotta got up from his seat and went to Venerable \textsanskrit{Mahāmoggallāna}, and exchanged greetings with him. When the greetings and polite conversation were over, he sat down to one side. He asked \textsanskrit{Mahāmoggallāna} the same questions, and received the same answers. 

He\marginnote{7.1} said, “It’s incredible, Master \textsanskrit{Moggallāna}, it’s amazing. How the meaning and the phrasing of the teacher and the disciple fit together and agree without contradiction when it comes to the chief matter! Just now I went to the ascetic Gotama and asked him about this matter. And he explained it to me with these words and phrases, just like Master \textsanskrit{Moggallāna}. It’s incredible, Master \textsanskrit{Moggallāna}, it’s amazing! How the meaning and the phrasing of the teacher and the disciple fit together and agree without contradiction when it comes to the chief matter!” 

%
\section*{{\suttatitleacronym SN 44.9}{\suttatitletranslation The Debating Hall }{\suttatitleroot Kutūhalasālāsutta}}
\addcontentsline{toc}{section}{\tocacronym{SN 44.9} \toctranslation{The Debating Hall } \tocroot{Kutūhalasālāsutta}}
\markboth{The Debating Hall }{Kutūhalasālāsutta}
\extramarks{SN 44.9}{SN 44.9}

Then\marginnote{1.1} the wanderer Vacchagotta went up to the Buddha and exchanged greetings with him. When the greetings and polite conversation were over, he sat down to one side, and said to the Buddha: 

“Master\marginnote{2.1} Gotama, a few days ago several ascetics, brahmins, and wanderers who follow various other paths were sitting together in the debating hall, and this discussion came up among them: ‘This \textsanskrit{Pūraṇa} Kassapa leads an order and a community, and teaches a community. He’s a well-known and famous religious founder, regarded as holy by many people. When a disciple passes away, he declares that this one is reborn here, while that one is reborn there. And as for a disciple who is a supreme person, highest of people, having reached the highest point, when they pass away he also declares that this one is reborn here, while that one is reborn there. 

This\marginnote{3.1} Makkhali \textsanskrit{Gosāla} … \textsanskrit{Nigaṇṭha} \textsanskrit{Nāṭaputta} … \textsanskrit{Sañjaya} \textsanskrit{Belaṭṭhiputta} … Pakudha \textsanskrit{Kaccāyana} … Ajita Kesakambala leads an order and a community, and teaches a community. He’s a well-known and famous religious founder, regarded as holy by many people. When a disciple passes away, he declares that this one is reborn here, while that one is reborn there. And as for a disciple who is a supreme person, highest of people, having reached the highest point, when they pass away he also declares that this one is reborn here, while that one is reborn there. 

This\marginnote{4.1} ascetic Gotama leads an order and a community, and teaches a community. He’s a well-known and famous religious founder, regarded as holy by many people. When a disciple passes away, he declares that this one is reborn here, while that one is reborn there. 

But\marginnote{4.4} as for a disciple who is a supreme person, highest of people, having reached the highest point, when they pass away he doesn’t declare that this one is reborn here, while that one is reborn there. Rather, he declares that they have cut off craving, untied the fetters, and by rightly comprehending conceit have made an end of suffering. I had doubt and uncertainty about that: ‘How on earth can I understand the ascetic Gotama’s teaching?’” 

“Vaccha,\marginnote{5.1} no wonder you’re doubting and uncertain. Doubt has come up in you about an uncertain matter. 

I\marginnote{5.3} describe rebirth for someone who grasps fuel, not for someone who doesn’t grasp fuel. It’s like a fire which only burns with fuel, not without fuel. In the same way I describe rebirth for someone who grasps fuel, not for someone who doesn’t grasp fuel.” 

“But\marginnote{6.1} when a flame is blown away by the wind, what do you say is its fuel then?” 

“At\marginnote{6.2} such a time, I say that it’s fueled by wind. For the wind is its fuel then.” 

“But\marginnote{6.4} when someone who is attached has laid down this body and has not been reborn in one of the realms, what does Master Gotama say is their fuel then?” 

“When\marginnote{6.5} someone who is attached has laid down this body, Vaccha, and has not been reborn in one of the realms, I say they’re fueled by craving. For craving is their fuel then.” 

%
\section*{{\suttatitleacronym SN 44.10}{\suttatitletranslation With Ānanda }{\suttatitleroot Ānandasutta}}
\addcontentsline{toc}{section}{\tocacronym{SN 44.10} \toctranslation{With Ānanda } \tocroot{Ānandasutta}}
\markboth{With Ānanda }{Ānandasutta}
\extramarks{SN 44.10}{SN 44.10}

Then\marginnote{1.1} the wanderer Vacchagotta went up to the Buddha and exchanged greetings with him. When the greetings and polite conversation were over, he sat down to one side and said to the Buddha: 

“Master\marginnote{1.3} Gotama, does the self survive?” But when he said this, the Buddha kept silent. 

“Then\marginnote{1.5} does the self not survive?” But for a second time the Buddha kept silent. Then the wanderer Vacchagotta got up from his seat and left. 

And\marginnote{2.1} then, not long after Vacchagotta had left, Venerable Ānanda said to the Buddha: 

“Sir,\marginnote{2.2} why didn’t you answer Vacchagotta’s question?” 

“Ānanda,\marginnote{2.3} when Vacchagotta asked me whether the self survives, if I had answered that ‘the self survives’ I would have been siding with the ascetics and brahmins who are eternalists. When Vacchagotta asked me whether the self does not survive, if I had answered that ‘the self does not survive’ I would have been siding with the ascetics and brahmins who are annihilationists. 

When\marginnote{2.5} Vacchagotta asked me whether the self survives, if I had answered that ‘the self survives’ would that have helped give rise to the knowledge that all things are not-self?” 

“No,\marginnote{2.7} sir.” 

“When\marginnote{2.8} Vacchagotta asked me whether the self does not survive, if I had answered that ‘the self does not survive’, Vacchagotta—who is already confused—would have got even more confused, thinking: ‘It seems that the self that I once had no longer survives.’” 

%
\section*{{\suttatitleacronym SN 44.11}{\suttatitletranslation With Sabhiya Kaccāna }{\suttatitleroot Sabhiyakaccānasutta}}
\addcontentsline{toc}{section}{\tocacronym{SN 44.11} \toctranslation{With Sabhiya Kaccāna } \tocroot{Sabhiyakaccānasutta}}
\markboth{With Sabhiya Kaccāna }{Sabhiyakaccānasutta}
\extramarks{SN 44.11}{SN 44.11}

At\marginnote{1.1} one time Venerable Sabhiya \textsanskrit{Kaccāna} was staying at \textsanskrit{Nādika} in the brick house. Then the wanderer Vacchagotta went up to him, and exchanged greetings with him. When the greetings and polite conversation were over, he sat down to one side, and said to Sabhiya \textsanskrit{Kaccāna}: 

“Master\marginnote{1.4} \textsanskrit{Kaccāna}, does a Realized One exist after death?” 

“Vaccha,\marginnote{1.5} this has not been declared by the Buddha.” 

“Well\marginnote{1.7} then, does a Realized One not exist after death?” 

“This\marginnote{1.8} too has not been declared by the Buddha.” 

“Well\marginnote{2.1} then, does a Realized One both exist and not exist after death?” 

“This\marginnote{2.2} has not been declared by the Buddha.” 

“Well\marginnote{2.4} then, does a Realized One neither exist nor not exist after death?” 

“This\marginnote{2.5} too has not been declared by the Buddha.” 

“Master\marginnote{3.1} \textsanskrit{Kaccāna}, when asked these questions, you say that this has not been declared by the Buddha. What’s the cause, what’s the reason why this has not been declared by the Buddha?” 

“In\marginnote{3.14} order to describe him as ‘possessing form’ or ‘formless’ or ‘percipient’ or ‘non-percipient’ or ‘neither percipient nor non-percipient’, there must be some cause or reason for doing so. But if that cause and reason were to totally and utterly cease without anything left over, how could you describe him in any such terms?” 

“Master\marginnote{3.16} \textsanskrit{Kaccāna}, how long has it been since you went forth?” 

“Not\marginnote{3.17} long, reverend: three years.” 

“Well,\marginnote{3.18} you’ve learned a lot already, let alone what lies ahead!” 

\scendsutta{The Linked Discourses on undeclared questions are complete. }

\scendbook{The Book of the Six Sense Fields is finished. }

%
\backmatter%
\chapter*{Colophon}
\addcontentsline{toc}{chapter}{Colophon}
\markboth{Colophon}{Colophon}

\section*{The Translator}

Bhikkhu Sujato was born as Anthony Aidan Best on 4/11/1966 in Perth, Western Australia. He grew up in the pleasant suburbs of Mt Lawley and Attadale alongside his sister Nicola, who was the good child. His mother, Margaret Lorraine Huntsman née Pinder, said “he’ll either be a priest or a poet”, while his father, Anthony Thomas Best, advised him to “never do anything for money”. He attended Aquinas College, a Catholic school, where he decided to become an atheist. At the University of WA he studied philosophy, aiming to learn what he wanted to do with his life. Finding that what he wanted to do was play guitar, he dropped out. His main band was named Martha’s Vineyard, which achieved modest success in the indie circuit. 

A seemingly random encounter with a roadside joey took him to Thailand, where he entered his first meditation retreat at Wat Ram Poeng, Chieng Mai in 1992. Feeling the call to the Buddha’s path, he took full ordination in Wat Pa Nanachat in 1994, where his teachers were Ajahn Pasanno and Ajahn Jayasaro. In 1997 he returned to Perth to study with Ajahn Brahm at Bodhinyana Monastery. 

He spent several years practicing in seclusion in Malaysia and Thailand before establishing Santi Forest Monastery in Bundanoon, NSW, in 2003. There he was instrumental in supporting the establishment of the Theravada bhikkhuni order in Australia and advocating for women’s rights. He continues to teach in Australia and globally, with a special concern for the moral implications of climate change and other forms of environmental destruction. He has published a series of books of original and groundbreaking research on early Buddhism. 

In 2005 he founded SuttaCentral together with Rod Bucknell and John Kelly. In 2015, seeing the need for a complete, accurate, plain English translation of the Pali texts, he undertook the task, spending nearly three years in isolation on the isle of Qi Mei off the coast of the nation of Taiwan. He completed the four main \textsanskrit{Nikāyas} in 2018, and the early books of the Khuddaka \textsanskrit{Nikāya} were complete by 2021. All this work is dedicated to the public domain and is entirely free of copyright encumbrance. 

In 2019 he returned to Sydney where he established Lokanta Vihara (The Monastery at the End of the World). 

\section*{Creation Process}

Primary source was the digital \textsanskrit{Mahāsaṅgīti} edition of the Pali \textsanskrit{Tipiṭaka}. Translated from the Pali, with reference to several English translations, especially those of Bhikkhu Bodhi.

\section*{The Translation}

This translation was part of a project to translate the four Pali \textsanskrit{Nikāyas} with the following aims: plain, approachable English; consistent terminology; accurate rendition of the Pali; free of copyright. It was made during 2016–2018 while Bhikkhu Sujato was staying in Qimei, Taiwan.

\section*{About SuttaCentral}

SuttaCentral publishes early Buddhist texts. Since 2005 we have provided root texts in Pali, Chinese, Sanskrit, Tibetan, and other languages, parallels between these texts, and translations in many modern languages. We build on the work of generations of scholars, and offer our contribution freely.

SuttaCentral is driven by volunteer contributions, and in addition we employ professional developers. We offer a sponsorship program for high quality translations from the original languages. Financial support for SuttaCentral is handled by the SuttaCentral Development Trust, a charitable trust registered in Australia.

\section*{About Bilara}

“Bilara” means “cat” in Pali, and it is the name of our Computer Assisted Translation (CAT) software. Bilara is a web app that enables translators to translate early Buddhist texts into their own language. These translations are published on SuttaCentral with the root text and translation side by side.

\section*{About SuttaCentral Editions}

The SuttaCentral Editions project makes high quality books from selected Bilara translations. These are published in formats including HTML, EPUB, PDF, and print.

If you want to print any of our Editions, please let us know and we will help prepare a file to your specifications.

%
\end{document}