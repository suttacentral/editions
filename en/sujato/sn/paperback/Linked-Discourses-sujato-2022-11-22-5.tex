\documentclass[12pt,openany]{book}%
\usepackage{lastpage}%
%
\usepackage[inner=1in, outer=1in, top=.7in, bottom=1in, papersize={6in,9in}, headheight=13pt]{geometry}
\usepackage{polyglossia}
\usepackage[12pt]{moresize}
\usepackage{soul}%
\usepackage{microtype}
\usepackage{tocbasic}
\usepackage{realscripts}
\usepackage{epigraph}%
\usepackage{setspace}%
\usepackage{sectsty}
\usepackage{fontspec}
\usepackage{marginnote}
\usepackage[bottom]{footmisc}
\usepackage{enumitem}
\usepackage{fancyhdr}
\usepackage{extramarks}
\usepackage{graphicx}
\usepackage{verse}
\usepackage{relsize}
\usepackage{etoolbox}
\usepackage[a-3u]{pdfx}

\hypersetup{
colorlinks=true,
urlcolor=black,
linkcolor=black,
citecolor=black
}

% use a small amount of tracking on small caps
\SetTracking[ spacing = {25*,166, } ]{ encoding = *, shape = sc }{ 25 }

% add a blank page
\newcommand{\blankpage}{
\newpage
\thispagestyle{empty}
\mbox{}
\newpage
}

% define languages
\setdefaultlanguage[]{english}
\setotherlanguage[script=Latin]{sanskrit}

%\usepackage{pagegrid}
%\pagegridsetup{top-left, step=.25in}

% define fonts
% use if arno sanskrit is unavailable
%\setmainfont{Gentium Plus}
%\newfontfamily\Semiboldsubheadfont[]{Gentium Plus}
%\newfontfamily\Semiboldnormalfont[]{Gentium Plus}
%\newfontfamily\Lightfont[]{Gentium Plus}
%\newfontfamily\Marginalfont[]{Gentium Plus}
%\newfontfamily\Allsmallcapsfont[RawFeature=+c2sc]{Gentium Plus}
%\newfontfamily\Noligaturefont[Renderer=Basic]{Gentium Plus}
%\newfontfamily\Noligaturecaptionfont[Renderer=Basic]{Gentium Plus}
%\newfontfamily\Fleuronfont[Ornament=1]{Gentium Plus}

% use if arno sanskrit is available. display is applied to \chapter and \part, subhead to \section and \subsection. When specifying semibold, the italic must be defined.
\setmainfont[Numbers=OldStyle]{Arno Pro}
\newfontfamily\Semibolddisplayfont[BoldItalicFont = Arno Pro Semibold Italic Display]{Arno Pro Semibold Display} %
\newfontfamily\Semiboldsubheadfont[BoldItalicFont = Arno Pro Semibold Italic Subhead]{Arno Pro Semibold Subhead}
\newfontfamily\Semiboldnormalfont[BoldItalicFont = Arno Pro Semibold Italic]{Arno Pro Semibold}
\newfontfamily\Marginalfont[RawFeature=+subs]{Arno Pro Regular}
\newfontfamily\Allsmallcapsfont[RawFeature=+c2sc]{Arno Pro}
\newfontfamily\Noligaturefont[Renderer=Basic]{Arno Pro}
\newfontfamily\Noligaturecaptionfont[Renderer=Basic]{Arno Pro Caption}

% chinese fonts
\newfontfamily\cjk{Noto Serif TC}
\newcommand*{\langlzh}[1]{\cjk{#1}\normalfont}%

% logo
\newfontfamily\Logofont{sclogo.ttf}
\newcommand*{\sclogo}[1]{\large\Logofont{#1}}

% use subscript numerals for margin notes
\renewcommand*{\marginfont}{\Marginalfont}

% ensure margin notes have consistent vertical alignment
\renewcommand*{\marginnotevadjust}{-.17em}

% use compact lists
\setitemize{noitemsep,leftmargin=1em}
\setenumerate{noitemsep,leftmargin=1em}
\setdescription{noitemsep, style=unboxed, leftmargin=0em}

% style ToC
\DeclareTOCStyleEntries[
  raggedentrytext,
  linefill=\hfill,
  pagenumberwidth=.5in,
  pagenumberformat=\normalfont,
  entryformat=\normalfont
]{tocline}{chapter,section}


  \setlength\topsep{0pt}%
  \setlength\parskip{0pt}%

% define new \centerpars command for use in ToC. This ensures centering, proper wrapping, and no page break after
\def\startcenter{%
  \par
  \begingroup
  \leftskip=0pt plus 1fil
  \rightskip=\leftskip
  \parindent=0pt
  \parfillskip=0pt
}
\def\stopcenter{%
  \par
  \endgroup
}
\long\def\centerpars#1{\startcenter#1\stopcenter}

% redefine part, so that it adds a toc entry without page number
\let\oldcontentsline\contentsline
\newcommand{\nopagecontentsline}[3]{\oldcontentsline{#1}{#2}{}}

    \makeatletter
\renewcommand*\l@part[2]{%
  \ifnum \c@tocdepth >-2\relax
    \addpenalty{-\@highpenalty}%
    \addvspace{0em \@plus\p@}%
    \setlength\@tempdima{3em}%
    \begingroup
      \parindent \z@ \rightskip \@pnumwidth
      \parfillskip -\@pnumwidth
      {\leavevmode
       \setstretch{.85}\large\scshape\centerpars{#1}\vspace*{-1em}\llap{#2}}\par
       \nobreak
         \global\@nobreaktrue
         \everypar{\global\@nobreakfalse\everypar{}}%
    \endgroup
  \fi}
\makeatother

\makeatletter
\def\@pnumwidth{2em}
\makeatother

% define new sectioning command, which is only used in volumes where the pannasa is found in some parts but not others, especially in an and sn

\newcommand*{\pannasa}[1]{\clearpage\thispagestyle{empty}\begin{center}\vspace*{14em}\setstretch{.85}\huge\itshape\scshape\MakeLowercase{#1}\end{center}}

    \makeatletter
\newcommand*\l@pannasa[2]{%
  \ifnum \c@tocdepth >-2\relax
    \addpenalty{-\@highpenalty}%
    \addvspace{.5em \@plus\p@}%
    \setlength\@tempdima{3em}%
    \begingroup
      \parindent \z@ \rightskip \@pnumwidth
      \parfillskip -\@pnumwidth
      {\leavevmode
       \setstretch{.85}\large\itshape\scshape\lowercase{\centerpars{#1}}\vspace*{-1em}\llap{#2}}\par
       \nobreak
         \global\@nobreaktrue
         \everypar{\global\@nobreakfalse\everypar{}}%
    \endgroup
  \fi}
\makeatother

% don't put page number on first page of toc (relies on etoolbox)
\patchcmd{\chapter}{plain}{empty}{}{}

% global line height
\setstretch{1.05}

% allow linebreak after em-dash
\catcode`\—=13
\protected\def—{\unskip\textemdash\allowbreak}

% style headings with secsty. chapter and section are defined per-edition
\partfont{\setstretch{.85}\normalfont\centering\textsc}
\subsectionfont{\setstretch{.85}\Semiboldsubheadfont}%
\subsubsectionfont{\setstretch{.85}\Semiboldnormalfont}

% style elements of suttatitle
\newcommand*{\suttatitleacronym}[1]{\smaller[2]{#1}\vspace*{.3em}}
\newcommand*{\suttatitletranslation}[1]{\linebreak{#1}}
\newcommand*{\suttatitleroot}[1]{\linebreak\smaller[2]\itshape{#1}}

\DeclareTOCStyleEntries[
  indent=3.3em,
  dynindent,
  beforeskip=.2em plus -2pt minus -1pt,
]{tocline}{section}

\DeclareTOCStyleEntries[
  indent=0em,
  dynindent,
  beforeskip=.4em plus -2pt minus -1pt,
]{tocline}{chapter}

\newcommand*{\tocacronym}[1]{\hspace*{-3.3em}{#1}\quad}
\newcommand*{\toctranslation}[1]{#1}
\newcommand*{\tocroot}[1]{(\textit{#1})}
\newcommand*{\tocchapterline}[1]{\bfseries\itshape{#1}}


% redefine paragraph and subparagraph headings to not be inline
\makeatletter
% Change the style of paragraph headings %
\renewcommand\paragraph{\@startsection{paragraph}{4}{\z@}%
            {-2.5ex\@plus -1ex \@minus -.25ex}%
            {1.25ex \@plus .25ex}%
            {\noindent\Semiboldnormalfont\normalsize}}

% Change the style of subparagraph headings %
\renewcommand\subparagraph{\@startsection{subparagraph}{5}{\z@}%
            {-2.5ex\@plus -1ex \@minus -.25ex}%
            {1.25ex \@plus .25ex}%
            {\noindent\Semiboldnormalfont\small}}
\makeatother

% use etoolbox to suppress page numbers on \part
\patchcmd{\part}{\thispagestyle{plain}}{\thispagestyle{empty}}
  {}{\errmessage{Cannot patch \string\part}}

% and to reduce margins on quotation
\patchcmd{\quotation}{\rightmargin}{\leftmargin 1.2em \rightmargin}{}{}
\AtBeginEnvironment{quotation}{\small}

% titlepage
\newcommand*{\titlepageTranslationTitle}[1]{{\begin{center}\begin{large}{#1}\end{large}\end{center}}}
\newcommand*{\titlepageCreatorName}[1]{{\begin{center}\begin{normalsize}{#1}\end{normalsize}\end{center}}}

% halftitlepage
\newcommand*{\halftitlepageTranslationTitle}[1]{\setstretch{2.5}{\begin{Huge}\uppercase{\so{#1}}\end{Huge}}}
\newcommand*{\halftitlepageTranslationSubtitle}[1]{\setstretch{1.2}{\begin{large}{#1}\end{large}}}
\newcommand*{\halftitlepageFleuron}[1]{{\begin{large}\Fleuronfont{{#1}}\end{large}}}
\newcommand*{\halftitlepageByline}[1]{{\begin{normalsize}\textit{{#1}}\end{normalsize}}}
\newcommand*{\halftitlepageCreatorName}[1]{{\begin{LARGE}{\textsc{#1}}\end{LARGE}}}
\newcommand*{\halftitlepageVolumeNumber}[1]{{\begin{normalsize}{\Allsmallcapsfont{\textsc{#1}}}\end{normalsize}}}
\newcommand*{\halftitlepageVolumeAcronym}[1]{{\begin{normalsize}{#1}\end{normalsize}}}
\newcommand*{\halftitlepageVolumeTranslationTitle}[1]{{\begin{Large}{\textsc{#1}}\end{Large}}}
\newcommand*{\halftitlepageVolumeRootTitle}[1]{{\begin{normalsize}{\Allsmallcapsfont{\textsc{\itshape #1}}}\end{normalsize}}}
\newcommand*{\halftitlepagePublisher}[1]{{\begin{large}{\Noligaturecaptionfont\textsc{#1}}\end{large}}}

% epigraph
\renewcommand{\epigraphflush}{center}
\renewcommand*{\epigraphwidth}{.85\textwidth}
\newcommand*{\epigraphTranslatedTitle}[1]{\vspace*{.5em}\footnotesize\textsc{#1}\\}%
\newcommand*{\epigraphRootTitle}[1]{\footnotesize\textit{#1}\\}%
\newcommand*{\epigraphReference}[1]{\footnotesize{#1}}%

% custom commands for html styling classes
\newcommand*{\scnamo}[1]{\begin{center}\textit{#1}\end{center}}
\newcommand*{\scendsection}[1]{\begin{center}\textit{#1}\end{center}}
\newcommand*{\scendsutta}[1]{\begin{center}\textit{#1}\end{center}}
\newcommand*{\scendbook}[1]{\begin{center}\uppercase{#1}\end{center}}
\newcommand*{\scendkanda}[1]{\begin{center}\textbf{#1}\end{center}}
\newcommand*{\scend}[1]{\begin{center}\textit{#1}\end{center}}
\newcommand*{\scuddanaintro}[1]{\textit{#1}}
\newcommand*{\scendvagga}[1]{\begin{center}\textbf{#1}\end{center}}
\newcommand*{\scrule}[1]{\textbf{#1}}
\newcommand*{\scadd}[1]{\textit{#1}}
\newcommand*{\scevam}[1]{\textsc{#1}}
\newcommand*{\scspeaker}[1]{\hspace{2em}\textit{#1}}
\newcommand*{\scbyline}[1]{\begin{flushright}\textit{#1}\end{flushright}\bigskip}

% custom command for thematic break = hr
\newcommand*{\thematicbreak}{\begin{center}\rule[.5ex]{6em}{.4pt}\begin{normalsize}\quad\Fleuronfont{•}\quad\end{normalsize}\rule[.5ex]{6em}{.4pt}\end{center}}

% manage and style page header and footer. "fancy" has header and footer, "plain" has footer only

\pagestyle{fancy}
\fancyhf{}
\fancyfoot[RE,LO]{\thepage}
\fancyfoot[LE,RO]{\footnotesize\lastleftxmark}
\fancyhead[CE]{\setstretch{.85}\Noligaturefont\MakeLowercase{\textsc{\firstrightmark}}}
\fancyhead[CO]{\setstretch{.85}\Noligaturefont\MakeLowercase{\textsc{\firstleftmark}}}
\renewcommand{\headrulewidth}{0pt}
\fancypagestyle{plain}{ %
\fancyhf{} % remove everything
\fancyfoot[RE,LO]{\thepage}
\fancyfoot[LE,RO]{\footnotesize\lastleftxmark}
\renewcommand{\headrulewidth}{0pt}
\renewcommand{\footrulewidth}{0pt}}

% style footnotes
\setlength{\skip\footins}{1em}

\makeatletter
\newcommand{\@makefntextcustom}[1]{%
    \parindent 0em%
    \thefootnote.\enskip #1%
}
\renewcommand{\@makefntext}[1]{\@makefntextcustom{#1}}
\makeatother

% hang quotes (requires microtype)
\microtypesetup{
  protrusion = true,
  expansion  = true,
  tracking   = true,
  factor     = 1000,
  patch      = all,
  final
}

% Custom protrusion rules to allow hanging punctuation
\SetProtrusion
{ encoding = *}
{
% char   right left
  {-} = {    , 500 },
  % Double Quotes
  \textquotedblleft
      = {1000,     },
  \textquotedblright
      = {    , 1000},
  \quotedblbase
      = {1000,     },
  % Single Quotes
  \textquoteleft
      = {1000,     },
  \textquoteright
      = {    , 1000},
  \quotesinglbase
      = {1000,     }
}

% make latex use actual font em for parindent, not Computer Modern Roman
\AtBeginDocument{\setlength{\parindent}{1em}}%
%

% Default values; a bit sloppier than normal
\tolerance 1414
\hbadness 1414
\emergencystretch 1.5em
\hfuzz 0.3pt
\clubpenalty = 10000
\widowpenalty = 10000
\displaywidowpenalty = 10000
\hfuzz \vfuzz
 \raggedbottom%

\title{Linked Discourses}
\author{Bhikkhu Sujato}
\date{}%
% define a different fleuron for each edition
\newfontfamily\Fleuronfont[Ornament=40]{Arno Pro}

% Define heading styles per edition for chapter and section. Suttatitle can be either of these, depending on the volume. 

\let\oldfrontmatter\frontmatter
\renewcommand{\frontmatter}{%
\chapterfont{\setstretch{.85}\normalfont\centering}%
\sectionfont{\setstretch{.85}\Semiboldsubheadfont}%
\oldfrontmatter}

\let\oldmainmatter\mainmatter
\renewcommand{\mainmatter}{%
\chapterfont{\setstretch{.85}\normalfont\centering}%
\sectionfont{\setstretch{.85}\normalfont\centering}%
\oldmainmatter}

\let\oldbackmatter\backmatter
\renewcommand{\backmatter}{%
\chapterfont{\setstretch{.85}\normalfont\centering}%
\sectionfont{\setstretch{.85}\Semiboldsubheadfont}%
\oldbackmatter}
%
%
\begin{document}%
\normalsize%
\frontmatter%
\setlength{\parindent}{0cm}

\pagestyle{empty}

\maketitle

\blankpage%
\begin{center}

\vspace*{2.2em}

\halftitlepageTranslationTitle{Linked Discourses}

\vspace*{1em}

\halftitlepageTranslationSubtitle{A plain translation of the Saṁyutta Nikāya}

\vspace*{2em}

\halftitlepageFleuron{•}

\vspace*{2em}

\halftitlepageByline{translated and introduced by}

\vspace*{.5em}

\halftitlepageCreatorName{Bhikkhu Sujato}

\vspace*{4em}

\halftitlepageVolumeNumber{Volume 5}

\smallskip

\halftitlepageVolumeAcronym{SN 45–56}

\smallskip

\halftitlepageVolumeTranslationTitle{The Group of Linked Discourses on the Path}

\smallskip

\halftitlepageVolumeRootTitle{Mahāvaggasaṁyutta}

\vspace*{\fill}

\sclogo{0}
 \halftitlepagePublisher{SuttaCentral}

\end{center}

\newpage
%
\setstretch{1.05}

\begin{footnotesize}

\textit{Linked Discourses} is a translation of the Saṁyuttanikāya by Bhikkhu Sujato.

\medskip

Creative Commons Zero (CC0)

To the extent possible under law, Bhikkhu Sujato has waived all copyright and related or neighboring rights to \textit{Linked Discourses}.

\medskip

This work is published from Australia.

\begin{center}
\textit{This translation is an expression of an ancient spiritual text that has been passed down by the Buddhist tradition for the benefit of all sentient beings. It is dedicated to the public domain via Creative Commons Zero (CC0). You are encouraged to copy, reproduce, adapt, alter, or otherwise make use of this translation. The translator respectfully requests that any use be in accordance with the values and principles of the Buddhist community.}
\end{center}

\medskip

\begin{description}
    \item[Web publication date] 2018
    \item[This edition] 2022-11-22 08:17:58
    \item[Publication type] paperback
    \item[Edition] ed5
    \item[Number of volumes] 5
    \item[Publication ISBN] 978-1-76132-086-6
    \item[Publication URL] https://suttacentral.net/editions/sn/en/sujato
    \item[Source URL] https://github.com/suttacentral/bilara-data/tree/published/translation/en/sujato/sutta/sn
    \item[Publication number] scpub4
\end{description}

\medskip

Published by SuttaCentral

\medskip

\textit{SuttaCentral,\\
c/o Alwis \& Alwis Pty Ltd\\
Kaurna Country,\\
Suite 12,\\
198 Greenhill Road,\\
Eastwood,\\
SA 5063,\\
Australia}

\end{footnotesize}

\newpage

\setlength{\parindent}{1.5em}%%
\tableofcontents
\newpage
\pagestyle{fancy}
%
\mainmatter%
\pagestyle{fancy}%
%
%
\addtocontents{toc}{\let\protect\contentsline\protect\nopagecontentsline}
\part*{Linked Discourses on the Eightfold Path }
\addcontentsline{toc}{part}{Linked Discourses on the Eightfold Path }
\markboth{}{}
\addtocontents{toc}{\let\protect\contentsline\protect\oldcontentsline}

%
\addtocontents{toc}{\let\protect\contentsline\protect\nopagecontentsline}
\chapter*{The Chapter on Ignorance }
\addcontentsline{toc}{chapter}{\tocchapterline{The Chapter on Ignorance }}
\addtocontents{toc}{\let\protect\contentsline\protect\oldcontentsline}

%
\section*{{\suttatitleacronym SN 45.1}{\suttatitletranslation Ignorance }{\suttatitleroot Avijjāsutta}}
\addcontentsline{toc}{section}{\tocacronym{SN 45.1} \toctranslation{Ignorance } \tocroot{Avijjāsutta}}
\markboth{Ignorance }{Avijjāsutta}
\extramarks{SN 45.1}{SN 45.1}

\scevam{So\marginnote{1.1} I have heard. }At one time the Buddha was staying near \textsanskrit{Sāvatthī} in Jeta’s Grove, \textsanskrit{Anāthapiṇḍika}’s monastery. There the Buddha addressed the mendicants, “Mendicants!” 

“Venerable\marginnote{1.5} sir,” they replied. The Buddha said this: 

“Mendicants,\marginnote{2.1} ignorance precedes the attainment of unskillful qualities, with lack of conscience and prudence following along. An ignoramus, sunk in ignorance, gives rise to wrong view. Wrong view gives rise to wrong thought. Wrong thought gives rise to wrong speech. Wrong speech gives rise to wrong action. Wrong action gives rise to wrong livelihood. Wrong livelihood gives rise to wrong effort. Wrong effort gives rise to wrong mindfulness. Wrong mindfulness gives rise to wrong immersion. 

Knowledge\marginnote{3.1} precedes the attainment of skillful qualities, with conscience and prudence following along. A sage, firm in knowledge, gives rise to right view. Right view gives rise to right thought. Right thought gives rise to right speech. Right speech gives rise to right action. Right action gives rise to right livelihood. Right livelihood gives rise to right effort. Right effort gives rise to right mindfulness. Right mindfulness gives rise to right immersion.” 

%
\section*{{\suttatitleacronym SN 45.2}{\suttatitletranslation Half the Spiritual Life }{\suttatitleroot Upaḍḍhasutta}}
\addcontentsline{toc}{section}{\tocacronym{SN 45.2} \toctranslation{Half the Spiritual Life } \tocroot{Upaḍḍhasutta}}
\markboth{Half the Spiritual Life }{Upaḍḍhasutta}
\extramarks{SN 45.2}{SN 45.2}

\scevam{So\marginnote{1.1} I have heard. }At one time the Buddha was staying in the land of the Sakyans, where they have a town named Townsville. Then Venerable Ānanda went up to the Buddha, bowed, sat down to one side, and said to him: 

“Sir,\marginnote{1.4} good friends, companions, and associates are half the spiritual life.” 

“Not\marginnote{2.1} so, Ānanda! Not so, Ānanda! Good friends, companions, and associates are the whole of the spiritual life. A mendicant with good friends, companions, and associates can expect to develop and cultivate the noble eightfold path. 

And\marginnote{3.1} how does a mendicant with good friends develop and cultivate the noble eightfold path? It’s when a mendicant develops right view, which relies on seclusion, fading away, and cessation, and ripens as letting go. They develop right thought … right speech … right action … right livelihood … right effort … right mindfulness … right immersion, which relies on seclusion, fading away, and cessation, and ripens as letting go. That’s how a mendicant with good friends develops and cultivates the noble eightfold path. 

And\marginnote{4.1} here’s another way to understand how good friends are the whole of the spiritual life. For, by relying on me as a good friend, sentient beings who are liable to rebirth, old age, and death, to sorrow, lamentation, pain, sadness, and distress are freed from all these things. This is another way to understand how good friends are the whole of the spiritual life.” 

%
\section*{{\suttatitleacronym SN 45.3}{\suttatitletranslation Sāriputta }{\suttatitleroot Sāriputtasutta}}
\addcontentsline{toc}{section}{\tocacronym{SN 45.3} \toctranslation{Sāriputta } \tocroot{Sāriputtasutta}}
\markboth{Sāriputta }{Sāriputtasutta}
\extramarks{SN 45.3}{SN 45.3}

At\marginnote{1.1} \textsanskrit{Sāvatthī}. 

Then\marginnote{1.2} \textsanskrit{Sāriputta} went up to the Buddha, bowed, sat down to one side, and said to him: 

“Sir,\marginnote{1.3} good friends, companions, and associates are the whole of the spiritual life.” 

“Good,\marginnote{2.1} good, \textsanskrit{Sāriputta}! Good friends, companions, and associates are the whole of the spiritual life. A mendicant with good friends, companions, and associates can expect to develop and cultivate the noble eightfold path. And how does a mendicant with good friends develop and cultivate the noble eightfold path? 

It’s\marginnote{3.1} when a mendicant develops right view, right thought, right speech, right action, right livelihood, right effort, right mindfulness, and right immersion, which rely on seclusion, fading away, and cessation, and ripen as letting go. That’s how a mendicant with good friends develops and cultivates the noble eightfold path. 

And\marginnote{4.1} here’s another way to understand how good friends are the whole of the spiritual life. For, by relying on me as a good friend, sentient beings who are liable to rebirth, old age, and death, to sorrow, lamentation, pain, sadness, and distress are freed from all these things. This is another way to understand how good friends are the whole of the spiritual life.” 

%
\section*{{\suttatitleacronym SN 45.4}{\suttatitletranslation Regarding the Brahmin Jāṇussoṇi }{\suttatitleroot Jāṇussoṇibrāhmaṇasutta}}
\addcontentsline{toc}{section}{\tocacronym{SN 45.4} \toctranslation{Regarding the Brahmin Jāṇussoṇi } \tocroot{Jāṇussoṇibrāhmaṇasutta}}
\markboth{Regarding the Brahmin Jāṇussoṇi }{Jāṇussoṇibrāhmaṇasutta}
\extramarks{SN 45.4}{SN 45.4}

At\marginnote{1.1} \textsanskrit{Sāvatthī}. 

Then\marginnote{1.2} Venerable Ānanda robed up in the morning and, taking his bowl and robe, entered \textsanskrit{Sāvatthī} for alms. He saw the brahmin \textsanskrit{Jāṇussoṇi} driving out of \textsanskrit{Sāvatthī} in a splendid all-white chariot drawn by mares. The yoked horses were pure white, as were the ornaments, chariot, upholstery, reins, goad, and canopy. And his turban, robes, sandals were white, as was the chowry fanning him. 

When\marginnote{1.5} people saw it they exclaimed, “Wow! That’s a \textsanskrit{Brahmā} vehicle! It’s a vehicle fit for \textsanskrit{Brahmā}!” 

Then\marginnote{2.1} Ānanda wandered for alms in \textsanskrit{Sāvatthī}. After the meal, on his return from almsround, he went to the Buddha, bowed, sat down to one side, and told him what had happened, adding, “Sir, can you point out a \textsanskrit{Brahmā} vehicle in this teaching and training?” 

“I\marginnote{4.1} can, Ānanda,” said the Buddha. 

“These\marginnote{4.2} are all terms for the noble eightfold path: ‘vehicle of \textsanskrit{Brahmā}’, or else ‘vehicle of truth’, or else ‘supreme victory in battle’. 

When\marginnote{5.1} right view is developed and cultivated it culminates with the removal of greed, hate, and delusion. When right thought … right speech … right action … right livelihood … right effort … right mindfulness … right immersion is developed and cultivated it culminates with the removal of greed, hate, and delusion. 

This\marginnote{6.1} is a way to understand how these are all terms for the noble eightfold path: ‘vehicle of \textsanskrit{Brahmā}’, or else ‘vehicle of truth’, or else ‘supreme victory in battle’.” 

That\marginnote{6.3} is what the Buddha said. 

Then\marginnote{7.1} the Holy One, the Teacher, went on to say: 

\begin{verse}%
“Its\marginnote{8.1} qualities of faith and wisdom \\
are always yoked to the shaft. \\
Conscience is its pole, mind its strap, \\
and mindfulness its careful driver. 

The\marginnote{9.1} chariot’s equipped with ethics, \\
its axle is absorption, and energy its wheel. \\
Equanimity and immersion are the carriage-shaft, \\
and it’s upholstered with desirelessness. 

Good\marginnote{10.1} will, harmlessness, and seclusion \\
are its weapons, \\
patience its shield and armor, \\
as it rolls on to sanctuary. 

This\marginnote{11.1} supreme \textsanskrit{Brahmā} vehicle \\
arises in oneself. \\
The wise leave the world in it, \\
sure of winning the victory.” 

%
\end{verse}

%
\section*{{\suttatitleacronym SN 45.5}{\suttatitletranslation What’s the Purpose }{\suttatitleroot Kimatthiyasutta}}
\addcontentsline{toc}{section}{\tocacronym{SN 45.5} \toctranslation{What’s the Purpose } \tocroot{Kimatthiyasutta}}
\markboth{What’s the Purpose }{Kimatthiyasutta}
\extramarks{SN 45.5}{SN 45.5}

At\marginnote{1.1} \textsanskrit{Sāvatthī}. 

Then\marginnote{1.2} several mendicants went up to the Buddha … and said to him: 

“Sir,\marginnote{2.1} sometimes wanderers who follow other paths ask us: ‘Reverends, what’s the purpose of leading the spiritual life under the ascetic Gotama?’ We answer them like this: ‘The purpose of leading the spiritual life under the Buddha is to completely understand suffering.’ 

Answering\marginnote{2.5} this way, we trust that we repeat what the Buddha has said, and don’t misrepresent him with an untruth. We trust our explanation is in line with the teaching, and that there are no legitimate grounds for rebuke or criticism.” 

“Indeed,\marginnote{3.1} in answering this way you repeat what I’ve said, and don’t misrepresent me with an untruth. Your explanation is in line with the teaching, and there are no legitimate grounds for rebuke or criticism. For the purpose of leading the spiritual life under me is to completely understand suffering. 

If\marginnote{3.3} wanderers who follow other paths were to ask you: ‘Is there a path and a practice for completely understanding that suffering?’ You should answer them like this: ‘There is.’ 

And\marginnote{4.1} what is that path? It is simply this noble eightfold path, that is: right view, right thought, right speech, right action, right livelihood, right effort, right mindfulness, and right immersion. This is the path and the practice for completely understanding suffering. When questioned by wanderers who follow other paths, that’s how you should answer them.” 

%
\section*{{\suttatitleacronym SN 45.6}{\suttatitletranslation A Mendicant (1st) }{\suttatitleroot Paṭhamaaññatarabhikkhusutta}}
\addcontentsline{toc}{section}{\tocacronym{SN 45.6} \toctranslation{A Mendicant (1st) } \tocroot{Paṭhamaaññatarabhikkhusutta}}
\markboth{A Mendicant (1st) }{Paṭhamaaññatarabhikkhusutta}
\extramarks{SN 45.6}{SN 45.6}

At\marginnote{1.1} \textsanskrit{Sāvatthī}. 

Then\marginnote{1.2} a mendicant went up to the Buddha … and asked him, “Sir, they speak of this thing called the ‘spiritual path’. What is the spiritual path? And what is the culmination of the spiritual path?” 

“Mendicant,\marginnote{2.1} the spiritual path is simply this noble eightfold path, that is: right view, right thought, right speech, right action, right livelihood, right effort, right mindfulness, and right immersion. The ending of greed, hate, and delusion. This is the culmination of the spiritual path.” 

%
\section*{{\suttatitleacronym SN 45.7}{\suttatitletranslation A Mendicant (2nd) }{\suttatitleroot Dutiyaaññatarabhikkhusutta}}
\addcontentsline{toc}{section}{\tocacronym{SN 45.7} \toctranslation{A Mendicant (2nd) } \tocroot{Dutiyaaññatarabhikkhusutta}}
\markboth{A Mendicant (2nd) }{Dutiyaaññatarabhikkhusutta}
\extramarks{SN 45.7}{SN 45.7}

At\marginnote{1.1} \textsanskrit{Sāvatthī}. 

Then\marginnote{1.2} a mendicant went up to the Buddha … and said to him: 

“Sir,\marginnote{2.1} they speak of ‘the removal of greed, hate, and delusion’. What is this a term for?” 

“Mendicant,\marginnote{2.4} the removal of greed, hate, and delusion is a term for the element of extinguishment. It’s used to speak of the ending of defilements.” 

When\marginnote{3.1} he said this, the mendicant said to the Buddha: 

“Sir,\marginnote{3.2} they speak of ‘the deathless’. What is the deathless? And what is the path that leads to the deathless?” 

“The\marginnote{3.4} ending of greed, hate, and delusion. This is called the deathless. The path that leads to the deathless is simply this noble eightfold path, that is: right view, right thought, right speech, right action, right livelihood, right effort, right mindfulness, and right immersion.” 

%
\section*{{\suttatitleacronym SN 45.8}{\suttatitletranslation Analysis }{\suttatitleroot Vibhaṅgasutta}}
\addcontentsline{toc}{section}{\tocacronym{SN 45.8} \toctranslation{Analysis } \tocroot{Vibhaṅgasutta}}
\markboth{Analysis }{Vibhaṅgasutta}
\extramarks{SN 45.8}{SN 45.8}

At\marginnote{1.1} \textsanskrit{Sāvatthī}. 

“Mendicants,\marginnote{1.2} I will teach and analyze for you the noble eightfold path. Listen and pay close attention, I will speak.” 

“Yes,\marginnote{1.4} sir,” they replied. The Buddha said this: 

“And\marginnote{2.1} what is the noble eightfold path? It is right view, right thought, right speech, right action, right livelihood, right effort, right mindfulness, and right immersion. 

And\marginnote{3.1} what is right view? Knowing about suffering, the origin of suffering, the cessation of suffering, and the practice that leads to the cessation of suffering. This is called right view. 

And\marginnote{4.1} what is right thought? It is the thought of renunciation, good will, and harmlessness. This is called right thought. 

And\marginnote{5.1} what is right speech? Avoiding speech that’s false, divisive, harsh, or nonsensical. This is called right speech. 

And\marginnote{6.1} what is right action? Avoiding killing living creatures, stealing, and sexual activity. This is called right action. 

And\marginnote{7.1} what is right livelihood? It’s when a noble disciple gives up wrong livelihood and earns a living by right livelihood. This is called right livelihood. 

And\marginnote{8.1} what is right effort? It’s when a mendicant generates enthusiasm, tries, makes an effort, exerts the mind, and strives so that bad, unskillful qualities don’t arise. They generate enthusiasm, try, make an effort, exert the mind, and strive so that bad, unskillful qualities that have arisen are given up. They generate enthusiasm, try, make an effort, exert the mind, and strive so that skillful qualities that have not arisen do arise. They generate enthusiasm, try, make an effort, exert the mind, and strive so that skillful qualities that have arisen remain, are not lost, but increase, mature, and are fulfilled by development. This is called right effort. 

And\marginnote{9.1} what is right mindfulness? It’s when a mendicant meditates by observing an aspect of the body—keen, aware, and mindful, rid of desire and aversion for the world. They meditate observing an aspect of feelings—keen, aware, and mindful, rid of desire and aversion for the world. They meditate observing an aspect of the mind—keen, aware, and mindful, rid of desire and aversion for the world. They meditate observing an aspect of principles—keen, aware, and mindful, rid of desire and aversion for the world. This is called right mindfulness. 

And\marginnote{10.1} what is right immersion? It’s when a mendicant, quite secluded from sensual pleasures, secluded from unskillful qualities, enters and remains in the first absorption, which has the rapture and bliss born of seclusion, while placing the mind and keeping it connected. As the placing of the mind and keeping it connected are stilled, they enter and remain in the second absorption, which has the rapture and bliss born of immersion, with internal clarity and confidence, and unified mind, without placing the mind and keeping it connected. And with the fading away of rapture, they enter and remain in the third absorption, where they meditate with equanimity, mindful and aware, personally experiencing the bliss of which the noble ones declare, ‘Equanimous and mindful, one meditates in bliss.’ Giving up pleasure and pain, and ending former happiness and sadness, they enter and remain in the fourth absorption, without pleasure or pain, with pure equanimity and mindfulness. This is called right immersion.” 

%
\section*{{\suttatitleacronym SN 45.9}{\suttatitletranslation A Spike }{\suttatitleroot Sūkasutta}}
\addcontentsline{toc}{section}{\tocacronym{SN 45.9} \toctranslation{A Spike } \tocroot{Sūkasutta}}
\markboth{A Spike }{Sūkasutta}
\extramarks{SN 45.9}{SN 45.9}

At\marginnote{1.1} \textsanskrit{Sāvatthī}. 

“Mendicants,\marginnote{1.2} suppose a spike of rice or barley was pointing the wrong way. If you trod on it with hand or foot, there’s no way it could break the skin and produce blood. Why is that? Because the spike is pointing the wrong way. 

In\marginnote{1.5} the same way, a mendicant whose view and development of the path is pointing the wrong way cannot break ignorance, produce knowledge, and realize extinguishment. Why is that? Because their view is pointing the wrong way. 

Suppose\marginnote{2.1} a spike of rice or barley was pointing the right way. If you trod on it with hand or foot, it may well break the skin and produce blood. Why is that? Because the spike is pointing the right way. 

In\marginnote{2.4} the same way, a mendicant whose view and development of the path is pointing the right way may well break ignorance, produce knowledge, and realize extinguishment. Why is that? Because their view is pointing the right way. 

And\marginnote{3.1} how does a mendicant whose view and development of the path is pointing the right way break ignorance, give rise to knowledge, and realize extinguishment? It’s when a mendicant develops right view, right thought, right speech, right action, right livelihood, right effort, right mindfulness, and right immersion, which rely on seclusion, fading away, and cessation, and ripen as letting go. That’s how a mendicant whose view and development of the path is pointing the right way breaks ignorance, gives rise to knowledge, and realizes extinguishment.” 

%
\section*{{\suttatitleacronym SN 45.10}{\suttatitletranslation With Nandiya }{\suttatitleroot Nandiyasutta}}
\addcontentsline{toc}{section}{\tocacronym{SN 45.10} \toctranslation{With Nandiya } \tocroot{Nandiyasutta}}
\markboth{With Nandiya }{Nandiyasutta}
\extramarks{SN 45.10}{SN 45.10}

At\marginnote{1.1} \textsanskrit{Sāvatthī}. 

Then\marginnote{1.2} the wanderer Nandiya went up to the Buddha, and exchanged greetings with him. When the greetings and polite conversation were over, he sat down to one side, and said to the Buddha: 

“Master\marginnote{1.4} Gotama, how many things, when developed and cultivated, have extinguishment as their culmination, destination, and end?” 

“These\marginnote{2.1} eight things, when developed and cultivated, have extinguishment as their culmination, destination, and end. What eight? They are: right view, right thought, right speech, right action, right livelihood, right effort, right mindfulness, and right immersion. These eight things, when developed and cultivated, have extinguishment as their culmination, destination, and end.” 

When\marginnote{2.5} he said this, the wanderer Nandiya said to the Buddha, “Excellent, Master Gotama! Excellent! … From this day forth, may Master Gotama remember me as a lay follower who has gone for refuge for life.” 

%
\addtocontents{toc}{\let\protect\contentsline\protect\nopagecontentsline}
\chapter*{The Chapter on Meditation }
\addcontentsline{toc}{chapter}{\tocchapterline{The Chapter on Meditation }}
\addtocontents{toc}{\let\protect\contentsline\protect\oldcontentsline}

%
\section*{{\suttatitleacronym SN 45.11}{\suttatitletranslation Meditation (1st) }{\suttatitleroot Paṭhamavihārasutta}}
\addcontentsline{toc}{section}{\tocacronym{SN 45.11} \toctranslation{Meditation (1st) } \tocroot{Paṭhamavihārasutta}}
\markboth{Meditation (1st) }{Paṭhamavihārasutta}
\extramarks{SN 45.11}{SN 45.11}

At\marginnote{1.1} \textsanskrit{Sāvatthī}. 

“Mendicants,\marginnote{1.2} I wish to go on retreat for a fortnight. No-one should approach me, except for the one who brings my almsfood.” 

“Yes,\marginnote{1.4} sir,” replied those mendicants. And no-one approached him, except for the one who brought the almsfood. 

Then\marginnote{2.1} after a fortnight had passed, the Buddha came out of retreat and addressed the mendicants: 

“Mendicants,\marginnote{2.2} I’ve been practicing part of the meditation I practiced when I was first awakened. I understand that there’s feeling conditioned by wrong view and feeling conditioned by right view. … There’s feeling conditioned by wrong immersion, and feeling conditioned by right immersion. 

There’s\marginnote{2.8} feeling conditioned by desire, by thought, and by perception. As long as desire, thought, and perception are not stilled, there is feeling conditioned by that. When desire, thought, and perception are stilled, there is feeling conditioned by that. 

There\marginnote{2.13} is effort to attain the unattained. When that state has been attained, there is also feeling conditioned by that.” 

%
\section*{{\suttatitleacronym SN 45.12}{\suttatitletranslation Meditation (2nd) }{\suttatitleroot Dutiyavihārasutta}}
\addcontentsline{toc}{section}{\tocacronym{SN 45.12} \toctranslation{Meditation (2nd) } \tocroot{Dutiyavihārasutta}}
\markboth{Meditation (2nd) }{Dutiyavihārasutta}
\extramarks{SN 45.12}{SN 45.12}

At\marginnote{1.1} \textsanskrit{Sāvatthī}. 

“Mendicants,\marginnote{1.2} I wish to go on retreat for three months. No-one should approach me, except for the one who brings my almsfood.” 

“Yes,\marginnote{1.4} sir,” replied those mendicants. And no-one approached him, except for the one who brought the almsfood. 

Then\marginnote{2.1} after three months had passed, the Buddha came out of retreat and addressed the mendicants: 

“Mendicants,\marginnote{2.2} I’ve been practicing part of the meditation I practiced when I was first awakened. 

I\marginnote{2.3} understand that there’s feeling conditioned by wrong view and by the stilling of wrong view, by right view and by the stilling of right view. … There’s feeling conditioned by wrong immersion and by the stilling of wrong immersion, by right immersion and by the stilling of right immersion. 

There’s\marginnote{2.11} feeling conditioned by desire and by the stilling of desire, by thought and by the stilling of thought, by perception and by the stilling of perception. As long as desire, thought, and perception are not stilled, there is feeling conditioned by that. When desire, thought, and perception are stilled, there is feeling conditioned by that. 

There\marginnote{2.19} is effort to attain the unattained. When that state has been attained, there is also feeling conditioned by that.” 

%
\section*{{\suttatitleacronym SN 45.13}{\suttatitletranslation A Trainee }{\suttatitleroot Sekkhasutta}}
\addcontentsline{toc}{section}{\tocacronym{SN 45.13} \toctranslation{A Trainee } \tocroot{Sekkhasutta}}
\markboth{A Trainee }{Sekkhasutta}
\extramarks{SN 45.13}{SN 45.13}

At\marginnote{1.1} \textsanskrit{Sāvatthī}. 

Then\marginnote{1.2} a mendicant went up to the Buddha … and asked him, “Sir, they speak of this person called ‘a trainee’. How is a trainee defined?” 

“Mendicant,\marginnote{2.1} it’s someone who has a trainee’s right view, right thought, right speech, right action, right livelihood, right effort, right mindfulness, and right immersion. That’s how a trainee is defined.” 

%
\section*{{\suttatitleacronym SN 45.14}{\suttatitletranslation Arising (1st) }{\suttatitleroot Paṭhamauppādasutta}}
\addcontentsline{toc}{section}{\tocacronym{SN 45.14} \toctranslation{Arising (1st) } \tocroot{Paṭhamauppādasutta}}
\markboth{Arising (1st) }{Paṭhamauppādasutta}
\extramarks{SN 45.14}{SN 45.14}

At\marginnote{1.1} \textsanskrit{Sāvatthī}. 

“Mendicants,\marginnote{1.2} these eight things don’t arise to be developed and cultivated except when a Realized One, a perfected one, a fully awakened Buddha has appeared. What eight? They are: right view, right thought, right speech, right action, right livelihood, right effort, right mindfulness, and right immersion. These eight things don’t arise to be developed and cultivated except when a Realized One, a perfected one, a fully awakened Buddha has appeared.” 

%
\section*{{\suttatitleacronym SN 45.15}{\suttatitletranslation Arising (2nd) }{\suttatitleroot Dutiyauppādasutta}}
\addcontentsline{toc}{section}{\tocacronym{SN 45.15} \toctranslation{Arising (2nd) } \tocroot{Dutiyauppādasutta}}
\markboth{Arising (2nd) }{Dutiyauppādasutta}
\extramarks{SN 45.15}{SN 45.15}

At\marginnote{1.1} \textsanskrit{Sāvatthī}. 

“Mendicants,\marginnote{1.2} these eight things don’t arise to be developed and cultivated apart from the Holy One’s training. What eight? They are: right view, right thought, right speech, right action, right livelihood, right effort, right mindfulness, and right immersion. These are the eight things that don’t arise to be developed and cultivated apart from the Holy One’s training.” 

%
\section*{{\suttatitleacronym SN 45.16}{\suttatitletranslation Purified (1st) }{\suttatitleroot Paṭhamaparisuddhasutta}}
\addcontentsline{toc}{section}{\tocacronym{SN 45.16} \toctranslation{Purified (1st) } \tocroot{Paṭhamaparisuddhasutta}}
\markboth{Purified (1st) }{Paṭhamaparisuddhasutta}
\extramarks{SN 45.16}{SN 45.16}

At\marginnote{1.1} \textsanskrit{Sāvatthī}. 

“Mendicants,\marginnote{1.2} these eight things don’t arise to be purified, bright, flawless, and rid of corruptions except when a Realized One, a perfected one, a fully awakened Buddha has appeared. What eight? They are: right view, right thought, right speech, right action, right livelihood, right effort, right mindfulness, and right immersion. These eight things don’t arise to be purified, bright, flawless, and rid of corruptions except when a Realized One, a perfected one, a fully awakened Buddha has appeared.” 

%
\section*{{\suttatitleacronym SN 45.17}{\suttatitletranslation Purified (2nd) }{\suttatitleroot Dutiyaparisuddhasutta}}
\addcontentsline{toc}{section}{\tocacronym{SN 45.17} \toctranslation{Purified (2nd) } \tocroot{Dutiyaparisuddhasutta}}
\markboth{Purified (2nd) }{Dutiyaparisuddhasutta}
\extramarks{SN 45.17}{SN 45.17}

At\marginnote{1.1} \textsanskrit{Sāvatthī}. 

“Mendicants,\marginnote{1.2} these eight things don’t arise to be purified, bright, flawless, and rid of corruptions apart from the Holy One’s training. What eight? They are: right view, right thought, right speech, right action, right livelihood, right effort, right mindfulness, and right immersion. These eight things don’t arise to be purified, bright, flawless, and rid of corruptions apart from the Holy One’s training.” 

%
\section*{{\suttatitleacronym SN 45.18}{\suttatitletranslation At the Chicken Monastery (1st) }{\suttatitleroot Paṭhamakukkuṭārāmasutta}}
\addcontentsline{toc}{section}{\tocacronym{SN 45.18} \toctranslation{At the Chicken Monastery (1st) } \tocroot{Paṭhamakukkuṭārāmasutta}}
\markboth{At the Chicken Monastery (1st) }{Paṭhamakukkuṭārāmasutta}
\extramarks{SN 45.18}{SN 45.18}

\scevam{So\marginnote{1.1} I have heard. }At one time the venerables Ānanda and Bhadda were staying near \textsanskrit{Pāṭaliputta}, in the Chicken Monastery. Then in the late afternoon, Venerable Bhadda came out of retreat, went to Venerable Ānanda, and exchanged greetings with him. When the greetings and polite conversation were over, he sat down to one side and said to Ānanda: 

“Reverend,\marginnote{2.1} they speak of this thing called ‘not the spiritual path’. What is not the spiritual path?” 

“Good,\marginnote{2.3} good, Reverend Bhadda! Your approach and articulation are excellent, and it’s a good question. For you asked: ‘They speak of this thing called “not the spiritual path”. What is not the spiritual path?’” 

“Yes,\marginnote{2.8} reverend.” 

“What\marginnote{2.9} is not the spiritual path is simply the wrong eightfold path, that is: wrong view, wrong thought, wrong speech, wrong action, wrong livelihood, wrong effort, wrong mindfulness, and wrong immersion.” 

%
\section*{{\suttatitleacronym SN 45.19}{\suttatitletranslation At the Chicken Monastery (2nd) }{\suttatitleroot Dutiyakukkuṭārāmasutta}}
\addcontentsline{toc}{section}{\tocacronym{SN 45.19} \toctranslation{At the Chicken Monastery (2nd) } \tocroot{Dutiyakukkuṭārāmasutta}}
\markboth{At the Chicken Monastery (2nd) }{Dutiyakukkuṭārāmasutta}
\extramarks{SN 45.19}{SN 45.19}

At\marginnote{1.1} \textsanskrit{Pāṭaliputta}. 

“Reverend,\marginnote{1.2} they speak of this thing called the ‘spiritual path’. What is the spiritual path? And what is the culmination of the spiritual path?” 

“Good,\marginnote{1.4} good, Reverend Bhadda! Your approach and articulation are excellent, and it’s a good question. For you asked: ‘They speak of this thing called “the spiritual path”. What is the spiritual path? And what is the culmination of the spiritual path?’” 

“Yes,\marginnote{1.9} reverend.” 

“The\marginnote{1.10} spiritual path is simply this noble eightfold path, that is: right view, right thought, right speech, right action, right livelihood, right effort, right mindfulness, and right immersion. 

The\marginnote{1.12} ending of greed, hate, and delusion: this is the culmination of the spiritual path.” 

%
\section*{{\suttatitleacronym SN 45.20}{\suttatitletranslation At the Chicken Monastery (3rd) }{\suttatitleroot Tatiyakukkuṭārāmasutta}}
\addcontentsline{toc}{section}{\tocacronym{SN 45.20} \toctranslation{At the Chicken Monastery (3rd) } \tocroot{Tatiyakukkuṭārāmasutta}}
\markboth{At the Chicken Monastery (3rd) }{Tatiyakukkuṭārāmasutta}
\extramarks{SN 45.20}{SN 45.20}

At\marginnote{1.1} \textsanskrit{Pāṭaliputta}. 

“Reverend,\marginnote{1.2} they speak of this thing called the ‘spiritual path’. What is the spiritual path? Who is someone on the spiritual path? And what is the culmination of the spiritual path?” 

“Good,\marginnote{1.4} good, Reverend Bhadda! Your approach and articulation are excellent, and it’s a good question. … 

The\marginnote{1.10} spiritual path is simply this noble eightfold path, that is: right view, right thought, right speech, right action, right livelihood, right effort, right mindfulness, and right immersion. 

Someone\marginnote{1.12} who possesses this noble eightfold path is called someone on the spiritual path. 

The\marginnote{1.14} ending of greed, hate, and delusion: this is the culmination of the spiritual path.” 

%
\addtocontents{toc}{\let\protect\contentsline\protect\nopagecontentsline}
\chapter*{The Chapter on the Wrong Way }
\addcontentsline{toc}{chapter}{\tocchapterline{The Chapter on the Wrong Way }}
\addtocontents{toc}{\let\protect\contentsline\protect\oldcontentsline}

%
\section*{{\suttatitleacronym SN 45.21}{\suttatitletranslation The Wrong Way }{\suttatitleroot Micchattasutta}}
\addcontentsline{toc}{section}{\tocacronym{SN 45.21} \toctranslation{The Wrong Way } \tocroot{Micchattasutta}}
\markboth{The Wrong Way }{Micchattasutta}
\extramarks{SN 45.21}{SN 45.21}

At\marginnote{1.1} \textsanskrit{Sāvatthī}. 

“Mendicants,\marginnote{1.2} I will teach you the wrong way and the right way. Listen … 

And\marginnote{1.4} what is the wrong way? It is wrong view, wrong thought, wrong speech, wrong action, wrong livelihood, wrong effort, wrong mindfulness, and wrong immersion. This is called the wrong way. 

And\marginnote{1.7} what is the right way? It is right view, right thought, right speech, right action, right livelihood, right effort, right mindfulness, and right immersion. This is called the right way.” 

%
\section*{{\suttatitleacronym SN 45.22}{\suttatitletranslation Unskillful Qualities }{\suttatitleroot Akusaladhammasutta}}
\addcontentsline{toc}{section}{\tocacronym{SN 45.22} \toctranslation{Unskillful Qualities } \tocroot{Akusaladhammasutta}}
\markboth{Unskillful Qualities }{Akusaladhammasutta}
\extramarks{SN 45.22}{SN 45.22}

At\marginnote{1.1} \textsanskrit{Sāvatthī}. 

“Mendicants,\marginnote{1.2} I will teach you skillful and unskillful qualities. Listen … 

And\marginnote{1.4} what are unskillful qualities? They are wrong view, wrong thought, wrong speech, wrong action, wrong livelihood, wrong effort, wrong mindfulness, and wrong immersion. These are called unskillful qualities. 

And\marginnote{1.7} what are skillful qualities? They are right view, right thought, right speech, right action, right livelihood, right effort, right mindfulness, and right immersion. These are called skillful qualities.” 

%
\section*{{\suttatitleacronym SN 45.23}{\suttatitletranslation Practice (1st) }{\suttatitleroot Paṭhamapaṭipadāsutta}}
\addcontentsline{toc}{section}{\tocacronym{SN 45.23} \toctranslation{Practice (1st) } \tocroot{Paṭhamapaṭipadāsutta}}
\markboth{Practice (1st) }{Paṭhamapaṭipadāsutta}
\extramarks{SN 45.23}{SN 45.23}

At\marginnote{1.1} \textsanskrit{Sāvatthī}. 

“Mendicants,\marginnote{1.2} I will teach you the wrong practice and the right practice. Listen … 

And\marginnote{1.4} what’s the wrong practice? It is wrong view, wrong thought, wrong speech, wrong action, wrong livelihood, wrong effort, wrong mindfulness, and wrong immersion. This is called the wrong practice. 

And\marginnote{1.7} what’s the right practice? It is right view, right thought, right speech, right action, right livelihood, right effort, right mindfulness, and right immersion. This is called the right practice.” 

%
\section*{{\suttatitleacronym SN 45.24}{\suttatitletranslation Practice (2nd) }{\suttatitleroot Dutiyapaṭipadāsutta}}
\addcontentsline{toc}{section}{\tocacronym{SN 45.24} \toctranslation{Practice (2nd) } \tocroot{Dutiyapaṭipadāsutta}}
\markboth{Practice (2nd) }{Dutiyapaṭipadāsutta}
\extramarks{SN 45.24}{SN 45.24}

At\marginnote{1.1} \textsanskrit{Sāvatthī}. 

“Mendicants,\marginnote{1.2} I don’t praise wrong practice for laypeople or renunciates. Because of wrong practice, neither laypeople nor renunciates succeed in the procedure of the skillful teaching. 

And\marginnote{2.1} what’s the wrong practice? It is wrong view, wrong thought, wrong speech, wrong action, wrong livelihood, wrong effort, wrong mindfulness, and wrong immersion. This is called the wrong practice. I don’t praise wrong practice for lay people or renunciates. Because of wrong practice, neither laypeople nor renunciates succeed in the procedure of the skillful teaching. 

I\marginnote{3.1} praise right practice for laypeople and renunciates. Because of right practice, both laypeople and renunciates succeed in the procedure of the skillful teaching. And what’s the right practice? It is right view, right thought, right speech, right action, right livelihood, right effort, right mindfulness, and right immersion. This is called the right practice. I praise right practice for laypeople and renunciates. 

Because\marginnote{3.7} of right practice, both laypeople and renunciates succeed in the procedure of the skillful teaching.” 

%
\section*{{\suttatitleacronym SN 45.25}{\suttatitletranslation A Good Person (1st) }{\suttatitleroot Paṭhamaasappurisasutta}}
\addcontentsline{toc}{section}{\tocacronym{SN 45.25} \toctranslation{A Good Person (1st) } \tocroot{Paṭhamaasappurisasutta}}
\markboth{A Good Person (1st) }{Paṭhamaasappurisasutta}
\extramarks{SN 45.25}{SN 45.25}

At\marginnote{1.1} \textsanskrit{Sāvatthī}. 

“Mendicants,\marginnote{1.2} I will teach you a bad person and a good person. Listen … 

And\marginnote{1.4} what is a bad person? It’s someone who has wrong view, wrong thought, wrong speech, wrong action, wrong livelihood, wrong effort, wrong mindfulness, and wrong immersion. This is called a bad person. 

And\marginnote{2.1} what is a good person? It’s someone who has right view, right thought, right speech, right action, right livelihood, right effort, right mindfulness, and right immersion. This is called a good person.” 

%
\section*{{\suttatitleacronym SN 45.26}{\suttatitletranslation A Good Person (2nd) }{\suttatitleroot Dutiyaasappurisasutta}}
\addcontentsline{toc}{section}{\tocacronym{SN 45.26} \toctranslation{A Good Person (2nd) } \tocroot{Dutiyaasappurisasutta}}
\markboth{A Good Person (2nd) }{Dutiyaasappurisasutta}
\extramarks{SN 45.26}{SN 45.26}

At\marginnote{1.1} \textsanskrit{Sāvatthī}. 

“Mendicants,\marginnote{1.2} I will teach you a bad person and a worse person, a good person and a better person. Listen … 

And\marginnote{1.5} what is a bad person? It’s someone who has wrong view, wrong thought, wrong speech, wrong action, wrong livelihood, wrong effort, wrong mindfulness, and wrong immersion. This is called a bad person. 

And\marginnote{2.1} what is a worse person? It’s someone who has wrong view, wrong thought, wrong speech, wrong action, wrong livelihood, wrong effort, wrong mindfulness, wrong immersion, wrong knowledge, and wrong freedom. This is called a worse person. 

And\marginnote{3.1} what is a good person? It’s someone who has right view, right thought, right speech, right action, right livelihood, right effort, right mindfulness, and right immersion. This is called a good person. 

And\marginnote{4.1} what is a better person? It’s someone who has right view, right thought, right speech, right action, right livelihood, right effort, right mindfulness, right immersion, right knowledge, and right freedom. This is called a better person.” 

%
\section*{{\suttatitleacronym SN 45.27}{\suttatitletranslation Pots }{\suttatitleroot Kumbhasutta}}
\addcontentsline{toc}{section}{\tocacronym{SN 45.27} \toctranslation{Pots } \tocroot{Kumbhasutta}}
\markboth{Pots }{Kumbhasutta}
\extramarks{SN 45.27}{SN 45.27}

At\marginnote{1.1} \textsanskrit{Sāvatthī}. 

“A\marginnote{1.2} pot without a stand is easy to overturn, but if it has a stand it’s hard to overturn. In the same way, a mind without a stand is easy to overturn, but if it has a stand it’s hard to overturn. 

And\marginnote{1.4} what’s the stand for the mind? It is simply this noble eightfold path, that is: right view, right thought, right speech, right action, right livelihood, right effort, right mindfulness, and right immersion. This is the stand for the mind. 

A\marginnote{1.8} pot without a stand is easy to overturn, but if it has a stand it’s hard to overturn. In the same way, a mind without a stand is easy to overturn, but if it has a stand it’s hard to overturn.” 

%
\section*{{\suttatitleacronym SN 45.28}{\suttatitletranslation Immersion }{\suttatitleroot Samādhisutta}}
\addcontentsline{toc}{section}{\tocacronym{SN 45.28} \toctranslation{Immersion } \tocroot{Samādhisutta}}
\markboth{Immersion }{Samādhisutta}
\extramarks{SN 45.28}{SN 45.28}

At\marginnote{1.1} \textsanskrit{Sāvatthī}. 

“Mendicants,\marginnote{1.2} I will teach you noble right immersion with its vital conditions and its prerequisites. Listen … 

And\marginnote{1.4} what is noble right immersion with its vital conditions and its prerequisites? There are right view, right thought, right speech, right action, right livelihood, right effort, and right mindfulness. 

Unification\marginnote{1.6} of mind with these seven factors as prerequisites is called noble right immersion ‘with its vital conditions’ and ‘with its prerequisites’.” 

%
\section*{{\suttatitleacronym SN 45.29}{\suttatitletranslation Feeling }{\suttatitleroot Vedanāsutta}}
\addcontentsline{toc}{section}{\tocacronym{SN 45.29} \toctranslation{Feeling } \tocroot{Vedanāsutta}}
\markboth{Feeling }{Vedanāsutta}
\extramarks{SN 45.29}{SN 45.29}

At\marginnote{1.1} \textsanskrit{Sāvatthī}. 

“Mendicants,\marginnote{1.2} there are these three feelings. What three? Pleasant, painful, and neutral feeling. These are the three feelings. 

The\marginnote{1.6} noble eightfold path should be developed to completely understand these three feelings. What is the noble eightfold path? It is right view, right thought, right speech, right action, right livelihood, right effort, right mindfulness, and right immersion. 

This\marginnote{1.9} noble eightfold path should be developed to completely understand these three feelings.” 

%
\section*{{\suttatitleacronym SN 45.30}{\suttatitletranslation With Uttiya }{\suttatitleroot Uttiyasutta}}
\addcontentsline{toc}{section}{\tocacronym{SN 45.30} \toctranslation{With Uttiya } \tocroot{Uttiyasutta}}
\markboth{With Uttiya }{Uttiyasutta}
\extramarks{SN 45.30}{SN 45.30}

At\marginnote{1.1} \textsanskrit{Sāvatthī}. 

Then\marginnote{1.2} Venerable Uttiya went up to the Buddha … and asked him, “Just now, sir, as I was in private retreat this thought came to mind. ‘The Buddha has spoken of the five kinds of sensual stimulation. What are they?’” 

“Good,\marginnote{1.6} good, Uttiya! I have spoken of these five kinds of sensual stimulation. What five? Sights known by the eye that are likable, desirable, agreeable, pleasant, sensual, and arousing. Sounds known by the ear … Smells known by the nose … Tastes known by the tongue … Touches known by the body that are likable, desirable, agreeable, pleasant, sensual, and arousing. These are the five kinds of sensual stimulation that I’ve spoken of. 

The\marginnote{1.15} noble eightfold path should be developed to give up these five kinds of sensual stimulation. What is the noble eightfold path? It is right view, right thought, right speech, right action, right livelihood, right effort, right mindfulness, and right immersion. This is the noble eightfold path that should be developed to give up these five kinds of sensual stimulation.” 

%
\addtocontents{toc}{\let\protect\contentsline\protect\nopagecontentsline}
\chapter*{The Chapter on Practice }
\addcontentsline{toc}{chapter}{\tocchapterline{The Chapter on Practice }}
\addtocontents{toc}{\let\protect\contentsline\protect\oldcontentsline}

%
\section*{{\suttatitleacronym SN 45.31}{\suttatitletranslation Practice (1st) }{\suttatitleroot Paṭhamapaṭipattisutta}}
\addcontentsline{toc}{section}{\tocacronym{SN 45.31} \toctranslation{Practice (1st) } \tocroot{Paṭhamapaṭipattisutta}}
\markboth{Practice (1st) }{Paṭhamapaṭipattisutta}
\extramarks{SN 45.31}{SN 45.31}

At\marginnote{1.1} \textsanskrit{Sāvatthī}. 

“Mendicants,\marginnote{1.2} I will teach you the wrong practice and the right practice. Listen … 

And\marginnote{1.4} what’s the wrong practice? It is wrong view, wrong thought, wrong speech, wrong action, wrong livelihood, wrong effort, wrong mindfulness, and wrong immersion. This is called the wrong practice. 

And\marginnote{1.7} what’s the right practice? It is right view, right thought, right speech, right action, right livelihood, right effort, right mindfulness, and right immersion. This is called the right practice.” 

%
\section*{{\suttatitleacronym SN 45.32}{\suttatitletranslation Practice (2nd) }{\suttatitleroot Dutiyapaṭipattisutta}}
\addcontentsline{toc}{section}{\tocacronym{SN 45.32} \toctranslation{Practice (2nd) } \tocroot{Dutiyapaṭipattisutta}}
\markboth{Practice (2nd) }{Dutiyapaṭipattisutta}
\extramarks{SN 45.32}{SN 45.32}

At\marginnote{1.1} \textsanskrit{Sāvatthī}. 

“Mendicants,\marginnote{1.2} I will teach you one practicing wrongly and one practicing rightly. Listen … 

And\marginnote{1.4} who is practicing wrongly? It’s someone who has wrong view, wrong thought, wrong speech, wrong action, wrong livelihood, wrong effort, wrong mindfulness, and wrong immersion. This is called one practicing wrongly. 

And\marginnote{1.7} who is practicing rightly? It’s someone who has right view, right thought, right speech, right action, right livelihood, right effort, right mindfulness, and right immersion. This is called one practicing rightly.” 

%
\section*{{\suttatitleacronym SN 45.33}{\suttatitletranslation Missed Out }{\suttatitleroot Viraddhasutta}}
\addcontentsline{toc}{section}{\tocacronym{SN 45.33} \toctranslation{Missed Out } \tocroot{Viraddhasutta}}
\markboth{Missed Out }{Viraddhasutta}
\extramarks{SN 45.33}{SN 45.33}

At\marginnote{1.1} \textsanskrit{Sāvatthī}. 

“Mendicants,\marginnote{1.2} whoever has missed out on the noble eightfold path has missed out on the noble path to the complete ending of suffering. Whoever has undertaken the noble eightfold path has undertaken the noble path to the complete ending of suffering. 

And\marginnote{1.4} what is the noble eightfold path? It is right view, right thought, right speech, right action, right livelihood, right effort, right mindfulness, and right immersion. 

Whoever\marginnote{1.6} has missed out on the noble eightfold path has missed out on the noble path to the complete ending of suffering. Whoever has undertaken the noble eightfold path has undertaken the noble path to the complete ending of suffering.” 

%
\section*{{\suttatitleacronym SN 45.34}{\suttatitletranslation Going to the Far Shore }{\suttatitleroot Pāraṅgamasutta}}
\addcontentsline{toc}{section}{\tocacronym{SN 45.34} \toctranslation{Going to the Far Shore } \tocroot{Pāraṅgamasutta}}
\markboth{Going to the Far Shore }{Pāraṅgamasutta}
\extramarks{SN 45.34}{SN 45.34}

At\marginnote{1.1} \textsanskrit{Sāvatthī}. 

“Mendicants,\marginnote{1.2} when these eight things are developed and cultivated they lead to going from the near shore to the far shore. What eight? They are right view, right thought, right speech, right action, right livelihood, right effort, right mindfulness, and right immersion. When these eight things are developed and cultivated they lead to going from the near shore to the far shore.” 

That\marginnote{2.1} is what the Buddha said. Then the Holy One, the Teacher, went on to say: 

\begin{verse}%
“Few\marginnote{3.1} are those among humans \\
who cross to the far shore. \\
The rest just run \\
around on the near shore. 

When\marginnote{4.1} the teaching is well explained, \\
those who practice accordingly \\
are the ones who will cross over \\
Death’s domain so hard to pass. 

Rid\marginnote{5.1} of dark qualities, \\
an astute person should develop the bright. \\
Leaving home behind \\
for the seclusion so hard to enjoy, 

you\marginnote{6.1} should try to find delight there, \\
having left behind sensual pleasures. \\
With no possessions, an astute person \\
should cleanse themselves of mental corruptions. 

And\marginnote{7.1} those whose minds are rightly developed \\
in the awakening factors; \\
letting go of attachments, \\
they delight in not grasping. \\
With defilements ended, brilliant, \\
they are extinguished in this world.” 

%
\end{verse}

%
\section*{{\suttatitleacronym SN 45.35}{\suttatitletranslation The Ascetic Life (1st) }{\suttatitleroot Paṭhamasāmaññasutta}}
\addcontentsline{toc}{section}{\tocacronym{SN 45.35} \toctranslation{The Ascetic Life (1st) } \tocroot{Paṭhamasāmaññasutta}}
\markboth{The Ascetic Life (1st) }{Paṭhamasāmaññasutta}
\extramarks{SN 45.35}{SN 45.35}

At\marginnote{1.1} \textsanskrit{Sāvatthī}. 

“Mendicants,\marginnote{1.2} I will teach you the ascetic life and the fruits of the ascetic life. Listen … 

And\marginnote{1.4} what is the ascetic life? It is simply this noble eightfold path, that is: right view, right thought, right speech, right action, right livelihood, right effort, right mindfulness, and right immersion. This is called the ascetic life. 

And\marginnote{1.8} what are the fruits of the ascetic life? The fruits of stream-entry, once-return, non-return, and perfection. These are called the fruits of the ascetic life.” 

%
\section*{{\suttatitleacronym SN 45.36}{\suttatitletranslation The Ascetic Life (2nd) }{\suttatitleroot Dutiyasāmaññasutta}}
\addcontentsline{toc}{section}{\tocacronym{SN 45.36} \toctranslation{The Ascetic Life (2nd) } \tocroot{Dutiyasāmaññasutta}}
\markboth{The Ascetic Life (2nd) }{Dutiyasāmaññasutta}
\extramarks{SN 45.36}{SN 45.36}

At\marginnote{1.1} \textsanskrit{Sāvatthī}. 

“Mendicants,\marginnote{1.2} I will teach you the ascetic life and the goal of the ascetic life. Listen … 

And\marginnote{1.4} what is the ascetic life? It is simply this noble eightfold path, that is: right view, right thought, right speech, right action, right livelihood, right effort, right mindfulness, and right immersion. This is called the ascetic life. 

And\marginnote{1.8} what is the goal of the ascetic life? The ending of greed, hate, and delusion. This is called the goal of the ascetic life.” 

%
\section*{{\suttatitleacronym SN 45.37}{\suttatitletranslation The Brahmin Life (1st) }{\suttatitleroot Paṭhamabrahmaññasutta}}
\addcontentsline{toc}{section}{\tocacronym{SN 45.37} \toctranslation{The Brahmin Life (1st) } \tocroot{Paṭhamabrahmaññasutta}}
\markboth{The Brahmin Life (1st) }{Paṭhamabrahmaññasutta}
\extramarks{SN 45.37}{SN 45.37}

At\marginnote{1.1} \textsanskrit{Sāvatthī}. 

“Mendicants,\marginnote{1.2} I will teach you life as a brahmin and the fruits of life as a brahmin. Listen … 

And\marginnote{1.4} what is life as a brahmin? It is simply this noble eightfold path, that is: right view, right thought, right speech, right action, right livelihood, right effort, right mindfulness, and right immersion. This is called life as a brahmin. 

And\marginnote{1.8} what are the fruits of life as a brahmin? The fruits of stream-entry, once-return, non-return, and perfection. These are called the fruits of life as a brahmin.” 

%
\section*{{\suttatitleacronym SN 45.38}{\suttatitletranslation The Brahmin Life (2nd) }{\suttatitleroot Dutiyabrahmaññasutta}}
\addcontentsline{toc}{section}{\tocacronym{SN 45.38} \toctranslation{The Brahmin Life (2nd) } \tocroot{Dutiyabrahmaññasutta}}
\markboth{The Brahmin Life (2nd) }{Dutiyabrahmaññasutta}
\extramarks{SN 45.38}{SN 45.38}

At\marginnote{1.1} \textsanskrit{Sāvatthī}. 

“Mendicants,\marginnote{1.2} I will teach you life as a brahmin and the goal of life as a brahmin. Listen … 

And\marginnote{1.4} what is life as a brahmin? It is simply this noble eightfold path, that is: right view, right thought, right speech, right action, right livelihood, right effort, right mindfulness, and right immersion. This is called life as a brahmin. 

And\marginnote{1.8} what is the goal of life as a brahmin? The ending of greed, hate, and delusion. This is called the goal of life as a brahmin.” 

%
\section*{{\suttatitleacronym SN 45.39}{\suttatitletranslation The Spiritual Path (1st) }{\suttatitleroot Paṭhamabrahmacariyasutta}}
\addcontentsline{toc}{section}{\tocacronym{SN 45.39} \toctranslation{The Spiritual Path (1st) } \tocroot{Paṭhamabrahmacariyasutta}}
\markboth{The Spiritual Path (1st) }{Paṭhamabrahmacariyasutta}
\extramarks{SN 45.39}{SN 45.39}

At\marginnote{1.1} \textsanskrit{Sāvatthī}. 

“Mendicants,\marginnote{1.2} I will teach you the spiritual path and the fruits of the spiritual path. Listen … 

And\marginnote{1.4} what is the spiritual path? It is simply this noble eightfold path, that is: right view, right thought, right speech, right action, right livelihood, right effort, right mindfulness, and right immersion. This is called the spiritual path. 

And\marginnote{1.8} what are the fruits of the spiritual path? The fruits of stream-entry, once-return, non-return, and perfection. These are called the fruits of the spiritual path.” 

%
\section*{{\suttatitleacronym SN 45.40}{\suttatitletranslation The Spiritual Path (2nd) }{\suttatitleroot Dutiyabrahmacariyasutta}}
\addcontentsline{toc}{section}{\tocacronym{SN 45.40} \toctranslation{The Spiritual Path (2nd) } \tocroot{Dutiyabrahmacariyasutta}}
\markboth{The Spiritual Path (2nd) }{Dutiyabrahmacariyasutta}
\extramarks{SN 45.40}{SN 45.40}

At\marginnote{1.1} \textsanskrit{Sāvatthī}. 

“Mendicants,\marginnote{1.2} I will teach you the spiritual path and the goal of the spiritual path. Listen … 

And\marginnote{1.4} what is the spiritual path? It is simply this noble eightfold path, that is: right view, right thought, right speech, right action, right livelihood, right effort, right mindfulness, and right immersion. This is called the spiritual path. 

And\marginnote{1.8} what is the goal of the spiritual path? The ending of greed, hate, and delusion. This is called the goal of the spiritual path.” 

%
\addtocontents{toc}{\let\protect\contentsline\protect\nopagecontentsline}
\chapter*{The Chapter of Abbreviated Texts on Followers of Other Paths }
\addcontentsline{toc}{chapter}{\tocchapterline{The Chapter of Abbreviated Texts on Followers of Other Paths }}
\addtocontents{toc}{\let\protect\contentsline\protect\oldcontentsline}

%
\section*{{\suttatitleacronym SN 45.41}{\suttatitletranslation The Fading Away of Greed }{\suttatitleroot Rāgavirāgasutta}}
\addcontentsline{toc}{section}{\tocacronym{SN 45.41} \toctranslation{The Fading Away of Greed } \tocroot{Rāgavirāgasutta}}
\markboth{The Fading Away of Greed }{Rāgavirāgasutta}
\extramarks{SN 45.41}{SN 45.41}

At\marginnote{1.1} \textsanskrit{Sāvatthī}. 

“Mendicants,\marginnote{1.2} if wanderers who follow another path were to ask you: ‘Reverends, what’s the purpose of leading the spiritual life under the ascetic Gotama?’ You should answer them like this: ‘The purpose of leading the spiritual life under the Buddha is the fading away of greed.’ 

If\marginnote{1.5} wanderers who follow other paths were to ask you: ‘Is there a path and a practice for the fading away of greed?’ You should answer them like this: ‘There is a path and a practice for the fading away of greed.’ 

And\marginnote{1.8} what is that path, what is that practice for the fading away of greed? It is simply this noble eightfold path, that is: right view, right thought, right speech, right action, right livelihood, right effort, right mindfulness, and right immersion. This is the path, this is the practice for the fading away of greed. 

When\marginnote{1.12} questioned by wanderers who follow other paths, that’s how you should answer them.” 

%
\section*{{\suttatitleacronym SN 45.42–47}{\suttatitletranslation Six Discourses on Giving Up Fetters, Etc. }{\suttatitleroot Saṁyojanappahānādisuttachakka}}
\addcontentsline{toc}{section}{\tocacronym{SN 45.42–47} \toctranslation{Six Discourses on Giving Up Fetters, Etc. } \tocroot{Saṁyojanappahānādisuttachakka}}
\markboth{Six Discourses on Giving Up Fetters, Etc. }{Saṁyojanappahānādisuttachakka}
\extramarks{SN 45.42–47}{SN 45.42–47}

“Mendicants,\marginnote{1.1} if wanderers who follow another path were to ask you: ‘Reverends, what’s the purpose of leading the spiritual life under the ascetic Gotama?’ You should answer them like this: 

‘The\marginnote{1.3} purpose of leading the spiritual life under the Buddha is to give up the fetters.’ … 

‘…\marginnote{1.4} to uproot the underlying tendencies.’ … 

‘…\marginnote{1.5} to completely understand the course of time.’ … 

‘…\marginnote{1.6} to end the defilements.’ … 

‘…\marginnote{1.7} to realize the fruit of knowledge and freedom.’ … 

‘…\marginnote{1.8} for knowledge and vision.’ …” 

%
\section*{{\suttatitleacronym SN 45.48}{\suttatitletranslation Extinguishment by Not Grasping }{\suttatitleroot Anupādāparinibbānasutta}}
\addcontentsline{toc}{section}{\tocacronym{SN 45.48} \toctranslation{Extinguishment by Not Grasping } \tocroot{Anupādāparinibbānasutta}}
\markboth{Extinguishment by Not Grasping }{Anupādāparinibbānasutta}
\extramarks{SN 45.48}{SN 45.48}

At\marginnote{1.1} \textsanskrit{Sāvatthī}. 

“Mendicants,\marginnote{1.2} if wanderers who follow another path were to ask you: ‘Reverends, what’s the purpose of leading the spiritual life under the ascetic Gotama?’ You should answer them like this: ‘The purpose of leading the spiritual life under the Buddha is extinguishment by not grasping.’ 

If\marginnote{1.5} wanderers who follow other paths were to ask you: ‘Is there a path and a practice for extinguishment by not grasping?’ You should answer them like this: ‘There is a path and a practice for extinguishment by not grasping.’ 

And\marginnote{1.8} what is that path, what is that practice for extinguishment by not grasping? It is simply this noble eightfold path, that is: right view, right thought, right speech, right action, right livelihood, right effort, right mindfulness, and right immersion. This is the path, this is the practice for extinguishment by not grasping. 

When\marginnote{1.12} questioned by wanderers who follow other paths, that’s how you should answer them.” 

%
\addtocontents{toc}{\let\protect\contentsline\protect\nopagecontentsline}
\chapter*{The Chapter of Abbreviated Texts on the Sun }
\addcontentsline{toc}{chapter}{\tocchapterline{The Chapter of Abbreviated Texts on the Sun }}
\addtocontents{toc}{\let\protect\contentsline\protect\oldcontentsline}

%
\section*{{\suttatitleacronym SN 45.49}{\suttatitletranslation Good Friends (1st) }{\suttatitleroot Kalyāṇamittasutta}}
\addcontentsline{toc}{section}{\tocacronym{SN 45.49} \toctranslation{Good Friends (1st) } \tocroot{Kalyāṇamittasutta}}
\markboth{Good Friends (1st) }{Kalyāṇamittasutta}
\extramarks{SN 45.49}{SN 45.49}

At\marginnote{1.1} \textsanskrit{Sāvatthī}. 

“Mendicants,\marginnote{1.2} the dawn is the forerunner and precursor of the sunrise. In the same way good friendship is the forerunner and precursor of the noble eightfold path for a mendicant. A mendicant with good friends can expect to develop and cultivate the noble eightfold path. 

And\marginnote{1.5} how does a mendicant with good friends develop and cultivate the noble eightfold path? It’s when a mendicant develops right view, right thought, right speech, right action, right livelihood, right effort, right mindfulness, and right immersion, which rely on seclusion, fading away, and cessation, and ripen as letting go. That’s how a mendicant with good friends develops and cultivates the noble eightfold path.” 

%
\section*{{\suttatitleacronym SN 45.50–54}{\suttatitletranslation Five Discourses on Accomplishment in Ethics, Etc. (1st) }{\suttatitleroot Sīlasampadādisuttapañcaka}}
\addcontentsline{toc}{section}{\tocacronym{SN 45.50–54} \toctranslation{Five Discourses on Accomplishment in Ethics, Etc. (1st) } \tocroot{Sīlasampadādisuttapañcaka}}
\markboth{Five Discourses on Accomplishment in Ethics, Etc. (1st) }{Sīlasampadādisuttapañcaka}
\extramarks{SN 45.50–54}{SN 45.50–54}

“Mendicants,\marginnote{1.1} the dawn is the forerunner and precursor of the sunrise. In the same way accomplishment in ethics is the forerunner and precursor of the noble eightfold path for a mendicant. A mendicant accomplished in ethics can expect …” … 

“…\marginnote{1.4} accomplished in enthusiasm …” 

“…\marginnote{1.5} accomplished in self-development …” 

“…\marginnote{1.6} accomplished in view …” 

“…\marginnote{1.7} accomplished in diligence …” 

%
\section*{{\suttatitleacronym SN 45.55}{\suttatitletranslation Accomplishment in Proper Attention (1st) }{\suttatitleroot Yonisomanasikārasampadāsutta}}
\addcontentsline{toc}{section}{\tocacronym{SN 45.55} \toctranslation{Accomplishment in Proper Attention (1st) } \tocroot{Yonisomanasikārasampadāsutta}}
\markboth{Accomplishment in Proper Attention (1st) }{Yonisomanasikārasampadāsutta}
\extramarks{SN 45.55}{SN 45.55}

“Mendicants,\marginnote{1.1} the dawn is the forerunner and precursor of the sunrise. In the same way accomplishment in proper attention is the forerunner and precursor of the noble eightfold path for a mendicant. A mendicant accomplished in proper attention can expect to develop and cultivate the noble eightfold path. 

And\marginnote{1.4} how does a mendicant accomplished in proper attention develop and cultivate the noble eightfold path? It’s when a mendicant develops right view, right thought, right speech, right action, right livelihood, right effort, right mindfulness, and right immersion, which rely on seclusion, fading away, and cessation, and ripen as letting go. That’s how a mendicant accomplished in proper attention develops and cultivates the noble eightfold path.” 

%
\section*{{\suttatitleacronym SN 45.56}{\suttatitletranslation Good Friends (2nd) }{\suttatitleroot Dutiyakalyāṇamittasutta}}
\addcontentsline{toc}{section}{\tocacronym{SN 45.56} \toctranslation{Good Friends (2nd) } \tocroot{Dutiyakalyāṇamittasutta}}
\markboth{Good Friends (2nd) }{Dutiyakalyāṇamittasutta}
\extramarks{SN 45.56}{SN 45.56}

“Mendicants,\marginnote{1.1} the dawn is the forerunner and precursor of the sunrise. In the same way good friendship is the forerunner and precursor of the noble eightfold path for a mendicant. A mendicant with good friends can expect to develop and cultivate the noble eightfold path. 

And\marginnote{1.4} how does a mendicant with good friends develop and cultivate the noble eightfold path? It’s when a mendicant develops right view, right thought, right speech, right action, right livelihood, right effort, right mindfulness, and right immersion, which culminate in the removal of greed, hate, and delusion. That’s how a mendicant with good friends develops and cultivates the noble eightfold path.” 

%
\section*{{\suttatitleacronym SN 45.57–61}{\suttatitletranslation Five Discourses on Accomplishment in Ethics, Etc. (2nd) }{\suttatitleroot Dutiyasīlasampadādisuttapañcaka}}
\addcontentsline{toc}{section}{\tocacronym{SN 45.57–61} \toctranslation{Five Discourses on Accomplishment in Ethics, Etc. (2nd) } \tocroot{Dutiyasīlasampadādisuttapañcaka}}
\markboth{Five Discourses on Accomplishment in Ethics, Etc. (2nd) }{Dutiyasīlasampadādisuttapañcaka}
\extramarks{SN 45.57–61}{SN 45.57–61}

“Mendicants,\marginnote{1.1} the dawn is the forerunner and precursor of the sunrise. In the same way accomplishment in ethics is the forerunner and precursor of the noble eightfold path for a mendicant. …” 

“…\marginnote{1.3} accomplishment in enthusiasm …” 

“…\marginnote{1.4} accomplishment in self-development …” 

“…\marginnote{1.5} accomplishment in view …” 

“…\marginnote{1.6} accomplishment in diligence …” 

%
\section*{{\suttatitleacronym SN 45.62}{\suttatitletranslation Accomplishment in Proper Attention (2nd) }{\suttatitleroot Dutiyayonisomanasikārasampadāsutta}}
\addcontentsline{toc}{section}{\tocacronym{SN 45.62} \toctranslation{Accomplishment in Proper Attention (2nd) } \tocroot{Dutiyayonisomanasikārasampadāsutta}}
\markboth{Accomplishment in Proper Attention (2nd) }{Dutiyayonisomanasikārasampadāsutta}
\extramarks{SN 45.62}{SN 45.62}

“…\marginnote{1.1} accomplishment in proper attention. A mendicant accomplished in proper attention can expect to develop and cultivate the noble eightfold path. 

And\marginnote{1.3} how does a mendicant accomplished in proper attention develop and cultivate the noble eightfold path? It’s when a mendicant develops right view, right thought, right speech, right action, right livelihood, right effort, right mindfulness, and right immersion, which culminate in the removal of greed, hate, and delusion. That’s how a mendicant accomplished in proper attention develops and cultivates the noble eightfold path.” 

%
\addtocontents{toc}{\let\protect\contentsline\protect\nopagecontentsline}
\chapter*{The Chapter of Abbreviated Texts on One Thing }
\addcontentsline{toc}{chapter}{\tocchapterline{The Chapter of Abbreviated Texts on One Thing }}
\addtocontents{toc}{\let\protect\contentsline\protect\oldcontentsline}

%
\section*{{\suttatitleacronym SN 45.63}{\suttatitletranslation Good Friends (1st) }{\suttatitleroot Kalyāṇamittasutta}}
\addcontentsline{toc}{section}{\tocacronym{SN 45.63} \toctranslation{Good Friends (1st) } \tocroot{Kalyāṇamittasutta}}
\markboth{Good Friends (1st) }{Kalyāṇamittasutta}
\extramarks{SN 45.63}{SN 45.63}

At\marginnote{1.1} \textsanskrit{Sāvatthī}. 

“Mendicants,\marginnote{1.2} one thing helps give rise to the noble eightfold path. What one thing? It’s good friendship. A mendicant with good friends can expect to develop and cultivate the noble eightfold path. 

And\marginnote{1.6} how does a mendicant with good friends develop and cultivate the noble eightfold path? It’s when a mendicant develops right view, right thought, right speech, right action, right livelihood, right effort, right mindfulness, and right immersion, which rely on seclusion, fading away, and cessation, and ripen as letting go. That’s how a mendicant with good friends develops and cultivates the noble eightfold path.” 

%
\section*{{\suttatitleacronym SN 45.64–68}{\suttatitletranslation Five Discourses on Accomplishment in Ethics, Etc. (1st) }{\suttatitleroot Sīlasampadādisuttapañcaka}}
\addcontentsline{toc}{section}{\tocacronym{SN 45.64–68} \toctranslation{Five Discourses on Accomplishment in Ethics, Etc. (1st) } \tocroot{Sīlasampadādisuttapañcaka}}
\markboth{Five Discourses on Accomplishment in Ethics, Etc. (1st) }{Sīlasampadādisuttapañcaka}
\extramarks{SN 45.64–68}{SN 45.64–68}

“Mendicants,\marginnote{1.1} one thing helps give rise to the noble eightfold path. What one thing? It’s accomplishment in ethics. …” 

“…\marginnote{1.4} accomplishment in enthusiasm …” 

“…\marginnote{1.5} accomplishment in self-development …” 

“…\marginnote{1.6} accomplishment in view …” 

“…\marginnote{1.7} accomplishment in diligence …” 

%
\section*{{\suttatitleacronym SN 45.69}{\suttatitletranslation Accomplishment in Proper Attention (1st) }{\suttatitleroot Yonisomanasikārasampadāsutta}}
\addcontentsline{toc}{section}{\tocacronym{SN 45.69} \toctranslation{Accomplishment in Proper Attention (1st) } \tocroot{Yonisomanasikārasampadāsutta}}
\markboth{Accomplishment in Proper Attention (1st) }{Yonisomanasikārasampadāsutta}
\extramarks{SN 45.69}{SN 45.69}

“…\marginnote{1.1} accomplishment in proper attention. A mendicant accomplished in proper attention can expect to develop and cultivate the noble eightfold path. 

And\marginnote{1.3} how does a mendicant accomplished in proper attention develop and cultivate the noble eightfold path? It’s when a mendicant develops right view, right thought, right speech, right action, right livelihood, right effort, right mindfulness, and right immersion, which rely on seclusion, fading away, and cessation, and ripen as letting go. That’s how a mendicant accomplished in proper attention develops and cultivates the noble eightfold path.” 

%
\section*{{\suttatitleacronym SN 45.70}{\suttatitletranslation Good Friends (2nd) }{\suttatitleroot Dutiyakalyāṇamittasutta}}
\addcontentsline{toc}{section}{\tocacronym{SN 45.70} \toctranslation{Good Friends (2nd) } \tocroot{Dutiyakalyāṇamittasutta}}
\markboth{Good Friends (2nd) }{Dutiyakalyāṇamittasutta}
\extramarks{SN 45.70}{SN 45.70}

At\marginnote{1.1} \textsanskrit{Sāvatthī}. 

“Mendicants,\marginnote{1.2} one thing helps give rise to the noble eightfold path. What one thing? It’s good friendship. A mendicant with good friends can expect to develop and cultivate the noble eightfold path. 

And\marginnote{1.6} how does a mendicant with good friends develop and cultivate the noble eightfold path? It’s when a mendicant develops right view, right thought, right speech, right action, right livelihood, right effort, right mindfulness, and right immersion, which culminate in the removal of greed, hate, and delusion. That’s how a mendicant with good friends develops and cultivates the noble eightfold path.” 

%
\section*{{\suttatitleacronym SN 45.71–75}{\suttatitletranslation Five Discourses on Accomplishment in Ethics, Etc. (2nd) }{\suttatitleroot Dutiyasīlasampadādisuttapañcaka}}
\addcontentsline{toc}{section}{\tocacronym{SN 45.71–75} \toctranslation{Five Discourses on Accomplishment in Ethics, Etc. (2nd) } \tocroot{Dutiyasīlasampadādisuttapañcaka}}
\markboth{Five Discourses on Accomplishment in Ethics, Etc. (2nd) }{Dutiyasīlasampadādisuttapañcaka}
\extramarks{SN 45.71–75}{SN 45.71–75}

At\marginnote{1.1} \textsanskrit{Sāvatthī}. 

“Mendicants,\marginnote{1.2} one thing helps give rise to the noble eightfold path. What one thing? It’s accomplishment in ethics. …” 

“…\marginnote{1.5} accomplishment in enthusiasm …” 

“…\marginnote{1.6} accomplishment in self-development …” 

“…\marginnote{1.7} accomplishment in view …” 

“…\marginnote{1.8} accomplishment in diligence …” 

%
\section*{{\suttatitleacronym SN 45.76}{\suttatitletranslation Accomplishment in Proper Attention (2nd) }{\suttatitleroot Dutiyayonisomanasikārasampadāsutta}}
\addcontentsline{toc}{section}{\tocacronym{SN 45.76} \toctranslation{Accomplishment in Proper Attention (2nd) } \tocroot{Dutiyayonisomanasikārasampadāsutta}}
\markboth{Accomplishment in Proper Attention (2nd) }{Dutiyayonisomanasikārasampadāsutta}
\extramarks{SN 45.76}{SN 45.76}

“…\marginnote{1.1} accomplishment in proper attention. A mendicant accomplished in proper attention can expect to develop and cultivate the noble eightfold path. 

And\marginnote{1.3} how does a mendicant accomplished in proper attention develop and cultivate the noble eightfold path? It’s when a mendicant develops right view, right thought, right speech, right action, right livelihood, right effort, right mindfulness, and right immersion, which culminate in the removal of greed, hate, and delusion. That’s how a mendicant accomplished in proper attention develops and cultivates the noble eightfold path.” 

%
\addtocontents{toc}{\let\protect\contentsline\protect\nopagecontentsline}
\chapter*{The Chapter of Abbreviated Texts on One Thing }
\addcontentsline{toc}{chapter}{\tocchapterline{The Chapter of Abbreviated Texts on One Thing }}
\addtocontents{toc}{\let\protect\contentsline\protect\oldcontentsline}

%
\section*{{\suttatitleacronym SN 45.77}{\suttatitletranslation Good Friends }{\suttatitleroot Kalyāṇamittasutta}}
\addcontentsline{toc}{section}{\tocacronym{SN 45.77} \toctranslation{Good Friends } \tocroot{Kalyāṇamittasutta}}
\markboth{Good Friends }{Kalyāṇamittasutta}
\extramarks{SN 45.77}{SN 45.77}

At\marginnote{1.1} \textsanskrit{Sāvatthī}. 

“Mendicants,\marginnote{1.2} I do not see a single thing that gives rise to the noble eightfold path, or, if it’s already arisen, fully develops it like good friendship. A mendicant with good friends can expect to develop and cultivate the noble eightfold path. 

And\marginnote{1.4} how does a mendicant with good friends develop and cultivate the noble eightfold path? It’s when a mendicant develops right view, right thought, right speech, right action, right livelihood, right effort, right mindfulness, and right immersion, which rely on seclusion, fading away, and cessation, and ripen as letting go. That’s how a mendicant with good friends develops and cultivates the noble eightfold path.” 

%
\section*{{\suttatitleacronym SN 45.78–82}{\suttatitletranslation Five Discourses on Accomplishment in Ethics, Etc. }{\suttatitleroot Sīlasampadādisuttapañcaka}}
\addcontentsline{toc}{section}{\tocacronym{SN 45.78–82} \toctranslation{Five Discourses on Accomplishment in Ethics, Etc. } \tocroot{Sīlasampadādisuttapañcaka}}
\markboth{Five Discourses on Accomplishment in Ethics, Etc. }{Sīlasampadādisuttapañcaka}
\extramarks{SN 45.78–82}{SN 45.78–82}

“Mendicants,\marginnote{1.1} I do not see a single thing that gives rise to the noble eightfold path, or, if it’s already arisen, fully develops it like accomplishment in ethics. …” 

“…\marginnote{1.2} accomplishment in enthusiasm …” 

“…\marginnote{1.3} accomplishment in self-development …” 

“…\marginnote{1.4} accomplishment in view …” 

“…\marginnote{1.5} accomplishment in diligence …” 

%
\section*{{\suttatitleacronym SN 45.83}{\suttatitletranslation Accomplishment in Proper Attention }{\suttatitleroot Yonisomanasikārasampadāsutta}}
\addcontentsline{toc}{section}{\tocacronym{SN 45.83} \toctranslation{Accomplishment in Proper Attention } \tocroot{Yonisomanasikārasampadāsutta}}
\markboth{Accomplishment in Proper Attention }{Yonisomanasikārasampadāsutta}
\extramarks{SN 45.83}{SN 45.83}

“…\marginnote{1.1} accomplishment in proper attention. A mendicant accomplished in proper attention can expect to develop and cultivate the noble eightfold path. 

And\marginnote{1.3} how does a mendicant accomplished in proper attention develop and cultivate the noble eightfold path? It’s when a mendicant develops right view, right thought, right speech, right action, right livelihood, right effort, right mindfulness, and right immersion, which rely on seclusion, fading away, and cessation, and ripen as letting go. That’s how a mendicant accomplished in proper attention develops and cultivates the noble eightfold path.” 

%
\section*{{\suttatitleacronym SN 45.84}{\suttatitletranslation Good Friends (2nd) }{\suttatitleroot Dutiyakalyāṇamittasutta}}
\addcontentsline{toc}{section}{\tocacronym{SN 45.84} \toctranslation{Good Friends (2nd) } \tocroot{Dutiyakalyāṇamittasutta}}
\markboth{Good Friends (2nd) }{Dutiyakalyāṇamittasutta}
\extramarks{SN 45.84}{SN 45.84}

“Mendicants,\marginnote{1.1} I do not see a single thing that gives rise to the noble eightfold path, or, if it’s already arisen, fully develops it like good friendship. A mendicant with good friends can expect to develop and cultivate the noble eightfold path. 

And\marginnote{1.3} how does a mendicant with good friends develop and cultivate the noble eightfold path? It’s when a mendicant develops right view, right thought, right speech, right action, right livelihood, right effort, right mindfulness, and right immersion, which culminate in the removal of greed, hate, and delusion. That’s how a mendicant with good friends develops and cultivates the noble eightfold path.” 

%
\section*{{\suttatitleacronym SN 45.85–89}{\suttatitletranslation Five Discourses on Accomplishment in Ethics, Etc. }{\suttatitleroot Dutiyasīlasampadādisuttapañcaka}}
\addcontentsline{toc}{section}{\tocacronym{SN 45.85–89} \toctranslation{Five Discourses on Accomplishment in Ethics, Etc. } \tocroot{Dutiyasīlasampadādisuttapañcaka}}
\markboth{Five Discourses on Accomplishment in Ethics, Etc. }{Dutiyasīlasampadādisuttapañcaka}
\extramarks{SN 45.85–89}{SN 45.85–89}

“Mendicants,\marginnote{1.1} I do not see a single thing that gives rise to the noble eightfold path, or, if it’s already arisen, fully develops it like accomplishment in ethics. …” 

“…\marginnote{1.2} accomplishment in enthusiasm …” 

“…\marginnote{1.3} accomplishment in self-development …” 

“…\marginnote{1.4} accomplishment in view …” 

“…\marginnote{1.5} accomplishment in diligence …” 

%
\section*{{\suttatitleacronym SN 45.90}{\suttatitletranslation Accomplishment in Proper Attention (2nd) }{\suttatitleroot Dutiyayonisomanasikārasampadāsutta}}
\addcontentsline{toc}{section}{\tocacronym{SN 45.90} \toctranslation{Accomplishment in Proper Attention (2nd) } \tocroot{Dutiyayonisomanasikārasampadāsutta}}
\markboth{Accomplishment in Proper Attention (2nd) }{Dutiyayonisomanasikārasampadāsutta}
\extramarks{SN 45.90}{SN 45.90}

“…\marginnote{1.1} accomplishment in proper attention. A mendicant accomplished in proper attention can expect to develop and cultivate the noble eightfold path. 

And\marginnote{1.3} how does a mendicant accomplished in proper attention develop and cultivate the noble eightfold path? It’s when a mendicant develops right view, right thought, right speech, right action, right livelihood, right effort, right mindfulness, and right immersion, which culminate in the removal of greed, hate, and delusion. That’s how a mendicant accomplished in proper attention develops and cultivates the noble eightfold path.” 

%
\addtocontents{toc}{\let\protect\contentsline\protect\nopagecontentsline}
\chapter*{The Chapter of Abbreviated Texts on the Ganges }
\addcontentsline{toc}{chapter}{\tocchapterline{The Chapter of Abbreviated Texts on the Ganges }}
\addtocontents{toc}{\let\protect\contentsline\protect\oldcontentsline}

%
\section*{{\suttatitleacronym SN 45.91}{\suttatitletranslation Slanting East }{\suttatitleroot Paṭhamapācīnaninnasutta}}
\addcontentsline{toc}{section}{\tocacronym{SN 45.91} \toctranslation{Slanting East } \tocroot{Paṭhamapācīnaninnasutta}}
\markboth{Slanting East }{Paṭhamapācīnaninnasutta}
\extramarks{SN 45.91}{SN 45.91}

At\marginnote{1.1} \textsanskrit{Sāvatthī}. 

“Mendicants,\marginnote{1.2} the Ganges river slants, slopes, and inclines to the east. In the same way, a mendicant who develops and cultivates the noble eightfold path slants, slopes, and inclines to extinguishment. 

And\marginnote{1.4} how does a mendicant who develops the noble eightfold path slant, slope, and incline to extinguishment? It’s when a mendicant develops right view, right thought, right speech, right action, right livelihood, right effort, right mindfulness, and right immersion, which rely on seclusion, fading away, and cessation, and ripen as letting go. That’s how a mendicant who develops and cultivates the noble eightfold path slants, slopes, and inclines to extinguishment.” 

%
\section*{{\suttatitleacronym SN 45.92–95}{\suttatitletranslation Four Discourses on Slanting East }{\suttatitleroot Dutiyādipācīnaninnasuttacatukka}}
\addcontentsline{toc}{section}{\tocacronym{SN 45.92–95} \toctranslation{Four Discourses on Slanting East } \tocroot{Dutiyādipācīnaninnasuttacatukka}}
\markboth{Four Discourses on Slanting East }{Dutiyādipācīnaninnasuttacatukka}
\extramarks{SN 45.92–95}{SN 45.92–95}

“Mendicants,\marginnote{1.1} the \textsanskrit{Yamunā} river slants, slopes, and inclines to the east. …” 

“…\marginnote{1.3} the \textsanskrit{Aciravatī} river …” 

“…\marginnote{1.5} the \textsanskrit{Sarabhū} river …” 

“…\marginnote{1.7} the \textsanskrit{Mahī} river …” 

%
\section*{{\suttatitleacronym SN 45.96}{\suttatitletranslation Sixth Discourse on Slanting East }{\suttatitleroot Chaṭṭhapācīnaninnasutta}}
\addcontentsline{toc}{section}{\tocacronym{SN 45.96} \toctranslation{Sixth Discourse on Slanting East } \tocroot{Chaṭṭhapācīnaninnasutta}}
\markboth{Sixth Discourse on Slanting East }{Chaṭṭhapācīnaninnasutta}
\extramarks{SN 45.96}{SN 45.96}

“Mendicants,\marginnote{1.1} all the great rivers—that is, the Ganges, \textsanskrit{Yamunā}, \textsanskrit{Aciravatī}, \textsanskrit{Sarabhū}, and \textsanskrit{Mahī}—slant, slope, and incline towards the east. In the same way, a mendicant who develops and cultivates the noble eightfold path slants, slopes, and inclines to extinguishment. 

And\marginnote{1.4} how does a mendicant who develops the noble eightfold path slant, slope, and incline to extinguishment? It’s when a mendicant develops right view, right thought, right speech, right action, right livelihood, right effort, right mindfulness, and right immersion, which rely on seclusion, fading away, and cessation, and ripen as letting go. That’s how a mendicant who develops and cultivates the noble eightfold path slants, slopes, and inclines to extinguishment.” 

%
\section*{{\suttatitleacronym SN 45.97}{\suttatitletranslation Slanting to the Ocean }{\suttatitleroot Paṭhamasamuddaninnasutta}}
\addcontentsline{toc}{section}{\tocacronym{SN 45.97} \toctranslation{Slanting to the Ocean } \tocroot{Paṭhamasamuddaninnasutta}}
\markboth{Slanting to the Ocean }{Paṭhamasamuddaninnasutta}
\extramarks{SN 45.97}{SN 45.97}

“Mendicants,\marginnote{1.1} the Ganges river slants, slopes, and inclines to the ocean. In the same way, a mendicant who develops the noble eightfold path slants, slopes, and inclines to extinguishment. …” 

%
\section*{{\suttatitleacronym SN 45.98–102}{\suttatitletranslation Five Discourses on Slanting to the Ocean }{\suttatitleroot Dutiyādisamuddaninnasuttapañcaka}}
\addcontentsline{toc}{section}{\tocacronym{SN 45.98–102} \toctranslation{Five Discourses on Slanting to the Ocean } \tocroot{Dutiyādisamuddaninnasuttapañcaka}}
\markboth{Five Discourses on Slanting to the Ocean }{Dutiyādisamuddaninnasuttapañcaka}
\extramarks{SN 45.98–102}{SN 45.98–102}

“Mendicants,\marginnote{1.1} the \textsanskrit{Yamunā} river slants, slopes, and inclines to the ocean. …” 

“…\marginnote{1.3} the \textsanskrit{Aciravatī} river …” 

“…\marginnote{1.5} the \textsanskrit{Sarabhū} river …” 

“…\marginnote{1.7} the \textsanskrit{Mahī} river …” 

“…\marginnote{1.9} all the great rivers …” 

%
\addtocontents{toc}{\let\protect\contentsline\protect\nopagecontentsline}
\chapter*{The Chapter of Abbreviated Texts on the Ganges }
\addcontentsline{toc}{chapter}{\tocchapterline{The Chapter of Abbreviated Texts on the Ganges }}
\addtocontents{toc}{\let\protect\contentsline\protect\oldcontentsline}

%
\section*{{\suttatitleacronym SN 45.103}{\suttatitletranslation Slanting East }{\suttatitleroot Paṭhamapācīnaninnasutta}}
\addcontentsline{toc}{section}{\tocacronym{SN 45.103} \toctranslation{Slanting East } \tocroot{Paṭhamapācīnaninnasutta}}
\markboth{Slanting East }{Paṭhamapācīnaninnasutta}
\extramarks{SN 45.103}{SN 45.103}

“Mendicants,\marginnote{1.1} the Ganges river slants, slopes, and inclines to the east. In the same way, a mendicant who develops and cultivates the noble eightfold path slants, slopes, and inclines to extinguishment. 

And\marginnote{1.3} how does a mendicant who develops the noble eightfold path slant, slope, and incline to extinguishment? It’s when a mendicant develops right view, right thought, right speech, right action, right livelihood, right effort, right mindfulness, and right immersion, which culminate in the removal of greed, hate, and delusion. That’s how a mendicant who develops and cultivates the noble eightfold path slants, slopes, and inclines to extinguishment.” 

%
\section*{{\suttatitleacronym SN 45.104–108}{\suttatitletranslation Five Discourses on Sloping to the East }{\suttatitleroot Dutiyādipācīnaninnasuttapañcaka}}
\addcontentsline{toc}{section}{\tocacronym{SN 45.104–108} \toctranslation{Five Discourses on Sloping to the East } \tocroot{Dutiyādipācīnaninnasuttapañcaka}}
\markboth{Five Discourses on Sloping to the East }{Dutiyādipācīnaninnasuttapañcaka}
\extramarks{SN 45.104–108}{SN 45.104–108}

“Mendicants,\marginnote{1.1} the \textsanskrit{Yamunā} river slants, slopes, and inclines to the east. …” 

“…\marginnote{1.1} the \textsanskrit{Aciravatī} river …” 

“…\marginnote{1.1} the \textsanskrit{Sarabhū} river …” 

“…\marginnote{1.1} the \textsanskrit{Mahī} river …” 

“…\marginnote{1.1} all the great rivers …” 

%
\section*{{\suttatitleacronym SN 45.109}{\suttatitletranslation Slanting to the Ocean }{\suttatitleroot Paṭhamasamuddaninnasutta}}
\addcontentsline{toc}{section}{\tocacronym{SN 45.109} \toctranslation{Slanting to the Ocean } \tocroot{Paṭhamasamuddaninnasutta}}
\markboth{Slanting to the Ocean }{Paṭhamasamuddaninnasutta}
\extramarks{SN 45.109}{SN 45.109}

“Mendicants,\marginnote{1.1} the Ganges river slants, slopes, and inclines to the ocean. In the same way, a mendicant who develops and cultivates the noble eightfold path slants, slopes, and inclines to extinguishment. 

And\marginnote{1.3} how does a mendicant who develops the noble eightfold path slant, slope, and incline to extinguishment? It’s when a mendicant develops right view, right thought, right speech, right action, right livelihood, right effort, right mindfulness, and right immersion, which culminate in the removal of greed, hate, and delusion. That’s how a mendicant who develops and cultivates the noble eightfold path slants, slopes, and inclines to extinguishment.” 

%
\section*{{\suttatitleacronym SN 45.110–114}{\suttatitletranslation Slanting to the Ocean }{\suttatitleroot Dutiyādisamuddaninnasutta}}
\addcontentsline{toc}{section}{\tocacronym{SN 45.110–114} \toctranslation{Slanting to the Ocean } \tocroot{Dutiyādisamuddaninnasutta}}
\markboth{Slanting to the Ocean }{Dutiyādisamuddaninnasutta}
\extramarks{SN 45.110–114}{SN 45.110–114}

“Mendicants,\marginnote{1.1} the \textsanskrit{Yamunā} river slants, slopes, and inclines to the ocean. …” 

“…\marginnote{1.1} the \textsanskrit{Aciravatī} river …” 

“…\marginnote{1.1} the \textsanskrit{Sarabhū} river …” 

“…\marginnote{1.1} the \textsanskrit{Mahī} river …” 

“…\marginnote{1.1} all the great rivers …” 

%
\section*{{\suttatitleacronym SN 45.115}{\suttatitletranslation Slanting East }{\suttatitleroot Paṭhamapācīnaninnasutta}}
\addcontentsline{toc}{section}{\tocacronym{SN 45.115} \toctranslation{Slanting East } \tocroot{Paṭhamapācīnaninnasutta}}
\markboth{Slanting East }{Paṭhamapācīnaninnasutta}
\extramarks{SN 45.115}{SN 45.115}

“Mendicants,\marginnote{1.1} the Ganges river slants, slopes, and inclines to the east. In the same way, a mendicant who develops and cultivates the noble eightfold path slants, slopes, and inclines to extinguishment. 

And\marginnote{1.3} how does a mendicant who develops the noble eightfold path slant, slope, and incline to extinguishment? It’s when a mendicant develops right view, right thought, right speech, right action, right livelihood, right effort, right mindfulness, and right immersion, which culminate, finish, and end in the deathless. That’s how a mendicant who develops and cultivates the noble eightfold path slants, slopes, and inclines to extinguishment.” 

%
\section*{{\suttatitleacronym SN 45.116–120}{\suttatitletranslation Slanting East }{\suttatitleroot Dutiyādipācīnaninnasutta}}
\addcontentsline{toc}{section}{\tocacronym{SN 45.116–120} \toctranslation{Slanting East } \tocroot{Dutiyādipācīnaninnasutta}}
\markboth{Slanting East }{Dutiyādipācīnaninnasutta}
\extramarks{SN 45.116–120}{SN 45.116–120}

“Mendicants,\marginnote{1.1} the \textsanskrit{Yamunā} river slants, slopes, and inclines to the east. …” 

“…\marginnote{1.1} the \textsanskrit{Aciravatī} river …” 

“…\marginnote{1.1} the \textsanskrit{Sarabhū} river …” 

“…\marginnote{1.1} the \textsanskrit{Mahī} river …” 

“…\marginnote{1.1} all the great rivers …” 

%
\section*{{\suttatitleacronym SN 45.121}{\suttatitletranslation Slanting to the Ocean }{\suttatitleroot Paṭhamasamuddaninnasutta}}
\addcontentsline{toc}{section}{\tocacronym{SN 45.121} \toctranslation{Slanting to the Ocean } \tocroot{Paṭhamasamuddaninnasutta}}
\markboth{Slanting to the Ocean }{Paṭhamasamuddaninnasutta}
\extramarks{SN 45.121}{SN 45.121}

“Mendicants,\marginnote{1.1} the Ganges river slants, slopes, and inclines to the ocean. In the same way, a mendicant who develops and cultivates the noble eightfold path slants, slopes, and inclines to extinguishment. 

And\marginnote{1.3} how does a mendicant who develops the noble eightfold path slant, slope, and incline to extinguishment? It’s when a mendicant develops right view, right thought, right speech, right action, right livelihood, right effort, right mindfulness, and right immersion, which culminate, finish, and end in the deathless. That’s how a mendicant who develops and cultivates the noble eightfold path slants, slopes, and inclines to extinguishment.” 

%
\section*{{\suttatitleacronym SN 45.122–126}{\suttatitletranslation Sloping to the Ocean }{\suttatitleroot Dutiyādisamuddaninnasutta}}
\addcontentsline{toc}{section}{\tocacronym{SN 45.122–126} \toctranslation{Sloping to the Ocean } \tocroot{Dutiyādisamuddaninnasutta}}
\markboth{Sloping to the Ocean }{Dutiyādisamuddaninnasutta}
\extramarks{SN 45.122–126}{SN 45.122–126}

“Mendicants,\marginnote{1.1} the \textsanskrit{Yamunā} river slants, slopes, and inclines to the ocean. …” 

“…\marginnote{1.3} the \textsanskrit{Aciravatī} river …” 

“…\marginnote{1.5} the \textsanskrit{Sarabhū} river …” 

“…\marginnote{1.7} the \textsanskrit{Mahī} river …” 

“…\marginnote{1.9} all the great rivers …” 

%
\section*{{\suttatitleacronym SN 45.127}{\suttatitletranslation Slanting East }{\suttatitleroot Paṭhamapācīnaninnasutta}}
\addcontentsline{toc}{section}{\tocacronym{SN 45.127} \toctranslation{Slanting East } \tocroot{Paṭhamapācīnaninnasutta}}
\markboth{Slanting East }{Paṭhamapācīnaninnasutta}
\extramarks{SN 45.127}{SN 45.127}

“Mendicants,\marginnote{1.1} the Ganges river slants, slopes, and inclines to the east. In the same way, a mendicant who develops and cultivates the noble eightfold path slants, slopes, and inclines to extinguishment. 

And\marginnote{1.3} how does a mendicant who develops the noble eightfold path slant, slope, and incline to extinguishment? It’s when a mendicant develops right view, right thought, right speech, right action, right livelihood, right effort, right mindfulness, and right immersion, which slants, slopes, and inclines to extinguishment. That’s how a mendicant who develops and cultivates the noble eightfold path slants, slopes, and inclines to extinguishment.” 

%
\section*{{\suttatitleacronym SN 45.128–132}{\suttatitletranslation Slanting East }{\suttatitleroot Dutiyādipācīnaninnasutta}}
\addcontentsline{toc}{section}{\tocacronym{SN 45.128–132} \toctranslation{Slanting East } \tocroot{Dutiyādipācīnaninnasutta}}
\markboth{Slanting East }{Dutiyādipācīnaninnasutta}
\extramarks{SN 45.128–132}{SN 45.128–132}

“Mendicants,\marginnote{1.1} the \textsanskrit{Yamunā} river slants, slopes, and inclines to the east. …” 

“…\marginnote{1.3} the \textsanskrit{Aciravatī} river …” 

“…\marginnote{1.5} the \textsanskrit{Sarabhū} river …” 

“…\marginnote{1.7} the \textsanskrit{Mahī} river …” 

“…\marginnote{1.9} all the great rivers …” 

%
\section*{{\suttatitleacronym SN 45.133}{\suttatitletranslation Slanting to the Ocean }{\suttatitleroot Paṭhamasamuddaninnasutta}}
\addcontentsline{toc}{section}{\tocacronym{SN 45.133} \toctranslation{Slanting to the Ocean } \tocroot{Paṭhamasamuddaninnasutta}}
\markboth{Slanting to the Ocean }{Paṭhamasamuddaninnasutta}
\extramarks{SN 45.133}{SN 45.133}

“Mendicants,\marginnote{1.1} the Ganges river slants, slopes, and inclines to the ocean. In the same way, a mendicant who develops and cultivates the noble eightfold path slants, slopes, and inclines to extinguishment. 

And\marginnote{1.3} how does a mendicant who develops the noble eightfold path slant, slope, and incline to extinguishment? It’s when a mendicant develops right view, right thought, right speech, right action, right livelihood, right effort, right mindfulness, and right immersion, which slants, slopes, and inclines to extinguishment. That’s how a mendicant who develops and cultivates the noble eightfold path slants, slopes, and inclines to extinguishment.” 

%
\section*{{\suttatitleacronym SN 45.134–138}{\suttatitletranslation Slanting to the Ocean }{\suttatitleroot Dutiyādisamuddaninnasutta}}
\addcontentsline{toc}{section}{\tocacronym{SN 45.134–138} \toctranslation{Slanting to the Ocean } \tocroot{Dutiyādisamuddaninnasutta}}
\markboth{Slanting to the Ocean }{Dutiyādisamuddaninnasutta}
\extramarks{SN 45.134–138}{SN 45.134–138}

“Mendicants,\marginnote{1.1} the \textsanskrit{Yamunā} river slants, slopes, and inclines to the ocean. …” 

“…\marginnote{1.3} the \textsanskrit{Aciravatī} river …” 

“…\marginnote{1.5} the \textsanskrit{Sarabhū} river …” 

“…\marginnote{1.7} the \textsanskrit{Mahī} river …” 

“…\marginnote{1.9} all the great rivers …” 

%
\addtocontents{toc}{\let\protect\contentsline\protect\nopagecontentsline}
\chapter*{The Chapter of Abbreviated Texts on Diligence }
\addcontentsline{toc}{chapter}{\tocchapterline{The Chapter of Abbreviated Texts on Diligence }}
\addtocontents{toc}{\let\protect\contentsline\protect\oldcontentsline}

%
\section*{{\suttatitleacronym SN 45.139}{\suttatitletranslation The Realized One }{\suttatitleroot Tathāgatasutta}}
\addcontentsline{toc}{section}{\tocacronym{SN 45.139} \toctranslation{The Realized One } \tocroot{Tathāgatasutta}}
\markboth{The Realized One }{Tathāgatasutta}
\extramarks{SN 45.139}{SN 45.139}

At\marginnote{1.1} \textsanskrit{Sāvatthī}. 

“Mendicants,\marginnote{1.2} the Realized One, the perfected one, the fully awakened Buddha, is said to be the best of all sentient beings—be they footless, with two feet, four feet, or many feet; with form or formless; with perception or without perception or with neither perception nor non-perception. In the same way, all skillful qualities are rooted in diligence and meet at diligence, and diligence is said to be the best of them. A mendicant who is diligent can expect to develop and cultivate the noble eightfold path. 

And\marginnote{1.6} how does a mendicant who is diligent develop and cultivate the noble eightfold path? It’s when a mendicant develops right view, right thought, right speech, right action, right livelihood, right effort, right mindfulness, and right immersion, which rely on seclusion, fading away, and cessation, and ripen as letting go. That’s how a mendicant who is diligent develops and cultivates the noble eightfold path. 

Mendicants,\marginnote{2.1} the Realized One, the perfected one, the fully awakened Buddha, is said to be the best of all sentient beings—be they footless, with two feet, four feet, or many feet; with form or formless; with perception or without perception or with neither perception nor non-perception. In the same way, all skillful qualities are rooted in diligence and meet at diligence, and diligence is said to be the best of them. A mendicant who is diligent can expect to develop and cultivate the noble eightfold path. 

And\marginnote{2.4} how does a mendicant who is diligent develop and cultivate the noble eightfold path? It’s when a mendicant develops right view, right thought, right speech, right action, right livelihood, right effort, right mindfulness, and right immersion, which culminate in the removal of greed, hate, and delusion. That’s how a mendicant who is diligent develops and cultivates the noble eightfold path. 

Mendicants,\marginnote{3.1} the Realized One, the perfected one, the fully awakened Buddha, is said to be the best of all sentient beings—be they footless, with two feet, four feet, or many feet; with form or formless; with perception or without perception or with neither perception nor non-perception. In the same way, all skillful qualities are rooted in diligence and meet at diligence, and diligence is said to be the best of them. A mendicant who is diligent can expect to develop and cultivate the noble eightfold path. 

And\marginnote{3.4} how does a mendicant who is diligent develop and cultivate the noble eightfold path? It’s when a mendicant develops right view, right thought, right speech, right action, right livelihood, right effort, right mindfulness, and right immersion, which culminate, finish, and end in the deathless. That’s how a mendicant who is diligent develops and cultivates the noble eightfold path. 

Mendicants,\marginnote{4.1} the Realized One, the perfected one, the fully awakened Buddha, is said to be the best of all sentient beings—be they footless, with two feet, four feet, or many feet; with form or formless; with perception or without perception or with neither perception nor non-perception. In the same way, all skillful qualities are rooted in diligence and meet at diligence, and diligence is said to be the best of them. A mendicant who is diligent can expect to develop and cultivate the noble eightfold path. 

And\marginnote{4.5} how does a mendicant who is diligent develop and cultivate the noble eightfold path? It’s when a mendicant develops right view, right thought, right speech, right action, right livelihood, right effort, right mindfulness, and right immersion, which slants, slopes, and inclines to extinguishment. That’s how a mendicant who is diligent develops and cultivates the noble eightfold path.” 

%
\section*{{\suttatitleacronym SN 45.140}{\suttatitletranslation Footprints }{\suttatitleroot Padasutta}}
\addcontentsline{toc}{section}{\tocacronym{SN 45.140} \toctranslation{Footprints } \tocroot{Padasutta}}
\markboth{Footprints }{Padasutta}
\extramarks{SN 45.140}{SN 45.140}

“The\marginnote{1.1} footprints of all creatures that walk can fit inside an elephant’s footprint. So an elephant’s footprint is said to be the biggest of them all. In the same way, all skillful qualities are rooted in diligence and meet at diligence, and diligence is said to be the best of them. A mendicant who is diligent can expect to develop and cultivate the noble eightfold path. 

And\marginnote{1.5} how does a mendicant who is diligent develop and cultivate the noble eightfold path? It’s when a mendicant develops right view, right thought, right speech, right action, right livelihood, right effort, right mindfulness, and right immersion, which rely on seclusion, fading away, and cessation, and ripen as letting go. … That’s how a mendicant who is diligent develops and cultivates the noble eightfold path.” 

%
\section*{{\suttatitleacronym SN 45.141–145}{\suttatitletranslation A Roof Peak }{\suttatitleroot Kūṭādisutta}}
\addcontentsline{toc}{section}{\tocacronym{SN 45.141–145} \toctranslation{A Roof Peak } \tocroot{Kūṭādisutta}}
\markboth{A Roof Peak }{Kūṭādisutta}
\extramarks{SN 45.141–145}{SN 45.141–145}

“Mendicants,\marginnote{1.1} the rafters of a bungalow all lean to the peak, slope to the peak, and meet at the peak, so the peak is said to be the topmost of them all. In the same way …” 

(This\marginnote{1.3} should be told in full as in the previous discourse.) 

“Of\marginnote{2.1} all kinds of fragrant root, spikenard is said to be the best. …” 

“Of\marginnote{3.1} all kinds of fragrant heartwood, red sandalwood is said to be the best. …” 

“Of\marginnote{4.1} all kinds of fragrant flower, jasmine is said to be the best. …” 

“All\marginnote{5.1} lesser kings are vassals of a wheel-turning monarch, so the wheel-turning monarch is said to be the foremost of them all. …” 

%
\section*{{\suttatitleacronym SN 45.146–148}{\suttatitletranslation The Moon, Etc. }{\suttatitleroot Candimādisutta}}
\addcontentsline{toc}{section}{\tocacronym{SN 45.146–148} \toctranslation{The Moon, Etc. } \tocroot{Candimādisutta}}
\markboth{The Moon, Etc. }{Candimādisutta}
\extramarks{SN 45.146–148}{SN 45.146–148}

“The\marginnote{1.1} radiance of all the stars is not worth a sixteenth part of the moon’s radiance, so the moon’s radiance is said to be the best of them all. …” 

“After\marginnote{2.1} the rainy season the sky is clear and cloudless. And when the sun rises, it dispels all the darkness from the sky as it shines and glows and radiates. …” 

“Mendicants,\marginnote{3.1} cloth from \textsanskrit{Kāsī} is said to be the best kind of woven cloth. …” 

\scendsection{(These should all be expanded as in the section on the Realized One.) }

%
\addtocontents{toc}{\let\protect\contentsline\protect\nopagecontentsline}
\chapter*{The Chapter on Hard Work }
\addcontentsline{toc}{chapter}{\tocchapterline{The Chapter on Hard Work }}
\addtocontents{toc}{\let\protect\contentsline\protect\oldcontentsline}

%
\section*{{\suttatitleacronym SN 45.149}{\suttatitletranslation Hard Work }{\suttatitleroot Balasutta}}
\addcontentsline{toc}{section}{\tocacronym{SN 45.149} \toctranslation{Hard Work } \tocroot{Balasutta}}
\markboth{Hard Work }{Balasutta}
\extramarks{SN 45.149}{SN 45.149}

At\marginnote{1.1} \textsanskrit{Sāvatthī}. 

“Mendicants,\marginnote{1.2} all the hard work that gets done depends on the earth and is grounded on the earth. In the same way, a mendicant develops and cultivates the noble eightfold path depending on and grounded on ethics. 

And\marginnote{1.4} how does a mendicant grounded on ethics develop and cultivate the noble eightfold path? It’s when a mendicant develops right view, right thought, right speech, right action, right livelihood, right effort, right mindfulness, and right immersion, which rely on seclusion, fading away, and cessation, and ripen as letting go. That’s how a mendicant grounded on ethics develops and cultivates the noble eightfold path.” 

“…\marginnote{3.3} which culminate in the removal of greed, hate, and delusion …” 

“…\marginnote{4.3} culminate, finish, and end in the deathless …” 

“…\marginnote{5.3} slants, slopes, and inclines to extinguishment …” 

%
\section*{{\suttatitleacronym SN 45.150}{\suttatitletranslation Seeds }{\suttatitleroot Bījasutta}}
\addcontentsline{toc}{section}{\tocacronym{SN 45.150} \toctranslation{Seeds } \tocroot{Bījasutta}}
\markboth{Seeds }{Bījasutta}
\extramarks{SN 45.150}{SN 45.150}

“All\marginnote{1.1} the plants and seeds that achieve growth, increase, and maturity do so depending on the earth and grounded on the earth. In the same way, a mendicant develops and cultivates the noble eightfold path depending on and grounded on ethics, achieving growth, increase, and maturity in good qualities. 

And\marginnote{1.3} how does a mendicant develop the noble eightfold path depending on and grounded on ethics, achieving growth, increase, and maturity in good qualities? It’s when a mendicant develops right view, right thought, right speech, right action, right livelihood, right effort, right mindfulness, and right immersion, which rely on seclusion, fading away, and cessation, and ripen as letting go. That’s how a mendicant develops and cultivates the noble eightfold path depending on and grounded on ethics, achieving growth, increase, and maturity in good qualities.” 

%
\section*{{\suttatitleacronym SN 45.151}{\suttatitletranslation Dragons }{\suttatitleroot Nāgasutta}}
\addcontentsline{toc}{section}{\tocacronym{SN 45.151} \toctranslation{Dragons } \tocroot{Nāgasutta}}
\markboth{Dragons }{Nāgasutta}
\extramarks{SN 45.151}{SN 45.151}

“Mendicants,\marginnote{1.1} dragons grow and wax strong supported by the Himalayas, the king of mountains. When they’re strong they dive into the pools. Then they dive into the lakes, the streams, the rivers, and finally the ocean. There they acquire a great and abundant body. In the same way, a mendicant develops and cultivates the noble eightfold path depending on and grounded on ethics, acquiring great and abundant good qualities. 

And\marginnote{1.4} how does a mendicant develop the noble eightfold path depending on and grounded on ethics, acquiring great and abundant good qualities? It’s when a mendicant develops right view, right thought, right speech, right action, right livelihood, right effort, right mindfulness, and right immersion, which rely on seclusion, fading away, and cessation, and ripen as letting go. That’s how a mendicant develops and cultivates the noble eightfold path depending on and grounded on ethics, acquiring great and abundant good qualities.” 

%
\section*{{\suttatitleacronym SN 45.152}{\suttatitletranslation Trees }{\suttatitleroot Rukkhasutta}}
\addcontentsline{toc}{section}{\tocacronym{SN 45.152} \toctranslation{Trees } \tocroot{Rukkhasutta}}
\markboth{Trees }{Rukkhasutta}
\extramarks{SN 45.152}{SN 45.152}

“Mendicants,\marginnote{1.1} suppose a tree slants, slopes, and inclines to the east. If it was cut off at the root, where would it fall?” 

“Sir,\marginnote{1.3} it would fall in the direction that it slants, slopes, and inclines.” 

“In\marginnote{1.4} the same way, a mendicant who develops and cultivates the noble eightfold path slants, slopes, and inclines to extinguishment. 

And\marginnote{1.5} how does a mendicant who develops the noble eightfold path slant, slope, and incline to extinguishment? It’s when a mendicant develops right view, right thought, right speech, right action, right livelihood, right effort, right mindfulness, and right immersion, which rely on seclusion, fading away, and cessation, and ripen as letting go. That’s how a mendicant who develops and cultivates the noble eightfold path slants, slopes, and inclines to extinguishment.” 

%
\section*{{\suttatitleacronym SN 45.153}{\suttatitletranslation Pots }{\suttatitleroot Kumbhasutta}}
\addcontentsline{toc}{section}{\tocacronym{SN 45.153} \toctranslation{Pots } \tocroot{Kumbhasutta}}
\markboth{Pots }{Kumbhasutta}
\extramarks{SN 45.153}{SN 45.153}

“Mendicants,\marginnote{1.1} suppose a pot full of water is tipped over, so the water drains out and doesn’t go back in. In the same way, a mendicant who develops and cultivates the noble eightfold path expels bad, unskillful qualities and doesn’t let them back in. 

And\marginnote{1.3} how does a mendicant who develops the noble eightfold path expel bad, unskillful qualities and not let them back in? It’s when a mendicant develops right view, right thought, right speech, right action, right livelihood, right effort, right mindfulness, and right immersion, which rely on seclusion, fading away, and cessation, and ripen as letting go. That’s how a mendicant who develops and cultivates the noble eightfold path expels bad, unskillful qualities and doesn’t let them back in.” 

%
\section*{{\suttatitleacronym SN 45.154}{\suttatitletranslation A Spike }{\suttatitleroot Sūkasutta}}
\addcontentsline{toc}{section}{\tocacronym{SN 45.154} \toctranslation{A Spike } \tocroot{Sūkasutta}}
\markboth{A Spike }{Sūkasutta}
\extramarks{SN 45.154}{SN 45.154}

“Mendicants,\marginnote{1.1} suppose a spike of rice or barley was pointing the right way. If you trod on it with hand or foot, it may well break the skin and produce blood. Why is that? Because the spike is pointing the right way. 

In\marginnote{1.4} the same way, a mendicant whose view and development of the path is pointing the right way may well break ignorance, produce knowledge, and realize extinguishment. Why is that? Because their view is pointing the right way. 

And\marginnote{1.7} how does a mendicant whose view and development of the path is pointing the right way break ignorance, give rise to knowledge, and realize extinguishment? It’s when a mendicant develops right view, right thought, right speech, right action, right livelihood, right effort, right mindfulness, and right immersion, which rely on seclusion, fading away, and cessation, and ripen as letting go. That’s how a mendicant whose view and development of the path is pointing the right way breaks ignorance, gives rise to knowledge, and realizes extinguishment.” 

%
\section*{{\suttatitleacronym SN 45.155}{\suttatitletranslation The Sky }{\suttatitleroot Ākāsasutta}}
\addcontentsline{toc}{section}{\tocacronym{SN 45.155} \toctranslation{The Sky } \tocroot{Ākāsasutta}}
\markboth{The Sky }{Ākāsasutta}
\extramarks{SN 45.155}{SN 45.155}

“Mendicants,\marginnote{1.1} various winds blow in the sky. Winds blow from the east, the west, the north, and the south. There are winds that are dusty and dustless, cool and warm, weak and strong. In the same way, when the noble eightfold path is developed and cultivated the following are fully developed: the four kinds of mindfulness meditation, the four right efforts, the four bases of psychic power, the five faculties, the five powers, and the seven awakening factors. 

And\marginnote{1.4} how are they fully developed? It’s when a mendicant develops right view, right thought, right speech, right action, right livelihood, right effort, right mindfulness, and right immersion, which rely on seclusion, fading away, and cessation, and ripen as letting go. That’s how they’re fully developed.” 

%
\section*{{\suttatitleacronym SN 45.156}{\suttatitletranslation Storms (1st) }{\suttatitleroot Paṭhamameghasutta}}
\addcontentsline{toc}{section}{\tocacronym{SN 45.156} \toctranslation{Storms (1st) } \tocroot{Paṭhamameghasutta}}
\markboth{Storms (1st) }{Paṭhamameghasutta}
\extramarks{SN 45.156}{SN 45.156}

“Mendicants,\marginnote{1.1} in the last month of summer, when the dust and dirt is stirred up, a large sudden storm disperses and settles it on the spot. In the same way, a mendicant who develops and cultivates the noble eightfold path disperses and stills bad, unskillful qualities on the spot. 

How\marginnote{1.3} does a mendicant who develops the noble eightfold path disperse and still bad, unskillful qualities on the spot? It’s when a mendicant develops right view, right thought, right speech, right action, right livelihood, right effort, right mindfulness, and right immersion, which rely on seclusion, fading away, and cessation, and ripen as letting go. That’s how a mendicant who develops and cultivates the noble eightfold path disperses and stills bad, unskillful qualities on the spot.” 

%
\section*{{\suttatitleacronym SN 45.157}{\suttatitletranslation Storms (2nd) }{\suttatitleroot Dutiyameghasutta}}
\addcontentsline{toc}{section}{\tocacronym{SN 45.157} \toctranslation{Storms (2nd) } \tocroot{Dutiyameghasutta}}
\markboth{Storms (2nd) }{Dutiyameghasutta}
\extramarks{SN 45.157}{SN 45.157}

“Mendicants,\marginnote{1.1} when a large storm has arisen, a strong wind disperses and settles it as it proceeds. In the same way, a mendicant who develops and cultivates the noble eightfold path disperses and stills bad, unskillful qualities as they proceed. 

And\marginnote{1.3} how does a mendicant who develops the noble eightfold path disperse and still bad, unskillful qualities as they proceed? It’s when a mendicant develops right view, right thought, right speech, right action, right livelihood, right effort, right mindfulness, and right immersion, which rely on seclusion, fading away, and cessation, and ripen as letting go. That’s how a mendicant who develops and cultivates the noble eightfold path disperses and stills bad, unskillful qualities as they proceed.” 

%
\section*{{\suttatitleacronym SN 45.158}{\suttatitletranslation A Ship }{\suttatitleroot Nāvāsutta}}
\addcontentsline{toc}{section}{\tocacronym{SN 45.158} \toctranslation{A Ship } \tocroot{Nāvāsutta}}
\markboth{A Ship }{Nāvāsutta}
\extramarks{SN 45.158}{SN 45.158}

“Mendicants,\marginnote{1.1} suppose there was a sea-faring ship bound together with ropes. For six months they deteriorated in the water. Then in the cold season it was hauled up on dry land, where the ropes were weathered by wind and sun. When the clouds soaked it with rain, the ropes would readily collapse and rot away. In the same way, when a mendicant develops and cultivates the noble eightfold path their fetters readily collapse and rot away. 

And\marginnote{1.3} how do they develop and cultivate the noble eightfold path so that their fetters readily collapse and rot away? It’s when a mendicant develops right view, right thought, right speech, right action, right livelihood, right effort, right mindfulness, and right immersion, which rely on seclusion, fading away, and cessation, and ripen as letting go. That’s how they develop and cultivate the noble eightfold path so that their fetters readily collapse and rot away.” 

%
\section*{{\suttatitleacronym SN 45.159}{\suttatitletranslation A Guest House }{\suttatitleroot Āgantukasutta}}
\addcontentsline{toc}{section}{\tocacronym{SN 45.159} \toctranslation{A Guest House } \tocroot{Āgantukasutta}}
\markboth{A Guest House }{Āgantukasutta}
\extramarks{SN 45.159}{SN 45.159}

“Mendicants,\marginnote{1.1} suppose there was a guest house. Lodgers come from the east, west, north, and south. Aristocrats, brahmins, merchants, and workers all stay there. In the same way, a mendicant who develops and cultivates the noble eightfold path completely understands by direct knowledge the things that should be completely understood by direct knowledge. They give up by direct knowledge the things that should be given up by direct knowledge. They realize by direct knowledge the things that should be realized by direct knowledge. They develop by direct knowledge the things that should be developed by direct knowledge. 

And\marginnote{2.1} what are the things that should be completely understood by direct knowledge? It should be said: the five grasping aggregates. What five? That is: form, feeling, perception, choices, and consciousness. These are the things that should be completely understood by direct knowledge. And what are the things that should be given up by direct knowledge? Ignorance and craving for continued existence. These are the things that should be given up by direct knowledge. And what are the things that should be realized by direct knowledge? Knowledge and freedom. These are the things that should be realized by direct knowledge. And what are the things that should be developed by direct knowledge? Serenity and discernment. These are the things that should be developed by direct knowledge. 

And\marginnote{2.15} how does a mendicant develop the noble eightfold path in this way? It’s when a mendicant develops right view, right thought, right speech, right action, right livelihood, right effort, right mindfulness, and right immersion, which rely on seclusion, fading away, and cessation, and ripen as letting go. That’s how a mendicant develops and cultivates the eightfold path in this way.” 

%
\section*{{\suttatitleacronym SN 45.160}{\suttatitletranslation A River }{\suttatitleroot Nadīsutta}}
\addcontentsline{toc}{section}{\tocacronym{SN 45.160} \toctranslation{A River } \tocroot{Nadīsutta}}
\markboth{A River }{Nadīsutta}
\extramarks{SN 45.160}{SN 45.160}

“Mendicants,\marginnote{1.1} suppose that, although the Ganges river slants, slopes, and inclines to the east, a large crowd were to come along with a spade and basket, saying: ‘We’ll make this Ganges river slant, slope, and incline to the west!’ 

What\marginnote{1.4} do you think, mendicants? Would they succeed?” 

“No,\marginnote{1.6} sir. Why is that? The Ganges river slants, slopes, and inclines to the east. It’s not easy to make it slant, slope, and incline to the west. That large crowd will eventually get weary and frustrated.” 

“In\marginnote{1.11} the same way, while a mendicant develops and cultivates the noble eightfold path, if rulers or their ministers, friends or colleagues, relatives or family should invite them to accept wealth, saying: ‘Please, mister, why let these ocher robes torment you? Why follow the practice of shaving your head and carrying an alms bowl? Come, return to a lesser life, enjoy wealth, and make merit!’ It’s simply impossible for a mendicant who develops and cultivates the noble eightfold path to resign the training and return to a lesser life. Why is that? Because for a long time that mendicant’s mind has slanted, sloped, and inclined to seclusion. So it’s impossible for them to return to a lesser life. 

And\marginnote{1.18} how does a mendicant develop the noble eightfold path? It’s when a mendicant develops right view, right thought, right speech, right action, right livelihood, right effort, right mindfulness, and right immersion, which rely on seclusion, fading away, and cessation, and ripen as letting go. That’s how a mendicant develops and cultivates the noble eightfold path.” 

%
\addtocontents{toc}{\let\protect\contentsline\protect\nopagecontentsline}
\chapter*{The Chapter on Searches }
\addcontentsline{toc}{chapter}{\tocchapterline{The Chapter on Searches }}
\addtocontents{toc}{\let\protect\contentsline\protect\oldcontentsline}

%
\section*{{\suttatitleacronym SN 45.161}{\suttatitletranslation Searches }{\suttatitleroot Esanāsutta}}
\addcontentsline{toc}{section}{\tocacronym{SN 45.161} \toctranslation{Searches } \tocroot{Esanāsutta}}
\markboth{Searches }{Esanāsutta}
\extramarks{SN 45.161}{SN 45.161}

At\marginnote{1.1} \textsanskrit{Sāvatthī}. 

“Mendicants,\marginnote{1.2} there are these three searches. What three? The search for sensual pleasures, the search for continued existence, and the search for a spiritual path. These are the three searches. 

The\marginnote{1.6} noble eightfold path should be developed to directly know these three searches. What is the noble eightfold path? It’s when a mendicant develops right view, right thought, right speech, right action, right livelihood, right effort, right mindfulness, and right immersion, which rely on seclusion, fading away, and cessation, and ripen as letting go. 

This\marginnote{1.9} is the noble eightfold path that should be developed to directly know these three searches.” 

“Mendicants,\marginnote{2.1} there are these three searches. What three? The search for sensual pleasures, the search for continued existence, and the search for a spiritual path. These are the three searches. 

The\marginnote{2.5} noble eightfold path should be developed to directly know these three searches. What is the noble eightfold path? It’s when a mendicant develops right view, right thought, right speech, right action, right livelihood, right effort, right mindfulness, and right immersion, which culminate in the removal of greed, hate, and delusion. 

This\marginnote{2.8} is the noble eightfold path that should be developed to directly know these three searches.” 

“Mendicants,\marginnote{3.1} there are these three searches. What three? The search for sensual pleasures, the search for continued existence, and the search for a spiritual path. These are the three searches. 

The\marginnote{3.5} noble eightfold path should be developed to directly know these three searches. What is the noble eightfold path? It’s when a mendicant develops right view, right thought, right speech, right action, right livelihood, right effort, right mindfulness, and right immersion, which culminate, finish, and end in the deathless. 

This\marginnote{3.8} is the noble eightfold path that should be developed to directly know these three searches.” 

“Mendicants,\marginnote{4.1} there are these three searches. What three? The search for sensual pleasures, the search for continued existence, and the search for a spiritual path. These are the three searches. 

The\marginnote{4.5} noble eightfold path should be developed to directly know these three searches. What is the noble eightfold path? It’s when a mendicant develops right view, right thought, right speech, right action, right livelihood, right effort, right mindfulness, and right immersion, which slants, slopes, and inclines to extinguishment. 

This\marginnote{4.8} is the noble eightfold path that should be developed to directly know these three searches.” 

“Mendicants,\marginnote{5.1} there are these three searches. What three? The search for sensual pleasures, the search for continued existence, and the search for a spiritual path. These are the three searches. 

The\marginnote{5.5} noble eightfold path should be developed to completely understand …” 

(This\marginnote{5.6} should be expanded with “completely understand” instead of “directly know”.) 

“Mendicants,\marginnote{6.1} there are these three searches. What three? The search for sensual pleasures, the search for continued existence, and the search for a spiritual path. These are the three searches. 

The\marginnote{6.5} noble eightfold path should be developed to finish …” 

(This\marginnote{6.6} should be expanded with “finish” instead of “directly know”.) 

“Mendicants,\marginnote{7.1} there are these three searches. What three? The search for sensual pleasures, the search for continued existence, and the search for a spiritual path. These are the three searches. 

The\marginnote{7.5} noble eightfold path should be developed to give up …” 

(This\marginnote{7.9} should be expanded with “give up” instead of “directly know”.) 

%
\section*{{\suttatitleacronym SN 45.162}{\suttatitletranslation Discriminations }{\suttatitleroot Vidhāsutta}}
\addcontentsline{toc}{section}{\tocacronym{SN 45.162} \toctranslation{Discriminations } \tocroot{Vidhāsutta}}
\markboth{Discriminations }{Vidhāsutta}
\extramarks{SN 45.162}{SN 45.162}

“Mendicants,\marginnote{1.1} there are three kinds of discrimination. What three? One discriminates, thinking that ‘I’m better’ or ‘I’m equal’ or ‘I’m worse’. These are the three kinds of discrimination. 

The\marginnote{1.5} noble eightfold path should be developed for the direct knowledge, complete understanding, finishing, and giving up of these three kinds of discrimination. What is the noble eightfold path? It’s when a mendicant develops right view, right thought, right speech, right action, right livelihood, right effort, right mindfulness, and right immersion, which rely on seclusion, fading away, and cessation, and ripen as letting go. This is the noble eightfold path that should be developed for the direct knowledge, complete understanding, finishing, and giving up of these three kinds of discrimination.” 

(This\marginnote{1.9} should be expanded as in the section on searches.) 

%
\section*{{\suttatitleacronym SN 45.163}{\suttatitletranslation Defilements }{\suttatitleroot Āsavasutta}}
\addcontentsline{toc}{section}{\tocacronym{SN 45.163} \toctranslation{Defilements } \tocroot{Āsavasutta}}
\markboth{Defilements }{Āsavasutta}
\extramarks{SN 45.163}{SN 45.163}

“Mendicants,\marginnote{1.1} there are these three defilements. What three? The defilements of sensuality, desire to be reborn, and ignorance. These are the three defilements. 

The\marginnote{1.5} noble eightfold path should be developed for the direct knowledge, complete understanding, finishing, and giving up of these three defilements.” 

%
\section*{{\suttatitleacronym SN 45.164}{\suttatitletranslation States of Existence }{\suttatitleroot Bhavasutta}}
\addcontentsline{toc}{section}{\tocacronym{SN 45.164} \toctranslation{States of Existence } \tocroot{Bhavasutta}}
\markboth{States of Existence }{Bhavasutta}
\extramarks{SN 45.164}{SN 45.164}

“There\marginnote{1.1} are these three states of existence. What three? Existence in the sensual realm, the realm of luminous form, and the formless realm. These are the three states of existence. 

The\marginnote{1.5} noble eightfold path should be developed for the direct knowledge, complete understanding, finishing, and giving up of these three states of existence.” 

%
\section*{{\suttatitleacronym SN 45.165}{\suttatitletranslation Forms of Suffering }{\suttatitleroot Dukkhatāsutta}}
\addcontentsline{toc}{section}{\tocacronym{SN 45.165} \toctranslation{Forms of Suffering } \tocroot{Dukkhatāsutta}}
\markboth{Forms of Suffering }{Dukkhatāsutta}
\extramarks{SN 45.165}{SN 45.165}

“Mendicants,\marginnote{1.1} there are these three forms of suffering. What three? The suffering inherent in painful feeling; the suffering inherent in conditions; and the suffering inherent in perishing. These are the three forms of suffering. 

The\marginnote{1.5} noble eightfold path should be developed for the direct knowledge, complete understanding, finishing, and giving up of these three forms of suffering.” 

%
\section*{{\suttatitleacronym SN 45.166}{\suttatitletranslation Kinds of Barrenness }{\suttatitleroot Khilasutta}}
\addcontentsline{toc}{section}{\tocacronym{SN 45.166} \toctranslation{Kinds of Barrenness } \tocroot{Khilasutta}}
\markboth{Kinds of Barrenness }{Khilasutta}
\extramarks{SN 45.166}{SN 45.166}

“Mendicants,\marginnote{1.1} there are these three kinds of barrenness. What three? Greed, hate, and delusion. These are the three kinds of barrenness. 

The\marginnote{1.5} noble eightfold path should be developed for the direct knowledge, complete understanding, finishing, and giving up of these three kinds of barrenness.” 

%
\section*{{\suttatitleacronym SN 45.167}{\suttatitletranslation Stains }{\suttatitleroot Malasutta}}
\addcontentsline{toc}{section}{\tocacronym{SN 45.167} \toctranslation{Stains } \tocroot{Malasutta}}
\markboth{Stains }{Malasutta}
\extramarks{SN 45.167}{SN 45.167}

“Mendicants,\marginnote{1.1} there are these three stains. What three? Greed, hate, and delusion. These are the three stains. 

The\marginnote{1.5} noble eightfold path should be developed for the direct knowledge, complete understanding, finishing, and giving up of these three stains.” 

%
\section*{{\suttatitleacronym SN 45.168}{\suttatitletranslation Troubles }{\suttatitleroot Nīghasutta}}
\addcontentsline{toc}{section}{\tocacronym{SN 45.168} \toctranslation{Troubles } \tocroot{Nīghasutta}}
\markboth{Troubles }{Nīghasutta}
\extramarks{SN 45.168}{SN 45.168}

“Mendicants,\marginnote{1.1} there are these three troubles. What three? Greed, hate, and delusion. These are the three troubles. 

The\marginnote{1.5} noble eightfold path should be developed for the direct knowledge, complete understanding, finishing, and giving up of these three troubles.” 

%
\section*{{\suttatitleacronym SN 45.169}{\suttatitletranslation Feelings }{\suttatitleroot Vedanāsutta}}
\addcontentsline{toc}{section}{\tocacronym{SN 45.169} \toctranslation{Feelings } \tocroot{Vedanāsutta}}
\markboth{Feelings }{Vedanāsutta}
\extramarks{SN 45.169}{SN 45.169}

“Mendicants,\marginnote{1.1} there are these three feelings: What three? Pleasant, painful, and neutral feeling. These are the three feelings. 

The\marginnote{1.5} noble eightfold path should be developed for the direct knowledge, complete understanding, finishing, and giving up of these three feelings.” 

%
\section*{{\suttatitleacronym SN 45.170}{\suttatitletranslation Craving }{\suttatitleroot Taṇhāsutta}}
\addcontentsline{toc}{section}{\tocacronym{SN 45.170} \toctranslation{Craving } \tocroot{Taṇhāsutta}}
\markboth{Craving }{Taṇhāsutta}
\extramarks{SN 45.170}{SN 45.170}

“Mendicants,\marginnote{1.1} there are these three cravings. What three? Craving for sensual pleasures, craving to continue existence, and craving to end existence. These are the three cravings. 

The\marginnote{1.5} noble eightfold path should be developed for the direct knowledge, complete understanding, finishing, and giving up of these three cravings. What is the noble eightfold path? It’s when a mendicant develops right view, right thought, right speech, right action, right livelihood, right effort, right mindfulness, and right immersion, which rely on seclusion, fading away, and cessation, and ripen as letting go. 

This\marginnote{1.8} is the noble eightfold path that should be developed for the direct knowledge, complete understanding, finishing, and giving up of these three cravings.” 

\subsection*{Thirst }

“Mendicants,\marginnote{2.1} there are these three thirsts. What three? Thirst for sensual pleasures, thirst to continue existence, and thirst to end existence. 

For\marginnote{2.4} the direct knowledge, complete understanding, finishing, and giving up of these three thirsts … … which culminates in the removal of greed, hate, and delusion. … which culminates, finishes, and ends in the deathless. … which slants, slopes, and inclines to extinguishment. 

The\marginnote{2.8} noble eightfold path should be developed for the direct knowledge, complete understanding, finishing, and giving up of these three thirsts.” 

%
\addtocontents{toc}{\let\protect\contentsline\protect\nopagecontentsline}
\chapter*{The Chapter on Floods }
\addcontentsline{toc}{chapter}{\tocchapterline{The Chapter on Floods }}
\addtocontents{toc}{\let\protect\contentsline\protect\oldcontentsline}

%
\section*{{\suttatitleacronym SN 45.171}{\suttatitletranslation Floods }{\suttatitleroot Oghasutta}}
\addcontentsline{toc}{section}{\tocacronym{SN 45.171} \toctranslation{Floods } \tocroot{Oghasutta}}
\markboth{Floods }{Oghasutta}
\extramarks{SN 45.171}{SN 45.171}

At\marginnote{1.1} \textsanskrit{Sāvatthī}. 

“Mendicants,\marginnote{1.2} there are these four floods. What four? The floods of sensuality, desire to be reborn, views, and ignorance. These are the four floods. 

The\marginnote{1.6} noble eightfold path should be developed for the direct knowledge, complete understanding, finishing, and giving up of these four floods.” 

(This\marginnote{1.7} should be expanded as in the section on searches.) 

%
\section*{{\suttatitleacronym SN 45.172}{\suttatitletranslation Attachments }{\suttatitleroot Yogasutta}}
\addcontentsline{toc}{section}{\tocacronym{SN 45.172} \toctranslation{Attachments } \tocroot{Yogasutta}}
\markboth{Attachments }{Yogasutta}
\extramarks{SN 45.172}{SN 45.172}

“Mendicants,\marginnote{1.1} there are these four attachments. What four? The attachment to sensual pleasures, future lives, views, and ignorance. These are the four attachments. 

The\marginnote{1.5} noble eightfold path should be developed for the direct knowledge, complete understanding, finishing, and giving up of these four attachments.” 

%
\section*{{\suttatitleacronym SN 45.173}{\suttatitletranslation Grasping }{\suttatitleroot Upādānasutta}}
\addcontentsline{toc}{section}{\tocacronym{SN 45.173} \toctranslation{Grasping } \tocroot{Upādānasutta}}
\markboth{Grasping }{Upādānasutta}
\extramarks{SN 45.173}{SN 45.173}

“Mendicants,\marginnote{1.1} there are these four kinds of grasping. What four? Grasping at sensual pleasures, views, precepts and observances, and theories of a self. These are the four kinds of grasping. 

The\marginnote{1.5} noble eightfold path should be developed for the direct knowledge, complete understanding, finishing, and giving up of these four kinds of grasping.” 

%
\section*{{\suttatitleacronym SN 45.174}{\suttatitletranslation Personal Ties }{\suttatitleroot Ganthasutta}}
\addcontentsline{toc}{section}{\tocacronym{SN 45.174} \toctranslation{Personal Ties } \tocroot{Ganthasutta}}
\markboth{Personal Ties }{Ganthasutta}
\extramarks{SN 45.174}{SN 45.174}

“Mendicants,\marginnote{1.1} there are these four ties. What four? The personal ties to covetousness, ill will, misapprehension of precepts and observances, and the insistence that this is the only truth. These are the four ties. 

The\marginnote{1.5} noble eightfold path should be developed for the direct knowledge, complete understanding, finishing, and giving up of these four ties.” 

%
\section*{{\suttatitleacronym SN 45.175}{\suttatitletranslation Tendencies }{\suttatitleroot Anusayasutta}}
\addcontentsline{toc}{section}{\tocacronym{SN 45.175} \toctranslation{Tendencies } \tocroot{Anusayasutta}}
\markboth{Tendencies }{Anusayasutta}
\extramarks{SN 45.175}{SN 45.175}

“Mendicants,\marginnote{1.1} there are these seven underlying tendencies. What seven? The underlying tendencies of sensual desire, repulsion, views, doubt, conceit, desire to be reborn, and ignorance. These are the seven underlying tendencies. 

The\marginnote{1.5} noble eightfold path should be developed for the direct knowledge, complete understanding, finishing, and giving up of these seven underlying tendencies.” 

%
\section*{{\suttatitleacronym SN 45.176}{\suttatitletranslation Kinds of Sensual Stimulation }{\suttatitleroot Kāmaguṇasutta}}
\addcontentsline{toc}{section}{\tocacronym{SN 45.176} \toctranslation{Kinds of Sensual Stimulation } \tocroot{Kāmaguṇasutta}}
\markboth{Kinds of Sensual Stimulation }{Kāmaguṇasutta}
\extramarks{SN 45.176}{SN 45.176}

“Mendicants,\marginnote{1.1} there are these five kinds of sensual stimulation. What five? Sights known by the eye that are likable, desirable, agreeable, pleasant, sensual, and arousing. Sounds known by the ear … Smells known by the nose … Tastes known by the tongue … Touches known by the body that are likable, desirable, agreeable, pleasant, sensual, and arousing. These are the five kinds of sensual stimulation. 

The\marginnote{1.8} noble eightfold path should be developed for the direct knowledge, complete understanding, finishing, and giving up of these five kinds of sensual stimulation.” 

%
\section*{{\suttatitleacronym SN 45.177}{\suttatitletranslation Hindrances }{\suttatitleroot Nīvaraṇasutta}}
\addcontentsline{toc}{section}{\tocacronym{SN 45.177} \toctranslation{Hindrances } \tocroot{Nīvaraṇasutta}}
\markboth{Hindrances }{Nīvaraṇasutta}
\extramarks{SN 45.177}{SN 45.177}

“Mendicants,\marginnote{1.1} there are these five hindrances. What five? The hindrances of sensual desire, ill will, dullness and drowsiness, restlessness and remorse, and doubt. These are the five hindrances. 

The\marginnote{1.5} noble eightfold path should be developed for the direct knowledge, complete understanding, finishing, and giving up of these five hindrances.” 

%
\section*{{\suttatitleacronym SN 45.178}{\suttatitletranslation Grasping Aggregates }{\suttatitleroot Upādānakkhandhasutta}}
\addcontentsline{toc}{section}{\tocacronym{SN 45.178} \toctranslation{Grasping Aggregates } \tocroot{Upādānakkhandhasutta}}
\markboth{Grasping Aggregates }{Upādānakkhandhasutta}
\extramarks{SN 45.178}{SN 45.178}

“Mendicants,\marginnote{1.1} there are these five grasping aggregates. What five? The grasping aggregates of form, feeling, perception, choices, and consciousness. These are the five grasping aggregates. 

The\marginnote{1.5} noble eightfold path should be developed for the direct knowledge, complete understanding, finishing, and giving up of these five grasping aggregates.” 

%
\section*{{\suttatitleacronym SN 45.179}{\suttatitletranslation Lower Fetters }{\suttatitleroot Orambhāgiyasutta}}
\addcontentsline{toc}{section}{\tocacronym{SN 45.179} \toctranslation{Lower Fetters } \tocroot{Orambhāgiyasutta}}
\markboth{Lower Fetters }{Orambhāgiyasutta}
\extramarks{SN 45.179}{SN 45.179}

“Mendicants,\marginnote{1.1} there are five lower fetters. What five? Identity view, doubt, misapprehension of precepts and observances, sensual desire, and ill will. These are the five lower fetters. 

The\marginnote{1.5} noble eightfold path should be developed for the direct knowledge, complete understanding, finishing, and giving up of these five lowers fetters.” 

%
\section*{{\suttatitleacronym SN 45.180}{\suttatitletranslation Higher Fetters }{\suttatitleroot Uddhambhāgiyasutta}}
\addcontentsline{toc}{section}{\tocacronym{SN 45.180} \toctranslation{Higher Fetters } \tocroot{Uddhambhāgiyasutta}}
\markboth{Higher Fetters }{Uddhambhāgiyasutta}
\extramarks{SN 45.180}{SN 45.180}

“Mendicants,\marginnote{1.1} there are five higher fetters. What five? Desire for rebirth in the realm of luminous form, desire for rebirth in the formless realm, conceit, restlessness, and ignorance. These are the five higher fetters. 

The\marginnote{1.5} noble eightfold path should be developed for the direct knowledge, complete understanding, finishing, and giving up of these five higher fetters. What is the noble eightfold path? It’s when a mendicant develops right view, right thought, right speech, right action, right livelihood, right effort, right mindfulness, and right immersion, which rely on seclusion, fading away, and cessation, and ripen as letting go. 

This\marginnote{1.8} is the noble eightfold path that should be developed for the direct knowledge, complete understanding, finishing, and giving up of these five higher fetters.” 

“Mendicants,\marginnote{2.1} there are five higher fetters. What five? Desire for rebirth in the realm of luminous form, desire for rebirth in the formless realm, conceit, restlessness, and ignorance. These are the five higher fetters. 

The\marginnote{2.5} noble eightfold path should be developed for the direct knowledge, complete understanding, finishing, and giving up of these five higher fetters. What is the noble eightfold path? It’s when a mendicant develops right view, right thought, right speech, right action, right livelihood, right effort, right mindfulness, and right immersion, which culminate in the removal of greed, hate, and delusion …” “… which culminate, finish, and end in the deathless …” “… which have extinguishment as their culmination, destination, and end. 

This\marginnote{2.10} is the noble eightfold path that should be developed for the direct knowledge, complete understanding, finishing, and giving up of these five higher fetters.” 

\scendsutta{The Linked Discourses on the Path is the first section. }

%
\addtocontents{toc}{\let\protect\contentsline\protect\nopagecontentsline}
\part*{Linked Discourses on the Awakening Factors }
\addcontentsline{toc}{part}{Linked Discourses on the Awakening Factors }
\markboth{}{}
\addtocontents{toc}{\let\protect\contentsline\protect\oldcontentsline}

%
\addtocontents{toc}{\let\protect\contentsline\protect\nopagecontentsline}
\chapter*{The Chapter on Mountains }
\addcontentsline{toc}{chapter}{\tocchapterline{The Chapter on Mountains }}
\addtocontents{toc}{\let\protect\contentsline\protect\oldcontentsline}

%
\section*{{\suttatitleacronym SN 46.1}{\suttatitletranslation The Himalaya }{\suttatitleroot Himavantasutta}}
\addcontentsline{toc}{section}{\tocacronym{SN 46.1} \toctranslation{The Himalaya } \tocroot{Himavantasutta}}
\markboth{The Himalaya }{Himavantasutta}
\extramarks{SN 46.1}{SN 46.1}

At\marginnote{1.1} \textsanskrit{Sāvatthī}. 

“Mendicants,\marginnote{1.2} dragons grow and wax strong supported by the Himalayas, the king of mountains. When they’re strong they dive into the pools. Then they dive into the lakes, the streams, the rivers, and finally the ocean. There they acquire a great and abundant body. 

In\marginnote{1.4} the same way, a mendicant develops and cultivates the seven awakening factors depending on and grounded on ethics, acquiring great and abundant good qualities. And how does a mendicant develop the seven awakening factors depending on and grounded on ethics, acquiring great and abundant good qualities? 

It’s\marginnote{1.6} when a mendicant develops the awakening factor of mindfulness, which relies on seclusion, fading away, and cessation, and ripens as letting go. 

They\marginnote{1.7} develop the awakening factor of investigation of principles … 

They\marginnote{1.8} develop the awakening factor of energy … 

They\marginnote{1.9} develop the awakening factor of rapture … 

They\marginnote{1.10} develop the awakening factor of tranquility … 

They\marginnote{1.11} develop the awakening factor of immersion … 

They\marginnote{1.12} develop the awakening factor of equanimity, which relies on seclusion, fading away, and cessation, and ripens as letting go. 

That’s\marginnote{1.13} how a mendicant develops and cultivates the seven awakening factors depending on and grounded on ethics, acquiring great and abundant good qualities.” 

%
\section*{{\suttatitleacronym SN 46.2}{\suttatitletranslation The Body }{\suttatitleroot Kāyasutta}}
\addcontentsline{toc}{section}{\tocacronym{SN 46.2} \toctranslation{The Body } \tocroot{Kāyasutta}}
\markboth{The Body }{Kāyasutta}
\extramarks{SN 46.2}{SN 46.2}

At\marginnote{1.1} \textsanskrit{Sāvatthī}. 

“Mendicants,\marginnote{1.2} this body is sustained by food. It depends on food to continue, and without food it doesn’t continue. In the same way, the five hindrances are sustained by fuel. They depend on fuel to continue, and without fuel they don’t continue. 

And\marginnote{2.1} what fuels the arising of sensual desire, or, when it has arisen, makes it increase and grow? There is the feature of beauty. Frequent improper attention to that fuels the arising of sensual desire, or, when it has arisen, makes it increase and grow. 

And\marginnote{3.1} what fuels the arising of ill will, or, when it has arisen, makes it increase and grow? There is the feature of harshness. Frequent improper attention to that fuels the arising of ill will, or, when it has arisen, makes it increase and grow. 

And\marginnote{4.1} what fuels the arising of dullness and drowsiness, or, when they have arisen, makes them increase and grow? There is discontent, sloth, yawning, sleepiness after eating, and mental sluggishness. Frequent improper attention to them fuels the arising of dullness and drowsiness, or, when they have arisen, makes them increase and grow. 

And\marginnote{5.1} what fuels the arising of restlessness and remorse, or, when they have arisen, makes them increase and grow? There is the unsettled mind. Frequent improper attention to that fuels the arising of restlessness and remorse, or, when they have arisen, makes them increase and grow. 

And\marginnote{6.1} what fuels the arising of doubt, or, when it has arisen, makes it increase and grow? There are things that are grounds for doubt. Frequent improper attention to them fuels the arising of doubt, or, when it has arisen, makes it increase and grow. 

This\marginnote{7.1} body is sustained by food. It depends on food to continue, and without food it doesn’t continue. In the same way, the five hindrances are sustained by fuel. They depend on fuel to continue, and without fuel they don’t continue. 

This\marginnote{8.1} body is sustained by food. It depends on food to continue, and without food it doesn’t continue. In the same way, the seven awakening factors are sustained by fuel. They depend on fuel to continue, and without fuel they don’t continue. 

And\marginnote{9.1} what fuels the arising of the awakening factor of mindfulness, or, when it has arisen, fully develops it? There are things that are grounds for the awakening factor of mindfulness. Frequent proper attention to them fuels the arising of the awakening factor of mindfulness, or, when it has arisen, fully develops it. 

And\marginnote{10.1} what fuels the arising of the awakening factor of investigation of principles, or, when it has arisen, fully develops it? There are qualities that are skillful and unskillful, blameworthy and blameless, inferior and superior, and those on the side of dark and bright. Frequent proper attention to them fuels the arising of the awakening factor of investigation of principles, or, when it has arisen, fully develops it. 

And\marginnote{11.1} what fuels the arising of the awakening factor of energy, or, when it has arisen, fully develops it? There are the elements of initiative, persistence, and exertion. Frequent proper attention to them fuels the arising of the awakening factor of energy, or, when it has arisen, fully develops it. 

And\marginnote{12.1} what fuels the arising of the awakening factor of rapture, or, when it has arisen, fully develops it? There are things that are grounds for the awakening factor of rapture. Frequent proper attention to them fuels the arising of the awakening factor of rapture, or, when it has arisen, fully develops it. 

And\marginnote{13.1} what fuels the arising of the awakening factor of tranquility, or, when it has arisen, fully develops it? There is tranquility of the body and of the mind. Frequent proper attention to that fuels the arising of the awakening factor of tranquility, or, when it has arisen, fully develops it. 

And\marginnote{14.1} what fuels the arising of the awakening factor of immersion, or, when it has arisen, fully develops it? There are things that are the foundation of serenity and freedom from distraction. Frequent proper attention to them fuels the arising of the awakening factor of immersion, or, when it has arisen, fully develops it. 

And\marginnote{15.1} what fuels the arising of the awakening factor of equanimity, or, when it has arisen, fully develops it? There are things that are grounds for the awakening factor of equanimity. Frequent proper attention to them fuels the arising of the awakening factor of equanimity, or, when it has arisen, fully develops it. 

This\marginnote{16.1} body is sustained by food. It depends on food to continue, and without food it doesn’t continue. In the same way, the seven awakening factors are sustained by fuel. They depend on fuel to continue, and without fuel they don’t continue.” 

%
\section*{{\suttatitleacronym SN 46.3}{\suttatitletranslation Ethics }{\suttatitleroot Sīlasutta}}
\addcontentsline{toc}{section}{\tocacronym{SN 46.3} \toctranslation{Ethics } \tocroot{Sīlasutta}}
\markboth{Ethics }{Sīlasutta}
\extramarks{SN 46.3}{SN 46.3}

“Mendicants,\marginnote{1.1} when a mendicant is accomplished in ethics, immersion, knowledge, freedom, or the knowledge and vision of freedom, even the sight of them is very helpful, I say. Even to hear them, approach them, pay homage to them, recollect them, or go forth following them is very helpful, I say. Why is that? Because after hearing the teaching of such mendicants, a mendicant will live withdrawn in both body and mind, as they recollect and think about that teaching. 

At\marginnote{2.1} such a time, a mendicant has activated the awakening factor of mindfulness; they develop it and perfect it. As they live mindfully in this way they investigate, explore, and inquire into that teaching with wisdom. 

At\marginnote{3.1} such a time, a mendicant has activated the awakening factor of investigation of principles; they develop it and perfect it. As they investigate principles with wisdom in this way their energy is roused up and unflagging. 

At\marginnote{4.1} such a time, a mendicant has activated the awakening factor of energy; they develop it and perfect it. When they’re energetic, spiritual rapture arises. 

At\marginnote{5.1} such a time, a mendicant has activated the awakening factor of rapture; they develop it and perfect it. When the mind is full of rapture, the body and mind become tranquil. 

At\marginnote{6.1} such a time, a mendicant has activated the awakening factor of tranquility; they develop it and perfect it. When the body is tranquil and one feels bliss, the mind becomes immersed in \textsanskrit{samādhi}. 

At\marginnote{7.1} such a time, a mendicant has activated the awakening factor of immersion; they develop it and perfect it. They closely watch over that mind immersed in \textsanskrit{samādhi}. 

At\marginnote{8.1} such a time, a mendicant has activated the awakening factor of equanimity; they develop it and perfect it. 

When\marginnote{9.1} the seven awakening factors are developed and cultivated in this way they can expect seven fruits and benefits. What seven? They attain enlightenment early on in this very life. If not, they attain enlightenment at the time of death. If not, with the ending of the five lower fetters, they’re extinguished between one life and the next. If not, with the ending of the five lower fetters they’re extinguished upon landing. If not, with the ending of the five lower fetters they’re extinguished without extra effort. If not, with the ending of the five lower fetters they’re extinguished with extra effort. If not, with the ending of the five lower fetters they head upstream, going to the \textsanskrit{Akaniṭṭha} realm. When the seven awakening factors are developed and cultivated in this way these are the seven fruits and benefits they can expect.” 

%
\section*{{\suttatitleacronym SN 46.4}{\suttatitletranslation Clothes }{\suttatitleroot Vatthasutta}}
\addcontentsline{toc}{section}{\tocacronym{SN 46.4} \toctranslation{Clothes } \tocroot{Vatthasutta}}
\markboth{Clothes }{Vatthasutta}
\extramarks{SN 46.4}{SN 46.4}

At\marginnote{1.1} one time Venerable \textsanskrit{Sāriputta} was staying near \textsanskrit{Sāvatthī} in Jeta’s Grove, \textsanskrit{Anāthapiṇḍika}’s monastery. There \textsanskrit{Sāriputta} addressed the mendicants: “Reverends, mendicants!” 

“Reverend,”\marginnote{1.4} they replied. \textsanskrit{Sāriputta} said this: 

“There\marginnote{2.1} are these seven awakening factors. What seven? The awakening factors of mindfulness, investigation of principles, energy, rapture, tranquility, immersion, and equanimity. These are the seven awakening factors. 

In\marginnote{2.5} the morning, I meditate on whichever of these seven awakening factors I want. At midday, and in the evening, I meditate on whichever of these seven awakening factors I want. If it’s the awakening factor of mindfulness, I know that it’s limitless and that it’s properly implemented. While it remains I understand that it remains. And if it subsides in me I understand the specific reason it subsides. … If it’s the awakening factor of equanimity, I know that it’s limitless and that it’s properly implemented. While it remains I understand that it remains. And if it subsides I understand the specific reason it subsides. 

Suppose\marginnote{3.1} that a ruler or their minister had a chest full of garments of different colors. In the morning, they’d don whatever pair of garments they wanted. At midday, and in the evening, they’d don whatever pair of garments they wanted. 

In\marginnote{3.5} the same way, in the morning, at midday, and in the evening, I meditate on whichever of these seven awakening factors I want. If it’s the awakening factor of mindfulness, I know that it’s limitless and that it’s properly implemented. While it remains I understand that it remains. And if it subsides I understand the specific reason it subsides. … If it’s the awakening factor of equanimity, I know that it’s limitless and that it’s properly implemented. While it remains I understand that it remains. And if it subsides I understand the specific reason it subsides.” 

%
\section*{{\suttatitleacronym SN 46.5}{\suttatitletranslation A Monk }{\suttatitleroot Bhikkhusutta}}
\addcontentsline{toc}{section}{\tocacronym{SN 46.5} \toctranslation{A Monk } \tocroot{Bhikkhusutta}}
\markboth{A Monk }{Bhikkhusutta}
\extramarks{SN 46.5}{SN 46.5}

At\marginnote{1.1} \textsanskrit{Sāvatthī}. 

Then\marginnote{1.2} a mendicant went up to the Buddha … and asked him, “Sir, they speak of the ‘awakening factors’. How are the awakening factors defined?” 

“Mendicant,\marginnote{1.5} they’re called awakening factors because they lead to awakening. 

A\marginnote{1.6} mendicant develops the awakening factors of mindfulness, investigation of principles, energy, rapture, tranquility, immersion, and equanimity, which rely on seclusion, fading away, and cessation, and ripen as letting go. As they develop the seven awakening factors, their mind is freed from the defilements of sensuality, desire to be reborn, and ignorance. When they’re freed, they know they’re freed. They understand: ‘Rebirth is ended, the spiritual journey has been completed, what had to be done has been done, there is no return to any state of existence.’ 

They’re\marginnote{1.11} called awakening factors because they lead to awakening.” 

%
\section*{{\suttatitleacronym SN 46.6}{\suttatitletranslation Kuṇḍaliya }{\suttatitleroot Kuṇḍaliyasutta}}
\addcontentsline{toc}{section}{\tocacronym{SN 46.6} \toctranslation{Kuṇḍaliya } \tocroot{Kuṇḍaliyasutta}}
\markboth{Kuṇḍaliya }{Kuṇḍaliyasutta}
\extramarks{SN 46.6}{SN 46.6}

At\marginnote{1.1} one time the Buddha was staying near \textsanskrit{Sāketa} in the deer part at the \textsanskrit{Añjana} Wood. Then the wanderer \textsanskrit{Kuṇḍaliya} went up to the Buddha, and exchanged greetings with him. When the greetings and polite conversation were over, he sat down to one side and said to the Buddha: 

“Master\marginnote{1.4} Gotama, I like to hang around the monasteries and visit the assemblies. When I’ve finished breakfast, it’s my habit to wander from monastery to monastery, from park to park. There I see some ascetics and brahmins speaking for the sake of winning debates and finding fault. But what benefit does Master Gotama live for?” 

“The\marginnote{1.9} benefit the Realized One lives for, \textsanskrit{Kuṇḍaliya}, is the fruit of knowledge and freedom.” 

“But\marginnote{2.1} what things must be developed and cultivated in order to fulfill knowledge and freedom?” 

“The\marginnote{2.2} seven awakening factors.” 

“But\marginnote{2.3} what things must be developed and cultivated in order to fulfill the seven awakening factors?” 

“The\marginnote{2.4} four kinds of mindfulness meditation.” 

“But\marginnote{2.5} what things must be developed and cultivated in order to fulfill the four kinds of mindfulness meditation?” 

“The\marginnote{2.6} three kinds of good conduct.” 

“But\marginnote{2.7} what things must be developed and cultivated in order to fulfill the three kinds of good conduct?” 

“Sense\marginnote{2.8} restraint. 

And\marginnote{3.1} \textsanskrit{Kuṇḍaliya}, how is sense restraint developed and cultivated so as to fulfill the three kinds of good conduct? A mendicant sees an agreeable sight with their eye. They don’t desire it or enjoy it, and they don’t give rise to greed. Their mind and body are steady internally, well settled and well freed. But if they see a disagreeable sight they’re not dismayed; their mind isn’t hardened, dejected, or full of ill will. Their mind and body are steady internally, well settled and well freed. 

Furthermore,\marginnote{4.1} a mendicant hears an agreeable sound with the ear … smells an agreeable odor with the nose … tastes an agreeable flavor with the tongue … feels an agreeable touch with the body … knows an agreeable thought with their mind. They don’t desire it or enjoy it, and they don’t give rise to greed. Their mind and body are steady internally, well settled and well freed. But if they know a disagreeable thought they’re not dismayed; their mind isn’t hardened, dejected, or full of ill will. Their mind and body are steady internally, well settled and well freed. 

When\marginnote{5.1} a mendicant’s mind and body are steady internally, they’re well settled and well freed when it comes to both agreeable and disagreeable sights, sounds, smells, tastes, touches, and thoughts. That’s how sense restraint is developed and cultivated so as to fulfill the three kinds of good conduct. 

And\marginnote{6.1} how are the three kinds of good conduct developed and cultivated so as to fulfill the four kinds of mindfulness meditation? A mendicant gives up bad conduct by way of body, speech, and mind, and develops good conduct by way of body, speech, and mind. That’s how the three kinds of good conduct are developed and cultivated so as to fulfill the four kinds of mindfulness meditation. 

And\marginnote{7.1} how are the four kinds of mindfulness meditation developed and cultivated so as to fulfill the seven awakening factors? A mendicant meditates by observing an aspect of the body—keen, aware, and mindful, rid of desire and aversion for the world. They meditate observing an aspect of feelings … mind … principles—keen, aware, and mindful, rid of desire and aversion for the world. That’s how the four kinds of mindfulness meditation are developed and cultivated so as to fulfill the seven awakening factors. 

And\marginnote{8.1} how are the seven awakening factors developed and cultivated so as to fulfill knowledge and freedom? A mendicant develops the awakening factors of mindfulness, investigation of principles, energy, rapture, tranquility, immersion, and equanimity, which rely on seclusion, fading away, and cessation, and ripen as letting go. That’s how the seven awakening factors are developed and cultivated so as to fulfill knowledge and freedom.” 

When\marginnote{9.1} he said this, the wanderer \textsanskrit{Kuṇḍaliya} said to the Buddha, “Excellent, Master Gotama! Excellent! As if he were righting the overturned, or revealing the hidden, or pointing out the path to the lost, or lighting a lamp in the dark so people with good eyes can see what’s there, Master Gotama has made the teaching clear in many ways. I go for refuge to Master Gotama, to the teaching, and to the mendicant \textsanskrit{Saṅgha}. From this day forth, may Master Gotama remember me as a lay follower who has gone for refuge for life.” 

%
\section*{{\suttatitleacronym SN 46.7}{\suttatitletranslation A Bungalow }{\suttatitleroot Kūṭāgārasutta}}
\addcontentsline{toc}{section}{\tocacronym{SN 46.7} \toctranslation{A Bungalow } \tocroot{Kūṭāgārasutta}}
\markboth{A Bungalow }{Kūṭāgārasutta}
\extramarks{SN 46.7}{SN 46.7}

“Mendicants,\marginnote{1.1} the rafters of a bungalow all slant, slope, and incline to the peak. In the same way, a mendicant who develops and cultivates the seven awakening factors slants, slopes, and inclines to extinguishment. 

And\marginnote{2.1} how does a mendicant who develops the seven awakening factors slant, slope, and incline to extinguishment? It’s when a mendicant develops the awakening factors of mindfulness, investigation of principles, energy, rapture, tranquility, immersion, and equanimity, which rely on seclusion, fading away, and cessation, and ripen as letting go. That’s how a mendicant who develops and cultivates the seven awakening factors slants, slopes, and inclines to extinguishment.” 

%
\section*{{\suttatitleacronym SN 46.8}{\suttatitletranslation With Upavāna }{\suttatitleroot Upavānasutta}}
\addcontentsline{toc}{section}{\tocacronym{SN 46.8} \toctranslation{With Upavāna } \tocroot{Upavānasutta}}
\markboth{With Upavāna }{Upavānasutta}
\extramarks{SN 46.8}{SN 46.8}

At\marginnote{1.1} one time the venerables \textsanskrit{Upavāna} and \textsanskrit{Sāriputta} were staying near Kosambi, in Ghosita’s Monastery. Then in the late afternoon, Venerable \textsanskrit{Sāriputta} came out of retreat, went to Venerable \textsanskrit{Upavāna} and exchanged greetings with him. When the greetings and polite conversation were over, he sat down to one side and said to \textsanskrit{Upavāna}: 

“Reverend\marginnote{2.1} \textsanskrit{Upavāna}, can a mendicant know by investigating inside themselves that the seven awakening factors are well implemented so that they lead to living at ease?” 

“They\marginnote{2.2} can, Reverend \textsanskrit{Sāriputta}. 

As\marginnote{3.1} a mendicant rouses up the awakening factor of mindfulness, they understand: ‘My mind is well freed. I’ve eradicated dullness and drowsiness, and eliminated restlessness and remorse. My energy is roused up, and my mind is sharply focused, not sluggish.’ … 

As\marginnote{3.2} they rouse up the awakening factor of equanimity, they understand: ‘My mind is well freed. I’ve eradicated dullness and drowsiness, and eliminated restlessness and remorse. My energy is roused up, and my mind is sharply focused, not sluggish.’ 

That’s\marginnote{3.3} how a mendicant can know by investigating inside themselves that the seven awakening factors are well implemented so that they lead to living at ease.” 

%
\section*{{\suttatitleacronym SN 46.9}{\suttatitletranslation Arisen (1st) }{\suttatitleroot Paṭhamauppannasutta}}
\addcontentsline{toc}{section}{\tocacronym{SN 46.9} \toctranslation{Arisen (1st) } \tocroot{Paṭhamauppannasutta}}
\markboth{Arisen (1st) }{Paṭhamauppannasutta}
\extramarks{SN 46.9}{SN 46.9}

“Mendicants,\marginnote{1.1} these seven awakening factors don’t arise to be developed and cultivated except when a Realized One, a perfected one, a fully awakened Buddha has appeared. What seven? The awakening factors of mindfulness, investigation of principles, energy, rapture, tranquility, immersion, and equanimity. These seven awakening factors don’t arise to be developed and cultivated except when a Realized One, a perfected one, a fully awakened Buddha has appeared.” 

%
\section*{{\suttatitleacronym SN 46.10}{\suttatitletranslation Arisen (2nd) }{\suttatitleroot Dutiyauppannasutta}}
\addcontentsline{toc}{section}{\tocacronym{SN 46.10} \toctranslation{Arisen (2nd) } \tocroot{Dutiyauppannasutta}}
\markboth{Arisen (2nd) }{Dutiyauppannasutta}
\extramarks{SN 46.10}{SN 46.10}

“Mendicants,\marginnote{1.1} these seven awakening factors don’t arise to be developed and cultivated apart from the Holy One’s training. What seven? The awakening factors of mindfulness, investigation of principles, energy, rapture, tranquility, immersion, and equanimity. These seven awakening factors don’t arise to be developed and cultivated apart from the Holy One’s training.” 

%
\addtocontents{toc}{\let\protect\contentsline\protect\nopagecontentsline}
\chapter*{The Chapter on Sick }
\addcontentsline{toc}{chapter}{\tocchapterline{The Chapter on Sick }}
\addtocontents{toc}{\let\protect\contentsline\protect\oldcontentsline}

%
\section*{{\suttatitleacronym SN 46.11}{\suttatitletranslation Living Creatures }{\suttatitleroot Pāṇasutta}}
\addcontentsline{toc}{section}{\tocacronym{SN 46.11} \toctranslation{Living Creatures } \tocroot{Pāṇasutta}}
\markboth{Living Creatures }{Pāṇasutta}
\extramarks{SN 46.11}{SN 46.11}

“Mendicants,\marginnote{1.1} living creatures engage in the four postures: sometimes walking, sometimes standing, sometimes sitting, sometimes lying down. They do so depending on the earth and grounded on the earth. In the same way, a mendicant develops and cultivates the seven awakening factors depending on and grounded on ethics. 

And\marginnote{2.1} how does a mendicant develop the seven awakening factors depending on and grounded on ethics? It’s when a mendicant develops the awakening factors of mindfulness, investigation of principles, energy, rapture, tranquility, immersion, and equanimity, which rely on seclusion, fading away, and cessation, and ripen as letting go. That’s how a mendicant develops and cultivates the seven awakening factors depending on and grounded on ethics.” 

%
\section*{{\suttatitleacronym SN 46.12}{\suttatitletranslation The Simile of the Sun (1st) }{\suttatitleroot Paṭhamasūriyūpamasutta}}
\addcontentsline{toc}{section}{\tocacronym{SN 46.12} \toctranslation{The Simile of the Sun (1st) } \tocroot{Paṭhamasūriyūpamasutta}}
\markboth{The Simile of the Sun (1st) }{Paṭhamasūriyūpamasutta}
\extramarks{SN 46.12}{SN 46.12}

“Mendicants,\marginnote{1.1} the dawn is the forerunner and precursor of the sunrise. In the same way, for a mendicant good friendship is the forerunner and precursor of the arising of the seven awakening factors. A mendicant with good friends can expect to develop and cultivate the seven awakening factors. 

And\marginnote{2.1} how does a mendicant with good friends develop and cultivate the seven awakening factors? It’s when a mendicant develops the awakening factors of mindfulness, investigation of principles, energy, rapture, tranquility, immersion, and equanimity, which rely on seclusion, fading away, and cessation, and ripen as letting go. That’s how a mendicant with good friends develops and cultivates the seven awakening factors.” 

%
\section*{{\suttatitleacronym SN 46.13}{\suttatitletranslation The Simile of the Sun (2nd) }{\suttatitleroot Dutiyasūriyūpamasutta}}
\addcontentsline{toc}{section}{\tocacronym{SN 46.13} \toctranslation{The Simile of the Sun (2nd) } \tocroot{Dutiyasūriyūpamasutta}}
\markboth{The Simile of the Sun (2nd) }{Dutiyasūriyūpamasutta}
\extramarks{SN 46.13}{SN 46.13}

“Mendicants,\marginnote{1.1} the dawn is the forerunner and precursor of the sunrise. In the same way, for a mendicant proper attention is the forerunner and precursor of the arising of the seven awakening factors. A mendicant accomplished in proper attention can expect to develop and cultivate the seven awakening factors. 

And\marginnote{2.1} how does a mendicant with proper attention develop and cultivate the seven awakening factors? It’s when a mendicant develops the awakening factors of mindfulness, investigation of principles, energy, rapture, tranquility, immersion, and equanimity, which rely on seclusion, fading away, and cessation, and ripen as letting go. That’s how a mendicant with proper attention develops and cultivates the seven awakening factors.” 

%
\section*{{\suttatitleacronym SN 46.14}{\suttatitletranslation Sick (1st) }{\suttatitleroot Paṭhamagilānasutta}}
\addcontentsline{toc}{section}{\tocacronym{SN 46.14} \toctranslation{Sick (1st) } \tocroot{Paṭhamagilānasutta}}
\markboth{Sick (1st) }{Paṭhamagilānasutta}
\extramarks{SN 46.14}{SN 46.14}

At\marginnote{1.1} one time the Buddha was staying near \textsanskrit{Rājagaha}, in the Bamboo Grove, the squirrels’ feeding ground. Now at that time Venerable \textsanskrit{Mahākassapa} was staying in the Pipphali cave, and he was sick, suffering, gravely ill. Then in the late afternoon, the Buddha came out of retreat, went to Venerable \textsanskrit{Mahākassapa}, sat down on the seat spread out, and said to him: 

“I\marginnote{2.1} hope you’re keeping well, Kassapa; I hope you’re alright. And I hope the pain is fading, not growing, that its fading is evident, not its growing.” 

“Sir,\marginnote{2.2} I’m not keeping well, I’m not alright. The pain is terrible and growing, not fading; its growing is evident, not its fading.” 

“Kassapa,\marginnote{3.1} I’ve rightly explained these seven awakening factors. When developed and cultivated, they lead to direct knowledge, to awakening, and to extinguishment. What seven? The awakening factors of mindfulness, investigation of principles, energy, rapture, tranquility, immersion, and equanimity. These are the seven awakening factors that I’ve rightly explained. When developed and cultivated, they lead to direct knowledge, to awakening, and to extinguishment.” 

“Indeed,\marginnote{3.5} Blessed One, these are awakening factors! Indeed, Holy One, these are awakening factors!” 

That\marginnote{4.1} is what the Buddha said. Satisfied, Venerable \textsanskrit{Mahākassapa} was happy with what the Buddha said. And that’s how he recovered from that illness. 

%
\section*{{\suttatitleacronym SN 46.15}{\suttatitletranslation Sick (2nd) }{\suttatitleroot Dutiyagilānasutta}}
\addcontentsline{toc}{section}{\tocacronym{SN 46.15} \toctranslation{Sick (2nd) } \tocroot{Dutiyagilānasutta}}
\markboth{Sick (2nd) }{Dutiyagilānasutta}
\extramarks{SN 46.15}{SN 46.15}

At\marginnote{1.1} one time the Buddha was staying near \textsanskrit{Rājagaha}, in the Bamboo Grove, the squirrels’ feeding ground. Now at that time Venerable \textsanskrit{Mahāmoggallāna} was staying on the Vulture’s Peak mountain, and he was sick, suffering, gravely ill. Then in the late afternoon, the Buddha came out of retreat, went to Venerable \textsanskrit{Moggallāna}, sat down on the seat spread out, and said to him: 

“I\marginnote{2.1} hope you’re keeping well, \textsanskrit{Moggallāna}; I hope you’re alright. And I hope the pain is fading, not growing, that its fading is evident, not its growing.” 

“Sir,\marginnote{2.2} I’m not keeping well, I’m not alright. The pain is terrible and growing, not fading; its growing is evident, not its fading.” 

“\textsanskrit{Moggallāna},\marginnote{3.1} I’ve rightly explained these seven awakening factors. When developed and cultivated, they lead to direct knowledge, to awakening, and to extinguishment. What seven? The awakening factors of mindfulness, investigation of principles, energy, rapture, tranquility, immersion, and equanimity. These are the seven awakening factors that I’ve rightly explained. When developed and cultivated, they lead to direct knowledge, to awakening, and to extinguishment.” 

“Indeed,\marginnote{3.5} Blessed One, these are awakening factors! Indeed, Holy One, these are awakening factors!” 

That\marginnote{4.1} is what the Buddha said. Satisfied, Venerable \textsanskrit{Mahāmoggallāna} was happy with what the Buddha said. And that’s how he recovered from that illness. 

%
\section*{{\suttatitleacronym SN 46.16}{\suttatitletranslation Sick (3rd) }{\suttatitleroot Tatiyagilānasutta}}
\addcontentsline{toc}{section}{\tocacronym{SN 46.16} \toctranslation{Sick (3rd) } \tocroot{Tatiyagilānasutta}}
\markboth{Sick (3rd) }{Tatiyagilānasutta}
\extramarks{SN 46.16}{SN 46.16}

At\marginnote{1.1} one time the Buddha was staying near \textsanskrit{Rājagaha}, in the Bamboo Grove, the squirrels’ feeding ground. Now at that time he was sick, suffering, gravely ill. Then Venerable \textsanskrit{Mahācunda} went up to the Buddha, bowed, and sat down to one side. The Buddha said to him: 

“Cunda,\marginnote{1.4} express your understanding of the awakening factors.” 

“Sir,\marginnote{2.1} the Buddha has rightly explained these seven awakening factors. When developed and cultivated, they lead to direct knowledge, to awakening, and to extinguishment. What seven? The awakening factors of mindfulness, investigation of principles, energy, rapture, tranquility, immersion, and equanimity. These are the seven awakening factors that the Buddha has rightly explained. When developed and cultivated, they lead to direct knowledge, to awakening, and to extinguishment.” 

“Indeed,\marginnote{2.5} Cunda, these are awakening factors! Indeed, Cunda, these are awakening factors!” 

This\marginnote{3.1} is what Cunda said, and the teacher approved. And that’s how the Buddha recovered from that illness. 

%
\section*{{\suttatitleacronym SN 46.17}{\suttatitletranslation Going to the Far Shore }{\suttatitleroot Pāraṅgamasutta}}
\addcontentsline{toc}{section}{\tocacronym{SN 46.17} \toctranslation{Going to the Far Shore } \tocroot{Pāraṅgamasutta}}
\markboth{Going to the Far Shore }{Pāraṅgamasutta}
\extramarks{SN 46.17}{SN 46.17}

“Mendicants,\marginnote{1.1} when these seven awakening factors are developed and cultivated they lead to going from the near shore to the far shore. What seven? The awakening factors of mindfulness, investigation of principles, energy, rapture, tranquility, immersion, and equanimity. These seven awakening factors, when developed and cultivated, lead to going from the near shore to the far shore. 

\begin{verse}%
Few\marginnote{2.1} are those among humans \\
who cross to the far shore. \\
The rest just run \\
around on the near shore. 

When\marginnote{3.1} the teaching is well explained, \\
those who practice accordingly \\
are the ones who will cross over \\
Death’s domain so hard to pass. 

Rid\marginnote{4.1} of dark qualities, \\
an astute person should develop the bright. \\
Leaving home behind \\
for the seclusion so hard to enjoy, 

you\marginnote{5.1} should try to find delight there, \\
having left behind sensual pleasures. \\
With no possessions, an astute person \\
should cleanse themselves of mental corruptions. 

And\marginnote{6.1} those whose minds are rightly developed \\
in the awakening factors; \\
letting go of attachments, \\
they delight in not grasping. \\
With defilements ended, brilliant, \\
they are extinguished in this world.” 

%
\end{verse}

%
\section*{{\suttatitleacronym SN 46.18}{\suttatitletranslation Missed Out }{\suttatitleroot Viraddhasutta}}
\addcontentsline{toc}{section}{\tocacronym{SN 46.18} \toctranslation{Missed Out } \tocroot{Viraddhasutta}}
\markboth{Missed Out }{Viraddhasutta}
\extramarks{SN 46.18}{SN 46.18}

“Mendicants,\marginnote{1.1} whoever has missed out on the seven awakening factors has missed out on the noble path to the complete ending of suffering. Whoever has undertaken the seven awakening factors has undertaken the noble path to the complete ending of suffering. What seven? The awakening factors of mindfulness, investigation of principles, energy, rapture, tranquility, immersion, and equanimity. Whoever has missed out on these seven awakening factors has missed out on the noble path to the complete ending of suffering. Whoever has undertaken these seven awakening factors has undertaken the noble path to the complete ending of suffering.” 

%
\section*{{\suttatitleacronym SN 46.19}{\suttatitletranslation Noble }{\suttatitleroot Ariyasutta}}
\addcontentsline{toc}{section}{\tocacronym{SN 46.19} \toctranslation{Noble } \tocroot{Ariyasutta}}
\markboth{Noble }{Ariyasutta}
\extramarks{SN 46.19}{SN 46.19}

“Mendicants,\marginnote{1.1} when these seven awakening factors are developed and cultivated they are noble and emancipating, and bring one who practices them to the complete ending of suffering. What seven? The awakening factors of mindfulness, investigation of principles, energy, rapture, tranquility, immersion, and equanimity. When these seven awakening factors are developed and cultivated they are noble and emancipating, and bring one who practices them to the complete ending of suffering.” 

%
\section*{{\suttatitleacronym SN 46.20}{\suttatitletranslation Disillusionment }{\suttatitleroot Nibbidāsutta}}
\addcontentsline{toc}{section}{\tocacronym{SN 46.20} \toctranslation{Disillusionment } \tocroot{Nibbidāsutta}}
\markboth{Disillusionment }{Nibbidāsutta}
\extramarks{SN 46.20}{SN 46.20}

“Mendicants,\marginnote{1.1} the seven awakening factors, when developed and cultivated, lead solely to disillusionment, dispassion, cessation, peace, insight, awakening, and extinguishment. What seven? The awakening factors of mindfulness, investigation of principles, energy, rapture, tranquility, immersion, and equanimity. These seven awakening factors, when developed and cultivated, lead solely to disillusionment, dispassion, cessation, peace, insight, awakening, and extinguishment.” 

%
\addtocontents{toc}{\let\protect\contentsline\protect\nopagecontentsline}
\chapter*{The Chapter with Udāyī }
\addcontentsline{toc}{chapter}{\tocchapterline{The Chapter with Udāyī }}
\addtocontents{toc}{\let\protect\contentsline\protect\oldcontentsline}

%
\section*{{\suttatitleacronym SN 46.21}{\suttatitletranslation To Awakening }{\suttatitleroot Bodhāyasutta}}
\addcontentsline{toc}{section}{\tocacronym{SN 46.21} \toctranslation{To Awakening } \tocroot{Bodhāyasutta}}
\markboth{To Awakening }{Bodhāyasutta}
\extramarks{SN 46.21}{SN 46.21}

Then\marginnote{1.1} a mendicant went up to the Buddha … and said to him: 

“Sir,\marginnote{2.1} they speak of the ‘awakening factors’. How are the awakening factors defined?” 

“Mendicant,\marginnote{2.3} they’re called awakening factors because they lead to awakening. A mendicant develops the awakening factors of mindfulness, investigation of principles, energy, rapture, tranquility, immersion, and equanimity, which rely on seclusion, fading away, and cessation, and ripen as letting go. They’re called awakening factors because they lead to awakening.” 

%
\section*{{\suttatitleacronym SN 46.22}{\suttatitletranslation A Teaching on the Awakening Factors }{\suttatitleroot Bojjhaṅgadesanāsutta}}
\addcontentsline{toc}{section}{\tocacronym{SN 46.22} \toctranslation{A Teaching on the Awakening Factors } \tocroot{Bojjhaṅgadesanāsutta}}
\markboth{A Teaching on the Awakening Factors }{Bojjhaṅgadesanāsutta}
\extramarks{SN 46.22}{SN 46.22}

“Mendicants,\marginnote{1.1} I will teach you the seven awakening factors. Listen … 

And\marginnote{1.3} what are the seven awakening factors? The awakening factors of mindfulness, investigation of principles, energy, rapture, tranquility, immersion, and equanimity. These are the seven awakening factors.” 

%
\section*{{\suttatitleacronym SN 46.23}{\suttatitletranslation Grounds }{\suttatitleroot Ṭhāniyasutta}}
\addcontentsline{toc}{section}{\tocacronym{SN 46.23} \toctranslation{Grounds } \tocroot{Ṭhāniyasutta}}
\markboth{Grounds }{Ṭhāniyasutta}
\extramarks{SN 46.23}{SN 46.23}

“Mendicants,\marginnote{1.1} when you frequently attend improperly to things that are grounds for sensual greed, sensual desire arises, and once arisen it increases and grows. When you frequently attend improperly to things that are grounds for ill will, ill will arises, and once arisen it increases and grows. When you frequently attend improperly to things that are grounds for dullness and drowsiness, dullness and drowsiness arise, and once arisen they increase and grow. When you frequently attend improperly to things that are grounds for restlessness and remorse, restlessness and remorse arise, and once arisen they increase and grow. When you frequently attend improperly to things that are grounds for doubt, doubt arises, and once arisen it increases and grows. 

When\marginnote{2.1} you frequently attend properly on things that are grounds for the awakening factor of mindfulness, the awakening factor of mindfulness arises, and once arisen it’s fully developed. … When you frequently attend properly on things that are grounds for the awakening factor of equanimity, the awakening factor of equanimity arises, and once arisen it’s fully developed.” 

%
\section*{{\suttatitleacronym SN 46.24}{\suttatitletranslation Improper Attention }{\suttatitleroot Ayonisomanasikārasutta}}
\addcontentsline{toc}{section}{\tocacronym{SN 46.24} \toctranslation{Improper Attention } \tocroot{Ayonisomanasikārasutta}}
\markboth{Improper Attention }{Ayonisomanasikārasutta}
\extramarks{SN 46.24}{SN 46.24}

“Mendicants,\marginnote{1.1} when you attend improperly, sensual desire, ill will, dullness and drowsiness, restlessness and remorse, and doubt arise, and once arisen they increase and grow. And the awakening factors of mindfulness, investigation of principles, energy, rapture, tranquility, immersion, and equanimity don’t arise, or if they’ve already arisen, they cease. 

When\marginnote{2.1} you attend properly, sensual desire, ill will, dullness and drowsiness, restlessness and remorse, and doubt don’t arise, or if they’ve already arisen they’re given up. 

And\marginnote{3.1} the awakening factors of mindfulness, investigation of principles, energy, rapture, tranquility, immersion, and equanimity arise, and once they’ve arisen, they’re fully developed.” 

%
\section*{{\suttatitleacronym SN 46.25}{\suttatitletranslation Non-decline }{\suttatitleroot Aparihāniyasutta}}
\addcontentsline{toc}{section}{\tocacronym{SN 46.25} \toctranslation{Non-decline } \tocroot{Aparihāniyasutta}}
\markboth{Non-decline }{Aparihāniyasutta}
\extramarks{SN 46.25}{SN 46.25}

“Mendicants,\marginnote{1.1} I will teach you seven principles that guard against decline. Listen … 

And\marginnote{1.3} what are the seven principles that guard against decline? They are the seven awakening factors. What seven? The awakening factors of mindfulness, investigation of principles, energy, rapture, tranquility, immersion, and equanimity. These are the seven principles that guard against decline.” 

%
\section*{{\suttatitleacronym SN 46.26}{\suttatitletranslation The Ending of Craving }{\suttatitleroot Taṇhakkhayasutta}}
\addcontentsline{toc}{section}{\tocacronym{SN 46.26} \toctranslation{The Ending of Craving } \tocroot{Taṇhakkhayasutta}}
\markboth{The Ending of Craving }{Taṇhakkhayasutta}
\extramarks{SN 46.26}{SN 46.26}

“Mendicants,\marginnote{1.1} you should develop the path and the practice that leads to the ending of craving. And what is the path and the practice that leads to the ending of craving? It is the seven awakening factors. What seven? The awakening factors of mindfulness, investigation of principles, energy, rapture, tranquility, immersion, and equanimity.” When he said this, \textsanskrit{Udāyī} said to him: 

“Sir,\marginnote{1.8} how are the seven awakening factors developed and cultivated so as to lead to the ending of craving?” 

“\textsanskrit{Udāyī},\marginnote{2.1} it’s when a mendicant develops the awakening factor of mindfulness, which relies on seclusion, fading away, and cessation, and ripens as letting go. And it is abundant, expansive, limitless, and free of ill will. As they do so, craving is given up. When craving is given up, deeds are given up. When deeds are given up, suffering is given up. … 

A\marginnote{2.5} mendicant develops the awakening factor of equanimity, which relies on seclusion, fading away, and cessation, and ripens as letting go. And it is abundant, expansive, limitless, and free of ill will. As they do so, craving is given up. When craving is given up, deeds are given up. When deeds are given up, suffering is given up. 

And\marginnote{2.9} so, \textsanskrit{Udāyī}, when craving ends, deeds end; when deeds end suffering ends.” 

%
\section*{{\suttatitleacronym SN 46.27}{\suttatitletranslation The Cessation of Craving }{\suttatitleroot Taṇhānirodhasutta}}
\addcontentsline{toc}{section}{\tocacronym{SN 46.27} \toctranslation{The Cessation of Craving } \tocroot{Taṇhānirodhasutta}}
\markboth{The Cessation of Craving }{Taṇhānirodhasutta}
\extramarks{SN 46.27}{SN 46.27}

“Mendicants,\marginnote{1.1} you should develop the path and the practice that leads to the cessation of craving. And what is the path and the practice that leads to the cessation of craving? It is the seven awakening factors. What seven? The awakening factors of mindfulness, investigation of principles, energy, rapture, tranquility, immersion, and equanimity. And how are the seven awakening factors developed and cultivated so as to lead to the cessation of craving? 

It’s\marginnote{2.1} when a mendicant develops the awakening factors of mindfulness, investigation of principles, energy, rapture, tranquility, immersion, 

and\marginnote{2.2} equanimity, which rely on seclusion, fading away, and cessation, and ripen as letting go. This is how the seven awakening factors are developed and cultivated so as to lead to the cessation of craving.” 

%
\section*{{\suttatitleacronym SN 46.28}{\suttatitletranslation Helping Penetration }{\suttatitleroot Nibbedhabhāgiyasutta}}
\addcontentsline{toc}{section}{\tocacronym{SN 46.28} \toctranslation{Helping Penetration } \tocroot{Nibbedhabhāgiyasutta}}
\markboth{Helping Penetration }{Nibbedhabhāgiyasutta}
\extramarks{SN 46.28}{SN 46.28}

“Mendicants,\marginnote{1.1} I will teach you a path that helps penetration. Listen … 

And\marginnote{1.3} what is the path that helps penetration? It is the seven awakening factors. What seven? The awakening factors of mindfulness, investigation of principles, energy, rapture, tranquility, immersion, and equanimity.” When he said this, \textsanskrit{Udāyī} said to him: 

“Sir,\marginnote{1.9} how are the seven awakening factors developed and cultivated so as to lead to penetration?” 

“\textsanskrit{Udāyī},\marginnote{2.1} it’s when a mendicant develops the awakening factor of mindfulness, which relies on seclusion, fading away, and cessation, and ripens as letting go. And it is abundant, expansive, limitless, and free of ill will. With a mind that has developed the awakening factor of mindfulness, they penetrate and shatter the mass of greed, the mass of hate, and the mass of delusion for the first time. … 

A\marginnote{2.5} mendicant develops the awakening factor of equanimity, which relies on seclusion, fading away, and cessation, and ripens as letting go. And it is abundant, expansive, limitless, and free of ill will. With a mind that has developed the awakening factor of equanimity, they penetrate and shatter the mass of greed, the mass of hate, and the mass of delusion for the first time. 

This\marginnote{2.9} is how are the seven awakening factors are developed and cultivated so as to lead to penetration.” 

%
\section*{{\suttatitleacronym SN 46.29}{\suttatitletranslation One Thing }{\suttatitleroot Ekadhammasutta}}
\addcontentsline{toc}{section}{\tocacronym{SN 46.29} \toctranslation{One Thing } \tocroot{Ekadhammasutta}}
\markboth{One Thing }{Ekadhammasutta}
\extramarks{SN 46.29}{SN 46.29}

“Mendicants,\marginnote{1.1} I do not see a single thing that, when it is developed and cultivated like this, leads to giving up the things that are prone to being fettered like the seven awakening factors. What seven? The awakening factors of mindfulness, investigation of principles, energy, rapture, tranquility, immersion, and equanimity. And how are the seven awakening factors developed and cultivated so as to lead to giving up the things that are prone to being fettered? 

It’s\marginnote{2.1} when a mendicant develops the awakening factors of mindfulness, investigation of principles, energy, rapture, tranquility, immersion, 

and\marginnote{2.2} equanimity, which rely on seclusion, fading away, and cessation, and ripen as letting go. That’s how the seven awakening factors are developed and cultivated so as to lead to giving up the things that are prone to being fettered. 

And\marginnote{3.1} what are the things that are prone to being fettered? The eye is something that’s prone to being fettered. This is where these fetters, shackles, and attachments arise. The ear … nose … tongue … body … mind is something that’s prone to being fettered. This is where these fetters, shackles, and attachments arise. These are called the things that are prone to being fettered.” 

%
\section*{{\suttatitleacronym SN 46.30}{\suttatitletranslation With Udāyī }{\suttatitleroot Udāyisutta}}
\addcontentsline{toc}{section}{\tocacronym{SN 46.30} \toctranslation{With Udāyī } \tocroot{Udāyisutta}}
\markboth{With Udāyī }{Udāyisutta}
\extramarks{SN 46.30}{SN 46.30}

At\marginnote{1.1} one time the Buddha was staying in the land of the Sumbhas, near the town of the Sumbhas called Sedaka. Then Venerable \textsanskrit{Udāyī} went up to the Buddha … and said to him: 

“It’s\marginnote{2.1} incredible, sir, it’s amazing! How helpful my love and respect for the Buddha have been, and my sense of conscience and prudence. For when I was still a layman, I wasn’t helped much by the teaching or the \textsanskrit{Saṅgha}. But when I considered my love and respect for the Buddha, and my sense of conscience and prudence, I went forth from the lay life to homelessness. The Buddha taught me the Dhamma: ‘Such is form, such is the origin of form, such is the ending of form. Such is feeling … Such is perception … Such are choices … Such is consciousness, such is the origin of consciousness, such is the ending of consciousness.’ 

Then,\marginnote{3.1} while staying in an empty hut, I followed the churning of the five grasping aggregates. I truly understood: ‘This is suffering’ … ‘This is the origin of suffering’ … ‘This is the cessation of suffering’ … ‘This is the practice that leads to the cessation of suffering’. I comprehended the teaching; I acquired the path. When developed and cultivated as I’m living in such a way, it will bring me to such a state that I will understand: ‘Rebirth is ended, the spiritual journey has been completed, what had to be done has been done, there is no return to any state of existence.’ 

I\marginnote{4.1} acquired the awakening factors of mindfulness, investigation of principles, energy, rapture, tranquility, immersion, and equanimity. When developed and cultivated as I’m living in such a way, they will bring me to such a state that I will understand: ‘Rebirth is ended, the spiritual journey has been completed, what had to be done has been done, there is no return to any state of existence.’ This is the path that I acquired. When developed and cultivated as I’m living in such a way, it will bring me to such a state that I will understand: ‘Rebirth is ended, the spiritual journey has been completed, what had to be done has been done, there is no return to any state of existence.’” 

“Good,\marginnote{5.1} good, \textsanskrit{Udāyī}! For that is indeed the path that you acquired. When developed and cultivated as you’re living in such a way, it will bring you to such a state that you will understand: ‘Rebirth is ended, the spiritual journey has been completed, what had to be done has been done, there is no return to any state of existence.’” 

%
\addtocontents{toc}{\let\protect\contentsline\protect\nopagecontentsline}
\chapter*{The Chapter on Hindrances }
\addcontentsline{toc}{chapter}{\tocchapterline{The Chapter on Hindrances }}
\addtocontents{toc}{\let\protect\contentsline\protect\oldcontentsline}

%
\section*{{\suttatitleacronym SN 46.31}{\suttatitletranslation Skillful (1st) }{\suttatitleroot Paṭhamakusalasutta}}
\addcontentsline{toc}{section}{\tocacronym{SN 46.31} \toctranslation{Skillful (1st) } \tocroot{Paṭhamakusalasutta}}
\markboth{Skillful (1st) }{Paṭhamakusalasutta}
\extramarks{SN 46.31}{SN 46.31}

“Mendicants,\marginnote{1.1} whatever qualities are skillful, part of the skillful, on the side of the skillful, all of them are rooted in diligence and meet at diligence, and diligence is said to be the best of them. A mendicant who is diligent can expect to develop and cultivate the seven awakening factors. 

And\marginnote{2.1} how does a diligent mendicant develop and cultivate the seven awakening factors? It’s when a mendicant develops the awakening factors of mindfulness, investigation of principles, energy, rapture, tranquility, immersion, and equanimity, which rely on seclusion, fading away, and cessation, and ripen as letting go. That’s how a diligent mendicant develops and cultivates the seven awakening factors.” 

%
\section*{{\suttatitleacronym SN 46.32}{\suttatitletranslation Skillful (2nd) }{\suttatitleroot Dutiyakusalasutta}}
\addcontentsline{toc}{section}{\tocacronym{SN 46.32} \toctranslation{Skillful (2nd) } \tocroot{Dutiyakusalasutta}}
\markboth{Skillful (2nd) }{Dutiyakusalasutta}
\extramarks{SN 46.32}{SN 46.32}

“Mendicants,\marginnote{1.1} whatever qualities are skillful, part of the skillful, on the side of the skillful, all of them are rooted in proper attention and meet at proper attention, and proper attention is said to be the best of them. A mendicant accomplished in proper attention can expect to develop and cultivate the seven awakening factors. 

And\marginnote{2.1} how does a mendicant with proper attention develop and cultivate the seven awakening factors? It’s when a mendicant develops the awakening factors of mindfulness, investigation of principles, energy, rapture, tranquility, immersion, and equanimity, which rely on seclusion, fading away, and cessation, and ripen as letting go. That’s how a mendicant with proper attention develops and cultivates the seven awakening factors.” 

%
\section*{{\suttatitleacronym SN 46.33}{\suttatitletranslation Corruptions }{\suttatitleroot Upakkilesasutta}}
\addcontentsline{toc}{section}{\tocacronym{SN 46.33} \toctranslation{Corruptions } \tocroot{Upakkilesasutta}}
\markboth{Corruptions }{Upakkilesasutta}
\extramarks{SN 46.33}{SN 46.33}

“Mendicants,\marginnote{1.1} there are these five corruptions of gold. When gold is corrupted by these it’s not pliable, workable, or radiant, but is brittle and not completely ready for working. What five? Iron, copper, tin, lead, and silver. When gold is corrupted by these five corruptions it’s not pliable, workable, or radiant, but is brittle and not completely ready for working. 

In\marginnote{2.1} the same way, there are these five corruptions of the mind. When the mind is corrupted by these it’s not pliable, workable, or radiant. It’s brittle, and not completely immersed in \textsanskrit{samādhi} for the ending of defilements. What five? Sensual desire, ill will, dullness and drowsiness, restlessness and remorse, and doubt. These are the five corruptions of the mind. When the mind is corrupted by these it’s not pliable, workable, or radiant. It’s brittle, and not completely immersed in \textsanskrit{samādhi} for the ending of defilements.” 

%
\section*{{\suttatitleacronym SN 46.34}{\suttatitletranslation Not Corruptions }{\suttatitleroot Anupakkilesasutta}}
\addcontentsline{toc}{section}{\tocacronym{SN 46.34} \toctranslation{Not Corruptions } \tocroot{Anupakkilesasutta}}
\markboth{Not Corruptions }{Anupakkilesasutta}
\extramarks{SN 46.34}{SN 46.34}

“Mendicants,\marginnote{1.1} these seven awakening factors are not obstacles, hindrances, or corruptions of the mind. When developed and cultivated they lead to the realization of the fruit of knowledge and freedom. What seven? The awakening factors of mindfulness, investigation of principles, energy, rapture, tranquility, immersion, and equanimity. These seven awakening factors are not obstacles, hindrances, or corruptions of the mind. When developed and cultivated they lead to the realization of the fruit of knowledge and freedom.” 

%
\section*{{\suttatitleacronym SN 46.35}{\suttatitletranslation Improper Attention }{\suttatitleroot Yonisomanasikārasutta}}
\addcontentsline{toc}{section}{\tocacronym{SN 46.35} \toctranslation{Improper Attention } \tocroot{Yonisomanasikārasutta}}
\markboth{Improper Attention }{Yonisomanasikārasutta}
\extramarks{SN 46.35}{SN 46.35}

“Mendicants,\marginnote{1.1} when you attend improperly, sensual desire, ill will, dullness and drowsiness, restlessness and remorse, and doubt arise, and once arisen they increase and grow.” 

“Mendicants,\marginnote{2.1} when you attend properly, the awakening factors of mindfulness, investigation of principles, energy, rapture, tranquility, immersion, and equanimity arise, and once they’ve arisen, they’re fully developed.” 

%
\section*{{\suttatitleacronym SN 46.36}{\suttatitletranslation Growth }{\suttatitleroot Buddhisutta}}
\addcontentsline{toc}{section}{\tocacronym{SN 46.36} \toctranslation{Growth } \tocroot{Buddhisutta}}
\markboth{Growth }{Buddhisutta}
\extramarks{SN 46.36}{SN 46.36}

“Mendicants,\marginnote{1.1} when the seven awakening factors are developed and cultivated they lead to growth and progress. What seven? The awakening factors of mindfulness, investigation of principles, energy, rapture, tranquility, immersion, and equanimity. When these seven awakening factors are developed and cultivated they lead to growth and progress.” 

%
\section*{{\suttatitleacronym SN 46.37}{\suttatitletranslation Obstacles }{\suttatitleroot Āvaraṇanīvaraṇasutta}}
\addcontentsline{toc}{section}{\tocacronym{SN 46.37} \toctranslation{Obstacles } \tocroot{Āvaraṇanīvaraṇasutta}}
\markboth{Obstacles }{Āvaraṇanīvaraṇasutta}
\extramarks{SN 46.37}{SN 46.37}

“Mendicants,\marginnote{1.1} there are these five obstacles and hindrances, corruptions of the heart that weaken wisdom. What five? Sensual desire, ill will, dullness and drowsiness, restlessness and remorse, and doubt. These are the five obstacles and hindrances, corruptions of the heart that weaken wisdom. 

There\marginnote{2.1} are these seven awakening factors that are not obstacles, hindrances, or corruptions of the mind. When developed and cultivated they lead to the realization of the fruit of knowledge and freedom. What seven? The awakening factors of mindfulness, investigation of principles, energy, rapture, tranquility, immersion, and equanimity. These seven awakening factors are not obstacles, hindrances, or corruptions of the mind. When developed and cultivated they lead to the realization of the fruit of knowledge and freedom. 

%
\section*{{\suttatitleacronym SN 46.38}{\suttatitletranslation Without Obstacles }{\suttatitleroot Anīvaraṇasutta}}
\addcontentsline{toc}{section}{\tocacronym{SN 46.38} \toctranslation{Without Obstacles } \tocroot{Anīvaraṇasutta}}
\markboth{Without Obstacles }{Anīvaraṇasutta}
\extramarks{SN 46.38}{SN 46.38}

Mendicants,\marginnote{1.1} sometimes a mendicant pays heed, pays attention, engages wholeheartedly, and lends an ear to the teaching. At such a time the five hindrances are absent, and the seven awakening factors are fully developed. 

What\marginnote{2.1} are the five hindrances that are absent? Sensual desire, ill will, dullness and drowsiness, restlessness and remorse, and doubt. These are the five hindrances that are absent. 

And\marginnote{3.1} what are the seven awakening factors that are fully developed? The awakening factors of mindfulness, investigation of principles, energy, rapture, tranquility, immersion, and equanimity. These are the seven awakening factors that are fully developed. Sometimes a mendicant pays heed, pays attention, engages wholeheartedly, and lends an ear to the teaching. At such a time the five hindrances are absent, and the seven awakening factors are fully developed.” 

%
\section*{{\suttatitleacronym SN 46.39}{\suttatitletranslation Trees }{\suttatitleroot Rukkhasutta}}
\addcontentsline{toc}{section}{\tocacronym{SN 46.39} \toctranslation{Trees } \tocroot{Rukkhasutta}}
\markboth{Trees }{Rukkhasutta}
\extramarks{SN 46.39}{SN 46.39}

“Mendicants,\marginnote{1.1} there are large trees with tiny seeds and big trunks, which grow up and around other trees as parasites. The trees they engulf break apart, collapse, and fall. And what are those large trees with tiny seeds and big trunks? The bodhi, banyan, wavy leaf fig, cluster fig, Moreton Bay fig, and wood apple. These are the large trees with tiny seeds and big trunks, which grow up and around other trees as parasites. The trees they engulf break apart, collapse, and fall. 

In\marginnote{1.5} the same way, take a certain gentleman who has gone forth from the lay life to homelessness, abandoning sensual pleasures. But beset by sensual pleasures that are similar, or even worse, he breaks apart, collapses, and falls. 

There\marginnote{2.1} are these five obstacles and hindrances, parasites of the mind that weaken wisdom. What five? Sensual desire, ill will, dullness and drowsiness, restlessness and remorse, and doubt. These are the five obstacles and hindrances, parasites of the mind that weaken wisdom. 

These\marginnote{3.1} seven awakening factors are not obstacles, hindrances, or parasites of the mind. When developed and cultivated they lead to the realization of the fruit of knowledge and freedom. What seven? The awakening factors of mindfulness, investigation of principles, energy, rapture, tranquility, immersion, and equanimity. These seven awakening factors are not obstacles, hindrances, or parasites of the mind. When developed and cultivated they lead to the realization of the fruit of knowledge and freedom.” 

%
\section*{{\suttatitleacronym SN 46.40}{\suttatitletranslation Hindrances }{\suttatitleroot Nīvaraṇasutta}}
\addcontentsline{toc}{section}{\tocacronym{SN 46.40} \toctranslation{Hindrances } \tocroot{Nīvaraṇasutta}}
\markboth{Hindrances }{Nīvaraṇasutta}
\extramarks{SN 46.40}{SN 46.40}

“Mendicants,\marginnote{1.1} these five hindrances are destroyers of sight, vision, and knowledge. They block wisdom, they’re on the side of anguish, and they don’t lead to extinguishment. What five? Sensual desire, ill will, dullness and drowsiness, restlessness and remorse, and doubt. These five hindrances are destroyers of sight, vision, and knowledge. They block wisdom, they’re on the side of anguish, and they don’t lead to extinguishment. 

These\marginnote{2.1} seven awakening factors are creators of vision and knowledge. They grow wisdom, they’re on the side of solace, and they lead to extinguishment. What seven? The awakening factors of mindfulness, investigation of principles, energy, rapture, tranquility, immersion, and equanimity. These seven awakening factors are creators of vision and knowledge. They grow wisdom, they’re on the side of solace, and they lead to extinguishment.” 

%
\addtocontents{toc}{\let\protect\contentsline\protect\nopagecontentsline}
\chapter*{The Chapter on the Wheel-Turning Monarch }
\addcontentsline{toc}{chapter}{\tocchapterline{The Chapter on the Wheel-Turning Monarch }}
\addtocontents{toc}{\let\protect\contentsline\protect\oldcontentsline}

%
\section*{{\suttatitleacronym SN 46.41}{\suttatitletranslation Discriminations }{\suttatitleroot Vidhāsutta}}
\addcontentsline{toc}{section}{\tocacronym{SN 46.41} \toctranslation{Discriminations } \tocroot{Vidhāsutta}}
\markboth{Discriminations }{Vidhāsutta}
\extramarks{SN 46.41}{SN 46.41}

At\marginnote{1.1} \textsanskrit{Sāvatthī}. 

“Mendicants,\marginnote{1.2} all the ascetics and brahmins in the past who have given up the three discriminations have done so by developing and cultivating the seven awakening factors. All the ascetics and brahmins in the future who will give up the three discriminations will do so by developing and cultivating the seven awakening factors. All the ascetics and brahmins in the present who are giving up the three discriminations do so by developing and cultivating the seven awakening factors. 

What\marginnote{1.5} seven? The awakening factors of mindfulness, investigation of principles, energy, rapture, tranquility, immersion, and equanimity. All the ascetics and brahmins in the past … future … and present who give up the three discriminations do so by developing and cultivating the seven awakening factors.” 

%
\section*{{\suttatitleacronym SN 46.42}{\suttatitletranslation A Wheel-Turning Monarch }{\suttatitleroot Cakkavattisutta}}
\addcontentsline{toc}{section}{\tocacronym{SN 46.42} \toctranslation{A Wheel-Turning Monarch } \tocroot{Cakkavattisutta}}
\markboth{A Wheel-Turning Monarch }{Cakkavattisutta}
\extramarks{SN 46.42}{SN 46.42}

“Mendicants,\marginnote{1.1} when a Wheel-Turning Monarch appears seven treasures appear. What seven? The wheel, the elephant, the horse, the jewel, the woman, the treasurer, and the counselor. When a Wheel-Turning Monarch appears these seven treasures appear. 

When\marginnote{2.1} a Realized One, a perfected one, a fully awakened Buddha appears the seven treasures of the awakening factors appear. What seven? The treasures of the awakening factors of mindfulness, investigation of principles, energy, rapture, tranquility, immersion, and equanimity. When a Realized One, a perfected one, a fully awakened Buddha appears these seven treasures of the awakening factors appear.” 

%
\section*{{\suttatitleacronym SN 46.43}{\suttatitletranslation About Māra }{\suttatitleroot Mārasutta}}
\addcontentsline{toc}{section}{\tocacronym{SN 46.43} \toctranslation{About Māra } \tocroot{Mārasutta}}
\markboth{About Māra }{Mārasutta}
\extramarks{SN 46.43}{SN 46.43}

“Mendicants,\marginnote{1.1} I will teach you a path for crushing \textsanskrit{Māra}’s army. Listen … 

And\marginnote{1.3} what is that path? It is the seven awakening factors. What seven? The awakening factors of mindfulness, investigation of principles, energy, rapture, tranquility, immersion, and equanimity. This is the path for crushing \textsanskrit{Māra}’s army.” 

%
\section*{{\suttatitleacronym SN 46.44}{\suttatitletranslation Witless }{\suttatitleroot Duppaññasutta}}
\addcontentsline{toc}{section}{\tocacronym{SN 46.44} \toctranslation{Witless } \tocroot{Duppaññasutta}}
\markboth{Witless }{Duppaññasutta}
\extramarks{SN 46.44}{SN 46.44}

Then\marginnote{1.1} a mendicant went up to the Buddha … and asked him, “Sir, they speak of ‘a witless idiot’. How is a witless idiot defined?” 

“Mendicant,\marginnote{1.4} they’re called a witless idiot because they haven’t developed and cultivated the seven awakening factors. What seven? The awakening factors of mindfulness, investigation of principles, energy, rapture, tranquility, immersion, and equanimity. They’re called a witless idiot because they haven’t developed and cultivated these seven awakening factors.” 

%
\section*{{\suttatitleacronym SN 46.45}{\suttatitletranslation Wise }{\suttatitleroot Paññavantasutta}}
\addcontentsline{toc}{section}{\tocacronym{SN 46.45} \toctranslation{Wise } \tocroot{Paññavantasutta}}
\markboth{Wise }{Paññavantasutta}
\extramarks{SN 46.45}{SN 46.45}

“Sir\marginnote{1.1} they speak of a person who is ‘wise, no idiot’. How is a person who is wise, no idiot defined?” 

“Mendicant,\marginnote{1.3} they’re called wise, no idiot because they’ve developed and cultivated the seven awakening factors. What seven? The awakening factors of mindfulness, investigation of principles, energy, rapture, tranquility, immersion, and equanimity. They’re called wise, no idiot because they’ve developed and cultivated these seven awakening factors.” 

%
\section*{{\suttatitleacronym SN 46.46}{\suttatitletranslation Poor }{\suttatitleroot Daliddasutta}}
\addcontentsline{toc}{section}{\tocacronym{SN 46.46} \toctranslation{Poor } \tocroot{Daliddasutta}}
\markboth{Poor }{Daliddasutta}
\extramarks{SN 46.46}{SN 46.46}

“Sir,\marginnote{1.1} they speak of someone who is ‘poor’. How is a poor person defined?” 

“Mendicant,\marginnote{1.3} they’re called poor because they haven’t developed and cultivated the seven awakening factors. What seven? The awakening factors of mindfulness, investigation of principles, energy, rapture, tranquility, immersion, and equanimity. They’re called poor because they haven’t developed and cultivated these seven awakening factors.” 

%
\section*{{\suttatitleacronym SN 46.47}{\suttatitletranslation Prosperous }{\suttatitleroot Adaliddasutta}}
\addcontentsline{toc}{section}{\tocacronym{SN 46.47} \toctranslation{Prosperous } \tocroot{Adaliddasutta}}
\markboth{Prosperous }{Adaliddasutta}
\extramarks{SN 46.47}{SN 46.47}

“Sir,\marginnote{1.1} they speak of someone who is ‘prosperous’. How is a prosperous person defined?” 

“Mendicant,\marginnote{1.3} they’re called prosperous because they’ve developed and cultivated the seven awakening factors. What seven? The awakening factors of mindfulness, investigation of principles, energy, rapture, tranquility, immersion, and equanimity. They’re called prosperous because they’ve developed and cultivated these seven awakening factors.” 

%
\section*{{\suttatitleacronym SN 46.48}{\suttatitletranslation The Sun }{\suttatitleroot Ādiccasutta}}
\addcontentsline{toc}{section}{\tocacronym{SN 46.48} \toctranslation{The Sun } \tocroot{Ādiccasutta}}
\markboth{The Sun }{Ādiccasutta}
\extramarks{SN 46.48}{SN 46.48}

“Mendicants,\marginnote{1.1} the dawn is the forerunner and precursor of the sunrise. 

In\marginnote{1.2} the same way, for a mendicant good friendship is the forerunner and precursor of the arising of the seven awakening factors. A mendicant with good friends can expect to develop and cultivate the seven awakening factors. 

And\marginnote{1.4} how does a mendicant with good friends develop and cultivate the seven awakening factors? It’s when a mendicant develops the awakening factors of mindfulness, investigation of principles, energy, rapture, tranquility, immersion, and equanimity, which rely on seclusion, fading away, and cessation, and ripen as letting go. That’s how a mendicant with good friends develops and cultivates the seven awakening factors.” 

%
\section*{{\suttatitleacronym SN 46.49}{\suttatitletranslation Interior }{\suttatitleroot Ajjhattikaṅgasutta}}
\addcontentsline{toc}{section}{\tocacronym{SN 46.49} \toctranslation{Interior } \tocroot{Ajjhattikaṅgasutta}}
\markboth{Interior }{Ajjhattikaṅgasutta}
\extramarks{SN 46.49}{SN 46.49}

“Taking\marginnote{1.1} into account interior factors, mendicants, I do not see a single one that gives rise to the seven awakening factors like proper attention. …” 

%
\section*{{\suttatitleacronym SN 46.50}{\suttatitletranslation Exterior }{\suttatitleroot Bāhiraṅgasutta}}
\addcontentsline{toc}{section}{\tocacronym{SN 46.50} \toctranslation{Exterior } \tocroot{Bāhiraṅgasutta}}
\markboth{Exterior }{Bāhiraṅgasutta}
\extramarks{SN 46.50}{SN 46.50}

“Taking\marginnote{1.1} into account exterior factors, mendicants, I do not see a single one that gives rise to the seven awakening factors like good friendship. …” 

%
\addtocontents{toc}{\let\protect\contentsline\protect\nopagecontentsline}
\chapter*{The Chapter on Discussion }
\addcontentsline{toc}{chapter}{\tocchapterline{The Chapter on Discussion }}
\addtocontents{toc}{\let\protect\contentsline\protect\oldcontentsline}

%
\section*{{\suttatitleacronym SN 46.51}{\suttatitletranslation Nourishing }{\suttatitleroot Āhārasutta}}
\addcontentsline{toc}{section}{\tocacronym{SN 46.51} \toctranslation{Nourishing } \tocroot{Āhārasutta}}
\markboth{Nourishing }{Āhārasutta}
\extramarks{SN 46.51}{SN 46.51}

At\marginnote{1.1} \textsanskrit{Sāvatthī}. 

“Mendicants,\marginnote{1.2} I will teach you what fuels and what starves the five hindrances and the seven awakening factors. Listen … 

And\marginnote{1.4} what fuels the arising of sensual desire, or, when it has arisen, makes it increase and grow? There is the feature of beauty. Frequent improper attention to that fuels the arising of sensual desire, or, when it has arisen, makes it increase and grow. 

And\marginnote{2.1} what fuels the arising of ill will, or, when it has arisen, makes it increase and grow? There is the feature of harshness. Frequent improper attention to that fuels the arising of ill will, or, when it has arisen, makes it increase and grow. 

And\marginnote{3.1} what fuels the arising of dullness and drowsiness, or, when they have arisen, makes them increase and grow? There is discontent, sloth, yawning, sleepiness after eating, and mental sluggishness. Frequent improper attention to that fuels the arising of dullness and drowsiness, or, when they have arisen, makes them increase and grow. 

And\marginnote{4.1} what fuels the arising of restlessness and remorse, or, when they have arisen, makes them increase and grow? There is the unsettled mind. Frequent improper attention to that fuels the arising of restlessness and remorse, or, when they have arisen, makes them increase and grow. 

And\marginnote{5.1} what fuels the arising of doubt, or, when it has arisen, makes it increase and grow? There are things that are grounds for doubt. Frequent improper attention to them fuels the arising of doubt, or, when it has arisen, makes it increase and grow. 

And\marginnote{6.1} what fuels the arising of the awakening factor of mindfulness, or, when it has arisen, fully develops it? There are things that are grounds for the awakening factor of mindfulness. Frequent proper attention to them fuels the arising of the awakening factor of mindfulness, or, when it has arisen, fully develops it. 

And\marginnote{7.1} what fuels the arising of the awakening factor of investigation of principles, or, when it has arisen, fully develops it? There are qualities that are skillful and unskillful, blameworthy and blameless, inferior and superior, and those on the side of dark and bright. Frequent proper attention to them fuels the arising of the awakening factor of investigation of principles, or, when it has arisen, fully develops it. 

And\marginnote{8.1} what fuels the arising of the awakening factor of energy, or, when it has arisen, fully develops it? There are the elements of initiative, persistence, and exertion. Frequent proper attention to them fuels the arising of the awakening factor of energy, or, when it has arisen, fully develops it. 

And\marginnote{9.1} what fuels the arising of the awakening factor of rapture, or, when it has arisen, fully develops it? There are things that are grounds for the awakening factor of rapture. Frequent proper attention to them fuels the arising of the awakening factor of rapture, or, when it has arisen, fully develops it. 

And\marginnote{10.1} what fuels the arising of the awakening factor of tranquility, or, when it has arisen, fully develops it? There is tranquility of the body and of the mind. Frequent proper attention to that fuels the arising of the awakening factor of tranquility, or, when it has arisen, fully develops it. 

And\marginnote{11.1} what fuels the arising of the awakening factor of immersion, or, when it has arisen, fully develops it? There are things that are the foundation of serenity and freedom from distraction. Frequent proper attention to them fuels the arising of the awakening factor of immersion, or, when it has arisen, fully develops it. 

And\marginnote{12.1} what fuels the arising of the awakening factor of equanimity, or, when it has arisen, fully develops it? There are things that are grounds for the awakening factor of equanimity. Frequent proper attention to them fuels the arising of the awakening factor of equanimity, or, when it has arisen, fully develops it. 

And\marginnote{13.1} what starves the arising of sensual desire, or, when it has arisen, starves its increase and growth? There is the feature of ugliness. Frequent proper attention to that starves the arising of sensual desire, or, when it has arisen, starves its increase and growth. 

And\marginnote{14.1} what starves the arising of ill will, or, when it has arisen, starves its increase and growth? There is the heart’s release by love. Frequent proper attention to that starves the arising of ill will, or, when it has arisen, starves its increase and growth. 

And\marginnote{15.1} what starves the arising of dullness and drowsiness, or, when they have arisen, starves their increase and growth? There are the elements of initiative, persistence, and exertion. Frequent proper attention to them starves the arising of dullness and drowsiness, or, when they have arisen, starves their increase and growth. 

And\marginnote{16.1} what starves the arising of restlessness and remorse, or, when they have arisen, starves their increase and growth? There is the settled mind. Frequent proper attention to that starves the arising of restlessness and remorse, or, when they have arisen, starves their increase and growth. 

And\marginnote{17.1} what starves the arising of doubt, or, when it has arisen, starves its increase and growth? There are qualities that are skillful and unskillful, blameworthy and blameless, inferior and superior, and those on the side of dark and bright. Frequent proper attention to them starves the arising of doubt, or, when it has arisen, starves its increase and growth. 

And\marginnote{18.1} what starves the arising of the awakening factor of mindfulness, or, when it has arisen, starves its full development? There are things that are grounds for the awakening factor of mindfulness. Not frequently focusing on them starves the arising of the awakening factor of mindfulness, or, when it has arisen, starves its full development. 

And\marginnote{19.1} what starves the arising of the awakening factor of investigation of principles, or, when it has arisen, starves its full development? There are qualities that are skillful and unskillful, blameworthy and blameless, inferior and superior, and those on the side of dark and bright. Not frequently focusing on them starves the arising of the awakening factor of investigation of principles, or, when it has arisen, starves its full development. 

And\marginnote{20.1} what starves the arising of the awakening factor of energy, or, when it has arisen, starves its full development? There are the elements of initiative, persistence, and exertion. Not frequently focusing on them starves the arising of the awakening factor of energy, or, when it has arisen, starves its full development. 

And\marginnote{21.1} what starves the arising of the awakening factor of rapture, or, when it has arisen, starves its full development? There are things that are grounds for the awakening factor of rapture. Not frequently focusing on them starves the arising of the awakening factor of rapture, or, when it has arisen, starves its full development. 

And\marginnote{22.1} what starves the arising of the awakening factor of tranquility, or, when it has arisen, starves its full development? There is tranquility of the body and of the mind. Not frequently attending to that starves the arising of the awakening factor of tranquility, or, when it has arisen, starves its full development. 

And\marginnote{23.1} what starves the arising of the awakening factor of immersion, or, when it has arisen, starves its full development? There are things that are the foundation of serenity and freedom from distraction. Not frequently focusing on them starves the arising of the awakening factor of immersion, or, when it has arisen, starves its full development. 

And\marginnote{24.1} what starves the arising of the awakening factor of equanimity, or, when it has arisen, starves its full development? There are things that are grounds for the awakening factor of equanimity. Not frequently focusing on them starves the arising of the awakening factor of equanimity, or, when it has arisen, starves its full development.” 

%
\section*{{\suttatitleacronym SN 46.52}{\suttatitletranslation Is There a Way? }{\suttatitleroot Pariyāyasutta}}
\addcontentsline{toc}{section}{\tocacronym{SN 46.52} \toctranslation{Is There a Way? } \tocroot{Pariyāyasutta}}
\markboth{Is There a Way? }{Pariyāyasutta}
\extramarks{SN 46.52}{SN 46.52}

Then\marginnote{1.1} several mendicants robed up in the morning and, taking their bowls and robes, entered \textsanskrit{Sāvatthī} for alms. Then it occurred to him, “It’s too early to wander for alms in \textsanskrit{Sāvatthī}. Why don’t we go to the monastery of the wanderers who follow other paths?” 

Then\marginnote{2.1} they went to the monastery of the wanderers who follow other paths, and exchanged greetings with the wanderers there. When the greetings and polite conversation were over, they sat down to one side. The wanderers said to them: 

“Reverends,\marginnote{3.1} the ascetic Gotama teaches his disciples like this: ‘Mendicants, please give up the five hindrances—corruptions of the heart that weaken wisdom—and truly develop the seven awakening factors.’ We too teach our disciples: ‘Reverends, please give up the five hindrances—corruptions of the heart that weaken wisdom—and truly develop the seven awakening factors.’ What, then, is the difference between the ascetic Gotama’s teaching and instruction and ours?” 

Those\marginnote{4.1} mendicants neither approved nor dismissed that statement of the wanderers who follow other paths. They got up from their seat, thinking: 

“We\marginnote{4.3} will learn the meaning of this statement from the Buddha himself.” Then, after the meal, when they returned from almsround, they went up to the Buddha, bowed, sat down to one side, and told him what had happened. 

“Mendicants,\marginnote{8.1} when wanderers who follow other paths say this, you should say to them: ‘But reverends, is there a way in which the five hindrances become ten and the seven awakening factors become fourteen?’ Questioned like this, the wanderers who follow other paths would be stumped, and, in addition, would get frustrated. Why is that? Because they’re out of their element. I don’t see anyone in this world—with its gods, \textsanskrit{Māras}, and \textsanskrit{Brahmās}, this population with its ascetics and brahmins, its gods and humans—who could provide a satisfying answer to these questions except for the Realized One or his disciple or someone who has heard it from them. 

And\marginnote{9.1} what is the way in which the five hindrances become ten? Sensual desire for what is internal is a hindrance; and sensual desire for what is external is also a hindrance. That’s how what is concisely referred to as ‘the hindrance of sensual desire’ becomes twofold. Ill will for what is internal is a hindrance; and ill will for what is external is also a hindrance. That’s how what is concisely referred to as ‘the hindrance of ill will’ becomes twofold. Dullness is a hindrance; and drowsiness is also a hindrance. That’s how what is concisely referred to as ‘the hindrance of dullness and drowsiness’ becomes twofold. Restlessness is a hindrance; and remorse is also a hindrance. That’s how what is concisely referred to as ‘the hindrance of restlessness and remorse’ becomes twofold. Doubt about internal things is a hindrance; and doubt about external things is also a hindrance. That’s how what is concisely referred to as ‘the hindrance of doubt’ becomes twofold. This is the way in which the five hindrances become ten. 

And\marginnote{10.1} what is the way in which the seven awakening factors become fourteen? Mindfulness of internal things is the awakening factor of mindfulness; and mindfulness of external things is also the awakening factor of mindfulness. That’s how what is concisely referred to as ‘the awakening factor of mindfulness’ becomes twofold. 

Investigating,\marginnote{11.1} exploring, and inquiring into internal things with wisdom is the awakening factor of investigation of principles; and investigating, exploring, and inquiring into external things with wisdom is also the awakening factor of investigation of principles. That’s how what is concisely referred to as ‘the awakening factor of investigation of principles’ becomes twofold. 

Physical\marginnote{12.1} energy is the awakening factor of energy; and mental energy is also the awakening factor of energy. That’s how what is concisely referred to as ‘the awakening factor of energy’ becomes twofold. 

Rapture\marginnote{13.1} while placing the mind and keeping it connected is the awakening factor of rapture; and rapture without placing the mind and keeping it connected is also the awakening factor of rapture. In this way what is concisely referred to as ‘the awakening factor of rapture’ becomes twofold. 

Physical\marginnote{14.1} tranquility is the awakening factor of tranquility; and mental tranquility is also the awakening factor of tranquility. In this way what is concisely referred to as ‘the awakening factor of tranquility’ becomes twofold. 

Immersion\marginnote{15.1} while placing the mind and keeping it connected is the awakening factor of immersion; and immersion without placing the mind and keeping it connected is also the awakening factor of immersion. In this way what is concisely referred to as ‘the awakening factor of immersion’ becomes twofold. 

Equanimity\marginnote{16.1} for internal things is the awakening factor of equanimity; and equanimity for external things is also the awakening factor of equanimity. In this way what is concisely referred to as ‘the awakening factor of equanimity’ becomes twofold. This is the way in which the seven awakening factors become fourteen.” 

%
\section*{{\suttatitleacronym SN 46.53}{\suttatitletranslation Fire }{\suttatitleroot Aggisutta}}
\addcontentsline{toc}{section}{\tocacronym{SN 46.53} \toctranslation{Fire } \tocroot{Aggisutta}}
\markboth{Fire }{Aggisutta}
\extramarks{SN 46.53}{SN 46.53}

Then\marginnote{1.1} several mendicants robed up in the morning and, taking their bowls and robes, entered \textsanskrit{Sāvatthī} for alms. 

(The\marginnote{1.2} same as the previous discourse.) 

“Mendicants,\marginnote{2.1} when wanderers who follow other paths say this, you should say to them: ‘Reverends, which awakening factors should not be developed when the mind is sluggish? And which awakening factors should be developed at that time? Which awakening factors should not be developed when the mind is restless? And which awakening factors should be developed at that time?’ Questioned like this, the wanderers who follow other paths would be stumped, and, in addition, would get frustrated. Why is that? Because they’re out of their element. 

I\marginnote{3.1} don’t see anyone in this world—with its gods, \textsanskrit{Māras}, and \textsanskrit{Brahmās}, this population with its ascetics and brahmins, its gods and humans—who could provide a satisfying answer to these questions except for the Realized One or his disciple or someone who has heard it from them. 

When\marginnote{4.1} the mind is sluggish, it’s the wrong time to develop the awakening factors of tranquility, immersion, and equanimity. Why is that? Because it’s hard to stimulate a sluggish mind with these things. 

Suppose\marginnote{5.1} someone wanted to make a small fire flare up. If they toss wet grass, cow-dung, and timber on it, spray it with water, and scatter dirt on it, could they make it flare up?” 

“No,\marginnote{5.4} sir.” 

“In\marginnote{6.1} the same way, when the mind is sluggish, it’s the wrong time to develop the awakening factors of tranquility, immersion, and equanimity. Why is that? Because it’s hard to stimulate a sluggish mind with these things. 

When\marginnote{7.1} the mind is sluggish, it’s the right time to develop the awakening factors of investigation of principles, energy, and rapture. Why is that? Because it’s easy to stimulate a sluggish mind with these things. 

Suppose\marginnote{8.1} someone wanted to make a small fire flare up. If they toss dry grass, cow-dung, and timber on it, blow on it, and don’t scatter dirt on it, could they make it flare up?” 

“Yes,\marginnote{8.4} sir.” 

“In\marginnote{9.1} the same way, when the mind is sluggish, it’s the right time to develop the awakening factors of investigation of principles, energy, and rapture. Why is that? Because it’s easy to stimulate a sluggish mind with these things. 

When\marginnote{10.1} the mind is restless, it’s the wrong time to develop the awakening factors of investigation of principles, energy, and rapture. Why is that? Because it’s hard to settle a restless mind with these things. 

Suppose\marginnote{11.1} someone wanted to extinguish a bonfire. If they toss dry grass, cow-dung, and timber on it, blow on it, and don’t scatter dirt on it, could they extinguish it?” 

“No,\marginnote{11.4} sir.” 

“In\marginnote{12.1} the same way, when the mind is restless, it’s the wrong time to develop the awakening factors of investigation of principles, energy, and rapture. Why is that? Because it’s hard to settle a restless mind with these things. 

When\marginnote{13.1} the mind is restless, it’s the right time to develop the awakening factors of tranquility, immersion, and equanimity. Why is that? Because it’s easy to settle a restless mind with these things. 

Suppose\marginnote{14.1} someone wanted to extinguish a bonfire. If they toss wet grass, cow-dung, and timber on it, spray it with water, and scatter dirt on it, could they extinguish it?” 

“Yes,\marginnote{14.4} sir.” 

“In\marginnote{15.1} the same way, when the mind is restless, it’s the right time to develop the awakening factors of tranquility, immersion, and equanimity. Why is that? Because it’s easy to settle a restless mind with these things. But mindfulness is always useful, I say.” 

%
\section*{{\suttatitleacronym SN 46.54}{\suttatitletranslation Full of Love }{\suttatitleroot Mettāsahagatasutta}}
\addcontentsline{toc}{section}{\tocacronym{SN 46.54} \toctranslation{Full of Love } \tocroot{Mettāsahagatasutta}}
\markboth{Full of Love }{Mettāsahagatasutta}
\extramarks{SN 46.54}{SN 46.54}

At\marginnote{1.1} one time the Buddha was staying in the land of the Koliyans, where they have a town called Haliddavasana. Then several mendicants robed up in the morning and, taking their bowls and robes, entered Haliddavasana for alms. Then it occurred to him, “It’s too early to wander for alms in Haliddavasana. Why don’t we go to the monastery of the wanderers who follow other paths?” 

Then\marginnote{2.1} they went to the monastery of the wanderers who follow other paths, and exchanged greetings with the wanderers there. When the greetings and polite conversation were over, they sat down to one side. The wanderers said to them: 

“Reverends,\marginnote{3.1} the ascetic Gotama teaches his disciples like this: ‘Come, mendicants, give up these five hindrances, corruptions of the heart that weaken wisdom, and meditate spreading a heart full of love to one direction, and to the second, and to the third, and to the fourth. In the same way above, below, across, everywhere, all around, spread a heart full of love to the whole world—abundant, expansive, limitless, free of enmity and ill will. Meditate spreading a heart full of compassion to one direction, and to the second, and to the third, and to the fourth. In the same way above, below, across, everywhere, all around, spread a heart full of compassion to the whole world—abundant, expansive, limitless, free of enmity and ill will. Meditate spreading a heart full of rejoicing to one direction, and to the second, and to the third, and to the fourth. In the same way above, below, across, everywhere, all around, spread a heart full of rejoicing to the whole world—abundant, expansive, limitless, free of enmity and ill will. Meditate spreading a heart full of equanimity to one direction, and to the second, and to the third, and to the fourth. In the same way above, below, across, everywhere, all around, they spread a heart full of equanimity to the whole world—abundant, expansive, limitless, free of enmity and ill will.’ 

We\marginnote{4.1} too teach our disciples in just the same way. What, then, is the difference between the ascetic Gotama’s teaching and instruction and ours?” 

Those\marginnote{5.1} mendicants neither approved nor dismissed that statement of the wanderers who follow other paths. They got up from their seat, thinking: 

“We\marginnote{5.3} will learn the meaning of this statement from the Buddha himself.” Then, after the meal, when they returned from almsround, they went up to the Buddha, bowed, sat down to one side, and told him what had happened. 

“Mendicants,\marginnote{11.1} when wanderers who follow other paths say this, you should say to them: ‘But reverends, how is the heart’s release by love developed? What is its destination, apex, fruit, and end? How is the heart’s release by compassion developed? What is its destination, apex, fruit, and end? How is the heart’s release by rejoicing developed? What is its destination, apex, fruit, and end? How is the heart’s release by equanimity developed? What is its destination, apex, fruit, and end?’ Questioned like this, the wanderers who follow other paths would be stumped, and, in addition, would get frustrated. Why is that? Because they’re out of their element. I don’t see anyone in this world—with its gods, \textsanskrit{Māras}, and \textsanskrit{Brahmās}, this population with its ascetics and brahmins, its gods and humans—who could provide a satisfying answer to these questions except for the Realized One or his disciple or someone who has heard it from them. 

And\marginnote{12.1} how is the heart’s release by love developed? What is its destination, apex, fruit, and end? It’s when a mendicant develops the heart’s release by love together with the awakening factors of mindfulness, investigation of principles, energy, rapture, tranquility, immersion, and equanimity, which rely on seclusion, fading away, and cessation, and ripen as letting go. If they wish: ‘May I meditate perceiving the repulsive in the unrepulsive,’ that’s what they do. If they wish: ‘May I meditate perceiving the unrepulsive in the repulsive,’ that’s what they do. If they wish: ‘May I meditate perceiving the repulsive in the unrepulsive and the repulsive,’ that’s what they do. If they wish: ‘May I meditate perceiving the unrepulsive in the repulsive and the unrepulsive,’ that’s what they do. If they wish: ‘May I meditate staying equanimous, mindful and aware, rejecting both the repulsive and the unrepulsive,’ that’s what they do. The apex of the heart’s release by love is the beautiful, I say, for a mendicant who has not penetrated to a higher freedom. 

And\marginnote{13.1} how is the heart’s release by compassion developed? What is its destination, apex, fruit, and end? It’s when a mendicant develops the heart’s release by compassion together with the awakening factors of mindfulness, investigation of principles, energy, rapture, tranquility, immersion, and equanimity, which rely on seclusion, fading away, and cessation, and ripen as letting go. If they wish: ‘May I meditate perceiving the repulsive in the unrepulsive,’ that’s what they do. … If they wish: ‘May I meditate staying equanimous, mindful and aware, rejecting both the repulsive and the unrepulsive,’ that’s what they do. Or else, going totally beyond perceptions of form, with the ending of perceptions of impingement, not focusing on perceptions of diversity, aware that ‘space is infinite’, they enter and remain in the dimension of infinite space. The apex of the heart’s release by compassion is the dimension of infinite space, I say, for a mendicant who has not penetrated to a higher freedom. 

And\marginnote{14.1} how is the heart’s release by rejoicing developed? What is its destination, apex, fruit, and end? It’s when a mendicant develops the heart’s release by rejoicing together with the awakening factors of mindfulness, investigation of principles, energy, rapture, tranquility, immersion, and equanimity, which rely on seclusion, fading away, and cessation, and ripen as letting go. If they wish: ‘May I meditate perceiving the repulsive in the unrepulsive,’ that’s what they do. … If they wish: ‘May I meditate staying equanimous, mindful and aware, rejecting both the repulsive and the unrepulsive,’ that’s what they do. Or else, going totally beyond the dimension of infinite space, aware that ‘consciousness is infinite’, they enter and remain in the dimension of infinite consciousness. The apex of the heart’s release by rejoicing is the dimension of infinite consciousness, I say, for a mendicant who has not penetrated to a higher freedom. 

And\marginnote{15.1} how is the heart’s release by equanimity developed? What is its destination, apex, fruit, and end? It’s when a mendicant develops the heart’s release by equanimity together with the awakening factors of mindfulness, investigation of principles, energy, rapture, tranquility, immersion, and equanimity, which rely on seclusion, fading away, and cessation, and ripen as letting go. If they wish: ‘May I meditate perceiving the repulsive in the unrepulsive,’ that’s what they do. If they wish: ‘May I meditate perceiving the unrepulsive in the repulsive,’ that’s what they do. If they wish: ‘May I meditate perceiving the repulsive in the unrepulsive and the repulsive,’ that’s what they do. If they wish: ‘May I meditate perceiving the unrepulsive in the repulsive and the unrepulsive,’ that’s what they do. If they wish: ‘May I meditate staying equanimous, mindful and aware, rejecting both the repulsive and the unrepulsive,’ that’s what they do. Or else, going totally beyond the dimension of infinite consciousness, aware that ‘there is nothing at all’, they enter and remain in the dimension of nothingness. The apex of the heart’s release by equanimity is the dimension of nothingness, I say, for a mendicant who has not penetrated to a higher freedom.” 

%
\section*{{\suttatitleacronym SN 46.55}{\suttatitletranslation With Saṅgārava }{\suttatitleroot Saṅgāravasutta}}
\addcontentsline{toc}{section}{\tocacronym{SN 46.55} \toctranslation{With Saṅgārava } \tocroot{Saṅgāravasutta}}
\markboth{With Saṅgārava }{Saṅgāravasutta}
\extramarks{SN 46.55}{SN 46.55}

At\marginnote{1.1} \textsanskrit{Sāvatthī}. 

Then\marginnote{1.2} \textsanskrit{Saṅgārava} the brahmin went up to the Buddha, and exchanged greetings with him. When the greetings and polite conversation were over, he sat down to one side and said to the Buddha: 

“What\marginnote{2.1} is the cause, Master Gotama, what is the reason why sometimes even hymns that are long-practiced don’t spring to mind, let alone those that are not practiced? And why is it that sometimes even hymns that are long-unpracticed spring to mind, let alone those that are practiced?” 

“Brahmin,\marginnote{3.1} there’s a time when your heart is overcome and mired in sensual desire and you don’t truly understand the escape from sensual desire that has arisen. At that time you don’t truly know or see what is good for yourself, good for another, or good for both. Even hymns that are long-practiced don’t spring to mind, let alone those that are not practiced. 

Suppose\marginnote{4.1} there was a bowl of water that was mixed with dye such as red lac, turmeric, indigo, or rose madder. Even a person with good eyesight checking their own reflection wouldn’t truly know it or see it. 

In\marginnote{4.3} the same way, when your heart is overcome and mired in sensual desire … Even hymns that are long-practiced don’t spring to mind, let alone those that are not practiced. 

Furthermore,\marginnote{5.1} when your heart is overcome and mired in ill will … Even hymns that are long-practiced don’t spring to mind, let alone those that are not practiced. 

Suppose\marginnote{6.1} there was a bowl of water that was heated by fire, boiling and bubbling. Even a person with good eyesight checking their own reflection wouldn’t truly know it or see it. 

In\marginnote{6.3} the same way, when your heart is overcome and mired in ill will … Even hymns that are long-practiced don’t spring to mind, let alone those that are not practiced. 

Furthermore,\marginnote{7.1} when your heart is overcome and mired in dullness and drowsiness … Even hymns that are long-practiced don’t spring to mind, let alone those that are not practiced. 

Suppose\marginnote{8.1} there was a bowl of water overgrown with moss and aquatic plants. Even a person with good eyesight checking their own reflection wouldn’t truly know it or see it. 

In\marginnote{8.3} the same way, when your heart is overcome and mired in dullness and drowsiness … Even hymns that are long-practiced don’t spring to mind, let alone those that are not practiced. 

Furthermore,\marginnote{9.1} when your heart is overcome and mired in restlessness and remorse … Even hymns that are long-practiced don’t spring to mind, let alone those that are not practiced. 

Suppose\marginnote{10.1} there was a bowl of water stirred by the wind, churning, swirling, and rippling. Even a person with good eyesight checking their own reflection wouldn’t truly know it or see it. 

In\marginnote{10.3} the same way, when your heart is overcome and mired in restlessness and remorse … Even hymns that are long-practiced don’t spring to mind, let alone those that are not practiced. 

Furthermore,\marginnote{11.1} when your heart is overcome and mired in doubt … Even hymns that are long-practiced don’t spring to mind, let alone those that are not practiced. 

Suppose\marginnote{12.1} there was a bowl of water that was cloudy, murky, and muddy, hidden in the darkness. Even a person with good eyesight checking their own reflection wouldn’t truly know it or see it. 

In\marginnote{12.3} the same way, there’s a time when your heart is overcome and mired in doubt and you don’t truly understand the escape from doubt that has arisen. At that time you don’t truly know or see what is good for yourself, good for another, or good for both. Even hymns that are long-practiced don’t spring to mind, let alone those that are not practiced. This is the cause, brahmin, this is the reason why sometimes even hymns that are long-practiced don’t spring to mind, let alone those that are not practiced. 

There’s\marginnote{13.1} a time when your heart is not overcome and mired in sensual desire and you truly understand the escape from sensual desire that has arisen. At that time you truly know and see what is good for yourself, good for another, and good for both. Even hymns that are long-unpracticed spring to mind, let alone those that are practiced. 

Suppose\marginnote{14.1} there was a bowl of water that was not mixed with dye such as red lac, turmeric, indigo, or rose madder. A person with good eyesight checking their own reflection would truly know it and see it. 

In\marginnote{14.3} the same way, when your heart is not overcome and mired in sensual desire … Even hymns that are long-unpracticed spring to mind, let alone those that are practiced. 

Furthermore,\marginnote{15.1} when your heart is not overcome and mired in ill will … Even hymns that are long-unpracticed spring to mind, let alone those that are practiced. 

Suppose\marginnote{16.1} there is a bowl of water that is not heated by a fire, boiling and bubbling. A person with good eyesight checking their own reflection would truly know it and see it. 

In\marginnote{16.3} the same way, when your heart is not overcome and mired in ill will … Even hymns that are long-unpracticed spring to mind, let alone those that are practiced. 

Furthermore,\marginnote{17.1} when your heart is not overcome and mired in dullness and drowsiness … Even hymns that are long-unpracticed spring to mind, let alone those that are practiced. 

Suppose\marginnote{18.1} there is a bowl of water that is not overgrown with moss and aquatic plants. A person with good eyesight checking their own reflection would truly know it and see it. 

In\marginnote{18.3} the same way, when your heart is not overcome and mired in dullness and drowsiness … Even hymns that are long-unpracticed spring to mind, let alone those that are practiced. 

Furthermore,\marginnote{19.1} when your heart is not overcome and mired in restlessness and remorse … Even hymns that are long-unpracticed spring to mind, let alone those that are practiced. 

Suppose\marginnote{20.1} there is a bowl of water that is not stirred by the wind, churning, swirling, and rippling. A person with good eyesight checking their own reflection would truly know it and see it. 

In\marginnote{20.3} the same way, when your heart is not overcome and mired in restlessness and remorse … Even hymns that are long-unpracticed spring to mind, let alone those that are practiced. 

Furthermore,\marginnote{21.1} when your heart is not overcome and mired in doubt … Even hymns that are long-unpracticed spring to mind, let alone those that are practiced. 

Suppose\marginnote{22.1} there was a bowl of water that was transparent, clear, and unclouded, brought into the light. A person with good eyesight checking their own reflection would truly know it and see it. 

In\marginnote{22.3} the same way, there’s a time when your heart is not overcome and mired in doubt and you truly understand the escape from doubt that has arisen. At that time you truly know and see what is good for yourself, good for another, and good for both. Even hymns that are long-unpracticed spring to mind, let alone those that are practiced. This is the cause, brahmin, this is the reason why sometimes even hymns that are long-unpracticed do spring to mind, let alone those that are practiced. 

These\marginnote{23.1} seven awakening factors are not obstacles, hindrances, or corruptions of the mind. When developed and cultivated they lead to the realization of the fruit of knowledge and freedom. What seven? The awakening factors of mindfulness, investigation of principles, energy, rapture, tranquility, immersion, and equanimity. These seven awakening factors are not obstacles, hindrances, or corruptions of the mind. When developed and cultivated they lead to the realization of the fruit of knowledge and freedom.” 

When\marginnote{24.1} he said this, \textsanskrit{Saṅgārava} said to the Buddha, “Excellent, Master Gotama! … From this day forth, may Master Gotama remember me as a lay follower who has gone for refuge for life.” 

%
\section*{{\suttatitleacronym SN 46.56}{\suttatitletranslation A Place Without Fear }{\suttatitleroot Abhayasutta}}
\addcontentsline{toc}{section}{\tocacronym{SN 46.56} \toctranslation{A Place Without Fear } \tocroot{Abhayasutta}}
\markboth{A Place Without Fear }{Abhayasutta}
\extramarks{SN 46.56}{SN 46.56}

\scevam{So\marginnote{1.1} I have heard. }At one time the Buddha was staying near \textsanskrit{Rājagaha}, on the Vulture’s Peak Mountain. Then Prince Abhaya went up to the Buddha, bowed, sat down to one side, and said to him: 

“Sir,\marginnote{1.4} \textsanskrit{Pūraṇa} Kassapa says this: ‘There is no cause or reason for not knowing and not seeing. Not knowing and not seeing have no cause or reason. There is no cause or reason for knowing and seeing. Knowing and seeing have no cause or reason.’ What does the Buddha say about this?” 

“Prince,\marginnote{1.10} there are causes and reasons for not knowing and not seeing. Not knowing and not seeing have causes and reasons. There are causes and reasons for knowing and seeing. Knowing and seeing have causes and reasons.” 

“But\marginnote{2.1} sir, what is the cause and reason for not knowing and not seeing? How do not knowing and not seeing have causes and reasons?” 

“There’s\marginnote{2.3} a time when the heart is overcome and mired in sensual desire, without truly knowing and seeing the escape from sensual desire that has arisen. This is a cause and reason for not knowing and not seeing. And this is how not knowing and not seeing have causes and reasons. 

Furthermore,\marginnote{3.1} there’s a time when the heart is overcome and mired in ill will … dullness and drowsiness … restlessness and remorse … doubt, without truly knowing and seeing the escape from doubt that has arisen. This is a cause and reason for not knowing and not seeing. And this is how not knowing and not seeing have causes and reasons.” 

“Sir,\marginnote{4.1} what is the name of this exposition of the teaching?” 

“These\marginnote{4.2} are called the ‘hindrances’, prince.” 

“Indeed,\marginnote{4.3} Blessed One, these are hindrances! Indeed, Holy One, these are hindrances! Overcome by even a single hindrance you wouldn’t truly know or see, let alone all five hindrances. 

But\marginnote{5.1} sir, what is the cause and reason for knowing and seeing? How do knowing and seeing have causes and reasons?” 

“It’s\marginnote{5.3} when a mendicant develops the awakening factor of mindfulness, which relies on seclusion, fading away, and cessation, and ripens as letting go. They truly know and see with a mind that has developed the awakening factor of mindfulness. This is a cause and reason for knowing and seeing. And this is how knowing and seeing have causes and reasons. 

Furthermore,\marginnote{6.1} a mendicant develops the awakening factor of investigation of principles … energy … rapture … tranquility … immersion … equanimity, which rely on seclusion, fading away, and cessation, and ripen as letting go. They truly know and see with a mind that has developed the awakening factor of equanimity. This is a cause and reason for knowing and seeing. And this is how knowing and seeing have causes and reasons.” 

“Sir,\marginnote{7.1} what is the name of this exposition of the teaching?” 

“These\marginnote{7.2} are called the ‘awakening factors’, prince.” 

“Indeed,\marginnote{7.3} Blessed One, these are awakening factors! Indeed, Holy One, these are awakening factors! Endowed with even a single awakening factor you would truly know and see, let alone all seven awakening factors. When climbing Mount Vulture’s Peak I became fatigued in body and mind. But this has now faded away. And I’ve comprehended the teaching.” 

%
\addtocontents{toc}{\let\protect\contentsline\protect\nopagecontentsline}
\chapter*{The Chapter on Breathing }
\addcontentsline{toc}{chapter}{\tocchapterline{The Chapter on Breathing }}
\addtocontents{toc}{\let\protect\contentsline\protect\oldcontentsline}

%
\section*{{\suttatitleacronym SN 46.57}{\suttatitletranslation A Skeleton }{\suttatitleroot Aṭṭhikamahapphalasutta}}
\addcontentsline{toc}{section}{\tocacronym{SN 46.57} \toctranslation{A Skeleton } \tocroot{Aṭṭhikamahapphalasutta}}
\markboth{A Skeleton }{Aṭṭhikamahapphalasutta}
\extramarks{SN 46.57}{SN 46.57}

At\marginnote{1.1} \textsanskrit{Sāvatthī}. 

“Mendicants,\marginnote{1.2} when the perception of a skeleton is developed and cultivated it’s very fruitful and beneficial. How so? It’s when a mendicant develops the perception of a skeleton together with the awakening factors of mindfulness, investigation of principles, energy, rapture, tranquility, immersion, and equanimity, which rely on seclusion, fading away, and cessation, and ripen as letting go. That’s how the perception of a skeleton, when developed and cultivated, is very fruitful and beneficial.” 

“When\marginnote{2.1} the perception of a skeleton is developed and cultivated you can expect one of two results: enlightenment in the present life, or if there’s something left over, non-return. How so?…” 

“Mendicants,\marginnote{3.1} when the perception of a skeleton is developed and cultivated it leads to great benefit. How so?…” 

“Mendicants,\marginnote{4.1} when the perception of a skeleton is developed and cultivated it leads to great sanctuary. How so?…” 

“Mendicants,\marginnote{5.1} when the perception of a skeleton is developed and cultivated it leads to great inspiration. How so?…” 

“Mendicants,\marginnote{6.1} when the perception of a skeleton is developed and cultivated it leads to dwelling in great comfort. How so?…” 

%
\section*{{\suttatitleacronym SN 46.58}{\suttatitletranslation Worm-Infested }{\suttatitleroot Puḷavakasutta}}
\addcontentsline{toc}{section}{\tocacronym{SN 46.58} \toctranslation{Worm-Infested } \tocroot{Puḷavakasutta}}
\markboth{Worm-Infested }{Puḷavakasutta}
\extramarks{SN 46.58}{SN 46.58}

“Mendicants,\marginnote{1.1} when the perception of a worm-infested corpse is developed and cultivated it’s very fruitful and beneficial. …” 

%
\section*{{\suttatitleacronym SN 46.59}{\suttatitletranslation Livid }{\suttatitleroot Vinīlakasutta}}
\addcontentsline{toc}{section}{\tocacronym{SN 46.59} \toctranslation{Livid } \tocroot{Vinīlakasutta}}
\markboth{Livid }{Vinīlakasutta}
\extramarks{SN 46.59}{SN 46.59}

“Mendicants,\marginnote{1.1} when the perception of a livid corpse is developed and cultivated it’s very fruitful and beneficial. …” 

%
\section*{{\suttatitleacronym SN 46.60}{\suttatitletranslation Split Open }{\suttatitleroot Vicchiddakasutta}}
\addcontentsline{toc}{section}{\tocacronym{SN 46.60} \toctranslation{Split Open } \tocroot{Vicchiddakasutta}}
\markboth{Split Open }{Vicchiddakasutta}
\extramarks{SN 46.60}{SN 46.60}

“Mendicants,\marginnote{1.1} when the perception of a split open corpse is developed and cultivated it’s very fruitful and beneficial. …” 

%
\section*{{\suttatitleacronym SN 46.61}{\suttatitletranslation Bloated }{\suttatitleroot Uddhumātakasutta}}
\addcontentsline{toc}{section}{\tocacronym{SN 46.61} \toctranslation{Bloated } \tocroot{Uddhumātakasutta}}
\markboth{Bloated }{Uddhumātakasutta}
\extramarks{SN 46.61}{SN 46.61}

“Mendicants,\marginnote{1.1} when the perception of a bloated corpse is developed and cultivated it’s very fruitful and beneficial. …” 

%
\section*{{\suttatitleacronym SN 46.62}{\suttatitletranslation Love }{\suttatitleroot Mettāsutta}}
\addcontentsline{toc}{section}{\tocacronym{SN 46.62} \toctranslation{Love } \tocroot{Mettāsutta}}
\markboth{Love }{Mettāsutta}
\extramarks{SN 46.62}{SN 46.62}

“Mendicants,\marginnote{1.1} when love is developed and cultivated it’s very fruitful and beneficial. …” 

%
\section*{{\suttatitleacronym SN 46.63}{\suttatitletranslation Compassion }{\suttatitleroot Karuṇāsutta}}
\addcontentsline{toc}{section}{\tocacronym{SN 46.63} \toctranslation{Compassion } \tocroot{Karuṇāsutta}}
\markboth{Compassion }{Karuṇāsutta}
\extramarks{SN 46.63}{SN 46.63}

“Mendicants,\marginnote{1.1} when compassion is developed and cultivated it’s very fruitful and beneficial. …” 

%
\section*{{\suttatitleacronym SN 46.64}{\suttatitletranslation Rejoicing }{\suttatitleroot Muditāsutta}}
\addcontentsline{toc}{section}{\tocacronym{SN 46.64} \toctranslation{Rejoicing } \tocroot{Muditāsutta}}
\markboth{Rejoicing }{Muditāsutta}
\extramarks{SN 46.64}{SN 46.64}

“Mendicants,\marginnote{1.1} when rejoicing is developed and cultivated it’s very fruitful and beneficial. …” 

%
\section*{{\suttatitleacronym SN 46.65}{\suttatitletranslation Equanimity }{\suttatitleroot Upekkhāsutta}}
\addcontentsline{toc}{section}{\tocacronym{SN 46.65} \toctranslation{Equanimity } \tocroot{Upekkhāsutta}}
\markboth{Equanimity }{Upekkhāsutta}
\extramarks{SN 46.65}{SN 46.65}

“Mendicants,\marginnote{1.1} when equanimity is developed and cultivated it’s very fruitful and beneficial. …” 

%
\section*{{\suttatitleacronym SN 46.66}{\suttatitletranslation Breathing }{\suttatitleroot Ānāpānasutta}}
\addcontentsline{toc}{section}{\tocacronym{SN 46.66} \toctranslation{Breathing } \tocroot{Ānāpānasutta}}
\markboth{Breathing }{Ānāpānasutta}
\extramarks{SN 46.66}{SN 46.66}

“Mendicants,\marginnote{1.1} when mindfulness of breathing is developed and cultivated it’s very fruitful and beneficial. …” 

%
\addtocontents{toc}{\let\protect\contentsline\protect\nopagecontentsline}
\chapter*{The Chapter on Cessation }
\addcontentsline{toc}{chapter}{\tocchapterline{The Chapter on Cessation }}
\addtocontents{toc}{\let\protect\contentsline\protect\oldcontentsline}

%
\section*{{\suttatitleacronym SN 46.67}{\suttatitletranslation Ugliness }{\suttatitleroot Asubhasutta}}
\addcontentsline{toc}{section}{\tocacronym{SN 46.67} \toctranslation{Ugliness } \tocroot{Asubhasutta}}
\markboth{Ugliness }{Asubhasutta}
\extramarks{SN 46.67}{SN 46.67}

“Mendicants,\marginnote{1.1} when the perception of ugliness is developed and cultivated it’s very fruitful and beneficial. …” 

%
\section*{{\suttatitleacronym SN 46.68}{\suttatitletranslation Death }{\suttatitleroot Maraṇasutta}}
\addcontentsline{toc}{section}{\tocacronym{SN 46.68} \toctranslation{Death } \tocroot{Maraṇasutta}}
\markboth{Death }{Maraṇasutta}
\extramarks{SN 46.68}{SN 46.68}

“Mendicants,\marginnote{1.1} when the perception of death is developed and cultivated it’s very fruitful and beneficial. …” 

%
\section*{{\suttatitleacronym SN 46.69}{\suttatitletranslation Repulsiveness of Food }{\suttatitleroot Āhārepaṭikūlasutta}}
\addcontentsline{toc}{section}{\tocacronym{SN 46.69} \toctranslation{Repulsiveness of Food } \tocroot{Āhārepaṭikūlasutta}}
\markboth{Repulsiveness of Food }{Āhārepaṭikūlasutta}
\extramarks{SN 46.69}{SN 46.69}

“Mendicants,\marginnote{1.1} when the perception of repulsiveness of food is developed and cultivated it’s very fruitful and beneficial. …” 

%
\section*{{\suttatitleacronym SN 46.70}{\suttatitletranslation Dissatisfaction }{\suttatitleroot Anabhiratisutta}}
\addcontentsline{toc}{section}{\tocacronym{SN 46.70} \toctranslation{Dissatisfaction } \tocroot{Anabhiratisutta}}
\markboth{Dissatisfaction }{Anabhiratisutta}
\extramarks{SN 46.70}{SN 46.70}

“Mendicants,\marginnote{1.1} when the perception of dissatisfaction with the whole world is developed and cultivated it’s very fruitful and beneficial. …” 

%
\section*{{\suttatitleacronym SN 46.71}{\suttatitletranslation Impermanence }{\suttatitleroot Aniccasutta}}
\addcontentsline{toc}{section}{\tocacronym{SN 46.71} \toctranslation{Impermanence } \tocroot{Aniccasutta}}
\markboth{Impermanence }{Aniccasutta}
\extramarks{SN 46.71}{SN 46.71}

“Mendicants,\marginnote{1.1} when the perception of impermanence is developed and cultivated it’s very fruitful and beneficial. …” 

%
\section*{{\suttatitleacronym SN 46.72}{\suttatitletranslation Suffering in Impermanence }{\suttatitleroot Dukkhasutta}}
\addcontentsline{toc}{section}{\tocacronym{SN 46.72} \toctranslation{Suffering in Impermanence } \tocroot{Dukkhasutta}}
\markboth{Suffering in Impermanence }{Dukkhasutta}
\extramarks{SN 46.72}{SN 46.72}

“Mendicants,\marginnote{1.1} when the perception of suffering in impermanence is developed and cultivated it’s very fruitful and beneficial. …” 

%
\section*{{\suttatitleacronym SN 46.73}{\suttatitletranslation Not-Self in Suffering }{\suttatitleroot Anattasutta}}
\addcontentsline{toc}{section}{\tocacronym{SN 46.73} \toctranslation{Not-Self in Suffering } \tocroot{Anattasutta}}
\markboth{Not-Self in Suffering }{Anattasutta}
\extramarks{SN 46.73}{SN 46.73}

“Mendicants,\marginnote{1.1} when the perception of not-self in suffering is developed and cultivated it’s very fruitful and beneficial. …” 

%
\section*{{\suttatitleacronym SN 46.74}{\suttatitletranslation Giving Up }{\suttatitleroot Pahānasutta}}
\addcontentsline{toc}{section}{\tocacronym{SN 46.74} \toctranslation{Giving Up } \tocroot{Pahānasutta}}
\markboth{Giving Up }{Pahānasutta}
\extramarks{SN 46.74}{SN 46.74}

“Mendicants,\marginnote{1.1} when the perception of giving up is developed and cultivated it’s very fruitful and beneficial. …” 

%
\section*{{\suttatitleacronym SN 46.75}{\suttatitletranslation Dispassion }{\suttatitleroot Virāgasutta}}
\addcontentsline{toc}{section}{\tocacronym{SN 46.75} \toctranslation{Dispassion } \tocroot{Virāgasutta}}
\markboth{Dispassion }{Virāgasutta}
\extramarks{SN 46.75}{SN 46.75}

“Mendicants,\marginnote{1.1} when the perception of dispassion is developed and cultivated it’s very fruitful and beneficial. …” 

%
\section*{{\suttatitleacronym SN 46.76}{\suttatitletranslation Cessation }{\suttatitleroot Nirodhasutta}}
\addcontentsline{toc}{section}{\tocacronym{SN 46.76} \toctranslation{Cessation } \tocroot{Nirodhasutta}}
\markboth{Cessation }{Nirodhasutta}
\extramarks{SN 46.76}{SN 46.76}

“Mendicants,\marginnote{1.1} when the perception of cessation is developed and cultivated it’s very fruitful and beneficial. How so? It’s when a mendicant develops the perception of cessation together with the awakening factors of mindfulness, investigation of principles, energy, rapture, tranquility, immersion, and equanimity, which rely on seclusion, fading away, and cessation, and ripen as letting go. That’s how, when the perception of cessation is developed and cultivated, it’s very fruitful and beneficial. 

When\marginnote{2.1} the perception of cessation is developed and cultivated you can expect one of two results: enlightenment in the present life, or if there’s something left over, non-return. How so? It’s when a mendicant develops the perception of cessation together with the awakening factors of mindfulness, investigation of principles, energy, rapture, tranquility, immersion, and equanimity, which rely on seclusion, fading away, and cessation, and ripen as letting go. When the perception of cessation is developed and cultivated in this way you can expect one of two results: enlightenment in the present life, or if there’s something left over, non-return.” 

“The\marginnote{3.1} perception of cessation, when developed and cultivated, leads to great benefit … great sanctuary … great inspiration … great ease. How so? It’s when a mendicant develops the perception of cessation together with the awakening factors of mindfulness, investigation of principles, energy, rapture, tranquility, immersion, and equanimity, which rely on seclusion, fading away, and cessation, and ripen as letting go. That’s how the perception of cessation, when developed and cultivated, leads to great benefit … great sanctuary … great inspiration … great ease.” 

%
\addtocontents{toc}{\let\protect\contentsline\protect\nopagecontentsline}
\chapter*{The Chapter of Abbreviated Texts on the Ganges }
\addcontentsline{toc}{chapter}{\tocchapterline{The Chapter of Abbreviated Texts on the Ganges }}
\addtocontents{toc}{\let\protect\contentsline\protect\oldcontentsline}

%
\section*{{\suttatitleacronym SN 46.77–88}{\suttatitletranslation The Ganges River, Etc. }{\suttatitleroot Gaṅgāpeyyālavagga}}
\addcontentsline{toc}{section}{\tocacronym{SN 46.77–88} \toctranslation{The Ganges River, Etc. } \tocroot{Gaṅgāpeyyālavagga}}
\markboth{The Ganges River, Etc. }{Gaṅgāpeyyālavagga}
\extramarks{SN 46.77–88}{SN 46.77–88}

“Mendicants,\marginnote{1.1} the Ganges river slants, slopes, and inclines to the east. In the same way, a mendicant who develops and cultivates the seven awakening factors slants, slopes, and inclines to extinguishment. 

And\marginnote{1.3} how does a mendicant who develops the seven awakening factors slant, slope, and incline to extinguishment? It’s when a mendicant develops the awakening factors of mindfulness, investigation of principles, energy, rapture, tranquility, immersion, and equanimity, which rely on seclusion, fading away, and cessation, and ripen as letting go. That’s how a mendicant who develops and cultivates the seven awakening factors slants, slopes, and inclines to extinguishment.” 

(To\marginnote{1.7} be expanded for each of the different rivers as in SN 45.91–102.) 

%
\addtocontents{toc}{\let\protect\contentsline\protect\nopagecontentsline}
\chapter*{The Chapter on Diligence }
\addcontentsline{toc}{chapter}{\tocchapterline{The Chapter on Diligence }}
\addtocontents{toc}{\let\protect\contentsline\protect\oldcontentsline}

%
\section*{{\suttatitleacronym SN 46.89–98}{\suttatitletranslation A Realized One, Etc. }{\suttatitleroot Appamādavagga}}
\addcontentsline{toc}{section}{\tocacronym{SN 46.89–98} \toctranslation{A Realized One, Etc. } \tocroot{Appamādavagga}}
\markboth{A Realized One, Etc. }{Appamādavagga}
\extramarks{SN 46.89–98}{SN 46.89–98}

“Mendicants,\marginnote{1.1} the Realized One, the perfected one, the fully awakened Buddha, is said to be the best of all sentient beings—be they footless, with two feet, four feet, or many feet …” 

(To\marginnote{3.1} be expanded as in SN 45.139–148.) 

%
\addtocontents{toc}{\let\protect\contentsline\protect\nopagecontentsline}
\chapter*{The Chapter on Hard Work }
\addcontentsline{toc}{chapter}{\tocchapterline{The Chapter on Hard Work }}
\addtocontents{toc}{\let\protect\contentsline\protect\oldcontentsline}

%
\section*{{\suttatitleacronym SN 46.99–110}{\suttatitletranslation Hard Work, Etc. }{\suttatitleroot Balakaraṇīyavagga}}
\addcontentsline{toc}{section}{\tocacronym{SN 46.99–110} \toctranslation{Hard Work, Etc. } \tocroot{Balakaraṇīyavagga}}
\markboth{Hard Work, Etc. }{Balakaraṇīyavagga}
\extramarks{SN 46.99–110}{SN 46.99–110}

“Mendicants,\marginnote{1.1} all the hard work that gets done depends on the earth and is grounded on the earth. …” 

(To\marginnote{3.1} be expanded as in SN 45.149–160.) 

%
\addtocontents{toc}{\let\protect\contentsline\protect\nopagecontentsline}
\chapter*{The Chapter on Searches }
\addcontentsline{toc}{chapter}{\tocchapterline{The Chapter on Searches }}
\addtocontents{toc}{\let\protect\contentsline\protect\oldcontentsline}

%
\section*{{\suttatitleacronym SN 46.111–120}{\suttatitletranslation Searches, Etc. }{\suttatitleroot Esanāvagga}}
\addcontentsline{toc}{section}{\tocacronym{SN 46.111–120} \toctranslation{Searches, Etc. } \tocroot{Esanāvagga}}
\markboth{Searches, Etc. }{Esanāvagga}
\extramarks{SN 46.111–120}{SN 46.111–120}

“Mendicants,\marginnote{1.1} there are these three searches. What three? The search for sensual pleasures, the search for continued existence, and the search for a spiritual path. …” 

(To\marginnote{3.1} be expanded as in SN 45.161–170.) 

%
\addtocontents{toc}{\let\protect\contentsline\protect\nopagecontentsline}
\chapter*{The Chapter on Floods }
\addcontentsline{toc}{chapter}{\tocchapterline{The Chapter on Floods }}
\addtocontents{toc}{\let\protect\contentsline\protect\oldcontentsline}

%
\section*{{\suttatitleacronym SN 46.121–129}{\suttatitletranslation Floods }{\suttatitleroot Oghādisutta}}
\addcontentsline{toc}{section}{\tocacronym{SN 46.121–129} \toctranslation{Floods } \tocroot{Oghādisutta}}
\markboth{Floods }{Oghādisutta}
\extramarks{SN 46.121–129}{SN 46.121–129}

“Mendicants,\marginnote{1.1} there are these four floods. What four? The floods of sensuality, desire to be reborn, views, and ignorance.” (To be expanded as in SN 45.171–179.) 

%
\section*{{\suttatitleacronym SN 46.130}{\suttatitletranslation Higher Fetters }{\suttatitleroot Uddhambhāgiyasutta}}
\addcontentsline{toc}{section}{\tocacronym{SN 46.130} \toctranslation{Higher Fetters } \tocroot{Uddhambhāgiyasutta}}
\markboth{Higher Fetters }{Uddhambhāgiyasutta}
\extramarks{SN 46.130}{SN 46.130}

At\marginnote{1.1} \textsanskrit{Sāvatthī}. 

“Mendicants,\marginnote{1.2} there are five higher fetters. What five? Desire for rebirth in the realm of luminous form, desire for rebirth in the formless realm, conceit, restlessness, and ignorance. These are the five higher fetters. The seven awakening factors should be developed for the direct knowledge, complete understanding, finishing, and giving up of these five higher fetters. 

What\marginnote{1.7} seven? It’s when a mendicant develops the awakening factors of mindfulness, investigation of principles, energy, rapture, tranquility, immersion, and equanimity, which culminate in the removal of greed, hate, and delusion. …” 

“…\marginnote{1.10} which culminate, finish, and end in the deathless …” 

“…\marginnote{1.11} which culminate, finish, and end in extinguishment … 

The\marginnote{1.12} seven awakening factors should be developed for the direct knowledge, complete understanding, finishing, and giving up of these five higher fetters.” 

%
\addtocontents{toc}{\let\protect\contentsline\protect\nopagecontentsline}
\chapter*{Another Chapter of Abbreviated Texts on the Ganges, Etc. }
\addcontentsline{toc}{chapter}{\tocchapterline{Another Chapter of Abbreviated Texts on the Ganges, Etc. }}
\addtocontents{toc}{\let\protect\contentsline\protect\oldcontentsline}

%
\section*{{\suttatitleacronym SN 46.131–142}{\suttatitletranslation More on the Ganges River, Etc. }{\suttatitleroot Punagaṅgāpeyyālavagga}}
\addcontentsline{toc}{section}{\tocacronym{SN 46.131–142} \toctranslation{More on the Ganges River, Etc. } \tocroot{Punagaṅgāpeyyālavagga}}
\markboth{More on the Ganges River, Etc. }{Punagaṅgāpeyyālavagga}
\extramarks{SN 46.131–142}{SN 46.131–142}

(To\marginnote{3.1} be expanded as in SN 45.103–114, removal of greed version.) 

%
\addtocontents{toc}{\let\protect\contentsline\protect\nopagecontentsline}
\chapter*{Another Chapter on Diligence }
\addcontentsline{toc}{chapter}{\tocchapterline{Another Chapter on Diligence }}
\addtocontents{toc}{\let\protect\contentsline\protect\oldcontentsline}

%
\section*{{\suttatitleacronym SN 46.143–152}{\suttatitletranslation Another Series on a Realized One, Etc. }{\suttatitleroot Punaappamādavagga}}
\addcontentsline{toc}{section}{\tocacronym{SN 46.143–152} \toctranslation{Another Series on a Realized One, Etc. } \tocroot{Punaappamādavagga}}
\markboth{Another Series on a Realized One, Etc. }{Punaappamādavagga}
\extramarks{SN 46.143–152}{SN 46.143–152}

(To\marginnote{3.1} be expanded as in SN 45.139–148, removal of greed version.) 

%
\addtocontents{toc}{\let\protect\contentsline\protect\nopagecontentsline}
\chapter*{Another Chapter on Hard Work }
\addcontentsline{toc}{chapter}{\tocchapterline{Another Chapter on Hard Work }}
\addtocontents{toc}{\let\protect\contentsline\protect\oldcontentsline}

%
\section*{{\suttatitleacronym SN 46.153–164}{\suttatitletranslation Hard Work, Etc. }{\suttatitleroot Punabalakaraṇīyavagga}}
\addcontentsline{toc}{section}{\tocacronym{SN 46.153–164} \toctranslation{Hard Work, Etc. } \tocroot{Punabalakaraṇīyavagga}}
\markboth{Hard Work, Etc. }{Punabalakaraṇīyavagga}
\extramarks{SN 46.153–164}{SN 46.153–164}

(To\marginnote{3.1} be expanded as in SN 45.149–160, removal of greed version.) 

%
\addtocontents{toc}{\let\protect\contentsline\protect\nopagecontentsline}
\chapter*{Another Chapter on Searches }
\addcontentsline{toc}{chapter}{\tocchapterline{Another Chapter on Searches }}
\addtocontents{toc}{\let\protect\contentsline\protect\oldcontentsline}

%
\section*{{\suttatitleacronym SN 46.165–174}{\suttatitletranslation Another Series on Searches, Etc. }{\suttatitleroot Punaesanāvagga}}
\addcontentsline{toc}{section}{\tocacronym{SN 46.165–174} \toctranslation{Another Series on Searches, Etc. } \tocroot{Punaesanāvagga}}
\markboth{Another Series on Searches, Etc. }{Punaesanāvagga}
\extramarks{SN 46.165–174}{SN 46.165–174}

\scendsection{(To be expanded as in SN 45.161–170.) }

%
\addtocontents{toc}{\let\protect\contentsline\protect\nopagecontentsline}
\chapter*{Another Chapter on Floods }
\addcontentsline{toc}{chapter}{\tocchapterline{Another Chapter on Floods }}
\addtocontents{toc}{\let\protect\contentsline\protect\oldcontentsline}

%
\section*{{\suttatitleacronym SN 46.175–184}{\suttatitletranslation Another Series on Floods, Etc. }{\suttatitleroot Punaoghavagga}}
\addcontentsline{toc}{section}{\tocacronym{SN 46.175–184} \toctranslation{Another Series on Floods, Etc. } \tocroot{Punaoghavagga}}
\markboth{Another Series on Floods, Etc. }{Punaoghavagga}
\extramarks{SN 46.175–184}{SN 46.175–184}

\scendsection{(To be expanded as in SN 45.171–180.) }

(All\marginnote{3.1} should be expanded as in the chapter on removal of greed, hate, and delusion.) 

(The\marginnote{3.2} Linked Discourses on Awakening Factors should be expanded just as the Linked Discourses on the Path.) 

\scendsutta{The Linked Discourses on the Awakening Factors is the second section. }

%
\addtocontents{toc}{\let\protect\contentsline\protect\nopagecontentsline}
\part*{Linked Discourses on Mindfulness Meditation }
\addcontentsline{toc}{part}{Linked Discourses on Mindfulness Meditation }
\markboth{}{}
\addtocontents{toc}{\let\protect\contentsline\protect\oldcontentsline}

%
\addtocontents{toc}{\let\protect\contentsline\protect\nopagecontentsline}
\chapter*{The Chapter on Ambapālī the Courtesan }
\addcontentsline{toc}{chapter}{\tocchapterline{The Chapter on Ambapālī the Courtesan }}
\addtocontents{toc}{\let\protect\contentsline\protect\oldcontentsline}

%
\section*{{\suttatitleacronym SN 47.1}{\suttatitletranslation In Ambapālī’s Wood }{\suttatitleroot Ambapālisutta}}
\addcontentsline{toc}{section}{\tocacronym{SN 47.1} \toctranslation{In Ambapālī’s Wood } \tocroot{Ambapālisutta}}
\markboth{In Ambapālī’s Wood }{Ambapālisutta}
\extramarks{SN 47.1}{SN 47.1}

\scevam{So\marginnote{1.1} I have heard. }At one time the Buddha was staying near \textsanskrit{Vesālī}, in \textsanskrit{Ambapālī}’s Wood. There the Buddha addressed the mendicants, “Mendicants!” 

“Venerable\marginnote{1.5} sir,” they replied. The Buddha said this: 

“Mendicants,\marginnote{2.1} the four kinds of mindfulness meditation are the path to convergence. They are in order to purify sentient beings, to get past sorrow and crying, to make an end of pain and sadness, to end the cycle of suffering, and to realize extinguishment. What four? 

It’s\marginnote{2.3} when a mendicant meditates by observing an aspect of the body—keen, aware, and mindful, rid of desire and aversion for the world. 

They\marginnote{2.4} meditate observing an aspect of feelings—keen, aware, and mindful, rid of desire and aversion for the world. 

They\marginnote{2.5} meditate observing an aspect of the mind—keen, aware, and mindful, rid of desire and aversion for the world. 

They\marginnote{2.6} meditate observing an aspect of principles—keen, aware, and mindful, rid of desire and aversion for the world. 

The\marginnote{2.7} four kinds of mindfulness meditation are the path to convergence. They are in order to purify sentient beings, to get past sorrow and crying, to make an end of pain and sadness, to end the cycle of suffering, and to realize extinguishment.” 

That\marginnote{3.1} is what the Buddha said. Satisfied, the mendicants were happy with what the Buddha said. 

%
\section*{{\suttatitleacronym SN 47.2}{\suttatitletranslation Mindful }{\suttatitleroot Satisutta}}
\addcontentsline{toc}{section}{\tocacronym{SN 47.2} \toctranslation{Mindful } \tocroot{Satisutta}}
\markboth{Mindful }{Satisutta}
\extramarks{SN 47.2}{SN 47.2}

At\marginnote{1.1} one time the Buddha was staying near \textsanskrit{Vesālī}, in \textsanskrit{Ambapālī}’s Wood. There the Buddha addressed the mendicants, “Mendicants!” 

“Venerable\marginnote{1.4} sir,” they replied. The Buddha said this: 

“Mendicants,\marginnote{2.1} a mendicant should live mindful and aware. This is my instruction to you. 

And\marginnote{2.3} how is a mendicant mindful? It’s when a mendicant meditates by observing an aspect of the body—keen, aware, and mindful, rid of desire and aversion for the world. They meditate observing an aspect of feelings … mind … principles—keen, aware, and mindful, rid of desire and aversion for the world. That’s how a mendicant is mindful. 

And\marginnote{3.1} how is a mendicant aware? It’s when a mendicant acts with situational awareness when going out and coming back; when looking ahead and aside; when bending and extending the limbs; when bearing the outer robe, bowl and robes; when eating, drinking, chewing, and tasting; when urinating and defecating; when walking, standing, sitting, sleeping, waking, speaking, and keeping silent. That’s how a mendicant acts with situational awareness. A mendicant should live mindful and aware. This is my instruction to you.” 

%
\section*{{\suttatitleacronym SN 47.3}{\suttatitletranslation A Monk }{\suttatitleroot Bhikkhusutta}}
\addcontentsline{toc}{section}{\tocacronym{SN 47.3} \toctranslation{A Monk } \tocroot{Bhikkhusutta}}
\markboth{A Monk }{Bhikkhusutta}
\extramarks{SN 47.3}{SN 47.3}

At\marginnote{1.1} one time the Buddha was staying near \textsanskrit{Sāvatthī} in Jeta’s Grove, \textsanskrit{Anāthapiṇḍika}’s monastery. 

Then\marginnote{1.2} a mendicant went up to the Buddha, bowed, sat down to one side, and said to him, “Sir, may the Buddha please teach me Dhamma in brief. When I’ve heard it, I’ll live alone, withdrawn, diligent, keen, and resolute.” 

“This\marginnote{1.4} is exactly how some foolish people ask me for something. But when the teaching has been explained they think only of following me around.” 

“Sir,\marginnote{1.5} may the Buddha please teach me Dhamma in brief! May the Holy One teach me the Dhamma in brief! Hopefully I can understand the meaning of what the Buddha says! Hopefully I can be an heir of the Buddha’s teaching!” 

“Well\marginnote{1.6} then, mendicant, you should purify the starting point of skillful qualities. What is the starting point of skillful qualities? Well purified ethics and correct view. When your ethics are well purified and your view is correct, you should develop the four kinds of mindfulness meditation in three ways, depending on and grounded on ethics. 

What\marginnote{2.1} four? 

Meditate\marginnote{2.2} observing an aspect of the body internally—keen, aware, and mindful, rid of desire and aversion for the world. Or meditate observing an aspect of the body externally—keen, aware, and mindful, rid of desire and aversion for the world. Or meditate observing an aspect of the body internally and externally—keen, aware, and mindful, rid of desire and aversion for the world. 

Or\marginnote{2.5} meditate observing an aspect of feelings internally … externally … internally and externally—keen, aware, and mindful, rid of desire and aversion for the world. 

Or\marginnote{2.8} meditate observing an aspect of the mind internally … externally … internally and externally—keen, aware, and mindful, rid of desire and aversion for the world. 

Or\marginnote{2.11} meditate observing an aspect of principles internally … externally … internally and externally—keen, aware, and mindful, rid of desire and aversion for the world. When you develop the four kinds of mindfulness meditation in these three ways, depending on and grounded on ethics, you can expect growth, not decline, in skillful qualities, whether by day or by night.” 

And\marginnote{3.1} then that mendicant approved and agreed with what the Buddha said. He got up from his seat, bowed, and respectfully circled the Buddha, keeping him on his right, before leaving. 

Then\marginnote{3.2} that mendicant, living alone, withdrawn, diligent, keen, and resolute, soon realized the supreme end of the spiritual path in this very life. He lived having achieved with his own insight the goal for which gentlemen rightly go forth from the lay life to homelessness. 

He\marginnote{3.3} understood: “Rebirth is ended; the spiritual journey has been completed; what had to be done has been done; there is no return to any state of existence.” And that mendicant became one of the perfected. 

%
\section*{{\suttatitleacronym SN 47.4}{\suttatitletranslation At Sālā }{\suttatitleroot Sālasutta}}
\addcontentsline{toc}{section}{\tocacronym{SN 47.4} \toctranslation{At Sālā } \tocroot{Sālasutta}}
\markboth{At Sālā }{Sālasutta}
\extramarks{SN 47.4}{SN 47.4}

At\marginnote{1.1} one time the Buddha was staying in the land of the Kosalans near the brahmin village of \textsanskrit{Sālā}. There the Buddha addressed the mendicants: 

“Mendicants,\marginnote{2.1} those mendicants who are junior—recently gone forth, newly come to this teaching and training—should be encouraged, supported, and established in the four kinds of mindfulness meditation. What four? Please, reverends, meditate observing an aspect of the body—keen, aware, at one, with minds that are clear, immersed in \textsanskrit{samādhi}, and unified, so as to truly know the body. Meditate observing an aspect of feelings—keen, aware, at one, with minds that are clear, immersed in \textsanskrit{samādhi}, and unified, so as to truly know feelings. Meditate observing an aspect of the mind—keen, aware, at one, with minds that are clear, immersed in \textsanskrit{samādhi}, and unified, so as to truly know the mind. Meditate observing an aspect of principles—keen, aware, at one, with minds that are clear, immersed in \textsanskrit{samādhi}, and unified, so as to truly know principles. 

Those\marginnote{3.1} mendicants who are trainees—who haven’t achieved their heart’s desire, but live aspiring to the supreme sanctuary—also meditate observing an aspect of the body—keen, aware, at one, with minds that are clear, immersed in \textsanskrit{samādhi}, and unified, so as to fully understand the body. They meditate observing an aspect of feelings—keen, aware, at one, with minds that are clear, immersed in \textsanskrit{samādhi}, and unified, so as to fully understand feelings. They meditate observing an aspect of the mind—keen, aware, at one, with minds that are clear, immersed in \textsanskrit{samādhi}, and unified, so as to fully understand the mind. They meditate observing an aspect of principles—keen, aware, at one, with minds that are clear, immersed in \textsanskrit{samādhi}, and unified, so as to fully understand principles. 

Those\marginnote{4.1} mendicants who are perfected—who have ended the defilements, completed the spiritual journey, done what had to be done, laid down the burden, achieved their own goal, utterly ended the fetters of rebirth, and are rightly freed through enlightenment—also meditate observing an aspect of the body—keen, aware, at one, with minds that are clear, immersed in \textsanskrit{samādhi}, and unified, detached from the body. They meditate observing an aspect of feelings—keen, aware, at one, with minds that are clear, immersed in \textsanskrit{samādhi}, and unified, detached from feelings. They meditate observing an aspect of the mind—keen, aware, at one, with minds that are clear, immersed in \textsanskrit{samādhi}, and unified, detached from the mind. They meditate observing an aspect of principles—keen, aware, at one, with minds that are clear, immersed in \textsanskrit{samādhi}, and unified, detached from principles. 

Those\marginnote{5.1} mendicants who are junior—recently gone forth, newly come to this teaching and training—should be encouraged, supported, and established in these four kinds of mindfulness meditation.” 

%
\section*{{\suttatitleacronym SN 47.5}{\suttatitletranslation A Heap of the Unskillful }{\suttatitleroot Akusalarāsisutta}}
\addcontentsline{toc}{section}{\tocacronym{SN 47.5} \toctranslation{A Heap of the Unskillful } \tocroot{Akusalarāsisutta}}
\markboth{A Heap of the Unskillful }{Akusalarāsisutta}
\extramarks{SN 47.5}{SN 47.5}

At\marginnote{1.1} \textsanskrit{Sāvatthī}. 

There\marginnote{1.2} the Buddha said: 

“Rightly\marginnote{1.3} speaking, mendicants, you’d call these five hindrances a ‘heap of the unskillful’. For these five hindrances are entirely a heap of the unskillful. What five? The hindrances of sensual desire, ill will, dullness and drowsiness, restlessness and remorse, and doubt. Rightly speaking, you’d call these five hindrances a ‘heap of the unskillful’. For these five hindrances are entirely a heap of the unskillful. 

Rightly\marginnote{2.1} speaking, you’d call these four kinds of mindfulness meditation a ‘heap of the skillful’. For these four kinds of mindfulness meditation are entirely a heap of the skillful. What four? It’s when a mendicant meditates by observing an aspect of the body—keen, aware, and mindful, rid of desire and aversion for the world. They meditate observing an aspect of feelings … They meditate observing an aspect of the mind … They meditate observing an aspect of principles—keen, aware, and mindful, rid of desire and aversion for the world. Rightly speaking, you’d call these four kinds of mindfulness meditation a ‘heap of the skillful’. For these four kinds of mindfulness meditation are entirely a heap of the skillful.” 

%
\section*{{\suttatitleacronym SN 47.6}{\suttatitletranslation A Hawk }{\suttatitleroot Sakuṇagghisutta}}
\addcontentsline{toc}{section}{\tocacronym{SN 47.6} \toctranslation{A Hawk } \tocroot{Sakuṇagghisutta}}
\markboth{A Hawk }{Sakuṇagghisutta}
\extramarks{SN 47.6}{SN 47.6}

“Once\marginnote{1.1} upon a time, mendicants, a hawk suddenly swooped down and grabbed a quail. And as the quail was being carried off he wailed, ‘I’m so unlucky, so unfortunate, to have roamed out of my territory into the domain of others. If today I’d roamed within my own territory, the domain of my fathers, this hawk wouldn’t have been able to beat me by fighting.’ 

‘So,\marginnote{1.5} quail, what is your own territory, the domain of your fathers?’ 

‘It’s\marginnote{1.6} a ploughed field covered with clods of earth.’ 

Confident\marginnote{1.7} in her own strength, the hawk was not daunted or intimidated. She released the quail, saying, ‘Go now, quail. But even there you won’t escape me!’ 

Then\marginnote{2.1} the quail went to a ploughed field covered with clods of earth. He climbed up a big clod, and standing there, he said to the hawk: ‘Come get me, hawk! Come get me, hawk!’ 

Confident\marginnote{2.3} in her own strength, the hawk was not daunted or intimidated. She folded her wings and suddenly swooped down on the quail. When the quail knew that the hawk was nearly there, he slipped under that clod. But the hawk crashed chest-first right there. 

That’s\marginnote{2.6} what happens when you roam out of your territory into the domain of others. 

So,\marginnote{3.1} mendicants, don’t roam out of your own territory into the domain of others. If you roam out of your own territory into the domain of others, \textsanskrit{Māra} will find a vulnerability and get hold of you. 

And\marginnote{3.3} what is not a mendicant’s own territory but the domain of others? It’s the five kinds of sensual stimulation. What five? Sights known by the eye that are likable, desirable, agreeable, pleasant, sensual, and arousing. Sounds known by the ear … Smells known by the nose … Tastes known by the tongue … Touches known by the body that are likable, desirable, agreeable, pleasant, sensual, and arousing. This is not a mendicant’s own territory but the domain of others. 

You\marginnote{4.1} should roam inside your own territory, the domain of your fathers. If you roam inside your own territory, the domain of your fathers, \textsanskrit{Māra} won’t find a vulnerability or get hold of you. 

And\marginnote{4.3} what is a mendicant’s own territory, the domain of the fathers? It’s the four kinds of mindfulness meditation. What four? It’s when a mendicant meditates by observing an aspect of the body—keen, aware, and mindful, rid of desire and aversion for the world. They meditate observing an aspect of feelings … mind … principles—keen, aware, and mindful, rid of desire and aversion for the world. This is a mendicant’s own territory, the domain of the fathers.” 

%
\section*{{\suttatitleacronym SN 47.7}{\suttatitletranslation A Monkey }{\suttatitleroot Makkaṭasutta}}
\addcontentsline{toc}{section}{\tocacronym{SN 47.7} \toctranslation{A Monkey } \tocroot{Makkaṭasutta}}
\markboth{A Monkey }{Makkaṭasutta}
\extramarks{SN 47.7}{SN 47.7}

“Mendicants,\marginnote{1.1} in the Himalayas there are regions that are rugged and impassable. In some such regions, neither monkeys nor humans can go, while in others, monkeys can go but not humans. There are also level, pleasant places where both monkeys and humans can go. There hunters lay snares of tar on the monkey trails to catch the monkeys. 

The\marginnote{2.1} monkeys who are not foolhardy and reckless see the tar and avoid it from afar. But a foolish and reckless monkey goes up to the tar and grabs it with a hand. He gets stuck there. Thinking to free his hand, he grabs it with his other hand. He gets stuck there. Thinking to free both hands, he grabs it with a foot. He gets stuck there. Thinking to free both hands and foot, he grabs it with his other foot. He gets stuck there. Thinking to free both hands and feet, he grabs it with his snout. He gets stuck there. 

And\marginnote{2.12} so the monkey, trapped at five points, just lies there screeching. He’d meet with tragedy and disaster, and the hunter can do what he wants with him. The hunter spears him, pries him off that tarred block of wood, and goes wherever he wants. 

That’s\marginnote{2.14} what happens when you roam out of your territory into the domain of others. 

So,\marginnote{3.1} mendicants, don’t roam out of your own territory into the domain of others. If you roam out of your own territory into the domain of others, \textsanskrit{Māra} will find a vulnerability and get hold of you. 

And\marginnote{3.3} what is not a mendicant’s own territory but the domain of others? It’s the five kinds of sensual stimulation. What five? Sights known by the eye that are likable, desirable, agreeable, pleasant, sensual, and arousing. Sounds known by the ear … Smells known by the nose … Tastes known by the tongue … Touches known by the body that are likable, desirable, agreeable, pleasant, sensual, and arousing. This is not a mendicant’s own territory but the domain of others. 

You\marginnote{4.1} should roam inside your own territory, the domain of your fathers. If you roam inside your own territory, the domain of your fathers, \textsanskrit{Māra} won’t find a vulnerability or get hold of you. 

And\marginnote{4.3} what is a mendicant’s own territory, the domain of the fathers? It’s the four kinds of mindfulness meditation. What four? It’s when a mendicant meditates by observing an aspect of the body—keen, aware, and mindful, rid of desire and aversion for the world. They meditate observing an aspect of feelings … mind … principles—keen, aware, and mindful, rid of desire and aversion for the world. This is a mendicant’s own territory, the domain of the fathers.” 

%
\section*{{\suttatitleacronym SN 47.8}{\suttatitletranslation Cooks }{\suttatitleroot Sūdasutta}}
\addcontentsline{toc}{section}{\tocacronym{SN 47.8} \toctranslation{Cooks } \tocroot{Sūdasutta}}
\markboth{Cooks }{Sūdasutta}
\extramarks{SN 47.8}{SN 47.8}

“Mendicants,\marginnote{1.1} suppose a foolish, incompetent, unskillful cook was to serve a ruler or their minister with an excessive variety of curries: superbly sour, bitter, pungent, and sweet; hot and mild, and salty and bland. 

But\marginnote{2.1} that cook didn’t take their master’s hint: ‘Today my master preferred this sauce, or he reached for it, or he took a lot of it, or he praised it. Today my master preferred the sour or bitter or pungent or sweet or hot or mild or salty sauce. Or he preferred the bland sauce, or he reached for the bland one, or he took a lot of it, or he praised it.’ 

That\marginnote{3.1} foolish, incompetent, unskillful cook doesn’t get presented with clothes, wages, or bonuses. Why is that? Because they don’t take their master’s hint. 

In\marginnote{3.4} the same way, a foolish, incompetent, unskillful mendicant meditates by observing an aspect of the body—keen, aware, and mindful, rid of desire and aversion for the world. As they meditate observing an aspect of the body, their mind doesn’t enter immersion, and their corruptions aren’t given up. But they don’t take the hint. They meditate observing an aspect of feelings … mind … principles—keen, aware, and mindful, rid of desire and aversion for the world. As they meditate observing an aspect of principles, the mind doesn’t enter immersion, and the corruptions aren’t given up. But they don’t take the hint. 

That\marginnote{4.1} foolish, incompetent, unskillful mendicant doesn’t get blissful meditations in this very life, nor do they get mindfulness and situational awareness. Why is that? Because they don’t take their mind’s hint. 

Suppose\marginnote{5.1} an astute, competent, skillful cook was to serve a ruler or their minister with an excessive variety of curries: superbly sour, bitter, pungent, and sweet; hot and mild, and salty and bland. 

And\marginnote{6.1} that cook took their master’s hint: ‘Today my master preferred this sauce, or he reached for it, or he took a lot of it, or he praised it. Today my master preferred the sour or bitter or pungent or sweet or hot or mild or salty sauce. Or he preferred the bland sauce, or he reached for the bland one, or he took a lot of it, or he praised it.’ 

That\marginnote{7.1} astute, competent, skillful cook gets presented with clothes, wages, and bonuses. Why is that? Because they take their master’s hint. 

In\marginnote{7.4} the same way, an astute, competent, skillful mendicant meditates by observing an aspect of the body—keen, aware, and mindful, rid of desire and aversion for the world. As they meditate observing an aspect of the body, their mind enters immersion, and their corruptions are given up. They take the hint. They meditate observing an aspect of feelings … mind … principles—keen, aware, and mindful, rid of desire and aversion for the world. As they meditate observing an aspect of principles, their mind enters immersion, and their corruptions are given up. They take the hint. 

That\marginnote{8.1} astute, competent, skillful mendicant gets blissful meditations in this very life, and they get mindfulness and situational awareness. Why is that? Because they take their mind’s hint.” 

%
\section*{{\suttatitleacronym SN 47.9}{\suttatitletranslation Sick }{\suttatitleroot Gilānasutta}}
\addcontentsline{toc}{section}{\tocacronym{SN 47.9} \toctranslation{Sick } \tocroot{Gilānasutta}}
\markboth{Sick }{Gilānasutta}
\extramarks{SN 47.9}{SN 47.9}

\scevam{So\marginnote{1.1} I have heard. }At one time the Buddha was staying near \textsanskrit{Vesālī}, at the little village of Beluva. There the Buddha addressed the mendicants: “Mendicants, please enter the rainy season residence with whatever friends or acquaintances you have around \textsanskrit{Vesālī}. I’ll commence the rainy season residence right here in the little village of Beluva.” 

“Yes,\marginnote{1.6} sir,” those mendicants replied. They did as the Buddha said, while the Buddha commenced the rainy season residence right there in the little village of Beluva. 

After\marginnote{2.1} the Buddha had commenced the rainy season residence, he fell severely ill, struck by dreadful pains, close to death. But he endured unbothered, with mindfulness and situational awareness. Then it occurred to the Buddha: 

“It\marginnote{2.4} would not be appropriate for me to become fully extinguished before informing my attendants and taking leave of the mendicant \textsanskrit{Saṅgha}. Why don’t I forcefully suppress this illness, stabilize the life force, and live on?” So that is what he did. Then the Buddha’s illness died down. 

Soon\marginnote{3.1} after the Buddha had recovered from that sickness, he left his dwelling and sat in the shade of the porch on the seat spread out. Then Venerable Ānanda went up to the Buddha, bowed, sat down to one side, and said to him: 

“Sir,\marginnote{3.3} it’s fantastic that the Buddha is comfortable, that he’s well, and that he’s alright. Because when the Buddha was sick, my body felt like it was drugged. I was disorientated, and the teachings didn’t spring to mind. Still, at least I was consoled by the thought that the Buddha won’t become fully extinguished without making some statement regarding the \textsanskrit{Saṅgha} of mendicants.” 

“But\marginnote{4.1} what could the mendicant \textsanskrit{Saṅgha} expect from me now, Ānanda? I’ve taught the Dhamma without making any distinction between secret and public teachings. The Realized One doesn’t have the closed fist of a teacher when it comes to the teachings. 

If\marginnote{4.4} there’s anyone who thinks: ‘I’ll take charge of the \textsanskrit{Saṅgha} of mendicants,’ or ‘the \textsanskrit{Saṅgha} of mendicants is meant for me,’ let them make a statement regarding the \textsanskrit{Saṅgha}. But the Realized One doesn’t think like this, so why should he make some statement regarding the \textsanskrit{Saṅgha}? 

Now\marginnote{4.9} I am old, elderly and senior. I’m advanced in years and have reached the final stage of life. I’m currently eighty years old. Just as a decrepit cart keeps going by relying on straps, in the same way, the Realized One’s body keeps going by relying on straps, or so you’d think. 

Sometimes\marginnote{5.1} the Realized One, not focusing on any signs, and with the cessation of certain feelings, enters and remains in the signless immersion of the heart. Only then does the Realized One’s body become more comfortable. 

So\marginnote{5.2} Ānanda, live as your own island, your own refuge, with no other refuge. Let the teaching be your island and your refuge, with no other refuge. 

And\marginnote{6.1} how does a mendicant do this? It’s when a mendicant meditates by observing an aspect of the body—keen, aware, and mindful, rid of desire and aversion for the world. They meditate observing an aspect of feelings … mind … principles—keen, aware, and mindful, rid of desire and aversion for the world. That’s how a mendicant lives as their own island, their own refuge, with no other refuge. That’s how the teaching is their island and their refuge, with no other refuge. 

Whether\marginnote{6.7} now or after I have passed, any who shall live as their own island, their own refuge, with no other refuge; with the teaching as their island and their refuge, with no other refuge—those mendicants of mine who want to train shall be among the best of the best.” 

%
\section*{{\suttatitleacronym SN 47.10}{\suttatitletranslation The Nuns’ Quarters }{\suttatitleroot Bhikkhunupassayasutta}}
\addcontentsline{toc}{section}{\tocacronym{SN 47.10} \toctranslation{The Nuns’ Quarters } \tocroot{Bhikkhunupassayasutta}}
\markboth{The Nuns’ Quarters }{Bhikkhunupassayasutta}
\extramarks{SN 47.10}{SN 47.10}

Then\marginnote{1.1} Venerable Ānanda robed up in the morning and, taking his bowl and robe, went to the nuns’ quarters, and sat down on the seat spread out. Then several nuns went up to Venerable Ānanda bowed, sat down to one side, and said to him: 

“Sir,\marginnote{2.1} Ānanda, several nuns meditate with their minds firmly established in the four kinds of mindfulness meditation. They have realized a higher distinction than they had before.” 

“That’s\marginnote{2.2} how it is, sisters! That’s how it is, sisters! Any monk or nun who meditates with their mind firmly established in the four kinds of mindfulness meditation can expect to realize a higher distinction than they had before.” 

Then\marginnote{3.1} Ānanda educated, encouraged, fired up, and inspired those nuns with a Dhamma talk, after which he got up from his seat and left. Then Ānanda wandered for alms in \textsanskrit{Sāvatthī}. After the meal, on his return from almsround, he went to the Buddha, bowed, sat down to one side, and told him what had happened. 

“That’s\marginnote{5.1} so true, Ānanda! That’s so true! Any monk or nun who meditates with their mind firmly established in the four kinds of mindfulness meditation can expect to realize a higher distinction than they had before. 

What\marginnote{6.1} four? It’s when a mendicant meditates by observing an aspect of the body—keen, aware, and mindful, rid of desire and aversion for the world. As they meditate observing an aspect of the body, based on the body there arises physical tension, or mental sluggishness, or the mind is externally scattered. That mendicant should direct their mind towards an inspiring foundation. As they do so, joy springs up. Being joyful, rapture springs up. When the mind is full of rapture, the body becomes tranquil. When the body is tranquil, one feels bliss. And when blissful, the mind becomes immersed in \textsanskrit{samādhi}. Then they reflect: ‘I have accomplished the goal for which I directed my mind. Let me now pull back.’ They pull back, and neither place the mind nor keep it connected. They understand: ‘I’m neither placing the mind nor keeping it connected. Mindful within myself, I’m happy.’ 

Furthermore,\marginnote{7.1} a mendicant meditates by observing an aspect of feelings … mind … principles—keen, aware, and mindful, rid of desire and aversion for the world. As they meditate observing an aspect of principles, based on principles there arises physical tension, or mental sluggishness, or the mind is externally scattered. That mendicant should direct their mind towards an inspiring foundation. As they do so, joy springs up. Being joyful, rapture springs up. When the mind is full of rapture, the body becomes tranquil. When the body is tranquil, one feels bliss. And when blissful, the mind becomes immersed in \textsanskrit{samādhi}. Then they reflect: ‘I have accomplished the goal for which I directed my mind. Let me now pull back.’ They pull back, and neither place the mind nor keep it connected. They understand: ‘I’m neither placing the mind nor keeping it connected. Mindful within myself, I’m happy.’ That’s how there is directed development. 

And\marginnote{8.1} how is there undirected development? Not directing their mind externally, a mendicant understands: ‘My mind is not directed externally.’ And they understand: ‘Over a period of time it’s unconstricted, freed, and undirected.’ And they also understand: ‘I meditate observing an aspect of the body—keen, aware, mindful; I am happy.’ Not directing their mind externally, a mendicant understands: ‘My mind is not directed externally.’ And they understand: ‘Over a period of time it’s unconstricted, freed, and undirected.’ And they also understand: ‘I meditate observing an aspect of feelings—keen, aware, mindful; I am happy.’ Not directing their mind externally, a mendicant understands: ‘My mind is not directed externally.’ And they understand: ‘Over a period of time it’s unconstricted, freed, and undirected.’ And they also understand: ‘I meditate observing an aspect of the mind—keen, aware, mindful; I am happy.’ Not directing their mind externally, a mendicant understands: ‘My mind is not directed externally.’ And they understand: ‘Over a period of time it’s unconstricted, freed, and undirected.’ And they also understand: ‘I meditate observing an aspect of principles—keen, aware, mindful; I am happy.’ That’s how there is undirected development. 

So,\marginnote{9.1} Ānanda, I’ve taught you directed development and undirected development. Out of compassion, I’ve done what a teacher should do who wants what’s best for their disciples. Here are these roots of trees, and here are these empty huts. Practice absorption, mendicants! Don’t be negligent! Don’t regret it later! This is my instruction to you.” 

That\marginnote{10.1} is what the Buddha said. Satisfied, Venerable Ānanda was happy with what the Buddha said. 

%
\addtocontents{toc}{\let\protect\contentsline\protect\nopagecontentsline}
\chapter*{The Chapter at Nālandā }
\addcontentsline{toc}{chapter}{\tocchapterline{The Chapter at Nālandā }}
\addtocontents{toc}{\let\protect\contentsline\protect\oldcontentsline}

%
\section*{{\suttatitleacronym SN 47.11}{\suttatitletranslation A Great Man }{\suttatitleroot Mahāpurisasutta}}
\addcontentsline{toc}{section}{\tocacronym{SN 47.11} \toctranslation{A Great Man } \tocroot{Mahāpurisasutta}}
\markboth{A Great Man }{Mahāpurisasutta}
\extramarks{SN 47.11}{SN 47.11}

At\marginnote{1.1} \textsanskrit{Sāvatthī}. 

Then\marginnote{1.2} \textsanskrit{Sāriputta} went up to the Buddha, bowed, sat down to one side, and said to the Buddha: 

“Sir,\marginnote{1.3} they speak of ‘a great man’. How is a great man defined?” 

“\textsanskrit{Sāriputta},\marginnote{1.5} someone whose mind is free is a great man, I say. If their mind is not free, I say they’re not a great man. 

And\marginnote{2.1} how does someone have a free mind? It’s when a mendicant meditates by observing an aspect of the body—keen, aware, and mindful, rid of desire and aversion for the world. As they meditate observing an aspect of the body, their mind becomes dispassionate, and is freed from the defilements by not grasping. They meditate observing an aspect of feelings … mind … principles—keen, aware, and mindful, rid of desire and aversion for the world. As they meditate observing an aspect of principles, their mind becomes dispassionate, and is freed from the defilements by not grasping. That’s how someone has a free mind. 

Someone\marginnote{2.9} whose mind is free is a great man, I say. If their mind is not free, I say they’re not a great man.” 

%
\section*{{\suttatitleacronym SN 47.12}{\suttatitletranslation At Nālandā }{\suttatitleroot Nālandasutta}}
\addcontentsline{toc}{section}{\tocacronym{SN 47.12} \toctranslation{At Nālandā } \tocroot{Nālandasutta}}
\markboth{At Nālandā }{Nālandasutta}
\extramarks{SN 47.12}{SN 47.12}

At\marginnote{1.1} one time the Buddha was staying near \textsanskrit{Nālandā} in \textsanskrit{Pāvārika}’s mango grove. Then \textsanskrit{Sāriputta} went up to the Buddha, bowed, sat down to one side, and said to him: 

“Sir,\marginnote{1.3} I have such confidence in the Buddha that I believe there’s no other ascetic or brahmin—whether past, future, or present—whose direct knowledge is superior to the Buddha when it comes to awakening.” 

“That’s\marginnote{1.5} a grand and dramatic statement, \textsanskrit{Sāriputta}. You’ve roared a definitive, categorical lion’s roar, saying: ‘I have such confidence in the Buddha that I believe there’s no other ascetic or brahmin—whether past, future, or present—whose direct knowledge is superior to the Buddha when it comes to awakening.’ 

What\marginnote{2.1} about all the perfected ones, the fully awakened Buddhas who lived in the past? Have you comprehended their minds to know that those Buddhas had such ethics, or such qualities, or such wisdom, or such meditation, or such freedom?” 

“No,\marginnote{2.3} sir.” 

“And\marginnote{3.1} what about all the perfected ones, the fully awakened Buddhas who will live in the future? Have you comprehended their minds to know that those Buddhas will have such ethics, or such qualities, or such wisdom, or such meditation, or such freedom?” 

“No,\marginnote{3.3} sir.” 

“And\marginnote{4.1} what about me, the perfected one, the fully awakened Buddha at present? Have you comprehended my mind to know that I have such ethics, or such qualities, or such wisdom, or such meditation, or such freedom?” 

“No,\marginnote{4.3} sir.” 

“Well\marginnote{5.1} then, \textsanskrit{Sāriputta}, given that you don’t comprehend the minds of Buddhas past, future, or present, what exactly are you doing, making such a grand and dramatic statement, roaring such a definitive, categorical lion’s roar?” 

“Sir,\marginnote{6.1} though I don’t comprehend the minds of Buddhas past, future, and present, still I understand this by inference from the teaching. Suppose there was a king’s frontier citadel with fortified embankments, ramparts, and arches, and a single gate. And it has a gatekeeper who is astute, competent, and intelligent. He keeps strangers out and lets known people in. As he walks around the patrol path, he doesn’t see a hole or cleft in the wall, not even one big enough for a cat to slip out. He thinks, ‘Whatever sizable creatures enter or leave the citadel, all of them do so via this gate.’ 

In\marginnote{6.8} the same way, I understand this by inference from the teaching: ‘All the perfected ones, fully awakened Buddhas—whether past, future, or present—give up the five hindrances, corruptions of the heart that weaken wisdom. Their mind is firmly established in the four kinds of mindfulness meditation. They correctly develop the seven awakening factors. And they wake up to the supreme perfect awakening.’” 

“Good,\marginnote{7.1} good, \textsanskrit{Sāriputta}! So \textsanskrit{Sāriputta}, you should frequently speak this exposition of the teaching to the monks, nuns, laymen, and laywomen. Though there will be some foolish people who have doubt or uncertainty regarding the Realized One, when they hear this exposition of the teaching they’ll give up that doubt or uncertainty.” 

%
\section*{{\suttatitleacronym SN 47.13}{\suttatitletranslation With Cunda }{\suttatitleroot Cundasutta}}
\addcontentsline{toc}{section}{\tocacronym{SN 47.13} \toctranslation{With Cunda } \tocroot{Cundasutta}}
\markboth{With Cunda }{Cundasutta}
\extramarks{SN 47.13}{SN 47.13}

At\marginnote{1.1} one time the Buddha was staying near \textsanskrit{Sāvatthī} in Jeta’s Grove, \textsanskrit{Anāthapiṇḍika}’s monastery. At that time Venerable \textsanskrit{Sāriputta} was staying in the Magadhan lands near the little village of \textsanskrit{Nālaka}, and he was sick, suffering, gravely ill. And the novice Cunda was his carer. 

Then\marginnote{2.1} Venerable \textsanskrit{Sāriputta} became fully extinguished because of that sickness. Then Cunda took \textsanskrit{Sāriputta}’s bowl and robes and set out for \textsanskrit{Sāvatthī}. He went to see Venerable Ānanda at Jeta’s grove, \textsanskrit{Anāthapiṇḍika}’s monastery, bowed, sat down to one side, and said to him: 

“Sir,\marginnote{2.3} Venerable \textsanskrit{Sāriputta} has become fully extinguished. This is his bowl and robe.” 

“Reverend\marginnote{3.1} Cunda, we should see the Buddha about this matter. Come, let’s go to the Buddha and inform him about this.” 

“Yes,\marginnote{3.3} sir,” replied Cunda. 

Then\marginnote{4.1} Ānanda and Cunda went to the Buddha, bowed, sat down to one side, and said to him: 

“Sir,\marginnote{4.2} this novice Cunda says that Venerable \textsanskrit{Sāriputta} has become fully extinguished. This is his bowl and robe. Since I heard this, my body feels like it’s drugged. I’m disorientated, and the teachings don’t spring to mind.” 

“Well,\marginnote{5.1} Ānanda, when \textsanskrit{Sāriputta} became fully extinguished, did he take away your entire spectrum of ethical conduct, of immersion, of wisdom, of freedom, or of the knowledge and vision of freedom?” 

“No,\marginnote{5.2} sir, he did not. But Venerable \textsanskrit{Sāriputta} was my adviser and counselor. He educated, encouraged, fired up, and inspired me. He never tired of teaching the Dhamma, and he supported his spiritual companions. I remember the nectar of the teaching, the riches of the teaching, the support of the teaching given by Venerable \textsanskrit{Sāriputta}.” 

“Ānanda,\marginnote{6.1} did I not prepare for this when I explained that we must be parted and separated from all we hold dear and beloved? How could it possibly be so that what is born, created, conditioned, and liable to wear out should not wear out? That is not possible. 

Suppose\marginnote{6.5} there was a large tree standing with heartwood, and the largest branch fell off. In the same way, in the great \textsanskrit{Saṅgha} that stands with heartwood, \textsanskrit{Sāriputta} has become fully extinguished. 

How\marginnote{6.7} could it possibly be so that what is born, created, conditioned, and liable to wear out should not wear out? That is not possible. 

So\marginnote{6.9} Ānanda, live as your own island, your own refuge, with no other refuge. Let the teaching be your island and your refuge, with no other refuge. 

And\marginnote{7.1} how does a mendicant do this? It’s when a mendicant meditates by observing an aspect of the body—keen, aware, and mindful, rid of desire and aversion for the world. They meditate observing an aspect of feelings … mind … principles—keen, aware, and mindful, rid of desire and aversion for the world. 

That’s\marginnote{7.6} how a mendicant lives as their own island, their own refuge, with no other refuge. That’s how the teaching is their island and their refuge, with no other refuge. 

Whether\marginnote{7.7} now or after I have passed, any who shall live as their own island, their own refuge, with no other refuge; with the teaching as their island and their refuge, with no other refuge—those mendicants of mine who want to train shall be among the best of the best.” 

%
\section*{{\suttatitleacronym SN 47.14}{\suttatitletranslation At Ukkacelā }{\suttatitleroot Ukkacelasutta}}
\addcontentsline{toc}{section}{\tocacronym{SN 47.14} \toctranslation{At Ukkacelā } \tocroot{Ukkacelasutta}}
\markboth{At Ukkacelā }{Ukkacelasutta}
\extramarks{SN 47.14}{SN 47.14}

At\marginnote{1.1} one time the Buddha was staying in the land of the \textsanskrit{Vajjīs} near \textsanskrit{Ukkacelā} on the bank of the Ganges river, together with a large \textsanskrit{Saṅgha} of mendicants. It was not long after \textsanskrit{Sāriputta} and \textsanskrit{Moggallāna} had become fully extinguished. Now, at that time the Buddha was sitting in the open, surrounded by the \textsanskrit{Saṅgha} of mendicants. 

Then\marginnote{2.1} the Buddha looked around the \textsanskrit{Saṅgha} of mendicants, who were silent. He addressed them: 

“Mendicants,\marginnote{2.2} this assembly seems empty to me now that \textsanskrit{Sāriputta} and \textsanskrit{Moggallāna} have become fully extinguished. When \textsanskrit{Sāriputta} and \textsanskrit{Moggallāna} were alive, my assembly was never empty; I had no concern for any region where they stayed. The Buddhas of the past or the future have pairs of chief disciples who are no better than \textsanskrit{Sāriputta} and \textsanskrit{Moggallāna} were to me. 

It’s\marginnote{2.7} an incredible and amazing quality of such disciples that they fulfill the Teacher’s instructions and follow his advice. And they’re liked and approved, respected and admired by the four assemblies. 

And\marginnote{2.9} it’s an incredible and amazing quality of the Realized One that when such a pair of disciples becomes fully extinguished he does not sorrow or lament. How could it possibly be so that what is born, created, conditioned, and liable to wear out should not wear out? That is not possible. 

Suppose\marginnote{2.13} there was a large tree standing with heartwood, and the largest branches fell off. In the same way, in the great \textsanskrit{Saṅgha} that stands with heartwood, \textsanskrit{Sāriputta} and \textsanskrit{Moggallāna} have become fully extinguished. 

How\marginnote{2.15} could it possibly be so that what is born, created, conditioned, and liable to wear out should not wear out? That is not possible. 

So\marginnote{2.17} mendicants, live as your own island, your own refuge, with no other refuge. Let the teaching be your island and your refuge, with no other refuge. 

And\marginnote{3.1} how does a mendicant do this? It’s when a mendicant meditates by observing an aspect of the body—keen, aware, and mindful, rid of desire and aversion for the world. They meditate observing an aspect of feelings … mind … principles—keen, aware, and mindful, rid of desire and aversion for the world. 

That’s\marginnote{3.6} how a mendicant lives as their own island, their own refuge, with no other refuge. That’s how the teaching is their island and their refuge, with no other refuge. 

Whether\marginnote{3.7} now or after I have passed, any who shall live as their own island, their own refuge, with no other refuge; with the teaching as their island and their refuge, with no other refuge—those mendicants of mine who want to train shall be among the best of the best.” 

%
\section*{{\suttatitleacronym SN 47.15}{\suttatitletranslation With Bāhiya }{\suttatitleroot Bāhiyasutta}}
\addcontentsline{toc}{section}{\tocacronym{SN 47.15} \toctranslation{With Bāhiya } \tocroot{Bāhiyasutta}}
\markboth{With Bāhiya }{Bāhiyasutta}
\extramarks{SN 47.15}{SN 47.15}

At\marginnote{1.1} \textsanskrit{Sāvatthī}. 

Then\marginnote{1.2} Venerable \textsanskrit{Bāhiya} went up to the Buddha, bowed, sat down to one side, and said to him: 

“Sir,\marginnote{1.3} may the Buddha please teach me Dhamma in brief. When I’ve heard it, I’ll live alone, withdrawn, diligent, keen, and resolute.” 

“Well\marginnote{1.4} then, \textsanskrit{Bāhiya}, you should purify the starting point of skillful qualities. What is the starting point of skillful qualities? Well purified ethics and correct view. When your ethics are well purified and your view is correct, you should develop the four kinds of mindfulness meditation, depending on and grounded on ethics. 

What\marginnote{2.1} four? Meditate observing an aspect of the body—keen, aware, and mindful, rid of desire and aversion for the world. Meditate observing an aspect of feelings … mind … principles—keen, aware, and mindful, rid of desire and aversion for the world. When you develop these four kinds of mindfulness meditation, depending on and grounded on ethics, you can expect growth, not decline, in skillful qualities, whether by day or by night.” 

And\marginnote{3.1} then Venerable \textsanskrit{Bāhiya} approved and agreed with what the Buddha said. He got up from his seat, bowed, and respectfully circled the Buddha, keeping him on his right, before leaving. Then \textsanskrit{Bāhiya}, living alone, withdrawn, diligent, keen, and resolute, soon realized the supreme end of the spiritual path in this very life. He lived having achieved with his own insight the goal for which gentlemen rightly go forth from the lay life to homelessness. 

He\marginnote{3.3} understood: “Rebirth is ended; the spiritual journey has been completed; what had to be done has been done; there is no return to any state of existence.” And Venerable \textsanskrit{Bāhiya} became one of the perfected. 

%
\section*{{\suttatitleacronym SN 47.16}{\suttatitletranslation With Uttiya }{\suttatitleroot Uttiyasutta}}
\addcontentsline{toc}{section}{\tocacronym{SN 47.16} \toctranslation{With Uttiya } \tocroot{Uttiyasutta}}
\markboth{With Uttiya }{Uttiyasutta}
\extramarks{SN 47.16}{SN 47.16}

At\marginnote{1.1} \textsanskrit{Sāvatthī}. 

Then\marginnote{1.2} Venerable Uttiya went up to the Buddha … and asked him, “Sir, may the Buddha please teach me Dhamma in brief. When I’ve heard it, I’ll live alone, withdrawn, diligent, keen, and resolute.” 

“Well\marginnote{1.4} then, Uttiya, you should purify the starting point of skillful qualities. What is the starting point of skillful qualities? Well purified ethics and correct view. When your ethics are well purified and your view is correct, you should develop the four kinds of mindfulness meditation, depending on and grounded on ethics. 

What\marginnote{2.1} four? Meditate observing an aspect of the body—keen, aware, and mindful, rid of desire and aversion for the world. Meditate observing an aspect of feelings … mind … principles—keen, aware, and mindful, rid of desire and aversion for the world. When you develop these four kinds of mindfulness meditation, depending on and grounded on ethics, you’ll pass beyond Death’s domain.” 

And\marginnote{3.1} then Venerable Uttiya approved and agreed with what the Buddha said. He got up from his seat, bowed, and respectfully circled the Buddha, keeping him on his right, before leaving. Then Uttiya, living alone, withdrawn, diligent, keen, and resolute, soon realized the supreme end of the spiritual path in this very life. He lived having achieved with his own insight the goal for which gentlemen rightly go forth from the lay life to homelessness. 

He\marginnote{3.3} understood: “Rebirth is ended; the spiritual journey has been completed; what had to be done has been done; there is no return to any state of existence.” And Venerable Uttiya became one of the perfected. 

%
\section*{{\suttatitleacronym SN 47.17}{\suttatitletranslation Noble }{\suttatitleroot Ariyasutta}}
\addcontentsline{toc}{section}{\tocacronym{SN 47.17} \toctranslation{Noble } \tocroot{Ariyasutta}}
\markboth{Noble }{Ariyasutta}
\extramarks{SN 47.17}{SN 47.17}

“Mendicants,\marginnote{1.1} when these four kinds of mindfulness meditation are developed and cultivated they are noble and emancipating, and bring one who practices them to the complete ending of suffering. What four? 

It’s\marginnote{1.3} when a mendicant meditates by observing an aspect of the body—keen, aware, and mindful, rid of desire and aversion for the world. They meditate observing an aspect of feelings … mind … principles—keen, aware, and mindful, rid of desire and aversion for the world. 

When\marginnote{1.7} these four kinds of mindfulness meditation are developed and cultivated they are noble and emancipating, and bring one who practices them to the complete ending of suffering.” 

%
\section*{{\suttatitleacronym SN 47.18}{\suttatitletranslation With Brahmā }{\suttatitleroot Brahmasutta}}
\addcontentsline{toc}{section}{\tocacronym{SN 47.18} \toctranslation{With Brahmā } \tocroot{Brahmasutta}}
\markboth{With Brahmā }{Brahmasutta}
\extramarks{SN 47.18}{SN 47.18}

At\marginnote{1.1} one time, when he was first awakened, the Buddha was staying near \textsanskrit{Uruvelā} at the goatherd’s banyan tree on the bank of the \textsanskrit{Nerañjarā} River. 

Then\marginnote{1.2} as he was in private retreat this thought came to his mind, “The four kinds of mindfulness meditation are the path to convergence. They are in order to purify sentient beings, to get past sorrow and crying, to make an end of pain and sadness, to end the cycle of suffering, and to realize extinguishment. 

What\marginnote{2.1} four? A mendicant would meditate observing an aspect of the body—keen, aware, and mindful, rid of desire and aversion for the world. Or they’d meditate observing an aspect of feelings … or mind … or principles—keen, aware, and mindful, rid of desire and aversion for the world. The four kinds of mindfulness meditation are the path to convergence. They are in order to purify sentient beings, to get past sorrow and crying, to make an end of pain and sadness, to end the cycle of suffering, and to realize extinguishment.” 

Then\marginnote{3.1} \textsanskrit{Brahmā} Sahampati knew what the Buddha was thinking. As easily as a strong person would extend or contract their arm, he vanished from the \textsanskrit{Brahmā} realm and reappeared in front of the Buddha. He arranged his robe over one shoulder, raised his joined palms toward the Buddha, and said: 

“That’s\marginnote{3.3} so true, Blessed One! That’s so true, Holy One! Sir, the four kinds of mindfulness meditation are the path to convergence. They are in order to purify sentient beings, to get past sorrow and crying, to make an end of pain and sadness, to end the cycle of suffering, and to realize extinguishment. 

What\marginnote{4.1} four? A mendicant would meditate observing an aspect of the body—keen, aware, and mindful, rid of desire and aversion for the world. Or they’d meditate observing an aspect of feelings … or mind … or principles—keen, aware, and mindful, rid of desire and aversion for the world. The four kinds of mindfulness meditation are the path to convergence. They are in order to purify sentient beings, to get past sorrow and crying, to make an end of pain and sadness, to end the cycle of suffering, and to realize extinguishment.” 

That’s\marginnote{5.1} what \textsanskrit{Brahmā} Sahampati said. Then he went on to say: 

\begin{verse}%
“The\marginnote{6.1} compassionate one, who sees the ending of rebirth, \\
understands the path to convergence. \\
By this path people crossed over before, \\
will cross, and are crossing.” 

%
\end{verse}

%
\section*{{\suttatitleacronym SN 47.19}{\suttatitletranslation At Sedaka }{\suttatitleroot Sedakasutta}}
\addcontentsline{toc}{section}{\tocacronym{SN 47.19} \toctranslation{At Sedaka } \tocroot{Sedakasutta}}
\markboth{At Sedaka }{Sedakasutta}
\extramarks{SN 47.19}{SN 47.19}

At\marginnote{1.1} one time the Buddha was staying in the land of the Sumbhas, near the town of the Sumbhas called Sedaka. There the Buddha addressed the mendicants: 

“Once\marginnote{1.3} upon a time, mendicants, an acrobat set up his bamboo pole and said to his apprentice \textsanskrit{Medakathālikā}, ‘Come now, dear \textsanskrit{Medakathālikā}, climb up the bamboo pole and stand on my shoulders.’ 

‘Yes,\marginnote{1.5} teacher,’ she replied. She climbed up the bamboo pole and stood on her teacher’s shoulders. 

Then\marginnote{1.6} the acrobat said to \textsanskrit{Medakathālikā}, ‘You look after me, dear \textsanskrit{Medakathālikā}, and I’ll look after you. That’s how, guarding and looking after each other, we’ll display our skill, collect our fee, and get down safely from the bamboo pole.’ 

When\marginnote{1.9} he said this, \textsanskrit{Medakathālikā} said to her teacher, ‘That’s not how it is, teacher! You should look after yourself, and I’ll look after myself. That’s how, guarding and looking after ourselves, we’ll display our skill, collect our fee, and get down safely from the bamboo pole.’ 

That’s\marginnote{2.1} the way,” said the Buddha. “It’s just as \textsanskrit{Medakathālikā} said to her teacher. Thinking ‘I’ll look after myself,’ you should cultivate mindfulness meditation. Thinking ‘I’ll look after others,’ you should cultivate mindfulness meditation. Looking after yourself, you look after others; and looking after others, you look after yourself. 

And\marginnote{3.1} how do you look after others by looking after yourself? By development, cultivation, and practice of meditation. And how do you look after yourself by looking after others? By acceptance, harmlessness, love, and sympathy. 

Thinking\marginnote{3.7} ‘I’ll look after myself,’ you should cultivate mindfulness meditation. Thinking ‘I’ll look after others,’ you should cultivate mindfulness meditation. Looking after yourself, you look after others; and looking after others, you look after yourself.” 

%
\section*{{\suttatitleacronym SN 47.20}{\suttatitletranslation The Finest Lady in the Land }{\suttatitleroot Janapadakalyāṇīsutta}}
\addcontentsline{toc}{section}{\tocacronym{SN 47.20} \toctranslation{The Finest Lady in the Land } \tocroot{Janapadakalyāṇīsutta}}
\markboth{The Finest Lady in the Land }{Janapadakalyāṇīsutta}
\extramarks{SN 47.20}{SN 47.20}

\scevam{So\marginnote{1.1} I have heard. }At one time the Buddha was staying in the land of the Sumbhas, near the town of the Sumbhas called Sedaka. There the Buddha addressed the mendicants, “Mendicants!” 

“Venerable\marginnote{1.5} sir,” they replied. The Buddha said this: 

“Mendicants,\marginnote{2.1} suppose that on hearing, ‘The finest lady in the land! The finest lady in the land!’ a large crowd would gather. And the finest lady in the land would dance and sing in a most thrilling way. On hearing, ‘The finest lady in the land is dancing and singing! The finest lady in the land is dancing and singing!’ an even larger crowd would gather. 

Then\marginnote{2.4} a person would come along who wants to live and doesn’t want to die, who wants to be happy and recoils from pain. They’d say to him, ‘Mister, this is a bowl full to the brim with oil. You must carry it in between this large crowd and the finest lady in the land. And a man with a drawn sword will follow right behind you. Wherever you spill even a drop, he’ll chop off your head right there.’ 

What\marginnote{2.9} do you think, mendicants? Would that person lose focus on that bowl, and negligently get distracted outside?” 

“No,\marginnote{2.11} sir.” 

“I’ve\marginnote{3.1} made up this simile to make a point. And this is what it means. ‘A bowl of oil filled to the brim’ is a term for mindfulness of the body. 

So\marginnote{3.4} you should train like this: ‘We will develop mindfulness of the body. We’ll cultivate it, make it our vehicle and our basis, keep it up, consolidate it, and properly implement it.’ That’s how you should train.” 

%
\addtocontents{toc}{\let\protect\contentsline\protect\nopagecontentsline}
\chapter*{The Chapter on Ethics and Duration }
\addcontentsline{toc}{chapter}{\tocchapterline{The Chapter on Ethics and Duration }}
\addtocontents{toc}{\let\protect\contentsline\protect\oldcontentsline}

%
\section*{{\suttatitleacronym SN 47.21}{\suttatitletranslation Ethics }{\suttatitleroot Sīlasutta}}
\addcontentsline{toc}{section}{\tocacronym{SN 47.21} \toctranslation{Ethics } \tocroot{Sīlasutta}}
\markboth{Ethics }{Sīlasutta}
\extramarks{SN 47.21}{SN 47.21}

\scevam{So\marginnote{1.1} I have heard. }At one time the venerables Ānanda and Bhadda were staying near \textsanskrit{Pāṭaliputta}, in the Chicken Monastery. Then in the late afternoon, Venerable Bhadda came out of retreat, went to Venerable Ānanda, and exchanged greetings with him. When the greetings and polite conversation were over, he sat down to one side and said to Ānanda: 

“Reverend\marginnote{1.5} Ānanda, the Buddha has spoken of skillful ethics. What’s their purpose?” 

“Good,\marginnote{2.1} good, Reverend Bhadda! Your approach and articulation are excellent, and it’s a good question. For you asked: ‘The Buddha has spoken of skillful ethics. What’s their purpose?’” 

“Yes,\marginnote{2.5} reverend.” 

“The\marginnote{2.6} Buddha has spoken of skillful ethics to the extent necessary for developing the four kinds of mindfulness meditation. 

What\marginnote{3.1} four? It’s when a mendicant meditates by observing an aspect of the body—keen, aware, and mindful, rid of desire and aversion for the world. They meditate observing an aspect of feelings … mind … principles—keen, aware, and mindful, rid of desire and aversion for the world. The Buddha has spoken of skillful ethics to the extent necessary for developing the four kinds of mindfulness meditation.” 

%
\section*{{\suttatitleacronym SN 47.22}{\suttatitletranslation Long Lasting }{\suttatitleroot Ciraṭṭhitisutta}}
\addcontentsline{toc}{section}{\tocacronym{SN 47.22} \toctranslation{Long Lasting } \tocroot{Ciraṭṭhitisutta}}
\markboth{Long Lasting }{Ciraṭṭhitisutta}
\extramarks{SN 47.22}{SN 47.22}

The\marginnote{1.1} same setting. 

“What\marginnote{1.3} is the cause, Reverend Ānanda, what is the reason why the true teaching does not last long after the final extinguishment of the Realized One? What is the cause, what is the reason why the true teaching does last long after the final extinguishment of the Realized One?” 

“Good,\marginnote{2.1} good, Reverend Bhadda! Your approach and articulation are excellent, and it’s a good question. For you asked: ‘What is the cause, Reverend Ānanda, what is the reason why the true teaching does not last long after the final extinguishment of the Realized One? What is the cause, what is the reason why the true teaching does last long after the final extinguishment of the Realized One?’” 

“Yes,\marginnote{2.6} reverend.” 

“It’s\marginnote{2.7} because of not developing and cultivating the four kinds of mindfulness meditation that the true teaching doesn’t last long after the final extinguishment of the Realized One. It’s because of developing and cultivating the four kinds of mindfulness meditation that the true teaching does last long after the final extinguishment of the Realized One. 

What\marginnote{3.1} four? It’s when a mendicant meditates by observing an aspect of the body—keen, aware, and mindful, rid of desire and aversion for the world. They meditate observing an aspect of feelings … mind … principles—keen, aware, and mindful, rid of desire and aversion for the world. It’s because of not developing and cultivating these four kinds of mindfulness meditation that the true teaching doesn’t last long after the final extinguishment of the Realized One. It’s because of developing and cultivating these four kinds of mindfulness meditation that the true teaching does last long after the final extinguishment of the Realized One.” 

%
\section*{{\suttatitleacronym SN 47.23}{\suttatitletranslation Decline }{\suttatitleroot Parihānasutta}}
\addcontentsline{toc}{section}{\tocacronym{SN 47.23} \toctranslation{Decline } \tocroot{Parihānasutta}}
\markboth{Decline }{Parihānasutta}
\extramarks{SN 47.23}{SN 47.23}

At\marginnote{1.1} one time the venerables Ānanda and Bhadda were staying near \textsanskrit{Pāṭaliputta}, in the Chicken Monastery. Then in the late afternoon, Venerable Bhadda came out of retreat, went to Venerable Ānanda, and exchanged greetings with him. When the greetings and polite conversation were over, he sat down to one side and said to Ānanda: 

“What’s\marginnote{1.4} the cause, Reverend Ānanda, what’s the reason why the true teaching declines? And what’s the cause, what’s the reason why the true teaching doesn’t decline?” 

“Good,\marginnote{2.1} good, Reverend Bhadda! Your approach and articulation are excellent, and it’s a good question. For you asked: ‘What’s the cause, what’s the reason why the true teaching declines? And what’s the cause, what’s the reason why the true teaching doesn’t decline?’” 

“Yes,\marginnote{2.6} reverend.” 

“It’s\marginnote{2.7} because of not developing and cultivating the four kinds of mindfulness meditation that the true teaching declines. It’s because of developing and cultivating the four kinds of mindfulness meditation that the true teaching doesn’t decline. 

What\marginnote{3.1} four? It’s when a mendicant meditates by observing an aspect of the body—keen, aware, and mindful, rid of desire and aversion for the world. They meditate observing an aspect of feelings … mind … principles—keen, aware, and mindful, rid of desire and aversion for the world. It’s because of not developing and cultivating these four kinds of mindfulness meditation that the true teaching declines. And it’s because of developing and cultivating these four kinds of mindfulness meditation that the true teaching doesn’t decline.” 

%
\section*{{\suttatitleacronym SN 47.24}{\suttatitletranslation Plain Version }{\suttatitleroot Suddhasutta}}
\addcontentsline{toc}{section}{\tocacronym{SN 47.24} \toctranslation{Plain Version } \tocroot{Suddhasutta}}
\markboth{Plain Version }{Suddhasutta}
\extramarks{SN 47.24}{SN 47.24}

At\marginnote{1.1} \textsanskrit{Sāvatthī}. 

“Mendicants,\marginnote{1.2} there are these four kinds of mindfulness meditation. What four? It’s when a mendicant meditates by observing an aspect of the body—keen, aware, and mindful, rid of desire and aversion for the world. They meditate observing an aspect of feelings … mind … principles—keen, aware, and mindful, rid of desire and aversion for the world. 

These\marginnote{1.8} are the four kinds of mindfulness meditation.” 

%
\section*{{\suttatitleacronym SN 47.25}{\suttatitletranslation A Certain Brahmin }{\suttatitleroot Aññatarabrāhmaṇasutta}}
\addcontentsline{toc}{section}{\tocacronym{SN 47.25} \toctranslation{A Certain Brahmin } \tocroot{Aññatarabrāhmaṇasutta}}
\markboth{A Certain Brahmin }{Aññatarabrāhmaṇasutta}
\extramarks{SN 47.25}{SN 47.25}

\scevam{So\marginnote{1.1} I have heard. }At one time the Buddha was staying near \textsanskrit{Sāvatthī} in Jeta’s Grove, \textsanskrit{Anāthapiṇḍika}’s monastery. Then a certain brahmin went up to the Buddha, and exchanged greetings with him. When the greetings and polite conversation were over, he sat down to one side and said to the Buddha: 

“What\marginnote{1.5} is the cause, Master Gotama, what is the reason why the true teaching does not last long after the final extinguishment of the Realized One? And what is the cause, what is the reason why the true teaching does last long after the final extinguishment of the Realized One?” 

“Brahmin,\marginnote{2.1} it’s because of not developing and cultivating the four kinds of mindfulness meditation that the true teaching doesn’t last long after the final extinguishment of the Realized One. It’s because of developing and cultivating the four kinds of mindfulness meditation that the true teaching does last long after the final extinguishment of the Realized One. 

What\marginnote{3.1} four? It’s when a mendicant meditates by observing an aspect of the body—keen, aware, and mindful, rid of desire and aversion for the world. They meditate observing an aspect of feelings … mind … principles—keen, aware, and mindful, rid of desire and aversion for the world. It’s because of not developing and cultivating these four kinds of mindfulness meditation that the true teaching doesn’t last long after the final extinguishment of the Realized One. It’s because of developing and cultivating these four kinds of mindfulness meditation that the true teaching does last long after the final extinguishment of the Realized One.” 

When\marginnote{4.1} he said this, the brahmin said to the Buddha, “Excellent, Master Gotama! Excellent! … From this day forth, may Master Gotama remember me as a lay follower who has gone for refuge for life.” 

%
\section*{{\suttatitleacronym SN 47.26}{\suttatitletranslation Partly }{\suttatitleroot Padesasutta}}
\addcontentsline{toc}{section}{\tocacronym{SN 47.26} \toctranslation{Partly } \tocroot{Padesasutta}}
\markboth{Partly }{Padesasutta}
\extramarks{SN 47.26}{SN 47.26}

At\marginnote{1.1} one time the venerables \textsanskrit{Sāriputta}, \textsanskrit{Mahāmoggallāna}, and Anuruddha were staying near \textsanskrit{Sāketa}, in the Thorny Wood. Then in the late afternoon, \textsanskrit{Sāriputta} and \textsanskrit{Mahāmoggallāna} came out of retreat, went to Anuruddha, and exchanged greetings with him. When the greetings and polite conversation were over, they sat down to one side. \textsanskrit{Sāriputta} said to Anuruddha: 

“Reverend,\marginnote{1.4} they speak of this person called ‘a trainee’. How is a trainee defined?” 

“Reverends,\marginnote{1.6} a trainee is someone who has partly developed the four kinds of mindfulness meditation. 

What\marginnote{2.1} four? It’s when a mendicant meditates by observing an aspect of the body—keen, aware, and mindful, rid of desire and aversion for the world. They meditate observing an aspect of feelings … mind … principles—keen, aware, and mindful, rid of desire and aversion for the world. A trainee is someone who has partly developed the four kinds of mindfulness meditation.” 

%
\section*{{\suttatitleacronym SN 47.27}{\suttatitletranslation Completely }{\suttatitleroot Samattasutta}}
\addcontentsline{toc}{section}{\tocacronym{SN 47.27} \toctranslation{Completely } \tocroot{Samattasutta}}
\markboth{Completely }{Samattasutta}
\extramarks{SN 47.27}{SN 47.27}

The\marginnote{1.1} same setting. 

“Reverend,\marginnote{1.3} they speak of this person called ‘an adept’. How is an adept defined?” 

“Reverends,\marginnote{1.5} an adept is someone who has completely developed the four kinds of mindfulness meditation. 

What\marginnote{2.1} four? It’s when a mendicant meditates by observing an aspect of the body—keen, aware, and mindful, rid of desire and aversion for the world. They meditate observing an aspect of feelings … mind … principles—keen, aware, and mindful, rid of desire and aversion for the world. An adept is someone who has completely developed the four kinds of mindfulness meditation.” 

%
\section*{{\suttatitleacronym SN 47.28}{\suttatitletranslation The World }{\suttatitleroot Lokasutta}}
\addcontentsline{toc}{section}{\tocacronym{SN 47.28} \toctranslation{The World } \tocroot{Lokasutta}}
\markboth{The World }{Lokasutta}
\extramarks{SN 47.28}{SN 47.28}

The\marginnote{1.1} same setting. “Reverend Anuruddha, what things have you developed and cultivated to attain great direct knowledge?” 

“Reverend,\marginnote{1.4} I attained great direct knowledge by developing and cultivating the four kinds of mindfulness meditation. 

What\marginnote{2.1} four? It’s when I meditate observing an aspect of the body—keen, aware, and mindful, rid of desire and aversion for the world. I meditate observing an aspect of feelings … mind … principles—keen, aware, and mindful, rid of desire and aversion for the world. I attained great direct knowledge by developing and cultivating these four kinds of mindfulness meditation. 

And\marginnote{2.7} it’s because of developing and cultivating these four kinds of mindfulness meditation that I directly know the entire galaxy.” 

%
\section*{{\suttatitleacronym SN 47.29}{\suttatitletranslation With Sirivaḍḍha }{\suttatitleroot Sirivaḍḍhasutta}}
\addcontentsline{toc}{section}{\tocacronym{SN 47.29} \toctranslation{With Sirivaḍḍha } \tocroot{Sirivaḍḍhasutta}}
\markboth{With Sirivaḍḍha }{Sirivaḍḍhasutta}
\extramarks{SN 47.29}{SN 47.29}

At\marginnote{1.1} one time Venerable Ānanda was staying near \textsanskrit{Rājagaha}, in the Bamboo Grove, the squirrels’ feeding ground. Now at that time the householder \textsanskrit{Sirivaḍḍha} was sick, suffering, gravely ill. Then he addressed a man: 

“Please,\marginnote{1.4} mister, go to Venerable Ānanda, and in my name bow with your head to his feet. Say to him: ‘Sir, the householder \textsanskrit{Sirivaḍḍha} is sick, suffering, gravely ill. He bows with his head to your feet.’ And then say: ‘Sir, please visit him at his home out of compassion.’” 

“Yes,\marginnote{1.9} sir,” that man replied. He did as \textsanskrit{Sirivaḍḍha} asked. Ānanda consented in silence. 

Then\marginnote{2.1} Venerable Ānanda robed up in the morning and, taking his bowl and robe, went to the home of the householder \textsanskrit{Sirivaḍḍha}, sat down on the seat spread out, and said to him: 

“I\marginnote{2.2} hope you’re keeping well, householder; I hope you’re alright. And I hope the pain is fading, not growing, that its fading is evident, not its growing.” 

“Sir,\marginnote{2.3} I’m not keeping well, I’m not alright. The pain is terrible and growing, not fading; its growing is evident, not its fading.” 

“So\marginnote{3.1} you should train like this: ‘I’ll meditate observing an aspect of the body—keen, aware, and mindful, rid of desire and aversion for the world. I’ll meditate on an aspect of feelings … mind … principles—keen, aware, and mindful, rid of desire and aversion for the world.’ That’s how you should train.” 

“These\marginnote{4.1} four kinds of mindfulness meditation that were taught by the Buddha are found in me, and I am seen in them. For I meditate observing an aspect of the body—keen, aware, and mindful, rid of desire and aversion for the world. I meditate observing an aspect of feelings … mind … principles—keen, aware, and mindful, rid of desire and aversion for the world. And of the five lower fetters taught by the Buddha, I don’t see any that I haven’t given up.” 

“You’re\marginnote{4.7} fortunate, householder, so very fortunate! You have declared the fruit of non-return.” 

%
\section*{{\suttatitleacronym SN 47.30}{\suttatitletranslation With Mānadinna }{\suttatitleroot Mānadinnasutta}}
\addcontentsline{toc}{section}{\tocacronym{SN 47.30} \toctranslation{With Mānadinna } \tocroot{Mānadinnasutta}}
\markboth{With Mānadinna }{Mānadinnasutta}
\extramarks{SN 47.30}{SN 47.30}

The\marginnote{1.1} same setting. Now at that time the householder \textsanskrit{Mānadinna} was sick, suffering, gravely ill. Then he addressed a man: 

“Please,\marginnote{1.4} mister, go to Venerable Ānanda …” … 

“Sir,\marginnote{1.5} I’m not keeping well, I’m not alright. The pain is terrible and growing, not fading; its growing is evident, not its fading. When I experience such painful feelings I meditate observing an aspect of the body—keen, aware, and mindful, rid of desire and aversion for the world. I meditate observing an aspect of feelings … mind … principles—keen, aware, and mindful, rid of desire and aversion for the world. And of the five lower fetters taught by the Buddha, I don’t see any that I haven’t given up.” 

“You’re\marginnote{1.11} fortunate, householder, so very fortunate! You have declared the fruit of non-return.” 

%
\addtocontents{toc}{\let\protect\contentsline\protect\nopagecontentsline}
\chapter*{The Chapter on Not Learned From Anyone Else }
\addcontentsline{toc}{chapter}{\tocchapterline{The Chapter on Not Learned From Anyone Else }}
\addtocontents{toc}{\let\protect\contentsline\protect\oldcontentsline}

%
\section*{{\suttatitleacronym SN 47.31}{\suttatitletranslation Not Learned From Anyone Else }{\suttatitleroot Ananussutasutta}}
\addcontentsline{toc}{section}{\tocacronym{SN 47.31} \toctranslation{Not Learned From Anyone Else } \tocroot{Ananussutasutta}}
\markboth{Not Learned From Anyone Else }{Ananussutasutta}
\extramarks{SN 47.31}{SN 47.31}

At\marginnote{1.1} \textsanskrit{Sāvatthī}. 

“‘This\marginnote{1.2} is the observation of an aspect of the body.’ Such, mendicants, was the vision, knowledge, wisdom, realization, and light that arose in me regarding teachings not learned before from another. ‘This observation of an aspect of the body should be developed.’ … ‘This observation of an aspect of the body has been developed.’ Such was the vision, knowledge, wisdom, realization, and light that arose in me regarding teachings not learned before from another. 

‘This\marginnote{2.1} is the observation of an aspect of feelings.’ … ‘This observation of an aspect of feelings should be developed.’ … ‘This observation of an aspect of feelings has been developed.’ … 

‘This\marginnote{3.1} is the observation of an aspect of the mind.’ … ‘This observation of an aspect of the mind should be developed.’ … ‘This observation of an aspect of the mind has been developed.’ … 

‘This\marginnote{4.1} is the observation of an aspect of principles.’ … ‘This observation of an aspect of principles should be developed.’ … ‘This observation of an aspect of principles has been developed.’ Such was the vision, knowledge, wisdom, realization, and light that arose in me regarding teachings not learned before from another.” 

%
\section*{{\suttatitleacronym SN 47.32}{\suttatitletranslation Fading Away }{\suttatitleroot Virāgasutta}}
\addcontentsline{toc}{section}{\tocacronym{SN 47.32} \toctranslation{Fading Away } \tocroot{Virāgasutta}}
\markboth{Fading Away }{Virāgasutta}
\extramarks{SN 47.32}{SN 47.32}

“Mendicants,\marginnote{1.1} these four kinds of mindfulness meditation, when developed and cultivated, lead solely to disillusionment, dispassion, cessation, peace, insight, awakening, and extinguishment. 

What\marginnote{2.1} four? It’s when a mendicant meditates by observing an aspect of the body—keen, aware, and mindful, rid of desire and aversion for the world. They meditate observing an aspect of feelings … mind … principles—keen, aware, and mindful, rid of desire and aversion for the world. 

These\marginnote{2.6} four kinds of mindfulness meditation, when developed and cultivated, lead solely to disillusionment, dispassion, cessation, peace, insight, awakening, and extinguishment.” 

%
\section*{{\suttatitleacronym SN 47.33}{\suttatitletranslation Missed Out }{\suttatitleroot Viraddhasutta}}
\addcontentsline{toc}{section}{\tocacronym{SN 47.33} \toctranslation{Missed Out } \tocroot{Viraddhasutta}}
\markboth{Missed Out }{Viraddhasutta}
\extramarks{SN 47.33}{SN 47.33}

“Mendicants,\marginnote{1.1} whoever has missed out on the four kinds of mindfulness meditation has missed out on the noble path to the complete ending of suffering. Whoever has undertaken the four kinds of mindfulness meditation has undertaken the noble path to the complete ending of suffering. 

What\marginnote{2.1} four? It’s when a mendicant meditates by observing an aspect of the body—keen, aware, and mindful, rid of desire and aversion for the world. They meditate observing an aspect of feelings … mind … principles—keen, aware, and mindful, rid of desire and aversion for the world. 

Whoever\marginnote{2.6} has missed out on these four kinds of mindfulness meditation has missed out on the noble path to the complete ending of suffering. Whoever has undertaken these four kinds of mindfulness meditation has undertaken the noble path to the complete ending of suffering.” 

%
\section*{{\suttatitleacronym SN 47.34}{\suttatitletranslation Developed }{\suttatitleroot Bhāvitasutta}}
\addcontentsline{toc}{section}{\tocacronym{SN 47.34} \toctranslation{Developed } \tocroot{Bhāvitasutta}}
\markboth{Developed }{Bhāvitasutta}
\extramarks{SN 47.34}{SN 47.34}

“Mendicants,\marginnote{1.1} when these four kinds of mindfulness meditation are developed and cultivated they lead to going from the near shore to the far shore. 

What\marginnote{2.1} four? It’s when a mendicant meditates by observing an aspect of the body—keen, aware, and mindful, rid of desire and aversion for the world. They meditate observing an aspect of feelings … mind … principles—keen, aware, and mindful, rid of desire and aversion for the world. 

When\marginnote{2.6} these four kinds of mindfulness meditation are developed and cultivated they lead to going from the near shore to the far shore.” 

%
\section*{{\suttatitleacronym SN 47.35}{\suttatitletranslation Mindful }{\suttatitleroot Satisutta}}
\addcontentsline{toc}{section}{\tocacronym{SN 47.35} \toctranslation{Mindful } \tocroot{Satisutta}}
\markboth{Mindful }{Satisutta}
\extramarks{SN 47.35}{SN 47.35}

At\marginnote{1.1} \textsanskrit{Sāvatthī}. 

“Mendicants,\marginnote{1.2} a mendicant should live mindful and aware. This is my instruction to you. 

And\marginnote{2.1} how is a mendicant mindful? It’s when a mendicant meditates by observing an aspect of the body—keen, aware, and mindful, rid of desire and aversion for the world. They meditate observing an aspect of feelings … mind … principles—keen, aware, and mindful, rid of desire and aversion for the world. That’s how a mendicant is mindful. 

And\marginnote{3.1} how is a mendicant aware? It’s when a mendicant knows feelings as they arise, as they remain, and as they go away. They know thoughts as they arise, as they remain, and as they go away. They know perceptions as they arise, as they remain, and as they go away. That’s how a mendicant is aware. A mendicant should live mindful and aware. This is my instruction to you.” 

%
\section*{{\suttatitleacronym SN 47.36}{\suttatitletranslation Enlightenment }{\suttatitleroot Aññāsutta}}
\addcontentsline{toc}{section}{\tocacronym{SN 47.36} \toctranslation{Enlightenment } \tocroot{Aññāsutta}}
\markboth{Enlightenment }{Aññāsutta}
\extramarks{SN 47.36}{SN 47.36}

“Mendicants,\marginnote{1.1} there are these four kinds of mindfulness meditation. What four? It’s when a mendicant meditates by observing an aspect of the body—keen, aware, and mindful, rid of desire and aversion for the world. They meditate observing an aspect of feelings … mind … principles—keen, aware, and mindful, rid of desire and aversion for the world. These are the four kinds of mindfulness meditation. 

Because\marginnote{1.8} of developing and cultivating these four kinds of mindfulness meditation, one of two results can be expected: enlightenment in the present life, or if there’s something left over, non-return.” 

%
\section*{{\suttatitleacronym SN 47.37}{\suttatitletranslation Desire }{\suttatitleroot Chandasutta}}
\addcontentsline{toc}{section}{\tocacronym{SN 47.37} \toctranslation{Desire } \tocroot{Chandasutta}}
\markboth{Desire }{Chandasutta}
\extramarks{SN 47.37}{SN 47.37}

“Mendicants,\marginnote{1.1} there are these four kinds of mindfulness meditation. What four? It’s when a mendicant meditates by observing an aspect of the body—keen, aware, and mindful, rid of desire and aversion for the world. As they do so they give up desire for the body. When desire is given up they realize the deathless. 

They\marginnote{2.1} meditate observing an aspect of feelings—keen, aware, and mindful, rid of desire and aversion for the world. As they do so they give up desire for feelings. When desire is given up they realize the deathless. 

They\marginnote{3.1} meditate observing an aspect of the mind—keen, aware, and mindful, rid of desire and aversion for the world. As they do so they give up desire for the mind. When desire is given up they realize the deathless. 

They\marginnote{4.1} meditate observing an aspect of principles—keen, aware, and mindful, rid of desire and aversion for the world. As they do so they give up desire for principles. When desire is given up they realize the deathless.” 

%
\section*{{\suttatitleacronym SN 47.38}{\suttatitletranslation Complete Understanding }{\suttatitleroot Pariññātasutta}}
\addcontentsline{toc}{section}{\tocacronym{SN 47.38} \toctranslation{Complete Understanding } \tocroot{Pariññātasutta}}
\markboth{Complete Understanding }{Pariññātasutta}
\extramarks{SN 47.38}{SN 47.38}

“Mendicants,\marginnote{1.1} there are these four kinds of mindfulness meditation. What four? It’s when a mendicant meditates by observing an aspect of the body—keen, aware, and mindful, rid of desire and aversion for the world. As they do so they completely understand the body. When the body is completely understood they realize the deathless. 

They\marginnote{2.1} meditate observing an aspect of feelings—keen, aware, and mindful, rid of desire and aversion for the world. As they do so they completely understand feelings. When feelings are completely understood they realize the deathless. 

They\marginnote{3.1} meditate observing an aspect of the mind—keen, aware, and mindful, rid of desire and aversion for the world. As they do so they completely understand the mind. When the mind is completely understood they realize the deathless. 

They\marginnote{4.1} meditate observing an aspect of principles—keen, aware, and mindful, rid of desire and aversion for the world. As they do so they completely understand principles. When principles are completely understood they realize the deathless.” 

%
\section*{{\suttatitleacronym SN 47.39}{\suttatitletranslation Development }{\suttatitleroot Bhāvanāsutta}}
\addcontentsline{toc}{section}{\tocacronym{SN 47.39} \toctranslation{Development } \tocroot{Bhāvanāsutta}}
\markboth{Development }{Bhāvanāsutta}
\extramarks{SN 47.39}{SN 47.39}

“Mendicants,\marginnote{1.1} I will teach you the development of the four kinds of mindfulness meditation. Listen … 

And\marginnote{1.3} what is the development of the four kinds of mindfulness meditation? It’s when a mendicant meditates by observing an aspect of the body—keen, aware, and mindful, rid of desire and aversion for the world. They meditate observing an aspect of feelings … mind … principles—keen, aware, and mindful, rid of desire and aversion for the world. 

This\marginnote{1.8} is the development of the four kinds of mindfulness meditation.” 

%
\section*{{\suttatitleacronym SN 47.40}{\suttatitletranslation Analysis }{\suttatitleroot Vibhaṅgasutta}}
\addcontentsline{toc}{section}{\tocacronym{SN 47.40} \toctranslation{Analysis } \tocroot{Vibhaṅgasutta}}
\markboth{Analysis }{Vibhaṅgasutta}
\extramarks{SN 47.40}{SN 47.40}

“Mendicants,\marginnote{1.1} I will teach you mindfulness meditation, the development of mindfulness meditation, and the practice that leads to the development of mindfulness meditation. Listen … 

And\marginnote{1.3} what is mindfulness meditation? It’s when a mendicant meditates by observing an aspect of the body—keen, aware, and mindful, rid of desire and aversion for the world. They meditate observing an aspect of feelings … mind … principles—keen, aware, and mindful, rid of desire and aversion for the world. This is called mindfulness meditation. 

And\marginnote{2.1} what is the development of mindfulness meditation? It’s when a mendicant meditates observing the body as liable to originate, as liable to vanish, and as liable to originate and vanish—keen, aware, and mindful, rid of desire and aversion for the world. They observe feelings … mind … principles as liable to originate, as liable to vanish, and as liable to originate and vanish—keen, aware, and mindful, rid of desire and aversion for the world. This is called the development of mindfulness meditation. 

And\marginnote{3.1} what is the practice that leads to the development of mindfulness meditation? It is simply this noble eightfold path, that is: right view, right thought, right speech, right action, right livelihood, right effort, right mindfulness, and right immersion. This is called the practice that leads to the development of mindfulness meditation.” 

%
\addtocontents{toc}{\let\protect\contentsline\protect\nopagecontentsline}
\chapter*{The Chapter on the Deathless }
\addcontentsline{toc}{chapter}{\tocchapterline{The Chapter on the Deathless }}
\addtocontents{toc}{\let\protect\contentsline\protect\oldcontentsline}

%
\section*{{\suttatitleacronym SN 47.41}{\suttatitletranslation The Deathless }{\suttatitleroot Amatasutta}}
\addcontentsline{toc}{section}{\tocacronym{SN 47.41} \toctranslation{The Deathless } \tocroot{Amatasutta}}
\markboth{The Deathless }{Amatasutta}
\extramarks{SN 47.41}{SN 47.41}

At\marginnote{1.1} \textsanskrit{Sāvatthī}. 

“Mendicants,\marginnote{1.2} you should meditate with your mind firmly established in the four kinds of mindfulness meditation. Don’t let the deathless escape you. What four? It’s when a mendicant meditates by observing an aspect of the body—keen, aware, and mindful, rid of desire and aversion for the world. They meditate observing an aspect of feelings … mind … principles—keen, aware, and mindful, rid of desire and aversion for the world. You should meditate with your mind firmly established in the four kinds of mindfulness meditation. Don’t let the deathless escape you.” 

%
\section*{{\suttatitleacronym SN 47.42}{\suttatitletranslation Origin }{\suttatitleroot Samudayasutta}}
\addcontentsline{toc}{section}{\tocacronym{SN 47.42} \toctranslation{Origin } \tocroot{Samudayasutta}}
\markboth{Origin }{Samudayasutta}
\extramarks{SN 47.42}{SN 47.42}

“Mendicants,\marginnote{1.1} I will teach you the origin and the ending of the four kinds of mindfulness meditation. Listen … 

And\marginnote{1.3} what is the origin of the body? The body originates from food. When food ceases, the body ends. 

Feelings\marginnote{1.6} originate from contact. When contact ceases, feelings end. 

The\marginnote{1.8} mind originates from name and form. When name and form cease, the mind ends. 

Principles\marginnote{1.10} originate from attention. When focus ends, principles end.” 

%
\section*{{\suttatitleacronym SN 47.43}{\suttatitletranslation The Path }{\suttatitleroot Maggasutta}}
\addcontentsline{toc}{section}{\tocacronym{SN 47.43} \toctranslation{The Path } \tocroot{Maggasutta}}
\markboth{The Path }{Maggasutta}
\extramarks{SN 47.43}{SN 47.43}

At\marginnote{1.1} \textsanskrit{Sāvatthī}. 

There\marginnote{1.2} the Buddha addressed the mendicants: “Mendicants, at one time, when I was first awakened, I was staying near \textsanskrit{Uruvelā} at the goatherd’s banyan tree on the bank of the \textsanskrit{Nerañjarā} River. As I was in private retreat this thought came to mind: ‘The four kinds of mindfulness meditation are the path to convergence. They are in order to purify sentient beings, to get past sorrow and crying, to make an end of pain and sadness, to end the cycle of suffering, and to realize extinguishment. 

What\marginnote{2.1} four? A mendicant would meditate observing an aspect of the body—keen, aware, and mindful, rid of desire and aversion for the world. Or they’d meditate observing an aspect of feelings … or mind … or principles—keen, aware, and mindful, rid of desire and aversion for the world. The four kinds of mindfulness meditation are the path to convergence. They are in order to purify sentient beings, to get past sorrow and crying, to make an end of pain and sadness, to end the cycle of suffering, and to realize extinguishment.’ 

And\marginnote{3.1} then \textsanskrit{Brahmā} Sahampati, knowing what I was thinking, as easily as a strong person would extend or contract their arm, vanished from the \textsanskrit{Brahmā} realm and reappeared in front of me. He arranged his robe over one shoulder, raised his joined palms toward the Buddha, and said: ‘That’s so true, Blessed One! That’s so true, Holy One! Sir, the four kinds of mindfulness meditation are the path to convergence. They are in order to purify sentient beings, to get past sorrow and crying, to make an end of pain and sadness, to end the cycle of suffering, and to realize extinguishment. 

What\marginnote{4.1} four? A mendicant would meditate observing an aspect of the body—keen, aware, and mindful, rid of desire and aversion for the world. Or they’d meditate observing an aspect of feelings … or mind … or principles—keen, aware, and mindful, rid of desire and aversion for the world. The four kinds of mindfulness meditation are the path to convergence. They are in order to purify sentient beings, to get past sorrow and crying, to make an end of pain and sadness, to end the cycle of suffering, and to realize extinguishment.’ 

That’s\marginnote{5.1} what \textsanskrit{Brahmā} Sahampati said. Then he went on to say: 

\begin{verse}%
‘The\marginnote{6.1} compassionate one, who sees the ending of rebirth, \\
understands the path to convergence. \\
By this path people crossed over before, \\
will cross over, and are crossing over.’” 

%
\end{verse}

%
\section*{{\suttatitleacronym SN 47.44}{\suttatitletranslation Mindful }{\suttatitleroot Satisutta}}
\addcontentsline{toc}{section}{\tocacronym{SN 47.44} \toctranslation{Mindful } \tocroot{Satisutta}}
\markboth{Mindful }{Satisutta}
\extramarks{SN 47.44}{SN 47.44}

“Mendicants,\marginnote{1.1} a mendicant should live mindfully. This is my instruction to you. 

And\marginnote{1.3} how is a mendicant mindful? It’s when a mendicant meditates by observing an aspect of the body—keen, aware, and mindful, rid of desire and aversion for the world. They meditate observing an aspect of feelings … mind … principles—keen, aware, and mindful, rid of desire and aversion for the world. That’s how a mendicant is mindful. A mendicant should live mindfully. This is my instruction to you.” 

%
\section*{{\suttatitleacronym SN 47.45}{\suttatitletranslation A Heap of the Skillful }{\suttatitleroot Kusalarāsisutta}}
\addcontentsline{toc}{section}{\tocacronym{SN 47.45} \toctranslation{A Heap of the Skillful } \tocroot{Kusalarāsisutta}}
\markboth{A Heap of the Skillful }{Kusalarāsisutta}
\extramarks{SN 47.45}{SN 47.45}

“Rightly\marginnote{1.1} speaking, mendicants, you’d call these four kinds of mindfulness meditation a ‘heap of the skillful’. For these four kinds of mindfulness meditation are entirely a heap of the skillful. 

What\marginnote{2.1} four? It’s when a mendicant meditates by observing an aspect of the body—keen, aware, and mindful, rid of desire and aversion for the world. They meditate observing an aspect of feelings … mind … principles—keen, aware, and mindful, rid of desire and aversion for the world. 

Rightly\marginnote{2.6} speaking, you’d call these four kinds of mindfulness meditation a ‘heap of the skillful’. For these four kinds of mindfulness meditation are entirely a heap of the skillful.” 

%
\section*{{\suttatitleacronym SN 47.46}{\suttatitletranslation Restraint in the Monastic Code }{\suttatitleroot Pātimokkhasaṁvarasutta}}
\addcontentsline{toc}{section}{\tocacronym{SN 47.46} \toctranslation{Restraint in the Monastic Code } \tocroot{Pātimokkhasaṁvarasutta}}
\markboth{Restraint in the Monastic Code }{Pātimokkhasaṁvarasutta}
\extramarks{SN 47.46}{SN 47.46}

Then\marginnote{1.1} a mendicant went up to the Buddha, bowed, sat down to one side, and said to him: 

“Sir,\marginnote{2.1} may the Buddha please teach me Dhamma in brief. When I’ve heard it, I’ll live alone, withdrawn, diligent, keen, and resolute.” 

“Well\marginnote{2.2} then, mendicant, you should purify the starting point of skillful qualities. What is the starting point of skillful qualities? Live restrained in the monastic code, conducting yourself well and seeking alms in suitable places. Seeing danger in the slightest fault, keep the rules you’ve undertaken. When you’ve done this, you should develop the four kinds of mindfulness meditation, depending on and grounded on ethics. 

What\marginnote{3.1} four? Meditate observing an aspect of the body—keen, aware, and mindful, rid of desire and aversion for the world. Meditate observing an aspect of feelings … mind … principles—keen, aware, and mindful, rid of desire and aversion for the world. 

When\marginnote{3.6} you develop the four kinds of mindfulness meditation in this way, depending on and grounded on ethics, you can expect growth, not decline, in skillful qualities, whether by day or by night.” 

And\marginnote{3.7} then that mendicant approved and agreed with what the Buddha said. He got up from his seat, bowed, and respectfully circled the Buddha, keeping him on his right, before leaving. 

Then\marginnote{4.1} that mendicant, living alone, withdrawn, diligent, keen, and resolute, soon realized the supreme end of the spiritual path in this very life. He lived having achieved with his own insight the goal for which gentlemen rightly go forth from the lay life to homelessness. 

He\marginnote{4.2} understood: “Rebirth is ended; the spiritual journey has been completed; what had to be done has been done; there is no return to any state of existence.” And that mendicant became one of the perfected. 

%
\section*{{\suttatitleacronym SN 47.47}{\suttatitletranslation Bad Conduct }{\suttatitleroot Duccaritasutta}}
\addcontentsline{toc}{section}{\tocacronym{SN 47.47} \toctranslation{Bad Conduct } \tocroot{Duccaritasutta}}
\markboth{Bad Conduct }{Duccaritasutta}
\extramarks{SN 47.47}{SN 47.47}

Then\marginnote{1.1} a mendicant went up to the Buddha … and said: 

“Sir,\marginnote{1.2} may the Buddha please teach me Dhamma in brief. When I’ve heard it, I’ll live alone, withdrawn, diligent, keen, and resolute.” 

“Well\marginnote{1.3} then, mendicant, you should purify skillful qualities starting from the beginning. What is the beginning of skillful qualities? Give up bad conduct by way of body, speech, and mind and develop good conduct by way of body, speech, and mind. When you’ve done this, you should develop the four kinds of mindfulness meditation, depending on and grounded on ethics. 

What\marginnote{2.1} four? Meditate observing an aspect of the body—keen, aware, and mindful, rid of desire and aversion for the world. Meditate observing an aspect of feelings … mind … principles—keen, aware, and mindful, rid of desire and aversion for the world. When you develop the four kinds of mindfulness meditation in this way, depending on and grounded on ethics, you can expect growth, not decline, in skillful qualities, whether by day or by night.” … 

And\marginnote{2.7} that mendicant became one of the perfected. 

%
\section*{{\suttatitleacronym SN 47.48}{\suttatitletranslation Friends }{\suttatitleroot Mittasutta}}
\addcontentsline{toc}{section}{\tocacronym{SN 47.48} \toctranslation{Friends } \tocroot{Mittasutta}}
\markboth{Friends }{Mittasutta}
\extramarks{SN 47.48}{SN 47.48}

“Mendicants,\marginnote{1.1} those for whom you have sympathy, and those worth listening to—friends and colleagues, relatives and family—should be encouraged, supported, and established in the development of the four kinds of mindfulness meditation. 

What\marginnote{2.1} four? It’s when a mendicant meditates by observing an aspect of the body—keen, aware, and mindful, rid of desire and aversion for the world. They meditate observing an aspect of feelings … mind … principles—keen, aware, and mindful, rid of desire and aversion for the world. 

Those\marginnote{2.6} for whom you have sympathy, and those worth listening to—friends and colleagues, relatives and family—should be encouraged, supported, and established in the development of the four kinds of mindfulness meditation.” 

%
\section*{{\suttatitleacronym SN 47.49}{\suttatitletranslation Feelings }{\suttatitleroot Vedanāsutta}}
\addcontentsline{toc}{section}{\tocacronym{SN 47.49} \toctranslation{Feelings } \tocroot{Vedanāsutta}}
\markboth{Feelings }{Vedanāsutta}
\extramarks{SN 47.49}{SN 47.49}

“Mendicants,\marginnote{1.1} there are these three feelings. What three? Pleasant, painful, and neutral feeling. These are the three feelings. The four kinds of mindfulness meditation should be developed to completely understand these three feelings. 

What\marginnote{2.1} four? It’s when a mendicant meditates by observing an aspect of the body—keen, aware, and mindful, rid of desire and aversion for the world. They meditate observing an aspect of feelings … mind … principles—keen, aware, and mindful, rid of desire and aversion for the world. 

These\marginnote{2.6} four kinds of mindfulness meditation should be developed to completely understand these three feelings.” 

%
\section*{{\suttatitleacronym SN 47.50}{\suttatitletranslation Defilements }{\suttatitleroot Āsavasutta}}
\addcontentsline{toc}{section}{\tocacronym{SN 47.50} \toctranslation{Defilements } \tocroot{Āsavasutta}}
\markboth{Defilements }{Āsavasutta}
\extramarks{SN 47.50}{SN 47.50}

“Mendicants,\marginnote{1.1} there are these three defilements. What three? The defilements of sensuality, desire to be reborn, and ignorance. These are the three defilements. 

The\marginnote{1.5} four kinds of mindfulness meditation should be developed to give up these three defilements. 

What\marginnote{2.1} four? It’s when a mendicant meditates by observing an aspect of the body—keen, aware, and mindful, rid of desire and aversion for the world. They meditate observing an aspect of feelings … mind … principles—keen, aware, and mindful, rid of desire and aversion for the world. 

These\marginnote{2.6} four kinds of mindfulness meditation should be developed to give up these three defilements.” 

%
\addtocontents{toc}{\let\protect\contentsline\protect\nopagecontentsline}
\chapter*{The Chapter of Abbreviated Texts on the Ganges }
\addcontentsline{toc}{chapter}{\tocchapterline{The Chapter of Abbreviated Texts on the Ganges }}
\addtocontents{toc}{\let\protect\contentsline\protect\oldcontentsline}

%
\section*{{\suttatitleacronym SN 47.51–62}{\suttatitletranslation Twelve Discourses on the Ganges River, Etc. }{\suttatitleroot Gaṅgāpeyyālavagga}}
\addcontentsline{toc}{section}{\tocacronym{SN 47.51–62} \toctranslation{Twelve Discourses on the Ganges River, Etc. } \tocroot{Gaṅgāpeyyālavagga}}
\markboth{Twelve Discourses on the Ganges River, Etc. }{Gaṅgāpeyyālavagga}
\extramarks{SN 47.51–62}{SN 47.51–62}

“Mendicants,\marginnote{1.1} the Ganges river slants, slopes, and inclines to the east. In the same way, a mendicant who develops and cultivates the four kinds of mindfulness meditation slants, slopes, and inclines to extinguishment. 

And\marginnote{2.1} how does a mendicant who develops the four kinds of mindfulness meditation slant, slope, and incline to extinguishment? It’s when a mendicant meditates by observing an aspect of the body—keen, aware, and mindful, rid of desire and aversion for the world. They meditate observing an aspect of feelings … mind … principles—keen, aware, and mindful, rid of desire and aversion for the world. 

That’s\marginnote{2.6} how a mendicant who develops and cultivates the four kinds of mindfulness meditation slants, slopes, and inclines to extinguishment.” 

\scendsection{(To be expanded for each of the different rivers as in SN 45.91–102.) }

\begin{quotation}%
Six\marginnote{3.1} on slanting to the east, \\
and six on slanting to the ocean; \\
these two sixes make twelve, \\
and that’s how this chapter is recited. 

%
\end{quotation}

%
\addtocontents{toc}{\let\protect\contentsline\protect\nopagecontentsline}
\chapter*{The Chapter on Diligence }
\addcontentsline{toc}{chapter}{\tocchapterline{The Chapter on Diligence }}
\addtocontents{toc}{\let\protect\contentsline\protect\oldcontentsline}

%
\section*{{\suttatitleacronym SN 47.63–72}{\suttatitletranslation The Realized One }{\suttatitleroot Appamādavagga}}
\addcontentsline{toc}{section}{\tocacronym{SN 47.63–72} \toctranslation{The Realized One } \tocroot{Appamādavagga}}
\markboth{The Realized One }{Appamādavagga}
\extramarks{SN 47.63–72}{SN 47.63–72}

“Mendicants,\marginnote{1.1} the Realized One, the perfected one, the fully awakened Buddha, is said to be the best of all sentient beings—be they footless, with two feet, four feet, or many feet …” 

\scendsection{(To be expanded as in SN 45.139–148.) }

\begin{quotation}%
The\marginnote{2.1} Realized One, footprint, roof peak, \\
roots, heartwood, jasmine, \\
monarch, sun and moon, \\
and cloth is the tenth. 

%
\end{quotation}

%
\addtocontents{toc}{\let\protect\contentsline\protect\nopagecontentsline}
\chapter*{The Chapter on Hard Work }
\addcontentsline{toc}{chapter}{\tocchapterline{The Chapter on Hard Work }}
\addtocontents{toc}{\let\protect\contentsline\protect\oldcontentsline}

%
\section*{{\suttatitleacronym SN 47.73–84}{\suttatitletranslation Hard Work, Etc. }{\suttatitleroot Balakaraṇīyavagga}}
\addcontentsline{toc}{section}{\tocacronym{SN 47.73–84} \toctranslation{Hard Work, Etc. } \tocroot{Balakaraṇīyavagga}}
\markboth{Hard Work, Etc. }{Balakaraṇīyavagga}
\extramarks{SN 47.73–84}{SN 47.73–84}

“Mendicants,\marginnote{1.1} all the hard work that gets done depends on the earth and is grounded on the earth. …” 

\scendsection{(To be expanded as in SN 45.149–160.) }

\begin{quotation}%
Hard\marginnote{2.1} work, seeds, and dragons, \\
a tree, a pot, and a spike, \\
the sky, and two on clouds, \\
a ship, a guest house, and a river. 

%
\end{quotation}

%
\addtocontents{toc}{\let\protect\contentsline\protect\nopagecontentsline}
\chapter*{The Chapter on Searches }
\addcontentsline{toc}{chapter}{\tocchapterline{The Chapter on Searches }}
\addtocontents{toc}{\let\protect\contentsline\protect\oldcontentsline}

%
\section*{{\suttatitleacronym SN 47.85–94}{\suttatitletranslation Searches, Etc. }{\suttatitleroot Esanāvagga}}
\addcontentsline{toc}{section}{\tocacronym{SN 47.85–94} \toctranslation{Searches, Etc. } \tocroot{Esanāvagga}}
\markboth{Searches, Etc. }{Esanāvagga}
\extramarks{SN 47.85–94}{SN 47.85–94}

“Mendicants,\marginnote{1.1} there are these three searches. What three? The search for sensual pleasures, the search for continued existence, and the search for a spiritual path. …” 

\scendsection{(To be expanded as in SN 45.161–170.) }

\begin{quotation}%
Searches,\marginnote{2.1} discriminations, defilements, \\
states of existence, three kinds of suffering, \\
barrenness, stains, and troubles, \\
feelings, craving, and thirst. 

%
\end{quotation}

%
\addtocontents{toc}{\let\protect\contentsline\protect\nopagecontentsline}
\chapter*{The Chapter on Floods }
\addcontentsline{toc}{chapter}{\tocchapterline{The Chapter on Floods }}
\addtocontents{toc}{\let\protect\contentsline\protect\oldcontentsline}

%
\section*{{\suttatitleacronym SN 47.95–104}{\suttatitletranslation Higher Fetters, Etc. }{\suttatitleroot Oghavagga}}
\addcontentsline{toc}{section}{\tocacronym{SN 47.95–104} \toctranslation{Higher Fetters, Etc. } \tocroot{Oghavagga}}
\markboth{Higher Fetters, Etc. }{Oghavagga}
\extramarks{SN 47.95–104}{SN 47.95–104}

(To\marginnote{1.1} be expanded as in SN 45.171–179, with the following as the final discourse.) “Mendicants, there are five higher fetters. What five? Desire for rebirth in the realm of luminous form, desire for rebirth in the formless realm, conceit, restlessness, and ignorance. These are the five higher fetters. 

The\marginnote{1.5} four kinds of mindfulness meditation should be developed for the direct knowledge, complete understanding, finishing, and giving up of these five higher fetters. 

What\marginnote{2.1} four? It’s when a mendicant meditates by observing an aspect of the body—keen, aware, and mindful, rid of desire and aversion for the world. They meditate observing an aspect of feelings … mind … principles—keen, aware, and mindful, rid of desire and aversion for the world. 

These\marginnote{2.6} four kinds of mindfulness meditation should be developed for the direct knowledge, complete understanding, finishing, and giving up of these five higher fetters.” 

(The\marginnote{3.1} Linked Discourses on Mindfulness Meditation should be expanded as in the Linked Discourses on the Path.) 

\begin{quotation}%
Floods,\marginnote{4.1} bonds, grasping, \\
ties, and underlying tendencies, \\
kinds of sensual stimulation, hindrances, \\
aggregates, and fetters high and low. 

%
\end{quotation}

\scendsutta{The Linked Discourses on Mindfulness Meditation is the third section. }

%
\addtocontents{toc}{\let\protect\contentsline\protect\nopagecontentsline}
\part*{Linked Discourses on the Faculties }
\addcontentsline{toc}{part}{Linked Discourses on the Faculties }
\markboth{}{}
\addtocontents{toc}{\let\protect\contentsline\protect\oldcontentsline}

%
\addtocontents{toc}{\let\protect\contentsline\protect\nopagecontentsline}
\chapter*{The Chapter on the Plain Version }
\addcontentsline{toc}{chapter}{\tocchapterline{The Chapter on the Plain Version }}
\addtocontents{toc}{\let\protect\contentsline\protect\oldcontentsline}

%
\section*{{\suttatitleacronym SN 48.1}{\suttatitletranslation Plain Version }{\suttatitleroot Suddhikasutta}}
\addcontentsline{toc}{section}{\tocacronym{SN 48.1} \toctranslation{Plain Version } \tocroot{Suddhikasutta}}
\markboth{Plain Version }{Suddhikasutta}
\extramarks{SN 48.1}{SN 48.1}

At\marginnote{1.1} \textsanskrit{Sāvatthī}. 

There\marginnote{1.2} the Buddha said: 

“Mendicants,\marginnote{1.3} there are these five faculties. What five? The faculties of faith, energy, mindfulness, immersion, and wisdom. These are the five faculties.” 

%
\section*{{\suttatitleacronym SN 48.2}{\suttatitletranslation A Stream-Enterer (1st) }{\suttatitleroot Paṭhamasotāpannasutta}}
\addcontentsline{toc}{section}{\tocacronym{SN 48.2} \toctranslation{A Stream-Enterer (1st) } \tocroot{Paṭhamasotāpannasutta}}
\markboth{A Stream-Enterer (1st) }{Paṭhamasotāpannasutta}
\extramarks{SN 48.2}{SN 48.2}

“Mendicants,\marginnote{1.1} there are these five faculties. What five? The faculties of faith, energy, mindfulness, immersion, and wisdom. A noble disciple comes to truly understand these five faculties’ gratification, drawback, and escape. Such a noble disciple is called a stream-enterer, not liable to be reborn in the underworld, bound for awakening.” 

%
\section*{{\suttatitleacronym SN 48.3}{\suttatitletranslation A Stream-Enterer (2nd) }{\suttatitleroot Dutiyasotāpannasutta}}
\addcontentsline{toc}{section}{\tocacronym{SN 48.3} \toctranslation{A Stream-Enterer (2nd) } \tocroot{Dutiyasotāpannasutta}}
\markboth{A Stream-Enterer (2nd) }{Dutiyasotāpannasutta}
\extramarks{SN 48.3}{SN 48.3}

“Mendicants,\marginnote{1.1} there are these five faculties. What five? The faculties of faith, energy, mindfulness, immersion, and wisdom. A noble disciple comes to truly understand these five faculties’ origin, ending, gratification, drawback, and escape. Such a noble disciple is called a stream-enterer, not liable to be reborn in the underworld, bound for awakening.” 

%
\section*{{\suttatitleacronym SN 48.4}{\suttatitletranslation A Perfected One (1st) }{\suttatitleroot Paṭhamaarahantasutta}}
\addcontentsline{toc}{section}{\tocacronym{SN 48.4} \toctranslation{A Perfected One (1st) } \tocroot{Paṭhamaarahantasutta}}
\markboth{A Perfected One (1st) }{Paṭhamaarahantasutta}
\extramarks{SN 48.4}{SN 48.4}

“Mendicants,\marginnote{1.1} there are these five faculties. What five? The faculties of faith, energy, mindfulness, immersion, and wisdom. A noble disciple comes to be freed by not grasping after truly understanding these five faculties’ gratification, drawback, and escape. Such a mendicant is called a perfected one, with defilements ended, who has completed the spiritual journey, done what had to be done, laid down the burden, achieved their own true goal, utterly ended the fetters of rebirth, and is rightly freed through enlightenment.” 

%
\section*{{\suttatitleacronym SN 48.5}{\suttatitletranslation A Perfected One (2nd) }{\suttatitleroot Dutiyaarahantasutta}}
\addcontentsline{toc}{section}{\tocacronym{SN 48.5} \toctranslation{A Perfected One (2nd) } \tocroot{Dutiyaarahantasutta}}
\markboth{A Perfected One (2nd) }{Dutiyaarahantasutta}
\extramarks{SN 48.5}{SN 48.5}

“Mendicants,\marginnote{1.1} there are these five faculties. What five? The faculties of faith, energy, mindfulness, immersion, and wisdom. A mendicant comes to be freed by not grasping after truly understanding these five faculties’ origin, ending, gratification, drawback, and escape. Such a mendicant is called a perfected one, with defilements ended, who has completed the spiritual journey, done what had to be done, laid down the burden, achieved their own true goal, utterly ended the fetters of rebirth, and is rightly freed through enlightenment.” 

%
\section*{{\suttatitleacronym SN 48.6}{\suttatitletranslation Ascetics and Brahmins (1st) }{\suttatitleroot Paṭhamasamaṇabrāhmaṇasutta}}
\addcontentsline{toc}{section}{\tocacronym{SN 48.6} \toctranslation{Ascetics and Brahmins (1st) } \tocroot{Paṭhamasamaṇabrāhmaṇasutta}}
\markboth{Ascetics and Brahmins (1st) }{Paṭhamasamaṇabrāhmaṇasutta}
\extramarks{SN 48.6}{SN 48.6}

“Mendicants,\marginnote{1.1} there are these five faculties. What five? The faculties of faith, energy, mindfulness, immersion, and wisdom. 

There\marginnote{1.4} are ascetics and brahmins who don’t truly understand the gratification, drawback, and escape when it comes to these five faculties. I don’t regard them as true ascetics and brahmins. Those venerables don’t realize the goal of life as an ascetic or brahmin, and don’t live having realized it with their own insight. 

There\marginnote{2.1} are ascetics and brahmins who do truly understand the gratification, drawback, and escape when it comes to these five faculties. I regard them as true ascetics and brahmins. Those venerables realize the goal of life as an ascetic or brahmin, and live having realized it with their own insight.” 

%
\section*{{\suttatitleacronym SN 48.7}{\suttatitletranslation Ascetics and Brahmins (2nd) }{\suttatitleroot Dutiyasamaṇabrāhmaṇasutta}}
\addcontentsline{toc}{section}{\tocacronym{SN 48.7} \toctranslation{Ascetics and Brahmins (2nd) } \tocroot{Dutiyasamaṇabrāhmaṇasutta}}
\markboth{Ascetics and Brahmins (2nd) }{Dutiyasamaṇabrāhmaṇasutta}
\extramarks{SN 48.7}{SN 48.7}

“Mendicants,\marginnote{1.1} there are ascetics and brahmins who don’t understand the faculty of faith, its origin, its cessation, and the practice that leads to its cessation. They don’t understand the faculty of energy … mindfulness … immersion … wisdom, its origin, its cessation, and the practice that leads to its cessation. I don’t regard them as true ascetics and brahmins. Those venerables don’t realize the goal of life as an ascetic or brahmin, and don’t live having realized it with their own insight. 

There\marginnote{2.1} are ascetics and brahmins who do understand the faculty of faith, its origin, its cessation, and the practice that leads to its cessation. They do understand the faculty of energy … mindfulness … immersion … wisdom, its origin, its cessation, and the practice that leads to its cessation. I regard them as true ascetics and brahmins. Those venerables realize the goal of life as an ascetic or brahmin, and live having realized it with their own insight.” 

%
\section*{{\suttatitleacronym SN 48.8}{\suttatitletranslation Should Be Seen }{\suttatitleroot Daṭṭhabbasutta}}
\addcontentsline{toc}{section}{\tocacronym{SN 48.8} \toctranslation{Should Be Seen } \tocroot{Daṭṭhabbasutta}}
\markboth{Should Be Seen }{Daṭṭhabbasutta}
\extramarks{SN 48.8}{SN 48.8}

“Mendicants,\marginnote{1.1} there are these five faculties. What five? The faculties of faith, energy, mindfulness, immersion, and wisdom. 

And\marginnote{1.4} where should the faculty of faith be seen? In the four factors of stream-entry. 

And\marginnote{1.7} where should the faculty of energy be seen? In the four right efforts. 

And\marginnote{1.10} where should the faculty of mindfulness be seen? In the four kinds of mindfulness meditation. 

And\marginnote{1.13} where should the faculty of immersion be seen? In the four absorptions. 

And\marginnote{1.16} where should the faculty of wisdom be seen? In the four noble truths. 

These\marginnote{1.19} are the five faculties.” 

%
\section*{{\suttatitleacronym SN 48.9}{\suttatitletranslation Analysis (1st) }{\suttatitleroot Paṭhamavibhaṅgasutta}}
\addcontentsline{toc}{section}{\tocacronym{SN 48.9} \toctranslation{Analysis (1st) } \tocroot{Paṭhamavibhaṅgasutta}}
\markboth{Analysis (1st) }{Paṭhamavibhaṅgasutta}
\extramarks{SN 48.9}{SN 48.9}

“Mendicants,\marginnote{1.1} there are these five faculties. What five? The faculties of faith, energy, mindfulness, immersion, and wisdom. 

And\marginnote{1.4} what is the faculty of faith? It’s when a noble disciple has faith in the Realized One’s awakening: ‘That Blessed One is perfected, a fully awakened Buddha, accomplished in knowledge and conduct, holy, knower of the world, supreme guide for those who wish to train, teacher of gods and humans, awakened, blessed.’ This is called the faculty of faith. 

And\marginnote{2.1} what is the faculty of energy? It’s when a mendicant lives with energy roused up for giving up unskillful qualities and embracing skillful qualities. They’re strong, staunchly vigorous, not slacking off when it comes to developing skillful qualities. This is called the faculty of energy. 

And\marginnote{3.1} what is the faculty of mindfulness? It’s when a noble disciple is mindful. They have utmost mindfulness and alertness, and can remember and recall what was said and done long ago. This is called the faculty of mindfulness. 

And\marginnote{4.1} what is the faculty of immersion? It’s when a noble disciple, relying on letting go, gains immersion, gains unification of mind. This is called the faculty of immersion. 

And\marginnote{5.1} what is the faculty of wisdom? It’s when a noble disciple is wise. They have the wisdom of arising and passing away which is noble, penetrative, and leads to the complete ending of suffering. This is called the faculty of wisdom. 

These\marginnote{5.4} are the five faculties.” 

%
\section*{{\suttatitleacronym SN 48.10}{\suttatitletranslation Analysis (2nd) }{\suttatitleroot Dutiyavibhaṅgasutta}}
\addcontentsline{toc}{section}{\tocacronym{SN 48.10} \toctranslation{Analysis (2nd) } \tocroot{Dutiyavibhaṅgasutta}}
\markboth{Analysis (2nd) }{Dutiyavibhaṅgasutta}
\extramarks{SN 48.10}{SN 48.10}

“Mendicants,\marginnote{1.1} there are these five faculties. What five? The faculties of faith, energy, mindfulness, immersion, and wisdom. 

And\marginnote{1.4} what is the faculty of faith? It’s when a noble disciple has faith in the Realized One’s awakening: ‘That Blessed One is perfected, a fully awakened Buddha, accomplished in knowledge and conduct, holy, knower of the world, supreme guide for those who wish to train, teacher of gods and humans, awakened, blessed.’ This is called the faculty of faith. 

And\marginnote{2.1} what is the faculty of energy? It’s when a mendicant lives with energy roused up for giving up unskillful qualities and embracing skillful qualities. They’re strong, staunchly vigorous, not slacking off when it comes to developing skillful qualities. They generate enthusiasm, try, make an effort, exert the mind, and strive so that bad, unskillful qualities don’t arise. They generate enthusiasm, try, make an effort, exert the mind, and strive so that bad, unskillful qualities that have arisen are given up. They generate enthusiasm, try, make an effort, exert the mind, and strive so that skillful qualities arise. They generate enthusiasm, try, make an effort, exert the mind, and strive so that skillful qualities that have arisen remain, are not lost, but increase, mature, and are completed by development. This is called the faculty of energy. 

And\marginnote{3.1} what is the faculty of mindfulness? It’s when a noble disciple is mindful. They have utmost mindfulness and alertness, and can remember and recall what was said and done long ago. They meditate observing an aspect of the body—keen, aware, and mindful, rid of desire and aversion for the world. They meditate observing an aspect of feelings … mind … principles—keen, aware, and mindful, rid of desire and aversion for the world. This is called the faculty of mindfulness. 

And\marginnote{4.1} what is the faculty of immersion? It’s when a noble disciple, relying on letting go, gains immersion, gains unification of mind. Quite secluded from sensual pleasures, secluded from unskillful qualities, they enter and remain in the first absorption, which has the rapture and bliss born of seclusion, while placing the mind and keeping it connected. As the placing of the mind and keeping it connected are stilled, they enter and remain in the second absorption, which has the rapture and bliss born of immersion, with internal clarity and confidence, and unified mind, without placing the mind and keeping it connected. And with the fading away of rapture, they enter and remain in the third absorption, where they meditate with equanimity, mindful and aware, personally experiencing the bliss of which the noble ones declare, ‘Equanimous and mindful, one meditates in bliss.’ Giving up pleasure and pain, and ending former happiness and sadness, they enter and remain in the fourth absorption, without pleasure or pain, with pure equanimity and mindfulness. This is called the faculty of immersion. 

And\marginnote{5.1} what is the faculty of wisdom? It’s when a noble disciple is wise. They have the wisdom of arising and passing away which is noble, penetrative, and leads to the complete ending of suffering. They truly understand: ‘This is suffering’ … ‘This is the origin of suffering’ … ‘This is the cessation of suffering’ … ‘This is the practice that leads to the cessation of suffering’. This is called the faculty of wisdom. 

These\marginnote{5.5} are the five faculties.” 

%
\addtocontents{toc}{\let\protect\contentsline\protect\nopagecontentsline}
\chapter*{The Chapter on Weaker }
\addcontentsline{toc}{chapter}{\tocchapterline{The Chapter on Weaker }}
\addtocontents{toc}{\let\protect\contentsline\protect\oldcontentsline}

%
\section*{{\suttatitleacronym SN 48.11}{\suttatitletranslation Gain }{\suttatitleroot Paṭilābhasutta}}
\addcontentsline{toc}{section}{\tocacronym{SN 48.11} \toctranslation{Gain } \tocroot{Paṭilābhasutta}}
\markboth{Gain }{Paṭilābhasutta}
\extramarks{SN 48.11}{SN 48.11}

“Mendicants,\marginnote{1.1} there are these five faculties. What five? The faculties of faith, energy, mindfulness, immersion, and wisdom. 

And\marginnote{1.4} what is the faculty of faith? It’s when a noble disciple has faith in the Realized One’s awakening: ‘That Blessed One is perfected, a fully awakened Buddha, accomplished in knowledge and conduct, holy, knower of the world, supreme guide for those who wish to train, teacher of gods and humans, awakened, blessed.’ This is called the faculty of faith. 

And\marginnote{2.1} what is the faculty of energy? The energy that’s gained in connection with the four right efforts. This is called the faculty of energy. 

And\marginnote{3.1} what is the faculty of mindfulness? The mindfulness that’s gained in connection with the four kinds of mindfulness meditation. This is called the faculty of mindfulness. 

And\marginnote{4.1} what is the faculty of immersion? It’s when a noble disciple, relying on letting go, gains immersion, gains unification of mind. This is called the faculty of immersion. 

And\marginnote{5.1} what is the faculty of wisdom? It’s when a noble disciple is wise. They have the wisdom of arising and passing away which is noble, penetrative, and leads to the complete ending of suffering. This is called the faculty of wisdom. 

These\marginnote{5.4} are the five faculties.” 

%
\section*{{\suttatitleacronym SN 48.12}{\suttatitletranslation In Brief (1st) }{\suttatitleroot Paṭhamasaṁkhittasutta}}
\addcontentsline{toc}{section}{\tocacronym{SN 48.12} \toctranslation{In Brief (1st) } \tocroot{Paṭhamasaṁkhittasutta}}
\markboth{In Brief (1st) }{Paṭhamasaṁkhittasutta}
\extramarks{SN 48.12}{SN 48.12}

“Mendicants,\marginnote{1.1} there are these five faculties. What five? The faculties of faith, energy, mindfulness, immersion, and wisdom. These are the five faculties. 

Someone\marginnote{1.5} who has completed and fulfilled these five faculties is a perfected one. If they are weaker than that, they’re a non-returner. If they are weaker still, they’re a once-returner. If they are weaker still, they’re a stream-enterer. If they’re weaker still, they’re a follower of the teachings. If they’re weaker still, they’re a follower by faith.” 

%
\section*{{\suttatitleacronym SN 48.13}{\suttatitletranslation In Brief (2nd) }{\suttatitleroot Dutiyasaṁkhittasutta}}
\addcontentsline{toc}{section}{\tocacronym{SN 48.13} \toctranslation{In Brief (2nd) } \tocroot{Dutiyasaṁkhittasutta}}
\markboth{In Brief (2nd) }{Dutiyasaṁkhittasutta}
\extramarks{SN 48.13}{SN 48.13}

“Mendicants,\marginnote{1.1} there are these five faculties. What five? The faculties of faith, energy, mindfulness, immersion, and wisdom. These are the five faculties. 

Someone\marginnote{1.5} who has completed and fulfilled these five faculties is a perfected one. If they are weaker than that, they’re a non-returner … a once-returner … a stream-enterer … a follower of the teachings … a follower by faith. 

So\marginnote{1.6} from a diversity of faculties there’s a diversity of fruits. And from a diversity of fruits there’s a diversity of persons.” 

%
\section*{{\suttatitleacronym SN 48.14}{\suttatitletranslation In Brief (3rd) }{\suttatitleroot Tatiyasaṁkhittasutta}}
\addcontentsline{toc}{section}{\tocacronym{SN 48.14} \toctranslation{In Brief (3rd) } \tocroot{Tatiyasaṁkhittasutta}}
\markboth{In Brief (3rd) }{Tatiyasaṁkhittasutta}
\extramarks{SN 48.14}{SN 48.14}

“Mendicants,\marginnote{1.1} there are these five faculties. What five? The faculties of faith, energy, mindfulness, immersion, and wisdom. These are the five faculties. 

Someone\marginnote{1.5} who has completed and fulfilled these five faculties is a perfected one. If they are weaker than that, they’re a non-returner … a once-returner … a stream-enterer … a follower of the teachings … a follower by faith. 

So,\marginnote{1.6} mendicants, if you practice partially you succeed partially. If you practice fully you succeed fully. These five faculties are not a waste, I say.” 

%
\section*{{\suttatitleacronym SN 48.15}{\suttatitletranslation In Detail (1st) }{\suttatitleroot Paṭhamavitthārasutta}}
\addcontentsline{toc}{section}{\tocacronym{SN 48.15} \toctranslation{In Detail (1st) } \tocroot{Paṭhamavitthārasutta}}
\markboth{In Detail (1st) }{Paṭhamavitthārasutta}
\extramarks{SN 48.15}{SN 48.15}

“Mendicants,\marginnote{1.1} there are these five faculties. What five? The faculties of faith, energy, mindfulness, immersion, and wisdom. These are the five faculties. 

Someone\marginnote{1.5} who has completed and fulfilled these five faculties is a perfected one. If they are weaker than that, they’re one who is extinguished between one life and the next … one who is extinguished upon landing … one who is extinguished without extra effort … one who is extinguished with extra effort … one who heads upstream, going to the \textsanskrit{Akaniṭṭha} realm … a once-returner … a stream-enterer … a follower of the teachings … a follower by faith.” 

%
\section*{{\suttatitleacronym SN 48.16}{\suttatitletranslation In Detail (2nd) }{\suttatitleroot Dutiyavitthārasutta}}
\addcontentsline{toc}{section}{\tocacronym{SN 48.16} \toctranslation{In Detail (2nd) } \tocroot{Dutiyavitthārasutta}}
\markboth{In Detail (2nd) }{Dutiyavitthārasutta}
\extramarks{SN 48.16}{SN 48.16}

“Mendicants,\marginnote{1.1} there are these five faculties. What five? The faculties of faith, energy, mindfulness, immersion, and wisdom. These are the five faculties. 

Someone\marginnote{1.5} who has completed and fulfilled these five faculties is a perfected one. If they are weaker than that, they’re one who is extinguished between one life and the next … one who is extinguished upon landing … one who is extinguished without extra effort … one who is extinguished with extra effort … one who heads upstream, going to the \textsanskrit{Akaniṭṭha} realm … a once-returner … a stream-enterer … a follower of the teachings … a follower by faith. 

So\marginnote{1.6} from a diversity of faculties there’s a diversity of fruits. And from a diversity of fruits there’s a diversity of persons.” 

%
\section*{{\suttatitleacronym SN 48.17}{\suttatitletranslation In Detail (3rd) }{\suttatitleroot Tatiyavitthārasutta}}
\addcontentsline{toc}{section}{\tocacronym{SN 48.17} \toctranslation{In Detail (3rd) } \tocroot{Tatiyavitthārasutta}}
\markboth{In Detail (3rd) }{Tatiyavitthārasutta}
\extramarks{SN 48.17}{SN 48.17}

“Mendicants,\marginnote{1.1} there are these five faculties. What five? The faculties of faith, energy, mindfulness, immersion, and wisdom. These are the five faculties. 

Someone\marginnote{1.5} who has completed and fulfilled these five faculties is a perfected one. If they are weaker than that, they’re one who is extinguished between one life and the next … one who is extinguished upon landing … one who is extinguished without extra effort … one who is extinguished with extra effort … one who heads upstream, going to the \textsanskrit{Akaniṭṭha} realm … a once-returner … a stream-enterer … a follower of the teachings … a follower by faith. 

So,\marginnote{1.6} mendicants, if you practice fully you succeed fully. If you practice partially you succeed partially. These five faculties are not a waste, I say.” 

%
\section*{{\suttatitleacronym SN 48.18}{\suttatitletranslation Practicing }{\suttatitleroot Paṭipannasutta}}
\addcontentsline{toc}{section}{\tocacronym{SN 48.18} \toctranslation{Practicing } \tocroot{Paṭipannasutta}}
\markboth{Practicing }{Paṭipannasutta}
\extramarks{SN 48.18}{SN 48.18}

“Mendicants,\marginnote{1.1} there are these five faculties. What five? The faculties of faith, energy, mindfulness, immersion, and wisdom. These are the five faculties. 

Someone\marginnote{1.5} who has completed and fulfilled these five faculties is a perfected one. If they are weaker than that, they’re practicing to realize the fruit of perfection … a non-returner … practicing to realize the fruit of non-return … a once-returner … practicing to realize the fruit of once-return … a stream-enterer … practicing to realize the fruit of stream-entry. Someone who totally and utterly lacks these five faculties is an outsider who belongs with the ordinary persons, I say.” 

%
\section*{{\suttatitleacronym SN 48.19}{\suttatitletranslation Endowed }{\suttatitleroot Sampannasutta}}
\addcontentsline{toc}{section}{\tocacronym{SN 48.19} \toctranslation{Endowed } \tocroot{Sampannasutta}}
\markboth{Endowed }{Sampannasutta}
\extramarks{SN 48.19}{SN 48.19}

Then\marginnote{1.1} a mendicant went up to the Buddha, bowed, sat down to one side, and said to him: 

“Sir,\marginnote{2.1} they speak of someone who is ‘accomplished regarding the faculties’. How is someone accomplished regarding the faculties defined?” 

“Mendicant,\marginnote{2.3} it’s when a mendicant develops the faculties of faith, energy, mindfulness, immersion, and wisdom that lead to peace and awakening. This is how someone who is accomplished regarding the faculties is defined.” 

%
\section*{{\suttatitleacronym SN 48.20}{\suttatitletranslation The Ending of Defilements }{\suttatitleroot Āsavakkhayasutta}}
\addcontentsline{toc}{section}{\tocacronym{SN 48.20} \toctranslation{The Ending of Defilements } \tocroot{Āsavakkhayasutta}}
\markboth{The Ending of Defilements }{Āsavakkhayasutta}
\extramarks{SN 48.20}{SN 48.20}

“Mendicants,\marginnote{1.1} there are these five faculties. What five? The faculties of faith, energy, mindfulness, immersion, and wisdom. These are the five faculties. 

It’s\marginnote{1.5} because of developing and cultivating these five faculties that a mendicant realizes the undefiled freedom of heart and freedom by wisdom in this very life. And they live having realized it with their own insight due to the ending of defilements.” 

%
\addtocontents{toc}{\let\protect\contentsline\protect\nopagecontentsline}
\chapter*{The Chapter on the Six Faculties }
\addcontentsline{toc}{chapter}{\tocchapterline{The Chapter on the Six Faculties }}
\addtocontents{toc}{\let\protect\contentsline\protect\oldcontentsline}

%
\section*{{\suttatitleacronym SN 48.21}{\suttatitletranslation Future Lives }{\suttatitleroot Punabbhavasutta}}
\addcontentsline{toc}{section}{\tocacronym{SN 48.21} \toctranslation{Future Lives } \tocroot{Punabbhavasutta}}
\markboth{Future Lives }{Punabbhavasutta}
\extramarks{SN 48.21}{SN 48.21}

“Mendicants,\marginnote{1.1} there are these five faculties. What five? The faculties of faith, energy, mindfulness, immersion, and wisdom. As long as I didn’t truly understand these five faculties’ gratification, drawback, and escape, I didn’t announce my supreme perfect awakening in this world with its gods, \textsanskrit{Māras}, and \textsanskrit{Brahmās}, this population with its ascetics and brahmins, its gods and humans. 

But\marginnote{1.5} when I did truly understand these five faculties’ gratification, drawback, and escape, I announced my supreme perfect awakening in this world with its gods, \textsanskrit{Māras}, and \textsanskrit{Brahmās}, this population with its ascetics and brahmins, its gods and humans. 

Knowledge\marginnote{1.6} and vision arose in me: ‘My freedom is unshakable; this is my last rebirth; now there’ll be no more future lives.’” 

%
\section*{{\suttatitleacronym SN 48.22}{\suttatitletranslation The Life Faculty }{\suttatitleroot Jīvitindriyasutta}}
\addcontentsline{toc}{section}{\tocacronym{SN 48.22} \toctranslation{The Life Faculty } \tocroot{Jīvitindriyasutta}}
\markboth{The Life Faculty }{Jīvitindriyasutta}
\extramarks{SN 48.22}{SN 48.22}

“Mendicants,\marginnote{1.1} there are these three faculties. What three? The faculties of femininity, masculinity, and life. These are the three faculties.” 

%
\section*{{\suttatitleacronym SN 48.23}{\suttatitletranslation The Faculty of Enlightenment }{\suttatitleroot Aññindriyasutta}}
\addcontentsline{toc}{section}{\tocacronym{SN 48.23} \toctranslation{The Faculty of Enlightenment } \tocroot{Aññindriyasutta}}
\markboth{The Faculty of Enlightenment }{Aññindriyasutta}
\extramarks{SN 48.23}{SN 48.23}

“Mendicants,\marginnote{1.1} there are these three faculties. What three? The faculty of understanding that one’s enlightenment is imminent. The faculty of enlightenment. The faculty of one who is enlightened. These are the three faculties.” 

%
\section*{{\suttatitleacronym SN 48.24}{\suttatitletranslation A One-Seeder }{\suttatitleroot Ekabījīsutta}}
\addcontentsline{toc}{section}{\tocacronym{SN 48.24} \toctranslation{A One-Seeder } \tocroot{Ekabījīsutta}}
\markboth{A One-Seeder }{Ekabījīsutta}
\extramarks{SN 48.24}{SN 48.24}

“Mendicants,\marginnote{1.1} there are these five faculties. What five? The faculties of faith, energy, mindfulness, immersion, and wisdom. These are the five faculties. 

Someone\marginnote{1.5} who has completed and fulfilled these five faculties is a perfected one. If they are weaker than that, they’re one who is extinguished between one life and the next … one who is extinguished upon landing … one who is extinguished without extra effort … one who is extinguished with extra effort … one who heads upstream, going to the \textsanskrit{Akaniṭṭha} realm … a once-returner … a one-seeder … one who goes from family to family … one who has seven rebirths at most … a follower of the teachings … a follower by faith.” 

%
\section*{{\suttatitleacronym SN 48.25}{\suttatitletranslation Plain Version }{\suttatitleroot Suddhakasutta}}
\addcontentsline{toc}{section}{\tocacronym{SN 48.25} \toctranslation{Plain Version } \tocroot{Suddhakasutta}}
\markboth{Plain Version }{Suddhakasutta}
\extramarks{SN 48.25}{SN 48.25}

“Mendicants,\marginnote{1.1} there are these six faculties. What six? The faculties of the eye, ear, nose, tongue, body, and mind. These are the six faculties.” 

%
\section*{{\suttatitleacronym SN 48.26}{\suttatitletranslation A Stream-Enterer }{\suttatitleroot Sotāpannasutta}}
\addcontentsline{toc}{section}{\tocacronym{SN 48.26} \toctranslation{A Stream-Enterer } \tocroot{Sotāpannasutta}}
\markboth{A Stream-Enterer }{Sotāpannasutta}
\extramarks{SN 48.26}{SN 48.26}

“Mendicants,\marginnote{1.1} there are these six faculties. What six? The faculties of the eye, ear, nose, tongue, body, and mind. A noble disciple comes to truly understand these six faculties’ origin, ending, gratification, drawback, and escape. Such a noble disciple is called a stream-enterer, not liable to be reborn in the underworld, bound for awakening.” 

%
\section*{{\suttatitleacronym SN 48.27}{\suttatitletranslation A Perfected One }{\suttatitleroot Arahantasutta}}
\addcontentsline{toc}{section}{\tocacronym{SN 48.27} \toctranslation{A Perfected One } \tocroot{Arahantasutta}}
\markboth{A Perfected One }{Arahantasutta}
\extramarks{SN 48.27}{SN 48.27}

“Mendicants,\marginnote{1.1} there are these six faculties. What six? The faculties of the eye, ear, nose, tongue, body, and mind. A mendicant comes to be freed by not grasping after truly understanding these six faculties’ origin, ending, gratification, drawback, and escape. 

Such\marginnote{1.5} a mendicant is called a perfected one, with defilements ended, who has completed the spiritual journey, done what had to be done, laid down the burden, achieved their own true goal, utterly ended the fetters of rebirth, and is rightly freed through enlightenment.” 

%
\section*{{\suttatitleacronym SN 48.28}{\suttatitletranslation Awakened }{\suttatitleroot Sambuddhasutta}}
\addcontentsline{toc}{section}{\tocacronym{SN 48.28} \toctranslation{Awakened } \tocroot{Sambuddhasutta}}
\markboth{Awakened }{Sambuddhasutta}
\extramarks{SN 48.28}{SN 48.28}

“Mendicants,\marginnote{1.1} there are these six faculties. What six? The faculties of the eye, ear, nose, tongue, body, and mind. 

As\marginnote{1.4} long as I didn’t truly understand these six faculties’ gratification, drawback, and escape, I didn’t announce my supreme perfect awakening in this world with its gods, \textsanskrit{Māras}, and \textsanskrit{Brahmās}, this population with its ascetics and brahmins, its gods and humans. 

But\marginnote{1.5} when I did truly understand these six faculties’ gratification, drawback, and escape, I announced my supreme perfect awakening in this world with its gods, \textsanskrit{Māras}, and \textsanskrit{Brahmās}, this population with its ascetics and brahmins, its gods and humans. 

Knowledge\marginnote{1.6} and vision arose in me: ‘My freedom is unshakable; this is my last rebirth; now there’ll be no more future lives.’” 

%
\section*{{\suttatitleacronym SN 48.29}{\suttatitletranslation Ascetics and Brahmins (1st) }{\suttatitleroot Paṭhamasamaṇabrāhmaṇasutta}}
\addcontentsline{toc}{section}{\tocacronym{SN 48.29} \toctranslation{Ascetics and Brahmins (1st) } \tocroot{Paṭhamasamaṇabrāhmaṇasutta}}
\markboth{Ascetics and Brahmins (1st) }{Paṭhamasamaṇabrāhmaṇasutta}
\extramarks{SN 48.29}{SN 48.29}

“Mendicants,\marginnote{1.1} there are these six faculties. What six? The faculties of the eye, ear, nose, tongue, body, and mind. 

There\marginnote{1.4} are ascetics and brahmins who don’t truly understand the origin, ending, gratification, drawback, and escape when it comes to these six faculties. I don’t regard them as true ascetics and brahmins. Those venerables don’t realize the goal of life as an ascetic or brahmin, and don’t live having realized it with their own insight. 

There\marginnote{1.6} are ascetics and brahmins who do truly understand the origin, ending, gratification, drawback, and escape when it comes to these six faculties. I regard them as true ascetics and brahmins. Those venerables realize the goal of life as an ascetic or brahmin, and live having realized it with their own insight.” 

%
\section*{{\suttatitleacronym SN 48.30}{\suttatitletranslation Ascetics and Brahmins (2nd) }{\suttatitleroot Dutiyasamaṇabrāhmaṇasutta}}
\addcontentsline{toc}{section}{\tocacronym{SN 48.30} \toctranslation{Ascetics and Brahmins (2nd) } \tocroot{Dutiyasamaṇabrāhmaṇasutta}}
\markboth{Ascetics and Brahmins (2nd) }{Dutiyasamaṇabrāhmaṇasutta}
\extramarks{SN 48.30}{SN 48.30}

“Mendicants,\marginnote{1.1} there are ascetics and brahmins who don’t understand the eye faculty, its origin, its cessation, and the practice that leads to its cessation. They don’t understand the ear faculty … nose faculty … tongue faculty … body faculty … mind faculty, its origin, its cessation, and the practice that leads to its cessation. I don’t regard them as true ascetics and brahmins. Those venerables don’t realize the goal of life as an ascetic or brahmin, and don’t live having realized it with their own insight. 

There\marginnote{2.1} are ascetics and brahmins who do understand the eye faculty, its origin, its cessation, and the practice that leads to its cessation. They understand the ear faculty … nose faculty … tongue faculty … body faculty … mind faculty, its origin, its cessation, and the practice that leads to its cessation. I regard them as true ascetics and brahmins. Those venerables realize the goal of life as an ascetic or brahmin, and live having realized it with their own insight.” 

%
\addtocontents{toc}{\let\protect\contentsline\protect\nopagecontentsline}
\chapter*{The Chapter on the Pleasure Faculty }
\addcontentsline{toc}{chapter}{\tocchapterline{The Chapter on the Pleasure Faculty }}
\addtocontents{toc}{\let\protect\contentsline\protect\oldcontentsline}

%
\section*{{\suttatitleacronym SN 48.31}{\suttatitletranslation Plain Version }{\suttatitleroot Suddhikasutta}}
\addcontentsline{toc}{section}{\tocacronym{SN 48.31} \toctranslation{Plain Version } \tocroot{Suddhikasutta}}
\markboth{Plain Version }{Suddhikasutta}
\extramarks{SN 48.31}{SN 48.31}

“Mendicants,\marginnote{1.1} there are these five faculties. What five? The faculties of pleasure, pain, happiness, sadness, and equanimity. These are the five faculties.” 

%
\section*{{\suttatitleacronym SN 48.32}{\suttatitletranslation A Stream-Enterer }{\suttatitleroot Sotāpannasutta}}
\addcontentsline{toc}{section}{\tocacronym{SN 48.32} \toctranslation{A Stream-Enterer } \tocroot{Sotāpannasutta}}
\markboth{A Stream-Enterer }{Sotāpannasutta}
\extramarks{SN 48.32}{SN 48.32}

“Mendicants,\marginnote{1.1} there are these five faculties. What five? The faculties of pleasure, pain, happiness, sadness, and equanimity. A noble disciple comes to truly understand these five faculties’ origin, ending, gratification, drawback, and escape. Such a noble disciple is called a stream-enterer, not liable to be reborn in the underworld, bound for awakening.” 

%
\section*{{\suttatitleacronym SN 48.33}{\suttatitletranslation A Perfected One }{\suttatitleroot Arahantasutta}}
\addcontentsline{toc}{section}{\tocacronym{SN 48.33} \toctranslation{A Perfected One } \tocroot{Arahantasutta}}
\markboth{A Perfected One }{Arahantasutta}
\extramarks{SN 48.33}{SN 48.33}

“Mendicants,\marginnote{1.1} there are these five faculties. What five? The faculties of pleasure, pain, happiness, sadness, and equanimity. A mendicant comes to be freed by not grasping after truly understanding these five faculties’ origin, ending, gratification, drawback, and escape. 

Such\marginnote{1.5} a mendicant is called a perfected one, with defilements ended, who has completed the spiritual journey, done what had to be done, laid down the burden, achieved their own true goal, utterly ended the fetters of rebirth, and is rightly freed through enlightenment.” 

%
\section*{{\suttatitleacronym SN 48.34}{\suttatitletranslation Ascetics and Brahmins (1st) }{\suttatitleroot Paṭhamasamaṇabrāhmaṇasutta}}
\addcontentsline{toc}{section}{\tocacronym{SN 48.34} \toctranslation{Ascetics and Brahmins (1st) } \tocroot{Paṭhamasamaṇabrāhmaṇasutta}}
\markboth{Ascetics and Brahmins (1st) }{Paṭhamasamaṇabrāhmaṇasutta}
\extramarks{SN 48.34}{SN 48.34}

“Mendicants,\marginnote{1.1} there are these five faculties. What five? The faculties of pleasure, pain, happiness, sadness, and equanimity. 

There\marginnote{1.4} are ascetics and brahmins who don’t truly understand the origin, ending, gratification, drawback, and escape when it comes to these five faculties. I don’t regard them as true ascetics and brahmins. Those venerables don’t realize the goal of life as an ascetic or brahmin, and don’t live having realized it with their own insight. 

There\marginnote{2.1} are ascetics and brahmins who do truly understand the origin, ending, gratification, drawback, and escape when it comes to these five faculties. I regard them as true ascetics and brahmins. Those venerables realize the goal of life as an ascetic or brahmin, and live having realized it with their own insight.” 

%
\section*{{\suttatitleacronym SN 48.35}{\suttatitletranslation Ascetics and Brahmins (2nd) }{\suttatitleroot Dutiyasamaṇabrāhmaṇasutta}}
\addcontentsline{toc}{section}{\tocacronym{SN 48.35} \toctranslation{Ascetics and Brahmins (2nd) } \tocroot{Dutiyasamaṇabrāhmaṇasutta}}
\markboth{Ascetics and Brahmins (2nd) }{Dutiyasamaṇabrāhmaṇasutta}
\extramarks{SN 48.35}{SN 48.35}

“Mendicants,\marginnote{1.1} there are these five faculties. What five? The faculties of pleasure, pain, happiness, sadness, and equanimity. 

“Mendicants,\marginnote{1.4} there are ascetics and brahmins who don’t understand the faculty of pleasure, its origin, its cessation, and the practice that leads to its cessation. There are ascetics and brahmins who don’t understand the faculty of pain … happiness … sadness … equanimity, its origin, its cessation, and the practice that leads to its cessation. I don’t regard them as true ascetics and brahmins. Those venerables don’t realize the goal of life as an ascetic or brahmin, and don’t live having realized it with their own insight. 

There\marginnote{2.1} are ascetics and brahmins who do understand the faculty of pleasure, its origin, its cessation, and the practice that leads to its cessation. There are ascetics and brahmins who do understand the faculty of pain … happiness … sadness … equanimity, its origin, its cessation, and the practice that leads to its cessation. I regard them as true ascetics and brahmins. Those venerables realize the goal of life as an ascetic or brahmin, and live having realized it with their own insight.” 

%
\section*{{\suttatitleacronym SN 48.36}{\suttatitletranslation Analysis (1st) }{\suttatitleroot Paṭhamavibhaṅgasutta}}
\addcontentsline{toc}{section}{\tocacronym{SN 48.36} \toctranslation{Analysis (1st) } \tocroot{Paṭhamavibhaṅgasutta}}
\markboth{Analysis (1st) }{Paṭhamavibhaṅgasutta}
\extramarks{SN 48.36}{SN 48.36}

“Mendicants,\marginnote{1.1} there are these five faculties. What five? The faculties of pleasure, pain, happiness, sadness, and equanimity. 

And\marginnote{2.1} what is the faculty of pleasure? Physical enjoyment, physical pleasure, the enjoyable, pleasant feeling that’s born from physical contact. This is called the faculty of pleasure. 

And\marginnote{3.1} what is the faculty of pain? Physical pain, physical displeasure, the painful, unpleasant feeling that’s born from physical contact. This is called the faculty of pain. 

And\marginnote{4.1} what is the faculty of happiness? Mental enjoyment, mental pleasure, the enjoyable, pleasant feeling that’s born from mind contact. This is called the faculty of happiness. 

And\marginnote{5.1} what is the faculty of sadness? Mental pain, mental displeasure, the painful, unpleasant feeling that’s born from mind contact. This is called the faculty of sadness. 

And\marginnote{6.1} what is the faculty of equanimity? Neither pleasant nor unpleasant feeling, whether physical or mental. This is the faculty of equanimity. 

These\marginnote{6.4} are the five faculties.” 

%
\section*{{\suttatitleacronym SN 48.37}{\suttatitletranslation Analysis (2nd) }{\suttatitleroot Dutiyavibhaṅgasutta}}
\addcontentsline{toc}{section}{\tocacronym{SN 48.37} \toctranslation{Analysis (2nd) } \tocroot{Dutiyavibhaṅgasutta}}
\markboth{Analysis (2nd) }{Dutiyavibhaṅgasutta}
\extramarks{SN 48.37}{SN 48.37}

“Mendicants,\marginnote{1.1} there are these five faculties. What five? The faculties of pleasure, pain, happiness, sadness, and equanimity. 

And\marginnote{2.1} what is the faculty of pleasure? Physical enjoyment, physical pleasure, the enjoyable, pleasant feeling that’s born from physical contact. This is called the faculty of pleasure. 

And\marginnote{3.1} what is the faculty of pain? Physical pain, physical displeasure, the painful, unpleasant feeling that’s born from physical contact. This is called the faculty of pain. 

And\marginnote{4.1} what is the faculty of happiness? Mental enjoyment, mental pleasure, the enjoyable, pleasant feeling that’s born from mind contact. This is called the faculty of happiness. 

And\marginnote{5.1} what is the faculty of sadness? Mental pain, mental displeasure, the painful, unpleasant feeling that’s born from mind contact. This is called the faculty of sadness. 

And\marginnote{6.1} what is the faculty of equanimity? Neither pleasant nor unpleasant feeling, whether physical or mental. This is the faculty of equanimity. 

In\marginnote{7.1} this context, the faculties of pleasure and happiness should be seen as pleasant feeling. The faculties of pain and sadness should be seen as painful feeling. The faculty of equanimity should be seen as neutral feeling. 

These\marginnote{7.4} are the five faculties.” 

%
\section*{{\suttatitleacronym SN 48.38}{\suttatitletranslation Analysis (3rd) }{\suttatitleroot Tatiyavibhaṅgasutta}}
\addcontentsline{toc}{section}{\tocacronym{SN 48.38} \toctranslation{Analysis (3rd) } \tocroot{Tatiyavibhaṅgasutta}}
\markboth{Analysis (3rd) }{Tatiyavibhaṅgasutta}
\extramarks{SN 48.38}{SN 48.38}

“Mendicants,\marginnote{1.1} there are these five faculties. What five? The faculties of pleasure, pain, happiness, sadness, and equanimity. 

And\marginnote{2.1} what is the faculty of pleasure? Physical enjoyment, physical pleasure, the enjoyable, pleasant feeling that’s born from physical contact. This is called the faculty of pleasure. 

And\marginnote{3.1} what is the faculty of pain? Physical pain, physical displeasure, the painful, unpleasant feeling that’s born from physical contact. This is called the faculty of pain. 

And\marginnote{4.1} what is the faculty of happiness? Mental enjoyment, mental pleasure, the enjoyable, pleasant feeling that’s born from mind contact. This is called the faculty of happiness. 

And\marginnote{5.1} what is the faculty of sadness? Mental pain, mental displeasure, the painful, unpleasant feeling that’s born from mind contact. This is called the faculty of sadness. 

And\marginnote{6.1} what is the faculty of equanimity? Neither pleasant nor unpleasant feeling, whether physical or mental. This is the faculty of equanimity. 

In\marginnote{7.1} this context, the faculties of pleasure and happiness should be seen as pleasant feeling. The faculties of pain and sadness should be seen as painful feeling. The faculty of equanimity should be seen as neutral feeling. 

That’s\marginnote{7.4} how these five faculties, depending on how they’re explained, having been five become three, and having been three become five.” 

%
\section*{{\suttatitleacronym SN 48.39}{\suttatitletranslation The Simile of the Fire Sticks }{\suttatitleroot Kaṭṭhopamasutta}}
\addcontentsline{toc}{section}{\tocacronym{SN 48.39} \toctranslation{The Simile of the Fire Sticks } \tocroot{Kaṭṭhopamasutta}}
\markboth{The Simile of the Fire Sticks }{Kaṭṭhopamasutta}
\extramarks{SN 48.39}{SN 48.39}

“Mendicants,\marginnote{1.1} there are these five faculties. What five? The faculties of pleasure, pain, happiness, sadness, and equanimity. 

The\marginnote{1.4} faculty of pleasure arises dependent on a contact to be experienced as pleasant. When in a state of pleasure, you understand: ‘I’m in a state of pleasure.’ With the cessation of that contact to be experienced as pleasant, you understand that the corresponding faculty of pleasure ceases and stops. 

The\marginnote{2.1} faculty of pain arises dependent on a contact to be experienced as painful. When in a state of pain, you understand: ‘I’m in a state of pain.’ With the cessation of that contact to be experienced as painful, you understand that the corresponding faculty of pain ceases and stops. 

The\marginnote{3.1} faculty of happiness arises dependent on a contact to be experienced as happiness. When in a state of happiness, you understand: ‘I’m in a state of happiness.’ With the cessation of that contact to be experienced as happiness, you understand that the corresponding faculty of happiness ceases and stops. 

The\marginnote{4.1} faculty of sadness arises dependent on a contact to be experienced as sadness. When in a state of sadness, you understand: ‘I’m in a state of sadness.’ With the cessation of that contact to be experienced as sadness, you understand that the corresponding faculty of sadness ceases and stops. 

The\marginnote{5.1} faculty of equanimity arises dependent on a contact to be experienced as equanimous. When in a state of equanimity, you understand: ‘I’m in a state of equanimity.’ With the cessation of that contact to be experienced as equanimous, you understand that the corresponding faculty of equanimity ceases and stops. 

When\marginnote{6.1} you rub two sticks together, heat is generated and fire is produced. But when you part the sticks and lay them aside, any corresponding heat ceases and stops. 

In\marginnote{6.2} the same way, the faculty of pleasure arises dependent on a contact to be experienced as pleasant. When in a state of pleasure, you understand: ‘I’m in a state of pleasure.’ With the cessation of that contact to be experienced as pleasant, you understand that the corresponding faculty of pleasure ceases and stops. 

The\marginnote{7.1} faculty of pain … happiness … sadness … equanimity arises dependent on a contact to be experienced as equanimous. When in a state of equanimity, you understand: ‘I’m in a state of equanimity.’ With the cessation of that contact to be experienced as equanimous, you understand that the corresponding faculty of equanimity ceases and stops.” 

%
\section*{{\suttatitleacronym SN 48.40}{\suttatitletranslation Irregular Order }{\suttatitleroot Uppaṭipāṭikasutta}}
\addcontentsline{toc}{section}{\tocacronym{SN 48.40} \toctranslation{Irregular Order } \tocroot{Uppaṭipāṭikasutta}}
\markboth{Irregular Order }{Uppaṭipāṭikasutta}
\extramarks{SN 48.40}{SN 48.40}

“Mendicants,\marginnote{1.1} there are these five faculties. What five? The faculties of pain, sadness, pleasure, happiness, and equanimity. 

While\marginnote{1.4} a mendicant is meditating—diligent, keen, and resolute—the faculty of pain arises. They understand: ‘The faculty of pain has arisen in me. And that has a foundation, a source, a condition, and a reason. It’s not possible for the faculty of pain to arise without a foundation, a source, a condition, or a reason.’ They understand the faculty of pain, its origin, its cessation, and where that faculty of pain that’s arisen ceases without anything left over. And where does that faculty of pain that’s arisen cease without anything left over? It’s when a mendicant, quite secluded from sensual pleasures, secluded from unskillful qualities, enters and remains in the first absorption, which has the rapture and bliss born of seclusion, while placing the mind and keeping it connected. That’s where the faculty of pain that’s arisen ceases without anything left over. They’re called a mendicant who understands the cessation of the faculty of pain, and who applies their mind to that end. 

While\marginnote{2.1} a mendicant is meditating—diligent, keen, and resolute—the faculty of sadness arises. They understand: ‘The faculty of sadness has arisen in me. And that has a foundation, a source, a condition, and a reason. It’s not possible for the faculty of sadness to arise without a foundation, a source, a condition, or a reason.’ They understand the faculty of sadness, its origin, its cessation, and where that faculty of sadness that’s arisen ceases without anything left over. And where does that faculty of sadness that’s arisen cease without anything left over? It’s when, as the placing of the mind and keeping it connected are stilled, a mendicant enters and remains in the second absorption, which has the rapture and bliss born of immersion, with internal clarity and confidence, and unified mind, without placing the mind and keeping it connected. That’s where the faculty of sadness that’s arisen ceases without anything left over. They’re called a mendicant who understands the cessation of the faculty of sadness, and who applies their mind to that end. 

While\marginnote{3.1} a mendicant is meditating—diligent, keen, and resolute—the faculty of pleasure arises. They understand: ‘The faculty of pleasure has arisen in me. And that has a foundation, a source, a condition, and a reason. It’s not possible for the faculty of pleasure to arise without a foundation, a source, a condition, or a reason.’ They understand the faculty of pleasure, its origin, its cessation, and where that faculty of pleasure that’s arisen ceases without anything left over. And where does that faculty of pleasure that’s arisen cease without anything left over? It’s when, with the fading away of rapture, a mendicant enters and remains in the third absorption, where they meditate with equanimity, mindful and aware, personally experiencing the bliss of which the noble ones declare, ‘Equanimous and mindful, one meditates in bliss.’ That’s where the faculty of pleasure that’s arisen ceases without anything left over. They’re called a mendicant who understands the cessation of the faculty of pleasure, and who applies their mind to that end. 

While\marginnote{4.1} a mendicant is meditating—diligent, keen, and resolute—the faculty of happiness arises. They understand: ‘The faculty of happiness has arisen in me. And that has a foundation, a source, a condition, and a reason. It’s not possible for the faculty of happiness to arise without a foundation, a source, a condition, or a reason.’ They understand the faculty of happiness, its origin, its cessation, and where that faculty of happiness that’s arisen ceases without anything left over. And where does that faculty of happiness that’s arisen cease without anything left over? It’s when, giving up pleasure and pain, and ending former happiness and sadness, a mendicant enters and remains in the fourth absorption, without pleasure or pain, with pure equanimity and mindfulness. That’s where the faculty of happiness that’s arisen ceases without anything left over. They’re called a mendicant who understands the cessation of the faculty of happiness, and who applies their mind to that end. 

While\marginnote{5.1} a mendicant is meditating—diligent, keen, and resolute—the faculty of equanimity arises. They understand: ‘The faculty of equanimity has arisen in me. And that has a foundation, a source, a condition, and a reason. It’s not possible for the faculty of equanimity to arise without a foundation, a source, a condition, or a reason.’ They understand the faculty of equanimity, its origin, its cessation, and where that faculty of equanimity that’s arisen ceases without anything left over. And where does that faculty of equanimity that’s arisen cease without anything left over? It’s when a mendicant, going totally beyond the dimension of neither perception nor non-perception, enters and remains in the cessation of perception and feeling. That’s where the faculty of equanimity that’s arisen ceases without anything left over. They’re called a mendicant who understands the cessation of the faculty of equanimity, and who applies their mind to that end.” 

%
\addtocontents{toc}{\let\protect\contentsline\protect\nopagecontentsline}
\chapter*{The Chapter on Old Age }
\addcontentsline{toc}{chapter}{\tocchapterline{The Chapter on Old Age }}
\addtocontents{toc}{\let\protect\contentsline\protect\oldcontentsline}

%
\section*{{\suttatitleacronym SN 48.41}{\suttatitletranslation Old Age }{\suttatitleroot Jarādhammasutta}}
\addcontentsline{toc}{section}{\tocacronym{SN 48.41} \toctranslation{Old Age } \tocroot{Jarādhammasutta}}
\markboth{Old Age }{Jarādhammasutta}
\extramarks{SN 48.41}{SN 48.41}

\scevam{So\marginnote{1.1} I have heard. }At one time the Buddha was staying near \textsanskrit{Sāvatthī} in the Eastern Monastery, the stilt longhouse of \textsanskrit{Migāra}’s mother. Then in the late afternoon, the Buddha came out of retreat and sat warming his back in the last rays of the sun. 

Then\marginnote{2.1} Venerable Ānanda went up to the Buddha, bowed, and while massaging the Buddha’s limbs he said: 

“It’s\marginnote{2.2} incredible, sir, it’s amazing, how the complexion of your skin is no longer pure and bright. Your limbs are flaccid and wrinkled, and your body is stooped. And it’s apparent that there has been a deterioration in your faculties of eye, ear, nose, tongue, and body.” 

“That’s\marginnote{3.1} how it is, Ānanda. When young you’re liable to grow old; when healthy you’re liable to get sick; and when alive you’re liable to die. The complexion of the skin is no longer pure and bright. The limbs are flaccid and wrinkled, and the body is stooped. And it’s apparent that there has been a deterioration in the faculties of eye, ear, nose, tongue, and body.” 

That\marginnote{4.1} is what the Buddha said. Then the Holy One, the Teacher, went on to say: 

\begin{verse}%
“Curse\marginnote{5.1} this wretched old age, \\
which makes you so ugly. \\
That’s how much this delightful puppet \\
is ground down by old age. 

Even\marginnote{6.1} if you live for a hundred years, \\
you’ll still end up dying. \\
Death spares no-one, \\
but crushes all underfoot.” 

%
\end{verse}

%
\section*{{\suttatitleacronym SN 48.42}{\suttatitletranslation The Brahmin Uṇṇābha }{\suttatitleroot Uṇṇābhabrāhmaṇasutta}}
\addcontentsline{toc}{section}{\tocacronym{SN 48.42} \toctranslation{The Brahmin Uṇṇābha } \tocroot{Uṇṇābhabrāhmaṇasutta}}
\markboth{The Brahmin Uṇṇābha }{Uṇṇābhabrāhmaṇasutta}
\extramarks{SN 48.42}{SN 48.42}

At\marginnote{1.1} \textsanskrit{Sāvatthī}. 

Then\marginnote{1.2} \textsanskrit{Uṇṇābha} the brahmin went up to the Buddha, and exchanged greetings with him. When the greetings and polite conversation were over, he sat down to one side and said to the Buddha: 

“Master\marginnote{2.1} Gotama, these five faculties have different scopes and different ranges, and don’t experience each others’ scope and range. What five? The faculties of the eye, ear, nose, tongue, and body. What do these five faculties, with their different scopes and ranges, have recourse to? What experiences their scopes and ranges?” 

“Brahmin,\marginnote{3.1} these five faculties have different scopes and different ranges, and don’t experience each others’ scope and range. What five? The faculties of the eye, ear, nose, tongue, and body. These five faculties, with their different scopes and ranges, have recourse to the mind. And the mind experiences their scopes and ranges.” 

“But\marginnote{4.1} Master Gotama, what does the mind have recourse to?” 

“The\marginnote{4.2} mind has recourse to mindfulness.” 

“But\marginnote{4.3} what does mindfulness have recourse to?” 

“Mindfulness\marginnote{4.4} has recourse to freedom.” 

“But\marginnote{4.5} what does freedom have recourse to?” 

“Freedom\marginnote{4.6} has recourse to extinguishment.” 

“But\marginnote{4.7} what does extinguishment have recourse to?” 

“This\marginnote{4.8} question goes too far, brahmin! You weren’t able to grasp the limit of questioning. For extinguishment is the culmination, destination, and end of the spiritual life.” 

And\marginnote{5.1} then the brahmin \textsanskrit{Uṇṇābha} approved and agreed with what the Buddha said. He got up from his seat, bowed, and respectfully circled the Buddha, keeping him on his right, before leaving. 

Then,\marginnote{6.1} not long after he had left, the Buddha addressed the mendicants: “Suppose there was a bungalow or a hall with a peaked roof, with windows on the eastern side. When the sun rises and a ray of light enters through a window, where would it land?” 

“On\marginnote{6.3} the western wall, sir.” 

“In\marginnote{6.4} the same way, the brahmin \textsanskrit{Uṇṇābha}’s faith in the Realized One is settled, rooted, and planted deep. It’s strong and can’t be shifted by any ascetic or brahmin or god or \textsanskrit{Māra} or \textsanskrit{Brahmā} or by anyone in the world. If he were to pass away at this time, he would be bound by no fetter that might return him to this world.” 

%
\section*{{\suttatitleacronym SN 48.43}{\suttatitletranslation At Sāketa }{\suttatitleroot Sāketasutta}}
\addcontentsline{toc}{section}{\tocacronym{SN 48.43} \toctranslation{At Sāketa } \tocroot{Sāketasutta}}
\markboth{At Sāketa }{Sāketasutta}
\extramarks{SN 48.43}{SN 48.43}

\scevam{So\marginnote{1.1} I have heard. }At one time the Buddha was staying near \textsanskrit{Sāketa} in the deer part at the \textsanskrit{Añjana} Wood. There the Buddha addressed the mendicants: “Mendicants, is there a way in which the five faculties become the five powers, and the five powers become the five faculties?” 

“Our\marginnote{2.1} teachings are rooted in the Buddha. He is our guide and our refuge. Sir, may the Buddha himself please clarify the meaning of this. The mendicants will listen and remember it.” 

“Mendicants,\marginnote{3.1} there is a way in which the five faculties become the five powers, and the five powers become the five faculties. 

And\marginnote{4.1} what is that method? The faculty of faith is the power of faith, and the power of faith is the faculty of faith. The faculty of energy is the power of energy, and the power of energy is the faculty of energy. The faculty of mindfulness is the power of mindfulness, and the power of mindfulness is the faculty of mindfulness. The faculty of immersion is the power of immersion, and the power of immersion is the faculty of immersion. The faculty of wisdom is the power of wisdom, and the power of wisdom is the faculty of wisdom. 

Suppose\marginnote{4.7} that there was a river slanting, sloping, and inclining to the east, and in the middle was an island. There’s a way in which that river can be reckoned to have just one stream. But there’s also a way in which that river can be reckoned to have two streams. 

And\marginnote{5.1} what’s the way in which that river can be reckoned to have just one stream? By taking into account the water to the east and the west of the island, that river can be reckoned to have just one stream. 

And\marginnote{6.1} what’s the way in which that river can be reckoned to have two streams? By taking into account the water to the north and the south of the island, that river can be reckoned to have two streams. 

In\marginnote{6.4} the same way, the faculty of faith is the power of faith, and the power of faith is the faculty of faith. The faculty of energy is the power of energy, and the power of energy is the faculty of energy. The faculty of mindfulness is the power of mindfulness, and the power of mindfulness is the faculty of mindfulness. The faculty of immersion is the power of immersion, and the power of immersion is the faculty of immersion. The faculty of wisdom is the power of wisdom, and the power of wisdom is the faculty of wisdom. 

It’s\marginnote{6.9} because of developing and cultivating the five faculties that a mendicant realizes the undefiled freedom of heart and freedom by wisdom in this very life. And they live having realized it with their own insight due to the ending of defilements.” 

%
\section*{{\suttatitleacronym SN 48.44}{\suttatitletranslation At the Eastern Gate }{\suttatitleroot Pubbakoṭṭhakasutta}}
\addcontentsline{toc}{section}{\tocacronym{SN 48.44} \toctranslation{At the Eastern Gate } \tocroot{Pubbakoṭṭhakasutta}}
\markboth{At the Eastern Gate }{Pubbakoṭṭhakasutta}
\extramarks{SN 48.44}{SN 48.44}

\scevam{So\marginnote{1.1} I have heard. }At one time the Buddha was staying in \textsanskrit{Sāvatthī} at the eastern gate. Then the Buddha said to Venerable \textsanskrit{Sāriputta}: 

“\textsanskrit{Sāriputta},\marginnote{1.4} do you have faith that the faculties of faith, energy, mindfulness, immersion, and wisdom, when developed and cultivated, culminate, finish, and end in the deathless?” 

“Sir,\marginnote{2.1} in this case I don’t rely on faith in the Buddha’s claim that the faculties of faith, energy, mindfulness, immersion, and wisdom, when developed and cultivated, culminate, finish, and end in the deathless. There are those who have not known or seen or understood or realized or experienced this with wisdom. They may rely on faith in this matter. But there are those who have known, seen, understood, realized, and experienced this with wisdom. They have no doubts or uncertainties in this matter. I have known, seen, understood, realized, and experienced this with wisdom. I have no doubts or uncertainties that the faculties of faith, energy, mindfulness, immersion, and wisdom, when developed and cultivated, culminate, finish, and end in the deathless.” 

“Good,\marginnote{3.1} good, \textsanskrit{Sāriputta}! There are those who have not known or seen or understood or realized or experienced this with wisdom. They may rely on faith in this matter. But there are those who have known, seen, understood, realized, and experienced this with wisdom. They have no doubts or uncertainties that the faculties of faith, energy, mindfulness, immersion, and wisdom, when developed and cultivated, culminate, finish, and end in the deathless.” 

%
\section*{{\suttatitleacronym SN 48.45}{\suttatitletranslation At the Eastern Monastery (1st) }{\suttatitleroot Paṭhamapubbārāmasutta}}
\addcontentsline{toc}{section}{\tocacronym{SN 48.45} \toctranslation{At the Eastern Monastery (1st) } \tocroot{Paṭhamapubbārāmasutta}}
\markboth{At the Eastern Monastery (1st) }{Paṭhamapubbārāmasutta}
\extramarks{SN 48.45}{SN 48.45}

\scevam{So\marginnote{1.1} I have heard. }At one time the Buddha was staying near \textsanskrit{Sāvatthī} in the Eastern Monastery, the stilt longhouse of \textsanskrit{Migāra}’s mother. There the Buddha addressed the mendicants: “Mendicants, how many faculties must a mendicant develop and cultivate so that they can declare enlightenment: ‘I understand: “Rebirth is ended, the spiritual journey has been completed, what had to be done has been done, there is no return to any state of existence”’?” 

“Our\marginnote{2.1} teachings are rooted in the Buddha. …” 

“A\marginnote{2.2} mendicant must develop and cultivate one faculty so that they can declare enlightenment. What one? The faculty of wisdom. When a noble disciple has wisdom, the faith, energy, mindfulness, and immersion that follow along with that become stabilized. This is the one faculty that a mendicant must develop and cultivate so that they can declare enlightenment: ‘I understand: “Rebirth is ended, the spiritual journey has been completed, what had to be done has been done, there is no return to any state of existence”’.” 

%
\section*{{\suttatitleacronym SN 48.46}{\suttatitletranslation At the Eastern Monastery (2nd) }{\suttatitleroot Dutiyapubbārāmasutta}}
\addcontentsline{toc}{section}{\tocacronym{SN 48.46} \toctranslation{At the Eastern Monastery (2nd) } \tocroot{Dutiyapubbārāmasutta}}
\markboth{At the Eastern Monastery (2nd) }{Dutiyapubbārāmasutta}
\extramarks{SN 48.46}{SN 48.46}

The\marginnote{1.1} same setting. 

“Mendicants,\marginnote{1.2} how many faculties must a mendicant develop and cultivate so that they can declare enlightenment: ‘I understand: “Rebirth is ended, the spiritual journey has been completed, what had to be done has been done, there is no return to any state of existence”’?” 

“Our\marginnote{2.1} teachings are rooted in the Buddha. …” 

“A\marginnote{2.2} mendicant must develop and cultivate two faculties so that they can declare enlightenment. What two? Noble wisdom and noble freedom. For their noble wisdom is the faculty of wisdom. And their noble freedom is the faculty of immersion. 

These\marginnote{2.8} are the two faculties that a mendicant must develop and cultivate so that they can declare enlightenment: ‘I understand: “Rebirth is ended, the spiritual journey has been completed, what had to be done has been done, there is no return to any state of existence”’.” 

%
\section*{{\suttatitleacronym SN 48.47}{\suttatitletranslation At the Eastern Monastery (3rd) }{\suttatitleroot Tatiyapubbārāmasutta}}
\addcontentsline{toc}{section}{\tocacronym{SN 48.47} \toctranslation{At the Eastern Monastery (3rd) } \tocroot{Tatiyapubbārāmasutta}}
\markboth{At the Eastern Monastery (3rd) }{Tatiyapubbārāmasutta}
\extramarks{SN 48.47}{SN 48.47}

The\marginnote{1.1} same setting. 

“Mendicants,\marginnote{1.2} how many faculties must a mendicant develop and cultivate so that they can declare enlightenment: ‘I understand: “Rebirth is ended, the spiritual journey has been completed, what had to be done has been done, there is no return to any state of existence”’?” 

“Our\marginnote{2.1} teachings are rooted in the Buddha. …” 

“A\marginnote{2.2} mendicant must develop and cultivate four faculties so that they can declare enlightenment. What four? The faculties of energy, mindfulness, immersion, and wisdom. 

These\marginnote{2.6} are the four faculties that a mendicant must develop and cultivate so that they can declare enlightenment: ‘I understand: “Rebirth is ended, the spiritual journey has been completed, what had to be done has been done, there is no return to any state of existence”’.” 

%
\section*{{\suttatitleacronym SN 48.48}{\suttatitletranslation At the Eastern Monastery (4th) }{\suttatitleroot Catutthapubbārāmasutta}}
\addcontentsline{toc}{section}{\tocacronym{SN 48.48} \toctranslation{At the Eastern Monastery (4th) } \tocroot{Catutthapubbārāmasutta}}
\markboth{At the Eastern Monastery (4th) }{Catutthapubbārāmasutta}
\extramarks{SN 48.48}{SN 48.48}

The\marginnote{1.1} same setting. 

“Mendicants,\marginnote{1.2} how many faculties must a mendicant develop and cultivate so that they can declare enlightenment: ‘I understand: “Rebirth is ended, the spiritual journey has been completed, what had to be done has been done, there is no return to any state of existence”’?” 

“Our\marginnote{2.1} teachings are rooted in the Buddha. …” 

“A\marginnote{2.2} mendicant must develop and cultivate five faculties so that they can declare enlightenment. What five? The faculties of faith, energy, mindfulness, immersion, and wisdom. 

These\marginnote{2.6} are the five faculties that a mendicant must develop and cultivate so that they can declare enlightenment: ‘I understand: “Rebirth is ended, the spiritual journey has been completed, what had to be done has been done, there is no return to any state of existence”’.” 

%
\section*{{\suttatitleacronym SN 48.49}{\suttatitletranslation About Bhāradvāja the Alms-gatherer }{\suttatitleroot Piṇḍolabhāradvājasutta}}
\addcontentsline{toc}{section}{\tocacronym{SN 48.49} \toctranslation{About Bhāradvāja the Alms-gatherer } \tocroot{Piṇḍolabhāradvājasutta}}
\markboth{About Bhāradvāja the Alms-gatherer }{Piṇḍolabhāradvājasutta}
\extramarks{SN 48.49}{SN 48.49}

\scevam{So\marginnote{1.1} I have heard. }At one time the Buddha was staying near Kosambi, in Ghosita’s Monastery. 

Now\marginnote{1.3} at that time Venerable \textsanskrit{Bhāradvāja} the Alms-gatherer had declared enlightenment: “I understand: ‘Rebirth is ended, the spiritual journey has been completed, what had to be done has been done, there is no return to any state of existence.’” 

Then\marginnote{1.5} several mendicants went up to the Buddha, bowed, sat down to one side, and told him what had happened. Then they said, “What reason does \textsanskrit{Bhāradvāja} the Alms-gatherer see for doing this?” 

“It’s\marginnote{3.1} because \textsanskrit{Bhāradvāja} the Alms-gatherer has developed and cultivated three faculties that he declares enlightenment: ‘I understand: “Rebirth is ended, the spiritual journey has been completed, what had to be done has been done, there is no return to any state of existence.”’ 

What\marginnote{3.3} three? The faculties of mindfulness, immersion, and wisdom. 

It’s\marginnote{3.5} because \textsanskrit{Bhāradvāja} the Alms-gatherer has developed and cultivated these three faculties that he declares enlightenment. 

What’s\marginnote{3.7} the culmination of these three faculties? They culminate in ending. In the ending of what? Of rebirth, old age, and death. 

It’s\marginnote{3.11} because he sees that they culminate in the ending of rebirth, old age, and death that \textsanskrit{Bhāradvāja} the Alms-gatherer declares enlightenment: ‘I understand: “Rebirth is ended, the spiritual journey has been completed, what had to be done has been done, there is no return to any state of existence.”’” 

%
\section*{{\suttatitleacronym SN 48.50}{\suttatitletranslation At Āpaṇa }{\suttatitleroot Āpaṇasutta}}
\addcontentsline{toc}{section}{\tocacronym{SN 48.50} \toctranslation{At Āpaṇa } \tocroot{Āpaṇasutta}}
\markboth{At Āpaṇa }{Āpaṇasutta}
\extramarks{SN 48.50}{SN 48.50}

\scevam{So\marginnote{1.1} I have heard. }At one time the Buddha was staying in the land of the \textsanskrit{Aṅgas}, near the \textsanskrit{Aṅgan} town called \textsanskrit{Āpaṇa}. Then the Buddha said to Venerable \textsanskrit{Sāriputta}: 

“\textsanskrit{Sāriputta},\marginnote{1.4} would a noble disciple who is sure and devoted to the Realized One have any doubt or uncertainty about the Realized One or his instructions?” 

“Sir,\marginnote{2.1} a noble disciple who is sure and devoted to the Realized One would have no doubt or uncertainty about the Realized One or his instructions. 

You\marginnote{2.2} can expect that a faithful noble disciple will live with energy roused up for giving up unskillful qualities and embracing skillful qualities. They’re strong, staunchly vigorous, not slacking off when it comes to developing skillful qualities. For their energy is the faculty of energy. 

You\marginnote{3.1} can expect that a faithful and energetic noble disciple will be mindful, with utmost mindfulness and alertness, able to remember and recall what was said and done long ago. For their mindfulness is the faculty of mindfulness. 

You\marginnote{4.1} can expect that a faithful, energetic, and mindful noble disciple will, relying on letting go, gain immersion, gain unification of mind. For their \textsanskrit{samādhi} is the faculty of immersion. 

You\marginnote{5.1} can expect that a faithful, energetic, mindful noble disciple with their mind immersed in \textsanskrit{samādhi} will understand this: ‘Transmigration has no known beginning. No first point is found of sentient beings roaming and transmigrating, shrouded by ignorance and fettered by craving. But when that dark mass of ignorance fades away and ceases with nothing left over, that state is peaceful and sublime. That is, the stilling of all activities, the letting go of all attachments, the ending of craving, fading away, cessation, extinguishment.’ For their noble wisdom is the faculty of wisdom. 

When\marginnote{6.1} a noble disciple has tried again and again, recollected again and again, entered immersion again and again, and understood with wisdom again and again, they will be confident of this: ‘I have previously heard of these things. But now I have direct meditative experience of them, and see them with penetrating wisdom.’ For their faith is the faculty of faith.” 

“Good,\marginnote{7.1} good, \textsanskrit{Sāriputta}! 

“\textsanskrit{Sāriputta},\marginnote{7.2} a noble disciple who is sure and devoted to the Realized One would have no doubt or uncertainty about the Realized One or his instructions. …” 

(The\marginnote{7.3} Buddha then repeated \textsanskrit{Sāriputta}’s answer word for word.) 

%
\addtocontents{toc}{\let\protect\contentsline\protect\nopagecontentsline}
\chapter*{The Chapter on the Boar’s Cave }
\addcontentsline{toc}{chapter}{\tocchapterline{The Chapter on the Boar’s Cave }}
\addtocontents{toc}{\let\protect\contentsline\protect\oldcontentsline}

%
\section*{{\suttatitleacronym SN 48.51}{\suttatitletranslation At Sālā }{\suttatitleroot Sālasutta}}
\addcontentsline{toc}{section}{\tocacronym{SN 48.51} \toctranslation{At Sālā } \tocroot{Sālasutta}}
\markboth{At Sālā }{Sālasutta}
\extramarks{SN 48.51}{SN 48.51}

\scevam{So\marginnote{1.1} I have heard. }At one time the Buddha was staying in the land of the Kosalans near the brahmin village of \textsanskrit{Sālā}. There the Buddha addressed the mendicants: “Mendicants, the lion, king of beasts, is said to be the best of animals in terms of strength, speed, and courage. In the same way, the faculty of wisdom is said to be the best of the qualities that lead to awakening in terms of becoming awakened. 

And\marginnote{2.1} what are the qualities that lead to awakening? The faculties of faith, energy, mindfulness, immersion, and wisdom are qualities that lead to awakening, in that they lead to becoming awakened. The lion, king of beasts, is said to be the best of animals in terms of strength, speed, and courage. In the same way, the faculty of wisdom is said to be the best of the qualities that lead to awakening in terms of becoming awakened.” 

%
\section*{{\suttatitleacronym SN 48.52}{\suttatitletranslation In the Land of the Mallas }{\suttatitleroot Mallikasutta}}
\addcontentsline{toc}{section}{\tocacronym{SN 48.52} \toctranslation{In the Land of the Mallas } \tocroot{Mallikasutta}}
\markboth{In the Land of the Mallas }{Mallikasutta}
\extramarks{SN 48.52}{SN 48.52}

\scevam{So\marginnote{1.1} I have heard. }At one time the Buddha was staying in the land of the Mallas, near the Mallian town called Uruvelakappa. There the Buddha addressed the mendicants: 

“Mendicants,\marginnote{1.4} as long as noble knowledge hasn’t arisen for a noble disciple the four faculties are not stable and fixed. But when noble knowledge has arisen for a noble disciple the four faculties become stable and fixed. 

It’s\marginnote{2.1} just like in a bungalow. As long as the roof peak is not lifted into place, the rafters are not stable or fixed. But when the roof peak is lifted into place, the rafters become stable and fixed. 

In\marginnote{2.3} the same way, as long as noble knowledge hasn’t arisen for a noble disciple the four faculties are not stable and fixed. But when noble knowledge has arisen for a noble disciple the four faculties become stable and fixed. 

What\marginnote{3.1} four? The faculties of faith, energy, mindfulness, and immersion. When a noble disciple has wisdom, the faith, energy, mindfulness, and immersion that follow along with that become stabilized.” 

%
\section*{{\suttatitleacronym SN 48.53}{\suttatitletranslation A Trainee }{\suttatitleroot Sekhasutta}}
\addcontentsline{toc}{section}{\tocacronym{SN 48.53} \toctranslation{A Trainee } \tocroot{Sekhasutta}}
\markboth{A Trainee }{Sekhasutta}
\extramarks{SN 48.53}{SN 48.53}

\scevam{So\marginnote{1.1} I have heard. }At one time the Buddha was staying near Kosambi, in Ghosita’s Monastery. There the Buddha addressed the mendicants: “Mendicants, is there a way that a mendicant who is a trainee, standing at the level of a trainee, can understand that they are a trainee? And that a mendicant who is an adept, standing at the level of an adept, can understand that they are an adept?” 

“Our\marginnote{2.1} teachings are rooted in the Buddha. …” 

“There\marginnote{2.2} is a way that a mendicant who is a trainee, standing at the level of a trainee, can understand that they are a trainee, and that a mendicant who is an adept, standing at the level of an adept, can understand that they are an adept. 

And\marginnote{3.1} what is a way that a mendicant who is a trainee can understand that they are a trainee? It’s when a mendicant who is a trainee truly understands: ‘This is suffering’ … ‘This is the origin of suffering’ … ‘This is the cessation of suffering’ … ‘This is the practice that leads to the cessation of suffering’. This is a way that a mendicant who is a trainee can understand that they are a trainee. 

Furthermore,\marginnote{4.1} a mendicant who is a trainee reflects: ‘Is there any other ascetic or brahmin elsewhere whose teaching is as true, as real, as accurate as that of the Buddha?’ They understand: ‘There is no other ascetic or brahmin elsewhere whose teaching is as true, as real, as accurate as that of the Buddha.’ This too is a way that a mendicant who is a trainee can understand that they are a trainee. 

Furthermore,\marginnote{5.1} a mendicant who is a trainee understands the five faculties: faith, energy, mindfulness, immersion, and wisdom. And although they don’t have direct meditative experience of their destination, apex, fruit, and culmination, they do see them with penetrating wisdom. This too is a way that a mendicant who is a trainee can understand that they are a trainee. 

And\marginnote{6.1} what is the way that a mendicant who is an adept can understand that they are an adept? It’s when a mendicant who is an adept understands the five faculties: faith, energy, mindfulness, immersion, and wisdom. They have direct meditative experience of their destination, apex, fruit, and culmination, and they see them with penetrating wisdom. This is a way that a mendicant who is an adept can understand that they are an adept. 

Furthermore,\marginnote{7.1} a mendicant who is an adept understands the six faculties: eye, ear, nose, tongue, body, and mind. They understand: ‘These six faculties will totally and utterly cease without anything left over. And no other six faculties will arise anywhere anyhow.’ This too is a way that a mendicant who is an adept can understand that they are an adept.” 

%
\section*{{\suttatitleacronym SN 48.54}{\suttatitletranslation Footprints }{\suttatitleroot Padasutta}}
\addcontentsline{toc}{section}{\tocacronym{SN 48.54} \toctranslation{Footprints } \tocroot{Padasutta}}
\markboth{Footprints }{Padasutta}
\extramarks{SN 48.54}{SN 48.54}

“The\marginnote{1.1} footprints of all creatures that walk can fit inside an elephant’s footprint, so an elephant’s footprint is said to be the biggest of them all. In the same way, the faculty of wisdom is said to be the best of the steps that lead to awakening in terms of becoming awakened. 

And\marginnote{1.3} what are the steps that lead to awakening? The faculties of faith, energy, mindfulness, immersion, and wisdom are steps that lead to awakening, in that they lead to becoming awakened. 

The\marginnote{1.9} footprints of all creatures that walk can fit inside an elephant’s footprint, so an elephant’s footprint is said to be the biggest of them all. In the same way, the faculty of wisdom is said to be the best of the steps that lead to awakening in terms of becoming awakened.” 

%
\section*{{\suttatitleacronym SN 48.55}{\suttatitletranslation Heartwood }{\suttatitleroot Sārasutta}}
\addcontentsline{toc}{section}{\tocacronym{SN 48.55} \toctranslation{Heartwood } \tocroot{Sārasutta}}
\markboth{Heartwood }{Sārasutta}
\extramarks{SN 48.55}{SN 48.55}

“Of\marginnote{1.1} all kinds of fragrant heartwood, red sandalwood is said to be the best. In the same way, the faculty of wisdom is said to be the best of the qualities that lead to awakening in terms of becoming awakened. 

And\marginnote{1.3} what are the qualities that lead to awakening? The faculties of faith, energy, mindfulness, immersion, and wisdom are qualities that lead to awakening, in that they lead to becoming awakened. 

Of\marginnote{1.9} all kinds of fragrant heartwood, red sandalwood is said to be the best. In the same way, the faculty of wisdom is said to be the best of the qualities that lead to awakening in terms of becoming awakened.” 

%
\section*{{\suttatitleacronym SN 48.56}{\suttatitletranslation Grounded }{\suttatitleroot Patiṭṭhitasutta}}
\addcontentsline{toc}{section}{\tocacronym{SN 48.56} \toctranslation{Grounded } \tocroot{Patiṭṭhitasutta}}
\markboth{Grounded }{Patiṭṭhitasutta}
\extramarks{SN 48.56}{SN 48.56}

“Mendicants,\marginnote{1.1} when a mendicant is grounded in one thing the five faculties become developed, well developed. What one thing? Diligence. And what is diligence? It’s when a mendicant looks after their mind when it comes to defilements and things that stimulate defilements. As they do so the faculties of faith, energy, mindfulness, immersion, and wisdom are fully developed. That’s how when a mendicant is grounded in one thing the five faculties become developed, well developed.” 

%
\section*{{\suttatitleacronym SN 48.57}{\suttatitletranslation With Brahmā Sahampati }{\suttatitleroot Sahampatibrahmasutta}}
\addcontentsline{toc}{section}{\tocacronym{SN 48.57} \toctranslation{With Brahmā Sahampati } \tocroot{Sahampatibrahmasutta}}
\markboth{With Brahmā Sahampati }{Sahampatibrahmasutta}
\extramarks{SN 48.57}{SN 48.57}

At\marginnote{1.1} one time, when he was first awakened, the Buddha was staying near \textsanskrit{Uruvelā} at the goatherd’s banyan tree on the bank of the \textsanskrit{Nerañjarā} River. 

Then\marginnote{1.2} as he was in private retreat this thought came to his mind, “When these five faculties are developed and cultivated they culminate, finish, and end in the deathless. What five? The faculties of faith, energy, mindfulness, immersion, and wisdom. When these five faculties are developed and cultivated they culminate, finish, and end in the deathless.” 

Then\marginnote{2.1} \textsanskrit{Brahmā} Sahampati knew what the Buddha was thinking. As easily as a strong person would extend or contract their arm, he vanished from the \textsanskrit{Brahmā} realm and reappeared in front of the Buddha. He arranged his robe over one shoulder, raised his joined palms toward the Buddha, and said: 

“That’s\marginnote{2.3} so true, Blessed One! That’s so true, Holy One! When these five faculties are developed and cultivated they culminate, finish, and end in the deathless. What five? The faculties of faith, energy, mindfulness, immersion, and wisdom. When these five faculties are developed and cultivated they culminate, finish, and end in the deathless. 

Once\marginnote{3.1} upon a time, sir, I lived the spiritual life under the fully awakened Buddha Kassapa. There they knew me as the mendicant Sahaka. Because of developing and cultivating these same five faculties I lost desire for sensual pleasures. When my body broke up, after death, I was reborn in a good place, in the \textsanskrit{Brahmā} realm. There they know me as \textsanskrit{Brahmā} Sahampati. 

That’s\marginnote{3.7} so true, Blessed One! That’s so true, Holy One! I know and see how when these five faculties are developed and cultivated they culminate, finish, and end in the deathless.” 

%
\section*{{\suttatitleacronym SN 48.58}{\suttatitletranslation The Boar’s Cave }{\suttatitleroot Sūkarakhatasutta}}
\addcontentsline{toc}{section}{\tocacronym{SN 48.58} \toctranslation{The Boar’s Cave } \tocroot{Sūkarakhatasutta}}
\markboth{The Boar’s Cave }{Sūkarakhatasutta}
\extramarks{SN 48.58}{SN 48.58}

At\marginnote{1.1} one time the Buddha was staying near \textsanskrit{Rājagaha}, on the Vulture’s Peak Mountain in the Boar’s Cave. Then the Buddha said to Venerable \textsanskrit{Sāriputta}: 

“\textsanskrit{Sāriputta},\marginnote{1.3} considering what benefit does a mendicant with defilements ended, while still alive, continue to show utmost devotion for the Realized One or his instructions?” 

“Sir,\marginnote{1.4} it is considering the supreme sanctuary that a mendicant with defilements ended, while still alive, continues to show utmost devotion for the Realized One or his instructions.” 

“Good,\marginnote{1.5} good, \textsanskrit{Sāriputta}! For it is considering the supreme sanctuary that a mendicant whose defilements are ended, while still alive, continues to show utmost devotion for the Realized One or his instructions. 

And\marginnote{2.1} what is that supreme sanctuary?” 

“It’s\marginnote{2.2} when a mendicant with defilements ended develops the faculties of faith, energy, mindfulness, immersion, and wisdom, which lead to peace and awakening. It is considering this supreme sanctuary that a mendicant with defilements ended, while still alive, continues to show utmost devotion for the Realized One or his instructions.” 

“Good,\marginnote{2.8} good, \textsanskrit{Sāriputta}! For this is that supreme sanctuary. 

And\marginnote{3.1} what is that utmost devotion that a mendicant with defilements ended, while still alive, continues to show towards the Realized One or his instructions?” 

“It’s\marginnote{3.2} when a mendicant with defilements ended maintains respect and reverence for the Teacher, the teaching, the \textsanskrit{Saṅgha}, the training, and immersion. This is that utmost devotion.” 

“Good,\marginnote{3.4} good, \textsanskrit{Sāriputta}! For this is that utmost devotion that a mendicant with defilements ended, while still alive, continues to show towards the Realized One or his instructions.” 

%
\section*{{\suttatitleacronym SN 48.59}{\suttatitletranslation Arising (1st) }{\suttatitleroot Paṭhamauppādasutta}}
\addcontentsline{toc}{section}{\tocacronym{SN 48.59} \toctranslation{Arising (1st) } \tocroot{Paṭhamauppādasutta}}
\markboth{Arising (1st) }{Paṭhamauppādasutta}
\extramarks{SN 48.59}{SN 48.59}

At\marginnote{1.1} \textsanskrit{Sāvatthī}. 

“Mendicants,\marginnote{1.2} these five faculties don’t arise to be developed and cultivated except when a Realized One, a perfected one, a fully awakened Buddha has appeared. What five? The faculties of faith, energy, mindfulness, immersion, and wisdom. These five faculties don’t arise to be developed and cultivated except when a Realized One, a perfected one, a fully awakened Buddha has appeared.” 

%
\section*{{\suttatitleacronym SN 48.60}{\suttatitletranslation Arising (2nd) }{\suttatitleroot Dutiyauppādasutta}}
\addcontentsline{toc}{section}{\tocacronym{SN 48.60} \toctranslation{Arising (2nd) } \tocroot{Dutiyauppādasutta}}
\markboth{Arising (2nd) }{Dutiyauppādasutta}
\extramarks{SN 48.60}{SN 48.60}

“Mendicants,\marginnote{1.1} these five faculties don’t arise to be developed and cultivated apart from the Holy One’s training. What five? The faculties of faith, energy, mindfulness, immersion, and wisdom. These five faculties don’t arise to be developed and cultivated apart from the Holy One’s training.” 

%
\addtocontents{toc}{\let\protect\contentsline\protect\nopagecontentsline}
\chapter*{The Chapter on Leading to Awakening }
\addcontentsline{toc}{chapter}{\tocchapterline{The Chapter on Leading to Awakening }}
\addtocontents{toc}{\let\protect\contentsline\protect\oldcontentsline}

%
\section*{{\suttatitleacronym SN 48.61}{\suttatitletranslation Fetters }{\suttatitleroot Saṁyojanasutta}}
\addcontentsline{toc}{section}{\tocacronym{SN 48.61} \toctranslation{Fetters } \tocroot{Saṁyojanasutta}}
\markboth{Fetters }{Saṁyojanasutta}
\extramarks{SN 48.61}{SN 48.61}

At\marginnote{1.1} \textsanskrit{Sāvatthī}. 

“Mendicants,\marginnote{1.2} when these five faculties are developed and cultivated they lead to giving up the fetters. What five? The faculties of faith, energy, mindfulness, immersion, and wisdom. When these five faculties are developed and cultivated they lead to giving up the fetters.” 

%
\section*{{\suttatitleacronym SN 48.62}{\suttatitletranslation Tendencies }{\suttatitleroot Anusayasutta}}
\addcontentsline{toc}{section}{\tocacronym{SN 48.62} \toctranslation{Tendencies } \tocroot{Anusayasutta}}
\markboth{Tendencies }{Anusayasutta}
\extramarks{SN 48.62}{SN 48.62}

“Mendicants,\marginnote{1.1} when these five faculties are developed and cultivated they lead to uprooting the underlying tendencies. What five? The faculties of faith, energy, mindfulness, immersion, and wisdom. When these five faculties are developed and cultivated they lead to uprooting the underlying tendencies.” 

%
\section*{{\suttatitleacronym SN 48.63}{\suttatitletranslation Complete Understanding }{\suttatitleroot Pariññāsutta}}
\addcontentsline{toc}{section}{\tocacronym{SN 48.63} \toctranslation{Complete Understanding } \tocroot{Pariññāsutta}}
\markboth{Complete Understanding }{Pariññāsutta}
\extramarks{SN 48.63}{SN 48.63}

“Mendicants,\marginnote{1.1} when these five faculties are developed and cultivated they lead to the complete understanding of the course of time. What five? The faculties of faith, energy, mindfulness, immersion, and wisdom. When these five faculties are developed and cultivated they lead to the complete understanding of the course of time.” 

%
\section*{{\suttatitleacronym SN 48.64}{\suttatitletranslation Ending of Defilements }{\suttatitleroot Āsavakkhayasutta}}
\addcontentsline{toc}{section}{\tocacronym{SN 48.64} \toctranslation{Ending of Defilements } \tocroot{Āsavakkhayasutta}}
\markboth{Ending of Defilements }{Āsavakkhayasutta}
\extramarks{SN 48.64}{SN 48.64}

“Mendicants,\marginnote{1.1} when these five faculties are developed and cultivated they lead to the ending of defilements. What five? The faculties of faith, energy, mindfulness, immersion, and wisdom. When these five faculties are developed and cultivated they lead to the ending of defilements.” 

“Mendicants,\marginnote{2.1} when these five faculties are developed and cultivated they lead to giving up the fetters, uprooting the underlying tendencies, completely understanding the course of time, and ending the defilements. What five? The faculties of faith, energy, mindfulness, immersion, and wisdom. When these five faculties are developed and cultivated they lead to giving up the fetters, uprooting the underlying tendencies, completely understanding the course of time, and ending the defilements.” 

%
\section*{{\suttatitleacronym SN 48.65}{\suttatitletranslation Two Fruits }{\suttatitleroot Paṭhamaphalasutta}}
\addcontentsline{toc}{section}{\tocacronym{SN 48.65} \toctranslation{Two Fruits } \tocroot{Paṭhamaphalasutta}}
\markboth{Two Fruits }{Paṭhamaphalasutta}
\extramarks{SN 48.65}{SN 48.65}

“Mendicants,\marginnote{1.1} there are these five faculties. What five? The faculties of faith, energy, mindfulness, immersion, and wisdom. These are the five faculties. 

Because\marginnote{1.5} of developing and cultivating these five faculties, one of two results can be expected: enlightenment in the present life, or if there’s something left over, non-return.” 

%
\section*{{\suttatitleacronym SN 48.66}{\suttatitletranslation Seven Benefits }{\suttatitleroot Dutiyaphalasutta}}
\addcontentsline{toc}{section}{\tocacronym{SN 48.66} \toctranslation{Seven Benefits } \tocroot{Dutiyaphalasutta}}
\markboth{Seven Benefits }{Dutiyaphalasutta}
\extramarks{SN 48.66}{SN 48.66}

“Mendicants,\marginnote{1.1} there are these five faculties. What five? The faculties of faith, energy, mindfulness, immersion, and wisdom. These are the five faculties. 

Because\marginnote{1.5} of developing and cultivating these five faculties, seven fruits and benefits can be expected. What seven? They attain enlightenment early on in this very life. If not, they attain enlightenment at the time of death. If not, with the ending of the five lower fetters, they’re extinguished between one life and the next … they’re extinguished upon landing … they’re extinguished without extra effort … they’re extinguished with extra effort … they head upstream, going to the \textsanskrit{Akaniṭṭha} realm. 

Because\marginnote{1.14} of developing and cultivating these five faculties, these seven fruits and benefits can be expected.” 

%
\section*{{\suttatitleacronym SN 48.67}{\suttatitletranslation A Tree (1st) }{\suttatitleroot Paṭhamarukkhasutta}}
\addcontentsline{toc}{section}{\tocacronym{SN 48.67} \toctranslation{A Tree (1st) } \tocroot{Paṭhamarukkhasutta}}
\markboth{A Tree (1st) }{Paṭhamarukkhasutta}
\extramarks{SN 48.67}{SN 48.67}

“Mendicants,\marginnote{1.1} of all the trees in India, the rose-apple is said to be the best. In the same way, the faculty of wisdom is said to be the best of the qualities that lead to awakening in terms of becoming awakened. And what are the qualities that lead to awakening? The faculties of faith, energy, mindfulness, immersion, and wisdom are qualities that lead to awakening, in that they lead to becoming awakened. Of all the trees in India, the rose-apple is said to be the best. In the same way, the faculty of wisdom is said to be the best of the qualities that lead to awakening in terms of becoming awakened.” 

%
\section*{{\suttatitleacronym SN 48.68}{\suttatitletranslation A Tree (2nd) }{\suttatitleroot Dutiyarukkhasutta}}
\addcontentsline{toc}{section}{\tocacronym{SN 48.68} \toctranslation{A Tree (2nd) } \tocroot{Dutiyarukkhasutta}}
\markboth{A Tree (2nd) }{Dutiyarukkhasutta}
\extramarks{SN 48.68}{SN 48.68}

“Mendicants,\marginnote{1.1} of all the trees belonging to the gods of the Thirty-Three, the Shady Orchid Tree is said to be the best. In the same way, the faculty of wisdom is said to be the best of the qualities that lead to awakening in terms of becoming awakened. 

And\marginnote{1.3} what are the qualities that lead to awakening? The faculties of faith, energy, mindfulness, immersion, and wisdom are qualities that lead to awakening, in that they lead to becoming awakened. 

Of\marginnote{1.9} all the trees belonging to the gods of the Thirty-Three, the Shady Orchid Tree is said to be the best. In the same way, the faculty of wisdom is said to be the best of the qualities that lead to awakening in terms of becoming awakened.” 

%
\section*{{\suttatitleacronym SN 48.69}{\suttatitletranslation A Tree (3rd) }{\suttatitleroot Tatiyarukkhasutta}}
\addcontentsline{toc}{section}{\tocacronym{SN 48.69} \toctranslation{A Tree (3rd) } \tocroot{Tatiyarukkhasutta}}
\markboth{A Tree (3rd) }{Tatiyarukkhasutta}
\extramarks{SN 48.69}{SN 48.69}

“Mendicants,\marginnote{1.1} of all the trees belonging to the demons, the trumpet-flower tree is said to be the best. In the same way, the faculty of wisdom is said to be the best of the qualities that lead to awakening in terms of becoming awakened. And what are the qualities that lead to awakening? The faculties of faith, energy, mindfulness, immersion, and wisdom are qualities that lead to awakening, in that they lead to becoming awakened. Of all the trees belonging to the demons, the trumpet-flower tree is said to be the best. In the same way, the faculty of wisdom is said to be the best of the qualities that lead to awakening in terms of becoming awakened.” 

%
\section*{{\suttatitleacronym SN 48.70}{\suttatitletranslation A Tree (4th) }{\suttatitleroot Catuttharukkhasutta}}
\addcontentsline{toc}{section}{\tocacronym{SN 48.70} \toctranslation{A Tree (4th) } \tocroot{Catuttharukkhasutta}}
\markboth{A Tree (4th) }{Catuttharukkhasutta}
\extramarks{SN 48.70}{SN 48.70}

“Mendicants,\marginnote{1.1} of all the trees belonging to the phoenixes, the red silk-cotton tree is said to be the best. In the same way, the faculty of wisdom is said to be the best of the qualities that lead to awakening in terms of becoming awakened. And what are the qualities that lead to awakening? The faculties of faith, energy, mindfulness, immersion, and wisdom are qualities that lead to awakening, in that they lead to becoming awakened. Of all the trees belonging to the phoenixes, the red silk-cotton tree is said to be the best. In the same way, the faculty of wisdom is said to be the best of the qualities that lead to awakening in terms of becoming awakened.” 

%
\addtocontents{toc}{\let\protect\contentsline\protect\nopagecontentsline}
\chapter*{The Chapter of Abbreviated Texts on the Ganges }
\addcontentsline{toc}{chapter}{\tocchapterline{The Chapter of Abbreviated Texts on the Ganges }}
\addtocontents{toc}{\let\protect\contentsline\protect\oldcontentsline}

%
\section*{{\suttatitleacronym SN 48.71–82}{\suttatitletranslation Slanting East, Etc. }{\suttatitleroot Gaṅgāpeyyālavagga}}
\addcontentsline{toc}{section}{\tocacronym{SN 48.71–82} \toctranslation{Slanting East, Etc. } \tocroot{Gaṅgāpeyyālavagga}}
\markboth{Slanting East, Etc. }{Gaṅgāpeyyālavagga}
\extramarks{SN 48.71–82}{SN 48.71–82}

“Mendicants,\marginnote{1.1} the Ganges river slants, slopes, and inclines to the east. In the same way, a mendicant developing and cultivating the five faculties slants, slopes, and inclines to extinguishment. 

How\marginnote{1.3} so? It’s when a mendicant develops the faculties of faith, energy, mindfulness, immersion, and wisdom, which rely on seclusion, fading away, and cessation, and ripen as letting go. That’s how a mendicant developing and cultivating the five faculties slants, slopes, and inclines to extinguishment.” 

\scendsection{(To be expanded for each of the different rivers as in SN 45.91–102.) }

\begin{quotation}%
Six\marginnote{2.1} on slanting to the east, \\
and six on slanting to the ocean; \\
these two sixes make twelve, \\
and that’s how this chapter is recited. 

%
\end{quotation}

%
\addtocontents{toc}{\let\protect\contentsline\protect\nopagecontentsline}
\chapter*{The Chapter on Diligence }
\addcontentsline{toc}{chapter}{\tocchapterline{The Chapter on Diligence }}
\addtocontents{toc}{\let\protect\contentsline\protect\oldcontentsline}

%
\section*{{\suttatitleacronym SN 48.83–92}{\suttatitletranslation Diligence }{\suttatitleroot Appamādavagga}}
\addcontentsline{toc}{section}{\tocacronym{SN 48.83–92} \toctranslation{Diligence } \tocroot{Appamādavagga}}
\markboth{Diligence }{Appamādavagga}
\extramarks{SN 48.83–92}{SN 48.83–92}

(To\marginnote{1.1} be expanded as in the chapter on diligence at SN 45.139–148.) 

\begin{quotation}%
The\marginnote{2.1} Realized One, footprint, roof peak, \\
roots, heartwood, jasmine, \\
monarch, sun and moon, \\
and cloth is the tenth. 

%
\end{quotation}

%
\addtocontents{toc}{\let\protect\contentsline\protect\nopagecontentsline}
\chapter*{The Chapter on Hard Work }
\addcontentsline{toc}{chapter}{\tocchapterline{The Chapter on Hard Work }}
\addtocontents{toc}{\let\protect\contentsline\protect\oldcontentsline}

%
\section*{{\suttatitleacronym SN 48.93–104}{\suttatitletranslation Hard Work }{\suttatitleroot Balakaraṇīyavagga}}
\addcontentsline{toc}{section}{\tocacronym{SN 48.93–104} \toctranslation{Hard Work } \tocroot{Balakaraṇīyavagga}}
\markboth{Hard Work }{Balakaraṇīyavagga}
\extramarks{SN 48.93–104}{SN 48.93–104}

(To\marginnote{1.1} be expanded as in the chapter on hard work at SN 45.149–160.) 

\begin{quotation}%
Hard\marginnote{2.1} work, seeds, and dragons, \\
a tree, a pot, and a spike, \\
the sky, and two on clouds, \\
a ship, a guest house, and a river. 

%
\end{quotation}

%
\addtocontents{toc}{\let\protect\contentsline\protect\nopagecontentsline}
\chapter*{The Chapter on Searches }
\addcontentsline{toc}{chapter}{\tocchapterline{The Chapter on Searches }}
\addtocontents{toc}{\let\protect\contentsline\protect\oldcontentsline}

%
\section*{{\suttatitleacronym SN 48.105–114}{\suttatitletranslation Searches }{\suttatitleroot Esanāvagga}}
\addcontentsline{toc}{section}{\tocacronym{SN 48.105–114} \toctranslation{Searches } \tocroot{Esanāvagga}}
\markboth{Searches }{Esanāvagga}
\extramarks{SN 48.105–114}{SN 48.105–114}

(To\marginnote{1.1} be expanded as in the chapter on searches at SN 45.161–170.) 

\begin{quotation}%
Searches,\marginnote{2.1} discriminations, defilements, \\
states of existence, three kinds of suffering, \\
barrenness, stains, and troubles, \\
feelings, craving, and thirst. 

%
\end{quotation}

%
\addtocontents{toc}{\let\protect\contentsline\protect\nopagecontentsline}
\chapter*{The Chapter on Floods }
\addcontentsline{toc}{chapter}{\tocchapterline{The Chapter on Floods }}
\addtocontents{toc}{\let\protect\contentsline\protect\oldcontentsline}

%
\section*{{\suttatitleacronym SN 48.115–124}{\suttatitletranslation Floods }{\suttatitleroot Oghavagga}}
\addcontentsline{toc}{section}{\tocacronym{SN 48.115–124} \toctranslation{Floods } \tocroot{Oghavagga}}
\markboth{Floods }{Oghavagga}
\extramarks{SN 48.115–124}{SN 48.115–124}

“Mendicants,\marginnote{1.1} there are five higher fetters. What five? Desire for rebirth in the realm of luminous form, desire for rebirth in the formless realm, conceit, restlessness, and ignorance. These are the five higher fetters. 

The\marginnote{1.5} five faculties should be developed for the direct knowledge, complete understanding, finishing, and giving up of these five higher fetters. What five? It’s when a mendicant develops the faculties of faith, energy, mindfulness, immersion, and wisdom, which rely on seclusion, fading away, and cessation, and ripen as letting go. 

These\marginnote{1.8} five faculties should be developed for the direct knowledge, complete understanding, finishing, and giving up of these five higher fetters.” 

\scendsection{(To be expanded as in the Linked Discourses on the Path, SN 45.171–179, with the above as the final discourse.) }

\begin{quotation}%
Floods,\marginnote{2.1} bonds, grasping, \\
ties, and underlying tendencies, \\
kinds of sensual stimulation, hindrances, \\
aggregates, and fetters high and low. 

%
\end{quotation}

%
\addtocontents{toc}{\let\protect\contentsline\protect\nopagecontentsline}
\chapter*{Another Chapter of Abbreviated Texts on the Ganges, Etc. }
\addcontentsline{toc}{chapter}{\tocchapterline{Another Chapter of Abbreviated Texts on the Ganges, Etc. }}
\addtocontents{toc}{\let\protect\contentsline\protect\oldcontentsline}

%
\section*{{\suttatitleacronym SN 48.125–136}{\suttatitletranslation Another on Sloping East, Etc. }{\suttatitleroot Punagaṅgāpeyyālavagga}}
\addcontentsline{toc}{section}{\tocacronym{SN 48.125–136} \toctranslation{Another on Sloping East, Etc. } \tocroot{Punagaṅgāpeyyālavagga}}
\markboth{Another on Sloping East, Etc. }{Punagaṅgāpeyyālavagga}
\extramarks{SN 48.125–136}{SN 48.125–136}

“Mendicants,\marginnote{1.1} the Ganges river slants, slopes, and inclines to the east. In the same way, a mendicant developing and cultivating the five faculties slants, slopes, and inclines to extinguishment. 

How\marginnote{1.3} so? It’s when a mendicant develops the faculties of faith, energy, mindfulness, immersion, and wisdom, which culminate in the removal of greed, hate, and delusion. That’s how a mendicant developing and cultivating the five faculties slants, slopes, and inclines to extinguishment.” 

\scendsection{(To be expanded for each of the different rivers as in SN 45.91–102.) }

\begin{quotation}%
Six\marginnote{2.1} on slanting to the east, \\
and six on slanting to the ocean; \\
these two sixes make twelve, \\
and that’s how this chapter is recited. 

%
\end{quotation}

%
\addtocontents{toc}{\let\protect\contentsline\protect\nopagecontentsline}
\chapter*{Another Chapter on Diligence }
\addcontentsline{toc}{chapter}{\tocchapterline{Another Chapter on Diligence }}
\addtocontents{toc}{\let\protect\contentsline\protect\oldcontentsline}

%
\section*{{\suttatitleacronym SN 48.137–146}{\suttatitletranslation Another Chapter on Diligence }{\suttatitleroot Punaappamādavagga}}
\addcontentsline{toc}{section}{\tocacronym{SN 48.137–146} \toctranslation{Another Chapter on Diligence } \tocroot{Punaappamādavagga}}
\markboth{Another Chapter on Diligence }{Punaappamādavagga}
\extramarks{SN 48.137–146}{SN 48.137–146}

(This\marginnote{1.1} text consists of the title only. To be expanded as in SN 45.139–148, removal of greed version.) 

%
\addtocontents{toc}{\let\protect\contentsline\protect\nopagecontentsline}
\chapter*{Another Chapter on Hard Work }
\addcontentsline{toc}{chapter}{\tocchapterline{Another Chapter on Hard Work }}
\addtocontents{toc}{\let\protect\contentsline\protect\oldcontentsline}

%
\section*{{\suttatitleacronym SN 48.147–158}{\suttatitletranslation Another Chapter on Hard Work }{\suttatitleroot Punagaṅgāpeyyālavagga}}
\addcontentsline{toc}{section}{\tocacronym{SN 48.147–158} \toctranslation{Another Chapter on Hard Work } \tocroot{Punagaṅgāpeyyālavagga}}
\markboth{Another Chapter on Hard Work }{Punagaṅgāpeyyālavagga}
\extramarks{SN 48.147–158}{SN 48.147–158}

(This\marginnote{1.1} text consists of the title only. To be expanded as in SN 45.149–160, removal of greed version.) 

%
\addtocontents{toc}{\let\protect\contentsline\protect\nopagecontentsline}
\chapter*{Another Chapter on Searches }
\addcontentsline{toc}{chapter}{\tocchapterline{Another Chapter on Searches }}
\addtocontents{toc}{\let\protect\contentsline\protect\oldcontentsline}

%
\section*{{\suttatitleacronym SN 48.159–168}{\suttatitletranslation Another Chapter on Searches }{\suttatitleroot Punaesanāvagga}}
\addcontentsline{toc}{section}{\tocacronym{SN 48.159–168} \toctranslation{Another Chapter on Searches } \tocroot{Punaesanāvagga}}
\markboth{Another Chapter on Searches }{Punaesanāvagga}
\extramarks{SN 48.159–168}{SN 48.159–168}

(To\marginnote{1.1} be expanded as in SN 45.161–170, removal of greed version.) 

%
\addtocontents{toc}{\let\protect\contentsline\protect\nopagecontentsline}
\chapter*{Another Chapter on Floods }
\addcontentsline{toc}{chapter}{\tocchapterline{Another Chapter on Floods }}
\addtocontents{toc}{\let\protect\contentsline\protect\oldcontentsline}

%
\section*{{\suttatitleacronym SN 48.169–178}{\suttatitletranslation Another Series on Floods, Etc. }{\suttatitleroot Punaoghavagga}}
\addcontentsline{toc}{section}{\tocacronym{SN 48.169–178} \toctranslation{Another Series on Floods, Etc. } \tocroot{Punaoghavagga}}
\markboth{Another Series on Floods, Etc. }{Punaoghavagga}
\extramarks{SN 48.169–178}{SN 48.169–178}

“Mendicants,\marginnote{1.1} there are five higher fetters. What five? Desire for rebirth in the realm of luminous form, desire for rebirth in the formless realm, conceit, restlessness, and ignorance. These are the five higher fetters. 

The\marginnote{1.5} five faculties should be developed for the direct knowledge, complete understanding, finishing, and giving up of these five higher fetters. What five? It’s when a mendicant develops the faculties of faith, energy, mindfulness, immersion, and wisdom, which culminate in the removal of greed, hate, and delusion. 

These\marginnote{1.8} five faculties should be developed for the direct knowledge, complete understanding, finishing, and giving up of these five higher fetters.” 

\scendsection{(To be expanded as in SN 45.171–179, with the above as the final discourse, removal of greed version.) }

\begin{quotation}%
Floods,\marginnote{2.1} bonds, grasping, \\
ties, and underlying tendencies, \\
kinds of sensual stimulation, hindrances, \\
aggregates, and fetters high and low. 

%
\end{quotation}

\scendsutta{The Linked Discourses on the Faculties is the fourth section. }

%
\addtocontents{toc}{\let\protect\contentsline\protect\nopagecontentsline}
\part*{Linked Discourses on the Right Efforts }
\addcontentsline{toc}{part}{Linked Discourses on the Right Efforts }
\markboth{}{}
\addtocontents{toc}{\let\protect\contentsline\protect\oldcontentsline}

%
\addtocontents{toc}{\let\protect\contentsline\protect\nopagecontentsline}
\chapter*{The Chapter of Abbreviated Texts on the Ganges }
\addcontentsline{toc}{chapter}{\tocchapterline{The Chapter of Abbreviated Texts on the Ganges }}
\addtocontents{toc}{\let\protect\contentsline\protect\oldcontentsline}

%
\section*{{\suttatitleacronym SN 49.1–12}{\suttatitletranslation Sloping East, Etc. }{\suttatitleroot Gaṅgāpeyyālavagga}}
\addcontentsline{toc}{section}{\tocacronym{SN 49.1–12} \toctranslation{Sloping East, Etc. } \tocroot{Gaṅgāpeyyālavagga}}
\markboth{Sloping East, Etc. }{Gaṅgāpeyyālavagga}
\extramarks{SN 49.1–12}{SN 49.1–12}

At\marginnote{1.1} \textsanskrit{Sāvatthī}. 

There\marginnote{1.2} the Buddha said: 

“Mendicants,\marginnote{1.3} there are these four right efforts. What four? 

It’s\marginnote{1.5} when a mendicant generates enthusiasm, tries, makes an effort, exerts the mind, and strives so that bad, unskillful qualities don’t arise. 

They\marginnote{1.6} generate enthusiasm, try, make an effort, exert the mind, and strive so that bad, unskillful qualities that have arisen are given up. 

They\marginnote{1.7} generate enthusiasm, try, make an effort, exert the mind, and strive so that skillful qualities arise. 

They\marginnote{1.8} generate enthusiasm, try, make an effort, exert the mind, and strive so that skillful qualities that have arisen remain, are not lost, but increase, mature, and are completed by development. These are the four right efforts. 

The\marginnote{2.1} Ganges river slants, slopes, and inclines to the east. In the same way, a mendicant who develops and cultivates the four right efforts slants, slopes, and inclines to extinguishment. 

And\marginnote{2.3} how does a mendicant who develops the four right efforts slant, slope, and incline to extinguishment? 

They\marginnote{2.4} generate enthusiasm, try, make an effort, exert the mind, and strive so that bad, unskillful qualities don’t arise. 

They\marginnote{2.5} generate enthusiasm, try, make an effort, exert the mind, and strive so that bad, unskillful qualities that have arisen are given up. 

They\marginnote{2.6} generate enthusiasm, try, make an effort, exert the mind, and strive so that skillful qualities arise. 

They\marginnote{2.7} generate enthusiasm, try, make an effort, exert the mind, and strive so that skillful qualities that have arisen remain, are not lost, but increase, mature, and are completed by development. 

That’s\marginnote{2.8} how a mendicant who develops and cultivates the four right efforts slants, slopes, and inclines to extinguishment.” 

(To\marginnote{3.1} be expanded as in SN 45.92–102.) 

\begin{quotation}%
Six\marginnote{4.1} on slanting to the east, \\
and six on slanting to the ocean; \\
these two sixes make twelve, \\
and that’s how this chapter is recited. 

%
\end{quotation}

%
\addtocontents{toc}{\let\protect\contentsline\protect\nopagecontentsline}
\chapter*{The Chapter on Diligence }
\addcontentsline{toc}{chapter}{\tocchapterline{The Chapter on Diligence }}
\addtocontents{toc}{\let\protect\contentsline\protect\oldcontentsline}

%
\section*{{\suttatitleacronym SN 49.13–22}{\suttatitletranslation Diligence }{\suttatitleroot Appamādavagga}}
\addcontentsline{toc}{section}{\tocacronym{SN 49.13–22} \toctranslation{Diligence } \tocroot{Appamādavagga}}
\markboth{Diligence }{Appamādavagga}
\extramarks{SN 49.13–22}{SN 49.13–22}

(To\marginnote{1.1} be expanded as in SN 45.139–148.) 

\begin{quotation}%
The\marginnote{2.1} Realized One, footprint, roof peak, \\
roots, heartwood, jasmine, \\
monarch, sun and moon, \\
and cloth is the tenth. 

%
\end{quotation}

%
\addtocontents{toc}{\let\protect\contentsline\protect\nopagecontentsline}
\chapter*{The Chapter on Hard Work }
\addcontentsline{toc}{chapter}{\tocchapterline{The Chapter on Hard Work }}
\addtocontents{toc}{\let\protect\contentsline\protect\oldcontentsline}

%
\section*{{\suttatitleacronym SN 49.23–34}{\suttatitletranslation Hard Work, Etc. }{\suttatitleroot Balakaraṇīyavagga}}
\addcontentsline{toc}{section}{\tocacronym{SN 49.23–34} \toctranslation{Hard Work, Etc. } \tocroot{Balakaraṇīyavagga}}
\markboth{Hard Work, Etc. }{Balakaraṇīyavagga}
\extramarks{SN 49.23–34}{SN 49.23–34}

“Mendicants,\marginnote{1.1} all the hard work that gets done depends on the earth and is grounded on the earth. In the same way, a mendicant develops and cultivates the four right efforts depending on and grounded on ethics. 

How\marginnote{1.3} so? It’s when a mendicant generates enthusiasm, tries, makes an effort, exerts the mind, and strives so that bad, unskillful qualities don’t arise. … so that skillful qualities that have arisen remain, are not lost, but increase, mature, and are completed by development. 

That’s\marginnote{1.6} how a mendicant develops and cultivates the four right efforts depending on and grounded on ethics.” 

(To\marginnote{1.7} be expanded as in SN 45.149–160.) 

\begin{quotation}%
Hard\marginnote{2.1} work, seeds, and dragons, \\
a tree, a pot, and a spike, \\
the sky, and two on clouds, \\
a ship, a guest house, and a river. 

%
\end{quotation}

%
\addtocontents{toc}{\let\protect\contentsline\protect\nopagecontentsline}
\chapter*{The Chapter on Searches }
\addcontentsline{toc}{chapter}{\tocchapterline{The Chapter on Searches }}
\addtocontents{toc}{\let\protect\contentsline\protect\oldcontentsline}

%
\section*{{\suttatitleacronym SN 49.35–44}{\suttatitletranslation Ten Discourses on Searches, Etc. }{\suttatitleroot Esanāvagga}}
\addcontentsline{toc}{section}{\tocacronym{SN 49.35–44} \toctranslation{Ten Discourses on Searches, Etc. } \tocroot{Esanāvagga}}
\markboth{Ten Discourses on Searches, Etc. }{Esanāvagga}
\extramarks{SN 49.35–44}{SN 49.35–44}

“Mendicants,\marginnote{1.1} there are these three searches. What three? The search for sensual pleasures, the search for continued existence, and the search for a spiritual path. These are the three searches. 

The\marginnote{1.5} four right efforts should be developed for the direct knowledge, complete understanding, finishing, and giving up of these three searches. What four? It’s when a mendicant generates enthusiasm, tries, makes an effort, exerts the mind, and strives so that bad, unskillful qualities don’t arise. … so that skillful qualities that have arisen remain, are not lost, but increase, mature, and are completed by development. 

These\marginnote{1.9} four right efforts should be developed for the direct knowledge, complete understanding, finishing, and giving up of these three searches.” 

(To\marginnote{1.10} be expanded as in SN 45.161–170.) 

\begin{quotation}%
Searches,\marginnote{2.1} discriminations, defilements, \\
states of existence, three kinds of suffering, \\
barrenness, stains, and troubles, \\
feelings, craving, and thirst. 

%
\end{quotation}

%
\addtocontents{toc}{\let\protect\contentsline\protect\nopagecontentsline}
\chapter*{The Chapter on Floods }
\addcontentsline{toc}{chapter}{\tocchapterline{The Chapter on Floods }}
\addtocontents{toc}{\let\protect\contentsline\protect\oldcontentsline}

%
\section*{{\suttatitleacronym SN 49.45–54}{\suttatitletranslation Floods, Etc. }{\suttatitleroot Oghavagga}}
\addcontentsline{toc}{section}{\tocacronym{SN 49.45–54} \toctranslation{Floods, Etc. } \tocroot{Oghavagga}}
\markboth{Floods, Etc. }{Oghavagga}
\extramarks{SN 49.45–54}{SN 49.45–54}

“Mendicants,\marginnote{1.1} there are five higher fetters. What five? Desire for rebirth in the realm of luminous form, desire for rebirth in the formless realm, conceit, restlessness, and ignorance. These are the five higher fetters. 

The\marginnote{1.5} four right efforts should be developed for the direct knowledge, complete understanding, finishing, and giving up of these five higher fetters. What four? It’s when a mendicant generates enthusiasm, tries, makes an effort, exerts the mind, and strives so that bad, unskillful qualities don’t arise. … so that skillful qualities that have arisen remain, are not lost, but increase, mature, and are completed by development. 

These\marginnote{1.9} four right efforts should be developed for the direct knowledge, complete understanding, finishing, and giving up of these five higher fetters.” 

(To\marginnote{1.10} be expanded as in SN 45.171–179, with the above as the final discourse.) 

\begin{quotation}%
Floods,\marginnote{2.1} bonds, grasping, \\
ties, and underlying tendencies, \\
kinds of sensual stimulation, hindrances, \\
aggregates, and fetters high and low. 

%
\end{quotation}

\scendsutta{The Linked Discourses on the Right Efforts is the fifth section. }

%
\addtocontents{toc}{\let\protect\contentsline\protect\nopagecontentsline}
\part*{Linked Discourses on the Five Powers }
\addcontentsline{toc}{part}{Linked Discourses on the Five Powers }
\markboth{}{}
\addtocontents{toc}{\let\protect\contentsline\protect\oldcontentsline}

%
\addtocontents{toc}{\let\protect\contentsline\protect\nopagecontentsline}
\chapter*{The Chapter of Abbreviated Texts on the Ganges }
\addcontentsline{toc}{chapter}{\tocchapterline{The Chapter of Abbreviated Texts on the Ganges }}
\addtocontents{toc}{\let\protect\contentsline\protect\oldcontentsline}

%
\section*{{\suttatitleacronym SN 50.1–12}{\suttatitletranslation Sloping East, Etc. }{\suttatitleroot Gaṅgāpeyyālavagga}}
\addcontentsline{toc}{section}{\tocacronym{SN 50.1–12} \toctranslation{Sloping East, Etc. } \tocroot{Gaṅgāpeyyālavagga}}
\markboth{Sloping East, Etc. }{Gaṅgāpeyyālavagga}
\extramarks{SN 50.1–12}{SN 50.1–12}

“Mendicants,\marginnote{1.1} there are these five powers. What five? The powers of faith, energy, mindfulness, immersion, and wisdom. These are the five powers. The Ganges river slants, slopes, and inclines to the east. In the same way, a mendicant who develops and cultivates the five powers slants, slopes, and inclines to extinguishment. 

And\marginnote{1.7} how does a mendicant who develops the five powers slant, slope, and incline to extinguishment? It’s when a mendicant develops the powers of faith, energy, mindfulness, immersion, and wisdom, which rely on seclusion, fading away, and cessation, and ripen as letting go. That’s how a mendicant who develops and cultivates the five powers slants, slopes, and inclines to extinguishment.” 

\scendsection{(To be expanded for each of the different rivers as in SN 45.91–102.) }

\begin{quotation}%
Six\marginnote{2.1} on slanting to the east, \\
and six on slanting to the ocean; \\
these two sixes make twelve, \\
and that’s how this chapter is recited. 

%
\end{quotation}

%
\addtocontents{toc}{\let\protect\contentsline\protect\nopagecontentsline}
\chapter*{The Chapter on Diligence }
\addcontentsline{toc}{chapter}{\tocchapterline{The Chapter on Diligence }}
\addtocontents{toc}{\let\protect\contentsline\protect\oldcontentsline}

%
\section*{{\suttatitleacronym SN 50.13–22}{\suttatitletranslation Diligence }{\suttatitleroot Appamādavagga}}
\addcontentsline{toc}{section}{\tocacronym{SN 50.13–22} \toctranslation{Diligence } \tocroot{Appamādavagga}}
\markboth{Diligence }{Appamādavagga}
\extramarks{SN 50.13–22}{SN 50.13–22}

(To\marginnote{1.1} be expanded as in the chapter on diligence at SN 45.139–148.) 

\begin{quotation}%
The\marginnote{2.1} Realized One, footprint, roof peak, \\
roots, heartwood, jasmine, \\
monarch, sun and moon, \\
and cloth is the tenth. 

%
\end{quotation}

%
\addtocontents{toc}{\let\protect\contentsline\protect\nopagecontentsline}
\chapter*{The Chapter on Hard Work }
\addcontentsline{toc}{chapter}{\tocchapterline{The Chapter on Hard Work }}
\addtocontents{toc}{\let\protect\contentsline\protect\oldcontentsline}

%
\section*{{\suttatitleacronym SN 50.23–34}{\suttatitletranslation Hard Work }{\suttatitleroot Balakaraṇīyavagga}}
\addcontentsline{toc}{section}{\tocacronym{SN 50.23–34} \toctranslation{Hard Work } \tocroot{Balakaraṇīyavagga}}
\markboth{Hard Work }{Balakaraṇīyavagga}
\extramarks{SN 50.23–34}{SN 50.23–34}

(To\marginnote{1.1} be expanded as in the chapter on hard work at SN 45.149–160.) 

\begin{quotation}%
Hard\marginnote{2.1} work, seeds, and dragons, \\
a tree, a pot, and a spike, \\
the sky, and two on clouds, \\
a ship, a guest house, and a river. 

%
\end{quotation}

%
\addtocontents{toc}{\let\protect\contentsline\protect\nopagecontentsline}
\chapter*{The Chapter on Searches }
\addcontentsline{toc}{chapter}{\tocchapterline{The Chapter on Searches }}
\addtocontents{toc}{\let\protect\contentsline\protect\oldcontentsline}

%
\section*{{\suttatitleacronym SN 50.35–44}{\suttatitletranslation Searches }{\suttatitleroot Esanāvagga}}
\addcontentsline{toc}{section}{\tocacronym{SN 50.35–44} \toctranslation{Searches } \tocroot{Esanāvagga}}
\markboth{Searches }{Esanāvagga}
\extramarks{SN 50.35–44}{SN 50.35–44}

(To\marginnote{1.1} be expanded as in the chapter on searches at SN 45.161–170.) 

\begin{quotation}%
Searches,\marginnote{2.1} discriminations, defilements, \\
states of existence, three kinds of suffering, \\
barrenness, stains, and troubles, \\
feelings, craving, and thirst. 

%
\end{quotation}

%
\addtocontents{toc}{\let\protect\contentsline\protect\nopagecontentsline}
\chapter*{The Chapter on Floods }
\addcontentsline{toc}{chapter}{\tocchapterline{The Chapter on Floods }}
\addtocontents{toc}{\let\protect\contentsline\protect\oldcontentsline}

%
\section*{{\suttatitleacronym SN 50.45–54}{\suttatitletranslation Floods, Etc. }{\suttatitleroot Oghavagga}}
\addcontentsline{toc}{section}{\tocacronym{SN 50.45–54} \toctranslation{Floods, Etc. } \tocroot{Oghavagga}}
\markboth{Floods, Etc. }{Oghavagga}
\extramarks{SN 50.45–54}{SN 50.45–54}

“Mendicants,\marginnote{1.1} there are five higher fetters. What five? Desire for rebirth in the realm of luminous form, desire for rebirth in the formless realm, conceit, restlessness, and ignorance. These are the five higher fetters. 

The\marginnote{1.5} five powers should be developed for the direct knowledge, complete understanding, finishing, and giving up of these five higher fetters. What five? It’s when a mendicant develops the powers of faith, energy, mindfulness, immersion, and wisdom, which rely on seclusion, fading away, and cessation, and ripen as letting go. 

These\marginnote{1.12} five powers should be developed for the direct knowledge, complete understanding, finishing, and giving up of these five higher fetters.” 

(To\marginnote{1.13} be expanded as in SN 45.171–179, with the above as the final discourse.) 

%
\addtocontents{toc}{\let\protect\contentsline\protect\nopagecontentsline}
\chapter*{Another Chapter of Abbreviated Texts on the Ganges, Etc. }
\addcontentsline{toc}{chapter}{\tocchapterline{Another Chapter of Abbreviated Texts on the Ganges, Etc. }}
\addtocontents{toc}{\let\protect\contentsline\protect\oldcontentsline}

%
\section*{{\suttatitleacronym SN 50.55–66}{\suttatitletranslation Sloping East, Etc. }{\suttatitleroot Punagaṅgāpeyyālavagga}}
\addcontentsline{toc}{section}{\tocacronym{SN 50.55–66} \toctranslation{Sloping East, Etc. } \tocroot{Punagaṅgāpeyyālavagga}}
\markboth{Sloping East, Etc. }{Punagaṅgāpeyyālavagga}
\extramarks{SN 50.55–66}{SN 50.55–66}

“Mendicants,\marginnote{1.1} the Ganges river slants, slopes, and inclines to the east. In the same way, a mendicant who develops and cultivates the five powers slants, slopes, and inclines to extinguishment. 

And\marginnote{1.3} how does a mendicant who develops the five powers slant, slope, and incline to extinguishment? It’s when a mendicant develops the powers of faith, energy, mindfulness, immersion, and wisdom, which culminate in the removal of greed, hate, and delusion. 

That’s\marginnote{1.5} how a mendicant who develops and cultivates the five powers slants, slopes, and inclines to extinguishment.” 

(To\marginnote{1.6} be expanded for each of the different rivers as in SN 45.91–102, removal of greed version.) 

\begin{quotation}%
Six\marginnote{2.1} on slanting to the east, \\
and six on slanting to the ocean; \\
these two sixes make twelve, \\
and that’s how this chapter is recited. 

%
\end{quotation}

%
\addtocontents{toc}{\let\protect\contentsline\protect\nopagecontentsline}
\chapter*{Another Chapter on Diligence }
\addcontentsline{toc}{chapter}{\tocchapterline{Another Chapter on Diligence }}
\addtocontents{toc}{\let\protect\contentsline\protect\oldcontentsline}

%
\section*{{\suttatitleacronym SN 50.67–76}{\suttatitletranslation Another Chapter on Diligence }{\suttatitleroot Punaappamādavagga}}
\addcontentsline{toc}{section}{\tocacronym{SN 50.67–76} \toctranslation{Another Chapter on Diligence } \tocroot{Punaappamādavagga}}
\markboth{Another Chapter on Diligence }{Punaappamādavagga}
\extramarks{SN 50.67–76}{SN 50.67–76}

(This\marginnote{1.1} text consists of the title only. To be expanded as in SN 45.139–148, removal of greed version.) 

%
\addtocontents{toc}{\let\protect\contentsline\protect\nopagecontentsline}
\chapter*{Another Chapter on Hard Work }
\addcontentsline{toc}{chapter}{\tocchapterline{Another Chapter on Hard Work }}
\addtocontents{toc}{\let\protect\contentsline\protect\oldcontentsline}

%
\section*{{\suttatitleacronym SN 50.77–88}{\suttatitletranslation Another Chapter on Hard Work }{\suttatitleroot Punabalakaraṇīyavagga}}
\addcontentsline{toc}{section}{\tocacronym{SN 50.77–88} \toctranslation{Another Chapter on Hard Work } \tocroot{Punabalakaraṇīyavagga}}
\markboth{Another Chapter on Hard Work }{Punabalakaraṇīyavagga}
\extramarks{SN 50.77–88}{SN 50.77–88}

(To\marginnote{1.1} be expanded as in SN 45.149–160, removal of greed version.) 

%
\addtocontents{toc}{\let\protect\contentsline\protect\nopagecontentsline}
\chapter*{Another Chapter on Searches }
\addcontentsline{toc}{chapter}{\tocchapterline{Another Chapter on Searches }}
\addtocontents{toc}{\let\protect\contentsline\protect\oldcontentsline}

%
\section*{{\suttatitleacronym SN 50.89–98}{\suttatitletranslation Another Series on Searches, Etc. }{\suttatitleroot Punaesanāvagga}}
\addcontentsline{toc}{section}{\tocacronym{SN 50.89–98} \toctranslation{Another Series on Searches, Etc. } \tocroot{Punaesanāvagga}}
\markboth{Another Series on Searches, Etc. }{Punaesanāvagga}
\extramarks{SN 50.89–98}{SN 50.89–98}

(To\marginnote{1.1} be expanded as in SN 45.161–170, removal of greed version.) 

\begin{quotation}%
Searches,\marginnote{2.1} discriminations, defilements, \\
states of existence, three kinds of suffering, \\
barrenness, stains, and troubles, \\
feelings, craving, and thirst. 

%
\end{quotation}

%
\addtocontents{toc}{\let\protect\contentsline\protect\nopagecontentsline}
\chapter*{Another Chapter on Floods }
\addcontentsline{toc}{chapter}{\tocchapterline{Another Chapter on Floods }}
\addtocontents{toc}{\let\protect\contentsline\protect\oldcontentsline}

%
\section*{{\suttatitleacronym SN 50.99–108}{\suttatitletranslation Another Series on Floods, Etc. }{\suttatitleroot Punaoghavagga}}
\addcontentsline{toc}{section}{\tocacronym{SN 50.99–108} \toctranslation{Another Series on Floods, Etc. } \tocroot{Punaoghavagga}}
\markboth{Another Series on Floods, Etc. }{Punaoghavagga}
\extramarks{SN 50.99–108}{SN 50.99–108}

“Mendicants,\marginnote{1.1} there are five higher fetters. What five? Desire for rebirth in the realm of luminous form, desire for rebirth in the formless realm, conceit, restlessness, and ignorance. These are the five higher fetters. 

The\marginnote{1.5} five powers should be developed for the direct knowledge, complete understanding, finishing, and giving up of these five higher fetters. What five? A mendicant develops the powers of faith, energy, mindfulness, immersion, and wisdom, which culminate in the removal of greed, hate, and delusion. 

These\marginnote{1.8} five powers should be developed for the direct knowledge, complete understanding, finishing, and giving up of these five higher fetters.” 

\scendsection{(To be expanded as in SN 45.171–179, with the above as the final discourse.) }

\begin{quotation}%
Floods,\marginnote{2.1} bonds, grasping, \\
ties, and underlying tendencies, \\
kinds of sensual stimulation, hindrances, \\
aggregates, and fetters high and low. 

%
\end{quotation}

\scendsutta{The Linked Discourses on the Powers is the sixth section. }

%
\addtocontents{toc}{\let\protect\contentsline\protect\nopagecontentsline}
\part*{Linked Discourses on the Bases of Psychic Power }
\addcontentsline{toc}{part}{Linked Discourses on the Bases of Psychic Power }
\markboth{}{}
\addtocontents{toc}{\let\protect\contentsline\protect\oldcontentsline}

%
\addtocontents{toc}{\let\protect\contentsline\protect\nopagecontentsline}
\chapter*{The Chapter at the Cāpāla Shrine }
\addcontentsline{toc}{chapter}{\tocchapterline{The Chapter at the Cāpāla Shrine }}
\addtocontents{toc}{\let\protect\contentsline\protect\oldcontentsline}

%
\section*{{\suttatitleacronym SN 51.1}{\suttatitletranslation From the Near Shore }{\suttatitleroot Apārasutta}}
\addcontentsline{toc}{section}{\tocacronym{SN 51.1} \toctranslation{From the Near Shore } \tocroot{Apārasutta}}
\markboth{From the Near Shore }{Apārasutta}
\extramarks{SN 51.1}{SN 51.1}

“Mendicants,\marginnote{1.1} when these four bases of psychic power are developed and cultivated they lead to going from the near shore to the far shore. What four? It’s when a mendicant develops the basis of psychic power that has immersion due to enthusiasm, and active effort. They develop the basis of psychic power that has immersion due to energy, and active effort. They develop the basis of psychic power that has immersion due to mental development, and active effort. They develop the basis of psychic power that has immersion due to inquiry, and active effort. When these four bases of psychic power are developed and cultivated they lead to going from the near shore to the far shore.” 

%
\section*{{\suttatitleacronym SN 51.2}{\suttatitletranslation Missed Out }{\suttatitleroot Viraddhasutta}}
\addcontentsline{toc}{section}{\tocacronym{SN 51.2} \toctranslation{Missed Out } \tocroot{Viraddhasutta}}
\markboth{Missed Out }{Viraddhasutta}
\extramarks{SN 51.2}{SN 51.2}

“Mendicants,\marginnote{1.1} whoever has missed out on the four bases of psychic power has missed out on the noble path to the complete ending of suffering. Whoever has undertaken the four bases of psychic power has undertaken the noble path to the complete ending of suffering. What four? It’s when a mendicant develops the basis of psychic power that has immersion due to enthusiasm … energy … mental development … inquiry, and active effort. Whoever has missed out on these four bases of psychic power has missed out on the noble path to the complete ending of suffering. Whoever has undertaken these four bases of psychic power has undertaken the noble path to the complete ending of suffering.” 

%
\section*{{\suttatitleacronym SN 51.3}{\suttatitletranslation A Noble One }{\suttatitleroot Ariyasutta}}
\addcontentsline{toc}{section}{\tocacronym{SN 51.3} \toctranslation{A Noble One } \tocroot{Ariyasutta}}
\markboth{A Noble One }{Ariyasutta}
\extramarks{SN 51.3}{SN 51.3}

“Mendicants,\marginnote{1.1} when these four bases of psychic power are developed and cultivated they are noble and emancipating, and bring one who practices them to the complete ending of suffering. What four? It’s when a mendicant develops the basis of psychic power that has immersion due to enthusiasm … energy … mental development … inquiry, and active effort. When these four bases of psychic power are developed and cultivated they are noble and emancipating, and bring one who practices them to the complete ending of suffering.” 

%
\section*{{\suttatitleacronym SN 51.4}{\suttatitletranslation Disillusionment }{\suttatitleroot Nibbidāsutta}}
\addcontentsline{toc}{section}{\tocacronym{SN 51.4} \toctranslation{Disillusionment } \tocroot{Nibbidāsutta}}
\markboth{Disillusionment }{Nibbidāsutta}
\extramarks{SN 51.4}{SN 51.4}

“Mendicants,\marginnote{1.1} these four bases of psychic power, when developed and cultivated, lead solely to disillusionment, dispassion, cessation, peace, insight, awakening, and extinguishment. What four? It’s when a mendicant develops the basis of psychic power that has immersion due to enthusiasm … energy … mental development … inquiry, and active effort. These four bases of psychic power, when developed and cultivated, lead solely to disillusionment, dispassion, cessation, peace, insight, awakening, and extinguishment.” 

%
\section*{{\suttatitleacronym SN 51.5}{\suttatitletranslation Partly }{\suttatitleroot Iddhipadesasutta}}
\addcontentsline{toc}{section}{\tocacronym{SN 51.5} \toctranslation{Partly } \tocroot{Iddhipadesasutta}}
\markboth{Partly }{Iddhipadesasutta}
\extramarks{SN 51.5}{SN 51.5}

“Mendicants,\marginnote{1.1} all the ascetics and brahmins in the past who have partly manifested psychic powers have done so by developing and cultivating the four bases of psychic power. All the ascetics and brahmins in the future who will partly manifest psychic powers will do so by developing and cultivating the four bases of psychic power. All the ascetics and brahmins in the present who are partly manifesting psychic powers do so by developing and cultivating the four bases of psychic power. 

What\marginnote{2.1} four? It’s when a mendicant develops the basis of psychic power that has immersion due to enthusiasm … energy … mental development … inquiry, and active effort. All the ascetics and brahmins in the past who have partly manifested psychic powers have done so by developing and cultivating these four bases of psychic power. All the ascetics and brahmins in the future who will partly manifest psychic powers will do so by developing and cultivating these four bases of psychic power. All the ascetics and brahmins in the present who are partly manifesting psychic powers do so by developing and cultivating these four bases of psychic power.” 

%
\section*{{\suttatitleacronym SN 51.6}{\suttatitletranslation Completely }{\suttatitleroot Samattasutta}}
\addcontentsline{toc}{section}{\tocacronym{SN 51.6} \toctranslation{Completely } \tocroot{Samattasutta}}
\markboth{Completely }{Samattasutta}
\extramarks{SN 51.6}{SN 51.6}

“Mendicants,\marginnote{1.1} all the ascetics and brahmins in the past who have completely manifested psychic powers have done so by developing and cultivating the four bases of psychic power. All the ascetics and brahmins in the future who will completely manifest psychic powers will do so by developing and cultivating the four bases of psychic power. All the ascetics and brahmins in the present who are completely manifesting psychic powers do so by developing and cultivating the four bases of psychic power. 

What\marginnote{2.1} four? It’s when a mendicant develops the basis of psychic power that has immersion due to enthusiasm … energy … mental development … inquiry, and active effort. All the ascetics and brahmins in the past who have completely manifested psychic powers have done so by developing and cultivating these four bases of psychic power. All the ascetics and brahmins in the future who will completely manifest psychic powers will do so by developing and cultivating these four bases of psychic power. All the ascetics and brahmins in the present who are completely manifesting psychic powers do so by developing and cultivating these four bases of psychic power.” 

%
\section*{{\suttatitleacronym SN 51.7}{\suttatitletranslation A Mendicant }{\suttatitleroot Bhikkhusutta}}
\addcontentsline{toc}{section}{\tocacronym{SN 51.7} \toctranslation{A Mendicant } \tocroot{Bhikkhusutta}}
\markboth{A Mendicant }{Bhikkhusutta}
\extramarks{SN 51.7}{SN 51.7}

“Mendicants,\marginnote{1.1} all the mendicants in the past … future … present who realize the undefiled freedom of heart and freedom by wisdom in this very life, and who live having realized it with their own insight due to the ending of defilements, do so by developing and cultivating the four bases of psychic power. 

What\marginnote{2.1} four? It’s when a mendicant develops the basis of psychic power that has immersion due to enthusiasm … energy … mental development … inquiry, and active effort. All the mendicants in the past … future … present who realize the undefiled freedom of heart and freedom by wisdom in this very life, and who live having realized it with their own insight due to the ending of defilements, do so by developing and cultivating these four bases of psychic power.” 

%
\section*{{\suttatitleacronym SN 51.8}{\suttatitletranslation Awakened }{\suttatitleroot Buddhasutta}}
\addcontentsline{toc}{section}{\tocacronym{SN 51.8} \toctranslation{Awakened } \tocroot{Buddhasutta}}
\markboth{Awakened }{Buddhasutta}
\extramarks{SN 51.8}{SN 51.8}

“Mendicants,\marginnote{1.1} there are these four bases of psychic power. What four? It’s when a mendicant develops the basis of psychic power that has immersion due to enthusiasm … energy … mental development … inquiry, and active effort. These are the four bases of psychic power. It is because he has developed and cultivated these four bases of psychic power that the Realized One is called ‘the perfected one, the fully awakened Buddha’.” 

%
\section*{{\suttatitleacronym SN 51.9}{\suttatitletranslation Knowledge }{\suttatitleroot Ñāṇasutta}}
\addcontentsline{toc}{section}{\tocacronym{SN 51.9} \toctranslation{Knowledge } \tocroot{Ñāṇasutta}}
\markboth{Knowledge }{Ñāṇasutta}
\extramarks{SN 51.9}{SN 51.9}

“Mendicants:\marginnote{1.1} ‘This is the basis of psychic power that has immersion due to enthusiasm, and active effort.’ Such was the vision, knowledge, wisdom, realization, and light that arose in me regarding teachings not learned before from another. ‘This basis of psychic power … should be developed.’ … ‘This basis of psychic power … has been developed.’ Such was the vision, knowledge, wisdom, realization, and light that arose in me regarding teachings not learned before from another. 

‘This\marginnote{2.1} is the basis of psychic power that has immersion due to energy, and active effort.’ … ‘This basis of psychic power … should be developed.’ … ‘This basis of psychic power … has been developed.’ Such was the vision, knowledge, wisdom, realization, and light that arose in me regarding teachings not learned before from another. 

‘This\marginnote{3.1} is the basis of psychic power that has immersion due to mental development, and active effort.’ … ‘This basis of psychic power … should be developed.’ … ‘This basis of psychic power … has been developed.’ Such was the vision, knowledge, wisdom, realization, and light that arose in me regarding teachings not learned before from another. 

‘This\marginnote{4.1} is the basis of psychic power that has immersion due to inquiry, and active effort.’ … ‘This basis of psychic power … should be developed.’ … ‘This basis of psychic power … has been developed.’ Such was the vision, knowledge, wisdom, realization, and light that arose in me regarding teachings not learned before from another.” 

%
\section*{{\suttatitleacronym SN 51.10}{\suttatitletranslation At the Cāpāla Shrine }{\suttatitleroot Cetiyasutta}}
\addcontentsline{toc}{section}{\tocacronym{SN 51.10} \toctranslation{At the Cāpāla Shrine } \tocroot{Cetiyasutta}}
\markboth{At the Cāpāla Shrine }{Cetiyasutta}
\extramarks{SN 51.10}{SN 51.10}

\scevam{So\marginnote{1.1} I have heard. }At one time the Buddha was staying near \textsanskrit{Vesālī}, at the Great Wood, in the hall with the peaked roof. Then the Buddha robed up in the morning and, taking his bowl and robe, entered \textsanskrit{Vesālī} for alms. Then, after the meal, on his return from almsround, he addressed Venerable Ānanda: “Ānanda, get your sitting cloth. Let’s go to the \textsanskrit{Cāpāla} shrine for the day’s meditation.” 

“Yes,\marginnote{1.7} sir,” replied Ānanda. Taking his sitting cloth he followed behind the Buddha. 

Then\marginnote{2.1} the Buddha went up to the \textsanskrit{Cāpāla} shrine, and sat down on the seat spread out. Ānanda bowed to the Buddha and sat down to one side. The Buddha said to him: 

“Ānanda,\marginnote{3.1} \textsanskrit{Vesālī} is lovely. And the Udena, Gotamaka, Sattamba, Bahuputta, \textsanskrit{Sārandada}, and \textsanskrit{Cāpāla} Tree-shrines are all lovely. Whoever has developed and cultivated the four bases of psychic power—made them a vehicle and a basis, kept them up, consolidated them, and properly implemented them—may, if they wish, live on for the eon or what’s left of the eon. The Realized One has developed and cultivated the four bases of psychic power, made them a vehicle and a basis, kept them up, consolidated them, and properly implemented them. If he wished, the Realized One could live on for the eon or what’s left of the eon.” 

But\marginnote{4.1} Ānanda didn’t get it, even though the Buddha dropped such an obvious hint, such a clear sign. He didn’t beg the Buddha, “Sir, may the Blessed One please remain for the eon! May the Holy One please remain for the eon! That would be for the welfare and happiness of the people, out of compassion for the world, for the benefit, welfare, and happiness of gods and humans.” For his mind was as if possessed by \textsanskrit{Māra}. 

For\marginnote{5.1} a second time … and for a third time, the Buddha said to Ānanda: 

“Ānanda,\marginnote{5.3} \textsanskrit{Vesālī} is lovely. And the Udena, Gotamaka, Sattamba, Bahuputta, \textsanskrit{Sārandada}, and \textsanskrit{Cāpāla} Tree-shrines are all lovely. Whoever has developed and cultivated the four bases of psychic power—made them a vehicle and a basis, kept them up, consolidated them, and properly implemented them—may, if they wish, live on for the eon or what’s left of the eon. The Realized One has developed and cultivated the four bases of psychic power, made them a vehicle and a basis, kept them up, consolidated them, and properly implemented them. If he wished, the Realized One could live on for the eon or what’s left of the eon.” 

But\marginnote{6.1} Ānanda didn’t get it, even though the Buddha dropped such an obvious hint, such a clear sign. He didn’t beg the Buddha, “Sir, may the Blessed One please remain for the eon! May the Holy One please remain for the eon! That would be for the welfare and happiness of the people, out of compassion for the world, for the benefit, welfare, and happiness of gods and humans.” For his mind was as if possessed by \textsanskrit{Māra}. 

Then\marginnote{7.1} the Buddha said to him, “Go now, Ānanda, at your convenience.” 

“Yes,\marginnote{7.4} sir,” replied Ānanda. He rose from his seat, bowed, and respectfully circled the Buddha, keeping him on his right, before sitting at the root of a tree close by. 

And\marginnote{8.1} then, not long after Ānanda had left, \textsanskrit{Māra} the Wicked went up to the the Buddha and said to him: 

“Sir,\marginnote{9.1} may the Blessed One now become fully extinguished! May the Holy One now become fully extinguished! Now is the time for the Buddha to become fully extinguished. Sir, you once made this statement: ‘Wicked One, I will not become fully extinguished until I have monk disciples who are competent, educated, assured, learned, have memorized the teachings, and practice in line with the teachings. Not until they practice properly, living in line with the teaching. Not until they’ve learned their own tradition, and explain, teach, assert, establish, disclose, analyze, and make it clear. Not until they can legitimately and completely refute the doctrines of others that come up, and teach with a demonstrable basis.’ 

Today\marginnote{10.1} you do have such monk disciples. May the Blessed One now become fully extinguished! May the Holy One now become fully extinguished! Now is the time for the Buddha to become fully extinguished. 

Sir,\marginnote{11.1} you once made this statement: ‘Wicked One, I will not become fully extinguished until I have nun disciples who are competent, educated, assured, learned …’ … 

Today\marginnote{12.1} you do have such nun disciples. May the Blessed One now become fully extinguished! May the Holy One now become fully extinguished! Now is the time for the Buddha to become fully extinguished. 

Sir,\marginnote{13.1} you once made this statement: ‘Wicked One, I will not become fully extinguished until I have layman disciples … and laywoman disciples who are competent, educated, assured, learned …’ … 

Today\marginnote{14.1} you do have such layman and laywoman disciples. May the Blessed One now become fully extinguished! May the Holy One now become fully extinguished! Now is the time for the Buddha to become fully extinguished. 

Sir,\marginnote{15.1} you once made this statement: ‘Wicked One, I will not become fully extinguished until my spiritual path is successful and prosperous, extensive, popular, widespread, and well proclaimed wherever there are gods and humans.’ Today your spiritual path is successful and prosperous, extensive, popular, widespread, and well proclaimed wherever there are gods and humans. May the Blessed One now become fully extinguished! May the Holy One now become fully extinguished! Now is the time for the Buddha to become fully extinguished.” 

When\marginnote{16.1} this was said, the Buddha said to \textsanskrit{Māra}, “Relax, Wicked One. The final extinguishment of the Realized One will be soon. Three months from now the Realized One will finally be extinguished.” 

So\marginnote{17.1} at the \textsanskrit{Cāpāla} Tree-shrine the Buddha, mindful and aware, surrendered the life force. When he did so there was a great earthquake, awe-inspiring and hair-raising, and thunder cracked the sky. Then, understanding this matter, on that occasion the Buddha expressed this heartfelt sentiment: 

\begin{verse}%
“Weighing\marginnote{18.1} up the incomparable against an extension of life, \\
the sage surrendered the life force. \\
Happy inside, serene, \\
he burst out of this self-made chain like a suit of armor.” 

%
\end{verse}

%
\addtocontents{toc}{\let\protect\contentsline\protect\nopagecontentsline}
\chapter*{The Chapter on Shaking the Stilt Longhouse }
\addcontentsline{toc}{chapter}{\tocchapterline{The Chapter on Shaking the Stilt Longhouse }}
\addtocontents{toc}{\let\protect\contentsline\protect\oldcontentsline}

%
\section*{{\suttatitleacronym SN 51.11}{\suttatitletranslation Before }{\suttatitleroot Pubbasutta}}
\addcontentsline{toc}{section}{\tocacronym{SN 51.11} \toctranslation{Before } \tocroot{Pubbasutta}}
\markboth{Before }{Pubbasutta}
\extramarks{SN 51.11}{SN 51.11}

At\marginnote{1.1} \textsanskrit{Sāvatthī}. 

“Mendicants,\marginnote{1.2} before my awakening—when I was still unawakened but intent on awakening—I thought: ‘What’s the cause, what’s the reason for the development of the bases of psychic power?’ Then it occurred to me: ‘It’s when a mendicant develops the basis of psychic power that has immersion due to enthusiasm, and active effort. They think: “My enthusiasm won’t be too lax or too tense. And it’ll be neither constricted internally nor scattered externally.” And they meditate perceiving continuity: as before, so after; as after, so before; as below, so above; as above, so below; as by day, so by night; as by night, so by day. And so, with an open and unenveloped heart, they develop a mind that’s full of radiance. 

They\marginnote{2.1} develop the basis of psychic power that has immersion due to energy … mental development … inquiry, and active effort. They think: “My inquiry won’t be too lax or too tense. And it’ll be neither constricted internally nor scattered externally.” And they meditate perceiving continuity: as before, so after; as after, so before; as below, so above; as above, so below; as by day, so by night; as by night, so by day. And so, with an open and unenveloped heart, they develop a mind that’s full of radiance.’ 

When\marginnote{5.1} the four bases of psychic power have been developed and cultivated in this way, they wield the many kinds of psychic power: multiplying themselves and becoming one again; appearing and disappearing; going unimpeded through a wall, a rampart, or a mountain as if through space; diving in and out of the earth as if it were water; walking on water as if it were earth; flying cross-legged through the sky like a bird; touching and stroking with the hand the sun and moon, so mighty and powerful; controlling the body as far as the \textsanskrit{Brahmā} realm. 

When\marginnote{6.1} the four bases of psychic power have been developed and cultivated in this way, they hear both kinds of sounds, human and divine, whether near or far. 

When\marginnote{7.1} the four bases of psychic power have been developed and cultivated in this way, they understand the minds of other beings and individuals, having comprehended them with their own mind. They understand mind with greed as ‘mind with greed’, and mind without greed as ‘mind without greed’. They understand mind with hate … mind without hate … mind with delusion … mind without delusion … constricted mind … scattered mind … expansive mind … unexpansive mind … mind that is not supreme … mind that is supreme … mind immersed in \textsanskrit{samādhi} … mind not immersed in \textsanskrit{samādhi} … freed mind … They understand unfreed mind as ‘unfreed mind’. 

When\marginnote{8.1} the four bases of psychic power have been developed and cultivated in this way, they recollect many kinds of past lives. That is: one, two, three, four, five, ten, twenty, thirty, forty, fifty, a hundred, a thousand, a hundred thousand rebirths; many eons of the world contracting, many eons of the world expanding, many eons of the world contracting and expanding. They remember: ‘There, I was named this, my clan was that, I looked like this, and that was my food. This was how I felt pleasure and pain, and that was how my life ended. When I passed away from that place I was reborn somewhere else. There, too, I was named this, my clan was that, I looked like this, and that was my food. This was how I felt pleasure and pain, and that was how my life ended. When I passed away from that place I was reborn here.’ And so they recollect their many kinds of past lives, with features and details. 

When\marginnote{9.1} the four bases of psychic power have been developed and cultivated in this way, with clairvoyance that is purified and superhuman, they see sentient beings passing away and being reborn—inferior and superior, beautiful and ugly, in a good place or a bad place. They understand how sentient beings are reborn according to their deeds. ‘These dear beings did bad things by way of body, speech, and mind. They spoke ill of the noble ones; they had wrong view; and they chose to act out of that wrong view. When their body breaks up, after death, they’re reborn in a place of loss, a bad place, the underworld, hell. These dear beings, however, did good things by way of body, speech, and mind. They never spoke ill of the noble ones; they had right view; and they chose to act out of that right view. When their body breaks up, after death, they’re reborn in a good place, a heavenly realm.’ And so, with clairvoyance that is purified and superhuman, they see sentient beings passing away and being reborn—inferior and superior, beautiful and ugly, in a good place or a bad place. They understand how sentient beings are reborn according to their deeds. 

When\marginnote{10.1} the four bases of psychic power have been developed and cultivated in this way, they realize the undefiled freedom of heart and freedom by wisdom in this very life. And they live having realized it with their own insight due to the ending of defilements.” 

%
\section*{{\suttatitleacronym SN 51.12}{\suttatitletranslation Very Fruitful }{\suttatitleroot Mahapphalasutta}}
\addcontentsline{toc}{section}{\tocacronym{SN 51.12} \toctranslation{Very Fruitful } \tocroot{Mahapphalasutta}}
\markboth{Very Fruitful }{Mahapphalasutta}
\extramarks{SN 51.12}{SN 51.12}

“Mendicants,\marginnote{1.1} when the four bases of psychic power are developed and cultivated they’re very fruitful and beneficial. How so? It’s when a mendicant develops the basis of psychic power that has immersion due to enthusiasm, and active effort. They think: ‘My enthusiasm won’t be too lax or too tense. And it’ll be neither constricted internally nor scattered externally.’ And they meditate perceiving continuity: as before, so after; as after, so before; as below, so above; as above, so below; as by day, so by night; as by night, so by day. And so, with an open and unenveloped heart, they develop a mind that’s full of radiance. 

They\marginnote{2.1} develop the basis of psychic power that has immersion due to energy … mental development … inquiry, and active effort. They think: ‘My inquiry won’t be too lax or too tense. And it’ll be neither constricted internally nor scattered externally.’ And they meditate perceiving continuity: as before, so after; as after, so before; as below, so above; as above, so below; as by day, so by night; as by night, so by day. And so, with an open and unenveloped heart, they develop a mind that’s full of radiance. When the four bases of psychic power have been developed and cultivated in this way they’re very fruitful and beneficial. 

When\marginnote{3.1} the four bases of psychic power have been developed and cultivated in this way, a mendicant wields the many kinds of psychic power: multiplying themselves and becoming one again … controlling the body as far as the \textsanskrit{Brahmā} realm. … 

When\marginnote{4.1} the four bases of psychic power have been developed and cultivated in this way, they realize the undefiled freedom of heart and freedom by wisdom in this very life. And they live having realized it with their own insight due to the ending of defilements.” 

%
\section*{{\suttatitleacronym SN 51.13}{\suttatitletranslation Immersion Due to Enthusiasm }{\suttatitleroot Chandasamādhisutta}}
\addcontentsline{toc}{section}{\tocacronym{SN 51.13} \toctranslation{Immersion Due to Enthusiasm } \tocroot{Chandasamādhisutta}}
\markboth{Immersion Due to Enthusiasm }{Chandasamādhisutta}
\extramarks{SN 51.13}{SN 51.13}

“Mendicants,\marginnote{1.1} if a mendicant depends on enthusiasm in order to gain immersion, gain unification of mind, this is called immersion due to enthusiasm. They generate enthusiasm, try, make an effort, exert the mind, and strive so that bad, unskillful qualities don’t arise. They generate enthusiasm, try, make an effort, exert the mind, and strive so that bad, unskillful qualities that have arisen are given up. They generate enthusiasm, try, make an effort, exert the mind, and strive so that skillful qualities arise. They generate enthusiasm, try, make an effort, exert the mind, and strive so that skillful qualities that have arisen remain, are not lost, but increase, mature, and are fulfilled by development. These are called active efforts. And so there is this enthusiasm, this immersion due to enthusiasm, and these active efforts. This is called the basis of psychic power that has immersion due to enthusiasm, and active effort. 

If\marginnote{2.1} a mendicant depends on energy in order to gain immersion, gain unification of mind, this is called immersion due to energy. They generate enthusiasm, try, make an effort, exert the mind, and strive so that bad, unskillful qualities don’t arise … so that skillful qualities that have arisen remain, are not lost, but increase, mature, and are fulfilled by development. These are called active efforts. And so there is this energy, this immersion due to energy, and these active efforts. This is called the basis of psychic power that has immersion due to energy, and active effort. 

If\marginnote{3.1} a mendicant depends on mental development in order to gain immersion, gain unification of mind, this is called immersion due to mental development. They generate enthusiasm, try, make an effort, exert the mind, and strive so that bad, unskillful qualities don’t arise … so that skillful qualities that have arisen remain, are not lost, but increase, mature, and are fulfilled by development. These are called active efforts. And so there is this mental development, this immersion due to mental development, and these active efforts. This is called the basis of psychic power that has immersion due to mental development, and active effort. 

If\marginnote{4.1} a mendicant depends on inquiry in order to gain immersion, gain unification of mind, this is called immersion due to inquiry. They generate enthusiasm, try, make an effort, exert the mind, and strive so that bad, unskillful qualities don’t arise … so that skillful qualities that have arisen remain, are not lost, but increase, mature, and are fulfilled by development. These are called active efforts. And so there is this inquiry, this immersion due to inquiry, and these active efforts. This is called the basis of psychic power that has immersion due to inquiry, and active effort.” 

%
\section*{{\suttatitleacronym SN 51.14}{\suttatitletranslation With Moggallāna }{\suttatitleroot Moggallānasutta}}
\addcontentsline{toc}{section}{\tocacronym{SN 51.14} \toctranslation{With Moggallāna } \tocroot{Moggallānasutta}}
\markboth{With Moggallāna }{Moggallānasutta}
\extramarks{SN 51.14}{SN 51.14}

\scevam{So\marginnote{1.1} I have heard. }At one time the Buddha was staying near \textsanskrit{Sāvatthī} in the Eastern Monastery, the stilt longhouse of \textsanskrit{Migāra}’s mother. Now at that time several mendicants were staying beneath the longhouse. They were restless, insolent, fickle, scurrilous, loose-tongued, unmindful, lacking situational awareness and immersion, with straying minds and undisciplined faculties. 

Then\marginnote{2.1} the Buddha addressed Venerable \textsanskrit{Mahāmoggallāna}, “These spiritual companions of yours staying beneath the longhouse are restless, insolent, fickle, scurrilous, loose-tongued, unmindful, lacking situational awareness and immersion, with wandering mind and undisciplined faculties. Go, \textsanskrit{Moggallāna}, and strike awe in those mendicants!” 

“Yes,\marginnote{3.1} sir,” replied \textsanskrit{Mahāmoggallāna}. Then he used his psychic power to make the longhouse shake and rock and tremble with his toe. Then those mendicants stood to one side, shocked and awestruck. 

“It’s\marginnote{3.3} incredible, it’s amazing! There’s no wind at all; and this stilt longhouse of \textsanskrit{Migāra}’s mother has deep foundations. It’s firmly embedded, imperturbable and unshakable. And yet it shakes and rocks and trembles!” 

Then\marginnote{4.1} the Buddha went up to those mendicants and said: 

“Why\marginnote{4.2} do you, mendicants, stand to one side, shocked and awestruck?” 

“It’s\marginnote{4.3} incredible, sir, it’s amazing! There’s no wind at all; and this stilt longhouse of \textsanskrit{Migāra}’s mother has deep foundations. It’s firmly embedded, imperturbable and unshakable. And yet it shakes and rocks and trembles!” 

“Wanting\marginnote{4.5} to strike awe in you, the mendicant \textsanskrit{Moggallāna} made the longhouse shake and rock and tremble with his toe. 

What\marginnote{4.6} do you think, mendicants? What things has the mendicant \textsanskrit{Moggallāna} developed and cultivated so as to have such power and might?” 

“Our\marginnote{4.8} teachings are rooted in the Buddha. He is our guide and our refuge. Sir, may the Buddha himself please clarify the meaning of this. The mendicants will listen and remember it.” 

“Well\marginnote{5.1} then, mendicants, listen. The mendicant \textsanskrit{Moggallāna} has become so powerful and mighty by developing and cultivating the four bases of psychic power. What four? \textsanskrit{Moggallāna} develops the basis of psychic power that has immersion due to enthusiasm … energy … mental development … inquiry, and active effort. He thinks: ‘My inquiry won’t be too lax or too tense. And it’ll be neither constricted internally nor scattered externally.’ And he meditates perceiving continuity: as before, so after; as after, so before; as below, so above; as above, so below; as by day, so by night; as by night, so by day. And so, with an open and unenveloped heart, he develops a mind that’s full of radiance. The mendicant \textsanskrit{Moggallāna} has become so powerful and mighty by developing and cultivating these four bases of psychic power. 

And\marginnote{5.16} by developing and cultivating these four bases of psychic power, the mendicant \textsanskrit{Moggallāna} wields the many kinds of psychic power … controlling the body as far as the \textsanskrit{Brahmā} realm. … 

And\marginnote{5.17} by developing and cultivating these four bases of psychic power, the mendicant \textsanskrit{Moggallāna} realizes the undefiled freedom of heart and freedom by wisdom in this very life. And he lives having realized it with his own insight due to the ending of defilements.” 

%
\section*{{\suttatitleacronym SN 51.15}{\suttatitletranslation The Brahmin Uṇṇābha }{\suttatitleroot Uṇṇābhabrāhmaṇasutta}}
\addcontentsline{toc}{section}{\tocacronym{SN 51.15} \toctranslation{The Brahmin Uṇṇābha } \tocroot{Uṇṇābhabrāhmaṇasutta}}
\markboth{The Brahmin Uṇṇābha }{Uṇṇābhabrāhmaṇasutta}
\extramarks{SN 51.15}{SN 51.15}

\scevam{So\marginnote{1.1} I have heard. }At one time Venerable Ānanda was staying near Kosambi, in Ghosita’s Monastery. Then \textsanskrit{Uṇṇābha} the brahmin went up to Venerable Ānanda, and exchanged greetings with him. When the greetings and polite conversation were over, he sat down to one side and said to him, “Master Ānanda, what’s the purpose of leading the spiritual life under the ascetic Gotama?” 

“The\marginnote{1.6} purpose of leading the spiritual life under the Buddha, brahmin, is to give up desire.” 

“But\marginnote{2.1} is there a path and a practice for giving up that desire?” 

“There\marginnote{2.2} is.” 

“What\marginnote{3.1} is that path?” 

“It’s\marginnote{3.2} when a mendicant develops the basis of psychic power that has immersion due to enthusiasm … energy … mental development … inquiry, and active effort. This is the path and the practice for giving up that desire.” 

“This\marginnote{4.1} being the case, Master Ānanda, the path is endless, not finite. For it’s not possible to give up desire by means of desire.” 

“Well\marginnote{4.3} then, brahmin, I’ll ask you about this in return, and you can answer as you like. What do you think, brahmin? Have you ever had a desire to walk to the park, but when you arrived at the park, the corresponding desire faded away?” 

“Yes,\marginnote{4.7} sir.” 

“Have\marginnote{4.8} you ever had the energy to walk to the park, but when you arrived at the park, the corresponding energy faded away?” 

“Yes,\marginnote{4.10} sir.” 

“Have\marginnote{4.11} you ever had the idea to walk to the park, but when you arrived at the park, the corresponding idea faded away?” 

“Yes,\marginnote{4.13} sir.” 

“Have\marginnote{4.14} you ever inquired regarding a walk to the park, but when you arrived at the park, the corresponding inquiry faded away?” 

“Yes,\marginnote{4.16} sir.” 

“In\marginnote{5.1} the same way, take a mendicant who is perfected—with defilements ended, who has completed the spiritual journey, done what had to be done, laid down the burden, achieved their own true goal, utterly ended the fetters of rebirth, and is rightly freed through enlightenment. They formerly had the desire to attain perfection, but when they attained perfection the corresponding desire faded away. They formerly had the energy to attain perfection, but when they attained perfection the corresponding energy faded away. They formerly had the idea to attain perfection, but when they attained perfection the corresponding idea faded away. They formerly inquired regarding attaining perfection, but when they attained perfection the corresponding inquiry faded away. What do you think, brahmin? This being the case, is the path endless or finite?” 

“Clearly,\marginnote{6.1} Master Ānanda, this being the case, the path is finite, not endless. Excellent, Master Ānanda! Excellent! As if he were righting the overturned, or revealing the hidden, or pointing out the path to the lost, or lighting a lamp in the dark so people with good eyes can see what’s there, Master Ānanda has made the teaching clear in many ways. I go for refuge to Master Gotama, to the teaching, and to the mendicant \textsanskrit{Saṅgha}. From this day forth, may Master Ānanda remember me as a lay follower who has gone for refuge for life.” 

%
\section*{{\suttatitleacronym SN 51.16}{\suttatitletranslation Ascetics and Brahmins (1st) }{\suttatitleroot Paṭhamasamaṇabrāhmaṇasutta}}
\addcontentsline{toc}{section}{\tocacronym{SN 51.16} \toctranslation{Ascetics and Brahmins (1st) } \tocroot{Paṭhamasamaṇabrāhmaṇasutta}}
\markboth{Ascetics and Brahmins (1st) }{Paṭhamasamaṇabrāhmaṇasutta}
\extramarks{SN 51.16}{SN 51.16}

“Mendicants,\marginnote{1.1} all the ascetics and brahmins in the past, future, or present who are mighty and powerful have become so by developing and cultivating the four bases of psychic power. 

What\marginnote{2.1} four? It’s when a mendicant develops the basis of psychic power that has immersion due to enthusiasm … energy … mental development … inquiry, and active effort. All the ascetics and brahmins in the past, future, or present who are mighty and powerful have become so by developing and cultivating the four bases of psychic power.” 

%
\section*{{\suttatitleacronym SN 51.17}{\suttatitletranslation Ascetics and Brahmins (2nd) }{\suttatitleroot Dutiyasamaṇabrāhmaṇasutta}}
\addcontentsline{toc}{section}{\tocacronym{SN 51.17} \toctranslation{Ascetics and Brahmins (2nd) } \tocroot{Dutiyasamaṇabrāhmaṇasutta}}
\markboth{Ascetics and Brahmins (2nd) }{Dutiyasamaṇabrāhmaṇasutta}
\extramarks{SN 51.17}{SN 51.17}

“Mendicants,\marginnote{1.1} all the ascetics and brahmins in the past, future, or present who wield the various kinds of psychic power—multiplying themselves and becoming one again; appearing and disappearing; going unimpeded through a wall, a rampart, or a mountain as if through space; diving in and out of the earth as if it were water; walking on water as if it were earth; flying cross-legged through the sky like a bird; touching and stroking with the hand the sun and moon, so mighty and powerful; controlling their body as far as the \textsanskrit{Brahmā} realm—do so by developing and cultivating the four bases of psychic power. 

What\marginnote{4.1} four? It’s when a mendicant develops the basis of psychic power that has immersion due to enthusiasm … energy … mental development … inquiry, and active effort. Mendicants, all the ascetics and brahmins in the past, future, or present who wield the many kinds of psychic power—multiplying themselves and becoming one again … controlling their body as far as the \textsanskrit{Brahmā} realm—do so by developing and cultivating these four bases of psychic power.” 

%
\section*{{\suttatitleacronym SN 51.18}{\suttatitletranslation A Mendicant }{\suttatitleroot Bhikkhusutta}}
\addcontentsline{toc}{section}{\tocacronym{SN 51.18} \toctranslation{A Mendicant } \tocroot{Bhikkhusutta}}
\markboth{A Mendicant }{Bhikkhusutta}
\extramarks{SN 51.18}{SN 51.18}

“Mendicants,\marginnote{1.1} by developing and cultivating the four bases of psychic power, a mendicant realizes the undefiled freedom of heart and freedom by wisdom in this very life. And they live having realized it with their own insight due to the ending of defilements. 

What\marginnote{2.1} four? It’s when a mendicant develops the basis of psychic power that has immersion due to enthusiasm … energy … mental development … inquiry, and active effort. By developing and cultivating these four bases of psychic power, a mendicant realizes the undefiled freedom of heart and freedom by wisdom in this very life. And they live having realized it with their own insight due to the ending of defilements.” 

%
\section*{{\suttatitleacronym SN 51.19}{\suttatitletranslation A Teaching on Psychic Power, Etc. }{\suttatitleroot Iddhādidesanāsutta}}
\addcontentsline{toc}{section}{\tocacronym{SN 51.19} \toctranslation{A Teaching on Psychic Power, Etc. } \tocroot{Iddhādidesanāsutta}}
\markboth{A Teaching on Psychic Power, Etc. }{Iddhādidesanāsutta}
\extramarks{SN 51.19}{SN 51.19}

“Mendicants,\marginnote{1.1} I will teach you psychic power, the bases of psychic power, the development of the bases of psychic power, and the practice that leads to the development of the bases of psychic power. Listen … 

And\marginnote{2.1} what is psychic power? It’s when a mendicant wields the many kinds of psychic power: multiplying themselves and becoming one again … controlling the body as far as the \textsanskrit{Brahmā} realm. This is called psychic power. 

And\marginnote{3.1} what is the basis of psychic power? The path and practice that leads to gaining psychic power. This is called the basis of psychic power. 

And\marginnote{4.1} what is the development of the bases of psychic power? It’s when a mendicant develops the basis of psychic power that has immersion due to enthusiasm … energy … mental development … inquiry, and active effort. This is called the development of the bases of psychic power. 

And\marginnote{5.1} what is the practice that leads to the development of the bases of psychic power? It is simply this noble eightfold path, that is: right view, right thought, right speech, right action, right livelihood, right effort, right mindfulness, and right immersion. This is called the practice that leads to the development of the bases of psychic power.” 

%
\section*{{\suttatitleacronym SN 51.20}{\suttatitletranslation Analysis }{\suttatitleroot Vibhaṅgasutta}}
\addcontentsline{toc}{section}{\tocacronym{SN 51.20} \toctranslation{Analysis } \tocroot{Vibhaṅgasutta}}
\markboth{Analysis }{Vibhaṅgasutta}
\extramarks{SN 51.20}{SN 51.20}

“Mendicants,\marginnote{1.1} when the four bases of psychic power are developed and cultivated they’re very fruitful and beneficial. 

How\marginnote{2.1} so? It’s when a mendicant develops the basis of psychic power that has immersion due to enthusiasm, and active effort. They think: ‘My enthusiasm won’t be too lax or too tense. And it’ll be neither constricted internally nor scattered externally.’ And they meditate perceiving continuity: as before, so after; as after, so before; as below, so above; as above, so below; as by day, so by night; as by night, so by day. And so, with an open and unenveloped heart, they develop a mind that’s full of radiance. 

They\marginnote{2.9} develop the basis of psychic power that has immersion due to energy … mental development … inquiry, and active effort. They think: ‘My inquiry won’t be too lax or too tense. And it’ll be neither constricted internally nor scattered externally.’ And they meditate perceiving continuity: as before, so after; as after, so before; as below, so above; as above, so below; as by day, so by night; as by night, so by day. And so, with an open and unenveloped heart, they develop a mind that’s full of radiance. 

And\marginnote{3.1} what is enthusiasm that’s too lax? It’s when enthusiasm is combined with laziness. This is called lax enthusiasm. 

And\marginnote{4.1} what is enthusiasm that’s too tense? It’s when enthusiasm is combined with restlessness. This is called tense enthusiasm. 

And\marginnote{5.1} what is enthusiasm that’s constricted internally? It’s when enthusiasm is combined with dullness and drowsiness. This is called enthusiasm constricted internally. 

And\marginnote{6.1} what is enthusiasm that’s distracted externally? It’s when enthusiasm is frequently distracted and diffused externally on account of the five kinds of sensual stimulation. This is called enthusiasm distracted externally. 

And\marginnote{7.1} how does a mendicant meditate perceiving continuity: as before, so after; as after, so before? It’s when the perception of continuity is properly grasped, attended, borne in mind, and comprehended with wisdom by a mendicant. That’s how a mendicant meditates perceiving continuity: as before, so after; as after, so before. 

And\marginnote{8.1} how does a mendicant meditate as below, so above; as above, so below? It’s when a mendicant examines their own body up from the soles of the feet and down from the tips of the hairs, wrapped in skin and full of many kinds of filth. ‘In this body there is head hair, body hair, nails, teeth, skin, flesh, sinews, bones, bone marrow, kidneys, heart, liver, diaphragm, spleen, lungs, intestines, mesentery, undigested food, feces, bile, phlegm, pus, blood, sweat, fat, tears, grease, saliva, snot, synovial fluid, urine.’ That’s how a mendicant meditates as below, so above; as above, so below. 

And\marginnote{9.1} how does a mendicant meditate as by day, so by night; as by night, so by day? It’s when a mendicant develops the basis of psychic power that has immersion due to enthusiasm, and active effort, with the same features, attributes, and signs by day as by night. And they develop it with the same features, attributes, and signs by night as by day. That’s how a mendicant meditates as by day, so by night; as by night, so by day. 

And\marginnote{10.1} how, with an open and unenveloped heart, does a mendicant develop a mind that’s full of radiance? It’s when a mendicant has properly grasped the perception of light, and has properly grasped the perception of day. That’s how, with an open and unenveloped heart, a mendicant develops a mind that’s full of radiance. 

And\marginnote{11.1} what is energy that’s too lax? … 

And\marginnote{16.1} what is mental development that’s too lax? … 

And\marginnote{20.1} what is inquiry that’s too lax? It’s when inquiry is combined with laziness. This is called lax inquiry. 

And\marginnote{21.1} what is inquiry that’s too tense? It’s when inquiry is combined with restlessness. This is called tense inquiry. 

And\marginnote{22.1} what is inquiry that’s constricted internally? It’s when inquiry is combined with dullness and drowsiness. This is called inquiry constricted internally. 

And\marginnote{23.1} what is inquiry that’s distracted externally? It’s when inquiry is frequently distracted and diffused externally on account of the five kinds of sensual stimulation. This is called inquiry distracted externally. … That’s how, with an open and unenveloped heart, a mendicant develops a mind that’s full of radiance. When the four bases of psychic power have been developed and cultivated in this way they’re very fruitful and beneficial. 

When\marginnote{24.1} the four bases of psychic power have been developed and cultivated in this way, a mendicant wields the many kinds of psychic power: multiplying themselves and becoming one again … controlling the body as far as the \textsanskrit{Brahmā} realm. When the four bases of psychic power have been developed and cultivated in this way, they realize the undefiled freedom of heart and freedom by wisdom in this very life. And they live having realized it with their own insight due to the ending of defilements.” 

%
\addtocontents{toc}{\let\protect\contentsline\protect\nopagecontentsline}
\chapter*{The Chapter on the Iron Ball }
\addcontentsline{toc}{chapter}{\tocchapterline{The Chapter on the Iron Ball }}
\addtocontents{toc}{\let\protect\contentsline\protect\oldcontentsline}

%
\section*{{\suttatitleacronym SN 51.21}{\suttatitletranslation The Path }{\suttatitleroot Maggasutta}}
\addcontentsline{toc}{section}{\tocacronym{SN 51.21} \toctranslation{The Path } \tocroot{Maggasutta}}
\markboth{The Path }{Maggasutta}
\extramarks{SN 51.21}{SN 51.21}

At\marginnote{1.1} \textsanskrit{Sāvatthī}. 

“Mendicants,\marginnote{1.2} before my awakening—when I was still unawakened but intent on awakening—I thought: ‘What’s the path and practice for developing the bases of psychic power?’ Then it occurred to me: ‘It’s when a mendicant develops the basis of psychic power that has immersion due to enthusiasm, and active effort. They think: “My enthusiasm won’t be too lax or too tense. And it’ll be neither constricted internally nor scattered externally.” And they meditate perceiving continuity: as before, so after; as after, so before; as below, so above; as above, so below; as by day, so by night; as by night, so by day. And so, with an open and unenveloped heart, they develop a mind that’s full of radiance. 

They\marginnote{1.12} develop the basis of psychic power that has immersion due to energy … mental development … inquiry, and active effort. They think: “My inquiry won’t be too lax or too tense. And it’ll be neither constricted internally nor scattered externally.” And they meditate perceiving continuity: as before, so after; as after, so before; as below, so above; as above, so below; as by day, so by night; as by night, so by day. And so, with an open and unenveloped heart, they develop a mind that’s full of radiance.’ 

When\marginnote{2.1} the four bases of psychic power have been developed and cultivated in this way, a mendicant wields the many kinds of psychic power: multiplying themselves and becoming one again … controlling the body as far as the \textsanskrit{Brahmā} realm. 

When\marginnote{2.2} the four bases of psychic power have been developed and cultivated in this way, they realize the undefiled freedom of heart and freedom by wisdom in this very life. And they live having realized it with their own insight due to the ending of defilements.” 

\scendsection{(The six direct knowledges should also be expanded.) }

%
\section*{{\suttatitleacronym SN 51.22}{\suttatitletranslation The Iron Ball }{\suttatitleroot Ayoguḷasutta}}
\addcontentsline{toc}{section}{\tocacronym{SN 51.22} \toctranslation{The Iron Ball } \tocroot{Ayoguḷasutta}}
\markboth{The Iron Ball }{Ayoguḷasutta}
\extramarks{SN 51.22}{SN 51.22}

At\marginnote{1.1} \textsanskrit{Sāvatthī}. 

Then\marginnote{1.2} Venerable Ānanda went up to the Buddha, bowed, sat down to one side, and said to him: 

“Sir,\marginnote{1.3} do you have personal experience of going to the \textsanskrit{Brahmā} realm by psychic power with a mind-made body?” 

“I\marginnote{1.4} do, Ānanda.” 

“But\marginnote{1.5} do you have personal experience of going to the \textsanskrit{Brahmā} realm by psychic power with this body made up of the four primary elements?” 

“I\marginnote{1.6} do, Ānanda.” 

“It’s\marginnote{2.1} incredible and amazing that the Buddha is capable of going to the \textsanskrit{Brahmā} realm by psychic power with a mind-made body! And that he has personal experience of going to the \textsanskrit{Brahmā} realm by psychic power with this body made up of the four primary elements!” 

“Ānanda,\marginnote{2.2} the Realized Ones are incredible and have incredible qualities. They’re amazing and have amazing qualities. 

Sometimes\marginnote{3.1} the Realized One submerges his body in his mind and his mind in his body. He meditates after sinking into a perception of bliss and lightness in the body. At that time his body becomes lighter, softer, more workable, and more radiant. 

Suppose\marginnote{4.1} there was an iron ball that had been heated all day. It’d become lighter, softer, more workable, and more radiant. In the same way, sometimes the Realized One submerges his body in his mind and his mind in his body. He meditates after sinking into a perception of bliss and lightness in the body. At that time his body becomes lighter, softer, more workable, and more radiant. 

Sometimes\marginnote{5.1} the Realized One submerges his body in his mind and his mind in his body. He meditates after sinking into a perception of bliss and lightness in the body. At that time his body easily rises up from the ground into the air. He wields the many kinds of psychic power: multiplying himself and becoming one again … controlling the body as far as the \textsanskrit{Brahmā} realm. 

Suppose\marginnote{6.1} there was a light tuft of cotton-wool or kapok. Taken up by the wind, it would easily rise up from the ground into the air. In the same way, sometimes the Realized One submerges his body in his mind and his mind in his body. He meditates after sinking into a perception of bliss and lightness in the body. At that time his body easily rises up from the ground into the air. He wields the many kinds of psychic power: multiplying himself and becoming one again … controlling the body as far as the \textsanskrit{Brahmā} realm.” 

%
\section*{{\suttatitleacronym SN 51.23}{\suttatitletranslation A Mendicant }{\suttatitleroot Bhikkhusutta}}
\addcontentsline{toc}{section}{\tocacronym{SN 51.23} \toctranslation{A Mendicant } \tocroot{Bhikkhusutta}}
\markboth{A Mendicant }{Bhikkhusutta}
\extramarks{SN 51.23}{SN 51.23}

“Mendicants,\marginnote{1.1} there are these four bases of psychic power. What four? It’s when a mendicant develops the basis of psychic power that has immersion due to enthusiasm … energy … mental development … inquiry, and active effort. These are the four bases of psychic power. By developing and cultivating these four bases of psychic power, a mendicant realizes the undefiled freedom of heart and freedom by wisdom in this very life. And they live having realized it with their own insight due to the ending of defilements.” 

%
\section*{{\suttatitleacronym SN 51.24}{\suttatitletranslation Plain Version }{\suttatitleroot Suddhikasutta}}
\addcontentsline{toc}{section}{\tocacronym{SN 51.24} \toctranslation{Plain Version } \tocroot{Suddhikasutta}}
\markboth{Plain Version }{Suddhikasutta}
\extramarks{SN 51.24}{SN 51.24}

“Mendicants,\marginnote{1.1} there are these four bases of psychic power. What four? It’s when a mendicant develops the basis of psychic power that has immersion due to enthusiasm … energy … mental development … inquiry, and active effort. These are the four bases of psychic power.” 

%
\section*{{\suttatitleacronym SN 51.25}{\suttatitletranslation Fruits (1st) }{\suttatitleroot Paṭhamaphalasutta}}
\addcontentsline{toc}{section}{\tocacronym{SN 51.25} \toctranslation{Fruits (1st) } \tocroot{Paṭhamaphalasutta}}
\markboth{Fruits (1st) }{Paṭhamaphalasutta}
\extramarks{SN 51.25}{SN 51.25}

“Mendicants,\marginnote{1.1} there are these four bases of psychic power. What four? It’s when a mendicant develops the basis of psychic power that has immersion due to enthusiasm … energy … mental development … inquiry, and active effort. These are the four bases of psychic power. Because of developing and cultivating these four bases of psychic power, one of two results can be expected: enlightenment in the present life, or if there’s something left over, non-return.” 

%
\section*{{\suttatitleacronym SN 51.26}{\suttatitletranslation Fruits (2nd) }{\suttatitleroot Dutiyaphalasutta}}
\addcontentsline{toc}{section}{\tocacronym{SN 51.26} \toctranslation{Fruits (2nd) } \tocroot{Dutiyaphalasutta}}
\markboth{Fruits (2nd) }{Dutiyaphalasutta}
\extramarks{SN 51.26}{SN 51.26}

“Mendicants,\marginnote{1.1} there are these four bases of psychic power. What four? It’s when a mendicant develops the basis of psychic power that has immersion due to enthusiasm … energy … mental development … inquiry, and active effort. These are the four bases of psychic power. Because of developing and cultivating these four bases of psychic power, seven fruits and benefits can be expected. 

What\marginnote{2.1} seven? They attain enlightenment early on in this very life. If not, they attain enlightenment at the time of death. If not, with the ending of the five lower fetters, they’re extinguished between one life and the next … they’re extinguished upon landing … they’re extinguished without extra effort … they’re extinguished with extra effort … they head upstream, going to the \textsanskrit{Akaniṭṭha} realm. Because of developing and cultivating these four bases of psychic power, these seven fruits and benefits can be expected.” 

%
\section*{{\suttatitleacronym SN 51.27}{\suttatitletranslation With Ānanda (1st) }{\suttatitleroot Paṭhamaānandasutta}}
\addcontentsline{toc}{section}{\tocacronym{SN 51.27} \toctranslation{With Ānanda (1st) } \tocroot{Paṭhamaānandasutta}}
\markboth{With Ānanda (1st) }{Paṭhamaānandasutta}
\extramarks{SN 51.27}{SN 51.27}

At\marginnote{1.1} \textsanskrit{Sāvatthī}. 

Then\marginnote{1.2} Venerable Ānanda went up to the Buddha, bowed, sat down to one side, and said to him: 

“Sir,\marginnote{2.1} what is psychic power? What is the basis of psychic power? What is the development of the bases of psychic power? And what is the practice that leads to the development of the bases of psychic power?” 

“Ānanda,\marginnote{2.2} take a mendicant who wields the many kinds of psychic power: multiplying themselves and becoming one again … controlling the body as far as the \textsanskrit{Brahmā} realm. This is called psychic power. 

And\marginnote{3.1} what is the basis of psychic power? The path and practice that leads to gaining psychic power. This is called the basis of psychic power. 

And\marginnote{4.1} what is the development of the bases of psychic power? It’s when a mendicant develops the basis of psychic power that has immersion due to enthusiasm … energy … mental development … inquiry, and active effort. This is called the development of the bases of psychic power. 

And\marginnote{5.1} what is the practice that leads to the development of the bases of psychic power? It is simply this noble eightfold path, that is: right view, right thought, right speech, right action, right livelihood, right effort, right mindfulness, and right immersion. This is called the practice that leads to the development of the bases of psychic power.” 

%
\section*{{\suttatitleacronym SN 51.28}{\suttatitletranslation With Ānanda (2nd) }{\suttatitleroot Dutiyaānandasutta}}
\addcontentsline{toc}{section}{\tocacronym{SN 51.28} \toctranslation{With Ānanda (2nd) } \tocroot{Dutiyaānandasutta}}
\markboth{With Ānanda (2nd) }{Dutiyaānandasutta}
\extramarks{SN 51.28}{SN 51.28}

The\marginnote{1.1} Buddha said to him: “Ānanda, what is psychic power? What is the basis of psychic power? What is the development of the bases of psychic power? And what is the practice that leads to the development of the bases of psychic power?” 

“Our\marginnote{1.3} teachings are rooted in the Buddha. He is our guide and our refuge. …” 

“Ānanda,\marginnote{2.1} take a mendicant who wields the many kinds of psychic power: multiplying themselves and becoming one again … controlling the body as far as the \textsanskrit{Brahmā} realm. This is called psychic power. 

And\marginnote{3.1} what is the basis of psychic power? The path and practice that leads to gaining psychic power. This is called the basis of psychic power. 

And\marginnote{4.1} what is the development of the bases of psychic power? It’s when a mendicant develops the basis of psychic power that has immersion due to enthusiasm … energy … mental development … inquiry, and active effort. This is called the development of the bases of psychic power. 

And\marginnote{5.1} what is the practice that leads to the development of the bases of psychic power? It is simply this noble eightfold path, that is: right view, right thought, right speech, right action, right livelihood, right effort, right mindfulness, and right immersion. This is called the practice that leads to the development of the bases of psychic power.” 

%
\section*{{\suttatitleacronym SN 51.29}{\suttatitletranslation Several Mendicants (1st) }{\suttatitleroot Paṭhamabhikkhusutta}}
\addcontentsline{toc}{section}{\tocacronym{SN 51.29} \toctranslation{Several Mendicants (1st) } \tocroot{Paṭhamabhikkhusutta}}
\markboth{Several Mendicants (1st) }{Paṭhamabhikkhusutta}
\extramarks{SN 51.29}{SN 51.29}

Then\marginnote{1.1} several mendicants went up to the Buddha, bowed, sat down to one side, and said to him: 

“Sir,\marginnote{1.2} what is psychic power? What is the basis of psychic power? What is the development of the bases of psychic power? And what is the practice that leads to the development of the bases of psychic power?” 

“Mendicants,\marginnote{2.1} take a mendicant who wields the many kinds of psychic power: multiplying themselves and becoming one again … controlling the body as far as the \textsanskrit{Brahmā} realm. This is called psychic power. 

And\marginnote{3.1} what is the basis of psychic power? The path and practice that leads to gaining psychic power. This is called the basis of psychic power. 

And\marginnote{4.1} what is the development of the bases of psychic power? It’s when a mendicant develops the basis of psychic power that has immersion due to enthusiasm … energy … mental development … inquiry, and active effort. This is called the development of the bases of psychic power. 

And\marginnote{5.1} what is the practice that leads to the development of the bases of psychic power? It is simply this noble eightfold path, that is: right view, right thought, right speech, right action, right livelihood, right effort, right mindfulness, and right immersion. This is called the practice that leads to the development of the bases of psychic power.” 

%
\section*{{\suttatitleacronym SN 51.30}{\suttatitletranslation Several Mendicants (2nd) }{\suttatitleroot Dutiyabhikkhusutta}}
\addcontentsline{toc}{section}{\tocacronym{SN 51.30} \toctranslation{Several Mendicants (2nd) } \tocroot{Dutiyabhikkhusutta}}
\markboth{Several Mendicants (2nd) }{Dutiyabhikkhusutta}
\extramarks{SN 51.30}{SN 51.30}

Then\marginnote{1.1} several mendicants went up to the Buddha … The Buddha said to them: 

“Mendicants,\marginnote{1.3} what is psychic power? What is the basis of psychic power? What is the development of the bases of psychic power? And what is the practice that leads to the development of the bases of psychic power?” 

“Our\marginnote{1.4} teachings are rooted in the Buddha. He is our guide and our refuge. …” 

“And\marginnote{2.1} what is psychic power? It’s a mendicant who wields the many kinds of psychic power: multiplying themselves and becoming one again … controlling the body as far as the \textsanskrit{Brahmā} realm. This is called psychic power. 

And\marginnote{3.1} what is the basis of psychic power? The path and practice that leads to gaining psychic power. This is called the basis of psychic power. 

And\marginnote{4.1} what is the development of the bases of psychic power? It’s when a mendicant develops the basis of psychic power that has immersion due to enthusiasm … energy … mental development … inquiry, and active effort. This is called the development of the bases of psychic power. 

And\marginnote{5.1} what is the practice that leads to the development of the bases of psychic power? It is simply this noble eightfold path, that is: right view, right thought, right speech, right action, right livelihood, right effort, right mindfulness, and right immersion. This is called the practice that leads to the development of the bases of psychic power.” 

%
\section*{{\suttatitleacronym SN 51.31}{\suttatitletranslation About Moggallāna }{\suttatitleroot Moggallānasutta}}
\addcontentsline{toc}{section}{\tocacronym{SN 51.31} \toctranslation{About Moggallāna } \tocroot{Moggallānasutta}}
\markboth{About Moggallāna }{Moggallānasutta}
\extramarks{SN 51.31}{SN 51.31}

There\marginnote{1.1} the Buddha addressed the mendicants: “What do you think, mendicants? What things has the mendicant \textsanskrit{Moggallāna} developed and cultivated so as to have such power and might?” 

“Our\marginnote{2.1} teachings are rooted in the Buddha. He is our guide and our refuge. …” 

“The\marginnote{2.2} mendicant \textsanskrit{Moggallāna} has become so powerful and mighty by developing and cultivating the four bases of psychic power. 

What\marginnote{3.1} four? \textsanskrit{Moggallāna} develops the basis of psychic power that has immersion due to enthusiasm, and active effort. He thinks: ‘My enthusiasm won’t be too lax or too tense. And it’ll be neither constricted internally nor scattered externally.’ And he meditates perceiving continuity: as before, so after; as after, so before; as below, so above; as above, so below; as by day, so by night; as by night, so by day. And so, with an open and unenveloped heart, he develops a mind that’s full of radiance. 

He\marginnote{3.9} develops the basis of psychic power that has immersion due to energy … mental development … inquiry, and active effort. He thinks: ‘My inquiry won’t be too lax or too tense. And it’ll be neither constricted internally nor scattered externally.’ … And so, with an open and unenveloped heart, he develops a mind that’s full of radiance. The mendicant \textsanskrit{Moggallāna} has become so powerful and mighty by developing and cultivating these four bases of psychic power. 

And\marginnote{4.1} by developing and cultivating these four bases of psychic power, the mendicant \textsanskrit{Moggallāna} wields the many kinds of psychic power: multiplying himself and becoming one again … controlling the body as far as the \textsanskrit{Brahmā} realm. 

And\marginnote{4.2} by developing and cultivating these four bases of psychic power, the mendicant \textsanskrit{Moggallāna} realizes the undefiled freedom of heart and freedom by wisdom in this very life. And he lives having realized it with his own insight due to the ending of defilements.” 

%
\section*{{\suttatitleacronym SN 51.32}{\suttatitletranslation The Realized One }{\suttatitleroot Tathāgatasutta}}
\addcontentsline{toc}{section}{\tocacronym{SN 51.32} \toctranslation{The Realized One } \tocroot{Tathāgatasutta}}
\markboth{The Realized One }{Tathāgatasutta}
\extramarks{SN 51.32}{SN 51.32}

There\marginnote{1.1} the Buddha addressed the mendicants: “What do you think, mendicants? What things has the Realized One developed and cultivated so as to have such power and might?” 

“Our\marginnote{2.1} teachings are rooted in the Buddha. …” 

“The\marginnote{2.2} Realized One has become so powerful and mighty by developing and cultivating the four bases of psychic power. 

What\marginnote{3.1} four? It’s when a mendicant develops the basis of psychic power that has immersion due to enthusiasm, and active effort. He thinks: ‘My enthusiasm won’t be too lax or too tense. And it’ll be neither constricted internally nor scattered externally.’ And he meditates perceiving continuity: as before, so after; as after, so before; as below, so above; as above, so below; as by day, so by night; as by night, so by day. And so, with an open and unenveloped heart, he develops a mind that’s full of radiance. 

He\marginnote{3.9} develops the basis of psychic power that has immersion due to energy … mental development … inquiry, and active effort. He thinks: ‘My inquiry won’t be too lax or too tense. And it’ll be neither constricted internally nor scattered externally.’ And he meditates perceiving continuity: as before, so after; as after, so before; as below, so above; as above, so below; as by day, so by night; as by night, so by day. And so, with an open and unenveloped heart, he develops a mind that’s full of radiance. 

The\marginnote{3.18} Realized One has become so powerful and mighty by developing and cultivating these four bases of psychic power. 

And\marginnote{4.1} by developing and cultivating these four bases of psychic power, the Realized One wields the many kinds of psychic power: multiplying himself and becoming one again … controlling the body as far as the \textsanskrit{Brahmā} realm. 

And\marginnote{4.2} by developing and cultivating these four bases of psychic power, the Realized One realizes the undefiled freedom of heart and freedom by wisdom in this very life. And he lives having realized it with his own insight due to the ending of defilements.” 

\scendsection{(The six direct knowledges should also be expanded.) }

%
\addtocontents{toc}{\let\protect\contentsline\protect\nopagecontentsline}
\chapter*{The Chapter of Abbreviated Texts on the Ganges }
\addcontentsline{toc}{chapter}{\tocchapterline{The Chapter of Abbreviated Texts on the Ganges }}
\addtocontents{toc}{\let\protect\contentsline\protect\oldcontentsline}

%
\section*{{\suttatitleacronym SN 51.33–44}{\suttatitletranslation The Ganges River, Etc. }{\suttatitleroot Gaṅgāpeyyālavagga}}
\addcontentsline{toc}{section}{\tocacronym{SN 51.33–44} \toctranslation{The Ganges River, Etc. } \tocroot{Gaṅgāpeyyālavagga}}
\markboth{The Ganges River, Etc. }{Gaṅgāpeyyālavagga}
\extramarks{SN 51.33–44}{SN 51.33–44}

“Mendicants,\marginnote{1.1} the Ganges river slants, slopes, and inclines to the east. In the same way, a mendicant who develops and cultivates the four bases of psychic power slants, slopes, and inclines to extinguishment. 

And\marginnote{1.3} how does a mendicant who develops the four bases of psychic power slant, slope, and incline to extinguishment? It’s when a mendicant develops the basis of psychic power that has immersion due to enthusiasm … energy … mental development … inquiry, and active effort. 

In\marginnote{2.1} the same way, a mendicant who develops and cultivates the four bases of psychic power slants, slopes, and inclines to extinguishment.” 

\scendsection{(To be expanded for each of the different rivers as in SN 45.91–102.) }

\begin{quotation}%
Six\marginnote{3.1} on slanting to the east, \\
and six on slanting to the ocean; \\
these two sixes make twelve, \\
and that’s how this chapter is recited. 

%
\end{quotation}

%
\addtocontents{toc}{\let\protect\contentsline\protect\nopagecontentsline}
\chapter*{The Chapter on Diligence }
\addcontentsline{toc}{chapter}{\tocchapterline{The Chapter on Diligence }}
\addtocontents{toc}{\let\protect\contentsline\protect\oldcontentsline}

%
\section*{{\suttatitleacronym SN 51.45–54}{\suttatitletranslation Diligence }{\suttatitleroot Appamādavagga}}
\addcontentsline{toc}{section}{\tocacronym{SN 51.45–54} \toctranslation{Diligence } \tocroot{Appamādavagga}}
\markboth{Diligence }{Appamādavagga}
\extramarks{SN 51.45–54}{SN 51.45–54}

(To\marginnote{1.1} be expanded as in the chapter on diligence at SN 45.139–148.) 

\begin{quotation}%
The\marginnote{2.1} Realized One, footprint, roof peak, \\
roots, heartwood, jasmine, \\
monarch, sun and moon, \\
and cloth is the tenth. 

%
\end{quotation}

%
\addtocontents{toc}{\let\protect\contentsline\protect\nopagecontentsline}
\chapter*{The Chapter on Hard Work }
\addcontentsline{toc}{chapter}{\tocchapterline{The Chapter on Hard Work }}
\addtocontents{toc}{\let\protect\contentsline\protect\oldcontentsline}

%
\section*{{\suttatitleacronym SN 51.55–66}{\suttatitletranslation Hard Work }{\suttatitleroot Balakaraṇīyavagga}}
\addcontentsline{toc}{section}{\tocacronym{SN 51.55–66} \toctranslation{Hard Work } \tocroot{Balakaraṇīyavagga}}
\markboth{Hard Work }{Balakaraṇīyavagga}
\extramarks{SN 51.55–66}{SN 51.55–66}

(To\marginnote{1.1} be expanded as in the chapter on hard work at SN 45.149–160.) 

\begin{quotation}%
Hard\marginnote{2.1} work, seeds, and dragons, \\
a tree, a pot, and a spike, \\
the sky, and two on clouds, \\
a ship, a guest house, and a river. 

%
\end{quotation}

%
\addtocontents{toc}{\let\protect\contentsline\protect\nopagecontentsline}
\chapter*{The Chapter on Searches }
\addcontentsline{toc}{chapter}{\tocchapterline{The Chapter on Searches }}
\addtocontents{toc}{\let\protect\contentsline\protect\oldcontentsline}

%
\section*{{\suttatitleacronym SN 51.67–76}{\suttatitletranslation Searches }{\suttatitleroot Esanāvagga}}
\addcontentsline{toc}{section}{\tocacronym{SN 51.67–76} \toctranslation{Searches } \tocroot{Esanāvagga}}
\markboth{Searches }{Esanāvagga}
\extramarks{SN 51.67–76}{SN 51.67–76}

(To\marginnote{1.1} be expanded as in the chapter on searches at SN 45.161–170.) 

\begin{quotation}%
Searches,\marginnote{2.1} discriminations, defilements, \\
states of existence, three kinds of suffering, \\
barrenness, stains, and troubles, \\
feelings, craving, and thirst. 

%
\end{quotation}

%
\addtocontents{toc}{\let\protect\contentsline\protect\nopagecontentsline}
\chapter*{The Chapter on Floods }
\addcontentsline{toc}{chapter}{\tocchapterline{The Chapter on Floods }}
\addtocontents{toc}{\let\protect\contentsline\protect\oldcontentsline}

%
\section*{{\suttatitleacronym SN 51.77–86}{\suttatitletranslation Floods, Etc. }{\suttatitleroot Oghavagga}}
\addcontentsline{toc}{section}{\tocacronym{SN 51.77–86} \toctranslation{Floods, Etc. } \tocroot{Oghavagga}}
\markboth{Floods, Etc. }{Oghavagga}
\extramarks{SN 51.77–86}{SN 51.77–86}

“Mendicants,\marginnote{1.1} there are five higher fetters. What five? Desire for rebirth in the realm of luminous form, desire for rebirth in the formless realm, conceit, restlessness, and ignorance. These are the five higher fetters. 

The\marginnote{1.5} four bases of psychic power should be developed for the direct knowledge, complete understanding, finishing, and giving up of these five higher fetters. What four? It’s when a mendicant develops the basis of psychic power that has immersion due to enthusiasm … energy … mental development … inquiry, and active effort. These four bases of psychic power should be developed for the direct knowledge, complete understanding, finishing, and giving up of these five higher fetters.” 

\scendsection{(To be expanded as in the Linked Discourses on the Path at SN 45.171–180.) }

\begin{quotation}%
Floods,\marginnote{2.1} bonds, grasping, \\
ties, and underlying tendencies, \\
kinds of sensual stimulation, hindrances, \\
aggregates, and fetters high and low. 

%
\end{quotation}

\scendsutta{The Linked Discourses on the Bases of psychic Power is the seventh section. }

%
\addtocontents{toc}{\let\protect\contentsline\protect\nopagecontentsline}
\part*{Linked Discourses with Anuruddha }
\addcontentsline{toc}{part}{Linked Discourses with Anuruddha }
\markboth{}{}
\addtocontents{toc}{\let\protect\contentsline\protect\oldcontentsline}

%
\addtocontents{toc}{\let\protect\contentsline\protect\nopagecontentsline}
\chapter*{The Chapter on In Private }
\addcontentsline{toc}{chapter}{\tocchapterline{The Chapter on In Private }}
\addtocontents{toc}{\let\protect\contentsline\protect\oldcontentsline}

%
\section*{{\suttatitleacronym SN 52.1}{\suttatitletranslation In Private (1st) }{\suttatitleroot Paṭhamarahogatasutta}}
\addcontentsline{toc}{section}{\tocacronym{SN 52.1} \toctranslation{In Private (1st) } \tocroot{Paṭhamarahogatasutta}}
\markboth{In Private (1st) }{Paṭhamarahogatasutta}
\extramarks{SN 52.1}{SN 52.1}

\scevam{So\marginnote{1.1} I have heard. }At one time Venerable Anuruddha was staying near \textsanskrit{Sāvatthī} in Jeta’s Grove, \textsanskrit{Anāthapiṇḍika}’s monastery. Then as Anuruddha was in private retreat this thought came to his mind: 

“Whoever\marginnote{1.4} has missed out on these four kinds of mindfulness meditation has missed out on the noble path to the complete ending of suffering. Whoever has undertaken these four kinds of mindfulness meditation has undertaken the noble path to the complete ending of suffering.” 

Then\marginnote{2.1} Venerable \textsanskrit{Mahāmoggallāna} knew what Venerable Anuruddha was thinking. As easily as a strong person would extend or contract their arm, he reappeared in front of Anuruddha, and said to him: 

“Reverend\marginnote{2.3} Anuruddha, how do you define the undertaking of the four kinds of mindfulness meditation by a mendicant?” 

“Reverend,\marginnote{3.1} it’s when a mendicant meditates observing the body internally as liable to originate, as liable to vanish, and as liable to originate and vanish—keen, aware, and mindful, rid of desire and aversion for the world. They meditate observing the body externally as liable to originate, as liable to vanish, and as liable to originate and vanish—keen, aware, and mindful, rid of desire and aversion for the world. They meditate observing the body internally and externally as liable to originate, as liable to vanish, and as liable to originate and vanish—keen, aware, and mindful, rid of desire and aversion for the world. 

If\marginnote{4.1} they wish: ‘May I meditate perceiving the repulsive in the unrepulsive,’ that’s what they do. If they wish: ‘May I meditate perceiving the unrepulsive in the repulsive,’ that’s what they do. If they wish: ‘May I meditate perceiving the repulsive in the unrepulsive and the repulsive,’ that’s what they do. If they wish: ‘May I meditate perceiving the unrepulsive in the repulsive and the unrepulsive,’ that’s what they do. If they wish: ‘May I meditate staying equanimous, mindful and aware, rejecting both the repulsive and the unrepulsive,’ that’s what they do. 

They\marginnote{5.1} meditate observing feelings internally … externally … internally and externally as liable to originate, as liable to vanish, and as liable to originate and vanish … 

They\marginnote{7.1} meditate observing the mind internally … externally … internally and externally as liable to originate, as liable to vanish, and as liable to originate and vanish … 

They\marginnote{9.1} meditate observing principles internally … externally … internally and externally as liable to originate, as liable to vanish, and as liable to originate and vanish … 

If\marginnote{10.1} they wish: ‘May I meditate perceiving the repulsive in the unrepulsive,’ that’s what they do. … If they wish: ‘May I meditate staying equanimous, mindful and aware, ignoring both the repulsive and the unrepulsive,’ that’s what they do. 

That’s\marginnote{10.3} how to define the undertaking of the four kinds of mindfulness meditation by a mendicant.” 

%
\section*{{\suttatitleacronym SN 52.2}{\suttatitletranslation In Private (2nd) }{\suttatitleroot Dutiyarahogatasutta}}
\addcontentsline{toc}{section}{\tocacronym{SN 52.2} \toctranslation{In Private (2nd) } \tocroot{Dutiyarahogatasutta}}
\markboth{In Private (2nd) }{Dutiyarahogatasutta}
\extramarks{SN 52.2}{SN 52.2}

At\marginnote{1.1} \textsanskrit{Sāvatthī}. 

Then\marginnote{1.2} as Anuruddha was in private retreat this thought came to his mind: 

“Whoever\marginnote{1.3} has missed out on these four kinds of mindfulness meditation has missed out on the noble path to the complete ending of suffering. Whoever has undertaken these four kinds of mindfulness meditation has undertaken the noble path to the complete ending of suffering.” 

Then\marginnote{2.1} Venerable \textsanskrit{Mahāmoggallāna} knew what Venerable Anuruddha was thinking. As easily as a strong person would extend or contract their arm, he reappeared in front of Anuruddha and said to him: 

“Reverend\marginnote{3.1} Anuruddha, how do you define the undertaking of the four kinds of mindfulness meditation by a mendicant?” 

“Reverend,\marginnote{4.1} it’s when a mendicant meditates by observing an aspect of the body internally—keen, aware, and mindful, rid of desire and aversion for the world. They meditate observing an aspect of the body externally … internally and externally—keen, aware, and mindful, rid of desire and aversion for the world. 

They\marginnote{5.1} meditate observing an aspect of feelings internally … externally … internally and externally … 

They\marginnote{6.1} meditate observing an aspect of the mind internally … externally … internally and externally … 

They\marginnote{7.1} meditate observing an aspect of principles internally … externally … internally and externally—keen, aware, and mindful, rid of desire and aversion for the world. That’s how to define the undertaking of the four kinds of mindfulness meditation by a mendicant.” 

%
\section*{{\suttatitleacronym SN 52.3}{\suttatitletranslation On the Bank of the Sutanu }{\suttatitleroot Sutanusutta}}
\addcontentsline{toc}{section}{\tocacronym{SN 52.3} \toctranslation{On the Bank of the Sutanu } \tocroot{Sutanusutta}}
\markboth{On the Bank of the Sutanu }{Sutanusutta}
\extramarks{SN 52.3}{SN 52.3}

At\marginnote{1.1} one time Venerable Anuruddha was staying near \textsanskrit{Sāvatthī} on the bank of the Sutanu. Then several mendicants went up to Venerable Anuruddha, and exchanged greetings with him. When the greetings and polite conversation were over, they sat down to one side, and said to him: 

“What\marginnote{1.4} things has Venerable Anuruddha developed and cultivated to attain great direct knowledge?” 

“Reverends,\marginnote{2.1} I attained great direct knowledge by developing and cultivating the four kinds of mindfulness meditation. What four? I meditate observing an aspect of the body—keen, aware, and mindful, rid of desire and aversion for the world. I meditate observing an aspect of feelings … mind … principles—keen, aware, and mindful, rid of desire and aversion for the world. I attained great direct knowledge by developing and cultivating these four kinds of mindfulness meditation. 

And\marginnote{2.8} it was by developing and cultivating these four kinds of mindfulness meditation that I directly knew the lower realm as lower, the middle realm as middle, and the higher realm as higher.” 

%
\section*{{\suttatitleacronym SN 52.4}{\suttatitletranslation At Thorny Wood (1st) }{\suttatitleroot Paṭhamakaṇḍakīsutta}}
\addcontentsline{toc}{section}{\tocacronym{SN 52.4} \toctranslation{At Thorny Wood (1st) } \tocroot{Paṭhamakaṇḍakīsutta}}
\markboth{At Thorny Wood (1st) }{Paṭhamakaṇḍakīsutta}
\extramarks{SN 52.4}{SN 52.4}

At\marginnote{1.1} one time the venerables Anuruddha, \textsanskrit{Sāriputta}, and \textsanskrit{Mahāmoggallāna} were staying near \textsanskrit{Sāketa}, in the Thorny Wood. Then in the late afternoon, \textsanskrit{Sāriputta} and \textsanskrit{Mahāmoggallāna} came out of retreat, went to Anuruddha, and exchanged greetings with him. When the greetings and polite conversation were over, they sat down to one side. \textsanskrit{Sāriputta} said to Anuruddha: 

“Reverend\marginnote{1.4} Anuruddha, what things should a trainee mendicant enter and remain in?” 

“Reverend\marginnote{2.1} \textsanskrit{Sāriputta}, a trainee mendicant should enter and remain in the four kinds of mindfulness meditation. What four? It’s when a mendicant meditates by observing an aspect of the body—keen, aware, and mindful, rid of desire and aversion for the world. They meditate observing an aspect of feelings … mind … principles—keen, aware, and mindful, rid of desire and aversion for the world. A trainee mendicant should enter and remain in these four kinds of mindfulness meditation.” 

%
\section*{{\suttatitleacronym SN 52.5}{\suttatitletranslation At Thorny Wood (2nd) }{\suttatitleroot Dutiyakaṇḍakīsutta}}
\addcontentsline{toc}{section}{\tocacronym{SN 52.5} \toctranslation{At Thorny Wood (2nd) } \tocroot{Dutiyakaṇḍakīsutta}}
\markboth{At Thorny Wood (2nd) }{Dutiyakaṇḍakīsutta}
\extramarks{SN 52.5}{SN 52.5}

At\marginnote{1.1} \textsanskrit{Sāketa}. \textsanskrit{Sāriputta} said to Anuruddha: 

“Reverend\marginnote{1.3} Anuruddha, what things should a mendicant who is an adept enter and remain in?” 

“Reverend\marginnote{1.4} \textsanskrit{Sāriputta}, a mendicant who is an adept should enter and remain in the four kinds of mindfulness meditation. What four? It’s when a mendicant meditates by observing an aspect of the body—keen, aware, and mindful, rid of desire and aversion for the world. They meditate observing an aspect of feelings … mind … principles—keen, aware, and mindful, rid of desire and aversion for the world. A mendicant who is an adept should enter and remain in these four kinds of mindfulness meditation.” 

%
\section*{{\suttatitleacronym SN 52.6}{\suttatitletranslation At Thorny Wood (3rd) }{\suttatitleroot Tatiyakaṇḍakīsutta}}
\addcontentsline{toc}{section}{\tocacronym{SN 52.6} \toctranslation{At Thorny Wood (3rd) } \tocroot{Tatiyakaṇḍakīsutta}}
\markboth{At Thorny Wood (3rd) }{Tatiyakaṇḍakīsutta}
\extramarks{SN 52.6}{SN 52.6}

At\marginnote{1.1} \textsanskrit{Sāketa}. \textsanskrit{Sāriputta} said to Anuruddha: 

“What\marginnote{1.3} things has Venerable Anuruddha developed and cultivated to attain great direct knowledge?” 

“Reverend,\marginnote{1.4} I attained great direct knowledge by developing and cultivating the four kinds of mindfulness meditation. What four? I meditate observing an aspect of the body—keen, aware, and mindful, rid of desire and aversion for the world. I meditate observing an aspect of feelings … mind … principles—keen, aware, and mindful, rid of desire and aversion for the world. I attained great direct knowledge by developing and cultivating these four kinds of mindfulness meditation. 

And\marginnote{1.11} it’s because of developing and cultivating these four kinds of mindfulness meditation that I directly know the entire galaxy.” 

%
\section*{{\suttatitleacronym SN 52.7}{\suttatitletranslation The Ending of Craving }{\suttatitleroot Taṇhākkhayasutta}}
\addcontentsline{toc}{section}{\tocacronym{SN 52.7} \toctranslation{The Ending of Craving } \tocroot{Taṇhākkhayasutta}}
\markboth{The Ending of Craving }{Taṇhākkhayasutta}
\extramarks{SN 52.7}{SN 52.7}

At\marginnote{1.1} \textsanskrit{Sāvatthī}. 

There\marginnote{1.2} Venerable Anuruddha addressed the mendicants: “Reverends, mendicants!” 

“Reverend,”\marginnote{1.4} they replied. Anuruddha said this: 

“Reverends,\marginnote{2.1} when these four kinds of mindfulness meditation are developed and cultivated they lead to the ending of craving. What four? It’s when a mendicant meditates by observing an aspect of the body … feelings … mind … principles—keen, aware, and mindful, rid of desire and aversion for the world. When these four kinds of mindfulness meditation are developed and cultivated they lead to the ending of craving.” 

%
\section*{{\suttatitleacronym SN 52.8}{\suttatitletranslation The Frankincense-Tree Hut }{\suttatitleroot Salaḷāgārasutta}}
\addcontentsline{toc}{section}{\tocacronym{SN 52.8} \toctranslation{The Frankincense-Tree Hut } \tocroot{Salaḷāgārasutta}}
\markboth{The Frankincense-Tree Hut }{Salaḷāgārasutta}
\extramarks{SN 52.8}{SN 52.8}

At\marginnote{1.1} one time Venerable Anuruddha was staying near \textsanskrit{Sāvatthī} in the frankincense-tree hut. There Venerable Anuruddha addressed the mendicants: “Reverends, suppose that, although the Ganges river slants, slopes, and inclines to the east, a large crowd were to come along with a spade and basket, saying: ‘We’ll make this Ganges river slant, slope, and incline to the west!’ What do you think, reverends? Would they succeed?” 

“No,\marginnote{1.7} reverend. Why is that? The Ganges river slants, slopes, and inclines to the east. It’s not easy to make it slant, slope, and incline to the west. That large crowd will eventually get weary and frustrated.” 

“In\marginnote{2.1} the same way, while a mendicant develops and cultivates the four kinds of mindfulness meditation, if rulers or their ministers, friends or colleagues, relatives or family should invite them to accept wealth, saying: ‘Please, mister, why let these ocher robes torment you? Why follow the practice of shaving your head and carrying an alms bowl? Come, return to a lesser life, enjoy wealth, and make merit!’ 

It’s\marginnote{3.1} simply impossible for a mendicant who is developing and cultivating the four kinds of mindfulness meditation to resign the training and return to a lesser life. Why is that? Because for a long time that mendicant’s mind has slanted, sloped, and inclined to seclusion. So it’s impossible for them to return to a lesser life. 

And\marginnote{3.4} how does a mendicant develop the four kinds of mindfulness meditation? It’s when a mendicant meditates by observing an aspect of the body … feelings … mind … principles—keen, aware, and mindful, rid of desire and aversion for the world. That’s how a mendicant develops and cultivates the four kinds of mindfulness meditation.” 

%
\section*{{\suttatitleacronym SN 52.9}{\suttatitletranslation In Ambapālī’s Wood }{\suttatitleroot Ambapālivanasutta}}
\addcontentsline{toc}{section}{\tocacronym{SN 52.9} \toctranslation{In Ambapālī’s Wood } \tocroot{Ambapālivanasutta}}
\markboth{In Ambapālī’s Wood }{Ambapālivanasutta}
\extramarks{SN 52.9}{SN 52.9}

At\marginnote{1.1} one time the venerables Anuruddha and \textsanskrit{Sāriputta} were staying near \textsanskrit{Vesālī}, in \textsanskrit{Ambapālī}’s Wood. Then in the late afternoon, \textsanskrit{Sāriputta} came out of retreat, went to Anuruddha, and said to him: 

“Reverend\marginnote{2.1} Anuruddha, your faculties are so very clear, and your complexion is pure and bright. What kind of meditation are you usually practicing these days?” 

“These\marginnote{2.3} days, reverend, I usually meditate with my mind firmly established in the four kinds of mindfulness meditation. What four? I meditate observing an aspect of the body … feelings … mind … principles—keen, aware, and mindful, rid of desire and aversion for the world. These days I usually meditate with my mind firmly established in these four kinds of mindfulness meditation. A mendicant who is perfected—with defilements ended, who has completed the spiritual journey, done what had to be done, laid down the burden, achieved their own true goal, utterly ended the fetters of rebirth, and is rightly freed through enlightenment—usually meditates with their mind firmly established in these four kinds of mindfulness meditation.” 

“We’re\marginnote{3.1} so fortunate, reverend, so very fortunate, to have heard such a dramatic statement in the presence of Venerable Anuruddha.” 

%
\section*{{\suttatitleacronym SN 52.10}{\suttatitletranslation Gravely Ill }{\suttatitleroot Bāḷhagilānasutta}}
\addcontentsline{toc}{section}{\tocacronym{SN 52.10} \toctranslation{Gravely Ill } \tocroot{Bāḷhagilānasutta}}
\markboth{Gravely Ill }{Bāḷhagilānasutta}
\extramarks{SN 52.10}{SN 52.10}

At\marginnote{1.1} one time Venerable Anuruddha was staying near \textsanskrit{Sāvatthī} in the Dark Forest. And he was sick, suffering, gravely ill. Then several mendicants went up to Venerable Anuruddha, and said to him: 

“What\marginnote{2.1} meditation does Venerable Anuruddha practice so that physical pain doesn’t occupy his mind?” 

“Reverends,\marginnote{2.2} I meditate with my mind firmly established in the four kinds of mindfulness meditation so that physical pain doesn’t occupy my mind. What four? I meditate observing an aspect of the body … feelings … mind … principles—keen, aware, and mindful, rid of desire and aversion for the world. I meditate with my mind firmly established in these four kinds of mindfulness meditation so that physical pain doesn’t occupy my mind.” 

%
\addtocontents{toc}{\let\protect\contentsline\protect\nopagecontentsline}
\chapter*{Chapter Two }
\addcontentsline{toc}{chapter}{\tocchapterline{Chapter Two }}
\addtocontents{toc}{\let\protect\contentsline\protect\oldcontentsline}

%
\section*{{\suttatitleacronym SN 52.11}{\suttatitletranslation A Thousand Eons }{\suttatitleroot Kappasahassasutta}}
\addcontentsline{toc}{section}{\tocacronym{SN 52.11} \toctranslation{A Thousand Eons } \tocroot{Kappasahassasutta}}
\markboth{A Thousand Eons }{Kappasahassasutta}
\extramarks{SN 52.11}{SN 52.11}

At\marginnote{1.1} one time Venerable Anuruddha was staying near \textsanskrit{Sāvatthī} in Jeta’s Grove, \textsanskrit{Anāthapiṇḍika}’s monastery. Then several mendicants went up to Venerable Anuruddha, exchanged greetings with him … and said: 

“What\marginnote{2.1} things has Venerable Anuruddha developed and cultivated to attain great direct knowledge?” 

“Reverends,\marginnote{2.2} I attained great direct knowledge by developing and cultivating the four kinds of mindfulness meditation. What four? I meditate observing an aspect of the body … feelings … mind … principles—keen, aware, and mindful, rid of desire and aversion for the world. I attained great direct knowledge by developing and cultivating these four kinds of mindfulness meditation. 

And\marginnote{2.9} it’s because of developing and cultivating these four kinds of mindfulness meditation that I recollect a thousand eons.” 

%
\section*{{\suttatitleacronym SN 52.12}{\suttatitletranslation Psychic Powers }{\suttatitleroot Iddhividhasutta}}
\addcontentsline{toc}{section}{\tocacronym{SN 52.12} \toctranslation{Psychic Powers } \tocroot{Iddhividhasutta}}
\markboth{Psychic Powers }{Iddhividhasutta}
\extramarks{SN 52.12}{SN 52.12}

“…\marginnote{1.1} And it’s because of developing and cultivating these four kinds of mindfulness meditation that I wield the many kinds of psychic power: multiplying myself and becoming one again … controlling the body as far as the \textsanskrit{Brahmā} realm.” 

%
\section*{{\suttatitleacronym SN 52.13}{\suttatitletranslation Clairaudience }{\suttatitleroot Dibbasotasutta}}
\addcontentsline{toc}{section}{\tocacronym{SN 52.13} \toctranslation{Clairaudience } \tocroot{Dibbasotasutta}}
\markboth{Clairaudience }{Dibbasotasutta}
\extramarks{SN 52.13}{SN 52.13}

“…\marginnote{1.1} And it’s because of developing and cultivating these four kinds of mindfulness meditation that, with clairaudience that is purified and superhuman, I hear both kinds of sounds, human and divine, whether near or far.” 

%
\section*{{\suttatitleacronym SN 52.14}{\suttatitletranslation Comprehending the Mind }{\suttatitleroot Cetopariyasutta}}
\addcontentsline{toc}{section}{\tocacronym{SN 52.14} \toctranslation{Comprehending the Mind } \tocroot{Cetopariyasutta}}
\markboth{Comprehending the Mind }{Cetopariyasutta}
\extramarks{SN 52.14}{SN 52.14}

“…\marginnote{1.1} And it’s because of developing and cultivating these four kinds of mindfulness meditation that I understand the minds of other beings and individuals, having comprehended them with my mind. I understand mind with greed as ‘mind with greed’ … I understand unfreed mind as ‘unfreed mind’.” 

%
\section*{{\suttatitleacronym SN 52.15}{\suttatitletranslation Possible }{\suttatitleroot Ṭhānasutta}}
\addcontentsline{toc}{section}{\tocacronym{SN 52.15} \toctranslation{Possible } \tocroot{Ṭhānasutta}}
\markboth{Possible }{Ṭhānasutta}
\extramarks{SN 52.15}{SN 52.15}

“…\marginnote{1.1} And it’s because of developing and cultivating these four kinds of mindfulness meditation that I truly understand the possible as possible and the impossible as impossible.” 

%
\section*{{\suttatitleacronym SN 52.16}{\suttatitletranslation The Results of Deeds Undertaken }{\suttatitleroot Kammasamādānasutta}}
\addcontentsline{toc}{section}{\tocacronym{SN 52.16} \toctranslation{The Results of Deeds Undertaken } \tocroot{Kammasamādānasutta}}
\markboth{The Results of Deeds Undertaken }{Kammasamādānasutta}
\extramarks{SN 52.16}{SN 52.16}

“…\marginnote{1.1} And it’s because of developing and cultivating these four kinds of mindfulness meditation that I truly understand the result of deeds undertaken in the past, future, and present in terms of causes and reasons.” 

%
\section*{{\suttatitleacronym SN 52.17}{\suttatitletranslation Where All Paths of Practice Lead }{\suttatitleroot Sabbatthagāminisutta}}
\addcontentsline{toc}{section}{\tocacronym{SN 52.17} \toctranslation{Where All Paths of Practice Lead } \tocroot{Sabbatthagāminisutta}}
\markboth{Where All Paths of Practice Lead }{Sabbatthagāminisutta}
\extramarks{SN 52.17}{SN 52.17}

“…\marginnote{1.1} And it’s because of developing and cultivating these four kinds of mindfulness meditation that I truly understand where all paths of practice lead.” 

%
\section*{{\suttatitleacronym SN 52.18}{\suttatitletranslation Diverse Elements }{\suttatitleroot Nānādhātusutta}}
\addcontentsline{toc}{section}{\tocacronym{SN 52.18} \toctranslation{Diverse Elements } \tocroot{Nānādhātusutta}}
\markboth{Diverse Elements }{Nānādhātusutta}
\extramarks{SN 52.18}{SN 52.18}

“…\marginnote{1.1} And it’s because of developing and cultivating these four kinds of mindfulness meditation that I truly understand the world with its many and diverse elements.” 

%
\section*{{\suttatitleacronym SN 52.19}{\suttatitletranslation Diverse Beliefs }{\suttatitleroot Nānādhimuttisutta}}
\addcontentsline{toc}{section}{\tocacronym{SN 52.19} \toctranslation{Diverse Beliefs } \tocroot{Nānādhimuttisutta}}
\markboth{Diverse Beliefs }{Nānādhimuttisutta}
\extramarks{SN 52.19}{SN 52.19}

“…\marginnote{1.1} And it’s because of developing and cultivating these four kinds of mindfulness meditation that I truly understand the diverse convictions of sentient beings.” 

%
\section*{{\suttatitleacronym SN 52.20}{\suttatitletranslation Comprehending the Faculties of Others }{\suttatitleroot Indriyaparopariyattasutta}}
\addcontentsline{toc}{section}{\tocacronym{SN 52.20} \toctranslation{Comprehending the Faculties of Others } \tocroot{Indriyaparopariyattasutta}}
\markboth{Comprehending the Faculties of Others }{Indriyaparopariyattasutta}
\extramarks{SN 52.20}{SN 52.20}

“…\marginnote{1.1} And it’s because of developing and cultivating these four kinds of mindfulness meditation that I truly understand the faculties of other sentient beings and other individuals after comprehending them with my mind.” 

%
\section*{{\suttatitleacronym SN 52.21}{\suttatitletranslation Absorptions, Etc. }{\suttatitleroot Jhānādisutta}}
\addcontentsline{toc}{section}{\tocacronym{SN 52.21} \toctranslation{Absorptions, Etc. } \tocroot{Jhānādisutta}}
\markboth{Absorptions, Etc. }{Jhānādisutta}
\extramarks{SN 52.21}{SN 52.21}

“…\marginnote{1.1} And it’s because of developing and cultivating these four kinds of mindfulness meditation that I truly understand corruption, cleansing, and emergence regarding the absorptions, liberations, immersions, and attainments.” 

%
\section*{{\suttatitleacronym SN 52.22}{\suttatitletranslation Past Lives }{\suttatitleroot Pubbenivāsasutta}}
\addcontentsline{toc}{section}{\tocacronym{SN 52.22} \toctranslation{Past Lives } \tocroot{Pubbenivāsasutta}}
\markboth{Past Lives }{Pubbenivāsasutta}
\extramarks{SN 52.22}{SN 52.22}

“…\marginnote{1.1} And it’s because of developing and cultivating these four kinds of mindfulness meditation that I recollect my many kinds of past lives, with features and details.” 

%
\section*{{\suttatitleacronym SN 52.23}{\suttatitletranslation Clairvoyance }{\suttatitleroot Dibbacakkhusutta}}
\addcontentsline{toc}{section}{\tocacronym{SN 52.23} \toctranslation{Clairvoyance } \tocroot{Dibbacakkhusutta}}
\markboth{Clairvoyance }{Dibbacakkhusutta}
\extramarks{SN 52.23}{SN 52.23}

“…\marginnote{1.1} And it’s because of developing and cultivating these four kinds of mindfulness meditation that, with clairvoyance that is purified and superhuman, I understand how sentient beings are reborn according to their deeds.” 

%
\section*{{\suttatitleacronym SN 52.24}{\suttatitletranslation The Ending of Defilements }{\suttatitleroot Āsavakkhayasutta}}
\addcontentsline{toc}{section}{\tocacronym{SN 52.24} \toctranslation{The Ending of Defilements } \tocroot{Āsavakkhayasutta}}
\markboth{The Ending of Defilements }{Āsavakkhayasutta}
\extramarks{SN 52.24}{SN 52.24}

“…\marginnote{1.1} And it’s because of developing and cultivating these four kinds of mindfulness meditation that I realized the undefiled freedom of heart and freedom by wisdom in this very life. And I live having realized it with my own insight due to the ending of defilements.” 

\scendsutta{The Linked Discourses with Anuruddha are the eighth section. }

%
\addtocontents{toc}{\let\protect\contentsline\protect\nopagecontentsline}
\part*{Linked Discourses on Absorption }
\addcontentsline{toc}{part}{Linked Discourses on Absorption }
\markboth{}{}
\addtocontents{toc}{\let\protect\contentsline\protect\oldcontentsline}

%
\addtocontents{toc}{\let\protect\contentsline\protect\nopagecontentsline}
\chapter*{The Chapter of Abbreviated Texts on the Ganges }
\addcontentsline{toc}{chapter}{\tocchapterline{The Chapter of Abbreviated Texts on the Ganges }}
\addtocontents{toc}{\let\protect\contentsline\protect\oldcontentsline}

%
\section*{{\suttatitleacronym SN 53.1–12}{\suttatitletranslation Absorptions, Etc. }{\suttatitleroot Gaṅgāpeyyālavagga}}
\addcontentsline{toc}{section}{\tocacronym{SN 53.1–12} \toctranslation{Absorptions, Etc. } \tocroot{Gaṅgāpeyyālavagga}}
\markboth{Absorptions, Etc. }{Gaṅgāpeyyālavagga}
\extramarks{SN 53.1–12}{SN 53.1–12}

At\marginnote{1.1} \textsanskrit{Sāvatthī}. 

“Mendicants,\marginnote{1.3} there are these four absorptions. What four? 

It’s\marginnote{1.5} when a mendicant, quite secluded from sensual pleasures, secluded from unskillful qualities, enters and remains in the first absorption, which has the rapture and bliss born of seclusion, while placing the mind and keeping it connected. 

As\marginnote{1.6} the placing of the mind and keeping it connected are stilled, they enter and remain in the second absorption, which has the rapture and bliss born of immersion, with internal clarity and confidence, and unified mind, without placing the mind and keeping it connected. 

And\marginnote{1.7} with the fading away of rapture, they enter and remain in the third absorption, where they meditate with equanimity, mindful and aware, personally experiencing the bliss of which the noble ones declare, ‘Equanimous and mindful, one meditates in bliss.’ 

Giving\marginnote{1.8} up pleasure and pain, and ending former happiness and sadness, they enter and remain in the fourth absorption, without pleasure or pain, with pure equanimity and mindfulness. 

These\marginnote{1.9} are the four absorptions. 

The\marginnote{2.1} Ganges river slants, slopes, and inclines to the east. In the same way, a mendicant who develops and cultivates the four absorptions slants, slopes, and inclines to extinguishment. 

And\marginnote{2.3} how does a mendicant who develops and cultivates the four absorptions slant, slope, and incline to extinguishment? 

It’s\marginnote{2.4} when a mendicant, quite secluded from sensual pleasures, secluded from unskillful qualities, enters and remains in the first absorption, which has the rapture and bliss born of seclusion, while placing the mind and keeping it connected. 

As\marginnote{2.5} the placing of the mind and keeping it connected are stilled, they enter and remain in the second absorption … third absorption … fourth absorption. 

That’s\marginnote{2.8} how a mendicant who develops and cultivates the four absorptions slants, slopes, and inclines to extinguishment.” 

\scendsection{(To be expanded for each of the different rivers as in SN 45.91–102.) }

\begin{quotation}%
Six\marginnote{3.1} on slanting to the east, \\
and six on slanting to the ocean; \\
these two sixes make twelve, \\
and that’s how this chapter is recited. 

%
\end{quotation}

%
\addtocontents{toc}{\let\protect\contentsline\protect\nopagecontentsline}
\chapter*{The Chapter on Diligence }
\addcontentsline{toc}{chapter}{\tocchapterline{The Chapter on Diligence }}
\addtocontents{toc}{\let\protect\contentsline\protect\oldcontentsline}

%
\section*{{\suttatitleacronym SN 53.13–22}{\suttatitletranslation Diligence }{\suttatitleroot Appamādavagga}}
\addcontentsline{toc}{section}{\tocacronym{SN 53.13–22} \toctranslation{Diligence } \tocroot{Appamādavagga}}
\markboth{Diligence }{Appamādavagga}
\extramarks{SN 53.13–22}{SN 53.13–22}

(To\marginnote{1.1} be expanded as in the chapter on diligence at SN 45.139–148.) 

\begin{quotation}%
The\marginnote{2.1} Realized One, footprint, roof peak, \\
roots, heartwood, jasmine, \\
monarch, sun and moon, \\
and cloth is the tenth. 

%
\end{quotation}

%
\addtocontents{toc}{\let\protect\contentsline\protect\nopagecontentsline}
\chapter*{The Chapter on Hard Work }
\addcontentsline{toc}{chapter}{\tocchapterline{The Chapter on Hard Work }}
\addtocontents{toc}{\let\protect\contentsline\protect\oldcontentsline}

%
\section*{{\suttatitleacronym SN 53.23–34}{\suttatitletranslation Hard Work }{\suttatitleroot Balakaraṇīyavagga}}
\addcontentsline{toc}{section}{\tocacronym{SN 53.23–34} \toctranslation{Hard Work } \tocroot{Balakaraṇīyavagga}}
\markboth{Hard Work }{Balakaraṇīyavagga}
\extramarks{SN 53.23–34}{SN 53.23–34}

(To\marginnote{1.1} be expanded as in the chapter on hard work at SN 45.149–160.) 

\begin{quotation}%
Hard\marginnote{2.1} work, seeds, and dragons, \\
a tree, a pot, and a spike, \\
the sky, and two on clouds, \\
a ship, a guest house, and a river. 

%
\end{quotation}

%
\addtocontents{toc}{\let\protect\contentsline\protect\nopagecontentsline}
\chapter*{The Chapter on Searches }
\addcontentsline{toc}{chapter}{\tocchapterline{The Chapter on Searches }}
\addtocontents{toc}{\let\protect\contentsline\protect\oldcontentsline}

%
\section*{{\suttatitleacronym SN 53.35–44}{\suttatitletranslation Searches }{\suttatitleroot Esanāvagga}}
\addcontentsline{toc}{section}{\tocacronym{SN 53.35–44} \toctranslation{Searches } \tocroot{Esanāvagga}}
\markboth{Searches }{Esanāvagga}
\extramarks{SN 53.35–44}{SN 53.35–44}

(To\marginnote{1.1} be expanded as in the chapter on searches at SN 45.161–170.) 

\begin{quotation}%
Searches,\marginnote{2.1} discriminations, defilements, \\
states of existence, three kinds of suffering, \\
barrenness, stains, and troubles, \\
feelings, craving, and thirst. 

%
\end{quotation}

%
\addtocontents{toc}{\let\protect\contentsline\protect\nopagecontentsline}
\chapter*{The Chapter on Floods }
\addcontentsline{toc}{chapter}{\tocchapterline{The Chapter on Floods }}
\addtocontents{toc}{\let\protect\contentsline\protect\oldcontentsline}

%
\section*{{\suttatitleacronym SN 53.45–54}{\suttatitletranslation Floods, etc. }{\suttatitleroot Oghavagga}}
\addcontentsline{toc}{section}{\tocacronym{SN 53.45–54} \toctranslation{Floods, etc. } \tocroot{Oghavagga}}
\markboth{Floods, etc. }{Oghavagga}
\extramarks{SN 53.45–54}{SN 53.45–54}

“Mendicants,\marginnote{1.1} there are five higher fetters. What five? Desire for rebirth in the realm of luminous form, desire for rebirth in the formless realm, conceit, restlessness, and ignorance. These are the five higher fetters. 

The\marginnote{1.5} four absorptions should be developed for the direct knowledge, complete understanding, finishing, and giving up of these five higher fetters. What four? It’s when a mendicant, quite secluded from sensual pleasures, secluded from unskillful qualities, enters and remains in the first absorption, which has the rapture and bliss born of seclusion, while placing the mind and keeping it connected. As the placing of the mind and keeping it connected are stilled, they enter and remain in the second absorption … third absorption … fourth absorption. These four absorptions should be developed for the direct knowledge, complete understanding, finishing, and giving up of these five higher fetters.” 

\scendsection{(To be expanded as in the Linked Discourses on the Path at SN 45.171–180.) }

\begin{quotation}%
Floods,\marginnote{2.1} bonds, grasping, \\
ties, and underlying tendencies, \\
kinds of sensual stimulation, hindrances, \\
aggregates, and fetters high and low. 

%
\end{quotation}

\scendsutta{The Linked Discourses on Absorption are the ninth section. }

%
\addtocontents{toc}{\let\protect\contentsline\protect\nopagecontentsline}
\part*{Linked Discourses on Breath Meditation }
\addcontentsline{toc}{part}{Linked Discourses on Breath Meditation }
\markboth{}{}
\addtocontents{toc}{\let\protect\contentsline\protect\oldcontentsline}

%
\addtocontents{toc}{\let\protect\contentsline\protect\nopagecontentsline}
\chapter*{The Chapter on One Thing }
\addcontentsline{toc}{chapter}{\tocchapterline{The Chapter on One Thing }}
\addtocontents{toc}{\let\protect\contentsline\protect\oldcontentsline}

%
\section*{{\suttatitleacronym SN 54.1}{\suttatitletranslation One Thing }{\suttatitleroot Ekadhammasutta}}
\addcontentsline{toc}{section}{\tocacronym{SN 54.1} \toctranslation{One Thing } \tocroot{Ekadhammasutta}}
\markboth{One Thing }{Ekadhammasutta}
\extramarks{SN 54.1}{SN 54.1}

At\marginnote{1.1} \textsanskrit{Sāvatthī}. 

“Mendicants,\marginnote{1.3} when one thing is developed and cultivated it’s very fruitful and beneficial. What one thing? Mindfulness of breathing. 

And\marginnote{1.6} how is mindfulness of breathing developed and cultivated to be very fruitful and beneficial? 

It’s\marginnote{1.7} when a mendicant has gone to a wilderness, or to the root of a tree, or to an empty hut. They sit down cross-legged, with their body straight, and establish mindfulness right there. 

Just\marginnote{1.8} mindful, they breathe in. Mindful, they breathe out. 

When\marginnote{2.1} breathing in heavily they know: ‘I’m breathing in heavily.’ When breathing out heavily they know: ‘I’m breathing out heavily.’ When breathing in lightly they know: ‘I’m breathing in lightly.’ When breathing out lightly they know: ‘I’m breathing out lightly.’ They practice like this: ‘I’ll breathe in experiencing the whole body.’ They practice like this: ‘I’ll breathe out experiencing the whole body.’ They practice like this: ‘I’ll breathe in stilling physical processes.’ They practice like this: ‘I’ll breathe out stilling physical processes.’ 

They\marginnote{3.1} practice like this: ‘I’ll breathe in experiencing rapture.’ They practice like this: ‘I’ll breathe out experiencing rapture.’ They practice like this: ‘I’ll breathe in experiencing bliss.’ They practice like this: ‘I’ll breathe out experiencing bliss.’ They practice like this: ‘I’ll breathe in experiencing mental processes.’ They practice like this: ‘I’ll breathe out experiencing mental processes.’ They practice like this: ‘I’ll breathe in stilling mental processes.’ They practice like this: ‘I’ll breathe out stilling mental processes.’ 

They\marginnote{4.1} practice like this: ‘I’ll breathe in experiencing the mind.’ They practice like this: ‘I’ll breathe out experiencing the mind.’ They practice like this: ‘I’ll breathe in gladdening the mind.’ They practice like this: ‘I’ll breathe out gladdening the mind.’ They practice like this: ‘I’ll breathe in immersing the mind in \textsanskrit{samādhi}.’ They practice like this: ‘I’ll breathe out immersing the mind in \textsanskrit{samādhi}.’ They practice like this: ‘I’ll breathe in freeing the mind.’ They practice like this: ‘I’ll breathe out freeing the mind.’ They practice like this: ‘I’ll breathe in observing impermanence.’ They practice like this: ‘I’ll breathe out observing impermanence.’ 

They\marginnote{5.1} practice like this: ‘I’ll breathe in observing fading away.’ They practice like this: ‘I’ll breathe out observing fading away.’ They practice like this: ‘I’ll breathe in observing cessation.’ They practice like this: ‘I’ll breathe out observing cessation.’ They practice like this: ‘I’ll breathe in observing letting go.’ They practice like this: ‘I’ll breathe out observing letting go.’ 

Mindfulness\marginnote{6.1} of breathing, when developed and cultivated in this way, is very fruitful and beneficial.” 

%
\section*{{\suttatitleacronym SN 54.2}{\suttatitletranslation Awakening Factors }{\suttatitleroot Bojjhaṅgasutta}}
\addcontentsline{toc}{section}{\tocacronym{SN 54.2} \toctranslation{Awakening Factors } \tocroot{Bojjhaṅgasutta}}
\markboth{Awakening Factors }{Bojjhaṅgasutta}
\extramarks{SN 54.2}{SN 54.2}

“Mendicants,\marginnote{1.1} when mindfulness of breathing is developed and cultivated it’s very fruitful and beneficial. And how is mindfulness of breathing developed and cultivated to be very fruitful and beneficial? It’s when a mendicant develops mindfulness of breathing together with the awakening factors of mindfulness, investigation of principles, energy, rapture, tranquility, immersion, and equanimity, which rely on seclusion, fading away, and cessation, and ripen as letting go. Mindfulness of breathing, when developed and cultivated in this way, is very fruitful and beneficial.” 

%
\section*{{\suttatitleacronym SN 54.3}{\suttatitletranslation Plain Version }{\suttatitleroot Suddhikasutta}}
\addcontentsline{toc}{section}{\tocacronym{SN 54.3} \toctranslation{Plain Version } \tocroot{Suddhikasutta}}
\markboth{Plain Version }{Suddhikasutta}
\extramarks{SN 54.3}{SN 54.3}

“Mendicants,\marginnote{1.1} when mindfulness of breathing is developed and cultivated it’s very fruitful and beneficial. And how is mindfulness of breathing developed and cultivated to be very fruitful and beneficial? 

It’s\marginnote{1.3} when a mendicant has gone to a wilderness, or to the root of a tree, or to an empty hut. They sit down cross-legged, with their body straight, and establish mindfulness right there. 

Just\marginnote{1.4} mindful, they breathe in. Mindful, they breathe out. … 

They\marginnote{1.5} practice like this: ‘I’ll breathe in observing letting go.’ They practice like this: ‘I’ll breathe out observing letting go.’ 

Mindfulness\marginnote{1.6} of breathing, when developed and cultivated in this way, is very fruitful and beneficial.” 

%
\section*{{\suttatitleacronym SN 54.4}{\suttatitletranslation Fruits (1st) }{\suttatitleroot Paṭhamaphalasutta}}
\addcontentsline{toc}{section}{\tocacronym{SN 54.4} \toctranslation{Fruits (1st) } \tocroot{Paṭhamaphalasutta}}
\markboth{Fruits (1st) }{Paṭhamaphalasutta}
\extramarks{SN 54.4}{SN 54.4}

“Mendicants,\marginnote{1.1} when mindfulness of breathing is developed and cultivated it’s very fruitful and beneficial. And how is mindfulness of breathing developed and cultivated to be very fruitful and beneficial? 

It’s\marginnote{1.3} when a mendicant has gone to a wilderness, or to the root of a tree, or to an empty hut. They sit down cross-legged, with their body straight, and establish mindfulness right there. 

Just\marginnote{1.4} mindful, they breathe in. Mindful, they breathe out. … 

They\marginnote{1.5} practice like this: ‘I’ll breathe in observing letting go.’ They practice like this: ‘I’ll breathe out observing letting go.’ 

Mindfulness\marginnote{1.6} of breathing, when developed and cultivated in this way, is very fruitful and beneficial. When mindfulness of breathing is developed and cultivated in this way you can expect one of two results: enlightenment in the present life, or if there’s something left over, non-return.” 

%
\section*{{\suttatitleacronym SN 54.5}{\suttatitletranslation Fruits (2nd) }{\suttatitleroot Dutiyaphalasutta}}
\addcontentsline{toc}{section}{\tocacronym{SN 54.5} \toctranslation{Fruits (2nd) } \tocroot{Dutiyaphalasutta}}
\markboth{Fruits (2nd) }{Dutiyaphalasutta}
\extramarks{SN 54.5}{SN 54.5}

“Mendicants,\marginnote{1.1} when mindfulness of breathing is developed and cultivated it’s very fruitful and beneficial. And how is mindfulness of breathing developed and cultivated to be very fruitful and beneficial? 

It’s\marginnote{1.3} when a mendicant has gone to a wilderness, or to the root of a tree, or to an empty hut. They sit down cross-legged, with their body straight, and establish mindfulness right there. 

Just\marginnote{1.4} mindful, they breathe in. Mindful, they breathe out. … 

They\marginnote{1.5} practice like this: ‘I’ll breathe in observing letting go.’ They practice like this: ‘I’ll breathe out observing letting go.’ 

Mindfulness\marginnote{1.6} of breathing, when developed and cultivated in this way, is very fruitful and beneficial. 

When\marginnote{2.1} mindfulness of breathing is developed and cultivated in this way you can expect seven fruits and benefits. What seven? You attain enlightenment early on in this very life. If not, you attain enlightenment at the time of death. If not, with the ending of the five lower fetters you’re extinguished in between one life and the next … you’re extinguished upon landing … you’re extinguished without extra effort … you’re extinguished with extra effort … you head upstream, going to the \textsanskrit{Akaniṭṭha} realm … When mindfulness of breathing is developed and cultivated in this way you can expect these seven fruits and benefits.” 

%
\section*{{\suttatitleacronym SN 54.6}{\suttatitletranslation With Ariṭṭha }{\suttatitleroot Ariṭṭhasutta}}
\addcontentsline{toc}{section}{\tocacronym{SN 54.6} \toctranslation{With Ariṭṭha } \tocroot{Ariṭṭhasutta}}
\markboth{With Ariṭṭha }{Ariṭṭhasutta}
\extramarks{SN 54.6}{SN 54.6}

At\marginnote{1.1} \textsanskrit{Sāvatthī}. 

There\marginnote{1.2} the Buddha … said: 

“Mendicants,\marginnote{1.3} do you develop mindfulness of breathing?” When he said this, Venerable \textsanskrit{Ariṭṭha} said to him: 

“Sir,\marginnote{1.5} I develop mindfulness of breathing.” 

“But\marginnote{1.6} mendicant, how do you develop it?” 

“Sir,\marginnote{1.7} I’ve given up desire for sensual pleasures of the past. I’m rid of desire for sensual pleasures of the future. And I have eliminated perception of repulsion regarding phenomena internally and externally. Just mindful, I will breathe in. Mindful, I will breathe out. That’s how I develop mindfulness of breathing.” 

“That\marginnote{2.1} is mindfulness of breathing, \textsanskrit{Ariṭṭha}; I don’t deny it. But as to how mindfulness of breathing is fulfilled in detail, listen and pay close attention, I will speak.” 

“Yes,\marginnote{2.4} sir,” \textsanskrit{Ariṭṭha} replied. The Buddha said this: 

“And\marginnote{3.1} how is mindfulness of breathing fulfilled in detail? It’s when a mendicant has gone to a wilderness, or to the root of a tree, or to an empty hut. They sit down cross-legged, with their body straight, and establish mindfulness right there. Just mindful, they breathe in. Mindful, they breathe out. When breathing in heavily they know: ‘I’m breathing in heavily.’ When breathing out heavily they know: ‘I’m breathing out heavily.’ … They practice like this: ‘I’ll breathe in observing letting go.’ They practice like this: ‘I’ll breathe out observing letting go.’ This is how mindfulness of breathing is fulfilled in detail.” 

%
\section*{{\suttatitleacronym SN 54.7}{\suttatitletranslation About Mahākappina }{\suttatitleroot Mahākappinasutta}}
\addcontentsline{toc}{section}{\tocacronym{SN 54.7} \toctranslation{About Mahākappina } \tocroot{Mahākappinasutta}}
\markboth{About Mahākappina }{Mahākappinasutta}
\extramarks{SN 54.7}{SN 54.7}

At\marginnote{1.1} \textsanskrit{Sāvatthī}. 

Now\marginnote{1.2} at that time Venerable \textsanskrit{Mahākappina} was sitting not far from the Buddha, cross-legged, with his body straight, and mindfulness established right there. The Buddha saw him, and addressed the mendicants: 

“Mendicants,\marginnote{2.1} do you see any disturbance or trembling in that mendicant’s body?” 

“Sir,\marginnote{2.2} whenever we see that mendicant meditating—whether in the middle of the \textsanskrit{Saṅgha} or alone in private—we never see any disturbance or trembling in his body.” 

“Mendicants,\marginnote{3.1} when an immersion has been developed and cultivated there’s no disturbance or trembling of the body or mind. That mendicant gets such immersion when he wants, without trouble or difficulty. And what is that immersion? 

When\marginnote{4.1} immersion due to mindfulness of breathing has been developed and cultivated there’s no disturbance or trembling of the body or mind. And how is immersion due to mindfulness of breathing developed and cultivated in such a way? 

It’s\marginnote{5.1} when a mendicant—gone to a wilderness, or to the root of a tree, or to an empty hut—sits down cross-legged, with their body straight, and establishes their mindfulness right there. Just mindful, they breathe in. Mindful, they breathe out. … They practice like this: ‘I’ll breathe in observing letting go.’ They practice like this: ‘I’ll breathe out observing letting go.’ 

That’s\marginnote{5.4} how immersion due to mindfulness of breathing is developed and cultivated so that there’s no disturbance or trembling of the body or mind.” 

%
\section*{{\suttatitleacronym SN 54.8}{\suttatitletranslation The Simile of the Lamp }{\suttatitleroot Padīpopamasutta}}
\addcontentsline{toc}{section}{\tocacronym{SN 54.8} \toctranslation{The Simile of the Lamp } \tocroot{Padīpopamasutta}}
\markboth{The Simile of the Lamp }{Padīpopamasutta}
\extramarks{SN 54.8}{SN 54.8}

“Mendicants,\marginnote{1.1} when immersion due to mindfulness of breathing is developed and cultivated it’s very fruitful and beneficial. How so? 

It’s\marginnote{2.1} when a mendicant has gone to a wilderness, or to the root of a tree, or to an empty hut. They sit down cross-legged, with their body straight, and establish mindfulness right there. Just mindful, they breathe in. Mindful, they breathe out. When breathing in heavily they know: ‘I’m breathing in heavily.’ When breathing out heavily they know: ‘I’m breathing out heavily.’ … They practice like this: ‘I’ll breathe in observing letting go.’ They practice like this: ‘I’ll breathe out observing letting go.’ That’s how immersion due to mindfulness of breathing, when developed and cultivated, is very fruitful and beneficial. 

Before\marginnote{3.1} my awakening—when I was still unawakened but intent on awakening—I too usually practiced this kind of meditation. And while I was usually practicing this kind of meditation neither my body nor my eyes became fatigued. And my mind was freed from defilements by not grasping. 

Now,\marginnote{4.1} a mendicant might wish: ‘May neither my body nor my eyes became fatigued. And may my mind be freed from grasping without defilements.’ So let them closely focus on this immersion due to mindfulness of breathing. 

Now,\marginnote{5.1} a mendicant might wish: ‘May I give up memories and thoughts of the lay life.’ So let them closely focus on this immersion due to mindfulness of breathing. 

Now,\marginnote{6.1} a mendicant might wish: ‘May I meditate perceiving the repulsive in the unrepulsive.’ So let them closely focus on this immersion due to mindfulness of breathing. 

Now,\marginnote{7.1} a mendicant might wish: ‘May I meditate perceiving the unrepulsive in the repulsive.’ So let them closely focus on this immersion due to mindfulness of breathing. 

Now,\marginnote{8.1} a mendicant might wish: ‘May I meditate perceiving the repulsive in the unrepulsive and the repulsive.’ So let them closely focus on this immersion due to mindfulness of breathing. 

Now,\marginnote{9.1} a mendicant might wish: ‘May I meditate perceiving the unrepulsive in the repulsive and the unrepulsive.’ So let them closely focus on this immersion due to mindfulness of breathing. 

Now,\marginnote{10.1} a mendicant might wish: ‘May I meditate staying equanimous, mindful and aware, rejecting both the repulsive and the unrepulsive.’ So let them closely focus on this immersion due to mindfulness of breathing. 

Now,\marginnote{11.1} a mendicant might wish: ‘Quite secluded from sensual pleasures, secluded from unskillful qualities, may I enter and remain in the first absorption, which has the rapture and bliss born of seclusion, while placing the mind and keeping it connected.’ So let them closely focus on this immersion due to mindfulness of breathing. 

Now,\marginnote{12.1} a mendicant might wish: ‘As the placing of the mind and keeping it connected are stilled, may I enter and remain in the second absorption, which has the rapture and bliss born of immersion, with internal clarity and confidence, and unified mind, without placing the mind and keeping it connected.’ So let them closely focus on this immersion due to mindfulness of breathing. 

Now,\marginnote{13.1} a mendicant might wish: ‘With the fading away of rapture, may I enter and remain in the third absorption, where I will meditate with equanimity, mindful and aware, personally experiencing the bliss of which the noble ones declare, “Equanimous and mindful, one meditates in bliss.”’ So let them closely focus on this immersion due to mindfulness of breathing. 

Now,\marginnote{14.1} a mendicant might wish: ‘With the giving up of pleasure and pain, and the ending of former happiness and sadness, may I enter and remain in the fourth absorption, without pleasure or pain, with pure equanimity and mindfulness.’ So let them closely focus on this immersion due to mindfulness of breathing. 

Now,\marginnote{15.1} a mendicant might wish: ‘Going totally beyond perceptions of form, with the ending of perceptions of impingement, not focusing on perceptions of diversity, aware that “space is infinite”, may I enter and remain in the dimension of infinite space.’ So let them closely focus on this immersion due to mindfulness of breathing. 

Now,\marginnote{16.1} a mendicant might wish: ‘Going totally beyond the dimension of infinite space, aware that “consciousness is infinite”, may I enter and remain in the dimension of infinite consciousness.’ So let them closely focus on this immersion due to mindfulness of breathing. 

Now,\marginnote{17.1} a mendicant might wish: ‘Going totally beyond the dimension of infinite consciousness, aware that “there is nothing at all”, may I enter and remain in the dimension of nothingness.’ So let them closely focus on this immersion due to mindfulness of breathing. 

Now,\marginnote{18.1} a mendicant might wish: ‘Going totally beyond the dimension of nothingness, may I enter and remain in the dimension of neither perception nor non-perception.’ So let them closely focus on this immersion due to mindfulness of breathing. 

Now,\marginnote{19.1} a mendicant might wish: ‘Going totally beyond the dimension of neither perception nor non-perception, may I enter and remain in the cessation of perception and feeling.’ So let them closely focus on this immersion due to mindfulness of breathing. 

When\marginnote{20.1} mindfulness of breathing has been developed and cultivated in this way, if they feel a pleasant feeling, they understand that it’s impermanent, that they’re not attached to it, and that they don’t take pleasure in it. If they feel a painful feeling, they understand that it’s impermanent, that they’re not attached to it, and that they don’t take pleasure in it. If they feel a neutral feeling, they understand that it’s impermanent, that they’re not attached to it, and that they don’t take pleasure in it. 

If\marginnote{21.1} they feel a pleasant feeling, they feel it detached. If they feel a painful feeling, they feel it detached. If they feel a neutral feeling, they feel it detached. Feeling the end of the body approaching, they understand: ‘I feel the end of the body approaching.’ Feeling the end of life approaching, they understand: ‘I feel the end of life approaching.’ They understand: ‘When my body breaks up and my life has come to an end, everything that’s felt, since I no longer take pleasure in it, will become cool right here.’ 

Suppose\marginnote{22.1} an oil lamp depended on oil and a wick to burn. As the oil and the wick are used up, it would be extinguished due to lack of fuel. In the same way, feeling the end of the body approaching, they understand: ‘I feel the end of the body approaching.’ Feeling the end of life approaching, they understand: ‘I feel the end of life approaching.’ They understand: ‘When my body breaks up and my life has come to an end, everything that’s felt, since I no longer take pleasure in it, will become cool right here.’” 

%
\section*{{\suttatitleacronym SN 54.9}{\suttatitletranslation At Vesālī }{\suttatitleroot Vesālīsutta}}
\addcontentsline{toc}{section}{\tocacronym{SN 54.9} \toctranslation{At Vesālī } \tocroot{Vesālīsutta}}
\markboth{At Vesālī }{Vesālīsutta}
\extramarks{SN 54.9}{SN 54.9}

\scevam{So\marginnote{1.1} I have heard. }At one time the Buddha was staying near \textsanskrit{Vesālī}, at the Great Wood, in the hall with the peaked roof. Now at that time the Buddha spoke in many ways to the mendicants about the meditation on ugliness. He praised the meditation on ugliness and its development. 

Then\marginnote{2.1} the Buddha said to the mendicants, “Mendicants, I wish to go on retreat for a fortnight. No-one should approach me, except for the one who brings my almsfood.” 

“Yes,\marginnote{2.4} sir,” replied those mendicants. And no-one approached him, except for the one who brought the almsfood. 

Then\marginnote{3.1} those mendicants thought, “The Buddha spoke in many ways about the meditation on ugliness. He praised the meditation on ugliness and its development.” They committed themselves to developing the many different facets of the meditation on ugliness. Becoming horrified, repelled, and disgusted with this body, they looked for someone to slit their wrists. Each day ten, twenty, or thirty mendicants slit their wrists. 

Then\marginnote{4.1} after a fortnight had passed, the Buddha came out of retreat and addressed Ānanda, “Ānanda, why does the mendicant \textsanskrit{Saṅgha} seem so diminished?” 

Ānanda\marginnote{4.3} told the Buddha all that had happened, and said, “Sir, please explain another way for the mendicant \textsanskrit{Saṅgha} to get enlightened.” 

“Well\marginnote{5.1} then, Ānanda, gather all the mendicants staying in the vicinity of \textsanskrit{Vesālī} together in the assembly hall.” 

“Yes,\marginnote{5.2} sir,” replied Ānanda. He did what the Buddha asked, went up to him, and said, “Sir, the mendicant \textsanskrit{Saṅgha} has assembled. Please, sir, come at your convenience.” 

Then\marginnote{6.1} the Buddha went to the assembly hall, sat down on the seat spread out, and addressed the mendicants: 

“Mendicants,\marginnote{6.3} when this immersion due to mindfulness of breathing is developed and cultivated it’s peaceful and sublime, a deliciously pleasant meditation. And it disperses and settles unskillful qualities on the spot whenever they arise. 

In\marginnote{7.1} the last month of summer, when the dust and dirt is stirred up, a large sudden storm disperses and settles it on the spot. 

In\marginnote{7.2} the same way, when this immersion due to mindfulness of breathing is developed and cultivated it’s peaceful and sublime, a deliciously pleasant meditation. And it disperses and settles unskillful qualities on the spot whenever they arise. And how is it so developed and cultivated? 

It’s\marginnote{8.1} when a mendicant—gone to a wilderness, or to the root of a tree, or to an empty hut—sits down cross-legged, with their body straight, and focuses their mindfulness right there. 

Just\marginnote{8.2} mindful, they breathe in. Mindful, they breathe out. … 

They\marginnote{8.3} practice like this: ‘I’ll breathe in observing letting go.’ They practice like this: ‘I’ll breathe out observing letting go.’ 

That’s\marginnote{8.4} how this immersion due to mindfulness of breathing is developed and cultivated so that it’s peaceful and sublime, a deliciously pleasant meditation. And it disperses and settles unskillful qualities on the spot whenever they arise.” 

%
\section*{{\suttatitleacronym SN 54.10}{\suttatitletranslation With Kimbila }{\suttatitleroot Kimilasutta}}
\addcontentsline{toc}{section}{\tocacronym{SN 54.10} \toctranslation{With Kimbila } \tocroot{Kimilasutta}}
\markboth{With Kimbila }{Kimilasutta}
\extramarks{SN 54.10}{SN 54.10}

\scevam{So\marginnote{1.1} I have heard. }At one time the Buddha was staying near \textsanskrit{Kimbilā} in the Freshwater Mangrove Wood. Then the Buddha said to Venerable Kimbila, “Kimbila, how is immersion due to mindfulness of breathing developed and cultivated so that it is very fruitful and beneficial?” 

When\marginnote{2.1} he said this, Kimbila kept silent. 

For\marginnote{2.2} a second time … 

And\marginnote{2.3} for a third time, the Buddha said to him, “How is immersion due to mindfulness of breathing developed and cultivated so that it is very fruitful and beneficial?” And a second time and a third time Kimbila kept silent. 

When\marginnote{3.1} he said this, Venerable Ānanda said to the Buddha, “Now is the time, Blessed One! Now is the time, Holy One! Let the Buddha speak on immersion due to mindfulness of breathing. The mendicants will listen and remember it.” 

“Well\marginnote{4.1} then, Ānanda, listen and pay close attention, I will speak.” 

“Yes,\marginnote{4.2} sir,” Ānanda replied. The Buddha said this: 

“Ānanda,\marginnote{4.4} how is immersion due to mindfulness of breathing developed and cultivated so that it is very fruitful and beneficial? It’s when a mendicant has gone to a wilderness, or to the root of a tree, or to an empty hut, sits down cross-legged, with their body straight, and establishes mindfulness right there. 

Just\marginnote{4.6} mindful, they breathe in. Mindful, they breathe out. … 

They\marginnote{4.7} practice like this: ‘I’ll breathe in observing letting go.’ They practice like this: ‘I’ll breathe out observing letting go.’ That’s how immersion due to mindfulness of breathing, when developed and cultivated, is very fruitful and beneficial. 

When\marginnote{5.1} a mendicant is breathing in heavily they know: ‘I’m breathing in heavily.’ When breathing out heavily they know: ‘I’m breathing out heavily.’ When breathing in lightly they know: ‘I’m breathing in lightly.’ When breathing out lightly they know: ‘I’m breathing out lightly.’ They practice like this: ‘I’ll breathe in experiencing the whole body.’ They practice like this: ‘I’ll breathe out experiencing the whole body.’ They practice like this: ‘I’ll breathe in stilling the physical process.’ They practice like this: ‘I’ll breathe out stilling the physical process.’ At such a time a mendicant is meditating by observing an aspect of the body—keen, aware, and mindful, rid of desire and aversion for the world. Why is that? Because the breath is a certain aspect of the body, I say. Therefore, at such a time a mendicant is meditating by observing an aspect of the body—keen, aware, and mindful, rid of desire and aversion for the world. 

There’s\marginnote{6.1} a time when a mendicant practices like this: ‘I’ll breathe in experiencing rapture.’ They practice like this: ‘I’ll breathe out experiencing rapture.’ They practice like this: ‘I’ll breathe in experiencing bliss.’ They practice like this: ‘I’ll breathe out experiencing bliss.’ They practice like this: ‘I’ll breathe in experiencing the mental processes.’ They practice like this: ‘I’ll breathe out experiencing the mental processes.’ They practice like this: ‘I’ll breathe in stilling the mental processes.’ They practice like this: ‘I’ll breathe out stilling the mental processes.’ At such a time a mendicant is meditating by observing an aspect of feelings—keen, aware, and mindful, rid of desire and aversion for the world. Why is that? Because close focus on the breath is a certain aspect of feelings, I say. Therefore, at such a time a mendicant is meditating by observing an aspect of feelings—keen, aware, and mindful, rid of desire and aversion for the world. 

There’s\marginnote{7.1} a time when a mendicant practices like this: ‘I’ll breathe in experiencing the mind.’ They practice like this: ‘I’ll breathe out experiencing the mind.’ They practice like this: ‘I’ll breathe in gladdening the mind.’ They practice like this: ‘I’ll breathe out gladdening the mind.’ They practice like this: ‘I’ll breathe in immersing the mind in \textsanskrit{samādhi}.’ They practice like this: ‘I’ll breathe out immersing the mind in \textsanskrit{samādhi}.’ They practice like this: ‘I’ll breathe in freeing the mind.’ They practice like this: ‘I’ll breathe out freeing the mind.’ At such a time a mendicant is meditating by observing an aspect of the mind—keen, aware, and mindful, rid of desire and aversion for the world. Why is that? Because there is no development of immersion due to mindfulness of breathing for someone who is unmindful and lacks awareness, I say. Therefore, at such a time a mendicant is meditating by observing an aspect of the mind—keen, aware, and mindful, rid of desire and aversion for the world. 

There’s\marginnote{8.1} a time when a mendicant practices like this: ‘I’ll breathe in observing impermanence.’ They practice like this: ‘I’ll breathe out observing impermanence.’ They practice like this: ‘I’ll breathe in observing fading away.’ They practice like this: ‘I’ll breathe out observing fading away.’ They practice like this: ‘I’ll breathe in observing cessation.’ They practice like this: ‘I’ll breathe out observing cessation.’ They practice like this: ‘I’ll breathe in observing letting go.’ They practice like this: ‘I’ll breathe out observing letting go.’ At such a time a mendicant is meditating by observing an aspect of principles—keen, aware, and mindful, rid of desire and aversion for the world. Having seen with wisdom the giving up of desire and aversion, they watch closely over with equanimity. Therefore, at such a time a mendicant is meditating by observing an aspect of principles—keen, aware, and mindful, rid of desire and aversion for the world. 

Suppose\marginnote{9.1} there was a large heap of sand at the crossroads. And a cart or chariot were to come by from the east, west, north, or south and destroy that heap of sand. 

In\marginnote{9.6} the same way, when a mendicant is meditating by observing an aspect of the body, feelings, mind, or principles, they destroy bad, unskillful qualities.” 

%
\addtocontents{toc}{\let\protect\contentsline\protect\nopagecontentsline}
\chapter*{Chapter Two }
\addcontentsline{toc}{chapter}{\tocchapterline{Chapter Two }}
\addtocontents{toc}{\let\protect\contentsline\protect\oldcontentsline}

%
\section*{{\suttatitleacronym SN 54.11}{\suttatitletranslation Icchānaṅgala }{\suttatitleroot Icchānaṅgalasutta}}
\addcontentsline{toc}{section}{\tocacronym{SN 54.11} \toctranslation{Icchānaṅgala } \tocroot{Icchānaṅgalasutta}}
\markboth{Icchānaṅgala }{Icchānaṅgalasutta}
\extramarks{SN 54.11}{SN 54.11}

At\marginnote{1.1} one time the Buddha was staying in a forest near \textsanskrit{Icchānaṅgala}. There he addressed the mendicants, “Mendicants, I wish to go on retreat for three months. No-one should approach me, except for the one who brings my almsfood.” 

“Yes,\marginnote{1.5} sir,” replied those mendicants. And no-one approached him, except for the one who brought the almsfood. 

Then\marginnote{2.1} after three months had passed, the Buddha came out of retreat and addressed the mendicants: 

“Mendicants,\marginnote{2.2} if wanderers who follow another path were to ask you: ‘Reverends, what was the ascetic Gotama’s usual meditation during the rainy season residence?’ You should answer them like this. ‘Reverends, the ascetic Gotama’s usual meditation during the rainy season residence was immersion due to mindfulness of breathing.’ 

In\marginnote{2.5} this regard: mindful, I breathe in. Mindful, I breathe out. 

When\marginnote{2.6} breathing in heavily I know: ‘I’m breathing in heavily.’ When breathing out heavily I know: ‘I’m breathing out heavily.’ When breathing in lightly I know: ‘I’m breathing in lightly.’ When breathing out lightly I know: ‘I’m breathing out lightly.’ I know: ‘I’ll breathe in experiencing the whole body.’ … 

I\marginnote{2.9} know: ‘I’ll breathe in observing letting go.’ I know: ‘I’ll breathe out observing letting go.’ 

For\marginnote{3.1} if anything should be rightly called ‘the meditation of a noble one’, or else ‘the meditation of a \textsanskrit{Brahmā}’, or else ‘the meditation of a realized one’, it’s immersion due to mindfulness of breathing. 

For\marginnote{3.5} those mendicants who are trainees—who haven’t achieved their heart’s desire, but live aspiring to the supreme sanctuary—the development and cultivation of immersion due to mindfulness of breathing leads to the ending of defilements. 

For\marginnote{3.6} those mendicants who are perfected—who have ended the defilements, completed the spiritual journey, done what had to be done, laid down the burden, achieved their own goal, utterly ended the fetters of rebirth, and are rightly freed through enlightenment—the development and cultivation of immersion due to mindfulness of breathing leads to blissful meditation in the present life, and to mindfulness and awareness. 

For\marginnote{4.1} if anything should be rightly called ‘the meditation of a noble one’, or else ‘the meditation of a \textsanskrit{Brahmā}’, or else ‘the meditation of a realized one’, it’s immersion due to mindfulness of breathing.” 

%
\section*{{\suttatitleacronym SN 54.12}{\suttatitletranslation In Doubt }{\suttatitleroot Kaṅkheyyasutta}}
\addcontentsline{toc}{section}{\tocacronym{SN 54.12} \toctranslation{In Doubt } \tocroot{Kaṅkheyyasutta}}
\markboth{In Doubt }{Kaṅkheyyasutta}
\extramarks{SN 54.12}{SN 54.12}

At\marginnote{1.1} one time Venerable \textsanskrit{Lomasavaṅgīsa} was staying in the land of the Sakyans, near Kapilavatthu in the Banyan Tree Monastery. Then \textsanskrit{Mahānāma} the Sakyan went up to Venerable \textsanskrit{Lomasavaṅgīsa}, bowed, sat down to one side, and said to him, “Sir, is the meditation of a trainee just the same as the meditation of a Realized One? Or is the meditation of a trainee different from the meditation of a Realized One?” 

“Reverend\marginnote{2.1} \textsanskrit{Mahānāma}, the meditation of a trainee and a realized one are not the same; they are different. Those mendicants who are trainees haven’t achieved their heart’s desire, but live aspiring for the supreme sanctuary. They meditate after giving up the five hindrances. What five? The hindrances of sensual desire, ill will, dullness and drowsiness, restlessness and remorse, and doubt. 

Those\marginnote{3.1} who are trainee mendicants … meditate after giving up the five hindrances. 

Those\marginnote{4.1} mendicants who are perfected—who have ended the defilements, completed the spiritual journey, done what had to be done, laid down the burden, achieved their own goal, utterly ended the fetters of rebirth, and are rightly freed through enlightenment—for them, the five hindrances are cut off at the root, made like a palm stump, obliterated, and unable to arise in the future. What five? The hindrances of sensual desire, ill will, dullness and drowsiness, restlessness and remorse, and doubt. 

Those\marginnote{5.1} mendicants who are perfected—who have ended the defilements … for them, the five hindrances are cut off at the root … and unable to arise in the future. And here’s another way to understand how the meditation of a trainee and a realized one are different. 

At\marginnote{6.1} one time the Buddha was staying in a forest near \textsanskrit{Icchānaṅgala}. There he addressed the mendicants, ‘Mendicants, I wish to go on retreat for three months. No-one should approach me, except for the one who brings my almsfood.’ 

‘Yes,\marginnote{6.5} sir,’ replied those mendicants. And no-one approached him, except for the one who brought the almsfood. 

Then\marginnote{7.1} after three months had passed, the Buddha came out of retreat and addressed the mendicants: 

‘Mendicants,\marginnote{7.2} if wanderers who follow another path were to ask you: “Reverends, what was the ascetic Gotama’s usual meditation during the rainy season residence?” You should answer them like this: “Reverends, the ascetic Gotama’s usual meditation during the rainy season residence was immersion due to mindfulness of breathing.” 

In\marginnote{7.5} this regard: mindful, I breathe in. Mindful, I breathe out. 

When\marginnote{7.6} breathing in heavily I know: “I’m breathing in heavily.” When breathing out heavily I know: “I’m breathing out heavily.” … 

I\marginnote{7.7} know: “I’ll breathe in observing letting go.” I know: “I’ll breathe out observing letting go.” 

For\marginnote{8.1} if anything should be rightly called “the meditation of a noble one”, or else “the meditation of a \textsanskrit{Brahmā}”, or else “the meditation of a realized one”, it’s immersion due to mindfulness of breathing. 

For\marginnote{9.1} those mendicants who are trainees—who haven’t achieved their heart’s desire, but live aspiring for the supreme sanctuary—the development and cultivation of immersion due to mindfulness of breathing leads to the ending of defilements. 

For\marginnote{10.1} those mendicants who are perfected—who have ended the defilements, completed the spiritual journey, done what had to be done, laid down the burden, achieved their own goal, utterly ended the fetters of rebirth, and are rightly freed through enlightenment—the development and cultivation of immersion due to mindfulness of breathing leads to blissful meditation in the present life, and to mindfulness and awareness. 

For\marginnote{11.1} if anything should be rightly called “the meditation of a noble one”, or else “the meditation of a \textsanskrit{Brahmā}”, or else “the meditation of a realized one”, it’s immersion due to mindfulness of breathing.’ 

This\marginnote{11.5} is another way to understand how the meditation of a trainee and a realized one are different.” 

%
\section*{{\suttatitleacronym SN 54.13}{\suttatitletranslation With Ānanda (1st) }{\suttatitleroot Paṭhamaānandasutta}}
\addcontentsline{toc}{section}{\tocacronym{SN 54.13} \toctranslation{With Ānanda (1st) } \tocroot{Paṭhamaānandasutta}}
\markboth{With Ānanda (1st) }{Paṭhamaānandasutta}
\extramarks{SN 54.13}{SN 54.13}

At\marginnote{1.1} \textsanskrit{Sāvatthī}. 

Then\marginnote{1.2} Venerable Ānanda went up to the Buddha, bowed, sat down to one side, and said to him: 

“Sir,\marginnote{1.3} is there one thing that, when developed and cultivated, fulfills four things; and those four things, when developed and cultivated, fulfill seven things; and those seven things, when developed and cultivated, fulfill two things?” 

“There\marginnote{2.1} is, Ānanda.” 

“Sir,\marginnote{3.1} what is that one thing?” 

“Immersion\marginnote{3.2} due to mindfulness of breathing is one thing that, when developed and cultivated, fulfills the four kinds of mindfulness meditation. And the four kinds of mindfulness meditation, when developed and cultivated, fulfill the seven awakening factors. And the seven awakening factors, when developed and cultivated, fulfill knowledge and freedom. 

And\marginnote{4.1} how is mindfulness of breathing developed and cultivated so as to fulfill the four kinds of mindfulness meditation? It’s when a mendicant has gone to a wilderness, or to the root of a tree, or to an empty hut, sits down cross-legged, with their body straight, and establishes mindfulness right there. Just mindful, they breathe in. Mindful, they breathe out. When breathing in heavily they know: ‘I’m breathing in heavily.’ When breathing out heavily they know: ‘I’m breathing out heavily.’ … They practice like this: ‘I’ll breathe in observing letting go.’ They practice like this: ‘I’ll breathe out observing letting go.’ 

When\marginnote{5.1} a mendicant is breathing in heavily they know: ‘I’m breathing in heavily.’ When breathing out heavily they know: ‘I’m breathing out heavily.’ … They practice like this: ‘I’ll breathe in stilling the physical process.’ They practice like this: ‘I’ll breathe out stilling the physical process.’ At such a time a mendicant is meditating by observing an aspect of the body—keen, aware, and mindful, rid of desire and aversion for the world. Why is that? Because the breath is a certain aspect of the body, I say. Therefore, at such a time a mendicant is meditating by observing an aspect of the body—keen, aware, and mindful, rid of desire and aversion for the world. 

There’s\marginnote{6.1} a time when a mendicant practices like this: ‘I’ll breathe in experiencing rapture … bliss … mind …’ … They practice like this: ‘I’ll breathe in stilling the mental processes.’ They practice like this: ‘I’ll breathe out stilling the mental processes.’ At such a time a mendicant is meditating by observing an aspect of feelings—keen, aware, and mindful, rid of desire and aversion for the world. Why is that? Because close focus on the breath is a certain aspect of feelings, I say. Therefore, at such a time a mendicant is meditating by observing an aspect of feelings—keen, aware, and mindful, rid of desire and aversion for the world. 

There’s\marginnote{7.1} a time when a mendicant practices like this: ‘I’ll breathe in experiencing the mind.’ They practice like this: ‘I’ll breathe out experiencing the mind.’ They practice like this: ‘I’ll breathe in gladdening the mind … immersing the mind in \textsanskrit{samādhi} … freeing the mind.’ They practice like this: ‘I’ll breathe out freeing the mind.’ At such a time a mendicant is meditating by observing an aspect of the mind—keen, aware, and mindful, rid of desire and aversion for the world. Why is that? Because there is no development of immersion due to mindfulness of breathing for someone who is unmindful and lacks awareness, I say. Therefore, at such a time a mendicant is meditating by observing an aspect of the mind—keen, aware, and mindful, rid of desire and aversion for the world. 

There’s\marginnote{8.1} a time when a mendicant practices like this: ‘I’ll breathe in observing impermanence … fading away … cessation … letting go.’ They practice like this: ‘I’ll breathe out observing letting go.’ At such a time a mendicant is meditating by observing an aspect of principles—keen, aware, and mindful, rid of desire and aversion for the world. Having seen with wisdom the giving up of desire and aversion, they watch closely over with equanimity. Therefore, at such a time a mendicant is meditating by observing an aspect of principles—keen, aware, and mindful, rid of desire and aversion for the world. 

That’s\marginnote{9.1} how immersion due to mindfulness of breathing is developed and cultivated so as to fulfill the four kinds of mindfulness meditation. 

And\marginnote{10.1} how are the four kinds of mindfulness meditation developed and cultivated so as to fulfill the seven awakening factors? Whenever a mendicant meditates by observing an aspect of the body, their mindfulness is established and lucid. At such a time, a mendicant has activated the awakening factor of mindfulness; they develop it and perfect it. 

As\marginnote{11.1} they live mindfully in this way they investigate, explore, and inquire into that principle with wisdom. At such a time, a mendicant has activated the awakening factor of investigation of principles; they develop it and perfect it. 

As\marginnote{12.1} they investigate principles with wisdom in this way their energy is roused up and unflagging. At such a time, a mendicant has activated the awakening factor of energy; they develop it and perfect it. 

When\marginnote{13.1} you’re energetic, spiritual rapture arises. At such a time, a mendicant has activated the awakening factor of rapture; they develop it and perfect it. 

When\marginnote{14.1} the mind is full of rapture, the body and mind become tranquil. At such a time, a mendicant has activated the awakening factor of tranquility; they develop it and perfect it. 

When\marginnote{15.1} the body is tranquil and one feels bliss, the mind becomes immersed in \textsanskrit{samādhi}. At such a time, a mendicant has activated the awakening factor of immersion; they develop it and perfect it. 

They\marginnote{16.1} closely watch over that mind immersed in \textsanskrit{samādhi}. At such a time, a mendicant has activated the awakening factor of equanimity; they develop it and perfect it. 

Whenever\marginnote{17.1} a mendicant meditates by observing an aspect of feelings … mind … principles, their mindfulness is established and lucid. At such a time, a mendicant has activated the awakening factor of mindfulness; they develop it and perfect it. … 

(This\marginnote{17.7} should be told in full as for the first kind of mindfulness meditation.) 

They\marginnote{18.1} closely watch over that mind immersed in \textsanskrit{samādhi}. At such a time, a mendicant has activated the awakening factor of equanimity; they develop it and perfect it. That’s how the four kinds of mindfulness meditation are developed and cultivated so as to fulfill the seven awakening factors. 

And\marginnote{19.1} how are the seven awakening factors developed and cultivated so as to fulfill knowledge and freedom? It’s when a mendicant develops the awakening factors of mindfulness, investigation of principles, energy, rapture, tranquility, immersion, and equanimity, which rely on seclusion, fading away, and cessation, and ripen as letting go. That’s how the seven awakening factors are developed and cultivated so as to fulfill knowledge and freedom.” 

%
\section*{{\suttatitleacronym SN 54.14}{\suttatitletranslation With Ānanda (2nd) }{\suttatitleroot Dutiyaānandasutta}}
\addcontentsline{toc}{section}{\tocacronym{SN 54.14} \toctranslation{With Ānanda (2nd) } \tocroot{Dutiyaānandasutta}}
\markboth{With Ānanda (2nd) }{Dutiyaānandasutta}
\extramarks{SN 54.14}{SN 54.14}

Then\marginnote{1.1} Venerable Ānanda went up to the Buddha, bowed, and sat down to one side. The Buddha said to him: “Ānanda, is there one thing that, when developed and cultivated, fulfills four things; and those four things, when developed and cultivated, fulfill seven things; and those seven things, when developed and cultivated, fulfill two things?” 

“Our\marginnote{1.3} teachings are rooted in the Buddha. …” 

“There\marginnote{1.4} is, Ānanda. 

And\marginnote{2.1} what is that one thing? Immersion due to mindfulness of breathing is one thing that, when developed and cultivated, fulfills the four kinds of mindfulness meditation. And the four kinds of mindfulness meditation, when developed and cultivated, fulfill the seven awakening factors. And the seven awakening factors, when developed and cultivated, fulfill knowledge and freedom. 

And\marginnote{2.3} how is mindfulness of breathing developed and cultivated so as to fulfill the four kinds of mindfulness meditation? … 

That’s\marginnote{2.5} how the seven awakening factors are developed and cultivated so as to fulfill knowledge and freedom.” 

%
\section*{{\suttatitleacronym SN 54.15}{\suttatitletranslation Several Mendicants (1st) }{\suttatitleroot Paṭhamabhikkhusutta}}
\addcontentsline{toc}{section}{\tocacronym{SN 54.15} \toctranslation{Several Mendicants (1st) } \tocroot{Paṭhamabhikkhusutta}}
\markboth{Several Mendicants (1st) }{Paṭhamabhikkhusutta}
\extramarks{SN 54.15}{SN 54.15}

Then\marginnote{1.1} several mendicants went up to the Buddha, bowed, sat down to one side, and said to him: 

“Sir,\marginnote{1.2} is there one thing that, when developed and cultivated, fulfills four things; and those four things, when developed and cultivated, fulfill seven things; and those seven things, when developed and cultivated, fulfill two things?” 

“There\marginnote{1.3} is, mendicants.” 

“Sir,\marginnote{2.1} what is that one thing?” 

“Immersion\marginnote{2.2} due to mindfulness of breathing is one thing that, when developed and cultivated, fulfills the four kinds of mindfulness meditation. And the four kinds of mindfulness meditation, when developed and cultivated, fulfill the seven awakening factors. And the seven awakening factors, when developed and cultivated, fulfill knowledge and freedom. 

And\marginnote{3.1} how is mindfulness of breathing developed and cultivated so as to fulfill the four kinds of mindfulness meditation? … 

That’s\marginnote{3.3} how the seven awakening factors are developed and cultivated so as to fulfill knowledge and freedom.” 

%
\section*{{\suttatitleacronym SN 54.16}{\suttatitletranslation Several Mendicants (2nd) }{\suttatitleroot Dutiyabhikkhusutta}}
\addcontentsline{toc}{section}{\tocacronym{SN 54.16} \toctranslation{Several Mendicants (2nd) } \tocroot{Dutiyabhikkhusutta}}
\markboth{Several Mendicants (2nd) }{Dutiyabhikkhusutta}
\extramarks{SN 54.16}{SN 54.16}

Then\marginnote{1.1} several mendicants went up to the Buddha, bowed, and sat down to one side. The Buddha said to them: 

“Mendicants,\marginnote{1.2} is there one thing that, when developed and cultivated, fulfills four things; and those four things, when developed and cultivated, fulfill seven things; and those seven things, when developed and cultivated, fulfill two things?” 

“Our\marginnote{1.3} teachings are rooted in the Buddha. …” 

“There\marginnote{1.4} is, mendicants. 

And\marginnote{2.1} what is that one thing? Immersion due to mindfulness of breathing is one thing that, when developed and cultivated, fulfills the four kinds of mindfulness meditation. And the four kinds of mindfulness meditation, when developed and cultivated, fulfill the seven awakening factors. And the seven awakening factors, when developed and cultivated, fulfill knowledge and freedom. 

And\marginnote{3.1} how is mindfulness of breathing developed and cultivated so as to fulfill the four kinds of mindfulness meditation? It’s when a mendicant has gone to a wilderness, or to the root of a tree, or to an empty hut. They sit down cross-legged, with their body straight, and establish mindfulness right there. … 

That’s\marginnote{18.1} how the seven awakening factors are developed and cultivated so as to fulfill knowledge and freedom.” 

%
\section*{{\suttatitleacronym SN 54.17}{\suttatitletranslation Giving Up the Fetters }{\suttatitleroot Saṁyojanappahānasutta}}
\addcontentsline{toc}{section}{\tocacronym{SN 54.17} \toctranslation{Giving Up the Fetters } \tocroot{Saṁyojanappahānasutta}}
\markboth{Giving Up the Fetters }{Saṁyojanappahānasutta}
\extramarks{SN 54.17}{SN 54.17}

“Mendicants,\marginnote{1.1} when immersion due to mindfulness of breathing is developed and cultivated it leads to giving up the fetters …” 

%
\section*{{\suttatitleacronym SN 54.18}{\suttatitletranslation Uprooting the Tendencies }{\suttatitleroot Anusayasamugghātasutta}}
\addcontentsline{toc}{section}{\tocacronym{SN 54.18} \toctranslation{Uprooting the Tendencies } \tocroot{Anusayasamugghātasutta}}
\markboth{Uprooting the Tendencies }{Anusayasamugghātasutta}
\extramarks{SN 54.18}{SN 54.18}

“Mendicants,\marginnote{1.1} when immersion due to mindfulness of breathing is developed and cultivated it leads to uprooting the underlying tendencies …” 

%
\section*{{\suttatitleacronym SN 54.19}{\suttatitletranslation Completely Understanding the Course of Time }{\suttatitleroot Addhānapariññāsutta}}
\addcontentsline{toc}{section}{\tocacronym{SN 54.19} \toctranslation{Completely Understanding the Course of Time } \tocroot{Addhānapariññāsutta}}
\markboth{Completely Understanding the Course of Time }{Addhānapariññāsutta}
\extramarks{SN 54.19}{SN 54.19}

“Mendicants,\marginnote{1.1} when immersion due to mindfulness of breathing is developed and cultivated it leads to completely understanding the course of time …” 

%
\section*{{\suttatitleacronym SN 54.20}{\suttatitletranslation The Ending of Defilements }{\suttatitleroot Āsavakkhayasutta}}
\addcontentsline{toc}{section}{\tocacronym{SN 54.20} \toctranslation{The Ending of Defilements } \tocroot{Āsavakkhayasutta}}
\markboth{The Ending of Defilements }{Āsavakkhayasutta}
\extramarks{SN 54.20}{SN 54.20}

“Mendicants,\marginnote{1.1} when immersion due to mindfulness of breathing is developed and cultivated it leads to the ending of defilements. And how is immersion due to mindfulness of breathing developed and cultivated so as to lead to giving up the fetters, uprooting the underlying tendencies, completely understanding the course of time, and ending the defilements? 

It’s\marginnote{1.6} when a mendicant—gone to a wilderness, or to the root of a tree, or to an empty hut—sits down cross-legged, with their body straight, and focuses their mindfulness right there. … 

They\marginnote{1.7} practice like this: ‘I’ll breathe in observing letting go.’ They practice like this: ‘I’ll breathe out observing letting go.’ 

That’s\marginnote{1.8} how immersion due to mindfulness of breathing is developed and cultivated so as to lead to giving up the fetters, uprooting the underlying tendencies, completely understanding the course of time, and ending the defilements.” 

\scendsutta{The Linked Discourses on Mindfulness of Breathing is the tenth section. }

%
\addtocontents{toc}{\let\protect\contentsline\protect\nopagecontentsline}
\part*{Linked Discourses on Stream-Entry }
\addcontentsline{toc}{part}{Linked Discourses on Stream-Entry }
\markboth{}{}
\addtocontents{toc}{\let\protect\contentsline\protect\oldcontentsline}

%
\addtocontents{toc}{\let\protect\contentsline\protect\nopagecontentsline}
\chapter*{The Chapter at Bamboo Gate }
\addcontentsline{toc}{chapter}{\tocchapterline{The Chapter at Bamboo Gate }}
\addtocontents{toc}{\let\protect\contentsline\protect\oldcontentsline}

%
\section*{{\suttatitleacronym SN 55.1}{\suttatitletranslation A Wheel-Turning Monarch }{\suttatitleroot Cakkavattirājasutta}}
\addcontentsline{toc}{section}{\tocacronym{SN 55.1} \toctranslation{A Wheel-Turning Monarch } \tocroot{Cakkavattirājasutta}}
\markboth{A Wheel-Turning Monarch }{Cakkavattirājasutta}
\extramarks{SN 55.1}{SN 55.1}

At\marginnote{1.1} \textsanskrit{Sāvatthī}. 

There\marginnote{1.2} the Buddha … said: 

“Mendicants,\marginnote{1.3} suppose a wheel-turning monarch were to rule as sovereign lord over these four continents. And when his body breaks up, after death, he’s reborn in a good place, a heavenly realm, in the company of the gods of the Thirty-Three. There he entertains himself in the Garden of Delight, escorted by a band of nymphs, and supplied and provided with the five kinds of heavenly sensual stimulation. Still, as he’s lacking four things, he’s not exempt from hell, the animal realm, or the ghost realm. He’s not exempt from places of loss, bad places, the underworld. 

Now\marginnote{1.4} suppose a noble disciple wears rags and feeds on scraps of almsfood. Still, as they have four things, they’re exempt from hell, the animal realm, or the ghost realm. They’re exempt from places of loss, bad places, the underworld. 

What\marginnote{2.1} four? It’s when a noble disciple has experiential confidence in the Buddha: ‘That Blessed One is perfected, a fully awakened Buddha, accomplished in knowledge and conduct, holy, knower of the world, supreme guide for those who wish to train, teacher of gods and humans, awakened, blessed.’ 

They\marginnote{2.4} have experiential confidence in the teaching: ‘The teaching is well explained by the Buddha—visible in this very life, immediately effective, inviting inspection, relevant, so that sensible people can know it for themselves.’ 

They\marginnote{2.6} have experiential confidence in the \textsanskrit{Saṅgha}: ‘The \textsanskrit{Saṅgha} of the Buddha’s disciples is practicing the way that’s good, direct, methodical, and proper. It consists of the four pairs, the eight individuals. This is the \textsanskrit{Saṅgha} of the Buddha’s disciples that is worthy of offerings dedicated to the gods, worthy of hospitality, worthy of a religious donation, worthy of greeting with joined palms, and is the supreme field of merit for the world.’ 

Furthermore,\marginnote{2.8} a noble disciple’s ethical conduct is loved by the noble ones, unbroken, impeccable, spotless, and unmarred, liberating, praised by sensible people, not mistaken, and leading to immersion. 

These\marginnote{2.9} are the four factors of stream-entry that they have. 

And,\marginnote{2.10} mendicants, gaining these four continents is not worth a sixteenth part of gaining these four things.” 

%
\section*{{\suttatitleacronym SN 55.2}{\suttatitletranslation The Culmination of the Spiritual Life }{\suttatitleroot Brahmacariyogadhasutta}}
\addcontentsline{toc}{section}{\tocacronym{SN 55.2} \toctranslation{The Culmination of the Spiritual Life } \tocroot{Brahmacariyogadhasutta}}
\markboth{The Culmination of the Spiritual Life }{Brahmacariyogadhasutta}
\extramarks{SN 55.2}{SN 55.2}

“Mendicants,\marginnote{1.1} a noble disciple who has four things is a stream-enterer, not liable to be reborn in the underworld, bound for awakening. 

What\marginnote{2.1} four? It’s when a noble disciple has experiential confidence in the Buddha … the teaching … the \textsanskrit{Saṅgha} … And they have the ethical conduct loved by the noble ones … leading to immersion. A noble disciple who has these four things is a stream-enterer, not liable to be reborn in the underworld, bound for awakening.” 

That\marginnote{3.1} is what the Buddha said. Then the Holy One, the Teacher, went on to say: 

\begin{verse}%
“Those\marginnote{4.1} who have faith and ethics, \\
confidence, and vision of the truth, \\
in time arrive at happiness, \\
the culmination of the spiritual life.” 

%
\end{verse}

%
\section*{{\suttatitleacronym SN 55.3}{\suttatitletranslation With Dīghāvu }{\suttatitleroot Dīghāvuupāsakasutta}}
\addcontentsline{toc}{section}{\tocacronym{SN 55.3} \toctranslation{With Dīghāvu } \tocroot{Dīghāvuupāsakasutta}}
\markboth{With Dīghāvu }{Dīghāvuupāsakasutta}
\extramarks{SN 55.3}{SN 55.3}

At\marginnote{1.1} one time the Buddha was staying near \textsanskrit{Rājagaha}, in the Bamboo Grove, the squirrels’ feeding ground. 

Now\marginnote{1.2} at that time the lay follower \textsanskrit{Dhīgāvu} was sick, suffering, gravely ill. Then he addressed his father, the householder Jotika, “Please, householder, go to the Buddha, and in my name bow with your head to his feet. Say to him: ‘Sir, the lay follower \textsanskrit{Dhīgāvu} is sick, suffering, gravely ill. He bows with his head to your feet.’ And then say: ‘Sir, please visit him at his home out of compassion.’” 

“Yes,\marginnote{1.9} dear,” replied Jotika. He did as \textsanskrit{Dīghāvu} asked. The Buddha consented in silence. 

Then\marginnote{2.1} the Buddha robed up in the morning and, taking his bowl and robe, went to the home of the lay follower \textsanskrit{Dīghāvu}, sat down on the seat spread out, and said to him, “I hope you’re keeping well, \textsanskrit{Dīghāvu}; I hope you’re alright. I hope that your pain is fading, not growing, that its fading is evident, not its growing.” 

“Sir,\marginnote{2.3} I’m not keeping well, I’m not alright. The pain is terrible and growing, not fading; its growing is evident, not its fading.” 

“So,\marginnote{2.4} \textsanskrit{Dīghāvu}, you should train like this: ‘I will have experiential confidence in the Buddha … the teaching … the \textsanskrit{Saṅgha} … And I will have the ethical conduct loved by the noble ones … leading to immersion.’ That’s how you should train.” 

“Sir,\marginnote{3.1} these four factors of stream-entry that were taught by the Buddha are found in me, and I am seen in them. For I have experiential confidence in the Buddha … the teaching … the \textsanskrit{Saṅgha} … And I have the ethical conduct loved by the noble ones … leading to immersion.” 

“In\marginnote{3.6} that case, \textsanskrit{Dīghāvu}, grounded on these four factors of stream-entry you should further develop these six things that play a part in realization. You should meditate observing the impermanence of all conditions, perceiving suffering in impermanence, perceiving not-self in suffering, perceiving giving up, perceiving fading away, and perceiving cessation. That’s how you should train.” 

“These\marginnote{4.1} six things that play a part in realization that were taught by the Buddha are found in me, and I embody them. For I meditate observing the impermanence of all conditions, perceiving suffering in impermanence, perceiving not-self in suffering, perceiving giving up, perceiving fading away, and perceiving cessation. 

But\marginnote{4.3} still, sir, I think, ‘I hope Jotika doesn’t suffer anguish when I’ve gone.’” Jotika said, “Dear \textsanskrit{Dīghāvu}, don’t focus on that. Come on, dear \textsanskrit{Dīghāvu}, you should closely focus on what the Buddha is saying.” 

When\marginnote{5.1} the Buddha had given this advice he got up from his seat and left. Not long after the Buddha left, \textsanskrit{Dīghāvu} passed away. Then several mendicants went up to the Buddha, bowed, sat down to one side, and said to him: 

“Sir,\marginnote{5.4} the lay follower named \textsanskrit{Dīghāvu}, who was advised in brief by the Buddha, has passed away. Where has he been reborn in his next life?” 

“Mendicants,\marginnote{5.6} the lay follower \textsanskrit{Dīghāvu} was astute. He practiced in line with the teachings, and did not trouble me about the teachings. With the ending of the five lower fetters, he’s been reborn spontaneously, and will become extinguished there, not liable to return from that world.” 

%
\section*{{\suttatitleacronym SN 55.4}{\suttatitletranslation With Sāriputta (1st) }{\suttatitleroot Paṭhamasāriputtasutta}}
\addcontentsline{toc}{section}{\tocacronym{SN 55.4} \toctranslation{With Sāriputta (1st) } \tocroot{Paṭhamasāriputtasutta}}
\markboth{With Sāriputta (1st) }{Paṭhamasāriputtasutta}
\extramarks{SN 55.4}{SN 55.4}

At\marginnote{1.1} one time Venerable \textsanskrit{Sāriputta} was staying near \textsanskrit{Sāvatthī} in Jeta’s Grove, \textsanskrit{Anāthapiṇḍika}’s monastery. Then in the late afternoon, Venerable Ānanda came out of retreat … and said to \textsanskrit{Sāriputta}: 

“Reverend,\marginnote{1.3} how many things do people have to possess in order for the Buddha to declare that they’re a stream-enterer, not liable to be reborn in the underworld, bound for awakening?” 

“Reverend,\marginnote{1.4} people have to possess four things in order for the Buddha to declare that they’re a stream-enterer, not liable to be reborn in the underworld, bound for awakening. 

What\marginnote{2.1} four? It’s when a noble disciple has experiential confidence in the Buddha … the teaching … the \textsanskrit{Saṅgha} … And they have the ethical conduct loved by the noble ones … leading to immersion. People have to possess these four things in order for the Buddha to declare that they’re a stream-enterer, not liable to be reborn in the underworld, bound for awakening.” 

%
\section*{{\suttatitleacronym SN 55.5}{\suttatitletranslation With Sāriputta (2nd) }{\suttatitleroot Dutiyasāriputtasutta}}
\addcontentsline{toc}{section}{\tocacronym{SN 55.5} \toctranslation{With Sāriputta (2nd) } \tocroot{Dutiyasāriputtasutta}}
\markboth{With Sāriputta (2nd) }{Dutiyasāriputtasutta}
\extramarks{SN 55.5}{SN 55.5}

Then\marginnote{1.1} \textsanskrit{Sāriputta} went up to the Buddha, bowed, and sat down to one side. The Buddha said to him: 

“\textsanskrit{Sāriputta},\marginnote{1.2} they speak of a ‘factor of stream-entry’. What is a factor of stream-entry?” 

“Sir,\marginnote{1.4} the factors of stream-entry are associating with good people, listening to the true teaching, proper attention, and practicing in line with the teaching.” 

“Good,\marginnote{1.5} good, \textsanskrit{Sāriputta}! For the factors of stream-entry are associating with good people, listening to the true teaching, proper attention, and practicing in line with the teaching. 

\textsanskrit{Sāriputta},\marginnote{2.1} they speak of ‘the stream’. What is the stream?” 

“Sir,\marginnote{2.3} the stream is simply this noble eightfold path, that is: right view, right thought, right speech, right action, right livelihood, right effort, right mindfulness, and right immersion.” 

“Good,\marginnote{2.5} good, \textsanskrit{Sāriputta}! For the stream is simply this noble eightfold path, that is: right view, right thought, right speech, right action, right livelihood, right effort, right mindfulness, and right immersion. 

\textsanskrit{Sāriputta},\marginnote{3.1} they speak of ‘a stream-enterer’. What is a stream-enterer?” 

“Sir,\marginnote{3.3} anyone who possesses this noble eightfold path is called a stream-enterer, the venerable of such and such name and clan.” 

“Good,\marginnote{3.4} good, \textsanskrit{Sāriputta}! For anyone who possesses this noble eightfold path is called a stream-enterer, the venerable of such and such name and clan.” 

%
\section*{{\suttatitleacronym SN 55.6}{\suttatitletranslation The Chamberlains }{\suttatitleroot Thapatisutta}}
\addcontentsline{toc}{section}{\tocacronym{SN 55.6} \toctranslation{The Chamberlains } \tocroot{Thapatisutta}}
\markboth{The Chamberlains }{Thapatisutta}
\extramarks{SN 55.6}{SN 55.6}

At\marginnote{1.1} \textsanskrit{Sāvatthī}. At that time several mendicants were making a robe for the Buddha, thinking that when his robe was finished and the three months of the rains residence had passed the Buddha would set out wandering. Now at that time the chamberlains Isidatta and \textsanskrit{Purāṇa} were residing in \textsanskrit{Sādhuka} on some business. They heard about this. 

So\marginnote{2.1} they posted someone on the road, saying: 

“My\marginnote{2.2} good man, let us know when you see the Blessed One coming, the perfected one, the fully awakened Buddha.” And that person stood there for two or three days before they saw the Buddha coming off in the distance. When they saw him, they went to the chamberlains and said: 

“Sirs,\marginnote{2.5} the Blessed One, the perfected one, the fully awakened Buddha is coming. Please come at your convenience.” 

Then\marginnote{3.1} the chamberlains went up to the Buddha, bowed, and followed behind him. And then the Buddha left the road, went to the root of a certain tree, and sat down on the seat spread out. The chamberlains Isidatta and \textsanskrit{Purāṇa} bowed, sat down to one side, and said to the Buddha: 

“Sir,\marginnote{4.1} when we hear that you will be setting out from \textsanskrit{Sāvatthī} to wander in the Kosalan lands, we’re sad and upset, thinking that you will be far from us. And when we hear that you are setting out from \textsanskrit{Sāvatthī} to wander in the Kosalan lands, we’re sad and upset, thinking that you are far from us. 

And\marginnote{5.1} when we hear that you will be setting out from the Kosalan lands to wander in the Mallian lands, we’re sad and upset, thinking that you will be far from us. And when we hear that you are setting out from the Kosalan lands to wander in the Mallian lands, we’re sad and upset, thinking that you are far from us. 

And\marginnote{6.1} when we hear that you will be setting out from the Mallian lands to wander in the Vajjian lands … 

you\marginnote{7.1} will be setting out from the Vajjian lands to wander in the \textsanskrit{Kāsian} lands … 

you\marginnote{8.1} will be setting out from the \textsanskrit{Kāsian} lands to wander in the \textsanskrit{Māgadhan} lands … 

you\marginnote{8.4} are setting out from the \textsanskrit{Kāsian} lands to wander in the \textsanskrit{Māgadhan} lands, we’re sad and upset, thinking that you are far from us. 

But\marginnote{9.1} when we hear that you will be setting out from the \textsanskrit{Māgadhan} lands to wander in the \textsanskrit{Kāsian} lands, we’re happy and joyful, thinking that you will be near to us. And when we hear that you are setting out from the \textsanskrit{Māgadhan} lands to wander in the \textsanskrit{Kāsian} lands … 

you\marginnote{10.1} will be setting out from the \textsanskrit{Kāsian} lands to wander in the Vajjian lands … 

you\marginnote{11.1} will be setting out from the Vajjian lands to wander in the Mallian lands … 

you\marginnote{12.1} will be setting out from the Mallian lands to wander in the Kosalan lands … 

you\marginnote{13.1} will be setting out in the Kosalan lands to wander to \textsanskrit{Sāvatthī}, we’re happy and joyful, thinking that you will be near to us. 

And\marginnote{13.4} when we hear that you are staying near \textsanskrit{Sāvatthī} in Jeta’s Grove, \textsanskrit{Anāthapiṇḍika}’s monastery we have no little happiness and joy, thinking that you are near to us.” 

“Well\marginnote{14.1} then, chamberlains, living in a house is cramped and dirty, but the life of one gone forth is wide open. Just this much is enough to be diligent.” 

“Sir,\marginnote{14.3} for us there is something that’s even more cramped than that, and is considered as such.” 

“What\marginnote{14.4} is that?” 

“Sir,\marginnote{15.1} it’s when King Pasenadi of Kosala wants to go and visit a park. We have to harness and prepare his royal elephants. Then we have to seat his dear and beloved wives on the elephants, one in front of us, and one behind. Those sisters smell like a freshly opened perfume box; that’s how the royal ladies smell with makeup on. The touch of those sisters is like a tuft of cotton-wool or kapok; that’s how dainty the royal ladies are. Now at that time we must look after the elephants, the sisters, and ourselves. But we don’t recall having a bad thought regarding those sisters. This is that thing that’s even more cramped than that, and is considered as such.” 

“Well\marginnote{16.1} then, chamberlains, living in a house is cramped and dirty, but the life of one gone forth is wide open. Just this much is enough to be diligent. A noble disciple who has four things is a stream-enterer, not liable to be reborn in the underworld, bound for awakening. 

What\marginnote{17.1} four? It’s when a noble disciple has experiential confidence in the Buddha … the teaching … the \textsanskrit{Saṅgha} … They live at home rid of the stain of stinginess, freely generous, open-handed, loving to let go, committed to charity, loving to give and to share. A noble disciple who has these four things is a stream-enterer, not liable to be reborn in the underworld, bound for awakening. 

And\marginnote{18.1} you have experiential confidence in the Buddha … the teaching … the \textsanskrit{Saṅgha} … And whatever there is in your family that’s available to give, you share it all with those who are ethical, of good character. 

What\marginnote{18.6} do you think, chamberlains? How many people among the Kosalans are your equal when it comes to giving and sharing?” 

“We’re\marginnote{18.9} fortunate, sir, so very fortunate, in that the Buddha understands us like this.” 

%
\section*{{\suttatitleacronym SN 55.7}{\suttatitletranslation The People of Bamboo Gate }{\suttatitleroot Veḷudvāreyyasutta}}
\addcontentsline{toc}{section}{\tocacronym{SN 55.7} \toctranslation{The People of Bamboo Gate } \tocroot{Veḷudvāreyyasutta}}
\markboth{The People of Bamboo Gate }{Veḷudvāreyyasutta}
\extramarks{SN 55.7}{SN 55.7}

\scevam{So\marginnote{1.1} I have heard. }At one time the Buddha was wandering in the land of the Kosalans together with a large \textsanskrit{Saṅgha} of mendicants when he arrived at a village of the Kosalan brahmins named Bamboo Gate. The brahmins and householders of Bamboo Gate heard: 

“It\marginnote{1.4} seems the ascetic Gotama—a Sakyan, gone forth from a Sakyan family—has arrived at Bamboo Gate, together with a large \textsanskrit{Saṅgha} of mendicants. He has this good reputation: ‘That Blessed One is perfected, a fully awakened Buddha, accomplished in knowledge and conduct, holy, knower of the world, supreme guide for those who wish to train, teacher of gods and humans, awakened, blessed.’ He has realized with his own insight this world—with its gods, \textsanskrit{Māras} and \textsanskrit{Brahmās}, this population with its ascetics and brahmins, gods and humans—and he makes it known to others. He teaches Dhamma that’s good in the beginning, good in the middle, and good in the end, meaningful and well-phrased. And he reveals a spiritual practice that’s entirely full and pure. It’s good to see such perfected ones.” 

Then\marginnote{2.1} the brahmins and householders of Bamboo Gate went up to the Buddha. Before sitting down to one side, some bowed, some exchanged greetings and polite conversation, some held up their joined palms toward the Buddha, some announced their name and clan, while some kept silent. Seated to one side they said to the Buddha: 

“Master\marginnote{2.2} Gotama, these are our wishes, desires, and hopes. We wish to live at home with our children; to use sandalwood imported from \textsanskrit{Kāsi}; to wear garlands, perfumes, and makeup; and to accept gold and money. And when our body breaks up, after death, we wish to be reborn in a good place, a heavenly realm. Given that we have such wishes, may the Buddha teach us the Dhamma so that we may achieve them.” 

“Householders,\marginnote{3.1} I will teach you an explanation of the Dhamma that applies to oneself. Listen and pay close attention, I will speak.” 

“Yes,\marginnote{3.3} sir,” they replied. The Buddha said this: 

“And\marginnote{4.1} what is the explanation of the Dhamma that applies to oneself? 

It’s\marginnote{4.2} when a noble disciple reflects: ‘I want to live and don’t want to die; I want to be happy and recoil from pain. Since this is so, if someone were to take my life, I wouldn’t like that. But others also want to live and don’t want to die; they want to be happy and recoil from pain. So if I were to take the life of someone else, they wouldn’t like that either. The thing that is disliked by me is also disliked by others. Since I dislike this thing, how can I inflict it on someone else?’ Reflecting in this way, they give up killing living creatures themselves. And they encourage others to give up killing living creatures, praising the giving up of killing living creatures. So their bodily behavior is purified in three points. 

Furthermore,\marginnote{5.1} a noble disciple reflects: ‘If someone were to steal from me, I wouldn’t like that. But if I were to steal from someone else, they wouldn’t like that either. The thing that is disliked by me is also disliked by others. Since I dislike this thing, how can I inflict it on someone else?’ Reflecting in this way, they give up stealing themselves. And they encourage others to give up stealing, praising the giving up of stealing. So their bodily behavior is purified in three points. 

Furthermore,\marginnote{6.1} a noble disciple reflects: ‘If someone were to have sexual relations with my wives, I wouldn’t like it. But if I were to have sexual relations with someone else’s wives, he wouldn’t like that either. The thing that is disliked by me is also disliked by others. Since I dislike this thing, how can I inflict it on others?’ Reflecting in this way, they give up sexual misconduct themselves. And they encourage others to give up sexual misconduct, praising the giving up of sexual misconduct. So their bodily behavior is purified in three points. 

Furthermore,\marginnote{7.1} a noble disciple reflects: ‘If someone were to distort my meaning by lying, I wouldn’t like it. But if I were to distort someone else’s meaning by lying, they wouldn’t like it either. The thing that is disliked by me is also disliked by someone else. Since I dislike this thing, how can I inflict it on others?’ Reflecting in this way, they give up lying themselves. And they encourage others to give up lying, praising the giving up of lying. So their verbal behavior is purified in three points. 

Furthermore,\marginnote{8.1} a noble disciple reflects: ‘If someone were to break me up from my friends by divisive speech, I wouldn’t like it. But if I were to break someone else from their friends by divisive speech, they wouldn’t like it either. …’ So their verbal behavior is purified in three points. 

Furthermore,\marginnote{9.1} a noble disciple reflects: ‘If someone were to attack me with harsh speech, I wouldn’t like it. But if I were to attack someone else with harsh speech, they wouldn’t like it either. …’ So their verbal behavior is purified in three points. 

Furthermore,\marginnote{10.1} a noble disciple reflects: ‘If someone were to annoy me by talking silliness and nonsense, I wouldn’t like it. But if I were to annoy someone else by talking silliness and nonsense, they wouldn’t like it either.’ The thing that is disliked by me is also disliked by another. Since I dislike this thing, how can I inflict it on another?’ Reflecting in this way, they give up talking nonsense themselves. And they encourage others to give up talking nonsense, praising the giving up of talking nonsense. So their verbal behavior is purified in three points. 

And\marginnote{11.1} they have experiential confidence in the Buddha … the teaching … the \textsanskrit{Saṅgha} … And they have the ethical conduct loved by the noble ones … leading to immersion. When a noble disciple has these seven good qualities and these four desirable states they may, if they wish, declare of themselves: ‘I’ve finished with rebirth in hell, the animal realm, and the ghost realm. I’ve finished with all places of loss, bad places, the underworld. I am a stream-enterer! I’m not liable to be reborn in the underworld, and am bound for awakening.’” 

When\marginnote{12.1} he had spoken, the brahmins and householders of Bamboo Gate said to the Buddha, “Excellent, Master Gotama! … We go for refuge to Master Gotama, to the teaching, and to the mendicant \textsanskrit{Saṅgha}. From this day forth, may Master Gotama remember us as lay followers who have gone for refuge for life.” 

%
\section*{{\suttatitleacronym SN 55.8}{\suttatitletranslation In the Brick Hall (1st) }{\suttatitleroot Paṭhamagiñjakāvasathasutta}}
\addcontentsline{toc}{section}{\tocacronym{SN 55.8} \toctranslation{In the Brick Hall (1st) } \tocroot{Paṭhamagiñjakāvasathasutta}}
\markboth{In the Brick Hall (1st) }{Paṭhamagiñjakāvasathasutta}
\extramarks{SN 55.8}{SN 55.8}

\scevam{So\marginnote{1.1} I have heard. }At one time the Buddha was staying at \textsanskrit{Nādika} in the brick house. Then Venerable Ānanda went up to the Buddha, bowed, sat down to one side, and said to him: 

“Sir,\marginnote{2.1} the monk named \textsanskrit{Sāḷha} has passed away. Where has he been reborn in his next life? The nun named \textsanskrit{Nandā}, the layman named Sudatta, and the laywoman named \textsanskrit{Sujātā} have passed away. Where have they been reborn in the next life?” 

“Ānanda,\marginnote{3.1} the monk \textsanskrit{Sāḷha} passed away having realized the undefiled freedom of heart and freedom by wisdom in this very life, having realized it with his own insight due to the ending of defilements. 

The\marginnote{3.2} nun \textsanskrit{Nandā} passed away having ended the five lower fetters. She’s been reborn spontaneously, and will be extinguished there, not liable to return from that world. 

The\marginnote{3.3} layman Sudatta passed away having ended three fetters, and weakened greed, hate, and delusion. He’s a once-returner; he will come back to this world once only, then make an end of suffering. 

The\marginnote{3.4} laywoman \textsanskrit{Sujātā} passed away having ended three fetters. She’s a stream-enterer, not liable to be reborn in the underworld, bound for awakening. 

It’s\marginnote{4.1} hardly surprising that a human being should pass away. But if you should come and ask me about it each and every time someone dies that would be a bother for me. So Ānanda, I will teach you the explanation of the Dhamma called ‘the mirror of the teaching’. A noble disciple who has this may declare of themselves: ‘I’ve finished with rebirth in hell, the animal realm, and the ghost realm. I’ve finished with all places of loss, bad places, the underworld. I am a stream-enterer! I’m not liable to be reborn in the underworld, and am bound for awakening.’ 

And\marginnote{5.1} what is that mirror of the teaching? 

It’s\marginnote{6.1} when a noble disciple has experiential confidence in the Buddha … the teaching … the \textsanskrit{Saṅgha} … And they have the ethical conduct loved by the noble ones … leading to immersion. This is that mirror of the teaching. A noble disciple who has this may declare of themselves: ‘I’ve finished with rebirth in hell, the animal realm, and the ghost realm. I’ve finished with all places of loss, bad places, the underworld. I am a stream-enterer! I’m not liable to be reborn in the underworld, and am bound for awakening.’” 

\scendsection{(The following two discourses have the same setting.) }

%
\section*{{\suttatitleacronym SN 55.9}{\suttatitletranslation At the Brick Hall (2nd) }{\suttatitleroot Dutiyagiñjakāvasathasutta}}
\addcontentsline{toc}{section}{\tocacronym{SN 55.9} \toctranslation{At the Brick Hall (2nd) } \tocroot{Dutiyagiñjakāvasathasutta}}
\markboth{At the Brick Hall (2nd) }{Dutiyagiñjakāvasathasutta}
\extramarks{SN 55.9}{SN 55.9}

Ānanda\marginnote{1.1} said to the Buddha: 

“Sir,\marginnote{1.2} the monk named Asoka has passed away. Where has he been reborn in his next life? The nun named \textsanskrit{Asokā}, the layman named Asoka, and the laywoman named \textsanskrit{Asokā} have passed away. Where have they been reborn in the next life?” 

“Ānanda,\marginnote{2.1} the monk Asoka passed away having realized the undefiled freedom of heart and freedom by wisdom in this very life … 

(And\marginnote{2.2} all is explained as in SN 55.8) 

This\marginnote{3.1} is that mirror of the teaching. A noble disciple who has this may declare of themselves: ‘I’ve finished with rebirth in hell, the animal realm, and the ghost realm. I’ve finished with all places of loss, bad places, the underworld. I am a stream-enterer! I’m not liable to be reborn in the underworld, and am bound for awakening.’” 

%
\section*{{\suttatitleacronym SN 55.10}{\suttatitletranslation At the Brick Hall (3rd) }{\suttatitleroot Tatiyagiñjakāvasathasutta}}
\addcontentsline{toc}{section}{\tocacronym{SN 55.10} \toctranslation{At the Brick Hall (3rd) } \tocroot{Tatiyagiñjakāvasathasutta}}
\markboth{At the Brick Hall (3rd) }{Tatiyagiñjakāvasathasutta}
\extramarks{SN 55.10}{SN 55.10}

Ānanda\marginnote{1.1} said to the Buddha: 

“Sir,\marginnote{1.2} the layman named \textsanskrit{Kakkaṭa} has passed away in \textsanskrit{Nādika}. Where has he been reborn in his next life? The laymen named \textsanskrit{Kaḷibha}, Nikata, \textsanskrit{Kaṭissaha}, \textsanskrit{Tuṭṭha}, \textsanskrit{Santuṭṭha}, Bhadda, and Subhadda have passed away in \textsanskrit{Nādika}. Where have they been reborn in the next life?” 

“Ānanda,\marginnote{2.1} the laymen \textsanskrit{Kakkaṭa}, \textsanskrit{Kaḷibha}, Nikata, \textsanskrit{Kaṭissaha}, \textsanskrit{Tuṭṭha}, \textsanskrit{Santuṭṭha}, Bhadda, and Subhadda passed away having ended the five lower fetters. They’ve been reborn spontaneously, and will be extinguished there, not liable to return from that world. 

Over\marginnote{3.1} fifty laymen in \textsanskrit{Nādika} have passed away having ended the five lower fetters. They’ve been reborn spontaneously, and will be extinguished there, not liable to return from that world. 

More\marginnote{3.2} than ninety laymen in \textsanskrit{Nādika} have passed away having ended three fetters, and weakened greed, hate, and delusion. They’re once-returners, who will come back to this world once only, then make an end of suffering. 

In\marginnote{3.3} excess of five hundred laymen in \textsanskrit{Nādika} have passed away having ended three fetters. They’re stream-enterers, not liable to be reborn in the underworld, bound for awakening. 

It’s\marginnote{4.1} hardly surprising that a human being should pass away. But if you should come and ask me about it each and every time someone passes away, that would be a bother for me. So Ānanda, I will teach you the explanation of the Dhamma called ‘the mirror of the teaching’. A noble disciple who has this may declare of themselves: ‘I’ve finished with rebirth in hell, the animal realm, and the ghost realm. I’ve finished with all places of loss, bad places, the underworld. I am a stream-enterer! I’m not liable to be reborn in the underworld, and am bound for awakening.’ 

And\marginnote{5.1} what is that mirror of the teaching? 

It’s\marginnote{6.1} when a noble disciple has experiential confidence in the Buddha … the teaching … the \textsanskrit{Saṅgha} … And they have the ethical conduct loved by the noble ones … leading to immersion. This is that mirror of the teaching. A noble disciple who has this may declare of themselves: ‘I’ve finished with rebirth in hell, the animal realm, and the ghost realm. I’ve finished with all places of loss, bad places, the underworld. I am a stream-enterer! I’m not liable to be reborn in the underworld, and am bound for awakening.’” 

%
\addtocontents{toc}{\let\protect\contentsline\protect\nopagecontentsline}
\chapter*{The Chapter on the Royal Monastery }
\addcontentsline{toc}{chapter}{\tocchapterline{The Chapter on the Royal Monastery }}
\addtocontents{toc}{\let\protect\contentsline\protect\oldcontentsline}

%
\section*{{\suttatitleacronym SN 55.11}{\suttatitletranslation A Saṅgha of a Thousand Nuns }{\suttatitleroot Sahassabhikkhunisaṁghasutta}}
\addcontentsline{toc}{section}{\tocacronym{SN 55.11} \toctranslation{A Saṅgha of a Thousand Nuns } \tocroot{Sahassabhikkhunisaṁghasutta}}
\markboth{A Saṅgha of a Thousand Nuns }{Sahassabhikkhunisaṁghasutta}
\extramarks{SN 55.11}{SN 55.11}

At\marginnote{1.1} one time the Buddha was staying near \textsanskrit{Sāvatthī} in the Royal Monastery. Then a \textsanskrit{Saṅgha} of a thousand nuns went up to the Buddha, bowed, and stood to one side. The Buddha said to them: 

“Nuns,\marginnote{2.1} a noble disciple who has four things is a stream-enterer, not liable to be reborn in the underworld, bound for awakening. What four? It’s when a noble disciple has experiential confidence in the Buddha … the teaching … the \textsanskrit{Saṅgha} … And they have the ethical conduct loved by the noble ones … leading to immersion. A noble disciple who has these four things is a stream-enterer, not liable to be reborn in the underworld, bound for awakening.” 

%
\section*{{\suttatitleacronym SN 55.12}{\suttatitletranslation The Brahmins }{\suttatitleroot Brāhmaṇasutta}}
\addcontentsline{toc}{section}{\tocacronym{SN 55.12} \toctranslation{The Brahmins } \tocroot{Brāhmaṇasutta}}
\markboth{The Brahmins }{Brāhmaṇasutta}
\extramarks{SN 55.12}{SN 55.12}

At\marginnote{1.1} \textsanskrit{Sāvatthī}. “Mendicants, the brahmins advocate a practice called ‘get up and go’. They encourage their disciples: ‘Please, good people, rising early you should face east and walk. Do not avoid a pit, a cliff, a stump, thorny ground, a swamp, or a sewer. You should await death in the place that you fall. And when your body breaks up, after death, you’ll be reborn in a good place, a heaven realm.’ 

But\marginnote{2.1} this practice of the brahmins is a foolish procedure, a stupid procedure. It doesn’t lead to disillusionment, dispassion, cessation, peace, insight, awakening, or extinguishment. But in the training of the Noble One I advocate a ‘get up and go’ practice which does lead solely to disillusionment, dispassion, cessation, peace, insight, awakening, and extinguishment. 

And\marginnote{3.1} what is that ‘get up and go’ practice? It’s when a noble disciple has experiential confidence in the Buddha … the teaching … the \textsanskrit{Saṅgha} … And they have the ethical conduct loved by the noble ones … leading to immersion. This is that ‘get up and go’ practice which does lead solely to disillusionment, dispassion, cessation, peace, insight, awakening, and extinguishment.” 

%
\section*{{\suttatitleacronym SN 55.13}{\suttatitletranslation With the Senior Monk Ānanda }{\suttatitleroot Ānandattherasutta}}
\addcontentsline{toc}{section}{\tocacronym{SN 55.13} \toctranslation{With the Senior Monk Ānanda } \tocroot{Ānandattherasutta}}
\markboth{With the Senior Monk Ānanda }{Ānandattherasutta}
\extramarks{SN 55.13}{SN 55.13}

At\marginnote{1.1} one time the venerables Ānanda and \textsanskrit{Sāriputta} were staying near \textsanskrit{Sāvatthī} in Jeta’s Grove, \textsanskrit{Anāthapiṇḍika}’s monastery. Then in the late afternoon, Venerable \textsanskrit{Sāriputta} came out of retreat, went to Venerable Ānanda, and exchanged greetings with him. When the greetings and polite conversation were over, he sat down to one side and said to him: 

“Reverend,\marginnote{1.4} how many things do people have to give up and how many do they have to possess in order for the Buddha to declare that they’re a stream-enterer, not liable to be reborn in the underworld, bound for awakening?” 

“Reverend,\marginnote{1.5} people have to give up four things and possess four things in order for the Buddha to declare that they’re a stream-enterer, not liable to be reborn in the underworld, bound for awakening. 

What\marginnote{2.1} four? They don’t have the distrust in the Buddha that causes an unlearned ordinary person to be reborn—when their body breaks up, after death—in a place of loss, a bad place, the underworld, hell. And they do have the experiential confidence in the Buddha that causes a learned noble disciple to be reborn—when their body breaks up, after death—in a good place, a heavenly realm. ‘That Blessed One is perfected, a fully awakened Buddha, accomplished in knowledge and conduct, holy, knower of the world, supreme guide for those who wish to train, teacher of gods and humans, awakened, blessed.’ 

They\marginnote{3.1} don’t have the distrust in the teaching that causes an unlearned ordinary person to be reborn—when their body breaks up, after death—in a place of loss, a bad place, the underworld, hell. And they do have the experiential confidence in the teaching that causes a learned noble disciple to be reborn—when their body breaks up, after death—in a good place, a heavenly realm. ‘The teaching is well explained by the Buddha—visible in this very life, immediately effective, inviting inspection, relevant, so that sensible people can know it for themselves.’ 

They\marginnote{4.1} don’t have the distrust in the \textsanskrit{Saṅgha} that causes an unlearned ordinary person to be reborn—when their body breaks up, after death—in a place of loss, a bad place, the underworld, hell. And they do have the experiential confidence in the \textsanskrit{Saṅgha} that causes a learned noble disciple to be reborn—when their body breaks up, after death—in a good place, a heavenly realm. ‘The \textsanskrit{Saṅgha} of the Buddha’s disciples is practicing the way that’s good, direct, methodical, and proper. It consists of the four pairs, the eight individuals. This \textsanskrit{Saṅgha} of the Buddha’s disciples is worthy of offerings dedicated to the gods, worthy of hospitality, worthy of a religious donation, and worthy of veneration with joined palms.’ 

They\marginnote{5.1} don’t have the unethical conduct that causes an unlearned ordinary person to be reborn—when their body breaks up, after death—in a place of loss, a bad place, the underworld, hell. And they do have the ethical conduct loved by the noble ones that causes a learned noble disciple to be reborn—when their body breaks up, after death—in a good place, a heavenly realm. Their ethical conduct is loved by the noble ones, unbroken, impeccable, spotless, and unmarred, liberating, praised by sensible people, not mistaken, and leading to immersion. People have to give up these four things and possess these four things in order for the Buddha to declare that they’re a stream-enterer, not liable to be reborn in the underworld, bound for awakening.” 

%
\section*{{\suttatitleacronym SN 55.14}{\suttatitletranslation Fear of the Bad Place }{\suttatitleroot Duggatibhayasutta}}
\addcontentsline{toc}{section}{\tocacronym{SN 55.14} \toctranslation{Fear of the Bad Place } \tocroot{Duggatibhayasutta}}
\markboth{Fear of the Bad Place }{Duggatibhayasutta}
\extramarks{SN 55.14}{SN 55.14}

“Mendicants,\marginnote{1.1} a noble disciple who has four things has gone beyond all fear of being reborn in a bad place. What four? It’s when a noble disciple has experiential confidence in the Buddha … the teaching … the \textsanskrit{Saṅgha} … And they have the ethical conduct loved by the noble ones … leading to immersion. A noble disciple who has these four things has gone beyond all fear of being reborn in a bad place.” 

%
\section*{{\suttatitleacronym SN 55.15}{\suttatitletranslation Fear of the Bad Place, the Underworld }{\suttatitleroot Duggativinipātabhayasutta}}
\addcontentsline{toc}{section}{\tocacronym{SN 55.15} \toctranslation{Fear of the Bad Place, the Underworld } \tocroot{Duggativinipātabhayasutta}}
\markboth{Fear of the Bad Place, the Underworld }{Duggativinipātabhayasutta}
\extramarks{SN 55.15}{SN 55.15}

“Mendicants,\marginnote{1.1} a noble disciple who has four things has gone beyond all fear of being reborn in a bad place, the underworld. What four? It’s when a noble disciple has experiential confidence in the Buddha … the teaching … the \textsanskrit{Saṅgha} … And they have the ethical conduct loved by the noble ones … leading to immersion. A noble disciple who has these four things has gone beyond all fear of being reborn in a bad place, the underworld.” 

%
\section*{{\suttatitleacronym SN 55.16}{\suttatitletranslation Friends and Colleagues (1st) }{\suttatitleroot Paṭhamamittāmaccasutta}}
\addcontentsline{toc}{section}{\tocacronym{SN 55.16} \toctranslation{Friends and Colleagues (1st) } \tocroot{Paṭhamamittāmaccasutta}}
\markboth{Friends and Colleagues (1st) }{Paṭhamamittāmaccasutta}
\extramarks{SN 55.16}{SN 55.16}

“Mendicants,\marginnote{1.1} those who you have sympathy for, and those worth listening to—friends and colleagues, relatives and family—should be encouraged, supported, and established in the four factors of stream-entry. What four? Experiential confidence in the Buddha … the teaching … the \textsanskrit{Saṅgha} … And the ethical conduct loved by the noble ones … leading to immersion. Those who you have sympathy for, and those worth listening to—friends and colleagues, relatives and family—should be encouraged, supported, and established in these four factors of stream-entry.” 

%
\section*{{\suttatitleacronym SN 55.17}{\suttatitletranslation Friends and Colleagues (2nd) }{\suttatitleroot Dutiyamittāmaccasutta}}
\addcontentsline{toc}{section}{\tocacronym{SN 55.17} \toctranslation{Friends and Colleagues (2nd) } \tocroot{Dutiyamittāmaccasutta}}
\markboth{Friends and Colleagues (2nd) }{Dutiyamittāmaccasutta}
\extramarks{SN 55.17}{SN 55.17}

“Mendicants,\marginnote{1.1} those who you have sympathy for, and those worth listening to—friends and colleagues, relatives and family—should be encouraged, supported, and established in the four factors of stream-entry. What four? Experiential confidence in the Buddha … 

There\marginnote{2.1} might be change in the four primary elements—earth, water, fire, and air—but a noble disciple with experiential confidence in the Buddha would never change. In this context, ‘change’ means that such a noble disciple will be reborn in hell, the animal realm, or the ghost realm: this is not possible. 

Experiential\marginnote{2.3} confidence in the teaching … 

Experiential\marginnote{2.4} confidence in the \textsanskrit{Saṅgha} … 

The\marginnote{2.5} ethical conduct loved by the noble ones … leading to immersion. There might be change in the four primary elements—earth, water, fire, and air—but a noble disciple with the ethical conduct loved by the noble ones would never change. In this context, ‘change’ means that such a noble disciple will be reborn in hell, the animal realm, or the ghost realm: this is not possible. 

Those\marginnote{2.8} who you have sympathy for, and those worth listening to—friends and colleagues, relatives and family—should be encouraged, supported, and established in these four factors of stream-entry.” 

%
\section*{{\suttatitleacronym SN 55.18}{\suttatitletranslation A Visit to the Gods (1st) }{\suttatitleroot Paṭhamadevacārikasutta}}
\addcontentsline{toc}{section}{\tocacronym{SN 55.18} \toctranslation{A Visit to the Gods (1st) } \tocroot{Paṭhamadevacārikasutta}}
\markboth{A Visit to the Gods (1st) }{Paṭhamadevacārikasutta}
\extramarks{SN 55.18}{SN 55.18}

At\marginnote{1.1} \textsanskrit{Sāvatthī}. And then Venerable \textsanskrit{Mahāmoggallāna}, as easily as a strong person would extend or contract their arm, vanished from Jeta’s Grove and reappeared among the gods of the Thirty-Three. Then several deities of the company of the Thirty-Three went up to Venerable \textsanskrit{Mahāmoggallāna}, bowed, and stood to one side. \textsanskrit{Moggallāna} said to them: 

“Reverends,\marginnote{2.1} it’s good to have experiential confidence in the Buddha. … It’s the reason why some sentient beings, when their body breaks up, after death, are reborn in a good place, a heavenly realm. It’s good to have experiential confidence in the teaching. … the \textsanskrit{Saṅgha} … and to have the ethical conduct that’s loved by the noble ones … leading to immersion. It’s the reason why some sentient beings, when their body breaks up, after death, are reborn in a good place, a heavenly realm.” 

“My\marginnote{3.1} good \textsanskrit{Moggallāna}, it’s good to have experiential confidence in the Buddha … It’s the reason why some sentient beings, when their body breaks up, after death, are reborn in a good place, a heavenly realm. It’s good to have experiential confidence in the teaching. … the \textsanskrit{Saṅgha} … and to have the ethical conduct that’s loved by the noble ones … leading to immersion. It’s the reason why some sentient beings, when their body breaks up, after death, are reborn in a good place, a heavenly realm.” 

%
\section*{{\suttatitleacronym SN 55.19}{\suttatitletranslation A Visit to the Gods (2nd) }{\suttatitleroot Dutiyadevacārikasutta}}
\addcontentsline{toc}{section}{\tocacronym{SN 55.19} \toctranslation{A Visit to the Gods (2nd) } \tocroot{Dutiyadevacārikasutta}}
\markboth{A Visit to the Gods (2nd) }{Dutiyadevacārikasutta}
\extramarks{SN 55.19}{SN 55.19}

At\marginnote{1.1} \textsanskrit{Sāvatthī}. 

And\marginnote{1.2} then Venerable \textsanskrit{Mahāmoggallāna}, as easily as a strong person would extend or contract their arm, vanished from Jeta’s Grove and reappeared among the gods of the Thirty-Three. Then several deities of the company of the Thirty-Three went up to Venerable \textsanskrit{Mahāmoggallāna}, bowed, and stood to one side. \textsanskrit{Moggallāna} said to them: 

“Reverends,\marginnote{2.1} it’s good to have experiential confidence in the Buddha. … It’s the reason why some sentient beings, when their body breaks up, after death, have been reborn in a good place, a heavenly realm. It’s good to have experiential confidence in the teaching. … the \textsanskrit{Saṅgha} … and to have the ethical conduct that’s loved by the noble ones … leading to immersion. It’s the reason why some sentient beings, when their body breaks up, after death, have been reborn in a good place, a heavenly realm.” 

“My\marginnote{3.1} good \textsanskrit{Moggallāna}, it’s good to have experiential confidence in the Buddha … It’s the reason why some sentient beings, when their body breaks up, after death, have been reborn in a good place, a heavenly realm. It’s good to have experiential confidence in the teaching. … the \textsanskrit{Saṅgha} … and to have the ethical conduct that’s loved by the noble ones … leading to immersion. It’s the reason why some sentient beings, when their body breaks up, after death, have been reborn in a good place, a heavenly realm.” 

%
\section*{{\suttatitleacronym SN 55.20}{\suttatitletranslation A Visit to the Gods (3rd) }{\suttatitleroot Tatiyadevacārikasutta}}
\addcontentsline{toc}{section}{\tocacronym{SN 55.20} \toctranslation{A Visit to the Gods (3rd) } \tocroot{Tatiyadevacārikasutta}}
\markboth{A Visit to the Gods (3rd) }{Tatiyadevacārikasutta}
\extramarks{SN 55.20}{SN 55.20}

Then\marginnote{1.1} the Buddha, as easily as a strong person would extend or contract their arm, vanished from Jeta’s Grove and reappeared among the gods of the Thirty-Three. Then several deities of the company of the Thirty-Three went up to the Buddha, bowed, and stood to one side. The Buddha said to them: 

“Reverends,\marginnote{2.1} it’s good to have experiential confidence in the Buddha. … It’s the reason why some sentient beings are stream-enterers, not liable to be reborn in the underworld, bound for awakening. It’s good to have experiential confidence in the teaching. … the \textsanskrit{Saṅgha} … and to have the ethical conduct that’s loved by the noble ones … leading to immersion. It’s the reason why some sentient beings are stream-enterers, not liable to be reborn in the underworld, bound for awakening.” 

“Good\marginnote{3.1} sir, it’s good to have experiential confidence in the Buddha … It’s the reason why some sentient beings are stream-enterers, not liable to be reborn in the underworld, bound for awakening. It’s good to have experiential confidence in the teaching. … the \textsanskrit{Saṅgha} … and to have the ethical conduct that’s loved by the noble ones … leading to immersion. It’s the reason why some sentient beings are stream-enterers, not liable to be reborn in the underworld, bound for awakening.” 

%
\addtocontents{toc}{\let\protect\contentsline\protect\nopagecontentsline}
\chapter*{The Chapter with Sarakāni }
\addcontentsline{toc}{chapter}{\tocchapterline{The Chapter with Sarakāni }}
\addtocontents{toc}{\let\protect\contentsline\protect\oldcontentsline}

%
\section*{{\suttatitleacronym SN 55.21}{\suttatitletranslation With Mahānāma (1st) }{\suttatitleroot Paṭhamamahānāmasutta}}
\addcontentsline{toc}{section}{\tocacronym{SN 55.21} \toctranslation{With Mahānāma (1st) } \tocroot{Paṭhamamahānāmasutta}}
\markboth{With Mahānāma (1st) }{Paṭhamamahānāmasutta}
\extramarks{SN 55.21}{SN 55.21}

\scevam{So\marginnote{1.1} I have heard. }At one time the Buddha was staying in the land of the Sakyans, near Kapilavatthu in the Banyan Tree Monastery. Then \textsanskrit{Mahānāma} the Sakyan went up to the Buddha, bowed, sat down to one side, and said to him: 

“Sir,\marginnote{1.4} this Kapilavatthu is successful and prosperous and full of people, with cramped cul-de-sacs. In the late afternoon, after paying homage to the Buddha or an esteemed mendicant, I enter Kapilavatthu. I encounter a stray elephant, horse, chariot, cart, or person. At that time I lose mindfulness regarding the Buddha, the teaching, and the \textsanskrit{Saṅgha}. I think: ‘If I were to die at this time, where would I be reborn in my next life?’” 

“Do\marginnote{2.1} not fear, \textsanskrit{Mahānāma}, do not fear! Your death will not be a bad one; your passing will not be a bad one. Take someone whose mind has for a long time been imbued with faith, ethics, learning, generosity, and wisdom. Their body consists of form, made up of the four primary elements, produced by mother and father, built up from rice and porridge, liable to impermanence, to wearing away and erosion, to breaking up and destruction. Right here the crows, vultures, hawks, dogs, jackals, and many kinds of little creatures devour it. But their mind rises up, headed for a higher place. 

Suppose\marginnote{3.1} a person was to sink a pot of ghee or oil into a deep lake and break it open. Its shards and chips would sink down, while the ghee or oil in it would rise up, headed for a higher place. 

In\marginnote{3.3} the same way, take someone whose mind has for a long time been imbued with faith, ethics, learning, generosity, and wisdom. Their body consists of form, made up of the four elements, produced by mother and father, built up from rice and porridge, liable to impermanence, to wearing away and erosion, to breaking up and destruction. Right here the crows, vultures, hawks, dogs, jackals, and many kinds of little creatures devour it. But their mind rises up, headed for a higher place. 

Your\marginnote{3.6} mind, \textsanskrit{Mahānāma}, has for a long time been imbued with faith, ethics, learning, generosity, and wisdom. Do not fear, \textsanskrit{Mahānāma}, do not fear! Your death will not be a bad one; your passing will not be a bad one.” 

%
\section*{{\suttatitleacronym SN 55.22}{\suttatitletranslation With Mahānāma (2nd) }{\suttatitleroot Dutiyamahānāmasutta}}
\addcontentsline{toc}{section}{\tocacronym{SN 55.22} \toctranslation{With Mahānāma (2nd) } \tocroot{Dutiyamahānāmasutta}}
\markboth{With Mahānāma (2nd) }{Dutiyamahānāmasutta}
\extramarks{SN 55.22}{SN 55.22}

\scevam{So\marginnote{1.1} I have heard. }At one time the Buddha was staying in the land of the Sakyans, near Kapilavatthu in the Banyan Tree Monastery. Then \textsanskrit{Mahānāma} the Sakyan went up to the Buddha, bowed, sat down to one side, and said to him: 

“Sir,\marginnote{1.4} this Kapilavatthu is successful and prosperous and full of people, with cramped cul-de-sacs. In the late afternoon, after paying homage to the Buddha or an esteemed mendicant, I enter Kapilavatthu. I encounter a stray elephant, horse, chariot, cart, or person. At that time I lose mindfulness regarding the Buddha, the teaching, and the \textsanskrit{Saṅgha}. I think: ‘If I were to die at this time, where would I be reborn in my next life?’” 

“Do\marginnote{2.1} not fear, \textsanskrit{Mahānāma}, do not fear! Your death will not be a bad one; your passing will not be a bad one. A noble disciple who has four things slants, slopes, and inclines towards extinguishment. What four? It’s when a noble disciple has experiential confidence in the Buddha … the teaching … the \textsanskrit{Saṅgha} … And they have the ethical conduct loved by the noble ones … leading to immersion. 

Suppose\marginnote{3.1} there was a tree that slants, slopes, and inclines to the east. If it was cut off at the root where would it fall?” 

“Sir,\marginnote{3.2} it would fall in the direction that it slants, slopes, and inclines.” 

“In\marginnote{3.3} the same way, a noble disciple who has four things slants, slopes, and inclines towards extinguishment.” 

%
\section*{{\suttatitleacronym SN 55.23}{\suttatitletranslation With Godhā the Sakyan }{\suttatitleroot Godhasakkasutta}}
\addcontentsline{toc}{section}{\tocacronym{SN 55.23} \toctranslation{With Godhā the Sakyan } \tocroot{Godhasakkasutta}}
\markboth{With Godhā the Sakyan }{Godhasakkasutta}
\extramarks{SN 55.23}{SN 55.23}

At\marginnote{1.1} Kapilavatthu. Then \textsanskrit{Mahānāma} the Sakyan went up to \textsanskrit{Godhā} the Sakyan, and said to him, “\textsanskrit{Godhā}, how many things must a person have for you to recognize them as a stream-enterer, not liable to be reborn in the underworld, bound for awakening?” 

“\textsanskrit{Mahānāma},\marginnote{2.1} a person must have three things for me to recognize them as a stream-enterer. What three? It’s when a noble disciple has experiential confidence in the Buddha … the teaching … and the \textsanskrit{Saṅgha} … When a person has these three things I recognize them as a stream-enterer. 

But\marginnote{3.1} \textsanskrit{Mahānāma}, how many things must a person have for \emph{you} to recognize them as a stream-enterer?” 

“\textsanskrit{Godhā},\marginnote{3.2} a person must have four things for me to recognize them as a stream-enterer. What four? It’s when a noble disciple has experiential confidence in the Buddha … the teaching … the \textsanskrit{Saṅgha} … And they have the ethical conduct loved by the noble ones … leading to immersion. When a person has these four things I recognize them as a stream-enterer.” 

“Hold\marginnote{4.1} on, \textsanskrit{Mahānāma}, hold on! Only the Buddha would know whether or not they have these things.” 

“Come,\marginnote{4.3} \textsanskrit{Godhā}, let’s go to the Buddha and inform him about this.” 

Then\marginnote{4.4} \textsanskrit{Mahānāma} and \textsanskrit{Godhā} went to the Buddha, bowed, and sat down to one side. \textsanskrit{Mahānāma} told the Buddha all that had happened, and then said: 

“Sir,\marginnote{8.1} some issue regarding the teaching might come up. The Buddha might take one side, and the \textsanskrit{Saṅgha} of monks the other. I’d side with the Buddha. May the Buddha remember me as having such confidence. Some issue regarding the teaching might come up. The Buddha might take one side, and the \textsanskrit{Saṅgha} of monks and the \textsanskrit{Saṅgha} of nuns the other. … The Buddha might take one side, and the \textsanskrit{Saṅgha} of monks and the \textsanskrit{Saṅgha} of nuns and the laymen the other. … The Buddha might take one side, and the \textsanskrit{Saṅgha} of monks and the \textsanskrit{Saṅgha} of nuns and the laymen and the laywomen the other. … The Buddha might take one side, and the \textsanskrit{Saṅgha} of monks and the \textsanskrit{Saṅgha} of nuns and the laymen and the laywomen and the world—with its gods, \textsanskrit{Māras} and \textsanskrit{Brahmās}, this population with its ascetics and brahmins, gods and humans—the other. I’d side with the Buddha. May the Buddha remember me as having such confidence.” 

“\textsanskrit{Godhā},\marginnote{8.19} what do you have to say to \textsanskrit{Mahānāma} when he speaks like this?” 

“Sir,\marginnote{8.20} I have nothing to say to \textsanskrit{Mahānāma} when he speaks like this, except what is good and wholesome.” 

%
\section*{{\suttatitleacronym SN 55.24}{\suttatitletranslation About Sarakāni (1st) }{\suttatitleroot Paṭhamasaraṇānisakkasutta}}
\addcontentsline{toc}{section}{\tocacronym{SN 55.24} \toctranslation{About Sarakāni (1st) } \tocroot{Paṭhamasaraṇānisakkasutta}}
\markboth{About Sarakāni (1st) }{Paṭhamasaraṇānisakkasutta}
\extramarks{SN 55.24}{SN 55.24}

At\marginnote{1.1} Kapilavatthu. 

Now\marginnote{1.2} at that time \textsanskrit{Sarakāni} the Sakyan had passed away. The Buddha declared that he was a stream-enterer, not liable to be reborn in the underworld, bound for awakening. 

At\marginnote{1.5} that, several Sakyans came together complaining, grumbling, and objecting, “It’s incredible, it’s amazing! Who can’t become a stream-enterer these days? For the Buddha even declared \textsanskrit{Sarakāni} to be a stream-enterer after he passed away. \textsanskrit{Sarakāni} was too weak for the training; he used to drink alcohol.” 

Then\marginnote{2.1} \textsanskrit{Mahānāma} the Sakyan went up to the Buddha, bowed, sat down to one side, and told him what had happened. The Buddha said: 

“\textsanskrit{Mahānāma},\marginnote{3.1} when a lay follower has for a long time gone for refuge to the Buddha, the teaching, and the \textsanskrit{Saṅgha}, how could they go to the underworld? And if anyone should rightly be said to have for a long time gone for refuge to the Buddha, the teaching, and the \textsanskrit{Saṅgha}, it’s \textsanskrit{Sarakāni} the Sakyan. \textsanskrit{Sarakāni} the Sakyan has for a long time gone for refuge to the Buddha, the teaching, and the \textsanskrit{Saṅgha}. How could he go to the underworld? 

Take\marginnote{4.1} a certain person who has experiential confidence in the Buddha … the teaching … the \textsanskrit{Saṅgha} … They have laughing wisdom and swift wisdom, and are endowed with freedom. They’ve realized the undefiled freedom of heart and freedom by wisdom in this very life. And they live having realized it with their own insight due to the ending of defilements. This person is exempt from hell, the animal realm, and the ghost realm. They’re exempt from places of loss, bad places, the underworld. 

Take\marginnote{5.1} another person who has experiential confidence in the Buddha … the teaching … the \textsanskrit{Saṅgha} … They have laughing wisdom and swift wisdom, but are not endowed with freedom. With the ending of the five lower fetters they’re reborn spontaneously. They are extinguished there, and are not liable to return from that world. This person, too, is exempt from hell, the animal realm, and the ghost realm. They’re exempt from places of loss, bad places, the underworld. 

Take\marginnote{6.1} another person who has experiential confidence in the Buddha … the teaching … the \textsanskrit{Saṅgha} … But they don’t have laughing wisdom or swift wisdom, nor are they endowed with freedom. With the ending of three fetters, and the weakening of greed, hate, and delusion, they’re a once-returner. They come back to this world once only, then make an end of suffering. This person, too, is exempt from hell, the animal realm, and the ghost realm. They’re exempt from places of loss, bad places, the underworld. 

Take\marginnote{7.1} another person who has experiential confidence in the Buddha … the teaching … the \textsanskrit{Saṅgha} … But they don’t have laughing wisdom or swift wisdom, nor are they endowed with freedom. With the ending of three fetters they’re a stream-enterer, not liable to be reborn in the underworld, bound for awakening. This person, too, is exempt from hell, the animal realm, and the ghost realm. They’re exempt from places of loss, bad places, the underworld. 

Take\marginnote{8.1} another person who doesn’t have experiential confidence in the Buddha … the teaching … the \textsanskrit{Saṅgha} … They don’t have laughing wisdom or swift wisdom, nor are they endowed with freedom. Still, they have these qualities: the faculties of faith, energy, mindfulness, immersion, and wisdom. And they accept the principles proclaimed by the Realized One after considering them with a degree of wisdom. This person, too, doesn’t go to hell, the animal realm, and the ghost realm. They don’t go to places of loss, bad places, the underworld. 

Take\marginnote{9.1} another person who doesn’t have experiential confidence in the Buddha … the teaching … the \textsanskrit{Saṅgha} … They don’t have laughing wisdom or swift wisdom, nor are they endowed with freedom. Still, they have these qualities: the faculties of faith, energy, mindfulness, immersion, and wisdom. And they have a degree of faith and love for the Buddha. This person, too, doesn’t go to hell, the animal realm, and the ghost realm. They don’t go to places of loss, bad places, the underworld. 

If\marginnote{9.9} these great sal trees could understand what was well said and poorly said, I’d declare them to be stream-enterers. Why can’t this apply to \textsanskrit{Sarakāni}? \textsanskrit{Mahānāma}, \textsanskrit{Sarakāni} the Sakyan undertook the training at the time of his death.” 

%
\section*{{\suttatitleacronym SN 55.25}{\suttatitletranslation About Sarakāni the Sakyan (2nd) }{\suttatitleroot Dutiyasaraṇānisakkasutta}}
\addcontentsline{toc}{section}{\tocacronym{SN 55.25} \toctranslation{About Sarakāni the Sakyan (2nd) } \tocroot{Dutiyasaraṇānisakkasutta}}
\markboth{About Sarakāni the Sakyan (2nd) }{Dutiyasaraṇānisakkasutta}
\extramarks{SN 55.25}{SN 55.25}

At\marginnote{1.1} Kapilavatthu. 

Now\marginnote{1.2} at that time \textsanskrit{Sarakāni} the Sakyan had passed away. The Buddha declared that he was a stream-enterer, not liable to be reborn in the underworld, bound for awakening. 

At\marginnote{1.5} that, several Sakyans came together complaining, grumbling, and objecting, “It’s incredible, it’s amazing! Who can’t become a stream-enterer these days? For the Buddha even declared \textsanskrit{Sarakāni} to be a stream-enterer after he passed away. \textsanskrit{Sarakāni} didn’t fulfill the training.” 

Then\marginnote{1.12} \textsanskrit{Mahānāma} the Sakyan went up to the Buddha, bowed, sat down to one side, and told him what had happened. The Buddha said: 

“\textsanskrit{Mahānāma},\marginnote{3.1} when a lay follower has for a long time gone for refuge to the Buddha, the teaching, and the \textsanskrit{Saṅgha}, how could they go to the underworld? And if anyone should rightly be said to have for a long time gone for refuge to the Buddha, the teaching, and the \textsanskrit{Saṅgha}, it’s \textsanskrit{Sarakāni} the Sakyan. \textsanskrit{Sarakāni} the Sakyan has for a long time gone for refuge to the Buddha, the teaching, and the \textsanskrit{Saṅgha}. 

Take\marginnote{4.1} a certain person who is sure and devoted to the Buddha … the teaching … the \textsanskrit{Saṅgha} … They have laughing wisdom and swift wisdom, and are endowed with freedom. They realize the undefiled freedom of heart and freedom by wisdom in this very life. And they live having realized it with their own insight due to the ending of defilements. This person is exempt from hell, the animal realm, and the ghost realm. They’re exempt from places of loss, bad places, the underworld. 

Take\marginnote{5.1} another person who is sure and devoted to the Buddha … the teaching … the \textsanskrit{Saṅgha} … They have laughing wisdom and swift wisdom, and are endowed with freedom. With the ending of the five lower fetters, they’re extinguished between one life and the next … they’re extinguished upon landing … they’re extinguished without extra effort … they’re extinguished with extra effort … they head upstream, going to the \textsanskrit{Akaniṭṭha} realm. This person, too, is exempt from hell, the animal realm, and the ghost realm. They’re exempt from places of loss, bad places, the underworld. 

Take\marginnote{6.1} another person who is sure and devoted to the Buddha … the teaching … the \textsanskrit{Saṅgha} … But they don’t have laughing wisdom or swift wisdom, nor are they endowed with freedom. With the ending of three fetters, and the weakening of greed, hate, and delusion, they’re a once-returner. They come back to this world once only, then make an end of suffering. This person, too, is exempt from hell, the animal realm, and the ghost realm. They’re exempt from places of loss, bad places, the underworld. 

Take\marginnote{7.1} another person who is sure and devoted to the Buddha … the teaching … the \textsanskrit{Saṅgha} … But they don’t have laughing wisdom or swift wisdom, nor are they endowed with freedom. With the ending of three fetters they’re a stream-enterer, not liable to be reborn in the underworld, bound for awakening. This person, too, is exempt from hell, the animal realm, and the ghost realm. They’re exempt from places of loss, bad places, the underworld. 

Take\marginnote{8.1} another person who isn’t sure or devoted to the Buddha … the teaching … the \textsanskrit{Saṅgha} … They don’t have laughing wisdom or swift wisdom, nor are they endowed with freedom. Still, they have these qualities: the faculties of faith, energy, mindfulness, immersion, and wisdom. And they accept the principles proclaimed by the Realized One after considering them with a degree of wisdom. This person, too, doesn’t go to hell, the animal realm, and the ghost realm. They don’t go to places of loss, bad places, the underworld. 

Take\marginnote{9.1} another person who isn’t sure or devoted to the Buddha … the teaching … the \textsanskrit{Saṅgha} … They don’t have laughing wisdom or swift wisdom, nor are they endowed with freedom. Still, they have these qualities: the faculties of faith, energy, mindfulness, immersion, and wisdom. And they have a degree of faith and love for the Buddha. This person, too, doesn’t go to hell, the animal realm, and the ghost realm. They don’t go to places of loss, bad places, the underworld. 

Suppose\marginnote{10.1} there was a barren field, a barren ground, with uncleared stumps. And you had seeds that were broken, spoiled, weather-damaged, infertile, and ill kept. And the heavens didn’t provide enough rain. Would those seeds grow, increase, and mature?” 

“No,\marginnote{10.3} sir.” 

“In\marginnote{10.4} the same way, take a teaching that’s badly explained and badly propounded, not emancipating, not leading to peace, proclaimed by someone who is not a fully awakened Buddha. This is what I call a barren field. A disciple remains in such a teaching, practicing in line with that teaching, practicing it properly, living in line with that teaching. This is what I call a bad seed. 

Suppose\marginnote{11.1} there was a fertile field, a fertile ground, well-cleared of stumps. And you had seeds that were intact, unspoiled, not weather-damaged, fertile, and well-kept. And there’s plenty of rainfall. Would those seeds grow, increase, and mature?” 

“Yes,\marginnote{11.4} sir.” 

“In\marginnote{11.5} the same way, take a teaching that’s well explained and well propounded, emancipating, leading to peace, proclaimed by someone who is a fully awakened Buddha. This is what I call a fertile field. A disciple remains in such a teaching, practicing in line with that teaching, practicing it properly, living in line with that teaching. This is what I call a good seed. Why can’t this apply to \textsanskrit{Sarakāni}? \textsanskrit{Mahānāma}, \textsanskrit{Sarakāni} the Sakyan fulfilled the training at the time of his death.” 

%
\section*{{\suttatitleacronym SN 55.26}{\suttatitletranslation Anāthapiṇḍika (1st) }{\suttatitleroot Paṭhamaanāthapiṇḍikasutta}}
\addcontentsline{toc}{section}{\tocacronym{SN 55.26} \toctranslation{Anāthapiṇḍika (1st) } \tocroot{Paṭhamaanāthapiṇḍikasutta}}
\markboth{Anāthapiṇḍika (1st) }{Paṭhamaanāthapiṇḍikasutta}
\extramarks{SN 55.26}{SN 55.26}

At\marginnote{1.1} \textsanskrit{Sāvatthī}. 

Now\marginnote{1.2} at that time the householder \textsanskrit{Anāthapiṇḍika} was sick, suffering, gravely ill. Then he addressed a man, “Please, mister, go to Venerable \textsanskrit{Sāriputta}, and in my name bow with your head to his feet. Say to him: ‘Sir, the householder \textsanskrit{Anāthapiṇḍika} is sick, suffering, gravely ill. He bows with his head to your feet.’ And then say: ‘Sir, please visit him at his home out of compassion.’” 

“Yes,\marginnote{2.1} sir,” that man replied. He did as \textsanskrit{Anāthapiṇḍika} asked. \textsanskrit{Sāriputta} consented in silence. 

Then\marginnote{4.1} Venerable \textsanskrit{Sāriputta} robed up in the morning and, taking his bowl and robe, went with Venerable Ānanda as his second monk to \textsanskrit{Anāthapiṇḍika}’s home. He sat down on the seat spread out, and said to \textsanskrit{Anāthapiṇḍika}, “I hope you’re keeping well, householder; I hope you’re alright. And I hope the pain is fading, not growing, that its fading is evident, not its growing.” 

“Sir,\marginnote{4.3} I’m not keeping well, I’m not alright. The pain is terrible and growing, not fading; its growing is evident, not its fading.” 

“Householder,\marginnote{5.1} you don’t have the distrust in the Buddha that causes an unlearned ordinary person to be reborn—when their body breaks up, after death—in a place of loss, a bad place, the underworld, hell. And you have experiential confidence in the Buddha: ‘That Blessed One is perfected, a fully awakened Buddha, accomplished in knowledge and conduct, holy, knower of the world, supreme guide for those who wish to train, teacher of gods and humans, awakened, blessed.’ Seeing in yourself that experiential confidence in the teaching, that pain may die down on the spot. 

You\marginnote{6.1} don’t have the distrust in the teaching that causes an unlearned ordinary person to be reborn—when their body breaks up, after death—in a place of loss, a bad place, the underworld, hell. And you have experiential confidence in the teaching: ‘The teaching is well explained by the Buddha—visible in this very life, immediately effective, inviting inspection, relevant, so that sensible people can know it for themselves.’ Seeing in yourself that experiential confidence in the teaching, that pain may die down on the spot. 

You\marginnote{7.1} don’t have the distrust in the \textsanskrit{Saṅgha} that causes an unlearned ordinary person to be reborn—when their body breaks up, after death—in a place of loss, a bad place, the underworld, hell. And you have experiential confidence in the \textsanskrit{Saṅgha}: ‘The \textsanskrit{Saṅgha} of the Buddha’s disciples is practicing the way that’s good, direct, methodical, and proper. It consists of the four pairs, the eight individuals. This \textsanskrit{Saṅgha} of the Buddha’s disciples is worthy of offerings dedicated to the gods, worthy of hospitality, worthy of a religious donation, and worthy of veneration with joined palms. It is the supreme field of merit for the world.’ Seeing in yourself that experiential confidence in the \textsanskrit{Saṅgha}, that pain may die down on the spot. 

You\marginnote{8.1} don’t have the unethical conduct that causes an unlearned ordinary person to be reborn—when their body breaks up, after death—in a place of loss, a bad place, the underworld, hell. Your ethical conduct is loved by the noble ones, unbroken, impeccable, spotless, and unmarred, liberating, praised by sensible people, not mistaken, and leading to immersion. Seeing in yourself that ethical conduct loved by the noble ones, that pain may die down on the spot. 

You\marginnote{9.1} don’t have the wrong view that causes an unlearned ordinary person to be reborn—when their body breaks up, after death—in a place of loss, a bad place, the underworld, hell. You have right view. Seeing in yourself that right view, that pain may die down on the spot. 

You\marginnote{10.1} don’t have the wrong thought … 

wrong\marginnote{11.1} speech … 

wrong\marginnote{12.1} action … 

wrong\marginnote{13.1} livelihood … 

wrong\marginnote{14.1} effort … 

wrong\marginnote{15.1} mindfulness … 

wrong\marginnote{16.1} immersion … 

wrong\marginnote{17.1} knowledge … 

wrong\marginnote{18.1} freedom … You have right freedom. Seeing in yourself that right freedom, that pain may die down on the spot.” 

And\marginnote{19.1} then \textsanskrit{Anāthapiṇḍika}’s pain died down on the spot. Then he served \textsanskrit{Sāriputta} and Ānanda from his own dish. When \textsanskrit{Sāriputta} had eaten and washed his hand and bowl, \textsanskrit{Anāthapiṇḍika} took a low seat and sat to one side. Venerable \textsanskrit{Sāriputta} expressed his appreciation to him with these verses. 

\begin{verse}%
“Whoever\marginnote{20.1} has faith in the Realized One, \\
unwavering and well grounded; \\
whose ethical conduct is good, \\
praised and loved by the noble ones; 

who\marginnote{21.1} has confidence in the \textsanskrit{Saṅgha}, \\
and correct view: \\
they’re said to be prosperous, \\
their life is not in vain. 

So\marginnote{22.1} let the wise devote themselves \\
to faith, ethical behaviour, \\
confidence, and insight into the teaching, \\
remembering the instructions of the Buddhas.” 

%
\end{verse}

After\marginnote{23.1} expressing his appreciation to \textsanskrit{Anāthapiṇḍika} with these verses, \textsanskrit{Sāriputta} got up from his seat and left. 

Then\marginnote{24.1} Ānanda went up to the Buddha, bowed, and sat down to one side. The Buddha said to him, “So, Ānanda, where are you coming from in the middle of the day?” 

“Sir,\marginnote{24.3} Venerable \textsanskrit{Sāriputta} advised the householder \textsanskrit{Anāthapiṇḍika} in this way and that.” 

“\textsanskrit{Sāriputta}\marginnote{24.4} is astute, Ānanda. He has great wisdom, since he can analyze the four factors of stream-entry in ten respects.” 

%
\section*{{\suttatitleacronym SN 55.27}{\suttatitletranslation With Anāthapiṇḍika (2nd) }{\suttatitleroot Dutiyaanāthapiṇḍikasutta}}
\addcontentsline{toc}{section}{\tocacronym{SN 55.27} \toctranslation{With Anāthapiṇḍika (2nd) } \tocroot{Dutiyaanāthapiṇḍikasutta}}
\markboth{With Anāthapiṇḍika (2nd) }{Dutiyaanāthapiṇḍikasutta}
\extramarks{SN 55.27}{SN 55.27}

At\marginnote{1.1} \textsanskrit{Sāvatthī}. 

Now\marginnote{1.2} at that time the householder \textsanskrit{Anāthapiṇḍika} was sick, suffering, gravely ill. Then he addressed a man, “Please, mister, go to Venerable Ānanda, and in my name bow with your head to his feet. Say to him: ‘Sir, the householder \textsanskrit{Anāthapiṇḍika} is sick, suffering, gravely ill. He bows with his head to your feet.’ And then say: ‘Sir, please visit him at his home out of compassion.’” 

“Yes,\marginnote{2.1} sir,” that man replied. He did as \textsanskrit{Anāthapiṇḍika} asked. Ānanda consented in silence. 

Then\marginnote{3.1} Venerable Ānanda robed up in the morning and, taking his bowl and robe, went to the home of the householder \textsanskrit{Anāthapiṇḍika}. He sat down on the seat spread out and said to \textsanskrit{Anāthapiṇḍika}, “I hope you’re keeping well, householder; I hope you’re alright. And I hope the pain is fading, not growing; that its fading, not its growing, is apparent.” 

“Sir,\marginnote{3.3} I’m not keeping well, I’m not alright. The pain is terrible and growing, not fading; its growing is evident, not its fading.” 

“Householder,\marginnote{4.1} when an unlearned ordinary person has four things, they’re frightened and terrified, and fear what awaits them after death. What four? Firstly, an unlearned ordinary person distrusts the Buddha. Seeing in themselves that distrust of the Buddha, they’re frightened and terrified, and fear what awaits them after death. 

Furthermore,\marginnote{5.1} an unlearned ordinary person distrusts the teaching … 

Furthermore,\marginnote{6.1} an unlearned ordinary person distrusts the \textsanskrit{Saṅgha} … 

Furthermore,\marginnote{7.1} an unlearned ordinary person has unethical conduct. Seeing in themselves that unethical conduct, they’re frightened and terrified, and fear what awaits them after death. When an unlearned ordinary person has these four things, they’re frightened and terrified, and fear what awaits them after death. 

When\marginnote{8.1} a learned noble disciple has four things, they’re not frightened or terrified, and don’t fear what awaits them after death. What four? Firstly, a noble disciple has experiential confidence in the Buddha … Seeing in themselves that experiential confidence in the Buddha, they’re not frightened or terrified, and don’t fear what awaits them after death. 

Furthermore,\marginnote{9.1} a noble disciple has experiential confidence in the teaching … 

Furthermore,\marginnote{10.1} a noble disciple has experiential confidence in the \textsanskrit{Saṅgha} … 

Furthermore,\marginnote{11.1} a noble disciple’s ethical conduct is loved by the noble ones, unbroken, impeccable, spotless, and unmarred, liberating, praised by sensible people, not mistaken, and leading to immersion. Seeing in themselves that ethical conduct loved by the noble ones, they’re not frightened or terrified, and don’t fear what awaits them after death. 

When\marginnote{11.3} a learned noble disciple has these four things, they’re not frightened or terrified, and don’t fear what awaits them after death.” 

“Sir,\marginnote{12.1} Ānanda, I am not afraid. What have I to fear? For I have experiential confidence in the Buddha … the teaching … the \textsanskrit{Saṅgha} … And of the training rules appropriate for laypeople taught by the Buddha, I don’t see any that I have broken.” 

“You’re\marginnote{12.9} fortunate, householder, so very fortunate, You have declared the fruit of stream-entry.” 

%
\section*{{\suttatitleacronym SN 55.28}{\suttatitletranslation Dangers and Threats (1st) }{\suttatitleroot Paṭhamabhayaverūpasantasutta}}
\addcontentsline{toc}{section}{\tocacronym{SN 55.28} \toctranslation{Dangers and Threats (1st) } \tocroot{Paṭhamabhayaverūpasantasutta}}
\markboth{Dangers and Threats (1st) }{Paṭhamabhayaverūpasantasutta}
\extramarks{SN 55.28}{SN 55.28}

At\marginnote{1.1} \textsanskrit{Sāvatthī}. 

Seated\marginnote{1.2} to one side, the Buddha said to the householder \textsanskrit{Anāthapiṇḍika}: 

“Householder,\marginnote{1.3} when a noble disciple has quelled five dangers and threats, has the four factors of stream-entry, and has clearly seen and comprehended the noble cycle with wisdom, they may, if they wish, declare of themselves: ‘I’ve finished with rebirth in hell, the animal realm, and the ghost realm. I’ve finished with all places of loss, bad places, the underworld. I am a stream-enterer! I’m not liable to be reborn in the underworld, and am bound for awakening.’ 

What\marginnote{2.1} are the five dangers and threats they have quelled? Anyone who kills living creatures creates dangers and threats both in the present life and in lives to come, and experiences mental pain and sadness. So that danger and threat is quelled for anyone who refrains from killing living creatures. Anyone who steals … Anyone who commits sexual misconduct … Anyone who lies … Anyone who uses alcoholic drinks that cause negligence creates dangers and threats both in the present life and in lives to come, and experiences mental pain and sadness. So that danger and threat is quelled for anyone who refrains from alcoholic drinks that cause negligence. These are the five dangers and threats they have quelled. 

What\marginnote{3.1} are the four factors of stream-entry that they have? It’s when a noble disciple has experiential confidence in the Buddha … the teaching … the \textsanskrit{Saṅgha} … And they have the ethical conduct loved by the noble ones … leading to immersion. These are the four factors of stream-entry that they have. 

And\marginnote{4.1} what is the noble cycle that they have clearly seen and comprehended with wisdom? A noble disciple properly attends to dependent origination itself: ‘When this exists, that is; due to the arising of this, that arises. When this doesn’t exist, that is not; due to the cessation of this, that ceases.' That is: Ignorance is a condition for choices. Choices are a condition for consciousness. Consciousness is a condition for name and form. Name and form are a condition for the six sense fields. The six sense fields are conditions for contact. Contact is a condition for feeling. Feeling is a condition for craving. Craving is a condition for grasping. Grasping is a condition for continued existence. Continued existence is a condition for rebirth. Rebirth is a condition for old age and death, sorrow, lamentation, pain, sadness, and distress to come to be. That is how this entire mass of suffering originates. When ignorance fades away and ceases with nothing left over, choices cease. When choices cease, consciousness ceases. When consciousness ceases, name and form cease. When name and form cease, the six sense fields cease. When the six sense fields cease, contact ceases. When contact ceases, feeling ceases. When feeling ceases, craving ceases. When craving ceases, grasping ceases. When grasping ceases, continued existence ceases. When continued existence ceases, rebirth ceases. When rebirth ceases, old age and death, sorrow, lamentation, pain, sadness, and distress cease. That is how this entire mass of suffering ceases. This is the noble cycle that they have clearly seen and comprehended with wisdom. 

When\marginnote{5.1} a noble disciple has quelled five dangers and threats, has the four factors of stream-entry, and has clearly seen and comprehended the noble cycle with wisdom, they may, if they wish, declare of themselves: ‘I’ve finished with rebirth in hell, the animal realm, and the ghost realm. I’ve finished with all places of loss, bad places, the underworld. I am a stream-enterer! I’m not liable to be reborn in the underworld, and am bound for awakening.’” 

%
\section*{{\suttatitleacronym SN 55.29}{\suttatitletranslation Dangers and Threats (2nd) }{\suttatitleroot Dutiyabhayaverūpasantasutta}}
\addcontentsline{toc}{section}{\tocacronym{SN 55.29} \toctranslation{Dangers and Threats (2nd) } \tocroot{Dutiyabhayaverūpasantasutta}}
\markboth{Dangers and Threats (2nd) }{Dutiyabhayaverūpasantasutta}
\extramarks{SN 55.29}{SN 55.29}

At\marginnote{1.1} \textsanskrit{Sāvatthī}. … “Mendicants, when a noble disciple has quelled five dangers and threats, has the four factors of stream-entry, and has clearly seen and comprehended the noble cycle with wisdom, they may, if they wish, declare of themselves: ‘I’ve finished with rebirth in hell, the animal realm, and the ghost realm. I’ve finished with all places of loss, bad places, the underworld. I am a stream-enterer! I’m not liable to be reborn in the underworld, and am bound for awakening.’” 

%
\section*{{\suttatitleacronym SN 55.30}{\suttatitletranslation With Nandaka the Licchavi }{\suttatitleroot Nandakalicchavisutta}}
\addcontentsline{toc}{section}{\tocacronym{SN 55.30} \toctranslation{With Nandaka the Licchavi } \tocroot{Nandakalicchavisutta}}
\markboth{With Nandaka the Licchavi }{Nandakalicchavisutta}
\extramarks{SN 55.30}{SN 55.30}

At\marginnote{1.1} one time the Buddha was staying near \textsanskrit{Vesālī}, at the Great Wood, in the hall with the peaked roof. Then Nandaka the Licchavi minister went up to the Buddha, bowed, and sat down to one side. The Buddha said to him: 

“Nandaka,\marginnote{2.1} a noble disciple who has four things is a stream-enterer, not liable to be reborn in the underworld, bound for awakening. What four? It’s when a noble disciple has experiential confidence in the Buddha … the teaching … the \textsanskrit{Saṅgha} … And they have the ethical conduct loved by the noble ones … leading to immersion. A noble disciple who has these four things is a stream-enterer, not liable to be reborn in the underworld, bound for awakening. 

A\marginnote{3.1} noble disciple who has these four things is guaranteed long life, beauty, happiness, fame, and sovereignty, both human and divine. Now, I don’t say this because I’ve heard it from some other ascetic or brahmin. I only say it because I’ve known, seen, and realized it for myself.” 

When\marginnote{4.1} he had spoken, a certain person said to Nandaka: 

“Sir,\marginnote{4.2} it is time to bathe.” 

“Enough\marginnote{4.3} now, my man, with that exterior bath. This interior bathing will do for me, that is, confidence in the Buddha.” 

%
\addtocontents{toc}{\let\protect\contentsline\protect\nopagecontentsline}
\chapter*{The Chapter on Overflowing Merit }
\addcontentsline{toc}{chapter}{\tocchapterline{The Chapter on Overflowing Merit }}
\addtocontents{toc}{\let\protect\contentsline\protect\oldcontentsline}

%
\section*{{\suttatitleacronym SN 55.31}{\suttatitletranslation Overflowing Merit (1st) }{\suttatitleroot Paṭhamapuññābhisandasutta}}
\addcontentsline{toc}{section}{\tocacronym{SN 55.31} \toctranslation{Overflowing Merit (1st) } \tocroot{Paṭhamapuññābhisandasutta}}
\markboth{Overflowing Merit (1st) }{Paṭhamapuññābhisandasutta}
\extramarks{SN 55.31}{SN 55.31}

At\marginnote{1.1} \textsanskrit{Sāvatthī}. 

“Mendicants,\marginnote{1.2} there are these four kinds of overflowing merit, overflowing goodness that nurture happiness. What four? Firstly, a noble disciple has experiential confidence in the Buddha … This is the first kind of overflowing merit, overflowing goodness that nurtures happiness. 

Furthermore,\marginnote{2.1} a noble disciple has experiential confidence in the teaching … This is the second kind of overflowing merit, overflowing goodness that nurtures happiness. 

Furthermore,\marginnote{3.1} a noble disciple has experiential confidence in the \textsanskrit{Saṅgha} … This is the third kind of overflowing merit, overflowing goodness that nurtures happiness. 

Furthermore,\marginnote{4.1} a noble disciple’s ethical conduct is loved by the noble ones, unbroken, impeccable, spotless, and unmarred, liberating, praised by sensible people, not mistaken, and leading to immersion. This is the fourth kind of overflowing merit, overflowing goodness that nurtures happiness. These are the four kinds of overflowing merit, overflowing goodness that nurture happiness.” 

%
\section*{{\suttatitleacronym SN 55.32}{\suttatitletranslation Overflowing Merit (2nd) }{\suttatitleroot Dutiyapuññābhisandasutta}}
\addcontentsline{toc}{section}{\tocacronym{SN 55.32} \toctranslation{Overflowing Merit (2nd) } \tocroot{Dutiyapuññābhisandasutta}}
\markboth{Overflowing Merit (2nd) }{Dutiyapuññābhisandasutta}
\extramarks{SN 55.32}{SN 55.32}

“Mendicants,\marginnote{1.1} there are these four kinds of overflowing merit, overflowing goodness that nurture happiness. What four? It’s when a noble disciple has experiential confidence in the Buddha … the teaching … the \textsanskrit{Saṅgha} … 

Furthermore,\marginnote{3.1} a noble disciple lives at home rid of the stain of stinginess, freely generous, open-handed, loving to let go, committed to charity, loving to give and to share. This is the fourth kind of overflowing merit, overflowing goodness that nurtures happiness. These are the four kinds of overflowing merit, overflowing goodness that nurture happiness.” 

%
\section*{{\suttatitleacronym SN 55.33}{\suttatitletranslation Overflowing Merit (3rd) }{\suttatitleroot Tatiyapuññābhisandasutta}}
\addcontentsline{toc}{section}{\tocacronym{SN 55.33} \toctranslation{Overflowing Merit (3rd) } \tocroot{Tatiyapuññābhisandasutta}}
\markboth{Overflowing Merit (3rd) }{Tatiyapuññābhisandasutta}
\extramarks{SN 55.33}{SN 55.33}

“Mendicants,\marginnote{1.1} there are these four kinds of overflowing merit, overflowing goodness that nurture happiness. What four? It’s when a noble disciple has experiential confidence in the Buddha … the teaching … the \textsanskrit{Saṅgha} … 

Furthermore,\marginnote{3.1} a noble disciple is wise. They have the wisdom of arising and passing away which is noble, penetrative, and leads to the complete ending of suffering. This is the fourth kind of overflowing merit, overflowing goodness that nurtures happiness. These are the four kinds of overflowing merit, overflowing goodness that nurture happiness.” 

%
\section*{{\suttatitleacronym SN 55.34}{\suttatitletranslation Footprints of the Gods (1st) }{\suttatitleroot Paṭhamadevapadasutta}}
\addcontentsline{toc}{section}{\tocacronym{SN 55.34} \toctranslation{Footprints of the Gods (1st) } \tocroot{Paṭhamadevapadasutta}}
\markboth{Footprints of the Gods (1st) }{Paṭhamadevapadasutta}
\extramarks{SN 55.34}{SN 55.34}

At\marginnote{1.1} \textsanskrit{Sāvatthī}. 

“Mendicants,\marginnote{1.2} these four footprints of the gods are in order to purify unpurified beings and cleanse unclean beings. 

What\marginnote{2.1} four? Firstly, a noble disciple has experiential confidence in the Buddha … This is the first footprint of the gods in order to purify unpurified beings and cleanse unclean beings. 

Furthermore,\marginnote{3.1} a noble disciple has experiential confidence in the teaching … the \textsanskrit{Saṅgha} … 

Furthermore,\marginnote{4.1} a noble disciple’s ethical conduct is loved by the noble ones, unbroken, impeccable, spotless, and unmarred, liberating, praised by sensible people, not mistaken, and leading to immersion. This is the fourth footprint of the gods in order to purify unpurified beings and cleanse unclean beings. These four footprints of the gods are in order to purify unpurified beings and cleanse unclean beings.” 

%
\section*{{\suttatitleacronym SN 55.35}{\suttatitletranslation Footprints of the Gods (2nd) }{\suttatitleroot Dutiyadevapadasutta}}
\addcontentsline{toc}{section}{\tocacronym{SN 55.35} \toctranslation{Footprints of the Gods (2nd) } \tocroot{Dutiyadevapadasutta}}
\markboth{Footprints of the Gods (2nd) }{Dutiyadevapadasutta}
\extramarks{SN 55.35}{SN 55.35}

“Mendicants,\marginnote{1.1} these four footprints of the gods are in order to purify unpurified beings and cleanse unclean beings. 

What\marginnote{2.1} four? Firstly, a noble disciple has experiential confidence in the Buddha: ‘That Blessed One is perfected, a fully awakened Buddha, accomplished in knowledge and conduct, holy, knower of the world, supreme guide for those who wish to train, teacher of gods and humans, awakened, blessed.’ Then they reflect: ‘What now is the footprint of the gods?’ They understand: ‘I hear that these days the gods consider non-harming to be supreme. But I don’t hurt any creature firm or frail. I definitely live in possession of a footprint of the gods.’ This is the first footprint of the gods in order to purify unpurified beings and cleanse unclean beings. 

Furthermore,\marginnote{3.1} a noble disciple has experiential confidence in the teaching … the \textsanskrit{Saṅgha} … 

Furthermore,\marginnote{4.1} a noble disciple’s ethical conduct is loved by the noble ones, unbroken, impeccable, spotless, and unmarred, liberating, praised by sensible people, not mistaken, and leading to immersion. Then they reflect: ‘What now is the footprint of the gods?’ They understand: ‘I hear that these days the gods consider non-harming to be supreme. But I don’t hurt any creature firm or frail. I definitely live in possession of a footprint of the gods.’ This is the fourth footprint of the gods in order to purify unpurified beings and cleanse unclean beings. These four footprints of the gods are in order to purify unpurified beings and cleanse unclean beings.” 

%
\section*{{\suttatitleacronym SN 55.36}{\suttatitletranslation In Common With the Gods }{\suttatitleroot Devasabhāgatasutta}}
\addcontentsline{toc}{section}{\tocacronym{SN 55.36} \toctranslation{In Common With the Gods } \tocroot{Devasabhāgatasutta}}
\markboth{In Common With the Gods }{Devasabhāgatasutta}
\extramarks{SN 55.36}{SN 55.36}

“Mendicants,\marginnote{1.1} when someone has four things the gods are pleased and speak of what they have in common. What four? Firstly, a noble disciple has experiential confidence in the Buddha … There are deities with experiential confidence in the Buddha who passed away from here and were reborn there. They think: ‘Having such experiential confidence in the Buddha, we passed away from there and were reborn here. That noble disciple has the same kind of experiential confidence in the Buddha, so they will come into the presence of the gods.’ 

Furthermore,\marginnote{2.1} a noble disciple has experiential confidence in the teaching … the \textsanskrit{Saṅgha} … And they have the ethical conduct loved by the noble ones … leading to immersion. There are deities with the ethical conduct loved by the noble ones who passed away from here and were reborn there. They think: ‘Having such ethical conduct loved by the noble ones, we passed away from there and were reborn here. That noble disciple has the same kind of ethical conduct loved by the noble ones, so they will come into the presence of the gods.’ When someone has four things the gods are pleased and speak of what they have in common.” 

%
\section*{{\suttatitleacronym SN 55.37}{\suttatitletranslation With Mahānāma }{\suttatitleroot Mahānāmasutta}}
\addcontentsline{toc}{section}{\tocacronym{SN 55.37} \toctranslation{With Mahānāma } \tocroot{Mahānāmasutta}}
\markboth{With Mahānāma }{Mahānāmasutta}
\extramarks{SN 55.37}{SN 55.37}

At\marginnote{1.1} one time the Buddha was staying in the land of the Sakyans, near Kapilavatthu in the Banyan Tree Monastery. Then \textsanskrit{Mahānāma} the Sakyan went up to the Buddha, bowed, sat down to one side, and said to him: 

“Sir,\marginnote{2.1} how is a lay follower defined?” 

“\textsanskrit{Mahānāma},\marginnote{2.2} when you’ve gone for refuge to the Buddha, the teaching, and the \textsanskrit{Saṅgha}, you’re considered to be a lay follower.” 

“But\marginnote{3.1} how is an ethical lay follower defined?” 

“When\marginnote{3.2} a lay follower doesn’t kill living creatures, steal, commit sexual misconduct, lie, or consume alcoholic drinks that cause negligence, they’re considered to be an ethical lay follower.” 

“But\marginnote{4.1} how is a faithful lay follower defined?” 

“It’s\marginnote{4.2} when a lay follower has faith in the Realized One’s awakening: ‘That Blessed One is perfected, a fully awakened Buddha, accomplished in knowledge and conduct, holy, knower of the world, supreme guide for those who wish to train, teacher of gods and humans, awakened, blessed.’ Then they’re considered to be a faithful lay follower.” 

“But\marginnote{5.1} how is a generous lay follower defined?” 

“It’s\marginnote{5.2} when a lay follower lives at home rid of the stain of stinginess, freely generous, open-handed, loving to let go, committed to charity, loving to give and to share. Then they’re considered to be a generous lay follower.” 

“But\marginnote{6.1} how is a wise lay follower defined?” 

“It’s\marginnote{6.2} when a lay follower is wise. They have the wisdom of arising and passing away which is noble, penetrative, and leads to the complete ending of suffering. Then they’re considered to be a wise lay follower.” 

%
\section*{{\suttatitleacronym SN 55.38}{\suttatitletranslation Rain }{\suttatitleroot Vassasutta}}
\addcontentsline{toc}{section}{\tocacronym{SN 55.38} \toctranslation{Rain } \tocroot{Vassasutta}}
\markboth{Rain }{Vassasutta}
\extramarks{SN 55.38}{SN 55.38}

“Mendicants,\marginnote{1.1} suppose it rains heavily on a mountain top, and the water flows downhill to fill the hollows, crevices, and creeks. As they become full, they fill up the pools. The pools fill up the lakes, the lakes fill up the streams, and the streams fill up the rivers. And as the rivers become full, they fill up the ocean. In the same way, a noble disciple has experiential confidence in the Buddha, the teaching, and the \textsanskrit{Saṅgha}, and the ethics loved by the noble ones. These things flow onwards; and, after crossing to the far shore, they lead to the ending of defilements.” 

%
\section*{{\suttatitleacronym SN 55.39}{\suttatitletranslation With Kāḷigodhā }{\suttatitleroot Kāḷigodhasutta}}
\addcontentsline{toc}{section}{\tocacronym{SN 55.39} \toctranslation{With Kāḷigodhā } \tocroot{Kāḷigodhasutta}}
\markboth{With Kāḷigodhā }{Kāḷigodhasutta}
\extramarks{SN 55.39}{SN 55.39}

At\marginnote{1.1} one time the Buddha was staying in the land of the Sakyans, near Kapilavatthu in the Banyan Tree Monastery. Then the Buddha robed up in the morning and, taking his bowl and robe, went to the home of \textsanskrit{Kāḷigodhā} the Sakyan lady, where he sat on the seat spread out. Then \textsanskrit{Kāḷigodhā} went up to the Buddha, bowed, and sat down to one side. The Buddha said to her: 

“\textsanskrit{Godhā},\marginnote{2.1} a female noble disciple who has four things is a stream-enterer, not liable to be reborn in the underworld, bound for awakening. What four? It’s when a noble disciple has experiential confidence in the Buddha … the teaching … the \textsanskrit{Saṅgha} … And they live at home rid of the stain of stinginess, freely generous, open-handed, loving to let go, committed to charity, loving to give and to share. A female noble disciple who has these four things is a stream-enterer, not liable to be reborn in the underworld, bound for awakening.” 

“Sir,\marginnote{3.1} these four factors of stream-entry that were taught by the Buddha are found in me, and I am seen in them. For I have experiential confidence in the Buddha … the teaching … the \textsanskrit{Saṅgha} … And I share without reservation all the gifts available to give in our family with those who are ethical and of good character.” 

“You’re\marginnote{3.7} fortunate, \textsanskrit{Godhā}, so very fortunate, You have declared the fruit of stream-entry.” 

%
\section*{{\suttatitleacronym SN 55.40}{\suttatitletranslation Nandiya the Sakyan }{\suttatitleroot Nandiyasakkasutta}}
\addcontentsline{toc}{section}{\tocacronym{SN 55.40} \toctranslation{Nandiya the Sakyan } \tocroot{Nandiyasakkasutta}}
\markboth{Nandiya the Sakyan }{Nandiyasakkasutta}
\extramarks{SN 55.40}{SN 55.40}

At\marginnote{1.1} one time the Buddha was staying in the land of the Sakyans, near Kapilavatthu in the Banyan Tree Monastery. Then Nandiya the Sakyan went up to the Buddha, bowed, sat down to one side, and said to him: 

“Sir,\marginnote{1.3} if a noble disciple were to totally and utterly lack the four factors of stream-entry, would they live negligently?” 

“Nandiya,\marginnote{2.1} someone who totally and utterly lacks these four factors of stream-entry is an outsider who belongs with the ordinary persons, I say. Neverthless, Nandiya, as to how a noble disciple lives negligently and how they live diligently, listen and attend closely, I will speak. 

“Yes,\marginnote{2.4} sir,” Nandiya replied. The Buddha said this: 

“And\marginnote{3.1} how does a noble disciple live negligently? Firstly, a noble disciple has experiential confidence in the Buddha … They’re content with that confidence, and don’t make a further effort for solitude by day or retreat by night. When they live negligently, there’s no joy. When there’s no joy, there’s no rapture. When there’s no rapture, there’s no tranquility. When there’s no tranquility, there’s suffering. When one is suffering, the mind does not become immersed in \textsanskrit{samādhi}. When the mind is not immersed in \textsanskrit{samādhi}, principles do not become clear. Because principles have not become clear, they’re reckoned to live negligently. 

Furthermore,\marginnote{4.1} a noble disciple has experiential confidence in the teaching … the \textsanskrit{Saṅgha} … And they have the ethical conduct loved by the noble ones … leading to immersion. They’re content with that ethical conduct loved by the noble ones, and don’t make a further effort for solitude by day or retreat by night. When they live negligently, there’s no joy. When there’s no joy, there’s no rapture. When there’s no rapture, there’s no tranquility. When there’s no tranquility, there’s suffering. When one is suffering, the mind does not become immersed in \textsanskrit{samādhi}. When the mind is not immersed in \textsanskrit{samādhi}, principles do not become clear. Because principles have not become clear, they’re reckoned to live negligently. That’s how a noble disciple lives negligently. 

And\marginnote{5.1} how does a noble disciple live diligently? Firstly, a noble disciple has experiential confidence in the Buddha … But they’re not content with that confidence, and make a further effort for solitude by day and retreat by night. When they live diligently, joy springs up. Being joyful, rapture springs up. When the mind is full of rapture, the body becomes tranquil. When the body is tranquil, they feel bliss. And when blissful, the mind becomes immersed in \textsanskrit{samādhi}. When the mind is immersed in \textsanskrit{samādhi}, principles become clear. Because principles have become clear, they’re reckoned to live diligently. 

Furthermore,\marginnote{6.1} a noble disciple has experiential confidence in the teaching … the \textsanskrit{Saṅgha} … And they have the ethical conduct loved by the noble ones … leading to immersion. But they’re not content with that ethical conduct loved by the noble ones, and make a further effort for solitude by day and retreat by night. When they live diligently, joy springs up. Being joyful, rapture springs up. When the mind is full of rapture, the body becomes tranquil. When the body is tranquil, they feel bliss. And when blissful, the mind becomes immersed in \textsanskrit{samādhi}. When the mind is immersed in \textsanskrit{samādhi}, principles become clear. Because principles have become clear, they’re reckoned to live diligently. That’s how a noble disciple lives diligently.” 

%
\addtocontents{toc}{\let\protect\contentsline\protect\nopagecontentsline}
\chapter*{The Chapter on Overflowing Merit, With Verses }
\addcontentsline{toc}{chapter}{\tocchapterline{The Chapter on Overflowing Merit, With Verses }}
\addtocontents{toc}{\let\protect\contentsline\protect\oldcontentsline}

%
\section*{{\suttatitleacronym SN 55.41}{\suttatitletranslation Overflowing Merit (1st) }{\suttatitleroot Paṭhamaabhisandasutta}}
\addcontentsline{toc}{section}{\tocacronym{SN 55.41} \toctranslation{Overflowing Merit (1st) } \tocroot{Paṭhamaabhisandasutta}}
\markboth{Overflowing Merit (1st) }{Paṭhamaabhisandasutta}
\extramarks{SN 55.41}{SN 55.41}

“Mendicants,\marginnote{1.1} there are these four kinds of overflowing merit, overflowing goodness that nurture happiness. What four? It’s when a noble disciple has experiential confidence in the Buddha … the teaching … the \textsanskrit{Saṅgha} … 

Furthermore,\marginnote{3.1} they have the ethical conduct loved by the noble ones … leading to immersion. … These are the four kinds of overflowing merit, overflowing goodness that nurture happiness. 

When\marginnote{4.1} a noble disciple has these four kinds of overflowing merit and goodness, it’s not easy to measure how much merit they have by saying that this is the extent of their overflowing merit, overflowing goodness that nurtures happiness. It’s simply reckoned as an incalculable, immeasurable, great mass of merit. 

It’s\marginnote{5.1} like trying to measure how much water is in the ocean. It’s not easy to say how many gallons, how many hundreds, thousands, hundreds of thousands of gallons there are. It’s simply reckoned as an incalculable, immeasurable, great mass of water. 

In\marginnote{5.4} the same way, when a noble disciple has these four kinds of overflowing merit and goodness, it’s not easy to measure how much merit they have by saying that this is the extent of their overflowing merit, overflowing goodness that nurtures happiness. It’s simply reckoned as an incalculable, immeasurable, great mass of merit.” 

That\marginnote{6.1} is what the Buddha said. Then the Holy One, the Teacher, went on to say: 

\begin{verse}%
“Hosts\marginnote{7.1} of people use the rivers, \\
and though the rivers are many, \\
all reach the great deep, the boundless ocean, \\
the cruel sea that’s home to precious gems. 

So\marginnote{8.1} too, when a person gives food, drink, and clothes; \\
and they’re a giver of beds, seats, and mats—\\
the streams of merit reach that astute person, \\
as the rivers bring their waters to the sea.” 

%
\end{verse}

%
\section*{{\suttatitleacronym SN 55.42}{\suttatitletranslation Overflowing Merit (2nd) }{\suttatitleroot Dutiyaabhisandasutta}}
\addcontentsline{toc}{section}{\tocacronym{SN 55.42} \toctranslation{Overflowing Merit (2nd) } \tocroot{Dutiyaabhisandasutta}}
\markboth{Overflowing Merit (2nd) }{Dutiyaabhisandasutta}
\extramarks{SN 55.42}{SN 55.42}

“Mendicants,\marginnote{1.1} there are these four kinds of overflowing merit, overflowing goodness that nurture happiness. What four? It’s when a noble disciple has experiential confidence in the Buddha … the teaching … the \textsanskrit{Saṅgha} … 

Furthermore,\marginnote{3.1} a noble disciple lives at home rid of the stain of stinginess, freely generous, open-handed, loving to let go, committed to charity, loving to give and to share. This is the fourth kind of overflowing merit, overflowing goodness that nurtures happiness. These are the four kinds of overflowing merit, overflowing goodness that nurture happiness. 

When\marginnote{4.1} a noble disciple has these four kinds of overflowing merit and goodness, it’s not easy to measure how much merit they have by saying that this is the extent of their overflowing merit, overflowing goodness that nurtures happiness. It’s simply reckoned as an incalculable, immeasurable, great mass of merit. 

There\marginnote{5.1} are places where the great rivers—the Ganges, Yamuna, \textsanskrit{Aciravatī}, \textsanskrit{Sarabhū}, and \textsanskrit{Mahī}—come together and converge. It’s not easy measure how much water is in such places by saying how many gallons, how many hundreds, thousands, hundreds of thousands of gallons there are. It’s simply reckoned as an incalculable, immeasurable, great mass of water. 

In\marginnote{5.5} the same way, when a noble disciple has these four kinds of overflowing merit and goodness, it’s not easy to measure how much merit they have by saying that this is the extent of their overflowing merit, overflowing goodness that nurtures happiness. It’s simply reckoned as an incalculable, immeasurable, great mass of merit.” 

That\marginnote{5.8} is what the Buddha said. Then the Holy One, the Teacher, went on to say: 

\begin{verse}%
“Hosts\marginnote{6.1} of people use the rivers, \\
and though the rivers are many, \\
all reach the great deep, the boundless ocean, \\
the cruel sea that’s home to precious gems. 

So\marginnote{7.1} too, when a person gives food, drink, and clothes; \\
and they’re a giver of beds, seats, and mats—\\
the streams of merit reach that astute person, \\
as the rivers bring their waters to the sea.” 

%
\end{verse}

%
\section*{{\suttatitleacronym SN 55.43}{\suttatitletranslation Overflowing Merit (3rd) }{\suttatitleroot Tatiyaabhisandasutta}}
\addcontentsline{toc}{section}{\tocacronym{SN 55.43} \toctranslation{Overflowing Merit (3rd) } \tocroot{Tatiyaabhisandasutta}}
\markboth{Overflowing Merit (3rd) }{Tatiyaabhisandasutta}
\extramarks{SN 55.43}{SN 55.43}

“Mendicants,\marginnote{1.1} there are these four kinds of overflowing merit, overflowing goodness that nurture happiness. What four? It’s when a noble disciple has experiential confidence in the Buddha … the teaching … the \textsanskrit{Saṅgha} … 

Furthermore,\marginnote{3.1} a noble disciple is wise. They have the wisdom of arising and passing away which is noble, penetrative, and leads to the complete ending of suffering. This is the fourth kind of overflowing merit, overflowing goodness that nurtures happiness. These are the four kinds of overflowing merit, overflowing goodness that nurture happiness. 

When\marginnote{4.1} a noble disciple has these four kinds of overflowing merit and goodness, it’s not easy to measure how much merit they have by saying that this is the extent of their overflowing merit, overflowing goodness that nurtures happiness. It’s simply reckoned as an incalculable, immeasurable, great mass of merit.” 

That\marginnote{4.4} is what the Buddha said. Then the Holy One, the Teacher, went on to say: 

\begin{verse}%
“One\marginnote{5.1} who desires merit, grounded in the skillful, \\
develops the eightfold path for realizing the deathless. \\
Once they’ve reached the heart of the teaching, delighting in ending, \\
they don’t tremble at the approach of the King of Death.” 

%
\end{verse}

%
\section*{{\suttatitleacronym SN 55.44}{\suttatitletranslation Rich (1st) }{\suttatitleroot Paṭhamamahaddhanasutta}}
\addcontentsline{toc}{section}{\tocacronym{SN 55.44} \toctranslation{Rich (1st) } \tocroot{Paṭhamamahaddhanasutta}}
\markboth{Rich (1st) }{Paṭhamamahaddhanasutta}
\extramarks{SN 55.44}{SN 55.44}

“Mendicants,\marginnote{1.1} a noble disciple who has four things is said to be rich, affluent, and wealthy. 

What\marginnote{2.1} four? It’s when a noble disciple has experiential confidence in the Buddha … the teaching … the \textsanskrit{Saṅgha} … And they have the ethical conduct loved by the noble ones … leading to immersion. A noble disciple who has these four things is said to be rich, affluent, and wealthy.” 

%
\section*{{\suttatitleacronym SN 55.45}{\suttatitletranslation Rich (2nd) }{\suttatitleroot Dutiyamahaddhanasutta}}
\addcontentsline{toc}{section}{\tocacronym{SN 55.45} \toctranslation{Rich (2nd) } \tocroot{Dutiyamahaddhanasutta}}
\markboth{Rich (2nd) }{Dutiyamahaddhanasutta}
\extramarks{SN 55.45}{SN 55.45}

“Mendicants,\marginnote{1.1} a noble disciple who has four things is said to be rich, affluent, wealthy, and famous. 

What\marginnote{2.1} four? It’s when a noble disciple has experiential confidence in the Buddha … the teaching … the \textsanskrit{Saṅgha} … And they have the ethical conduct loved by the noble ones … leading to immersion. A noble disciple who has these four things is said to be rich, affluent, wealthy, and famous.” 

%
\section*{{\suttatitleacronym SN 55.46}{\suttatitletranslation Plain Version }{\suttatitleroot Suddhakasutta}}
\addcontentsline{toc}{section}{\tocacronym{SN 55.46} \toctranslation{Plain Version } \tocroot{Suddhakasutta}}
\markboth{Plain Version }{Suddhakasutta}
\extramarks{SN 55.46}{SN 55.46}

“Mendicants,\marginnote{1.1} a noble disciple who has four things is a stream-enterer, not liable to be reborn in the underworld, bound for awakening. 

What\marginnote{2.1} four? It’s when a noble disciple has experiential confidence in the Buddha … the teaching … the \textsanskrit{Saṅgha} … And they have the ethical conduct loved by the noble ones … leading to immersion. A noble disciple who has these four things is a stream-enterer, not liable to be reborn in the underworld, bound for awakening.” 

%
\section*{{\suttatitleacronym SN 55.47}{\suttatitletranslation With Nandiya }{\suttatitleroot Nandiyasutta}}
\addcontentsline{toc}{section}{\tocacronym{SN 55.47} \toctranslation{With Nandiya } \tocroot{Nandiyasutta}}
\markboth{With Nandiya }{Nandiyasutta}
\extramarks{SN 55.47}{SN 55.47}

At\marginnote{1.1} Kapilavatthu. Seated to one side, the Buddha said to Nandiya the Sakyan: 

“Nandiya,\marginnote{1.3} a noble disciple who has four things is a stream-enterer, not liable to be reborn in the underworld, bound for awakening. 

What\marginnote{2.1} four? It’s when a noble disciple has experiential confidence in the Buddha … the teaching … the \textsanskrit{Saṅgha} … And they have the ethical conduct loved by the noble ones … leading to immersion. A noble disciple who has these four things is a stream-enterer, not liable to be reborn in the underworld, bound for awakening.” 

%
\section*{{\suttatitleacronym SN 55.48}{\suttatitletranslation With Bhaddiya }{\suttatitleroot Bhaddiyasutta}}
\addcontentsline{toc}{section}{\tocacronym{SN 55.48} \toctranslation{With Bhaddiya } \tocroot{Bhaddiyasutta}}
\markboth{With Bhaddiya }{Bhaddiyasutta}
\extramarks{SN 55.48}{SN 55.48}

At\marginnote{1.1} Kapilavatthu. Seated to one side, the Buddha said to Bhaddiya the Sakyan: 

“Bhaddiya,\marginnote{1.3} a noble disciple who has four things is a stream-enterer …” 

%
\section*{{\suttatitleacronym SN 55.49}{\suttatitletranslation With Mahānāma }{\suttatitleroot Mahānāmasutta}}
\addcontentsline{toc}{section}{\tocacronym{SN 55.49} \toctranslation{With Mahānāma } \tocroot{Mahānāmasutta}}
\markboth{With Mahānāma }{Mahānāmasutta}
\extramarks{SN 55.49}{SN 55.49}

At\marginnote{1.1} Kapilavatthu. Seated to one side, the Buddha said to \textsanskrit{Mahānāma} the Sakyan: 

“\textsanskrit{Mahānāma},\marginnote{1.3} a noble disciple who has four things is a stream-enterer …” 

%
\section*{{\suttatitleacronym SN 55.50}{\suttatitletranslation Factors }{\suttatitleroot Aṅgasutta}}
\addcontentsline{toc}{section}{\tocacronym{SN 55.50} \toctranslation{Factors } \tocroot{Aṅgasutta}}
\markboth{Factors }{Aṅgasutta}
\extramarks{SN 55.50}{SN 55.50}

“Mendicants,\marginnote{1.1} there are these four factors of stream-entry. What four? Associating with good people, listening to the true teaching, proper attention, and practicing in line with the teaching. These are the four factors of stream-entry.” 

%
\addtocontents{toc}{\let\protect\contentsline\protect\nopagecontentsline}
\chapter*{The Chapter on a Wise Person }
\addcontentsline{toc}{chapter}{\tocchapterline{The Chapter on a Wise Person }}
\addtocontents{toc}{\let\protect\contentsline\protect\oldcontentsline}

%
\section*{{\suttatitleacronym SN 55.51}{\suttatitletranslation With Verses }{\suttatitleroot Sagāthakasutta}}
\addcontentsline{toc}{section}{\tocacronym{SN 55.51} \toctranslation{With Verses } \tocroot{Sagāthakasutta}}
\markboth{With Verses }{Sagāthakasutta}
\extramarks{SN 55.51}{SN 55.51}

“Mendicants,\marginnote{1.1} a noble disciple who has four things is a stream-enterer, not liable to be reborn in the underworld, bound for awakening. 

What\marginnote{2.1} four? It’s when a noble disciple has experiential confidence in the Buddha … the teaching … the \textsanskrit{Saṅgha} … And they have the ethical conduct loved by the noble ones … leading to immersion. A noble disciple who has these four things is a stream-enterer, not liable to be reborn in the underworld, bound for awakening.” 

That\marginnote{2.8} is what the Buddha said. Then the Holy One, the Teacher, went on to say: 

\begin{verse}%
“Whoever\marginnote{3.1} has faith in the Realized One, \\
unwavering and well grounded; \\
whose ethical conduct is good, \\
praised and loved by the noble ones; 

who\marginnote{4.1} has confidence in the \textsanskrit{Saṅgha}, \\
and correct view: \\
they’re said to be prosperous, \\
their life is not in vain. 

So\marginnote{5.1} let the wise devote themselves \\
to faith, ethical behaviour, \\
confidence, and insight into the teaching, \\
remembering the instructions of the Buddhas.” 

%
\end{verse}

%
\section*{{\suttatitleacronym SN 55.52}{\suttatitletranslation One Who Completed the Rains }{\suttatitleroot Vassaṁvutthasutta}}
\addcontentsline{toc}{section}{\tocacronym{SN 55.52} \toctranslation{One Who Completed the Rains } \tocroot{Vassaṁvutthasutta}}
\markboth{One Who Completed the Rains }{Vassaṁvutthasutta}
\extramarks{SN 55.52}{SN 55.52}

At\marginnote{1.1} one time the Buddha was staying near \textsanskrit{Sāvatthī} in Jeta’s Grove, \textsanskrit{Anāthapiṇḍika}’s monastery. Now at that time a certain mendicant who had completed the rainy season residence in \textsanskrit{Sāvatthī} arrived at Kapilavatthu on some business. The Sakyans of Kapilavatthu heard about this. 

They\marginnote{2.1} went to that mendicant, bowed, sat down to one side, and said to him, “Sir, we hope that you’re healthy and well.” 

“I\marginnote{2.3} am, good sirs.” 

“And\marginnote{3.1} we hope that \textsanskrit{Sāriputta} and \textsanskrit{Moggallāna} are healthy and well.” 

“They\marginnote{3.2} are.” 

“And\marginnote{4.1} we hope that the mendicant \textsanskrit{Saṅgha} is healthy and well.” 

“It\marginnote{4.2} is.” 

“But\marginnote{5.1} sir, during this rains residence did you hear and learn anything in the presence of the Buddha?” 

“Good\marginnote{5.2} sirs, I heard and learned this in the presence of the Buddha: ‘There are fewer mendicants who realize the undefiled freedom of heart and freedom by wisdom in this very life, and live having realized it with their own insight due to the ending of defilements. There are more mendicants who, having ended the five lower fetters, are reborn spontaneously, and will be extinguished there, not liable to return from that world.’ 

In\marginnote{6.1} addition, I heard and learned this in the presence of the Buddha: ‘There are fewer mendicants who, having ended the five lower fetters, are reborn spontaneously, and will be extinguished there, not liable to return from that world. There are more mendicants who, with the ending of three fetters, and the weakening of greed, hate, and delusion, are once-returners, who come back to this world once only, then make an end of suffering.’ 

In\marginnote{7.1} addition, I heard and learned this in the presence of the Buddha: ‘There are fewer mendicants who, with the ending of three fetters, and the weakening of greed, hate, and delusion, are once-returners, who come back to this world once only, then make an end of suffering. There are more mendicants who, with the ending of three fetters are stream-enterers, not liable to be reborn in the underworld, bound for awakening.’” 

%
\section*{{\suttatitleacronym SN 55.53}{\suttatitletranslation With Dhammadinna }{\suttatitleroot Dhammadinnasutta}}
\addcontentsline{toc}{section}{\tocacronym{SN 55.53} \toctranslation{With Dhammadinna } \tocroot{Dhammadinnasutta}}
\markboth{With Dhammadinna }{Dhammadinnasutta}
\extramarks{SN 55.53}{SN 55.53}

At\marginnote{1.1} one time the Buddha was staying near Benares, in the deer park at Isipatana. Then the lay follower Dhammadinna, together with five hundred lay followers, went up to the Buddha, bowed, sat down to one side, and said to him: 

“May\marginnote{1.3} the Buddha please advise and instruct us. It will be for our lasting welfare and happiness.” 

“So,\marginnote{2.1} Dhammadinna, you should train like this: ‘From time to time we will undertake and dwell upon the discourses spoken by the Realized One that are deep, profound, transcendent, dealing with emptiness.’ That’s how you should train yourselves.” 

“Sir,\marginnote{2.4} we live at home with our children, using sandalwood imported from \textsanskrit{Kāsi}, wearing garlands, perfumes, and makeup, and accepting gold and money. It’s not easy for us to undertake and dwell from time to time upon the discourses spoken by the Realized One that are deep, profound, transcendent, dealing with emptiness. Since we are established in the five training rules, please teach us further.” 

“So,\marginnote{3.1} Dhammadinna, you should train like this: ‘We will have experiential confidence in the Buddha … the teaching … the \textsanskrit{Saṅgha} … And we will have the ethical conduct loved by the noble ones … leading to immersion.’ That’s how you should train yourselves.” 

“Sir,\marginnote{4.1} these four factors of stream-entry that were taught by the Buddha are found in us, and we embody them. For we have experiential confidence in the Buddha … the teaching … the \textsanskrit{Saṅgha} … And we have the ethical conduct loved by the noble ones … leading to immersion.” 

“You’re\marginnote{4.7} fortunate, Dhammadinna, so very fortunate! You have all declared the fruit of stream-entry.” 

%
\section*{{\suttatitleacronym SN 55.54}{\suttatitletranslation Sick }{\suttatitleroot Gilānasutta}}
\addcontentsline{toc}{section}{\tocacronym{SN 55.54} \toctranslation{Sick } \tocroot{Gilānasutta}}
\markboth{Sick }{Gilānasutta}
\extramarks{SN 55.54}{SN 55.54}

At\marginnote{1.1} one time the Buddha was staying in the land of the Sakyans, near Kapilavatthu in the Banyan Tree Monastery. 

At\marginnote{1.2} that time several mendicants were making a robe for the Buddha, thinking that when his robe was finished and the three months of the rains residence had passed the Buddha would set out wandering. 

\textsanskrit{Mahānāma}\marginnote{1.4} the Sakyan heard about this. Then he went up to the Buddha, bowed, sat down to one side, and told him that he had heard that the Buddha was leaving. He added, “Sir, I haven’t heard and learned it in the presence of the Buddha how a wise lay follower should advise another wise lay follower who is sick, suffering, gravely ill.” 

“\textsanskrit{Mahānāma},\marginnote{2.1} a wise lay follower should put at ease another wise lay follower who is sick, suffering, gravely ill with four consolations. ‘Be at ease, sir. You have experiential confidence in the Buddha … the teaching … the \textsanskrit{Saṅgha} … And you have the ethical conduct loved by the noble ones … leading to immersion.’ 

When\marginnote{3.1} a wise lay follower has put at ease another wise lay follower who is sick, suffering, gravely ill with these four consolations, they should say: ‘Are you concerned for your mother and father?’ If they reply, ‘I am,’ they should say: ‘But sir, it’s your nature to die. Whether or not you are concerned for your mother and father, you will die anyway. It would be good to give up concern for your mother and father.’ 

If\marginnote{4.1} they reply, ‘I have given up concern for my mother and father,’ they should say: ‘But are you concerned for your partners and children?’ If they reply, ‘I am,’ they should say: ‘But sir, it’s your nature to die. Whether or not you are concerned for your partners and children, you will die anyway. It would be good to give up concern for your partners and children.’ 

If\marginnote{5.1} they reply, ‘I have given up concern for my partners and children,’ they should say: ‘But are you concerned for the five kinds of human sensual stimulation?’ If they reply, ‘I am,’ they should say: ‘Good sir, heavenly sensual pleasures are better than human sensual pleasures. It would be good to turn your mind away from human sensual pleasures and fix it on the gods of the Four Great Kings.’ 

If\marginnote{6.1} they reply, ‘I have done so,’ they should say: ‘Good sir, the gods of the Thirty-Three are better than the gods of the Four Great Kings … 

Good\marginnote{7.1} sir, the gods of Yama … the Joyful Gods … the Gods Who Love to Create … the Gods Who Control the Creations of Others … the Gods of the \textsanskrit{Brahmā} realm are better than the Gods Who Control the Creations of Others. It would be good to turn your mind away from the Gods Who Control the Creations of Others and fix it on the Gods of the \textsanskrit{Brahmā} realm.’ If they reply, ‘I have done so,’ they should say: ‘Good sir, the \textsanskrit{Brahmā} realm is impermanent, not lasting, and included within identity. It would be good to turn your mind away from the \textsanskrit{Brahmā} realm and apply it to the cessation of identity.’ 

If\marginnote{8.1} they reply, ‘I have done so,’ then there is no difference between a lay follower whose mind is freed in this way and a mendicant whose mind is freed from defilements; that is, between the freedom of one and the other.” 

%
\section*{{\suttatitleacronym SN 55.55}{\suttatitletranslation The Fruit of Stream-Entry }{\suttatitleroot Sotāpattiphalasutta}}
\addcontentsline{toc}{section}{\tocacronym{SN 55.55} \toctranslation{The Fruit of Stream-Entry } \tocroot{Sotāpattiphalasutta}}
\markboth{The Fruit of Stream-Entry }{Sotāpattiphalasutta}
\extramarks{SN 55.55}{SN 55.55}

“Mendicants,\marginnote{1.1} when four things are developed and cultivated they lead to the realization of the fruit of stream-entry. What four? Associating with good people, listening to the true teaching, proper attention, and practicing in line with the teaching. When these four things are developed and cultivated they lead to the realization of the fruit of stream-entry.” 

%
\section*{{\suttatitleacronym SN 55.56}{\suttatitletranslation The Fruit of Once-Return }{\suttatitleroot Sakadāgāmiphalasutta}}
\addcontentsline{toc}{section}{\tocacronym{SN 55.56} \toctranslation{The Fruit of Once-Return } \tocroot{Sakadāgāmiphalasutta}}
\markboth{The Fruit of Once-Return }{Sakadāgāmiphalasutta}
\extramarks{SN 55.56}{SN 55.56}

“Mendicants,\marginnote{1.1} when four things are developed and cultivated they lead to the realization of the fruit of once-return. …” 

%
\section*{{\suttatitleacronym SN 55.57}{\suttatitletranslation The Fruit of Non-Return }{\suttatitleroot Anāgāmiphalasutta}}
\addcontentsline{toc}{section}{\tocacronym{SN 55.57} \toctranslation{The Fruit of Non-Return } \tocroot{Anāgāmiphalasutta}}
\markboth{The Fruit of Non-Return }{Anāgāmiphalasutta}
\extramarks{SN 55.57}{SN 55.57}

“Mendicants,\marginnote{1.1} when four things are developed and cultivated they lead to the realization of the fruit of non-return. …” 

%
\section*{{\suttatitleacronym SN 55.58}{\suttatitletranslation The Fruit of Perfection }{\suttatitleroot Arahattaphalasutta}}
\addcontentsline{toc}{section}{\tocacronym{SN 55.58} \toctranslation{The Fruit of Perfection } \tocroot{Arahattaphalasutta}}
\markboth{The Fruit of Perfection }{Arahattaphalasutta}
\extramarks{SN 55.58}{SN 55.58}

“Mendicants,\marginnote{1.1} when four things are developed and cultivated they lead to the realization of the fruit of perfection. …” 

%
\section*{{\suttatitleacronym SN 55.59}{\suttatitletranslation The Getting of Wisdom }{\suttatitleroot Paññāpaṭilābhasutta}}
\addcontentsline{toc}{section}{\tocacronym{SN 55.59} \toctranslation{The Getting of Wisdom } \tocroot{Paññāpaṭilābhasutta}}
\markboth{The Getting of Wisdom }{Paññāpaṭilābhasutta}
\extramarks{SN 55.59}{SN 55.59}

“Mendicants,\marginnote{1.1} when four things are developed and cultivated they lead to the getting of wisdom. …” 

%
\section*{{\suttatitleacronym SN 55.60}{\suttatitletranslation The Growth of Wisdom }{\suttatitleroot Paññāvuddhisutta}}
\addcontentsline{toc}{section}{\tocacronym{SN 55.60} \toctranslation{The Growth of Wisdom } \tocroot{Paññāvuddhisutta}}
\markboth{The Growth of Wisdom }{Paññāvuddhisutta}
\extramarks{SN 55.60}{SN 55.60}

“Mendicants,\marginnote{1.1} when four things are developed and cultivated they lead to the growth of wisdom. …” 

%
\section*{{\suttatitleacronym SN 55.61}{\suttatitletranslation The Increase of Wisdom }{\suttatitleroot Paññāvepullasutta}}
\addcontentsline{toc}{section}{\tocacronym{SN 55.61} \toctranslation{The Increase of Wisdom } \tocroot{Paññāvepullasutta}}
\markboth{The Increase of Wisdom }{Paññāvepullasutta}
\extramarks{SN 55.61}{SN 55.61}

“Mendicants,\marginnote{1.1} when four things are developed and cultivated they lead to the increase of wisdom. …” 

%
\addtocontents{toc}{\let\protect\contentsline\protect\nopagecontentsline}
\chapter*{The Chapter on Great Wisdom }
\addcontentsline{toc}{chapter}{\tocchapterline{The Chapter on Great Wisdom }}
\addtocontents{toc}{\let\protect\contentsline\protect\oldcontentsline}

%
\section*{{\suttatitleacronym SN 55.62}{\suttatitletranslation Great Wisdom }{\suttatitleroot Mahāpaññāsutta}}
\addcontentsline{toc}{section}{\tocacronym{SN 55.62} \toctranslation{Great Wisdom } \tocroot{Mahāpaññāsutta}}
\markboth{Great Wisdom }{Mahāpaññāsutta}
\extramarks{SN 55.62}{SN 55.62}

“Mendicants,\marginnote{1.1} when four things are developed and cultivated they lead to great wisdom. What four? Associating with good people, listening to the true teaching, proper attention, and practicing in line with the teaching. When these four things are developed and cultivated they lead to great wisdom.” 

%
\section*{{\suttatitleacronym SN 55.63}{\suttatitletranslation Widespread Wisdom }{\suttatitleroot Puthupaññāsutta}}
\addcontentsline{toc}{section}{\tocacronym{SN 55.63} \toctranslation{Widespread Wisdom } \tocroot{Puthupaññāsutta}}
\markboth{Widespread Wisdom }{Puthupaññāsutta}
\extramarks{SN 55.63}{SN 55.63}

“Mendicants,\marginnote{1.1} when four things are developed and cultivated they lead to widespread wisdom …” 

%
\section*{{\suttatitleacronym SN 55.64}{\suttatitletranslation Abundant Wisdom }{\suttatitleroot Vipulapaññāsutta}}
\addcontentsline{toc}{section}{\tocacronym{SN 55.64} \toctranslation{Abundant Wisdom } \tocroot{Vipulapaññāsutta}}
\markboth{Abundant Wisdom }{Vipulapaññāsutta}
\extramarks{SN 55.64}{SN 55.64}

“Mendicants,\marginnote{1.1} when four things are developed and cultivated they lead to abundant wisdom …” 

%
\section*{{\suttatitleacronym SN 55.65}{\suttatitletranslation Deep Wisdom }{\suttatitleroot Gambhīrapaññāsutta}}
\addcontentsline{toc}{section}{\tocacronym{SN 55.65} \toctranslation{Deep Wisdom } \tocroot{Gambhīrapaññāsutta}}
\markboth{Deep Wisdom }{Gambhīrapaññāsutta}
\extramarks{SN 55.65}{SN 55.65}

“Mendicants,\marginnote{1.1} when four things are developed and cultivated they lead to deep wisdom …” 

%
\section*{{\suttatitleacronym SN 55.66}{\suttatitletranslation Extraordinary Wisdom }{\suttatitleroot Appamattapaññāsutta}}
\addcontentsline{toc}{section}{\tocacronym{SN 55.66} \toctranslation{Extraordinary Wisdom } \tocroot{Appamattapaññāsutta}}
\markboth{Extraordinary Wisdom }{Appamattapaññāsutta}
\extramarks{SN 55.66}{SN 55.66}

“Mendicants,\marginnote{1.1} when four things are developed and cultivated they lead to extraordinary wisdom …” 

%
\section*{{\suttatitleacronym SN 55.67}{\suttatitletranslation Vast Wisdom }{\suttatitleroot Bhūripaññāsutta}}
\addcontentsline{toc}{section}{\tocacronym{SN 55.67} \toctranslation{Vast Wisdom } \tocroot{Bhūripaññāsutta}}
\markboth{Vast Wisdom }{Bhūripaññāsutta}
\extramarks{SN 55.67}{SN 55.67}

“Mendicants,\marginnote{1.1} when four things are developed and cultivated they lead to vast wisdom …” 

%
\section*{{\suttatitleacronym SN 55.68}{\suttatitletranslation Much Wisdom }{\suttatitleroot Paññābāhullasutta}}
\addcontentsline{toc}{section}{\tocacronym{SN 55.68} \toctranslation{Much Wisdom } \tocroot{Paññābāhullasutta}}
\markboth{Much Wisdom }{Paññābāhullasutta}
\extramarks{SN 55.68}{SN 55.68}

“Mendicants,\marginnote{1.1} when four things are developed and cultivated they lead to much wisdom …” 

%
\section*{{\suttatitleacronym SN 55.69}{\suttatitletranslation Fast Wisdom }{\suttatitleroot Sīghapaññāsutta}}
\addcontentsline{toc}{section}{\tocacronym{SN 55.69} \toctranslation{Fast Wisdom } \tocroot{Sīghapaññāsutta}}
\markboth{Fast Wisdom }{Sīghapaññāsutta}
\extramarks{SN 55.69}{SN 55.69}

“Mendicants,\marginnote{1.1} when four things are developed and cultivated they lead to fast wisdom …” 

%
\section*{{\suttatitleacronym SN 55.70}{\suttatitletranslation Light Wisdom }{\suttatitleroot Lahupaññāsutta}}
\addcontentsline{toc}{section}{\tocacronym{SN 55.70} \toctranslation{Light Wisdom } \tocroot{Lahupaññāsutta}}
\markboth{Light Wisdom }{Lahupaññāsutta}
\extramarks{SN 55.70}{SN 55.70}

“Mendicants,\marginnote{1.1} when four things are developed and cultivated they lead to light wisdom …” 

%
\section*{{\suttatitleacronym SN 55.71}{\suttatitletranslation Laughing Wisdom }{\suttatitleroot Hāsapaññāsutta}}
\addcontentsline{toc}{section}{\tocacronym{SN 55.71} \toctranslation{Laughing Wisdom } \tocroot{Hāsapaññāsutta}}
\markboth{Laughing Wisdom }{Hāsapaññāsutta}
\extramarks{SN 55.71}{SN 55.71}

“Mendicants,\marginnote{1.1} when four things are developed and cultivated they lead to laughing wisdom …” 

%
\section*{{\suttatitleacronym SN 55.72}{\suttatitletranslation Swift Wisdom }{\suttatitleroot Javanapaññāsutta}}
\addcontentsline{toc}{section}{\tocacronym{SN 55.72} \toctranslation{Swift Wisdom } \tocroot{Javanapaññāsutta}}
\markboth{Swift Wisdom }{Javanapaññāsutta}
\extramarks{SN 55.72}{SN 55.72}

“Mendicants,\marginnote{1.1} when four things are developed and cultivated they lead to swift wisdom …” 

%
\section*{{\suttatitleacronym SN 55.73}{\suttatitletranslation Sharp Wisdom }{\suttatitleroot Tikkhapaññāsutta}}
\addcontentsline{toc}{section}{\tocacronym{SN 55.73} \toctranslation{Sharp Wisdom } \tocroot{Tikkhapaññāsutta}}
\markboth{Sharp Wisdom }{Tikkhapaññāsutta}
\extramarks{SN 55.73}{SN 55.73}

“Mendicants,\marginnote{1.1} when four things are developed and cultivated they lead to sharp wisdom …” 

%
\section*{{\suttatitleacronym SN 55.74}{\suttatitletranslation Penetrating Wisdom }{\suttatitleroot Nibbedhikapaññāsutta}}
\addcontentsline{toc}{section}{\tocacronym{SN 55.74} \toctranslation{Penetrating Wisdom } \tocroot{Nibbedhikapaññāsutta}}
\markboth{Penetrating Wisdom }{Nibbedhikapaññāsutta}
\extramarks{SN 55.74}{SN 55.74}

“Mendicants,\marginnote{1.1} when four things are developed and cultivated they lead to penetrating wisdom. What four? Associating with good people, listening to the true teaching, proper attention, and practicing in line with the teaching. When these four things are developed and cultivated they lead to penetrating wisdom.” 

\scendsutta{The Linked Discourses on Stream-Entry, the eleventh section. }

%
\addtocontents{toc}{\let\protect\contentsline\protect\nopagecontentsline}
\part*{Linked Discourses on the Truths }
\addcontentsline{toc}{part}{Linked Discourses on the Truths }
\markboth{}{}
\addtocontents{toc}{\let\protect\contentsline\protect\oldcontentsline}

%
\addtocontents{toc}{\let\protect\contentsline\protect\nopagecontentsline}
\chapter*{The Chapter on Immersion }
\addcontentsline{toc}{chapter}{\tocchapterline{The Chapter on Immersion }}
\addtocontents{toc}{\let\protect\contentsline\protect\oldcontentsline}

%
\section*{{\suttatitleacronym SN 56.1}{\suttatitletranslation Immersion }{\suttatitleroot Samādhisutta}}
\addcontentsline{toc}{section}{\tocacronym{SN 56.1} \toctranslation{Immersion } \tocroot{Samādhisutta}}
\markboth{Immersion }{Samādhisutta}
\extramarks{SN 56.1}{SN 56.1}

At\marginnote{1.1} \textsanskrit{Sāvatthī}. 

“Mendicants,\marginnote{1.3} develop immersion. A mendicant who has immersion truly understands. What do they truly understand? They truly understand: ‘This is suffering’ … ‘This is the origin of suffering’ … ‘This is the cessation of suffering’ … ‘This is the practice that leads to the cessation of suffering’. Develop immersion. A mendicant who has immersion truly understands. 

That’s\marginnote{2.1} why you should practice meditation to understand: ‘This is suffering’ … ‘This is the origin of suffering’ … ‘This is the cessation of suffering’ … ‘This is the practice that leads to the cessation of suffering’.” 

%
\section*{{\suttatitleacronym SN 56.2}{\suttatitletranslation Retreat }{\suttatitleroot Paṭisallānasutta}}
\addcontentsline{toc}{section}{\tocacronym{SN 56.2} \toctranslation{Retreat } \tocroot{Paṭisallānasutta}}
\markboth{Retreat }{Paṭisallānasutta}
\extramarks{SN 56.2}{SN 56.2}

“Mendicants,\marginnote{1.1} meditate in retreat. A mendicant in retreat truly understands. What do they truly understand? They truly understand: ‘This is suffering’ … ‘This is the origin of suffering’ … ‘This is the cessation of suffering’ … ‘This is the practice that leads to the cessation of suffering’. Meditate in retreat. A mendicant in retreat truly understands. 

That’s\marginnote{2.1} why you should practice meditation to understand: ‘This is suffering’ … ‘This is the origin of suffering’ … ‘This is the cessation of suffering’ … ‘This is the practice that leads to the cessation of suffering’.” 

%
\section*{{\suttatitleacronym SN 56.3}{\suttatitletranslation A Gentleman (1st) }{\suttatitleroot Paṭhamakulaputtasutta}}
\addcontentsline{toc}{section}{\tocacronym{SN 56.3} \toctranslation{A Gentleman (1st) } \tocroot{Paṭhamakulaputtasutta}}
\markboth{A Gentleman (1st) }{Paṭhamakulaputtasutta}
\extramarks{SN 56.3}{SN 56.3}

“Mendicants,\marginnote{1.1} whatever gentlemen—past, future, or present—rightly go forth from the lay life to homelessness, all of them do so in order to truly comprehend the four noble truths. 

What\marginnote{2.1} four? The noble truths of suffering, the origin of suffering, the cessation of suffering, and the practice that leads to the cessation of suffering. Whatever gentlemen—past, future, or present—rightly go forth from the lay life to homelessness, all of them do so in order to truly comprehend the four noble truths. 

That’s\marginnote{3.1} why you should practice meditation to understand: ‘This is suffering’ … ‘This is the origin of suffering’ … ‘This is the cessation of suffering’ … ‘This is the practice that leads to the cessation of suffering’.” 

%
\section*{{\suttatitleacronym SN 56.4}{\suttatitletranslation A Gentleman (2nd) }{\suttatitleroot Dutiyakulaputtasutta}}
\addcontentsline{toc}{section}{\tocacronym{SN 56.4} \toctranslation{A Gentleman (2nd) } \tocroot{Dutiyakulaputtasutta}}
\markboth{A Gentleman (2nd) }{Dutiyakulaputtasutta}
\extramarks{SN 56.4}{SN 56.4}

“Mendicants,\marginnote{1.1} whatever gentlemen—past, future, or present—truly comprehend after rightly going forth from the lay life to homelessness, all of them truly comprehend the four noble truths. 

What\marginnote{2.1} four? The noble truths of suffering, the origin of suffering, the cessation of suffering, and the practice that leads to the cessation of suffering. … 

That’s\marginnote{3.1} why you should practice meditation …” 

%
\section*{{\suttatitleacronym SN 56.5}{\suttatitletranslation Ascetics and Brahmins (1st) }{\suttatitleroot Paṭhamasamaṇabrāhmaṇasutta}}
\addcontentsline{toc}{section}{\tocacronym{SN 56.5} \toctranslation{Ascetics and Brahmins (1st) } \tocroot{Paṭhamasamaṇabrāhmaṇasutta}}
\markboth{Ascetics and Brahmins (1st) }{Paṭhamasamaṇabrāhmaṇasutta}
\extramarks{SN 56.5}{SN 56.5}

“Mendicants,\marginnote{1.1} whatever ascetics and brahmins truly wake up—in the past, future, or present—all of them truly wake up to the four noble truths. 

What\marginnote{2.1} four? The noble truths of suffering, the origin of suffering, the cessation of suffering, and the practice that leads to the cessation of suffering. … 

That’s\marginnote{3.1} why you should practice meditation …” 

%
\section*{{\suttatitleacronym SN 56.6}{\suttatitletranslation Ascetics and Brahmins (2nd) }{\suttatitleroot Dutiyasamaṇabrāhmaṇasutta}}
\addcontentsline{toc}{section}{\tocacronym{SN 56.6} \toctranslation{Ascetics and Brahmins (2nd) } \tocroot{Dutiyasamaṇabrāhmaṇasutta}}
\markboth{Ascetics and Brahmins (2nd) }{Dutiyasamaṇabrāhmaṇasutta}
\extramarks{SN 56.6}{SN 56.6}

“Mendicants,\marginnote{1.1} whatever ascetics and brahmins—past, future, or present—reveal that they are awakened, all of them reveal that they truly awakened to the four noble truths. 

What\marginnote{2.1} four? The noble truths of suffering, the origin of suffering, the cessation of suffering, and the practice that leads to the cessation of suffering. … 

That’s\marginnote{3.1} why you should practice meditation …” 

%
\section*{{\suttatitleacronym SN 56.7}{\suttatitletranslation Thoughts }{\suttatitleroot Vitakkasutta}}
\addcontentsline{toc}{section}{\tocacronym{SN 56.7} \toctranslation{Thoughts } \tocroot{Vitakkasutta}}
\markboth{Thoughts }{Vitakkasutta}
\extramarks{SN 56.7}{SN 56.7}

“Mendicants,\marginnote{1.1} don’t think bad, unskillful thoughts, such as sensual, malicious, and cruel thoughts. Why is that? Because those thoughts aren’t beneficial or relevant to the fundamentals of the spiritual life. They don’t lead to disillusionment, dispassion, cessation, peace, insight, awakening, and extinguishment. 

When\marginnote{2.1} you think, you should think: ‘This is suffering’ … ‘This is the origin of suffering’ … ‘This is the cessation of suffering’ … ‘This is the practice that leads to the cessation of suffering’. Why is that? Because those thoughts are beneficial and relevant to the fundamentals of the spiritual life. They lead to disillusionment, dispassion, cessation, peace, insight, awakening, and extinguishment. 

That’s\marginnote{3.1} why you should practice meditation …” 

%
\section*{{\suttatitleacronym SN 56.8}{\suttatitletranslation Thought }{\suttatitleroot Cintasutta}}
\addcontentsline{toc}{section}{\tocacronym{SN 56.8} \toctranslation{Thought } \tocroot{Cintasutta}}
\markboth{Thought }{Cintasutta}
\extramarks{SN 56.8}{SN 56.8}

“Mendicants,\marginnote{1.1} don’t think up a bad, unskillful idea. For example: the cosmos is eternal, or not eternal, or finite, or infinite; the soul and the body are the same thing, or they are different things; after death, a Realized One exists, or doesn’t exist, or both exists and doesn’t exist, or neither exists nor doesn’t exist. Why is that? Because those thoughts aren’t beneficial or relevant to the fundamentals of the spiritual life. They don’t lead to disillusionment, dispassion, cessation, peace, insight, awakening, and extinguishment. 

When\marginnote{2.1} you think something up, you should think: ‘This is suffering’ … ‘This is the origin of suffering’ … ‘This is the cessation of suffering’ … ‘This is the practice that leads to the cessation of suffering’. Why is that? Because those thoughts are beneficial and relevant to the fundamentals of the spiritual life. They lead to disillusionment, dispassion, cessation, peace, insight, awakening, and extinguishment. 

That’s\marginnote{3.1} why you should practice meditation …” 

%
\section*{{\suttatitleacronym SN 56.9}{\suttatitletranslation Arguments }{\suttatitleroot Viggāhikakathāsutta}}
\addcontentsline{toc}{section}{\tocacronym{SN 56.9} \toctranslation{Arguments } \tocroot{Viggāhikakathāsutta}}
\markboth{Arguments }{Viggāhikakathāsutta}
\extramarks{SN 56.9}{SN 56.9}

“Mendicants,\marginnote{1.1} don’t get into arguments, such as: ‘You don’t understand this teaching and training. I understand this teaching and training. What, you understand this teaching and training? You’re practicing wrong. I’m practicing right. I stay on topic, you don’t. You said last what you should have said first. You said first what you should have said last. What you’ve thought so much about has been disproved. Your doctrine is refuted. Go on, save your doctrine! You’re trapped; get yourself out of this—if you can!’ Why is that? Because those discussions aren’t beneficial or relevant to the fundamentals of the spiritual life. They don’t lead to disillusionment, dispassion, cessation, peace, insight, awakening, and extinguishment. 

When\marginnote{2.1} you discuss, you should discuss: ‘This is suffering’ … ‘This is the origin of suffering’ … ‘This is the cessation of suffering’ … ‘This is the practice that leads to the cessation of suffering’. … 

That’s\marginnote{2.2} why you should practice meditation …” 

%
\section*{{\suttatitleacronym SN 56.10}{\suttatitletranslation Unworthy Talk }{\suttatitleroot Tiracchānakathāsutta}}
\addcontentsline{toc}{section}{\tocacronym{SN 56.10} \toctranslation{Unworthy Talk } \tocroot{Tiracchānakathāsutta}}
\markboth{Unworthy Talk }{Tiracchānakathāsutta}
\extramarks{SN 56.10}{SN 56.10}

“Mendicants,\marginnote{1.1} don’t engage in all kinds of unworthy talk, such as talk about kings, bandits, and ministers; talk about armies, threats, and wars; talk about food, drink, clothes, and beds; talk about garlands and fragrances; talk about family, vehicles, villages, towns, cities, and countries; talk about women and heroes; street talk and talk at the well; talk about the departed; motley talk; tales of land and sea; and talk about being reborn in this or that state of existence. Why is that? Because those discussions aren’t beneficial or relevant to the fundamentals of the spiritual life. They don’t lead to disillusionment, dispassion, cessation, peace, insight, awakening, and extinguishment. 

When\marginnote{2.1} you discuss, you should discuss: ‘This is suffering’ … ‘This is the origin of suffering’ … ‘This is the cessation of suffering’ … ‘This is the practice that leads to the cessation of suffering’. … 

That’s\marginnote{3.1} why you should practice meditation …” 

%
\addtocontents{toc}{\let\protect\contentsline\protect\nopagecontentsline}
\chapter*{The Chapter on Rolling Forth the Wheel of Dhamma }
\addcontentsline{toc}{chapter}{\tocchapterline{The Chapter on Rolling Forth the Wheel of Dhamma }}
\addtocontents{toc}{\let\protect\contentsline\protect\oldcontentsline}

%
\section*{{\suttatitleacronym SN 56.11}{\suttatitletranslation Rolling Forth the Wheel of Dhamma }{\suttatitleroot Dhammacakkappavattanasutta}}
\addcontentsline{toc}{section}{\tocacronym{SN 56.11} \toctranslation{Rolling Forth the Wheel of Dhamma } \tocroot{Dhammacakkappavattanasutta}}
\markboth{Rolling Forth the Wheel of Dhamma }{Dhammacakkappavattanasutta}
\extramarks{SN 56.11}{SN 56.11}

At\marginnote{1.1} one time the Buddha was staying near Benares, in the deer park at Isipatana. There the Buddha addressed the group of five mendicants: 

“Mendicants,\marginnote{2.1} these two extremes should not be cultivated by one who has gone forth. What two? Indulgence in sensual pleasures, which is low, crude, ordinary, ignoble, and pointless. And indulgence in self-mortification, which is painful, ignoble, and pointless. Avoiding these two extremes, the Realized One woke up by understanding the middle way of practice, which gives vision and knowledge, and leads to peace, direct knowledge, awakening, and extinguishment. 

And\marginnote{3.1} what is that middle way of practice? It is simply this noble eightfold path, that is: right view, right thought, right speech, right action, right livelihood, right effort, right mindfulness, and right immersion. This is that middle way of practice, which gives vision and knowledge, and leads to peace, direct knowledge, awakening, and extinguishment. 

Now\marginnote{4.1} this is the noble truth of suffering. Rebirth is suffering; old age is suffering; illness is suffering; death is suffering; association with the disliked is suffering; separation from the liked is suffering; not getting what you wish for is suffering. In brief, the five grasping aggregates are suffering. 

Now\marginnote{4.3} this is the noble truth of the origin of suffering. It’s the craving that leads to future lives, mixed up with relishing and greed, chasing pleasure in various realms. That is, craving for sensual pleasures, craving to continue existence, and craving to end existence. 

Now\marginnote{4.6} this is the noble truth of the cessation of suffering. It’s the fading away and cessation of that very same craving with nothing left over; giving it away, letting it go, releasing it, and not adhering to it. 

Now\marginnote{4.8} this is the noble truth of the practice that leads to the cessation of suffering. It is simply this noble eightfold path, that is: right view, right thought, right speech, right action, right livelihood, right effort, right mindfulness, and right immersion. 

‘This\marginnote{5.1} is the noble truth of suffering.’ Such was the vision, knowledge, wisdom, realization, and light that arose in me regarding teachings not learned before from another. ‘This noble truth of suffering should be completely understood.’ Such was the vision that arose in me … ‘This noble truth of suffering has been completely understood.’ Such was the vision that arose in me … 

‘This\marginnote{6.1} is the noble truth of the origin of suffering.’ Such was the vision that arose in me … ‘This noble truth of the origin of suffering should be given up.’ Such was the vision that arose in me … ‘This noble truth of the origin of suffering has been given up.’ Such was the vision that arose in me … 

‘This\marginnote{7.1} is the noble truth of the cessation of suffering.’ Such was the vision that arose in me … ‘This noble truth of the cessation of suffering should be realized.’ Such was the vision that arose in me … ‘This noble truth of the cessation of suffering has been realized.’ Such was the vision that arose in me … 

‘This\marginnote{8.1} is the noble truth of the practice that leads to the cessation of suffering.’ Such was the vision that arose in me … ‘This noble truth of the practice that leads to the cessation of suffering should be developed.’ Such was the vision that arose in me … ‘This noble truth of the practice that leads to the cessation of suffering has been developed.’ Such was the vision, knowledge, wisdom, realization, and light that arose in me regarding teachings not learned before from another. 

As\marginnote{9.1} long as my true knowledge and vision about these four noble truths was not fully purified in these three perspectives and twelve aspects, I didn’t announce my supreme perfect awakening in this world with its gods, \textsanskrit{Māras}, and \textsanskrit{Brahmās}, this population with its ascetics and brahmins, its gods and humans. 

But\marginnote{10.1} when my true knowledge and vision about these four noble truths was fully purified in these three perspectives and twelve aspects, I announced my supreme perfect awakening in this world with its gods, \textsanskrit{Māras}, and \textsanskrit{Brahmās}, this population with its ascetics and brahmins, its gods and humans. 

Knowledge\marginnote{10.2} and vision arose in me: ‘My freedom is unshakable; this is my last rebirth; now there’ll be no more future lives.’” 

That\marginnote{10.4} is what the Buddha said. Satisfied, the group of five mendicants was happy with what the Buddha said. 

And\marginnote{11.1} while this discourse was being spoken, the stainless, immaculate vision of the Dhamma arose in Venerable \textsanskrit{Koṇḍañña}: “Everything that has a beginning has an end.” 

And\marginnote{12.1} when the Buddha rolled forth the Wheel of Dhamma, the earth gods raised the cry: “Near Benares, in the deer park at Isipatana, the Buddha has rolled forth the supreme Wheel of Dhamma. And that wheel cannot be rolled back by any ascetic or brahmin or god or \textsanskrit{Māra} or \textsanskrit{Brahmā} or by anyone in the world.” 

Hearing\marginnote{12.3} the cry of the Earth Gods, the Gods of the Four Great Kings … the Gods of the Thirty-Three … the Gods of Yama … the Joyful Gods … the Gods Who Love to Create … the Gods Who Control the Creations of Others … the Gods of \textsanskrit{Brahmā}’s Host raised the cry: “Near Benares, in the deer park at Isipatana, the Buddha has rolled forth the supreme Wheel of Dhamma. And that wheel cannot be rolled back by any ascetic or brahmin or god or \textsanskrit{Māra} or \textsanskrit{Brahmā} or by anyone in the world.” 

And\marginnote{13.1} so at that moment, in that instant, the cry soared up to the \textsanskrit{Brahmā} realm. And this galaxy shook and rocked and trembled. And an immeasurable, magnificent light appeared in the world, surpassing the glory of the gods. 

Then\marginnote{14.1} the Buddha expressed this heartfelt sentiment: “\textsanskrit{Koṇḍañña} has really understood! \textsanskrit{Koṇḍañña} has really understood!” 

And\marginnote{14.3} that’s how Venerable \textsanskrit{Koṇḍañña} came to be known as “\textsanskrit{Koṇḍañña} Who Understood”. 

%
\section*{{\suttatitleacronym SN 56.12}{\suttatitletranslation The Realized Ones }{\suttatitleroot Tathāgatasutta}}
\addcontentsline{toc}{section}{\tocacronym{SN 56.12} \toctranslation{The Realized Ones } \tocroot{Tathāgatasutta}}
\markboth{The Realized Ones }{Tathāgatasutta}
\extramarks{SN 56.12}{SN 56.12}

“‘This\marginnote{1.1} is the noble truth of suffering.’ Such was the vision, knowledge, wisdom, realization, and light that arose in the Realized Ones regarding teachings not learned before from another. ‘This noble truth of suffering should be completely understood.’ … ‘This noble truth of suffering has been completely understood.’ … 

‘This\marginnote{2.1} is the noble truth of the origin of suffering.’ … ‘This noble truth of the origin of suffering should be given up.’ … ‘This noble truth of the origin of suffering has been given up.’ … 

‘This\marginnote{3.1} is the noble truth of the cessation of suffering.’ … ‘This noble truth of the cessation of suffering should be realized.’ … ‘This noble truth of the cessation of suffering has been realized.’ … 

‘This\marginnote{4.1} is the noble truth of the practice that leads to the cessation of suffering.’ … ‘This noble truth of the practice that leads to the cessation of suffering should be developed.’ … ‘This noble truth of the practice that leads to the cessation of suffering has been developed.’ Such was the vision, knowledge, wisdom, realization, and light that arose in the Realized Ones regarding teachings not learned before from another.” 

%
\section*{{\suttatitleacronym SN 56.13}{\suttatitletranslation Aggregates }{\suttatitleroot Khandhasutta}}
\addcontentsline{toc}{section}{\tocacronym{SN 56.13} \toctranslation{Aggregates } \tocroot{Khandhasutta}}
\markboth{Aggregates }{Khandhasutta}
\extramarks{SN 56.13}{SN 56.13}

“Mendicants,\marginnote{1.1} there are these four noble truths. What four? The noble truths of suffering, the origin of suffering, the cessation of suffering, and the practice that leads to the cessation of suffering. 

And\marginnote{2.1} what is the noble truth of suffering? You should say: ‘The five grasping aggregates’. That is: form, feeling, perception, choices, and consciousness. This is called the noble truth of suffering. 

And\marginnote{3.1} what is the noble truth of the origin of suffering? It’s the craving that leads to future lives, mixed up with relishing and greed, chasing pleasure in various realms. That is, craving for sensual pleasures, craving to continue existence, and craving to end existence. This is called the noble truth of the origin of suffering. 

And\marginnote{4.1} what is the noble truth of the cessation of suffering? It’s the fading away and cessation of that very same craving with nothing left over; giving it away, letting it go, releasing it, and not adhering to it. This is called the noble truth of the cessation of suffering. 

And\marginnote{5.1} what is the noble truth of the practice that leads to the cessation of suffering? It is simply this noble eightfold path, that is: right view, right thought, right speech, right action, right livelihood, right effort, right mindfulness, and right immersion. This is called the noble truth of the practice that leads to the cessation of suffering. These are the four noble truths. 

That’s\marginnote{6.1} why you should practice meditation …” 

%
\section*{{\suttatitleacronym SN 56.14}{\suttatitletranslation Interior Sense Fields }{\suttatitleroot Ajjhattikāyatanasutta}}
\addcontentsline{toc}{section}{\tocacronym{SN 56.14} \toctranslation{Interior Sense Fields } \tocroot{Ajjhattikāyatanasutta}}
\markboth{Interior Sense Fields }{Ajjhattikāyatanasutta}
\extramarks{SN 56.14}{SN 56.14}

“Mendicants,\marginnote{1.1} there are these four noble truths. What four? The noble truths of suffering, the origin of suffering, the cessation of suffering, and the practice that leads to the cessation of suffering. 

And\marginnote{2.1} what is the noble truth of suffering? You should say: ‘The six interior sense fields’. What six? The sense fields of the eye, ear, nose, tongue, body, and mind. This is called the noble truth of suffering. …” 

%
\section*{{\suttatitleacronym SN 56.15}{\suttatitletranslation Remembering (1st) }{\suttatitleroot Paṭhamadhāraṇasutta}}
\addcontentsline{toc}{section}{\tocacronym{SN 56.15} \toctranslation{Remembering (1st) } \tocroot{Paṭhamadhāraṇasutta}}
\markboth{Remembering (1st) }{Paṭhamadhāraṇasutta}
\extramarks{SN 56.15}{SN 56.15}

“Mendicants,\marginnote{1.1} do you remember the four noble truths that I taught?” When he said this, one of the mendicants said to the Buddha: 

“I\marginnote{1.3} do, sir.” 

“How\marginnote{1.4} so, mendicant?” 

“Sir,\marginnote{1.5} I remember that suffering is the first noble truth you’ve taught; the origin of suffering is the second; the cessation of suffering is the third; and the practice that leads to the cessation of suffering is the fourth. That’s how I remember the four noble truths as you’ve taught them.” 

“Good,\marginnote{2.1} good, mendicant! It’s good that you remember the four noble truths as I’ve taught them. Suffering is the first noble truth I’ve taught, and that’s how you should remember it. The origin of suffering is the second; the cessation of suffering is the third; and the practice that leads to the cessation of suffering is the fourth. That’s how you should remember the four noble truths as I’ve taught them. 

That’s\marginnote{3.1} why you should practice meditation …” 

%
\section*{{\suttatitleacronym SN 56.16}{\suttatitletranslation Remembering (2nd) }{\suttatitleroot Dutiyadhāraṇasutta}}
\addcontentsline{toc}{section}{\tocacronym{SN 56.16} \toctranslation{Remembering (2nd) } \tocroot{Dutiyadhāraṇasutta}}
\markboth{Remembering (2nd) }{Dutiyadhāraṇasutta}
\extramarks{SN 56.16}{SN 56.16}

“Mendicants,\marginnote{1.1} do you remember the four noble truths that I taught?” When he said this, one of the mendicants said to the Buddha: 

“I\marginnote{1.3} do, sir.” 

“How\marginnote{2.1} so, mendicant?” 

“Sir,\marginnote{2.2} I remember that suffering is the first noble truth you’ve taught. For if any ascetic or brahmin should say this: ‘What the ascetic Gotama teaches is not the first noble truth of suffering. I’ll reject this first noble truth of suffering and describe another first noble truth of suffering.’ That is not possible. The origin of suffering … The cessation of suffering … The practice that leads to the cessation of suffering is the fourth noble truth you’ve taught. For if any ascetic or brahmin should say this: ‘What the ascetic Gotama teaches is not the fourth noble truth of the practice that leads to the cessation of suffering. I’ll reject this fourth noble truth of the practice that leads to the cessation of suffering and describe another fourth noble truth of the practice that leads to the cessation of suffering.’ That is not possible. That’s how I remember the four noble truths as you’ve taught them.” 

“Good,\marginnote{3.1} good, mendicant! It’s good that you remember the four noble truths as I’ve taught them. Suffering is the first noble truth I’ve taught, and that’s how you should remember it. For if any ascetic or brahmin should say this: ‘What the ascetic Gotama teaches is not the first noble truth of suffering. I’ll reject this first noble truth of suffering and describe another first noble truth of suffering.’ That is not possible. The origin of suffering … The cessation of suffering … The practice that leads to the cessation of suffering is the fourth noble truth I’ve taught, and that’s how you should remember it. For if any ascetic or brahmin should say this: ‘What the ascetic Gotama teaches is not the fourth noble truth of the practice that leads to the cessation of suffering. I’ll reject this fourth noble truth of the practice that leads to the cessation of suffering and describe another fourth noble truth of the practice that leads to the cessation of suffering.’ That is not possible. That’s how you should remember the four noble truths as I’ve taught them. 

That’s\marginnote{4.1} why you should practice meditation …” 

%
\section*{{\suttatitleacronym SN 56.17}{\suttatitletranslation Ignorance }{\suttatitleroot Avijjāsutta}}
\addcontentsline{toc}{section}{\tocacronym{SN 56.17} \toctranslation{Ignorance } \tocroot{Avijjāsutta}}
\markboth{Ignorance }{Avijjāsutta}
\extramarks{SN 56.17}{SN 56.17}

Seated\marginnote{1.1} to one side, that mendicant said to the Buddha: 

“Sir,\marginnote{1.2} they speak of this thing called ‘ignorance’. What is ignorance? And how is an ignorant person defined?” 

“Not\marginnote{1.5} knowing about suffering, the origin of suffering, the cessation of suffering, and the practice that leads to the cessation of suffering. This is called ignorance. And this is how an ignorant person is defined. 

That’s\marginnote{2.1} why you should practice meditation …” 

%
\section*{{\suttatitleacronym SN 56.18}{\suttatitletranslation Knowledge }{\suttatitleroot Vijjāsutta}}
\addcontentsline{toc}{section}{\tocacronym{SN 56.18} \toctranslation{Knowledge } \tocroot{Vijjāsutta}}
\markboth{Knowledge }{Vijjāsutta}
\extramarks{SN 56.18}{SN 56.18}

Then\marginnote{1.1} a mendicant went up to the Buddha, bowed, sat down to one side, and said to him: 

“Sir,\marginnote{1.2} they speak of this thing called ‘knowledge’. What is knowledge? And how is a knowledgeable person defined?” 

“Knowing\marginnote{1.5} about suffering, the origin of suffering, the cessation of suffering, and the practice that leads to the cessation of suffering. This is called knowledge. And this is how a knowledgeable person is defined. 

That’s\marginnote{2.1} why you should practice meditation …” 

%
\section*{{\suttatitleacronym SN 56.19}{\suttatitletranslation Expressions }{\suttatitleroot Saṅkāsanasutta}}
\addcontentsline{toc}{section}{\tocacronym{SN 56.19} \toctranslation{Expressions } \tocroot{Saṅkāsanasutta}}
\markboth{Expressions }{Saṅkāsanasutta}
\extramarks{SN 56.19}{SN 56.19}

“Mendicants,\marginnote{1.1} I’ve declared: ‘This is the noble truth of suffering.’ And there are limitless explanations, phrases, and expressions about that: ‘This is another way of saying that this is the noble truth of suffering.’ I’ve declared: ‘This is the noble truth of the origin of suffering.’ … I’ve declared: ‘This is the noble truth of the cessation of suffering.’ … I’ve declared: ‘This is the noble truth of the practice that leads to the cessation of suffering.’ And there are limitless explanations, phrases, and expressions about that: ‘This is another way of saying that this is the noble truth of the practice that leads to the cessation of suffering.’ 

That’s\marginnote{2.1} why you should practice meditation …” 

%
\section*{{\suttatitleacronym SN 56.20}{\suttatitletranslation Real }{\suttatitleroot Tathasutta}}
\addcontentsline{toc}{section}{\tocacronym{SN 56.20} \toctranslation{Real } \tocroot{Tathasutta}}
\markboth{Real }{Tathasutta}
\extramarks{SN 56.20}{SN 56.20}

“Mendicants,\marginnote{1.1} these four things are real, not unreal, not otherwise. What four? ‘This is suffering’ … ‘This is the origin of suffering’ … ‘This is the cessation of suffering’ … ‘This is the practice that leads to the cessation of suffering’ … These four things are real, not unreal, not otherwise. 

That’s\marginnote{2.1} why you should practice meditation …” 

%
\addtocontents{toc}{\let\protect\contentsline\protect\nopagecontentsline}
\chapter*{The Chapter at the Village of Koṭi }
\addcontentsline{toc}{chapter}{\tocchapterline{The Chapter at the Village of Koṭi }}
\addtocontents{toc}{\let\protect\contentsline\protect\oldcontentsline}

%
\section*{{\suttatitleacronym SN 56.21}{\suttatitletranslation At the Village of Koṭi (1st) }{\suttatitleroot Paṭhamakoṭigāmasutta}}
\addcontentsline{toc}{section}{\tocacronym{SN 56.21} \toctranslation{At the Village of Koṭi (1st) } \tocroot{Paṭhamakoṭigāmasutta}}
\markboth{At the Village of Koṭi (1st) }{Paṭhamakoṭigāmasutta}
\extramarks{SN 56.21}{SN 56.21}

At\marginnote{1.1} one time the Buddha was staying in the land of the Vajjis at the village of \textsanskrit{Koṭi}. There the Buddha addressed the mendicants: “Mendicants, not understanding and not penetrating four noble truths, both you and I have wandered and transmigrated for such a very long time. 

What\marginnote{2.1} four? The noble truths of suffering, the origin of suffering, the cessation of suffering, and the practice that leads to the cessation of suffering. These noble truths of suffering, origin, cessation, and the path have been understood and comprehended. Craving for continued existence has been cut off; the conduit to rebirth is ended; now there’ll be no more future lives.” 

That\marginnote{3.1} is what the Buddha said. Then the Holy One, the Teacher, went on to say: 

\begin{verse}%
“Because\marginnote{4.1} of not truly seeing \\
the four noble truths, \\
we have transmigrated for a long time \\
from one rebirth to the next. 

But\marginnote{5.1} now that these truths have been seen, \\
the conduit to rebirth is eradicated. \\
The root of suffering is cut off, \\
now there’ll be no more future lives.” 

%
\end{verse}

%
\section*{{\suttatitleacronym SN 56.22}{\suttatitletranslation At the Village of Koṭi (2nd) }{\suttatitleroot Dutiyakoṭigāmasutta}}
\addcontentsline{toc}{section}{\tocacronym{SN 56.22} \toctranslation{At the Village of Koṭi (2nd) } \tocroot{Dutiyakoṭigāmasutta}}
\markboth{At the Village of Koṭi (2nd) }{Dutiyakoṭigāmasutta}
\extramarks{SN 56.22}{SN 56.22}

“Mendicants,\marginnote{1.1} there are ascetics and brahmins who don’t truly understand about suffering, its origin, its cessation, and the path. I don’t regard them as true ascetics and brahmins. Those venerables don’t realize the goal of life as an ascetic or brahmin, and don’t live having realized it with their own insight. 

There\marginnote{2.1} are ascetics and brahmins who do truly understand about suffering, its origin, its cessation, and the path. I regard them as true ascetics and brahmins. Those venerables realize the goal of life as an ascetic or brahmin, and live having realized it with their own insight.” 

That\marginnote{3.1} is what the Buddha said. Then the Holy One, the Teacher, went on to say: 

\begin{verse}%
“There\marginnote{4.1} are those who don’t understand suffering \\
and suffering’s cause, \\
and where all suffering \\
ceases with nothing left over. 

And\marginnote{5.1} they don’t know the path \\
that leads to the stilling of suffering. \\
They lack the heart’s release, \\
as well as the release by wisdom. \\
Unable to make an end, \\
they continue to be reborn and grow old. 

But\marginnote{6.1} there are those who understand suffering \\
and suffering’s cause, \\
and where all suffering \\
ceases with nothing left over. 

And\marginnote{7.1} they understand the path \\
that leads to the stilling of suffering. \\
They’re endowed with the heart’s release, \\
as well as the release by wisdom. \\
Able to make an end, \\
they don’t continue to be reborn and grow old.” 

%
\end{verse}

%
\section*{{\suttatitleacronym SN 56.23}{\suttatitletranslation The Fully Awakened Buddha }{\suttatitleroot Sammāsambuddhasutta}}
\addcontentsline{toc}{section}{\tocacronym{SN 56.23} \toctranslation{The Fully Awakened Buddha } \tocroot{Sammāsambuddhasutta}}
\markboth{The Fully Awakened Buddha }{Sammāsambuddhasutta}
\extramarks{SN 56.23}{SN 56.23}

At\marginnote{1.1} \textsanskrit{Sāvatthī}. 

“Mendicants,\marginnote{1.2} there are these four noble truths. What four? The noble truths of suffering, the origin of suffering, the cessation of suffering, and the practice that leads to the cessation of suffering. These are the four noble truths. It is because he has truly woken up to these four noble truths that the Realized One is called ‘the perfected one, the fully awakened Buddha’. 

That’s\marginnote{2.1} why you should practice meditation …” 

%
\section*{{\suttatitleacronym SN 56.24}{\suttatitletranslation The Perfected Ones }{\suttatitleroot Arahantasutta}}
\addcontentsline{toc}{section}{\tocacronym{SN 56.24} \toctranslation{The Perfected Ones } \tocroot{Arahantasutta}}
\markboth{The Perfected Ones }{Arahantasutta}
\extramarks{SN 56.24}{SN 56.24}

At\marginnote{1.1} \textsanskrit{Sāvatthī}. 

“Mendicants,\marginnote{1.2} whatever perfected ones, fully awakened Buddhas truly wake up—in the past, future, or present—all of them truly wake up to the four noble truths. 

What\marginnote{2.1} four? The noble truths of suffering, the origin of suffering, the cessation of suffering, and the practice that leads to the cessation of suffering. Whatever perfected ones, fully awakened Buddhas truly wake up—in the past, future, or present—all of them truly wake up to the four noble truths. 

That’s\marginnote{3.1} why you should practice meditation …” 

%
\section*{{\suttatitleacronym SN 56.25}{\suttatitletranslation The Ending of Defilements }{\suttatitleroot Āsavakkhayasutta}}
\addcontentsline{toc}{section}{\tocacronym{SN 56.25} \toctranslation{The Ending of Defilements } \tocroot{Āsavakkhayasutta}}
\markboth{The Ending of Defilements }{Āsavakkhayasutta}
\extramarks{SN 56.25}{SN 56.25}

“Mendicants,\marginnote{1.1} I say that the ending of defilements is for one who knows and sees, not for one who does not know or see. For one who knows and sees what? The ending of defilements is for one who knows and sees suffering, its origin, its cessation, and the path. The ending of the defilements is for one who knows and sees this. 

That’s\marginnote{2.1} why you should practice meditation …” 

%
\section*{{\suttatitleacronym SN 56.26}{\suttatitletranslation Friends }{\suttatitleroot Mittasutta}}
\addcontentsline{toc}{section}{\tocacronym{SN 56.26} \toctranslation{Friends } \tocroot{Mittasutta}}
\markboth{Friends }{Mittasutta}
\extramarks{SN 56.26}{SN 56.26}

“Mendicants,\marginnote{1.1} those who you have sympathy for, and those worth listening to—friends and colleagues, relatives and family—should be encouraged, supported, and established in the true comprehension of the four noble truths. 

What\marginnote{2.1} four? The noble truths of suffering, the origin of suffering, the cessation of suffering, and the practice that leads to the cessation of suffering. Those who you have sympathy for, and those worth listening to—friends and colleagues, relatives and family—should be encouraged, supported, and established in the true comprehension of these four noble truths. 

That’s\marginnote{3.1} why you should practice meditation …” 

%
\section*{{\suttatitleacronym SN 56.27}{\suttatitletranslation Real }{\suttatitleroot Tathasutta}}
\addcontentsline{toc}{section}{\tocacronym{SN 56.27} \toctranslation{Real } \tocroot{Tathasutta}}
\markboth{Real }{Tathasutta}
\extramarks{SN 56.27}{SN 56.27}

“Mendicants,\marginnote{1.1} there are these four noble truths. What four? The noble truths of suffering, the origin of suffering, the cessation of suffering, and the practice that leads to the cessation of suffering. These four noble truths are real, not unreal, not otherwise. That’s why they’re called ‘noble truths’. 

That’s\marginnote{2.1} why you should practice meditation …” 

%
\section*{{\suttatitleacronym SN 56.28}{\suttatitletranslation The World }{\suttatitleroot Lokasutta}}
\addcontentsline{toc}{section}{\tocacronym{SN 56.28} \toctranslation{The World } \tocroot{Lokasutta}}
\markboth{The World }{Lokasutta}
\extramarks{SN 56.28}{SN 56.28}

“Mendicants,\marginnote{1.1} there are these four noble truths. What four? The noble truths of suffering, the origin of suffering, the cessation of suffering, and the practice that leads to the cessation of suffering. In this world with its gods, \textsanskrit{Māras}, and \textsanskrit{Brahmās}, this population with its ascetics and brahmins, its gods and humans, the Realized One is the Noble One. That’s why they’re called ‘noble truths’. 

That’s\marginnote{2.1} why you should practice meditation …” 

%
\section*{{\suttatitleacronym SN 56.29}{\suttatitletranslation Should Be Completely Understood }{\suttatitleroot Pariññeyyasutta}}
\addcontentsline{toc}{section}{\tocacronym{SN 56.29} \toctranslation{Should Be Completely Understood } \tocroot{Pariññeyyasutta}}
\markboth{Should Be Completely Understood }{Pariññeyyasutta}
\extramarks{SN 56.29}{SN 56.29}

“Mendicants,\marginnote{1.1} there are these four noble truths. What four? The noble truths of suffering, the origin of suffering, the cessation of suffering, and the practice that leads to the cessation of suffering. These are the four noble truths. Of these four noble truths, there is one to be completely understood, one to be given up, one to be realized, and one to be developed. 

And\marginnote{2.1} which noble truth should be completely understood? The noble truth of suffering should be completely understood. The noble truth of the origin of suffering should be given up. The noble truth of the cessation of suffering should be realized. The noble truth of the practice that leads to the cessation of suffering should be developed. 

That’s\marginnote{3.1} why you should practice meditation …” 

%
\section*{{\suttatitleacronym SN 56.30}{\suttatitletranslation With Gavampati }{\suttatitleroot Gavampatisutta}}
\addcontentsline{toc}{section}{\tocacronym{SN 56.30} \toctranslation{With Gavampati } \tocroot{Gavampatisutta}}
\markboth{With Gavampati }{Gavampatisutta}
\extramarks{SN 56.30}{SN 56.30}

At\marginnote{1.1} one time several mendicants were staying in the land of the Cetis at \textsanskrit{Sahajāti}. Now at that time, after the meal, on return from almsround, several senior mendicants sat together in the pavilion and this discussion came up among them: 

“Reverends,\marginnote{1.3} does someone who sees suffering also see the origin of suffering, the cessation of suffering, and the practice that leads to the cessation of suffering?” 

When\marginnote{2.1} they said this, Venerable Gavampati said to those senior mendicants: 

“Reverends,\marginnote{2.2} I have heard and learned this in the presence of the Buddha: ‘Someone who sees suffering also sees the origin of suffering, the cessation of suffering, and the practice that leads to the cessation of suffering. Someone who sees the origin of suffering also sees suffering, the cessation of suffering, and the practice that leads to the cessation of suffering. Someone who sees the cessation of suffering also sees suffering, the origin of suffering, and the practice that leads to the cessation of suffering. Someone who sees the practice that leads to the cessation of suffering also sees suffering, the origin of suffering, and the cessation of suffering.’” 

%
\addtocontents{toc}{\let\protect\contentsline\protect\nopagecontentsline}
\chapter*{The Chapter in a Rosewood Forest }
\addcontentsline{toc}{chapter}{\tocchapterline{The Chapter in a Rosewood Forest }}
\addtocontents{toc}{\let\protect\contentsline\protect\oldcontentsline}

%
\section*{{\suttatitleacronym SN 56.31}{\suttatitletranslation In a Rosewood Forest }{\suttatitleroot Sīsapāvanasutta}}
\addcontentsline{toc}{section}{\tocacronym{SN 56.31} \toctranslation{In a Rosewood Forest } \tocroot{Sīsapāvanasutta}}
\markboth{In a Rosewood Forest }{Sīsapāvanasutta}
\extramarks{SN 56.31}{SN 56.31}

At\marginnote{1.1} one time the Buddha was staying near \textsanskrit{Kosambī} in a rosewood forest. Then the Buddha picked up a few rosewood leaves in his hand and addressed the mendicants: “What do you think, mendicants? Which is more: the few leaves in my hand, or those in the forest above me?” 

“Sir,\marginnote{1.6} the few leaves in your hand are a tiny amount. There are far more leaves in the forest above.” 

“In\marginnote{1.8} the same way, there is much more that I have directly known but have not explained to you. What I have explained is a tiny amount. And why haven’t I explained it? Because it’s not beneficial or relevant to the fundamentals of the spiritual life. It doesn’t lead to disillusionment, dispassion, cessation, peace, insight, awakening, and extinguishment. That’s why I haven’t explained it. 

And\marginnote{2.1} what have I explained? I have explained: ‘This is suffering’ … ‘This is the origin of suffering’ … ‘This is the cessation of suffering’ … ‘This is the practice that leads to the cessation of suffering’. 

And\marginnote{3.1} why have I explained this? Because it’s beneficial and relevant to the fundamentals of the spiritual life. It leads to disillusionment, dispassion, cessation, peace, insight, awakening, and extinguishment. That’s why I’ve explained it. 

That’s\marginnote{4.1} why you should practice meditation …” 

%
\section*{{\suttatitleacronym SN 56.32}{\suttatitletranslation Acacia Leaves }{\suttatitleroot Khadirapattasutta}}
\addcontentsline{toc}{section}{\tocacronym{SN 56.32} \toctranslation{Acacia Leaves } \tocroot{Khadirapattasutta}}
\markboth{Acacia Leaves }{Khadirapattasutta}
\extramarks{SN 56.32}{SN 56.32}

“Mendicants,\marginnote{1.1} suppose someone were to say: ‘Without truly comprehending the noble truths of suffering, its origin, its cessation, and the path, I will completely make an end of suffering.’ That is not possible. 

It’s\marginnote{2.1} as if someone were to say: ‘I’ll make a basket out of acacia leaves or pine needles or myrobalan leaves, and use it to carry water or a palm frond.’ That is not possible. In the same way, suppose someone were to say: ‘Without truly comprehending the noble truths of suffering, its origin, its cessation, and the path, I will completely make an end of suffering.’ That is not possible. 

But\marginnote{3.1} suppose someone were to say: ‘After truly comprehending the noble truths of suffering, its origin, its cessation, and the path, I will completely make an end of suffering.’ That is possible. 

It’s\marginnote{4.1} as if someone were to say: ‘I’ll make a basket out of lotus leaves or flame-of-the-forest leaves or camel’s foot creeper leaves, and use it to carry water or a palm frond.’ That is possible. In the same way, suppose someone were to say: ‘After truly comprehending the noble truths of suffering, its origin, its cessation, and the path, I will completely make an end of suffering.’ That is possible. 

That’s\marginnote{5.1} why you should practice meditation …” 

%
\section*{{\suttatitleacronym SN 56.33}{\suttatitletranslation A Stick }{\suttatitleroot Daṇḍasutta}}
\addcontentsline{toc}{section}{\tocacronym{SN 56.33} \toctranslation{A Stick } \tocroot{Daṇḍasutta}}
\markboth{A Stick }{Daṇḍasutta}
\extramarks{SN 56.33}{SN 56.33}

“Mendicants,\marginnote{1.1} suppose a stick was tossed up in the air. Sometimes it’d fall on its bottom and sometimes the top. It’s the same for sentient beings roaming and transmigrating, shrouded by ignorance and fettered by craving. Sometimes they go from this world to the other world, and sometimes they come from the other world to this world. Why is that? It’s because they haven’t seen the four noble truths. What four? The noble truths of suffering, its origin, its cessation, and the path. 

That’s\marginnote{2.1} why you should practice meditation …” 

%
\section*{{\suttatitleacronym SN 56.34}{\suttatitletranslation Clothes }{\suttatitleroot Celasutta}}
\addcontentsline{toc}{section}{\tocacronym{SN 56.34} \toctranslation{Clothes } \tocroot{Celasutta}}
\markboth{Clothes }{Celasutta}
\extramarks{SN 56.34}{SN 56.34}

“Mendicants,\marginnote{1.1} if your clothes or head were on fire, what would you do about it?” 

“Sir,\marginnote{1.2} if our clothes or head were on fire, we’d apply intense enthusiasm, effort, zeal, vigor, perseverance, mindfulness, and situational awareness in order to extinguish it.” 

“Mendicants,\marginnote{2.1} regarding your burning head or clothes with equanimity, not paying attention to them, you should apply intense enthusiasm, effort, zeal, vigor, perseverance, mindfulness, and situational awareness to truly comprehending the four noble truths. What four? The noble truths of suffering, its origin, its cessation, and the path. 

That’s\marginnote{3.1} why you should practice meditation …” 

%
\section*{{\suttatitleacronym SN 56.35}{\suttatitletranslation A Hundred Spears }{\suttatitleroot Sattisatasutta}}
\addcontentsline{toc}{section}{\tocacronym{SN 56.35} \toctranslation{A Hundred Spears } \tocroot{Sattisatasutta}}
\markboth{A Hundred Spears }{Sattisatasutta}
\extramarks{SN 56.35}{SN 56.35}

“Mendicants,\marginnote{1.1} suppose there was a man with a lifespan of a hundred years. And someone might say to him: ‘Come now, my good man, they’ll strike you with a hundred spears in the morning, at midday, and in the late afternoon. And you’ll live for a hundred years being struck with three hundred spears every day. But when a hundred years have passed, you will comprehend the four noble truths for the first time.’ 

For\marginnote{2.1} an earnest gentleman this is sufficient reason to submit. 

Why\marginnote{2.2} is that? Transmigration has no known beginning. No first point is found of blows by spears, swords, arrows, and axes. Now this may be so. But the comprehension of the four noble truths doesn’t come with pain or sadness, I say. Rather, the comprehension of the four noble truths comes only with pleasure and happiness, I say. What four? The noble truths of suffering, its origin, its cessation, and the path. 

That’s\marginnote{3.1} why you should practice meditation …” 

%
\section*{{\suttatitleacronym SN 56.36}{\suttatitletranslation Living Creatures }{\suttatitleroot Pāṇasutta}}
\addcontentsline{toc}{section}{\tocacronym{SN 56.36} \toctranslation{Living Creatures } \tocroot{Pāṇasutta}}
\markboth{Living Creatures }{Pāṇasutta}
\extramarks{SN 56.36}{SN 56.36}

“Suppose\marginnote{1.1} a person was to strip all the grass, sticks, branches, and leaves in India, gather them together into one pile, and make them into stakes. Then they’d impale the large creatures in the ocean on large stakes; the medium-sized creatures on medium-sized stakes; and the small creatures on small stakes. They wouldn’t run out of sizable creatures in the ocean before using up all the grass, sticks, branches, and leaves in India. There are far more small creatures in the ocean than this, so it wouldn’t be feasible to impale them on stakes. Why is that? Because of the small size of those life-forms. That’s how big the plane of loss is. 

A\marginnote{2.6} person accomplished in view, exempt from that vast plane of loss, truly understands: ‘This is suffering’ … ‘This is the origin of suffering’ … ‘This is the cessation of suffering’ … ‘This is the practice that leads to the cessation of suffering’. 

That’s\marginnote{3.1} why you should practice meditation …” 

%
\section*{{\suttatitleacronym SN 56.37}{\suttatitletranslation The Simile of the Sun (1st) }{\suttatitleroot Paṭhamasūriyasutta}}
\addcontentsline{toc}{section}{\tocacronym{SN 56.37} \toctranslation{The Simile of the Sun (1st) } \tocroot{Paṭhamasūriyasutta}}
\markboth{The Simile of the Sun (1st) }{Paṭhamasūriyasutta}
\extramarks{SN 56.37}{SN 56.37}

“Mendicants,\marginnote{1.1} the dawn is the forerunner and precursor of the sunrise. 

In\marginnote{1.2} the same way, right view is the forerunner and precursor of truly comprehending the four noble truths. A mendicant with right view can expect to truly understand: ‘This is suffering’ … ‘This is the origin of suffering’ … ‘This is the cessation of suffering’ … ‘This is the practice that leads to the cessation of suffering’. 

That’s\marginnote{2.1} why you should practice meditation …” 

%
\section*{{\suttatitleacronym SN 56.38}{\suttatitletranslation The Simile of the Sun (2nd) }{\suttatitleroot Dutiyasūriyasutta}}
\addcontentsline{toc}{section}{\tocacronym{SN 56.38} \toctranslation{The Simile of the Sun (2nd) } \tocroot{Dutiyasūriyasutta}}
\markboth{The Simile of the Sun (2nd) }{Dutiyasūriyasutta}
\extramarks{SN 56.38}{SN 56.38}

“Mendicants,\marginnote{1.1} as long as the moon and the sun don’t arise in the world, no great light or great radiance appears. Darkness prevails then, utter darkness. Day and night aren’t found, nor months and fortnights, nor seasons and years. 

But\marginnote{2.1} when the moon and the sun arise in the world, a great light, a great radiance appears. Darkness no longer prevails. Day and night are found, and months and fortnights, and seasons and years. 

In\marginnote{2.4} the same way, as long as the Realized One doesn’t arise in the world, no great light or great radiance appears. Darkness prevails then, utter darkness. There’s no explanation of the four noble truths, no teaching, advocating, establishing, clarifying, analyzing, and revealing of them. 

But\marginnote{3.1} when the Realized One arises in the world, a great light, a great radiance appears. Darkness no longer prevails. Then there’s the explanation of the four noble truths, the teaching, advocating, establishing, clarifying, analyzing, and revealing of them. What four? The noble truths of suffering, its origin, its cessation, and the path. 

That’s\marginnote{4.1} why you should practice meditation …” 

%
\section*{{\suttatitleacronym SN 56.39}{\suttatitletranslation A Boundary Pillar }{\suttatitleroot Indakhīlasutta}}
\addcontentsline{toc}{section}{\tocacronym{SN 56.39} \toctranslation{A Boundary Pillar } \tocroot{Indakhīlasutta}}
\markboth{A Boundary Pillar }{Indakhīlasutta}
\extramarks{SN 56.39}{SN 56.39}

“Mendicants,\marginnote{1.1} there are ascetics and brahmins who don’t truly understand about suffering, its origin, its cessation, and the path. They gaze up at the face of another ascetic or brahmin, thinking: ‘Surely this worthy one knows and sees.’ 

Suppose\marginnote{2.1} there was a light tuft of cotton-wool or kapok which was taken up by the wind and landed on level ground. The east wind wafts it west; the west wind wafts it east; the north wind wafts it south; and the south wind wafts it north. Why is that? It’s because the tuft of cotton-wool is so light. 

In\marginnote{2.5} the same way, there are ascetics and brahmins who don’t truly understand about suffering, its origin, its cessation, and the path. They gaze up at the face of another ascetic or brahmin, thinking: ‘Surely this worthy one knows and sees.’ Why is that? It’s because they haven’t seen the four noble truths. 

There\marginnote{3.1} are ascetics and brahmins who truly understand about suffering, its origin, its cessation, and the path. They don’t gaze up at the face of another ascetic or brahmin, thinking: ‘Surely this worthy one knows and sees.’ 

Suppose\marginnote{4.1} there was an iron pillar or a boundary pillar with deep foundations, firmly embedded, imperturbable and unshakable. Even if violent storms were to blow up out of the east, the west, the north, and the south, they couldn’t make it shake or rock or tremble. Why is that? It’s because that boundary pillar is firmly embedded, with deep foundations. 

In\marginnote{4.5} the same way, there are ascetics and brahmins who truly understand about suffering, its origin, its cessation, and the path. They don’t gaze up at the face of another ascetic or brahmin, thinking: ‘Surely this worthy one knows and sees.’ Why is that? It’s because they have clearly seen the four noble truths. What four? The noble truths of suffering, its origin, its cessation, and the path. 

That’s\marginnote{5.1} why you should practice meditation …” 

%
\section*{{\suttatitleacronym SN 56.40}{\suttatitletranslation Looking For a Debate }{\suttatitleroot Vādatthikasutta}}
\addcontentsline{toc}{section}{\tocacronym{SN 56.40} \toctranslation{Looking For a Debate } \tocroot{Vādatthikasutta}}
\markboth{Looking For a Debate }{Vādatthikasutta}
\extramarks{SN 56.40}{SN 56.40}

“Mendicants,\marginnote{1.1} take any mendicant who truly understands: ‘This is suffering’ … ‘This is the origin of suffering’ … ‘This is the cessation of suffering’ … ‘This is the practice that leads to the cessation of suffering’. An ascetic or brahmin might come from the east, west, north, or south wanting to debate, seeking a debate, thinking: ‘I’ll refute their doctrine!’ It’s simply impossible for them to legitimately make that mendicant shake or rock or tremble. 

Suppose\marginnote{2.1} there was a stone pillar, sixteen feet long. Eight feet were buried underground, and eight above ground. Even if violent storms were to blow up out of the east, the west, the north, and the south, they couldn’t make it shake or rock or tremble. Why is that? It’s because that boundary pillar is firmly embedded, with deep foundations. 

In\marginnote{2.6} the same way, take any mendicant who truly understands: ‘This is suffering’ … ‘This is the origin of suffering’ … ‘This is the cessation of suffering’ … ‘This is the practice that leads to the cessation of suffering’. An ascetic or brahmin might come from the east, west, north, or south wanting to debate, seeking a debate, thinking: ‘I’ll refute their doctrine!’ It’s simply impossible for them to legitimately make that mendicant shake or rock or tremble. Why is that? It’s because they have clearly seen the four noble truths. What four? The noble truths of suffering, its origin, its cessation, and the path. 

That’s\marginnote{3.1} why you should practice meditation …” 

%
\addtocontents{toc}{\let\protect\contentsline\protect\nopagecontentsline}
\chapter*{The Chapter on a Cliff }
\addcontentsline{toc}{chapter}{\tocchapterline{The Chapter on a Cliff }}
\addtocontents{toc}{\let\protect\contentsline\protect\oldcontentsline}

%
\section*{{\suttatitleacronym SN 56.41}{\suttatitletranslation Speculation About the World }{\suttatitleroot Lokacintāsutta}}
\addcontentsline{toc}{section}{\tocacronym{SN 56.41} \toctranslation{Speculation About the World } \tocroot{Lokacintāsutta}}
\markboth{Speculation About the World }{Lokacintāsutta}
\extramarks{SN 56.41}{SN 56.41}

At\marginnote{1.1} one time the Buddha was staying near \textsanskrit{Rājagaha}, in the Bamboo Grove, the squirrels’ feeding ground. There the Buddha addressed the mendicants: 

“Once\marginnote{1.3} upon a time, mendicants, a certain person left \textsanskrit{Rājagaha}, thinking ‘I’ll speculate about the world.’ They went to the \textsanskrit{Sumāgadhā} lotus pond and sat down on the bank speculating about the world. Then that person saw an army of four divisions enter a lotus stalk. When he saw this he thought, ‘I’ve gone mad, really, I’ve lost my mind! I’m seeing things that don’t exist in the world.’ 

Then\marginnote{2.1} that person entered the city and informed a large crowd, ‘I’ve gone mad, really, I’ve lost my mind! I’m seeing things that don’t exist in the world.’ 

‘But\marginnote{2.4} how is it that you’re mad? How have you lost your mind? And what have you seen that doesn’t exist in the world?’ 

‘Sirs,\marginnote{2.6} I left \textsanskrit{Rājagaha}, thinking “I’ll speculate about the world.” I went to the \textsanskrit{Sumāgadhā} lotus pond and sat down on the bank speculating about the world. Then I saw an army of four divisions enter a lotus stalk. That’s why I’m mad, that’s why I’ve lost my mind. And that’s what I’ve seen that doesn’t exist in the world.’ 

‘Well,\marginnote{2.10} mister, you’re definitely mad, you’ve definitely lost your mind. And you’re seeing things that don’t exist in the world.’ 

But\marginnote{3.1} what that person saw was in fact real, not unreal. Once upon a time, a battle was fought between the gods and the demons. In that battle the gods won and the demons lost. The defeated and terrified demons entered the citadel of the demons through the lotus stalk only to confuse the gods. 

So\marginnote{4.1} mendicants, don’t speculate about the world. For example: the cosmos is eternal, or not eternal, or finite, or infinite; the soul and the body are the same thing, or they are different things; after death, a Realized One exists, or doesn’t exist, or both exists and doesn’t exist, or neither exists nor doesn’t exist. Why is that? Because those thoughts aren’t beneficial or relevant to the fundamentals of the spiritual life. They don’t lead to disillusionment, dispassion, cessation, peace, insight, awakening, and extinguishment. 

When\marginnote{5.1} you think something up, you should think: ‘This is suffering’ … ‘This is the origin of suffering’ … ‘This is the cessation of suffering’ … ‘This is the practice that leads to the cessation of suffering’. Why is that? Because those thoughts are beneficial and relevant to the fundamentals of the spiritual life. They lead to disillusionment, dispassion, cessation, peace, insight, awakening, and extinguishment. 

That’s\marginnote{6.1} why you should practice meditation …” 

%
\section*{{\suttatitleacronym SN 56.42}{\suttatitletranslation A Cliff }{\suttatitleroot Papātasutta}}
\addcontentsline{toc}{section}{\tocacronym{SN 56.42} \toctranslation{A Cliff } \tocroot{Papātasutta}}
\markboth{A Cliff }{Papātasutta}
\extramarks{SN 56.42}{SN 56.42}

At\marginnote{1.1} one time the Buddha was staying near \textsanskrit{Rājagaha}, on the Vulture’s Peak Mountain. 

Then\marginnote{1.2} the Buddha said to the mendicants, “Come, mendicants, let’s go to Inspiration Peak for the day’s meditation. 

“Yes,\marginnote{1.4} sir,” they replied. Then the Buddha together with several mendicants went to Inspiration Peak. 

A\marginnote{1.6} certain mendicant saw the big cliff there and said to the Buddha, “Sir, that big cliff is really huge and scary. Is there any other cliff bigger and scarier than this one?” 

“There\marginnote{1.10} is, mendicant.” 

“But\marginnote{2.1} sir, what is it?” 

“Mendicant,\marginnote{2.2} there are ascetics and brahmins who don’t truly understand about suffering, its origin, its cessation, and the path. They take pleasure in choices that lead to rebirth, old age, and death, to sorrow, lamentation, pain, sadness, and distress. Since they take pleasure in such choices, they continue to make them. Having made choices that lead to rebirth, old age, and death, to sorrow, lamentation, pain, sadness, and distress, they fall down the cliff of rebirth, old age, and death, of sorrow, lamentation, pain, sadness, and distress. They’re not freed from rebirth, old age, and death, from sorrow, lamentation, pain, sadness, and distress. They’re not freed from suffering, I say. 

There\marginnote{3.1} are ascetics and brahmins who truly understand about suffering, its origin, its cessation, and the path. They don’t take pleasure in choices that lead to rebirth, old age, and death, to sorrow, lamentation, pain, sadness, and distress. Since they don’t take pleasure in such choices, they stop making them. Having stopped making choices that lead to rebirth, old age, and death, to sorrow, lamentation, pain, sadness, and distress, they don’t fall down the cliff of rebirth, old age, and death, of sorrow, lamentation, pain, sadness, and distress. They’re freed from rebirth, old age, and death, from sorrow, lamentation, pain, sadness, and distress. They’re freed from suffering, I say. 

That’s\marginnote{4.1} why you should practice meditation …” 

%
\section*{{\suttatitleacronym SN 56.43}{\suttatitletranslation The Mighty Fever }{\suttatitleroot Mahāpariḷāhasutta}}
\addcontentsline{toc}{section}{\tocacronym{SN 56.43} \toctranslation{The Mighty Fever } \tocroot{Mahāpariḷāhasutta}}
\markboth{The Mighty Fever }{Mahāpariḷāhasutta}
\extramarks{SN 56.43}{SN 56.43}

“Mendicants,\marginnote{1.1} there is a hell called ‘The Mighty Fever’. There, whatever sight you see with your eye is unlikable, not likable; undesirable, not desirable; unpleasant, not pleasant. Whatever sound you hear … Whatever odor you smell … Whatever flavor you taste … Whatever touch you feel … Whatever thought you know with your mind is unlikable, not likable; undesirable, not desirable; unpleasant, not pleasant.” 

When\marginnote{2.1} he said this, one of the mendicants said to the Buddha, “Sir, that fever really is mighty, so very mighty. Is there any other fever more mighty and terrifying than this one?” 

“There\marginnote{2.4} is, mendicant.” 

“But\marginnote{3.1} sir, what is it?” 

“Mendicants,\marginnote{3.2} there are ascetics and brahmins who don’t truly understand about suffering, its origin, its cessation, and the path. They take pleasure in choices that lead to rebirth … They continue to make such choices … Having made such choices, they burn with the fever of rebirth, old age, and death, of sorrow, lamentation, pain, sadness, and distress. They’re not freed from rebirth, old age, and death, from sorrow, lamentation, pain, sadness, and distress. They’re not freed from suffering, I say. 

There\marginnote{4.1} are ascetics and brahmins who truly understand about suffering, its origin, its cessation, and the path. They don’t take pleasure in choices that lead to rebirth … They stop making such choices … Having stopped making such choices, they don’t burn with the fever of rebirth, old age, and death, of sorrow, lamentation, pain, sadness, and distress. They’re freed from rebirth, old age, and death, from sorrow, lamentation, pain, sadness, and distress. They’re freed from suffering, I say. 

That’s\marginnote{5.1} why you should practice meditation …” 

%
\section*{{\suttatitleacronym SN 56.44}{\suttatitletranslation A Bungalow }{\suttatitleroot Kūṭāgārasutta}}
\addcontentsline{toc}{section}{\tocacronym{SN 56.44} \toctranslation{A Bungalow } \tocroot{Kūṭāgārasutta}}
\markboth{A Bungalow }{Kūṭāgārasutta}
\extramarks{SN 56.44}{SN 56.44}

“Mendicants,\marginnote{1.1} suppose someone were to say: ‘Without truly comprehending the noble truths of suffering, its origin, its cessation, and the path, I will completely make an end of suffering.’ That is not possible. 

It’s\marginnote{2.1} as if someone were to say: ‘Before the lower story of a bungalow is built, I will climb up to the upper story.’ That is not possible. In the same way, suppose someone were to say: ‘Without truly comprehending the noble truths of suffering, its origin, its cessation, and the path, I will completely make an end of suffering.’ That is not possible. 

But\marginnote{3.1} suppose someone were to say: ‘After truly comprehending the noble truths of suffering, its origin, its cessation, and the path, I will completely make an end of suffering.’ That is possible. 

It’s\marginnote{4.1} as if someone were to say: ‘After the lower story of a bungalow is built, I will climb up to the upper story.’ That is possible. In the same way, suppose someone were to say: ‘After truly comprehending the noble truths of suffering, its origin, its cessation, and the path, I will completely make an end of suffering.’ That is possible. 

That’s\marginnote{5.1} why you should practice meditation …” 

%
\section*{{\suttatitleacronym SN 56.45}{\suttatitletranslation Splitting Hairs }{\suttatitleroot Vālasutta}}
\addcontentsline{toc}{section}{\tocacronym{SN 56.45} \toctranslation{Splitting Hairs } \tocroot{Vālasutta}}
\markboth{Splitting Hairs }{Vālasutta}
\extramarks{SN 56.45}{SN 56.45}

At\marginnote{1.1} one time the Buddha was staying near \textsanskrit{Vesālī}, at the Great Wood, in the hall with the peaked roof. 

Then\marginnote{1.2} Venerable Ānanda robed up in the morning and, taking his bowl and robe, entered \textsanskrit{Vesālī} for alms. He saw several Licchavi youths practicing archery near the town hall. They were shooting arrows from a distance through a small keyhole, shot after shot without missing. 

When\marginnote{1.4} he saw this he thought, “These Licchavi youths really are trained, so well trained, in that they shoot arrows from a distance through a small keyhole, shot after shot without missing.” 

Then\marginnote{2.1} Ānanda wandered for alms in \textsanskrit{Vesālī}. After the meal, on his return from almsround, he went to the Buddha, bowed, sat down to one side, and told him what had happened. 

“What\marginnote{3.1} do you think, Ānanda? Which is harder and more challenging: to shoot arrows from a distance through a small keyhole, shot after shot without missing? Or to take a horsehair split into seven strands and penetrate one tip with another tip?” 

“It’s\marginnote{3.4} more difficult and challenging, sir, to take a horsehair split into seven strands and penetrate one tip with another tip.” 

“Still,\marginnote{3.5} Ānanda, those who truly penetrate suffering, its origin, its cessation, and the path penetrate something tougher than that. 

That’s\marginnote{4.1} why you should practice meditation …” 

%
\section*{{\suttatitleacronym SN 56.46}{\suttatitletranslation Darkness }{\suttatitleroot Andhakārasutta}}
\addcontentsline{toc}{section}{\tocacronym{SN 56.46} \toctranslation{Darkness } \tocroot{Andhakārasutta}}
\markboth{Darkness }{Andhakārasutta}
\extramarks{SN 56.46}{SN 56.46}

“Mendicants,\marginnote{1.1} the boundless desolation of interstellar space is so utterly dark that even the light of the moon and the sun, so mighty and powerful, makes no impression.” 

When\marginnote{2.1} he said this, one of the mendicants asked the Buddha, “Sir, that darkness really is mighty, so very mighty. Is there any other darkness more mighty and terrifying than this one?” 

“There\marginnote{2.4} is, mendicant.” 

“But\marginnote{3.1} sir, what is it?” 

“There\marginnote{3.2} are ascetics and brahmins who don’t truly understand about suffering, its origin, its cessation, and the path. They take pleasure in choices that lead to rebirth … They continue to make such choices … Having made such choices, they fall into the darkness of rebirth, old age, and death, of sorrow, lamentation, pain, sadness, and distress. They’re not freed from rebirth, old age, and death, from sorrow, lamentation, pain, sadness, and distress. They’re not freed from suffering, I say. 

There\marginnote{4.1} are ascetics and brahmins who truly understand about suffering, its origin, its cessation, and the path. They don’t take pleasure in choices that lead to rebirth … They stop making such choices … Having stopped making such choices, they don’t fall into the darkness of rebirth, old age, and death, of sorrow, lamentation, pain, sadness, and distress. They’re freed from rebirth, old age, and death, from sorrow, lamentation, pain, sadness, and distress. They’re freed from suffering, I say. 

That’s\marginnote{5.1} why you should practice meditation …” 

%
\section*{{\suttatitleacronym SN 56.47}{\suttatitletranslation A Yoke With a Hole (1st) }{\suttatitleroot Paṭhamachiggaḷayugasutta}}
\addcontentsline{toc}{section}{\tocacronym{SN 56.47} \toctranslation{A Yoke With a Hole (1st) } \tocroot{Paṭhamachiggaḷayugasutta}}
\markboth{A Yoke With a Hole (1st) }{Paṭhamachiggaḷayugasutta}
\extramarks{SN 56.47}{SN 56.47}

“Mendicants,\marginnote{1.1} suppose a person was to throw a yoke with a single hole into the ocean. And there was a one-eyed turtle who popped up once every hundred years. 

What\marginnote{1.2} do you think, mendicants? Would that one-eyed turtle, popping up once every hundred years, still poke its neck through the hole in that yoke?” 

“Only\marginnote{1.4} after a very long time, sir, if ever.” 

“That\marginnote{2.1} one-eyed turtle would poke its neck through the hole in that yoke sooner than a fool who has fallen to the underworld would be reborn as a human being, I say. 

Why\marginnote{3.1} is that? Because in that place there’s no principled or moral conduct, and no doing what is good and skillful. There they just prey on each other, preying on the weak. Why is that? It’s because they haven’t seen the four noble truths. What four? The noble truths of suffering, its origin, its cessation, and the path. 

That’s\marginnote{4.1} why you should practice meditation …” 

%
\section*{{\suttatitleacronym SN 56.48}{\suttatitletranslation A Yoke With a Hole (2nd) }{\suttatitleroot Dutiyachiggaḷayugasutta}}
\addcontentsline{toc}{section}{\tocacronym{SN 56.48} \toctranslation{A Yoke With a Hole (2nd) } \tocroot{Dutiyachiggaḷayugasutta}}
\markboth{A Yoke With a Hole (2nd) }{Dutiyachiggaḷayugasutta}
\extramarks{SN 56.48}{SN 56.48}

“Mendicants,\marginnote{1.1} suppose the earth was entirely covered with water. And a person threw a yoke with a single hole into it. The east wind wafts it west; the west wind wafts it east; the north wind wafts it south; and the south wind wafts it north. And there was a one-eyed turtle who popped up once every hundred years. 

What\marginnote{1.5} do you think, mendicants? Would that one-eyed turtle, popping up once every hundred years, still poke its neck through the hole in that yoke?” 

“It’s\marginnote{1.7} unlikely, sir.” 

“That’s\marginnote{2.1} how unlikely it is to get reborn as a human being. And that’s how unlikely it is for a Realized One to arise in the world, a perfected one, a fully awakened Buddha. And that’s how unlikely it is for the teaching and training proclaimed by a Realized One to shine in the world. And now, mendicants, you have been reborn as a human being. A Realized One has arisen in the world, a perfected one, a fully awakened Buddha. And the teaching and training proclaimed by a Realized One shines in the world. 

That’s\marginnote{3.1} why you should practice meditation …” 

%
\section*{{\suttatitleacronym SN 56.49}{\suttatitletranslation Sineru, King of Mountains (1st) }{\suttatitleroot Paṭhamasinerupabbatarājasutta}}
\addcontentsline{toc}{section}{\tocacronym{SN 56.49} \toctranslation{Sineru, King of Mountains (1st) } \tocroot{Paṭhamasinerupabbatarājasutta}}
\markboth{Sineru, King of Mountains (1st) }{Paṭhamasinerupabbatarājasutta}
\extramarks{SN 56.49}{SN 56.49}

“Mendicants,\marginnote{1.1} suppose a person was to place down on Sineru, the king of mountains, seven pebbles the size of mung beans. 

What\marginnote{1.2} do you think, mendicants? Which is more: the seven pebbles the size of mung beans? Or Sineru, the king of mountains?” 

“Sir,\marginnote{1.4} Sineru, the king of mountains, is certainly more. The seven pebbles the size of mung beans are tiny. Compared to Sineru, they don’t count, there’s no comparison, they’re not worth a fraction.” 

“In\marginnote{1.7} the same way, for a person with comprehension, a noble disciple accomplished in view, the suffering that’s over and done with is more, what’s left is tiny. Compared to the mass of suffering in the past that’s over and done with, it doesn’t count, there’s no comparison, it’s not worth a fraction, since there are at most seven more lives. Such a person truly understands about suffering, its origin, its cessation, and the path. 

That’s\marginnote{2.1} why you should practice meditation …” 

%
\section*{{\suttatitleacronym SN 56.50}{\suttatitletranslation Sineru, King of Mountains (2nd) }{\suttatitleroot Dutiyasinerupabbatarājasutta}}
\addcontentsline{toc}{section}{\tocacronym{SN 56.50} \toctranslation{Sineru, King of Mountains (2nd) } \tocroot{Dutiyasinerupabbatarājasutta}}
\markboth{Sineru, King of Mountains (2nd) }{Dutiyasinerupabbatarājasutta}
\extramarks{SN 56.50}{SN 56.50}

“Mendicants,\marginnote{1.1} suppose Sineru, the king of mountains, was worn away and eroded except for seven pebbles the size of mustard seeds. 

What\marginnote{1.2} do you think, mendicants? Which is more: the portion of Sineru, the king of mountains, that has been worn away and eroded? Or the seven pebbles the size of mustard seeds that are left?” 

“Sir,\marginnote{1.4} the portion of Sineru, the king of mountains, that has been worn away and eroded is certainly more. The seven pebbles the size of mustard seeds are tiny. Compared to Sineru, they don’t count, there’s no comparison, they’re not worth a fraction.” 

“In\marginnote{1.7} the same way, for a person with comprehension, a noble disciple accomplished in view, the suffering that’s over and done with is more, what’s left is tiny. Compared to the mass of suffering in the past that’s over and done with, it doesn’t count, there’s no comparison, it’s not worth a fraction, since there are at most seven more lives. Such a person truly understands about suffering, its origin, its cessation, and the path. 

That’s\marginnote{2.1} why you should practice meditation …” 

%
\addtocontents{toc}{\let\protect\contentsline\protect\nopagecontentsline}
\chapter*{The Chapter on Comprehension }
\addcontentsline{toc}{chapter}{\tocchapterline{The Chapter on Comprehension }}
\addtocontents{toc}{\let\protect\contentsline\protect\oldcontentsline}

%
\section*{{\suttatitleacronym SN 56.51}{\suttatitletranslation A Fingernail }{\suttatitleroot Nakhasikhāsutta}}
\addcontentsline{toc}{section}{\tocacronym{SN 56.51} \toctranslation{A Fingernail } \tocroot{Nakhasikhāsutta}}
\markboth{A Fingernail }{Nakhasikhāsutta}
\extramarks{SN 56.51}{SN 56.51}

Then\marginnote{1.1} the Buddha, picking up a little bit of dirt under his fingernail, addressed the mendicants: “What do you think, mendicants? Which is more: the little bit of dirt under my fingernail, or this great earth?” 

“Sir,\marginnote{1.4} the great earth is certainly more. The little bit of dirt under your fingernail is tiny. Compared to the great earth, it doesn’t count, there’s no comparison, it’s not worth a fraction.” 

“In\marginnote{1.6} the same way, for a person with comprehension, a noble disciple accomplished in view, the suffering that’s over and done with is more, what’s left is tiny. Compared to the mass of suffering in the past that’s over and done with, it doesn’t count, there’s no comparison, it’s not worth a fraction, since there are at most seven more lives. Such a person truly understands about suffering, its origin, its cessation, and the path. 

That’s\marginnote{2.1} why you should practice meditation …” 

%
\section*{{\suttatitleacronym SN 56.52}{\suttatitletranslation A Lotus Pond }{\suttatitleroot Pokkharaṇīsutta}}
\addcontentsline{toc}{section}{\tocacronym{SN 56.52} \toctranslation{A Lotus Pond } \tocroot{Pokkharaṇīsutta}}
\markboth{A Lotus Pond }{Pokkharaṇīsutta}
\extramarks{SN 56.52}{SN 56.52}

“Mendicants,\marginnote{1.1} suppose there was a lotus pond that was fifty leagues long, fifty leagues wide, and fifty leagues deep, full to the brim so a crow could drink from it. Then a person would pick up some water on the tip of a blade of grass. 

What\marginnote{1.3} do you think, mendicants? Which is more: the water on the tip of the blade of grass, or the water in the lotus pond?” 

“Sir,\marginnote{1.5} the water in the lotus pond is certainly more. The water on the tip of a blade of grass is tiny. Compared to the water in the lotus pond, it doesn’t count, there’s no comparison, it’s not worth a fraction.” 

“In\marginnote{1.7} the same way, for a noble disciple … 

That’s\marginnote{1.8} why you should practice meditation …” 

%
\section*{{\suttatitleacronym SN 56.53}{\suttatitletranslation Where the Waters Flow Together (1st) }{\suttatitleroot Paṭhamasambhejjasutta}}
\addcontentsline{toc}{section}{\tocacronym{SN 56.53} \toctranslation{Where the Waters Flow Together (1st) } \tocroot{Paṭhamasambhejjasutta}}
\markboth{Where the Waters Flow Together (1st) }{Paṭhamasambhejjasutta}
\extramarks{SN 56.53}{SN 56.53}

“Mendicants,\marginnote{1.1} there are places where the great rivers—the Ganges, Yamuna, \textsanskrit{Aciravatī}, \textsanskrit{Sarabhū}, and \textsanskrit{Mahī}—come together and converge. Suppose a person was to draw two or three drops of water from such a place. 

What\marginnote{1.3} do you think, mendicants? Which is more: the two or three drops drawn out or the water in the confluence?” 

“Sir,\marginnote{1.5} the water in the confluence is certainly more. The two or three drops drawn out are tiny. Compared to the water in the confluence, it doesn’t count, there’s no comparison, it’s not worth a fraction.” 

“In\marginnote{1.7} the same way, for a noble disciple … 

That’s\marginnote{1.8} why you should practice meditation …” 

%
\section*{{\suttatitleacronym SN 56.54}{\suttatitletranslation Where the Waters Flow Together (2nd) }{\suttatitleroot Dutiyasambhejjasutta}}
\addcontentsline{toc}{section}{\tocacronym{SN 56.54} \toctranslation{Where the Waters Flow Together (2nd) } \tocroot{Dutiyasambhejjasutta}}
\markboth{Where the Waters Flow Together (2nd) }{Dutiyasambhejjasutta}
\extramarks{SN 56.54}{SN 56.54}

“Mendicants,\marginnote{1.1} there are places where the great rivers—the Ganges, Yamuna, \textsanskrit{Aciravatī}, \textsanskrit{Sarabhū}, and \textsanskrit{Mahī}—come together and converge. Suppose that water dried up and evaporated except for two or three drops. 

What\marginnote{1.3} do you think, mendicants? Which is more: the water in the confluence that has dried up and evaporated, or the two or three drops left?” 

“Sir,\marginnote{1.5} the water in the confluence that has dried up and evaporated is certainly more. The two or three drops left are tiny. Compared to the water in the confluence that has dried up and evaporated, it doesn’t count, there’s no comparison, it’s not worth a fraction.” 

“In\marginnote{1.7} the same way, for a noble disciple … 

That’s\marginnote{1.8} why you should practice meditation …” 

%
\section*{{\suttatitleacronym SN 56.55}{\suttatitletranslation The Earth (1st) }{\suttatitleroot Paṭhamamahāpathavīsutta}}
\addcontentsline{toc}{section}{\tocacronym{SN 56.55} \toctranslation{The Earth (1st) } \tocroot{Paṭhamamahāpathavīsutta}}
\markboth{The Earth (1st) }{Paṭhamamahāpathavīsutta}
\extramarks{SN 56.55}{SN 56.55}

“Mendicants,\marginnote{1.1} suppose a person was to place seven clay balls the size of jujube seeds on the great earth. 

What\marginnote{1.2} do you think, mendicants? Which is more: the seven clay balls the size of jujube seeds, or the great earth?” 

“Sir,\marginnote{1.4} the great earth is certainly more. The seven clay balls the size of jujube seeds are tiny. Compared to the great earth, they don’t count, there’s no comparison, they’re not worth a fraction.” 

“In\marginnote{1.6} the same way, for a noble disciple … 

That’s\marginnote{1.7} why you should practice meditation …” 

%
\section*{{\suttatitleacronym SN 56.56}{\suttatitletranslation The Earth (2nd) }{\suttatitleroot Dutiyamahāpathavīsutta}}
\addcontentsline{toc}{section}{\tocacronym{SN 56.56} \toctranslation{The Earth (2nd) } \tocroot{Dutiyamahāpathavīsutta}}
\markboth{The Earth (2nd) }{Dutiyamahāpathavīsutta}
\extramarks{SN 56.56}{SN 56.56}

“Mendicants,\marginnote{1.1} suppose the great earth was worn away and eroded except for seven clay balls the size of jujube seeds. 

What\marginnote{1.2} do you think, mendicants? Which is more: the great earth that has been worn away and eroded, or the seven clay balls the size of jujube seeds that are left?” 

“Sir,\marginnote{1.4} the great earth that has been worn away and eroded is certainly more. The seven clay balls the size of jujube seeds are tiny. Compared to the great earth that has been worn away and eroded, they don’t count, there’s no comparison, they’re not worth a fraction.” 

“In\marginnote{1.6} the same way, for a noble disciple … 

That’s\marginnote{1.7} why you should practice meditation …” 

%
\section*{{\suttatitleacronym SN 56.57}{\suttatitletranslation The Ocean (1st) }{\suttatitleroot Paṭhamamahāsamuddasutta}}
\addcontentsline{toc}{section}{\tocacronym{SN 56.57} \toctranslation{The Ocean (1st) } \tocroot{Paṭhamamahāsamuddasutta}}
\markboth{The Ocean (1st) }{Paṭhamamahāsamuddasutta}
\extramarks{SN 56.57}{SN 56.57}

“Mendicants,\marginnote{1.1} suppose a man was to draw up two or three drops of water from the ocean. 

What\marginnote{1.2} do you think, mendicants? Which is more: the two or three drops drawn out or the water in the ocean?” 

“Sir,\marginnote{1.4} the water in the ocean is certainly more. The two or three drops drawn out are tiny. Compared to the water in the ocean, it doesn’t count, there’s no comparison, it’s not worth a fraction.” 

“In\marginnote{1.6} the same way, for a noble disciple … 

That’s\marginnote{1.7} why you should practice meditation …” 

%
\section*{{\suttatitleacronym SN 56.58}{\suttatitletranslation The Ocean (2nd) }{\suttatitleroot Dutiyamahāsamuddasutta}}
\addcontentsline{toc}{section}{\tocacronym{SN 56.58} \toctranslation{The Ocean (2nd) } \tocroot{Dutiyamahāsamuddasutta}}
\markboth{The Ocean (2nd) }{Dutiyamahāsamuddasutta}
\extramarks{SN 56.58}{SN 56.58}

“Mendicants,\marginnote{1.1} suppose the water in the ocean dried up and evaporated except for two or three drops. 

What\marginnote{1.2} do you think, mendicants? Which is more: the water in the ocean that has dried up and evaporated, or the two or three drops left?” 

“Sir,\marginnote{1.4} the water in the ocean that has dried up and evaporated is certainly more. The two or three drops left are tiny. Compared to the water in the ocean that has dried up and evaporated, it doesn’t count, there’s no comparison, it’s not worth a fraction.” 

“In\marginnote{1.6} the same way, for a noble disciple … 

That’s\marginnote{1.7} why you should practice meditation …” 

%
\section*{{\suttatitleacronym SN 56.59}{\suttatitletranslation A Mountain (1st) }{\suttatitleroot Paṭhamapabbatūpamasutta}}
\addcontentsline{toc}{section}{\tocacronym{SN 56.59} \toctranslation{A Mountain (1st) } \tocroot{Paṭhamapabbatūpamasutta}}
\markboth{A Mountain (1st) }{Paṭhamapabbatūpamasutta}
\extramarks{SN 56.59}{SN 56.59}

“Mendicants,\marginnote{1.1} suppose a person was to place seven pebbles the size of mustard seeds on the Himalayas, the king of mountains. 

What\marginnote{1.2} do you think, mendicants? Which is more: the seven pebbles the size of mustard seeds, or the Himalayas, the king of mountains?” 

“Sir,\marginnote{1.4} the Himalayas, the king of mountains, is certainly more. The seven pebbles the size of mustard seeds are tiny. Compared to the Himalayas, they don’t count, there’s no comparison, they’re not worth a fraction.” 

“In\marginnote{1.6} the same way, for a noble disciple … 

That’s\marginnote{1.7} why you should practice meditation …” 

%
\section*{{\suttatitleacronym SN 56.60}{\suttatitletranslation A Mountain (2nd) }{\suttatitleroot Dutiyapabbatūpamasutta}}
\addcontentsline{toc}{section}{\tocacronym{SN 56.60} \toctranslation{A Mountain (2nd) } \tocroot{Dutiyapabbatūpamasutta}}
\markboth{A Mountain (2nd) }{Dutiyapabbatūpamasutta}
\extramarks{SN 56.60}{SN 56.60}

“Mendicants,\marginnote{1.1} suppose the Himalayas, the king of mountains, was worn away and eroded except for seven pebbles the size of mustard seeds. 

What\marginnote{1.2} do you think, mendicants? Which is more: the portion of the Himalayas, the king of mountains, that has been worn away and eroded, or the seven pebbles the size of mustard seeds that are left?” 

“Sir,\marginnote{1.4} the portion of the Himalayas, the king of mountains, that has been worn away and eroded is certainly more. The seven pebbles the size of mustard seeds are tiny. Compared to the Himalayas, they don’t count, there’s no comparison, they’re not worth a fraction.” 

“In\marginnote{1.6} the same way, for a person with comprehension, a noble disciple accomplished in view, the suffering that’s over and done with is more, what’s left is tiny. Compared to the mass of suffering in the past that’s over and done with, it doesn’t count, there’s no comparison, it’s not worth a fraction, since there are at most seven more lives. Such a person truly understands about suffering, its origin, its cessation, and the path. 

That’s\marginnote{2.1} why you should practice meditation …” 

%
\addtocontents{toc}{\let\protect\contentsline\protect\nopagecontentsline}
\chapter*{The First Chapter of Abbreviated Texts on Raw Grain }
\addcontentsline{toc}{chapter}{\tocchapterline{The First Chapter of Abbreviated Texts on Raw Grain }}
\addtocontents{toc}{\let\protect\contentsline\protect\oldcontentsline}

%
\section*{{\suttatitleacronym SN 56.61}{\suttatitletranslation Not Human }{\suttatitleroot Aññatrasutta}}
\addcontentsline{toc}{section}{\tocacronym{SN 56.61} \toctranslation{Not Human } \tocroot{Aññatrasutta}}
\markboth{Not Human }{Aññatrasutta}
\extramarks{SN 56.61}{SN 56.61}

Then\marginnote{1.1} the Buddha, picking up a little bit of dirt under his fingernail, addressed the mendicants: “What do you think, mendicants? Which is more: the little bit of dirt under my fingernail, or this great earth?” 

“Sir,\marginnote{1.4} the great earth is certainly more. The little bit of dirt under your fingernail is tiny. Compared to the great earth, it doesn’t count, there’s no comparison, it’s not worth a fraction.” 

“In\marginnote{2.1} the same way, the sentient beings reborn as humans are few, while those not reborn as humans are many. Why is that? It’s because they haven’t seen the four noble truths. What four? The noble truths of suffering, its origin, its cessation, and the path. 

That’s\marginnote{3.1} why you should practice meditation …” 

%
\section*{{\suttatitleacronym SN 56.62}{\suttatitletranslation In the Borderlands }{\suttatitleroot Paccantasutta}}
\addcontentsline{toc}{section}{\tocacronym{SN 56.62} \toctranslation{In the Borderlands } \tocroot{Paccantasutta}}
\markboth{In the Borderlands }{Paccantasutta}
\extramarks{SN 56.62}{SN 56.62}

Then\marginnote{1.1} the Buddha, picking up a little bit of dirt under his fingernail, addressed the mendicants: “What do you think, mendicants? Which is more: the little bit of dirt under my fingernail, or this great earth?” 

“Sir,\marginnote{1.4} the great earth is certainly more. The little bit of dirt under your fingernail is tiny. Compared to the great earth, it doesn’t count, there’s no comparison, it’s not worth a fraction.” 

“In\marginnote{2.1} the same way, the sentient beings reborn in central countries are few, while those reborn in the borderlands, among strange barbarian tribes, are many. …” 

%
\section*{{\suttatitleacronym SN 56.63}{\suttatitletranslation Wisdom }{\suttatitleroot Paññāsutta}}
\addcontentsline{toc}{section}{\tocacronym{SN 56.63} \toctranslation{Wisdom } \tocroot{Paññāsutta}}
\markboth{Wisdom }{Paññāsutta}
\extramarks{SN 56.63}{SN 56.63}

“…\marginnote{1.1} the sentient beings who have the noble eye of wisdom are few, while those who are ignorant and confused are many. …” 

%
\section*{{\suttatitleacronym SN 56.64}{\suttatitletranslation Alcohol and Drugs }{\suttatitleroot Surāmerayasutta}}
\addcontentsline{toc}{section}{\tocacronym{SN 56.64} \toctranslation{Alcohol and Drugs } \tocroot{Surāmerayasutta}}
\markboth{Alcohol and Drugs }{Surāmerayasutta}
\extramarks{SN 56.64}{SN 56.64}

“…\marginnote{1.1} the sentient beings who refrain from alcoholic drinks that cause negligence, are few, while those who don’t refrain are many. …” 

%
\section*{{\suttatitleacronym SN 56.65}{\suttatitletranslation Born in Water }{\suttatitleroot Odakasutta}}
\addcontentsline{toc}{section}{\tocacronym{SN 56.65} \toctranslation{Born in Water } \tocroot{Odakasutta}}
\markboth{Born in Water }{Odakasutta}
\extramarks{SN 56.65}{SN 56.65}

“…\marginnote{1.1} the sentient beings born on land are few, while those born in water are many. …” 

%
\section*{{\suttatitleacronym SN 56.66}{\suttatitletranslation Respect Mother }{\suttatitleroot Matteyyasutta}}
\addcontentsline{toc}{section}{\tocacronym{SN 56.66} \toctranslation{Respect Mother } \tocroot{Matteyyasutta}}
\markboth{Respect Mother }{Matteyyasutta}
\extramarks{SN 56.66}{SN 56.66}

“…\marginnote{1.1} the sentient beings who respect their mothers are few, while those who don’t are many. …” 

%
\section*{{\suttatitleacronym SN 56.67}{\suttatitletranslation Respect Father }{\suttatitleroot Petteyyasutta}}
\addcontentsline{toc}{section}{\tocacronym{SN 56.67} \toctranslation{Respect Father } \tocroot{Petteyyasutta}}
\markboth{Respect Father }{Petteyyasutta}
\extramarks{SN 56.67}{SN 56.67}

“…\marginnote{1.1} the sentient beings who respect their fathers are few, while those who don’t are many. …” 

%
\section*{{\suttatitleacronym SN 56.68}{\suttatitletranslation Respect Ascetics }{\suttatitleroot Sāmaññasutta}}
\addcontentsline{toc}{section}{\tocacronym{SN 56.68} \toctranslation{Respect Ascetics } \tocroot{Sāmaññasutta}}
\markboth{Respect Ascetics }{Sāmaññasutta}
\extramarks{SN 56.68}{SN 56.68}

“…\marginnote{1.1} the sentient beings who respect ascetics are few, while those who don’t are many. …” 

%
\section*{{\suttatitleacronym SN 56.69}{\suttatitletranslation Respect Brahmins }{\suttatitleroot Brahmaññasutta}}
\addcontentsline{toc}{section}{\tocacronym{SN 56.69} \toctranslation{Respect Brahmins } \tocroot{Brahmaññasutta}}
\markboth{Respect Brahmins }{Brahmaññasutta}
\extramarks{SN 56.69}{SN 56.69}

“…\marginnote{1.1} the sentient beings who respect brahmins are few, while those who don’t are many. …” 

%
\section*{{\suttatitleacronym SN 56.70}{\suttatitletranslation Honor the Elders }{\suttatitleroot Pacāyikasutta}}
\addcontentsline{toc}{section}{\tocacronym{SN 56.70} \toctranslation{Honor the Elders } \tocroot{Pacāyikasutta}}
\markboth{Honor the Elders }{Pacāyikasutta}
\extramarks{SN 56.70}{SN 56.70}

“…\marginnote{1.1} the sentient beings who honor the elders in the family are few, while those who don’t are many. …” 

%
\addtocontents{toc}{\let\protect\contentsline\protect\nopagecontentsline}
\chapter*{The Second Chapter of Abbreviated Texts on Raw Grain }
\addcontentsline{toc}{chapter}{\tocchapterline{The Second Chapter of Abbreviated Texts on Raw Grain }}
\addtocontents{toc}{\let\protect\contentsline\protect\oldcontentsline}

%
\section*{{\suttatitleacronym SN 56.71}{\suttatitletranslation Killing Living Creatures }{\suttatitleroot Pāṇātipātasutta}}
\addcontentsline{toc}{section}{\tocacronym{SN 56.71} \toctranslation{Killing Living Creatures } \tocroot{Pāṇātipātasutta}}
\markboth{Killing Living Creatures }{Pāṇātipātasutta}
\extramarks{SN 56.71}{SN 56.71}

“…\marginnote{1.1} the sentient beings who refrain from killing living creatures are few, while those who don’t refrain are many. …” 

%
\section*{{\suttatitleacronym SN 56.72}{\suttatitletranslation Stealing }{\suttatitleroot Adinnādānasutta}}
\addcontentsline{toc}{section}{\tocacronym{SN 56.72} \toctranslation{Stealing } \tocroot{Adinnādānasutta}}
\markboth{Stealing }{Adinnādānasutta}
\extramarks{SN 56.72}{SN 56.72}

“…\marginnote{1.1} the sentient beings who refrain from stealing are few, while those who don’t refrain are many. …” 

%
\section*{{\suttatitleacronym SN 56.73}{\suttatitletranslation Sexual Misconduct }{\suttatitleroot Kāmesumicchācārasutta}}
\addcontentsline{toc}{section}{\tocacronym{SN 56.73} \toctranslation{Sexual Misconduct } \tocroot{Kāmesumicchācārasutta}}
\markboth{Sexual Misconduct }{Kāmesumicchācārasutta}
\extramarks{SN 56.73}{SN 56.73}

“…\marginnote{1.1} the sentient beings who refrain from sexual misconduct are few, while those who don’t refrain are many. …” 

%
\section*{{\suttatitleacronym SN 56.74}{\suttatitletranslation Lying }{\suttatitleroot Musāvādasutta}}
\addcontentsline{toc}{section}{\tocacronym{SN 56.74} \toctranslation{Lying } \tocroot{Musāvādasutta}}
\markboth{Lying }{Musāvādasutta}
\extramarks{SN 56.74}{SN 56.74}

“…\marginnote{1.1} the sentient beings who refrain from lying are few, while those who don’t refrain are many. …” 

%
\section*{{\suttatitleacronym SN 56.75}{\suttatitletranslation Divisive Speech }{\suttatitleroot Pesuññasutta}}
\addcontentsline{toc}{section}{\tocacronym{SN 56.75} \toctranslation{Divisive Speech } \tocroot{Pesuññasutta}}
\markboth{Divisive Speech }{Pesuññasutta}
\extramarks{SN 56.75}{SN 56.75}

“…\marginnote{1.1} the sentient beings who refrain from divisive speech are few, while those who don’t refrain are many. …” 

%
\section*{{\suttatitleacronym SN 56.76}{\suttatitletranslation Harsh Speech }{\suttatitleroot Pharusavācāsutta}}
\addcontentsline{toc}{section}{\tocacronym{SN 56.76} \toctranslation{Harsh Speech } \tocroot{Pharusavācāsutta}}
\markboth{Harsh Speech }{Pharusavācāsutta}
\extramarks{SN 56.76}{SN 56.76}

“…\marginnote{1.1} the sentient beings who refrain from harsh speech are few, while those who don’t refrain are many. …” 

%
\section*{{\suttatitleacronym SN 56.77}{\suttatitletranslation Nonsense }{\suttatitleroot Samphappalāpasutta}}
\addcontentsline{toc}{section}{\tocacronym{SN 56.77} \toctranslation{Nonsense } \tocroot{Samphappalāpasutta}}
\markboth{Nonsense }{Samphappalāpasutta}
\extramarks{SN 56.77}{SN 56.77}

“…\marginnote{1.1} the sentient beings who refrain from talking nonsense are few, while those who don’t refrain are many. …” 

%
\section*{{\suttatitleacronym SN 56.78}{\suttatitletranslation Plants }{\suttatitleroot Bījagāmasutta}}
\addcontentsline{toc}{section}{\tocacronym{SN 56.78} \toctranslation{Plants } \tocroot{Bījagāmasutta}}
\markboth{Plants }{Bījagāmasutta}
\extramarks{SN 56.78}{SN 56.78}

“…\marginnote{1.1} the sentient beings who refrain from injuring plants and seeds are few, while those who don’t refrain are many. …” 

%
\section*{{\suttatitleacronym SN 56.79}{\suttatitletranslation Food at the Wrong Time }{\suttatitleroot Vikālabhojanasutta}}
\addcontentsline{toc}{section}{\tocacronym{SN 56.79} \toctranslation{Food at the Wrong Time } \tocroot{Vikālabhojanasutta}}
\markboth{Food at the Wrong Time }{Vikālabhojanasutta}
\extramarks{SN 56.79}{SN 56.79}

“…\marginnote{1.1} the sentient beings who refrain from food at the wrong time are few, while those who don’t refrain are many. …” 

%
\section*{{\suttatitleacronym SN 56.80}{\suttatitletranslation Perfumes and Makeup }{\suttatitleroot Gandhavilepanasutta}}
\addcontentsline{toc}{section}{\tocacronym{SN 56.80} \toctranslation{Perfumes and Makeup } \tocroot{Gandhavilepanasutta}}
\markboth{Perfumes and Makeup }{Gandhavilepanasutta}
\extramarks{SN 56.80}{SN 56.80}

“…\marginnote{1.1} the sentient beings who refrain from beautifying and adorning themselves with garlands, perfumes, and makeup are few, while those who don’t refrain are many …” 

%
\addtocontents{toc}{\let\protect\contentsline\protect\nopagecontentsline}
\chapter*{The Third Chapter of Abbreviated Texts on Raw Grain }
\addcontentsline{toc}{chapter}{\tocchapterline{The Third Chapter of Abbreviated Texts on Raw Grain }}
\addtocontents{toc}{\let\protect\contentsline\protect\oldcontentsline}

%
\section*{{\suttatitleacronym SN 56.81}{\suttatitletranslation Dancing and Singing }{\suttatitleroot Naccagītasutta}}
\addcontentsline{toc}{section}{\tocacronym{SN 56.81} \toctranslation{Dancing and Singing } \tocroot{Naccagītasutta}}
\markboth{Dancing and Singing }{Naccagītasutta}
\extramarks{SN 56.81}{SN 56.81}

….\marginnote{1.1} “… the sentient beings who refrain from dancing, singing, music, and seeing shows are few, while those who don’t refrain are many …” 

%
\section*{{\suttatitleacronym SN 56.82}{\suttatitletranslation High Beds }{\suttatitleroot Uccāsayanasutta}}
\addcontentsline{toc}{section}{\tocacronym{SN 56.82} \toctranslation{High Beds } \tocroot{Uccāsayanasutta}}
\markboth{High Beds }{Uccāsayanasutta}
\extramarks{SN 56.82}{SN 56.82}

“…\marginnote{1.1} the sentient beings who refrain from high and luxurious beds are few, while those who don’t refrain are many. …” 

%
\section*{{\suttatitleacronym SN 56.83}{\suttatitletranslation Gold and Money }{\suttatitleroot Jātarūparajatasutta}}
\addcontentsline{toc}{section}{\tocacronym{SN 56.83} \toctranslation{Gold and Money } \tocroot{Jātarūparajatasutta}}
\markboth{Gold and Money }{Jātarūparajatasutta}
\extramarks{SN 56.83}{SN 56.83}

“…\marginnote{1.1} the sentient beings who refrain from receiving gold and money are few, while those who don’t refrain are many. …” 

%
\section*{{\suttatitleacronym SN 56.84}{\suttatitletranslation Raw Grain }{\suttatitleroot Āmakadhaññasutta}}
\addcontentsline{toc}{section}{\tocacronym{SN 56.84} \toctranslation{Raw Grain } \tocroot{Āmakadhaññasutta}}
\markboth{Raw Grain }{Āmakadhaññasutta}
\extramarks{SN 56.84}{SN 56.84}

“…\marginnote{1.1} the sentient beings who refrain from receiving raw grain are few, while those who don’t refrain are many. …” 

%
\section*{{\suttatitleacronym SN 56.85}{\suttatitletranslation Raw Meat }{\suttatitleroot Āmakamaṁsasutta}}
\addcontentsline{toc}{section}{\tocacronym{SN 56.85} \toctranslation{Raw Meat } \tocroot{Āmakamaṁsasutta}}
\markboth{Raw Meat }{Āmakamaṁsasutta}
\extramarks{SN 56.85}{SN 56.85}

“…\marginnote{1.1} the sentient beings who refrain from receiving raw meat are few, while those who don’t refrain are many. …” 

%
\section*{{\suttatitleacronym SN 56.86}{\suttatitletranslation Women and Girls }{\suttatitleroot Kumārikasutta}}
\addcontentsline{toc}{section}{\tocacronym{SN 56.86} \toctranslation{Women and Girls } \tocroot{Kumārikasutta}}
\markboth{Women and Girls }{Kumārikasutta}
\extramarks{SN 56.86}{SN 56.86}

“…\marginnote{1.1} the sentient beings who refrain from receiving women and girls are few, while those who don’t refrain are many. …” 

%
\section*{{\suttatitleacronym SN 56.87}{\suttatitletranslation Bondservants }{\suttatitleroot Dāsidāsasutta}}
\addcontentsline{toc}{section}{\tocacronym{SN 56.87} \toctranslation{Bondservants } \tocroot{Dāsidāsasutta}}
\markboth{Bondservants }{Dāsidāsasutta}
\extramarks{SN 56.87}{SN 56.87}

“…\marginnote{1.1} the sentient beings who refrain from receiving male and female bondservants are few, while those who don’t refrain are many. …” 

%
\section*{{\suttatitleacronym SN 56.88}{\suttatitletranslation Goats and Sheep }{\suttatitleroot Ajeḷakasutta}}
\addcontentsline{toc}{section}{\tocacronym{SN 56.88} \toctranslation{Goats and Sheep } \tocroot{Ajeḷakasutta}}
\markboth{Goats and Sheep }{Ajeḷakasutta}
\extramarks{SN 56.88}{SN 56.88}

“…\marginnote{1.1} the sentient beings who refrain from receiving goats and sheep are few, while those who don’t refrain are many. …” 

%
\section*{{\suttatitleacronym SN 56.89}{\suttatitletranslation Chickens and Pigs }{\suttatitleroot Kukkuṭasūkarasutta}}
\addcontentsline{toc}{section}{\tocacronym{SN 56.89} \toctranslation{Chickens and Pigs } \tocroot{Kukkuṭasūkarasutta}}
\markboth{Chickens and Pigs }{Kukkuṭasūkarasutta}
\extramarks{SN 56.89}{SN 56.89}

“…\marginnote{1.1} the sentient beings who refrain from receiving chickens and pigs are few, while those who don’t refrain are many. …” 

%
\section*{{\suttatitleacronym SN 56.90}{\suttatitletranslation Elephants and Cows }{\suttatitleroot Hatthigavassasutta}}
\addcontentsline{toc}{section}{\tocacronym{SN 56.90} \toctranslation{Elephants and Cows } \tocroot{Hatthigavassasutta}}
\markboth{Elephants and Cows }{Hatthigavassasutta}
\extramarks{SN 56.90}{SN 56.90}

“…\marginnote{1.1} the sentient beings who refrain from receiving elephants, cows, horses, and mares are few, while those who don’t refrain are many. …” 

%
\addtocontents{toc}{\let\protect\contentsline\protect\nopagecontentsline}
\chapter*{The Fourth Chapter of Abbreviated Texts on Raw Grain }
\addcontentsline{toc}{chapter}{\tocchapterline{The Fourth Chapter of Abbreviated Texts on Raw Grain }}
\addtocontents{toc}{\let\protect\contentsline\protect\oldcontentsline}

%
\section*{{\suttatitleacronym SN 56.91}{\suttatitletranslation Fields and Land }{\suttatitleroot Khettavatthusutta}}
\addcontentsline{toc}{section}{\tocacronym{SN 56.91} \toctranslation{Fields and Land } \tocroot{Khettavatthusutta}}
\markboth{Fields and Land }{Khettavatthusutta}
\extramarks{SN 56.91}{SN 56.91}

“…\marginnote{1.1} the sentient beings who refrain from receiving fields and land are few, while those who don’t refrain are many. …” 

%
\section*{{\suttatitleacronym SN 56.92}{\suttatitletranslation Buying and Selling }{\suttatitleroot Kayavikkayasutta}}
\addcontentsline{toc}{section}{\tocacronym{SN 56.92} \toctranslation{Buying and Selling } \tocroot{Kayavikkayasutta}}
\markboth{Buying and Selling }{Kayavikkayasutta}
\extramarks{SN 56.92}{SN 56.92}

“…\marginnote{1.1} the sentient beings who refrain from buying and selling are few, while those who don’t refrain are many. …” 

%
\section*{{\suttatitleacronym SN 56.93}{\suttatitletranslation Errands }{\suttatitleroot Dūteyyasutta}}
\addcontentsline{toc}{section}{\tocacronym{SN 56.93} \toctranslation{Errands } \tocroot{Dūteyyasutta}}
\markboth{Errands }{Dūteyyasutta}
\extramarks{SN 56.93}{SN 56.93}

“…\marginnote{1.1} the sentient beings who refrain from running errands and messages are few, while those who don’t refrain are many. …” 

%
\section*{{\suttatitleacronym SN 56.94}{\suttatitletranslation False Weights }{\suttatitleroot Tulākūṭasutta}}
\addcontentsline{toc}{section}{\tocacronym{SN 56.94} \toctranslation{False Weights } \tocroot{Tulākūṭasutta}}
\markboth{False Weights }{Tulākūṭasutta}
\extramarks{SN 56.94}{SN 56.94}

“…\marginnote{1.1} the sentient beings who refrain from falsifying weights, metals, or measures are few, while those who don’t refrain are many. …” 

%
\section*{{\suttatitleacronym SN 56.95}{\suttatitletranslation Bribery }{\suttatitleroot Ukkoṭanasutta}}
\addcontentsline{toc}{section}{\tocacronym{SN 56.95} \toctranslation{Bribery } \tocroot{Ukkoṭanasutta}}
\markboth{Bribery }{Ukkoṭanasutta}
\extramarks{SN 56.95}{SN 56.95}

“…\marginnote{1.1} the sentient beings who refrain from bribery, fraud, cheating, and duplicity are few, while those who don’t refrain are many. …” 

%
\section*{{\suttatitleacronym SN 56.96–101}{\suttatitletranslation Mutilation, Etc. }{\suttatitleroot Chedanādisutta}}
\addcontentsline{toc}{section}{\tocacronym{SN 56.96–101} \toctranslation{Mutilation, Etc. } \tocroot{Chedanādisutta}}
\markboth{Mutilation, Etc. }{Chedanādisutta}
\extramarks{SN 56.96–101}{SN 56.96–101}

“…\marginnote{1.1} the sentient beings who refrain from mutilation, murder, abduction, banditry, plunder, and violence are few, while those who don’t refrain are many. Why is that? It’s because they haven’t seen the four noble truths. What four? The noble truths of suffering, its origin, its cessation, and the path. 

That’s\marginnote{2.1} why you should practice meditation …” 

%
\addtocontents{toc}{\let\protect\contentsline\protect\nopagecontentsline}
\chapter*{The Chapter of Abbreviated Texts on Five Destinations }
\addcontentsline{toc}{chapter}{\tocchapterline{The Chapter of Abbreviated Texts on Five Destinations }}
\addtocontents{toc}{\let\protect\contentsline\protect\oldcontentsline}

%
\section*{{\suttatitleacronym SN 56.102}{\suttatitletranslation Passing Away as Humans and Reborn in Hell }{\suttatitleroot Manussacutinirayasutta}}
\addcontentsline{toc}{section}{\tocacronym{SN 56.102} \toctranslation{Passing Away as Humans and Reborn in Hell } \tocroot{Manussacutinirayasutta}}
\markboth{Passing Away as Humans and Reborn in Hell }{Manussacutinirayasutta}
\extramarks{SN 56.102}{SN 56.102}

“…\marginnote{1.8} the sentient beings who die as humans and are reborn as humans are few, while those who die as humans and are reborn in hell are many …” 

%
\section*{{\suttatitleacronym SN 56.103}{\suttatitletranslation Passing Away as Humans and Reborn as Animals }{\suttatitleroot Manussacutitiracchānasutta}}
\addcontentsline{toc}{section}{\tocacronym{SN 56.103} \toctranslation{Passing Away as Humans and Reborn as Animals } \tocroot{Manussacutitiracchānasutta}}
\markboth{Passing Away as Humans and Reborn as Animals }{Manussacutitiracchānasutta}
\extramarks{SN 56.103}{SN 56.103}

“…\marginnote{1.1} the sentient beings who die as humans and are reborn as humans are few, while those who die as humans and are reborn in the animal realm are many …” 

%
\section*{{\suttatitleacronym SN 56.104}{\suttatitletranslation Passing Away as Humans and Reborn as Ghosts }{\suttatitleroot Manussacutipettivisayasutta}}
\addcontentsline{toc}{section}{\tocacronym{SN 56.104} \toctranslation{Passing Away as Humans and Reborn as Ghosts } \tocroot{Manussacutipettivisayasutta}}
\markboth{Passing Away as Humans and Reborn as Ghosts }{Manussacutipettivisayasutta}
\extramarks{SN 56.104}{SN 56.104}

“…\marginnote{1.1} the sentient beings who die as humans and are reborn as humans are few, while those who die as humans and are reborn in the ghost realm are many …” 

%
\section*{{\suttatitleacronym SN 56.105–107}{\suttatitletranslation Passing Away as Humans and Reborn as Gods }{\suttatitleroot Manussacutidevanirayādisutta}}
\addcontentsline{toc}{section}{\tocacronym{SN 56.105–107} \toctranslation{Passing Away as Humans and Reborn as Gods } \tocroot{Manussacutidevanirayādisutta}}
\markboth{Passing Away as Humans and Reborn as Gods }{Manussacutidevanirayādisutta}
\extramarks{SN 56.105–107}{SN 56.105–107}

“…\marginnote{1.1} the sentient beings who die as humans and are reborn as gods are few, while those who die as humans and are reborn in hell, or the animal realm, or the ghost realm are many.” 

%
\section*{{\suttatitleacronym SN 56.108–110}{\suttatitletranslation Passing Away as Gods and Reborn as Gods }{\suttatitleroot Devacutinirayādisutta}}
\addcontentsline{toc}{section}{\tocacronym{SN 56.108–110} \toctranslation{Passing Away as Gods and Reborn as Gods } \tocroot{Devacutinirayādisutta}}
\markboth{Passing Away as Gods and Reborn as Gods }{Devacutinirayādisutta}
\extramarks{SN 56.108–110}{SN 56.108–110}

“…\marginnote{1.1} the sentient beings who die as gods and are reborn as gods are few, while those who die as gods and are reborn in hell, or the animal realm, or the ghost realm are many.” 

%
\section*{{\suttatitleacronym SN 56.111–113}{\suttatitletranslation Dying as Gods and Reborn as Humans }{\suttatitleroot Devamanussanirayādisutta}}
\addcontentsline{toc}{section}{\tocacronym{SN 56.111–113} \toctranslation{Dying as Gods and Reborn as Humans } \tocroot{Devamanussanirayādisutta}}
\markboth{Dying as Gods and Reborn as Humans }{Devamanussanirayādisutta}
\extramarks{SN 56.111–113}{SN 56.111–113}

“…\marginnote{1.1} the sentient beings who die as gods and are reborn as humans are few, while those who die as gods and are reborn in hell, or the animal realm, or the ghost realm are many.” 

%
\section*{{\suttatitleacronym SN 56.114–116}{\suttatitletranslation Dying in Hell and Reborn as Humans }{\suttatitleroot Nirayamanussanirayādisutta}}
\addcontentsline{toc}{section}{\tocacronym{SN 56.114–116} \toctranslation{Dying in Hell and Reborn as Humans } \tocroot{Nirayamanussanirayādisutta}}
\markboth{Dying in Hell and Reborn as Humans }{Nirayamanussanirayādisutta}
\extramarks{SN 56.114–116}{SN 56.114–116}

“…\marginnote{1.1} the sentient beings who die in hell and are reborn as humans are few, while those who die in hell and are reborn in hell, or the animal realm, or the ghost realm are many.” 

%
\section*{{\suttatitleacronym SN 56.117–119}{\suttatitletranslation Dying in Hell and Reborn as Gods }{\suttatitleroot Nirayadevanirayādisutta}}
\addcontentsline{toc}{section}{\tocacronym{SN 56.117–119} \toctranslation{Dying in Hell and Reborn as Gods } \tocroot{Nirayadevanirayādisutta}}
\markboth{Dying in Hell and Reborn as Gods }{Nirayadevanirayādisutta}
\extramarks{SN 56.117–119}{SN 56.117–119}

“…\marginnote{1.1} the sentient beings who die in hell and are reborn as gods are few, while those who die in hell and are reborn in hell, or the animal realm, or the ghost realm are many.” 

%
\section*{{\suttatitleacronym SN 56.120–122}{\suttatitletranslation Dying as Animals and Reborn as Humans }{\suttatitleroot Tiracchānamanussanirayādisutta}}
\addcontentsline{toc}{section}{\tocacronym{SN 56.120–122} \toctranslation{Dying as Animals and Reborn as Humans } \tocroot{Tiracchānamanussanirayādisutta}}
\markboth{Dying as Animals and Reborn as Humans }{Tiracchānamanussanirayādisutta}
\extramarks{SN 56.120–122}{SN 56.120–122}

“…\marginnote{1.1} the sentient beings who die as animals and are reborn as humans are few, while those who die as animals and are reborn in hell, or the animal realm, or the ghost realm are many.” 

%
\section*{{\suttatitleacronym SN 56.123–125}{\suttatitletranslation Dying as Animals and Reborn as Gods }{\suttatitleroot Tiracchānadevanirayādisutta}}
\addcontentsline{toc}{section}{\tocacronym{SN 56.123–125} \toctranslation{Dying as Animals and Reborn as Gods } \tocroot{Tiracchānadevanirayādisutta}}
\markboth{Dying as Animals and Reborn as Gods }{Tiracchānadevanirayādisutta}
\extramarks{SN 56.123–125}{SN 56.123–125}

“…\marginnote{1.1} the sentient beings who die as animals and are reborn as gods are few, while those who die as animals and are reborn in hell, or the animal realm, or the ghost realm are many.” 

%
\section*{{\suttatitleacronym SN 56.126–128}{\suttatitletranslation Dying as Ghosts and Reborn as Humans }{\suttatitleroot Pettimanussanirayādisutta}}
\addcontentsline{toc}{section}{\tocacronym{SN 56.126–128} \toctranslation{Dying as Ghosts and Reborn as Humans } \tocroot{Pettimanussanirayādisutta}}
\markboth{Dying as Ghosts and Reborn as Humans }{Pettimanussanirayādisutta}
\extramarks{SN 56.126–128}{SN 56.126–128}

“…\marginnote{1.1} the sentient beings who die as ghosts and are reborn as humans are few, while those who die as ghosts and are reborn in hell, or the animal realm, or the ghost realm are many.” 

%
\section*{{\suttatitleacronym SN 56.129–130}{\suttatitletranslation Dying as Ghosts and Reborn as Gods }{\suttatitleroot Pettidevanirayādisutta}}
\addcontentsline{toc}{section}{\tocacronym{SN 56.129–130} \toctranslation{Dying as Ghosts and Reborn as Gods } \tocroot{Pettidevanirayādisutta}}
\markboth{Dying as Ghosts and Reborn as Gods }{Pettidevanirayādisutta}
\extramarks{SN 56.129–130}{SN 56.129–130}

“…\marginnote{1.1} the sentient beings who die as ghosts and are reborn as gods are few, while those who die as ghosts and are reborn in hell are many.” 

“…\marginnote{1.2} the sentient beings who die as ghosts and are reborn as gods are few, while those who die as ghosts and are reborn in the animal realm are many.” 

%
\section*{{\suttatitleacronym SN 56.131}{\suttatitletranslation Dying as Ghosts and Reborn as Ghosts }{\suttatitleroot Pettidevapettivisayasutta}}
\addcontentsline{toc}{section}{\tocacronym{SN 56.131} \toctranslation{Dying as Ghosts and Reborn as Ghosts } \tocroot{Pettidevapettivisayasutta}}
\markboth{Dying as Ghosts and Reborn as Ghosts }{Pettidevapettivisayasutta}
\extramarks{SN 56.131}{SN 56.131}

“…\marginnote{1.1} the sentient beings who die as ghosts and are reborn as gods are few, while those who die as ghosts and are reborn in the ghost realm are many. Why is that? It’s because they haven’t seen the four noble truths. What four? The noble truths of suffering, the origin of suffering, the cessation of suffering, and the practice that leads to the cessation of suffering. 

That’s\marginnote{2.1} why you should practice meditation to understand: ‘This is suffering’ … ‘This is the origin of suffering’ … ‘This is the cessation of suffering’ … ‘This is the practice that leads to the cessation of suffering’.” 

That\marginnote{3.1} is what the Buddha said. Satisfied, the mendicants were happy with what the Buddha said. 

\scendsutta{The Linked Discourses on the Truths, the twelfth section. }

\scendbook{The Great Book is finished. }

\scendbook{The Linked Discourses is completed. }

%
\backmatter%
\chapter*{Colophon}
\addcontentsline{toc}{chapter}{Colophon}
\markboth{Colophon}{Colophon}

\section*{The Translator}

Bhikkhu Sujato was born as Anthony Aidan Best on 4/11/1966 in Perth, Western Australia. He grew up in the pleasant suburbs of Mt Lawley and Attadale alongside his sister Nicola, who was the good child. His mother, Margaret Lorraine Huntsman née Pinder, said “he’ll either be a priest or a poet”, while his father, Anthony Thomas Best, advised him to “never do anything for money”. He attended Aquinas College, a Catholic school, where he decided to become an atheist. At the University of WA he studied philosophy, aiming to learn what he wanted to do with his life. Finding that what he wanted to do was play guitar, he dropped out. His main band was named Martha’s Vineyard, which achieved modest success in the indie circuit. 

A seemingly random encounter with a roadside joey took him to Thailand, where he entered his first meditation retreat at Wat Ram Poeng, Chieng Mai in 1992. Feeling the call to the Buddha’s path, he took full ordination in Wat Pa Nanachat in 1994, where his teachers were Ajahn Pasanno and Ajahn Jayasaro. In 1997 he returned to Perth to study with Ajahn Brahm at Bodhinyana Monastery. 

He spent several years practicing in seclusion in Malaysia and Thailand before establishing Santi Forest Monastery in Bundanoon, NSW, in 2003. There he was instrumental in supporting the establishment of the Theravada bhikkhuni order in Australia and advocating for women’s rights. He continues to teach in Australia and globally, with a special concern for the moral implications of climate change and other forms of environmental destruction. He has published a series of books of original and groundbreaking research on early Buddhism. 

In 2005 he founded SuttaCentral together with Rod Bucknell and John Kelly. In 2015, seeing the need for a complete, accurate, plain English translation of the Pali texts, he undertook the task, spending nearly three years in isolation on the isle of Qi Mei off the coast of the nation of Taiwan. He completed the four main \textsanskrit{Nikāyas} in 2018, and the early books of the Khuddaka \textsanskrit{Nikāya} were complete by 2021. All this work is dedicated to the public domain and is entirely free of copyright encumbrance. 

In 2019 he returned to Sydney where he established Lokanta Vihara (The Monastery at the End of the World). 

\section*{Creation Process}

Primary source was the digital \textsanskrit{Mahāsaṅgīti} edition of the Pali \textsanskrit{Tipiṭaka}. Translated from the Pali, with reference to several English translations, especially those of Bhikkhu Bodhi.

\section*{The Translation}

This translation was part of a project to translate the four Pali \textsanskrit{Nikāyas} with the following aims: plain, approachable English; consistent terminology; accurate rendition of the Pali; free of copyright. It was made during 2016–2018 while Bhikkhu Sujato was staying in Qimei, Taiwan.

\section*{About SuttaCentral}

SuttaCentral publishes early Buddhist texts. Since 2005 we have provided root texts in Pali, Chinese, Sanskrit, Tibetan, and other languages, parallels between these texts, and translations in many modern languages. We build on the work of generations of scholars, and offer our contribution freely.

SuttaCentral is driven by volunteer contributions, and in addition we employ professional developers. We offer a sponsorship program for high quality translations from the original languages. Financial support for SuttaCentral is handled by the SuttaCentral Development Trust, a charitable trust registered in Australia.

\section*{About Bilara}

“Bilara” means “cat” in Pali, and it is the name of our Computer Assisted Translation (CAT) software. Bilara is a web app that enables translators to translate early Buddhist texts into their own language. These translations are published on SuttaCentral with the root text and translation side by side.

\section*{About SuttaCentral Editions}

The SuttaCentral Editions project makes high quality books from selected Bilara translations. These are published in formats including HTML, EPUB, PDF, and print.

If you want to print any of our Editions, please let us know and we will help prepare a file to your specifications.

%
\end{document}