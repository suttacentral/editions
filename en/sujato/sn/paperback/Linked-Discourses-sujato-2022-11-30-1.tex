\documentclass[12pt,openany]{book}%
\usepackage{lastpage}%
%
\usepackage[inner=1in, outer=1in, top=.7in, bottom=1in, papersize={6in,9in}, headheight=13pt]{geometry}
\usepackage{polyglossia}
\usepackage[12pt]{moresize}
\usepackage{soul}%
\usepackage{microtype}
\usepackage{tocbasic}
\usepackage{realscripts}
\usepackage{epigraph}%
\usepackage{setspace}%
\usepackage{sectsty}
\usepackage{fontspec}
\usepackage{marginnote}
\usepackage[bottom]{footmisc}
\usepackage{enumitem}
\usepackage{fancyhdr}
\usepackage{extramarks}
\usepackage{graphicx}
\usepackage{verse}
\usepackage{relsize}
\usepackage{etoolbox}
\usepackage[a-3u]{pdfx}

\hypersetup{
colorlinks=true,
urlcolor=black,
linkcolor=black,
citecolor=black
}

% use a small amount of tracking on small caps
\SetTracking[ spacing = {25*,166, } ]{ encoding = *, shape = sc }{ 25 }

% add a blank page
\newcommand{\blankpage}{
\newpage
\thispagestyle{empty}
\mbox{}
\newpage
}

% define languages
\setdefaultlanguage[]{english}
\setotherlanguage[script=Latin]{sanskrit}

%\usepackage{pagegrid}
%\pagegridsetup{top-left, step=.25in}

% define fonts
% use if arno sanskrit is unavailable
%\setmainfont{Gentium Plus}
%\newfontfamily\Semiboldsubheadfont[]{Gentium Plus}
%\newfontfamily\Semiboldnormalfont[]{Gentium Plus}
%\newfontfamily\Lightfont[]{Gentium Plus}
%\newfontfamily\Marginalfont[]{Gentium Plus}
%\newfontfamily\Allsmallcapsfont[RawFeature=+c2sc]{Gentium Plus}
%\newfontfamily\Noligaturefont[Renderer=Basic]{Gentium Plus}
%\newfontfamily\Noligaturecaptionfont[Renderer=Basic]{Gentium Plus}
%\newfontfamily\Fleuronfont[Ornament=1]{Gentium Plus}

% use if arno sanskrit is available. display is applied to \chapter and \part, subhead to \section and \subsection. When specifying semibold, the italic must be defined.
\setmainfont[Numbers=OldStyle]{Arno Pro}
\newfontfamily\Semibolddisplayfont[BoldItalicFont = Arno Pro Semibold Italic Display]{Arno Pro Semibold Display} %
\newfontfamily\Semiboldsubheadfont[BoldItalicFont = Arno Pro Semibold Italic Subhead]{Arno Pro Semibold Subhead}
\newfontfamily\Semiboldnormalfont[BoldItalicFont = Arno Pro Semibold Italic]{Arno Pro Semibold}
\newfontfamily\Marginalfont[RawFeature=+subs]{Arno Pro Regular}
\newfontfamily\Allsmallcapsfont[RawFeature=+c2sc]{Arno Pro}
\newfontfamily\Noligaturefont[Renderer=Basic]{Arno Pro}
\newfontfamily\Noligaturecaptionfont[Renderer=Basic]{Arno Pro Caption}

% chinese fonts
\newfontfamily\cjk{Noto Serif TC}
\newcommand*{\langlzh}[1]{\cjk{#1}\normalfont}%

% logo
\newfontfamily\Logofont{sclogo.ttf}
\newcommand*{\sclogo}[1]{\large\Logofont{#1}}

% use subscript numerals for margin notes
\renewcommand*{\marginfont}{\Marginalfont}

% ensure margin notes have consistent vertical alignment
\renewcommand*{\marginnotevadjust}{-.17em}

% use compact lists
\setitemize{noitemsep,leftmargin=1em}
\setenumerate{noitemsep,leftmargin=1em}
\setdescription{noitemsep, style=unboxed, leftmargin=0em}

% style ToC
\DeclareTOCStyleEntries[
  raggedentrytext,
  linefill=\hfill,
  pagenumberwidth=.5in,
  pagenumberformat=\normalfont,
  entryformat=\normalfont
]{tocline}{chapter,section}


  \setlength\topsep{0pt}%
  \setlength\parskip{0pt}%

% define new \centerpars command for use in ToC. This ensures centering, proper wrapping, and no page break after
\def\startcenter{%
  \par
  \begingroup
  \leftskip=0pt plus 1fil
  \rightskip=\leftskip
  \parindent=0pt
  \parfillskip=0pt
}
\def\stopcenter{%
  \par
  \endgroup
}
\long\def\centerpars#1{\startcenter#1\stopcenter}

% redefine part, so that it adds a toc entry without page number
\let\oldcontentsline\contentsline
\newcommand{\nopagecontentsline}[3]{\oldcontentsline{#1}{#2}{}}

    \makeatletter
\renewcommand*\l@part[2]{%
  \ifnum \c@tocdepth >-2\relax
    \addpenalty{-\@highpenalty}%
    \addvspace{0em \@plus\p@}%
    \setlength\@tempdima{3em}%
    \begingroup
      \parindent \z@ \rightskip \@pnumwidth
      \parfillskip -\@pnumwidth
      {\leavevmode
       \setstretch{.85}\large\scshape\centerpars{#1}\vspace*{-1em}\llap{#2}}\par
       \nobreak
         \global\@nobreaktrue
         \everypar{\global\@nobreakfalse\everypar{}}%
    \endgroup
  \fi}
\makeatother

\makeatletter
\def\@pnumwidth{2em}
\makeatother

% define new sectioning command, which is only used in volumes where the pannasa is found in some parts but not others, especially in an and sn

\newcommand*{\pannasa}[1]{\clearpage\thispagestyle{empty}\begin{center}\vspace*{14em}\setstretch{.85}\huge\itshape\scshape\MakeLowercase{#1}\end{center}}

    \makeatletter
\newcommand*\l@pannasa[2]{%
  \ifnum \c@tocdepth >-2\relax
    \addpenalty{-\@highpenalty}%
    \addvspace{.5em \@plus\p@}%
    \setlength\@tempdima{3em}%
    \begingroup
      \parindent \z@ \rightskip \@pnumwidth
      \parfillskip -\@pnumwidth
      {\leavevmode
       \setstretch{.85}\large\itshape\scshape\lowercase{\centerpars{#1}}\vspace*{-1em}\llap{#2}}\par
       \nobreak
         \global\@nobreaktrue
         \everypar{\global\@nobreakfalse\everypar{}}%
    \endgroup
  \fi}
\makeatother

% don't put page number on first page of toc (relies on etoolbox)
\patchcmd{\chapter}{plain}{empty}{}{}

% global line height
\setstretch{1.05}

% allow linebreak after em-dash
\catcode`\—=13
\protected\def—{\unskip\textemdash\allowbreak}

% style headings with secsty. chapter and section are defined per-edition
\partfont{\setstretch{.85}\normalfont\centering\textsc}
\subsectionfont{\setstretch{.85}\Semiboldsubheadfont}%
\subsubsectionfont{\setstretch{.85}\Semiboldnormalfont}

% style elements of suttatitle
\newcommand*{\suttatitleacronym}[1]{\smaller[2]{#1}\vspace*{.3em}}
\newcommand*{\suttatitletranslation}[1]{\linebreak{#1}}
\newcommand*{\suttatitleroot}[1]{\linebreak\smaller[2]\itshape{#1}}

\DeclareTOCStyleEntries[
  indent=3.3em,
  dynindent,
  beforeskip=.2em plus -2pt minus -1pt,
]{tocline}{section}

\DeclareTOCStyleEntries[
  indent=0em,
  dynindent,
  beforeskip=.4em plus -2pt minus -1pt,
]{tocline}{chapter}

\newcommand*{\tocacronym}[1]{\hspace*{-3.3em}{#1}\quad}
\newcommand*{\toctranslation}[1]{#1}
\newcommand*{\tocroot}[1]{(\textit{#1})}
\newcommand*{\tocchapterline}[1]{\bfseries\itshape{#1}}


% redefine paragraph and subparagraph headings to not be inline
\makeatletter
% Change the style of paragraph headings %
\renewcommand\paragraph{\@startsection{paragraph}{4}{\z@}%
            {-2.5ex\@plus -1ex \@minus -.25ex}%
            {1.25ex \@plus .25ex}%
            {\noindent\Semiboldnormalfont\normalsize}}

% Change the style of subparagraph headings %
\renewcommand\subparagraph{\@startsection{subparagraph}{5}{\z@}%
            {-2.5ex\@plus -1ex \@minus -.25ex}%
            {1.25ex \@plus .25ex}%
            {\noindent\Semiboldnormalfont\small}}
\makeatother

% use etoolbox to suppress page numbers on \part
\patchcmd{\part}{\thispagestyle{plain}}{\thispagestyle{empty}}
  {}{\errmessage{Cannot patch \string\part}}

% and to reduce margins on quotation
\patchcmd{\quotation}{\rightmargin}{\leftmargin 1.2em \rightmargin}{}{}
\AtBeginEnvironment{quotation}{\small}

% titlepage
\newcommand*{\titlepageTranslationTitle}[1]{{\begin{center}\begin{large}{#1}\end{large}\end{center}}}
\newcommand*{\titlepageCreatorName}[1]{{\begin{center}\begin{normalsize}{#1}\end{normalsize}\end{center}}}

% halftitlepage
\newcommand*{\halftitlepageTranslationTitle}[1]{\setstretch{2.5}{\begin{Huge}\uppercase{\so{#1}}\end{Huge}}}
\newcommand*{\halftitlepageTranslationSubtitle}[1]{\setstretch{1.2}{\begin{large}{#1}\end{large}}}
\newcommand*{\halftitlepageFleuron}[1]{{\begin{large}\Fleuronfont{{#1}}\end{large}}}
\newcommand*{\halftitlepageByline}[1]{{\begin{normalsize}\textit{{#1}}\end{normalsize}}}
\newcommand*{\halftitlepageCreatorName}[1]{{\begin{LARGE}{\textsc{#1}}\end{LARGE}}}
\newcommand*{\halftitlepageVolumeNumber}[1]{{\begin{normalsize}{\Allsmallcapsfont{\textsc{#1}}}\end{normalsize}}}
\newcommand*{\halftitlepageVolumeAcronym}[1]{{\begin{normalsize}{#1}\end{normalsize}}}
\newcommand*{\halftitlepageVolumeTranslationTitle}[1]{{\begin{Large}{\textsc{#1}}\end{Large}}}
\newcommand*{\halftitlepageVolumeRootTitle}[1]{{\begin{normalsize}{\Allsmallcapsfont{\textsc{\itshape #1}}}\end{normalsize}}}
\newcommand*{\halftitlepagePublisher}[1]{{\begin{large}{\Noligaturecaptionfont\textsc{#1}}\end{large}}}

% epigraph
\renewcommand{\epigraphflush}{center}
\renewcommand*{\epigraphwidth}{.85\textwidth}
\newcommand*{\epigraphTranslatedTitle}[1]{\vspace*{.5em}\footnotesize\textsc{#1}\\}%
\newcommand*{\epigraphRootTitle}[1]{\footnotesize\textit{#1}\\}%
\newcommand*{\epigraphReference}[1]{\footnotesize{#1}}%

% custom commands for html styling classes
\newcommand*{\scnamo}[1]{\begin{center}\textit{#1}\end{center}}
\newcommand*{\scendsection}[1]{\begin{center}\textit{#1}\end{center}}
\newcommand*{\scendsutta}[1]{\begin{center}\textit{#1}\end{center}}
\newcommand*{\scendbook}[1]{\begin{center}\uppercase{#1}\end{center}}
\newcommand*{\scendkanda}[1]{\begin{center}\textbf{#1}\end{center}}
\newcommand*{\scend}[1]{\begin{center}\textit{#1}\end{center}}
\newcommand*{\scuddanaintro}[1]{\textit{#1}}
\newcommand*{\scendvagga}[1]{\begin{center}\textbf{#1}\end{center}}
\newcommand*{\scrule}[1]{\textbf{#1}}
\newcommand*{\scadd}[1]{\textit{#1}}
\newcommand*{\scevam}[1]{\textsc{#1}}
\newcommand*{\scspeaker}[1]{\hspace{2em}\textit{#1}}
\newcommand*{\scbyline}[1]{\begin{flushright}\textit{#1}\end{flushright}\bigskip}

% custom command for thematic break = hr
\newcommand*{\thematicbreak}{\begin{center}\rule[.5ex]{6em}{.4pt}\begin{normalsize}\quad\Fleuronfont{•}\quad\end{normalsize}\rule[.5ex]{6em}{.4pt}\end{center}}

% manage and style page header and footer. "fancy" has header and footer, "plain" has footer only

\pagestyle{fancy}
\fancyhf{}
\fancyfoot[RE,LO]{\thepage}
\fancyfoot[LE,RO]{\footnotesize\lastleftxmark}
\fancyhead[CE]{\setstretch{.85}\Noligaturefont\MakeLowercase{\textsc{\firstrightmark}}}
\fancyhead[CO]{\setstretch{.85}\Noligaturefont\MakeLowercase{\textsc{\firstleftmark}}}
\renewcommand{\headrulewidth}{0pt}
\fancypagestyle{plain}{ %
\fancyhf{} % remove everything
\fancyfoot[RE,LO]{\thepage}
\fancyfoot[LE,RO]{\footnotesize\lastleftxmark}
\renewcommand{\headrulewidth}{0pt}
\renewcommand{\footrulewidth}{0pt}}

% style footnotes
\setlength{\skip\footins}{1em}

\makeatletter
\newcommand{\@makefntextcustom}[1]{%
    \parindent 0em%
    \thefootnote.\enskip #1%
}
\renewcommand{\@makefntext}[1]{\@makefntextcustom{#1}}
\makeatother

% hang quotes (requires microtype)
\microtypesetup{
  protrusion = true,
  expansion  = true,
  tracking   = true,
  factor     = 1000,
  patch      = all,
  final
}

% Custom protrusion rules to allow hanging punctuation
\SetProtrusion
{ encoding = *}
{
% char   right left
  {-} = {    , 500 },
  % Double Quotes
  \textquotedblleft
      = {1000,     },
  \textquotedblright
      = {    , 1000},
  \quotedblbase
      = {1000,     },
  % Single Quotes
  \textquoteleft
      = {1000,     },
  \textquoteright
      = {    , 1000},
  \quotesinglbase
      = {1000,     }
}

% make latex use actual font em for parindent, not Computer Modern Roman
\AtBeginDocument{\setlength{\parindent}{1em}}%
%

% Default values; a bit sloppier than normal
\tolerance 1414
\hbadness 1414
\emergencystretch 1.5em
\hfuzz 0.3pt
\clubpenalty = 10000
\widowpenalty = 10000
\displaywidowpenalty = 10000
\hfuzz \vfuzz
 \raggedbottom%

\title{Linked Discourses}
\author{Bhikkhu Sujato}
\date{}%
% define a different fleuron for each edition
\newfontfamily\Fleuronfont[Ornament=40]{Arno Pro}

% Define heading styles per edition for chapter and section. Suttatitle can be either of these, depending on the volume. 

\let\oldfrontmatter\frontmatter
\renewcommand{\frontmatter}{%
\chapterfont{\setstretch{.85}\normalfont\centering}%
\sectionfont{\setstretch{.85}\Semiboldsubheadfont}%
\oldfrontmatter}

\let\oldmainmatter\mainmatter
\renewcommand{\mainmatter}{%
\chapterfont{\setstretch{.85}\normalfont\centering}%
\sectionfont{\setstretch{.85}\normalfont\centering}%
\oldmainmatter}

\let\oldbackmatter\backmatter
\renewcommand{\backmatter}{%
\chapterfont{\setstretch{.85}\normalfont\centering}%
\sectionfont{\setstretch{.85}\Semiboldsubheadfont}%
\oldbackmatter}
%
%
\begin{document}%
\normalsize%
\frontmatter%
\setlength{\parindent}{0cm}

\pagestyle{empty}

\maketitle

\blankpage%
\begin{center}

\vspace*{2.2em}

\halftitlepageTranslationTitle{Linked Discourses}

\vspace*{1em}

\halftitlepageTranslationSubtitle{A plain translation of the Saṁyutta Nikāya}

\vspace*{2em}

\halftitlepageFleuron{•}

\vspace*{2em}

\halftitlepageByline{translated and introduced by}

\vspace*{.5em}

\halftitlepageCreatorName{Bhikkhu Sujato}

\vspace*{4em}

\halftitlepageVolumeNumber{Volume 1}

\smallskip

\halftitlepageVolumeAcronym{SN 1–11}

\smallskip

\halftitlepageVolumeTranslationTitle{The Group of Linked Discourses With Verses}

\smallskip

\halftitlepageVolumeRootTitle{Sagāthāvaggasaṁyutta}

\vspace*{\fill}

\sclogo{0}
 \halftitlepagePublisher{SuttaCentral}

\end{center}

\newpage
%
\setstretch{1.05}

\begin{footnotesize}

\textit{Linked Discourses} is a translation of the Saṁyuttanikāya by Bhikkhu Sujato.

\medskip

Creative Commons Zero (CC0)

To the extent possible under law, Bhikkhu Sujato has waived all copyright and related or neighboring rights to \textit{Linked Discourses}.

\medskip

This work is published from Australia.

\begin{center}
\textit{This translation is an expression of an ancient spiritual text that has been passed down by the Buddhist tradition for the benefit of all sentient beings. It is dedicated to the public domain via Creative Commons Zero (CC0). You are encouraged to copy, reproduce, adapt, alter, or otherwise make use of this translation. The translator respectfully requests that any use be in accordance with the values and principles of the Buddhist community.}
\end{center}

\medskip

\begin{description}
    \item[Web publication date] 2018
    \item[This edition] 2022-11-30 08:48:21
    \item[Publication type] paperback
    \item[Edition] ed5
    \item[Number of volumes] 5
    \item[Publication ISBN] 978-1-76132-086-6
    \item[Publication URL] https://suttacentral.net/editions/sn/en/sujato
    \item[Source URL] https://github.com/suttacentral/bilara-data/tree/published/translation/en/sujato/sutta/sn
    \item[Publication number] scpub4
\end{description}

\medskip

Published by SuttaCentral

\medskip

\textit{SuttaCentral,\\
c/o Alwis \& Alwis Pty Ltd\\
Kaurna Country,\\
Suite 12,\\
198 Greenhill Road,\\
Eastwood,\\
SA 5063,\\
Australia}

\end{footnotesize}

\newpage

\setlength{\parindent}{1.5em}%%
\newpage

\vspace*{\fill}

\begin{center}
\epigraph{Truth itself is the undying word:\\
this is an ancient principle.\\
Good people say that the teaching and its meaning\\
are grounded in the truth.}
{
\epigraphTranslatedTitle{“Well-Spoken Words”}
\epigraphRootTitle{\textsanskrit{Subhāsitasutta}}
\epigraphReference{\textsanskrit{Saṁyutta} \textsanskrit{Nikāya} 8.5}
}
\end{center}

\vspace*{2in}

\vspace*{\fill}

\blankpage%

\setlength{\parindent}{1em}
%
\tableofcontents
\newpage
\pagestyle{fancy}
%
\chapter*{The SuttaCentral Editions Series}
\addcontentsline{toc}{chapter}{The SuttaCentral Editions Series}
\markboth{The SuttaCentral Editions Series}{The SuttaCentral Editions Series}

Since 2005 SuttaCentral has provided access to the texts, translations, and parallels of early Buddhist texts. In 2018 we started creating and publishing our own translations of these seminal spiritual classics. The “Editions” series now makes selected translations available as books in various forms, including print, PDF, and EPUB.

Editions are selected from our most complete, well-crafted, and reliable translations. They aim to bring these texts to a wider audience in forms that reward mindful reading. Care is taken with every detail of the production, and we aim to meet or exceed professional best standards in every way. These are the core scriptures underlying the entire Buddhist tradition, and we believe that they deserve to be preserved and made available in highest quality without compromise.

SuttaCentral is a charitable organization. Our work is accomplished by volunteers and through the generosity of our donors. Everything we create is offered to all of humanity free of any copyright or licensing restrictions. 

%
\chapter*{Preface to \emph{Linked Discourses}}
\addcontentsline{toc}{chapter}{Preface to \emph{Linked Discourses}}
\markboth{Preface to \emph{Linked Discourses}}{Preface to \emph{Linked Discourses}}

I was introduced to Buddhism through the \textsanskrit{Theravāda} tradition. I found my way to it through meditation, and pursued my studies so I could more deeply understand my experiences in meditation. It didn’t take me long to notice the rather odd fact that, while meditation was supposed to reveal direct, experiential truths, there was a lot of disagreement about what these things were and what they meant. It seemed that even direct experience was filtered by beliefs and expectations. 

I found that, while many modern teachers dismissed the role of theory, insisting that experience alone was the standard, this was not the case in the Suttas. The Buddha placed right view at the start of the path, insisting that the framing of our ideas is what gives meaning to our experiences.

Living in Thailand, I was surrounded by Thai-flavored Theravada. My English-speaking monastic community at Wat Pa Nanachat was somewhat more eclectic. There were monks and visitors from all over, and while the “official” teachings were mainstream Theravada and forest tradition, behind the scenes there was a whole range of spiritual ideas and priorities. I learned about different flavors of Buddhism, but I didn’t really have a way of understanding how it all fit together, or how it related to my meditation. Many of the things I heard about seemed quite silly or far from the Buddha’s teaching, and I confess, I became quite dogmatic, convinced that Theravada was the one and only original and true way. It was the great Sri Lankan monk K. Sri Dhammananda who checked this impulse, gently reminding me of the respect we owe to all practitioners of Dhamma.

I read some books of Buddhist history, notably A.K. Warder’s \textit{Indian Buddhism}, and from them learned that there were early discourses in languages other than Pali, especially Chinese. This was at once exciting and a little disturbing. One monk said to me that when he thought of the existence of parallels, it was like a nervous, lurking anxiety: what if we’re wrong? What if the things we take to be true turn out out to be no more than an institutional dogma, or an accidental artefact of history? Some modern schools of interpretation take this doubt as a starting point to dismantle the very idea that we can know what the Buddha taught, replacing knowledge with destructive nihilism.

Some years later, I met Rod Bucknell, from whom, during our discussions when starting SuttaCentral, I learned of a different approach. The Taiwanese monk Master Yin Shun (\langlzh{印順}) had developed a powerful theory of the origins and shared teachings in Buddhism. His insight was based on a comparative reading of all the texts (he read the Pali canon mainly from the Japanese translation), and was sparked by one of the great Mahayana treatises, Asanga’s \textsanskrit{Yogacārabhūmi}. 

Yin Shun posited that the \textsanskrit{Saṁyutta} was the first and primary collection of texts in Buddhism. By this he meant the original \textsanskrit{Saṁyutta}, not the developed forms we have today in the Pali \textsanskrit{Saṁyuttanikāya}, multiple Chinese \textsanskrit{Saṁyuktāgama}, and various portions in Sanskrit and Tibetan. These are, like all extant collections, the outcome of a redaction process which left its discernable fingerprints. Nonetheless, the close relation between all these texts suggests that changes have been minor.

Once you take this idea on board, the signs leap out from everywhere. For example, some accounts of the First Council speak of the \textsanskrit{Saṁyutta} being recited first, and we find that the first three sermons are in the \textsanskrit{Saṁyutta}. Most convincing, however, is the observation that the list of topics found in the \textsanskrit{Saṁyutta} is central to the Buddha’s teaching: the aggregates, the senses, dependent origination, the elements, the path, the four noble truths. When the Suttas list the Buddha’s essential teachings, they do so with topics from the \textsanskrit{Saṁyutta}. And when later generations organized the Buddha’s teachings into coherent wholes, the same list of topics provided a handy scaffold.

This is far from a complete theory of early Buddhist texts; it is a complex situation, and there are many factors at play. But Yin Shun’s fundamental thesis offered a compelling framework to make sense of the vast mass of texts and their interrelations.

The \textsanskrit{Saṁyutta} theory suggests a simple guideline for interpreting Suttas: look for the main understanding of key teachings in the simple Suttas of the \textsanskrit{Saṁyutta}, and see other discourses, especially those of the Majjhima and \textsanskrit{Dīgha}, as being built upon these foundations. The theory doesn’t mean that all \textsanskrit{Saṁyutta} discourses are earlier than others. It simply means that they were organized in this collection before other collections. It clarifies priority and perspective.

This was a revelation for me, and I pursued this insight in my book \textit{A History of Mindfulness}. And ever since, I have found it to be a reliable means of sorting out what the Buddha really taught.

It was challenging. Over and over again, I had to confront my own expectations and biasses. The anxieties of my friend turned out to be not completely baseless. I found that I could no longer believe in Theravada as the one true, original sources of Buddhism: it was, rather, one of many schools, and like all schools it preserved much and changed much. I could see the many different flavors of Buddhism—some quite alien to me—while recognizing that underneath them all lie the same fundamental teachings of the \textsanskrit{Saṁyutta}.

At a deeper level, this helped me realize how my own biasses and expectations had been shaping my understanding of my meditation experiences, and hence what I did in meditation, and hence the nature of those experiences. If you think meditation is a certain way, you will meditate in accord, and your experiences will confirm your ideas. And if you tell yourself, “This is my personal experience, and has nothing to do with theory”, then you’ll never find a way out of the cycle. For that, we need critical inquiry based on the best available facts. 

The truth was literally right in front of my nose all along, yet for years I did not see it. I was practicing “mindfulness” in the belief that this was the way of insight (\textit{\textsanskrit{vipassanā}}). But along the way, I found that mindfulness led to serenity (\textit{samatha}). This wasn’t what I had learned to expect from the \textsanskrit{Satipaṭṭhāna} Sutta as explained by modern teachers, for whom mindfulness and insight are intrinsically linked. 

Studying the \textsanskrit{Saṁyutta} closely I noticed that the teachings on insight—hundreds of Suttas—almost never mention mindfulness. And the teachings on mindfulness emphasize how it leads to serenity. It is serenity—the deep, immersive peace of mind called \textit{\textsanskrit{samādhi}} or \textit{\textsanskrit{jhāna}}—which then leads to liberating insight. 

Suddenly my own experiences made sense. And so I changed the way I approached meditation in all sorts of subtle ways. I had thought of meditation as “noting” various “objects” with “momentary concentration” that would give rise to “dry insight”. But I realized that none of these words or ideas was found in the Suttas at all. It simply isn’t how the Buddha taught. Rather, he spoke of breathing mindfully, of the natural process of settling the mind, of how when one is informed by right view, wisdom emerges from a mind at peace.

This is my experience, and yours is different. It is not that the traditions, schools, and methods are wrong. They are fine for what they are, but what they are not is carbon copies of the Buddha’s teachings. Think of them as degrees of approximation. It is in the Suttas, and especially the \textsanskrit{Saṁyutta}, that we find the closest thing to the Buddha’s words. I found that those words matched my experiences and informed my practice in ways that the schools and methods did not. And I am always grateful for those teachers who have made it possible for to learn from the greatest teacher of them all.

%
\chapter*{The Linked Discourses: the blueprint for Buddhist philosophy}
\addcontentsline{toc}{chapter}{The Linked Discourses: the blueprint for Buddhist philosophy}
\markboth{The Linked Discourses: the blueprint for Buddhist philosophy}{The Linked Discourses: the blueprint for Buddhist philosophy}

\scbyline{Bhikkhu Sujato, 2019}

The \textsanskrit{Saṁyutta} \textsanskrit{Nikāya} is the third of the four main divisions in the Sutta \textsanskrit{Piṭaka} of the Pali Canon (\textit{\textsanskrit{tipiṭaka}}). It is translated here as \textit{Linked Discourses}, and has been previously translated as the \textit{Connected Discourses} or the \textit{Kindred Sayings}. As the title suggests, its discourses are grouped thematically. These thematic groups of texts are called \textit{\textsanskrit{saṁyuttas}}, and the \textsanskrit{Saṁyutta} \textsanskrit{Nikāya} is the collection of such \textit{\textsanskrit{saṁyuttas}}.

The \textsanskrit{Saṁyutta} \textsanskrit{Nikāya} consists of 56 \textit{\textsanskrit{saṁyuttas}} collected in five large “books” (\textit{vagga}), containing over a thousand discourses. The \textsanskrit{Mahāsaṅgīti} text as used on SuttaCentral contains 2837 discourses; but the total number is somewhat arbitrary, as it depends on how the abbreviated texts are expanded, which differs in different editions. This variation, however, applies only to the way the texts are counted, and does not affect the content, which is virtually identical in every edition.

It is in the \textsanskrit{Saṁyutta} \textsanskrit{Nikāya} that we find the core doctrines that have formed the basis for all subsequent Buddhist philosophy. It is largely structured around major doctrinal sections that correspond with the template of the four noble truths.

\begin{description}%
\item[Suffering] Aggregates (\href{https://suttacentral.net/sn22}{SN 22})%
\item[The origin of suffering] Six sense media (\href{https://suttacentral.net/sn35}{SN 35})%
\item[The cessation of suffering] Dependent origination (\href{https://suttacentral.net/sn12}{SN 12})%
\item[The practice that leads to the end of suffering] Dependent cessation (\href{https://suttacentral.net/sn12}{SN 12})%
\end{description}

The four noble truths themselves are treated in the final chapter, the Sacca \textsanskrit{Saṁyutta} (\href{https://suttacentral.net/sn56}{SN 56}).

Not all the \textit{\textsanskrit{saṁyuttas}} fit so easily into this scheme. There are many minor \textit{\textsanskrit{saṁyuttas}}, which are sometimes connected with a major \textit{\textsanskrit{saṁyutta}}, and sometimes not. In addition, the first book, the \textsanskrit{Sagāthāvagga}, is not organized by subject. Rather, the thematic linking here is the type of person involved in the discourse. These texts are also unified in literary form; they are in mixed prose and verse.

This collection has a full parallel in the \textsanskrit{Saṁyuktāgama} (SA) of the \textsanskrit{Sarvāstivāda} school in Chinese translation. In addition, there are two partial collections in Chinese (SA-2 and SA-3) as well as a number of miscellaneous or fragmentary texts in Chinese, Sanskrit, and Tibetan. Much of the organizational structure of SN is shared with SA, suggesting that this structure preceded the split between these two collections.

\section*{How the \textsanskrit{Saṁyutta} is Organized}

The \textsanskrit{Saṁyutta} \textsanskrit{Nikāya} is conveniently divided into five large \textit{vaggas} or “books”. As noted in the General Guide, the \textsanskrit{Saṁyutta} is an example of the “nested \textit{vagga}” structure, where the (unusual) \textit{vagga} as “book” includes many of the normal kind of “small” \textit{vagga}, i.e. groups of about ten suttas.

Within each of the five “big” \textit{vaggas} there are several \textit{\textsanskrit{saṁyuttas}}, each containing a set of discourses that are linked by person or theme (sometimes both). For example, each of the discourses in \href{https://suttacentral.net/sn5}{SN 5} features a nun (\textit{\textsanskrit{bhikkhunī}}), while each discourse in \href{https://suttacentral.net/sn24}{SN 24} deals with the subject of “views” (\textit{\textsanskrit{diṭṭhi}}).

In SuttaCentral, the discourses of the \textsanskrit{Saṁyutta} are referenced by \textit{\textsanskrit{saṁyutta}} and \textit{sutta}. Thus \href{https://suttacentral.net/sn1.1}{SN 1.1} is the first discourse of the first \textit{\textsanskrit{saṁyutta}}, while \href{https://suttacentral.net/sn56.11}{SN 56.11} is the eleventh discourse of the fifty-sixth \textit{\textsanskrit{saṁyutta}}. The SuttaCentral system is the same as that used by Bhikkhu Bodhi in his \textit{Connected Discourses of the Buddha} and on Access to Insight.

The five books are named according to various principles:

\begin{itemize}%
\item Vol. 1 \textsanskrit{Sagāthāvagga} contains sets of discourses that contain verses, as indicated by the title.%
\item Vols. 2–4 are each named after the first and largest \textit{\textsanskrit{saṁyutta}} of the book.%
\item Vol. 5 is called the “Great Book” (\textit{\textsanskrit{Mahāvagga}}) due to its size. The Chinese version is called, appropriately, “The Book of the Path” (\textit{Maggavagga}).%
\end{itemize}

In this essay I will give an overview of each of the five books. However, I will not summarize each of the 56 \textit{\textsanskrit{saṁyuttas}}, for that would make it far too long. For such summaries, see the lists of suttas on SuttaCentral, which include explanations of the various structural levels of the \textit{\textsanskrit{saṁyutta}}, as well as individual discourses. Here I will focus more on general questions of content and interpretation.

\section*{The Book With Verses}

The “Book With Verses” (\textsanskrit{Sagāthāvagga}) is divided into eleven \textit{\textsanskrit{saṁyuttas}}, with a total of 271 suttas.

While most of the \textsanskrit{Saṁyutta} is organized around subject matter, here the organizational principle is \emph{people}. Each \textit{\textsanskrit{saṁyutta}} depicts a conversation involving the Buddha or his disciples with a different person or kind of person, such as gods, kings, nuns, or brahmins.

A typical sutta has a bare narrative framework, where someone comes to the Buddha and utters a verse, and the Buddha replies with a better one. In some cases, notably the Sakka \textsanskrit{Saṁyutta}, the narrative element is developed into a lively exchange.

\subsection*{Verse \& Prose}

Each of the suttas in this collection contains verse with a prose narrative background, although in many cases the prose has been omitted through abbreviation. This kind of literary form is common in Indic literature, so it is worth spending a little time to understand it.

The oldest Indic literature is the Ṛg Veda, a collection of about 10,600 verses. These were passed down in the oral tradition of the brahmins for thousands of years. One of the keys to accurate transmission of this sacred lore was the use of metre: rhythmic patterns of long and short syllables. Such metres provide a scaffolding that organize words, and hence knowledge, in a form that is as memorable as a song; and indeed, they would have been sung in a simple melody. In this way, the verses become set in a precise and defined form, one that may be preserved and passed down unaltered through the generations.

But poetry is not just technically complex; it is ecstatic, inspired, divine. The brahmins did not see the Vedas as being authored in the normal sense, but channeled as the divine word of god (\textsanskrit{Brahmā}) through human sages (Pali: \textit{isi}; Sanskrit: \textit{\textsanskrit{ṛṣi}}). The Vedic verses constantly allude to stories, myths, and events—for example, the slaying of the dragon \textsanskrit{Vṛtra} by the god-hero Indra—that were well known to their audience, and thus did not require spelling out in the text itself. The verses are in fact hymns, invoked in ritual to heighten the emotional response, to inspire awe, fear, or devotion. They are given meaning and context by the background understanding of the mythology. Thus the verses imply a story, of which they are the emotional and narrative climax.

So we can think of a verse as the seed crystal around which a more flexible prose narrative grows and evolves. The prose may be adjusted to time and place, presented in greater or lesser detail, or adapted for the audience. It may comment on contemporary events or express a personal perspective; but the verse is (in theory) always the same.

We are speaking here of the verses found in the \textsanskrit{Sagāthāvagga}. But it is worth bearing in mind that there are plenty of verses in the \textit{\textsanskrit{nikāyas}} outside the \textsanskrit{Sagāthāvagga}, and they are not all of the same type. Here is a brief summary of the main verse types you will encounter. This is just to help a reader get a rough orientation, and exceptions and blurred lines are easily found.

\begin{description}%
\item[Climactic verse] As in the \textsanskrit{Sagāthāvagga}, such verses appear at the climax of a narrative. The narrative may be very thin, or even absent, yet it is always assumed. Sometimes it is supplied in later commentaries. This form is used outside the \textsanskrit{Sagāthāvagga} in such texts as the Dhammapada, \textsanskrit{Udāna}, and \textsanskrit{Jātakas}. We might also consider under this head longer devotional verses such as those of Sela (\href{https://suttacentral.net/mn92}{MN 92}, \href{https://suttacentral.net/snp3.7}{Snp 3.7}).%
\item[Independent poems] A set of verses that makes up a unified literary and thematic whole, and is independent of a prose narrative. There are relatively few of these in the four \textit{\textsanskrit{nikāyas}}, but they dominate the Sutta \textsanskrit{Nipāta}. The last \textit{vagga} of that book contains a series of such independent poems, all united within a narrative set in verse. Some of the verses of the \textsanskrit{Sagāthāvagga} might be considered under this head, if the prose narrative is dismissed as negligible.%
\item[Devotional invocations] Such texts as the \textsanskrit{Mahāsamaya} Sutta (\href{https://suttacentral.net/dn20}{DN 20}), the \textsanskrit{Āṭāṇātiya} Sutta (\href{https://suttacentral.net/dn32}{DN 32}), or the Isigili Sutta (\href{https://suttacentral.net/mn116}{MN 116}) occupy an unusual place in the early Buddhist corpus. Thin in their doctrinal content, they appear more as incantations for protection or blessings.%
\item[Summary verse] Like the climactic verses, these accompany prose. But rather than being an emotional highlight, they serve as a mnemonic device to help preserve the content of the prose. These are most familiar in the \textit{\textsanskrit{uddānas}} that appear at the end of \textit{vaggas} and other sections throughout the EBTs, typically listing a keyword from each text and thus acting as a kind of table of contents. These are not to be confused with the genre of climactic verse known as \textit{\textsanskrit{udāna}}, “inspired saying”, which, despite the similar spelling, is a completely different word. In addition to the formal \textit{\textsanskrit{uddānas}}, we can consider under this head many of the verses of the \textsanskrit{Aṅguttara}, especially in the Fours, which often serve purely to summarize the content of the prose, although occasionally they are developed into a more satisfying poetic reflection on the theme. Occasionally a longer sutta will contain mixed portions of prose and summary verse, notably the \textsanskrit{Dvayatānupassanā} Sutta (\href{https://suttacentral.net/snp3.12}{Snp 3.12}). A much later development of this style is found in the \textsanskrit{Lakkhaṇa} Sutta (\href{https://suttacentral.net/dn30}{DN 30}).%
\end{description}

In my translations, I have rendered verse as prose broken into lines, rarely attempting poetic virtue. To render these highly didactic verses, dense with doctrinal terms, into genuine English verse is no easy task. In many cases, especially with the summary verses, the text in Pali has little in the way of literary merit. Other texts, especially the later verses, display learned command of complex and sophisticated literary forms such as is rare to find, even among writers of English poetry. Combined with the often obscure vocabulary, rare and archaic grammatical forms, and syntactic flexibility of Pali verses, the task of rendering them in readable and accurate English is hard and time-consuming, even without aspiring to poetic beauty. So my verse is workmanlike, and I can only hope that poets take up the task of rendering selected verses with the beauty they deserve.

\subsection*{The Play of the Gods}

In the Book With Verses we see the ancient Vedic pattern adopted to serve a Buddhist purpose. It is no coincidence that here we meet various deities, many of whom hail from Vedic mythology, in contexts that sometimes directly respond to specific Vedic or brahmanical passages.

The casual appearance of deities throughout these texts is, of course, problematic. These days, we don’t normally see gods manifesting with glorious light at spiritual gatherings. So how are we to understand this?

One obvious answer is that such texts are literally true: gods of these names did appear in exactly the way depicted and have these exact conversations. If so, why are such things not seen in our day? One might be tempted to point regretfully to the decline of religious and ethical life in modern times. But this is just another unverifiable claim: how could we possibly know such a thing? And it creates an even bigger problem. For when we see the past as a uniquely privileged era, one blessed with a degree of purity and wholesomeness that is lost to us, then what is the point of practice? Are we not better off pining for the glories of old, and wishing for the renewal of the Dhamma under the future Buddha Metteyya? Such views forget a basic principle of the teaching: it is \textit{\textsanskrit{akāliko}}—we can realize it here \& now, no matter when we live.

So perhaps we are better off adopting a skeptical view: such deities do not exist, and such events did not happen. They are simply religious propaganda, fictions whose purpose is to convert simple people by importing a familiar Indic cosmology. If there is any reality at all to them, it is purely psychological; such beings represent different aspects of the mind. Despite its scientific appearance, this reductive view, too, is unsustainable. The ideas of rebirth and the existence of multiple dimensions of existence are not found just in popular narratives, but are central to core teachings such as dependent origination and the four noble truths—the second noble truth is precisely “the craving that leads to future rebirth” (\textit{\textsanskrit{yāyaṁ} \textsanskrit{taṇhā} \textsanskrit{ponobbhavikā}}). They can’t be simply written off as an uncritical inheritance from Indian culture.

These views are polar opposites; and like all pairs of opposites, they share more in common than they like to admit. Both of them are concerned with \emph{facts}, with whether these events were true or not. But the texts as we have them are not collections of facts: they are stories. And the significance of a story lies in its meaning. Whether a story is real or not is at best secondary, and often beside the point entirely. It serves to engage an audience, provoking wonder, surprise, awe, or joy.

The Buddhist traditions understood this well, as evidenced by the textual situation. While in some cases the verses and story are tightly linked, it is very common for the same verse to be accompanied by completely different background narratives, or by no background at all. The verses, which convey the essential Dhamma teaching, the core of meaning and emotion, remain the same, while the story varies. To insist on the factuality or otherwise of the story is to miss the point. The story provides a context that brings the teaching in the verses to life for an audience.

Thus the best lens through which to see such texts is neither as history nor as propaganda but as sacred story; that is, as myth. Each of the short suttas tells a story that conveys a timeless spiritual truth in a way that spoke to the audience of that time and place. They take place within a wider mythology that helps people find their place in a vast and unknown cosmos.

As always, the early Buddhist response to the earlier religious traditions is complex and nuanced. And, while it is true that many of the details of both literary form and subject matter are drawn from the Vedas, it is in the differences that the distinctively Buddhist character of the texts shows itself.

In the Vedas, the human agents are merely the transmitters of the sacred word of the gods. Exactly how this happened is unclear, but it probably involved a combination of drugs (\textit{soma} is one of the great Vedic deities), ritual, creative inspiration, traditional lore, devotion, and communal empowerment, all of which inspired the sacred poets to heights of ecstatic visioning through which the words of the gods manifested. But regardless of the details, the key point is that the traditions regarded the human agent in the relationship as incidental, and the real value of the texts as stemming from the divine.

In the Buddhist texts, the situation is reversed. The gods do not inspire human hosts, they speak for themselves. And they are no infallible reserves of Truth; they may be right or wrong, skillful or foolish, just like anyone else. While the magnificence of their presence is emphasized, the ultimate effect is to show the worthlessness of such displays, for the gods are constantly being schooled by the Buddha. The most characteristic form of dialogue is where a god presents an idea that is pretty good, within in a limited, mundane (i.e. Vedic) world view, but which the Buddha elevates to an entirely new level.

It is an elementary axiom of Buddhism that the gods are not metaphysical, in the sense that they do not exist in a separate realm governed by different principles than our own. On the contrary, they are impermanent and suffering, trapped in the cycle of transmigration just like us. It follows from this that they do not have access to any special form of knowledge or wisdom. Buddhists do not look to the gods for teachings; rather, the Buddha is “teacher of gods and humans”.

I have focused on the interactions with divine beings in this collection, as these require the most contextualizing. But not all of the collections feature divine beings. Many of the \textit{\textsanskrit{saṁyuttas}} feature kinds of people familiar from other texts of the time. And even when divine beings are involved, in the majority of cases, the verses themselves do not require a divine setting, as there’s nothing about the gods and their divine dramas in the verses themselves.

Some of the texts in this collection are well known and widely quoted, such as the invitation of \textsanskrit{Brahmā} or the nun \textsanskrit{Vajirā}’s simile of the person as a chariot. Most of the suttas here have parallels in the Chinese \textsanskrit{Saṁyuktāgama} translations; the partial translations SA-2 and SA-3 include \textsanskrit{Sagāthāvagga} material. In addition, many of the verses have parallels elsewhere throughout the Buddhist literature in all languages.

\section*{The Book of Causation}

The Book of Causation (\textsanskrit{Nidānavagga}) is the second of the five books of the Linked Discourses. It is named after the first and longest section, the \textsanskrit{Nidāna} \textsanskrit{Saṁyutta}. This deals with causation through the core Buddhist teaching of dependent origination, which explains how rebirth happens without a soul. The next three \textit{\textsanskrit{saṁyuttas}} can be seen as appendices to the Book of Causation, dealing with the elimination of the suffering of transmigration (\href{https://suttacentral.net/sn13}{SN 13}), various sets of conditioned elements (\href{https://suttacentral.net/sn14}{SN 14}), and the unknowability of the extent of transmigration (\href{https://suttacentral.net/sn15}{SN 15}). The remaining six \textit{\textsanskrit{saṁyuttas}} are not related thematically. Instead, they are mostly grouped by person rather than subject.

The theme of causation runs through all the Buddha’s teaching. We find it in contexts such as meditation practice, societal ills, biological evolution, medicine, psychological stress, and many more. However, when we refer to dependent origination we are not speaking about a general principle of causality—although such a principle is presented at \href{https://suttacentral.net/sn12.21}{SN 12.21}—but of a specific series of conditional links laying bare how suffering originates and how it ends. As such, it is an extended treatment of the second and third noble truths (\href{https://suttacentral.net/sn12.27}{SN 12.27}). It integrates psychological and existential aspects of suffering, showing how, when bound by craving, we make choices that bind us to transmigrating into future lives (\href{https://suttacentral.net/sn12.38}{SN 12.38}). The reason why we have not escaped the process of rebirth is that we do not understand dependent origination (\href{https://suttacentral.net/sn12.60}{SN 12.60}). Thus one of the core purposes of the teaching is to explain how rebirth takes place without speaking in terms of “me” or “mine” (\href{https://suttacentral.net/sn12.37}{SN 12.37}).

It is a deeply human need to want to understand how things came to be. Virtually every religious or spiritual path feels the need to offer some kind of explanation of where this world came from and what is our place in it. Such creation myths are found all over the world, and bear striking resemblances. They speak of a time when the world was formless, covered in a watery darkness, before light appeared and the world took shape. In the usual way of myths, these stories work at multiple levels, reflecting both the physical evolution of the planet (macrocosm) and the growth of an individual in the womb (microcosm).

Long before the Buddha, the Nasadiya Sukta of the Ṛg Veda (10.129) told the story of creation in a radically new way. It drew upon the motifs of the classic creation myth—water, darkness, formlessness—but showed their development with a new emphasis on desire and agency. Creation evolved not from divine decree, but due to the energies found within. And we cannot know what came before this process; even the highest God came afterwards.

The Buddha shared the epistemological humility of the Nasadiya Sukta, insisting that the ultimate origin of things was unknowable (\href{https://suttacentral.net/sn15.1}{SN 15.1}). Dependent origination, indeed, took things much further, entirely dispensing with both theology and mythology. However, it retained the richness and depth of the mythology, encapsulating within its sparse formulation both immediate experience and cosmic transmigration.

The \textsanskrit{Nidāna} \textsanskrit{Saṁyutta} begins by stating (\href{https://suttacentral.net/sn12.1}{SN 12.1}) and defining (\href{https://suttacentral.net/sn12.2}{SN 12.2}) each of the terms in the standard 12-linked chain, definitions which are assumed to apply throughout. Remaining discourses iterate on this theme, introducing new perspectives and formulations. These sometimes vary the standard 12 links, and so can shed light on unexpected nuances and hidden depths. Here’s a summary of the definitions given for the 12 links, together with some explanatory notes.

\begin{description}%
\item[Ignorance (\textit{\textsanskrit{avijjā}})] Not understanding the four noble truths.
This does not, of course, mean ignorance of everyday facts and details. A Buddha or an arahant is not omniscient.%
\item[Choices (\textit{\textsanskrit{saṅkhārā}})] Intentional acts (\textit{kamma}) of good or bad, which are expressed through body, speech, or mind.
The Indic term \emph{\textsanskrit{saṅkhāra}} may refer to any kind of activity that generates a result. It is used in a mundane context for such things as construction work. The Vedic ritual is a \emph{\textsanskrit{saṅkhāra}}, which was intended to produce a result of benefit in this life or the next. In Buddhism, \emph{\textsanskrit{saṅkhāra}} is sometimes used in a general sense of “conditioned (and conditioning) phenomena”. In dependent origination, however, it is defined as moral choices or intentions to do good or bad (\href{https://suttacentral.net/sn12.51}{SN 12.51}). A \emph{\textsanskrit{saṅkhāra}} is a force or energy in the mind that propels consciousness towards rebirth in a particular state. This may be consciously formulated as a wish or aspiration (see \href{https://suttacentral.net/mn120}{MN 120}), but is normally unconscious, i.e. born out of ignorance.%
\item[Consciousness (\textit{\textsanskrit{viññāṇa}})] The six kinds of sense consciousness.
In the suttas, all forms of consciousness are regarded as making up the “stream of consciousness” (\textit{\textsanskrit{viññāṇasota}}) that is established both in this life and the next (\href{https://suttacentral.net/dn28}{DN 28}). This consciousness is empowered and directed by the choices that we have made.%
\item[Name \& form (\textit{\textsanskrit{nāma}-\textsanskrit{rūpa}})] “Name” is various mental factors (feeling, perception, attention, contact, and intention), while “form” is the four material elements that make up the body.
This is a tricky concept. It stems from \textsanskrit{Upaniṣadic} usage, where it refers to the various individuated entities in the world, each with their own “form” and “name”. Each of the rivers on the earth, to take a metaphor from the Prasna \textsanskrit{Upaniṣad} (6.5), has its own individual shape, and is called by its own name; but when they return to the ocean they lose their names and shapes and are just known as the great ocean. The ocean in this metaphor stands for consciousness, which in the \textsanskrit{Upaniṣads} is taken to be the eternal and infinite divinity of the cosmos. The Buddha directly rebuts this idea by showing that consciousness and name \& form are dependent on each other. In \href{https://suttacentral.net/dn15}{DN 15} \textit{The Great Discourse on Causation} (\textit{\textsanskrit{Mahānidānasutta}}), name \& form is described as the embryo taking shape within the mother’s womb, while \href{https://suttacentral.net/mn38}{MN 38} \textit{The Longer Discourse on the Ending of Craving} (\textit{\textsanskrit{Mahātaṇhāsaṅkhaya}}) speaks of how the child then grows and matures. Thus it primarily means the individual organism with its mental and physical attributes. Since \textit{\textsanskrit{nāma}-\textsanskrit{rūpa}} stands in mutual dependence with consciousness, however, it is not correct to translate it as “mind \& body”—mind/body dualism has no place in early Buddhism. It was only in much later Abhidhamma texts that \textit{\textsanskrit{nāma}-\textsanskrit{rūpa}} came to be used as an umbrella term for all mental and physical properties, in which context the translation “mind \& body” is appropriate.%
\item[Six sense fields (\textit{\textsanskrit{saḷāyatana}})] The eye, ear, nose, tongue, body, and mind.
These are treated in detail in \href{https://suttacentral.net/sn35}{SN 35} \textsanskrit{Saḷāyatana} \textsanskrit{Saṁyutta}. In dependent origination they are said to develop and evolve as the individual grows up, enabling them to experience the world in ever more sophisticated ways. This is the first of four links that, like consciousness, are six-fold following the six senses. These integrate the process of immediate sense experience within the broader context of dependent origination.%
\item[Contact (\textit{phassa})] This is the operation of sense stimulus, when the six sense organs are activated and perform their function. It occurs with the coming together of the inner sense organ, the outer sense stimulus, and the corresponding consciousness.
The conscious individual does not exist in isolation, but can only live and grow in an environment that provides them with stimulation.%
\item[Feeling (\textit{\textsanskrit{vedanā}})] The pleasant, painful, or neutral tone that accompanies all conscious experience.
Certain kinds of experience are enjoyable, others are unpleasant, while some have no particular affect. Note that \textit{\textsanskrit{vedanā}} in Buddhism does not refer to feelings in the sense of “emotions”, which are far more complex. \textit{\textsanskrit{Vedanā}} is a fundamental property of all experience, and is treated in detail in \href{https://suttacentral.net/sn36}{SN 36} Vedana \textsanskrit{Saṁyutta}.%
\item[Craving (\textit{\textsanskrit{taṇhā}})] Craving or desire for the six sense stimuli.
Here the definition follows the theme of the six senses, rather than the definition given in the four noble truths, which is craving for future rebirth, i.e. sensual craving, craving to continue existence, and craving to end existence. In both cases, craving refers to a primal instinct that responds to sense stimulus, seeking to get more pleasure or to avoid pain.%
\item[Grasping (\textit{\textsanskrit{upādāna}})] Grasping at sensual pleasures, views, religious observances, and theories of self.
Apart from the first factor, the kinds of grasping are more sophisticated than the primal desires of “craving”. They require the development of thought and language. This represents a further stage in the growth of a person, as they mature and become fully responsible for their actions. It is for this reason that in Buddhism it is primarily mature humans who perform the deeds that generate rebirth and shape the course of future lives. Animals or children may indeed perform such deeds, but they are less weighty in effect.%
\item[Continued existence (\textit{bhava})] Existence may be in the sensual realm (\textit{\textsanskrit{kāma}-bhava}), the realm of luminous form (\textit{\textsanskrit{rūpa}-bhava}), or the formless realm (\textit{\textsanskrit{arūpa}-bhava}).
\textit{Bhava} may be translated as “existence” or “life”. It refers to the ongoing process of existence. By grasping at various aspects of the present life, beings generate kammic energy in accordance with that. Most beings are attached to the sensual realm, but those who have practiced advanced meditation may become attached to the realms of luminous form (through the four absorptions) or the formless (through the formless attainments). Such attachment or grasping stimulates and activates these aspects of existence, creating a corresponding rebirth. \textit{Bhava} is therefore like a thread that runs through the various steps of dependent origination; and indeed, the whole of dependent origination is sometimes called the \textit{bhavacakka}, the “Wheel of Existence”. \textit{Bhava} is a countable noun, so the older rendering as “becoming” is incorrect: you can’t speak of multiple “becomings”. Nevertheless, \textit{bhava} has a distinctly pregnant sense. While we might long for a life of stable and eternal joy, it is the nature of existence that, even as it passes away, it contains the seeds for a new life in the future. Thus in \href{https://suttacentral.net/an3.76}{AN 3.76} the Buddha explains \textit{bhava} by saying that “deeds are the field, consciousness is the seed, and craving is the moisture” (\textit{\textsanskrit{kammaṁ} \textsanskrit{khettaṁ}, \textsanskrit{viññāṇaṁ} \textsanskrit{bījaṁ}, \textsanskrit{taṇhā} sneho}) for generating a new life in the future (\textit{\textsanskrit{āyatiṁ} \textsanskrit{punabbhavābhinibbatti}}).%
\item[Rebirth (\textit{\textsanskrit{jāti}})] The rebirth or conception of the aggregates in the various orders of sentient beings.
The Pali \textit{\textsanskrit{jāti}} is often translated as “birth”, but in doctrinal contexts it always refers to rebirth in the sense of reincarnation in a new life. Despite the claims of some modern commentators, the suttas never use rebirth as a psychological metaphor.%
\item[Old age and death, sorrow, lamentation, pain, sadness, and distress] Old age is the breaking of teeth, wrinkling of skin, and failing of the faculties. Death is the laying down of the body at the end of life.
Like rebirth, old age and death are defined in purely physical terms. Psychological suffering is covered by the other terms.%
\end{description}

Dependent origination is core to the insight not just of the historical Buddha Gotama, but of other Buddhas of the past (\href{https://suttacentral.net/sn12.4}{SN 12.4–10}). His own realization was like a person who stumbles upon an ancient city, lost and overgrown in the jungle (\href{https://suttacentral.net/sn12.65}{SN 12.65}). Thus dependent origination is not an invention of the Buddha, but is a description of a natural principle, one that operates in the same way whether or not a Buddha appears (\href{https://suttacentral.net/sn12.20}{SN 12.20}).

The Buddha distinguished between the phenomena that are dependently originated and the process of dependent origination (\href{https://suttacentral.net/sn12.20}{SN 12.20}). While all of the dependently originated phemomena can be seen in the present, the reality of the process in the past and future must be inferred (\href{https://suttacentral.net/sn12.34}{SN 12.34}).

This is the difference between knowledge of dependent origination and the psychic powers of seeing the specific details of past lives and present rebirths. Such psychic visions arise from deep immersion in meditation, and are useful but not essential for understanding and letting go. Dependent origination is not about seeing the specifics of past and future lives, but understanding the principles by which rebirth operates. If memories of past lives are like watching a show on TV, insight into dependent origination is like understanding the science and technology of how television transmission works. This is why dependent origination is always regarded as unique and central to Buddhism, while psychic visions have only a limited role to play.

This conscious body, with its complex systems of mental and physical processes, was produced due to ignorance and craving in a past life (\href{https://suttacentral.net/sn12.19}{SN 12.19}). This is how our present existence came to be. So long as we continue to be trapped in craving for sensory experience, we repeat the cycle, fueling craving and generating yet another new body that will be reborn in the future. When we understand the theory of dependent origination it gives us an opening for developing insight.

It is, of course, self evident that all the factors of dependent origination may be observed in the present. But in terms of the functioning of the process, we begin by examining the central factors, the unfolding of sense experience. Gradually we realize that the implications of what we are seeing are far deeper than the mere present. Like a scientist who, examining tree rings or ice cores, realizes that they can make reliable inferences about the deep past, we understand that the same processes that brought us here will propel us into the future. And we see that it is that very understanding which is the beginning of the end for ignorance.

Thus while this teaching is a profoundly philosophical one, it is not mere theory, but a praxis as well (\href{https://suttacentral.net/sn12.3}{SN 12.3}). Its understanding unfolds as a natural consequence of developing faith in the Dhamma (\href{https://suttacentral.net/sn12.23}{SN 12.23}). It is realized by one who has experiential realization of the Dhamma, commonly known as a stream-enterer (\href{https://suttacentral.net/sn12.28}{SN 12.28}). This is true for both monastics (\href{https://suttacentral.net/sn12.29}{SN 12.29}) and lay people (\href{https://suttacentral.net/sn12.41}{SN 12.41}). This is why a noble disciple has no doubts about the meaning or origin of life: they have seen it for themselves (\href{https://suttacentral.net/sn12.49}{SN 12.49}). Such an individual is independent of others and need not rely on a teacher. One need not be a perfected one (\textit{arahant}) to understand dependent origination (\href{https://suttacentral.net/sn12.68}{SN 12.68}).

The Buddha illustrates dependent origination with many similes, the imagery often drawn from ideas of fire, fuel, or food. Focusing on things that give pleasure tends to stimulate craving, like a fire fueled by dry grass (\href{https://suttacentral.net/sn12.52}{SN 12.52}) or a tree drawing up sap (\href{https://suttacentral.net/sn12.58}{SN 12.58}). It is in this collection that we find the now-classic simile of the mind like a monkey, although the application is somewhat unexpected (\href{https://suttacentral.net/sn12.61}{SN 12.61}).

A number of suttas take up the idea of conditionality as “food” or “fuel” or “sustenance” and apply it to a set of four things: solid food, contact, intention, and consciousness (\href{https://suttacentral.net/sn12.11}{SN 12.11}). Each of these is, in turn, illustrated with similes that are as horrifying as they are unforgettable (\href{https://suttacentral.net/sn12.63}{SN 12.63}).

While the standard presentation of the 12 links might give the impression that they occur one after the other, like a series of dominoes falling down, the reality is more complex. Certain factors, such as consciousness and name \& form, are interdependent, relying on each other (\href{https://suttacentral.net/sn12.65}{SN 12.65}) like two sheaves of reeds (\href{https://suttacentral.net/sn12.67}{SN 12.67}). Sometimes a factor might be implied rather than mentioned outright (\href{https://suttacentral.net/sn12.13}{SN 12.13}), sometimes the sequence is altered (\href{https://suttacentral.net/sn12.43}{SN 12.43} and \href{https://suttacentral.net/sn12.44}{SN 12.44}), while elsewhere the teaching might be presented in a quite different way (\href{https://suttacentral.net/sn12.38}{SN 12.38}). Conditionality in Buddhism is always understood to be complex and ramified: many causes, many effects, all interacting. The simple schema is not meant to be reductive, but to clarify crucial aspects of the process in a way that is easily memorized and understood.

Philosophically, dependent origination is said to be the “middle teaching” that avoids extreme views. Such extreme views are the polar opposites that often define philosophical positions. They include views of moral responsibility: is the person who does the deed identical with the one who experiences the result? Or is it experienced by a different person? The Buddha rejects these alternatives: the deed has an effect, and that effect is suffering (\href{https://suttacentral.net/sn12.46}{SN 12.46}).

Similarly the Buddha rejects the ideas that “everything exists” or that “nothing exists” (\href{https://suttacentral.net/sn12.48}{SN 12.48}). While these notions might seem odd, even bizarre, to our way of thinking, they stem from the Indian philosophical tendency to see “being” as inherent, absolute, even divine. Thus if something exists, it exists in an absolute and essential sense, and if all exists, it means the reality of the cosmos itself is absolute and eternal. If nothing “exists”, it does not mean that there is nothing real in the world; it means that things do not have any essence, and so will perish. Thus the Buddha rejects these opposing views as forms of eternalism and annihilationism, the ideas that the self will last forever, and that the self will be destroyed (\href{https://suttacentral.net/sn12.15}{SN 12.15}).

Finally, perhaps the single most important thing to remember is that dependent “origination” is only half the picture. Of equal importance is “cessation”, the ending of each of the factors, which is what is called “extinguishment” (\textit{\textsanskrit{nibbāna}}). The true purpose of studying dependent origination is not for philosophical mastery, not for winning debates or passing a course, but for realizing the ending of suffering. Dependent origination stands as a truly empowering teaching, as it assumes that human understanding is quite sufficient to comprehend the essence of existence itself, to find salvation through wisdom.

\section*{The Book of the Aggregates}

The “Book of the Aggregates” is the third of the five books of the Linked Discourses. It is named after the first and longest \textit{\textsanskrit{saṁyutta}}, which deals with the core Buddhist teaching of the five aggregates in 159 discourses. Of the remaining twelve \textit{\textsanskrit{saṁyuttas}}, three also take up the theme of the aggregates, while the remainder deal with miscellaneous secondary themes, some organized by subject, others by person.

The “five grasping aggregates” (\textit{\textsanskrit{pañc}’\textsanskrit{upādānakkhandhā}}) were mentioned in the first sermon as the summary of the noble truth of suffering (\href{https://suttacentral.net/sn56.11}{SN 56.11}), and became a foundational teaching in all forms of Buddhism.

The basic idea of an “aggregate” (\textit{khandha}) is a set or class of phenomena. The “five aggregates” are the various sets of phenomena so classified.

The five aggregates are almost always said to be “grasping” aggregates. The term “grasping” (\textit{\textsanskrit{upādāna}}) has a complex and multi-layered relation to the basic term.

\begin{itemize}%
\item The aggregates are the \emph{subject} of grasping, in that they are the things that are normally attached to and taken to be the permanent “self”.%
\item But they are not merely passive spectators: they are also the \emph{functional support} of grasping, the things that make grasping work. This is probably the basic metaphor of the set, as the five aggregates correspond to the five fingers of a hand, which perform the act of grasping. Grasping is itself something that the aggregates do. In this metaphor, the “thumb” is consciousness, which stands against the other four.%
\item As active participants in the process of grasping, they \emph{stimulate} grasping to themselves (\textit{\textsanskrit{upādāniya}}).%
\item And finally, they are the \emph{product} of grasping in the sense that attachments in past lives have given rise to the aggregates in this life (\textit{\textsanskrit{upādiṇṇa}}).%
\end{itemize}

Here is a brief analysis of each of the five.

\begin{description}%
\item[Form (\textit{\textsanskrit{rūpa}})] “Physical phenomena”, or sometimes simply “body”, understood as consisting of the four primary physical properties: earth (solid), water (liquidity), fire (heat), and air (gas), and anything material derived from these, such as the impressions of the five material senses.
\textit{\textsanskrit{Rūpa}} is more extensive in scope than the Western concept of “matter”. It includes material properties that are perceived purely in the mind, such as shape or color seen as visions in meditation.%
\item[Feeling (\textit{\textsanskrit{vedanā}})] The pleasant, painful, or neutral tone of experience born from the six senses.%
\item[Perception (\textit{\textsanskrit{saññā}})] The recognition or interpretation of experience through the six senses.
Perception refers to that function of the mind which organizes the dizzyingly complex and chaotic input of present experience based on past experience. The eye, for example, does not see “blue” or “yellow”, it only sees light in various frequencies and amplitudes. Perception recognizes that these inputs correspond to the concept “blue” or “yellow” (\href{https://suttacentral.net/sn22.79}{SN 22.79}), and so it enables us to live in a world of (relatively) permanent and predictable entities and ideas. While perception thus makes consciousness possible, it also can trap us into seeing things only in terms of the past. In the legal discussions of the Vinaya, it is common to discuss cases where a mendicant’s actions are based on a perception that turns out to be incorrect.%
\item[Choices (\textit{\textsanskrit{saṅkhārā}})] Intention, will, or volition (\textit{\textsanskrit{cetanā}}); the choice to perform an act, especially one with an ethical dimension. It is choices that create the five aggregates (\href{https://suttacentral.net/sn22.79}{SN 22.79}).
Later forms of Buddhism, starting with the Abhidhamma texts, treated this aggregate as if it were a catch-all, whose purpose was to include everything not mentioned under the other aggregates. However this is not the case in the early texts, where there is no indication that \textit{\textsanskrit{saṅkhārā}} in this context means anything other than “volition, choice”.%
\item[Consciousness (\textit{\textsanskrit{viññāṇa}})] The subjective process of awareness itself.
As in dependent origination, consciousness is said to depend on name \& form.%
\end{description}

With the exception of perception, all of these are also found in dependent origination, where they have similar definitions. Whereas dependent origination shows the unfolding of the process of suffering in time, the teaching on the aggregates focuses on those aspects of present experience that are most apt to be taken as a self. In \href{https://suttacentral.net/sn22.5}{SN 22.5} the grasping to the aggregates is shown as the very same grasping that leads to rebirth as shown in dependent origination. \href{https://suttacentral.net/sn22.54}{SN 22.54} furthers this argument, asserting that it is impossible to speak of rebirth without referring to the aggregates.

From the very first teaching of the Buddha (\href{https://suttacentral.net/sn56.11}{SN 56.11}) we learn that the aggregates are suffering. In the second sermon—the Discourse on Not-Self (\textsanskrit{Anattalakkhaṇa} Sutta) at \href{https://suttacentral.net/sn22.59}{SN 22.59}—this brief statement is drawn out in a further dialogue with the group of five ascetics. Each of the aggregates leads to affliction and one cannot simply decree that the aggregates be whatever one wants; hence they cannot be a self. Further, each of the aggregates is impermanent and therefore suffering, which again rules out the possibility that they are a self. Seeing in this way, a practitioner lets go of attachment to the aggregates and realizes freedom. It was while listening to this discourse that the five ascetics all became perfected ones.

While the doctrine of the “three marks” is found throughout all Buddhist texts, it is here in the Khandha \textsanskrit{Saṁyutta} that it rises to prominence. Here is a brief outline.

\begin{description}%
\item[Impermanence (\textit{anicca})] All conditioned phenomena are produced and maintained by causes and hence can only last so long as the causes sustain them. Impermanence is a fractal phenomena; it is how reality is structured at every level. It applies equally to the grandest scale of universes and the lifespans of the gods as it does to the incessant breaking up and vanishing of conditions from moment to moment. But in the five aggregates the main focus is on the scale of human existence, where the emotional impact of impermanence is felt most keenly in death.%
\item[Suffering (\textit{dukkha})] At its simplest level this refers to painful feelings, whether physical or mental (\textit{dukkha-\textsanskrit{dukkhatā}}). By itself this is a profound observation, as virtually every moment of our waking lives is afflicted by some form of pain or irritation. But suffering runs deeper than that, for even when we do experience pleasure, it cannot be sustained. The second bite of a mango is delicious—but not quite as delicious as the first (\textit{\textsanskrit{vipariṇāma}-\textsanskrit{dukkhatā}}). Finally, even the most profound of pleasures, such as the bliss of deep meditation, is never as peaceful as \textit{\textsanskrit{nibbāna}}, since by its nature it is conditioned and unstable (\textit{\textsanskrit{saṅkhāradukkhatā}}). No experience is as peaceful as cessation.%
\item[Not-self (\textit{\textsanskrit{anattā}})] The most subtle and distinctive of the three marks, not-self is the most apt to be misunderstood. It is primarily an anti-metaphysical doctrine, not a psychological one. It is intended to rule out the various kinds of self or soul proposed by the philosophers at the time of the Buddha. It does this by pointing out that all the self doctrines end up identifying one or other of the aggregates as self; but they do not have the nature that the self is supposed to have. This meaning is quite different from the modern psychological notion of self, and it is inappropriate, and potentially harmful, to apply the teaching of not-self in cases where a person is suffering from a disorder of identity.%
\end{description}

Let us dwell a little further on the idea of self and not-self, which is best understood in its historical context. From a few centuries before the Buddha, Indian sages and philosophers had become fascinated by the subjective nature of experience. They wondered who it was, in the true and ultimate sense, that was the one referred to as “I”.

Initial theories built on simple animist notions, imaging the self as an external physical totem, or even as a little man who lived in the chest. Others theorized that the self was the heart, or the breath, or some other physical attribute. But all of these may be refuted by simple empirical observation. Sometimes a totem may be destroyed, yet a person lives. When you watch a person who is asleep, no little man may be observed leaving by the mouth. And when a trumpeter expels all their breath, they do not drop down dead.

So what then is this self if not something material? Perhaps, rather, it is feeling, the bliss experienced by one who goes to a beautiful realm after death. But this cannot be so, for feeling, too, is impermanent (\href{https://suttacentral.net/dn15}{DN 15}). Then could the self be perception (\href{https://suttacentral.net/dn1}{DN 1})? But no, perception too is tricky and unreliable, like an illusion. Is self then one’s choices? A man, after all, is defined by the decisions he makes. But these too are seen to be impermanent and unreliable; oft-times one makes bad choices, or the results of a choice are not what one hopes.

Unsatisfied, the sages of the \textsanskrit{Upaniṣads} rejected all such limited conceptions of the Self (\textsanskrit{Bṛhadāraṇyaka} \textsanskrit{Upaniṣad} 3.9.26: \textit{neti! neti!}). They arrived at their most profound thesis: the self in its highest sense was awareness itself, the sheer mass of consciousness (\textsanskrit{Bṛhadāraṇyaka} \textsanskrit{Upaniṣad} 2.4.12: \textit{\textsanskrit{vijnāna}-ghanam’eva}). The true nature of the self is the supreme divinity (\textsanskrit{Bṛhadāraṇyaka} \textsanskrit{Upaniṣad} 1.4.10: \textit{\textsanskrit{ahaṁ} \textsanskrit{brahmāsmīti}}; cp. \href{https://suttacentral.net/dn1\#2}{DN 1:2}, \href{https://suttacentral.net/dn11\#81}{DN 11:81}, \href{https://suttacentral.net/dn24\#2}{DN 24:2}: \textit{ahamasmi \textsanskrit{brahmā}}). This insight is expressed in the \textsanskrit{Upaniṣads} as the famous “thou art that” (\textsanskrit{Chāndogya} \textsanskrit{Upaniṣad} 6.8.7: \textit{tat tvam asi}), and the Pali texts as “I am that” ( \href{https://suttacentral.net/sn22.8}{SN 22.8}: \textit{eso hamasmi}). “That” may be anything one identifies as self. But to one who understands rightly (\textit{ya \textsanskrit{evaṁ} veda}) the divine self is nothing less than the entirety of the universe: “the self is identical with the cosmos” (\href{https://suttacentral.net/sn22.81}{SN 22.81}: \textit{so \textsanskrit{attā} so loko}, cp. \textsanskrit{Bṛhadāraṇyaka} \textsanskrit{Upaniṣad} 1.2.7: \textit{tasyeme \textsanskrit{lokā} \textsanskrit{ātmānaḥ}}, 4.5.7: \textit{\textsanskrit{idaṁ} \textsanskrit{brahmedaṁ} \textsanskrit{kṣatram} ime \textsanskrit{lokā} ime \textsanskrit{devā} ime \textsanskrit{vedā} \textsanskrit{imāni} \textsanskrit{bhūtānīdaṁ} \textsanskrit{sarvaṁ} yad ayam \textsanskrit{ātmā}}). This philosophy is most closely associated with Yajnavalkya, a brahmanical sage who lived in the same region as the Buddha (Mithila), perhaps a century or two earlier.

While the exact form of these arguments may seem archaic, we still cling to the aggregates in similar ways. We think of our possessions and belongings—homes, clothes, cars—as expressions of our self, and are upset when they are damaged or criticized. We attach, too, to our physical bodies, reveling in health, or imagining that we will survive through the propagation of our DNA. We attach to pleasure, thinking that happiness will last. We attach to our perceptions, such as our sense of belonging to a nation or religion, or our idea of ourselves as a good person. We attach to our choices, taking pride in our ability to make decisions. Finally, we attach to our consciousness, especially as we purify awareness in meditation.

Thus one of the key functions of the aggregates was to categorize theories of the self, moving from simple to profound. This seems to have been familiar to philosophers before the Buddha. The aggregates are mentioned in passing in the first sermon, as if it is taken for granted that the five ascetics would know them. Many of the sectarian views of self in \href{https://suttacentral.net/dn1}{DN 1} Brahmajala Sutta refer to the aggregates in one way or another. And elsewhere, the non-Buddhist ascetic Saccaka asserted that the five aggregates were the self (\href{https://suttacentral.net/mn35}{MN 35}). Nevertheless, the aggregates have not been identified in any pre-Buddhist texts.

Regardless of whether the set of categories was pre-Buddhist, the Buddha treated them in his own distinctive way, emphasizing that when examined, the aggregates turn out to lack the qualities of permanence, surety, and refuge that are intrinsic to the idea of a true self. But our grasping and identification are strong and have been built up over a long time, so it is not enough to merely acknowledge this on an intellectual level. Hence in the Khandha \textsanskrit{Saṁyutta} we find the core teachings emphasized again and again. The Buddha constantly reminds the mendicants that the aggregates lead to sorrow and despair (\href{https://suttacentral.net/sn22.7}{SN 22.7}), that they are aggravating (\href{https://suttacentral.net/sn22.79}{SN 22.79}), that desire for them must be given up (\href{https://suttacentral.net/sn22.137}{SN 22.137}), that they are alien (\href{https://suttacentral.net/sn22.33}{SN 22.33}). One who identifies with the aggregates is like a man who hires an assassin as a servant (\href{https://suttacentral.net/sn22.85}{SN 22.85}). They are suffering in the past and future just as they are today (\href{https://suttacentral.net/sn22.10}{SN 22.10}).

The view that the aggregates are self is called “identity view” (\textit{\textsanskrit{sakkāyadiṭṭhi}}). It is possible to identify with any or all of the aggregates in a myriad of ways, commonly set out as twenty forms of identity view (\href{https://suttacentral.net/sn22.1}{SN 22.1}, etc.). Identity view leashes an unenlightened person to transmigration like a dog tied to a post, pointlessly running around and around (\href{https://suttacentral.net/sn22.99}{SN 22.99}).

Several discourses emphasize that in order to understand the aggregates it is essential to develop the deep stillness of immersion meditation (\href{https://suttacentral.net/sn22.5}{SN 22.5}, \href{https://suttacentral.net/sn22.6}{SN 22.6}). But meditative realization is not something that just happens automatically; one must continually contemplate and observe the aggregates (\href{https://suttacentral.net/sn22.40}{SN 22.40}, etc.).

In this collection we find a large number of striking and lively narratives, showing how the aggregates could be a solace at the time of old age (\href{https://suttacentral.net/sn22.1}{SN 22.1}), a guide to the knotty theoretical debates on identity, or a framework for insight meditation.

\section*{The Book of the Six Sense Fields}

The “Book of the Six Sense Fields” is the fourth of the five books of the Linked Discourses. It is named after the first and longest \textit{\textsanskrit{saṁyutta}}. The second \textit{\textsanskrit{saṁyutta}} on Feelings also deals with a major doctrinal topic, one that is closely related to the main theme. The remaining eight \textit{\textsanskrit{saṁyuttas}} deal with secondary themes organized by subject or by person.

The number of discourses in the “Linked Discourses on the Six Sense Fields” varies between editions, mainly due to the way repetitions are counted; SuttaCentral follows Bhikkhu Bodhi’s translation in counting 248 discourses; see his introduction to this chapter for a discussion of the problems in counting the Suttas of this collection. They are collected in four \textit{\textsanskrit{paṇṇasakas}}.

This \textit{\textsanskrit{saṁyutta}} has an especially close relationship with the “Linked Discourses on the Aggregates”, one that goes far beyond the apparent thematic similarities. In fact, many of the discourses in the two collections are constructed on virtually identical lines. Bhikkhu Bodhi explores these connections with his notion of “template parallels”, which are found throughout the \textsanskrit{Saṁyutta} \textsanskrit{Nikāya}, but especially with these two sections.

The six sense fields complement the five aggregates as the summary of the noble truth of suffering. Where the aggregates focus on the functional structure of experience as basis for views of self, the emphasis here is on how sense experience stimulates desire.

The six sense fields are the means through which the world is known, and so each of them has two aspects. The “inner” aspect is the sense organs, for example the “eye” or the “ear”, which make it possible for an organism to experience the outside world by receiving sense stimuli. These are paired with the external sense stimuli, such as “sights” or “sounds”, which impact the sense organ (contact, \textit{phassa}) and give rise to the appropriate form of consciousness.

It’s best to avoid thinking of the external sense fields as “objects”, since in the suttas they are depicted in relation to the observing mind, and not as independently existing entities. There is no word for “object” in this sense in the early texts: existence is not objective, it is relational. The term \textit{\textsanskrit{ārammaṇa}}, which came to be used in this sense much later in the Abhidhamma, means “support” in the suttas.

The operation of the senses is relatively straightforward until we come to the last sense, the “mind” and “thoughts” or “mental phenomena”. To clear up a possible confusion, this “sixth sense” is simply the mental faculty and has nothing to do with psychic powers. And unlike the five external senses, the “inner” sense field is not a physical organ: \textit{mano} does not mean “brain”.

The exact meaning of “mind” (\textit{mano}) in this context is not spelled out in detail, so let us consider this first. The suttas use three main terms for the mind: \textit{mano}, \textit{citta}, and \textit{\textsanskrit{viññāṇa}}. In general these are synonyms and it is not possible to draw hard and fast distinctions between them (\href{https://suttacentral.net/an3.60}{AN 3.60}, \href{https://suttacentral.net/dn1\#2}{DN 1:2}, \href{https://suttacentral.net/sn12.61}{SN 12.61}). Nevertheless, they tend to be used in different contexts, each with a distinct nuance. These contexts can be understood in terms of the four noble truths; thus the different terms refer to the same thing, but imply a different aspect or response to that thing.

\begin{description}%
\item[\textit{\textsanskrit{Viññāṇa}}] In doctrinal contexts this is awareness itself, the sheer knowing of things. It appears in this sense in dependent origination, the aggregates, and the sense fields. Hence it pertains to the first noble truth, the suffering of the world, and it needs to be fully understood. In colloquial usage, however, it can take on a variety of shades of meaning, such as “understanding”.%
\item[\textit{Mano}] The mind in action, one of the three spheres of \textit{kamma}, a sense it inherits from the \textsanskrit{Upaniṣads}. It is that which creates results, as in the famous first line of the Dhammapada: \textit{mano \textsanskrit{pubbaṅgamā} \textsanskrit{dhammā}}, “mind is the forerunner of all things”. It is particularly used in ethical contexts, the performance of mental acts that bear fruit of either good or bad. So it may be understood as primarily relating to the second and third noble truths, the origination and ending of suffering.%
\item[\textit{Citta}] The most general, and the least tightly bound to a particular technical sense. It is used widely as “mind”, “thought”, “heart”, etc. But when found in technical contexts it refers to \textit{\textsanskrit{samādhi}}, to the purified awareness of deep meditative immersion. For this reason it is specially used in contexts relating to the path, the fourth noble truth.%
\end{description}

In the six senses, \textit{mano} is clearly not identical with the “knowing” (\textit{\textsanskrit{viññāṇa}}), as it gives rise to it. Nor is it the “known”, the phenomena of which the mind is aware, for that is \textit{\textsanskrit{dhammā}}. Nor is it the turning towards or paying attention to the thing known, as is revealed in \href{https://suttacentral.net/mn28}{MN 28} \textit{The Longer Simile of the Elephant’s Footprint} (\textit{\textsanskrit{Mahāhatthipadopamasutta}}):

\begin{quotation}%
Though the mind is intact internally, so long as exterior thoughts don’t come into range and there’s no corresponding attention, there’s no manifestation of the corresponding type of consciousness.

%
\end{quotation}

This passage suggests that, like the physical sense organs, \textit{mano} in some way pre-exists the actual moment of conscious awareness. This does not mean that it is some mystical substrata of consciousness, for as we have seen \textit{mano} is consistently used in the sense of the mind that performs acts, especially those with a moral dimension. So the \textit{mano} is that which has performed deeds in the past, fueling an ongoing mental continuum within which the results of those deeds may be experienced in the present. It is the mental faculty that bears the potential for conscious experience, created and conditioned by choices made in the past.

The “outer” aspect of the sixth sense is \textit{\textsanskrit{dhammā}}, a term so ambiguous its translation is always difficult. Here it refers to anything that may be known directly by the mind, distinct from the five physical senses. The most technically correct translation is probably “mental phenomena”. However, this is clumsy and opaque, so “thought” may be used as a more colloquial rendering, so long as it is understood to include ideas, imagination, and so on, not just verbalized cognition.

The term \textit{\textsanskrit{āyatana}} refers to something “stretched out”, a domain, field, or dimension of activity. However, the Visuddhimagga suggests that the sense of the word is primarily a “cause”, or perhaps “stimulus”:

\begin{quotation}%
… base (\textit{\textsanskrit{āyatana}}) should be understood as such (a) because of its actuating (\textit{\textsanskrit{āyatana}}), (b) because of being the range (\textit{tanana}) of the origins (\textit{\textsanskrit{āya}}), and (c) because of leading on (\textit{nayana}) what is actuated (\textit{\textsanskrit{āyata}}). \textit{Path of Purification}, XV.4, translated by Bhikkhu \textsanskrit{Ñāṇamoḷi}.

%
\end{quotation}

Bhikkhu \textsanskrit{Ñāṇamoḷi} rendered the term accordingly as “base”, which has been followed by Bhikkhu Bodhi. But this commentarial explanation is merely a series of false etymologies, or rather, puns. The point of such explanations is to provide material for teachers to reflect on and use in teaching, and they shouldn’t be taken uncritically. In fact the verbal root is not the obscure \textit{\textsanskrit{āyatati}} (“to actuate”) but \textit{\textsanskrit{āyamati}}, “to stretch, to extend”. \textit{Āyatana} is commonly used in this sense, and may be translated “field”, “dimension”, etc.

As so often, the context draws upon and redefines brahmanical terminology. The “six sense fields” (\textit{\textsanskrit{saḷāyatana}}) were first mentioned in the Buddha’s third teaching, the famous Fire Discourse (\textsanskrit{Ādittapariyāya} Sutta) which appears in this collection at \href{https://suttacentral.net/sn35.28}{SN 35.28}. This sermon was given to a large assembly of brahmanical ascetics, following a period when the Buddha stayed in their “fire house”, a kind of shrine room for worshiping the sacred flame. And in Sanskrit, this place is called an \textit{\textsanskrit{āyatana}}. The \textsanskrit{Upaniṣads} also call the senses \textit{\textsanskrit{āyatana}} in the sense of fields or scopes of activity and experience (eg. \textsanskrit{Bṛhadāraṇyaka} \textsanskrit{Upaniṣad} 4.1.4: \textit{\textsanskrit{cakṣur} \textsanskrit{evāyatanam}}, \textsanskrit{Bṛhadāraṇyaka} \textsanskrit{Upaniṣad} 6.1.5: \textit{mano \textsanskrit{vā} \textsanskrit{āyatanam}}; \textsanskrit{Chāndogya} \textsanskrit{Upaniṣad} 5.1.5: \textit{mano ha \textsanskrit{vā} \textsanskrit{āyatanam}}).

When the Buddha told those ascetics that “all is burning”, he was not giving an Abhidhamma analysis, for it was many centuries before Abhidhamma was developed. He was speaking in terms that the brahmins could understand.

One of the key projects of the brahmanical \textsanskrit{Upaniṣads} was to reinterpret the deities of the Vedas. Rather than thinking of them as entities who lived in the sky, they became forces or essences that imbued all of reality. So for the brahmin ascetics, the flame (\textit{agni}) was worshiped as the embodiment of a sacred energy that was imminent in all things.

The teachings of the Fire Sermon respond to several key \textsanskrit{Upaniṣadic} passages. In \textsanskrit{Bṛhadāraṇyaka} \textsanskrit{Upaniṣad} 1.3, it is told how evil entered into the world by the actions of the demons (\textit{asuras}). While the gods (\textit{devas}) were performing the ritual, they entered into the various senses and corrupted them, tainting them with evil and death. Hence when suffering is experienced through the senses, this is the reason. But those same senses can be freed from this corruption by being carried beyond death.

These purified, divine senses are further described at \textsanskrit{Bṛhadāraṇyaka} \textsanskrit{Upaniṣad} 2.5.1, the famous “Honey-Knowledge”, regarded as one of the highest and most secret teachings. It presents a template, applied to various different kinds of things. These are not organized so rationally as the Buddhist doctrines, but include quite different kinds of things in the same set, such as the elements, truth, the sun, etc. Nevertheless, the parallels with the teachings of the six senses are quite apparent.

\begin{quotation}%
\textit{ayam \textsanskrit{ādityaḥ} \textsanskrit{sarveṣāṁ} \textsanskrit{bhūtānāṁ} madhu} \\
This sun is the honey of all beings. \\
\textit{\textsanskrit{asyādityasya} \textsanskrit{sarvāṇi} \textsanskrit{bhūtāni} madhu} \\
All beings are the honey of the sun. \\
\textit{\textsanskrit{yaś} \textsanskrit{cāyam} asminn \textsanskrit{āditye} tejomayo ’\textsanskrit{mṛtamayaḥ} \textsanskrit{puruṣo} \textsanskrit{yaś} \textsanskrit{cāyam} \textsanskrit{adhyātmaṁ} \textsanskrit{cākṣuṣas} tejomayo ’\textsanskrit{mṛtamayaḥ} \textsanskrit{puruṣo} ’yam eva sa yo ’yam \textsanskrit{ātmā}} \\
This person in the sun made of fire and immortality, and this person in the internal eye made of fire and immortality: this is that—that which is the self. \\
\textit{idam \textsanskrit{amṛtam} \textsanskrit{idaṁ} \textsanskrit{brahmedaṁ} sarvam} \\
This is the immortal, this is the divine, this is the all. \\
(\textsanskrit{Bṛhadāraṇyaka} \textsanskrit{Upaniṣad} 2.5.5, translation by myself.)

%
\end{quotation}

The Brahmanical view is that all creation stems from \textsanskrit{Brahmā} and hence is, in its truest essence, overflowing with divinity and bliss—honey. Any suffering is merely a temporary imperfection.

This is how they handled the great challenge to any theistic system, the problem of evil. For the brahmins, to focus on suffering is to miss the point. This is not merely a facile “positive thinking” doctrine, it is a profoundly contemplative philosophy, worked out in great detail across many complex sacred texts, and informed by deep meditative practice. It does not deny the reality of suffering, but it evokes a deeper reality where suffering cannot reach.

Rather than tackling the textual and philosophical issues, the Buddha preferred to point directly at experience. Divested of theology, the experience of our senses is not “honey” but “fire”. And while our philosophy may say that fire is sacred, the reality is that it burns. The Buddha was showing the ascetics that there is no need to invoke deities and metaphysics in order to understand their experience: they could see how it worked right here.

The forces lighting that fire can be readily discerned: greed, hate, and delusion. This classic Buddhist presentation of the fundamental defilements appears first in this passage. It is correlated with the three feelings: pleasant feeling stimulates desire; painful feeling provokes hate; and neutral feeling slips into delusion (\href{https://suttacentral.net/mn44\#25}{MN 44:25}, \href{https://suttacentral.net/mn128\#28}{MN 128:28}, \href{https://suttacentral.net/sn36.3}{SN 36.3}).

The Fire Sermon, in its brevity, foreshadows several distinctive features of the teachings on the six senses as compared to the five aggregates. It is direct, emotional and powerful, speaking of the world that is burning, in contrast with the more intellectual approach of the aggregates.

By invoking the idea of the “all”, the Fire Sermon suggests that the scope of the six senses includes all that is experienced and known. This idea was expanded in multiple Suttas (\href{https://suttacentral.net/sn35.33}{SN 35.33–52}). By contrast, no such claim to completeness is made of the aggregates. And the text treats sense experience as a conditioned process, the immediately visible dimension of dependent origination.

Since the sense fields make experience possible, it is through them that suffering comes to be (\href{https://suttacentral.net/sn35.106}{SN 35.106}). It is in order to understand this suffering that one undertakes the spiritual path (\href{https://suttacentral.net/sn35.81}{SN 35.81}, \href{https://suttacentral.net/sn35.152}{SN 35.152}). The sense fields are, in fact, the world (\textit{loka}) that wears away (\textit{lujjati}; \href{https://suttacentral.net/sn35.82}{SN 35.82}, \href{https://suttacentral.net/sn35.84}{SN 35.84}), for “whatever in the world through which you perceive the world and conceive the world is called the world in the training of the noble one” (\href{https://suttacentral.net/sn35.116}{SN 35.116}). This world is empty of self (\href{https://suttacentral.net/sn35.85}{SN 35.85}).

Since the sense fields are produced by choices made in past lives, they are said to be “old kamma”; in this they contrast with the aggregates, for they include “choices”, which are the kamma made in the present. Having inherited the senses as the result of past deeds, however, we proceed to respond to them through thinking or conceiving of them in terms of a “self”, a process known in Pali as “identifying” (\textit{\textsanskrit{maññita}}; \href{https://suttacentral.net/sn35.146}{SN 35.146}, \href{https://suttacentral.net/sn35.30}{SN 35.30–32}, \href{https://suttacentral.net/sn35.90}{SN 35.90–91}, \href{https://suttacentral.net/sn35.248}{SN 35.248}).

“Conceiving” and the closely related “conceit” (\textit{\textsanskrit{māna}}) refer to the tendency of the mind to shape experience in terms of the self. Much of our thought is devoted to justifying, explaining, and interpreting our experience in ways that reinforce our notion of self. This can end up spinning out of control, in which case it is called “proliferation” (\textit{\textsanskrit{papañca}}). To cut through this process the Buddha urges us to stop short with sense experience (\href{https://suttacentral.net/sn35.94}{SN 35.94}, \href{https://suttacentral.net/sn35.95}{SN 35.95}).

It is significant that, while the texts repeatedly speak of how the aggregates form the basis for theories of self (\textit{\textsanskrit{sakkāya}}), the same is \emph{not} said of the sense fields. If the aggregates provoke grasping to theories, the sense fields provoke grasping at \emph{pleasure}, at the sheer vitality of sensory experience. Thus while the teachings on the aggregates emphasize \emph{views}, here the focus shifts to \emph{restraint}. A standard passage on sense restraint, familiar from the Gradual Training, speaks of preventing harmful qualities from invading the mind in the midst of sense experience (\href{https://suttacentral.net/sn35.120}{SN 35.120}, \href{https://suttacentral.net/sn35.127}{SN 35.127}, \href{https://suttacentral.net/sn35.239}{SN 35.239}, \href{https://suttacentral.net/sn35.240}{SN 35.240}). A person who chases the pleasure afforded by the senses is no less trapped by the pain they bring, and it is only by setting up mindfulness that one can achieve peace (\href{https://suttacentral.net/sn35.132}{SN 35.132}, \href{https://suttacentral.net/sn35.243}{SN 35.243–244}, \href{https://suttacentral.net/sn35.247}{SN 35.247}).

In this way, by choosing the sense fields as a locus of practice one cuts directly at the roots of craving. This is emphasized in the final two \textit{vaggas}, which are especially rich in unforgettable imagery. The senses are an ocean traversed during the spiritual journey (\href{https://suttacentral.net/sn35.228}{SN 35.228}). We’d be better off being tortured by hot pokers than being caught up in sense experience (\href{https://suttacentral.net/sn35.235}{SN 35.235}). If you wish to train in meditation, you must learn to withdraw the senses like a tortoise drawing in its limbs, becoming safe from predators (\href{https://suttacentral.net/sn35.240}{SN 35.240}). Pleasant experiences are the bait of \textsanskrit{Māra} (\href{https://suttacentral.net/sn35.230}{SN 35.230}). The six senses are like six very different animals, all tied together, and fighting to get to their own territory (\href{https://suttacentral.net/sn35.247}{SN 35.247}).

\section*{The Great Book}

The “Great Book” is the last and largest of the five books of the Linked Discourses. It consists of twelve \textit{\textsanskrit{saṁyuttas}}, almost all of which deal with an aspect of Buddhist practice, or the path. The first of these, indeed, is the “Section on the Path” (\textit{Magga \textsanskrit{Saṁyutta}}), and in the northern canons the book as a whole is referred to as the “Book of the Path” (\textit{Maggavagga}).

The first seven \textit{\textsanskrit{saṁyuttas}} offer a detailed treatment of seven sets of factors on Buddhist practice. These sets came to be known to the later traditions as the 37 \textit{\textsanskrit{bodhipakkhiyā} \textsanskrit{dhammā}}, or “qualities leading to awakening”. Note that this term is not used in this way in the suttas; it is, rather, applied to one of the sets, the five faculties (\href{https://suttacentral.net/sn48.55}{SN 48.55}, etc.). While the 37 factors are mentioned throughout the canon, it is in this book that we find the primary source for these teachings. Subsequent \textit{\textsanskrit{saṁyuttas}} deal with the path from different perspectives, while the final two deal with stream-entry and the four noble truths respectively.

While most books of the \textsanskrit{Saṁyutta} are dominated by one major collection, the Great Book features several \textit{\textsanskrit{saṁyuttas}} of comparable importance. For this reason I will briefly discuss most of the substantive \textit{\textsanskrit{saṁyuttas}}. I leave aside those that are merely sets of template repetitions, and also the final two \textit{\textsanskrit{saṁyuttas}} on stream entry and the truths, as I have covered these topics elsewhere. I preface the discussion of the individual \textit{\textsanskrit{saṁyuttas}} with a general discussion of the “qualities leading to awakening”.

The \textit{\textsanskrit{saṁyuttas}} in the Great Book display considerable complexity in their structure and use of repetitions. But for fear of overburdening the discussion, I refer anyone interested to the relevant sections of Ven Bodhi’s \textit{Connected Discourses}.

\subsection*{The 37 Qualities Leading to Awakening}

For the early Buddhist texts, the primary concern was the spiritual practice that leads to the escape from suffering. This is the fourth noble truth. From the very first discourse, this was spelled out by a specific set of factors comprising the path to awakening: the noble eightfold path. During the course of his long teaching career, the Buddha presented this path in many different ways, formally or informally, briefly or in detail, emphasizing different aspects to suit the occasion or the person.

Before his death, it seems, the Buddha had begun to systematize these various presentations, putting together seven sets of qualities pertaining to the path, totaling 37 factors. Each set presented the path to liberation from a slightly different perspective.

The seven primary \textit{\textsanskrit{saṁyuttas}} of the \textsanskrit{Mahāvagga} contain the same teachings, albeit in a different sequence. The \textsanskrit{Mahāvagga} begins with the noble eightfold path, due to its prestige and importance as \emph{the} teaching on the path. But when presented elsewhere in the suttas we find the sets arranged numerically.

\begin{itemize}%
\item Four kinds of mindfulness meditation. The observation of:
\begin{enumerate}%
\item body%
\item feelings%
\item mind%
\item principles%
\end{enumerate}

%
\item Four right efforts:
\begin{enumerate}%
\item to prevent the bad%
\item to give up the bad%
\item to give rise to the good%
\item to maintain and grow the good%
\end{enumerate}

%
\item Four bases of psychic power:
\begin{enumerate}%
\item enthusiasm%
\item energy%
\item mind%
\item inquiry%
\end{enumerate}

%
\item Five faculties:
\begin{enumerate}%
\item faith%
\item energy%
\item mindfulness%
\item immersion%
\item wisdom%
\end{enumerate}

%
\item Five powers:
\begin{enumerate}%
\item faith%
\item energy%
\item mindfulness%
\item immersion%
\item wisdom%
\end{enumerate}

%
\item Seven factors of awakening:
\begin{enumerate}%
\item mindfulness%
\item investigation of principles%
\item energy%
\item rapture%
\item tranquility%
\item immersion%
\item equanimity%
\end{enumerate}

%
\item Noble eightfold path:
\begin{enumerate}%
\item right view%
\item right thought%
\item right speech%
\item right action%
\item right livelihood%
\item right effort%
\item right mindfulness%
\item right immersion%
\end{enumerate}

%
\end{itemize}

A cursory glance at the Pali texts shows how influential and widespread this set of 37 qualities was. It appears in each of the four \textit{\textsanskrit{nikāyas}} (\href{https://suttacentral.net/dn28}{DN 28}, \href{https://suttacentral.net/dn29}{DN 29}, \href{https://suttacentral.net/dn16}{DN 16}, \href{https://suttacentral.net/mn103}{MN 103}, \href{https://suttacentral.net/mn104}{MN 104}, \href{https://suttacentral.net/sn22.81}{SN 22.81}, \href{https://suttacentral.net/sn22.101}{SN 22.101}, \href{https://suttacentral.net/sn43.12}{SN 43.12}, \href{https://suttacentral.net/an8.19}{AN 8.19}) as well as the \textsanskrit{Udāna} (\href{https://suttacentral.net/ud5.5}{Ud 5.5}). It is one of the few doctrinal teachings to be mentioned several times in the Vinaya (\href{https://suttacentral.net/pj4}{Pj 4}, \href{https://suttacentral.net/pc8}{Pc 8}, \href{https://suttacentral.net/pli{-}tv{-}kd19}{Kd 19}). It occurs constantly in the late canonical texts of the Khuddaka (\href{https://suttacentral.net/ne8}{Ne 8}, \href{https://suttacentral.net/cnd12}{Cnd 12}, \href{https://suttacentral.net/cnd15}{Cnd 15}, \href{https://suttacentral.net/cnd20}{Cnd 20}, \href{https://suttacentral.net/cnd22}{Cnd 22}, \href{https://suttacentral.net/mnd6}{Mnd 6}, \href{https://suttacentral.net/mnd7}{Mnd 7}, \href{https://suttacentral.net/mnd14}{Mnd 14}, \href{https://suttacentral.net/mnd16}{Mnd 16}, \href{https://suttacentral.net/ps1.5}{Ps 1.5}, \href{https://suttacentral.net/ps2.8}{Ps 2.8}, \href{https://suttacentral.net/ps2.9}{Ps 2.9}, \href{https://suttacentral.net/mil3.1.13}{Mil 3.1.13}, \href{https://suttacentral.net/mil6.4.1}{Mil 6.4.1}, \href{https://suttacentral.net/mil6.4.2}{Mil 6.4.2}, etc.) as well as the Abhidhamma (\href{https://suttacentral.net/vb17}{Vb 17}, \href{https://suttacentral.net/Dt1.2}{Dt 1.2}, \href{https://suttacentral.net/Dt2.1}{Dt 2.1}, \href{https://suttacentral.net/Dt2.6}{Dt 2.6}, \href{https://suttacentral.net/kv4.3}{Kv 4.3}, \href{https://suttacentral.net/kv12.5}{Kv 12.5}, \href{https://suttacentral.net/kv14.9}{Kv 14.9}, \href{https://suttacentral.net/kv15.6}{Kv 15.6}, \href{https://suttacentral.net/kv21.1}{Kv 21.1}, \href{https://suttacentral.net/kv21.5}{Kv 21.5}, etc.).

But its influence was not to stop there, for it remained a central doctrinal principle in later forms of Buddhism. In the \textsanskrit{Mahāyāna}, for example, the same 37 qualities came to be known as the “37 practices of the Bodhisattva”.

The Buddha declared that these teachings emerged from his own direct knowledge. Clearly they are factors of practice, to be developed and experienced by those on the spiritual journey. However, from their earliest appearances, they were also treated as teachings to be learned, memorized, and recited. From \href{https://suttacentral.net/dn29}{DN 29}:

\begin{quotation}%
You should all come together and recite in concert, without disputing, those things I have taught you from my direct knowledge, comparing meaning with meaning and phrasing with phrasing, so that this spiritual path may last for a long time.

%
\end{quotation}

Such passages place the 37 factors at the heart of the Buddha’s scriptural legacy. But what, exactly, was to be recited? Surely such a momentous teaching must have entailed something more than simply listing the factors. There must have been an agreed upon body of texts, a canon of scripture recited in unity by the early community. And what could that have been if not these very teachings, the collected discourses on the factors of the path found today in the \textsanskrit{Mahāvagga}? This is not to deny that expansion and elaboration of these has occurred, but the core teachings of the \textsanskrit{Mahāvagga} were, in all probability, the heart of the scriptures for the earliest Buddhists.

Certain of the sets focus on a specific area, such as mindfulness or effort, while others have a more overall view, such as the noble eightfold path. Nevertheless, they are deeply interconnected, with the same factors recurring in multiple sets. Overall, they strongly emphasize meditation, although other dimensions of spiritual practice, such as ethics and study, are also found. Here is a brief overview of the general distinctions in perspective between the groups. Note that the first three sets loosely correspond to the final three factors of the noble eightfold path: right effort, right mindfulness, and right immersion.

\begin{description}%
\item[Four kinds of mindfulness meditation] The practice of undertaking meditation leading to serenity and insight.%
\item[Four right efforts] The putting forth of effort in mental cultivation.%
\item[Four bases of psychic power] Development of deep immersion leading to various extraordinary abilities.%
\item[Five faculties] The mental qualities that lead to liberation, and which characterize the mind of one on the path.%
\item[Five powers] The same as the faculties.%
\item[Seven factors of awakening] Retention and investigation of teachings lead to the progressive deepening of the emotional qualities that ripen in liberation.%
\item[Noble eightfold path] The broadest in scope of the sets and the only one to explicitly mention ethics.%
\end{description}

As is common in the suttas, these sets sometimes refer to similar qualities with different terms. The quality of wisdom, for example, is called “observation of principles” (\textit{\textsanskrit{dhamānupassanā}}) as the fourth kind of mindfulness meditation, “inquiry” (\textit{\textsanskrit{vīmaṁsa}}) in the bases for psychic power, “wisdom” (\textit{\textsanskrit{paññā}}) in the faculties and powers, “investigation of principles” (\textit{dhammavicaya}) in the factors of awakening, and “right view” (\textit{\textsanskrit{sammādiṭṭhi}}) in the noble eightfold path. The relations between all these terms are analyzed in detail in the Abhidhamma and commentarial texts.

Bear in mind, though, that each context has its own integrity, its own specific purpose and orientation, and the choice of different terms is by no means arbitrary. “Right view”, being placed at the \emph{start} of the path, emphasizes the theoretical understanding gained by hearing the teaching. “Investigation of principles”, similarly located near the beginning, refers to the reflection and inquiry into these teachings as realized in oneself. “Observation of principles” and “inquiry” occur after the development of deep stillness in absorption meditation, and refer to the inquiry and investigation into the nature of that experience, and the meditative processes and conditions that shape such profound experiences. And “wisdom”, the culmination of all these, is the realization of the four noble truths, the liberating insight of the stream-enterer. So when considered on its own, as a distinct mental factor, they can be regarded as synonyms. But their true depth is realized only by understanding the role they play in their context.

When surveying these teachings and reflecting on them as a spiritual path, there is something rather odd about them. They appear quite different from the practices that one normally considers to be “religious”. Where are the rituals? The sacrifice? The devotion to deity? The allegiance to an institution? The symbols, rites, and mythology? These things are starkly, dramatically absent. To be sure, some such things may be found, in one form or another, elsewhere in the canon, and more so in later Buddhist traditions. But here, in the teachings regarded by the Buddha himself as his core message and practice, we find only balanced and reasoned development of the behavior, emotions, and intellect. It is an integrated and rational path, one that does not depend on cultural or historical specifics, but on universal human qualities. The factors that lead to awakening, all 37 of them, are things that every human may find within themselves. In pointing to these qualities, the Buddha was pointing to the spiritual potential of all beings, and offering us the means to grow and develop the best parts of ourselves.

\subsection*{SN 45: Linked Discourses on the Path}

The noble eightfold path was famously declared to be the “middle way” in the Buddha’s very first teaching (\href{https://suttacentral.net/sn56.11}{SN 56.11}). It covers the entire spiritual path (\href{https://suttacentral.net/sn45.6}{SN 45.6}, \href{https://suttacentral.net/sn45.19}{SN 45.19}, \href{https://suttacentral.net/sn45.20}{SN 45.20}), beginning with the acquisition of right view as the starting point (\href{https://suttacentral.net/sn45.1}{SN 45.1}), and leading to deep meditative immersion as the immediate precursor to the realization of the four noble truths.

The noble eightfold path is said to be a “divine vehicle” which carries us to awakening, its factors compared to the parts of a chariot (\href{https://suttacentral.net/sn45.4}{SN 45.4}). Practicing it leads to the end of suffering (\href{https://suttacentral.net/sn45.5}{SN 45.5}), but only if it begins with right view, else it will lead to harming oneself (\href{https://suttacentral.net/sn45.9}{SN 45.9}).

The factors are defined at \href{https://suttacentral.net/sn45.8}{SN 45.8}, as well as several other places in the canon.

\begin{description}%
\item[Right view (\textit{\textsanskrit{sammādiṭṭhi}})] Understanding the four noble truths.%
\item[Right thought (\textit{\textsanskrit{sammāsaṅkappa}})] Thoughts of letting go, love, and kindness.%
\item[Right speech (\textit{\textsanskrit{sammāvācā}})] Speech that is true, harmonious, gentle, and meaningful.%
\item[Right action (\textit{\textsanskrit{sammākammanta}})] Avoiding killing, stealing, and sexual misconduct.%
\item[Right livelihood (\textit{\textsanskrit{sammāājīva}})] Avoiding harmful livelihood.%
\item[Right effort (\textit{\textsanskrit{sammāvāyāma}})] The four right efforts.%
\item[Right mindfulness (\textit{\textsanskrit{sammāsati}})] The four kinds of mindfulness meditation.%
\item[Right immersion (\textit{\textsanskrit{sammāsamādhi}})] The four absorptions.%
\end{description}

The eight factors have a clear progressive aspect, as made clear from the beginning of this collection (\href{https://suttacentral.net/sn45.1}{SN 45.1}). They follow the same general course that is spelled out in detail in the Gradual Training, though with less emphasis on the monastic life, as both renunciates and lay folk should practice them (\href{https://suttacentral.net/sn45.24}{SN 45.24}). One hears the teaching and gains an initial understanding (right view). Then one determines to live in accordance with this (right thought), undertaking the essentials of ethical conduct in speech (right speech) and body (right action), and ensuring that one does not earn money in a manner that causes harm (right livelihood). With this foundation one makes an effort to purify the mind (right effort), undertaking meditation (right mindfulness) leading to deep absorption (right immersion) (\href{https://suttacentral.net/sn45.28}{SN 45.28}).

When all these factors have been fulfilled, the mind is ready to make the breakthrough to the realization of the four noble truths. In this way the understanding of four noble truths, beginning as a concept accepted on faith, gradually deepens throughout the spiritual journey, nourished by experience and reflection. Right view guides us on each step of the path, learning from mistakes, revealing our hidden motivations, and uncovering unexpected possibilities. Ultimately it transforms into the liberating insight of the noble ones (\href{https://suttacentral.net/sn45.13}{SN 45.13}, \href{https://suttacentral.net/sn45.35}{SN 45.35}, etc.). The key to this transformation is the brilliant clarity and stillness of meditative absorption, a higher consciousness that sees further and deeper than ever before, and which has the power to completely eradicate greed, hate, and delusion (\href{https://suttacentral.net/sn45.36}{SN 45.36}, etc.).

Nevertheless, despite this progression, it is obviously not the case that the factors are to be undertaken in a literal one-at-a-time fashion. The “path” is only a metaphor, and in real life spiritual development is more complex.

The factors of the path are best seen as providing a framework for reflecting on and if necessary changing one’s own life and practice. Each of these factors is essential, and if you find yourself missing out on higher factors, try asking whether you’ve put enough work into the basics. Sometimes people seem enthusiastic to get to the higher states of consciousness, without laying the broad and secure foundations offered by the simpler factors of the path. If developing deep meditation is proving difficult, then the answer is not to try to force it to ripen quicker, nor, worse, to explain it away as being somehow unnecessary. Rather, pay closer attention to improving right view through study and discussion of Dhamma; to developing right thought by becoming more generous and open-hearted; or to being more careful in one’s ethical and business conduct (\href{https://suttacentral.net/sn45.50}{SN 45.50–54}).

And remember, this path is not walked alone. For all the emphasis on solitary meditation, this \textit{\textsanskrit{saṁyutta}} reminds us that good friendship is the whole of the spiritual life (\href{https://suttacentral.net/sn45.2}{SN 45.2}, \href{https://suttacentral.net/sn45.3}{SN 45.3}), for good friendship precedes the noble eightfold path (\href{https://suttacentral.net/sn45.49}{SN 45.49}).

\subsection*{SN 46: Linked Discourses on the Awakening Factors}

These seven factors are called the “awakening factors” (\textit{\textsanskrit{bojjaṅga}}, i.e. \textit{bodhi} + \textit{\textsanskrit{aṅga}}) because they lead to awakening (\href{https://suttacentral.net/sn46.5}{SN 46.5}, \href{https://suttacentral.net/sn46.21}{SN 46.21}). Of themselves, they focus on the psychology of contemplation, but the \textit{\textsanskrit{saṁyutta}} makes it clear from the start that, like all presentations of the path, they rest on ethics (\href{https://suttacentral.net/sn46.1}{SN 46.1}).

Unlike the factors of the path, there is no explicit definition. Nevertheless, we should of course interpret these factors in the same way as they occur in the eightfold path and elsewhere. However, there are some new factors, as well as a few places that offer a new perspective on some familiar factors. Most of the following details come from \href{https://suttacentral.net/sn46.52}{SN 46.52}.

\begin{description}%
\item[Mindfulness (\textit{sati})] Includes both the recollection of teachings (\href{https://suttacentral.net/sn46.3}{SN 46.3}) as well as mindful awareness of phenomena internal and external.%
\item[Investigation of principles (\textit{dhammavicaya})] Includes both reflection and investigation of the teachings (\href{https://suttacentral.net/sn46.3}{SN 46.3}) as well as investigation into phenomena internal and external.%
\item[Energy (\textit{viriya})] Both mental and physical.%
\item[Rapture (\textit{\textsanskrit{pīti}})] This is the experience of uplifting joy that emerges as the mind becomes peaceful in meditation. It includes the rapture of the first and second absorptions.%
\item[Tranquility (\textit{passadhi})] Both physical and mental%
\item[Immersion (\textit{\textsanskrit{samādhi}})] The absorptions.%
\item[Equanimity (\textit{\textsanskrit{upekkhā}})] This may be both the equanimity of the higher states of immersion as well as that of deep insight.%
\end{description}

One detail of the preceding probably needs further explanation; that is, the idea that mindfulness includes recollection of the teachings. Mindfulness is defined throughout the suttas as the ability to recollect things that were said and done long ago (\href{https://suttacentral.net/dn33}{DN 33}, \href{https://suttacentral.net/dn34}{DN 34}, \href{https://suttacentral.net/mn53}{MN 53}, \href{https://suttacentral.net/sn48.9}{SN 48.9}, \href{https://suttacentral.net/sn48.50}{SN 48.50}, \href{https://suttacentral.net/an4.35}{AN 4.35}, \href{https://suttacentral.net/an8.13}{AN 8.13}, \href{https://suttacentral.net/an10.17}{AN 10.17}, etc.). The root meaning of the word \textit{sati} is in fact “memory” and in the Brahmanical traditions it refers to the memorized scriptures. But of course today we understand mindfulness as “clear awareness” of phenomena in the present.

This \textit{\textsanskrit{saṁyutta}} offers a clue that helps resolve these two senses. In \href{https://suttacentral.net/sn46.56}{SN 46.56}, a brahmin asks the Buddha why he can sometimes remember his chanting and sometimes cannot. The Buddha explains that the presence of the hindrances obscures his memory, giving an elaborate series of similes comparing water in various states with the various hindrances. How, we might wonder, does a reciter of oral texts achieve this? \emph{By maintaining continued and clear focus during the act of recitation}. When the mind wanders and gets distracted, the recitation is lost. \textit{Sati} does not mean the unstructured memories that happen to come to mind, but the steady flow and continuity of consciously focused awareness. And in this way the act of recollecting scriptures suddenly seems a lot like keeping attention on ones’ meditation.

The factors are sequential, with each serving as condition or fuel for the next (\href{https://suttacentral.net/sn46.3}{SN 46.3}). Multiple suttas stress this aspect of conditionality. Each of the awakening factors is nourished by a specific kind of fuel (\href{https://suttacentral.net/sn46.51}{SN 46.51}). The set as a whole emerges from the practice of the four kinds of mindfulness meditation and the series of practices that underlie them (\href{https://suttacentral.net/sn45.6}{SN 45.6}). They affect and condition the mind in distinct ways; thus when the mind is tired, it’s best to develop investigation, energy, and rapture, but when restless, develop tranquility, immersion, and equanimity. But mindfulness is always useful (\href{https://suttacentral.net/sn46.53}{SN 46.53}). And the factors themselves are the condition for awakening (\href{https://suttacentral.net/sn46.56}{SN 46.56}).

Nevertheless, even the perfected ones continue to practice them, donning any one of them whenever they wish, like a garment (\href{https://suttacentral.net/sn46.4}{SN 46.4}). Such a one has “acquired the path” and understands the true power of the awakening factors to lead to the end of rebirth (\href{https://suttacentral.net/sn46.30}{SN 46.30}).

The \textit{\textsanskrit{saṁyutta}} repeatedly opposes the awakening factors with their dark counterparts, the five hindrances of sensual desire, ill will, dullness and drowsiness, restlessness and remorse, and doubt. These are compared to corruptions in gold (\href{https://suttacentral.net/sn46.33}{SN 46.33}) or to parasites (\href{https://suttacentral.net/sn46.39}{SN 46.39}).

One of the unique aspect of the awakening factors is that their recitation is said to be effective in helping cure disease. Several suttas speak of how a sick monk—and even the Buddha himself—becomes inspired by hearing them recited and rises up cured (\href{https://suttacentral.net/sn46.14}{SN 46.14–16}). Understandably, this has ensured that reciting passages on the awakening factors for sick people remains popular in Theravadin culture. If such recitation seems less effective today than in the suttas, it should be borne in mind that these are cases of advanced and experienced meditators, perfected ones indeed, who had already developed these factors to completion. Their inspiration is on a different level than that of an ordinary person. And even so there is no guarantee: there are plenty of cases in the early texts where perfected ones fall ill with no cure.

Most of the awakening factors refer to the emotional aspects of spiritual path, the joy and peace of meditation. This is further emphasized in \href{https://suttacentral.net/sn46.54}{SN 46.54}, which connects the awakening factors with the four immeasurables or divine meditations—love, compassion, rejoicing, and equanimity. The Buddhist mendicants are challenged by followers of other paths, who say that they too teach the development of these things. The Buddha points out, however, that he describes \emph{how} to develop these things to their fullest potential. And to do this the immeasurables are empowered by the awakening factors.

\subsection*{SN 47: Linked Discourses on Mindfulness Meditation}

The Pali term \textit{\textsanskrit{satipaṭṭhāna}} means the “establishing of mindfulness”. I usually render it more colloquially as simply “mindfulness meditation”. While elsewhere \textit{sati} is defined as “memory”, here it is described as \textit{\textsanskrit{anupassanā}}, “sustained observation”. It refers to the meditative practice of setting up and maintaining continued and unbroken awareness in four distinct arenas:

\begin{description}%
\item[Body (\textit{\textsanskrit{kāya}})] Any aspect of the physical, including the breath, the postures, parts of the body, and so on.%
\item[Feelings (\textit{\textsanskrit{vedanā}})] Different kinds of feeling, whether painful, pleasurable, or neutral, spiritual or carnal.%
\item[Mind (\textit{citta})] States of awareness, whether under the influence of greed, hate, and delusion, or free of such.%
\item[Principles (\textit{\textsanskrit{dhammā}})] Understanding the causal relations that lead to suffering or to peace, especially by reflecting on the process of meditation itself.%
\end{description}

Each of these can include a diverse range of experience. But in meditation it is important to keep focus. The standard formula phrases this through the use of the reflexive idiom \textit{\textsanskrit{kāye} \textsanskrit{kāyānupassī}}. Here the locative case is used quite literally to mean “one of the bodies in the body”, or as we would say in English, a particular aspect of the body. Thus the meditator does not continually shift attention to whatever comes into mind, but maintains a steady, continuous awareness on a specific aspect of physical experience.

This is a progressive practice. The nature of this progress becomes more clear when it is recognized that mindfulness of breathing is a form of \textit{\textsanskrit{satipaṭṭhāna}} practice.

\begin{enumerate}%
\item Meditation begins with attention to the relatively coarse phenomena of the physical breath until it becomes calm and still.%
\item A subtle sense of joy and bliss pervades the breath and the body.%
\item The mind becomes freed, immersed in the singular experience of the bliss of release.%
\item One contemplates the changing process of meditation that has led to this point. The mind, empowered by immersion, lets go.%
\end{enumerate}

But \textit{\textsanskrit{satipaṭṭhāna}} is broader than I have indicated here, for it includes not only the positive experiences that evolve during meditation, but also the negative ones: the pain, the constricted mind, the hindrances. By encompassing the full range of experience, \textit{\textsanskrit{satipaṭṭhāna}} promotes a broad, inclusive approach to meditation, one based on awareness rather than control, laying the groundwork for the flowering of wisdom.

This \textit{\textsanskrit{saṁyutta}} presents a series of insightful and often delightful suttas on \textit{\textsanskrit{satipaṭṭhāna}}, but it does not define the scope of the meditation. The definitions above are derived from the longer discourses today found at \href{https://suttacentral.net/mn10}{MN 10} and \href{https://suttacentral.net/dn22}{DN 22}. However, these have clearly undergone considerable late development as compared with the short discourses of the \textit{\textsanskrit{saṁyutta}}, and one cannot simply assume that everything in \href{https://suttacentral.net/mn10}{MN 10} and \href{https://suttacentral.net/dn22}{DN 22} applies in the \textit{\textsanskrit{saṁyutta}}.

The Pali compound \textit{\textsanskrit{satipaṭṭhāna}} resolves to \textit{sati} + \textit{\textsanskrit{upaṭṭhāna}}. This phrase is familiar from the Gradual Training, where it refers to the moment when a practitioner sits down in seclusion and begins meditation by “establishing mindfulness” (\textit{\textsanskrit{satiṁ} \textsanskrit{upaṭṭhapetvā}}). It thus refers primarily to the formal practice of meditation.

Today it is common to speak of “mindfulness in daily life”, but in the suttas this is called \textit{\textsanskrit{sampajañña}}, which I translate as “situational awareness”. This is one of the series of practices in the Gradual Training that lays the groundwork for formal meditation. \href{https://suttacentral.net/sn47.2}{SN 47.2} makes plain the distinction between these two by treating them as two qualities the mendicant should develop. This is not to say, of course, that they are completely separate, for nothing in spiritual and mental development happens in isolation. \textit{\textsanskrit{Sampajañña}} is not limited to “mindfulness in daily life”, but plays a role in absorptions and insight as well (see \href{https://suttacentral.net/sn47.35}{SN 47.35}). But it is to say that these two practices are primarily distinct, with situational awareness helping to prepare the mind for mindfulness meditation.

The standard formula describes the mindful meditator with four terms. These refer back to the fundamental helper practices of the Gradual Training, reminding us that \textit{\textsanskrit{satipaṭṭhāna}} meditation does not happen in isolation:

\begin{description}%
\item[Keen (\textit{\textsanskrit{ātāpī}})] possessing persistent and unflagging energy.%
\item[Aware (\textit{\textsanskrit{sampajāno}})] possessing situational awareness.%
\item[Mindful (\textit{\textsanskrit{satimā}})] possessing mindfulness.%
\item[Rid of desire and aversion for the world (\textit{vineyya loke \textsanskrit{abhijjhādomanassaṁ}})] having eliminated the overt forms of desire and aversion through the practice of sense restraint. The phrase \textit{\textsanskrit{abhijjhādomanassa}} is elsewhere used only in the context of sense restraint (\href{https://suttacentral.net/dn10}{DN 10}, \href{https://suttacentral.net/mn33}{MN 33}, \href{https://suttacentral.net/sn35.120}{SN 35.120}, \href{https://suttacentral.net/an4.14}{AN 4.14}, etc.).%
\end{description}

In the eightfold path, the awakening factors, the faculties, and the powers, mindfulness meditation is one of the key factors leading to deep meditative stillness and immersion. It is defined elsewhere as “the basis for immersion in \textit{\textsanskrit{samādhi}}” (\href{https://suttacentral.net/mn44}{MN 44}: \textit{\textsanskrit{cattāro} \textsanskrit{satipaṭṭhānā} \textsanskrit{samādhinimittā}}). With the charming parable of a cook, \href{https://suttacentral.net/sn47.8}{SN 47.8} shows how a skillful mindfulness meditator, by understanding the characteristics of their own mind, enters immersion and abandons defilements, while a poor meditator fails. In \href{https://suttacentral.net/sn47.4}{SN 47.4} the Buddha urges all meditators, whether beginners or advanced, to practice mindfulness to the level of full immersion (\textit{\textsanskrit{ekodibhūtā} \textsanskrit{vippasannacittā} \textsanskrit{samāhitā} \textsanskrit{ekaggacittā}}; “at one, with minds that are clear, immersed in \textit{\textsanskrit{samādhi}}, and unified”).

The centrality of meditative immersion is reinforced by the saying that \textit{\textsanskrit{satipaṭṭhāna}} is the “path to convergence” (\textit{\textsanskrit{ekāyano} maggo}). This saying is famous from \href{https://suttacentral.net/mn10}{MN 10} but sourced from the \textsanskrit{Saṁyutta}, where the saying is placed in the mouth of \textsanskrit{Brahmā} (\href{https://suttacentral.net/sn47.1}{SN 47.1}, \href{https://suttacentral.net/sn47.18}{SN 47.18}, \href{https://suttacentral.net/sn47.43}{SN 47.43}). It is a term from the \textsanskrit{Upaniṣads}, which in contemplative contexts means “the place where all things come together as one” (\textsanskrit{Bṛhadāraṇyaka} \textsanskrit{Upaniṣad} 2.4:11).

The meditative absorptions (\textit{\textsanskrit{jhānas}}) are explicitly brought into \textit{\textsanskrit{satipaṭṭhāna}} in the extended and late passage on the four noble truths in \href{https://suttacentral.net/dn22}{DN 22}. However they are implicit in many places, including the observation of feelings under the notions of “spiritual rapture” and “spiritual bliss”, which are defined in terms of the absorptions (\href{https://suttacentral.net/sn36.31}{SN 36.31}); as well as in the observation of mind under the mind that is “expansive”, “unexcelled”, “immersed”, “freed”, all of which are terms for deep states of absorption; or the discussion at \href{https://suttacentral.net/mn125}{MN 125}.

This is not to say that insight (or discernment, \textit{\textsanskrit{vipassanā}}) has no place in \textit{\textsanskrit{satipaṭṭhāna}}. On the contrary, the fourth of the \textit{\textsanskrit{satipaṭṭhānas}}, the observation of principles, is primarily concerned with the insight that follows from meditative immersion. Here, as described in \href{https://suttacentral.net/mn10}{MN 10}, one does not merely observe the presence and absence of various factors, one understands the reason why they appear and disappear. And understanding causality is the heart of insight. This is reinforced in the teaching on mindfulness of breathing, which introduces the contemplation of impermanence at this point.

Two suttas bring the \textit{\textsanskrit{vipassanā}} aspect to the fore. In \href{https://suttacentral.net/sn47.40}{SN 47.40}, the Buddha first teaches the standard \textit{\textsanskrit{satipaṭṭhāna}} practice, then introduces the “development” of \textit{\textsanskrit{satipaṭṭhāna}}. In the suttas, “development” means the enhancement and expansion of what is already there. (\textit{\textsanskrit{Bhāvanā}} is derived from the causitive form of the word “to be”, i.e. “to make be more”.) This further development involves contemplating all four of the \textit{\textsanskrit{satipaṭṭhānas}} in terms of origin and cessation. The exact meaning of this is spelled out in \href{https://suttacentral.net/sn47.42}{SN 47.42}, which gives the origin of each of the four.

A distinctive feature of this collection is the number of charming parables, which are as memorable as they are amusing. In addition to the story of the cook which we mentioned above (\href{https://suttacentral.net/sn47.8}{SN 47.8}), we hear how a quail learned to escape a hawk (\href{https://suttacentral.net/sn47.6}{SN 47.6}), how a foolish monkey got trapped in tar (\href{https://suttacentral.net/sn47.7}{SN 47.7}), and how two acrobats support each other (\href{https://suttacentral.net/sn47.19}{SN 47.19}). Another discourse sets a seemingly impossible challenge for mindfulness practice: to walk, carrying a bowl of oil filled to the brim, between a popular performer and the crowd jostling to see her, while a man with a drawn sword waits to chop off your head if you spill a drop (\href{https://suttacentral.net/sn47.20}{SN 47.20})!

\subsection*{SN 48: Linked Discourses on the Faculties}

The word \textit{indriya} has a rather interesting history. It occurs 39 times in the Ṛg Veda in the general sense of “the power of Indra”, the great warrior-god and dragon-slayer known in Pali as Sakka. But the nature of this power is perhaps not what one might imagine, for more than two-thirds of these cases connect \textit{indriya} with \textit{soma}.

Now, \textit{soma} was of course a drug, probably a preparation from the amphetamine-like stimulant ephedra. It was drunk by the ancient Indo-Europeans to imbue warriors with a berserk energy on the battlefield. As well as taming the horse and inventing the fast two-wheeled chariot, drug-enhanced combat was one of the key innovations underlying the military success of the Indo-Europeans.

In the Vedic culture this was ritualized as religious practice: Indra himself drinks \textit{soma} to magnify his potency. He becomes unstoppable and crushes all his enemies before him. His devotees follow his example, manifesting the power of the god within themselves. The drug-induced high gave them the might of the gods. But the crucial point is that the power is not borrowed from Indra; rather, both god and devotee draw power from the same source. It was inside them all along, it just needed the \textit{soma} to bring it out.

By the time of the Buddha, the Vedic age was long-gone and the \textit{soma} largely forgotten. Later commentators, unfamiliar with its Vedic roots, defined \textit{indriya} as “rulership”, and the various \textit{indriyas} in Buddhism as the governing faculties that exercise control over their domains. But the use in the suttas shows that the meaning lies closer to the Vedic sense of “potency”. The \textit{indriyas} are innate potentials that can be manifested in the right conditions.

This is why, after the Buddha’s awakening, he surveyed the world and assessed the \textit{indriyas} of the many different beings in it. He saw the spiritual potential latent within each person to different degrees, and realized that this hidden potential could be drawn out with the right teaching and encouragement (\href{https://suttacentral.net/sn6.1}{SN 6.1}).

To formulate a teaching on the \textit{indriyas}, the Buddha drew upon a set of five qualities he had developed under his former teachers \textsanskrit{Āḷāra} \textsanskrit{Kālāma} and Uddaka \textsanskrit{Rāmaputta} (\href{https://suttacentral.net/mn26\#17}{MN 26:17}, etc.). That these really are a set of teachings in the brahmanical tradition is confirmed by their mention in the \textsanskrit{Yogasūtra} (1.20). He called this set the five \textit{indriyas}.

The same qualities were also known as the \textit{balas} or “powers”. At \href{https://suttacentral.net/sn48.43}{SN 48.43} the Buddha discusses the relation between these two sets, saying they are like a river that flows around an island. They are part of the same stream and go to the same place, but from a certain perspective they can be distinguished. The term \textit{bala}, like \textit{indriya}, is Vedic, with the same basic sense of potency or strength, and occurs in contexts featuring Indra and his \textit{soma}. The \textit{balas} have only some repetition templates in the \textit{\textsanskrit{saṁyutta}}, and are defined in the \textsanskrit{Aṅguttara} (\href{https://suttacentral.net/an5.14}{AN 5.14}).

The \textit{indriyas} (together with the \textit{balas}) came to be included in the 37 \textit{\textsanskrit{bodhipakkhiyadhammā}}, and form the heart of the Indriya \textsanskrit{Saṁyutta}, where they are defined as follows (\href{https://suttacentral.net/sn48.8}{SN 48.8}, \href{https://suttacentral.net/sn48.9}{SN 48.9}, \href{https://suttacentral.net/sn48.10}{SN 48.10}):

\begin{description}%
\item[Faith (\textit{\textsanskrit{saddhā}})] Faith in the Buddha’s awakening.%
\item[Energy (\textit{viriya})] The effort to give up the bad and develop the good.%
\item[Mindfulness (\textit{sati})] Recollection of things said and done long ago, and the four kinds of mindfulness meditation.%
\item[Immersion (\textit{\textsanskrit{samādhi}})] Unification of mind based on letting-go; further defined as the four absorptions (\textit{\textsanskrit{jhānas}}).%
\item[Wisdom (\textit{\textsanskrit{paññā}})] Understanding impermanence and the four noble truths.%
\end{description}

By beginning with faith (\textit{\textsanskrit{saddhā}}), the text introduces a quality not explicitly mentioned in the earlier sets. In Buddhism, faith is essential. In traditional Buddhist lands to this day, the quiet yet steadfast faith and devotion to the Buddha and his teachings is ever-present, expressed through offerings of flowers, through grace and humility in the presence of the sacred, or through uplifting recollection of the Buddha’s words. It is an emotional quality, often paired with \textit{pema}, “affection”. But the Buddha explicitly rejected blind or “baseless faith” (\href{https://suttacentral.net/mn95\#13}{MN 95:13}: \textit{\textsanskrit{amūlikā} \textsanskrit{saddhā}}) and urged his followers to develop a “grounded faith” (\href{https://suttacentral.net/mn47\#16}{MN 47:16}: \textit{\textsanskrit{ākāravatī} \textsanskrit{saddhā}}) based on careful and critical inquiry.

This is a faith that is akin to the confidence and trust that a scientist needs when relying on the findings and theories of others in their field. It is essential in order to get anywhere; but at the same time, it is completely provisional. If there is anything that is contradicted by the evidence, it should be rejected. And once you have seen the truth for yourself, there is no need for faith, as pointed out by Venerable \textsanskrit{Sāriputta} in \href{https://suttacentral.net/sn48.44}{SN 48.44}.

Following the pattern we have seen in previous sets of qualities, the mention of faith aligns the faculties with the progress of one following the Gradual Training. First one hears the teaching and gains faith, then one goes forth and applies energy in practice, undertaking mindfulness meditation, realizing the absorptions and the wisdom into impermanence that follows on from them. At this point a practitioner goes beyond simple belief or reasoned argument and sees the truth for themselves. Their faith is described as \textit{\textsanskrit{aveccappasāda}}, “experiential” or “confirmed” confidence. The word \textit{avecca} literally means “having undergone”. It is only at this point that faith is unshakable.

Many of suttas on the five faculties are built along the same kinds of patterns and templates as the \textit{\textsanskrit{saṁyuttas}} on the noble eightfold path or the awakening factors. But in two related respects they are quite distinctive. And both of these distinctive features stem from the root sense of \textit{indriya} as “potency” or “potential”.

The first of these two features is the use of the faculties to grade practitioners. One who truly understands the faculties is a stream-enterer (\href{https://suttacentral.net/sn48.2}{SN 48.2}, \href{https://suttacentral.net/sn48.3}{SN 48.3}), while one who, based on this understanding, completely lets go is a perfected one (\href{https://suttacentral.net/sn48.4}{SN 48.4}, \href{https://suttacentral.net/sn48.5}{SN 48.5}). This grading of practitioners based on their development of the faculties is extended in more detail in a further series of discourses (\href{https://suttacentral.net/sn48.12}{SN 48.12–18}, \href{https://suttacentral.net/sn48.24}{SN 48.24}).

So while, in one sense, we all have these faculties within us as a hidden potential, they do not manifest their strength until empowered by the path. Once that happens, at the moment of stream-entry, they are as unstoppable as Indra on a dragon-slaying rampage.

To understand the second distinctive feature, recall that in these \textit{\textsanskrit{saṁyuttas}} we are dealing with the path, the fourth of the noble truths, which is “to be developed” (\textit{\textsanskrit{bhāvetabba}}). And while the Indriya \textsanskrit{Saṁyutta}, like other \textit{\textsanskrit{saṁyuttas}} on the path, does indeed speak of the “development of the faculties”, a series of suttas also speaks of understanding the faculties in light of the four noble truths (\href{https://suttacentral.net/sn48.2}{SN 48.2–7}). Normally such phrasing is found in discourses dealing with the first noble truth, such as those on the aggregates or sense fields, which are “to be fully understood”. Here the texts are blurring the distinction between the first and fourth noble truths. To be sure, this is not unique; we have already noted that a couple of discourses on \textit{\textsanskrit{satipaṭṭhāna}} do a similar thing in a different way. But it is unusual, and certainly the emphasis is unique.

There is nothing doctrinally difficult about this; after all, the path is conditioned (\href{https://suttacentral.net/an4.34}{AN 4.34}), and all conditioned things are suffering. But the Buddha usually spoke of the path in glowingly positive terms, not about its suffering and drawbacks.

Once again, this makes sense when we consider the faculties as inner potentials, as something that we already possess in a latent form to one degree or another. In understanding the faculties we are understanding \emph{who we are} and \emph{who we might become}.

This idea that an \textit{indriya} is a potency or ability or strength possessed by a person is further developed in the remainder of the \textit{\textsanskrit{saṁyutta}}, which introduces a series of faculties beyond the basic five. Together with the five faculties these make up a list of 22 faculties, which became a standard set in the Abhidhamma (see \href{https://suttacentral.net/vb5}{Vb 5}). Here they are in the Abhidhamma sequence:

\begin{itemize}%
\item The six sense faculties (\href{https://suttacentral.net/sn48.25}{SN 48.25}).%
\item Three biological faculties: femininity, masculinity, and vitality (\href{https://suttacentral.net/sn48.22}{SN 48.22}).%
\item Five kinds of feeling as faculties (\href{https://suttacentral.net/sn48.31}{SN 48.31}).%
\item The five spiritual faculties.%
\item Three faculties relating to stages of awakening (\href{https://suttacentral.net/sn48.23}{SN 48.23}).%
\end{itemize}

Senses, feelings, even biological attributes, are things that everyone possesses. They must be understood as part of conditioned reality, and hence suffering, but they can be harnessed to empower the spiritual path.

\subsection*{SN 51: Linked Discourses on the Bases for Psychic Power}

We have learned that the terms \textit{indriya} and \textit{bala}, which we translate as “faculty” and “power” were Vedic terms closely associated with the divine might of the war-god Indra. The current \textit{\textsanskrit{saṁyutta}} deals with \textit{iddhi}, another Vedic term with a similar meaning of “success, power, potency”. Note that the Pali \textit{iddhi} is identical in meaning with two Vedic terms, \textit{siddhi} and \textit{\textsanskrit{ṛddhi}}, but formally it is derived from the latter. \textit{\textsanskrit{Pāda}} literally means “foot”, and since \textit{\textsanskrit{iddhipāda}} is defined as the “path or practice to gaining \textit{iddhi}” (\href{https://suttacentral.net/sn51.27}{SN 51.27}), it’s tempting to maintain the metaphor by speaking of the “four footsteps to psychic power”.

\textit{Iddhis} may refer to various kinds of success, potency, or power, but in this context they consist of various astonishing feats of psychic power or superpowers. Such feats have a long and colorful history in India. In the Vedas, as we have seen, they originated in the legendary military prowess of the gods, to which mortals aspired with the aid of stimulants. As the \textit{soma} vanished, it seems, other means of transcending normal human and physical limits were sought. Ascetics undertook punishing mortifications (\textit{tapas}), torturing their bodies in search of superpowers. While some pre-Buddhist religious practitioners—notably those of the Jains and the \textsanskrit{Upaniṣads}—had set themselves more lofty and worthy goals than mere powers, there remained many for whom spiritual practice was a means to these decidedly worldly ends.

The modern cultural fascination with superheroes shows that this is not bound to a specific cultural time or place. It is about the very human longing for transcendence and transformation, becoming other, becoming more. Superheroes display many of the same kinds of powers talked about in the Buddhist and other ancient Indian texts: mind-reading, enhanced senses, the ability to control the elements or to multiply one’s form, to fly in the sky and even through space (\href{https://suttacentral.net/sn51.11}{SN 51.11}). And the means by which powers are gained remain similar to the pre-Buddhist traditions: they may be of divine or alien origin; or derived from a drug or chemical agent; or the outcome of enduring trials and suffering.

Dispensing with these methods, however, the Buddha said that superpowers are gained through pure mental development or meditation. The focus shifted from the powers themselves to the means for attaining them; which, it turns out, also happens to be the path to awakening. The various powers extend or enhance ordinary human abilities, and they may be developed in the same way as any other ability is developed: by practice.

Despite their frequent mention in Buddhist texts, psychic powers are notably omitted when it comes to the things that are really important. They are side-effects of the spiritual path, things that may be fun and of some worth as preliminary exercises, but far from the true goal (see \href{https://suttacentral.net/sn12.70}{SN 12.70}). The Buddha in fact had a decidedly ambiguous attitude to powers, especially when they were shown off. He forbade the monastics from displaying them publicly, saying monks who make such displays were like a woman who shows her private parts for a cheap coin (\href{https://suttacentral.net/pli{-}tv{-}kd15\#8.2}{Kd 15:8.2}). Displays of psychic powers are moreover criticized because they seem like mere magic (\href{https://suttacentral.net/dn11\#5}{DN 11:5}, \href{https://suttacentral.net/an3.60}{AN 3.60}). And the possession of superpowers was by no means a sign of genuine spiritual attainment, for even Devadatta, the Buddha’s arch-nemesis, was said to have attained them (\href{https://suttacentral.net/pli{-}tv{-}kd17\#1.4}{Kd 17:1.4}).

None of this addresses the question of whether such powers are real. The suttas assume throughout that they are, and there is no reason to think this does not reflect the Buddha’s own views. Traditional Buddhism has always accepted the reality of experiences and powers beyond the normal, and Buddhist cultures are full of anecdotes and stories about such events. Rigorous studies, however, are harder to come by. The extraordinary \textit{Irreducible Mind}, a sustained critique of reductionist theories of mind, assembles hundreds of studies into various kinds of extraordinary phenomena. While a reasonable person may well remain skeptical, it seems there is a significant body of evidence in support of such things as mind-reading or recollection of past lives. The ability to fly or to touch the sun remain, sadly, unattested.

Normally in the suttas the term \textit{iddhi} is used for a specific set of psychic powers, which primarily exhibit mastery over the physical realm (\href{https://suttacentral.net/sn51.19}{SN 51.19}, etc.). These are typically included within a broader set of six “direct knowledges” (\textit{\textsanskrit{abhiññā}}), which are also mentioned in this \textit{\textsanskrit{saṁyutta}} (\href{https://suttacentral.net/sn51.11}{SN 51.11}). The final one of these is the ending of defilements and rebirth, the true goal of Buddhist practice.

As to the substance of the \textit{\textsanskrit{iddhipādas}}, there are four basic terms:

\begin{description}%
\item[Enthusiasm (\textit{chanda})] This is one of the most common words for “desire”. While not formally mentioned as an item in the other lists of the \textit{\textsanskrit{bodhipakkhiyadhammā}}, it appears in the formula for the four right efforts. It is the desire to do good, to practice, to escape suffering.%
\item[Energy (\textit{viriya})] This is the single most common factor among the 37 \textit{\textsanskrit{bodhipakkhiyadhammā}}. However in the bases for psychic power it receives special emphasis as it is not only one of the factors, but also qualifies each of the factors.%
\item[Mind (\textit{citta})] Thought, idea, resolve, or awareness (see below).%
\item[Inquiry (\textit{\textsanskrit{vīmaṁsā}})] Inquiry or investigation into the Dhamma, but especially into what obstructs meditation and what helps it. In this context, it is not too far in meaning from “curiosity”.%
\end{description}

Curiously enough, though the word \textit{citta} has a wide range of meanings, it is not clearly defined in this context. Even the Abhidhamma and commentaries offer little more than the usual list of synonyms for “mind” (\href{https://suttacentral.net/vb9}{Vb 9}). Normally in the context of the path, the mind is “to be developed” (see \href{https://suttacentral.net/sn51.9}{SN 51.9}) and such “development of mind” (\textit{\textsanskrit{cittabhāvanā}}) is a term for \textit{\textsanskrit{samādhi}} and the path to it. Accordingly, \textit{citta} falls between the energy and wisdom factors, in the place normally occupied by \textit{\textsanskrit{samādhi}} and mindfulness, and is said to be developed in the normal way of \textit{\textsanskrit{samādhi}} (see \href{https://suttacentral.net/sn51.11}{SN 51.11}). And \textit{\textsanskrit{samādhi}} itself, like energy, is constantly emphasized as essential to this practice at every point.

However, \textit{citta} is also the thought or intention that gets you to your destination (\href{https://suttacentral.net/sn51.15}{SN 51.15}). In line with this, one gains \textit{\textsanskrit{samādhi}} by relying on \textit{citta} (\href{https://suttacentral.net/sn51.13}{SN 51.13}), which suggests that \textit{citta} cannot be exactly identical with \textit{\textsanskrit{samādhi}}.

Perhaps the term \textit{citta} was used here precisely because of its breadth of meaning. It encompasses the “thought” of the Dhamma, of practice, or of the goal; the “idea” one has in mind that leads one on; the “resolve” that keeps attention focused; the growing “awareness” as the goal comes into view; and the purified “consciousness” of deep meditation. In this way \textit{citta} here covers the same ground as it does as one of the four \textit{\textsanskrit{satipaṭṭhānas}}: it refers to the mind state with which one develops the path, including, but not limited to, states of \textit{\textsanskrit{samādhi}}.

The four bases are almost always presented in a stock formula that consists of a long compound, the meaning of which is explained at \href{https://suttacentral.net/sn51.13}{SN 51.13}. Each of the four qualities may be relied on to develop deep unification of mind, or \textit{\textsanskrit{samādhi}}. This process involves making an active effort, defined in terms of the four right efforts. Thus each of the \textit{\textsanskrit{iddhipādas}} consists of these three aspects:

\begin{enumerate}%
\item One of the four qualities.%
\item The meditative immersion that results.%
\item The effort required.%
\end{enumerate}

At \href{https://suttacentral.net/sn51.20}{SN 51.20} we find the most detailed explanation of how these are applied in practice. This sutta brings in a number of practices familiar from elsewhere in the suttas, such as the contemplation of the 31 parts of the body. While most of these are straightforward, there is a somewhat obscure Pali idiom that begs a little clarification. That is the phrase “as before, so after; as after, so before” (\textit{\textsanskrit{yathā} pure \textsanskrit{tathā} \textsanskrit{pacchā}, \textsanskrit{yathā} \textsanskrit{pacchā} \textsanskrit{tathā} pure}), called the “perception of continuity” (\textit{\textsanskrit{pacchāpuresaññā}}). Similar phrases are found in several places in the context of meditation (\href{https://suttacentral.net/thag6.4}{Thag 6.4}, \href{https://suttacentral.net/sn47.10}{SN 47.10}, \href{https://suttacentral.net/an7.61}{AN 7.61}, \href{https://suttacentral.net/an3.90}{AN 3.90}). In the Vinaya, the same phrase is used to emphasize that the status of a mendicant remains unchanged. In meditation, it points to the need for constant and consistent effort in maintaining one’s focus. As part of a series of related idioms—as above, so below; as by day, so by night; as this is, so is that—it indicates how the process of meditation moves from diversity and differentiation towards unity and oneness.

When first encountering the bases for psychic power, students are often puzzled by an apparent paradox. Desire, so we’re told, is the cause of suffering, yet here we are supposed to develop it. This problem is addressed directly in \href{https://suttacentral.net/sn51.15}{SN 51.15}, where Ānanda explains to the brahmin \textsanskrit{Uṇṇābha} that the spiritual path is lived to give up desire, which is accomplished by developing the four bases of psychic power. But \textsanskrit{Uṇṇābha} protests, for desire is itself one of the four bases, and desire cannot be given up by means of desire. Ānanda resolves the contradiction with the simile of a man walking to a park. Before setting out, one has the desire, the energy, the idea, or the curiosity to reach the park. But when you get there, those things vanish. In the same way, the desire or enthusiasm to reach the goal of spiritual practice carries you to the goal, but once there it is no longer needed.

\section*{A Brief Textual History}

The \textsanskrit{Saṁyutta} \textsanskrit{Nikāya} was edited by M. Léon Feer on the basis of manuscripts in Sinhalese and Burmese scripts, and published in Latin script by the Pali Text Society from 1884 to 1898. The first translation followed in 1917–30 by Mrs C.A.F. Rhys Davids (vols. 1–2) and F.L. Woodward (vols. 3–5) under the title \textit{The Book of the Kindred Sayings}. In 1999 the PTS issued a new edition of the Pali text of vol. 1 \textsanskrit{Sagāthāvagga}, edited by G.A. Somaratne.

While several partial translations were subsequently made, there was no complete new translation in English until 2002, when Bhikkhu Bodhi published his \textit{The Connected Discourses of the Buddha}. As with his translation of the \textit{The Middle-Length Discourses of the Buddha}, this constituted a major leap forward, essentially rendering the earlier translations obsolete. Unlike the \textit{Middle-Length Discourses}, this was an entirely new translation. In an extensive introduction, Bhikkhu Bodhi spelled out his evolving approach to translation and presented a detailed thematic and structural analysis of the text.

Where the Pali was unclear I frequently referred to the earlier work of Bodhi, and rarely to Woodward/Rhys Davids and various translations of specific texts.

%
\chapter*{Acknowledgements}
\addcontentsline{toc}{chapter}{Acknowledgements}
\markboth{Acknowledgements}{Acknowledgements}

I remember with gratitude all those from whom I have learned the Dhamma, especially Ajahn Brahm and Bhikkhu Bodhi, the two monks who more than anyone else showed me the depth, meaning, and practical value of the Suttas.

Special thanks to Dustin and Keiko Cheah and family, who sponsored my stay in Qi Mei while I made this translation.

Thanks also for Blake Walshe, who provided essential software support for my translation work.

Throughout the process of translation, I have frequently sought feedback and suggestions from the community on the SuttaCentral community on our forum, “Discuss and Discover”. I want to thank all those who have made suggestions and contributed to my understanding, as well as to the moderators who have made the forum possible. A special thanks is due to \textsanskrit{Sabbamittā}, a true friend of all, who has tirelessly and precisely checked my work.

Finally my everlasting thanks to all those people, far too many to mention, who have supported SuttaCentral, and those who have supported my life as a monastic. None of this would be possible without you.

%
\mainmatter%
\pagestyle{fancy}%
%
%
\addtocontents{toc}{\let\protect\contentsline\protect\nopagecontentsline}
\part*{Linked Discourses With Deities }
\addcontentsline{toc}{part}{Linked Discourses With Deities }
\markboth{}{}
\addtocontents{toc}{\let\protect\contentsline\protect\oldcontentsline}

%
\addtocontents{toc}{\let\protect\contentsline\protect\nopagecontentsline}
\chapter*{The Chapter on a Reed }
\addcontentsline{toc}{chapter}{\tocchapterline{The Chapter on a Reed }}
\addtocontents{toc}{\let\protect\contentsline\protect\oldcontentsline}

%
\section*{{\suttatitleacronym SN 1.1}{\suttatitletranslation Crossing the Flood }{\suttatitleroot Oghataraṇasutta}}
\addcontentsline{toc}{section}{\tocacronym{SN 1.1} \toctranslation{Crossing the Flood } \tocroot{Oghataraṇasutta}}
\markboth{Crossing the Flood }{Oghataraṇasutta}
\extramarks{SN 1.1}{SN 1.1}

\scevam{So\marginnote{1.1} I have heard. }At one time the Buddha was staying near \textsanskrit{Sāvatthī} in Jeta’s Grove, \textsanskrit{Anāthapiṇḍika}’s Monastery. 

Then,\marginnote{1.3} late at night, a glorious deity, lighting up the entire Jeta’s Grove, went up to the Buddha, bowed, stood to one side, and said to him, “Good sir, how did you cross the flood?” 

“Neither\marginnote{1.5} standing nor swimming, sir, I crossed the flood.” 

“But\marginnote{1.6} in what way did you cross the flood neither standing nor swimming?” 

“When\marginnote{1.7} I stood still, I went under. And when I swam, I was swept away. That’s how I crossed the flood neither standing nor swimming.” 

\begin{verse}%
“After\marginnote{2.1} a long time I see \\
a brahmin extinguished. \\
Neither standing nor swimming, \\
he’s crossed over clinging to the world.” 

%
\end{verse}

This\marginnote{3.1} is what that deity said, and the teacher approved. Then that deity, knowing that the teacher approved, bowed and respectfully circled the Buddha, keeping him on his right, before vanishing right there. 

%
\section*{{\suttatitleacronym SN 1.2}{\suttatitletranslation Liberation }{\suttatitleroot Nimokkhasutta}}
\addcontentsline{toc}{section}{\tocacronym{SN 1.2} \toctranslation{Liberation } \tocroot{Nimokkhasutta}}
\markboth{Liberation }{Nimokkhasutta}
\extramarks{SN 1.2}{SN 1.2}

At\marginnote{1.1} \textsanskrit{Sāvatthī}. 

Then,\marginnote{1.2} late at night, a glorious deity, lighting up the entire Jeta’s Grove, went up to the Buddha, bowed, stood to one side, and said to him, “Good sir, do you understand liberation, emancipation, and seclusion for sentient beings?” 

“I\marginnote{3.1} do, sir.” 

“But\marginnote{4.1} how is it that you understand liberation, emancipation, and seclusion for sentient beings?” 

\begin{verse}%
“With\marginnote{5.1} the ending of relish for rebirth, \\
the finishing of perception and consciousness, \\
and the cessation and stilling of feelings: \\
that, sir, is how I understand liberation, \\
emancipation, and seclusion for sentient beings.” 

%
\end{verse}

%
\section*{{\suttatitleacronym SN 1.3}{\suttatitletranslation Led On }{\suttatitleroot Upanīyasutta}}
\addcontentsline{toc}{section}{\tocacronym{SN 1.3} \toctranslation{Led On } \tocroot{Upanīyasutta}}
\markboth{Led On }{Upanīyasutta}
\extramarks{SN 1.3}{SN 1.3}

At\marginnote{1.1} \textsanskrit{Sāvatthī}. 

Standing\marginnote{1.2} to one side, that deity recited this verse in the Buddha’s presence: 

\begin{verse}%
“This\marginnote{2.1} life, so very short, is led onward; \\
one led on to old age has no shelter. \\
Seeing this peril in death, \\
do good deeds that bring you to joy.” 

“This\marginnote{3.1} life, so very short, is led onward; \\
one led on to old age has no shelter. \\
Seeing this peril in death, \\
a seeker of peace would drop the world’s bait.” 

%
\end{verse}

%
\section*{{\suttatitleacronym SN 1.4}{\suttatitletranslation Time Flies }{\suttatitleroot Accentisutta}}
\addcontentsline{toc}{section}{\tocacronym{SN 1.4} \toctranslation{Time Flies } \tocroot{Accentisutta}}
\markboth{Time Flies }{Accentisutta}
\extramarks{SN 1.4}{SN 1.4}

At\marginnote{1.1} \textsanskrit{Sāvatthī}. 

Standing\marginnote{1.2} to one side, that deity recited this verse in the Buddha’s presence: 

\begin{verse}%
“Time\marginnote{2.1} flies, nights pass by, \\
the stages of life leave us one by one. \\
Seeing this peril in death, \\
do good deeds that bring you to joy.” 

“Time\marginnote{3.1} flies, nights pass by, \\
the stages of life leave us one by one. \\
Seeing this peril in death, \\
a seeker of peace would drop the world’s bait.” 

%
\end{verse}

%
\section*{{\suttatitleacronym SN 1.5}{\suttatitletranslation Cut How Many? }{\suttatitleroot Katichindasutta}}
\addcontentsline{toc}{section}{\tocacronym{SN 1.5} \toctranslation{Cut How Many? } \tocroot{Katichindasutta}}
\markboth{Cut How Many? }{Katichindasutta}
\extramarks{SN 1.5}{SN 1.5}

At\marginnote{1.1} \textsanskrit{Sāvatthī}. 

Standing\marginnote{1.2} to one side, that deity recited this verse in the Buddha’s presence: 

\begin{verse}%
“Cut\marginnote{2.1} how many? Drop how many? \\
How many more should be developed? \\
How many chains must a mendicant escape \\
before you call them a flood-crosser?” 

“Five\marginnote{3.1} to cut, five to drop, \\
and five more to develop. \\
A mendicant must escape five chains \\
before you call them a flood-crosser.” 

%
\end{verse}

%
\section*{{\suttatitleacronym SN 1.6}{\suttatitletranslation Awake }{\suttatitleroot Jāgarasutta}}
\addcontentsline{toc}{section}{\tocacronym{SN 1.6} \toctranslation{Awake } \tocroot{Jāgarasutta}}
\markboth{Awake }{Jāgarasutta}
\extramarks{SN 1.6}{SN 1.6}

At\marginnote{1.1} \textsanskrit{Sāvatthī}. 

Standing\marginnote{1.2} to one side, that deity recited this verse in the Buddha’s presence: 

\begin{verse}%
“How\marginnote{2.1} many sleep while others wake? \\
How many wake among the sleeping? \\
By how many do you gather dust? \\
By how many are you cleansed?” 

“Five\marginnote{3.1} sleep while others wake. \\
Five wake among the sleeping. \\
By five you gather dust. \\
By five you’re cleansed.” 

%
\end{verse}

%
\section*{{\suttatitleacronym SN 1.7}{\suttatitletranslation Not Comprehending }{\suttatitleroot Appaṭividitasutta}}
\addcontentsline{toc}{section}{\tocacronym{SN 1.7} \toctranslation{Not Comprehending } \tocroot{Appaṭividitasutta}}
\markboth{Not Comprehending }{Appaṭividitasutta}
\extramarks{SN 1.7}{SN 1.7}

At\marginnote{1.1} \textsanskrit{Sāvatthī}. 

Standing\marginnote{1.2} to one side, that deity recited this verse in the Buddha’s presence: 

\begin{verse}%
“Those\marginnote{2.1} who don’t comprehend the teachings, \\
who may be led astray by the doctrines of others; \\
asleep, they have not woken up: \\
it is time for them to wake!” 

“Those\marginnote{3.1} who clearly comprehend the teachings, \\
who won’t be led astray by the doctrines of others; \\
they’ve woken up, they rightly know, \\
they live smoothly in the rough.” 

%
\end{verse}

%
\section*{{\suttatitleacronym SN 1.8}{\suttatitletranslation Very Confused }{\suttatitleroot Susammuṭṭhasutta}}
\addcontentsline{toc}{section}{\tocacronym{SN 1.8} \toctranslation{Very Confused } \tocroot{Susammuṭṭhasutta}}
\markboth{Very Confused }{Susammuṭṭhasutta}
\extramarks{SN 1.8}{SN 1.8}

At\marginnote{1.1} \textsanskrit{Sāvatthī}. 

Standing\marginnote{1.2} to one side, that deity recited this verse in the Buddha’s presence: 

\begin{verse}%
“Those\marginnote{2.1} who are very confused about the teachings, \\
who may be led astray by the doctrines of others; \\
asleep, they have not woken up: \\
it is time for them to wake!” 

“Those\marginnote{3.1} who are unconfused about the teachings, \\
who won’t be led astray by the doctrines of others; \\
they’ve woken up, they rightly know, \\
they live smoothly in the rough.” 

%
\end{verse}

%
\section*{{\suttatitleacronym SN 1.9}{\suttatitletranslation Fond of Conceit }{\suttatitleroot Mānakāmasutta}}
\addcontentsline{toc}{section}{\tocacronym{SN 1.9} \toctranslation{Fond of Conceit } \tocroot{Mānakāmasutta}}
\markboth{Fond of Conceit }{Mānakāmasutta}
\extramarks{SN 1.9}{SN 1.9}

At\marginnote{1.1} \textsanskrit{Sāvatthī}. 

Standing\marginnote{1.2} to one side, that deity recited this verse in the Buddha’s presence: 

\begin{verse}%
“Someone\marginnote{2.1} who’s fond of conceit can’t be tamed, \\
and someone without immersion can’t be a sage. \\
Living negligent alone in the wilderness, \\
they can’t pass beyond Death’s domain.” 

“Having\marginnote{3.1} given up conceit, serene within oneself, \\
with a healthy heart, everywhere free; \\
living diligent alone in the wilderness, \\
they pass beyond Death’s domain.” 

%
\end{verse}

%
\section*{{\suttatitleacronym SN 1.10}{\suttatitletranslation Wilderness }{\suttatitleroot Araññasutta}}
\addcontentsline{toc}{section}{\tocacronym{SN 1.10} \toctranslation{Wilderness } \tocroot{Araññasutta}}
\markboth{Wilderness }{Araññasutta}
\extramarks{SN 1.10}{SN 1.10}

At\marginnote{1.1} \textsanskrit{Sāvatthī}. 

Standing\marginnote{1.2} to one side, that deity addressed the Buddha in verse: 

\begin{verse}%
“Living\marginnote{2.1} in the wilderness, \\
peaceful spiritual practitioners \\
eat just one meal a day: \\
so why is their complexion so clear?” 

“They\marginnote{3.1} don’t grieve for the past, \\
nor do they long for the future; \\
they feed on whatever comes that day, \\
that’s why their complexion’s so clear. 

Because\marginnote{4.1} they long for the future, \\
and grieve for the past, \\
fools wither away, \\
like a green reed mowed down.” 

%
\end{verse}

%
\addtocontents{toc}{\let\protect\contentsline\protect\nopagecontentsline}
\chapter*{The Chapter on the Garden of Delight }
\addcontentsline{toc}{chapter}{\tocchapterline{The Chapter on the Garden of Delight }}
\addtocontents{toc}{\let\protect\contentsline\protect\oldcontentsline}

%
\section*{{\suttatitleacronym SN 1.11}{\suttatitletranslation The Garden of Delight }{\suttatitleroot Nandanasutta}}
\addcontentsline{toc}{section}{\tocacronym{SN 1.11} \toctranslation{The Garden of Delight } \tocroot{Nandanasutta}}
\markboth{The Garden of Delight }{Nandanasutta}
\extramarks{SN 1.11}{SN 1.11}

\scevam{So\marginnote{1.1} I have heard. }At one time the Buddha was staying near \textsanskrit{Sāvatthī} in Jeta’s Grove, \textsanskrit{Anāthapiṇḍika}’s monastery. There the Buddha addressed the mendicants, “Mendicants!” 

“Venerable\marginnote{1.5} sir,” they replied. The Buddha said this: 

“Once\marginnote{2.1} upon a time, mendicants, a certain deity of the company of the Thirty-Three was amusing themselves in the Garden of Delight, escorted by a band of nymphs, and supplied and provided with the five kinds of heavenly sensual stimulation. On that occasion they recited this verse: 

\begin{verse}%
‘They\marginnote{3.1} don’t know pleasure \\
who don’t see the Garden of Delight! \\
It’s the abode of lordly gods, \\
the glorious host of Thirty!’ 

%
\end{verse}

When\marginnote{4.1} they had spoken, another deity replied with this verse: 

\begin{verse}%
‘Fool,\marginnote{5.1} don’t you understand \\
the saying of the perfected ones: \\
All conditions are impermanent, \\
their nature is to rise and fall; \\
having arisen, they cease; \\
their stilling is true bliss.’” 

%
\end{verse}

%
\section*{{\suttatitleacronym SN 1.12}{\suttatitletranslation Delight }{\suttatitleroot Nandatisutta}}
\addcontentsline{toc}{section}{\tocacronym{SN 1.12} \toctranslation{Delight } \tocroot{Nandatisutta}}
\markboth{Delight }{Nandatisutta}
\extramarks{SN 1.12}{SN 1.12}

At\marginnote{1.1} \textsanskrit{Sāvatthī}. 

Standing\marginnote{1.2} to one side, that deity recited this verse in the Buddha’s presence: 

\begin{verse}%
“Your\marginnote{2.1} children bring you delight! \\
Your cattle also bring you delight! \\
For attachments are a man’s delight; \\
without attachments there’s no delight.” 

“Your\marginnote{3.1} children bring you sorrow. \\
Your cattle also bring you sorrow. \\
For attachments are a man’s sorrow; \\
without attachments there are no sorrows.” 

%
\end{verse}

%
\section*{{\suttatitleacronym SN 1.13}{\suttatitletranslation There’s Nothing Like a Child }{\suttatitleroot Natthiputtasamasutta}}
\addcontentsline{toc}{section}{\tocacronym{SN 1.13} \toctranslation{There’s Nothing Like a Child } \tocroot{Natthiputtasamasutta}}
\markboth{There’s Nothing Like a Child }{Natthiputtasamasutta}
\extramarks{SN 1.13}{SN 1.13}

At\marginnote{1.1} \textsanskrit{Sāvatthī}. 

Standing\marginnote{1.2} to one side, that deity recited this verse in the Buddha’s presence: 

\begin{verse}%
“There’s\marginnote{2.1} no love like that for a child, \\
no wealth equal to cattle, \\
no light like that of the sun, \\
and of waters the ocean is paramount.” 

“There’s\marginnote{3.1} no love like that for oneself, \\
no wealth equal to grain, \\
no light like that of wisdom, \\
and of waters the rain is paramount.” 

%
\end{verse}

%
\section*{{\suttatitleacronym SN 1.14}{\suttatitletranslation Aristocrats }{\suttatitleroot Khattiyasutta}}
\addcontentsline{toc}{section}{\tocacronym{SN 1.14} \toctranslation{Aristocrats } \tocroot{Khattiyasutta}}
\markboth{Aristocrats }{Khattiyasutta}
\extramarks{SN 1.14}{SN 1.14}

\begin{verse}%
“An\marginnote{1.1} aristocrat is the best of bipeds, \\
an ox is the best of quadrupeds, \\
a maiden is the best of wives, \\
and a first-born the best of sons.” 

“A\marginnote{2.1} Buddha is the best of bipeds, \\
a thoroughbred, the best of quadrupeds, \\
a good listener is the best of wives, \\
and the best of sons is loyal.” 

%
\end{verse}

%
\section*{{\suttatitleacronym SN 1.15}{\suttatitletranslation Whispering }{\suttatitleroot Saṇamānasutta}}
\addcontentsline{toc}{section}{\tocacronym{SN 1.15} \toctranslation{Whispering } \tocroot{Saṇamānasutta}}
\markboth{Whispering }{Saṇamānasutta}
\extramarks{SN 1.15}{SN 1.15}

\begin{verse}%
“In\marginnote{1.1} the still of high noon, \\
when the birds have settled down, \\
the formidable jungle whispers to itself: \\
that seems so scary to me!” 

“In\marginnote{2.1} the still of high noon, \\
when the birds have settled down, \\
the formidable jungle whispers to itself: \\
that seems so delightful to me!” 

%
\end{verse}

%
\section*{{\suttatitleacronym SN 1.16}{\suttatitletranslation Sleepiness and Sloth }{\suttatitleroot Niddātandīsutta}}
\addcontentsline{toc}{section}{\tocacronym{SN 1.16} \toctranslation{Sleepiness and Sloth } \tocroot{Niddātandīsutta}}
\markboth{Sleepiness and Sloth }{Niddātandīsutta}
\extramarks{SN 1.16}{SN 1.16}

\begin{verse}%
“Sleepiness,\marginnote{1.1} sloth, and yawning, \\
discontent, and grogginess after eating: \\
because of this the noble path \\
doesn’t shine for creatures here.” 

“Sleepiness,\marginnote{2.1} sloth, and yawning, \\
discontent, and grogginess after eating: \\
when this is energetically fended off, \\
the noble path is purified.” 

%
\end{verse}

%
\section*{{\suttatitleacronym SN 1.17}{\suttatitletranslation Hard to Do }{\suttatitleroot Dukkarasutta}}
\addcontentsline{toc}{section}{\tocacronym{SN 1.17} \toctranslation{Hard to Do } \tocroot{Dukkarasutta}}
\markboth{Hard to Do }{Dukkarasutta}
\extramarks{SN 1.17}{SN 1.17}

\begin{verse}%
“Hard\marginnote{1.1} to do, hard to endure, \\
is the ascetic life for the inept, \\
for it has many narrow passes \\
where the fool founders.” 

“How\marginnote{2.1} many days could an ascetic live \\
without controlling the mind? \\
They’d founder with each step, \\
under the sway of thoughts. 

A\marginnote{3.1} mendicant should collect their thoughts \\
as a tortoise draws its limbs into its shell. \\
Independent, not disturbing others, \\
quenched: they wouldn’t blame anyone.” 

%
\end{verse}

%
\section*{{\suttatitleacronym SN 1.18}{\suttatitletranslation Conscience }{\suttatitleroot Hirīsutta}}
\addcontentsline{toc}{section}{\tocacronym{SN 1.18} \toctranslation{Conscience } \tocroot{Hirīsutta}}
\markboth{Conscience }{Hirīsutta}
\extramarks{SN 1.18}{SN 1.18}

\begin{verse}%
“Can\marginnote{1.1} a person constrained by conscience \\
be found in the world? \\
Who shies away from blame, \\
like a fine horse from the whip?” 

“Few\marginnote{2.1} are those constrained by conscience, \\
who live always mindful. \\
Having reached the end of suffering, \\
they live smoothly in the rough.” 

%
\end{verse}

%
\section*{{\suttatitleacronym SN 1.19}{\suttatitletranslation Little Hut }{\suttatitleroot Kuṭikāsutta}}
\addcontentsline{toc}{section}{\tocacronym{SN 1.19} \toctranslation{Little Hut } \tocroot{Kuṭikāsutta}}
\markboth{Little Hut }{Kuṭikāsutta}
\extramarks{SN 1.19}{SN 1.19}

\begin{verse}%
“Don’t\marginnote{1.1} you have a little hut? \\
Don’t you have a little nest? \\
Don’t you have any networks? \\
Aren’t you free of shackles?” 

“Indeed\marginnote{2.1} I have no little hut. \\
Indeed I have no little nest. \\
Indeed I have no networks. \\
Indeed I’m free from shackles.” 

“What\marginnote{3.1} do you think I call a little hut? \\
What do I call a little nest? \\
What do you think I call a network? \\
And what do I call a shackle?” 

“You\marginnote{4.1} call a mother a little hut; \\
and a wife a little nest. \\
You call children a network, \\
and you tell me craving’s a shackle.” 

“It’s\marginnote{5.1} good you have no little hut! \\
It’s good you have no little nest! \\
It’s good you have no networks! \\
And good that you’re free from shackles.” 

%
\end{verse}

%
\section*{{\suttatitleacronym SN 1.20}{\suttatitletranslation With Samiddhi }{\suttatitleroot Samiddhisutta}}
\addcontentsline{toc}{section}{\tocacronym{SN 1.20} \toctranslation{With Samiddhi } \tocroot{Samiddhisutta}}
\markboth{With Samiddhi }{Samiddhisutta}
\extramarks{SN 1.20}{SN 1.20}

\scevam{So\marginnote{1.1} I have heard. }At one time the Buddha was staying near \textsanskrit{Rājagaha} in the Hot Springs Monastery. 

Then\marginnote{1.3} Venerable Samiddhi rose at the crack of dawn and went to the hot springs to bathe. When he had bathed and emerged from the water he stood in one robe drying himself. 

Then,\marginnote{1.5} late at night, a glorious deity, lighting up the entire hot springs, went up to Samiddhi, and, standing in the air, addressed him in verse: 

\begin{verse}%
“Mendicant,\marginnote{2.1} you seek alms before you eat; \\
you wouldn’t seek alms after eating. \\
But you should eat first, then seek alms: \\
don’t let the time pass you by.” 

“I\marginnote{3.1} actually don’t know the time; \\
it’s hidden and unseen. \\
That’s why I seek alms before eating, \\
so that the time may not pass me by!” 

%
\end{verse}

Then\marginnote{4.1} that deity landed on the ground and said to Samiddhi, “You’ve gone forth while young, mendicant. You’re black-haired, blessed with youth, in the prime of life, and you’ve never flirted with sensual pleasures. Enjoy human sensual pleasures! Don’t give up what is visible in the present to chase after what takes effect over time.” 

“I’m\marginnote{5.1} not, good sir; I’m giving up what takes effect over time to chase after what is visible in the present. For the Buddha has said that sensual pleasures take effect over time, with much suffering and distress, and they’re all the more full of drawbacks. But this teaching is visible in this very life, immediately effective, inviting inspection, relevant, so that sensible people can know it for themselves.” 

“But\marginnote{6.1} in what way, mendicant, has the Buddha said that sensual pleasures take effect over time, with much suffering and distress, and they’re all the more full of drawbacks? And how is this teaching visible in this very life, immediately effective, inviting inspection, relevant, so that sensible people can know it for themselves?” 

“I’m\marginnote{7.1} junior, good sir, recently gone forth, newly come to this teaching and training. I’m not able to explain this in detail. But the Blessed One, the perfected one, the fully awakened Buddha is staying near \textsanskrit{Rājagaha} in the Hot Springs Monatery. You should go to him and ask about this matter. And you should remember it in line with the Buddha’s answer.” 

“It’s\marginnote{8.1} not easy for us to approach the Buddha, as he is surrounded by other illustrious deities. If you go to the Buddha and ask him about this matter, we’ll come along and listen to the teaching.” 

“Yes,\marginnote{8.3} good sir,” Venerable Samiddhi replied. He went to the Buddha, bowed, sat down to one side, and told him what had happened. Then he added: 

“Sir,\marginnote{17.1} if that deity spoke the truth, he’ll be close by.” 

When\marginnote{18.1} he had spoken, that deity said to Samiddhi, “Ask, mendicant, ask! For I have arrived.” 

Then\marginnote{19.1} the Buddha addressed the deity in verse: 

\begin{verse}%
“Sentient\marginnote{20.1} beings who perceive the communicable, \\
become established in the communicable. \\
Not understanding the communicable, \\
they fall under the yoke of Death. 

But\marginnote{21.1} having fully understood the communicable, \\
they don’t identify as a communicator, \\
for they have nothing \\
by which they might be described. \\
Tell me if you understand, spirit.” 

%
\end{verse}

“I\marginnote{22.1} don’t understand the detailed meaning of the Buddha’s brief statement. Please teach me this matter so I can understand the detailed meaning.” 

\begin{verse}%
“If\marginnote{23.1} you think that ‘I’m equal, \\
special, or worse’, you’ll get into arguments. \\
Unwavering in the face of the three discriminations, \\
you’ll have no thought ‘I’m equal or special’. \\
Tell me if you understand, spirit.” 

%
\end{verse}

“I\marginnote{24.1} don’t understand the detailed meaning of the Buddha’s brief statement. Please teach me this matter so I can understand the detailed meaning.” 

\begin{verse}%
“Judging\marginnote{25.1} is given up, conceit rejected; \\
craving for name and form is cut off right here. \\
They’ve cut the ties, untroubled, with no need for hope. \\
Though gods and humans search for them \\
in this world and the world beyond, they never find them, \\
not in heaven nor in any abode. 

%
\end{verse}

Tell\marginnote{26.1} me if you understand, spirit.” 

“This\marginnote{27.1} is how I understand the detailed meaning of the Buddha’s brief statement: 

\begin{verse}%
You\marginnote{28.1} should never do anything bad \\
by speech or mind or body in all the world. \\
Having given up sensual pleasures, mindful and aware, \\
you shouldn’t keep doing what’s painful and pointless.” 

%
\end{verse}

%
\addtocontents{toc}{\let\protect\contentsline\protect\nopagecontentsline}
\chapter*{The Chapter on a Sword }
\addcontentsline{toc}{chapter}{\tocchapterline{The Chapter on a Sword }}
\addtocontents{toc}{\let\protect\contentsline\protect\oldcontentsline}

%
\section*{{\suttatitleacronym SN 1.21}{\suttatitletranslation A Sword }{\suttatitleroot Sattisutta}}
\addcontentsline{toc}{section}{\tocacronym{SN 1.21} \toctranslation{A Sword } \tocroot{Sattisutta}}
\markboth{A Sword }{Sattisutta}
\extramarks{SN 1.21}{SN 1.21}

At\marginnote{1.1} \textsanskrit{Sāvatthī}. 

Standing\marginnote{1.2} to one side, that deity recited this verse in the Buddha’s presence: 

\begin{verse}%
“Like\marginnote{2.1} they’re struck by a sword, \\
like their head was on fire, \\
a mendicant, mindful, should go forth, \\
to give up sensual desire.” 

“Like\marginnote{3.1} they’re struck by a sword, \\
like their head was on fire, \\
a mendicant, mindful, should go forth, \\
to give up identity view.” 

%
\end{verse}

%
\section*{{\suttatitleacronym SN 1.22}{\suttatitletranslation Impact }{\suttatitleroot Phusatisutta}}
\addcontentsline{toc}{section}{\tocacronym{SN 1.22} \toctranslation{Impact } \tocroot{Phusatisutta}}
\markboth{Impact }{Phusatisutta}
\extramarks{SN 1.22}{SN 1.22}

\begin{verse}%
“It\marginnote{1.1} doesn’t impact a person who doesn’t impact others. \\
It impacts a person because they impact others. \\
That’s why it impacts one who impacts, \\
who wrongs one who’s done no wrong.” 

“Whoever\marginnote{2.1} wrongs a man who’s done no wrong, \\
a pure man who has not a blemish, \\
the evil backfires on the fool, \\
like fine dust thrown upwind.” 

%
\end{verse}

%
\section*{{\suttatitleacronym SN 1.23}{\suttatitletranslation A Tangle }{\suttatitleroot Jaṭāsutta}}
\addcontentsline{toc}{section}{\tocacronym{SN 1.23} \toctranslation{A Tangle } \tocroot{Jaṭāsutta}}
\markboth{A Tangle }{Jaṭāsutta}
\extramarks{SN 1.23}{SN 1.23}

\begin{verse}%
“Tangled\marginnote{1.1} within, tangled without: \\
these people are tangled in tangles. \\
I ask you this, Gotama: \\
who can untangle this tangle?” 

“A\marginnote{2.1} wise person grounded in ethics, \\
developing the mind and wisdom, \\
a keen and alert mendicant—\\
they can untangle this tangle. 

Those\marginnote{3.1} in whom greed, hate, and ignorance \\
have faded away; \\
the perfected ones with defilements ended—\\
they have untangled the tangle. 

And\marginnote{4.1} where name and form \\
cease with nothing left over; \\
as well as impingement and perception of form: \\
it’s there that the tangle is cut.” 

%
\end{verse}

%
\section*{{\suttatitleacronym SN 1.24}{\suttatitletranslation Shielding the Mind }{\suttatitleroot Manonivāraṇasutta}}
\addcontentsline{toc}{section}{\tocacronym{SN 1.24} \toctranslation{Shielding the Mind } \tocroot{Manonivāraṇasutta}}
\markboth{Shielding the Mind }{Manonivāraṇasutta}
\extramarks{SN 1.24}{SN 1.24}

\begin{verse}%
“Whatever\marginnote{1.1} you’ve shielded the mind from \\
can’t cause you suffering. \\
So you should shield the mind from everything, \\
then you’re freed from all suffering.” 

“You\marginnote{2.1} needn’t shield the mind from everything. \\
When the mind is under control, \\
you need only shield the mind \\
from where the bad things come.” 

%
\end{verse}

%
\section*{{\suttatitleacronym SN 1.25}{\suttatitletranslation A Perfected One }{\suttatitleroot Arahantasutta}}
\addcontentsline{toc}{section}{\tocacronym{SN 1.25} \toctranslation{A Perfected One } \tocroot{Arahantasutta}}
\markboth{A Perfected One }{Arahantasutta}
\extramarks{SN 1.25}{SN 1.25}

\begin{verse}%
“When\marginnote{1.1} a mendicant is perfected, proficient, \\
with defilements ended, bearing the final body: \\
would they say, ‘I speak’, \\
or even ‘they speak to me’?” 

“When\marginnote{2.1} a mendicant is perfected, proficient, \\
with defilements ended, bearing the final body: \\
they would say, ‘I speak’, \\
and also ‘they speak to me’. \\
Skillful, understanding the world’s conventions, \\
they’d use these terms as no more than expressions.” 

“When\marginnote{3.1} a mendicant is perfected, proficient, \\
with defilements ended, bearing the final body: \\
is such a mendicant drawing close to conceit \\
if they’d say, ‘I speak’, \\
or even ‘they speak to me’?” 

“Someone\marginnote{4.1} who has given up conceit has no ties, \\
the ties of conceit are all dissipated. \\
Though that clever person has transcended identity, \\
they’d still say, ‘I speak’, 

and\marginnote{5.1} also ‘they speak to me’. \\
Skillful, understanding the world’s conventions, \\
they’d use these terms as no more than expressions.” 

%
\end{verse}

%
\section*{{\suttatitleacronym SN 1.26}{\suttatitletranslation Lamps }{\suttatitleroot Pajjotasutta}}
\addcontentsline{toc}{section}{\tocacronym{SN 1.26} \toctranslation{Lamps } \tocroot{Pajjotasutta}}
\markboth{Lamps }{Pajjotasutta}
\extramarks{SN 1.26}{SN 1.26}

\begin{verse}%
“How\marginnote{1.1} many lamps are there \\
that light up the world? \\
We’ve come to ask the Buddha; \\
how are we to understand this?” 

“There\marginnote{2.1} are four lamps in the world, \\
a fifth is not found. \\
The sun blazes by day, \\
the moon glows at night, 

while\marginnote{3.1} a fire lights up both \\
by day and by night. \\
But a Buddha is the best of lights: \\
this is the supreme radiance.” 

%
\end{verse}

%
\section*{{\suttatitleacronym SN 1.27}{\suttatitletranslation Streams }{\suttatitleroot Sarasutta}}
\addcontentsline{toc}{section}{\tocacronym{SN 1.27} \toctranslation{Streams } \tocroot{Sarasutta}}
\markboth{Streams }{Sarasutta}
\extramarks{SN 1.27}{SN 1.27}

\begin{verse}%
“From\marginnote{1.1} where do streams turn back? \\
Where does the cycle spin no more? \\
Where do name and form \\
cease with nothing left over?” 

“Where\marginnote{2.1} water and earth, \\
fire and air find no footing. \\
From here the streams turn back; \\
here the cycle spins no more; \\
and here it is that name and form \\
cease with nothing left over.” 

%
\end{verse}

%
\section*{{\suttatitleacronym SN 1.28}{\suttatitletranslation Affluent }{\suttatitleroot Mahaddhanasutta}}
\addcontentsline{toc}{section}{\tocacronym{SN 1.28} \toctranslation{Affluent } \tocroot{Mahaddhanasutta}}
\markboth{Affluent }{Mahaddhanasutta}
\extramarks{SN 1.28}{SN 1.28}

\begin{verse}%
“The\marginnote{1.1} affluent and the wealthy, \\
even the aristocrats who rule the land, \\
are jealous of each other, \\
insatiable in sensual pleasures. 

Among\marginnote{2.1} those of such an avid nature, \\
flowing along the stream of lives, \\
who here has given up craving? \\
Who in the world is not avid?” 

“Having\marginnote{3.1} given up their home, their child, their cattle, \\
and all that they love, they went forth. \\
Having given up desire and hate, \\
having dispelled ignorance, \\
the perfected ones with defilements ended—\\
they in the world are not avid.” 

%
\end{verse}

%
\section*{{\suttatitleacronym SN 1.29}{\suttatitletranslation Four Wheels }{\suttatitleroot Catucakkasutta}}
\addcontentsline{toc}{section}{\tocacronym{SN 1.29} \toctranslation{Four Wheels } \tocroot{Catucakkasutta}}
\markboth{Four Wheels }{Catucakkasutta}
\extramarks{SN 1.29}{SN 1.29}

\begin{verse}%
“Four\marginnote{1.1} are its wheels, and nine its doors; \\
it’s stuffed full, bound with greed, \\
and born from a bog. \\
Great hero, how am I supposed to live like this?” 

“Having\marginnote{2.1} cut the strap and harness—\\
the wicked desire and greed—\\
and having plucked out craving, root and all: \\
that’s how you’re supposed to live like this.” 

%
\end{verse}

%
\section*{{\suttatitleacronym SN 1.30}{\suttatitletranslation Antelope Calves }{\suttatitleroot Eṇijaṅghasutta}}
\addcontentsline{toc}{section}{\tocacronym{SN 1.30} \toctranslation{Antelope Calves } \tocroot{Eṇijaṅghasutta}}
\markboth{Antelope Calves }{Eṇijaṅghasutta}
\extramarks{SN 1.30}{SN 1.30}

\begin{verse}%
“O\marginnote{1.1} hero so lean, with antelope calves, \\
not greedy, eating little, \\
an elephant, wandering alone like a lion, \\
you’re not concerned for sensual pleasures. \\
We’ve come to ask a question: \\
How is one released from all suffering?” 

“There\marginnote{2.1} are five kinds of sensual stimulation in the world, \\
and the mind is said to be the sixth. \\
When you’ve discarded desire for these, \\
you’re released from all suffering.” 

%
\end{verse}

%
\addtocontents{toc}{\let\protect\contentsline\protect\nopagecontentsline}
\chapter*{The Chapter on the Satullapa Group }
\addcontentsline{toc}{chapter}{\tocchapterline{The Chapter on the Satullapa Group }}
\addtocontents{toc}{\let\protect\contentsline\protect\oldcontentsline}

%
\section*{{\suttatitleacronym SN 1.31}{\suttatitletranslation Virtuous }{\suttatitleroot Sabbhisutta}}
\addcontentsline{toc}{section}{\tocacronym{SN 1.31} \toctranslation{Virtuous } \tocroot{Sabbhisutta}}
\markboth{Virtuous }{Sabbhisutta}
\extramarks{SN 1.31}{SN 1.31}

\scevam{So\marginnote{1.1} I have heard. }At one time the Buddha was staying near \textsanskrit{Sāvatthī} in Jeta’s Grove, \textsanskrit{Anāthapiṇḍika}’s Monastery. 

Then,\marginnote{1.3} late at night, several glorious deities of the Satullapa Group, lighting up the entire Jeta’s Grove, went up to the Buddha, bowed, and stood to one side. Standing to one side, one deity recited this verse in the Buddha’s presence: 

\begin{verse}%
“Associate\marginnote{2.1} only with the virtuous! \\
Try to get close to the virtuous! \\
Understanding the true teaching of the good, \\
things get better, not worse.” 

%
\end{verse}

Then\marginnote{3.1} another deity recited this verse in the Buddha’s presence: 

\begin{verse}%
“Associate\marginnote{4.1} only with the virtuous! \\
Try to get close to the virtuous! \\
Understanding the true teaching of the good, \\
wisdom is gained—but not from anyone else.” 

%
\end{verse}

Then\marginnote{5.1} another deity recited this verse in the Buddha’s presence: 

\begin{verse}%
“Associate\marginnote{6.1} only with the virtuous! \\
Try to get close to the virtuous! \\
Understanding the true teaching of the good, \\
you grieve not among the grieving.” 

%
\end{verse}

Then\marginnote{7.1} another deity recited this verse in the Buddha’s presence: 

\begin{verse}%
“Associate\marginnote{8.1} only with the virtuous! \\
Try to get close to the virtuous! \\
Understanding the true teaching of the good, \\
you shine among your relatives.” 

%
\end{verse}

Then\marginnote{9.1} another deity recited this verse in the Buddha’s presence: 

\begin{verse}%
“Associate\marginnote{10.1} only with the virtuous! \\
Try to get close to the virtuous! \\
Understanding the true teaching of the good, \\
sentient beings go to a good place.” 

%
\end{verse}

Then\marginnote{11.1} another deity recited this verse in the Buddha’s presence: 

\begin{verse}%
“Associate\marginnote{12.1} only with the virtuous! \\
Try to get close to the virtuous! \\
Understanding the true teaching of the good, \\
sentient beings live happily.” 

%
\end{verse}

Then\marginnote{13.1} another deity said to the Buddha, “Sir, who has spoken well?” 

“You’ve\marginnote{14.1} all spoken well in your own way. However, listen to me also: 

\begin{verse}%
Associate\marginnote{15.1} only with the virtuous! \\
Try to get close to the virtuous! \\
Understanding the true teaching of the good, \\
you’re released from all suffering.” 

%
\end{verse}

That\marginnote{16.1} is what the Buddha said. Then those deities, knowing that the teacher approved, bowed and respectfully circled the Buddha, keeping him on their right, before vanishing right there. 

%
\section*{{\suttatitleacronym SN 1.32}{\suttatitletranslation Stinginess }{\suttatitleroot Maccharisutta}}
\addcontentsline{toc}{section}{\tocacronym{SN 1.32} \toctranslation{Stinginess } \tocroot{Maccharisutta}}
\markboth{Stinginess }{Maccharisutta}
\extramarks{SN 1.32}{SN 1.32}

At\marginnote{1.1} one time the Buddha was staying near \textsanskrit{Sāvatthī} in Jeta’s Grove, \textsanskrit{Anāthapiṇḍika}’s monastery. 

Then,\marginnote{1.2} late at night, several glorious deities of the Satullapa Group, lighting up the entire Jeta’s Grove, went up to the Buddha, bowed, and stood to one side. Standing to one side, one deity recited this verse in the Buddha’s presence: 

\begin{verse}%
“Because\marginnote{2.1} of stinginess and negligence \\
a gift is not given. \\
Wanting merit, \\
a smart person would give.” 

%
\end{verse}

Then\marginnote{3.1} another deity recited these verses in the Buddha’s presence: 

\begin{verse}%
“When\marginnote{4.1} a miser fails to give because of fear, \\
the very thing they’re afraid of comes to pass. \\
The hunger and thirst \\
that a miser fears \\
hurt the fool \\
in this world and the next. 

So\marginnote{5.1} you should dispel stinginess, \\
overcoming that stain, and give a gift. \\
The good deeds of sentient beings \\
support them in the next world.” 

%
\end{verse}

Then\marginnote{6.1} another deity recited these verses in the Buddha’s presence: 

\begin{verse}%
“Among\marginnote{7.1} the dead they do not die, \\
those who, like fellow travelers on the road, \\
are happy to provide, though they have but little. \\
This is an eternal truth. 

Some\marginnote{8.1} who have little are happy to provide, \\
while some who have much don’t wish to give. \\
An offering given from little \\
is multiplied a thousand times.” 

%
\end{verse}

Then\marginnote{9.1} another deity recited these verses in the Buddha’s presence: 

\begin{verse}%
“Giving\marginnote{10.1} what’s hard to give, \\
doing what’s hard to do; \\
the wicked don’t act like this, \\
for the teaching of the good is hard to follow. 

That’s\marginnote{11.1} why the virtuous and the wicked \\
have different destinations after leaving this place. \\
The wicked go to hell, \\
while the virtuous are bound for heaven.” 

%
\end{verse}

Then\marginnote{12.1} another deity said to the Buddha, “Sir, who has spoken well?” 

“You’ve\marginnote{13.1} all spoken well in your own way. However, listen to me also: 

\begin{verse}%
A\marginnote{14.1} hundred thousand people making a thousand sacrifices \\
isn’t worth a fraction \\
of one who lives rightly, wandering for gleanings, \\
or one who supports their partner from what little they have.” 

%
\end{verse}

Then\marginnote{15.1} another deity addressed the Buddha in verse: 

\begin{verse}%
“Why\marginnote{16.1} doesn’t that sacrifice of theirs, so abundant and magnificent, \\
equal the value of a moral person’s gift? \\
How is it that a hundred thousand people making a thousand sacrifices \\
isn’t worth a fraction of what’s offered by such a person?” 

“Some\marginnote{17.1} give based on immorality—\\
after injuring, killing, and tormenting. \\
Such an offering—tearful, violent—\\
in no way equals the value of a moral person’s gift. 

That’s\marginnote{18.1} how it is that a hundred thousand people making a thousand sacrifices \\
isn’t worth a fraction of what’s offered by such a person.” 

%
\end{verse}

%
\section*{{\suttatitleacronym SN 1.33}{\suttatitletranslation Good }{\suttatitleroot Sādhusutta}}
\addcontentsline{toc}{section}{\tocacronym{SN 1.33} \toctranslation{Good } \tocroot{Sādhusutta}}
\markboth{Good }{Sādhusutta}
\extramarks{SN 1.33}{SN 1.33}

At\marginnote{1.1} \textsanskrit{Sāvatthī}. 

Then,\marginnote{1.2} late at night, several glorious deities of the Satullapa Group, lighting up the entire Jeta’s Grove, went up to the Buddha, bowed, and stood to one side. Standing to one side, one deity expressed this heartfelt sentiment in the Buddha’s presence: 

\begin{verse}%
“Good,\marginnote{2.1} sir, is giving! \\
Because of stinginess and negligence \\
a gift is not given. \\
Wanting merit, \\
a smart person would give.” 

%
\end{verse}

Then\marginnote{3.1} another deity expressed this heartfelt sentiment in the Buddha’s presence: 

\begin{verse}%
“Good,\marginnote{4.1} sir, is giving! \\
Even when one has little, giving is good. 

Some\marginnote{5.1} who have little are happy to provide, \\
while some who have much don’t wish to give. \\
An offering given from little \\
is multiplied a thousand times.” 

%
\end{verse}

Then\marginnote{6.1} another deity expressed this heartfelt sentiment in the Buddha’s presence: 

\begin{verse}%
“Good,\marginnote{7.1} sir, is giving! \\
Even when one has little, giving is good. \\
And it’s also good to give out of faith. \\
Giving and warfare are similar, they say, \\
for even a few of the good may conquer the many. \\
If a faithful person gives even a little, \\
it still brings them happiness in the hereafter.” 

%
\end{verse}

Then\marginnote{8.1} another deity expressed this heartfelt sentiment in the Buddha’s presence: 

\begin{verse}%
“Good,\marginnote{9.1} sir, is giving! \\
Even when one has little, giving is good. \\
And it’s also good to give out of faith. \\
And it’s also good to give legitimate wealth. 

A\marginnote{10.1} man who gives legitimate wealth, \\
earned by his efforts and initiative, \\
has passed over Yama’s \textsanskrit{Vetaraṇi} River; \\
that mortal arrives at celestial fields.” 

%
\end{verse}

Then\marginnote{11.1} another deity expressed this heartfelt sentiment in the Buddha’s presence: 

\begin{verse}%
“Good,\marginnote{12.1} sir, is giving! \\
Even when one has little, giving is good. \\
And it’s also good to give out of faith. \\
And it’s also good to give legitimate wealth. \\
And it’s also good to give intelligently. 

The\marginnote{13.1} Holy One praises giving intelligently \\
to those worthy of offerings here in the world of the living. \\
What’s given to these is very fruitful, \\
like seeds sown in a fertile field.” 

%
\end{verse}

Then\marginnote{14.1} another deity expressed this heartfelt sentiment in the Buddha’s presence: 

\begin{verse}%
“Good,\marginnote{15.1} sir, is giving! \\
Even when one has little, giving is good. \\
And it’s also good to give out of faith. \\
And it’s also good to give legitimate wealth. \\
And it’s also good to give intelligently. \\
And it’s also good to be restrained when it comes to living creatures. 

One\marginnote{16.1} who lives without harming any living being \\
never does bad because of others’ blame; \\
for in that case they praise the coward, not the brave; \\
and the virtuous never do bad out of fear.” 

%
\end{verse}

Then\marginnote{17.1} another deity said to the Buddha, “Sir, who has spoken well?” 

“You’ve\marginnote{18.1} all spoken well in your own way. However, listen to me also: 

\begin{verse}%
It’s\marginnote{19.1} true that giving is praised in many ways \\
but the path of the teaching is better than giving, \\
for in days old and older still, \\
the wise and virtuous even attained extinction.” 

%
\end{verse}

%
\section*{{\suttatitleacronym SN 1.34}{\suttatitletranslation There Are None }{\suttatitleroot Nasantisutta}}
\addcontentsline{toc}{section}{\tocacronym{SN 1.34} \toctranslation{There Are None } \tocroot{Nasantisutta}}
\markboth{There Are None }{Nasantisutta}
\extramarks{SN 1.34}{SN 1.34}

At\marginnote{1.1} one time the Buddha was staying near \textsanskrit{Sāvatthī} in Jeta’s Grove, \textsanskrit{Anāthapiṇḍika}’s Monastery. 

Then,\marginnote{1.2} late at night, several glorious deities of the Satullapa Group, lighting up the entire Jeta’s Grove, went up to the Buddha, bowed, and stood to one side. Standing to one side, one deity recited this verse in the Buddha’s presence: 

\begin{verse}%
“Among\marginnote{2.1} humans there are no sensual pleasures that are permanent. \\
Here there are sensuous things, bound to which, \\
drunk on which, there’s no coming back. \\
That person doesn’t return here from Death’s domain.” 

“Misery\marginnote{3.1} is born of desire; suffering is born of desire; \\
when desire is removed, misery is removed; \\
when misery is removed, suffering is removed.” 

“The\marginnote{4.1} world’s pretty things aren’t sensual pleasures. \\
Greedy intention is a person’s sensual pleasure. \\
The world’s pretty things stay just as they are, \\
but a wise one removes desire for them. 

Give\marginnote{5.1} up anger, get rid of conceit, \\
and get past all the fetters. \\
Sufferings don’t torment the one who has nothing, \\
not clinging to name and form. 

Judging’s\marginnote{6.1} given up, conceit rejected; \\
craving for name and form is cut off right here. \\
They’ve cut the ties, untroubled, with no need for hope. \\
Though gods and humans search for them \\
in this world and the world beyond, they never find them, \\
not in heaven nor in any abode.” 

“If\marginnote{7.1} neither gods nor humans see one freed in this way,” \scspeaker{said Venerable \textsanskrit{Mogharājā}, }\\
“in this world or the world beyond, \\
are those who revere that supreme person, \\
who lives for the good of mankind, also worthy of praise?” 

“The\marginnote{8.1} mendicants who revere one freed in this way,” \\
\scspeaker{said the Buddha, }\\
“are also worthy of praise, \textsanskrit{Mogharājā}. \\
But having understood the teaching and given up doubt, \\
those mendicants can escape their chains.” 

%
\end{verse}

%
\section*{{\suttatitleacronym SN 1.35}{\suttatitletranslation Disdain }{\suttatitleroot Ujjhānasaññisutta}}
\addcontentsline{toc}{section}{\tocacronym{SN 1.35} \toctranslation{Disdain } \tocroot{Ujjhānasaññisutta}}
\markboth{Disdain }{Ujjhānasaññisutta}
\extramarks{SN 1.35}{SN 1.35}

At\marginnote{1.1} one time the Buddha was staying near \textsanskrit{Sāvatthī} in Jeta’s Grove, \textsanskrit{Anāthapiṇḍika}’s Monastery. 

Then,\marginnote{1.2} late at night, several glorious deities of the Disdainful Group, lighting up the entire Jeta’s Grove, went up to the Buddha, and stood in the air. Standing in the air, one deity recited this verse in the Buddha’s presence: 

\begin{verse}%
“Someone\marginnote{2.1} who pretends \\
to be other than they really are, \\
is like a cheating gambler \\
who enjoys what was gained by theft. 

You\marginnote{3.1} should only say what you would do; \\
you shouldn’t say what you wouldn’t do. \\
The wise will recognize \\
one who talks without doing.” 

“Not\marginnote{4.1} just by speaking, \\
nor solely by listening, \\
are you able to progress \\
on this hard path, \\
by which wise ones practicing absorption \\
are released from \textsanskrit{Māra}’s bonds. 

The\marginnote{5.1} wise certainly don’t act like that, \\
for they understand the way of the world. \\
The wise are extinguished by understanding, \\
they’ve crossed over clinging to the world.” 

%
\end{verse}

Then\marginnote{6.1} those deities landed on the ground, bowed with their heads at the Buddha’s feet and said, “We have made a mistake, sir. It was foolish, stupid, and unskillful of us to imagine we could attack the Buddha! Please, sir, accept our mistake for what it is, so we will restrain ourselves in future.” 

At\marginnote{6.4} that, the Buddha smiled. 

Then\marginnote{6.5} those deities, becoming even more disdainful, flew up in the air. One deity recited this verse in the Buddha’s presence: 

\begin{verse}%
“If\marginnote{7.1} you don’t give your pardon \\
when a mistake is confessed, \\
with hidden anger and heavy hate, \\
you’re stuck in your enmity.” 

“If\marginnote{8.1} no mistake is found, \\
if no-one’s gone astray, \\
and enmities are settled, \\
then who could have been unskillful?” 

“Who\marginnote{9.1} makes no mistakes? \\
Who doesn’t go astray? \\
Who doesn’t fall into confusion? \\
Who is the wise one that’s ever mindful?” 

“The\marginnote{10.1} Realized One, the Buddha, \\
compassionate for all beings: \\
that’s who makes no mistakes, \\
and that’s who doesn’t go astray. \\
He doesn’t fall into confusion, \\
for he’s the wise one, ever mindful. 

If\marginnote{11.1} you don’t give your pardon \\
when a mistake is confessed, \\
with hidden anger and heavy hate, \\
you’re stuck in your enmity. \\
I don’t approve of such enmity, \\
and so I pardon your mistake.” 

%
\end{verse}

%
\section*{{\suttatitleacronym SN 1.36}{\suttatitletranslation Faith }{\suttatitleroot Saddhāsutta}}
\addcontentsline{toc}{section}{\tocacronym{SN 1.36} \toctranslation{Faith } \tocroot{Saddhāsutta}}
\markboth{Faith }{Saddhāsutta}
\extramarks{SN 1.36}{SN 1.36}

At\marginnote{1.1} one time the Buddha was staying near \textsanskrit{Sāvatthī} in Jeta’s Grove, \textsanskrit{Anāthapiṇḍika}’s monastery. 

Then,\marginnote{1.2} late at night, several glorious deities of the Satullapa Group, lighting up the entire Jeta’s Grove, went up to the Buddha, bowed, and stood to one side. Standing to one side, one deity recited this verse in the Buddha’s presence: 

\begin{verse}%
“Faith\marginnote{2.1} is a person’s partner. \\
If faithlessness doesn’t linger, \\
fame and renown are theirs, \\
and when they discard this corpse they go to heaven.” 

%
\end{verse}

Then\marginnote{3.1} another deity recited these verses in the Buddha’s presence: 

\begin{verse}%
“Give\marginnote{4.1} up anger, get rid of conceit, \\
and get past all the fetters. \\
Chains don’t torment one who has nothing, \\
not clinging to name and form.” 

“Fools\marginnote{5.1} and half-wits \\
devote themselves to negligence. \\
But the wise protect diligence \\
as their best treasure. 

Don’t\marginnote{6.1} devote yourself to negligence, \\
or delight in sexual intimacy. \\
For if you’re diligent and practice absorption, \\
you’ll attain ultimate happiness.” 

%
\end{verse}

%
\section*{{\suttatitleacronym SN 1.37}{\suttatitletranslation The Congregation }{\suttatitleroot Samayasutta}}
\addcontentsline{toc}{section}{\tocacronym{SN 1.37} \toctranslation{The Congregation } \tocroot{Samayasutta}}
\markboth{The Congregation }{Samayasutta}
\extramarks{SN 1.37}{SN 1.37}

\scevam{So\marginnote{1.1} I have heard. }At one time the Buddha was staying in the land of the Sakyans, near Kapilavatthu in the Great Wood, together with a large \textsanskrit{Saṅgha} of around five hundred mendicants, all of whom were perfected ones. And most of the deities from ten solar systems had gathered to see the Buddha and the \textsanskrit{Saṅgha} of mendicants. 

Then\marginnote{1.4} four deities of the Pure Abodes, aware of what was happening, thought: “Why don’t we go to the Buddha and each recite a verse in his presence?” 

Then,\marginnote{2.1} as easily as a strong person would extend or contract their arm, they vanished from the Pure Abodes and reappeared in front of the Buddha. They bowed to the Buddha and stood to one side. Standing to one side, one deity recited this verse in the Buddha’s presence: 

\begin{verse}%
“There’s\marginnote{3.1} a great congregation in the woods, \\
a host of gods have assembled. \\
We’ve come to this righteous congregation \\
to see the invincible \textsanskrit{Saṅgha}!” 

%
\end{verse}

Then\marginnote{4.1} another deity recited this verse in the Buddha’s presence: 

\begin{verse}%
“The\marginnote{5.1} mendicants there have immersion, \\
they’ve straightened out their own minds. \\
Like a charioteer who has taken the reins, \\
the astute ones protect their senses.” 

%
\end{verse}

Then\marginnote{6.1} another deity recited this verse in the Buddha’s presence: 

\begin{verse}%
“Having\marginnote{7.1} cut the stake and cut the bar, \\
they’re unstirred, with boundary post uprooted. \\
They live pure and immaculate, \\
the young dragons tamed by the seer.” 

%
\end{verse}

Then\marginnote{8.1} another deity recited this verse in the Buddha’s presence: 

\begin{verse}%
“Anyone\marginnote{9.1} who has gone to the Buddha for refuge \\
won’t go to a plane of loss. \\
After giving up this human body, \\
they swell the hosts of gods.” 

%
\end{verse}

%
\section*{{\suttatitleacronym SN 1.38}{\suttatitletranslation A Splinter }{\suttatitleroot Sakalikasutta}}
\addcontentsline{toc}{section}{\tocacronym{SN 1.38} \toctranslation{A Splinter } \tocroot{Sakalikasutta}}
\markboth{A Splinter }{Sakalikasutta}
\extramarks{SN 1.38}{SN 1.38}

\scevam{So\marginnote{1.1} I have heard. }At one time the Buddha was staying near \textsanskrit{Rājagaha} in the Maddakucchi deer park. 

Now\marginnote{1.3} at that time the Buddha’s foot had been cut by a splinter. The Buddha was stricken by harrowing pains; physical feelings that were painful, sharp, severe, acute, unpleasant, and disagreeable. But he endured unbothered, with mindfulness and situational awareness. And then he spread out his outer robe folded in four and laid down in the lion’s posture—on the right side, placing one foot on top of the other—mindful and aware. 

Then,\marginnote{2.1} late at night, seven hundred glorious deities of the Satullapa Group, lighting up the entire Maddakucchi, went up to the Buddha, bowed, and stood to one side. 

Standing\marginnote{2.2} to one side, one deity expressed this heartfelt sentiment in the Buddha’s presence: “The ascetic Gotama is such an elephant, sir! And as an elephant, he endures painful physical feelings that have come up—sharp, severe, acute, unpleasant, and disagreeable—unbothered, with mindfulness and situational awareness.” 

Then\marginnote{3.1} another deity expressed this heartfelt sentiment in the Buddha’s presence: “The ascetic Gotama is such a lion, sir! And as a lion, he endures painful physical feelings … unbothered.” 

Then\marginnote{4.1} another deity expressed this heartfelt sentiment in the Buddha’s presence: “The ascetic Gotama is such a thoroughbred, sir! And as a thoroughbred, he endures painful physical feelings … unbothered.” 

Then\marginnote{5.1} another deity expressed this heartfelt sentiment in the Buddha’s presence: “The ascetic Gotama is such a chief bull, sir! And as a chief bull, he endures painful physical feelings … unbothered.” 

Then\marginnote{6.1} another deity expressed this heartfelt sentiment in the Buddha’s presence: “The ascetic Gotama is such a behemoth, sir! And as a behemoth, he endures painful physical feelings … unbothered.” 

Then\marginnote{7.1} another deity expressed this heartfelt sentiment in the Buddha’s presence: “The ascetic Gotama is truly tamed, sir! And as someone tamed, he endures painful physical feelings … unbothered.” 

Then\marginnote{8.1} another deity expressed this heartfelt sentiment in the Buddha’s presence: “See, his immersion is so well developed, and his mind is so well freed—not leaning forward or pulling back, and not held in place by forceful suppression. If anyone imagines that they can overcome such an elephant of a man, a lion of a man, a thoroughbred of a man, a chief bull of a man, a behemoth of a man, a tamed man—what is that but a failure to see?” 

\begin{verse}%
“Learned\marginnote{9.1} in the five Vedas, brahmins practice \\
mortification for a full century. \\
But their minds are not properly freed, \\
for those of base character don’t cross to the far shore. 

Seized\marginnote{10.1} by craving, attached to precepts and observances, \\
they practice rough mortification for a hundred years. \\
But their minds are not properly freed, \\
for those of base character don’t cross to the far shore. 

Someone\marginnote{11.1} who’s fond of conceit can’t be tamed, \\
and someone without immersion can’t be a sage. \\
Living negligent alone in the wilderness, \\
they can’t pass beyond Death’s domain.” 

“Having\marginnote{12.1} given up conceit, serene within oneself, \\
with a healthy heart, everywhere free; \\
living diligent alone in the wilderness, \\
they pass beyond Death’s domain.” 

%
\end{verse}

%
\section*{{\suttatitleacronym SN 1.39}{\suttatitletranslation With Pajjunna’s Daughter (1st) }{\suttatitleroot Paṭhamapajjunnadhītusutta}}
\addcontentsline{toc}{section}{\tocacronym{SN 1.39} \toctranslation{With Pajjunna’s Daughter (1st) } \tocroot{Paṭhamapajjunnadhītusutta}}
\markboth{With Pajjunna’s Daughter (1st) }{Paṭhamapajjunnadhītusutta}
\extramarks{SN 1.39}{SN 1.39}

\scevam{So\marginnote{1.1} I have heard. }At one time the Buddha was staying near \textsanskrit{Vesālī}, at the Great Wood, in the hall with the peaked roof. 

Then,\marginnote{1.3} late at night, the beautiful \textsanskrit{Kokanadā}, Pajjunna’s daughter, lighting up the entire Great Wood, went up to the Buddha, bowed, stood to one side, and recited these verses in the Buddha’s presence: 

\begin{verse}%
“Staying\marginnote{2.1} in the woods of \textsanskrit{Vesālī} \\
is the Buddha, best of beings. \\
\textsanskrit{Kokanadā} am I who worships him, \\
\textsanskrit{Kokanadā}, Pajjuna’s daughter. 

Previously\marginnote{3.1} I had only heard \\
the teaching realized by the seer. \\
But now I know it as a witness \\
while the sage, the Holy One teaches. 

There\marginnote{4.1} are unintelligent people who go about \\
denouncing the teaching of the noble ones. \\
They fall into the terrible Hell of Screams \\
where they suffer long. 

There\marginnote{5.1} are those who have found acceptance and peace \\
in the teaching of the noble ones. \\
After giving up this human body, \\
they swell the hosts of gods.” 

%
\end{verse}

%
\section*{{\suttatitleacronym SN 1.40}{\suttatitletranslation With Pajjunna’s Daughter (2nd) }{\suttatitleroot Dutiyapajjunnadhītusuttaṁ}}
\addcontentsline{toc}{section}{\tocacronym{SN 1.40} \toctranslation{With Pajjunna’s Daughter (2nd) } \tocroot{Dutiyapajjunnadhītusuttaṁ}}
\markboth{With Pajjunna’s Daughter (2nd) }{Dutiyapajjunnadhītusuttaṁ}
\extramarks{SN 1.40}{SN 1.40}

\scevam{So\marginnote{1.1} I have heard. }At one time the Buddha was staying near \textsanskrit{Vesālī}, at the Great Wood, in the hall with the peaked roof. 

Then,\marginnote{1.3} late at night, the beautiful \textsanskrit{Kokanadā} the Younger, Pajjunna’s daughter, lighting up the entire Great Wood, went up to the Buddha, bowed, stood to one side, and recited these verses in the Buddha’s presence: 

\begin{verse}%
“\textsanskrit{Kokanadā},\marginnote{2.1} Pajjunna’s daughter, came here, \\
beautiful as a flash of lightning. \\
Revering the Buddha and the teaching, \\
she spoke these verses full of meaning. 

The\marginnote{3.1} teaching is such that \\
I could analyze it in many different ways. \\
However, I will state the meaning in brief \\
as far as I have learned it by heart. 

You\marginnote{4.1} should never do anything bad \\
by speech or mind or body in all the world. \\
Having given up sensual pleasures, mindful and aware, \\
you shouldn’t keep doing what’s painful and pointless.” 

%
\end{verse}

%
\addtocontents{toc}{\let\protect\contentsline\protect\nopagecontentsline}
\chapter*{The Chapter on Fire }
\addcontentsline{toc}{chapter}{\tocchapterline{The Chapter on Fire }}
\addtocontents{toc}{\let\protect\contentsline\protect\oldcontentsline}

%
\section*{{\suttatitleacronym SN 1.41}{\suttatitletranslation On Fire }{\suttatitleroot Ādittasutta}}
\addcontentsline{toc}{section}{\tocacronym{SN 1.41} \toctranslation{On Fire } \tocroot{Ādittasutta}}
\markboth{On Fire }{Ādittasutta}
\extramarks{SN 1.41}{SN 1.41}

\scevam{So\marginnote{1.1} I have heard. }At one time the Buddha was staying near \textsanskrit{Sāvatthī} in Jeta’s Grove, \textsanskrit{Anāthapiṇḍika}’s Monastery. 

Then,\marginnote{1.3} late at night, a glorious deity, lighting up the entire Jeta’s Grove, went up to the Buddha, bowed, stood to one side, and recited these verses in the Buddha’s presence: 

\begin{verse}%
“When\marginnote{2.1} your house is on fire, \\
you rescue the pot \\
that’s useful, \\
not the one that’s burnt. 

And\marginnote{3.1} as the world is on fire \\
with old age and death, \\
you should rescue by giving, \\
for what’s given is rescued. 

What’s\marginnote{4.1} given has happiness as its fruit, \\
but not what isn’t given. \\
Bandits take it, or rulers, \\
it’s consumed by fire, or lost. 

Then\marginnote{5.1} in the end this corpse is cast off, \\
along with all your possessions. \\
Knowing this, a clever person \\
would enjoy what they have and also give it away. \\
After giving and using according to their means, \\
blameless, they go to a heavenly place.” 

%
\end{verse}

%
\section*{{\suttatitleacronym SN 1.42}{\suttatitletranslation Giving What? }{\suttatitleroot Kiṁdadasutta}}
\addcontentsline{toc}{section}{\tocacronym{SN 1.42} \toctranslation{Giving What? } \tocroot{Kiṁdadasutta}}
\markboth{Giving What? }{Kiṁdadasutta}
\extramarks{SN 1.42}{SN 1.42}

\begin{verse}%
“Giving\marginnote{1.1} what do you give strength? \\
Giving what do you give beauty? \\
Giving what do you give happiness? \\
Giving what do you give vision? \\
And who is the giver of all? \\
Please answer my question.” 

“Giving\marginnote{2.1} food you give strength. \\
Giving clothes you give beauty. \\
Giving a vehicle you give happiness. \\
Giving a lamp you give vision. 

And\marginnote{3.1} the giver of all \\
is the one who gives a residence. \\
But a person who teaches the Dhamma \\
gives the gift of the Deathless.” 

%
\end{verse}

%
\section*{{\suttatitleacronym SN 1.43}{\suttatitletranslation Food }{\suttatitleroot Annasutta}}
\addcontentsline{toc}{section}{\tocacronym{SN 1.43} \toctranslation{Food } \tocroot{Annasutta}}
\markboth{Food }{Annasutta}
\extramarks{SN 1.43}{SN 1.43}

\begin{verse}%
“Both\marginnote{1.1} gods and humans \\
enjoy their food. \\
So what’s the name of the spirit \\
who doesn’t like food?” 

“Those\marginnote{2.1} who give with faith \\
and a clear and confident heart, \\
partake of food \\
in this world and the next. 

So\marginnote{3.1} you should dispel stinginess, \\
overcoming that stain, and give a gift. \\
The good deeds of sentient beings \\
support them in the next world.” 

%
\end{verse}

%
\section*{{\suttatitleacronym SN 1.44}{\suttatitletranslation One Root }{\suttatitleroot Ekamūlasutta}}
\addcontentsline{toc}{section}{\tocacronym{SN 1.44} \toctranslation{One Root } \tocroot{Ekamūlasutta}}
\markboth{One Root }{Ekamūlasutta}
\extramarks{SN 1.44}{SN 1.44}

\begin{verse}%
“One\marginnote{1.1} is the root, two are the whirlpools, \\
three are the stains, five the spreads, \\
twelve the ocean’s whirlpools: \\
such is the abyss crossed over by the hermit.” 

%
\end{verse}

%
\section*{{\suttatitleacronym SN 1.45}{\suttatitletranslation Peerless }{\suttatitleroot Anomasutta}}
\addcontentsline{toc}{section}{\tocacronym{SN 1.45} \toctranslation{Peerless } \tocroot{Anomasutta}}
\markboth{Peerless }{Anomasutta}
\extramarks{SN 1.45}{SN 1.45}

\begin{verse}%
“Behold\marginnote{1.1} him of peerless name who sees the subtle meaning, \\
giver of wisdom, unattached to the realm of sensuality, \\
the all-knower, so very intelligent, \\
the great hermit treading the noble road.” 

%
\end{verse}

%
\section*{{\suttatitleacronym SN 1.46}{\suttatitletranslation Nymphs }{\suttatitleroot Accharāsutta}}
\addcontentsline{toc}{section}{\tocacronym{SN 1.46} \toctranslation{Nymphs } \tocroot{Accharāsutta}}
\markboth{Nymphs }{Accharāsutta}
\extramarks{SN 1.46}{SN 1.46}

\begin{verse}%
“It’s\marginnote{1.1} resounding with a group of nymphs, \\
but haunted by a gang of goblins! \\
This grove is called ‘Delusion’. \\
How am I supposed to live like this?” 

“That\marginnote{2.1} path is called ‘the straight way’, \\
and it’s headed for the place called ‘fearless’. \\
The chariot is called ‘unswerving’, \\
fitted with wheels of skillful thoughts. 

Conscience\marginnote{3.1} is its bench-back, \\
mindfulness its upholstery. \\
I say the teaching is the driver, \\
with right view running out in front. 

Any\marginnote{4.1} woman or man \\
who has such a vehicle, \\
by means of this vehicle \\
has drawn near to extinguishment.” 

%
\end{verse}

%
\section*{{\suttatitleacronym SN 1.47}{\suttatitletranslation Planters }{\suttatitleroot Vanaropasutta}}
\addcontentsline{toc}{section}{\tocacronym{SN 1.47} \toctranslation{Planters } \tocroot{Vanaropasutta}}
\markboth{Planters }{Vanaropasutta}
\extramarks{SN 1.47}{SN 1.47}

\begin{verse}%
“Whose\marginnote{1.1} merit always grows \\
by day and by night? \\
Firm in principle, accomplished in conduct, \\
who is going to heaven?” 

“Planters\marginnote{2.1} of parks or groves, \\
those who build a bridge, \\
a drinking place and well, \\
and those who give a residence. 

Their\marginnote{3.1} merit always grows \\
by day and by night. \\
Firm in principle, accomplished in conduct, \\
they are going to heaven.” 

%
\end{verse}

%
\section*{{\suttatitleacronym SN 1.48}{\suttatitletranslation Jeta’s Grove }{\suttatitleroot Jetavanasutta}}
\addcontentsline{toc}{section}{\tocacronym{SN 1.48} \toctranslation{Jeta’s Grove } \tocroot{Jetavanasutta}}
\markboth{Jeta’s Grove }{Jetavanasutta}
\extramarks{SN 1.48}{SN 1.48}

\begin{verse}%
“This\marginnote{1.1} is indeed that Jeta’s Grove, \\
frequented by the \textsanskrit{Saṅgha} of hermits, \\
where the King of Dhamma stayed: \\
it brings me joy! 

Deeds,\marginnote{2.1} knowledge, and principle; \\
ethical conduct, an excellent livelihood; \\
by these are mortals purified, \\
not by clan or wealth. 

That’s\marginnote{3.1} why an astute person, \\
seeing what’s good for themselves, \\
would examine the teaching properly, \\
and thus be purified in it. 

\textsanskrit{Sāriputta}\marginnote{4.1} has true wisdom, \\
ethics, and also peace. \\
Any mendicant who has crossed over \\
can at best equal him.” 

%
\end{verse}

%
\section*{{\suttatitleacronym SN 1.49}{\suttatitletranslation Stingy }{\suttatitleroot Maccharisutta}}
\addcontentsline{toc}{section}{\tocacronym{SN 1.49} \toctranslation{Stingy } \tocroot{Maccharisutta}}
\markboth{Stingy }{Maccharisutta}
\extramarks{SN 1.49}{SN 1.49}

\begin{verse}%
“Those\marginnote{1.1} folk in the world who are stingy, \\
miserly and abusive, \\
setting up obstacles \\
for others who give. 

What\marginnote{2.1} kind of result do they reap? \\
What kind of future life? \\
We’ve come to ask the Buddha; \\
how are we to understand this?” 

“Those\marginnote{3.1} folk in the world who are stingy, \\
miserly and abusive, \\
setting up obstacles \\
for others who give: 

they’re\marginnote{4.1} reborn in hell, \\
the animal realm, or Yama’s world. \\
If they return to the human state, \\
they’re born in a poor family, 

where\marginnote{5.1} clothes, food, pleasure, and play \\
are hard to find. \\
They don’t even get \\
what they expect from others. \\
This is the result in the present life, \\
and in the next, a bad destination.” 

“We\marginnote{6.1} get what you’re saying, \\
and ask another question, Gotama. \\
What about those who’ve gained the human state, \\
who are bountiful and rid of stinginess, 

confident\marginnote{7.1} in the Buddha and the teaching, \\
with keen respect for the \textsanskrit{Saṅgha}? \\
What kind of result do they reap? \\
What kind of future life? \\
We’ve come to ask the Buddha; \\
how are we to understand this?” 

“Those\marginnote{8.1} who’ve gained the human state \\
who are bountiful and rid of stinginess, \\
confident in the Buddha and the teaching, \\
with keen respect for the \textsanskrit{Saṅgha}: \\
they illuminate the heavens \\
wherever they’re reborn. 

If\marginnote{9.1} they return to the human state, \\
they’re reborn in a rich family, \\
where clothes, food, pleasure, and play \\
are easy to find. 

They\marginnote{10.1} rejoice like those \\
who control the possessions of others. \\
This is the result in the present life, \\
and in the next, a good destination.” 

%
\end{verse}

%
\section*{{\suttatitleacronym SN 1.50}{\suttatitletranslation With Ghaṭīkāra }{\suttatitleroot Ghaṭīkārasutta}}
\addcontentsline{toc}{section}{\tocacronym{SN 1.50} \toctranslation{With Ghaṭīkāra } \tocroot{Ghaṭīkārasutta}}
\markboth{With Ghaṭīkāra }{Ghaṭīkārasutta}
\extramarks{SN 1.50}{SN 1.50}

\begin{verse}%
“Seven\marginnote{1.1} mendicants reborn in Aviha \\
have been freed. \\
With the complete ending of greed and hate, \\
they’ve crossed over clinging to the world.” 

“Who\marginnote{2.1} are those who’ve crossed the bog, \\
Death’s domain so hard to pass? \\
Who, after leaving behind the human body, \\
have risen above celestial yokes?” 

“Upaka\marginnote{3.1} and \textsanskrit{Palagaṇḍa}, \\
and \textsanskrit{Pukkusāti}, these three; \\
Bhaddiya and Bhaddadeva, \\
and \textsanskrit{Bāhudantī} and \textsanskrit{Piṅgiya}. \\
They, after leaving behind the human body, \\
have risen above celestial yokes.” 

“You\marginnote{4.1} speak well of them, \\
who have let go the snares of \textsanskrit{Māra}. \\
Whose teaching did they understand \\
to cut the bonds of rebirth?” 

“None\marginnote{5.1} other than the Blessed One! \\
None other than your instruction! \\
It was your teaching that they understood \\
to cut the bonds of rebirth. 

Where\marginnote{6.1} name and form \\
cease with nothing left over; \\
understanding this teaching, \\
they cut the bonds of rebirth.” 

“The\marginnote{7.1} words you say are deep, \\
hard to understand, so very hard to wake up to. \\
Whose teaching did you understand \\
that you can say such things?” 

“In\marginnote{8.1} the past I was a potter \\
in \textsanskrit{Vebhaliṅga} called \textsanskrit{Ghaṭīkāra}. \\
I took care of my parents \\
as a lay follower of Buddha Kassapa. 

I\marginnote{9.1} refrained from sexual intercourse, \\
I was celibate, spiritual. \\
We lived in the same village; \\
in the past I was your friend. 

I\marginnote{10.1} am the one who understands \\
that these seven mendicants have been freed. \\
With the complete ending of greed and hate, \\
they’ve crossed over clinging to the world.” 

“That’s\marginnote{11.1} exactly how it was, \\
just as you say, Bhaggava. \\
In the past you were a potter \\
in \textsanskrit{Vebhaliṅga} called \textsanskrit{Ghaṭīkāra}. \\
You took care of your parents \\
as a lay follower of Buddha Kassapa. 

You\marginnote{12.1} refrained from sexual intercourse, \\
you were celibate, spiritual. \\
We lived in the same village; \\
in the past you were my friend.” 

“That’s\marginnote{13.1} how it was \\
when those friends of old met again. \\
Both of them are evolved, \\
and bear their final body.” 

%
\end{verse}

%
\addtocontents{toc}{\let\protect\contentsline\protect\nopagecontentsline}
\chapter*{The Chapter on Old Age }
\addcontentsline{toc}{chapter}{\tocchapterline{The Chapter on Old Age }}
\addtocontents{toc}{\let\protect\contentsline\protect\oldcontentsline}

%
\section*{{\suttatitleacronym SN 1.51}{\suttatitletranslation Old Age }{\suttatitleroot Jarāsutta}}
\addcontentsline{toc}{section}{\tocacronym{SN 1.51} \toctranslation{Old Age } \tocroot{Jarāsutta}}
\markboth{Old Age }{Jarāsutta}
\extramarks{SN 1.51}{SN 1.51}

\begin{verse}%
“What’s\marginnote{1.1} still good in old age? \\
What’s good when grounded? \\
What is people’s treasure? \\
What’s hard for thieves to take?” 

“Ethics\marginnote{2.1} are still good in old age. \\
Faith is good when grounded. \\
Wisdom is people’s treasure. \\
Merit’s hard for thieves to take.” 

%
\end{verse}

%
\section*{{\suttatitleacronym SN 1.52}{\suttatitletranslation Getting Old }{\suttatitleroot Ajarasāsutta}}
\addcontentsline{toc}{section}{\tocacronym{SN 1.52} \toctranslation{Getting Old } \tocroot{Ajarasāsutta}}
\markboth{Getting Old }{Ajarasāsutta}
\extramarks{SN 1.52}{SN 1.52}

\begin{verse}%
“What’s\marginnote{1.1} good because it never gets old? \\
What's good when committed? \\
What is people’s treasure? \\
What can thieves never take?” 

“Ethics\marginnote{2.1} are good because they never grow old. \\
Faith is good when committed. \\
Wisdom is people’s treasure. \\
Merit’s what thieves can never take.” 

%
\end{verse}

%
\section*{{\suttatitleacronym SN 1.53}{\suttatitletranslation A Friend }{\suttatitleroot Mittasutta}}
\addcontentsline{toc}{section}{\tocacronym{SN 1.53} \toctranslation{A Friend } \tocroot{Mittasutta}}
\markboth{A Friend }{Mittasutta}
\extramarks{SN 1.53}{SN 1.53}

\begin{verse}%
“Who’s\marginnote{1.1} your friend abroad? \\
Who’s your friend at home? \\
Who’s your friend in need? \\
Who’s your friend in the next life?” 

“A\marginnote{2.1} caravan is your friend abroad. \\
Mother is your friend at home. \\
A comrade in a time of need \\
is a friend time and again. \\
But the good deeds you’ve done yourself—\\
that’s your friend in the next life.” 

%
\end{verse}

%
\section*{{\suttatitleacronym SN 1.54}{\suttatitletranslation Grounds }{\suttatitleroot Vatthusutta}}
\addcontentsline{toc}{section}{\tocacronym{SN 1.54} \toctranslation{Grounds } \tocroot{Vatthusutta}}
\markboth{Grounds }{Vatthusutta}
\extramarks{SN 1.54}{SN 1.54}

\begin{verse}%
“What\marginnote{1.1} is the ground of human beings? \\
What is the best companion here? \\
By what do the creatures who live off the earth \\
sustain their life?” 

“Children\marginnote{2.1} are the ground of human beings. \\
A wife is the best companion. \\
The creatures who live off the earth \\
sustain their life by rain.” 

%
\end{verse}

%
\section*{{\suttatitleacronym SN 1.55}{\suttatitletranslation Gives Birth (1st) }{\suttatitleroot Paṭhamajanasutta}}
\addcontentsline{toc}{section}{\tocacronym{SN 1.55} \toctranslation{Gives Birth (1st) } \tocroot{Paṭhamajanasutta}}
\markboth{Gives Birth (1st) }{Paṭhamajanasutta}
\extramarks{SN 1.55}{SN 1.55}

\begin{verse}%
“What\marginnote{1.1} gives birth to a person? \\
What do they have that runs about? \\
What enters transmigration? \\
What’s their greatest fear?” 

“Craving\marginnote{2.1} gives birth to a person. \\
Their mind is what runs about. \\
A sentient being enters transmigration. \\
Suffering is their greatest fear.” 

%
\end{verse}

%
\section*{{\suttatitleacronym SN 1.56}{\suttatitletranslation Gives Birth (2nd) }{\suttatitleroot Dutiyajanasutta}}
\addcontentsline{toc}{section}{\tocacronym{SN 1.56} \toctranslation{Gives Birth (2nd) } \tocroot{Dutiyajanasutta}}
\markboth{Gives Birth (2nd) }{Dutiyajanasutta}
\extramarks{SN 1.56}{SN 1.56}

\begin{verse}%
“What\marginnote{1.1} gives birth to a person? \\
What do they have that runs about? \\
What enters transmigration? \\
From what are they not free?” 

“Craving\marginnote{2.1} gives birth to a person. \\
Their mind is what runs about. \\
A sentient being enters transmigration. \\
From suffering they are not free.” 

%
\end{verse}

%
\section*{{\suttatitleacronym SN 1.57}{\suttatitletranslation Gives Birth (3rd) }{\suttatitleroot Tatiyajanasutta}}
\addcontentsline{toc}{section}{\tocacronym{SN 1.57} \toctranslation{Gives Birth (3rd) } \tocroot{Tatiyajanasutta}}
\markboth{Gives Birth (3rd) }{Tatiyajanasutta}
\extramarks{SN 1.57}{SN 1.57}

\begin{verse}%
“What\marginnote{1.1} gives birth to a person? \\
What do they have that runs about? \\
What enters transmigration? \\
What is their destiny?” 

“Craving\marginnote{2.1} gives birth to a person. \\
Their mind is what runs about. \\
A sentient being enters transmigration. \\
Deeds are their destiny.” 

%
\end{verse}

%
\section*{{\suttatitleacronym SN 1.58}{\suttatitletranslation Deviation }{\suttatitleroot Uppathasutta}}
\addcontentsline{toc}{section}{\tocacronym{SN 1.58} \toctranslation{Deviation } \tocroot{Uppathasutta}}
\markboth{Deviation }{Uppathasutta}
\extramarks{SN 1.58}{SN 1.58}

\begin{verse}%
“What’s\marginnote{1.1} declared to be a deviation? \\
What is ending day and night? \\
What’s the stain of celibacy? \\
What’s the waterless bath?” 

“Lust\marginnote{2.1} is declared to be a deviation. \\
Youth is ending day and night. \\
Women are the stain of celibacy, \\
to which this generation clings. \\
Austerity and celibacy \\
are the waterless bath.” 

%
\end{verse}

%
\section*{{\suttatitleacronym SN 1.59}{\suttatitletranslation A Partner }{\suttatitleroot Dutiyasutta}}
\addcontentsline{toc}{section}{\tocacronym{SN 1.59} \toctranslation{A Partner } \tocroot{Dutiyasutta}}
\markboth{A Partner }{Dutiyasutta}
\extramarks{SN 1.59}{SN 1.59}

\begin{verse}%
“What\marginnote{1.1} is a person’s partner? \\
What instructs them? \\
Enjoying what is a mortal \\
released from all suffering?” 

“Faith\marginnote{2.1} is a person’s partner. \\
Wisdom instructs them. \\
Delighting in extinguishment a mortal \\
is released from all suffering.” 

%
\end{verse}

%
\section*{{\suttatitleacronym SN 1.60}{\suttatitletranslation A Poet }{\suttatitleroot Kavisutta}}
\addcontentsline{toc}{section}{\tocacronym{SN 1.60} \toctranslation{A Poet } \tocroot{Kavisutta}}
\markboth{A Poet }{Kavisutta}
\extramarks{SN 1.60}{SN 1.60}

\begin{verse}%
“What’s\marginnote{1.1} the basis of verses? \\
What’s their detailed expression? \\
What do verses depend upon? \\
What underlies verses?” 

“Metre\marginnote{2.1} is the basis of verses. \\
Syllables are their detailed expression. \\
Verses depend on names. \\
A poet underlies verses.” 

%
\end{verse}

%
\addtocontents{toc}{\let\protect\contentsline\protect\nopagecontentsline}
\chapter*{The Chapter on Oppressed }
\addcontentsline{toc}{chapter}{\tocchapterline{The Chapter on Oppressed }}
\addtocontents{toc}{\let\protect\contentsline\protect\oldcontentsline}

%
\section*{{\suttatitleacronym SN 1.61}{\suttatitletranslation Name }{\suttatitleroot Nāmasutta}}
\addcontentsline{toc}{section}{\tocacronym{SN 1.61} \toctranslation{Name } \tocroot{Nāmasutta}}
\markboth{Name }{Nāmasutta}
\extramarks{SN 1.61}{SN 1.61}

\begin{verse}%
“What\marginnote{1.1} oppresses everything? \\
What is nothing bigger than? \\
What is the one thing \\
that has everything under its sway?” 

“Name\marginnote{2.1} oppresses everything. \\
Nothing’s bigger than name. \\
Name is the one thing \\
that has everything under its sway.” 

%
\end{verse}

%
\section*{{\suttatitleacronym SN 1.62}{\suttatitletranslation Mind }{\suttatitleroot Cittasutta}}
\addcontentsline{toc}{section}{\tocacronym{SN 1.62} \toctranslation{Mind } \tocroot{Cittasutta}}
\markboth{Mind }{Cittasutta}
\extramarks{SN 1.62}{SN 1.62}

\begin{verse}%
“What\marginnote{1.1} leads the world on? \\
What drags it around? \\
What is the one thing \\
that has everything under its sway?” 

“The\marginnote{2.1} mind leads the world on. \\
The mind drags it around. \\
Mind is the one thing \\
that has everything under its sway.” 

%
\end{verse}

%
\section*{{\suttatitleacronym SN 1.63}{\suttatitletranslation Craving }{\suttatitleroot Taṇhāsutta}}
\addcontentsline{toc}{section}{\tocacronym{SN 1.63} \toctranslation{Craving } \tocroot{Taṇhāsutta}}
\markboth{Craving }{Taṇhāsutta}
\extramarks{SN 1.63}{SN 1.63}

\begin{verse}%
“What\marginnote{1.1} leads the world on? \\
What drags it around? \\
What is the one thing \\
that has everything under its sway?” 

“Craving\marginnote{2.1} leads the world on. \\
Craving drags it around. \\
Craving is the one thing \\
that has everything under its sway.” 

%
\end{verse}

%
\section*{{\suttatitleacronym SN 1.64}{\suttatitletranslation Fetter }{\suttatitleroot Saṁyojanasutta}}
\addcontentsline{toc}{section}{\tocacronym{SN 1.64} \toctranslation{Fetter } \tocroot{Saṁyojanasutta}}
\markboth{Fetter }{Saṁyojanasutta}
\extramarks{SN 1.64}{SN 1.64}

\begin{verse}%
“What\marginnote{1.1} fetters the world? \\
How does it travel about? \\
With the giving up of what \\
is extinguishment spoken of?” 

“Delight\marginnote{2.1} fetters the world. \\
It travels about by means of thought. \\
With the giving up of craving \\
extinguishment is spoken of.” 

%
\end{verse}

%
\section*{{\suttatitleacronym SN 1.65}{\suttatitletranslation Bonds }{\suttatitleroot Bandhanasutta}}
\addcontentsline{toc}{section}{\tocacronym{SN 1.65} \toctranslation{Bonds } \tocroot{Bandhanasutta}}
\markboth{Bonds }{Bandhanasutta}
\extramarks{SN 1.65}{SN 1.65}

\begin{verse}%
“What\marginnote{1.1} binds the world? \\
How does it travel about? \\
With the giving up of what \\
are all bonds severed?” 

“Delight\marginnote{2.1} binds the world. \\
It travels about by means of thought. \\
With the giving up of craving \\
all bonds are severed.” 

%
\end{verse}

%
\section*{{\suttatitleacronym SN 1.66}{\suttatitletranslation Beaten Down }{\suttatitleroot Attahatasutta}}
\addcontentsline{toc}{section}{\tocacronym{SN 1.66} \toctranslation{Beaten Down } \tocroot{Attahatasutta}}
\markboth{Beaten Down }{Attahatasutta}
\extramarks{SN 1.66}{SN 1.66}

\begin{verse}%
“By\marginnote{1.1} what is the world beaten down? \\
By what is it surrounded? \\
What dart has laid it low? \\
With what is it always fuming?” 

“The\marginnote{2.1} world is beaten down by death. \\
It’s surrounded by old age. \\
The dart of craving has struck it down. \\
It’s always fuming with desire.” 

%
\end{verse}

%
\section*{{\suttatitleacronym SN 1.67}{\suttatitletranslation Trapped }{\suttatitleroot Uḍḍitasutta}}
\addcontentsline{toc}{section}{\tocacronym{SN 1.67} \toctranslation{Trapped } \tocroot{Uḍḍitasutta}}
\markboth{Trapped }{Uḍḍitasutta}
\extramarks{SN 1.67}{SN 1.67}

\begin{verse}%
“What\marginnote{1.1} has trapped the world? \\
By what is it surrounded? \\
What has the world fastened shut? \\
On what is the world grounded?” 

“Craving\marginnote{2.1} has trapped the world. \\
It’s surrounded by old age. \\
Mortality has the world fastened shut. \\
The world is grounded on suffering.” 

%
\end{verse}

%
\section*{{\suttatitleacronym SN 1.68}{\suttatitletranslation Fastened Shut }{\suttatitleroot Pihitasutta}}
\addcontentsline{toc}{section}{\tocacronym{SN 1.68} \toctranslation{Fastened Shut } \tocroot{Pihitasutta}}
\markboth{Fastened Shut }{Pihitasutta}
\extramarks{SN 1.68}{SN 1.68}

\begin{verse}%
“What\marginnote{1.1} has the world fastened shut? \\
On what is the world grounded? \\
What has trapped the world? \\
By what is it surrounded?” 

“Mortality\marginnote{2.1} has the world fastened shut. \\
The world is grounded on suffering. \\
Craving has trapped the world. \\
It’s surrounded by old age.” 

%
\end{verse}

%
\section*{{\suttatitleacronym SN 1.69}{\suttatitletranslation Desire }{\suttatitleroot Icchāsutta}}
\addcontentsline{toc}{section}{\tocacronym{SN 1.69} \toctranslation{Desire } \tocroot{Icchāsutta}}
\markboth{Desire }{Icchāsutta}
\extramarks{SN 1.69}{SN 1.69}

\begin{verse}%
“What\marginnote{1.1} is it that binds the world? \\
By removing what is it freed? \\
With the giving up of what \\
are all bonds severed?” 

“Desire\marginnote{2.1} is what binds the world. \\
By the removing of desire it’s freed. \\
With the giving up of craving, \\
all bonds are severed.” 

%
\end{verse}

%
\section*{{\suttatitleacronym SN 1.70}{\suttatitletranslation The World }{\suttatitleroot Lokasutta}}
\addcontentsline{toc}{section}{\tocacronym{SN 1.70} \toctranslation{The World } \tocroot{Lokasutta}}
\markboth{The World }{Lokasutta}
\extramarks{SN 1.70}{SN 1.70}

\begin{verse}%
“What\marginnote{1.1} has the world arisen in? \\
What does it get close to? \\
By grasping what \\
is the world troubled in what?” 

“The\marginnote{2.1} world’s arisen in six. \\
It gets close to six. \\
By grasping at these six, \\
the world’s troubled in six.” 

%
\end{verse}

%
\addtocontents{toc}{\let\protect\contentsline\protect\nopagecontentsline}
\chapter*{The Chapter on Incinerated }
\addcontentsline{toc}{chapter}{\tocchapterline{The Chapter on Incinerated }}
\addtocontents{toc}{\let\protect\contentsline\protect\oldcontentsline}

%
\section*{{\suttatitleacronym SN 1.71}{\suttatitletranslation Incinerated }{\suttatitleroot Chetvāsutta}}
\addcontentsline{toc}{section}{\tocacronym{SN 1.71} \toctranslation{Incinerated } \tocroot{Chetvāsutta}}
\markboth{Incinerated }{Chetvāsutta}
\extramarks{SN 1.71}{SN 1.71}

At\marginnote{1.1} \textsanskrit{Sāvatthī}. 

Standing\marginnote{1.2} to one side, that deity addressed the Buddha in verse: 

\begin{verse}%
“When\marginnote{2.1} what is incinerated do you sleep at ease? \\
When what is incinerated is there no sorrow? \\
What’s the one thing, Gotama, \\
whose killing you approve?” 

“When\marginnote{3.1} anger’s incinerated you sleep at ease. \\
When anger’s incinerated there is no sorrow. \\
O deity, anger has a poisonous root \\
and a honey tip. \\
The noble ones praise its killing, \\
for when it’s incinerated there is no sorrow.” 

%
\end{verse}

%
\section*{{\suttatitleacronym SN 1.72}{\suttatitletranslation A Chariot }{\suttatitleroot Rathasutta}}
\addcontentsline{toc}{section}{\tocacronym{SN 1.72} \toctranslation{A Chariot } \tocroot{Rathasutta}}
\markboth{A Chariot }{Rathasutta}
\extramarks{SN 1.72}{SN 1.72}

\begin{verse}%
“What’s\marginnote{1.1} the mark of a chariot? \\
What’s the mark of fire? \\
What’s the mark of a nation? \\
What’s the mark of a woman?” 

“A\marginnote{2.1} banner is the mark of a chariot. \\
Smoke is the mark of fire. \\
A ruler is a nation’s mark. \\
And a husband is the mark of a woman.” 

%
\end{verse}

%
\section*{{\suttatitleacronym SN 1.73}{\suttatitletranslation Wealth }{\suttatitleroot Vittasutta}}
\addcontentsline{toc}{section}{\tocacronym{SN 1.73} \toctranslation{Wealth } \tocroot{Vittasutta}}
\markboth{Wealth }{Vittasutta}
\extramarks{SN 1.73}{SN 1.73}

\begin{verse}%
“What’s\marginnote{1.1} a person’s best wealth? \\
What brings happiness when practiced well? \\
What’s the sweetest taste of all? \\
The one who they say has the best life: how do they live?” 

“Faith\marginnote{2.1} here is a person’s best wealth. \\
The teaching brings happiness when practiced well. \\
Truth is the sweetest taste of all. \\
The one who they say has the best life lives by wisdom.” 

%
\end{verse}

%
\section*{{\suttatitleacronym SN 1.74}{\suttatitletranslation Rain }{\suttatitleroot Vuṭṭhisutta}}
\addcontentsline{toc}{section}{\tocacronym{SN 1.74} \toctranslation{Rain } \tocroot{Vuṭṭhisutta}}
\markboth{Rain }{Vuṭṭhisutta}
\extramarks{SN 1.74}{SN 1.74}

\begin{verse}%
“What’s\marginnote{1.1} the best of things that rise? \\
And what’s the finest of things that fall? \\
And what of the things that go forth? \\
And who’s the finest speaker?” 

“A\marginnote{2.1} seed’s the best of things that rise. \\
Rain’s the finest thing that falls. \\
Cattle, of things that go forth. \\
And a child is the finest speaker.” 

“Knowledge\marginnote{3.1} is best of things that rise. \\
Ignorance the finest thing that falls. \\
The \textsanskrit{Saṅgha}, of things that go forth. \\
And the Buddha is the finest speaker.” 

%
\end{verse}

%
\section*{{\suttatitleacronym SN 1.75}{\suttatitletranslation Afraid }{\suttatitleroot Bhītāsutta}}
\addcontentsline{toc}{section}{\tocacronym{SN 1.75} \toctranslation{Afraid } \tocroot{Bhītāsutta}}
\markboth{Afraid }{Bhītāsutta}
\extramarks{SN 1.75}{SN 1.75}

\begin{verse}%
“Why\marginnote{1.1} are so many people here afraid, \\
when the path has been taught with so many dimensions? \\
I ask you, Gotama, whose wisdom is vast: \\
Standing on what need one not fear the next world?” 

“When\marginnote{2.1} speech and mind are directed right, \\
and you don’t do anything bad with the body \\
while dwelling at home with plenty of food and drink. \\
Faithful, gentle, charitable, and bountiful: \\
standing on these four principles, \\
standing on the teaching one need not fear the next world.” 

%
\end{verse}

%
\section*{{\suttatitleacronym SN 1.76}{\suttatitletranslation Getting Old }{\suttatitleroot Najīratisutta}}
\addcontentsline{toc}{section}{\tocacronym{SN 1.76} \toctranslation{Getting Old } \tocroot{Najīratisutta}}
\markboth{Getting Old }{Najīratisutta}
\extramarks{SN 1.76}{SN 1.76}

\begin{verse}%
“What\marginnote{1.1} gets old, what doesn’t grow old? \\
What’s called a deviation? \\
What’s a roadblock for skillful qualities? \\
What is ending day and night? \\
What’s the stain of celibacy? \\
What’s the waterless bath? 

How\marginnote{2.1} many holes are there in the world, \\
where one’s wealth leaks out? \\
We’ve come to ask the Buddha; \\
how are we to understand this?” 

“The\marginnote{3.1} physical form of mortals gets old, \\
but their name and clan don’t. \\
Lust is called a deviation, 

and\marginnote{4.1} greed obstructs skillful qualities. \\
Youth is ending day and night. \\
Women are the stain of celibacy, \\
to which this generation clings. \\
Austerity and celibacy \\
are the waterless bath. 

There\marginnote{5.1} are six holes in the world, \\
where one’s wealth leaks out: \\
laziness and negligence, \\
lack of initiative and lack of restraint, \\
sleepiness and sloth. \\
You should completely get rid of these holes!” 

%
\end{verse}

%
\section*{{\suttatitleacronym SN 1.77}{\suttatitletranslation Authority }{\suttatitleroot Issariyasutta}}
\addcontentsline{toc}{section}{\tocacronym{SN 1.77} \toctranslation{Authority } \tocroot{Issariyasutta}}
\markboth{Authority }{Issariyasutta}
\extramarks{SN 1.77}{SN 1.77}

\begin{verse}%
“What\marginnote{1.1} is authority in the world? \\
What’s the best of valuables? \\
What in the world is a rusty sword? \\
Who is a plague on the world? 

Who\marginnote{2.1} gets arrested when they take things away? \\
And who is loved when they take things away? \\
And who is approved by the astute \\
when they come again and again?” 

“Power\marginnote{3.1} is authority in the world. \\
A woman is the best of valuables. \\
Anger in the world is a rusty sword. \\
A bandit is a plague on the world. 

A\marginnote{4.1} bandit gets arrested when they take things away. \\
And an ascetic is loved when they take things away. \\
An ascetic is approved by the astute \\
when they come again and again.” 

%
\end{verse}

%
\section*{{\suttatitleacronym SN 1.78}{\suttatitletranslation Desire }{\suttatitleroot Kāmasutta}}
\addcontentsline{toc}{section}{\tocacronym{SN 1.78} \toctranslation{Desire } \tocroot{Kāmasutta}}
\markboth{Desire }{Kāmasutta}
\extramarks{SN 1.78}{SN 1.78}

\begin{verse}%
“What\marginnote{1.1} should one who desires the good not give away? \\
What should a mortal not reject? \\
What should be let out when it’s good, \\
but not when it’s bad?” 

“A\marginnote{2.1} man shouldn’t give away himself. \\
He shouldn’t reject himself. \\
Speech should be let out when it’s good, \\
but not when it’s bad.” 

%
\end{verse}

%
\section*{{\suttatitleacronym SN 1.79}{\suttatitletranslation Provisions }{\suttatitleroot Pātheyyasutta}}
\addcontentsline{toc}{section}{\tocacronym{SN 1.79} \toctranslation{Provisions } \tocroot{Pātheyyasutta}}
\markboth{Provisions }{Pātheyyasutta}
\extramarks{SN 1.79}{SN 1.79}

\begin{verse}%
“How\marginnote{1.1} should provisions be tied up? \\
What’s the lair of wealth? \\
What drags a person around? \\
What in the world is hard to give up? \\
What are many beings tied up with, \\
like birds in a snare?” 

“Provisions\marginnote{2.1} should be tied up with faith. \\
Glory is the lair of wealth. \\
Desire drags a person around. \\
Desire in the world is hard to give up. \\
Many beings are tied up with desire, \\
like birds in a snare.” 

%
\end{verse}

%
\section*{{\suttatitleacronym SN 1.80}{\suttatitletranslation Lamp }{\suttatitleroot Pajjotasutta}}
\addcontentsline{toc}{section}{\tocacronym{SN 1.80} \toctranslation{Lamp } \tocroot{Pajjotasutta}}
\markboth{Lamp }{Pajjotasutta}
\extramarks{SN 1.80}{SN 1.80}

\begin{verse}%
“What’s\marginnote{1.1} the lamp for the world? \\
What in the world is wakeful? \\
Who are one’s work colleagues? \\
What is one’s walk of life? 

What\marginnote{2.1} nurtures the idle and the tireless, \\
like a mother her child? \\
By what do the creatures who live off the earth \\
sustain their life?” 

“Wisdom\marginnote{3.1} is the lamp for the world. \\
Mindfulness in the world is wakeful. \\
Cattle are one’s work colleagues, \\
and the furrow is one’s walk of life. 

Rain\marginnote{4.1} nurtures the idle and the tireless, \\
like a mother her child. \\
The creatures who live on the earth \\
sustain their life by rain.” 

%
\end{verse}

%
\section*{{\suttatitleacronym SN 1.81}{\suttatitletranslation Without Conflict }{\suttatitleroot Araṇasutta}}
\addcontentsline{toc}{section}{\tocacronym{SN 1.81} \toctranslation{Without Conflict } \tocroot{Araṇasutta}}
\markboth{Without Conflict }{Araṇasutta}
\extramarks{SN 1.81}{SN 1.81}

\begin{verse}%
“Who\marginnote{1.1} in the world has no conflict? \\
Whose life is not lost? \\
Who here completely understands desire? \\
Who always lives as their own master? 

To\marginnote{2.1} whom do mother, father, and brothers \\
bow when they’re established? \\
Who here, though of low birth, \\
is bowed to even by aristocrats?” 

“Ascetics\marginnote{3.1} have no conflict in the world. \\
The life of ascetics is not lost. \\
Ascetics completely understand desire. \\
Ascetics always live as their own master. 

Mother,\marginnote{4.1} father, and brothers \\
bow to ascetics when they’re established. \\
Even though an ascetic is of low birth, \\
they’re bowed to even by aristocrats.” 

%
\end{verse}

\scendsutta{The Linked Discourses on Deities are complete. }

%
\addtocontents{toc}{\let\protect\contentsline\protect\nopagecontentsline}
\part*{Linked Discourses on Gods }
\addcontentsline{toc}{part}{Linked Discourses on Gods }
\markboth{}{}
\addtocontents{toc}{\let\protect\contentsline\protect\oldcontentsline}

%
\addtocontents{toc}{\let\protect\contentsline\protect\nopagecontentsline}
\chapter*{Chapter One }
\addcontentsline{toc}{chapter}{\tocchapterline{Chapter One }}
\addtocontents{toc}{\let\protect\contentsline\protect\oldcontentsline}

%
\section*{{\suttatitleacronym SN 2.1}{\suttatitletranslation With Kassapa (1st) }{\suttatitleroot Paṭhamakassapasutta}}
\addcontentsline{toc}{section}{\tocacronym{SN 2.1} \toctranslation{With Kassapa (1st) } \tocroot{Paṭhamakassapasutta}}
\markboth{With Kassapa (1st) }{Paṭhamakassapasutta}
\extramarks{SN 2.1}{SN 2.1}

\scevam{So\marginnote{1.1} I have heard. }At one time the Buddha was staying near \textsanskrit{Sāvatthī} in Jeta’s Grove, \textsanskrit{Anāthapiṇḍika}’s monastery. 

Then,\marginnote{1.3} late at night, the glorious god Kassapa, lighting up the entire Jeta’s Grove, went up to the Buddha, bowed, stood to one side, and said to him, “The Buddha has revealed the mendicant, but not his instructions to a mendicant.” 

“Well\marginnote{1.5} then, Kassapa, clarify this matter yourself.” 

\begin{verse}%
“They\marginnote{2.1} should train in following good advice, \\
in attending closely to ascetics, \\
in sitting alone in hidden places, \\
and in calming the mind.” 

%
\end{verse}

That’s\marginnote{3.1} what the god Kassapa said, and the teacher approved. Then Kassapa, knowing that the teacher approved, bowed and respectfully circled the Buddha, keeping him on his right, before vanishing right there. 

%
\section*{{\suttatitleacronym SN 2.2}{\suttatitletranslation With Kassapa (2nd) }{\suttatitleroot Dutiyakassapasutta}}
\addcontentsline{toc}{section}{\tocacronym{SN 2.2} \toctranslation{With Kassapa (2nd) } \tocroot{Dutiyakassapasutta}}
\markboth{With Kassapa (2nd) }{Dutiyakassapasutta}
\extramarks{SN 2.2}{SN 2.2}

At\marginnote{1.1} \textsanskrit{Sāvatthī}. 

Standing\marginnote{1.2} to one side, the god Kassapa recited this verse in the Buddha’s presence: 

\begin{verse}%
“Suppose\marginnote{2.1} a mendicant is a meditator, freed in mind. \\
If they want to reach the heart’s peace, \\
having known the arising and passing of the world, \\
healthy-minded, independent, that is their reward.” 

%
\end{verse}

%
\section*{{\suttatitleacronym SN 2.3}{\suttatitletranslation With Māgha }{\suttatitleroot Māghasutta}}
\addcontentsline{toc}{section}{\tocacronym{SN 2.3} \toctranslation{With Māgha } \tocroot{Māghasutta}}
\markboth{With Māgha }{Māghasutta}
\extramarks{SN 2.3}{SN 2.3}

At\marginnote{1.1} \textsanskrit{Sāvatthī}. 

Then,\marginnote{1.2} late at night, the glorious god \textsanskrit{Māgha}, lighting up the entire Jeta’s Grove, went up to the Buddha, bowed, stood to one side, and addressed the Buddha in verse: 

\begin{verse}%
“When\marginnote{2.1} what is incinerated do you sleep at ease? \\
When what is incinerated is there no sorrow? \\
What is the one thing \\
whose killing you approve?” 

“When\marginnote{3.1} anger’s incinerated you sleep at ease. \\
When anger’s incinerated there is no sorrow. \\
\textsanskrit{Vatrabhū}, anger has a poisonous root, \\
and a honey tip. \\
The noble ones praise the slaying of anger, \\
for when it’s incinerated there is no sorrow.” 

%
\end{verse}

%
\section*{{\suttatitleacronym SN 2.4}{\suttatitletranslation With Māghadha }{\suttatitleroot Māgadhasutta}}
\addcontentsline{toc}{section}{\tocacronym{SN 2.4} \toctranslation{With Māghadha } \tocroot{Māgadhasutta}}
\markboth{With Māghadha }{Māgadhasutta}
\extramarks{SN 2.4}{SN 2.4}

At\marginnote{1.1} \textsanskrit{Sāvatthī}. 

Standing\marginnote{1.2} to one side, the god \textsanskrit{Māgadha} addressed the Buddha in verse: 

\begin{verse}%
“How\marginnote{2.1} many lamps are there \\
that light up the world? \\
We’ve come to ask the Buddha; \\
how are we to understand this?” 

“There\marginnote{3.1} are four lamps in the world, \\
a fifth is not found. \\
The sun blazes by day, \\
the moon glows at night, 

while\marginnote{4.1} a fire lights up both \\
by day and by night. \\
But a Buddha is the best of lights: \\
this is the supreme radiance.” 

%
\end{verse}

%
\section*{{\suttatitleacronym SN 2.5}{\suttatitletranslation With Dāmali }{\suttatitleroot Dāmalisutta}}
\addcontentsline{toc}{section}{\tocacronym{SN 2.5} \toctranslation{With Dāmali } \tocroot{Dāmalisutta}}
\markboth{With Dāmali }{Dāmalisutta}
\extramarks{SN 2.5}{SN 2.5}

At\marginnote{1.1} \textsanskrit{Sāvatthī}. 

Then,\marginnote{1.2} late at night, the glorious god \textsanskrit{Dāmali}, lighting up the entire Jeta’s Grove, went up to the Buddha, bowed, stood to one side, and recited this verse in the Buddha’s presence: 

\begin{verse}%
“This\marginnote{2.1} is what should be done by a brahmin: \\
unrelenting striving. \\
Then, with the giving up of sensual pleasures, \\
they won’t hope to be reborn.” 

“The\marginnote{3.1} brahmin has nothing left to do,” \scspeaker{said the Buddha to \textsanskrit{Dāmali}, }\\
“for they’ve completed their task. \\
So long as a person fails to gain a footing in the river, \\
they strive with every limb. \\
But someone who has gained a footing and stands on dry land \\
need not strive, for they have reached the far shore. 

\textsanskrit{Dāmali},\marginnote{4.1} this is a simile for the brahmin, \\
alert, a meditator who has ended defilements. \\
Since they’ve reached the end of rebirth and death, \\
they need not strive, for they have reached the far shore.” 

%
\end{verse}

%
\section*{{\suttatitleacronym SN 2.6}{\suttatitletranslation With Kāmada }{\suttatitleroot Kāmadasutta}}
\addcontentsline{toc}{section}{\tocacronym{SN 2.6} \toctranslation{With Kāmada } \tocroot{Kāmadasutta}}
\markboth{With Kāmada }{Kāmadasutta}
\extramarks{SN 2.6}{SN 2.6}

At\marginnote{1.1} \textsanskrit{Sāvatthī}. 

Standing\marginnote{1.2} to one side, the god \textsanskrit{Kāmada} said to the Buddha, “It’s too hard, Blessed One! It’s just too hard!” 

\begin{verse}%
“They\marginnote{2.1} do it even though it’s hard,” \\
\scspeaker{said the Buddha to \textsanskrit{Kāmada}, }\\
“the stable trainees with ethics, and immersion. \\
For one who has entered the homeless life, \\
contentment brings happiness.” 

%
\end{verse}

“Such\marginnote{3.1} contentment, Blessed One, is hard to find.” 

\begin{verse}%
“They\marginnote{4.1} find it even though it’s hard,” \\
\scspeaker{said the Buddha to \textsanskrit{Kāmada}, }\\
“those who love peace of mind; \\
whose minds love to meditate \\
day and night.” 

%
\end{verse}

“But\marginnote{5.1} it’s hard, Blessed One, to immerse this mind in \textsanskrit{samādhi}.” 

\begin{verse}%
“They\marginnote{6.1} become immersed in \textsanskrit{samādhi} even though it’s hard,” \\
\scspeaker{said the Buddha to \textsanskrit{Kāmada}, }\\
“those who love calming the faculties. \\
Having cut through the net of Death, \\
the noble ones, \textsanskrit{Kāmada}, go on their way.” 

%
\end{verse}

“But\marginnote{7.1} this path, Blessed One, is rough and hard to travel.” 

\begin{verse}%
“Though\marginnote{8.1} it’s rough, hard to travel, \\
the noble ones, \textsanskrit{Kāmada}, go on their way. \\
The ignoble fall headfirst \\
on a rough path. \\
But the path of the noble ones is smooth, \\
for the noble ones are smooth amid the rough.” 

%
\end{verse}

%
\section*{{\suttatitleacronym SN 2.7}{\suttatitletranslation With Pañcālacaṇḍa }{\suttatitleroot Pañcālacaṇḍasutta}}
\addcontentsline{toc}{section}{\tocacronym{SN 2.7} \toctranslation{With Pañcālacaṇḍa } \tocroot{Pañcālacaṇḍasutta}}
\markboth{With Pañcālacaṇḍa }{Pañcālacaṇḍasutta}
\extramarks{SN 2.7}{SN 2.7}

At\marginnote{1.1} \textsanskrit{Sāvatthī}. 

Standing\marginnote{1.2} to one side, the god \textsanskrit{Pañcālacaṇḍa} recited this verse in the Buddha’s presence: 

\begin{verse}%
“The\marginnote{2.1} opening amid confinement \\
was discovered by the Buddha of vast intelligence, \\
who woke up to absorption, \\
the sage, the solitary bull.” 

“Even\marginnote{3.1} amid confinement they discover,” \scspeaker{said the Buddha to \textsanskrit{Pañcālacaṇḍa}, }\\
“the principle for attaining extinguishment. \\
Those who have acquired mindfulness \\
are perfectly serene in \textsanskrit{samādhi}.” 

%
\end{verse}

%
\section*{{\suttatitleacronym SN 2.8}{\suttatitletranslation With Tāyana }{\suttatitleroot Tāyanasutta}}
\addcontentsline{toc}{section}{\tocacronym{SN 2.8} \toctranslation{With Tāyana } \tocroot{Tāyanasutta}}
\markboth{With Tāyana }{Tāyanasutta}
\extramarks{SN 2.8}{SN 2.8}

At\marginnote{1.1} \textsanskrit{Sāvatthī}. 

Then,\marginnote{1.2} late at night, the glorious god \textsanskrit{Tāyana}, formerly a religious founder, lighting up the entire Jeta’s Grove, went up to the Buddha, bowed, stood to one side, and recited these verses in the Buddha’s presence: 

\begin{verse}%
“Strive\marginnote{2.1} and cut the stream! \\
Dispel sensual pleasures, brahmin. \\
A sage who doesn’t give up sensual pleasures \\
is not reborn in a unified state. 

If\marginnote{3.1} one is to do what should be done, \\
one should staunchly strive. \\
For the life gone forth when laxly led \\
just stirs up dust all the more. 

A\marginnote{4.1} bad deed is better left undone, \\
for it will plague you later on. \\
A good deed is better done, \\
one that does not plague you. 

When\marginnote{5.1} kusa grass is wrongly grasped \\
it only cuts the hand. \\
So too, the ascetic life, when wrongly taken, \\
drags you to hell. 

Any\marginnote{6.1} lax act, \\
any corrupt observance, \\
or suspicious spiritual life, \\
is not very fruitful.” 

%
\end{verse}

That’s\marginnote{7.1} what the god \textsanskrit{Tāyana} said. Then he bowed and respectfully circled the Buddha, keeping him on his right side, before vanishing right there. 

Then,\marginnote{8.1} when the night had passed, the Buddha told the mendicants all that had happened. 

“Mendicants,\marginnote{8.2} tonight, the glorious god \textsanskrit{Tāyana}, formerly a religious founder, lighting up the entire Jeta’s Grove, came to me, bowed, stood to one side, and recited these verses in my presence.” The Buddha repeated the verses in full, adding: 

“That’s\marginnote{14.1} what the god \textsanskrit{Tāyana} said. Then he bowed and respectfully circled me, keeping me on his right side, before vanishing right there. Mendicants, learn the verses of \textsanskrit{Tāyana}! Memorize the verses of \textsanskrit{Tāyana}! Remember the verses of \textsanskrit{Tāyana}! These verses are beneficial and relate to the fundamentals of the spiritual life.” 

%
\section*{{\suttatitleacronym SN 2.9}{\suttatitletranslation The Moon }{\suttatitleroot Candimasutta}}
\addcontentsline{toc}{section}{\tocacronym{SN 2.9} \toctranslation{The Moon } \tocroot{Candimasutta}}
\markboth{The Moon }{Candimasutta}
\extramarks{SN 2.9}{SN 2.9}

At\marginnote{1.1} \textsanskrit{Sāvatthī}. 

Now\marginnote{1.2} at that time the Moon God had been seized by \textsanskrit{Rāhu}, lord of demons. Then the Moon God, recollecting the Buddha, at that time recited this verse: 

\begin{verse}%
“Homage\marginnote{2.1} to you, Buddha, hero! \\
You’re free in every way. \\
I’ve wandered into confinement: \\
be my refuge!” 

%
\end{verse}

Then\marginnote{3.1} the Buddha addressed \textsanskrit{Rāhu} in verse concerning the Moon God: 

\begin{verse}%
“The\marginnote{4.1} Moon God has gone for refuge \\
to the Realized One, the perfected one. \\
The Buddhas have compassion for the world—\\
so \textsanskrit{Rāhu}, release the Moon!” 

%
\end{verse}

Then\marginnote{5.1} \textsanskrit{Rāhu}, having released the Moon, rushed to see Vepacitti, lord of demons and stood to one side, shocked and awestruck. Vepacitti addressed him in verse: 

\begin{verse}%
“Why\marginnote{6.1} the rush? \\
\textsanskrit{Rāhu}, you released the Moon \\
and came here looking like you’re in shock: \\
why do you stand there so scared?” 

“My\marginnote{7.1} head would have exploded in seven pieces, \\
I would have found no happiness in life, \\
if, when enchanted by the Buddha’s spell, \\
I had not released the Moon.” 

%
\end{verse}

%
\section*{{\suttatitleacronym SN 2.10}{\suttatitletranslation The Sun }{\suttatitleroot Sūriyasutta}}
\addcontentsline{toc}{section}{\tocacronym{SN 2.10} \toctranslation{The Sun } \tocroot{Sūriyasutta}}
\markboth{The Sun }{Sūriyasutta}
\extramarks{SN 2.10}{SN 2.10}

At\marginnote{1.1} \textsanskrit{Sāvatthī}. 

Now\marginnote{1.2} at that time the Sun God had been seized by \textsanskrit{Rāhu}, lord of demons. Then the Sun God, recollecting the Buddha, at that time recited this verse: 

\begin{verse}%
“Homage\marginnote{2.1} to you, Buddha, hero! \\
You’re everywhere free. \\
I’ve wandered into confinement: \\
be my refuge!” 

%
\end{verse}

Then\marginnote{3.1} the Buddha addressed \textsanskrit{Rāhu} in verse concerning the Sun God: 

\begin{verse}%
“The\marginnote{4.1} Sun God has gone for refuge \\
to the Realized One, the perfected one. \\
The Buddhas have compassion for the world—\\
so \textsanskrit{Rāhu}, release the Sun! 

He\marginnote{5.1} is a beacon in the darkness, \\
the blazing sun, circle of magnificent flame. \\
\textsanskrit{Rāhu}, do not swallow him as he traverses the sky. \\
\textsanskrit{Rāhu}, release my offspring, the Sun!” 

%
\end{verse}

Then\marginnote{6.1} \textsanskrit{Rāhu}, having released the Sun, rushed to see Vepacitti, lord of demons and stood to one side, shocked and awestruck. Vepacitti addressed him in verse: 

\begin{verse}%
“Why\marginnote{7.1} the rush? \\
\textsanskrit{Rāhu}, you released the Sun \\
and came here looking like you’re in shock: \\
why do you stand there so scared?” 

“My\marginnote{8.1} head would have exploded in seven pieces, \\
I would have found no joy in life, \\
if, when enchanted by the Buddha’s spell, \\
I had not released the Sun.” 

%
\end{verse}

%
\addtocontents{toc}{\let\protect\contentsline\protect\nopagecontentsline}
\chapter*{The Chapter with Anāthapiṇḍika }
\addcontentsline{toc}{chapter}{\tocchapterline{The Chapter with Anāthapiṇḍika }}
\addtocontents{toc}{\let\protect\contentsline\protect\oldcontentsline}

%
\section*{{\suttatitleacronym SN 2.11}{\suttatitletranslation With Candimasa }{\suttatitleroot Candimasasutta}}
\addcontentsline{toc}{section}{\tocacronym{SN 2.11} \toctranslation{With Candimasa } \tocroot{Candimasasutta}}
\markboth{With Candimasa }{Candimasasutta}
\extramarks{SN 2.11}{SN 2.11}

At\marginnote{1.1} \textsanskrit{Sāvatthī}. 

Then,\marginnote{1.2} late at night, the glorious god Candimasa, lighting up the entire Jeta’s Grove, went up to the Buddha, bowed, stood to one side, and recited this verse in the Buddha’s presence: 

\begin{verse}%
“Like\marginnote{2.1} deer in a mosquito-free marsh, \\
they will reach a safe place \\
having entered the absorptions, \\
unified, alert, and mindful.” 

“Like\marginnote{3.1} fish when the net is cut, \\
they will reach the far shore \\
having entered the absorptions, \\
diligent, with vices discarded.” 

%
\end{verse}

%
\section*{{\suttatitleacronym SN 2.12}{\suttatitletranslation With Vishnu }{\suttatitleroot Veṇḍusutta}}
\addcontentsline{toc}{section}{\tocacronym{SN 2.12} \toctranslation{With Vishnu } \tocroot{Veṇḍusutta}}
\markboth{With Vishnu }{Veṇḍusutta}
\extramarks{SN 2.12}{SN 2.12}

Standing\marginnote{1.1} to one side, the god Vishnu recited this verse in the Buddha’s presence: 

\begin{verse}%
“Happy\marginnote{2.1} are the children of Manu \\
who pay homage to the Holy One! \\
They apply themselves to Gotama’s instructions, \\
diligently training.” 

“Those\marginnote{3.1} who practice absorption in accord with the training”, \scspeaker{said the Buddha to Vishnu, }\\
“in the way of teaching I’ve proclaimed, \\
they’re in time to be diligent; \\
they won’t fall under the sway of Death.” 

%
\end{verse}

%
\section*{{\suttatitleacronym SN 2.13}{\suttatitletranslation With Dīghalaṭṭhi }{\suttatitleroot Dīghalaṭṭhisutta}}
\addcontentsline{toc}{section}{\tocacronym{SN 2.13} \toctranslation{With Dīghalaṭṭhi } \tocroot{Dīghalaṭṭhisutta}}
\markboth{With Dīghalaṭṭhi }{Dīghalaṭṭhisutta}
\extramarks{SN 2.13}{SN 2.13}

\scevam{So\marginnote{1.1} I have heard. }At one time the Buddha was staying near \textsanskrit{Rājagaha}, in the Bamboo Grove, the squirrels’ feeding ground. 

Then,\marginnote{1.3} late at night, the glorious god \textsanskrit{Dīghalaṭṭhi}, lighting up the entire Bamboo Grove, went up to the Buddha, bowed, stood to one side, and recited this verse in the Buddha’s presence: 

\begin{verse}%
“Suppose\marginnote{2.1} a mendicant is a meditator, freed in mind. \\
If they want to reach the heart’s peace, \\
having known the arising and passing of the world, \\
healthy-minded, independent, that is their reward.” 

%
\end{verse}

%
\section*{{\suttatitleacronym SN 2.14}{\suttatitletranslation With Nandana }{\suttatitleroot Nandanasutta}}
\addcontentsline{toc}{section}{\tocacronym{SN 2.14} \toctranslation{With Nandana } \tocroot{Nandanasutta}}
\markboth{With Nandana }{Nandanasutta}
\extramarks{SN 2.14}{SN 2.14}

Standing\marginnote{1.1} to one side, the god Nandana addressed the Buddha in verse: 

\begin{verse}%
“I\marginnote{2.1} ask you, Gotama, whose wisdom is vast, \\
the Blessed One of unhindered knowledge and vision. \\
What kind of person do they call ethical? \\
What kind of person do they call wise? \\
What kind of person lives on after transcending suffering? \\
What kind of person is worshipped by the deities?” 

“A\marginnote{3.1} person who is ethical, wise, evolved, \\
becomes serene, loving absorption, mindful, \\
who’s gotten rid of and given up all sorrows, \\
with defilements ended, they bear their final body. 

That’s\marginnote{4.1} the kind of person they call ethical. \\
That’s the kind of person they call wise. \\
That kind of person lives on after transcending suffering. \\
That kind of person is worshipped by the deities.” 

%
\end{verse}

%
\section*{{\suttatitleacronym SN 2.15}{\suttatitletranslation With Candana }{\suttatitleroot Candanasutta}}
\addcontentsline{toc}{section}{\tocacronym{SN 2.15} \toctranslation{With Candana } \tocroot{Candanasutta}}
\markboth{With Candana }{Candanasutta}
\extramarks{SN 2.15}{SN 2.15}

Standing\marginnote{1.1} to one side, the god Candana addressed the Buddha in verse: 

\begin{verse}%
“Who\marginnote{2.1} here crosses the flood, \\
tireless all day and night? \\
Who, not standing and unsupported, \\
does not sink in the deep?” 

“Someone\marginnote{3.1} who is always endowed with ethics, \\
wise and serene, \\
energetic and resolute, \\
crosses the flood so hard to cross. 

Someone\marginnote{4.1} who desists from sensual perception, \\
has moved past the fetter of form, \\
and has finished with relishing and greed \\
does not sink in the deep.” 

%
\end{verse}

%
\section*{{\suttatitleacronym SN 2.16}{\suttatitletranslation With Vāsudatta }{\suttatitleroot Vāsudattasutta}}
\addcontentsline{toc}{section}{\tocacronym{SN 2.16} \toctranslation{With Vāsudatta } \tocroot{Vāsudattasutta}}
\markboth{With Vāsudatta }{Vāsudattasutta}
\extramarks{SN 2.16}{SN 2.16}

Standing\marginnote{1.1} to one side, the god \textsanskrit{Vāsudatta} recited this verse in the Buddha’s presence: 

\begin{verse}%
“Like\marginnote{2.1} they’re struck by a sword, \\
like their head was on fire, \\
a mendicant should wander mindful, \\
to give up sensual desire.” 

“Like\marginnote{3.1} they’re struck by a sword, \\
like their head was on fire, \\
a mendicant should wander mindful, \\
to give up identity view.” 

%
\end{verse}

%
\section*{{\suttatitleacronym SN 2.17}{\suttatitletranslation With Subrahmā }{\suttatitleroot Subrahmasutta}}
\addcontentsline{toc}{section}{\tocacronym{SN 2.17} \toctranslation{With Subrahmā } \tocroot{Subrahmasutta}}
\markboth{With Subrahmā }{Subrahmasutta}
\extramarks{SN 2.17}{SN 2.17}

Standing\marginnote{1.1} to one side, the god \textsanskrit{Subrahmā} addressed the Buddha in verse: 

\begin{verse}%
“This\marginnote{2.1} mind is always anxious, \\
this mind is always stressed \\
about stresses that haven’t arisen \\
and those that have. \\
If there is a state free of anxiety, \\
please answer my question.” 

“Not\marginnote{3.1} without understanding and austerity, \\
not without restraining the sense faculties, \\
not without letting go of everything, \\
do I see safety for living creatures.” 

%
\end{verse}

That\marginnote{4.1} is what the Buddha said. … The god vanished right there. 

%
\section*{{\suttatitleacronym SN 2.18}{\suttatitletranslation With Kakudha }{\suttatitleroot Kakudhasutta}}
\addcontentsline{toc}{section}{\tocacronym{SN 2.18} \toctranslation{With Kakudha } \tocroot{Kakudhasutta}}
\markboth{With Kakudha }{Kakudhasutta}
\extramarks{SN 2.18}{SN 2.18}

\scevam{So\marginnote{1.1} I have heard. }At one time the Buddha was staying near \textsanskrit{Sāketa} in the deer park at the \textsanskrit{Añjana} Wood. 

Then,\marginnote{1.3} late at night, the glorious god Kakudha, lighting up the entire \textsanskrit{Añjana} Wood, went up to the Buddha, bowed, stood to one side, and said to him, “Do you delight, ascetic?” 

“What\marginnote{1.5} have I gained, sir?” 

“Well\marginnote{1.6} then, ascetic, do you sorrow?” 

“What\marginnote{1.7} have I lost, sir?” 

“Well\marginnote{1.8} then, ascetic, do you neither delight nor sorrow?” 

“Yes,\marginnote{1.9} sir.” 

\begin{verse}%
“I\marginnote{2.1} hope you’re untroubled, mendicant, \\
I hope that delight isn’t found in you. \\
I hope that discontent doesn’t \\
overwhelm you as you sit alone.” 

“I’m\marginnote{3.1} genuinely untroubled, spirit, \\
and no delight is found in me. \\
And also discontent doesn’t \\
overwhelm me as I sit alone.” 

“How\marginnote{4.1} are you untroubled, mendicant? \\
How is delight not found in you? \\
How does discontent not \\
overwhelm you as you sit alone?” 

“Delight\marginnote{5.1} is born from misery, \\
misery is born from delight; \\
sir, you should know me as \\
a mendicant free of delight and misery.” 

“After\marginnote{6.1} a long time I see \\
a brahmin extinguished. \\
A mendicant free of delight and misery, \\
he has crossed over clinging to the world.” 

%
\end{verse}

%
\section*{{\suttatitleacronym SN 2.19}{\suttatitletranslation With Uttara }{\suttatitleroot Uttarasutta}}
\addcontentsline{toc}{section}{\tocacronym{SN 2.19} \toctranslation{With Uttara } \tocroot{Uttarasutta}}
\markboth{With Uttara }{Uttarasutta}
\extramarks{SN 2.19}{SN 2.19}

At\marginnote{1.1} \textsanskrit{Rājagaha}. Standing to one side, the god Uttara recited this verse in the Buddha’s presence: 

\begin{verse}%
“This\marginnote{2.1} life, so very short, is led onward. \\
There’s no shelter for one led on by old age. \\
Seeing this peril in death, \\
do good deeds that bring happiness.” 

“This\marginnote{3.1} life, so very short, is led onward. \\
There’s no shelter for one led on by old age. \\
Seeing this peril in death, \\
a seeker of peace would drop the world’s bait.” 

%
\end{verse}

%
\section*{{\suttatitleacronym SN 2.20}{\suttatitletranslation With Anāthapiṇḍika }{\suttatitleroot Anāthapiṇḍikasutta}}
\addcontentsline{toc}{section}{\tocacronym{SN 2.20} \toctranslation{With Anāthapiṇḍika } \tocroot{Anāthapiṇḍikasutta}}
\markboth{With Anāthapiṇḍika }{Anāthapiṇḍikasutta}
\extramarks{SN 2.20}{SN 2.20}

Standing\marginnote{1.1} to one side, the god \textsanskrit{Anāthapiṇḍika} recited these verses in the Buddha’s presence: 

\begin{verse}%
“This\marginnote{2.1} is indeed that Jeta’s Grove, \\
frequented by the \textsanskrit{Saṅgha} of hermits, \\
where the King of Dhamma stayed: \\
it brings me joy! 

Deeds,\marginnote{3.1} knowledge, and principle; \\
ethical conduct, an excellent livelihood; \\
by these are mortals purified, \\
not by clan or wealth. 

That’s\marginnote{4.1} why an astute person, \\
seeing what’s good for themselves, \\
would examine the teaching rationally, \\
and thus be purified in it. 

\textsanskrit{Sāriputta}\marginnote{5.1} has true wisdom, \\
ethics, and also peace. \\
Any mendicant who has gone beyond \\
can at best equal him.” 

%
\end{verse}

This\marginnote{6.1} is what the god \textsanskrit{Anāthapiṇḍika} said. Then he bowed and respectfully circled the Buddha, keeping him on his right side, before vanishing right there. 

Then,\marginnote{7.1} when the night had passed, the Buddha addressed the mendicants: “Mendicants, tonight, a certain glorious god, lighting up the entire Jeta’s Grove, came to me, bowed, stood to one side, and recited these verses in my presence.” The Buddha then repeated the verses in full. 

When\marginnote{13.1} he said this, Venerable Ānanda said to the Buddha, “Sir, that god must surely have been \textsanskrit{Anāthapiṇḍika}. For the householder \textsanskrit{Anāthapiṇḍika} was devoted to Venerable \textsanskrit{Sāriputta}.” 

“Good,\marginnote{13.4} good, Ānanda. You’ve reached the logical conclusion, as far as logic goes. For that was indeed the god \textsanskrit{Anāthapiṇḍika}.” 

%
\addtocontents{toc}{\let\protect\contentsline\protect\nopagecontentsline}
\chapter*{The Chapter on Various Sectarians }
\addcontentsline{toc}{chapter}{\tocchapterline{The Chapter on Various Sectarians }}
\addtocontents{toc}{\let\protect\contentsline\protect\oldcontentsline}

%
\section*{{\suttatitleacronym SN 2.21}{\suttatitletranslation With Shiva }{\suttatitleroot Sivasutta}}
\addcontentsline{toc}{section}{\tocacronym{SN 2.21} \toctranslation{With Shiva } \tocroot{Sivasutta}}
\markboth{With Shiva }{Sivasutta}
\extramarks{SN 2.21}{SN 2.21}

\scevam{So\marginnote{1.1} I have heard. }At one time the Buddha was staying near \textsanskrit{Sāvatthī} in Jeta’s Grove, \textsanskrit{Anāthapiṇḍika}’s monastery. 

Then,\marginnote{1.3} late at night, the glorious god Shiva, lighting up the entire Jeta’s Grove, went up to the Buddha, bowed, stood to one side, and recited these verses in the Buddha’s presence: 

\begin{verse}%
“Associate\marginnote{2.1} only with the virtuous! \\
Try to get close to the virtuous! \\
Understanding the true teaching of the good, \\
things get better, not worse. 

Associate\marginnote{3.1} only with the virtuous! \\
Try to get close to the virtuous! \\
Understanding the true teaching of the good, \\
wisdom is gained—but not from anyone else. 

Associate\marginnote{4.1} only with the virtuous! \\
Try to get close to the virtuous! \\
Understanding the true teaching of the good, \\
you grieve not among the grieving. 

Associate\marginnote{5.1} only with the virtuous! \\
Try to get close to the virtuous! \\
Understanding the true teaching of the good, \\
you shine among your relatives. 

Associate\marginnote{6.1} only with the virtuous! \\
Try to get close to the virtuous! \\
Understanding the true teaching of the good, \\
sentient beings go to a good place. 

Associate\marginnote{7.1} only with the virtuous! \\
Try to get close to the virtuous! \\
Understanding the true teaching of the good, \\
sentient beings live happily.” 

%
\end{verse}

Then\marginnote{8.1} the Buddha replied to Shiva in verse: 

\begin{verse}%
“Associate\marginnote{9.1} only with the virtuous! \\
Try to get close to the virtuous! \\
Understanding the true teaching of the good, \\
you’re released from all suffering.” 

%
\end{verse}

%
\section*{{\suttatitleacronym SN 2.22}{\suttatitletranslation With Khema }{\suttatitleroot Khemasutta}}
\addcontentsline{toc}{section}{\tocacronym{SN 2.22} \toctranslation{With Khema } \tocroot{Khemasutta}}
\markboth{With Khema }{Khemasutta}
\extramarks{SN 2.22}{SN 2.22}

Standing\marginnote{1.1} to one side, the god Khema recited these verses in the Buddha’s presence: 

\begin{verse}%
“Witless\marginnote{2.1} fools behave \\
like their own worst enemies, \\
doing wicked deeds \\
that ripen as bitter fruit. 

It’s\marginnote{3.1} not good to do a deed \\
that plagues you later on, \\
for which you weep and wail, \\
as its effect stays with you. 

It\marginnote{4.1} is good to do a deed \\
that doesn’t plague you later on, \\
that gladdens and cheers, \\
as its effect stays with you.” 

“As\marginnote{5.1} a precaution, you should do \\
what you know is for your own welfare. \\
A thinker, a wise one would not proceed \\
thinking like the cart driver. 

Suppose\marginnote{6.1} a cart driver leaves the highway, \\
so even and well compacted. \\
They enter upon a rough road, \\
and fret when their axle breaks. 

So\marginnote{7.1} too, an idiot departs the good \\
to follow what’s against the good. \\
Fallen in the jaws of death, \\
they fret like their axle’s broken.” 

%
\end{verse}

%
\section*{{\suttatitleacronym SN 2.23}{\suttatitletranslation With Serī }{\suttatitleroot Serīsutta}}
\addcontentsline{toc}{section}{\tocacronym{SN 2.23} \toctranslation{With Serī } \tocroot{Serīsutta}}
\markboth{With Serī }{Serīsutta}
\extramarks{SN 2.23}{SN 2.23}

Standing\marginnote{1.1} to one side, the god \textsanskrit{Serī} addressed the Buddha in verse: 

\begin{verse}%
“Both\marginnote{2.1} gods and humans \\
enjoy their food. \\
So what’s the name of the spirit \\
who doesn’t like food?” 

“Those\marginnote{3.1} who give with faith \\
and a clear and confident heart, \\
partake of food \\
in this world and the next. 

So\marginnote{4.1} you should dispel stinginess, \\
overcoming that stain, and give a gift. \\
The good deeds of sentient beings \\
support them in the next world.” 

%
\end{verse}

“It’s\marginnote{5.1} incredible, sir, it’s amazing, how well said this was by Master Gotama.” He repeated the Buddha’s verses, and said: 

“Once\marginnote{8.1} upon a time, sir, I was a king named \textsanskrit{Serī}, a giver, a donor, who praised giving. I gave gifts at the four gates to ascetics and brahmins, to paupers, vagrants, travelers, and beggars. Then the ladies of my harem approached me and said, ‘Your Majesty gives gifts, but we don’t. Your Majesty, please support us to give gifts and make merit.’ Then it occurred to me, ‘I’m a giver, a donor, who praises giving. When they say, “We would give gifts”, what am I to say?’ And so I gave the first gate to the ladies of my harem. There they gave gifts, while my own giving dwindled. 

Then\marginnote{9.1} my aristocrat vassals approached me and said, ‘Your Majesty gives gifts, the ladies of your harem give gifts, but we don’t. Your Majesty, please support us to give gifts and make merit.’ Then it occurred to me, ‘I’m a giver, a donor, who praises giving. When they say, “We would give gifts”, what am I to say?’ And so I gave the second gate to my aristocrat vassals. There they gave gifts, while my own giving dwindled. 

Then\marginnote{10.1} my troops approached me and said, ‘Your Majesty gives gifts, the ladies of your harem give gifts, your aristocrat vassals give gifts, but we don’t. Your Majesty, please support us to give gifts and make merit.’ Then it occurred to me, ‘I’m a giver, a donor, who praises giving. When they say, “We would give gifts”, what am I to say?’ And so I gave the third gate to my troops. There they gave gifts, while my own giving dwindled. 

Then\marginnote{11.1} my brahmins and householders approached me and said, ‘Your Majesty gives gifts, the ladies of your harem give gifts, your aristocrat vassals give gifts, your troops give gifts, but we don’t. Your Majesty, please support us to give gifts and make merit.’ Then it occurred to me, ‘I’m a giver, a donor, who praises giving. When they say, “We would give gifts”, what am I to say?’ And so I gave the fourth gate to my brahmins and householders. There they gave gifts, while my own giving dwindled. 

Then\marginnote{12.1} my men approached me and said, ‘Now Your Majesty is not giving gifts at all!’ When they said this, I said to those men, ‘So then, my men, send half of the revenue from the outer districts to the royal compound. Then give half right there to ascetics and brahmins, to paupers, vagrants, travelers, and beggars.’ Sir, for a long time I made so much merit and did so many skillful deeds. I never reached any limit so as to say ‘there’s this much merit’ or ‘there’s this much result of merit’ or ‘for so long I’ll remain in heaven’. It’s incredible, sir, it’s amazing, how well said this was by Master Gotama: 

\begin{verse}%
‘Those\marginnote{13.1} who give with faith \\
and a clear and confident heart, \\
partake of food \\
in this world and the next. 

So\marginnote{14.1} you should dispel stinginess, \\
overcoming that stain, and give a gift. \\
The good deeds of sentient beings \\
support them in the next world.’” 

%
\end{verse}

%
\section*{{\suttatitleacronym SN 2.24}{\suttatitletranslation With Ghaṭīkāra }{\suttatitleroot Ghaṭīkārasutta}}
\addcontentsline{toc}{section}{\tocacronym{SN 2.24} \toctranslation{With Ghaṭīkāra } \tocroot{Ghaṭīkārasutta}}
\markboth{With Ghaṭīkāra }{Ghaṭīkārasutta}
\extramarks{SN 2.24}{SN 2.24}

Standing\marginnote{1.1} to one side, the god \textsanskrit{Ghaṭīkāra} recited this verse in the Buddha’s presence: 

\begin{verse}%
“Seven\marginnote{2.1} mendicants reborn in Aviha \\
have been freed. \\
With the complete ending of greed and hate, \\
they’ve crossed over clinging to the world.” 

“Who\marginnote{3.1} are those who’ve crossed the bog, \\
Death’s domain so hard to pass? \\
Who, after leaving behind the human body, \\
have risen above celestial yokes?” 

“Upaka\marginnote{4.1} and \textsanskrit{Palagaṇḍa}, \\
and \textsanskrit{Pukkusāti}, these three; \\
Bhaddiya and Bhaddadeva, \\
and \textsanskrit{Bāhudantī} and \textsanskrit{Piṅgiya}. \\
They, after leaving behind the human body, \\
have risen above celestial yokes.” 

“You\marginnote{5.1} speak well of them, \\
who have let go the snares of \textsanskrit{Māra}. \\
Whose teaching did they understand \\
to cut the bonds of rebirth?” 

“None\marginnote{6.1} other than the Blessed One! \\
None other than your instruction! \\
It was your teaching that they understood \\
to cut the bonds of rebirth. 

Where\marginnote{7.1} name and form \\
cease with nothing left over; \\
understanding this teaching, \\
they cut the bonds of rebirth.” 

“The\marginnote{8.1} words you say are deep, \\
hard to understand, so very hard to wake up to. \\
Whose teaching did you understand \\
to be able to say such things?” 

“In\marginnote{9.1} the past I was a potter \\
in \textsanskrit{Vebhaliṅga} called \textsanskrit{Ghaṭīkāra}. \\
I took care of my parents \\
as a lay follower of Buddha Kassapa. 

I\marginnote{10.1} refrained from sexual intercourse, \\
I was celibate, spiritual. \\
We lived in the same village; \\
in the past I was your friend. 

I\marginnote{11.1} am the one who understands \\
that these seven mendicants have been freed. \\
With the complete ending of greed and hate, \\
they’ve crossed over clinging to the world.” 

“That’s\marginnote{12.1} exactly how it was, \\
just as you say, Bhaggava. \\
In the past you were a potter \\
in \textsanskrit{Vebhaliṅga} called \textsanskrit{Ghaṭīkāra}. 

You\marginnote{13.1} took care of your parents \\
as a lay follower of Buddha Kassapa. \\
You refrained from sexual intercourse, \\
you were celibate, spiritual. \\
We lived in the same village; \\
in the past you were my friend.” 

“That’s\marginnote{14.1} how it was \\
when those friends of old met again. \\
Both of them are evolved, \\
and bear their final body.” 

%
\end{verse}

%
\section*{{\suttatitleacronym SN 2.25}{\suttatitletranslation With Jantu }{\suttatitleroot Jantusutta}}
\addcontentsline{toc}{section}{\tocacronym{SN 2.25} \toctranslation{With Jantu } \tocroot{Jantusutta}}
\markboth{With Jantu }{Jantusutta}
\extramarks{SN 2.25}{SN 2.25}

\scevam{So\marginnote{1.1} I have heard. }

At\marginnote{1.2} one time several mendicants were staying in the Kosalan lands, in a wilderness hut on the slopes of the Himalayas. They were restless, insolent, fickle, scurrilous, loose-tongued, unmindful, lacking situational awareness and immersion, with straying minds and undisciplined faculties. 

Then\marginnote{2.1} on the fifteenth day sabbath the god Jantu went up to those mendicants and addressed them in verse: 

\begin{verse}%
“The\marginnote{3.1} mendicants used to live happily, \\
as disciples of Gotama. \\
Desireless they sought alms; \\
desireless they used their lodgings. \\
Knowing that the world was impermanent \\
they made an end of suffering. 

But\marginnote{4.1} now they’ve made themselves hard to look after, \\
like chiefs in a village. \\
They eat and eat and then lie down, \\
unconscious in the homes of others. 

Having\marginnote{5.1} raised my joined palms to the \textsanskrit{Saṅgha}, \\
I speak here only about certain people. \\
They’re rejects, with no protector, \\
just like those who have passed away. 

I’m\marginnote{6.1} speaking about \\
those who live negligently. \\
To those who live diligently \\
I pay homage.” 

%
\end{verse}

%
\section*{{\suttatitleacronym SN 2.26}{\suttatitletranslation With Rohitassa }{\suttatitleroot Rohitassasutta}}
\addcontentsline{toc}{section}{\tocacronym{SN 2.26} \toctranslation{With Rohitassa } \tocroot{Rohitassasutta}}
\markboth{With Rohitassa }{Rohitassasutta}
\extramarks{SN 2.26}{SN 2.26}

At\marginnote{1.1} \textsanskrit{Sāvatthī}. 

Standing\marginnote{1.2} to one side, the god Rohitassa said to the Buddha: 

“Sir,\marginnote{1.3} is it possible to know or see or reach the end of the world by traveling to a place where there’s no being born, growing old, dying, passing away, or being reborn?” 

“Reverend,\marginnote{1.4} I say it’s not possible to know or see or reach the end of the world by traveling to a place where there’s no being born, growing old, dying, passing away, or being reborn.” 

“It’s\marginnote{2.1} incredible, sir, it’s amazing, how well said this was by Master Gotama. 

Once\marginnote{3.1} upon a time, I was a hermit called Rohitassa, son of Bhoja. I was a sky-walker with psychic power. I was as fast as a light arrow easily shot across the shadow of a palm tree by a well-trained expert archer with a strong bow. My stride was such that it could span from the eastern ocean to the western ocean. This wish came to me: ‘I will reach the end of the world by traveling.’ Having such speed and stride, I traveled for my whole lifespan of a hundred years—pausing only to eat and drink, go to the toilet, and sleep to dispel weariness—and I passed away along the way, never reaching the end of the world. 

It’s\marginnote{4.1} incredible, sir, it’s amazing, how well said this was by Master Gotama: ‘Reverend, I say it’s not possible to know or see or reach the end of the world by traveling to a place where there’s no being born, growing old, dying, passing away, or being reborn.’” 

“But\marginnote{5.1} Reverend, I also say there’s no making an end of suffering without reaching the end of the world. For it is in this fathom-long carcass with its perception and mind that I describe the world, its origin, its cessation, and the practice that leads to its cessation. 

\begin{verse}%
The\marginnote{6.1} end of the world can never \\
be reached by traveling. \\
But without reaching the end of the world, \\
there’s no release from suffering. 

So\marginnote{7.1} a clever person, understanding the world, \\
has completed the spiritual journey, and gone to the end of the world. \\
A peaceful one, knowing the end of the world, \\
does not long for this world or the next.” 

%
\end{verse}

%
\section*{{\suttatitleacronym SN 2.27}{\suttatitletranslation With Nanda }{\suttatitleroot Nandasutta}}
\addcontentsline{toc}{section}{\tocacronym{SN 2.27} \toctranslation{With Nanda } \tocroot{Nandasutta}}
\markboth{With Nanda }{Nandasutta}
\extramarks{SN 2.27}{SN 2.27}

Standing\marginnote{1.1} to one side, the god Nanda recited this verse in the Buddha’s presence: 

\begin{verse}%
“Time\marginnote{2.1} flies, nights pass by, \\
the stages of life leave us one by one. \\
Seeing this peril in death, \\
you should do good deeds that bring happiness.” 

“Time\marginnote{3.1} flies, nights pass by, \\
the stages of life leave us one by one. \\
Seeing this peril in death, \\
one looking for peace would drop the world’s bait.” 

%
\end{verse}

%
\section*{{\suttatitleacronym SN 2.28}{\suttatitletranslation With Nandivisāla }{\suttatitleroot Nandivisālasutta}}
\addcontentsline{toc}{section}{\tocacronym{SN 2.28} \toctranslation{With Nandivisāla } \tocroot{Nandivisālasutta}}
\markboth{With Nandivisāla }{Nandivisālasutta}
\extramarks{SN 2.28}{SN 2.28}

Standing\marginnote{1.1} to one side, the god \textsanskrit{Nandivisāla} addressed the Buddha in verse: 

\begin{verse}%
“Four\marginnote{2.1} are its wheels, and nine its doors; \\
it’s stuffed full, bound with greed, \\
and born from a bog. \\
Great hero, how am I supposed to live like this?” 

“Having\marginnote{3.1} cut the strap and harness—\\
wicked desire and greed—\\
and having plucked out craving, root and all: \\
that’s how you’re supposed to live like this.” 

%
\end{verse}

%
\section*{{\suttatitleacronym SN 2.29}{\suttatitletranslation With Susīma }{\suttatitleroot Susimasutta}}
\addcontentsline{toc}{section}{\tocacronym{SN 2.29} \toctranslation{With Susīma } \tocroot{Susimasutta}}
\markboth{With Susīma }{Susimasutta}
\extramarks{SN 2.29}{SN 2.29}

At\marginnote{1.1} \textsanskrit{Sāvatthī}. 

Then\marginnote{1.2} Venerable Ānanda went up to the Buddha, bowed, and sat down to one side. The Buddha said to him, “Ānanda, do you like \textsanskrit{Sāriputta}?” 

“Sir,\marginnote{2.1} who on earth would not like Venerable \textsanskrit{Sāriputta} unless they’re a fool, a hater, delusional, or mentally deranged? Venerable \textsanskrit{Sāriputta} is astute, he has great wisdom, widespread wisdom, laughing wisdom, swift wisdom, sharp wisdom, and penetrating wisdom. He has few wishes, he’s content, secluded, aloof, and energetic. He gives advice and accepts advice; he accuses and criticizes wickedness. Who on earth would not like Venerable \textsanskrit{Sāriputta} unless they’re a fool, a hater, delusional, or mentally deranged?” 

“That’s\marginnote{3.1} so true, Ānanda! That’s so true! Who on earth would not like Venerable \textsanskrit{Sāriputta} unless they’re a fool, a hater, delusional, or mentally deranged?” And the Buddha repeated all of Ānanda’s terms of praise. 

While\marginnote{4.1} this praise of \textsanskrit{Sāriputta} was being spoken, the god \textsanskrit{Susīma} approached the Buddha, escorted by a large assembly of gods. He bowed, stood to one side, and said to him: 

“That’s\marginnote{5.1} so true, Blessed One! That’s so true, Holy One! Who on earth would not like Venerable \textsanskrit{Sāriputta} unless they’re a fool, a hater, delusional, or mentally deranged?” And he too repeated all the terms of praise of \textsanskrit{Sāriputta}, adding, “For I too, sir, whenever I go to an assembly of gods, frequently hear the same terms of praise.” 

While\marginnote{7.1} this praise of \textsanskrit{Sāriputta} was being spoken, the gods of \textsanskrit{Susīma}’s assembly—uplifted and overjoyed, full of rapture and happiness—generated a rainbow of bright colors. 

Suppose\marginnote{8.1} there was a beryl gem that was naturally beautiful, eight-faceted, well-worked. When placed on a cream rug it would shine and glow and radiate. In the same way, the gods of \textsanskrit{Susīma}’s assembly generated a rainbow of bright colors. 

Suppose\marginnote{9.1} there was a pendant of river gold, fashioned by an expert smith, well wrought in the forge. When placed on a cream rug it would shine and glow and radiate. In the same way, the gods of \textsanskrit{Susīma}’s assembly generated a rainbow of bright colors. 

Suppose\marginnote{10.1} that after the rainy season the sky was clear and cloudless. At the crack of dawn, the Morning Star shines and glows and radiates. In the same way, the gods of \textsanskrit{Susīma}’s assembly generated a rainbow of bright colors. 

Suppose\marginnote{11.1} that after the rainy season the sky was clear and cloudless. As the sun rises, it would dispel all the darkness from the sky as it shines and glows and radiates. In the same way, the gods of \textsanskrit{Susīma}’s assembly generated a rainbow of bright colors. 

Then\marginnote{12.1} the god \textsanskrit{Susīma} recited this verse about Venerable \textsanskrit{Sāriputta} in the Buddha’s presence: 

\begin{verse}%
“He’s\marginnote{13.1} considered astute, \\
\textsanskrit{Sāriputta}, free of anger. \\
Few in wishes, sweet, tamed, \\
the hermit shines in the Teacher’s praise!” 

%
\end{verse}

Then\marginnote{14.1} the Buddha replied to \textsanskrit{Susīma} with this verse about Venerable \textsanskrit{Sāriputta}: 

\begin{verse}%
“He’s\marginnote{15.1} considered astute, \\
\textsanskrit{Sāriputta}, free of anger. \\
Few in wishes, sweet, tamed; \\
developed and well-tamed, he bides his time.” 

%
\end{verse}

%
\section*{{\suttatitleacronym SN 2.30}{\suttatitletranslation The Disciples of Various Sectarians }{\suttatitleroot Nānātitthiyasāvakasutta}}
\addcontentsline{toc}{section}{\tocacronym{SN 2.30} \toctranslation{The Disciples of Various Sectarians } \tocroot{Nānātitthiyasāvakasutta}}
\markboth{The Disciples of Various Sectarians }{Nānātitthiyasāvakasutta}
\extramarks{SN 2.30}{SN 2.30}

\scevam{So\marginnote{1.1} I have heard. }At one time the Buddha was staying near \textsanskrit{Rājagaha}, in the Bamboo Grove, the squirrels’ feeding ground. 

Then,\marginnote{1.3} late at night, several glorious gods lit up the entire Bamboo Grove. They were Asama, \textsanskrit{Sahalī}, \textsanskrit{Niṅka}, \textsanskrit{Ākoṭaka}, \textsanskrit{Vetambarī}, and \textsanskrit{Māṇavagāmiya}, and all of them were disciples of various sectarian teachers. They went up to the Buddha, bowed, and stood to one side. 

Standing\marginnote{1.4} to one side, the god Asama recited this verse about \textsanskrit{Pūraṇa} Kassapa in the Buddha’s presence: 

\begin{verse}%
“In\marginnote{2.1} injuring and killing here, \\
in beating and extortion, \\
Kassapa saw no evil, \\
nor any merit for oneself. \\
What he taught should truly be trusted, \\
he’s worthy of esteem as Teacher.” 

%
\end{verse}

Then\marginnote{3.1} the god \textsanskrit{Sahalī} recited this verse about Makkhali Gosala in the Buddha’s presence: 

\begin{verse}%
“Through\marginnote{4.1} mortification in disgust of sin he became well restrained. \\
He gave up arguing with people. \\
Refraining from false speech, he spoke the truth. \\
Surely such a man does no wrong!” 

%
\end{verse}

Then\marginnote{5.1} the god \textsanskrit{Niṅka} recited this verse about \textsanskrit{Nigaṇṭha} \textsanskrit{Nātaputta} in the Buddha’s presence: 

\begin{verse}%
“Disgusted\marginnote{6.1} at sin, an alert mendicant, \\
well restrained in the four controls; \\
explaining what is seen and heard: \\
surely he can be no sinner!” 

%
\end{verse}

Then\marginnote{7.1} the god \textsanskrit{Ākoṭaka} recited this verse about various sectarian teachers in the Buddha’s presence: 

\begin{verse}%
“Pakudhaka,\marginnote{8.1} \textsanskrit{Kātiyāna}, and \textsanskrit{Nigaṇṭha}, \\
as well as this Makkhali and \textsanskrit{Pūraṇa}: \\
Teachers of communities, attained ascetics, \\
surely they weren’t far from truly good men!” 

%
\end{verse}

Then\marginnote{9.1} the god \textsanskrit{Vetambarī} replied to the god \textsanskrit{Ākoṭaka} in verse: 

\begin{verse}%
“Though\marginnote{10.1} the wretched jackal howls along, \\
it never equals the lion. \\
A naked liar with suspicious conduct, \\
though they teach a community, is not like the good.” 

%
\end{verse}

Then\marginnote{11.1} \textsanskrit{Māra} the Wicked took possession of the god \textsanskrit{Vetambarī} and recited this verse in the Buddha’s presence: 

\begin{verse}%
“Those\marginnote{12.1} dedicated to mortification in disgust of sin, \\
safeguarding their seclusion, \\
attached to form, \\
they rejoice in the heavenly realm. \\
Indeed, those mortals give correct instructions \\
regarding the next world.” 

%
\end{verse}

Then\marginnote{13.1} the Buddha, knowing that this was \textsanskrit{Māra} the Wicked, replied to him in verse: 

\begin{verse}%
“Whatever\marginnote{14.1} forms there are in this world or the world beyond, \\
and those of shining beauty in the sky, \\
all of these you praise, Namuci, \\
like bait tossed out for catching fish.” 

%
\end{verse}

Then\marginnote{15.1} the god \textsanskrit{Māṇavagāmiya} recited this verse about the Buddha in his presence: 

\begin{verse}%
“Of\marginnote{16.1} all the mountains of \textsanskrit{Rājagaha}, \\
Vipula’s said to be the best. \\
Seta is the best of the Himalayan peaks, \\
and the sun, of travelers in space. 

The\marginnote{17.1} ocean is the best of seas, \\
and the moon, of lights that shine at night. \\
But in all the world with its gods, \\
the Buddha is declared foremost.” 

%
\end{verse}

\scendsutta{The Linked Discourses on Gods are complete. }

%
\addtocontents{toc}{\let\protect\contentsline\protect\nopagecontentsline}
\part*{Linked Discourses With King Pasenadi of Kosala }
\addcontentsline{toc}{part}{Linked Discourses With King Pasenadi of Kosala }
\markboth{}{}
\addtocontents{toc}{\let\protect\contentsline\protect\oldcontentsline}

%
\addtocontents{toc}{\let\protect\contentsline\protect\nopagecontentsline}
\chapter*{Chapter One }
\addcontentsline{toc}{chapter}{\tocchapterline{Chapter One }}
\addtocontents{toc}{\let\protect\contentsline\protect\oldcontentsline}

%
\section*{{\suttatitleacronym SN 3.1}{\suttatitletranslation Young }{\suttatitleroot Daharasutta}}
\addcontentsline{toc}{section}{\tocacronym{SN 3.1} \toctranslation{Young } \tocroot{Daharasutta}}
\markboth{Young }{Daharasutta}
\extramarks{SN 3.1}{SN 3.1}

\scevam{So\marginnote{1.1} I have heard. }At one time the Buddha was staying near \textsanskrit{Sāvatthī} in Jeta’s Grove, \textsanskrit{Anāthapiṇḍika}’s monastery. 

Then\marginnote{1.3} King Pasenadi of Kosala went up to the Buddha, and exchanged greetings with him. When the greetings and polite conversation were over, he sat down to one side and said to the Buddha, “Does Master Gotama claim to have awakened to the supreme perfect awakening?” 

“If\marginnote{1.6} anyone should rightly be said to have awakened to the supreme perfect awakening, it’s me. For, great king, I have awakened to the supreme perfect awakening.” 

“Well,\marginnote{2.1} there are those ascetics and brahmins who lead an order and a community, and teach a community. They’re well-known and famous religious founders, regarded as holy by many people. That is, \textsanskrit{Pūraṇa} Kassapa, Makkhali \textsanskrit{Gosāla}, \textsanskrit{Nigaṇṭha} \textsanskrit{Nāṭaputta}, \textsanskrit{Sañjaya} \textsanskrit{Belaṭṭhiputta}, Pakudha \textsanskrit{Kaccāyana}, and Ajita Kesakambala. I also asked them whether they claimed to have awakened to the supreme perfect awakening, but they made no such claim. So why do you, given that you’re so young in age and newly gone forth?” 

“Great\marginnote{3.1} king, these four things should not be looked down upon or disparaged because they are young. What four? An aristocrat, a serpent, a fire, and a mendicant. These four things should not be looked down upon or disparaged because they are young.” 

That\marginnote{4.1} is what the Buddha said. Then the Holy One, the Teacher, went on to say: 

\begin{verse}%
“A\marginnote{5.1} man should not despise \\
an aristocrat of impeccable lineage, \\
high-born and famous, \\
just because they’re young. 

For\marginnote{6.1} it’s possible that that lord of men, \\
as aristocrat, will gain the throne. \\
And in his anger he’ll execute a royal punishment, \\
and have you violently beaten. \\
Hence you should avoid him \\
for the sake of your own life. 

Whether\marginnote{7.1} in village or wilderness, \\
wherever a serpent is seen, \\
a man should not look down on it \\
or despise it for its youth. 

With\marginnote{8.1} its rainbow of colors, \\
the serpent of fiery breath glides along. \\
It lashes out and bites the fool, \\
both men and women alike. \\
Hence you should avoid it \\
for the sake of your own life. 

A\marginnote{9.1} fire devours a huge amount, \\
a conflagration with a blackened trail. \\
A man should not look down on it \\
just because it’s young. 

For\marginnote{10.1} once it gets fuel \\
it’ll become a huge conflagration. \\
It’ll lash out and burn the fool, \\
both men and women alike. \\
Hence you should avoid it \\
for the sake of your own life. 

When\marginnote{11.1} a forest is burned by fire, \\
a conflagration with a blackened trail, \\
the shoots will spring up there again, \\
with the passing of the days and nights. 

But\marginnote{12.1} if a mendicant endowed with ethics \\
burns you with their power, \\
you’ll have no sons or cattle, \\
nor will your heirs find wealth. \\
Childless and heirless you become, \\
like a palm-tree stump. 

That’s\marginnote{13.1} why an astute person, \\
seeing what’s good for themselves, \\
would always treat these properly: \\
a snake, a conflagration, \\
a famous aristocrat, \\
and a mendicant endowed with ethics.” 

%
\end{verse}

When\marginnote{14.1} this was said, King Pasenadi of Kosala said to the Buddha, “Excellent, sir! Excellent! As if he were righting the overturned, or revealing the hidden, or pointing out the path to the lost, or lighting a lamp in the dark so people with good eyes can see what’s there, the Buddha has made the teaching clear in many ways. I go for refuge to the Buddha, to the teaching, and to the mendicant \textsanskrit{Saṅgha}. From this day forth, may the Buddha remember me as a lay follower who has gone for refuge for life.” 

%
\section*{{\suttatitleacronym SN 3.2}{\suttatitletranslation A Person }{\suttatitleroot Purisasutta}}
\addcontentsline{toc}{section}{\tocacronym{SN 3.2} \toctranslation{A Person } \tocroot{Purisasutta}}
\markboth{A Person }{Purisasutta}
\extramarks{SN 3.2}{SN 3.2}

At\marginnote{1.1} \textsanskrit{Sāvatthī}. 

Then\marginnote{1.2} King Pasenadi of Kosala went up to the Buddha, bowed, sat down to one side, and said to the Buddha, “Sir, how many things arise inside a person for their harm, suffering, and discomfort?” 

“Great\marginnote{2.1} king, three things arise inside a person for their harm, suffering, and discomfort. What three? Greed, hate, and delusion. These three things arise inside a person for their harm, suffering, and discomfort.” 

That\marginnote{2.7} is what the Buddha said. … 

\begin{verse}%
“When\marginnote{3.1} greed, hate, and delusion, \\
have arisen inside oneself, \\
they harm a person of wicked heart, \\
as a reed is destroyed by its own fruit.” 

%
\end{verse}

%
\section*{{\suttatitleacronym SN 3.3}{\suttatitletranslation Old Age and Death }{\suttatitleroot Jarāmaraṇasutta}}
\addcontentsline{toc}{section}{\tocacronym{SN 3.3} \toctranslation{Old Age and Death } \tocroot{Jarāmaraṇasutta}}
\markboth{Old Age and Death }{Jarāmaraṇasutta}
\extramarks{SN 3.3}{SN 3.3}

At\marginnote{1.1} \textsanskrit{Sāvatthī}. 

Seated\marginnote{1.2} to one side, King Pasenadi said to the Buddha, “Sir, for someone who has been reborn, is there any exemption from old age and death?” 

“Great\marginnote{1.4} king, for someone who has been reborn, there’s no exemption from old age and death. Even for well-to-do aristocrats, brahmins, or householders—rich, affluent, and wealthy, with lots of gold and silver, lots of property and assets, and lots of money and grain—when they’re born, there’s no exemption from old age and death. Even for mendicants who are perfected—who have ended the defilements, completed the spiritual journey, done what had to be done, laid down the burden, achieved their own goal, utterly ended the fetters of rebirth, and are rightly freed through enlightenment—their bodies are liable to break up and be laid to rest.” 

That\marginnote{1.9} is what the Buddha said. … 

\begin{verse}%
“The\marginnote{2.1} fancy chariots of kings wear out, \\
and this body too gets old. \\
But goodness never gets old: \\
so the true and the good proclaim.” 

%
\end{verse}

%
\section*{{\suttatitleacronym SN 3.4}{\suttatitletranslation Loved }{\suttatitleroot Piyasutta}}
\addcontentsline{toc}{section}{\tocacronym{SN 3.4} \toctranslation{Loved } \tocroot{Piyasutta}}
\markboth{Loved }{Piyasutta}
\extramarks{SN 3.4}{SN 3.4}

At\marginnote{1.1} \textsanskrit{Sāvatthī}. 

Seated\marginnote{1.2} to one side, King Pasenadi said to the Buddha, “Just now, sir, as I was in private retreat this thought came to mind. ‘Who are those who love themselves? And who are those who don’t love themselves?’ 

Then\marginnote{1.5} it occurred to me: ‘Those who do bad things by way of body, speech, and mind don’t love themselves. Even though they may say: “I love myself”, they don’t really. Why is that? It’s because they treat themselves like an enemy. That’s why they don’t love themselves. 

Those\marginnote{1.13} who do good things by way of body, speech, and mind do love themselves. Even though they may say: “I don’t love myself”, they do really. Why is that? It’s because they treat themselves like a loved one. That’s why they do love themselves.’” 

“That’s\marginnote{2.1} so true, great king! That’s so true!” said the Buddha. And he repeated the king’s statement, adding: 

\begin{verse}%
“If\marginnote{3.1} you’d only love yourself, \\
you’d not yoke yourself to wickedness. \\
For happiness is not easy to find \\
by someone who does bad deeds. 

When\marginnote{4.1} you’re seized by the terminator \\
as you give up your human life, \\
what can you call your own? \\
What do you take when you go? \\
What goes with you, \\
like a shadow that never leaves? 

Both\marginnote{5.1} the good and the bad \\
that a mortal does in this life \\
is what they can call their own. \\
That’s what they take when they go. \\
That’s what goes with them, \\
like a shadow that never leaves. 

That’s\marginnote{6.1} why you should do good, \\
investing in the future life. \\
The good deeds of sentient beings \\
support them in the next world.” 

%
\end{verse}

%
\section*{{\suttatitleacronym SN 3.5}{\suttatitletranslation Self-Protected }{\suttatitleroot Attarakkhitasutta}}
\addcontentsline{toc}{section}{\tocacronym{SN 3.5} \toctranslation{Self-Protected } \tocroot{Attarakkhitasutta}}
\markboth{Self-Protected }{Attarakkhitasutta}
\extramarks{SN 3.5}{SN 3.5}

At\marginnote{1.1} \textsanskrit{Sāvatthī}. 

Seated\marginnote{1.2} to one side, King Pasenadi said to the Buddha, “Just now, sir, as I was in private retreat this thought came to mind. ‘Who are those who protect themselves? And who are those who don’t protect themselves?’ 

Then\marginnote{1.5} it occurred to me: ‘Those who do bad things by way of body, speech, and mind don’t protect themselves. Even if they’re protected by a company of elephants, cavalry, chariots, or infantry, they still don’t protect themselves. Why is that? Because such protection is exterior, not interior. That’s why they don’t protect themselves. 

Those\marginnote{1.13} who do good things by way of body, speech, and mind do protect themselves. Even if they’re not protected by a company of elephants, cavalry, chariots, or infantry, they still protect themselves. Why is that? Because such protection is interior, not exterior. That’s why they do protect themselves.’” 

“That’s\marginnote{2.1} so true, great king! That’s so true!” said the Buddha. And he repeated the king’s statement, adding: 

\begin{verse}%
“Restraint\marginnote{3.1} of the body is good; \\
restraint of speech is good; \\
restraint of mind is good; \\
everywhere, restraint is good. \\
A sincere person, restrained everywhere, \\
is said to be ‘protected’.” 

%
\end{verse}

%
\section*{{\suttatitleacronym SN 3.6}{\suttatitletranslation Few }{\suttatitleroot Appakasutta}}
\addcontentsline{toc}{section}{\tocacronym{SN 3.6} \toctranslation{Few } \tocroot{Appakasutta}}
\markboth{Few }{Appakasutta}
\extramarks{SN 3.6}{SN 3.6}

At\marginnote{1.1} \textsanskrit{Sāvatthī}. 

Seated\marginnote{1.2} to one side, King Pasenadi said to the Buddha, “Just now, sir, as I was in private retreat this thought came to mind: ‘Few are the sentient beings in the world who, when they obtain luxury possessions, don’t get indulgent and negligent, giving in to greed for sensual pleasures, and doing the wrong thing by others. There are many more who, when they obtain luxury possessions, do get indulgent and negligent, giving in to greed for sensual pleasures, and doing the wrong thing by others.’” 

“That’s\marginnote{2.1} so true, great king! That’s so true!” said the Buddha. And he repeated the king’s statement, adding: 

\begin{verse}%
“Full\marginnote{3.1} of desire for possessions and pleasures, \\
greedy, infatuated by sensual pleasures; \\
they don’t notice that they’ve gone too far, \\
like deer falling into a trap set out. \\
It’ll be bitter later on; \\
for the result will be bad for them.” 

%
\end{verse}

%
\section*{{\suttatitleacronym SN 3.7}{\suttatitletranslation Judgment }{\suttatitleroot Aḍḍakaraṇasutta}}
\addcontentsline{toc}{section}{\tocacronym{SN 3.7} \toctranslation{Judgment } \tocroot{Aḍḍakaraṇasutta}}
\markboth{Judgment }{Aḍḍakaraṇasutta}
\extramarks{SN 3.7}{SN 3.7}

At\marginnote{1.1} \textsanskrit{Sāvatthī}. 

Seated\marginnote{1.2} to one side, King Pasenadi said to the Buddha, “Sir, when I’m sitting in judgment I see well-to-do aristocrats, brahmins, and householders—rich, affluent, and wealthy, with lots of gold and silver, lots of property and assets, and lots of money and grain. But they tell deliberate lies for the sake of sensual pleasures. Then it occurred to me: ‘Enough with passing judgment today. Now my dear son will be known by the judgments he makes.’” 

“That’s\marginnote{2.1} so true, great king! That’s so true! Those who are well-to-do aristocrats, brahmins, and householders tell deliberate lies for the sake of sensual pleasures. That is for their lasting harm and suffering.” 

That\marginnote{2.4} is what the Buddha said. … 

\begin{verse}%
“Full\marginnote{3.1} of desire for possessions and pleasures, \\
greedy, infatuated by sensual pleasures; \\
they don’t notice that they’ve gone too far, \\
like fish entering a net set out. \\
It’ll be bitter later on; \\
for the result will be bad for them.” 

%
\end{verse}

%
\section*{{\suttatitleacronym SN 3.8}{\suttatitletranslation With Queen Mallikā }{\suttatitleroot Mallikāsutta}}
\addcontentsline{toc}{section}{\tocacronym{SN 3.8} \toctranslation{With Queen Mallikā } \tocroot{Mallikāsutta}}
\markboth{With Queen Mallikā }{Mallikāsutta}
\extramarks{SN 3.8}{SN 3.8}

At\marginnote{1.1} \textsanskrit{Sāvatthī}. 

Now\marginnote{1.2} at that time King Pasenadi of Kosala was upstairs in the royal longhouse together with Queen \textsanskrit{Mallikā}. 

Then\marginnote{1.3} the king said to the queen, “\textsanskrit{Mallikā}, is there anyone more dear to you than yourself?” 

“No,\marginnote{1.5} great king, there isn’t. But is there anyone more dear to you than yourself?” 

“For\marginnote{1.7} me also, \textsanskrit{Mallikā}, there’s no-one.” 

Then\marginnote{2.1} King Pasenadi of Kosala came downstairs from the stilt longhouse, went to the Buddha, bowed, sat down to one side, and told him what had happened. 

Then,\marginnote{3.1} understanding this matter, on that occasion the Buddha recited this verse: 

\begin{verse}%
“Having\marginnote{4.1} explored every quarter with the mind, \\
one finds no-one dearer than oneself. \\
Likewise for others, each holds themselves dear; \\
so one who loves themselves would harm no other.” 

%
\end{verse}

%
\section*{{\suttatitleacronym SN 3.9}{\suttatitletranslation Sacrifice }{\suttatitleroot Yaññasutta}}
\addcontentsline{toc}{section}{\tocacronym{SN 3.9} \toctranslation{Sacrifice } \tocroot{Yaññasutta}}
\markboth{Sacrifice }{Yaññasutta}
\extramarks{SN 3.9}{SN 3.9}

At\marginnote{1.1} \textsanskrit{Sāvatthī}. 

Now\marginnote{1.2} at that time a big sacrifice had been set up for King Pasenadi of Kosala. Five hundred chief bulls, five hundred bullocks, five hundred heifers, five hundred goats, and five hundred rams had been led to the pillar for the sacrifice. His bondservants, employees, and workers did their jobs under threat of punishment and danger, weeping with tearful faces. 

Then\marginnote{2.1} several mendicants robed up in the morning and, taking their bowls and robes, entered \textsanskrit{Sāvatthī} for alms. Then, after the meal, when they returned from almsround, they went up to the Buddha, bowed, sat down to one side, and told him what was happening. 

Then,\marginnote{3.1} understanding this matter, on that occasion the Buddha recited these verses: 

\begin{verse}%
“Horse\marginnote{4.1} sacrifice, human sacrifice, \\
the sacrifices of the ‘stick-casting’, \\
the ‘royal soma drinking’, and the ‘unbarred’—\\
these huge violent sacrifices yield no great fruit. 

The\marginnote{5.1} great sages of good conduct \\
don’t attend sacrifices \\
where goats, sheep, and cattle \\
and various creatures are killed. 

But\marginnote{6.1} the great sages of good conduct \\
do attend non-violent sacrifices \\
of regular family tradition, \\
where goats, sheep, and cattle, \\
and various creatures aren’t killed. 

A\marginnote{7.1} clever person should sacrifice like this, \\
for this sacrifice is very fruitful. \\
For a sponsor of sacrifices like this, \\
things get better, not worse. \\
Such a sacrifice is truly abundant, \\
and even the deities are pleased.” 

%
\end{verse}

%
\section*{{\suttatitleacronym SN 3.10}{\suttatitletranslation Shackles }{\suttatitleroot Bandhanasutta}}
\addcontentsline{toc}{section}{\tocacronym{SN 3.10} \toctranslation{Shackles } \tocroot{Bandhanasutta}}
\markboth{Shackles }{Bandhanasutta}
\extramarks{SN 3.10}{SN 3.10}

Now\marginnote{1.1} at that time a large group of people had been put in shackles by King Pasenadi of Kosala—some in ropes, some in manacles, some in chains. 

Then\marginnote{2.1} several mendicants robed up in the morning and, taking their bowls and robes, entered \textsanskrit{Sāvatthī} for alms. Then, after the meal, when they returned from almsround, they went up to the Buddha, bowed, sat down to one side, and told him what was happening. 

Then,\marginnote{3.1} understanding this matter, on that occasion the Buddha recited these verses: 

\begin{verse}%
“The\marginnote{4.1} wise say that shackle is not strong \\
that’s made of iron, wood, or knots. \\
But obsession with jeweled earrings, \\
concern for your partners and children: 

this,\marginnote{5.1} say the wise, is a strong shackle \\
dragging the indulgent down, hard to escape. \\
Having cut this one too they go forth, \\
unconcerned, having given up sensual pleasures.” 

%
\end{verse}

%
\addtocontents{toc}{\let\protect\contentsline\protect\nopagecontentsline}
\chapter*{Chapter Two }
\addcontentsline{toc}{chapter}{\tocchapterline{Chapter Two }}
\addtocontents{toc}{\let\protect\contentsline\protect\oldcontentsline}

%
\section*{{\suttatitleacronym SN 3.11}{\suttatitletranslation Seven Matted-Hair Ascetics }{\suttatitleroot Sattajaṭilasutta}}
\addcontentsline{toc}{section}{\tocacronym{SN 3.11} \toctranslation{Seven Matted-Hair Ascetics } \tocroot{Sattajaṭilasutta}}
\markboth{Seven Matted-Hair Ascetics }{Sattajaṭilasutta}
\extramarks{SN 3.11}{SN 3.11}

At\marginnote{1.1} one time the Buddha was staying near \textsanskrit{Sāvatthī} in the Eastern Monastery, the stilt longhouse of \textsanskrit{Migāra}’s mother. 

Then\marginnote{1.2} in the late afternoon, the Buddha came out of retreat and sat outside the gate. Then King Pasenadi of Kosala went up to the Buddha, bowed, and sat down to one side. 

Now\marginnote{2.1} at that time seven matted-hair ascetics, seven Jain ascetics, seven naked ascetics, seven one-cloth ascetics, and seven wanderers passed by not far from the Buddha. Their armpits and bodies were hairy, and their nails were long; and they carried their stuff with shoulder-poles. 

Then\marginnote{2.2} King Pasenadi got up from his seat, arranged his robe over one shoulder, knelt with his right knee on the ground, raised his joined palms toward those various ascetics, and pronounced his name three times: “Sirs, I am Pasenadi, king of Kosala! … I am Pasenadi, king of Kosala!” 

Then,\marginnote{3.1} soon after those ascetics had left, King Pasenadi went up to the Buddha, bowed, sat down to one side, and said to him, “Sir, are they among those in the world who are perfected ones or who are on the path to perfection?” 

“Great\marginnote{4.1} king, as a layman enjoying sensual pleasures, living at home with your children, using sandalwood imported from \textsanskrit{Kāsi}, wearing garlands, perfumes, and makeup, and accepting gold and money, it’s hard for you to know who is perfected or on the path to perfection. 

You\marginnote{5.1} can get to know a person’s ethics by living with them. But only after a long time, not casually; only when paying attention, not when inattentive; and only by the wise, not the witless. You can get to know a person’s purity by dealing with them. … You can get to know a person’s resilience in times of trouble. … You can get to know a person’s wisdom by discussion. But only after a long time, not casually; only when paying attention, not when inattentive; and only by the wise, not the witless.” 

“It’s\marginnote{6.1} incredible, sir, it’s amazing, how well said this was by the Buddha. … 

Sir,\marginnote{7.1} these are my spies, my undercover agents returning after spying on the country. First they go undercover, then I have them report to me. And now—when they have washed off the dust and dirt, and are nicely bathed and anointed, with hair and beard dressed, and dressed in white—they will amuse themselves, supplied and provided with the five kinds of sensual stimulation.” 

Then,\marginnote{8.1} understanding this matter, on that occasion the Buddha recited these verses: 

\begin{verse}%
“It’s\marginnote{9.1} not easy to know a man by his appearance. \\
You shouldn’t trust them at first sight. \\
For undisciplined men live in this world \\
disguised as the disciplined. 

Like\marginnote{10.1} a fake earring made of clay, \\
like a copper penny coated with gold, \\
they live hidden in the world, \\
corrupt inside but impressive outside.” 

%
\end{verse}

%
\section*{{\suttatitleacronym SN 3.12}{\suttatitletranslation Five Kings }{\suttatitleroot Pañcarājasutta}}
\addcontentsline{toc}{section}{\tocacronym{SN 3.12} \toctranslation{Five Kings } \tocroot{Pañcarājasutta}}
\markboth{Five Kings }{Pañcarājasutta}
\extramarks{SN 3.12}{SN 3.12}

At\marginnote{1.1} \textsanskrit{Sāvatthī}. 

Now\marginnote{1.2} at that time five kings headed by Pasenadi were amusing themselves, supplied and provided with the five kinds of sensual stimulation, and this discussion came up among them: “What’s the best of sensual pleasures?” 

Some\marginnote{1.4} of them said, “Sights are the best of sensual pleasures!” 

Others\marginnote{1.6} said, “Sounds are best!” 

Others\marginnote{1.8} said, “Smells are best!” 

Others\marginnote{1.10} said, “Tastes are best!” 

Others\marginnote{1.12} said, “Touches are best!” 

Since\marginnote{1.14} those kings were unable to persuade each other, King Pasenadi said to them, “Come, good sirs, let’s go to the Buddha and ask him about this. As he answers, so we’ll remember it.” 

“Yes,\marginnote{2.4} dear sir,” replied those kings. 

Then\marginnote{3.1} those five kings headed by Pasenadi went to the Buddha, bowed, and sat down to one side. King Pasenadi reported their conversation to the Buddha, and said, “Sir, what’s the best of sensual pleasures?” 

“Great\marginnote{4.1} king, which kind of sensual stimulation is best is defined by which is most agreeable, I say. The very same sights that are agreeable to some are disagreeable to others. When you’re happy with certain sights, as you’ve got all you wished for, you don’t want any other sight that’s better or finer. For you, those sights are perfect and supreme. 

The\marginnote{5.1} very same sounds … smells … tastes … touches that are agreeable to some are disagreeable to others. When you’re happy with certain touches, as you’ve got all you wished for, you don’t want any other touch that’s better or finer. For you, those touches are perfect and supreme.” 

Now\marginnote{9.1} at that time the lay follower \textsanskrit{Candanaṅgalika} was sitting in that assembly. Then he got up from his seat, arranged his robe over one shoulder, raised his joined palms toward the Buddha, and said, “I feel inspired to speak, Blessed One! I feel inspired to speak, Holy One!” 

“Then\marginnote{9.4} speak as you feel inspired,” said the Buddha. 

Then\marginnote{10.1} the lay follower \textsanskrit{Candanaṅgalika} extolled the Buddha in his presence with an appropriate verse: 

\begin{verse}%
“Like\marginnote{11.1} a fragrant pink lotus \\
that blooms in the morning, its fragrance unfaded—\\
see \textsanskrit{Aṅgīrasa} shine, \\
bright as the sun in the sky!” 

%
\end{verse}

Then\marginnote{12.1} those five kings clothed \textsanskrit{Candanaṅgalika} with five upper robes. And \textsanskrit{Candanaṅgalika} in turn endowed the Buddha with those robes. 

%
\section*{{\suttatitleacronym SN 3.13}{\suttatitletranslation A Bucket of Rice }{\suttatitleroot Doṇapākasutta}}
\addcontentsline{toc}{section}{\tocacronym{SN 3.13} \toctranslation{A Bucket of Rice } \tocroot{Doṇapākasutta}}
\markboth{A Bucket of Rice }{Doṇapākasutta}
\extramarks{SN 3.13}{SN 3.13}

At\marginnote{1.1} \textsanskrit{Sāvatthī}. 

Now\marginnote{1.2} at that time King Pasenadi of Kosala used to eat rice by the bucket. Then after eating King Pasenadi of Kosala went up to the Buddha, huffing and puffing. He bowed and sat down to one side. 

Then,\marginnote{2.1} knowing that King Pasenadi was huffing and puffing after eating, on that occasion the Buddha recited this verse: 

\begin{verse}%
“When\marginnote{3.1} a man is always mindful, \\
knowing moderation in eating, \\
his discomfort fades, \\
and he ages slowly, taking care of his life.” 

%
\end{verse}

Now\marginnote{4.1} at that time the brahmin student Sudassana was standing behind the king. Then King Pasenadi addressed him, “Please, dear Sudassana, memorize this verse in the Buddha’s presence and recite it to me whenever I am presented with a meal. I’ll set up a regular daily allowance of a hundred dollars for you.” 

“Yes,\marginnote{4.5} Your Majesty,” replied Sudassana. He memorized that verse in the Buddha’s presence, and then whenever the king was presented with a meal he would repeat it: 

\begin{verse}%
“When\marginnote{5.1} a man is always mindful, \\
knowing moderation in eating, \\
his discomfort fades, \\
and he ages slowly, taking care of his life.” 

%
\end{verse}

Then\marginnote{6.1} the king gradually got used to having no more than a pint of rice. After some time King Pasenadi’s body slimmed right down. Stroking his limbs with his hands, at that time he expressed this heartfelt sentiment: 

“In\marginnote{6.3} both ways the Buddha has compassion for me: in the good of the present life and the good of the next.” 

%
\section*{{\suttatitleacronym SN 3.14}{\suttatitletranslation Battle (1st) }{\suttatitleroot Paṭhamasaṅgāmasutta}}
\addcontentsline{toc}{section}{\tocacronym{SN 3.14} \toctranslation{Battle (1st) } \tocroot{Paṭhamasaṅgāmasutta}}
\markboth{Battle (1st) }{Paṭhamasaṅgāmasutta}
\extramarks{SN 3.14}{SN 3.14}

At\marginnote{1.1} \textsanskrit{Sāvatthī}. 

Then\marginnote{1.2} King \textsanskrit{Ajātasattu} Vedehiputta of Magadha mobilized an army of four divisions and marched to \textsanskrit{Kāsi} to attack King Pasenadi of Kosala. When King Pasenadi heard of this, he mobilized an army of four divisions and marched to \textsanskrit{Kāsi} to defend it against \textsanskrit{Ajātasattu}. Then the two kings met in battle. And in that battle \textsanskrit{Ajātasattu} defeated Pasenadi, who withdrew to his own capital at \textsanskrit{Sāvatthī}. 

Then\marginnote{2.1} several mendicants robed up in the morning and, taking their bowls and robes, entered \textsanskrit{Sāvatthī} for alms. Then, after the meal, when they returned from almsround, they went up to the Buddha, bowed, sat down to one side, and told him what had happened. Then the Buddha said: 

“Mendicants,\marginnote{4.1} King \textsanskrit{Ajātasattu} has bad friends, companions, and associates. But King Pasenadi has good friends, companions, and associates. Yet on this day King Pasenadi will have a bad night’s sleep as one defeated.” 

That\marginnote{4.4} is what the Buddha said. … 

\begin{verse}%
“Victory\marginnote{5.1} breeds enmity; \\
the defeated sleep badly. \\
The peaceful sleep at ease, \\
having left victory and defeat behind.” 

%
\end{verse}

%
\section*{{\suttatitleacronym SN 3.15}{\suttatitletranslation Battle (2nd) }{\suttatitleroot Dutiyasaṅgāmasutta}}
\addcontentsline{toc}{section}{\tocacronym{SN 3.15} \toctranslation{Battle (2nd) } \tocroot{Dutiyasaṅgāmasutta}}
\markboth{Battle (2nd) }{Dutiyasaṅgāmasutta}
\extramarks{SN 3.15}{SN 3.15}

Then\marginnote{1.1} King \textsanskrit{Ajātasattu} Vedehiputta of Magadha mobilized an army of four divisions and marched to \textsanskrit{Kāsi} to attack King Pasenadi of Kosala. When King Pasenadi heard of this, he mobilized an army of four divisions and marched to \textsanskrit{Kāsi} to defend it against \textsanskrit{Ajātasattu}. Then the two kings met in battle. And in that battle Pasenadi defeated \textsanskrit{Ajātasattu} and captured him alive. 

Then\marginnote{1.7} King Pasenadi thought, “Even though I’ve never betrayed this King \textsanskrit{Ajātasattu}, he betrayed me. Still, he is my nephew. Now that I’ve vanquished all of \textsanskrit{Ajātasattu}’s elephant troops, cavalry, chariots, and infantry, why don’t I let him loose with just his life?” 

And\marginnote{2.1} that’s what he did. 

Then\marginnote{3.1} several mendicants … told the Buddha what had happened. 

Then,\marginnote{6.1} understanding this matter, on that occasion the Buddha recited these verses: 

\begin{verse}%
“A\marginnote{7.1} man goes on plundering \\
as long as it serves his ends. \\
But as soon as others plunder him, \\
the plunderer is plundered. 

For\marginnote{8.1} the fool thinks they’ve got away with it \\
so long as their wickedness has not ripened. \\
But as soon as that wickedness ripens, \\
they fall into suffering. 

A\marginnote{9.1} killer creates a killer; \\
a conqueror creates a conqueror; \\
an abuser creates abuse, \\
and a bully creates a bully. \\
And so as deeds unfold \\
the plunderer is plundered.” 

%
\end{verse}

%
\section*{{\suttatitleacronym SN 3.16}{\suttatitletranslation A Daughter }{\suttatitleroot Mallikāsutta}}
\addcontentsline{toc}{section}{\tocacronym{SN 3.16} \toctranslation{A Daughter } \tocroot{Mallikāsutta}}
\markboth{A Daughter }{Mallikāsutta}
\extramarks{SN 3.16}{SN 3.16}

At\marginnote{1.1} \textsanskrit{Sāvatthī}. 

Then\marginnote{1.2} King Pasenadi of Kosala went up to the Buddha, bowed, and sat down to one side. Then a man went up to the king and whispered in his ear, “Your Majesty, Queen \textsanskrit{Mallikā} has given birth to a daughter.” When this was said, King Pasenadi was disappointed. 

Then,\marginnote{2.1} knowing that King Pasenadi was disappointed, on that occasion the Buddha recited these verses: 

\begin{verse}%
“Well,\marginnote{3.1} some women are better than men, \\
O ruler of the people. \\
Wise and virtuous, \\
a devoted wife who honors her mother in law. 

And\marginnote{4.1} when she has a son, \\
he becomes a hero, O lord of the land. \\
The son of such a blessed lady \\
may even rule the realm.” 

%
\end{verse}

%
\section*{{\suttatitleacronym SN 3.17}{\suttatitletranslation Diligence }{\suttatitleroot Appamādasutta}}
\addcontentsline{toc}{section}{\tocacronym{SN 3.17} \toctranslation{Diligence } \tocroot{Appamādasutta}}
\markboth{Diligence }{Appamādasutta}
\extramarks{SN 3.17}{SN 3.17}

At\marginnote{1.1} \textsanskrit{Sāvatthī}. 

Seated\marginnote{1.2} to one side, King Pasenadi said to the Buddha, “Sir, is there one thing that secures benefits for both the present life and lives to come?” 

“There\marginnote{2.1} is, great king.” 

“So\marginnote{3.1} what is that one thing?” 

“Diligence,\marginnote{4.1} great king, is one thing that secures benefits for both the present life and lives to come. The footprints of all creatures that walk can fit inside an elephant’s footprint. So an elephant’s footprint is said to be the biggest of them all. In the same way, diligence is one thing that secures benefits for both the present life and lives to come.” 

That\marginnote{4.6} is what the Buddha said. … 

\begin{verse}%
“For\marginnote{5.1} one who desires a continuous flow \\
of exceptional delights—\\
long life, beauty, and health, \\
heaven, and birth in an eminent family—

the\marginnote{6.1} astute praise diligence \\
in making merit. \\
Being diligent, an astute person \\
secures both benefits: 

the\marginnote{7.1} benefit in this life, \\
and in lives to come. \\
A wise one, comprehending the meaning, \\
is said to be astute.” 

%
\end{verse}

%
\section*{{\suttatitleacronym SN 3.18}{\suttatitletranslation Good Friends }{\suttatitleroot Kalyāṇamittasutta}}
\addcontentsline{toc}{section}{\tocacronym{SN 3.18} \toctranslation{Good Friends } \tocroot{Kalyāṇamittasutta}}
\markboth{Good Friends }{Kalyāṇamittasutta}
\extramarks{SN 3.18}{SN 3.18}

At\marginnote{1.1} \textsanskrit{Sāvatthī}. 

Seated\marginnote{1.2} to one side, King Pasenadi said to the Buddha, “Just now, sir, as I was in private retreat this thought came to mind. ‘The teaching is well explained by the Buddha. But it’s for someone with good friends, companions, and associates, not for someone with bad friends, companions, and associates.’” 

“That’s\marginnote{2.1} so true, great king! That’s so true!” said the Buddha. And he repeated the king’s statement, adding: 

“Great\marginnote{3.1} king, this one time I was staying in the land of the Sakyans where they have a town named Townsville. Then the mendicant Ānanda came to me, bowed, sat down to one side, and said: ‘Sir, good friends, companions, and associates are half the spiritual life.’ 

When\marginnote{4.1} he had spoken, I said to him: ‘Not so, Ānanda! Not so, Ānanda! Good friends, companions, and associates are the whole of the spiritual life. A mendicant with good friends, companions, and associates can expect to develop and cultivate the noble eightfold path. 

And\marginnote{5.1} how does a mendicant with good friends develop and cultivate the noble eightfold path? It’s when a mendicant develops right view, right thought, right speech, right action, right livelihood, right effort, right mindfulness, and right immersion, which rely on seclusion, fading away, and cessation, and ripen as letting go. That’s how a mendicant with good friends develops and cultivates the noble eightfold path. And here’s another way to understand how good friends are the whole of the spiritual life. 

For,\marginnote{6.1} by relying on me as a good friend, sentient beings who are liable to rebirth, old age, and death, to sorrow, lamentation, pain, sadness, and distress are freed from all these things. This is another way to understand how good friends are the whole of the spiritual life.’ 

So,\marginnote{7.1} great king, you should train like this: ‘I will have good friends, companions, and associates.’ That’s how you should train. 

When\marginnote{8.1} you have good friends, companions, and associates, you should live supported by one thing: diligence in skillful qualities. 

When\marginnote{9.1} you’re diligent, supported by diligence, your ladies of the harem, aristocrat vassals, troops, and people of town and country will think: ‘The king lives diligently, supported by diligence. We’d better live diligently, supported by diligence!’ 

When\marginnote{13.1} you’re diligent, supported by diligence, then not only you yourself, but your ladies of the harem, and your treasury and storehouses will be guarded and protected.” 

That\marginnote{13.3} is what the Buddha said. … 

\begin{verse}%
“For\marginnote{14.1} one who desires a continuous flow \\
of exceptional wealth, \\
the astute praise diligence \\
in making merit. \\
Being diligent, an astute person \\
secures both benefits: 

the\marginnote{15.1} benefit in this life, \\
and in lives to come. \\
A wise one, comprehending the meaning, \\
is said to be astute.” 

%
\end{verse}

%
\section*{{\suttatitleacronym SN 3.19}{\suttatitletranslation Childless (1st) }{\suttatitleroot Paṭhamaaputtakasutta}}
\addcontentsline{toc}{section}{\tocacronym{SN 3.19} \toctranslation{Childless (1st) } \tocroot{Paṭhamaaputtakasutta}}
\markboth{Childless (1st) }{Paṭhamaaputtakasutta}
\extramarks{SN 3.19}{SN 3.19}

At\marginnote{1.1} \textsanskrit{Sāvatthī}. 

Then\marginnote{1.2} King Pasenadi of Kosala went up to the Buddha in the middle of the day, bowed, and sat down to one side. The Buddha said to him, “So, great king, where are you coming from in the middle of the day?” 

“Sir,\marginnote{2.1} here in \textsanskrit{Sāvatthī} a financier householder has passed away. Since he died childless, I have come after transferring his fortune to the royal compound. There was eight million in gold, not to mention the silver. And yet that financier ate meals of rough gruel with pickles. He wore clothes consisting of three pieces of sunn hemp. He traveled around in a vehicle that was a dilapidated little cart, holding a leaf as sunshade.” 

“That’s\marginnote{3.1} so true, great king! That’s so true! When a bad person has acquired exceptional wealth they don’t make themselves happy and pleased. Nor do they make their mother and father, partners and children, bondservants, workers, and staff, and friends and colleagues happy and pleased. And they don’t establish an uplifting religious donation for ascetics and brahmins that’s conducive to heaven, ripens in happiness, and leads to heaven. Because they haven’t made proper use of that wealth, rulers or bandits take it, or fire consumes it, or flood sweeps it away, or unloved heirs take it. Since that wealth is not properly utilized, it’s wasted, not used. 

Suppose\marginnote{4.1} there was a lotus pond in an uninhabited region with clear, sweet, cool water, clean, with smooth banks, delightful. But people don’t collect it or drink it or bathe in it or use it for any purpose. Since that water is not properly utilized, it’s wasted, not used. 

In\marginnote{4.4} the same way, when a bad person has acquired exceptional wealth … it’s wasted, not used. 

When\marginnote{5.1} a good person has acquired exceptional wealth they make themselves happy and pleased. And they make their mother and father, partners and children, bondservants, workers, and staff, and friends and colleagues happy and pleased. And they establish an uplifting religious donation for ascetics and brahmins that’s conducive to heaven, ripens in happiness, and leads to heaven. Because they make proper use of that wealth, rulers or bandits don’t take it, fire doesn’t consume it, flood doesn’t sweep it away, and unloved heirs don’t take it. Since that wealth is properly utilized, it’s used, not wasted. 

Suppose\marginnote{6.1} there was a lotus pond not far from a town or village with clear, sweet, cool water, clean, with smooth banks, delightful. And people collected it and drank it and bathed in it and used it for their own purpose. Since that water is properly utilized, it’s used, not wasted. 

In\marginnote{6.4} the same way, when a good person has acquired exceptional wealth … it’s used, not wasted. 

\begin{verse}%
As\marginnote{7.1} cool water in an uninhabited region \\
evaporates when not drunk; \\
so too when a sinner acquires wealth, \\
they neither use it themselves nor give it away. 

But\marginnote{8.1} when a wise and sensible person gets hold of wealth, \\
they use it and do their duty. \\
That head, having supported the family unit, \\
blameless, goes to a heavenly place.” 

%
\end{verse}

%
\section*{{\suttatitleacronym SN 3.20}{\suttatitletranslation Childless (2nd) }{\suttatitleroot Dutiyaaputtakasutta}}
\addcontentsline{toc}{section}{\tocacronym{SN 3.20} \toctranslation{Childless (2nd) } \tocroot{Dutiyaaputtakasutta}}
\markboth{Childless (2nd) }{Dutiyaaputtakasutta}
\extramarks{SN 3.20}{SN 3.20}

Then\marginnote{1.1} King Pasenadi of Kosala went up to the Buddha in the middle of the day … The Buddha said to him, “So, great king, where are you coming from in the middle of the day?” 

“Sir,\marginnote{2.1} here in \textsanskrit{Sāvatthī} a financier householder has passed away. Since he died childless, I have come after transferring his fortune to the royal compound. There was ten million in gold, not to mention the silver. And yet that financier ate meals of rough gruel with pickles. He wore clothes consisting of three pieces of sunn hemp. He traveled around in a vehicle that was a dilapidated little cart, holding a leaf as sunshade.” 

“That’s\marginnote{3.1} so true, great king! That’s so true! Once upon a time, great king, that financier householder provided almsfood on behalf of a Buddha awakened for themselves named \textsanskrit{Tagarasikhī}. He instructed: ‘Give alms to that ascetic,’ before getting up from his seat and leaving. But after giving he regretted it: ‘It would have been better to feed the bondservants or workers with that almsfood.’ What’s more, he murdered his brother’s only child for the sake of his fortune. 

Because\marginnote{4.1} that financier provided \textsanskrit{Tagarasikhī} with almsfood, as a result of that deed he was reborn seven times in a good place, a heavenly realm. And as a residual result of that same deed he held the position of financier seven times right here in \textsanskrit{Sāvatthī}. But because that financier regretted giving alms, as a result of that deed his mind didn’t tend to enjoy nice food, clothes, vehicles, or the five refined kinds of sensual stimulation. And because that financier murdered his brother’s only child for the sake of his fortune, as a result of that deed he burned in hell for many years, for many hundreds, many thousands, many hundreds of thousands of years. And as a residual result of that same deed, he is childless for the seventh time, his fortune ending up in the royal treasury. Now the old merit of that financier has been used up, and he hasn’t accumulated new merit. Today, great king, that financier burns in the Great Hell of Screams.” 

“Really,\marginnote{4.9} sir, that financier has been reborn in the Great Hell of Screams?” 

“Yes\marginnote{4.10} he has, great king.” 

That\marginnote{4.11} is what the Buddha said. … 

\begin{verse}%
“Grain,\marginnote{5.1} wealth, silver, and gold, \\
or whatever other possessions there are; \\
bondservants, workers, employees, \\
and those dependent for their livelihood: 

you\marginnote{6.1} must go on without taking these; \\
all of them are left behind. \\
But the deeds you do \\
by body, speech, and mind—

that’s\marginnote{7.1} what you can call your own. \\
That’s what you take when you go. \\
That’s what goes with you, \\
like a shadow that never leaves. 

That’s\marginnote{8.1} why you should do good, \\
investing in the future life. \\
The good deeds of sentient beings \\
support them in the next world.” 

%
\end{verse}

%
\addtocontents{toc}{\let\protect\contentsline\protect\nopagecontentsline}
\chapter*{Chapter Three }
\addcontentsline{toc}{chapter}{\tocchapterline{Chapter Three }}
\addtocontents{toc}{\let\protect\contentsline\protect\oldcontentsline}

%
\section*{{\suttatitleacronym SN 3.21}{\suttatitletranslation Persons }{\suttatitleroot Puggalasutta}}
\addcontentsline{toc}{section}{\tocacronym{SN 3.21} \toctranslation{Persons } \tocroot{Puggalasutta}}
\markboth{Persons }{Puggalasutta}
\extramarks{SN 3.21}{SN 3.21}

At\marginnote{1.1} \textsanskrit{Sāvatthī}. 

Then\marginnote{1.2} King Pasenadi of Kosala went up to the Buddha, bowed, and sat down to one side. The Buddha said to him: 

“Great\marginnote{1.3} king, these four people are found in the world. What four? 

\begin{enumerate}%
\item The dark bound for darkness, %
\item the dark bound for light, %
\item the light bound for darkness, %
\item and the light bound for light. %
\end{enumerate}

And\marginnote{2.1} how is a person dark and bound for darkness? It’s when some person is reborn in a low family—a family of outcastes, bamboo-workers, hunters, chariot-makers, or waste-collectors—poor, with little to eat or drink, where life is tough, and food and shelter are hard to find. And they’re ugly, unsightly, deformed, chronically ill—one-eyed, crippled, lame, or half-paralyzed. They don’t get to have food, drink, clothes, and vehicles; garlands, perfumes, and makeup; or bed, house, and lighting. And they do bad things by way of body, speech, and mind. When their body breaks up, after death, they’re reborn in a place of loss, a bad place, the underworld, hell. 

This\marginnote{3.1} person is like someone who goes from darkness to darkness, from blackness to blackness, from bloodstain to bloodstain. That’s how a person is dark and bound for darkness. 

And\marginnote{4.1} how is a person dark and bound for light? It’s when some person is reborn in a low family—a family of outcastes, bamboo-workers, hunters, chariot-makers, or waste-collectors—poor, with little to eat or drink, where life is tough, and food and shelter are hard to find. And they’re ugly, unsightly, deformed, chronically ill—one-eyed, crippled, lame, or half-paralyzed. They don’t get to have food, drink, clothes, and vehicles; garlands, perfumes, and makeup; or bed, house, and lighting. But they do good things by way of body, speech, and mind. When their body breaks up, after death, they’re reborn in a good place, a heavenly realm. 

This\marginnote{5.1} person is like someone who ascends from the ground to a couch; from a couch to horseback; from horseback to an elephant; and from an elephant to a stilt longhouse. That’s how a person is dark and bound for light. 

And\marginnote{6.1} how is a person light and bound for darkness? It’s when some person is reborn in an eminent family—a well-to-do family of aristocrats, brahmins, or householders—rich, affluent, and wealthy, with lots of gold and silver, lots of property and assets, and lots of money and grain. And they’re attractive, good-looking, lovely, of surpassing beauty. They get to have food, drink, clothes, and vehicles; garlands, perfumes, and makeup; and bed, house, and lighting. But they do bad things by way of body, speech, and mind. When their body breaks up, after death, they’re reborn in a place of loss, a bad place, the underworld, hell. 

This\marginnote{7.1} person is like someone who descends from a stilt longhouse to an elephant; from an elephant to horseback; from horseback to a couch; and from a couch to the ground; and from the ground they enter darkness. That’s how a person is light and bound for darkness. 

And\marginnote{8.1} how is a person light and bound for light? It’s when some person is reborn in an eminent family—a well-to-do family of aristocrats, brahmins, or householders—rich, affluent, and wealthy, with lots of gold and silver, lots of property and assets, and lots of money and grain. And they’re attractive, good-looking, lovely, of surpassing beauty. They get to have food, drink, clothes, and vehicles; garlands, perfumes, and makeup; and bed, house, and lighting. And they do good things by way of body, speech, and mind. When their body breaks up, after death, they’re reborn in a good place, a heavenly realm. 

This\marginnote{9.1} person is like someone who shifts from one couch to another; from the back of one horse to another; from one elephant to another; or from one stilt longhouse to another. That’s how a person is light and bound for light. These are the four people found in the world.” 

That\marginnote{9.5} is what the Buddha said. … 

\begin{verse}%
“O\marginnote{10.1} king, some people are poor, \\
faithless and stingy. \\
Miserly, with bad intentions, \\
they lack regard, they have wrong view. 

They\marginnote{11.1} abuse and insult \\
ascetics and brahmins \\
and other renunciates. \\
They’re nihilists and bullies, 

who\marginnote{12.1} prevent others from giving \\
food to beggars. \\
O king, ruler of the people: \\
when such people die \\
they fall into the terrible hell—\\
from darkness they’re bound for darkness. 

O\marginnote{13.1} king, some people are poor, \\
but faithful and not stingy. \\
They give with best of intentions, \\
that peaceful-hearted person. 

They\marginnote{14.1} rise for and bow to \\
ascetics and brahmins \\
and other renunciates. \\
Training in moral conduct, 

they\marginnote{15.1} don’t prevent others from giving \\
food to beggars. \\
O king, ruler of the people: \\
when such people die \\
they go to the heaven of the Three and Thirty—\\
from darkness they’re bound for light. 

O\marginnote{16.1} king, some people are rich, \\
but faithless and stingy. \\
Miserly, with bad intentions, \\
they lack regard, they have wrong view. 

They\marginnote{17.1} abuse and insult \\
ascetics and brahmins \\
and other renunciates. \\
They’re nihilists and bullies, 

who\marginnote{18.1} prevent others from giving \\
food to beggars. \\
O king, ruler of the people: \\
when such people die \\
they fall into the terrible hell—\\
from light they’re bound for darkness. 

O\marginnote{19.1} king, some people are rich, \\
faithful and not stingy. \\
They give with best of intentions, \\
that peaceful-hearted person. 

They\marginnote{20.1} rise for and bow to \\
ascetics and brahmins \\
and other renunciates. \\
Training in moral conduct, 

they\marginnote{21.1} don’t prevent others from giving \\
food to beggars. \\
O king, ruler of the people: \\
when such people die \\
they go to the heaven of the Three and Thirty—\\
from light they’re bound for light.” 

%
\end{verse}

%
\section*{{\suttatitleacronym SN 3.22}{\suttatitletranslation Grandmother }{\suttatitleroot Ayyikāsutta}}
\addcontentsline{toc}{section}{\tocacronym{SN 3.22} \toctranslation{Grandmother } \tocroot{Ayyikāsutta}}
\markboth{Grandmother }{Ayyikāsutta}
\extramarks{SN 3.22}{SN 3.22}

At\marginnote{1.1} \textsanskrit{Sāvatthī}. 

King\marginnote{1.2} Pasenadi of Kosala sat to one side, and the Buddha said to him, “So, great king, where are you coming from in the middle of the day?” 

“Sir,\marginnote{2.1} my grandmother has passed away. She was old, elderly and senior. She was advanced in years and had reached the final stage of life; she was a hundred and twenty years old. But I loved my grandmother; she was dear to me. If by giving away the elephant-treasure I could get my grandmother back, I’d do it. If by giving away the horse-treasure I could get my grandmother back, I’d do it. If by giving away a prize village I could get my grandmother back, I’d do it. If by giving away the whole country I could get my grandmother back, I’d do it.” 

“Great\marginnote{2.11} king, all sentient beings are liable to die. Death is their end; they’re not exempt from death.” 

“It’s\marginnote{2.12} incredible, sir, it’s amazing, how well said this was by the Buddha: ‘All sentient beings are liable to die. Death is their end; they’re not exempt from death.’” 

“That’s\marginnote{3.1} so true, great king! That’s so true! All sentient beings are liable to die. Death is their end; they’re not exempt from death. It’s like the vessels made by potters. Whatever kind they are, whether baked or unbaked, all of them are liable to break apart. Breaking is their end; they’re not exempt from breakage. In the same way, all sentient beings are liable to die. Death is their end; they’re not exempt from death.” 

That\marginnote{3.5} is what the Buddha said. … 

\begin{verse}%
“All\marginnote{4.1} beings will die, \\
for life ends with death. \\
They pass on according to their deeds, \\
reaping the fruits of good and bad. \\
Those who do bad go to hell, \\
and if you do good you go to heaven. 

That’s\marginnote{5.1} why you should do good, \\
investing in the future life. \\
The good deeds of sentient beings \\
support them in the next world.” 

%
\end{verse}

%
\section*{{\suttatitleacronym SN 3.23}{\suttatitletranslation The World }{\suttatitleroot Lokasutta}}
\addcontentsline{toc}{section}{\tocacronym{SN 3.23} \toctranslation{The World } \tocroot{Lokasutta}}
\markboth{The World }{Lokasutta}
\extramarks{SN 3.23}{SN 3.23}

At\marginnote{1.1} \textsanskrit{Sāvatthī}. 

Seated\marginnote{1.2} to one side, King Pasenadi said to the Buddha, “Sir, how many things arise in the world for its harm, suffering, and discomfort?” 

“Great\marginnote{1.4} king, three things arise in the world for its harm, suffering, and discomfort. What three? Greed, hate, and delusion. These three things arise in the world for its harm, suffering, and discomfort.” 

That\marginnote{1.10} is what the Buddha said. … 

\begin{verse}%
“When\marginnote{2.1} greed, hate, and delusion, \\
have arisen inside oneself, \\
they harm a person of wicked heart, \\
as a reed is destroyed by its own fruit.” 

%
\end{verse}

%
\section*{{\suttatitleacronym SN 3.24}{\suttatitletranslation Archery }{\suttatitleroot Issattasutta}}
\addcontentsline{toc}{section}{\tocacronym{SN 3.24} \toctranslation{Archery } \tocroot{Issattasutta}}
\markboth{Archery }{Issattasutta}
\extramarks{SN 3.24}{SN 3.24}

At\marginnote{1.1} \textsanskrit{Sāvatthī}. 

Seated\marginnote{1.2} to one side, King Pasenadi said to the Buddha, “Sir, where should a gift be given?” 

“Wherever\marginnote{1.4} your heart feels inspired, great king.” 

“But\marginnote{1.5} sir, where is a gift very fruitful?” 

“Where\marginnote{1.6} a gift should be given is one thing, great king, but where a gift is very fruitful is another. A gift is very fruitful when it’s given to an ethical person, not so much to an unethical person. Well then, great king, I’ll ask you about this in return, and you can answer as you like. 

What\marginnote{1.9} do you think, great king? Suppose you were at war, ready to fight a battle. Then along comes an aristocrat youth who is untrained, inexpert, unfit, inexperienced. And he’s fearful, scared, nervous, quick to flee. Would you employ such a man? Would he be of any use to you?” 

“No,\marginnote{1.13} sir, I would have no use for such a man.” 

“What\marginnote{1.14} about a brahmin youth, a merchant youth, or a worker youth who was similar?” 

“No,\marginnote{1.17} sir, I would have no use for such a man.” 

“What\marginnote{2.1} do you think, great king? Suppose you were at war, ready to fight a battle. Then along comes an aristocrat youth who is trained, expert, fit, experienced. And he’s fearless, brave, bold, standing his ground. Would you employ such a man? Would he be of any use to you?” 

“Yes,\marginnote{2.5} sir, I would have a use for such a man.” 

“What\marginnote{2.6} about a brahmin youth, a merchant youth, or a worker youth who was similar? Would you employ such a man? Would he be of any use to you?” 

“Yes,\marginnote{2.10} sir, I would have a use for such a man.” 

“In\marginnote{3.1} the same way, a gift to anyone who has given up five factors and possesses five factors is very fruitful, no matter what family they’ve gone forth from. 

What\marginnote{3.2} are the five factors they’ve given up? Sensual desire, ill will, dullness and drowsiness, restlessness and remorse, and doubt. These are the five factors they’ve given up. 

What\marginnote{3.5} are the five factors they possess? The entire spectrum of an adept’s ethics, immersion, wisdom, freedom, and knowledge and vision of freedom. These are the five factors they possess. 

I\marginnote{3.8} say that a gift to anyone who has given up these five factors and possesses these five factors is very fruitful.” 

That\marginnote{3.9} is what the Buddha said. Then the Holy One, the Teacher, went on to say: 

\begin{verse}%
“Any\marginnote{4.1} youth skilled at archery, \\
powerful and vigorous, \\
would be employed by a king going to war—\\
one is not a coward by reason of birth. 

Just\marginnote{5.1} so, whoever is settled \\
in the qualities of patience and gentleness, \\
a clever person with noble conduct, \\
should be venerated even if they’re low born. 

You\marginnote{6.1} should build lovely hermitages \\
and settle learned people in them. \\
You should set up water supplies in barren regions \\
and passages in places hard to travel. 

Food,\marginnote{7.1} drink, edibles, \\
clothes, and lodgings \\
should be given to the upright ones, \\
with a clear and confident heart. 

The\marginnote{8.1} thundering rain cloud, \\
its hundred peaks wreathed in lightning, \\
pours down over the rich earth, \\
soaking the uplands and valleys. 

So\marginnote{9.1} too an astute person, \\
faithful and learned, \\
should prepare a meal to satisfy \\
renunciates with food and drink. 

Rejoicing,\marginnote{10.1} they strew gifts about, \\
crying ‘Give! give!’ \\
For that is their thunder, \\
like the gods when it rains. \\
That stream of merit so abundant \\
showers down on the giver.” 

%
\end{verse}

%
\section*{{\suttatitleacronym SN 3.25}{\suttatitletranslation The Simile of the Mountain }{\suttatitleroot Pabbatūpamasutta}}
\addcontentsline{toc}{section}{\tocacronym{SN 3.25} \toctranslation{The Simile of the Mountain } \tocroot{Pabbatūpamasutta}}
\markboth{The Simile of the Mountain }{Pabbatūpamasutta}
\extramarks{SN 3.25}{SN 3.25}

At\marginnote{1.1} \textsanskrit{Sāvatthī}. 

King\marginnote{1.2} Pasenadi of Kosala sat to one side, and the Buddha said to him, “So, great king, where are you coming from in the middle of the day?” 

“Sir,\marginnote{1.4} there are anointed aristocratic kings who are infatuated with authority, and obsessed with greed for sensual pleasures. They have attained stability in the country, occupying a vast conquered territory. Today I have been busy fulfilling the duties of such kings.” 

“What\marginnote{2.1} do you think, great king? Suppose a trustworthy and reliable man were to come from the east. He’d approach you and say: ‘Please sir, you should know this. I come from the east. There I saw a huge mountain that reached the clouds. And it was coming this way, crushing all creatures. So then, great king, do what you must!’ 

Then\marginnote{2.7} a second trustworthy and reliable man were to come from the west … a third from the north … and a fourth from the south. He’d approach you and say: ‘Please sir, you should know this. I come from the south. There I saw a huge mountain that reached the clouds. And it was coming this way, crushing all creatures. So then, great king, do what you must!’ 

Should\marginnote{2.14} such a dire threat arise—a terrible loss of human life, when human birth is so rare—what would you do?” 

“Sir,\marginnote{3.1} what could I do but practice the teachings, practice morality, doing skillful and good actions?” 

“I\marginnote{4.1} tell you, great king, I announce to you: old age and death are advancing upon you. Since old age and death are advancing upon you, what would you do?” 

“Sir,\marginnote{4.3} what can I do but practice the teachings, practice morality, doing skillful and good actions? 

Sir,\marginnote{4.4} there are anointed aristocratic kings who are infatuated with authority, and obsessed with greed for sensual pleasures. They have attained stability in the country, occupying a vast conquered territory. Such kings engage in battles of elephants, cavalry, chariots, or infantry. But there is no place, no scope for such battles when old age and death are advancing. 

In\marginnote{4.11} this royal court there are ministers of wise counsel who are capable of dividing an approaching enemy by wise counsel. But there is no place, no scope for such diplomatic battles when old age and death are advancing. 

In\marginnote{4.13} this royal court there is abundant gold coin and bullion stored in dungeons and towers. Using this wealth we can pay off an approaching enemy. But there is no place, no scope for such monetary battles when old age and death are advancing. 

When\marginnote{4.15} old age and death are advancing, what can I do but practice the teachings, practice morality, doing skillful and good actions?” 

“That’s\marginnote{5.1} so true, great king! That’s so true! When old age and death are advancing, what can you do but practice the teachings, practice morality, doing skillful and good actions?” 

That\marginnote{5.3} is what the Buddha said. Then the Holy One, the Teacher, went on to say: 

\begin{verse}%
“Suppose\marginnote{6.1} there were vast mountains \\
of solid rock touching the sky \\
drawing in from all sides \\
and crushing the four quarters. 

So\marginnote{7.1} too old age and death \\
advance upon all living creatures—\\
aristocrats, brahmins, merchants, \\
workers, outcastes, and scavengers. \\
They spare nothing. \\
They crush all beneath them. 

There’s\marginnote{8.1} nowhere for elephants to take a stand, \\
nor chariots nor infantry. \\
They can’t be defeated \\
by diplomatic battles or by wealth. 

That’s\marginnote{9.1} why an astute person, \\
seeing what’s good for themselves, \\
being wise, would place faith \\
in the Buddha, the teaching, and the \textsanskrit{Saṅgha}. 

Whoever\marginnote{10.1} lives by the teaching \\
in body, speech, and mind, \\
is praised in this life \\
and departs to rejoice in heaven.” 

%
\end{verse}

\scendsutta{The Linked Discourses with the Kosalan are completed. }

%
\addtocontents{toc}{\let\protect\contentsline\protect\nopagecontentsline}
\part*{Linked Discourses With Māra }
\addcontentsline{toc}{part}{Linked Discourses With Māra }
\markboth{}{}
\addtocontents{toc}{\let\protect\contentsline\protect\oldcontentsline}

%
\addtocontents{toc}{\let\protect\contentsline\protect\nopagecontentsline}
\chapter*{Chapter One }
\addcontentsline{toc}{chapter}{\tocchapterline{Chapter One }}
\addtocontents{toc}{\let\protect\contentsline\protect\oldcontentsline}

%
\section*{{\suttatitleacronym SN 4.1}{\suttatitletranslation Mortification }{\suttatitleroot Tapokammasutta}}
\addcontentsline{toc}{section}{\tocacronym{SN 4.1} \toctranslation{Mortification } \tocroot{Tapokammasutta}}
\markboth{Mortification }{Tapokammasutta}
\extramarks{SN 4.1}{SN 4.1}

\scevam{So\marginnote{1.1} I have heard. }At one time, when he was first awakened, the Buddha was staying near \textsanskrit{Uruvelā} at the root of the goatherd’s banyan tree on the bank of the \textsanskrit{Nerañjarā} River. 

Then\marginnote{1.3} as he was in private retreat this thought came to his mind, “I am truly freed from that grueling work! Thank goodness I’m freed from that pointless grueling work. Thank goodness that, steadfast and mindful, I have attained awakening.” 

And\marginnote{2.1} then \textsanskrit{Māra} the Wicked, knowing what the Buddha was thinking, went up to him and addressed him in verse: 

\begin{verse}%
“You’ve\marginnote{3.1} departed from the practice of mortification \\
by which humans purify themselves. \\
You’re impure, but think yourself pure; \\
you’ve strayed from the path of purity.” 

%
\end{verse}

Then\marginnote{4.1} the Buddha, knowing that this was \textsanskrit{Māra} the Wicked, replied to him in verse: 

\begin{verse}%
“I\marginnote{5.1} realized that it’s pointless; \\
all that mortification in search of immortality \\
is as futile \\
as oars and rudder on dry land. 

Ethics,\marginnote{6.1} immersion, and wisdom: \\
by developing this path to awakening \\
I attained ultimate purity. \\
You’re beaten, terminator!” 

%
\end{verse}

Then\marginnote{7.1} \textsanskrit{Māra} the Wicked, thinking, “The Buddha knows me! The Holy One knows me!” miserable and sad, vanished right there. 

%
\section*{{\suttatitleacronym SN 4.2}{\suttatitletranslation In the Form of an Elephant King }{\suttatitleroot Hatthirājavaṇṇasutta}}
\addcontentsline{toc}{section}{\tocacronym{SN 4.2} \toctranslation{In the Form of an Elephant King } \tocroot{Hatthirājavaṇṇasutta}}
\markboth{In the Form of an Elephant King }{Hatthirājavaṇṇasutta}
\extramarks{SN 4.2}{SN 4.2}

\scevam{So\marginnote{1.1} I have heard. }At one time, when he was first awakened, the Buddha was staying near \textsanskrit{Uruvelā} at the root of the goatherd’s banyan tree on the bank of the \textsanskrit{Nerañjarā} River. 

Now\marginnote{1.3} at that time the Buddha was meditating in the open during the dark of night, while a gentle rain drizzled down. 

Then\marginnote{1.4} \textsanskrit{Māra} the Wicked, wanting to make the Buddha feel fear, terror, and goosebumps, manifested in the form of a huge elephant king and approached him. Its head was like a huge block of soapstone. Its tusks were like pure silver. Its trunk was like a long plough pole. 

Then\marginnote{1.11} the Buddha, knowing that this was \textsanskrit{Māra} the Wicked, addressed him in verse: 

\begin{verse}%
“Transmigrating\marginnote{2.1} for such a long time, \\
you’ve made forms beautiful and ugly. \\
Enough of this, Wicked One! \\
You’re beaten, terminator!” 

%
\end{verse}

Then\marginnote{3.1} \textsanskrit{Māra} the Wicked, thinking, “The Buddha knows me! The Holy One knows me!” miserable and sad, vanished right there. 

%
\section*{{\suttatitleacronym SN 4.3}{\suttatitletranslation Beautiful }{\suttatitleroot Subhasutta}}
\addcontentsline{toc}{section}{\tocacronym{SN 4.3} \toctranslation{Beautiful } \tocroot{Subhasutta}}
\markboth{Beautiful }{Subhasutta}
\extramarks{SN 4.3}{SN 4.3}

\scevam{So\marginnote{1.1} I have heard. }At one time, when he was first awakened, the Buddha was staying near \textsanskrit{Uruvelā} at the root of the goatherd’s banyan tree on the bank of the \textsanskrit{Nerañjarā} River. 

Now\marginnote{1.3} at that time the Buddha was meditating in the open during the dark of night, while a gentle rain drizzled down. 

Then\marginnote{1.4} \textsanskrit{Māra} the Wicked, wanting to make the Buddha feel fear, terror, and goosebumps, approached him, and while not far away generated a rainbow of bright colors, both beautiful and ugly. 

Then\marginnote{1.5} the Buddha, knowing that this was \textsanskrit{Māra} the Wicked, replied to him in verse: 

\begin{verse}%
“Transmigrating\marginnote{2.1} for such a long time, \\
you’ve made forms beautiful and ugly. \\
Enough of this, Wicked One! \\
You’re beaten, terminator. 

Those\marginnote{3.1} who are well restrained \\
in body, speech, and mind \\
don’t fall under \textsanskrit{Māra}’s sway, \\
nor are they your lackies.” 

%
\end{verse}

Then\marginnote{4.1} \textsanskrit{Māra} … vanished right there. 

%
\section*{{\suttatitleacronym SN 4.4}{\suttatitletranslation Māra’s Snares (1st) }{\suttatitleroot Paṭhamamārapāsasutta}}
\addcontentsline{toc}{section}{\tocacronym{SN 4.4} \toctranslation{Māra’s Snares (1st) } \tocroot{Paṭhamamārapāsasutta}}
\markboth{Māra’s Snares (1st) }{Paṭhamamārapāsasutta}
\extramarks{SN 4.4}{SN 4.4}

\scevam{So\marginnote{1.1} I have heard. }At one time the Buddha was staying near Benares, in the deer park at Isipatana. There the Buddha addressed the mendicants, “Mendicants!” 

“Venerable\marginnote{1.5} sir,” they replied. The Buddha said this: 

“Mendicants,\marginnote{2.1} I have attained and realized supreme freedom through proper attention and proper effort. You too should attain and realize supreme freedom through proper attention and proper effort.” 

Then\marginnote{2.3} \textsanskrit{Māra} the Wicked went up to the Buddha and addressed him in verse: 

\begin{verse}%
“You’re\marginnote{3.1} bound by \textsanskrit{Māra}’s snares, \\
both human and divine. \\
You’re bound by \textsanskrit{Māra}’s bonds: \\
you won’t escape me, ascetic!” 

“I’m\marginnote{4.1} freed from \textsanskrit{Māra}’s snares, \\
both human and divine. \\
I’m freed from \textsanskrit{Māra}’s bonds. \\
You’re beaten, terminator!” 

%
\end{verse}

Then\marginnote{5.1} \textsanskrit{Māra} … vanished right there. 

%
\section*{{\suttatitleacronym SN 4.5}{\suttatitletranslation Māra’s Snares (2nd) }{\suttatitleroot Dutiyamārapāsasutta}}
\addcontentsline{toc}{section}{\tocacronym{SN 4.5} \toctranslation{Māra’s Snares (2nd) } \tocroot{Dutiyamārapāsasutta}}
\markboth{Māra’s Snares (2nd) }{Dutiyamārapāsasutta}
\extramarks{SN 4.5}{SN 4.5}

At\marginnote{1.1} one time the Buddha was staying near Benares, in the deer park at Isipatana. There the Buddha addressed the mendicants, “Mendicants!” 

“Venerable\marginnote{1.4} sir,” they replied. The Buddha said this: 

“Mendicants,\marginnote{2.1} I am freed from all snares, both human and divine. You are also freed from all snares, both human and divine. 

Wander\marginnote{2.3} forth, mendicants, for the welfare and happiness of the people, out of compassion for the world, for the benefit, welfare, and happiness of gods and humans. Let not two go by one road. 

Teach\marginnote{2.5} the Dhamma that’s good in the beginning, good in the middle, and good in the end, meaningful and well-phrased. And reveal a spiritual practice that’s entirely full and pure. There are beings with little dust in their eyes. They’re in decline because they haven’t heard the teaching. There will be those who understand the teaching! 

I\marginnote{2.8} will travel to \textsanskrit{Uruvelā}, the village of \textsanskrit{Senāni}, in order to teach the Dhamma.” 

Then\marginnote{2.9} \textsanskrit{Māra} the Wicked went up to the Buddha and addressed him in verse: 

\begin{verse}%
“You’re\marginnote{3.1} bound by all snares, \\
both human and divine. \\
You’re bound by the great bond: \\
you won’t escape me, ascetic!” 

“I’m\marginnote{4.1} freed from all snares, \\
both human and divine. \\
I’m freed from the great bonds; \\
You’re beaten, terminator!” 

%
\end{verse}

Then\marginnote{5.1} \textsanskrit{Māra} … vanished right there. 

%
\section*{{\suttatitleacronym SN 4.6}{\suttatitletranslation A Serpent }{\suttatitleroot Sappasutta}}
\addcontentsline{toc}{section}{\tocacronym{SN 4.6} \toctranslation{A Serpent } \tocroot{Sappasutta}}
\markboth{A Serpent }{Sappasutta}
\extramarks{SN 4.6}{SN 4.6}

\scevam{So\marginnote{1.1} I have heard. }At one time the Buddha was staying near \textsanskrit{Rājagaha}, in the Bamboo Grove, the squirrels’ feeding ground. 

Now\marginnote{1.3} at that time the Buddha was meditating in the open during the dark of night, while a gentle rain drizzled down. 

Then\marginnote{2.1} \textsanskrit{Māra} the Wicked, wanting to make the Buddha feel fear, terror, and goosebumps, manifested in the form of a huge serpent king and approached him. Its body was like a huge canoe carved from a single tree. Its hood was like a large brewer’s sieve. Its eyes were like those big bronze dishes from Kosala. Its tongue flickered from its mouth like lightning flashes in a thunderstorm. The sound of its breathing was like the puffing of a blacksmith’s bellows. 

Then\marginnote{3.1} the Buddha, knowing that this was \textsanskrit{Māra} the Wicked, replied to him in verse: 

\begin{verse}%
“A\marginnote{4.1} self-controlled sage frequents \\
empty buildings for lodging. \\
It’s appropriate for such a person \\
to live there after relinquishing. 

Though\marginnote{5.1} there are lots of creepy crawlies, \\
and lots of flies and snakes, \\
they wouldn’t stir a hair \\
of a great sage in that empty hut. 

Though\marginnote{6.1} the sky may split and the earth may quake, \\
and all creatures be stricken with fear; \\
and even if an arrow’s aimed at their breast, \\
the Buddhas take no shelter in attachments.” 

%
\end{verse}

Then\marginnote{7.1} \textsanskrit{Māra} the Wicked, thinking, “The Buddha knows me! The Holy One knows me!” miserable and sad, vanished right there. 

%
\section*{{\suttatitleacronym SN 4.7}{\suttatitletranslation Sleeping }{\suttatitleroot Supatisutta}}
\addcontentsline{toc}{section}{\tocacronym{SN 4.7} \toctranslation{Sleeping } \tocroot{Supatisutta}}
\markboth{Sleeping }{Supatisutta}
\extramarks{SN 4.7}{SN 4.7}

At\marginnote{1.1} one time the Buddha was staying near \textsanskrit{Rājagaha}, in the Bamboo Grove, the squirrels’ feeding ground. 

He\marginnote{1.2} spent most of the night walking mindfully in the open. At the crack of dawn he washed his feet and entered his dwelling. He laid down in the lion’s posture—on the right side, placing one foot on top of the other—mindful and aware, and focused on the time of getting up. 

Then\marginnote{1.4} \textsanskrit{Māra} the Wicked went up to the Buddha and addressed him in verse: 

\begin{verse}%
“What,\marginnote{2.1} you’re asleep? Really, you’re asleep? \\
You sleep like a loser—what’s up with that? \\
You sleep, thinking that the hut is empty. \\
You sleep when the sun has come up—what’s up with that?” 

“For\marginnote{3.1} them there is no craving—\\
the weaver, the clinger—to track them anywhere. \\
With the ending of all attachments the awakened Buddha sleeps. \\
What’s that got to do with you, \textsanskrit{Māra}?” 

%
\end{verse}

Then\marginnote{4.1} \textsanskrit{Māra} … vanished right there. 

%
\section*{{\suttatitleacronym SN 4.8}{\suttatitletranslation Delighting }{\suttatitleroot Nandatisutta}}
\addcontentsline{toc}{section}{\tocacronym{SN 4.8} \toctranslation{Delighting } \tocroot{Nandatisutta}}
\markboth{Delighting }{Nandatisutta}
\extramarks{SN 4.8}{SN 4.8}

\scevam{So\marginnote{1.1} I have heard. }At one time the Buddha was staying near \textsanskrit{Sāvatthī} in Jeta’s Grove, \textsanskrit{Anāthapiṇḍika}’s monastery. 

Then\marginnote{1.3} \textsanskrit{Māra} the Wicked went up to the Buddha and recited this verse in the Buddha’s presence: 

\begin{verse}%
“Your\marginnote{2.1} children bring you delight! \\
Your cattle also bring you delight! \\
For attachments are a man’s delight; \\
without attachments there’s no delight.” 

“Your\marginnote{3.1} children bring you sorrow. \\
Your cattle also bring you sorrow. \\
For attachments are a man’s sorrow; \\
without attachments there are no sorrows.” 

%
\end{verse}

Then\marginnote{4.1} \textsanskrit{Māra} the Wicked, thinking, “The Buddha knows me! The Holy One knows me!” miserable and sad, vanished right there. 

%
\section*{{\suttatitleacronym SN 4.9}{\suttatitletranslation Life Span (1st) }{\suttatitleroot Paṭhamaāyusutta}}
\addcontentsline{toc}{section}{\tocacronym{SN 4.9} \toctranslation{Life Span (1st) } \tocroot{Paṭhamaāyusutta}}
\markboth{Life Span (1st) }{Paṭhamaāyusutta}
\extramarks{SN 4.9}{SN 4.9}

\scevam{So\marginnote{1.1} I have heard. }At one time the Buddha was staying near \textsanskrit{Rājagaha}, in the Bamboo Grove, the squirrels’ feeding ground. There the Buddha addressed the mendicants, “Mendicants!” 

“Venerable\marginnote{1.5} sir,” they replied. The Buddha said this: 

“Mendicants,\marginnote{2.1} the life span of humans is short. You must go to the next life. So you should do what is skillful, you should practice the spiritual life. No-one born is immortal. A long life is a hundred years or a little more.” 

Then\marginnote{3.1} \textsanskrit{Māra} the Wicked went up to the Buddha and addressed him in verse: 

\begin{verse}%
“The\marginnote{4.1} life of humans is long! \\
A good person wouldn’t scorn it. \\
Live like a suckling babe, \\
for Death has not come for you.” 

“The\marginnote{5.1} life of humans is short, \\
and a good person scorns it. \\
They should live as though their head was on fire, \\
for Death comes for everyone.” 

%
\end{verse}

Then\marginnote{6.1} \textsanskrit{Māra} … vanished right there. 

%
\section*{{\suttatitleacronym SN 4.10}{\suttatitletranslation Life Span (2nd) }{\suttatitleroot Dutiyaāyusutta}}
\addcontentsline{toc}{section}{\tocacronym{SN 4.10} \toctranslation{Life Span (2nd) } \tocroot{Dutiyaāyusutta}}
\markboth{Life Span (2nd) }{Dutiyaāyusutta}
\extramarks{SN 4.10}{SN 4.10}

\scevam{So\marginnote{1.1} I have heard. }At one time the Buddha was staying near \textsanskrit{Rājagaha}, in the Bamboo Grove, the squirrels’ feeding ground. There the Buddha … said: 

“Mendicants,\marginnote{2.1} the life span of humans is short. You must go to the next life. So you should do what is skillful, you should practice the spiritual life. No-one born is immortal. A long life is a hundred years or a little more.” 

Then\marginnote{3.1} \textsanskrit{Māra} the Wicked went up to the Buddha and addressed him in verse: 

\begin{verse}%
“The\marginnote{4.1} days and nights don’t rush by, \\
and life isn’t cut short. \\
The life of mortals keeps rolling on, \\
like a chariot’s rim around the hub.” 

“The\marginnote{5.1} days and nights rush by, \\
and then life is cut short. \\
The life of mortals wastes away, \\
like the water in tiny streams.” 

%
\end{verse}

Then\marginnote{6.1} \textsanskrit{Māra} the Wicked, thinking, “The Buddha knows me! The Holy One knows me!” miserable and sad, vanished right there. 

%
\addtocontents{toc}{\let\protect\contentsline\protect\nopagecontentsline}
\chapter*{Chapter Two }
\addcontentsline{toc}{chapter}{\tocchapterline{Chapter Two }}
\addtocontents{toc}{\let\protect\contentsline\protect\oldcontentsline}

%
\section*{{\suttatitleacronym SN 4.11}{\suttatitletranslation Boulders }{\suttatitleroot Pāsāṇasutta}}
\addcontentsline{toc}{section}{\tocacronym{SN 4.11} \toctranslation{Boulders } \tocroot{Pāsāṇasutta}}
\markboth{Boulders }{Pāsāṇasutta}
\extramarks{SN 4.11}{SN 4.11}

At\marginnote{1.1} one time the Buddha was staying near \textsanskrit{Rājagaha}, on the Vulture’s Peak Mountain. Now at that time the Buddha was meditating in the open during the dark of night, while a gentle rain drizzled down. 

Then\marginnote{1.3} \textsanskrit{Māra} the Wicked, wanting to make the Buddha feel fear, terror, and goosebumps, approached him, and crushed some large boulders close by him. 

Then\marginnote{2.1} the Buddha, knowing that this was \textsanskrit{Māra} the Wicked, addressed him in verse: 

\begin{verse}%
“Even\marginnote{3.1} if you shake \\
this entire Vulture’s Peak, \\
the rightly released, \\
the awakened, are unshaken.” 

%
\end{verse}

Then\marginnote{4.1} \textsanskrit{Māra} the Wicked, thinking, “The Buddha knows me! The Holy One knows me!” miserable and sad, vanished right there. 

%
\section*{{\suttatitleacronym SN 4.12}{\suttatitletranslation Lion }{\suttatitleroot Kinnusīhasutta}}
\addcontentsline{toc}{section}{\tocacronym{SN 4.12} \toctranslation{Lion } \tocroot{Kinnusīhasutta}}
\markboth{Lion }{Kinnusīhasutta}
\extramarks{SN 4.12}{SN 4.12}

At\marginnote{1.1} one time the Buddha was staying near \textsanskrit{Sāvatthī} in Jeta’s Grove, \textsanskrit{Anāthapiṇḍika}’s monastery. Now, at that time the Buddha was teaching Dhamma, surrounded by a large assembly. 

Then\marginnote{2.1} \textsanskrit{Māra} thought, “The ascetic Gotama is teaching Dhamma, surrounded by a large assembly. Why don’t I go and pull the wool over their eyes?” 

Then\marginnote{2.4} \textsanskrit{Māra} the Wicked went up to the Buddha and addressed him in verse: 

\begin{verse}%
“Why\marginnote{3.1} now do you roar like a lion? \\
You’re so self-assured in the assembly! \\
For there is someone who’ll wrestle with you, \\
so why do you imagine you’re the victor?” 

“The\marginnote{4.1} great heroes they roar, \\
self-assured in the assemblies. \\
The Realized One, attained to power, \\
has crossed over clinging to the world.” 

%
\end{verse}

Then\marginnote{5.1} \textsanskrit{Māra} the Wicked, thinking, “The Buddha knows me! The Holy One knows me!” miserable and sad, vanished right there. 

%
\section*{{\suttatitleacronym SN 4.13}{\suttatitletranslation A Splinter }{\suttatitleroot Sakalikasutta}}
\addcontentsline{toc}{section}{\tocacronym{SN 4.13} \toctranslation{A Splinter } \tocroot{Sakalikasutta}}
\markboth{A Splinter }{Sakalikasutta}
\extramarks{SN 4.13}{SN 4.13}

\scevam{So\marginnote{1.1} I have heard. }At one time the Buddha was staying near \textsanskrit{Rājagaha} in the Maddakucchi deer park. 

Now\marginnote{1.3} at that time the Buddha’s foot had been cut by a splinter. The Buddha was stricken by harrowing pains; physical feelings that were painful, sharp, severe, acute, unpleasant, and disagreeable. But he endured unbothered, with mindfulness and situational awareness. And then he spread out his outer robe folded in four and laid down in the lion’s posture—on the right side, placing one foot on top of the other—mindful and aware. 

Then\marginnote{1.7} \textsanskrit{Māra} the Wicked went up to the Buddha and addressed him in verse: 

\begin{verse}%
“Are\marginnote{2.1} you feeble that you lie down? Or are you drunk on poetry? \\
Don’t you have all that you need? \\
Alone in a secluded lodging, \\
why this sleeping, sleepyhead?” 

“I’m\marginnote{3.1} not feeble that I lie down, nor am I drunk on poetry. \\
Having reached the goal, I’m rid of sorrow. \\
Alone in a secluded lodging, \\
I lie down full of compassion for all living creatures. 

Even\marginnote{4.1} those with a dart stuck in the breast, \\
piercing the heart again and again, \\
are able to get some sleep. \\
So why not I, whose dart is drawn out? 

I\marginnote{5.1} don’t lie awake tense, nor do I fear to sleep. \\
The days and nights don’t disturb me, \\
as I see no decline for myself in the world. \\
That’s why I lie down full of compassion for all living creatures.” 

%
\end{verse}

Then\marginnote{6.1} \textsanskrit{Māra} the Wicked, thinking, “The Buddha knows me! The Holy One knows me!” miserable and sad, vanished right there. 

%
\section*{{\suttatitleacronym SN 4.14}{\suttatitletranslation Appropriate }{\suttatitleroot Patirūpasutta}}
\addcontentsline{toc}{section}{\tocacronym{SN 4.14} \toctranslation{Appropriate } \tocroot{Patirūpasutta}}
\markboth{Appropriate }{Patirūpasutta}
\extramarks{SN 4.14}{SN 4.14}

At\marginnote{1.1} one time the Buddha was staying in the land of the Kosalans near the brahmin village of \textsanskrit{Ekasālā}. 

Now,\marginnote{1.2} at that time the Buddha was teaching Dhamma, surrounded by a large assembly of laypeople. 

Then\marginnote{2.1} \textsanskrit{Māra} thought, “The ascetic Gotama is teaching Dhamma, surrounded by a large assembly of laypeople. Why don’t I go and pull the wool over their eyes?” 

Then\marginnote{2.4} \textsanskrit{Māra} the Wicked went up to the Buddha and addressed him in verse: 

\begin{verse}%
“It’s\marginnote{3.1} not appropriate for you \\
to instruct others. \\
As you engage in this, \\
don’t get caught up in favoring and opposing.” 

“The\marginnote{4.1} Buddha instructs others \\
out of compassion for their welfare. \\
The Realized One is liberated \\
from favoring and opposing.” 

%
\end{verse}

Then\marginnote{5.1} \textsanskrit{Māra} the Wicked, thinking, “The Buddha knows me! The Holy One knows me!” miserable and sad, vanished right there. 

%
\section*{{\suttatitleacronym SN 4.15}{\suttatitletranslation A Mental Snare }{\suttatitleroot Mānasasutta}}
\addcontentsline{toc}{section}{\tocacronym{SN 4.15} \toctranslation{A Mental Snare } \tocroot{Mānasasutta}}
\markboth{A Mental Snare }{Mānasasutta}
\extramarks{SN 4.15}{SN 4.15}

\scevam{So\marginnote{1.1} I have heard. }At one time the Buddha was staying near \textsanskrit{Sāvatthī} in Jeta’s Grove, \textsanskrit{Anāthapiṇḍika}’s monastery. 

Then\marginnote{1.3} \textsanskrit{Māra} the Wicked went up to the Buddha and addressed him in verse: 

\begin{verse}%
“There’s\marginnote{2.1} a mental snare \\
wandering the sky. \\
I’ll bind you with it—\\
you won’t escape me, ascetic!” 

“Sights,\marginnote{3.1} sounds, tastes, smells, \\
and touches so delightful: \\
desire for these is gone from me. \\
You’re beaten, terminator!” 

%
\end{verse}

Then\marginnote{4.1} \textsanskrit{Māra} the Wicked, thinking, “The Buddha knows me! The Holy One knows me!” miserable and sad, vanished right there. 

%
\section*{{\suttatitleacronym SN 4.16}{\suttatitletranslation The Alms Bowls }{\suttatitleroot Pattasutta}}
\addcontentsline{toc}{section}{\tocacronym{SN 4.16} \toctranslation{The Alms Bowls } \tocroot{Pattasutta}}
\markboth{The Alms Bowls }{Pattasutta}
\extramarks{SN 4.16}{SN 4.16}

At\marginnote{1.1} \textsanskrit{Sāvatthī}. 

Now\marginnote{1.2} at that time the Buddha was educating, encouraging, firing up, and inspiring the mendicants with a Dhamma talk on the topic of the five grasping aggregates. And those mendicants were paying heed, paying attention, engaging wholeheartedly, and lending an ear. 

Then\marginnote{2.1} \textsanskrit{Māra} thought, “This ascetic Gotama is educating, encouraging, firing up, and inspiring the mendicants with a Dhamma talk on the topic of the five grasping aggregates. And the mendicants are paying heed, paying attention, engaging wholeheartedly, and lending an ear. Why don’t I go and pull the wool over their eyes?” 

At\marginnote{3.1} that time several alms bowls were placed in the open air. Then \textsanskrit{Māra} the Wicked manifested in the form of an ox and approached those bowls. 

One\marginnote{3.3} of the mendicants said to another, “Mendicant, mendicant, that ox will break the bowls.” 

When\marginnote{3.5} this was said, the Buddha said to that mendicant, “Mendicant, that’s no ox. That’s \textsanskrit{Māra} the Wicked come to pull the wool over your eyes!” 

Then\marginnote{3.8} the Buddha, knowing that this was \textsanskrit{Māra} the Wicked, addressed him in verse: 

\begin{verse}%
“Sights,\marginnote{4.1} feeling, and perception, \\
consciousness and what is chosen: \\
‘I am not this’ and ‘this is not mine’; \\
that’s how to be free of desire for them. 

When\marginnote{5.1} you’re detached, secure, \\
all fetters transcended, \\
though \textsanskrit{Māra} and his army chase everywhere \\
they never find you.” 

%
\end{verse}

Then\marginnote{6.1} \textsanskrit{Māra} … vanished right there. 

%
\section*{{\suttatitleacronym SN 4.17}{\suttatitletranslation The Six Fields of Contact }{\suttatitleroot Chaphassāyatanasutta}}
\addcontentsline{toc}{section}{\tocacronym{SN 4.17} \toctranslation{The Six Fields of Contact } \tocroot{Chaphassāyatanasutta}}
\markboth{The Six Fields of Contact }{Chaphassāyatanasutta}
\extramarks{SN 4.17}{SN 4.17}

At\marginnote{1.1} one time the Buddha was staying near \textsanskrit{Vesālī}, at the Great Wood, in the hall with the peaked roof. 

Now\marginnote{1.2} at that time the Buddha was educating, encouraging, firing up, and inspiring the mendicants with a Dhamma talk on the topic of the six fields of contact. And those mendicants were paying heed, paying attention, engaging wholeheartedly, and lending an ear. 

Then\marginnote{2.1} \textsanskrit{Māra} thought, “This ascetic Gotama is educating, encouraging, firing up, and inspiring the mendicants with a Dhamma talk on the topic of the six fields of contact. And those mendicants are paying heed, paying attention, engaging wholeheartedly, and lending an ear. Why don’t I go and pull the wool over their eyes?” 

Then\marginnote{2.5} \textsanskrit{Māra} the Wicked went up to the Buddha and made a terrifyingly loud noise close by him. It seemed as if the earth were shattering, so that one of the mendicants said to another, “Mendicant, mendicant, it seems like the earth is shattering!” 

When\marginnote{2.7} this was said, the Buddha said to that mendicant, “Mendicant, that’s not the earth shattering. That’s \textsanskrit{Māra} the Wicked come to pull the wool over your eyes!” 

Then\marginnote{2.10} the Buddha, knowing that this was \textsanskrit{Māra} the Wicked, addressed him in verse: 

\begin{verse}%
“Sights,\marginnote{3.1} sounds, tastes, smells, \\
touches, and thoughts, the lot of them—\\
this is the dreadful bait \\
that the world’s infatuated by. 

But\marginnote{4.1} a mindful disciple of the Buddha \\
has transcended all that. \\
Having slipped free of \textsanskrit{Māra}’s sway, \\
they shine like the sun.” 

%
\end{verse}

Then\marginnote{5.1} \textsanskrit{Māra} … vanished right there. 

%
\section*{{\suttatitleacronym SN 4.18}{\suttatitletranslation Alms Food }{\suttatitleroot Piṇḍasutta}}
\addcontentsline{toc}{section}{\tocacronym{SN 4.18} \toctranslation{Alms Food } \tocroot{Piṇḍasutta}}
\markboth{Alms Food }{Piṇḍasutta}
\extramarks{SN 4.18}{SN 4.18}

At\marginnote{1.1} one time the Buddha was staying in the land of the Magadhans near the brahmin village of \textsanskrit{Pañcasālā}. 

Now\marginnote{1.2} at that time in \textsanskrit{Pañcasālā} the young women were taking care of guests. Then the Buddha robed up in the morning and, taking his bowl and robe, entered \textsanskrit{Pañcasālā} for alms. 

Now\marginnote{1.4} at that time \textsanskrit{Māra} had possessed the brahmins and householders of \textsanskrit{Pañcasālā}, so that they thought, “Don’t let the ascetic Gotama get any alms!” 

Then\marginnote{2.1} the Buddha left the village with his bowl as clean-washed as it was when he entered for alms. 

Then\marginnote{2.2} \textsanskrit{Māra} the Wicked went up to the Buddha and said to him, “Well, ascetic, did you get any alms?” 

“Wicked\marginnote{2.4} One, did you make sure I didn’t get any alms?” 

“Well\marginnote{2.5} then, sir, let the Buddha enter \textsanskrit{Pañcasālā} a second time for alms. I’ll make sure you get alms.” 

\begin{verse}%
“\textsanskrit{Māra}’s\marginnote{3.1} made bad karma \\
in attacking the Realized One. \\
Wicked One, do you imagine that \\
your wickedness won’t bear fruit? 

Let\marginnote{4.1} us live so very happily, \\
we who have nothing. \\
We shall feed on rapture, \\
like the gods of streaming radiance.” 

%
\end{verse}

Then\marginnote{5.1} \textsanskrit{Māra} the Wicked, thinking, “The Buddha knows me! The Holy One knows me!” miserable and sad, vanished right there. 

%
\section*{{\suttatitleacronym SN 4.19}{\suttatitletranslation A Farmer }{\suttatitleroot Kassakasutta}}
\addcontentsline{toc}{section}{\tocacronym{SN 4.19} \toctranslation{A Farmer } \tocroot{Kassakasutta}}
\markboth{A Farmer }{Kassakasutta}
\extramarks{SN 4.19}{SN 4.19}

At\marginnote{1.1} \textsanskrit{Sāvatthī}. 

Now\marginnote{1.2} at that time the Buddha was educating, encouraging, firing up, and inspiring the mendicants with a Dhamma talk about extinguishment. And those mendicants were paying heed, paying attention, engaging wholeheartedly, and lending an ear. 

Then\marginnote{2.1} \textsanskrit{Māra} thought, “The ascetic Gotama is giving a Dhamma talk about extinguishment … and the mendicants are listening well. Why don’t I go and pull the wool over their eyes?” 

Then\marginnote{2.4} \textsanskrit{Māra} the Wicked manifested in the form of a farmer carrying a large plough on his shoulder. He held a long goad, his hair was messy, he was clad in sunn hemp, and his feet were muddy. He went up to the Buddha and said to him, “So, ascetic, did you happen to see any oxen?” 

“But\marginnote{2.6} what have you to do with oxen, Wicked One?” 

“Mine\marginnote{2.7} alone, ascetic, is the eye, mine are sights, mine is the field of eye contact consciousness. Where can you escape me, ascetic? Mine alone is the ear … nose … tongue … body … mind, mine are thoughts, mine is the field of mind contact consciousness. Where can you escape me, ascetic?” 

“Yours\marginnote{3.1} alone, Wicked One, is the eye, yours are sights, yours is the field of eye contact consciousness. Where there is no eye, no sights, no eye contact consciousness—you have no place there, Wicked One! Yours alone is the ear … nose … tongue … body … mind, yours are thoughts, yours is the field of mind contact consciousness. Where there is no mind, no thoughts, no mind contact consciousness—you have no place there, Wicked One!” 

\begin{verse}%
“The\marginnote{4.1} things they call ‘mine’, \\
and those who say ‘it’s mine’: \\
if your mind remains there, \\
you won’t escape me, ascetic!” 

“The\marginnote{5.1} things they speak of aren’t mine; \\
I’m not someone who speaks like that. \\
So know this, Wicked One: \\
you won’t even see the path I take.” 

%
\end{verse}

Then\marginnote{6.1} \textsanskrit{Māra} … vanished right there. 

%
\section*{{\suttatitleacronym SN 4.20}{\suttatitletranslation Ruling }{\suttatitleroot Rajjasutta}}
\addcontentsline{toc}{section}{\tocacronym{SN 4.20} \toctranslation{Ruling } \tocroot{Rajjasutta}}
\markboth{Ruling }{Rajjasutta}
\extramarks{SN 4.20}{SN 4.20}

At\marginnote{1.1} one time the Buddha was staying in the land of the Kosalans, in a wilderness hut on the slopes of the Himalayas. 

Then\marginnote{1.2} as he was in private retreat this thought came to his mind, “I wonder if it’s possible to rule legitimately, without killing or having someone kill for you; without conquering or having someone conquer for you; without sorrowing or causing sorrow?” 

And\marginnote{2.1} then \textsanskrit{Māra} the Wicked, knowing what the Buddha was thinking, went up to him and said, “Rule, Blessed One! Rule, Holy One! Rule legitimately, without killing or having someone kill for you; without conquering or having someone conquer for you; without sorrowing or causing sorrow!” 

“But\marginnote{2.3} what do you see, Wicked One, that you say this to me?” 

“The\marginnote{2.5} Blessed One, sir, has developed and cultivated the four bases for psychic power, made them a vehicle and a basis, kept them up, consolidated them, and properly implemented them. If he wished, the Blessed One need only determine that the Himalaya, king of mountains, was gold, and it would turn into gold.” 

\begin{verse}%
“Take\marginnote{3.1} a golden mountain, \\
made entirely of gold, and double it—\\
it’s still not enough for one! \\
Knowing this, live a moral life. 

When\marginnote{4.1} a person has seen where suffering comes from \\
how could they incline towards sensual pleasures? \\
Realizing that attachment is a snare in the world, \\
a person would train to remove it.” 

%
\end{verse}

Then\marginnote{5.1} \textsanskrit{Māra} the Wicked, thinking, “The Buddha knows me! The Holy One knows me!” miserable and sad, vanished right there. 

%
\addtocontents{toc}{\let\protect\contentsline\protect\nopagecontentsline}
\chapter*{Chapter Three }
\addcontentsline{toc}{chapter}{\tocchapterline{Chapter Three }}
\addtocontents{toc}{\let\protect\contentsline\protect\oldcontentsline}

%
\section*{{\suttatitleacronym SN 4.21}{\suttatitletranslation Several }{\suttatitleroot Sambahulasutta}}
\addcontentsline{toc}{section}{\tocacronym{SN 4.21} \toctranslation{Several } \tocroot{Sambahulasutta}}
\markboth{Several }{Sambahulasutta}
\extramarks{SN 4.21}{SN 4.21}

\scevam{So\marginnote{1.1} I have heard. }At one time the Buddha was staying in the land of the Sakyans near \textsanskrit{Silāvatī}. 

Now\marginnote{1.3} at that time several mendicants were meditating not far from the Buddha, diligent, keen, and resolute. 

Then\marginnote{1.4} \textsanskrit{Māra} the Wicked manifested in the form of a brahmin with a large matted dreadlock, wearing an antelope hide. He was old, bent double, wheezing, and held a staff made of cluster fig tree wood. He went up to those mendicants and said, “You’ve gone forth while young, reverends. You’re black-haired, blessed with youth, in the prime of life, and you’ve never flirted with sensual pleasures. Enjoy human sensual pleasures. Don’t give up what is visible in the present to chase after what takes effect over time.” 

“Brahmin,\marginnote{1.8} that’s not what we’re doing. We’re giving up what takes effect over time to chase after what is visible in the present. For the Buddha says that sensual pleasures take effect over time; they give much suffering and distress, and they are all the more full of drawbacks. But this teaching is visible in this very life, immediately effective, inviting inspection, relevant, so that sensible people can know it for themselves.” 

When\marginnote{1.12} they had spoken, \textsanskrit{Māra} the Wicked shook his head, waggled his tongue, raised his eyebrows until his brow puckered in three furrows, and departed leaning on his staff. 

Then\marginnote{2.1} those mendicants went up to the Buddha, bowed, sat down to one side, and told him what had happened. The Buddha said, “Mendicants, that was no brahmin. That was \textsanskrit{Māra} the Wicked who came to pull the wool over your eyes!” 

Then,\marginnote{3.3} understanding this matter, on that occasion the Buddha recited this verse: 

\begin{verse}%
“When\marginnote{4.1} a person has seen where suffering comes from \\
how could they incline towards sensual pleasures? \\
Realizing that attachment is a snare in the world, \\
a person would train to remove it.” 

%
\end{verse}

%
\section*{{\suttatitleacronym SN 4.22}{\suttatitletranslation With Samiddhi }{\suttatitleroot Samiddhisutta}}
\addcontentsline{toc}{section}{\tocacronym{SN 4.22} \toctranslation{With Samiddhi } \tocroot{Samiddhisutta}}
\markboth{With Samiddhi }{Samiddhisutta}
\extramarks{SN 4.22}{SN 4.22}

At\marginnote{1.1} one time the Buddha was staying in the land of the Sakyans near \textsanskrit{Silāvatī}. 

Now\marginnote{1.2} at that time Venerable Samiddhi was meditating not far from the Buddha, diligent, keen, and resolute. Then as Venerable Samiddhi was in private retreat this thought came to his mind, “I’m so fortunate, so very fortunate, to have a teacher who is a perfected one, a fully awakened Buddha! I’m so fortunate, so very fortunate, to have gone forth in a teaching and training so well explained! I’m so fortunate, so very fortunate, to have spiritual companions who are ethical and of good character.” 

And\marginnote{1.10} then \textsanskrit{Māra} the Wicked, knowing what Samiddhi was thinking, went up to him and made a terrifyingly loud noise close by him. It seemed as if the earth was shattering. 

Then\marginnote{2.1} Samiddhi went up to the Buddha, bowed, sat down to one side, and told him what had happened. The Buddha said, “Samiddhi, that’s not the earth shattering. That’s \textsanskrit{Māra} the Wicked come to pull the wool over your eyes! Go back to that same place, Samiddhi, and meditate, diligent, keen, and resolute.” 

“Yes,\marginnote{3.4} sir,” replied Samiddhi. He got up from his seat, bowed, and respectfully circled the Buddha, keeping him on his right, before leaving. 

And\marginnote{3.5} for a second time Samiddhi was meditating in that same place, diligent, ardent, and resolute. And for a second time he had the same thought … and \textsanskrit{Māra} made an earth-shattering noise. 

Then\marginnote{3.9} Samiddhi addressed \textsanskrit{Māra} the Wicked One in verse: 

\begin{verse}%
“I\marginnote{4.1} went forth out of faith \\
from the lay life to homelessness. \\
My mindfulness and wisdom are mature, \\
my mind is serene in immersion. \\
Make whatever illusions you want, \\
it won’t bother me.” 

%
\end{verse}

Then\marginnote{5.1} \textsanskrit{Māra} the Wicked, thinking, “The mendicant Samiddhi knows me!” miserable and sad, vanished right there. 

%
\section*{{\suttatitleacronym SN 4.23}{\suttatitletranslation With Godhika }{\suttatitleroot Godhikasutta}}
\addcontentsline{toc}{section}{\tocacronym{SN 4.23} \toctranslation{With Godhika } \tocroot{Godhikasutta}}
\markboth{With Godhika }{Godhikasutta}
\extramarks{SN 4.23}{SN 4.23}

\scevam{So\marginnote{1.1} I have heard. }At one time the Buddha was staying near \textsanskrit{Rājagaha}, in the Bamboo Grove, the squirrels’ feeding ground. 

Now\marginnote{1.3} at that time Venerable Godhika was staying on the slopes of Isigili at the Black Rock. Then Venerable Godhika, meditating diligent, keen, and resolute, experienced temporary freedom of heart. But then he fell away from that temporary freedom of heart. For a second … third … fourth … fifth … sixth time Godhika experienced temporary freedom of heart. But for a sixth time he fell away from it. For a seventh time Godhika, meditating diligent, keen, and resolute, experienced temporary freedom of heart. 

Then\marginnote{2.1} he thought, “I’ve fallen away from this temporary freedom of heart no less than six times. Why don’t I slit my wrists?” 

And\marginnote{2.4} then \textsanskrit{Māra} the Wicked, knowing what Godhika was thinking, went up to the Buddha and addressed him in verse: 

\begin{verse}%
“O\marginnote{3.1} great hero, O greatly wise! \\
Shining with power and glory. \\
You’ve gone beyond all threats and perils, \\
I bow to your feet, O seer! 

Great\marginnote{4.1} hero, master of death, \\
your disciple longs for death, \\
he’s planning for it. \\
Stop him, O light-bringer! 

For\marginnote{5.1} how, Blessed One, can a disciple of yours, \\
one who loves your teaching, \\
a trainee who hasn’t achieved their heart’s desire, \\
take his own life, O renowned one?” 

%
\end{verse}

Now\marginnote{6.1} at that time Venerable Godhika had already slit his wrists. 

Then\marginnote{6.2} the Buddha, knowing that this was \textsanskrit{Māra} the Wicked, addressed him in verse: 

\begin{verse}%
“This\marginnote{7.1} is how the wise act, \\
for they don’t long for life. \\
Having plucked out craving, root and all, \\
Godhika is extinguished.” 

%
\end{verse}

Then\marginnote{8.1} the Buddha said to the mendicants, “Come, mendicants, let’s go to the Black Rock on the slopes of Isigili where Godhika, who came from a good family, slit his wrists.” 

“Yes,\marginnote{8.3} sir,” they replied. 

Then\marginnote{9.1} the Buddha together with several mendicants went to the Black Rock on the slopes of Isigili. The Buddha saw Godhika off in the distance lying on his cot, having cast off the aggregates. 

Now\marginnote{9.3} at that time a cloud of black smoke was moving east, west, north, south, above, below, and in-between. 

Then\marginnote{10.1} the Buddha said to the mendicants, 

“Mendicants,\marginnote{10.2} do you see that cloud of black smoke moving east, west, north, south, above, below, and in-between?” 

“Yes,\marginnote{10.3} sir.” 

“That’s\marginnote{10.4} \textsanskrit{Māra} the Wicked searching for Godhika’s consciousness, wondering: ‘Where is Godhika’s consciousness established?’ But since his consciousness is not established, Godhika is extinguished.” 

Then\marginnote{10.7} \textsanskrit{Māra}, carrying his arched harp made from the pale timber of wood apple, went up to the Buddha and addressed him in verse: 

\begin{verse}%
“Above,\marginnote{11.1} below, all round, \\
in the four quarters and in-between, \\
I’ve been searching without success: \\
where has that Godhika got to?” 

“He\marginnote{12.1} was a wise and steadfast sage, \\
a meditator who loved absorption. \\
By day and by night he applied himself, \\
without concern for his life. 

He\marginnote{13.1} defeated the army of death, \\
and won’t return for any future life. \\
Having plucked out craving, root and all \\
Godhika is extinguished.” 

So\marginnote{14.1} stricken with sorrow \\
that his harp dropped from his armpit, \\
that spirit, downcast, \\
vanished right there. 

%
\end{verse}

%
\section*{{\suttatitleacronym SN 4.24}{\suttatitletranslation Seven Years of Following }{\suttatitleroot Sattavassānubandhasutta}}
\addcontentsline{toc}{section}{\tocacronym{SN 4.24} \toctranslation{Seven Years of Following } \tocroot{Sattavassānubandhasutta}}
\markboth{Seven Years of Following }{Sattavassānubandhasutta}
\extramarks{SN 4.24}{SN 4.24}

\scevam{So\marginnote{1.1} I have heard. }At one time the Buddha was staying near \textsanskrit{Uruvelā} at the goatherd’s banyan tree on the bank of the \textsanskrit{Nerañjarā} River. 

Now\marginnote{1.3} at that time \textsanskrit{Māra} the Wicked had been following the Buddha for seven years hoping to find a vulnerability without success. 

Then\marginnote{1.4} \textsanskrit{Māra} the Wicked went up to the Buddha and addressed him in verse: 

\begin{verse}%
“Are\marginnote{2.1} you swamped by sorrow that you meditate in the forest? \\
Have you lost a fortune, or do you long for one? \\
Or perhaps you’ve committed some crime in the village? \\
Why don’t you get too close to people? \\
And why does no-one get close to you?” 

“I’ve\marginnote{3.1} dug out the root of sorrow completely. \\
I practice absorption free of guilt or sorrow. \\
I’ve cut off all greed and prayer for future lives. \\
Undefiled, I practice absorption, O kinsman of the negligent!” 

“The\marginnote{4.1} things they call ‘mine’, \\
and those who say ‘it’s mine’: \\
if your mind remains there, \\
you won’t escape me, ascetic!” 

“The\marginnote{5.1} things they speak of aren’t mine; \\
I’m not someone who speaks like that. \\
So know this, Wicked One: \\
you won’t even see the path I take.” 

“If\marginnote{6.1} you’ve discovered the path \\
that’s safe, and leads to the deathless, \\
go and walk that path alone—\\
why teach it to anyone else?” 

“Those\marginnote{7.1} crossing to the far shore \\
ask what’s beyond the domain of Death. \\
When I’m asked, I explain to them \\
the truth without attachments.” 

%
\end{verse}

“Sir,\marginnote{8.1} suppose there was a lotus pond not far from a town or village, and a crab lived there. Then several boys or girls would leave the town or village and go to the pond, where they’d pull out the crab and put it on dry land. Whenever that crab extended a claw, those boys or girls would snap, crack, and break it off with a stick or a stone. And when that crab’s claws had all been snapped, cracked, and broken off it wouldn’t be able to return down into that lotus pond. 

In\marginnote{8.6} the same way, sir, the Buddha has snapped, cracked, and broken off all my tricks, dodges, and evasions. Now I’m not able to approach the Buddha again in hopes of finding a vulnerability.” 

Then\marginnote{8.8} \textsanskrit{Māra} the Wicked recited these verses of disappointment in the Buddha’s presence: 

\begin{verse}%
“A\marginnote{9.1} crow once circled a stone \\
that looked like a lump of fat. \\
‘Perhaps I’ll find something tender,’ it thought, \\
‘perhaps there’s something tasty.’ 

But\marginnote{10.1} finding nothing tasty, \\
the crow left that place. \\
Like the crow that pecked the stone, \\
I leave Gotama disappointed.” 

%
\end{verse}

%
\section*{{\suttatitleacronym SN 4.25}{\suttatitletranslation Māra’s Daughters }{\suttatitleroot Māradhītusutta}}
\addcontentsline{toc}{section}{\tocacronym{SN 4.25} \toctranslation{Māra’s Daughters } \tocroot{Māradhītusutta}}
\markboth{Māra’s Daughters }{Māradhītusutta}
\extramarks{SN 4.25}{SN 4.25}

And\marginnote{1.1} then \textsanskrit{Māra} the Wicked, after reciting these verses of disillusionment in the Buddha’s presence, left that place. He sat cross-legged on the ground not far from the Buddha, silent, embarrassed, shoulders drooping, downcast, depressed, with nothing to say, scratching the ground with a stick. 

Then\marginnote{1.2} \textsanskrit{Māra}’s daughters Craving, Delight, and Lust went up to \textsanskrit{Māra} the Wicked, and addressed him in verse: 

\begin{verse}%
“Why\marginnote{2.1} so downhearted, dad? \\
What man are you upset about? \\
We’ll catch him with the snare of lust, \\
like an elephant in the wild. \\
We’ll tie him up and bring him back—\\
he’ll fall under your sway!” 

“In\marginnote{3.1} this world he is the perfected one, the Holy One. \\
He’s not easily seduced by lust. \\
He has slipped free of \textsanskrit{Māra}’s sway; \\
that’s why I’m so upset.” 

%
\end{verse}

Then\marginnote{4.1} \textsanskrit{Māra}’s daughters Craving, Delight, and Lust went up to the Buddha, and said to him, “We serve at your feet, ascetic.” But the Buddha ignored them, since he was freed with the supreme ending of attachments. 

Then\marginnote{5.1} Craving, Delight, and Lust withdrew to one side to think up a plan. “Men have a diverse spectrum of tastes. Why don’t we each manifest in the form of a hundred young maidens?” 

So\marginnote{5.4} that’s what they did. Then they went up to the Buddha and said to him, “We serve at your feet, ascetic.” But the Buddha still ignored them, since he was freed with the supreme ending of attachments. 

Then\marginnote{6.1} Craving, Delight, and Lust withdrew to one side to think up a plan. “Men have a diverse spectrum of tastes. Why don’t we each manifest in the form of a hundred women who have never given birth?” So that’s what they did. Then they went up to the Buddha and said to him, “We serve at your feet, ascetic.” But the Buddha still ignored them, since he was freed with the supreme ending of attachments. 

Then\marginnote{7.1} Craving, Delight, and Lust … each manifested in the form of a hundred women who have given birth once … women who have given birth twice … middle-aged women … old women … But the Buddha still ignored them, since he was freed with the supreme ending of attachments. 

Then\marginnote{9.6} Craving, Delight, and Lust withdrew to one side and said, “What our father said is true: 

\begin{verse}%
‘In\marginnote{10.1} this world he is the perfected one, the Holy One. \\
He’s not easily seduced by lust. \\
He has slipped free of \textsanskrit{Māra}’s sway; \\
that’s why I’m so upset.’ 

%
\end{verse}

For\marginnote{11.1} if we had come on to any ascetic or brahmin like this who was not free of lust, his heart would explode, or he’d spew hot blood from his mouth, or he’d go mad and lose his mind. He’d dry up, wither away, and shrivel up like a green reed that was mowed down.” 

Then\marginnote{12.1} \textsanskrit{Māra}’s daughters Craving, Delight, and Lust went up to the Buddha, and stood to one side. \textsanskrit{Māra}’s daughter Craving addressed the Buddha in verse: 

\begin{verse}%
“Are\marginnote{13.1} you swamped by sorrow that you meditate in the forest? \\
Have you lost a fortune, or do you long for one? \\
Or perhaps you’ve committed some crime in the village? \\
Why don’t you get too close to people? \\
And why does no-one get close to you?” 

“I’ve\marginnote{14.1} reached the goal, peace of heart. \\
Having conquered the army of the likable and pleasant, \\
alone, practicing absorption, I awakened to bliss. \\
That’s why I don’t get too close to people, \\
and no-one gets too close to me.” 

%
\end{verse}

Then\marginnote{15.1} \textsanskrit{Māra}’s daughter Delight addressed the Buddha in verse: 

\begin{verse}%
“How\marginnote{16.1} does a mendicant who has crossed five floods \\
usually meditate here while crossing the sixth? \\
How do they usually practice absorption so that sensual perceptions \\
are kept out and don’t get hold of them?” 

“With\marginnote{17.1} tranquil body and mind well freed, \\
without making plans, mindful, homeless; \\
understanding the teaching, they practice absorption without placing the mind; \\
they’re not shaking or drifting or rigid. 

That’s\marginnote{18.1} how a mendicant who has crossed five floods \\
usually meditates here while crossing the sixth. \\
That’s how they usually practice absorption so that sensual perceptions \\
are kept out and don’t get hold of them.” 

%
\end{verse}

Then\marginnote{19.1} \textsanskrit{Māra}’s daughter Lust addressed the Buddha in verse: 

\begin{verse}%
“He\marginnote{20.1} lives with his community after cutting off craving, \\
and many of the faithful will cross over for sure. \\
Alas, this homeless one will snatch many men away, \\
and lead them past the King of Death!” 

“The\marginnote{21.1} great heroes they lead \\
by means of the true teaching. \\
When the Realized Ones are leading by the teaching, \\
how could anyone who knows be jealous?” 

%
\end{verse}

Then\marginnote{22.1} \textsanskrit{Māra}’s daughters Craving, Delight, and Lust went up to \textsanskrit{Māra} the Wicked. \textsanskrit{Māra} the Wicked saw them coming off in the distance, and addressed them in verse: 

\begin{verse}%
“Fools!\marginnote{23.1} You drill into a mountain \\
with lotus stalks! \\
You dig up a hill with your nails! \\
You chew iron with your teeth! 

You\marginnote{24.1} seek a footing in the deeps, as it were, \\
while lifting a rock with your head! \\
After attacking a stump with your breast, as it were, \\
you leave Gotama disappointed.” 

“They\marginnote{25.1} came in their splendor—\\
Craving, Delight, and Lust. \\
But the Teacher brushed them off right there, \\
like the breeze, a fallen tuft.” 

%
\end{verse}

\scendsutta{The Linked Discourses with \textsanskrit{Māra} are complete. }

%
\addtocontents{toc}{\let\protect\contentsline\protect\nopagecontentsline}
\part*{Linked Discourses With Nuns }
\addcontentsline{toc}{part}{Linked Discourses With Nuns }
\markboth{}{}
\addtocontents{toc}{\let\protect\contentsline\protect\oldcontentsline}

%
\addtocontents{toc}{\let\protect\contentsline\protect\nopagecontentsline}
\chapter*{The Chapter on Nuns }
\addcontentsline{toc}{chapter}{\tocchapterline{The Chapter on Nuns }}
\addtocontents{toc}{\let\protect\contentsline\protect\oldcontentsline}

%
\section*{{\suttatitleacronym SN 5.1}{\suttatitletranslation With Āḷavikā }{\suttatitleroot Āḷavikāsutta}}
\addcontentsline{toc}{section}{\tocacronym{SN 5.1} \toctranslation{With Āḷavikā } \tocroot{Āḷavikāsutta}}
\markboth{With Āḷavikā }{Āḷavikāsutta}
\extramarks{SN 5.1}{SN 5.1}

\scevam{So\marginnote{1.1} I have heard. }At one time the Buddha was staying near \textsanskrit{Sāvatthī} in Jeta’s Grove, \textsanskrit{Anāthapiṇḍika}’s monastery. 

Then\marginnote{1.3} the nun \textsanskrit{Āḷavikā} robed up in the morning and, taking her bowl and robe, entered \textsanskrit{Sāvatthī} for alms. She wandered for alms in \textsanskrit{Sāvatthī}. After the meal, on her return from almsround, she went to the Dark Forest seeking seclusion. 

Then\marginnote{1.5} \textsanskrit{Māra} the Wicked, wanting to make the nun \textsanskrit{Āḷavikā} feel fear, terror, and goosebumps, wanting to make her fall away from seclusion, went up to her and addressed her in verse: 

\begin{verse}%
“There’s\marginnote{2.1} no escape in the world, \\
so what will seclusion do for you? \\
Enjoy the delights of sensual pleasure; \\
don’t regret it later.” 

%
\end{verse}

Then\marginnote{3.1} the nun \textsanskrit{Āḷavikā} thought, “Who’s speaking this verse, a human or a non-human?” 

Then\marginnote{3.3} she thought, “This is \textsanskrit{Māra} the Wicked, wanting to make me feel fear, terror, and goosebumps, wanting to make me fall away from seclusion!” 

Then\marginnote{3.5} \textsanskrit{Āḷavikā}, knowing that this was \textsanskrit{Māra} the Wicked, replied to him in verse: 

\begin{verse}%
“There\marginnote{4.1} is an escape in the world, \\
and I’ve personally experienced it with wisdom. \\
O Wicked One, kinsman of the negligent, \\
you don’t know that place. 

Sensual\marginnote{5.1} pleasures are like swords and stakes; \\
the aggregates are their chopping block. \\
What you call sensual delight \\
has become no delight for me.” 

%
\end{verse}

Then\marginnote{6.1} \textsanskrit{Māra} the Wicked, thinking, “The nun \textsanskrit{Āḷavikā} knows me!” miserable and sad, vanished right there. 

%
\section*{{\suttatitleacronym SN 5.2}{\suttatitletranslation With Somā }{\suttatitleroot Somāsutta}}
\addcontentsline{toc}{section}{\tocacronym{SN 5.2} \toctranslation{With Somā } \tocroot{Somāsutta}}
\markboth{With Somā }{Somāsutta}
\extramarks{SN 5.2}{SN 5.2}

At\marginnote{1.1} \textsanskrit{Sāvatthī}. 

Then\marginnote{1.2} the nun \textsanskrit{Somā} robed up in the morning and, taking her bowl and robe, entered \textsanskrit{Sāvatthī} for alms. She wandered for alms in \textsanskrit{Sāvatthī}. After the meal, on her return from almsround, she went to the Dark Forest, plunged deep into it, and sat at the root of a tree for the day’s meditation. 

Then\marginnote{1.5} \textsanskrit{Māra} the Wicked, wanting to make the nun \textsanskrit{Somā} feel fear, terror, and goosebumps, wanting to make her fall away from immersion, went up to her and addressed her in verse: 

\begin{verse}%
“That\marginnote{2.1} state’s very challenging; \\
it’s for the sages to attain. \\
It’s not possible for a woman, \\
with her two-fingered wisdom.” 

%
\end{verse}

Then\marginnote{3.1} the nun \textsanskrit{Somā} thought, “Who’s speaking this verse, a human or a non-human?” 

Then\marginnote{3.3} she thought, “This is \textsanskrit{Māra} the Wicked, wanting to make me feel fear, terror, and goosebumps, wanting to make me fall away from immersion!” 

Then\marginnote{3.5} \textsanskrit{Somā}, knowing that this was \textsanskrit{Māra} the Wicked, replied to him in verse: 

\begin{verse}%
“What\marginnote{4.1} difference does womanhood make \\
when the mind is serene, \\
and knowledge is present \\
as you rightly discern the Dhamma. 

Surely\marginnote{5.1} someone who might think: \\
‘I am woman’, or ‘I am man’, \\
or ‘I am’ anything at all, \\
is fit for \textsanskrit{Māra} to address.” 

%
\end{verse}

Then\marginnote{6.1} \textsanskrit{Māra} the Wicked, thinking, “The nun \textsanskrit{Somā} knows me!” miserable and sad, vanished right there. 

%
\section*{{\suttatitleacronym SN 5.3}{\suttatitletranslation With Kisāgotamī }{\suttatitleroot Kisāgotamīsutta}}
\addcontentsline{toc}{section}{\tocacronym{SN 5.3} \toctranslation{With Kisāgotamī } \tocroot{Kisāgotamīsutta}}
\markboth{With Kisāgotamī }{Kisāgotamīsutta}
\extramarks{SN 5.3}{SN 5.3}

At\marginnote{1.1} \textsanskrit{Sāvatthī}. 

Then\marginnote{1.2} the nun \textsanskrit{Kisāgotamī} robed up in the morning and, taking her bowl and robe, entered \textsanskrit{Sāvatthī} for alms. She wandered for alms in \textsanskrit{Sāvatthī}. After the meal, on her return from almsround, she went to the Dark Forest, plunged deep into it, and sat at the root of a tree for the day’s meditation. 

Then\marginnote{1.5} \textsanskrit{Māra} the Wicked, wanting to make the nun \textsanskrit{Kisāgotamī} feel fear, terror, and goosebumps, wanting to make her fall away from immersion, went up to her and addressed her in verse: 

\begin{verse}%
“Why\marginnote{2.1} do you sit alone and cry \\
as if your children have died? \\
You’ve come to the woods all alone—\\
you must be looking for a man!” 

%
\end{verse}

Then\marginnote{3.1} the nun \textsanskrit{Kisāgotamī} thought, “Who’s speaking this verse, a human or a non-human?” 

Then\marginnote{3.3} she thought, “This is \textsanskrit{Māra} the Wicked, wanting to make me feel fear, terror, and goosebumps, wanting to make me fall away from immersion!” 

Then\marginnote{4.1} \textsanskrit{Kisāgotamī}, knowing that this was \textsanskrit{Māra} the Wicked, replied to him in verse: 

\begin{verse}%
“I’ve\marginnote{5.1} got over the death of children, \\
and I’m finished with men. \\
I don’t grieve or lament, \\
and I’m not afraid of you, sir! 

Relishing\marginnote{6.1} is destroyed in every respect, \\
and the mass of darkness is shattered. \\
I’ve defeated the army of death, \\
and live without defilements.” 

%
\end{verse}

Then\marginnote{7.1} \textsanskrit{Māra} the Wicked, thinking, “The nun \textsanskrit{Kisāgotamī} knows me!” miserable and sad, vanished right there. 

%
\section*{{\suttatitleacronym SN 5.4}{\suttatitletranslation With Vijayā }{\suttatitleroot Vijayāsutta}}
\addcontentsline{toc}{section}{\tocacronym{SN 5.4} \toctranslation{With Vijayā } \tocroot{Vijayāsutta}}
\markboth{With Vijayā }{Vijayāsutta}
\extramarks{SN 5.4}{SN 5.4}

At\marginnote{1.1} \textsanskrit{Sāvatthī}. 

Then\marginnote{1.2} the nun \textsanskrit{Vijayā} robed up in the morning … and sat at the root of a tree for the day’s meditation. 

Then\marginnote{1.4} \textsanskrit{Māra} the Wicked, wanting to make the nun \textsanskrit{Vijayā} feel fear, terror, and goosebumps, wanting to make her fall away from immersion, went up to her and addressed her in verse: 

\begin{verse}%
“You’re\marginnote{2.1} so young and beautiful, \\
and I’m a youth in my prime. \\
Come, my lady, let us enjoy \\
the music of a five-piece band.” 

%
\end{verse}

Then\marginnote{3.1} the nun \textsanskrit{Vijayā} thought, “Who’s speaking this verse, a human or a non-human?” 

Then\marginnote{3.3} she thought, “This is \textsanskrit{Māra} the Wicked, wanting to make me feel fear, terror, and goosebumps, wanting to make me fall away from immersion!” 

Then\marginnote{3.5} \textsanskrit{Vijayā}, knowing that this was \textsanskrit{Māra} the Wicked, replied to him in verse: 

\begin{verse}%
“Sights,\marginnote{4.1} sounds, tastes, smells, \\
and touches so delightful. \\
I hand them right back to you, \textsanskrit{Māra}, \\
for I have no use for them. 

This\marginnote{5.1} body is foul, \\
decaying and frail. \\
I’m horrified and repelled by it, \\
and I’ve eradicated sensual craving. 

There\marginnote{6.1} are beings in the realm of luminous form, \\
others established in the formless, \\
and also those peaceful attainments: \\
I’ve destroyed the darkness regarding all of them.” 

%
\end{verse}

Then\marginnote{7.1} \textsanskrit{Māra} the Wicked, thinking, “The nun \textsanskrit{Vijayā} knows me!” miserable and sad, vanished right there. 

%
\section*{{\suttatitleacronym SN 5.5}{\suttatitletranslation With Uppalavaṇṇā }{\suttatitleroot Uppalavaṇṇāsutta}}
\addcontentsline{toc}{section}{\tocacronym{SN 5.5} \toctranslation{With Uppalavaṇṇā } \tocroot{Uppalavaṇṇāsutta}}
\markboth{With Uppalavaṇṇā }{Uppalavaṇṇāsutta}
\extramarks{SN 5.5}{SN 5.5}

At\marginnote{1.1} \textsanskrit{Sāvatthī}. 

Then\marginnote{1.2} the nun \textsanskrit{Uppalavaṇṇā} robed up in the morning … and stood at the root of a sal tree in full flower. 

Then\marginnote{1.4} \textsanskrit{Māra} the Wicked, wanting to make the nun \textsanskrit{Uppalavaṇṇā} feel fear, terror, and goosebumps, wanting to make her fall away from immersion, went up to her and addressed her in verse: 

\begin{verse}%
“You’ve\marginnote{2.1} come to this sal tree all crowned with flowers, \\
and stand at its root all alone, O nun. \\
Your beauty is second to none; \\
silly girl, aren’t you afraid of rascals?” 

%
\end{verse}

Then\marginnote{3.1} the nun \textsanskrit{Uppalavaṇṇā} thought, “Who’s speaking this verse, a human or a non-human?” 

Then\marginnote{3.3} she thought, “This is \textsanskrit{Māra} the Wicked, wanting to make me feel fear, terror, and goosebumps, wanting to make me fall away from immersion!” 

Then\marginnote{3.5} \textsanskrit{Uppalavaṇṇā}, knowing that this was \textsanskrit{Māra} the Wicked, replied to him in verse: 

\begin{verse}%
“Even\marginnote{4.1} if 100,000 rascals like you \\
were to come here, \\
I’d stir not a hair nor panic. \\
I’m not scared of you, \textsanskrit{Māra}, even alone. 

I’ll\marginnote{5.1} vanish, \\
or I’ll enter your belly; \\
I could stand between your eyebrows \\
and you still wouldn’t see me. 

I’m\marginnote{6.1} the master of my own mind, \\
I’ve developed the bases of psychic power well. \\
I’m free from all bonds, \\
and I’m not afraid of you, sir!” 

%
\end{verse}

Then\marginnote{7.1} \textsanskrit{Māra} the Wicked, thinking, “The nun \textsanskrit{Uppalavaṇṇā} knows me!” miserable and sad, vanished right there. 

%
\section*{{\suttatitleacronym SN 5.6}{\suttatitletranslation With Cālā }{\suttatitleroot Cālāsutta}}
\addcontentsline{toc}{section}{\tocacronym{SN 5.6} \toctranslation{With Cālā } \tocroot{Cālāsutta}}
\markboth{With Cālā }{Cālāsutta}
\extramarks{SN 5.6}{SN 5.6}

At\marginnote{1.1} \textsanskrit{Sāvatthī}. 

Then\marginnote{1.2} the nun \textsanskrit{Cālā} robed up in the morning … and sat at the root of a tree for the day’s meditation. 

Then\marginnote{1.4} \textsanskrit{Māra} the Wicked went up to \textsanskrit{Cālā} and said to her, “Nun, what don’t you approve of?” 

“I\marginnote{1.6} don’t approve of rebirth, sir.” 

\begin{verse}%
“Why\marginnote{2.1} don’t you approve of rebirth? \\
When you’re born, you get to enjoy sensual pleasures. \\
Who put this idea in your head: \\
‘Nun, don’t approve of rebirth’?” 

“Death\marginnote{3.1} comes to those who are born, \\
when you’re born you undergo sufferings—\\
killing, caging, misery—\\
that’s why you shouldn’t approve of rebirth. 

The\marginnote{4.1} Buddha taught me the Dhamma \\
for passing beyond rebirth, \\
for giving up all suffering; \\
he settled me in the truth. 

There\marginnote{5.1} are beings in the realm of luminous form, \\
and others established in the formless. \\
Not understanding cessation, \\
they return in future lives.” 

%
\end{verse}

Then\marginnote{6.1} \textsanskrit{Māra} the Wicked, thinking, “The nun \textsanskrit{Cālā} knows me!” miserable and sad, vanished right there. 

%
\section*{{\suttatitleacronym SN 5.7}{\suttatitletranslation With Upacālā }{\suttatitleroot Upacālāsutta}}
\addcontentsline{toc}{section}{\tocacronym{SN 5.7} \toctranslation{With Upacālā } \tocroot{Upacālāsutta}}
\markboth{With Upacālā }{Upacālāsutta}
\extramarks{SN 5.7}{SN 5.7}

At\marginnote{1.1} \textsanskrit{Sāvatthī}. 

Then\marginnote{1.2} the nun \textsanskrit{Upacālā} robed up in the morning … and sat at the root of a tree for the day’s meditation. 

Then\marginnote{1.4} \textsanskrit{Māra} the Wicked went up to \textsanskrit{Upacālā} and said to her, “Nun, where do you want to be reborn?” 

“I\marginnote{1.6} don’t want to be reborn anywhere, sir.” 

\begin{verse}%
“There\marginnote{2.1} are the Gods of the Thirty-Three, and those of Yama; \\
also the Joyful Deities, \\
the Gods Who Love to Create, \\
and the Gods Who Control the Creations of Others. \\
Set your heart on such places, \\
and you’ll undergo delight.” 

“The\marginnote{3.1} Gods of the Thirty-Three, and those of Yama; \\
also the Joyful Deities, \\
the Gods Who Love to Create, \\
and the Gods Who Control the Creations of Others—\\
they’re bound with the bonds of sensuality; \\
they fall under your sway again. 

All\marginnote{4.1} the world is on fire, \\
all the world is smoldering, \\
all the world is ablaze, \\
all the world is rocking. 

My\marginnote{5.1} mind adores that place \\
where \textsanskrit{Māra} cannot go; \\
it’s not shaking or burning, \\
and not frequented by ordinary people.” 

%
\end{verse}

Then\marginnote{6.1} \textsanskrit{Māra} the Wicked, thinking, “The nun \textsanskrit{Upacālā} knows me!” miserable and sad, vanished right there. 

%
\section*{{\suttatitleacronym SN 5.8}{\suttatitletranslation With Sīsupacālā }{\suttatitleroot Sīsupacālāsutta}}
\addcontentsline{toc}{section}{\tocacronym{SN 5.8} \toctranslation{With Sīsupacālā } \tocroot{Sīsupacālāsutta}}
\markboth{With Sīsupacālā }{Sīsupacālāsutta}
\extramarks{SN 5.8}{SN 5.8}

At\marginnote{1.1} \textsanskrit{Sāvatthī}. 

Then\marginnote{1.2} the nun \textsanskrit{Sīsupacālā} robed up in the morning … and sat at the root of a tree for the day’s meditation. 

Then\marginnote{1.4} \textsanskrit{Māra} the Wicked went up to \textsanskrit{Sīsupacālā} and said to her, “Nun, whose creed do you believe in?” 

“I\marginnote{1.6} don’t believe in anyone’s creed, sir.” 

\begin{verse}%
“In\marginnote{2.1} whose name did you shave your head? \\
You look like an ascetic, \\
but you don’t believe in any creed. \\
Why do you live as if lost?” 

“Followers\marginnote{3.1} of other creeds \\
are confident in their views. \\
But I don’t believe in their teaching, \\
for they’re no experts in the Dhamma. 

But\marginnote{4.1} there is one born in the Sakyan clan, \\
the unrivaled Buddha, \\
champion, dispeller of \textsanskrit{Māra}, \\
everywhere undefeated, 

everywhere\marginnote{5.1} freed, and unattached, \\
the all-seeing seer. \\
He has attained the end of all deeds, \\
freed with the ending of attachments. \\
That Blessed One is my Teacher, \\
and I believe in his instruction.” 

%
\end{verse}

Then\marginnote{6.1} \textsanskrit{Māra} the Wicked, thinking, “The nun \textsanskrit{Sīsupacālā} knows me!” miserable and sad, vanished right there. 

%
\section*{{\suttatitleacronym SN 5.9}{\suttatitletranslation With Selā }{\suttatitleroot Selāsutta}}
\addcontentsline{toc}{section}{\tocacronym{SN 5.9} \toctranslation{With Selā } \tocroot{Selāsutta}}
\markboth{With Selā }{Selāsutta}
\extramarks{SN 5.9}{SN 5.9}

At\marginnote{1.1} \textsanskrit{Sāvatthī}. 

Then\marginnote{1.2} the nun \textsanskrit{Selā} robed up in the morning … and sat at the root of a tree for the day’s meditation. 

Then\marginnote{1.4} \textsanskrit{Māra} the Wicked, wanting to make the nun \textsanskrit{Selā} feel fear, terror, and goosebumps … addressed her in verse: 

\begin{verse}%
“Who\marginnote{2.1} created this puppet? \\
Where is its maker? \\
Where has the puppet arisen? \\
And where does it cease?” 

%
\end{verse}

Then\marginnote{3.1} the nun \textsanskrit{Selā} thought, “Who’s speaking this verse, a human or a non-human?” 

Then\marginnote{3.3} she thought, “This is \textsanskrit{Māra} the Wicked, wanting to make me feel fear, terror, and goosebumps, wanting to make me fall away from immersion!” 

Then\marginnote{3.5} \textsanskrit{Selā}, knowing that this was \textsanskrit{Māra} the Wicked, replied to him in verse: 

\begin{verse}%
“This\marginnote{4.1} puppet isn’t self-made, \\
nor is this misery made by another. \\
It comes to be because of a cause, \\
and ceases when the cause breaks up. 

It’s\marginnote{5.1} like a seed that’s sown \\
in a field; it grows \\
relying on both the soil’s nutrients \\
as well as moisture. 

In\marginnote{6.1} the same way the aggregates and elements \\
and these six sense fields \\
come to be because of a cause, \\
and cease when the cause breaks up.” 

%
\end{verse}

Then\marginnote{7.1} \textsanskrit{Māra} the Wicked, thinking, “The nun \textsanskrit{Selā} knows me!” miserable and sad, vanished right there. 

%
\section*{{\suttatitleacronym SN 5.10}{\suttatitletranslation With Vajirā }{\suttatitleroot Vajirāsutta}}
\addcontentsline{toc}{section}{\tocacronym{SN 5.10} \toctranslation{With Vajirā } \tocroot{Vajirāsutta}}
\markboth{With Vajirā }{Vajirāsutta}
\extramarks{SN 5.10}{SN 5.10}

At\marginnote{1.1} \textsanskrit{Sāvatthī}. 

Then\marginnote{1.2} the nun \textsanskrit{Vajirā} robed up in the morning and, taking her bowl and robe, entered \textsanskrit{Sāvatthī} for alms. She wandered for alms in \textsanskrit{Sāvatthī}. After the meal, on her return from almsround, she went to the Dark Forest, plunged deep into it, and sat at the root of a tree for the day’s meditation. 

Then\marginnote{1.5} \textsanskrit{Māra} the Wicked, wanting to make the nun \textsanskrit{Vajirā} feel fear, terror, and goosebumps, wanting to make her fall away from immersion, went up to her and addressed her in verse: 

\begin{verse}%
“Who\marginnote{2.1} created this sentient being? \\
Where is its maker? \\
Where has the being arisen? \\
And where does it cease?” 

%
\end{verse}

Then\marginnote{3.1} the nun \textsanskrit{Vajirā} thought, “Who’s speaking this verse, a human or a non-human?” 

Then\marginnote{3.3} she thought, “This is \textsanskrit{Māra} the Wicked, wanting to make me feel fear, terror, and goosebumps, wanting to make me fall away from immersion!” 

Then\marginnote{3.5} \textsanskrit{Vajirā}, knowing that this was \textsanskrit{Māra} the Wicked, replied to him in verse: 

\begin{verse}%
“Why\marginnote{4.1} do you believe there’s such a thing as a ‘sentient being’? \\
\textsanskrit{Māra}, is this your theory? \\
This is just a pile of conditions, \\
you won’t find a sentient being here. 

When\marginnote{5.1} the parts are assembled \\
we use the word ‘chariot’. \\
So too, when the aggregates are present \\
‘sentient being’ is the convention we use. 

But\marginnote{6.1} it’s only suffering that comes to be, \\
lasts a while, then disappears. \\
Naught but suffering comes to be, \\
naught but suffering ceases.” 

%
\end{verse}

Then\marginnote{7.1} \textsanskrit{Māra} the Wicked, thinking, “The nun \textsanskrit{Vajirā} knows me!” miserable and sad, vanished right there. 

\scendsutta{The Linked Discourses on Nuns are completed. }

%
\addtocontents{toc}{\let\protect\contentsline\protect\nopagecontentsline}
\part*{Linked Discourses With Brahmā Gods }
\addcontentsline{toc}{part}{Linked Discourses With Brahmā Gods }
\markboth{}{}
\addtocontents{toc}{\let\protect\contentsline\protect\oldcontentsline}

%
\addtocontents{toc}{\let\protect\contentsline\protect\nopagecontentsline}
\chapter*{Chapter One }
\addcontentsline{toc}{chapter}{\tocchapterline{Chapter One }}
\addtocontents{toc}{\let\protect\contentsline\protect\oldcontentsline}

%
\section*{{\suttatitleacronym SN 6.1}{\suttatitletranslation The Appeal of Brahmā }{\suttatitleroot Brahmāyācanasutta}}
\addcontentsline{toc}{section}{\tocacronym{SN 6.1} \toctranslation{The Appeal of Brahmā } \tocroot{Brahmāyācanasutta}}
\markboth{The Appeal of Brahmā }{Brahmāyācanasutta}
\extramarks{SN 6.1}{SN 6.1}

\scevam{So\marginnote{1.1} I have heard. }At one time, when he was first awakened, the Buddha was staying near \textsanskrit{Uruvelā} at the root of the goatherd’s banyan tree on the bank of the \textsanskrit{Nerañjarā} River. 

Then\marginnote{1.3} as he was in private retreat this thought came to his mind, “This principle I have discovered is deep, hard to see, hard to understand, peaceful, sublime, beyond the scope of logic, subtle, comprehensible to the astute. But people like attachment, they love it and enjoy it. It’s hard for them to see this thing; that is, specific conditionality, dependent origination. It’s also hard for them to see this thing; that is, the stilling of all activities, the letting go of all attachments, the ending of craving, fading away, cessation, extinguishment. And if I were to teach this principle, others might not understand me, which would be wearying and troublesome for me.” 

And\marginnote{1.9} then these verses, which were neither supernaturally inspired, nor learned before in the past, occurred to the Buddha: 

\begin{verse}%
“I’ve\marginnote{2.1} struggled hard to realize this, \\
enough with trying to explain it! \\
This principle is not easily understood \\
by those mired in greed and hate. 

Those\marginnote{3.1} besotted by greed can’t see \\
what’s subtle, going against the stream, \\
deep, hard to see, and very fine, \\
for they’re veiled in a mass of darkness.” 

%
\end{verse}

And\marginnote{4.1} as the Buddha reflected like this, his mind inclined to remaining passive, not to teaching the Dhamma. 

Then\marginnote{5.1} \textsanskrit{Brahmā} Sahampati, knowing what the Buddha was thinking, thought, “Oh my goodness! The world will be lost, the world will perish! For the mind of the Realized One, the perfected one, the fully awakened Buddha, inclines to remaining passive, not to teaching the Dhamma.” 

Then,\marginnote{5.3} as easily as a strong person would extend or contract their arm, he vanished from the \textsanskrit{Brahmā} realm and reappeared in front of the Buddha. He arranged his robe over one shoulder, knelt with his right knee on the ground, raised his joined palms toward the Buddha, and said: 

“Sir,\marginnote{5.5} let the Blessed One teach the Dhamma! Let the Holy One teach the Dhamma! There are beings with little dust in their eyes. They’re in decline because they haven’t heard the teaching. There will be those who understand the teaching!” 

This\marginnote{5.8} is what \textsanskrit{Brahmā} Sahampati said. Then he went on to say: 

\begin{verse}%
“Among\marginnote{6.1} the Magadhans there appeared in the past \\
an impure teaching thought up by those still stained. \\
Fling open the door to the deathless! \\
Let them hear the teaching the immaculate one discovered. 

Standing\marginnote{7.1} high on a rocky mountain, \\
you can see the people all around. \\
In just the same way, all-seer, wise one, \\
having ascended the Temple of Truth, \\
rid of sorrow, look upon the people \\
swamped with sorrow, oppressed by rebirth and old age. 

Rise,\marginnote{8.1} hero! Victor in battle, leader of the caravan, \\
wander the world without obligation. \\
Let the Blessed One teach the Dhamma! \\
There will be those who understand!” 

%
\end{verse}

Then\marginnote{9.1} the Buddha, understanding \textsanskrit{Brahmā}’s invitation, surveyed the world with the eye of a Buddha, because of his compassion for sentient beings. And the Buddha saw sentient beings with little dust in their eyes, and some with much dust in their eyes; with keen faculties and with weak faculties, with good qualities and with bad qualities, easy to teach and hard to teach. And some of them lived seeing the danger in the fault to do with the next world, while others did not. 

It’s\marginnote{9.3} like a pool with blue water lilies, or pink or white lotuses. Some of them sprout and grow in the water without rising above it, thriving underwater. Some of them sprout and grow in the water reaching the water’s surface. And some of them sprout and grow in the water but rise up above the water and stand with no water clinging to them. 

In\marginnote{9.4} the same way, the Buddha saw sentient beings with little dust in their eyes, and some with much dust in their eyes; with keen faculties and with weak faculties, with good qualities and with bad qualities, easy to teach and hard to teach. And some of them lived seeing the danger in the fault to do with the next world, while others did not. 

When\marginnote{9.5} he had seen this he replied in verse to \textsanskrit{Brahmā} Sahampati: 

\begin{verse}%
“Flung\marginnote{10.1} open are the doors to the deathless! \\
Let those with ears to hear commit to faith. \\
Thinking it would be troublesome, \textsanskrit{Brahmā}, I did not teach \\
the sophisticated, sublime Dhamma among humans.” 

%
\end{verse}

Then\marginnote{11.1} \textsanskrit{Brahmā} Sahampati, knowing that his request for the Buddha to teach the Dhamma had been granted, bowed and respectfully circled the Buddha, keeping him on his right, before vanishing right there. 

%
\section*{{\suttatitleacronym SN 6.2}{\suttatitletranslation Respect }{\suttatitleroot Gāravasutta}}
\addcontentsline{toc}{section}{\tocacronym{SN 6.2} \toctranslation{Respect } \tocroot{Gāravasutta}}
\markboth{Respect }{Gāravasutta}
\extramarks{SN 6.2}{SN 6.2}

\scevam{So\marginnote{1.1} I have heard. }At one time, when he was first awakened, the Buddha was staying near \textsanskrit{Uruvelā} at the root of the goatherd’s banyan tree on the bank of the \textsanskrit{Nerañjarā} River. 

Then\marginnote{1.3} as he was in private retreat this thought came to his mind, “It’s unpleasant to live without respect and reverence. What ascetic or brahmin should I honor and respect and rely on?” 

Then\marginnote{2.1} it occurred to him: 

“I’d\marginnote{2.2} honor and respect and rely on another ascetic or brahmin so as to complete the entire spectrum of ethics, if it were incomplete. But I don’t see any other ascetic or brahmin in this world—with its gods, \textsanskrit{Māras}, and \textsanskrit{Brahmās}, this population with its ascetics and brahmins, its gods and humans—who is more accomplished than myself in ethics, who I should honor and respect and rely on. 

I’d\marginnote{3.1} honor and respect and rely on another ascetic or brahmin so as to complete the entire spectrum of immersion, if it were incomplete. But I don’t see any other ascetic or brahmin … who is more accomplished than myself in immersion … 

I’d\marginnote{4.1} honor and respect and rely on another ascetic or brahmin so as to complete the entire spectrum of wisdom, if it were incomplete. But I don’t see any other ascetic or brahmin … who is more accomplished than myself in wisdom … 

I’d\marginnote{5.1} honor and respect and rely on another ascetic or brahmin so as to complete the entire spectrum of freedom, if it were incomplete. But I don’t see any other ascetic or brahmin … who is more accomplished than myself in freedom … 

I’d\marginnote{6.1} honor and respect and rely on another ascetic or brahmin so as to complete the entire spectrum of the knowledge and vision of freedom, if it were incomplete. But I don’t see any other ascetic or brahmin in this world—with its gods, \textsanskrit{Māras}, and \textsanskrit{Brahmās}, this population with its ascetics and brahmins, its gods and humans—who is more accomplished than myself in the knowledge and vision of freedom, who I should honor and respect and rely on. Why don’t I honor and respect and rely on the same Dhamma to which I was awakened?” 

Then\marginnote{7.1} \textsanskrit{Brahmā} Sahampati knew what the Buddha was thinking. As easily as a strong person would extend or contract their arm, he vanished from the \textsanskrit{Brahmā} realm and reappeared in front of the Buddha. He arranged his robe over one shoulder, raised his joined palms toward the Buddha, and said: 

“That’s\marginnote{7.3} so true, Blessed One! That’s so true, Holy One! All the perfected ones, the fully awakened Buddhas who lived in the past honored and respected and relied on this same teaching. All the perfected ones, the fully awakened Buddhas who will live in the future will honor and respect and rely on this same teaching. May the Blessed One, who is the perfected one, the fully awakened Buddha at present, also honor and respect and rely on this same teaching.” 

This\marginnote{7.7} is what \textsanskrit{Brahmā} Sahampati said. Then he went on to say: 

\begin{verse}%
“All\marginnote{8.1} Buddhas in the past, \\
the Buddhas of the future, \\
and the Buddha at present—\\
destroyer of the sorrows of many—

respecting\marginnote{9.1} the true teaching \\
they did live, they do live, \\
and they also will live. \\
This is the nature of the Buddhas. 

Therefore\marginnote{10.1} someone who cares for their own welfare, \\
and wants to become the very best they can be, \\
should respect the true teaching, \\
remembering the instructions of the Buddhas.” 

%
\end{verse}

%
\section*{{\suttatitleacronym SN 6.3}{\suttatitletranslation With Brahmadeva }{\suttatitleroot Brahmadevasutta}}
\addcontentsline{toc}{section}{\tocacronym{SN 6.3} \toctranslation{With Brahmadeva } \tocroot{Brahmadevasutta}}
\markboth{With Brahmadeva }{Brahmadevasutta}
\extramarks{SN 6.3}{SN 6.3}

\scevam{So\marginnote{1.1} I have heard. }At one time the Buddha was staying near \textsanskrit{Sāvatthī} in Jeta’s Grove, \textsanskrit{Anāthapiṇḍika}’s monastery. 

Now\marginnote{1.3} at that time a certain brahmin lady had a son called Brahmadeva, who had gone forth from the lay life to homelessness in the presence of the Buddha. 

Then\marginnote{2.1} Venerable Brahmadeva, living alone, withdrawn, diligent, keen, and resolute, soon realized the supreme end of the spiritual path in this very life. He lived having achieved with his own insight the goal for which gentlemen rightly go forth from the lay life to homelessness. 

He\marginnote{2.2} understood: “Rebirth is ended; the spiritual journey has been completed; what had to be done has been done; there is no return to any state of existence.” And Venerable Brahmadeva became one of the perfected. 

Then\marginnote{3.1} Brahmadeva robed up in the morning and, taking his bowl and robe, entered \textsanskrit{Sāvatthī} for alms. Wandering indiscriminately for almsfood in \textsanskrit{Sāvatthī}, he approached his own mother’s home. 

Now\marginnote{3.3} at that time Brahmadeva’s mother, the brahmin lady, was offering up a regular oblation to \textsanskrit{Brahmā}. 

Then\marginnote{3.4} \textsanskrit{Brahmā} Sahampati thought, “This Venerable Brahmadeva’s mother, the brahmin lady, offers up a regular oblation to \textsanskrit{Brahmā}. Why don’t I go and stir up a sense of urgency in her?” 

Then,\marginnote{3.7} as easily as a strong person would extend or contract their arm, he vanished from the \textsanskrit{Brahmā} realm and reappeared in the home of Brahmadeva’s mother. Then \textsanskrit{Brahmā} Sahampati, while standing in the air, addressed Brahmadeva’s mother in verse: 

\begin{verse}%
“Far\marginnote{4.1} from here is the \textsanskrit{Brahmā} realm, madam, \\
to which you offer a regular oblation. \\
But \textsanskrit{Brahmā} doesn’t eat that kind of food. \\
Why pray, when you don’t know the path to \textsanskrit{Brahmā}? 

This\marginnote{5.1} Brahmadeva, madam, \\
free of attachments, has surpassed the gods. \\
Owning nothing, providing for no other, a mendicant \\
has entered your house for alms. 

He’s\marginnote{6.1} worthy of offerings dedicated to the gods, a knowledge master, evolved. \\
He’s worthy of a religious donation from gods and men. \\
Having banished all evils, he’s unsullied. \\
Cool at heart, he wanders searching for food. 

He\marginnote{7.1} has no before and after, \\
peaceful, unclouded, untroubled, with no need for hope, \\
he has laid down the rod for all creatures firm and frail. \\
So let him enjoy your offering of choice alms. 

With\marginnote{8.1} peaceful mind, he has left the crowd, \\
he wanders like a tamed elephant, unperturbed. \\
He’s a mendicant fair in ethics, with heart well freed. \\
So let him enjoy your offering of choice alms. 

With\marginnote{9.1} unwavering confidence in him, \\
present your religious donation to one who is worthy of it. \\
Now that you’ve seen the sage who has crossed over, madam, \\
make merit for the sake of future happiness!” 

With\marginnote{10.1} unwavering confidence in him, \\
she presented her religious donation to one who is worthy of it. \\
After seeing the sage who had crossed over, the brahmin lady \\
made merit for the sake of future happiness. 

%
\end{verse}

%
\section*{{\suttatitleacronym SN 6.4}{\suttatitletranslation With Baka the Brahmā }{\suttatitleroot Bakabrahmasutta}}
\addcontentsline{toc}{section}{\tocacronym{SN 6.4} \toctranslation{With Baka the Brahmā } \tocroot{Bakabrahmasutta}}
\markboth{With Baka the Brahmā }{Bakabrahmasutta}
\extramarks{SN 6.4}{SN 6.4}

\scevam{So\marginnote{1.1} I have heard. }At one time the Buddha was staying near \textsanskrit{Sāvatthī} in Jeta’s Grove, \textsanskrit{Anāthapiṇḍika}’s monastery. 

Now\marginnote{1.3} at that time Baka the \textsanskrit{Brahmā} had the following harmful misconception: “This is permanent, this is everlasting, this is eternal, this is whole, this is imperishable. For this is where there’s no being born, growing old, dying, passing away, or being reborn. And there’s no other escape beyond this.” 

Then\marginnote{2.1} the Buddha knew what Baka the \textsanskrit{Brahmā} was thinking. As easily as a strong person would extend or contract their arm, he vanished from Jeta’s Grove and reappeared in that \textsanskrit{Brahmā} realm. 

Baka\marginnote{2.2} the \textsanskrit{Brahmā} saw the Buddha coming off in the distance and said to him, “Come, good sir! Welcome, good sir! It’s been a long time since you took the opportunity to come here. For this is permanent, this is everlasting, this is eternal, this is complete, this is imperishable. For this is where there’s no being born, growing old, dying, passing away, or being reborn. And there’s no other escape beyond this.” 

When\marginnote{3.1} he had spoken, the Buddha said to him, “Alas, Baka the \textsanskrit{Brahmā} is lost in ignorance! Alas, Baka the \textsanskrit{Brahmā} is lost in ignorance! Because what is actually impermanent, not lasting, transient, incomplete, and perishable, he says is permanent, everlasting, eternal, complete, and imperishable. And where there is being born, growing old, dying, passing away, and being reborn, he says that there’s no being born, growing old, dying, passing away, or being reborn. And although there is another escape beyond this, he says that there’s no other escape beyond this.” 

\begin{verse}%
“Gotama,\marginnote{4.1} we seventy-two merit-makers are now wielders of power, \\
having passed beyond rebirth and old age. \\
This is our last rebirth as \textsanskrit{Brahmā}, knowledge master. \\
And now many people pray to us.” 

“But,\marginnote{5.1} Baka, the life span here is short, not long, \\
though you think it’s long. \\
I know that your life span \\
is two quinquadecillion years, \textsanskrit{Brahmā}.” 

“Blessed\marginnote{6.1} One, I am the one of infinite vision, \\
who has gone beyond rebirth and old age and sorrow. \\
What precepts and observances did I practice in the past? \\
Explain to me so that I may understand.” 

“You\marginnote{7.1} gave drink to many people \\
who were oppressed by thirst and heat. \\
They’re the precepts and observances you practiced in the past. \\
I recollect it like one who has wakened from sleep. 

When\marginnote{8.1} people at Deer River Bank were seized, \\
you released the captives as they were led away. \\
That’s the precepts and observances you practiced in the past. \\
I recollect it like one who has wakened from sleep. 

When\marginnote{9.1} a boat on the Ganges River was seized \\
by a fierce dragon desiring human flesh, \\
you freed it wielding mighty force. \\
That’s the precepts and observances you practiced in the past. \\
I recollect it like one who has wakened from sleep. 

I\marginnote{10.1} used to be your servant named Kappa. \\
You thought he was intelligent and loyal. \\
That’s the precepts and observances you practiced in the past. \\
I recollect it like one who has wakened from sleep.” 

“You\marginnote{11.1} certainly understand this life span of mine. \\
And others, too, you know; that’s why you’re the Buddha. \\
And that’s why your blazing glory \\
lights up even the \textsanskrit{Brahmā} realm.” 

%
\end{verse}

%
\section*{{\suttatitleacronym SN 6.5}{\suttatitletranslation A Certain Brahmā }{\suttatitleroot Aññatarabrahmasutta}}
\addcontentsline{toc}{section}{\tocacronym{SN 6.5} \toctranslation{A Certain Brahmā } \tocroot{Aññatarabrahmasutta}}
\markboth{A Certain Brahmā }{Aññatarabrahmasutta}
\extramarks{SN 6.5}{SN 6.5}

At\marginnote{1.1} \textsanskrit{Sāvatthī}. 

Now\marginnote{1.2} at that time a certain \textsanskrit{Brahmā} had the following harmful misconception: “No ascetic or brahmin can come here!” 

Then\marginnote{1.4} the Buddha knew what that \textsanskrit{Brahmā} was thinking. As easily as a strong person would extend or contract their arm, he vanished from Jeta’s Grove and reappeared in that \textsanskrit{Brahmā} realm. Then the Buddha sat cross-legged in the air above that \textsanskrit{Brahmā}, having entered upon the fire element. 

Then\marginnote{2.1} Venerable \textsanskrit{Mahāmoggallāna} thought, “Where is the Buddha staying at present?” With clairvoyance that is purified and superhuman, he saw the Buddha seated cross-legged in the air above that \textsanskrit{Brahmā}, having entered upon the fire element. Then, as easily as a strong person would extend or contract their arm, he vanished from Jeta’s Grove and reappeared in that \textsanskrit{Brahmā} realm. Then \textsanskrit{Mahāmoggallāna}—positioning himself in the east, below the Buddha—sat cross-legged in the air above that \textsanskrit{Brahmā}, having entered upon the fire element. 

Then\marginnote{3.1} Venerable \textsanskrit{Mahākassapa} … positioned himself in the south … 

Venerable\marginnote{4.1} \textsanskrit{Mahākappina} … positioned himself in the west … 

Venerable\marginnote{5.1} Anuruddha … positioned himself in the north, below the Buddha, sitting cross-legged in the air above that \textsanskrit{Brahmā}, having entered upon the fire element. 

Then\marginnote{6.1} \textsanskrit{Mahāmoggallāna} addressed that \textsanskrit{Brahmā} in verse: 

\begin{verse}%
“Sir,\marginnote{7.1} do you still have the same view \\
that you had in the past? \\
Or do you see the radiance \\
transcending the \textsanskrit{Brahmā} realm?” 

“Good\marginnote{8.1} sir, I don’t have that view \\
that I had in the past. \\
I see the radiance \\
transcending the \textsanskrit{Brahmā} realm. \\
So how could I say today \\
that I am permanent and eternal?” 

%
\end{verse}

Having\marginnote{9.1} inspired a sense of awe in the \textsanskrit{Brahmā}, as easily as a strong person would extend or contract their arm, the Buddha vanished from the \textsanskrit{Brahmā} realm and reappeared in Jeta’s Grove. 

Then\marginnote{9.2} that \textsanskrit{Brahmā} addressed a member of his retinue, “Please, good sir, go up to Venerable \textsanskrit{Mahāmoggallāna} and say to him: ‘\textsanskrit{Moggallāna} my good sir, are there any other disciples of the Buddha who have power and might comparable to the masters \textsanskrit{Moggallāna}, Kassapa, Kappina, and Anuruddha?’” 

“Yes,\marginnote{9.6} good sir,” replied that retinue member. He went to \textsanskrit{Moggallāna} and asked as instructed. 

Then\marginnote{9.9} \textsanskrit{Mahāmoggallāna} addressed that member of \textsanskrit{Brahmā}’s retinue in verse: 

\begin{verse}%
“There\marginnote{10.1} are many disciples of the Buddha \\
who have the three knowledges, \\
and have attained psychic power, expert in reading minds, \\
they’re perfected ones with defilements ended.” 

%
\end{verse}

Then\marginnote{11.1} that member of \textsanskrit{Brahmā}’s retinue, having approved and agreed with what \textsanskrit{Mahāmoggallāna} said, went to that \textsanskrit{Brahmā} and said to him, “Good sir, Venerable \textsanskrit{Mahāmoggallāna} said this: 

\begin{verse}%
‘There\marginnote{12.1} are many disciples of the Buddha \\
who have the three knowledges, \\
and have attained psychic power, expert in reading minds, \\
they’re perfected ones with defilements ended.’” 

%
\end{verse}

That’s\marginnote{13.1} what that member of \textsanskrit{Brahmā}’s retinue said. Satisfied, that \textsanskrit{Brahmā} was happy with what the member of his retinue said. 

%
\section*{{\suttatitleacronym SN 6.6}{\suttatitletranslation The Negligent Brahmā }{\suttatitleroot Brahmalokasutta}}
\addcontentsline{toc}{section}{\tocacronym{SN 6.6} \toctranslation{The Negligent Brahmā } \tocroot{Brahmalokasutta}}
\markboth{The Negligent Brahmā }{Brahmalokasutta}
\extramarks{SN 6.6}{SN 6.6}

At\marginnote{1.1} \textsanskrit{Sāvatthī}. 

Now\marginnote{1.2} at that time the Buddha had gone into retreat for the day’s meditation. 

Then\marginnote{1.3} the independent \textsanskrit{brahmās} \textsanskrit{Subrahmā} and \textsanskrit{Suddhāvāsa} went to the Buddha and stationed themselves one by each door-post. But \textsanskrit{Subrahmā} said to \textsanskrit{Suddhāvāsa}, “Good sir, it’s the wrong time to pay homage to the Buddha. He has gone into retreat for the day’s meditation. But such and such \textsanskrit{Brahmā} realm is successful and prosperous, while the \textsanskrit{Brahmā} living there is negligent. Come, let’s go to that \textsanskrit{Brahmā} realm and inspire awe in that \textsanskrit{Brahmā}!” 

“Yes,\marginnote{1.8} good sir,” replied \textsanskrit{Suddhāvāsa}. 

Then,\marginnote{2.1} as easily as a strong person would extend or contract their arm, they vanished from in front of the Buddha and appeared in that \textsanskrit{Brahmā} realm. 

That\marginnote{2.2} \textsanskrit{Brahmā} saw those \textsanskrit{Brahmās} coming off in the distance and said to them, “Well now, good sirs, where have you come from?” 

“Good\marginnote{2.4} sir, we’ve come from the presence of the Blessed One, the perfected one, the fully awakened Buddha. Shouldn’t you go to attend on that Blessed One?” 

When\marginnote{3.1} they had spoken, that \textsanskrit{Brahmā} refused to accept their advice. He multiplied himself a thousand times and said to \textsanskrit{Subrahmā}, “Good sir, can’t you see that I have such psychic power?” 

“I\marginnote{3.3} see that, good sir.” 

“Since\marginnote{3.4} I have such psychic power and might, what other ascetic or brahmin should I go to and attend upon?” 

Then\marginnote{4.1} \textsanskrit{Subrahmā} multiplied himself two thousand times and said to that \textsanskrit{Brahmā}, “Good sir, can’t you see that I have such psychic power?” 

“I\marginnote{4.3} see that, good sir.” 

“That\marginnote{4.4} Buddha has even more psychic power and might than you or me. Shouldn’t you go to attend on that Blessed One?” 

Then\marginnote{4.6} that \textsanskrit{Brahmā} addressed \textsanskrit{Subrahmā} in verse: 

\begin{verse}%
“There\marginnote{5.1} are three hundreds of phoenixes, four of swans, and five of eagles. \\
This palace belongs to him who practiced absorption. \\
It shines, \textsanskrit{Brahmā}, \\
lighting up the northern quarter!” 

“So\marginnote{6.1} what if your palace shines, \\
lighting up the northern quarter? \\
A clever person who has seen the deficiency in form, \\
its chronic trembling, takes no pleasure in it.” 

%
\end{verse}

Then\marginnote{7.1} after inspiring awe in that \textsanskrit{Brahmā}, the independent \textsanskrit{brahmās} \textsanskrit{Subrahmā} and \textsanskrit{Suddhāvāsa} vanished right there. And after some time that \textsanskrit{Brahmā} went to attend on the Buddha. 

%
\section*{{\suttatitleacronym SN 6.7}{\suttatitletranslation About Kokālika }{\suttatitleroot Kokālikasutta}}
\addcontentsline{toc}{section}{\tocacronym{SN 6.7} \toctranslation{About Kokālika } \tocroot{Kokālikasutta}}
\markboth{About Kokālika }{Kokālikasutta}
\extramarks{SN 6.7}{SN 6.7}

At\marginnote{1.1} \textsanskrit{Sāvatthī}. 

Now\marginnote{1.2} at that time the Buddha had gone into retreat for the day’s meditation. Then the independent \textsanskrit{brahmās} \textsanskrit{Subrahmā} and \textsanskrit{Suddhāvāsa} went to the Buddha and stationed themselves one by each door-post. 

Then\marginnote{1.4} \textsanskrit{Subrahmā} recited this verse about the mendicant \textsanskrit{Kokālika} in the Buddha’s presence: 

\begin{verse}%
“What\marginnote{2.1} wise person here would judge \\
the immeasurable by measuring them? \\
I think anyone who’d do such a thing \\
must be an ordinary person, shrouded in darkness.” 

%
\end{verse}

%
\section*{{\suttatitleacronym SN 6.8}{\suttatitletranslation About Katamorakatissaka }{\suttatitleroot Katamodakatissasutta}}
\addcontentsline{toc}{section}{\tocacronym{SN 6.8} \toctranslation{About Katamorakatissaka } \tocroot{Katamodakatissasutta}}
\markboth{About Katamorakatissaka }{Katamodakatissasutta}
\extramarks{SN 6.8}{SN 6.8}

At\marginnote{1.1} \textsanskrit{Sāvatthī}. 

Now\marginnote{1.2} at that time the Buddha had gone into retreat for the day’s meditation. Then the independent \textsanskrit{brahmās} \textsanskrit{Subrahmā} and \textsanskrit{Suddhāvāsa} went to the Buddha and stationed themselves one by each door-post. 

Then\marginnote{1.4} \textsanskrit{Suddhāvāsa} recited this verse about the mendicant Katamorakatissaka in the Buddha’s presence: 

\begin{verse}%
“What\marginnote{2.1} wise person here would judge \\
the immeasurable by measuring them? \\
I think anyone who’d do such a thing \\
must be a fool, shrouded in darkness.” 

%
\end{verse}

%
\section*{{\suttatitleacronym SN 6.9}{\suttatitletranslation With the Brahmā Tudu }{\suttatitleroot Turūbrahmasutta}}
\addcontentsline{toc}{section}{\tocacronym{SN 6.9} \toctranslation{With the Brahmā Tudu } \tocroot{Turūbrahmasutta}}
\markboth{With the Brahmā Tudu }{Turūbrahmasutta}
\extramarks{SN 6.9}{SN 6.9}

At\marginnote{1.1} \textsanskrit{Sāvatthī}. 

Now\marginnote{1.2} at that time the mendicant \textsanskrit{Kokālika} was sick, suffering, gravely ill. 

Then,\marginnote{1.3} late at night, the beautiful independent \textsanskrit{brahmā} Tudu, lighting up the entire Jeta’s Grove, went up to the mendicant \textsanskrit{Kokālika}, and standing in the air he said to him, “\textsanskrit{Kokālika}, have confidence in \textsanskrit{Sāriputta} and \textsanskrit{Moggallāna}, they’re good monks.” 

“Who\marginnote{1.6} are you, reverend?” 

“I\marginnote{1.7} am Tudu the independent \textsanskrit{brahmā}.” 

“Didn’t\marginnote{1.8} the Buddha declare you a non-returner? So what exactly are you doing back here? See how far you have strayed!” 

\begin{verse}%
“A\marginnote{2.1} man is born \\
with an axe in his mouth. \\
A fool cuts themselves with it \\
when they say bad words. 

When\marginnote{3.1} you praise someone worthy of criticism, \\
or criticize someone worthy of praise, \\
you choose bad luck with your own mouth: \\
you’ll never find happiness that way. 

Bad\marginnote{4.1} luck at dice is a trivial thing, \\
if all you lose is your money \\
and all you own, even yourself. \\
What’s really terrible luck \\
is to hate the holy ones. 

For\marginnote{5.1} more than two quinquadecillion years, \\
and another five quattuordecillion years, \\
a slanderer of noble ones goes to hell, \\
having aimed bad words and thoughts at them.” 

%
\end{verse}

%
\section*{{\suttatitleacronym SN 6.10}{\suttatitletranslation With Kokālika }{\suttatitleroot Kokālikasutta}}
\addcontentsline{toc}{section}{\tocacronym{SN 6.10} \toctranslation{With Kokālika } \tocroot{Kokālikasutta}}
\markboth{With Kokālika }{Kokālikasutta}
\extramarks{SN 6.10}{SN 6.10}

At\marginnote{1.1} \textsanskrit{Sāvatthī}. 

Then\marginnote{1.2} the mendicant \textsanskrit{Kokālika} went up to the Buddha, bowed, sat down to one side, and said to him, “Sir, \textsanskrit{Sāriputta} and \textsanskrit{Moggallāna} have wicked desires. They’ve fallen under the sway of wicked desires.” 

When\marginnote{1.4} this was said, the Buddha said to \textsanskrit{Kokālika}, “Don’t say that, \textsanskrit{Kokālika}! Don’t say that, \textsanskrit{Kokālika}! Have confidence in \textsanskrit{Sāriputta} and \textsanskrit{Moggallāna}, they’re good monks.” 

For\marginnote{1.7} a second time \textsanskrit{Kokālika} said to the Buddha, “Despite my faith and trust in the Buddha, \textsanskrit{Sāriputta} and \textsanskrit{Moggallāna} have wicked desires. They’ve fallen under the sway of wicked desires.” 

For\marginnote{1.9} a second time, the Buddha said to \textsanskrit{Kokālika}, “Don’t say that, \textsanskrit{Kokālika}! Don’t say that, \textsanskrit{Kokālika}! Have confidence in \textsanskrit{Sāriputta} and \textsanskrit{Moggallāna}, they’re good monks.” 

For\marginnote{1.12} a third time \textsanskrit{Kokālika} said to the Buddha, “Despite my faith and trust in the Buddha, \textsanskrit{Sāriputta} and \textsanskrit{Moggallāna} have wicked desires. They’ve fallen under the sway of wicked desires.” 

For\marginnote{1.14} a third time, the Buddha said to \textsanskrit{Kokālika}, “Don’t say that, \textsanskrit{Kokālika}! Don’t say that, \textsanskrit{Kokālika}! Have confidence in \textsanskrit{Sāriputta} and \textsanskrit{Moggallāna}, they’re good monks.” 

Then\marginnote{2.1} \textsanskrit{Kokālika} got up from his seat, bowed, and respectfully circled the Buddha, keeping him on his right, before leaving. Not long after he left his body erupted with boils the size of mustard seeds. The boils grew to the size of mung beans, then chickpeas, then jujube seeds, then jujubes, then myrobalans, then unripe wood apples, then ripe wood apples. Finally they burst open, and pus and blood oozed out. Then the mendicant \textsanskrit{Kokālika} died of that illness. He was reborn in the Pink Lotus hell because of his resentment for \textsanskrit{Sāriputta} and \textsanskrit{Moggallāna}. 

Then,\marginnote{3.1} late at night, the beautiful \textsanskrit{Brahmā} Sahampati, lighting up the entire Jeta’s Grove, went up to the Buddha, bowed, stood to one side, and said to him, “Sir, the mendicant \textsanskrit{Kokālika} has passed away. He was reborn in the Pink Lotus hell because of his resentment for \textsanskrit{Sāriputta} and \textsanskrit{Moggallāna}.” 

That’s\marginnote{3.4} what \textsanskrit{Brahmā} Sahampati said. Then he bowed and respectfully circled the Buddha, keeping him on his right side, before vanishing right there. 

Then,\marginnote{4.1} when the night had passed, the Buddha told the mendicants all that had happened. 

When\marginnote{5.1} he said this, one of the mendicants asked the Buddha, “Sir, how long is the life span in the Pink Lotus hell?” 

“It’s\marginnote{5.3} long, mendicant. It’s not easy to calculate how many years, how many hundreds or thousands or hundreds of thousands of years it lasts.” 

“But\marginnote{5.6} sir, is it possible to give a simile?” 

“It’s\marginnote{5.7} possible,” said the Buddha. 

“Suppose\marginnote{6.1} there was a Kosalan cartload of twenty bushels of sesame seed. And at the end of every hundred years someone would remove a single seed from it. By this means the Kosalan cartload of twenty bushels of sesame seed would run out faster than a single lifetime in the Abbuda hell. 

Now,\marginnote{6.3} twenty lifetimes in the Abbuda hell equal one lifetime in the Nirabbuda hell. Twenty lifetimes in the Nirabbuda hell equal one lifetime in the Ababa hell. Twenty lifetimes in the Ababa hell equal one lifetime in the \textsanskrit{Aṭaṭa} hell. Twenty lifetimes in the \textsanskrit{Aṭaṭa} hell equal one lifetime in the Ahaha hell. Twenty lifetimes in the Ahaha hell equal one lifetime in the Yellow Lotus hell. Twenty lifetimes in the Yellow Lotus hell equal one lifetime in the Sweet-Smelling hell. Twenty lifetimes in the Sweet-Smelling hell equal one lifetime in the Blue Water Lily hell. Twenty lifetimes in the Blue Water Lily hell equal one lifetime in the White Lotus hell. Twenty lifetimes in the White Lotus hell equal one lifetime in the Pink Lotus hell. 

The\marginnote{6.12} mendicant \textsanskrit{Kokālika} has been reborn in the Pink Lotus hell because of his resentment for \textsanskrit{Sāriputta} and \textsanskrit{Moggallāna}.” 

That\marginnote{6.13} is what the Buddha said. Then the Holy One, the Teacher, went on to say: 

\begin{verse}%
“A\marginnote{7.1} man is born \\
with an axe in his mouth. \\
A fool cuts themselves with it \\
when they say bad words. 

When\marginnote{8.1} you praise someone worthy of criticism, \\
or criticize someone worthy of praise, \\
you choose bad luck with your own mouth: \\
you’ll never find happiness that way. 

Bad\marginnote{9.1} luck at dice is a trivial thing, \\
if all you lose is your money \\
and all you own, even yourself. \\
What’s really terrible luck \\
is to hate the holy ones. 

For\marginnote{10.1} more than two quinquadecillion years, \\
and another five quattuordecillion years, \\
a slanderer of noble ones goes to hell, \\
having aimed bad words and thoughts at them.” 

%
\end{verse}

%
\addtocontents{toc}{\let\protect\contentsline\protect\nopagecontentsline}
\chapter*{Chapter Two }
\addcontentsline{toc}{chapter}{\tocchapterline{Chapter Two }}
\addtocontents{toc}{\let\protect\contentsline\protect\oldcontentsline}

%
\section*{{\suttatitleacronym SN 6.11}{\suttatitletranslation With Sanaṅkumāra }{\suttatitleroot Sanaṅkumārasutta}}
\addcontentsline{toc}{section}{\tocacronym{SN 6.11} \toctranslation{With Sanaṅkumāra } \tocroot{Sanaṅkumārasutta}}
\markboth{With Sanaṅkumāra }{Sanaṅkumārasutta}
\extramarks{SN 6.11}{SN 6.11}

\scevam{So\marginnote{1.1} I have heard. }At one time the Buddha was staying near \textsanskrit{Rājagaha}, on the bank of the \textsanskrit{Sappinī} river. 

Then,\marginnote{1.3} late at night, the beautiful \textsanskrit{Brahmā} \textsanskrit{Sanaṅkumāra}, lighting up the entire \textsanskrit{Sappinī} riverbank, went up to the Buddha, bowed, stood to one side, and recited this verse in the Buddha’s presence: 

\begin{verse}%
“The\marginnote{2.1} aristocrat is first among people \\
who take clan as the standard. \\
But one accomplished in knowledge and conduct \\
is first among gods and humans.” 

%
\end{verse}

That’s\marginnote{3.1} what \textsanskrit{Brahmā} \textsanskrit{Sanaṅkumāra} said, and the teacher approved. Then \textsanskrit{Brahmā} \textsanskrit{Sanaṅkumāra}, knowing that the teacher approved, bowed and respectfully circled the Buddha, keeping him on his right, before vanishing right there. 

%
\section*{{\suttatitleacronym SN 6.12}{\suttatitletranslation About Devadatta }{\suttatitleroot Devadattasutta}}
\addcontentsline{toc}{section}{\tocacronym{SN 6.12} \toctranslation{About Devadatta } \tocroot{Devadattasutta}}
\markboth{About Devadatta }{Devadattasutta}
\extramarks{SN 6.12}{SN 6.12}

\scevam{So\marginnote{1.1} I have heard. }At one time the Buddha was staying near \textsanskrit{Rājagaha}, on the Vulture’s Peak Mountain, not long after Devadatta had left. 

Then,\marginnote{1.3} late at night, the beautiful \textsanskrit{Brahmā} Sahampati, lighting up the entire Vulture’s Peak, went up to the Buddha, bowed, stood to one side, and recited this verse in the Buddha’s presence: 

\begin{verse}%
“The\marginnote{2.1} banana tree is destroyed by its own fruit, \\
as are the bamboo and the reed. \\
Honor destroys a sinner, \\
as pregnancy destroys a mule.” 

%
\end{verse}

%
\section*{{\suttatitleacronym SN 6.13}{\suttatitletranslation At Andhakavinda }{\suttatitleroot Andhakavindasutta}}
\addcontentsline{toc}{section}{\tocacronym{SN 6.13} \toctranslation{At Andhakavinda } \tocroot{Andhakavindasutta}}
\markboth{At Andhakavinda }{Andhakavindasutta}
\extramarks{SN 6.13}{SN 6.13}

At\marginnote{1.1} one time the Buddha was staying in the land of the Magadhans at Andhakavinda. 

Now\marginnote{1.2} at that time the Buddha was meditating in the open during the dark of night, while a gentle rain drizzled down. 

Then,\marginnote{1.3} late at night, the beautiful \textsanskrit{Brahmā} Sahampati, lighting up the entirety of Andhakavinda, went up to the Buddha, bowed, stood to one side, and recited these verses in the Buddha’s presence: 

\begin{verse}%
“One\marginnote{2.1} should frequent secluded lodgings, \\
and practice to be released from fetters. \\
If you don’t find enjoyment there, \\
live in the \textsanskrit{Saṅgha}, guarded and mindful. 

Walking\marginnote{3.1} for alms from family to family, \\
with senses guarded, alert and mindful. \\
One should frequent secluded lodgings, \\
free of fear, freed in the fearless. 

Where\marginnote{4.1} dreadful serpents slither, \\
where the lightning flashes and the sky thunders \\
in the dark of the night; \\
there meditates a mendicant, free of goosebumps. 

For\marginnote{5.1} this has in fact been seen by me, \\
it isn’t just what the testament says. \\
Within a single spiritual dispensation \\
a thousand are conquerors of Death. 

And\marginnote{6.1} of trainees there are more than five hundred, \\
and ten times ten tens; \\
all are stream-enterers, \\
freed from rebirth in the animal realm. 

And\marginnote{7.1} as for other people \\
who I think have shared in merit—\\
I couldn’t even number them, \\
for fear of speaking falsely.” 

%
\end{verse}

%
\section*{{\suttatitleacronym SN 6.14}{\suttatitletranslation About Aruṇavatī }{\suttatitleroot Aruṇavatīsutta}}
\addcontentsline{toc}{section}{\tocacronym{SN 6.14} \toctranslation{About Aruṇavatī } \tocroot{Aruṇavatīsutta}}
\markboth{About Aruṇavatī }{Aruṇavatīsutta}
\extramarks{SN 6.14}{SN 6.14}

\scevam{So\marginnote{1.1} I have heard. }At one time the Buddha was staying near \textsanskrit{Sāvatthī}. There he addressed the mendicants, “Mendicants!” 

“Venerable\marginnote{1.5} sir,” they replied. The Buddha said this: 

“Once\marginnote{2.1} upon a time, mendicants, there was a king named \textsanskrit{Aruṇavā}. He had a capital named \textsanskrit{Aruṇavatī}. \textsanskrit{Sikhī} the Blessed One, the perfected one, the fully awakened Buddha lived supported by \textsanskrit{Aruṇavatī}. \textsanskrit{Sikhī} had a fine pair of chief disciples named \textsanskrit{Abhibhū} and Sambhava. 

Then\marginnote{2.5} the Buddha \textsanskrit{Sikhī} addressed the mendicant \textsanskrit{Abhibhū}, ‘Come, brahmin, let’s go to one of the \textsanskrit{brahmā} realms until it’s time for our meal.’ 

‘Yes,\marginnote{2.7} sir,’ replied \textsanskrit{Abhibhū}. Then, as easily as a strong person would extend or contract their arm, they vanished from \textsanskrit{Aruṇavatī} and appeared in that \textsanskrit{Brahmā} realm. 

Then\marginnote{3.1} the Buddha \textsanskrit{Sikhī} addressed the mendicant \textsanskrit{Abhibhū}, ‘Brahmin, teach the Dhamma as you feel inspired for that \textsanskrit{Brahmā}, his assembly, and the members of his retinue.’ 

‘Yes,\marginnote{3.3} sir,’ replied \textsanskrit{Abhibhū}. Then he educated, encouraged, fired up, and inspired them with a Dhamma talk. 

But\marginnote{3.4} the \textsanskrit{Brahmā}, his assembly, and his retinue complained, grumbled, and objected, ‘It’s incredible, it’s amazing! How on earth can a disciple teach Dhamma in the presence of the Teacher?’ 

Then\marginnote{4.1} the Buddha \textsanskrit{Sikhī} addressed the mendicant \textsanskrit{Abhibhū}, ‘Brahmin, \textsanskrit{Brahmā}, his assembly, and his retinue are complaining that a disciple teaches Dhamma in the presence of the Teacher. Well then, brahmin, stir them up even more!’ 

‘Yes,\marginnote{4.5} sir,’ replied \textsanskrit{Abhibhū}. Then he taught Dhamma with his body visible; with his body invisible; with the lower half visible and the upper half invisible; and with the upper half visible and the lower half invisible. 

And\marginnote{4.6} the \textsanskrit{Brahmā}, his assembly, and his retinue, their minds full of wonder and amazement, thought, ‘It’s incredible, it’s amazing! The ascetic has such psychic power and might!’ 

Then\marginnote{5.1} \textsanskrit{Abhibhū} said to the Buddha \textsanskrit{Sikhī}, ‘Sir, I recall having said this in the middle of the \textsanskrit{Saṅgha}: “Standing in the \textsanskrit{Brahmā} realm, I can make my voice heard throughout the galaxy.”’ 

‘Now\marginnote{5.4} is the time, brahmin! Now is the time, brahmin! Standing in the \textsanskrit{Brahmā} realm, make your voice heard throughout the galaxy.’ 

‘Yes,\marginnote{5.6} sir,’ replied \textsanskrit{Abhibhū}. Standing in the \textsanskrit{Brahmā} realm, he recited this verse: 

\begin{verse}%
‘Rouse\marginnote{6.1} yourselves! Try harder! \\
Devote yourselves to the teachings of the Buddha! \\
Crush the army of Death, \\
as an elephant a hut of reeds. 

Whoever\marginnote{7.1} shall meditate diligently \\
in this teaching and training, \\
giving up transmigration through rebirths, \\
will make an end of suffering.’ 

%
\end{verse}

Having\marginnote{8.1} inspired that \textsanskrit{Brahmā}, his assembly, and his retinue with a sense of awe, as easily as a strong person would extend or contract their arm, \textsanskrit{Sikhī} and \textsanskrit{Abhibhū} vanished from that \textsanskrit{Brahmā} realm and appeared in \textsanskrit{Aruṇavatī}. 

Then\marginnote{8.3} the Buddha \textsanskrit{Sikhī} addressed the mendicants, ‘Mendicants, did you hear the mendicant \textsanskrit{Abhibhū} speaking a verse while standing in a \textsanskrit{Brahmā} realm?’ 

‘We\marginnote{8.5} did, sir.’ 

‘But\marginnote{8.6} what exactly did you hear?’ 

‘This\marginnote{8.7} is what we heard, sir: 

\begin{verse}%
“Rouse\marginnote{9.1} yourselves! Try harder! \\
Devote yourselves to the teachings of the Buddha! \\
Crush the army of Death, \\
as an elephant a hut of reeds. 

Whoever\marginnote{10.1} shall meditate diligently \\
in this teaching and training, \\
giving up transmigration through rebirths, \\
will make an end of suffering.” 

%
\end{verse}

That’s\marginnote{11.1} what we heard, sir.’ 

‘Good,\marginnote{11.2} good, mendicants! It’s good that you heard the mendicant \textsanskrit{Abhibhū} speaking this verse while standing in a \textsanskrit{Brahmā} realm.’” 

That\marginnote{12.1} is what the Buddha said. Satisfied, the mendicants were happy with what the Buddha said. 

%
\section*{{\suttatitleacronym SN 6.15}{\suttatitletranslation Final Extinguishment }{\suttatitleroot Parinibbānasutta}}
\addcontentsline{toc}{section}{\tocacronym{SN 6.15} \toctranslation{Final Extinguishment } \tocroot{Parinibbānasutta}}
\markboth{Final Extinguishment }{Parinibbānasutta}
\extramarks{SN 6.15}{SN 6.15}

At\marginnote{1.1} one time the Buddha was staying between a pair of sal trees in the sal forest of the Mallas at Upavattana near \textsanskrit{Kusinārā} at the time of his final extinguishment. 

Then\marginnote{1.2} the Buddha said to the mendicants: “Come now, mendicants, I say to you all: ‘Conditions fall apart. Persist with diligence.’” 

These\marginnote{1.5} were the Realized One’s last words. 

Then\marginnote{2.1} the Buddha entered the first absorption. Emerging from that, he entered the second absorption. Emerging from that, he successively entered into and emerged from the third absorption, the fourth absorption, the dimension of infinite space, the dimension of infinite consciousness, the dimension of nothingness, and the dimension of neither perception nor non-perception. Then he entered the cessation of perception and feeling. 

Then\marginnote{3.1} he emerged from the cessation of perception and feeling and entered the dimension of neither perception nor non-perception. Emerging from that, he successively entered into and emerged from the dimension of nothingness, the dimension of infinite consciousness, the dimension of infinite space, the fourth absorption, the third absorption, the second absorption, and the first absorption. Emerging from that, he successively entered into and emerged from the second absorption and the third absorption. Then he entered the fourth absorption. Emerging from that the Buddha immediately became fully extinguished. 

When\marginnote{3.2} the Buddha became fully extinguished, along with the full extinguishment, \textsanskrit{Brahmā} Sahampati recited this verse: 

\begin{verse}%
“All\marginnote{4.1} creatures in this world \\
must lay down this bag of bones. \\
For even a Teacher such as this, \\
unrivaled in the world, \\
the Realized One, attained to power, \\
the Buddha became fully extinguished.” 

%
\end{verse}

When\marginnote{5.1} the Buddha became fully extinguished, Sakka, lord of gods, recited this verse: 

\begin{verse}%
“Oh!\marginnote{6.1} Conditions are impermanent, \\
their nature is to rise and fall; \\
having arisen, they cease; \\
their stilling is true bliss.” 

%
\end{verse}

When\marginnote{7.1} the Buddha became fully extinguished, Venerable Ānanda recited this verse: 

\begin{verse}%
“Then\marginnote{8.1} there was terror! \\
Then they had goosebumps! \\
When the Buddha, endowed with all fine qualities, \\
became fully extinguished.” 

%
\end{verse}

When\marginnote{9.1} the Buddha became fully extinguished, Venerable Anuruddha recited this verse: 

\begin{verse}%
“There\marginnote{10.1} was no more breathing \\
for the poised one of steady heart. \\
Imperturbable, committed to peace, \\
the seer became fully extinguished. 

He\marginnote{11.1} put up with painful feelings \\
without flinching. \\
The liberation of his heart \\
was like the extinguishing of a lamp.” 

%
\end{verse}

\scendsutta{The Linked Discourses on \textsanskrit{Brahmā} are complete. }

%
\addtocontents{toc}{\let\protect\contentsline\protect\nopagecontentsline}
\part*{Linked Discourses with Brahmins }
\addcontentsline{toc}{part}{Linked Discourses with Brahmins }
\markboth{}{}
\addtocontents{toc}{\let\protect\contentsline\protect\oldcontentsline}

%
\addtocontents{toc}{\let\protect\contentsline\protect\nopagecontentsline}
\chapter*{The Chapter on the Perfected Ones }
\addcontentsline{toc}{chapter}{\tocchapterline{The Chapter on the Perfected Ones }}
\addtocontents{toc}{\let\protect\contentsline\protect\oldcontentsline}

%
\section*{{\suttatitleacronym SN 7.1}{\suttatitletranslation With Dhanañjānī }{\suttatitleroot Dhanañjānīsutta}}
\addcontentsline{toc}{section}{\tocacronym{SN 7.1} \toctranslation{With Dhanañjānī } \tocroot{Dhanañjānīsutta}}
\markboth{With Dhanañjānī }{Dhanañjānīsutta}
\extramarks{SN 7.1}{SN 7.1}

\scevam{So\marginnote{1.1} I have heard. }At one time the Buddha was staying near \textsanskrit{Rājagaha}, in the Bamboo Grove, the squirrels’ feeding ground. 

Now\marginnote{1.3} at that time a certain brahmin lady of the \textsanskrit{Bhāradvāja} clan named \textsanskrit{Dhanañjānī} was devoted to the Buddha, the teaching, and the \textsanskrit{Saṅgha}. Once, while she was bringing her husband his meal she tripped and expressed this heartfelt sentiment three times: 

“Homage\marginnote{2.1} to that Blessed One, the perfected one, the fully awakened Buddha! 

Homage\marginnote{3.1} to that Blessed One, the perfected one, the fully awakened Buddha! 

Homage\marginnote{4.1} to that Blessed One, the perfected one, the fully awakened Buddha!” 

When\marginnote{5.1} she said this, the brahmin said to \textsanskrit{Dhanañjānī}: 

“That’d\marginnote{5.2} be right. For the slightest thing this lowlife woman spouts out praise for that bald ascetic. Right now, lowlife woman, I’m going to refute your teacher’s doctrine!” 

“Brahmin,\marginnote{5.4} I don’t see anyone in this world—with its gods, \textsanskrit{Māras}, and \textsanskrit{Brahmās}, this population with its ascetics and brahmins, its gods and humans—who can refute the doctrine of the Blessed One, the perfected one, the fully awakened Buddha. But anyway, you should go. When you’ve gone you’ll understand.” 

Then\marginnote{6.1} the brahmin of the \textsanskrit{Bhāradvāja} clan, angry and upset, went to the Buddha and exchanged greetings with him. When the greetings and polite conversation were over, he sat down to one side, and addressed the Buddha in verse: 

\begin{verse}%
“When\marginnote{7.1} what is incinerated do you sleep at ease? \\
When what is incinerated is there no sorrow? \\
What is the one thing \\
whose killing you approve?” 

“When\marginnote{8.1} anger’s incinerated you sleep at ease. \\
When anger’s incinerated there is no sorrow. \\
O brahmin, anger has a poisonous root \\
and a honey tip. \\
The noble ones praise its killing, \\
for when it’s incinerated there is no sorrow.” 

%
\end{verse}

When\marginnote{9.1} he said this, the brahmin said to the Buddha, “Excellent, Master Gotama! Excellent! As if he were righting the overturned, or revealing the hidden, or pointing out the path to the lost, or lighting a lamp in the dark so people with good eyes can see what’s there, Master Gotama has made the teaching clear in many ways. I go for refuge to Master Gotama, to the teaching, and to the mendicant \textsanskrit{Saṅgha}. Sir, may I receive the going forth, the ordination in the Buddha’s presence?” 

And\marginnote{10.1} the brahmin received the going forth, the ordination in the Buddha’s presence. Not long after his ordination, Venerable \textsanskrit{Bhāradvāja}, living alone, withdrawn, diligent, keen, and resolute, soon realized the supreme end of the spiritual path in this very life. He lived having achieved with his own insight the goal for which gentlemen rightly go forth from the lay life to homelessness. 

He\marginnote{10.3} understood: “Rebirth is ended; the spiritual journey has been completed; what had to be done has been done; there is no return to any state of existence.” And Venerable \textsanskrit{Bhāradvāja} became one of the perfected. 

%
\section*{{\suttatitleacronym SN 7.2}{\suttatitletranslation The Abuser }{\suttatitleroot Akkosasutta}}
\addcontentsline{toc}{section}{\tocacronym{SN 7.2} \toctranslation{The Abuser } \tocroot{Akkosasutta}}
\markboth{The Abuser }{Akkosasutta}
\extramarks{SN 7.2}{SN 7.2}

At\marginnote{1.1} one time the Buddha was staying near \textsanskrit{Rājagaha}, in the Bamboo Grove, the squirrels’ feeding ground. 

The\marginnote{1.2} brahmin \textsanskrit{Bhāradvāja} the Rude heard a rumor that a brahmin of the \textsanskrit{Bhāradvāja} clan had gone forth from the lay life to homelessness in the presence of the ascetic Gotama. Angry and displeased he went to the Buddha and abused and insulted him with rude, harsh words. When he had spoken, the Buddha said to him: 

“What\marginnote{2.2} do you think, brahmin? Do friends and colleagues, relatives and family members, and guests still come to visit you?” 

“Sometimes\marginnote{2.4} they do, Master Gotama.” 

“Do\marginnote{2.5} you then serve them with a variety of foods and savories?” 

“Sometimes\marginnote{2.6} I do.” 

“But\marginnote{2.7} if they don’t accept it, brahmin, who does it belong to?” 

“In\marginnote{2.8} that case it still belongs to me.” 

“In\marginnote{2.9} the same way, brahmin, when you abuse, harass, and attack us who do not abuse, harass, and attack, we don’t accept it. It still belongs to you, brahmin, it still belongs to you! 

Someone\marginnote{3.1} who, when abused, harassed, and attacked, abuses, harasses, and attacks in return is said to eat the food and have a reaction to it. But we neither eat your food nor do we have a reaction to it. It still belongs to you, brahmin, it still belongs to you!” 

“The\marginnote{3.5} king and his retinue believe that Master Gotama is a perfected one. And yet he still gets angry.” 

\begin{verse}%
“For\marginnote{4.1} one free of anger, tamed, living in balance, \\
freed by right knowledge, \\
a poised one who is at peace: \\
where would anger come from? 

When\marginnote{5.1} you get angry at an angry person \\
you just make things worse for yourself. \\
When you don’t get angry at an angry person \\
you win a battle hard to win. 

When\marginnote{6.1} you know that the other is angry, \\
you act for the good of both \\
yourself and the other \\
if you’re mindful and stay calm. 

People\marginnote{7.1} unfamiliar with the teaching \\
consider one who heals both \\
oneself and the other \\
to be a fool.” 

%
\end{verse}

When\marginnote{8.1} he had spoken, \textsanskrit{Bhāradvāja} the Rude said to the Buddha, “Excellent, Master Gotama! … I go for refuge to Master Gotama, to the teaching, and to the mendicant \textsanskrit{Saṅgha}. Sir, may I receive the going forth, the ordination in the Buddha’s presence?” 

And\marginnote{9.1} the brahmin \textsanskrit{Bhāradvāja} the Rude received the going forth, the ordination in the Buddha’s presence. Not long after his ordination, Venerable \textsanskrit{Bhāradvāja} the Rude, living alone, withdrawn, diligent, keen, and resolute, soon realized the supreme end of the spiritual path in this very life. He lived having achieved with his own insight the goal for which gentlemen rightly go forth from the lay life to homelessness. 

He\marginnote{9.3} understood: “Rebirth is ended; the spiritual journey has been completed; what had to be done has been done; there is no return to any state of existence.” And Venerable \textsanskrit{Bhāradvāja} became one of the perfected. 

%
\section*{{\suttatitleacronym SN 7.3}{\suttatitletranslation With Bhāradvāja the Fiend }{\suttatitleroot Asurindakasutta}}
\addcontentsline{toc}{section}{\tocacronym{SN 7.3} \toctranslation{With Bhāradvāja the Fiend } \tocroot{Asurindakasutta}}
\markboth{With Bhāradvāja the Fiend }{Asurindakasutta}
\extramarks{SN 7.3}{SN 7.3}

At\marginnote{1.1} one time the Buddha was staying near \textsanskrit{Rājagaha}, in the Bamboo Grove, the squirrels’ feeding ground. The brahmin \textsanskrit{Bhāradvāja} the Fiend heard a rumor to the effect that a brahmin of the \textsanskrit{Bhāradvāja} clan had gone forth from the lay life to homelessness in the presence of the ascetic Gotama. Angry and displeased he went to the Buddha and abused and insulted him with rude, harsh words. 

But\marginnote{1.4} when he said this, the Buddha kept silent. 

Then\marginnote{1.5} \textsanskrit{Bhāradvāja} the Fiend said to the Buddha, “You’re beaten, ascetic, you’re beaten!” 

\begin{verse}%
“‘Ha!\marginnote{2.1} I won!’ thinks the fool, \\
when speaking with harsh words. \\
Patience is the true victory \\
for those who understand. 

When\marginnote{3.1} you get angry at an angry person \\
you just make things worse for yourself. \\
When you don’t get angry at an angry person \\
you win a battle hard to win. 

When\marginnote{4.1} you know that the other is angry, \\
you act for the good of both \\
yourself and the other \\
if you’re mindful and stay calm. 

People\marginnote{5.1} unskilled in Dhamma \\
consider one who heals both \\
oneself and the other \\
to be a fool.” 

%
\end{verse}

When\marginnote{6.1} he had spoken, \textsanskrit{Bhāradvāja} the Fiend said to the Buddha, “Excellent, Master Gotama! …” … And Venerable \textsanskrit{Bhāradvāja} became one of the perfected. 

%
\section*{{\suttatitleacronym SN 7.4}{\suttatitletranslation With Bhāradvāja the Bitter }{\suttatitleroot Bilaṅgikasutta}}
\addcontentsline{toc}{section}{\tocacronym{SN 7.4} \toctranslation{With Bhāradvāja the Bitter } \tocroot{Bilaṅgikasutta}}
\markboth{With Bhāradvāja the Bitter }{Bilaṅgikasutta}
\extramarks{SN 7.4}{SN 7.4}

At\marginnote{1.1} one time the Buddha was staying near \textsanskrit{Rājagaha}, in the Bamboo Grove, the squirrels’ feeding ground. The brahmin \textsanskrit{Bhāradvāja} the Bitter heard a rumor that a brahmin of the \textsanskrit{Bhāradvāja} clan had gone forth from the lay life to homelessness in the presence of the ascetic Gotama. Angry and displeased he went to the Buddha and stood silently to one side. 

Then\marginnote{1.4} the Buddha, knowing what \textsanskrit{Bhāradvāja} the Bitter was thinking, addressed him in verse: 

\begin{verse}%
“Whoever\marginnote{2.1} wrongs a man who has done no wrong, \\
a pure man who has not a blemish, \\
the evil backfires on the fool, \\
like fine dust thrown upwind.” 

%
\end{verse}

When\marginnote{3.1} he said this, the brahmin \textsanskrit{Bhāradvāja} the Bitter said to the Buddha, “Excellent, Master Gotama! …” … And Venerable \textsanskrit{Bhāradvāja} became one of the perfected. 

%
\section*{{\suttatitleacronym SN 7.5}{\suttatitletranslation Harmless }{\suttatitleroot Ahiṁsakasutta}}
\addcontentsline{toc}{section}{\tocacronym{SN 7.5} \toctranslation{Harmless } \tocroot{Ahiṁsakasutta}}
\markboth{Harmless }{Ahiṁsakasutta}
\extramarks{SN 7.5}{SN 7.5}

At\marginnote{1.1} \textsanskrit{Sāvatthī}. 

Then\marginnote{1.2} the brahmin \textsanskrit{Bhāradvāja} the Harmless went up to the Buddha, and exchanged greetings with him. 

When\marginnote{1.3} the greetings and polite conversation were over, he sat down to one side and said, “I am Harmless, Master Gotama, I am Harmless!” 

\begin{verse}%
“If\marginnote{2.1} you were really like your name, \\
then you’d be Harmless. \\
But a truly harmless person \\
does no harm by way of \\
body, speech, or mind; \\
they don’t harm anyone else.” 

%
\end{verse}

When\marginnote{3.1} he had spoken, the brahmin \textsanskrit{Bhāradvāja} the Harmless said to the Buddha, “Excellent, Master Gotama! …” … And Venerable \textsanskrit{Bhāradvāja} the Harmless became one of the perfected. 

%
\section*{{\suttatitleacronym SN 7.6}{\suttatitletranslation With Bhāradvāja of the Matted Hair }{\suttatitleroot Jaṭāsutta}}
\addcontentsline{toc}{section}{\tocacronym{SN 7.6} \toctranslation{With Bhāradvāja of the Matted Hair } \tocroot{Jaṭāsutta}}
\markboth{With Bhāradvāja of the Matted Hair }{Jaṭāsutta}
\extramarks{SN 7.6}{SN 7.6}

At\marginnote{1.1} \textsanskrit{Sāvatthī}. 

Then\marginnote{1.2} the brahmin \textsanskrit{Bhāradvāja} of the Matted Hair went up to the Buddha, and exchanged greetings with him. 

When\marginnote{1.3} the greetings and polite conversation were over, he sat down to one side, and addressed the Buddha in verse: 

\begin{verse}%
“Tangled\marginnote{2.1} within, tangled without: \\
these people are tangled in tangles. \\
I ask you this, Gotama: \\
who can untangle this tangle?” 

“A\marginnote{3.1} wise person grounded in ethics, \\
developing the mind and wisdom, \\
a keen and alert mendicant—\\
they can untangle this tangle. 

Those\marginnote{4.1} in whom greed, hate, and ignorance \\
have faded away; \\
the perfected ones with defilements ended—\\
they have untangled the tangle. 

Where\marginnote{5.1} name and form \\
cease with nothing left over; \\
as well as impingement and perception of form: \\
it’s there that the tangle is cut.” 

%
\end{verse}

When\marginnote{6.1} he had spoken, \textsanskrit{Bhāradvāja} of the Matted Hair said to the Buddha, “Excellent, Master Gotama! …” … And Venerable \textsanskrit{Bhāradvāja} became one of the perfected. 

%
\section*{{\suttatitleacronym SN 7.7}{\suttatitletranslation With Bhāradvāja the Pure }{\suttatitleroot Suddhikasutta}}
\addcontentsline{toc}{section}{\tocacronym{SN 7.7} \toctranslation{With Bhāradvāja the Pure } \tocroot{Suddhikasutta}}
\markboth{With Bhāradvāja the Pure }{Suddhikasutta}
\extramarks{SN 7.7}{SN 7.7}

At\marginnote{1.1} \textsanskrit{Sāvatthī}. 

Then\marginnote{1.2} the brahmin \textsanskrit{Bhāradvāja} the Pure went up to the Buddha, and exchanged greetings with him. 

When\marginnote{1.3} the greetings and polite conversation were over, he sat down to one side, and recited this verse in his presence: 

\begin{verse}%
“No\marginnote{2.1} brahmin in the world is ever purified \\
even though he’s ethical and mortifies himself. \\
But one accomplished in knowledge and conduct \\
is purified, not these commoners.” 

“Even\marginnote{3.1} one who mutters many prayers \\
is no brahmin by birth \\
if they’re filthy and corrupt within, \\
supporting themselves by fraud. 

Regardless\marginnote{4.1} of whether you’re an aristocrat, \\
a brahmin, merchant, worker, or an outcaste or scavenger—\\
if you’re energetic and resolute, \\
always staunchly vigorous, \\
you’ll attain ultimate purity. \\
Know that for a fact, brahmin.” 

%
\end{verse}

When\marginnote{5.1} he had spoken, the brahmin \textsanskrit{Bhāradvāja} the Pure said to the Buddha, “Excellent, Master Gotama …” … And Venerable \textsanskrit{Bhāradvāja} became one of the perfected. 

%
\section*{{\suttatitleacronym SN 7.8}{\suttatitletranslation With Bhāradvāja the Fire-Worshiper }{\suttatitleroot Aggikasutta}}
\addcontentsline{toc}{section}{\tocacronym{SN 7.8} \toctranslation{With Bhāradvāja the Fire-Worshiper } \tocroot{Aggikasutta}}
\markboth{With Bhāradvāja the Fire-Worshiper }{Aggikasutta}
\extramarks{SN 7.8}{SN 7.8}

At\marginnote{1.1} one time the Buddha was staying near \textsanskrit{Rājagaha}, in the Bamboo Grove, the squirrels’ feeding ground. 

Now\marginnote{1.2} at that time ghee and milk-rice had been set out for the brahmin \textsanskrit{Bhāradvāja} the Fire-Worshiper, who thought, “I will serve the sacred flame! I will perform the fire sacrifice!” 

Then\marginnote{2.1} the Buddha robed up in the morning and, taking his bowl and robe, entered \textsanskrit{Rājagaha} for alms. Wandering indiscriminately for almsfood in \textsanskrit{Rājagaha}, he approached \textsanskrit{Bhāradvāja} the Fire-Worshiper’s home and stood to one side. 

\textsanskrit{Bhāradvāja}\marginnote{2.3} the Fire-Worshiper saw him standing for alms and addressed him in verse: 

\begin{verse}%
“One\marginnote{3.1} who’s accomplished in the three knowledges, \\
of good lineage and ample learning, \\
accomplished in knowledge and conduct \\
may enjoy this milk-rice.” 

“Even\marginnote{4.1} one who mutters many prayers \\
is no brahmin by birth \\
if they’re filthy and corrupt within, \\
with a following gained by fraud. 

But\marginnote{5.1} one who knows their past lives, \\
and sees heaven and places of loss, \\
and has attained the ending of rebirth, \\
that sage has perfect insight. 

Because\marginnote{6.1} of these three knowledges \\
a brahmin is a master of the three knowledges. \\
Accomplished in knowledge and conduct, \\
they may enjoy this milk-rice.” 

%
\end{verse}

“Eat,\marginnote{7.1} Master Gotama! you are truly a brahmin.” 

\begin{verse}%
“Food\marginnote{8.1} enchanted by a spell isn’t fit for me to eat. \\
That’s not the principle of those who see, brahmin. \\
The Buddhas reject things enchanted with spells. \\
Since there is such a principle, brahmin, that’s how they live. 

Serve\marginnote{9.1} with other food and drink \\
the consummate one, the great hermit, \\
with defilements ended and remorse stilled. \\
For he is the field for the seeker of merit.” 

%
\end{verse}

When\marginnote{10.1} he had spoken, the brahmin \textsanskrit{Bhāradvāja} the Fire-Worshiper said to the Buddha, “Excellent, Master Gotama! …” … And Venerable \textsanskrit{Bhāradvāja} the Fire-Worshiper became one of the perfected. 

%
\section*{{\suttatitleacronym SN 7.9}{\suttatitletranslation With Sundarikabhāradvāja }{\suttatitleroot Sundarikasutta}}
\addcontentsline{toc}{section}{\tocacronym{SN 7.9} \toctranslation{With Sundarikabhāradvāja } \tocroot{Sundarikasutta}}
\markboth{With Sundarikabhāradvāja }{Sundarikasutta}
\extramarks{SN 7.9}{SN 7.9}

At\marginnote{1.1} one time the Buddha was staying in the Kosalan lands on the bank of the \textsanskrit{Sundarikā} river. 

Now\marginnote{1.2} at that time the brahmin \textsanskrit{Sundarikabhāradvāja} was serving the sacred flame and performing the fire sacrifice on the bank of the \textsanskrit{Sundarikā}. 

Then\marginnote{1.3} he looked all around the four quarters, wondering, “Now who might eat the leftovers of this offering?” 

He\marginnote{1.5} saw the Buddha meditating at the root of a certain tree with his robe pulled over his head. Taking the leftovers of the offering in his left hand and a pitcher in the right he approached the Buddha. When he heard \textsanskrit{Sundarikabhāradvāja}’s footsteps the Buddha uncovered his head. 

\textsanskrit{Sundarikabhāradvāja}\marginnote{1.8} thought, “This man is shaven, he is shaven!” And he wanted to turn back. 

But\marginnote{1.9} he thought, “Even some brahmins are shaven. Why don’t I go to him and ask about his birth?” 

Then\marginnote{2.1} the brahmin \textsanskrit{Sundarikabhāradvāja} went up to the Buddha, and said to him, “Sir, in what caste were you born?” 

\begin{verse}%
“Don’t\marginnote{3.1} ask about birth, ask about conduct; \\
for any wood can surely generate fire. \\
A steadfast sage, even though from a low class family, \\
is a thoroughbred checked by conscience. 

Tamed\marginnote{4.1} by truth, fulfilled by taming, \\
a complete knowledge master who has completed the spiritual journey—\\
that’s who a sacrificer should introduce themselves to, \\
and make a timely offering to one worthy of a religious donation.” 

“My\marginnote{5.1} sacrificial offering must have been well performed, \\
since I have met such a knowledge master! \\
It’s because I’d never met anyone like you \\
that others ate the leftover offering. 

%
\end{verse}

Eat,\marginnote{6.1} Master Gotama, you are truly a brahmin.” 

\begin{verse}%
“Food\marginnote{7.1} enchanted by a spell isn’t fit for me to eat. \\
That’s not the principle of those who see, brahmin. \\
The Buddhas reject things enchanted with spells. \\
Since there is such a principle, brahmin, that’s how they live. 

Serve\marginnote{8.1} with other food and drink \\
the consummate one, the great hermit, \\
with defilements ended and remorse stilled. \\
For he is the field for the seeker of merit.” 

%
\end{verse}

“Then,\marginnote{9.1} Master Gotama, to whom should I give the leftovers of this offering?” 

“Brahmin,\marginnote{9.2} I don’t see anyone in this world—with its gods, \textsanskrit{Māras}, and \textsanskrit{Brahmās}, this population with its ascetics and brahmins, its gods and humans—who can properly digest these leftovers, except for the Realized One or one of his disciples. Well then, brahmin, throw out those leftovers where there is little that grows, or drop them into water that has no living creatures.” 

So\marginnote{10.1} \textsanskrit{Sundarikabhāradvāja} dropped the leftover offering in water that had no living creatures. And when those leftovers were placed in the water, they sizzled and hissed, steaming and fuming. Suppose there was an iron cauldron that had been heated all day. If you placed it in the water, it would sizzle and hiss, steaming and fuming. In the same way, when those leftovers were placed in the water, they sizzled and hissed, steaming and fuming. 

Then\marginnote{11.1} the brahmin \textsanskrit{Sundarikabhāradvāja}, shocked and awestruck, went up to the Buddha, and stood to one side. The Buddha addressed him in verse: 

\begin{verse}%
“When\marginnote{12.1} you’re kindling the wood, brahmin, \\
don’t imagine this is purity, for it’s just an external. \\
For experts say this is no way to purity, \\
when one seeks purity in externals. 

I’ve\marginnote{13.1} given up kindling firewood, brahmin, \\
now I just light the inner flame. \\
Always blazing, always serene, \\
I am a perfected one leading the spiritual life. 

Conceit,\marginnote{14.1} brahmin, is the burden of your possessions, \\
anger your smoke, and lies your ashes. \\
The tongue is the ladle and the heart the fire altar; \\
a well-tamed self is a person’s light. 

The\marginnote{15.1} teaching is a lake with shores of ethics, brahmin, \\
unclouded, praised by the fine to the good. \\
There the knowledge masters go to bathe, \\
and cross to the far shore without getting wet. 

Truth,\marginnote{16.1} principle, restraint, the spiritual life; \\
the attainment of the supreme based on the middle, brahmin. \\
Pay homage to the upright ones—\\
I declare that man to be one who follows the teaching.” 

%
\end{verse}

When\marginnote{17.1} he had spoken, the brahmin \textsanskrit{Sundarikabhāradvāja} said to the Buddha, “Excellent, Master Gotama …” … And Venerable \textsanskrit{Bhāradvāja} became one of the perfected. 

%
\section*{{\suttatitleacronym SN 7.10}{\suttatitletranslation Many Daughters }{\suttatitleroot Bahudhītarasutta}}
\addcontentsline{toc}{section}{\tocacronym{SN 7.10} \toctranslation{Many Daughters } \tocroot{Bahudhītarasutta}}
\markboth{Many Daughters }{Bahudhītarasutta}
\extramarks{SN 7.10}{SN 7.10}

At\marginnote{1.1} one time the Buddha was staying in the land of the Kosalans in a certain forest grove. 

Now\marginnote{1.2} at that time one of the brahmins of the \textsanskrit{Bhāradvāja} clan had lost fourteen oxen. While looking for them he went to that forest, where he saw the Buddha sitting down cross-legged, with his body straight, and mindfulness established right there. He went up to the Buddha, and recited these verses in the Buddha’s presence: 

\begin{verse}%
“This\marginnote{2.1} ascetic mustn’t have \\
fourteen oxen \\
missing for the past six days: \\
that’s why this ascetic is happy. 

This\marginnote{3.1} ascetic mustn’t have \\
a field of sesame ruined, \\
with just one or two leaves: \\
that’s why this ascetic is happy. 

This\marginnote{4.1} ascetic mustn’t have \\
rats in a vacant barn \\
dancing merrily: \\
that’s why this ascetic is happy. 

This\marginnote{5.1} ascetic mustn’t have \\
carpets that for seven months \\
have been infested with fleas: \\
that’s why this ascetic is happy. 

This\marginnote{6.1} ascetic mustn’t have \\
seven widowed daughters \\
with one or two children each: \\
that’s why this ascetic is happy. 

This\marginnote{7.1} ascetic mustn’t have \\
a wife with sallow, blotchy skin \\
to wake him with a kick: \\
that’s why this ascetic is happy. 

This\marginnote{8.1} ascetic mustn’t have \\
creditors knocking at dawn, \\
warning, ‘Pay up! Pay up!’: \\
that’s why this ascetic is happy.” 

“You’re\marginnote{9.1} right, brahmin, I don’t have \\
fourteen oxen \\
missing for the past six days: \\
that’s why I’m happy, brahmin. 

You’re\marginnote{10.1} right, brahmin, I don’t have \\
a field of sesame ruined, \\
with just one or two leaves: \\
that’s why I’m happy, brahmin. 

You’re\marginnote{11.1} right, brahmin, I don’t have \\
rats in a vacant barn \\
dancing merrily: \\
that’s why I’m happy, brahmin. 

You’re\marginnote{12.1} right, brahmin, I don’t have \\
carpets that for seven months \\
have been infested with fleas: \\
that’s why I’m happy, brahmin. 

You’re\marginnote{13.1} right, brahmin, I don’t have \\
seven widowed daughters \\
with one or two children each: \\
that’s why I’m happy, brahmin. 

You’re\marginnote{14.1} right, brahmin, I don’t have \\
a wife with blotchy, pockmarked skin \\
to wake me up with a kick: \\
that’s why I’m happy, brahmin. 

You’re\marginnote{15.1} right, brahmin, I don’t have \\
creditors knocking at dawn, \\
warning, ‘Pay up! Pay up!’: \\
that’s why I’m happy, brahmin.” 

%
\end{verse}

When\marginnote{16.1} he had spoken, the brahmin said to the Buddha, “Excellent, Master Gotama! Excellent! … As if he were righting the overturned, or revealing the hidden, or pointing out the path to the lost, or lighting a lamp in the dark so people with good eyes can see what’s there, Master Gotama has made the teaching clear in many ways. I go for refuge to Master Gotama, to the teaching, and to the mendicant \textsanskrit{Saṅgha}. Sir, may I receive the going forth, the ordination in the Buddha’s presence?” 

And\marginnote{17.1} the brahmin received the going forth, the ordination in the Buddha’s presence. Not long after his ordination, Venerable \textsanskrit{Bhāradvāja}, living alone, withdrawn, diligent, keen, and resolute, soon realized the supreme end of the spiritual path in this very life. He lived having achieved with his own insight the goal for which gentlemen rightly go forth from the lay life to homelessness. 

He\marginnote{17.3} understood: “Rebirth is ended; the spiritual journey has been completed; what had to be done has been done; there is no return to any state of existence.” And Venerable \textsanskrit{Bhāradvāja} became one of the perfected. 

%
\addtocontents{toc}{\let\protect\contentsline\protect\nopagecontentsline}
\chapter*{The Chapter on a Lay Follower }
\addcontentsline{toc}{chapter}{\tocchapterline{The Chapter on a Lay Follower }}
\addtocontents{toc}{\let\protect\contentsline\protect\oldcontentsline}

%
\section*{{\suttatitleacronym SN 7.11}{\suttatitletranslation With Bhāradvāja the Farmer }{\suttatitleroot Kasibhāradvājasutta}}
\addcontentsline{toc}{section}{\tocacronym{SN 7.11} \toctranslation{With Bhāradvāja the Farmer } \tocroot{Kasibhāradvājasutta}}
\markboth{With Bhāradvāja the Farmer }{Kasibhāradvājasutta}
\extramarks{SN 7.11}{SN 7.11}

\scevam{So\marginnote{1.1} I have heard. }At one time the Buddha was staying in the land of the Magadhans in the Southern Hills near the brahmin village of \textsanskrit{Ekanāḷa}. 

Now\marginnote{1.3} at that time the brahmin \textsanskrit{Bhāradvāja} the Farmer had harnessed around five hundred plows, it being the season for sowing. Then the Buddha robed up in the morning and, taking his bowl and robe, went to where \textsanskrit{Bhāradvāja} the Farmer was working. 

Now\marginnote{2.1} at that time \textsanskrit{Bhāradvāja} the Farmer was distributing food. Then the Buddha went to where the distribution was taking place and stood to one side. 

\textsanskrit{Bhāradvāja}\marginnote{2.3} the Farmer saw him standing for alms and said to him, “I plough and sow, ascetic, and then I eat. You too should plough and sow, then you may eat.” 

“I\marginnote{2.7} too plough and sow, brahmin, and then I eat.” 

“I\marginnote{2.8} don’t see Master Gotama with a yoke or plow or plowshare or goad or oxen, yet he says: ‘I too plough and sow, brahmin, and then I eat.’” 

Then\marginnote{2.10} \textsanskrit{Bhāradvāja} the Farmer addressed the Buddha in verse: 

\begin{verse}%
“You\marginnote{3.1} claim to be a farmer, \\
but I don’t see you farming. \\
Tell me how you’re a farmer when asked: \\
how am I to recognize your farming?” 

“Faith\marginnote{4.1} is my seed, austerity my rain, \\
and wisdom is my yoke and plough. \\
Conscience is my pole, mind my strap, \\
mindfulness my plowshare and goad. 

Guarded\marginnote{5.1} in body and speech, \\
I restrict my intake of food. \\
I use truth as my scythe, \\
and gentleness is my release. 

Energy\marginnote{6.1} is my beast of burden, \\
transporting me to a place of sanctuary. \\
It goes without turning back \\
to where there is no sorrow. 

That’s\marginnote{7.1} how to do the farming \\
that has the Deathless as its fruit. \\
When you finish this farming \\
you’re released from all suffering.” 

%
\end{verse}

“Eat,\marginnote{8.1} Master Gotama, you are truly a farmer. For Master Gotama does the farming that has the Deathless as its fruit.” 

\begin{verse}%
“Food\marginnote{9.1} enchanted by a spell isn’t fit for me to eat. \\
That’s not the principle of those who see, brahmin. \\
The Buddhas reject things enchanted with spells. \\
Since there is such a principle, brahmin, that’s how they live. 

Serve\marginnote{10.1} with other food and drink \\
the consummate one, the great hermit, \\
with defilements ended and remorse stilled. \\
For he is the field for the seeker of merit.” 

%
\end{verse}

When\marginnote{11.1} he had spoken, the brahmin \textsanskrit{Bhāradvāja} the Farmer said to the Buddha, “Excellent, Master Gotama … From this day forth, may Master Gotama remember me as a lay follower who has gone for refuge for life.” 

%
\section*{{\suttatitleacronym SN 7.12}{\suttatitletranslation With Udaya }{\suttatitleroot Udayasutta}}
\addcontentsline{toc}{section}{\tocacronym{SN 7.12} \toctranslation{With Udaya } \tocroot{Udayasutta}}
\markboth{With Udaya }{Udayasutta}
\extramarks{SN 7.12}{SN 7.12}

At\marginnote{1.1} \textsanskrit{Sāvatthī}. 

Then\marginnote{1.2} the Buddha robed up in the morning and, taking his bowl and robe, went to the home of the brahmin Udaya. Then Udaya filled the Buddha’s bowl with rice. The next day … and the day after that … Udaya filled the Buddha’s bowl with rice. 

But\marginnote{1.5} when he had filled the Buddha’s bowl for a third time, he said to the Buddha, “This insatiable ascetic Gotama keeps coming back again and again!” 

\begin{verse}%
“Again\marginnote{2.1} and again, they sow the seed; \\
again and again, the lord god sends rain; \\
again and again, farmers plough the field; \\
again and again, grain is produced for the nation. 

Again\marginnote{3.1} and again, the beggars beg; \\
again and again, the donors give. \\
Again and again, when the donors have given, \\
again and again, they take their place in heaven. 

Again\marginnote{4.1} and again, dairy farmers milk; \\
again and again, a calf cleaves to its mother; \\
again and again, oppressing and intimidating; \\
that idiot is reborn again and again. 

Again\marginnote{5.1} and again, you’re reborn and die; \\
again and again, you get carried to a charnel ground. \\
But when they’ve gained the path for no further rebirth, \\
one of vast wisdom is not reborn again and again.” 

%
\end{verse}

When\marginnote{6.1} he had spoken, the brahmin Udaya said to the Buddha, “Excellent, Master Gotama … From this day forth, may Master Gotama remember me as a lay follower who has gone for refuge for life.” 

%
\section*{{\suttatitleacronym SN 7.13}{\suttatitletranslation With Devahita }{\suttatitleroot Devahitasutta}}
\addcontentsline{toc}{section}{\tocacronym{SN 7.13} \toctranslation{With Devahita } \tocroot{Devahitasutta}}
\markboth{With Devahita }{Devahitasutta}
\extramarks{SN 7.13}{SN 7.13}

At\marginnote{1.1} \textsanskrit{Sāvatthī}. 

Now\marginnote{1.2} at that time the Buddha was afflicted by winds. Venerable \textsanskrit{Upavāṇa} was his carer. 

Then\marginnote{1.4} the Buddha said to \textsanskrit{Upavāṇa}, “Please, \textsanskrit{Upavāṇa}, find some hot water for me.” 

“Yes,\marginnote{1.6} sir,” replied \textsanskrit{Upavāṇa}. He robed up, and, taking his bowl and robe, went to the home of the brahmin Devahita, and stood silently to one side. 

Devahita\marginnote{1.7} saw him standing there and addressed him in verse: 

\begin{verse}%
“Silent\marginnote{2.1} stands the Master, \\
shaven, wrapped in his outer robe. \\
What do you want? What are you looking for? \\
What have you come here to ask for?” 

“The\marginnote{3.1} perfected one, the Holy One in the world, \\
the sage is afflicted by winds. \\
If there’s hot water, \\
give it to the sage, brahmin. 

I\marginnote{4.1} wish to bring it to the one \\
who is esteemed by the estimable, \\
honored by the honorable, \\
venerated by the venerable.” 

%
\end{verse}

Then\marginnote{5.1} Devahita had a man fetch a carrying-pole with hot water. He also presented \textsanskrit{Upavāṇa} with a jar of molasses. 

Then\marginnote{5.2} Venerable \textsanskrit{Upavāṇa} went up to the Buddha and bathed him with the hot water. Then he stirred molasses into hot water and presented it to the Buddha. Then the Buddha’s illness died down. 

Then\marginnote{6.1} the brahmin Devahita went up to the Buddha, and exchanged greetings with him. When the greetings and polite conversation were over, he sat down to one side, and addressed the Buddha in verse: 

\begin{verse}%
“Where\marginnote{7.1} should you give an available gift? \\
Where is a gift very fruitful? \\
How does one who is donating \\
ensure the success of their religious donation?” 

“One\marginnote{8.1} who knows their past lives, \\
and sees heaven and places of loss, \\
and has attained the ending of rebirth, \\
that sage has perfect insight. 

Here\marginnote{9.1} you should give an available gift; \\
what’s given here is very fruitful. \\
That’s how a sponsor of sacrifices \\
ensures the success of their religious donation.” 

%
\end{verse}

When\marginnote{10.1} he had spoken, the brahmin Devahita said to the Buddha, “Excellent, Master Gotama … From this day forth, may Master Gotama remember me as a lay follower who has gone for refuge for life.” 

%
\section*{{\suttatitleacronym SN 7.14}{\suttatitletranslation A well-to-do brahmin }{\suttatitleroot Mahāsālasutta}}
\addcontentsline{toc}{section}{\tocacronym{SN 7.14} \toctranslation{A well-to-do brahmin } \tocroot{Mahāsālasutta}}
\markboth{A well-to-do brahmin }{Mahāsālasutta}
\extramarks{SN 7.14}{SN 7.14}

At\marginnote{1.1} \textsanskrit{Sāvatthī}. 

Then\marginnote{1.2} a certain well-to-do brahmin, shabby, wearing a shabby cloak, went up to the Buddha, and exchanged greetings with him. 

When\marginnote{1.3} the greetings and polite conversation were over, he sat down to one side. The Buddha said to him, “Brahmin, why are you so shabby, wearing a shabby cloak?” 

“Master\marginnote{1.5} Gotama, I have four sons. At their wives’ bidding they expelled me from my house.” 

“Well\marginnote{1.7} then, brahmin, memorize these verses and recite them to your sons when you are all seated in the council hall with a large crowd. 

\begin{verse}%
‘I\marginnote{2.1} was overjoyed when they were born, \\
and wished for them the very best. \\
But at their wives’ bidding they chased me out, \\
like hounds after hogs. 

It\marginnote{3.1} turns out they’re wicked, those nasty men, \\
though they called me their dear old Dad. \\
They’re monsters in the shape of sons, \\
throwing me out as I’ve grown old. 

Like\marginnote{4.1} an old, useless horse \\
led away from its fodder, \\
the elderly father of those kids \\
begs for alms at others’ homes. 

Even\marginnote{5.1} my staff is better \\
than those disobedient sons, \\
for it wards off a wild bull, \\
and even a wild dog. 

It\marginnote{6.1} goes before me in the dark; \\
in deep waters it supports me. \\
By the wonderful power of this staff, \\
when I stumble, I stand firm again.’” 

%
\end{verse}

Having\marginnote{7.1} memorized those verses in the Buddha’s presence, the brahmin recited them to his sons when they were all seated in the council hall with a large crowd. … 

Then\marginnote{13.1} the brahmin’s sons led him back home, bathed him, and each clothed him with a fine pair of garments. Then the brahmin, taking one pair of garments, went to the Buddha and exchanged greetings with him. When the greetings and polite conversation were over, he sat down to one side and said to the Buddha: 

“Master\marginnote{13.4} Gotama, we brahmins seek a fee for our teacher. May Master Gotama please accept my teacher’s fee!” So the Buddha accepted it out of compassion. Then the well-to-do brahmin said to the Buddha, “Excellent, Master Gotama … From this day forth, may Master Gotama remember me as a lay follower who has gone for refuge for life.” 

%
\section*{{\suttatitleacronym SN 7.15}{\suttatitletranslation Stuck-Up }{\suttatitleroot Mānatthaddhasutta}}
\addcontentsline{toc}{section}{\tocacronym{SN 7.15} \toctranslation{Stuck-Up } \tocroot{Mānatthaddhasutta}}
\markboth{Stuck-Up }{Mānatthaddhasutta}
\extramarks{SN 7.15}{SN 7.15}

At\marginnote{1.1} \textsanskrit{Sāvatthī}. 

Now\marginnote{1.2} at that time a brahmin named Stuck-Up was residing in \textsanskrit{Sāvatthī}. He didn’t bow to his mother or father, his teacher, or his oldest brother. Now, at that time the Buddha was teaching Dhamma, surrounded by a large assembly. 

Then\marginnote{1.5} Stuck-Up thought, “The ascetic Gotama is teaching Dhamma, surrounded by a large assembly. Why don’t I approach him? If he speaks to me, I’ll speak to him. But if he doesn’t speak, neither will I.” 

Then\marginnote{1.10} the brahmin Stuck-Up went up to the Buddha, and stood silently to one side. But the Buddha didn’t speak to him. 

Then\marginnote{1.12} Stuck-Up thought, “This ascetic Gotama knows nothing!” And he wanted to go back from there right away. 

Then\marginnote{1.14} the Buddha, knowing what Stuck-Up was thinking, addressed him in verse: 

\begin{verse}%
“It’s\marginnote{2.1} not good to foster conceit \\
if you want what’s good for you, brahmin. \\
You should foster the goal \\
which brought you here.” 

%
\end{verse}

Then\marginnote{3.1} Stuck-Up thought, “The ascetic Gotama knows my mind!” He bowed with his head at the Buddha’s feet, caressing them and covering them with kisses, and pronounced his name: “Master Gotama, I am Stuck-Up! I am Stuck-Up!” 

Then\marginnote{3.4} that assembly was stunned: “It’s incredible, it’s amazing! This brahmin Stuck-Up doesn’t bow to his mother or father, his teacher, or his oldest brother. Yet he shows such utmost devotion to the ascetic Gotama!” 

Then\marginnote{3.8} the Buddha said to the brahmin Stuck-Up, “Enough, brahmin. Get up, and take your own seat. For your mind has confidence in me.” 

Then\marginnote{3.11} Stuck-Up took his seat and said to the Buddha: 

\begin{verse}%
“Regarding\marginnote{4.1} whom should you not be conceited? \\
Who should you respect? \\
Who should you esteem? \\
Who is it good to venerate properly?” 

“Your\marginnote{5.1} mother and father, \\
and also your oldest brother, \\
with teacher as fourth. \\
Regarding these you should not be conceited. \\
They are who you should respect. \\
They are who you should esteem. \\
And they’re who it’s good to venerate properly. 

And\marginnote{6.1} when you’ve humbled conceit, and aren’t stuck-up, \\
show supreme reverence for \\
the perfected ones, cooled, \\
their task complete, free of defilements.” 

%
\end{verse}

When\marginnote{7.1} he had spoken, the brahmin Stuck-Up said to the Buddha, “Excellent, Master Gotama … From this day forth, may Master Gotama remember me as a lay follower who has gone for refuge for life.” 

%
\section*{{\suttatitleacronym SN 7.16}{\suttatitletranslation The Contraphile }{\suttatitleroot Paccanīkasutta}}
\addcontentsline{toc}{section}{\tocacronym{SN 7.16} \toctranslation{The Contraphile } \tocroot{Paccanīkasutta}}
\markboth{The Contraphile }{Paccanīkasutta}
\extramarks{SN 7.16}{SN 7.16}

At\marginnote{1.1} \textsanskrit{Sāvatthī}. 

Now\marginnote{1.2} at that time a brahmin named Contraphile, who loved contradiction, was residing in \textsanskrit{Sāvatthī}. 

Then\marginnote{1.3} Contraphile thought, “Why don’t I go to the ascetic Gotama and contradict everything he says?” 

At\marginnote{1.6} that time the Buddha was walking mindfully in the open air. 

Then\marginnote{1.7} the brahmin Contraphile went up to the Buddha, and said to him as he was walking, “Ascetic, preach the Dhamma.” 

\begin{verse}%
“Fine\marginnote{2.1} words aren’t easy to understand \\
by a lover of contradiction, \\
whose mind is tainted \\
and full of aggression. 

But\marginnote{3.1} when you’ve dispelled aggression, \\
and your suspicious mind, \\
and you’ve got rid of resentment, \\
then you’ll understand fine words.” 

%
\end{verse}

When\marginnote{4.1} he had spoken, the brahmin Contraphile said to the Buddha, “Excellent, Master Gotama! Excellent! … From this day forth, may Master Gotama remember me as a lay follower who has gone for refuge for life.” 

%
\section*{{\suttatitleacronym SN 7.17}{\suttatitletranslation The Builder }{\suttatitleroot Navakammikasutta}}
\addcontentsline{toc}{section}{\tocacronym{SN 7.17} \toctranslation{The Builder } \tocroot{Navakammikasutta}}
\markboth{The Builder }{Navakammikasutta}
\extramarks{SN 7.17}{SN 7.17}

At\marginnote{1.1} one time the Buddha was staying in the land of the Kosalans in a certain forest grove. 

Now\marginnote{1.2} at that time the brahmin \textsanskrit{Bhāradvāja} the Builder was doing some building work in that jungle thicket. He saw the Buddha sitting down cross-legged at the root of a certain sal tree, with his body straight, and mindfulness established right there. 

Seeing\marginnote{1.4} this, it occurred to him, “I enjoy doing this building work here in the jungle. I wonder what the ascetic Gotama enjoys doing?” 

Then\marginnote{1.7} \textsanskrit{Bhāradvāja} the Builder went up to the Buddha and addressed him in verse: 

\begin{verse}%
“What\marginnote{2.1} kind of work do you do \\
as a mendicant in the sal jungle? \\
How do you find enjoyment \\
alone in the wilderness, Gotama?” 

“There’s\marginnote{3.1} nothing I need to do in the jungle; \\
my jungle’s cut down at the root, it’s withered away. \\
With jungle cleared and free of thorns, \\
I enjoy being alone in the jungle, having given up discontent.” 

%
\end{verse}

When\marginnote{4.1} he had spoken, the brahmin \textsanskrit{Bhāradvāja} the Builder said to the Buddha, “Excellent, Master Gotama … From this day forth, may Master Gotama remember me as a lay follower who has gone for refuge for life.” 

%
\section*{{\suttatitleacronym SN 7.18}{\suttatitletranslation Collecting Firewood }{\suttatitleroot Kaṭṭhahārasutta}}
\addcontentsline{toc}{section}{\tocacronym{SN 7.18} \toctranslation{Collecting Firewood } \tocroot{Kaṭṭhahārasutta}}
\markboth{Collecting Firewood }{Kaṭṭhahārasutta}
\extramarks{SN 7.18}{SN 7.18}

At\marginnote{1.1} one time the Buddha was staying in the land of the Kosalans in a certain forest grove. 

Then\marginnote{1.2} several youths, students of one of the \textsanskrit{Bhāradvāja} brahmins, approached a forest grove while collecting firewood. They saw the Buddha sitting down cross-legged at the root of a certain sal tree, with his body straight, and mindfulness established right there. Seeing this, they went up to \textsanskrit{Bhāradvāja} and said to him, “Please sir, you should know this. In such and such a forest grove the ascetic Gotama is sitting down cross-legged, with his body straight, and mindfulness established right there.” 

Then\marginnote{1.5} \textsanskrit{Bhāradvāja} together with those students went to that forest grove where he saw the Buddha sitting down cross-legged, with his body straight, and mindfulness established right there. 

He\marginnote{1.7} went up to the Buddha and addressed him in verse: 

\begin{verse}%
“Deep\marginnote{2.1} in the jungle so full of terrors, \\
you’ve plunged into the empty, desolate wilderness. \\
Still, steady, and graceful: \\
how beautifully you meditate, mendicant! 

Where\marginnote{3.1} there is no song or music, \\
a lonely sage resorts to the wilderness. \\
This strikes me as an amazing thing, \\
that you dwell so joyfully alone in the jungle. 

I\marginnote{4.1} suppose you wish to be reborn in the company \\
of the supreme sovereign of the heaven of the Three and Thirty. \\
Is that why you resort to the desolate wilderness, \\
to practice austerities for attaining \textsanskrit{Brahmā}?” 

“Any\marginnote{5.1} wishes and hopes that are always attached \\
to the many and various realms—\\
the yearnings sprung from the root of unknowing—\\
I’ve eliminated them all down to the root. 

So\marginnote{6.1} I’m wishless, unattached, disengaged; \\
amongst all things, my vision is clear. \\
I’ve attained the state of grace, the supreme awakening; \\
I meditate alone, brahmin, and self-assured.” 

%
\end{verse}

When\marginnote{7.1} he had spoken, \textsanskrit{Bhāradvāja} said to the Buddha, “Excellent, Master Gotama! Excellent! … From this day forth, may Master Gotama remember me as a lay follower who has gone for refuge for life.” 

%
\section*{{\suttatitleacronym SN 7.19}{\suttatitletranslation The Brahmin Who Provided for His Mother }{\suttatitleroot Mātuposakasutta}}
\addcontentsline{toc}{section}{\tocacronym{SN 7.19} \toctranslation{The Brahmin Who Provided for His Mother } \tocroot{Mātuposakasutta}}
\markboth{The Brahmin Who Provided for His Mother }{Mātuposakasutta}
\extramarks{SN 7.19}{SN 7.19}

At\marginnote{1.1} \textsanskrit{Sāvatthī}. 

Then\marginnote{1.2} a brahmin who provided for his mother went up to the Buddha, and exchanged greetings with him. 

When\marginnote{1.3} the greetings and polite conversation were over, he sat down to one side and said to the Buddha, “Master Gotama, I seek alms by legitimate means, which I use to provide for my mother and father. In doing so, am I doing my duty?” 

“Indeed,\marginnote{1.6} brahmin, in so doing you are doing your duty. Whoever seeks alms by legitimate means, and uses them to provide for their mother and father makes much merit. 

\begin{verse}%
A\marginnote{2.1} mortal provides for their mother \\
and father by legitimate means; \\
because they look after \\
their parents like this, \\
they’re praised in this life by the astute, \\
and they depart to rejoice in heaven.” 

%
\end{verse}

When\marginnote{3.1} he said this, the brahmin who provided for his mother said to the Buddha, “Excellent, Master Gotama! Excellent! … From this day forth, may Master Gotama remember me as a lay follower who has gone for refuge for life.” 

%
\section*{{\suttatitleacronym SN 7.20}{\suttatitletranslation A Beggar }{\suttatitleroot Bhikkhakasutta}}
\addcontentsline{toc}{section}{\tocacronym{SN 7.20} \toctranslation{A Beggar } \tocroot{Bhikkhakasutta}}
\markboth{A Beggar }{Bhikkhakasutta}
\extramarks{SN 7.20}{SN 7.20}

At\marginnote{1.1} \textsanskrit{Sāvatthī}. 

Then\marginnote{1.2} a begging brahmin went up to the Buddha, and exchanged greetings with him. 

When\marginnote{1.3} the greetings and polite conversation were over, he sat down to one side and said to the Buddha, “Master Gotama, both you and I are beggars. What, then, is the difference between us?” 

\begin{verse}%
“You\marginnote{2.1} don’t become a beggar \\
just by begging from others. \\
One who has undertaken domestic duties \\
has not yet become a mendicant. 

But\marginnote{3.1} one living a spiritual life \\
who has banished both merit and evil, \\
and wanders having assessed the world, \\
is said to be a mendicant.” 

%
\end{verse}

When\marginnote{4.1} he had spoken, the begging brahmin said to the Buddha, “Excellent, Master Gotama! Excellent! … From this day forth, may Master Gotama remember me as a lay follower who has gone for refuge for life.” 

%
\section*{{\suttatitleacronym SN 7.21}{\suttatitletranslation With Saṅgārava }{\suttatitleroot Saṅgāravasutta}}
\addcontentsline{toc}{section}{\tocacronym{SN 7.21} \toctranslation{With Saṅgārava } \tocroot{Saṅgāravasutta}}
\markboth{With Saṅgārava }{Saṅgāravasutta}
\extramarks{SN 7.21}{SN 7.21}

At\marginnote{1.1} \textsanskrit{Sāvatthī}. 

Now\marginnote{1.2} at that time there was a brahmin named \textsanskrit{Saṅgārava} staying in \textsanskrit{Sāvatthī}. He practiced purification by water, believing in purification by water. He lived committed to the practice of immersing himself in water at dawn and dusk. Then Venerable Ānanda robed up in the morning and, taking his bowl and robe, entered \textsanskrit{Sāvatthī} for alms. 

He\marginnote{1.4} wandered for alms in \textsanskrit{Sāvatthī}. After the meal, on his return from almsround, he went to the Buddha, bowed, sat down to one side, and said to him, “Sir, there is a brahmin named \textsanskrit{Saṅgārava} staying in \textsanskrit{Sāvatthī}. He practices purification by water, believing in purification by water. He lives committed to the practice of immersing himself in water at dawn and dusk. Please visit him at his home out of compassion.” The Buddha consented in silence. 

Then\marginnote{2.1} the Buddha robed up in the morning and, taking his bowl and robe, went to the home of the brahmin \textsanskrit{Saṅgārava}, and sat down on the seat spread out. Then the brahmin \textsanskrit{Saṅgārava} went up to the Buddha, and exchanged greetings with him. 

When\marginnote{2.3} the greetings and polite conversation were over, he sat down to one side. The Buddha said to him, “Is it really true, brahmin, that you practice purification by water, believing in purification by water; that you live committed to the practice of immersing yourself in water at dawn and dusk?” 

“Yes,\marginnote{2.5} Master Gotama.” 

“But\marginnote{2.6} brahmin, for what reason do you practice purification by water?” 

“It’s\marginnote{2.7} because, Master Gotama, whatever bad deeds I’ve done during the day I wash off by bathing at dusk; and whatever bad deeds I’ve done during the night, I wash off by bathing at dawn. That’s the reason why I practice purification by water.” 

\begin{verse}%
“The\marginnote{3.1} teaching is a lake with shores of ethics, brahmin, \\
unclouded, praised by the fine to the good. \\
There the knowledge masters go to bathe, \\
and cross to the far shore without getting wet.” 

%
\end{verse}

When\marginnote{4.1} he had spoken, \textsanskrit{Saṅgārava} said to the Buddha, “Excellent, Master Gotama! Excellent! … From this day forth, may Master Gotama remember me as a lay follower who has gone for refuge for life.” 

%
\section*{{\suttatitleacronym SN 7.22}{\suttatitletranslation At Khomadussa }{\suttatitleroot Khomadussasutta}}
\addcontentsline{toc}{section}{\tocacronym{SN 7.22} \toctranslation{At Khomadussa } \tocroot{Khomadussasutta}}
\markboth{At Khomadussa }{Khomadussasutta}
\extramarks{SN 7.22}{SN 7.22}

\scevam{So\marginnote{1.1} I have heard. }At one time the Buddha was staying in the land of the Sakyans, where they have a town named Khomadussa. 

Then\marginnote{1.3} the Buddha robed up in the morning and, taking his bowl and robe, entered Khomadussa for alms. Now at that time the brahmins and householders of Khomadussa were gathered in the council hall for some business, while a gentle rain drizzled down. 

Then\marginnote{1.5} the Buddha approached that council. The brahmins and householders saw the Buddha coming off in the distance, and said, “Who are these shavelings, these fake ascetics? Don’t they understand the council rules?” 

Then\marginnote{1.9} the Buddha addressed the brahmins and householders of Khomadussa in verse: 

\begin{verse}%
“If\marginnote{2.1} good people are not present it is no true council; \\
and those whose speak against principle are not good people. \\
Having given up greed, hate, and delusion, \\
speakers of principle are good people.” 

%
\end{verse}

When\marginnote{3.1} he had spoken, the brahmins and householders of Khomadussa said to the Buddha, “Excellent, Master Gotama! Excellent! As if he were righting the overturned, or revealing the hidden, or pointing out the path to the lost, or lighting a lamp in the dark so people with good eyes can see what’s there, Master Gotama has made the teaching clear in many ways. We go for refuge to Master Gotama, to the teaching, and to the mendicant \textsanskrit{Saṅgha}. From this day forth, may Master Gotama remember us as lay followers who have gone for refuge for life.” 

\scendsutta{The Linked Discourses with Brahmins are complete. }

%
\addtocontents{toc}{\let\protect\contentsline\protect\nopagecontentsline}
\part*{Linked Discourses With Vaṅgīsa }
\addcontentsline{toc}{part}{Linked Discourses With Vaṅgīsa }
\markboth{}{}
\addtocontents{toc}{\let\protect\contentsline\protect\oldcontentsline}

%
\addtocontents{toc}{\let\protect\contentsline\protect\nopagecontentsline}
\chapter*{The Chapter with the Poet Vaṅgīsa }
\addcontentsline{toc}{chapter}{\tocchapterline{The Chapter with the Poet Vaṅgīsa }}
\addtocontents{toc}{\let\protect\contentsline\protect\oldcontentsline}

%
\section*{{\suttatitleacronym SN 8.1}{\suttatitletranslation Renounced }{\suttatitleroot Nikkhantasutta}}
\addcontentsline{toc}{section}{\tocacronym{SN 8.1} \toctranslation{Renounced } \tocroot{Nikkhantasutta}}
\markboth{Renounced }{Nikkhantasutta}
\extramarks{SN 8.1}{SN 8.1}

\scevam{So\marginnote{1.1} I have heard. }At one time Venerable \textsanskrit{Vaṅgīsa} was staying near \textsanskrit{Āḷavī}, at the \textsanskrit{Aggāḷava} Tree-shrine, together with his mentor, Venerable Nigrodhakappa. 

Now\marginnote{1.3} at that time \textsanskrit{Vaṅgīsa} was junior, recently gone forth. He had been left behind to look after the dwelling. 

Then\marginnote{1.4} several women dressed in all their finery went to the monastery at \textsanskrit{Aggāḷava} in order to inspect the dwelling. When \textsanskrit{Vaṅgīsa} saw them he became dissatisfied, with lust infecting his mind. 

Then\marginnote{1.6} he thought, “It’s my loss, my misfortune, that I’ve become dissatisfied, with lust infecting my mind. How is it possible for someone else to dispel my discontent and give rise to satisfaction? Why don’t I do it myself?” 

Then,\marginnote{1.10} on the occasion of dispelling his own discontent and giving rise to satisfaction, he recited these verses: 

\begin{verse}%
“Now\marginnote{2.1} that I’ve renounced \\
the home life for homelessness \\
I’m overrun \\
by the rude thoughts of the Dark One. 

Even\marginnote{3.1} if a thousand mighty princes and great archers, \\
well trained, with strong bows, \\
were to completely surround me; \\
I would never flee. 

And\marginnote{4.1} even if women come, \\
many more than that, \\
they won’t scare me, \\
for I stand firm in the teaching. 

I\marginnote{5.1} heard this with my own ears \\
from the Buddha, kinsman of the Sun, \\
about the path going to extinguishment; \\
that’s what delights my mind. 

Wicked\marginnote{6.1} One, if you come near me \\
as I meditate like this, \\
I’ll make sure that you, Death, \\
won’t even see the path I take.” 

%
\end{verse}

%
\section*{{\suttatitleacronym SN 8.2}{\suttatitletranslation Dissatisfaction }{\suttatitleroot Aratīsutta}}
\addcontentsline{toc}{section}{\tocacronym{SN 8.2} \toctranslation{Dissatisfaction } \tocroot{Aratīsutta}}
\markboth{Dissatisfaction }{Aratīsutta}
\extramarks{SN 8.2}{SN 8.2}

At\marginnote{1.1} one time Venerable \textsanskrit{Vaṅgīsa} was staying near \textsanskrit{Āḷavī}, at the \textsanskrit{Aggāḷava} Tree-shrine, together with his mentor, Venerable Nigrodhakappa. 

Now\marginnote{1.2} at that time after Venerable Nigrodhakappa had finished his meal, on his return from almsround, he would enter his dwelling and not emerge for the rest of that day, or the next. 

And\marginnote{1.3} at that time Venerable \textsanskrit{Vaṅgīsa} became dissatisfied, as lust infected his mind. 

Then\marginnote{1.4} he thought, “It’s my loss, my misfortune, that I’ve become dissatisfied, with lust infecting my mind. How is it possible for someone else to dispel my discontent and give rise to satisfaction? Why don’t I do it myself?” 

Then,\marginnote{1.8} on the occasion of dispelling his own discontent and giving rise to satisfaction, he recited these verses: 

\begin{verse}%
“Giving\marginnote{2.1} up discontent and desire, \\
along with all thoughts of the lay life, \\
they wouldn’t get entangled in anything; \\
unentangled, undesiring: that’s a real mendicant. 

Whether\marginnote{3.1} on this earth or in the sky, \\
whatever in the world is included in form \\
wears out, it is all impermanent; \\
the thoughtful live having comprehended this truth. 

People\marginnote{4.1} are bound to their attachments, \\
to what is seen, heard, felt, and thought. \\
Unstirred, dispel desire for these things; \\
for one called ‘a sage’ does not cling to them. 

Attached\marginnote{5.1} to the sixty wrong views, and full of their own opinions, \\
ordinary people are fixed in wrong principles. \\
But that mendicant wouldn’t join a sectarian group, \\
still less would they utter lewd speech. 

Clever,\marginnote{6.1} long serene, \\
free of deceit, alert, without envy, \\
the sage has reached the state of peace; \\
and because he’s extinguished, he bides his time.” 

%
\end{verse}

%
\section*{{\suttatitleacronym SN 8.3}{\suttatitletranslation Good-Hearted }{\suttatitleroot Pesalasutta}}
\addcontentsline{toc}{section}{\tocacronym{SN 8.3} \toctranslation{Good-Hearted } \tocroot{Pesalasutta}}
\markboth{Good-Hearted }{Pesalasutta}
\extramarks{SN 8.3}{SN 8.3}

At\marginnote{1.1} one time Venerable \textsanskrit{Vaṅgīsa} was staying near \textsanskrit{Āḷavī}, at the \textsanskrit{Aggāḷava} Tree-shrine, together with his mentor, Venerable Nigrodhakappa. 

Now\marginnote{1.2} at that time Venerable \textsanskrit{Vaṅgīsa} looked down upon other good-hearted mendicants because of his own poetic virtuosity. 

Then\marginnote{1.3} he thought, “It’s my loss, my misfortune, that I look down on other good-hearted mendicants because of my own poetic virtuosity.” 

Then,\marginnote{1.6} on the occasion of arousing remorse in himself, he recited these verses: 

\begin{verse}%
“Give\marginnote{2.1} up conceit, Gotama! \\
Completely abandon the different kinds of conceit! \\
Besotted with the different kinds of conceit, \\
you’ve had regrets for a long time. 

Smeared\marginnote{3.1} by smears and slain by conceit, \\
people fall into hell. \\
When people slain by conceit are reborn in hell, \\
they grieve for a long time. 

But\marginnote{4.1} a mendicant who practices rightly, \\
winner of the path, never grieves. \\
They enjoy happiness and a good reputation, \\
and they rightly call him a ‘Seer of Truth’. 

So\marginnote{5.1} don’t be hard-hearted, be energetic, \\
with hindrances given up, be pure. \\
Then with conceit given up completely, \\
use knowledge to make an end, and be at peace.” 

%
\end{verse}

%
\section*{{\suttatitleacronym SN 8.4}{\suttatitletranslation With Ānanda }{\suttatitleroot Ānandasutta}}
\addcontentsline{toc}{section}{\tocacronym{SN 8.4} \toctranslation{With Ānanda } \tocroot{Ānandasutta}}
\markboth{With Ānanda }{Ānandasutta}
\extramarks{SN 8.4}{SN 8.4}

At\marginnote{1.1} one time Venerable Ānanda was staying near \textsanskrit{Sāvatthī} in Jeta’s Grove, \textsanskrit{Anāthapiṇḍika}’s monastery. 

Then\marginnote{1.2} Venerable Ānanda robed up in the morning and, taking his bowl and robe, entered \textsanskrit{Sāvatthī} for alms with Venerable \textsanskrit{Vaṅgīsa} as his second monk. 

And\marginnote{1.3} at that time Venerable \textsanskrit{Vaṅgīsa} became dissatisfied, as lust infected his mind. Then he addressed Ānanda in verse: 

\begin{verse}%
“I’ve\marginnote{2.1} got a burning desire for pleasure; \\
My mind is on fire! \\
Please, out of compassion, Gotama, \\
tell me how to quench the flames.” 

“Your\marginnote{3.1} mind is on fire \\
because of a perversion of perception. \\
Turn away from the feature of things \\
that’s attractive, provoking lust. 

See\marginnote{4.1} all conditioned phenomena as other, \\
as suffering and not-self. \\
Extinguish the great fire of lust, \\
don’t burn up again and again. 

With\marginnote{5.1} mind unified and serene, \\
meditate on the ugly aspects of the body. \\
With mindfulness immersed in the body, \\
be full of disillusionment. 

Meditate\marginnote{6.1} on the signless, \\
give up the underlying tendency to conceit; \\
and when you comprehend conceit, \\
you will live at peace.” 

%
\end{verse}

%
\section*{{\suttatitleacronym SN 8.5}{\suttatitletranslation Well-Spoken Words }{\suttatitleroot Subhāsitasutta}}
\addcontentsline{toc}{section}{\tocacronym{SN 8.5} \toctranslation{Well-Spoken Words } \tocroot{Subhāsitasutta}}
\markboth{Well-Spoken Words }{Subhāsitasutta}
\extramarks{SN 8.5}{SN 8.5}

At\marginnote{1.1} \textsanskrit{Sāvatthī}. 

There\marginnote{1.2} the Buddha addressed the mendicants, “Mendicants!” 

“Venerable\marginnote{1.4} sir,” they replied. The Buddha said this: 

“Mendicants,\marginnote{2.1} speech that has four factors is well spoken, not poorly spoken. It’s blameless and is not criticized by sensible people. What four? It’s when a mendicant speaks well, not poorly; they speak on the teaching, not against the teaching; they speak pleasantly, not unpleasantly; and they speak truthfully, not falsely. Speech with these four factors is well spoken, not poorly spoken. It’s blameless and is not criticized by sensible people.” 

That\marginnote{2.5} is what the Buddha said. Then the Holy One, the Teacher, went on to say: 

\begin{verse}%
“Good\marginnote{3.1} people say that well-spoken words are foremost; \\
second, speak on the teaching, not against it; \\
third, speak pleasantly, not unpleasantly; \\
and fourth, speak truthfully, not falsely.” 

%
\end{verse}

Then\marginnote{4.1} Venerable \textsanskrit{Vaṅgīsa} got up from his seat, arranged his robe over one shoulder, raised his joined palms toward the Buddha, and said, “I feel inspired to speak, Blessed One! I feel inspired to speak, Holy One!” 

“Then\marginnote{4.3} speak as you feel inspired,” said the Buddha. 

Then\marginnote{4.4} \textsanskrit{Vaṅgīsa} extolled the Buddha in his presence with fitting verses: 

\begin{verse}%
“Speak\marginnote{5.1} only such words \\
as do not hurt yourself \\
nor harm others; \\
such speech is truly well spoken. 

Speak\marginnote{6.1} only pleasing words, \\
words gladly welcomed. \\
Pleasing words are those \\
that bring nothing bad to others. 

Truth\marginnote{7.1} itself is the undying word: \\
this is an eternal truth. \\
Good people say that the teaching and its meaning \\
are grounded in the truth. 

The\marginnote{8.1} words spoken by the Buddha \\
for realizing the sanctuary, extinguishment, \\
for making an end of suffering: \\
this really is the best kind of speech.” 

%
\end{verse}

%
\section*{{\suttatitleacronym SN 8.6}{\suttatitletranslation With Sāriputta }{\suttatitleroot Sāriputtasutta}}
\addcontentsline{toc}{section}{\tocacronym{SN 8.6} \toctranslation{With Sāriputta } \tocroot{Sāriputtasutta}}
\markboth{With Sāriputta }{Sāriputtasutta}
\extramarks{SN 8.6}{SN 8.6}

At\marginnote{1.1} one time Venerable \textsanskrit{Sāriputta} was staying near \textsanskrit{Sāvatthī} in Jeta’s Grove, \textsanskrit{Anāthapiṇḍika}’s monastery. 

Now\marginnote{1.2} at that time Venerable \textsanskrit{Sāriputta} was educating, encouraging, firing up, and inspiring the mendicants in the assembly hall with a Dhamma talk. His words were polished, clear, articulate, and expressed the meaning. And those mendicants were paying heed, paying attention, engaging wholeheartedly, and lending an ear. 

Then\marginnote{1.4} Venerable \textsanskrit{Vaṅgīsa} thought, “This Venerable \textsanskrit{Sāriputta} is educating the mendicants. … And those mendicants are paying heed, paying attention, engaging wholeheartedly, and lending an ear. Why don’t I extoll him in his presence with fitting verses?” 

Then\marginnote{2.1} Venerable \textsanskrit{Vaṅgīsa} got up from his seat, arranged his robe over one shoulder, raised his joined palms toward \textsanskrit{Sāriputta}, and said, “I feel inspired to speak, Reverend \textsanskrit{Sāriputta}! I feel inspired to speak, Reverend \textsanskrit{Sāriputta}!” 

“Then\marginnote{2.3} speak as you feel inspired,” said \textsanskrit{Sāriputta}. 

Then\marginnote{2.4} \textsanskrit{Vaṅgīsa} extolled \textsanskrit{Sāriputta} in his presence with fitting verses: 

\begin{verse}%
“Deep\marginnote{3.1} in wisdom, intelligent, \\
expert in the variety of paths; \\
\textsanskrit{Sāriputta}, so greatly wise, \\
teaches Dhamma to the mendicants. 

He\marginnote{4.1} teaches in brief, \\
or he speaks at length. \\
His call, like a myna bird, \\
overflows with inspiration. 

While\marginnote{5.1} he teaches \\
the mendicants listen to his sweet voice, \\
sounding attractive, \\
clear and graceful. \\
They listen joyfully, \\
their hearts elated.” 

%
\end{verse}

%
\section*{{\suttatitleacronym SN 8.7}{\suttatitletranslation The Invitation to Admonish }{\suttatitleroot Pavāraṇāsutta}}
\addcontentsline{toc}{section}{\tocacronym{SN 8.7} \toctranslation{The Invitation to Admonish } \tocroot{Pavāraṇāsutta}}
\markboth{The Invitation to Admonish }{Pavāraṇāsutta}
\extramarks{SN 8.7}{SN 8.7}

At\marginnote{1.1} one time the Buddha was staying near \textsanskrit{Sāvatthī} in the Eastern Monastery, the stilt longhouse of \textsanskrit{Migāra}’s mother, together with a large \textsanskrit{Saṅgha} of around five hundred monks, all of whom were perfected ones. Now, at that time it was the sabbath—the full moon on the fifteenth day—and the Buddha was sitting in the open surrounded by the \textsanskrit{Saṅgha} of monks for the invitation to admonish. 

Then\marginnote{1.3} the Buddha looked around the \textsanskrit{Saṅgha} of monks, who were silent. He addressed them: “Come now, monks, I invite you all: Is there anything I’ve done by way of body or speech that you would criticize?” 

When\marginnote{2.1} he had spoken, Venerable \textsanskrit{Sāriputta} got up from his seat, arranged his robe over one shoulder, raised his joined palms toward the Buddha, and said: “There is nothing, sir, that you’ve done by way of body or speech that we would criticize. For the Blessed One gave rise to the unarisen path, gave birth to the unborn path, and explained the unexplained path. He is the knower of the path, the discoverer of the path, the expert on the path. And now the disciples live following the path; they acquire it later. And sir, I invite the Blessed One. Is there anything I’ve done by way of body or speech that you would criticize?” 

“There\marginnote{3.1} is nothing, \textsanskrit{Sāriputta}, that you’ve done by way of body or speech that I would criticize. \textsanskrit{Sāriputta}, you are astute. You have great wisdom, widespread wisdom, laughing wisdom, swift wisdom, sharp wisdom, penetrating wisdom. A wheel-turning monarch’s oldest son rightly keeps wielding the power set in motion by his father. In the same way, \textsanskrit{Sāriputta} rightly keeps rolling the supreme Wheel of Dhamma that was rolled forth by me.” 

“Since\marginnote{4.1} it seems I have done nothing worthy of the Blessed One’s criticism, is there anything these five hundred monks have done by way of body or speech that you would criticize?” 

“There\marginnote{4.3} is nothing, \textsanskrit{Sāriputta}, that these five hundred monks have done by way of body or speech that I would criticize. For of these five hundred monks, sixty have the three knowledges, sixty have the six direct knowledges, sixty are freed both ways, and the rest are freed by wisdom.” 

Then\marginnote{5.1} Venerable \textsanskrit{Vaṅgīsa} got up from his seat, arranged his robe over one shoulder, raised his joined palms toward the Buddha, and said, “I feel inspired to speak, Blessed One! I feel inspired to speak, Holy One!” 

“Then\marginnote{5.3} speak as you feel inspired,” said the Buddha. 

Then\marginnote{5.4} \textsanskrit{Vaṅgīsa} extolled the Buddha in his presence with fitting verses: 

\begin{verse}%
“Today,\marginnote{6.1} on the fifteenth day sabbath, \\
five hundred monks have gathered together to purify their precepts. \\
These untroubled sages have cut off their fetters and bonds, \\
they will not be reborn again. 

Just\marginnote{7.1} as a wheel-turning monarch \\
surrounded by ministers \\
travels all around this \\
land that’s girt by sea. 

So\marginnote{8.1} disciples with the three knowledges, \\
conquerors of death, \\
revere the winner of the battle, \\
the unsurpassed caravan leader. 

All\marginnote{9.1} are sons of the Blessed One—\\
there is no rubbish here. \\
I bow to the kinsman of the Sun, \\
destroyer of the dart of craving.” 

%
\end{verse}

%
\section*{{\suttatitleacronym SN 8.8}{\suttatitletranslation Over a Thousand }{\suttatitleroot Parosahassasutta}}
\addcontentsline{toc}{section}{\tocacronym{SN 8.8} \toctranslation{Over a Thousand } \tocroot{Parosahassasutta}}
\markboth{Over a Thousand }{Parosahassasutta}
\extramarks{SN 8.8}{SN 8.8}

At\marginnote{1.1} one time the Buddha was staying near \textsanskrit{Sāvatthī} in Jeta’s Grove, \textsanskrit{Anāthapiṇḍika}’s monastery, together with a large \textsanskrit{Saṅgha} of 1,250 mendicants. 

Now\marginnote{1.2} at that time the Buddha was educating, encouraging, firing up, and inspiring the mendicants with a Dhamma talk about extinguishment. And those mendicants were paying heed, paying attention, engaging wholeheartedly, and lending an ear. 

Then\marginnote{1.4} Venerable \textsanskrit{Vaṅgīsa} thought, “The Buddha is educating, encouraging, firing up, and inspiring the mendicants with a Dhamma talk about extinguishment. And those mendicants are paying heed, paying attention, engaging wholeheartedly, and lending an ear. Why don’t I extoll him in his presence with fitting verses?” 

Then\marginnote{2.1} Venerable \textsanskrit{Vaṅgīsa} got up from his seat, arranged his robe over one shoulder, raised his joined palms toward the Buddha, and said, “I feel inspired to speak, Blessed One! I feel inspired to speak, Holy One!” 

“Then\marginnote{2.3} speak as you feel inspired,” said the Buddha. 

Then\marginnote{2.4} \textsanskrit{Vaṅgīsa} extolled the Buddha in his presence with fitting verses: 

\begin{verse}%
“Over\marginnote{3.1} a thousand mendicants \\
revere the Holy One \\
as he teaches the immaculate Dhamma, \\
extinguishment, fearing nothing from any quarter. 

They\marginnote{4.1} listen to the immaculate Dhamma \\
taught by the fully awakened Buddha; \\
the Buddha is so brilliant, \\
at the fore of the mendicant \textsanskrit{Saṅgha}, 

Blessed\marginnote{5.1} One, your name is ‘Giant’, \\
seventh of the sages. \\
You are like a great cloud \\
that rains on your disciples. 

I’ve\marginnote{6.1} left my day’s meditation, \\
out of desire to see the teacher. \\
Great hero, your disciple \textsanskrit{Vaṅgīsa} \\
bows at your feet.” 

%
\end{verse}

“\textsanskrit{Vaṅgīsa},\marginnote{7.1} had you previously composed these verses, or did they spring to mind in the moment?” 

“They\marginnote{7.2} sprang to mind in the moment, sir.” 

“Well\marginnote{7.3} then, \textsanskrit{Vaṅgīsa}, speak some more spontaneously inspired verses.” 

“Yes,\marginnote{7.4} sir,” replied \textsanskrit{Vaṅgīsa}. Then he extolled the Buddha with some more spontaneously inspired verses, not previously composed: 

\begin{verse}%
“Having\marginnote{8.1} overcome \textsanskrit{Māra}’s devious path, \\
you wander with hard-heartedness dissolved. \\
See him, the liberator from bonds, unattached, \\
analyzing the teaching. 

You\marginnote{9.1} have explained in many ways \\
the path to cross the flood. \\
The Seers of Truth stand unfaltering \\
in the deathless you’ve explained. 

As\marginnote{10.1} the bringer of light who has pierced the truth, \\
you’ve seen what lies beyond all realms. \\
When you saw and realized this for yourself, \\
you taught it first to the group of five. 

When\marginnote{11.1} the Dhamma has been so well taught, \\
how could those who know it be negligent? \\
That’s why, being diligent, we should always train \\
respectfully in the Buddha’s teaching.” 

%
\end{verse}

%
\section*{{\suttatitleacronym SN 8.9}{\suttatitletranslation With Koṇḍañña }{\suttatitleroot Koṇḍaññasutta}}
\addcontentsline{toc}{section}{\tocacronym{SN 8.9} \toctranslation{With Koṇḍañña } \tocroot{Koṇḍaññasutta}}
\markboth{With Koṇḍañña }{Koṇḍaññasutta}
\extramarks{SN 8.9}{SN 8.9}

At\marginnote{1.1} one time the Buddha was staying near \textsanskrit{Rājagaha}, in the Bamboo Grove, the squirrels’ feeding ground. 

Then\marginnote{1.2} Venerable \textsanskrit{Koṇḍañña} Who Understood approached the Buddha after a very long absence. He bowed with his head at the Buddha’s feet, caressing them and covering them with kisses, and pronounced his name: “I am \textsanskrit{Koṇḍañña}, Blessed One! I am \textsanskrit{Koṇḍañña}, Holy One!” 

Then\marginnote{1.4} Venerable \textsanskrit{Vaṅgīsa} thought, “This Venerable \textsanskrit{Koṇḍañña} Who Understood has approached the Buddha after a very long absence. He bowed with his head at the Buddha’s feet, caressing them and covering them with kisses, and pronounced his name: ‘I am \textsanskrit{Koṇḍañña}, Blessed One! I am \textsanskrit{Koṇḍañña}, Holy One!’ Why don’t I extoll him in the Buddha’s presence with fitting verses?” 

Then\marginnote{2.1} Venerable \textsanskrit{Vaṅgīsa} got up from his seat, arranged his robe over one shoulder, raised his joined palms toward the Buddha, and said, “I feel inspired to speak, Blessed One! I feel inspired to speak, Holy One!” 

“Then\marginnote{2.3} speak as you feel inspired,” said the Buddha. 

Then\marginnote{2.4} \textsanskrit{Vaṅgīsa} extolled \textsanskrit{Koṇḍañña} in the Buddha’s presence with fitting verses: 

\begin{verse}%
“The\marginnote{3.1} senior monk who was awakened right after the Buddha, \\
\textsanskrit{Koṇḍañña}, is keenly energetic. \\
He regularly gains blissful meditative states, \\
and the three kinds of seclusion. 

Whatever\marginnote{4.1} can be attained by a disciple \\
who does the Teacher’s bidding, \\
he has attained it all, \\
through diligently training himself. 

With\marginnote{5.1} great power and the three knowledges, \\
expert in comprehending the minds of others, \\
\textsanskrit{Koṇḍañña}, the heir to the Buddha, \\
bows at the Teacher’s feet.” 

%
\end{verse}

%
\section*{{\suttatitleacronym SN 8.10}{\suttatitletranslation With Moggallāna }{\suttatitleroot Moggallānasutta}}
\addcontentsline{toc}{section}{\tocacronym{SN 8.10} \toctranslation{With Moggallāna } \tocroot{Moggallānasutta}}
\markboth{With Moggallāna }{Moggallānasutta}
\extramarks{SN 8.10}{SN 8.10}

At\marginnote{1.1} one time the Buddha was staying on the slopes of Isigili at the Black Rock, together with a large \textsanskrit{Saṅgha} of around five hundred mendicants, all of whom were perfected ones. Thereupon, with his mind, Venerable \textsanskrit{Mahāmoggallāna} checked to see whose mind was liberated and free of attachments. 

Then\marginnote{1.3} Venerable \textsanskrit{Vaṅgīsa} thought, “The Buddha is staying on the slopes of Isigili … with five hundred perfected ones. \textsanskrit{Mahāmoggallāna} is checking to see whose mind is liberated and free of attachments. Why don’t I extoll him in the Buddha’s presence with fitting verses?” 

Then\marginnote{2.1} Venerable \textsanskrit{Vaṅgīsa} got up from his seat, arranged his robe over one shoulder, raised his joined palms toward the Buddha, and said, “I feel inspired to speak, Blessed One! I feel inspired to speak, Holy One!” 

“Then\marginnote{2.3} speak as you feel inspired,” said the Buddha. 

Then\marginnote{2.4} \textsanskrit{Vaṅgīsa} extolled \textsanskrit{Mahāmoggallāna} in his presence with fitting verses: 

\begin{verse}%
“As\marginnote{3.1} the sage, who has gone beyond suffering, \\
sits upon the mountain slope, \\
he is revered by disciples with the three knowledges, \\
conquerors of death. 

\textsanskrit{Moggallāna},\marginnote{4.1} of great psychic power, \\
comprehends with his mind, \\
scrutinizing their minds, \\
liberated, free of attachments. 

So\marginnote{5.1} they revere Gotama, \\
the sage gone beyond suffering, \\
who is endowed with all path factors, \\
and with a multitude of attributes.” 

%
\end{verse}

%
\section*{{\suttatitleacronym SN 8.11}{\suttatitletranslation At Gaggarā }{\suttatitleroot Gaggarāsutta}}
\addcontentsline{toc}{section}{\tocacronym{SN 8.11} \toctranslation{At Gaggarā } \tocroot{Gaggarāsutta}}
\markboth{At Gaggarā }{Gaggarāsutta}
\extramarks{SN 8.11}{SN 8.11}

At\marginnote{1.1} one time the Buddha was staying near \textsanskrit{Campā} on the banks of the \textsanskrit{Gaggarā} Lotus Pond, together with a large \textsanskrit{Saṅgha} of around five hundred mendicants, seven hundred male and seven hundred female lay followers, and many thousands of deities. But the Buddha outshone them all in beauty and glory. 

Then\marginnote{1.3} Venerable \textsanskrit{Vaṅgīsa} thought, “The Buddha is staying near \textsanskrit{Campā} on the banks of the \textsanskrit{Gaggarā} Lotus Pond, together with a large \textsanskrit{Saṅgha} of around five hundred mendicants, seven hundred male and seven hundred female lay followers, and many thousands of deities. And he outshines them all in beauty and glory. Why don’t I extoll him in his presence with fitting verses?” 

Then\marginnote{2.1} Venerable \textsanskrit{Vaṅgīsa} got up from his seat, arranged his robe over one shoulder, raised his joined palms toward the Buddha, and said, “I feel inspired to speak, Blessed One! I feel inspired to speak, Holy One!” 

“Then\marginnote{2.3} speak as you feel inspired,” said the Buddha. 

Then\marginnote{2.4} \textsanskrit{Vaṅgīsa} extolled the Buddha in his presence with fitting verses: 

\begin{verse}%
“Like\marginnote{3.1} the moon on a cloudless night, \\
like the shining immaculate sun, \\
so too \textsanskrit{Aṅgīrasa}, O great sage, \\
your glory outshines the entire world.” 

%
\end{verse}

%
\section*{{\suttatitleacronym SN 8.12}{\suttatitletranslation With Vaṅgīsa }{\suttatitleroot Vaṅgīsasutta}}
\addcontentsline{toc}{section}{\tocacronym{SN 8.12} \toctranslation{With Vaṅgīsa } \tocroot{Vaṅgīsasutta}}
\markboth{With Vaṅgīsa }{Vaṅgīsasutta}
\extramarks{SN 8.12}{SN 8.12}

At\marginnote{1.1} one time Venerable \textsanskrit{Vaṅgīsa} was staying near \textsanskrit{Sāvatthī} in Jeta’s Grove, \textsanskrit{Anāthapiṇḍika}’s monastery. 

Now\marginnote{1.2} at that time \textsanskrit{Vaṅgīsa} had recently attained perfection. While experiencing the bliss of freedom, on that occasion he recited these verses: 

\begin{verse}%
“We\marginnote{2.1} used to wander, drunk on poetry, \\
village to village, town to town. \\
Then we saw the Buddha, \\
and faith arose in us. 

He\marginnote{3.1} taught me Dhamma: \\
the aggregates, sense fields, and elements. \\
When I heard his teaching \\
I went forth to homelessness. 

It\marginnote{4.1} was truly for the benefit of many \\
that the sage achieved awakening—\\
for the monks and for the nuns \\
who see that they’ve reached certainty. 

It\marginnote{5.1} was so welcome for me \\
to be in the presence of the Buddha. \\
I’ve attained the three knowledges, \\
and fulfilled the Buddha’s instructions. 

I\marginnote{6.1} know my past lives, \\
my clairvoyance is purified, \\
I am master of three knowledges, attained in psychic power, \\
expert in comprehending the minds of others.” 

%
\end{verse}

\scendsutta{The Linked Discourses with \textsanskrit{Vaṅgīsa} are complete. }

%
\addtocontents{toc}{\let\protect\contentsline\protect\nopagecontentsline}
\part*{Linked Discourses in the Woods }
\addcontentsline{toc}{part}{Linked Discourses in the Woods }
\markboth{}{}
\addtocontents{toc}{\let\protect\contentsline\protect\oldcontentsline}

%
\addtocontents{toc}{\let\protect\contentsline\protect\nopagecontentsline}
\chapter*{The Chapter on In the Woods }
\addcontentsline{toc}{chapter}{\tocchapterline{The Chapter on In the Woods }}
\addtocontents{toc}{\let\protect\contentsline\protect\oldcontentsline}

%
\section*{{\suttatitleacronym SN 9.1}{\suttatitletranslation Seclusion }{\suttatitleroot Vivekasutta}}
\addcontentsline{toc}{section}{\tocacronym{SN 9.1} \toctranslation{Seclusion } \tocroot{Vivekasutta}}
\markboth{Seclusion }{Vivekasutta}
\extramarks{SN 9.1}{SN 9.1}

\scevam{So\marginnote{1.1} I have heard. }At one time one of the mendicants was staying in the land of the Kosalans in a certain forest grove. 

Now\marginnote{1.3} at that time that mendicant, during their day’s meditation, was thinking bad, unskillful thoughts to do with the lay life. The deity haunting that forest had compassion for that mendicant, and wanted what’s best for them. So they approached that mendicant wanting to stir them up, and addressed them in verse: 

\begin{verse}%
“You\marginnote{2.1} entered the woods desiring seclusion, \\
yet your mind strays to outward things. \\
As a person, you should dispel the desire for people. \\
Then you’ll be happy, free of greed. 

Mindful,\marginnote{3.1} give up discontent; \\
let us remind you of the way of the good. \\
The dusty abyss is so hard to cross; \\
don’t let sensual dust drag you down. 

Just\marginnote{4.1} as a bird strewn with dirt \\
sheds that clingy dust with a shake; \\
so too, an energetic, mindful mendicant \\
sheds that clingy dust with a shake.” 

%
\end{verse}

Impelled\marginnote{5.1} by that deity, that mendicant was struck with a sense of urgency. 

%
\section*{{\suttatitleacronym SN 9.2}{\suttatitletranslation Getting Up }{\suttatitleroot Upaṭṭhānasutta}}
\addcontentsline{toc}{section}{\tocacronym{SN 9.2} \toctranslation{Getting Up } \tocroot{Upaṭṭhānasutta}}
\markboth{Getting Up }{Upaṭṭhānasutta}
\extramarks{SN 9.2}{SN 9.2}

At\marginnote{1.1} one time one of the mendicants was staying in the land of the Kosalans in a certain forest grove. 

Now\marginnote{1.2} at that time that mendicant fell asleep during the day’s meditation. The deity haunting that forest had compassion for that mendicant, and wanted what’s best for them. So they approached that mendicant wanting to stir them up, and addressed them in verse: 

\begin{verse}%
“Get\marginnote{2.1} up, mendicant! Why lie down? \\
What’s the point in sleeping? \\
How can the afflicted slumber \\
when injured by an arrow strike? 

You\marginnote{3.1} should amplify the faith \\
that led you to go forth \\
from the home life to homelessness. \\
Don’t fall under the sway of slumber.” 

“Sensual\marginnote{4.1} pleasures are impermanent and unstable, \\
but idiots still fall for them. \\
Among those who are bound, they’re free and unattached: \\
why bother a renunciate? 

By\marginnote{5.1} removing desire and greed, \\
by going beyond ignorance, \\
that knowledge has been perfectly cleansed: \\
why bother a renunciate? 

By\marginnote{6.1} breaking ignorance with knowledge, \\
by the ending of defilements, \\
they’re sorrowless, unstressed: \\
why bother a renunciate? 

Energetic,\marginnote{7.1} resolute, \\
always staunchly vigorous, \\
aspiring to extinguishment: \\
why bother a renunciate?” 

%
\end{verse}

%
\section*{{\suttatitleacronym SN 9.3}{\suttatitletranslation With Kassapagotta }{\suttatitleroot Kassapagottasutta}}
\addcontentsline{toc}{section}{\tocacronym{SN 9.3} \toctranslation{With Kassapagotta } \tocroot{Kassapagottasutta}}
\markboth{With Kassapagotta }{Kassapagottasutta}
\extramarks{SN 9.3}{SN 9.3}

At\marginnote{1.1} one time Venerable Kassapagotta was staying in the land of the Kosalans in a certain forest grove. 

Now\marginnote{1.2} at that time Venerable Kassapagotta, having withdrawn for his day’s meditation, tried to advise a tribal hunter. Then the deity haunting that forest approached Kassapagotta wanting to stir him up, and recited these verses: 

\begin{verse}%
“A\marginnote{2.1} tribal hunter wandering the rugged hills \\
is unintelligent, unthinking. \\
It’s a waste of time to advise him; \\
this mendicant seems to me like an idiot. 

The\marginnote{3.1} tribal hunter listens without understanding, \\
he looks without seeing. \\
Though the teaching is spoken, \\
the fool doesn’t get it. 

Even\marginnote{4.1} if you lit ten lamps \\
and brought them to him, Kassapa, \\
he wouldn’t see anything, \\
for he has no eyes to see.” 

%
\end{verse}

Impelled\marginnote{5.1} by that deity, Venerable Kassapagotta was struck with a sense of urgency. 

%
\section*{{\suttatitleacronym SN 9.4}{\suttatitletranslation Several Mendicants Set Out Wandering }{\suttatitleroot Sambahulasutta}}
\addcontentsline{toc}{section}{\tocacronym{SN 9.4} \toctranslation{Several Mendicants Set Out Wandering } \tocroot{Sambahulasutta}}
\markboth{Several Mendicants Set Out Wandering }{Sambahulasutta}
\extramarks{SN 9.4}{SN 9.4}

At\marginnote{1.1} one time several mendicants were staying in the land of the Kosalans in a certain forest grove. 

Then\marginnote{1.2} after completing the three months of the rainy season residence, those mendicants set out wandering. Not seeing those mendicants, the deity haunting that forest cried. And on that occasion they recited this verse: 

\begin{verse}%
“Seeing\marginnote{2.1} so many vacated seats today, \\
it seems to me that they must have become dissatisfied. \\
They were so learned, such brilliant speakers! \\
Where have these disciples of Gotama gone?” 

%
\end{verse}

When\marginnote{3.1} they had spoken, another deity replied with this verse: 

\begin{verse}%
“They’ve\marginnote{4.1} gone to Magadha, they’ve gone to Kosala, \\
and some are in the Vajjian lands. \\
Like deer that wander free of ties, \\
the mendicants live with no abode.” 

%
\end{verse}

%
\section*{{\suttatitleacronym SN 9.5}{\suttatitletranslation With Ānanda }{\suttatitleroot Ānandasutta}}
\addcontentsline{toc}{section}{\tocacronym{SN 9.5} \toctranslation{With Ānanda } \tocroot{Ānandasutta}}
\markboth{With Ānanda }{Ānandasutta}
\extramarks{SN 9.5}{SN 9.5}

At\marginnote{1.1} one time Venerable Ānanda was staying in the land of the Kosalans in a certain forest grove. 

Now\marginnote{1.2} at that time Ānanda was spending too much time informing the lay people. Then the deity haunting that forest had compassion for Ānanda, wanting what’s best for him. So they approached him wanting to stir him up, and recited these verses: 

\begin{verse}%
“You’ve\marginnote{2.1} left for the jungle, the root of a tree, \\
with quenching in your heart. \\
Practice absorption, Gotama, don’t be negligent! \\
What is this hullabaloo to you?” 

%
\end{verse}

Impelled\marginnote{3.1} by that deity, Venerable Ānanda was struck with a sense of urgency. 

%
\section*{{\suttatitleacronym SN 9.6}{\suttatitletranslation With Anuruddha }{\suttatitleroot Anuruddhasutta}}
\addcontentsline{toc}{section}{\tocacronym{SN 9.6} \toctranslation{With Anuruddha } \tocroot{Anuruddhasutta}}
\markboth{With Anuruddha }{Anuruddhasutta}
\extramarks{SN 9.6}{SN 9.6}

At\marginnote{1.1} one time Venerable Anuruddha was staying in the land of the Kosalans in a certain forest grove. 

Then\marginnote{1.2} a certain deity of the company of the Thirty-Three named Penelope had been Anuruddha’s partner in a former life. She went up to Anuruddha, and recited these verses: 

\begin{verse}%
“Set\marginnote{2.1} your heart there, \\
where you used to live; \\
among the gods of the Thirty-Three, \\
whose every desire is granted! \\
At the fore of a retinue \\
of divine maidens, you’ll shine!” 

“Divine\marginnote{3.1} maidens are in a sorry state, \\
stuck in self-identity. \\
And those beings too are in a sorry state, \\
who are attached to divine maidens.” 

“They\marginnote{4.1} don’t know pleasure \\
who don’t see the Garden of Delight! \\
It’s the abode of lordly gods, \\
the glorious host of Thirty!” 

“Fool,\marginnote{5.1} don’t you understand \\
the saying of the perfected ones: \\
all conditions are impermanent, \\
their nature is to rise and fall; \\
having arisen, they cease; \\
their stilling is true bliss. 

Penelope,\marginnote{6.1} weaver of the web, \\
there’ll be no more lives in the hosts of gods. \\
Transmigration through births is finished, \\
now there’ll be no more future lives.” 

%
\end{verse}

%
\section*{{\suttatitleacronym SN 9.7}{\suttatitletranslation With Nāgadatta }{\suttatitleroot Nāgadattasutta}}
\addcontentsline{toc}{section}{\tocacronym{SN 9.7} \toctranslation{With Nāgadatta } \tocroot{Nāgadattasutta}}
\markboth{With Nāgadatta }{Nāgadattasutta}
\extramarks{SN 9.7}{SN 9.7}

At\marginnote{1.1} one time Venerable \textsanskrit{Nāgadatta} was staying in the land of the Kosalans in a certain forest grove. 

Now\marginnote{1.2} at that time Venerable \textsanskrit{Nāgadatta} had been entering the village too early and returning late in the day. Then the deity haunting that forest had compassion for \textsanskrit{Nāgadatta}, wanting what’s best for him. So they approached him wanting to stir him up, and recited these verses: 

\begin{verse}%
“Entering\marginnote{2.1} too early, \\
and returning after spending too much of the day, \\
\textsanskrit{Nāgadatta} socializes with lay people, \\
sharing their joys and sorrows. 

I’m\marginnote{3.1} afraid for \textsanskrit{Nāgadatta}; he’s so reckless \\
in his attachment to families. \\
May he not come under the King of Death’s power, \\
under the sway of the terminator!” 

%
\end{verse}

Impelled\marginnote{4.1} by that deity, Venerable \textsanskrit{Nāgadatta} was struck with a sense of urgency. 

%
\section*{{\suttatitleacronym SN 9.8}{\suttatitletranslation The Mistress of the House }{\suttatitleroot Kulagharaṇīsutta}}
\addcontentsline{toc}{section}{\tocacronym{SN 9.8} \toctranslation{The Mistress of the House } \tocroot{Kulagharaṇīsutta}}
\markboth{The Mistress of the House }{Kulagharaṇīsutta}
\extramarks{SN 9.8}{SN 9.8}

At\marginnote{1.1} one time one of the monks was staying in the land of the Kosalans in a certain forest grove. 

Now\marginnote{1.2} at that time that monk had become too closely involved in the affairs of a certain family. The deity haunting that forest had compassion for that monk, wanting what’s best for him. So, wanting to stir him up, they manifested in the appearance of the mistress of that family, approached the monk, and addressed him in verse: 

\begin{verse}%
“On\marginnote{2.1} the banks of the rivers and in the guest houses, \\
in meeting halls and highways, \\
people come together and gossip: \\
what’s going on between you and me?” 

“There\marginnote{3.1} are lots of annoying sounds \\
that an austere ascetic must endure. \\
But they mustn’t be dismayed by that, \\
for that’s not what defiles you. 

If\marginnote{4.1} you’re startled by every little sound, \\
like a wind-deer in the wood, \\
they’ll call you ‘flighty minded’; \\
and your practice won’t succeed.” 

%
\end{verse}

%
\section*{{\suttatitleacronym SN 9.9}{\suttatitletranslation A Vajji }{\suttatitleroot Vajjiputtasutta}}
\addcontentsline{toc}{section}{\tocacronym{SN 9.9} \toctranslation{A Vajji } \tocroot{Vajjiputtasutta}}
\markboth{A Vajji }{Vajjiputtasutta}
\extramarks{SN 9.9}{SN 9.9}

At\marginnote{1.1} one time a certain Vajjian mendicant was staying near \textsanskrit{Vesālī} in a certain forest grove. 

Now\marginnote{1.2} at that time the Vajjis were holding an all-night event in \textsanskrit{Vesālī}. Then that mendicant, groaning at the noise of musical instruments being beaten and played, on that occasion recited this verse: 

\begin{verse}%
“We\marginnote{2.1} dwell alone in the wilderness, \\
like a cast-off log in the forest. \\
On a night like this, \\
who’s worse off than me?” 

%
\end{verse}

The\marginnote{3.1} deity haunting that forest had compassion for that mendicant, and wanted what’s best for them. So they approached that mendicant wanting to stir them up, and addressed them in verse: 

\begin{verse}%
“You\marginnote{4.1} dwell alone in the wilderness, \\
like a cast-off log in the forest. \\
Lots of people are jealous of you, \\
like beings in hell of those going to heaven.” 

%
\end{verse}

Impelled\marginnote{5.1} by that deity, that mendicant was struck with a sense of urgency. 

%
\section*{{\suttatitleacronym SN 9.10}{\suttatitletranslation Recitation }{\suttatitleroot Sajjhāyasutta}}
\addcontentsline{toc}{section}{\tocacronym{SN 9.10} \toctranslation{Recitation } \tocroot{Sajjhāyasutta}}
\markboth{Recitation }{Sajjhāyasutta}
\extramarks{SN 9.10}{SN 9.10}

At\marginnote{1.1} one time one of the mendicants was staying in the land of the Kosalans in a certain forest grove. 

Now\marginnote{1.2} at that time that mendicant had previously been spending too much time in recitation. But some time later they adhered to passivity and silence. Not hearing the teaching, the deity haunting that forest approached that mendicant, and addressed them in verse: 

\begin{verse}%
“Mendicant,\marginnote{2.1} why don’t you recite passages of the teaching, \\
living together with other mendicants? \\
When you hear the teaching confidence grows; \\
and the reciter is praised in the present life.” 

“I\marginnote{3.1} used to be enthusiastic about passages of the teaching, \\
so long as I’d not realized dispassion. \\
But then I realized dispassion, which the good call \\
the laying to rest by completely understanding \\
whatever is seen, heard, and thought.” 

%
\end{verse}

%
\section*{{\suttatitleacronym SN 9.11}{\suttatitletranslation Unskillful Thoughts }{\suttatitleroot Akusalavitakkasutta}}
\addcontentsline{toc}{section}{\tocacronym{SN 9.11} \toctranslation{Unskillful Thoughts } \tocroot{Akusalavitakkasutta}}
\markboth{Unskillful Thoughts }{Akusalavitakkasutta}
\extramarks{SN 9.11}{SN 9.11}

At\marginnote{1.1} one time one of the mendicants was staying in the land of the Kosalans in a certain forest grove. 

Now\marginnote{1.2} at that time that mendicant, during their day’s meditation, was thinking bad, unskillful thoughts, that is: sensual, malicious, and cruel thoughts. The deity haunting that forest had compassion for that mendicant, and wanted what’s best for them. So they approached that mendicant wanting to stir them up, and addressed them in verse: 

\begin{verse}%
“Because\marginnote{2.1} of improper attention, \\
you’re consumed by your thoughts. \\
When you’ve given up irrationality, \\
make sure your thoughts are rational. 

Thinking\marginnote{3.1} about the Teacher, the teaching, \\
the \textsanskrit{Saṅgha}, and your own ethics, \\
you’ll find gladness, \\
and rapture and bliss as well, no doubt. \\
And when you’re full of joy, \\
you’ll make an end to suffering.” 

%
\end{verse}

Impelled\marginnote{4.1} by that deity, that mendicant was struck with a sense of urgency. 

%
\section*{{\suttatitleacronym SN 9.12}{\suttatitletranslation Midday }{\suttatitleroot Majjhanhikasutta}}
\addcontentsline{toc}{section}{\tocacronym{SN 9.12} \toctranslation{Midday } \tocroot{Majjhanhikasutta}}
\markboth{Midday }{Majjhanhikasutta}
\extramarks{SN 9.12}{SN 9.12}

At\marginnote{1.1} one time one of the mendicants was staying in the land of the Kosalans in a certain forest grove. The deity haunting that forest approached that mendicant and recited this verse in their presence: 

\begin{verse}%
“In\marginnote{2.1} the still of high noon, \\
when the birds have settled down, \\
the formidable jungle whispers to itself: \\
that seems so scary to me!” 

“In\marginnote{3.1} the still of high noon, \\
when the birds have settled down, \\
the formidable jungle whispers to itself: \\
that seems so delightful to me!” 

%
\end{verse}

%
\section*{{\suttatitleacronym SN 9.13}{\suttatitletranslation Undisciplined Faculties }{\suttatitleroot Pākatindriyasutta}}
\addcontentsline{toc}{section}{\tocacronym{SN 9.13} \toctranslation{Undisciplined Faculties } \tocroot{Pākatindriyasutta}}
\markboth{Undisciplined Faculties }{Pākatindriyasutta}
\extramarks{SN 9.13}{SN 9.13}

At\marginnote{1.1} one time several mendicants were staying in the Kosalan lands in a certain forest grove. They were restless, insolent, fickle, scurrilous, loose-tongued, unmindful, lacking situational awareness and immersion, with straying minds and undisciplined faculties. 

The\marginnote{1.2} deity haunting that forest had compassion for those mendicants, and wanted what’s best for them. So they approached those mendicants wanting to stir them up, and addressed them in verse: 

\begin{verse}%
“The\marginnote{2.1} mendicants used to live happily, \\
as disciples of Gotama. \\
Desireless they sought alms; \\
desireless they used their lodgings. \\
Knowing that the world was impermanent \\
they made an end of suffering. 

But\marginnote{3.1} now they’ve made themselves hard to look after, \\
like chiefs in a village. \\
They eat and eat and then lie down, \\
unconscious in the homes of others. 

Having\marginnote{4.1} raised my joined palms to the \textsanskrit{Saṅgha}, \\
I speak here only about certain people. \\
They’re rejects, with no protector, \\
just like those who have passed away. 

I’m\marginnote{5.1} speaking about \\
those who live negligently. \\
To those who live diligently \\
I pay homage.” 

%
\end{verse}

Impelled\marginnote{6.1} by that deity, those mendicants were struck with a sense of urgency. 

%
\section*{{\suttatitleacronym SN 9.14}{\suttatitletranslation The Thief of Scent }{\suttatitleroot Gandhatthenasutta}}
\addcontentsline{toc}{section}{\tocacronym{SN 9.14} \toctranslation{The Thief of Scent } \tocroot{Gandhatthenasutta}}
\markboth{The Thief of Scent }{Gandhatthenasutta}
\extramarks{SN 9.14}{SN 9.14}

At\marginnote{1.1} one time one of the mendicants was staying in the land of the Kosalans in a certain forest grove. 

Now\marginnote{1.2} at that time, after the meal, on their return from almsround, that mendicant plunged into a lotus pond and sniffed a pink lotus. The deity haunting that forest had compassion for that mendicant, and wanted what’s best for them. So they approached that mendicant wanting to stir them up, and addressed them in verse: 

\begin{verse}%
“This\marginnote{2.1} water flower has not been given. \\
When you sniff it, \\
this is one factor of theft. \\
Good sir, you are a thief of scent!” 

“I\marginnote{3.1} do not take, nor do I break; \\
I sniff the water flower from afar. \\
So based on what evidence \\
do you call me a thief of scent? 

Why\marginnote{4.1} don’t you accuse someone \\
who does such vandalizing \\
as digging up the roots, \\
or breaking off the flowers?” 

“I\marginnote{5.1} have nothing to say \\
to a person who is a crude vandal, \\
soiled like a used nappy. \\
You’re the one who deserves to be spoken to. 

To\marginnote{6.1} the man who has not a blemish \\
who is always seeking purity, \\
even a hair-tip of evil \\
seems as big as a cloud.” 

“Indeed,\marginnote{7.1} O spirit, you understand me, \\
and you empathize with me. \\
Please speak to me again, \\
whenever you see something like this.” 

“I’m\marginnote{8.1} no dependent of yours, \\
nor am I your servant. \\
You yourself should know, mendicant, \\
the way that leads to a good place.” 

%
\end{verse}

Impelled\marginnote{9.1} by that deity, that mendicant was struck with a sense of urgency. 

\scendsutta{The Linked Discourses in the Forest are completed. }

%
\addtocontents{toc}{\let\protect\contentsline\protect\nopagecontentsline}
\part*{Linked Discourses with Spirits }
\addcontentsline{toc}{part}{Linked Discourses with Spirits }
\markboth{}{}
\addtocontents{toc}{\let\protect\contentsline\protect\oldcontentsline}

%
\addtocontents{toc}{\let\protect\contentsline\protect\nopagecontentsline}
\chapter*{The Chapter with Indaka }
\addcontentsline{toc}{chapter}{\tocchapterline{The Chapter with Indaka }}
\addtocontents{toc}{\let\protect\contentsline\protect\oldcontentsline}

%
\section*{{\suttatitleacronym SN 10.1}{\suttatitletranslation With Indaka }{\suttatitleroot Indakasutta}}
\addcontentsline{toc}{section}{\tocacronym{SN 10.1} \toctranslation{With Indaka } \tocroot{Indakasutta}}
\markboth{With Indaka }{Indakasutta}
\extramarks{SN 10.1}{SN 10.1}

\scevam{So\marginnote{1.1} I have heard. }At one time the Buddha was staying near \textsanskrit{Rājagaha} on Mount Indra’s Peak, the haunt of the native spirit Indaka. 

Then\marginnote{1.3} the native spirit Indaka went up to the Buddha, and addressed him in verse: 

\begin{verse}%
“The\marginnote{2.1} Buddhas say that form is not the soul. \\
Then how does this body manifest? \\
Where do the bones and liver come from? \\
And how does one cling on in the womb?” 

“First\marginnote{3.1} there’s a drop of coagulate; \\
from there a little bud appears; \\
next it becomes a piece of flesh; \\
which produces a swelling. \\
From that swelling the limbs appear, \\
the head hair, body hair, and teeth. 

And\marginnote{4.1} whatever the mother eats—\\
the food and drink that she consumes—\\
nourishes them there, \\
the person in the mother’s womb.” 

%
\end{verse}

%
\section*{{\suttatitleacronym SN 10.2}{\suttatitletranslation With a Spirit Named Sakka }{\suttatitleroot Sakkanāmasutta}}
\addcontentsline{toc}{section}{\tocacronym{SN 10.2} \toctranslation{With a Spirit Named Sakka } \tocroot{Sakkanāmasutta}}
\markboth{With a Spirit Named Sakka }{Sakkanāmasutta}
\extramarks{SN 10.2}{SN 10.2}

At\marginnote{1.1} one time the Buddha was staying near \textsanskrit{Rājagaha}, on the Vulture’s Peak Mountain. 

Then\marginnote{1.2} a spirit named Sakka went up to the Buddha, and addressed him in verse: 

\begin{verse}%
“You’ve\marginnote{2.1} given up all ties, \\
and are fully freed. \\
It’s not a good idea for you, ascetic, \\
to be instructing others.” 

“No\marginnote{3.1} matter what the apparent reason \\
why people are together, Sakka, \\
it’s unworthy for a wise person \\
to not think of the other with compassion. 

If\marginnote{4.1} you instruct others \\
with a mind clear and confident, \\
your compassion and empathy \\
don’t create attachments.” 

%
\end{verse}

%
\section*{{\suttatitleacronym SN 10.3}{\suttatitletranslation With Spiky }{\suttatitleroot Sūcilomasutta}}
\addcontentsline{toc}{section}{\tocacronym{SN 10.3} \toctranslation{With Spiky } \tocroot{Sūcilomasutta}}
\markboth{With Spiky }{Sūcilomasutta}
\extramarks{SN 10.3}{SN 10.3}

At\marginnote{1.1} one time the Buddha was staying near \textsanskrit{Gayā} on the cut-stone ledge in the haunt of Spiky the native spirit. 

Now\marginnote{1.2} at that time the native spirits Shaggy and Spiky were passing by not far from the Buddha. 

So\marginnote{1.3} Shaggy said to Spiky, “That’s an ascetic.” 

“That’s\marginnote{1.5} no ascetic, he’s a faker! I’ll soon find out whether he’s an ascetic or a faker.” 

Then\marginnote{2.1} Spiky went up to the Buddha and leaned up against his body, but the Buddha pulled away. 

Then\marginnote{2.3} Spiky said to the Buddha, “Are you afraid, ascetic?” 

“No,\marginnote{2.5} sir, I’m not afraid. But your touch is nasty.” 

“I\marginnote{2.7} will ask you a question, ascetic. If you don’t answer me, I’ll drive you insane, or explode your heart, or grab you by the feet and throw you to the far shore of the Ganges!” 

“I\marginnote{2.9} don’t see anyone in this world with its gods, \textsanskrit{Māras}, and \textsanskrit{Brahmās}, this population with its ascetics and brahmins, its gods and humans who could do that to me. But anyway, ask what you wish.” 

\begin{verse}%
“Where\marginnote{3.1} do greed and hate come from? \\
From where spring discontent, desire, and terror? \\
Where do the mind’s thoughts originate, \\
like a crow let loose by boys.” 

“Greed\marginnote{4.1} and hate come from here; \\
from here spring discontent, desire, and terror; \\
here’s where the mind’s thoughts originate, \\
like a crow let loose by boys. 

Born\marginnote{5.1} of affection, originating in oneself, \\
like the shoots from a banyan’s trunk; \\
the many kinds of attachment to sensual pleasures \\
are like camel’s foot creeper strung through the woods. 

Those\marginnote{6.1} who understand where they come from \\
get rid of them—listen up, spirit! \\
They cross this flood so hard to cross, \\
not crossed before, so as to not be reborn.” 

%
\end{verse}

%
\section*{{\suttatitleacronym SN 10.4}{\suttatitletranslation With Maṇibhadda }{\suttatitleroot Maṇibhaddasutta}}
\addcontentsline{toc}{section}{\tocacronym{SN 10.4} \toctranslation{With Maṇibhadda } \tocroot{Maṇibhaddasutta}}
\markboth{With Maṇibhadda }{Maṇibhaddasutta}
\extramarks{SN 10.4}{SN 10.4}

At\marginnote{1.1} one time the Buddha was staying in the land of the Magadhans at the \textsanskrit{Maṇimālika} Tree-shrine, the haunt of the native spirit \textsanskrit{Maṇibhadda}. 

Then\marginnote{1.2} the native spirit \textsanskrit{Maṇibhadda} went up to the Buddha, and recited this verse in the Buddha’s presence: 

\begin{verse}%
“It’s\marginnote{2.1} always auspicious for the mindful; \\
the mindful prosper in happiness. \\
Each new day is better for the mindful, \\
and they’re freed from enmity.” 

“It’s\marginnote{3.1} always auspicious for the mindful; \\
the mindful prosper in happiness. \\
Each new day is better for the mindful, \\
but they’re not freed from enmity. 

But\marginnote{4.1} someone whose mind delights in harmlessness, \\
all day and all night, \\
with love for all living creatures—\\
they have no enmity for anyone.” 

%
\end{verse}

%
\section*{{\suttatitleacronym SN 10.5}{\suttatitletranslation With Sānu }{\suttatitleroot Sānusutta}}
\addcontentsline{toc}{section}{\tocacronym{SN 10.5} \toctranslation{With Sānu } \tocroot{Sānusutta}}
\markboth{With Sānu }{Sānusutta}
\extramarks{SN 10.5}{SN 10.5}

At\marginnote{1.1} one time the Buddha was staying near \textsanskrit{Sāvatthī} in Jeta’s Grove, \textsanskrit{Anāthapiṇḍika}’s monastery. 

Now\marginnote{1.2} at that time a certain lay woman had a son named \textsanskrit{Sānu} who had been possessed by a native spirit. And as that lay woman wept, on that occasion she recited these verses: 

\begin{verse}%
“I\marginnote{2.1} have heard this from the perfected ones. \\
The native spirits will not mess with anyone \\
who leads the spiritual life \\
by observing the sabbath 

complete\marginnote{3.1} in all eight factors \\
on the fourteenth and the fifteenth days, \\
and the eighth day of the fortnight, \\
as well as on the fortnight of special displays. \\
But now today I see \\
native spirits messing with \textsanskrit{Sānu}.” 

“What\marginnote{4.1} you heard from the perfected ones is right. \\
The native spirits will not mess with anyone \\
who leads the spiritual life \\
by observing the sabbath 

complete\marginnote{5.1} in all eight factors \\
on the fourteenth and the fifteenth days, \\
and the eighth day of the fortnight, \\
as well as on the fortnight of special displays. 

When\marginnote{6.1} \textsanskrit{Sānu} regains consciousness tell him \\
this saying of the native spirits: \\
Don’t do bad deeds \\
either openly or in secret. 

If\marginnote{7.1} you should do a bad deed, \\
or you’re doing one now, \\
you won’t be freed from suffering, \\
though you fly away and flee.” 

“Mum,\marginnote{8.1} they cry for the dead, \\
or for one who’s alive but has disappeared. \\
I’m alive and you can see me, \\
so mum, why do you weep for me?” 

“Son,\marginnote{9.1} they cry for the dead, \\
or for one who’s alive but has disappeared. \\
But someone who has given up sensual pleasures \\
only to come back here again: \\
they cry for them as well, \\
for though still alive they’re really dead. 

My\marginnote{10.1} dear, you’ve been rescued from hot coals, \\
and you want to plunge right back in them! \\
My dear, you’ve been rescued from the abyss, \\
and you want to plunge right back there! 

Keep\marginnote{11.1} pushing forward, it’s what’s best for you! \\
Who have I got to complain to? \\
When your things have been saved from a fire, \\
would you want them to be burnt again?” 

%
\end{verse}

%
\section*{{\suttatitleacronym SN 10.6}{\suttatitletranslation With Piyaṅkara }{\suttatitleroot Piyaṅkarasutta}}
\addcontentsline{toc}{section}{\tocacronym{SN 10.6} \toctranslation{With Piyaṅkara } \tocroot{Piyaṅkarasutta}}
\markboth{With Piyaṅkara }{Piyaṅkarasutta}
\extramarks{SN 10.6}{SN 10.6}

At\marginnote{1.1} one time Venerable \textsanskrit{Sāriputta} was staying near \textsanskrit{Sāvatthī} in Jeta’s Grove, \textsanskrit{Anāthapiṇḍika}’s monastery. 

Now\marginnote{1.2} at that time Venerable Anuruddha rose at the crack of dawn and recited passages of the teaching. Then the native spirit \textsanskrit{Piyaṅkara}’s Mother soothed her little child, saying: 

\begin{verse}%
“Don’t\marginnote{2.1} make a sound, \textsanskrit{Piyaṅkara}! \\
A mendicant recites passages of the teaching. \\
When we understand a passage, \\
we can practice for our welfare. 

Let\marginnote{3.1} us keep from harming living creatures, \\
and speak no lying words. \\
We should train ourselves well in ethics, \\
and hopefully we’ll be freed from the goblin realm.” 

%
\end{verse}

%
\section*{{\suttatitleacronym SN 10.7}{\suttatitletranslation With Punabbasu }{\suttatitleroot Punabbasusutta}}
\addcontentsline{toc}{section}{\tocacronym{SN 10.7} \toctranslation{With Punabbasu } \tocroot{Punabbasusutta}}
\markboth{With Punabbasu }{Punabbasusutta}
\extramarks{SN 10.7}{SN 10.7}

At\marginnote{1.1} one time the Buddha was staying near \textsanskrit{Sāvatthī} in Jeta’s Grove, \textsanskrit{Anāthapiṇḍika}’s monastery. 

Now\marginnote{1.2} at that time the Buddha was educating, encouraging, firing up, and inspiring the mendicants with a Dhamma talk about extinguishment. And those mendicants were paying heed, paying attention, engaging wholeheartedly, and lending an ear. 

Then\marginnote{1.4} the native spirit Punabbasu’s Mother soothed her little children, saying: 

\begin{verse}%
“Hush,\marginnote{2.1} little \textsanskrit{Uttarā}! \\
Hush, Punabbasu! \\
For I want to listen to the teaching \\
of the Teacher, the supreme Buddha. 

Since\marginnote{3.1} the Blessed One spoke of extinguishment, \\
the release from all ties, \\
I have a lasting love \\
for this teaching. 

In\marginnote{4.1} this world, your own child is dear; \\
in this world, your own husband is dear; \\
but even greater than that is my love \\
for this teaching’s quest. 

For\marginnote{5.1} neither son nor husband, \\
dear as they are, can free you from suffering; \\
as listening to the true teaching \\
frees living creatures from suffering. 

In\marginnote{6.1} this world mired in suffering, \\
fettered by old age and death, \\
I want to listen to the teaching \\
that the Buddha awakened to, \\
which frees you from old age and death. \\
So hush, Punabbasu!” 

“Mom,\marginnote{7.1} I’m not speaking, \\
and \textsanskrit{Uttarā} is silent, too. \\
Focus just on the teaching, \\
for it’s nice to listen to the true teaching. \\
And it’s because we haven’t understood the teaching \\
that we live in suffering, Mom. 

He\marginnote{8.1} is a beacon for those who are lost \\
among gods and humans. \\
The Buddha, bearing his final body, \\
the Seer teaches Dhamma.” 

“It’s\marginnote{9.1} good that my child’s so astute, \\
this child I bore and suckled! \\
My child loves the pure teaching \\
of the supreme Buddha. 

Punabbasu,\marginnote{10.1} may you be happy! \\
Today, I rise. \\
Hear me too, \textsanskrit{Uttarā}: \\
I have seen the noble truths!” 

%
\end{verse}

%
\section*{{\suttatitleacronym SN 10.8}{\suttatitletranslation With Sudatta }{\suttatitleroot Sudattasutta}}
\addcontentsline{toc}{section}{\tocacronym{SN 10.8} \toctranslation{With Sudatta } \tocroot{Sudattasutta}}
\markboth{With Sudatta }{Sudattasutta}
\extramarks{SN 10.8}{SN 10.8}

At\marginnote{1.1} one time the Buddha was staying near \textsanskrit{Rājagaha} in the Cool Grove. 

Now\marginnote{1.2} at that time the householder \textsanskrit{Anāthapiṇḍika} had arrived at \textsanskrit{Rājagaha} on some business. He heard a rumor that a Buddha had arisen in the world. 

Right\marginnote{1.5} away he wanted to go and see the Buddha, but he thought, “It’s too late to go and see the Buddha today. I’ll go and see him tomorrow.” He went to bed thinking of the Buddha. 

During\marginnote{1.8} the night he got up three times thinking it was morning. Then he approached the Sivaka Gate, and non-human beings opened it for him. 

But\marginnote{1.11} as he was leaving the city, light vanished and darkness appeared to him. He felt fear, terror, and goosebumps, and wanted to turn back. 

Then\marginnote{1.12} the invisible spirit Sivaka called out: 

\begin{verse}%
“A\marginnote{2.1} hundred elephants, a hundred horses, \\
a hundred mule-drawn chariots, \\
a hundred thousand maidens \\
bedecked with jeweled earrings: \\
these are not worth a sixteenth part \\
of a single forward stride! 

Forward,\marginnote{3.1} householder! \\
Forward, householder! \\
Going forward is better for you, \\
not turning back!” 

%
\end{verse}

Then\marginnote{4.1} darkness vanished and light appeared to \textsanskrit{Anāthapiṇḍika}. His fear, terror, and goosebumps settled down. 

But\marginnote{4.2} for a second time, light vanished and darkness appeared to him. … 

For\marginnote{4.3} a second time the invisible spirit Sivaka called out … 

\begin{verse}%
“…\marginnote{6.1} Going forward is better for you, \\
not turning back!” 

%
\end{verse}

Then\marginnote{7.1} darkness vanished and light appeared to \textsanskrit{Anāthapiṇḍika}. His fear, terror, and goosebumps settled down. 

But\marginnote{7.2} for a third time, light vanished and darkness appeared to him. … 

For\marginnote{7.3} a third time the invisible spirit Sivaka called out … 

\begin{verse}%
“…\marginnote{9.1} Going forward is better for you, \\
not turning back!” 

%
\end{verse}

Then\marginnote{10.1} darkness vanished and light appeared to \textsanskrit{Anāthapiṇḍika}. His fear, terror, and goosebumps settled down. Then the householder \textsanskrit{Anāthapiṇḍika} went to the Cool Grove and approached the Buddha. 

Now\marginnote{11.1} at that time the Buddha had risen at the crack of dawn and was walking mindfully in the open. He saw \textsanskrit{Anāthapiṇḍika} coming off in the distance. So he stepped down from the walking path, sat down on the seat spread out, and said to \textsanskrit{Anāthapiṇḍika}, “Come, Sudatta.” 

Then\marginnote{11.6} \textsanskrit{Anāthapiṇḍika} thought, “The Buddha calls me by name!” Smiling and elated, he bowed with his head at the Buddha’s feet and said to him, “Sir, I trust the Buddha slept well?” 

\begin{verse}%
“A\marginnote{12.1} brahmin who is fully extinguished \\
always sleeps at ease. \\
Sensual pleasures slip off them, \\
they’re cooled, free of attachments. 

Since\marginnote{13.1} they’ve cut off all clinging, \\
and removed the stress from the heart, \\
the peaceful sleep at ease, \\
having found peace of mind.” 

%
\end{verse}

%
\section*{{\suttatitleacronym SN 10.9}{\suttatitletranslation With the Nun Sukkā (1st) }{\suttatitleroot Paṭhamasukkāsutta}}
\addcontentsline{toc}{section}{\tocacronym{SN 10.9} \toctranslation{With the Nun Sukkā (1st) } \tocroot{Paṭhamasukkāsutta}}
\markboth{With the Nun Sukkā (1st) }{Paṭhamasukkāsutta}
\extramarks{SN 10.9}{SN 10.9}

At\marginnote{1.1} one time the Buddha was staying near \textsanskrit{Rājagaha}, in the Bamboo Grove, the squirrels’ feeding ground. 

Now,\marginnote{1.2} at that time the nun \textsanskrit{Sukkā} was teaching Dhamma, surrounded by a large assembly. Then a native spirit was so devoted to \textsanskrit{Sukkā} that he went from street to street and from square to square, and on that occasion recited these verses: 

\begin{verse}%
“What’s\marginnote{2.1} up with these people in \textsanskrit{Rājagaha}? \\
They sleep like they’ve been drinking mead! \\
They don’t attend on \textsanskrit{Sukkā} \\
as she’s teaching the deathless state. 

But\marginnote{3.1} the wise—\\
it’s as if they drink it up, \\
so irresistible, delicious, and nutritious, \\
like travelers enjoying a cool cloud.” 

%
\end{verse}

%
\section*{{\suttatitleacronym SN 10.10}{\suttatitletranslation With the Nun Sukkā (2nd) }{\suttatitleroot Dutiyasukkāsutta}}
\addcontentsline{toc}{section}{\tocacronym{SN 10.10} \toctranslation{With the Nun Sukkā (2nd) } \tocroot{Dutiyasukkāsutta}}
\markboth{With the Nun Sukkā (2nd) }{Dutiyasukkāsutta}
\extramarks{SN 10.10}{SN 10.10}

At\marginnote{1.1} one time the Buddha was staying near \textsanskrit{Rājagaha}, in the Bamboo Grove, the squirrels’ feeding ground. 

Now\marginnote{1.2} at that time a certain lay follower gave food to the nun \textsanskrit{Sukkā}. Then a native spirit was so devoted to \textsanskrit{Sukkā} that he went from street to street and from square to square, and on that occasion recited these verses: 

\begin{verse}%
“O!\marginnote{2.1} He has made so much merit! \\
That lay follower is so very wise. \\
He just gave food to \textsanskrit{Sukkā}, \\
who is released from all ties.” 

%
\end{verse}

%
\section*{{\suttatitleacronym SN 10.11}{\suttatitletranslation With the Nun Cīrā }{\suttatitleroot Cīrāsutta}}
\addcontentsline{toc}{section}{\tocacronym{SN 10.11} \toctranslation{With the Nun Cīrā } \tocroot{Cīrāsutta}}
\markboth{With the Nun Cīrā }{Cīrāsutta}
\extramarks{SN 10.11}{SN 10.11}

\scevam{So\marginnote{1.1} I have heard. }At one time the Buddha was staying near \textsanskrit{Rājagaha}, in the Bamboo Grove, the squirrels’ feeding ground. 

Now\marginnote{1.3} at that time a certain lay follower gave a robe to the nun \textsanskrit{Cīrā}. Then a native spirit was so devoted to \textsanskrit{Cīrā} that he went from street to street and from square to square, and on that occasion recited these verses: 

\begin{verse}%
“O!\marginnote{2.1} He has made so much merit! \\
That lay-follower is so very wise. \\
He gave a robe to \textsanskrit{Cīrā}, \\
who is released from all bonds.” 

%
\end{verse}

%
\section*{{\suttatitleacronym SN 10.12}{\suttatitletranslation With Āḷavaka }{\suttatitleroot Āḷavakasutta}}
\addcontentsline{toc}{section}{\tocacronym{SN 10.12} \toctranslation{With Āḷavaka } \tocroot{Āḷavakasutta}}
\markboth{With Āḷavaka }{Āḷavakasutta}
\extramarks{SN 10.12}{SN 10.12}

\scevam{So\marginnote{1.1} I have heard. }At one time the Buddha was staying near \textsanskrit{Āḷavī} in the haunt of the native spirit \textsanskrit{Āḷavaka}. 

Then\marginnote{1.3} the native spirit \textsanskrit{Āḷavaka} went up to the Buddha, and said to him: “Get out, ascetic!” 

Saying,\marginnote{1.5} “All right, sir,” the Buddha went out. 

“Get\marginnote{1.6} in, ascetic!” 

Saying,\marginnote{1.7} “All right, sir,” the Buddha went in. 

And\marginnote{1.8} for a second time the native spirit \textsanskrit{Āḷavaka} said to the Buddha, “Get out, ascetic!” 

Saying,\marginnote{1.10} “All right, sir,” the Buddha went out. 

“Get\marginnote{1.11} in, ascetic!” 

Saying,\marginnote{1.12} “All right, sir,” the Buddha went in. 

And\marginnote{1.13} for a third time the native spirit \textsanskrit{Āḷavaka} said to the Buddha, “Get out, ascetic!” 

Saying,\marginnote{1.15} “All right, sir,” the Buddha went out. 

“Get\marginnote{1.16} in, ascetic!” 

Saying,\marginnote{1.17} “All right, sir,” the Buddha went in. 

And\marginnote{1.18} for a fourth time the native spirit \textsanskrit{Āḷavaka} said to the Buddha, 

“Get\marginnote{1.19} out, ascetic!” 

“No,\marginnote{1.20} sir, I won’t get out. Do what you must.” 

“I\marginnote{1.22} will ask you a question, ascetic. If you don’t answer me, I’ll drive you insane, or explode your heart, or grab you by the feet and throw you to the far shore of the Ganges!” 

“I\marginnote{1.24} don’t see anyone in this world with its gods, \textsanskrit{Māras}, and \textsanskrit{Brahmās}, this population with its ascetics and brahmins, its gods and humans who could do that to me. But anyway, ask what you wish.” 

\begin{verse}%
“What’s\marginnote{2.1} a person’s best wealth? \\
What brings happiness when practiced well? \\
What’s the sweetest taste of all? \\
The one who they say has the best life: how do they live?” 

“Faith\marginnote{3.1} here is a person’s best wealth. \\
The teaching brings happiness when practiced well. \\
Truth is the sweetest taste of all. \\
The one who they say has the best life lives by wisdom.” 

“How\marginnote{4.1} do you cross the flood? \\
How do you cross the deluge? \\
How do you get over suffering? \\
How do you get purified?” 

“By\marginnote{5.1} faith you cross the flood, \\
and by diligence the deluge. \\
By energy you get past suffering, \\
and you’re purified by wisdom.” 

“How\marginnote{6.1} do you get wisdom? \\
How do you earn wealth? \\
How do you get a good reputation? \\
How do you hold on to friends? \\
How do the departed not grieve \\
when passing from this world to the next?” 

“One\marginnote{7.1} who is diligent and discerning \\
gains wisdom by wanting to learn, \\
having faith in the perfected ones, \\
and the teaching for becoming extinguished. 

Being\marginnote{8.1} responsible, acting appropriately, \\
and working hard you earn wealth. \\
Truthfulness wins you a good reputation. \\
You hold on to friends by giving. \\
That’s how the departed do not grieve \\
when passing from this world to the next. 

A\marginnote{9.1} faithful householder \\
who has these four qualities \\
does not grieve after passing away: \\
truth, principle, steadfastness, and generosity. 

Go\marginnote{10.1} ahead, ask others as well, \\
there are many ascetics and brahmins. \\
See whether anything better is found \\
than truth, self-control, generosity, and patience.” 

“Why\marginnote{11.1} now would I question \\
the many ascetics and brahmins? \\
Today I understand \\
what’s good for the next life. 

It\marginnote{12.1} was truly for my benefit \\
that the Buddha came to stay at \textsanskrit{Āḷavī}. \\
Today I understand \\
where a gift is very fruitful. 

I\marginnote{13.1} myself will journey \\
village to village, town to town, \\
paying homage to the Buddha, \\
and the natural excellence of the teaching!” 

%
\end{verse}

\scendsutta{The Linked Discourses with Native Spirits are complete. }

%
\addtocontents{toc}{\let\protect\contentsline\protect\nopagecontentsline}
\part*{Linked Discourses with Sakka }
\addcontentsline{toc}{part}{Linked Discourses with Sakka }
\markboth{}{}
\addtocontents{toc}{\let\protect\contentsline\protect\oldcontentsline}

%
\addtocontents{toc}{\let\protect\contentsline\protect\nopagecontentsline}
\chapter*{Chapter One }
\addcontentsline{toc}{chapter}{\tocchapterline{Chapter One }}
\addtocontents{toc}{\let\protect\contentsline\protect\oldcontentsline}

%
\section*{{\suttatitleacronym SN 11.1}{\suttatitletranslation With Suvīra }{\suttatitleroot Suvīrasutta}}
\addcontentsline{toc}{section}{\tocacronym{SN 11.1} \toctranslation{With Suvīra } \tocroot{Suvīrasutta}}
\markboth{With Suvīra }{Suvīrasutta}
\extramarks{SN 11.1}{SN 11.1}

\scevam{So\marginnote{1.1} I have heard. }At one time the Buddha was staying near \textsanskrit{Sāvatthī} in Jeta’s Grove, \textsanskrit{Anāthapiṇḍika}’s monastery. There the Buddha addressed the mendicants, “Mendicants!” 

“Venerable\marginnote{1.5} sir,” they replied. The Buddha said this: 

“Once\marginnote{2.1} upon a time, mendicants, the demons marched against the gods. 

Then\marginnote{2.2} Sakka, lord of gods, addressed the god \textsanskrit{Suvīra}, ‘Dear \textsanskrit{Suvīra}, the demons march against the gods! Go, and march against the demons!’ 

‘Yes,\marginnote{2.5} lord,’ replied \textsanskrit{Suvīra}. But he fell into negligence. 

For\marginnote{2.6} a second time Sakka addressed \textsanskrit{Suvīra}, ‘Dear \textsanskrit{Suvīra}, the demons march against the gods! Go, and march against the demons!’ 

‘Yes,\marginnote{2.9} lord,’ replied \textsanskrit{Suvīra}. But for a second time he fell into negligence. 

For\marginnote{2.10} a third time Sakka addressed \textsanskrit{Suvīra}, ‘Dear \textsanskrit{Suvīra}, the demons march against the gods! Go, and march against the demons!’ 

‘Yes,\marginnote{2.13} lord,’ replied \textsanskrit{Suvīra}. But for a third time he fell into negligence. 

Then\marginnote{2.14} Sakka addressed the god \textsanskrit{Suvīra} in verse: 

\begin{verse}%
‘\textsanskrit{Suvīra},\marginnote{3.1} go to that place \\
where you can achieve happiness \\
without working for it or trying hard—\\
and take me with you!’ 

‘That\marginnote{4.1} a lazy man who does no work, \\
and doesn’t do his duty, \\
should fulfill all his desires: \\
Sakka, grant me this boon!’ 

‘\textsanskrit{Suvīra},\marginnote{5.1} go to that place \\
where a lazy man who does no work \\
prospers in unending happiness—\\
and take me with you!’ 

‘O\marginnote{6.1} Sakka, first among gods, \\
that we might find the happiness \\
that’s sorrowless, unstressed: \\
Sakka, grant me this boon!’ 

‘If\marginnote{7.1} there exists anywhere a place \\
where one can live happily without working, \\
that surely would be extinguishment’s path! \\
Go there, \textsanskrit{Suvīra}, \\
and take me with you!’ 

%
\end{verse}

So,\marginnote{8.1} mendicants, even Sakka, lord of gods—while living off of the fruit of his good and bad deeds, and ruling as sovereign lord over these gods of the Thirty-Three—will speak in praise of initiative and energy. But since you have gone forth in such a well explained teaching and training, it would be truly beautiful for you to try hard, strive, and make an effort to attain the unattained, achieve the unachieved, and realize the unrealized!” 

%
\section*{{\suttatitleacronym SN 11.2}{\suttatitletranslation With Susīma }{\suttatitleroot Susīmasutta}}
\addcontentsline{toc}{section}{\tocacronym{SN 11.2} \toctranslation{With Susīma } \tocroot{Susīmasutta}}
\markboth{With Susīma }{Susīmasutta}
\extramarks{SN 11.2}{SN 11.2}

At\marginnote{1.1} \textsanskrit{Sāvatthī}. 

There\marginnote{1.2} the Buddha addressed the mendicants, “Mendicants!” 

“Venerable\marginnote{1.4} sir,” they replied. The Buddha said this: 

“Once\marginnote{2.1} upon a time, mendicants, the demons marched against the gods. 

Then\marginnote{2.2} Sakka, lord of gods, addressed the god \textsanskrit{Susīma}, ‘Dear \textsanskrit{Susīma}, the demons march against the gods! Go, and march against the demons!’ 

‘Yes,\marginnote{2.5} lord,’ replied \textsanskrit{Susīma}. But he fell into negligence. 

For\marginnote{2.6} a second time … 

For\marginnote{2.8} a third time … 

Then\marginnote{2.10} Sakka addressed the god \textsanskrit{Susīma} in verse: 

\begin{verse}%
‘\textsanskrit{Susīma},\marginnote{3.1} go to that place \\
where you can achieve happiness \\
without working for it or trying hard—\\
and take me with you!’ 

‘That\marginnote{4.1} a lazy man who does no work, \\
and doesn’t do his duty, \\
should fulfill all his desires: \\
Sakka, grant me this boon!’ 

‘\textsanskrit{Susīma},\marginnote{5.1} go to that place \\
where a lazy man who does no work \\
prospers in unending happiness—\\
and take me with you!’ 

‘O\marginnote{6.1} Sakka, first among gods, \\
that we might find the happiness \\
that’s sorrowless, unstressed: \\
Sakka, grant me this boon!’ 

‘If\marginnote{7.1} there exists anywhere a place \\
where one can live happily without working, \\
that surely would be extinguishment’s path! \\
\textsanskrit{Susīma}, go to that place \\
and take me with you!’ 

%
\end{verse}

So,\marginnote{8.1} mendicants, even Sakka, lord of gods—while living off of the fruit of his good and bad deeds, and ruling as sovereign lord over these gods of the Thirty-Three—will speak in praise of initiative and energy. But since you have gone forth in such a well explained teaching and training, it would be truly beautiful for you to try hard, strive, and make an effort to attain the unattained, achieve the unachieved, and realize the unrealized!” 

%
\section*{{\suttatitleacronym SN 11.3}{\suttatitletranslation The Banner’s Crest }{\suttatitleroot Dhajaggasutta}}
\addcontentsline{toc}{section}{\tocacronym{SN 11.3} \toctranslation{The Banner’s Crest } \tocroot{Dhajaggasutta}}
\markboth{The Banner’s Crest }{Dhajaggasutta}
\extramarks{SN 11.3}{SN 11.3}

At\marginnote{1.1} \textsanskrit{Sāvatthī}. 

There\marginnote{1.2} the Buddha addressed the mendicants, “Mendicants!” 

“Venerable\marginnote{1.4} sir,” they replied. The Buddha said this: 

“Once\marginnote{2.1} upon a time, mendicants, a battle was fought between the gods and the demons. Then Sakka, lord of gods, addressed the gods of the Thirty-Three: 

‘Good\marginnote{3.1} sirs, when the gods are fighting, if you get scared or terrified, just look up at my banner’s crest. Then your fear and terror will go away. 

If\marginnote{4.1} you can’t see my banner’s crest, then look up at the banner’s crest of \textsanskrit{Pajāpati}, king of gods. Then your fear and terror will go away. 

If\marginnote{5.1} you can’t see his banner’s crest, then look up at the banner’s crest of \textsanskrit{Varuṇa}, king of gods. Then your fear and terror will go away. 

If\marginnote{6.1} you can’t see his banner’s crest, then look up at the banner’s crest of \textsanskrit{Īsāna}, king of gods. Then your fear and terror will go away.’ 

However,\marginnote{7.1} when they look up at those banner’s crests their fear and terror might go away or it might not. 

Why\marginnote{8.1} is that? Because Sakka is not free of greed, hate, and delusion. He is fearful, scared, nervous, quick to flee. 

But,\marginnote{9.1} mendicants, I say this: If you’ve gone to a wilderness, or to the root of a tree, or to an empty hut and you get scared or terrified, just recollect me: ‘That Blessed One is perfected, a fully awakened Buddha, accomplished in knowledge and conduct, holy, knower of the world, supreme guide for those who wish to train, teacher of gods and humans, awakened, blessed.’ Then your fear and terror will go away. 

If\marginnote{10.1} you can’t recollect me, then recollect the teaching: ‘The teaching is well explained by the Buddha—visible in this very life, immediately effective, inviting inspection, relevant, so that sensible people can know it for themselves.’ Then your fear and terror will go away. 

If\marginnote{11.1} you can’t recollect the teaching, then recollect the \textsanskrit{Saṅgha}: ‘The \textsanskrit{Saṅgha} of the Buddha’s disciples is practicing the way that’s good, direct, methodical, and proper. It consists of the four pairs, the eight individuals. This is the \textsanskrit{Saṅgha} of the Buddha’s disciples that is worthy of offerings dedicated to the gods, worthy of hospitality, worthy of a religious donation, worthy of greeting with joined palms, and is the supreme field of merit for the world.’ Then your fear and terror will go away. 

Why\marginnote{12.1} is that? Because the Realized One is free of greed, hate, and delusion. He is fearless, brave, bold, and stands his ground.” 

That\marginnote{12.3} is what the Buddha said. Then the Holy One, the Teacher, went on to say: 

\begin{verse}%
“In\marginnote{13.1} the wilderness, at a tree’s root, \\
or an empty hut, O mendicants, \\
recollect the Buddha, \\
and no fear will come to you. 

If\marginnote{14.1} you can’t recollect the Buddha—\\
the eldest in the world, the bull of a man—\\
then recollect the teaching, \\
emancipating, well taught. 

If\marginnote{15.1} you can’t recollect the teaching—\\
emancipating, well taught—\\
then recollect the \textsanskrit{Saṅgha}, \\
the supreme field of merit. 

Thus\marginnote{16.1} recollecting the Buddha, \\
the teaching, and the \textsanskrit{Saṅgha}, mendicants, \\
fear and terror \\
and goosebumps will be no more.” 

%
\end{verse}

%
\section*{{\suttatitleacronym SN 11.4}{\suttatitletranslation With Vepacitti }{\suttatitleroot Vepacittisutta}}
\addcontentsline{toc}{section}{\tocacronym{SN 11.4} \toctranslation{With Vepacitti } \tocroot{Vepacittisutta}}
\markboth{With Vepacitti }{Vepacittisutta}
\extramarks{SN 11.4}{SN 11.4}

At\marginnote{1.1} \textsanskrit{Sāvatthī}. 

“Once\marginnote{1.2} upon a time, mendicants, a battle was fought between the gods and the demons. 

Then\marginnote{1.3} Vepacitti, lord of demons, addressed the demons, ‘My good sirs, if the demons defeat the gods in this battle, bind Sakka, the lord of gods, by his limbs and neck and bring him to my presence in the citadel of the demons.’ 

Meanwhile,\marginnote{1.5} Sakka, lord of gods, addressed the gods of the Thirty-Three, ‘My good sirs, if the gods defeat the demons in this battle, bind Vepacitti by his limbs and neck and bring him to my presence in the Sudhamma hall of the gods.’ 

In\marginnote{1.7} that battle the gods won and the demons lost. So the gods of the Thirty-Three bound Vepacitti by his limbs and neck and brought him to Sakka’s presence in the Sudhamma hall of the gods. 

And\marginnote{1.9} as Sakka was entering and leaving the hall, Vepacitti abused and insulted him with rude, harsh words. So \textsanskrit{Mātali} the charioteer addressed Sakka in verse, 

\begin{verse}%
‘O\marginnote{2.1} \textsanskrit{Maghavā}, O Sakka, \\
is it from fear or from weakness \\
that you put up with such harsh words \\
in the presence of Vepacitti?’ 

‘It’s\marginnote{3.1} not out of fear or weakness \\
that I’m patient with Vepacitti. \\
For how can a sensible person like me \\
get in a fight with a fool?’ 

‘Fools\marginnote{4.1} would vent even more \\
if there’s no-one to put a stop to them. \\
So a wise one should stop \\
a fool with forceful punishment.’ 

‘I\marginnote{5.1} think that this is the only way \\
to put a stop to a fool, \\
when you know that the other is upset, \\
be mindful and stay calm.’ 

‘I\marginnote{6.1} see this fault, \textsanskrit{Vāsava}, \\
in just being patient. \\
When a fool thinks, \\
“He puts up with me out of fear,” \\
the idiot will go after you even harder, \\
like a cow chasing someone who runs away.’ 

‘Let\marginnote{7.1} him think this if he wishes, or not—\\
“He puts up with me out of fear.” \\
Of goals culminating in one’s own good, \\
none better than patience is found. 

When\marginnote{8.1} a strong person \\
puts up with a weakling, \\
they call that the ultimate patience, \\
for a weakling must always be patient. 

The\marginnote{9.1} strength of folly \\
is really just weakness, they say. \\
But no-one can challenge a person \\
who’s strong, guarded by the teaching. 

When\marginnote{10.1} you get angry at an angry person \\
you just make things worse for yourself. \\
When you don’t get angry at an angry person \\
you win a battle hard to win. 

When\marginnote{11.1} you know that the other is angry, \\
you act for the good of both \\
yourself and the other \\
if you’re mindful and stay calm. 

People\marginnote{12.1} unfamiliar with the teaching \\
consider one who heals both \\
oneself and the other \\
to be a fool.’ 

%
\end{verse}

So,\marginnote{13.1} mendicants, even Sakka, lord of gods—while living off of the fruit of his good and bad deeds, and ruling as sovereign lord over these gods of the Thirty-Three—will speak in praise of patience and gentleness. But since you have gone forth in such a well explained teaching and training, it would be truly beautiful for you to be patient and gentle!” 

%
\section*{{\suttatitleacronym SN 11.5}{\suttatitletranslation Victory by Good Speech }{\suttatitleroot Subhāsitajayasutta}}
\addcontentsline{toc}{section}{\tocacronym{SN 11.5} \toctranslation{Victory by Good Speech } \tocroot{Subhāsitajayasutta}}
\markboth{Victory by Good Speech }{Subhāsitajayasutta}
\extramarks{SN 11.5}{SN 11.5}

At\marginnote{1.1} \textsanskrit{Sāvatthī}. 

“Once\marginnote{1.2} upon a time, mendicants, a battle was fought between the gods and the demons. 

Then\marginnote{1.3} Vepacitti, lord of demons, said to Sakka, lord of gods, ‘Lord of gods, let there be victory by fine words!’ 

‘Vepacitti,\marginnote{1.5} let there be victory by fine words!’ 

Then\marginnote{1.6} the gods and the demons appointed a panel of judges, saying, ‘These will understand our good and bad statements.’ 

Then\marginnote{1.8} Vepacitti, lord of demons, said to Sakka, lord of gods, ‘Lord of gods, recite a verse!’ 

When\marginnote{1.10} he said this, Sakka said to him, ‘Vepacitti, you are the elder god here. Recite a verse.’ 

So\marginnote{1.13} Vepacitti recited this verse: 

\begin{verse}%
‘Fools\marginnote{2.1} would vent even more \\
if there’s no-one to put a stop to them. \\
So an intelligent person should stop \\
a fool with forceful punishment.’ 

%
\end{verse}

The\marginnote{3.1} demons applauded Vepacitti’s verse, while the gods kept silent. 

Then\marginnote{3.2} Vepacitti said to Sakka, ‘Lord of gods, recite a verse!’ So Sakka recited this verse: 

\begin{verse}%
‘I\marginnote{4.1} think that this is the only way \\
to put a stop to a fool, \\
when you know that the other is upset, \\
be mindful and stay calm.’ 

%
\end{verse}

The\marginnote{5.1} gods applauded Sakka’s verse, while the demons kept silent. 

Then\marginnote{5.2} Sakka said to Vepacitti, ‘Vepacitti, recite a verse!’ So Vepacitti recited this verse: 

\begin{verse}%
‘I\marginnote{6.1} see this fault, \textsanskrit{Vāsava}, \\
in just being patient. \\
When a fool thinks, \\
“He puts up with me out of fear,” \\
the idiot will go after you even harder, \\
like a cow chasing someone who runs away.’ 

%
\end{verse}

The\marginnote{7.1} demons applauded Vepacitti’s verse, while the gods kept silent. 

Then\marginnote{7.2} Vepacitti said to Sakka, ‘Lord of gods, recite a verse!’ So Sakka recited this verse: 

\begin{verse}%
‘Let\marginnote{8.1} him think this if he wishes, or not—\\
“He puts up with me out of fear.” \\
Of goals culminating in one’s own good, \\
none better than patience is found. 

When\marginnote{9.1} a strong person \\
puts up with a weakling, \\
they call that the ultimate patience, \\
for a weakling must always be patient. 

The\marginnote{10.1} strength of folly \\
is really just weakness, they say. \\
But no-one can challenge a person \\
who’s strong, guarded by the teaching. 

When\marginnote{11.1} you get angry at an angry person \\
you just make things worse for yourself. \\
When you don’t get angry at an angry person \\
you win a battle hard to win. 

When\marginnote{12.1} you know that the other is angry, \\
you act for the good of both \\
yourself and the other \\
if you’re mindful and stay calm. 

People\marginnote{13.1} unfamiliar with the teaching \\
consider one who heals both \\
oneself and the other \\
to be a fool.’ 

%
\end{verse}

The\marginnote{14.1} gods applauded Sakka’s verses, while the demons kept silent. 

Then\marginnote{14.2} the panel of judges consisting of both gods and demons said this, ‘The verses spoken by Vepacitti evoke punishment and violence. That’s how you get arguments, quarrels, and disputes. The verses spoken by Sakka don’t evoke punishment and violence. That’s how you stay free of arguments, quarrels, and disputes. 

Sakka,\marginnote{14.7} lord of gods, wins victory by fine words!’ 

And\marginnote{14.8} that’s how Sakka came to win victory by fine words.” 

%
\section*{{\suttatitleacronym SN 11.6}{\suttatitletranslation Bird Nests }{\suttatitleroot Kulāvakasutta}}
\addcontentsline{toc}{section}{\tocacronym{SN 11.6} \toctranslation{Bird Nests } \tocroot{Kulāvakasutta}}
\markboth{Bird Nests }{Kulāvakasutta}
\extramarks{SN 11.6}{SN 11.6}

At\marginnote{1.1} \textsanskrit{Sāvatthī}. 

“Once\marginnote{1.2} upon a time, mendicants, a battle was fought between the gods and the demons. In that battle the demons won and the gods lost. Defeated, the gods fled north with the demons in pursuit. 

Then\marginnote{1.5} Sakka, lord of gods, addressed his charioteer \textsanskrit{Mātali} in verse: 

\begin{verse}%
‘\textsanskrit{Mātali},\marginnote{2.1} don’t ram the bird nests \\
in the red silk-cotton woods with your chariot pole. \\
I’d rather give up our lives to the demons \\
than deprive these birds of their nests.’ 

%
\end{verse}

‘Yes,\marginnote{3.1} lord,’ replied \textsanskrit{Mātali}. And he turned the chariot back around, with its team of a thousand thoroughbreds. 

Then\marginnote{3.2} the demons thought, ‘Now Sakka’s chariot has turned back. The demons will have to fight the gods a second time!’ Terrified, they retreated right away to the citadel of the demons. 

And\marginnote{3.5} that’s how Sakka came to win victory by principle.” 

%
\section*{{\suttatitleacronym SN 11.7}{\suttatitletranslation Not Betray }{\suttatitleroot Nadubbhiyasutta}}
\addcontentsline{toc}{section}{\tocacronym{SN 11.7} \toctranslation{Not Betray } \tocroot{Nadubbhiyasutta}}
\markboth{Not Betray }{Nadubbhiyasutta}
\extramarks{SN 11.7}{SN 11.7}

At\marginnote{1.1} \textsanskrit{Sāvatthī}. 

“Once\marginnote{1.2} upon a time, mendicants, as Sakka, lord of gods, was in private retreat this thought came to his mind, ‘I should never betray even a sworn enemy.’ 

And\marginnote{1.4} then Vepacitti, lord of demons, knowing what Sakka was thinking, approached him. 

Sakka\marginnote{1.5} saw Vepacitti coming off in the distance, and said to him, ‘Stop, Vepacitti, you’re caught!’ 

‘Dear\marginnote{2.1} sir, don’t give up the idea you just had!’ 

‘Swear,\marginnote{3.1} Vepacitti, that you won’t betray me.’ 

\begin{verse}%
‘Whatever\marginnote{4.1} bad things happen to a liar, \\
or to someone who slanders the noble ones, \\
or to someone who betrays a friend, \\
or to someone who’s ungrateful, \\
the same bad things impact \\
anyone who betrays you, \textsanskrit{Sujā}’s husband.’” 

%
\end{verse}

%
\section*{{\suttatitleacronym SN 11.8}{\suttatitletranslation Verocana, Lord of Demons }{\suttatitleroot Verocanaasurindasutta}}
\addcontentsline{toc}{section}{\tocacronym{SN 11.8} \toctranslation{Verocana, Lord of Demons } \tocroot{Verocanaasurindasutta}}
\markboth{Verocana, Lord of Demons }{Verocanaasurindasutta}
\extramarks{SN 11.8}{SN 11.8}

Near\marginnote{1.1} \textsanskrit{Sāvatthī} in Jeta’s Grove. 

Now\marginnote{1.2} at that time the Buddha had gone into retreat for the day’s meditation. 

Then\marginnote{1.3} Sakka, lord of gods, and Verocana, lord of demons, approached the Buddha and stationed themselves one by each door-post. Then Verocana recited this verse in the Buddha’s presence: 

\begin{verse}%
“A\marginnote{2.1} man should make an effort \\
until his goal is accomplished. \\
When goals are accomplished they shine: \\
this is the word of Verocana!” 

“A\marginnote{3.1} man should make an effort \\
until his goal is accomplished. \\
Of goals that shine when accomplished, \\
none better than patience is found.” 

“All\marginnote{4.1} beings are goal-orientated, \\
as befits them in each case. \\
But connection is the ultimate \\
of pleasures for all living creatures. \\
When goals are accomplished they shine: \\
this is the word of Verocana!” 

“All\marginnote{5.1} beings are goal-orientated, \\
as befits them in each case. \\
But connection is the ultimate \\
of pleasures for all living creatures. \\
Of goals that shine when accomplished, \\
none better than patience is found.” 

%
\end{verse}

%
\section*{{\suttatitleacronym SN 11.9}{\suttatitletranslation Hermits in the Wilderness }{\suttatitleroot Araññāyatanaisisutta}}
\addcontentsline{toc}{section}{\tocacronym{SN 11.9} \toctranslation{Hermits in the Wilderness } \tocroot{Araññāyatanaisisutta}}
\markboth{Hermits in the Wilderness }{Araññāyatanaisisutta}
\extramarks{SN 11.9}{SN 11.9}

At\marginnote{1.1} \textsanskrit{Sāvatthī}. 

“Once\marginnote{1.2} upon a time, mendicants, several hermits who were ethical, of good character, settled in leaf huts in a wilderness region. 

Then\marginnote{1.3} Sakka, lord of gods, and Vepacitti, lord of demons, went to those hermits. Vepacitti put on his boots, strapped on his sword, and, carrying a sunshade, entered the hermitage through the main gate. He walked right past those hermits, keeping them at a distance. 

Sakka\marginnote{1.5} took off his boots, gave his sword to others, and, putting down his sunshade, entered the hermitage through a gate he happened upon. He stood downwind of those hermits, revering them with joined palms. 

Then\marginnote{1.6} those hermits addressed Sakka in verse: 

\begin{verse}%
‘When\marginnote{2.1} hermits have been long ordained, \\
the odor of their bodies goes with the wind. \\
You’d better leave, O thousand-eyed! \\
The odor of the hermits is unclean, king of gods.’ 

‘When\marginnote{3.1} hermits have been long ordained, \\
let the odor of their bodies go with the wind. \\
We yearn for this odor, sirs, \\
like a colorful crown of flowers. \\
The gods don’t see it as repulsive.’” 

%
\end{verse}

%
\section*{{\suttatitleacronym SN 11.10}{\suttatitletranslation Hermits by the Ocean }{\suttatitleroot Samuddakasutta}}
\addcontentsline{toc}{section}{\tocacronym{SN 11.10} \toctranslation{Hermits by the Ocean } \tocroot{Samuddakasutta}}
\markboth{Hermits by the Ocean }{Samuddakasutta}
\extramarks{SN 11.10}{SN 11.10}

At\marginnote{1.1} \textsanskrit{Sāvatthī}. 

“Once\marginnote{1.2} upon a time, mendicants, several hermits who were ethical, of good character, settled in leaf huts on the ocean shore. 

Now\marginnote{1.3} at that time a battle was fought between the gods and the demons. 

Then\marginnote{1.4} the hermits thought, ‘The gods are principled, the demons are unprincipled. We may be at risk from the demons. Why don’t we approach Sambara, lord of demons, and beg him for a pledge of safety.’ 

Then,\marginnote{1.8} as easily as a strong person would extend or contract their arm, those hermits vanished from those leaf huts on the ocean shore and reappeared in Sambara’s presence. Then those hermits addressed Sambara in verse: 

\begin{verse}%
‘The\marginnote{2.1} hermits have come to Sambara \\
to beg for a pledge of safety. \\
For you can give them what you wish, \\
whether danger or safety.’ 

‘There\marginnote{3.1} is no safety for hermits, \\
the hated associates of Sakka! \\
Though you beg me for your safety, \\
I’ll only give you fear!’ 

‘Though\marginnote{4.1} we beg you for our safety, \\
you give us only fear. \\
This is what we get from you: \\
may endless peril come to you! 

Whatever\marginnote{5.1} kind of seed you sow, \\
that is the fruit you reap. \\
A doer of good gets good, \\
a doer of bad gets bad. \\
You have sown your own seed, friend, \\
now you’ll experience the fruit.’ 

%
\end{verse}

Then\marginnote{6.1} those hermits, having cursed Sambara, as easily as a strong person would extend or contract their arm, vanished from Sambara’s presence and reappeared in those leaf huts on the ocean shore. 

But\marginnote{6.2} after being cursed by the hermits, Sambara woke in alarm three times that night.” 

%
\addtocontents{toc}{\let\protect\contentsline\protect\nopagecontentsline}
\chapter*{Chapter Two }
\addcontentsline{toc}{chapter}{\tocchapterline{Chapter Two }}
\addtocontents{toc}{\let\protect\contentsline\protect\oldcontentsline}

%
\section*{{\suttatitleacronym SN 11.11}{\suttatitletranslation Vows }{\suttatitleroot Vatapadasutta}}
\addcontentsline{toc}{section}{\tocacronym{SN 11.11} \toctranslation{Vows } \tocroot{Vatapadasutta}}
\markboth{Vows }{Vatapadasutta}
\extramarks{SN 11.11}{SN 11.11}

At\marginnote{1.1} \textsanskrit{Sāvatthī}. 

“Mendicants,\marginnote{1.2} in a former life, when Sakka was a human being, he undertook seven vows. And it was because of undertaking these that he achieved the status of Sakka. What seven? 

As\marginnote{1.4} long as I live, may I support my parents. As long as I live, may I honor the elders in the family. As long as I live, may I speak gently. As long as I live, may I not speak divisively. As long as I live, may I live at home rid of the stain of stinginess, freely generous, open-handed, loving to let go, committed to charity, loving to give and to share. As long as I live, may I speak the truth. As long as I live, may I be free of anger, or should anger arise, may I quickly get rid of it. 

In\marginnote{1.11} a former life, when Sakka was a human being, he undertook seven vows. And it was because of undertaking these that he achieved the status of Sakka. 

\begin{verse}%
A\marginnote{2.1} person who respects their parents, \\
and honors the elders in the family, \\
whose speech is gentle and courteous, \\
and has given up divisiveness; 

who’s\marginnote{3.1} committed to getting rid of stinginess, \\
is truthful, and has mastered anger: \\
the gods of the Thirty-Three \\
call them truly a good person.” 

%
\end{verse}

%
\section*{{\suttatitleacronym SN 11.12}{\suttatitletranslation Sakka’s Names }{\suttatitleroot Sakkanāmasutta}}
\addcontentsline{toc}{section}{\tocacronym{SN 11.12} \toctranslation{Sakka’s Names } \tocroot{Sakkanāmasutta}}
\markboth{Sakka’s Names }{Sakkanāmasutta}
\extramarks{SN 11.12}{SN 11.12}

Near\marginnote{1.1} \textsanskrit{Sāvatthī} in Jeta’s Grove. There the Buddha said to the mendicants: 

“Mendicants,\marginnote{1.3} in a former life, when Sakka was a human being, he was a brahmanical student named Magha. That’s why he’s called \textsanskrit{Maghavā}. 

In\marginnote{2.1} a former life, when Sakka was a human being, he gave gifts in stronghold after stronghold. That’s why he’s called Purindada, the Stronghold-Giver. 

In\marginnote{3.1} a former life, when Sakka was a human being, he gave gifts carefully. That’s why he’s called Sakka, the Careful. 

In\marginnote{4.1} a former life, when Sakka was a human being, he gave the gift of a guest house. That’s why he’s called \textsanskrit{Vāsava}, the Houser. 

Sakka\marginnote{5.1} thinks of a thousand things in a moment. That’s why he’s called Sahassakkha, the Thousand-Eye. 

Sakka’s\marginnote{6.1} wife is the demon maiden named \textsanskrit{Sujā}. That’s why he’s called Sujampati, \textsanskrit{Sujā}’s Husband. 

Sakka\marginnote{7.1} rules as sovereign lord over the gods of the Thirty-Three. That’s why he’s called lord of gods. 

In\marginnote{8.1} a former life, when Sakka was a human being, he undertook seven vows. And it was because of undertaking these that he achieved the status of Sakka. What seven? 

As\marginnote{8.3} long as I live, may I support my parents. As long as I live, may I honor the elders in the family. As long as I live, may I speak gently. As long as I live, may I not speak divisively. As long as I live, may I live at home rid of the stain of stinginess, freely generous, open-handed, loving to let go, committed to charity, loving to give and to share. As long as I live, may I speak the truth. As long as I live, may I be free of anger, or should anger arise, may I quickly get rid of it. 

In\marginnote{8.10} a former life, when Sakka was a human being, he undertook seven vows. And it was because of undertaking these that he achieved the status of Sakka. 

\begin{verse}%
A\marginnote{9.1} person who respects their parents, \\
and honors the elders in the family, \\
whose speech is gentle and courteous, \\
and has given up divisiveness; 

who’s\marginnote{10.1} committed to getting rid of stinginess, \\
is truthful, and has mastered anger: \\
the gods of the Thirty-Three \\
call them truly a good person.” 

%
\end{verse}

%
\section*{{\suttatitleacronym SN 11.13}{\suttatitletranslation With Mahāli }{\suttatitleroot Mahālisutta}}
\addcontentsline{toc}{section}{\tocacronym{SN 11.13} \toctranslation{With Mahāli } \tocroot{Mahālisutta}}
\markboth{With Mahāli }{Mahālisutta}
\extramarks{SN 11.13}{SN 11.13}

\scevam{So\marginnote{1.1} I have heard. }At one time the Buddha was staying near \textsanskrit{Vesālī}, at the Great Wood, in the hall with the peaked roof. 

Then\marginnote{1.3} \textsanskrit{Mahāli} the Licchavi went up to the Buddha, bowed, sat down to one side, and said to him, “Sir, have you seen Sakka, lord of gods?” 

“I\marginnote{3.1} have, \textsanskrit{Mahāli}.” 

“But\marginnote{4.1} surely, sir, you must have seen someone who looked like Sakka. For Sakka is hard to see.” 

“\textsanskrit{Mahāli},\marginnote{5.1} I understand Sakka. And I understand the things that he undertook and committed to, which enabled him to achieve the status of Sakka. 

In\marginnote{6.1} a former life, when Sakka was a human being, he was a brahmanical student named Magha. That’s why he’s called \textsanskrit{Maghavā}. 

In\marginnote{7.1} a former life, when Sakka was a human being, he gave gifts carefully. That’s why he’s called Sakka, the careful. 

In\marginnote{8.1} a former life, when Sakka was a human being, he gave gifts in stronghold after stronghold. That’s why he’s called Purindada, the stronghold-giver. 

In\marginnote{9.1} a former life, when Sakka was a human being, he gave the gift of a guest house. That’s why he’s called \textsanskrit{Vāsava}, the houser. 

Sakka\marginnote{10.1} thinks of a thousand things in a moment. That’s why he’s called Sahassakkha, Thousand-Eye. 

Sakka’s\marginnote{11.1} wife is the demon maiden named \textsanskrit{Sujā}. That’s why he’s called Sujampati, \textsanskrit{Sujā}’s husband. 

Sakka\marginnote{12.1} rules as sovereign lord over the gods of the Thirty-Three. That’s why he’s called lord of gods. 

In\marginnote{13.1} a former life, when Sakka was a human being, he undertook seven vows. And it was because of undertaking these that he achieved the status of Sakka. What seven? 

As\marginnote{13.3} long as I live, may I support my parents. As long as I live, may I honor the elders in the family. As long as I live, may I speak gently. As long as I live, may I not speak divisively. As long as I live, may I live at home rid of the stain of stinginess, freely generous, open-handed, loving to let go, committed to charity, loving to give and to share. As long as I live, may I speak the truth. As long as I live, may I be free of anger, or should anger arise, may I quickly get rid of it. 

In\marginnote{13.10} a former life, when Sakka was a human being, he undertook seven vows. And it was because of undertaking these that he achieved the status of Sakka. 

\begin{verse}%
A\marginnote{14.1} person who respects their parents, \\
and honors the elders in the family, \\
whose speech is gentle and courteous, \\
and has given up divisiveness; 

who’s\marginnote{15.1} committed to getting rid of stinginess, \\
is truthful, and has mastered anger: \\
the gods of the Thirty-Three \\
call them truly a good person.” 

%
\end{verse}

%
\section*{{\suttatitleacronym SN 11.14}{\suttatitletranslation Poor }{\suttatitleroot Daliddasutta}}
\addcontentsline{toc}{section}{\tocacronym{SN 11.14} \toctranslation{Poor } \tocroot{Daliddasutta}}
\markboth{Poor }{Daliddasutta}
\extramarks{SN 11.14}{SN 11.14}

At\marginnote{1.1} one time the Buddha was staying near \textsanskrit{Rājagaha}, in the Bamboo Grove, the squirrels’ feeding ground. There the Buddha addressed the mendicants, “Mendicants!” 

“Venerable\marginnote{1.4} sir,” they replied. The Buddha said this: 

“Once\marginnote{2.1} upon a time, mendicants, there was a poor person, destitute and pitiful. They took up faith, ethics, learning, generosity, and wisdom in the teaching and training proclaimed by the Realized One. After undertaking these things, when their body broke up, after death, they were reborn in a good place, a heavenly realm, in the company of the gods of the Thirty-Three. There they outshone the other gods in beauty and glory. 

But\marginnote{2.5} the gods of the Thirty-Three complained, grumbled, and objected, ‘It’s incredible, it’s amazing! For when this god was a human being in their past life they were poor, destitute, and pitiful. And when their body broke up, after death, they were reborn in a good place, a heavenly realm, in the company of the gods of the Thirty-Three. Here they outshine the other gods in beauty and glory.’ 

Then\marginnote{3.1} Sakka, lord of gods, addressed the gods of the Thirty-Three, ‘Good sirs, don’t complain about this god. When this god was a human being in their past life they took up faith, ethics, learning, generosity, and wisdom in the teaching and training proclaimed by the Realized One. After undertaking these things, when their body broke up, after death, they’ve been reborn in a good place, a heavenly realm, in the company of the gods of the Thirty-Three. Here they outshine the other gods in beauty and glory.’ 

Then\marginnote{3.6} Sakka, lord of gods, guiding the gods of the Thirty-Three, recited this verse: 

\begin{verse}%
‘Whoever\marginnote{4.1} has faith in the Realized One, \\
unwavering and well grounded; \\
whose ethical conduct is good, \\
praised and loved by the noble ones; 

who\marginnote{5.1} has confidence in the \textsanskrit{Saṅgha}, \\
and correct view: \\
they’re said to be prosperous, \\
their life is not in vain. 

So\marginnote{6.1} let the wise devote themselves \\
to faith, ethical behaviour, \\
confidence, and insight into the teaching, \\
remembering the instructions of the Buddhas.’” 

%
\end{verse}

%
\section*{{\suttatitleacronym SN 11.15}{\suttatitletranslation Delightful }{\suttatitleroot Rāmaṇeyyakasutta}}
\addcontentsline{toc}{section}{\tocacronym{SN 11.15} \toctranslation{Delightful } \tocroot{Rāmaṇeyyakasutta}}
\markboth{Delightful }{Rāmaṇeyyakasutta}
\extramarks{SN 11.15}{SN 11.15}

Near\marginnote{1.1} \textsanskrit{Sāvatthī} in Jeta’s Grove. 

And\marginnote{1.2} then Sakka, lord of gods, went up to the Buddha, bowed, stood to one side, and said to him, “Sir, what is a delightful place?” 

\begin{verse}%
“Shrines\marginnote{2.1} in parks and forests, \\
well-made lotus ponds, \\
are not worth a sixteenth part \\
of a delightful human being. 

Whether\marginnote{3.1} in village or wilderness, \\
in a valley or the uplands, \\
wherever the perfected ones live \\
is a delightful place.” 

%
\end{verse}

%
\section*{{\suttatitleacronym SN 11.16}{\suttatitletranslation Sponsoring Sacrifice }{\suttatitleroot Yajamānasutta}}
\addcontentsline{toc}{section}{\tocacronym{SN 11.16} \toctranslation{Sponsoring Sacrifice } \tocroot{Yajamānasutta}}
\markboth{Sponsoring Sacrifice }{Yajamānasutta}
\extramarks{SN 11.16}{SN 11.16}

At\marginnote{1.1} one time the Buddha was staying near \textsanskrit{Rājagaha}, on the Vulture’s Peak Mountain. 

And\marginnote{1.2} then Sakka, lord of gods, went up to the Buddha, bowed, stood to one side, and addressed him in verse: 

\begin{verse}%
“For\marginnote{2.1} humans, those merit-seeking creatures, \\
who sponsor sacrifices, \\
making worldly merit, \\
where is a gift very fruitful?” 

“Four\marginnote{3.1} practicing the path, \\
and four established in the fruit. \\
This is the upright \textsanskrit{Saṅgha}, \\
with wisdom, ethics, and immersion. 

For\marginnote{4.1} humans, those merit-seeking creatures, \\
who sponsor sacrifices, \\
making worldly merit, \\
what is given to the \textsanskrit{Saṅgha} is very fruitful.” 

%
\end{verse}

%
\section*{{\suttatitleacronym SN 11.17}{\suttatitletranslation Homage to the Buddha }{\suttatitleroot Buddhavandanāsutta}}
\addcontentsline{toc}{section}{\tocacronym{SN 11.17} \toctranslation{Homage to the Buddha } \tocroot{Buddhavandanāsutta}}
\markboth{Homage to the Buddha }{Buddhavandanāsutta}
\extramarks{SN 11.17}{SN 11.17}

Near\marginnote{1.1} \textsanskrit{Sāvatthī} in Jeta’s Grove. 

Now\marginnote{1.2} at that time the Buddha had gone into retreat for the day’s meditation. Then Sakka, lord of gods, and \textsanskrit{Brahmā} Sahampati approached the Buddha and stationed themselves one by each door-post. 

Then\marginnote{1.4} Sakka recited this verse in the Buddha’s presence: 

\begin{verse}%
“Rise,\marginnote{2.1} hero! Victor in battle, with burden put down, \\
wander the world without obligation. \\
Your mind is fully liberated, \\
like the moon on the fifteenth night.” 

%
\end{verse}

“Lord\marginnote{3.1} of gods, that’s not how to pay homage to the Realized Ones. This is how it should be done: 

\begin{verse}%
‘Rise,\marginnote{4.1} hero! Victor in battle, leader of the caravan, \\
wander the world without obligation. \\
Let the Blessed One teach the Dhamma! \\
There will be those who understand!’” 

%
\end{verse}

%
\section*{{\suttatitleacronym SN 11.18}{\suttatitletranslation Who Sakka Worships }{\suttatitleroot Gahaṭṭhavandanāsutta}}
\addcontentsline{toc}{section}{\tocacronym{SN 11.18} \toctranslation{Who Sakka Worships } \tocroot{Gahaṭṭhavandanāsutta}}
\markboth{Who Sakka Worships }{Gahaṭṭhavandanāsutta}
\extramarks{SN 11.18}{SN 11.18}

At\marginnote{1.1} \textsanskrit{Sāvatthī}. 

“Once\marginnote{1.3} upon a time, mendicants, Sakka, lord of gods, addressed his charioteer \textsanskrit{Mātali}, ‘My dear \textsanskrit{Mātali}, harness the chariot with its team of a thousand thoroughbreds. We will go to a park and see the scenery.’ 

‘Yes,\marginnote{1.5} lord,’ replied \textsanskrit{Mātali}. He harnessed the chariot and informed Sakka, ‘Good sir, the chariot with its team of a thousand thoroughbreds has been harnessed. Please go at your convenience.’ 

Then\marginnote{1.8} Sakka descended from the Palace of Victory, raised his joined palms, and revered the different quarters. 

So\marginnote{1.9} \textsanskrit{Mātali} the charioteer addressed Sakka in verse: 

\begin{verse}%
‘Those\marginnote{2.1} proficient in the three Vedas worship you, \\
as do all the aristocrats on earth, \\
the Four Great Kings, \\
and the glorious Thirty. \\
So what’s the name of the spirit \\
that you worship, Sakka?’ 

‘Those\marginnote{3.1} proficient in the three Vedas worship me, \\
as do all the aristocrats on earth, \\
the Four Great Kings, \\
and the glorious Thirty. 

But\marginnote{4.1} I revere those accomplished in ethics, \\
who have long trained in immersion, \\
who have rightly gone forth \\
committed to the spiritual life. 

I\marginnote{5.1} also worship those householders, \\
the ethical lay followers \\
who make merit, \textsanskrit{Mātali}, \\
supporting a partner in a principled manner.’ 

‘Those\marginnote{6.1} who you worship \\
seem to be the best in the world, Sakka. \\
I too will worship \\
those who you worship, Sakka.’ 

After\marginnote{7.1} saying this, \textsanskrit{Maghavā} the chief, \\
king of gods, \textsanskrit{Sujā}’s husband, \\
having worshipped the quarters \\
climbed into his chariot.” 

%
\end{verse}

%
\section*{{\suttatitleacronym SN 11.19}{\suttatitletranslation Who Sakka Worships }{\suttatitleroot Satthāravandanāsutta}}
\addcontentsline{toc}{section}{\tocacronym{SN 11.19} \toctranslation{Who Sakka Worships } \tocroot{Satthāravandanāsutta}}
\markboth{Who Sakka Worships }{Satthāravandanāsutta}
\extramarks{SN 11.19}{SN 11.19}

Near\marginnote{1.1} \textsanskrit{Sāvatthī} in Jeta’s Grove. 

“Once\marginnote{1.2} upon a time, mendicants, Sakka, lord of gods, addressed his charioteer \textsanskrit{Mātali}, ‘My dear \textsanskrit{Mātali}, harness the chariot with its team of a thousand thoroughbreds. We will go to a park and see the scenery.’ 

‘Yes,\marginnote{1.4} lord,’ replied \textsanskrit{Mātali}. He harnessed the chariot and informed Sakka, ‘Good sir, the chariot with its team of a thousand thoroughbreds has been harnessed. Please go at your convenience.’ 

Then\marginnote{1.7} Sakka descended from the Palace of Victory, raised his joined palms, and revered the Buddha. 

So\marginnote{1.8} \textsanskrit{Mātali} the charioteer addressed Sakka in verse: 

\begin{verse}%
‘Gods\marginnote{2.1} and men \\
worship you, \textsanskrit{Vāsava}. \\
So what’s the name of the spirit \\
that you worship, Sakka?’ 

‘It’s\marginnote{3.1} the fully awakened Buddha, \\
the Teacher of peerless name \\
in this world with its gods—\\
that’s who I worship, \textsanskrit{Mātali}. 

Those\marginnote{4.1} in whom greed, hate, and ignorance \\
have faded away; \\
the perfected ones with defilements ended—\\
they're who I worship, \textsanskrit{Mātali}. 

The\marginnote{5.1} trainees who take pleasure in decreasing suffering, \\
diligently pursuing the training \\
for getting rid of greed and hate, \\
and going past ignorance—\\
they’re who I worship, \textsanskrit{Mātali}.’ 

‘Those\marginnote{6.1} who you worship \\
seem to be the best in the world, Sakka. \\
I too will worship \\
those who you worship, Sakka.’ 

After\marginnote{7.1} saying this, \textsanskrit{Maghavā} the chief, \\
king of gods, \textsanskrit{Sujā}’s husband, \\
having worshipped the Buddha, \\
climbed into his chariot.” 

%
\end{verse}

%
\section*{{\suttatitleacronym SN 11.20}{\suttatitletranslation Who Sakka Worships }{\suttatitleroot Saṁghavandanāsutta}}
\addcontentsline{toc}{section}{\tocacronym{SN 11.20} \toctranslation{Who Sakka Worships } \tocroot{Saṁghavandanāsutta}}
\markboth{Who Sakka Worships }{Saṁghavandanāsutta}
\extramarks{SN 11.20}{SN 11.20}

Near\marginnote{1.1} \textsanskrit{Sāvatthī} in Jeta’s Grove. 

“Once\marginnote{1.3} upon a time, mendicants, Sakka, lord of gods, addressed his charioteer \textsanskrit{Mātali}, ‘My dear \textsanskrit{Mātali}, harness the chariot with its team of a thousand thoroughbreds. We will go to a park and see the scenery.’ 

‘Yes,\marginnote{1.5} lord,’ replied \textsanskrit{Mātali}. He harnessed the chariot and informed Sakka, ‘Good sir, the chariot with its team of a thousand thoroughbreds has been harnessed. Please go at your convenience.’ 

Then\marginnote{1.8} Sakka descended from the Palace of Victory, raised his joined palms, and revered the mendicant \textsanskrit{Saṅgha}. 

So\marginnote{1.9} \textsanskrit{Mātali} the charioteer addressed Sakka in verse: 

\begin{verse}%
‘It’s\marginnote{2.1} these who should worship you, \\
namely the humans stuck in their putrid bodies, \\
sunk in a corpse, \\
stricken by hunger and thirst. 

Why\marginnote{3.1} then do you envy those \\
who are homeless, \textsanskrit{Vāsava}? \\
Relate the hermits’ way of life, \\
let us hear what you say.’ 

‘This\marginnote{4.1} is why I envy the \\
homeless, \textsanskrit{Mātali}. \\
When they leave a village, \\
they proceed without concern. 

They\marginnote{5.1} hoard no goods in storerooms, \\
nor in pots or baskets. \\
They seek food prepared by others, \\
and, true to their vows, live on that. 

The\marginnote{6.1} wise whose words are full of wisdom, \\
live peacefully and quietly. \\
Gods fight with demons, \\
and mortals fight each other, \textsanskrit{Mātali}. 

Not\marginnote{7.1} fighting among those who fight, \\
extinguished among those who are armed, \\
not grasping among those who grasp: \\
they’re who I worship, \textsanskrit{Mātali}.’ 

‘Those\marginnote{8.1} who you worship \\
seem to be the best in the world, Sakka. \\
I too will worship \\
those who you worship, \textsanskrit{Vāsava}.’ 

After\marginnote{9.1} saying this, \textsanskrit{Maghavā} the chief, \\
king of gods, \textsanskrit{Sujā}’s husband, \\
having worshipped the mendicant \textsanskrit{Saṅgha}, \\
climbed into his chariot.” 

%
\end{verse}

%
\addtocontents{toc}{\let\protect\contentsline\protect\nopagecontentsline}
\chapter*{Chapter Three }
\addcontentsline{toc}{chapter}{\tocchapterline{Chapter Three }}
\addtocontents{toc}{\let\protect\contentsline\protect\oldcontentsline}

%
\section*{{\suttatitleacronym SN 11.21}{\suttatitletranslation Incinerated }{\suttatitleroot Chetvāsutta}}
\addcontentsline{toc}{section}{\tocacronym{SN 11.21} \toctranslation{Incinerated } \tocroot{Chetvāsutta}}
\markboth{Incinerated }{Chetvāsutta}
\extramarks{SN 11.21}{SN 11.21}

Near\marginnote{1.1} \textsanskrit{Sāvatthī} in Jeta’s Grove. 

And\marginnote{1.2} then Sakka, lord of gods, went up to the Buddha, bowed, stood to one side, and addressed him in verse: 

\begin{verse}%
“When\marginnote{2.1} what is incinerated do you sleep at ease? \\
When what is incinerated is there no sorrow? \\
What is the one thing \\
whose killing you approve?” 

“When\marginnote{3.1} anger’s incinerated you sleep at ease. \\
When anger’s incinerated there is no sorrow. \\
O \textsanskrit{Vāsava}, anger has a poisonous root \\
and a honey tip. \\
The noble ones praise its killing, \\
for when it’s incinerated there is no sorrow.” 

%
\end{verse}

%
\section*{{\suttatitleacronym SN 11.22}{\suttatitletranslation Ugly }{\suttatitleroot Dubbaṇṇiyasutta}}
\addcontentsline{toc}{section}{\tocacronym{SN 11.22} \toctranslation{Ugly } \tocroot{Dubbaṇṇiyasutta}}
\markboth{Ugly }{Dubbaṇṇiyasutta}
\extramarks{SN 11.22}{SN 11.22}

Near\marginnote{1.1} \textsanskrit{Sāvatthī} in Jeta’s Grove. 

“Once\marginnote{1.3} upon a time, mendicants, there was a native spirit who was ugly and deformed. He sat on the throne of Sakka, lord of gods. 

But\marginnote{1.4} the gods of the Thirty-Three complained, grumbled, and objected, ‘It’s incredible, it’s amazing! This ugly and deformed spirit is sitting on the throne of Sakka, the lord of gods.’ But the more the gods complained, the more attractive, good-looking, and lovely that spirit became. 

So\marginnote{2.1} the gods went up to Sakka and told him what had happened, adding, ‘Surely, good sir, that must be the anger-eating spirit!’ 

Then\marginnote{3.1} Sakka went up to that spirit, arranged his robe over one shoulder, knelt with his right knee on the ground, raised his joined palms toward the anger-eating spirit, and pronounced his name three times: ‘Good sir, I am Sakka, lord of gods! Good sir, I am Sakka, the lord of gods!’ But the more Sakka pronounced his name, the uglier and more deformed the spirit became, until eventually it vanished right there. 

Then\marginnote{3.5} Sakka, lord of gods, guiding the gods of the Thirty-Three, recited this verse: 

\begin{verse}%
‘My\marginnote{4.1} mind isn’t easily upset; \\
I’m not easily drawn into the maelstrom. \\
I don’t get angry for long, \\
anger doesn’t last in me. 

When\marginnote{5.1} I do get angry I don’t speak harshly, \\
nor do I advertise my own virtues. \\
I carefully restrain myself \\
out of regard for my own welfare.’” 

%
\end{verse}

%
\section*{{\suttatitleacronym SN 11.23}{\suttatitletranslation The Sambari Sorcery }{\suttatitleroot Sambarimāyāsutta}}
\addcontentsline{toc}{section}{\tocacronym{SN 11.23} \toctranslation{The Sambari Sorcery } \tocroot{Sambarimāyāsutta}}
\markboth{The Sambari Sorcery }{Sambarimāyāsutta}
\extramarks{SN 11.23}{SN 11.23}

At\marginnote{1.1} \textsanskrit{Sāvatthī}. 

The\marginnote{1.2} Buddha said this: 

“Once\marginnote{1.3} upon a time, mendicants, Vepacitti, lord of demons, was sick, suffering, gravely ill. So Sakka went to see him to ask after his illness. 

Vepacitti\marginnote{1.5} saw Sakka coming off in the distance, and said to him, ‘Heal me, lord of gods!’ 

‘Teach\marginnote{1.8} me, Vepacitti, the Sambari sorcery.’ 

‘I\marginnote{1.9} can’t do that, good sir, until I have consulted with the demons.’ 

Then\marginnote{1.10} Vepacitti, lord of demons, asked the demons, ‘Good sirs, may I teach the Sambari sorcery to Sakka, lord of gods?’ 

‘Do\marginnote{1.12} not, good sir, teach the Sambari sorcery to Sakka!’ 

So\marginnote{1.13} Vepacitti addressed Sakka in verse: 

\begin{verse}%
‘O\marginnote{2.1} \textsanskrit{Maghavā}, O Sakka, \\
king of gods, \textsanskrit{Sujā}’s husband, \\
a sorceror falls into the terrible hell—\\
like Sambara, for a hundred years.’” 

%
\end{verse}

%
\section*{{\suttatitleacronym SN 11.24}{\suttatitletranslation Transgression }{\suttatitleroot Accayasutta}}
\addcontentsline{toc}{section}{\tocacronym{SN 11.24} \toctranslation{Transgression } \tocroot{Accayasutta}}
\markboth{Transgression }{Accayasutta}
\extramarks{SN 11.24}{SN 11.24}

At\marginnote{1.1} \textsanskrit{Sāvatthī}. 

Now\marginnote{1.2} at that time two mendicants were overly attached, and one of them transgressed against the other. The transgressor confessed to the other mendicant, but they didn’t accept it. Then several mendicants went up to the Buddha, bowed, sat down to one side, and told him what had happened. 

“Mendicants,\marginnote{2.1} there are two fools. One who doesn’t recognize when they’ve made a mistake. And one who doesn’t properly accept the confession of someone who’s made a mistake. These are the two fools. 

There\marginnote{2.4} are two who are astute. One who recognizes when they’ve made a mistake. And one who properly accepts the confession of someone who’s made a mistake. These are the two who are astute. 

Once\marginnote{3.1} upon a time, mendicants, Sakka, lord of gods, guiding the gods of the Thirty-Three, recited this verse: 

\begin{verse}%
‘Control\marginnote{4.1} your anger; \\
don’t let friendships decay. \\
Don’t blame the blameless, \\
and don’t say divisive things. \\
For anger crushes bad people \\
like a mountain.’” 

%
\end{verse}

%
\section*{{\suttatitleacronym SN 11.25}{\suttatitletranslation Don’t Be Angry }{\suttatitleroot Akkodhasutta}}
\addcontentsline{toc}{section}{\tocacronym{SN 11.25} \toctranslation{Don’t Be Angry } \tocroot{Akkodhasutta}}
\markboth{Don’t Be Angry }{Akkodhasutta}
\extramarks{SN 11.25}{SN 11.25}

\scevam{So\marginnote{1.1} I have heard. }At one time the Buddha was staying near \textsanskrit{Sāvatthī} in Jeta’s Grove, \textsanskrit{Anāthapiṇḍika}’s monastery. 

There\marginnote{1.3} the Buddha addressed the mendicants: 

“Once\marginnote{1.4} upon a time, mendicants, Sakka, lord of gods, guiding the gods of the Thirty-Three, recited this verse: 

\begin{verse}%
‘Don’t\marginnote{2.1} let anger be your master, \\
don’t get angry at angry people. \\
Kindness and harmlessness \\
are always present in the noble ones. \\
For anger crushes bad people \\
like a mountain.’” 

%
\end{verse}

\scendsutta{The Linked Discourses with Sakka are complete. }

\scendbook{The Book With Verses is finished. }

%
\backmatter%
\chapter*{Colophon}
\addcontentsline{toc}{chapter}{Colophon}
\markboth{Colophon}{Colophon}

\section*{The Translator}

Bhikkhu Sujato was born as Anthony Aidan Best on 4/11/1966 in Perth, Western Australia. He grew up in the pleasant suburbs of Mt Lawley and Attadale alongside his sister Nicola, who was the good child. His mother, Margaret Lorraine Huntsman née Pinder, said “he’ll either be a priest or a poet”, while his father, Anthony Thomas Best, advised him to “never do anything for money”. He attended Aquinas College, a Catholic school, where he decided to become an atheist. At the University of WA he studied philosophy, aiming to learn what he wanted to do with his life. Finding that what he wanted to do was play guitar, he dropped out. His main band was named Martha’s Vineyard, which achieved modest success in the indie circuit. 

A seemingly random encounter with a roadside joey took him to Thailand, where he entered his first meditation retreat at Wat Ram Poeng, Chieng Mai in 1992. Feeling the call to the Buddha’s path, he took full ordination in Wat Pa Nanachat in 1994, where his teachers were Ajahn Pasanno and Ajahn Jayasaro. In 1997 he returned to Perth to study with Ajahn Brahm at Bodhinyana Monastery. 

He spent several years practicing in seclusion in Malaysia and Thailand before establishing Santi Forest Monastery in Bundanoon, NSW, in 2003. There he was instrumental in supporting the establishment of the Theravada bhikkhuni order in Australia and advocating for women’s rights. He continues to teach in Australia and globally, with a special concern for the moral implications of climate change and other forms of environmental destruction. He has published a series of books of original and groundbreaking research on early Buddhism. 

In 2005 he founded SuttaCentral together with Rod Bucknell and John Kelly. In 2015, seeing the need for a complete, accurate, plain English translation of the Pali texts, he undertook the task, spending nearly three years in isolation on the isle of Qi Mei off the coast of the nation of Taiwan. He completed the four main \textsanskrit{Nikāyas} in 2018, and the early books of the Khuddaka \textsanskrit{Nikāya} were complete by 2021. All this work is dedicated to the public domain and is entirely free of copyright encumbrance. 

In 2019 he returned to Sydney where he established Lokanta Vihara (The Monastery at the End of the World). 

\section*{Creation Process}

Primary source was the digital \textsanskrit{Mahāsaṅgīti} edition of the Pali \textsanskrit{Tipiṭaka}. Translated from the Pali, with reference to several English translations, especially those of Bhikkhu Bodhi.

\section*{The Translation}

This translation was part of a project to translate the four Pali \textsanskrit{Nikāyas} with the following aims: plain, approachable English; consistent terminology; accurate rendition of the Pali; free of copyright. It was made during 2016–2018 while Bhikkhu Sujato was staying in Qimei, Taiwan.

\section*{About SuttaCentral}

SuttaCentral publishes early Buddhist texts. Since 2005 we have provided root texts in Pali, Chinese, Sanskrit, Tibetan, and other languages, parallels between these texts, and translations in many modern languages. We build on the work of generations of scholars, and offer our contribution freely.

SuttaCentral is driven by volunteer contributions, and in addition we employ professional developers. We offer a sponsorship program for high quality translations from the original languages. Financial support for SuttaCentral is handled by the SuttaCentral Development Trust, a charitable trust registered in Australia.

\section*{About Bilara}

“Bilara” means “cat” in Pali, and it is the name of our Computer Assisted Translation (CAT) software. Bilara is a web app that enables translators to translate early Buddhist texts into their own language. These translations are published on SuttaCentral with the root text and translation side by side.

\section*{About SuttaCentral Editions}

The SuttaCentral Editions project makes high quality books from selected Bilara translations. These are published in formats including HTML, EPUB, PDF, and print.

If you want to print any of our Editions, please let us know and we will help prepare a file to your specifications.

%
\end{document}