\documentclass[12pt,openany]{book}%
\usepackage{lastpage}%
%
\usepackage[inner=1in, outer=1in, top=.7in, bottom=1in, papersize={6in,9in}, headheight=13pt]{geometry}
\usepackage{polyglossia}
\usepackage[12pt]{moresize}
\usepackage{soul}%
\usepackage{microtype}
\usepackage{tocbasic}
\usepackage{realscripts}
\usepackage{epigraph}%
\usepackage{setspace}%
\usepackage{sectsty}
\usepackage{fontspec}
\usepackage{marginnote}
\usepackage[bottom]{footmisc}
\usepackage{enumitem}
\usepackage{fancyhdr}
\usepackage{extramarks}
\usepackage{graphicx}
\usepackage{verse}
\usepackage{relsize}
\usepackage{etoolbox}
\usepackage[a-3u]{pdfx}

\hypersetup{
colorlinks=true,
urlcolor=black,
linkcolor=black,
citecolor=black
}

% use a small amount of tracking on small caps
\SetTracking[ spacing = {25*,166, } ]{ encoding = *, shape = sc }{ 25 }

% add a blank page
\newcommand{\blankpage}{
\newpage
\thispagestyle{empty}
\mbox{}
\newpage
}

% define languages
\setdefaultlanguage[]{english}
\setotherlanguage[script=Latin]{sanskrit}

%\usepackage{pagegrid}
%\pagegridsetup{top-left, step=.25in}

% define fonts
% use if arno sanskrit is unavailable
%\setmainfont{Gentium Plus}
%\newfontfamily\Semiboldsubheadfont[]{Gentium Plus}
%\newfontfamily\Semiboldnormalfont[]{Gentium Plus}
%\newfontfamily\Lightfont[]{Gentium Plus}
%\newfontfamily\Marginalfont[]{Gentium Plus}
%\newfontfamily\Allsmallcapsfont[RawFeature=+c2sc]{Gentium Plus}
%\newfontfamily\Noligaturefont[Renderer=Basic]{Gentium Plus}
%\newfontfamily\Noligaturecaptionfont[Renderer=Basic]{Gentium Plus}
%\newfontfamily\Fleuronfont[Ornament=1]{Gentium Plus}

% use if arno sanskrit is available. display is applied to \chapter and \part, subhead to \section and \subsection. When specifying semibold, the italic must be defined.
\setmainfont[Numbers=OldStyle]{Arno Pro}
\newfontfamily\Semibolddisplayfont[BoldItalicFont = Arno Pro Semibold Italic Display]{Arno Pro Semibold Display} %
\newfontfamily\Semiboldsubheadfont[BoldItalicFont = Arno Pro Semibold Italic Subhead]{Arno Pro Semibold Subhead}
\newfontfamily\Semiboldnormalfont[BoldItalicFont = Arno Pro Semibold Italic]{Arno Pro Semibold}
\newfontfamily\Marginalfont[RawFeature=+subs]{Arno Pro Regular}
\newfontfamily\Allsmallcapsfont[RawFeature=+c2sc]{Arno Pro}
\newfontfamily\Noligaturefont[Renderer=Basic]{Arno Pro}
\newfontfamily\Noligaturecaptionfont[Renderer=Basic]{Arno Pro Caption}

% chinese fonts
\newfontfamily\cjk{Noto Serif TC}
\newcommand*{\langlzh}[1]{\cjk{#1}\normalfont}%

% logo
\newfontfamily\Logofont{sclogo.ttf}
\newcommand*{\sclogo}[1]{\large\Logofont{#1}}

% use subscript numerals for margin notes
\renewcommand*{\marginfont}{\Marginalfont}

% ensure margin notes have consistent vertical alignment
\renewcommand*{\marginnotevadjust}{-.17em}

% use compact lists
\setitemize{noitemsep,leftmargin=1em}
\setenumerate{noitemsep,leftmargin=1em}
\setdescription{noitemsep, style=unboxed, leftmargin=0em}

% style ToC
\DeclareTOCStyleEntries[
  raggedentrytext,
  linefill=\hfill,
  pagenumberwidth=.5in,
  pagenumberformat=\normalfont,
  entryformat=\normalfont
]{tocline}{chapter,section}


  \setlength\topsep{0pt}%
  \setlength\parskip{0pt}%

% define new \centerpars command for use in ToC. This ensures centering, proper wrapping, and no page break after
\def\startcenter{%
  \par
  \begingroup
  \leftskip=0pt plus 1fil
  \rightskip=\leftskip
  \parindent=0pt
  \parfillskip=0pt
}
\def\stopcenter{%
  \par
  \endgroup
}
\long\def\centerpars#1{\startcenter#1\stopcenter}

% redefine part, so that it adds a toc entry without page number
\let\oldcontentsline\contentsline
\newcommand{\nopagecontentsline}[3]{\oldcontentsline{#1}{#2}{}}

    \makeatletter
\renewcommand*\l@part[2]{%
  \ifnum \c@tocdepth >-2\relax
    \addpenalty{-\@highpenalty}%
    \addvspace{0em \@plus\p@}%
    \setlength\@tempdima{3em}%
    \begingroup
      \parindent \z@ \rightskip \@pnumwidth
      \parfillskip -\@pnumwidth
      {\leavevmode
       \setstretch{.85}\large\scshape\centerpars{#1}\vspace*{-1em}\llap{#2}}\par
       \nobreak
         \global\@nobreaktrue
         \everypar{\global\@nobreakfalse\everypar{}}%
    \endgroup
  \fi}
\makeatother

\makeatletter
\def\@pnumwidth{2em}
\makeatother

% define new sectioning command, which is only used in volumes where the pannasa is found in some parts but not others, especially in an and sn

\newcommand*{\pannasa}[1]{\clearpage\thispagestyle{empty}\begin{center}\vspace*{14em}\setstretch{.85}\huge\itshape\scshape\MakeLowercase{#1}\end{center}}

    \makeatletter
\newcommand*\l@pannasa[2]{%
  \ifnum \c@tocdepth >-2\relax
    \addpenalty{-\@highpenalty}%
    \addvspace{.5em \@plus\p@}%
    \setlength\@tempdima{3em}%
    \begingroup
      \parindent \z@ \rightskip \@pnumwidth
      \parfillskip -\@pnumwidth
      {\leavevmode
       \setstretch{.85}\large\itshape\scshape\lowercase{\centerpars{#1}}\vspace*{-1em}\llap{#2}}\par
       \nobreak
         \global\@nobreaktrue
         \everypar{\global\@nobreakfalse\everypar{}}%
    \endgroup
  \fi}
\makeatother

% don't put page number on first page of toc (relies on etoolbox)
\patchcmd{\chapter}{plain}{empty}{}{}

% global line height
\setstretch{1.05}

% allow linebreak after em-dash
\catcode`\—=13
\protected\def—{\unskip\textemdash\allowbreak}

% style headings with secsty. chapter and section are defined per-edition
\partfont{\setstretch{.85}\normalfont\centering\textsc}
\subsectionfont{\setstretch{.85}\Semiboldsubheadfont}%
\subsubsectionfont{\setstretch{.85}\Semiboldnormalfont}

% style elements of suttatitle
\newcommand*{\suttatitleacronym}[1]{\smaller[2]{#1}\vspace*{.3em}}
\newcommand*{\suttatitletranslation}[1]{\linebreak{#1}}
\newcommand*{\suttatitleroot}[1]{\linebreak\smaller[2]\itshape{#1}}

\DeclareTOCStyleEntries[
  indent=3.3em,
  dynindent,
  beforeskip=.2em plus -2pt minus -1pt,
]{tocline}{section}

\DeclareTOCStyleEntries[
  indent=0em,
  dynindent,
  beforeskip=.4em plus -2pt minus -1pt,
]{tocline}{chapter}

\newcommand*{\tocacronym}[1]{\hspace*{-3.3em}{#1}\quad}
\newcommand*{\toctranslation}[1]{#1}
\newcommand*{\tocroot}[1]{(\textit{#1})}
\newcommand*{\tocchapterline}[1]{\bfseries\itshape{#1}}


% redefine paragraph and subparagraph headings to not be inline
\makeatletter
% Change the style of paragraph headings %
\renewcommand\paragraph{\@startsection{paragraph}{4}{\z@}%
            {-2.5ex\@plus -1ex \@minus -.25ex}%
            {1.25ex \@plus .25ex}%
            {\noindent\Semiboldnormalfont\normalsize}}

% Change the style of subparagraph headings %
\renewcommand\subparagraph{\@startsection{subparagraph}{5}{\z@}%
            {-2.5ex\@plus -1ex \@minus -.25ex}%
            {1.25ex \@plus .25ex}%
            {\noindent\Semiboldnormalfont\small}}
\makeatother

% use etoolbox to suppress page numbers on \part
\patchcmd{\part}{\thispagestyle{plain}}{\thispagestyle{empty}}
  {}{\errmessage{Cannot patch \string\part}}

% and to reduce margins on quotation
\patchcmd{\quotation}{\rightmargin}{\leftmargin 1.2em \rightmargin}{}{}
\AtBeginEnvironment{quotation}{\small}

% titlepage
\newcommand*{\titlepageTranslationTitle}[1]{{\begin{center}\begin{large}{#1}\end{large}\end{center}}}
\newcommand*{\titlepageCreatorName}[1]{{\begin{center}\begin{normalsize}{#1}\end{normalsize}\end{center}}}

% halftitlepage
\newcommand*{\halftitlepageTranslationTitle}[1]{\setstretch{2.5}{\begin{Huge}\uppercase{\so{#1}}\end{Huge}}}
\newcommand*{\halftitlepageTranslationSubtitle}[1]{\setstretch{1.2}{\begin{large}{#1}\end{large}}}
\newcommand*{\halftitlepageFleuron}[1]{{\begin{large}\Fleuronfont{{#1}}\end{large}}}
\newcommand*{\halftitlepageByline}[1]{{\begin{normalsize}\textit{{#1}}\end{normalsize}}}
\newcommand*{\halftitlepageCreatorName}[1]{{\begin{LARGE}{\textsc{#1}}\end{LARGE}}}
\newcommand*{\halftitlepageVolumeNumber}[1]{{\begin{normalsize}{\Allsmallcapsfont{\textsc{#1}}}\end{normalsize}}}
\newcommand*{\halftitlepageVolumeAcronym}[1]{{\begin{normalsize}{#1}\end{normalsize}}}
\newcommand*{\halftitlepageVolumeTranslationTitle}[1]{{\begin{Large}{\textsc{#1}}\end{Large}}}
\newcommand*{\halftitlepageVolumeRootTitle}[1]{{\begin{normalsize}{\Allsmallcapsfont{\textsc{\itshape #1}}}\end{normalsize}}}
\newcommand*{\halftitlepagePublisher}[1]{{\begin{large}{\Noligaturecaptionfont\textsc{#1}}\end{large}}}

% epigraph
\renewcommand{\epigraphflush}{center}
\renewcommand*{\epigraphwidth}{.85\textwidth}
\newcommand*{\epigraphTranslatedTitle}[1]{\vspace*{.5em}\footnotesize\textsc{#1}\\}%
\newcommand*{\epigraphRootTitle}[1]{\footnotesize\textit{#1}\\}%
\newcommand*{\epigraphReference}[1]{\footnotesize{#1}}%

% custom commands for html styling classes
\newcommand*{\scnamo}[1]{\begin{center}\textit{#1}\end{center}}
\newcommand*{\scendsection}[1]{\begin{center}\textit{#1}\end{center}}
\newcommand*{\scendsutta}[1]{\begin{center}\textit{#1}\end{center}}
\newcommand*{\scendbook}[1]{\begin{center}\uppercase{#1}\end{center}}
\newcommand*{\scendkanda}[1]{\begin{center}\textbf{#1}\end{center}}
\newcommand*{\scend}[1]{\begin{center}\textit{#1}\end{center}}
\newcommand*{\scuddanaintro}[1]{\textit{#1}}
\newcommand*{\scendvagga}[1]{\begin{center}\textbf{#1}\end{center}}
\newcommand*{\scrule}[1]{\textbf{#1}}
\newcommand*{\scadd}[1]{\textit{#1}}
\newcommand*{\scevam}[1]{\textsc{#1}}
\newcommand*{\scspeaker}[1]{\hspace{2em}\textit{#1}}
\newcommand*{\scbyline}[1]{\begin{flushright}\textit{#1}\end{flushright}\bigskip}

% custom command for thematic break = hr
\newcommand*{\thematicbreak}{\begin{center}\rule[.5ex]{6em}{.4pt}\begin{normalsize}\quad\Fleuronfont{•}\quad\end{normalsize}\rule[.5ex]{6em}{.4pt}\end{center}}

% manage and style page header and footer. "fancy" has header and footer, "plain" has footer only

\pagestyle{fancy}
\fancyhf{}
\fancyfoot[RE,LO]{\thepage}
\fancyfoot[LE,RO]{\footnotesize\lastleftxmark}
\fancyhead[CE]{\setstretch{.85}\Noligaturefont\MakeLowercase{\textsc{\firstrightmark}}}
\fancyhead[CO]{\setstretch{.85}\Noligaturefont\MakeLowercase{\textsc{\firstleftmark}}}
\renewcommand{\headrulewidth}{0pt}
\fancypagestyle{plain}{ %
\fancyhf{} % remove everything
\fancyfoot[RE,LO]{\thepage}
\fancyfoot[LE,RO]{\footnotesize\lastleftxmark}
\renewcommand{\headrulewidth}{0pt}
\renewcommand{\footrulewidth}{0pt}}

% style footnotes
\setlength{\skip\footins}{1em}

\makeatletter
\newcommand{\@makefntextcustom}[1]{%
    \parindent 0em%
    \thefootnote.\enskip #1%
}
\renewcommand{\@makefntext}[1]{\@makefntextcustom{#1}}
\makeatother

% hang quotes (requires microtype)
\microtypesetup{
  protrusion = true,
  expansion  = true,
  tracking   = true,
  factor     = 1000,
  patch      = all,
  final
}

% Custom protrusion rules to allow hanging punctuation
\SetProtrusion
{ encoding = *}
{
% char   right left
  {-} = {    , 500 },
  % Double Quotes
  \textquotedblleft
      = {1000,     },
  \textquotedblright
      = {    , 1000},
  \quotedblbase
      = {1000,     },
  % Single Quotes
  \textquoteleft
      = {1000,     },
  \textquoteright
      = {    , 1000},
  \quotesinglbase
      = {1000,     }
}

% make latex use actual font em for parindent, not Computer Modern Roman
\AtBeginDocument{\setlength{\parindent}{1em}}%
%

% Default values; a bit sloppier than normal
\tolerance 1414
\hbadness 1414
\emergencystretch 1.5em
\hfuzz 0.3pt
\clubpenalty = 10000
\widowpenalty = 10000
\displaywidowpenalty = 10000
\hfuzz \vfuzz
 \raggedbottom%

\title{Middle Discourses}
\author{Bhikkhu Sujato}
\date{}%
% define a different fleuron for each edition
\newfontfamily\Fleuronfont[Ornament=4]{Arno Pro}

% Define heading styles per edition for chapter, section, and subsection. Suttatitle can be any one of these, depending on the volume. 

\let\oldfrontmatter\frontmatter
\renewcommand{\frontmatter}{%
\chapterfont{\setstretch{.85}\normalfont\centering}%
\sectionfont{\setstretch{.85}\Semiboldsubheadfont}%
\oldfrontmatter}

\let\oldmainmatter\mainmatter
\renewcommand{\mainmatter}{%
\chapterfont{\thispagestyle{empty}\vspace*{4em}\setstretch{.85}\LARGE\normalfont\itshape\scshape\centering\MakeLowercase}
\sectionfont{\clearpage\thispagestyle{plain}\vspace*{2em}\setstretch{.85}\normalfont\centering}%
\oldmainmatter}

\let\oldbackmatter\backmatter
\renewcommand{\backmatter}{%
\chapterfont{\setstretch{.85}\normalfont\centering}%
\sectionfont{\setstretch{.85}\Semiboldsubheadfont}%
\oldbackmatter}
%
%
\begin{document}%
\normalsize%
\frontmatter%
\setlength{\parindent}{0cm}

\pagestyle{empty}

\maketitle

\blankpage%
\begin{center}

\vspace*{2.2em}

\halftitlepageTranslationTitle{Middle Discourses}

\vspace*{1em}

\halftitlepageTranslationSubtitle{A lucid translation of the Majjhima Nikāya}

\vspace*{2em}

\halftitlepageFleuron{•}

\vspace*{2em}

\halftitlepageByline{translated and introduced by}

\vspace*{.5em}

\halftitlepageCreatorName{Bhikkhu Sujato}

\vspace*{4em}

\halftitlepageVolumeNumber{Volume 2}

\smallskip

\halftitlepageVolumeAcronym{MN 51–100}

\smallskip

\halftitlepageVolumeTranslationTitle{The Middle Fifty}

\smallskip

\halftitlepageVolumeRootTitle{Majjhimapaṇṇāsa}

\vspace*{\fill}

\sclogo{0}
 \halftitlepagePublisher{SuttaCentral}

\end{center}

\newpage
%
\setstretch{1.05}

\begin{footnotesize}

\textit{Middle Discourses} is a translation of the Majjhimanikāya by Bhikkhu Sujato.

\medskip

Creative Commons Zero (CC0)

To the extent possible under law, Bhikkhu Sujato has waived all copyright and related or neighboring rights to \textit{Middle Discourses}.

\medskip

This work is published from Australia.

\begin{center}
\textit{This translation is an expression of an ancient spiritual text that has been passed down by the Buddhist tradition for the benefit of all sentient beings. It is dedicated to the public domain via Creative Commons Zero (CC0). You are encouraged to copy, reproduce, adapt, alter, or otherwise make use of this translation. The translator respectfully requests that any use be in accordance with the values and principles of the Buddhist community.}
\end{center}

\medskip

\begin{description}
    \item[Web publication date] 2018
    \item[This edition] 2022-11-08 08:10:33
    \item[Publication type] paperback
    \item[Edition] ed5
    \item[Number of volumes] 3
    \item[Publication ISBN] 978-1-76132-065-1
    \item[Publication URL] https://suttacentral.net/editions/mn/en/sujato
    \item[Source URL] https://github.com/suttacentral/bilara-data/tree/published/translation/en/sujato/sutta/mn
    \item[Publication number] scpub3
\end{description}

\medskip

Published by SuttaCentral

\medskip

\textit{SuttaCentral,\\
c/o Alwis \& Alwis Pty Ltd\\
Kaurna Country,\\
Suite 12,\\
198 Greenhill Road,\\
Eastwood,\\
SA 5063,\\
Australia}

\end{footnotesize}

\newpage

\setlength{\parindent}{1.5em}%%
\tableofcontents
\newpage
\pagestyle{fancy}
%
\chapter*{Summary of Contents}
\addcontentsline{toc}{chapter}{Summary of Contents}
\markboth{Summary of Contents}{Summary of Contents}

\begin{description}%
\item[The Chapter on Householders(\textit{\textsanskrit{Gahapativagga}})] This chapter is addressed to a diverse range of lay people.%
\item[MN 51: With Kandaraka (\textit{\textsanskrit{Kandarakasutta}})] The Buddha discusses mindfulness meditation with lay practitioners. Contrasting the openness of animals with the duplicity of humans, he explains how to practice in a way that causes no harm to oneself or others.%
\item[MN 52: The Man From the City of \textsanskrit{Aṭṭhaka} (\textit{\textsanskrit{Aṭṭhakanāgarasutta}})] Asked by a householder to teach a path to freedom, Venerable Ānanda explains no less than eleven meditative states that may serve as doors to the deathless.%
\item[MN 53: A Trainee (\textit{\textsanskrit{Sekhasutta}})] The Buddha is invited by his family, the Sakyans of Kapilavatthu, to inaugurate a new community hall. He invites Venerable Ānanda to explain in detail the stages of spiritual practice for a lay trainee.%
\item[MN 54: With Potaliya the Householder (\textit{\textsanskrit{Potaliyasutta}})] When Potaliya got upset at being referred to as “householder”, the Buddha quizzed him as to the true nature of attachment and renunciation.%
\item[MN 55: With \textsanskrit{Jīvaka} (\textit{\textsanskrit{Jīvakasutta}})] The Buddha’s personal doctor, \textsanskrit{Jīvaka}, hears criticisms of the Buddha’s policy regarding eating meat, and asks him about it.%
\item[MN 56: With \textsanskrit{Upāli} (\textit{\textsanskrit{Upālisutta}})] The Buddha disagrees with a Jain ascetic on the question of whether physical or mental deeds are more important. When he hears of this, the Jain disciple \textsanskrit{Upāli} decides to visit the Buddha and refute him, and proceeds despite all warnings.%
\item[MN 57: The Ascetic Who Behaved Like a Dog (\textit{\textsanskrit{Kukkuravatikasutta}})] Some ascetics in ancient India undertook extreme practices, such as a vow to behave like an ox or a dog. The Buddha meets two such individuals, and is reluctantly pressed to reveal the kammic outcomes of such practice.%
\item[MN 58: With Prince Abhaya (\textit{\textsanskrit{Abhayarājakumārasutta}})] The leader of the Jains, \textsanskrit{Nigaṇṭha} \textsanskrit{Nātaputta}, gives his disciple Prince Abhaya a dilemma to pose to the Buddha, supposing that this will show his weakness. Things don’t go quite as planned.%
\item[MN 59: The Many Kinds of Feeling (\textit{\textsanskrit{Bahuvedanīyasutta}})] The Buddha resolves a disagreement on the number of kinds of feelings that he taught, pointing out that different ways of teaching are appropriate in different contexts, and should not be a cause of disputes. He goes on to show the importance of pleasure in developing higher meditation.%
\item[MN 60: Guaranteed (\textit{\textsanskrit{Apaṇṇakasutta}})] The Buddha teaches a group of uncommitted householders how to use a rational reflection to arrive at practices and principles that are guaranteed to have a good outcome, even if we don’t know all the variables.%
\item[The Chapter on Mendicants(\textit{\textsanskrit{Bhikkhuvagga}})] Ten discourses to monks, many of them focusing on matters of discipline.%
\item[MN 61: Advice to \textsanskrit{Rāhula} at \textsanskrit{Ambalaṭṭhika} (\textit{\textsanskrit{Ambalaṭṭhikarāhulovādasutta}})] Using the “object lesson” of a cup of water, the Buddha explains to his son, \textsanskrit{Rāhula}, the importance of telling the truth and reflecting on one’s motives.%
\item[MN 62: The Longer Advice to \textsanskrit{Rāhula} (\textit{\textsanskrit{Mahārāhulovādasutta}})] The Buddha tells \textsanskrit{Rāhula} to meditate on not-self, which he immediately puts into practice. Seeing him, Venerable \textsanskrit{Sāriputta} advises him to develop breath meditation, but the Buddha suggests a wide range of different practices first.%
\item[MN 63: The Shorter Discourse With \textsanskrit{Māluṅkya} (\textit{\textsanskrit{Cūḷamālukyasutta}})] A monk demands that the Buddha answer his metaphysical questions, or else he will disrobe. The Buddha compares him to a man struck by an arrow, who refuses treatment until he can have all his questions about the arrow and the archer answered.%
\item[MN 64: The Longer Discourse With \textsanskrit{Māluṅkya} (\textit{\textsanskrit{Mahāmālukyasutta}})] A little baby has no wrong views or intentions, but the underlying tendency for these things is still there. Without practicing, they will inevitably recur.%
\item[MN 65: With \textsanskrit{Bhaddāli} (\textit{\textsanskrit{Bhaddālisutta}})] A monk refuses to follow the rule forbidding eating after noon, but is filled with remorse and forgiven.%
\item[MN 66: The Simile of the Quail (\textit{\textsanskrit{Laṭukikopamasutta}})] Again raising the rule regarding eating, but this time as a reflection of gratitude for the Buddha in eliminating things that cause complexity and stress. The Buddha emphasizes how attachment even to little things can be dangerous.%
\item[MN 67: At \textsanskrit{Cātumā} (\textit{\textsanskrit{Cātumasutta}})] After dismissing some unruly monks, the Buddha is persuaded to relent, and teaches them four dangers for those gone forth.%
\item[MN 68: At \textsanskrit{Naḷakapāna} (\textit{\textsanskrit{Naḷakapānasutta}})] Those who practice do so not because they are failures, but because they aspire to higher freedom. When he speaks of the attainments of disciples, the Buddha does so in order to inspire.%
\item[MN 69: With \textsanskrit{Gulissāni} (\textit{\textsanskrit{Goliyānisutta}})] A monk comes down to the community from the wilderness, but doesn’t behave properly. Venerable \textsanskrit{Sāriputta} explains how a mendicant should behave, whether in forest or town.%
\item[MN 70: At \textsanskrit{Kīṭāgiri} (\textit{\textsanskrit{Kīṭāgirisutta}})] A third discourse that presents the health benefits of eating in one part of the day, and the reluctance of some mendicants to follow this.%
\item[The Chapter on Wanderers(\textit{\textsanskrit{Paribbājakavagga}})] The Buddha in dialog with ascetics and wanderers.%
\item[MN 71: To Vacchagotta on the Three Knowledges (\textit{\textsanskrit{Tevijjavacchasutta}})] The Buddha denies being omniscient, and sets forth the three higher knowledges that form the core of his awakened insight.%
\item[MN 72: With Vacchagotta on Fire (\textit{\textsanskrit{Aggivacchasutta}})] Refusing to take a stance regarding useless metaphysical speculations, the Buddha illustrates the spiritual goal with the simile of a flame going out.%
\item[MN 73: The Longer Discourse With Vacchagotta (\textit{\textsanskrit{Mahāvacchasutta}})] In the final installment of the “Vacchagotta trilogy”, Vacchagotta lets go his obsession with meaningless speculation, and asks about practice.%
\item[MN 74: With \textsanskrit{Dīghanakha} (\textit{\textsanskrit{Dīghanakhasutta}})] Deftly outmaneuvering an extreme skeptic, the Buddha discusses the outcomes of belief and disbelief. Rather than getting stuck in abstractions, he encourages staying close to the feelings one experiences.%
\item[MN 75: With \textsanskrit{Māgaṇḍiya} (\textit{\textsanskrit{Māgaṇḍiyasutta}})] Accused by a hedonist of being too negative, the Buddha recounts the luxury of his upbringing, and his realization of how little value there was in such things. Through renunciation he found a far greater pleasure.%
\item[MN 76: With Sandaka (\textit{\textsanskrit{Sandakasutta}})] Venerable Ānanda teaches a group of wanderers how there are many different approaches to the spiritual life, many of which lead nowhere.%
\item[MN 77: The Longer Discourse with \textsanskrit{Sakuludāyī} (\textit{\textsanskrit{Mahāsakuludāyisutta}})] Unlike many teachers, the Buddha’s followers treat him with genuine love and respect, since they see the sincerity of his teaching and practice.%
\item[MN 78: With \textsanskrit{Uggāhamāna} \textsanskrit{Samaṇamuṇḍika} (\textit{\textsanskrit{Samaṇamuṇḍikasutta}})] A wanderer teaches that a person has reached the highest attainment when they keep four basic ethical precepts. The Buddha’s standards are considerably higher.%
\item[MN 79: The Shorter Discourse With \textsanskrit{Sakuludāyī} (\textit{\textsanskrit{Cūḷasakuludāyisutta}})] A wanderer teaches his doctrine of the “highest splendor” but is unable to give a satisfactory account of what that means. The Buddha memorably compares him to someone who is in love with an idealized women who he has never met.%
\item[MN 80: With Vekhanasa (\textit{\textsanskrit{Vekhanasasutta}})] Starting off similar to the previous, the Buddha goes on to explain that one is not converted to his teaching just because of clever arguments, but because you see in yourself the results of the practice.%
\item[The Chapter on Kings(\textit{\textsanskrit{Rājavagga}})] Various dialogs with kings and princes, many of whom followed the Buddha.%
\item[MN 81: With \textsanskrit{Ghaṭīkāra} (\textit{\textsanskrit{Ghaṭikārasutta}})] The Buddha relates an unusual account of a past life in the time of the previous Buddha, Kassapa. At that time he was not interested in Dhamma, and had to be forced to go see the Buddha. This discourse is important in understanding the development of the Bodhisattva doctrine.%
\item[MN 82: With \textsanskrit{Raṭṭhapāla} (\textit{\textsanskrit{Raṭṭhapālasutta}})] A wealthy young man, \textsanskrit{Raṭṭhapāla}, has a strong aspiration to go forth, but has to prevail against the reluctance of his parents. Even after he became a monk, his parents tried to persuade him to disrobe. The discourse ends with a moving series of teachings on the fragility of the world.%
\item[MN 83: About King \textsanskrit{Makhādeva} (\textit{\textsanskrit{Maghadevasutta}})] A rare extended mythic narrative, telling of an ancient kingly lineage and their eventual downfall.%
\item[MN 84: At \textsanskrit{Madhurā} (\textit{\textsanskrit{Madhurasutta}})] In \textsanskrit{Madhurā}, towards the north-eastern limit of the Buddha’s reach during his life, King Avantiputta asks Venerable \textsanskrit{Mahākaccāna} regarding the brahmanical claim to be the highest caste.%
\item[MN 85: With Prince Bodhi (\textit{\textsanskrit{Bodhirājakumārasutta}})] Admitting that he used to believe that pleasure was to be gained through pain, the Buddha explains how his practice showed him the fallacy of that idea.%
\item[MN 86: With \textsanskrit{Aṅgulimāla} (\textit{\textsanskrit{Aṅgulimālasutta}})] Ignoring warnings, the Buddha ventures into the domain of the notorious killer \textsanskrit{Aṅgulimāla} and succeeds in converting him to the path of non-violence. After becoming a monk \textsanskrit{Aṅgulimāla} still suffered for his past deeds, but only to a small extent. He uses his new commitment to non-violence to help a woman in labor.%
\item[MN 87: Born From the Beloved (\textit{\textsanskrit{Piyajātikasutta}})] A rare glimpse into the marital life of King Pasenadi, and how he is led to the Dhamma by his Queen, the incomparable \textsanskrit{Mallikā}. She confirms the Buddha’s teaching that our loved ones bring us sorrow; but that’s not something a husband, father, and king wants to hear.%
\item[MN 88: The Imported Cloth (\textit{\textsanskrit{Bāhitikasutta}})] King Pasenadi takes a chance to visit Venerable Ānanda, where he asks about skillful and unskillful behavior, and what is praised by the Buddha. He offers Ānanda a valuable cloth in gratitude.%
\item[MN 89: Shrines to the Teaching (\textit{\textsanskrit{Dhammacetiyasutta}})] King Pasenadi, near the end of his life, visits the Buddha, and shows moving devotion and love for his teacher.%
\item[MN 90: At \textsanskrit{Kaṇṇakatthala} (\textit{\textsanskrit{Kaṇṇakatthalasutta}})] King Pasenadi questions the Buddha on miscellaneous matters: caste, omniscience, and the gods among them.%
\item[The Chapter on Brahmins(\textit{\textsanskrit{Brāhmaṇavagga}})] The Buddha engages with the powerful caste of brahmins, contesting their claims to spiritual authority.%
\item[MN 91: With \textsanskrit{Brahmāyu} (\textit{\textsanskrit{Brahmāyusutta}})] The oldest and most respected brahmin of the age sends a student to examine the Buddha, and he spends several months following his every move before reporting back. Convinced that the Buddha fulfills an ancient prophecy of the Great Man, the brahmin becomes his disciple.%
\item[MN 92: With Sela (\textit{\textsanskrit{Selasutta}})] A brahmanical ascetic named \textsanskrit{Keṇiya} invites the entire \textsanskrit{Saṅgha} for a meal. When the brahmin Sela sees what is happening, he visits the Buddha and expresses his delight in a moving series of devotional verses.%
\item[MN 93: With \textsanskrit{Assalāyana} (\textit{\textsanskrit{Assalāyanasutta}})] A precocious brahmin student is encouraged against his wishes to challenge the Buddha on the question of caste. His reluctance turns out to be justified.%
\item[MN 94: With \textsanskrit{Ghoṭamukha} (\textit{\textsanskrit{Ghoṭamukhasutta}})] A brahmin denies that there is such a thing as a principled renunciate life, but Venerable Udena persuades him otherwise.%
\item[MN 95: With \textsanskrit{Caṅkī} (\textit{\textsanskrit{Caṅkīsutta}})] The reputed brahmin \textsanskrit{Caṅkī} goes with a large group to visit the Buddha, despite the reservations of other brahmins. A precocious student challenges the Buddha, affirming the validity of the Vedic scriptures. The Buddha gives a detailed explanation of how true understanding gradually emerges through spiritual education.%
\item[MN 96: With \textsanskrit{Esukārī} (\textit{\textsanskrit{Esukārīsutta}})] A brahmin claims that one deserves service and privilege depending on caste, but the Buddha counters that it is conduct, not caste, that show a person’s worth.%
\item[MN 97: With \textsanskrit{Dhanañjāni} (\textit{\textsanskrit{Dhanañjānisutta}})] A corrupt tax-collector is redeemed by his encounter with Venerable \textsanskrit{Sāriputta}.%
\item[MN 98: With \textsanskrit{Vāseṭṭha} (\textit{\textsanskrit{Vāseṭṭhasutta}})] Two brahmin students ask the Buddha about what makes a brahmin: birth or deeds? the Buddha points out that, while the species of animals are determined by birth, for humans what matters is how you chose to live. This discourse anticipates the modern view that there are no such things as clearly defined racial differences among humans.%
\item[MN 99: With Subha (\textit{\textsanskrit{Subhasutta}})] Working hard is not valuable in and of itself; what matters is the outcome. And just as in lay life, spiritual practice may or may not lead to fruitful results.%
\item[MN 100: With \textsanskrit{Saṅgārava} (\textit{\textsanskrit{Saṅgāravasutta}})] Angered by the devotion of a brahmin lady, a brahmin visits the Buddha. He positions himself against traditionalists and rationalists, as someone whose teaching is based on direct experience.%
\end{description}

%
\mainmatter%
\pagestyle{fancy}%
\addtocontents{toc}{\let\protect\contentsline\protect\nopagecontentsline}
\part*{The Middle Fifty}
\addcontentsline{toc}{part}{The Middle Fifty}
\markboth{}{}
\addtocontents{toc}{\let\protect\contentsline\protect\oldcontentsline}

%
\addtocontents{toc}{\let\protect\contentsline\protect\nopagecontentsline}
\chapter*{The Chapter on Householders}
\addcontentsline{toc}{chapter}{\tocchapterline{The Chapter on Householders}}
\addtocontents{toc}{\let\protect\contentsline\protect\oldcontentsline}

%
\section*{{\suttatitleacronym MN 51}{\suttatitletranslation With Kandaraka }{\suttatitleroot Kandarakasutta}}
\addcontentsline{toc}{section}{\tocacronym{MN 51} \toctranslation{With Kandaraka } \tocroot{Kandarakasutta}}
\markboth{With Kandaraka }{Kandarakasutta}
\extramarks{MN 51}{MN 51}

\scevam{So\marginnote{1.1} I have heard. }At one time the Buddha was staying near \textsanskrit{Campā} on the banks of the \textsanskrit{Gaggarā} Lotus Pond together with a large \textsanskrit{Saṅgha} of mendicants. 

Then\marginnote{1.3} Pessa the elephant driver’s son and Kandaraka the wanderer went to see the Buddha. When they had approached, Pessa bowed and sat down to one side. But the wanderer Kandaraka exchanged greetings with the Buddha and stood to one side. He looked around the mendicant \textsanskrit{Saṅgha}, who were so very silent, and said to the Buddha: 

“It’s\marginnote{2.1} incredible, Master Gotama, it’s amazing! How the mendicant \textsanskrit{Saṅgha} has been led to practice properly by Master Gotama! All the perfected ones, the fully awakened Buddhas in the past or the future who lead the mendicant \textsanskrit{Saṅgha} to practice properly will at best do so like Master Gotama does in the present.” 

“That’s\marginnote{3.1} so true, Kandaraka! That’s so true! All the perfected ones, the fully awakened Buddhas in the past or the future who lead the mendicant \textsanskrit{Saṅgha} to practice properly will at best do so like I do in the present. 

For\marginnote{3.6} in this mendicant \textsanskrit{Saṅgha} there are perfected mendicants, who have ended the defilements, completed the spiritual journey, done what had to be done, laid down the burden, achieved their own goal, utterly ended the fetters of rebirth, and are rightly freed through enlightenment. And in this mendicant \textsanskrit{Saṅgha} there are trainee mendicants who are consistently ethical, living consistently, alert, living alertly. They meditate with their minds firmly established in the four kinds of mindfulness meditation. What four? 

It’s\marginnote{3.10} when a mendicant meditates by observing an aspect of the body—keen, aware, and mindful, rid of desire and aversion for the world. They meditate observing an aspect of feelings—keen, aware, and mindful, rid of desire and aversion for the world. They meditate observing an aspect of the mind—keen, aware, and mindful, rid of desire and aversion for the world. They meditate observing an aspect of principles—keen, aware, and mindful, rid of desire and aversion for the world.” 

When\marginnote{4.1} he had spoken, Pessa said to the Buddha: 

“It’s\marginnote{4.2} incredible, sir, it’s amazing, how much the Buddha has clearly described the four kinds of mindfulness meditation! They are in order to purify sentient beings, to get past sorrow and crying, to make an end of pain and sadness, to end the cycle of suffering, and to realize extinguishment. For we white-clothed laypeople also from time to time meditate with our minds well established in the four kinds of mindfulness meditation. We meditate observing an aspect of the body … feelings … mind … principles—keen, aware, and mindful, rid of desire and aversion for the world. 

It’s\marginnote{4.9} incredible, sir, it’s amazing! How the Buddha knows what’s best for sentient beings, even though people continue to be so shady, rotten, and tricky. For human beings are shady, sir, while the animal is obvious. For I can drive an elephant in training, and while going back and forth in \textsanskrit{Campā} it’ll try all the tricks, bluffs, ruses, and feints that it can. But my bondservants, employees, and workers behave one way by body, another by speech, and their minds another. It’s incredible, sir, it’s amazing! How the Buddha knows what’s best for sentient beings, even though people continue to be so shady, rotten, and tricky. For human beings are shady, sir, while the animal is obvious.” 

“That’s\marginnote{5.1} so true, Pessa! That’s so true! For human beings are shady, while the animal is obvious. Pessa, these four people are found in the world. What four? 

\begin{enumerate}%
\item One person mortifies themselves, committed to the practice of mortifying themselves. %
\item One person mortifies others, committed to the practice of mortifying others. %
\item One person mortifies themselves and others, committed to the practice of mortifying themselves and others. %
\item One person doesn’t mortify either themselves or others, committed to the practice of not mortifying themselves or others. They live without wishes in the present life, extinguished, cooled, experiencing bliss, having become holy in themselves. %
\end{enumerate}

Which\marginnote{5.11} one of these four people do you like the sound of?” 

“Sir,\marginnote{5.12} I don’t like the sound of the first three people. I only like the sound of the last person, who doesn’t mortify either themselves or others.” 

“But\marginnote{6.1} why don’t you like the sound of those three people?” 

“Sir,\marginnote{6.2} the person who mortifies themselves does so even though they want to be happy and recoil from pain. That’s why I don’t like the sound of that person. The person who mortifies others does so even though others want to be happy and recoil from pain. That’s why I don’t like the sound of that person. The person who mortifies themselves and others does so even though both themselves and others want to be happy and recoil from pain. That’s why I don’t like the sound of that person. The person who doesn’t mortify either themselves or others—living without wishes, extinguished, cooled, experiencing bliss, having become holy in themselves—does not torment themselves or others, both of whom want to be happy and recoil from pain. That’s why I like the sound of that person. Well, now, sir, I must go. I have many duties, and much to do.” 

“Please,\marginnote{6.13} Pessa, go at your convenience.” And then Pessa the elephant driver’s son approved and agreed with what the Buddha said. He got up from his seat, bowed, and respectfully circled the Buddha, keeping him on his right, before leaving. 

Then,\marginnote{7.1} not long after he had left, the Buddha addressed the mendicants: “Mendicants, Pessa the elephant driver’s son is astute. He has great wisdom. If he had sat here a little longer so that I could have analyzed these four people in detail, he would have greatly benefited. Still, even with this much he has already greatly benefited.” 

“Now\marginnote{7.6} is the time, Blessed One! Now is the time, Holy One! May the Buddha analyze these four people in detail. The mendicants will listen and remember it.” 

“Well\marginnote{7.8} then, mendicants, listen and pay close attention, I will speak.” 

“Yes,\marginnote{7.9} sir,” they replied. The Buddha said this: 

“And\marginnote{8.1} what person mortifies themselves, committed to the practice of mortifying themselves? It’s when someone goes naked, ignoring conventions. They lick their hands, and don’t come or wait when called. They don’t consent to food brought to them, or food prepared for them, or an invitation for a meal. They don’t receive anything from a pot or bowl; or from someone who keeps sheep, or who has a weapon or a shovel in their home; or where a couple is eating; or where there is a woman who is pregnant, breastfeeding, or who has a man in her home; or where there’s a dog waiting or flies buzzing. They accept no fish or meat or liquor or wine, and drink no beer. They go to just one house for alms, taking just one mouthful, or two houses and two mouthfuls, up to seven houses and seven mouthfuls. They feed on one saucer a day, two saucers a day, up to seven saucers a day. They eat once a day, once every second day, up to once a week, and so on, even up to once a fortnight. They live committed to the practice of eating food at set intervals. 

They\marginnote{8.6} eat herbs, millet, wild rice, poor rice, water lettuce, rice bran, scum from boiling rice, sesame flour, grass, or cow dung. They survive on forest roots and fruits, or eating fallen fruit. 

They\marginnote{8.7} wear robes of sunn hemp, mixed hemp, corpse-wrapping cloth, rags, lodh tree bark, antelope hide (whole or in strips), kusa grass, bark, wood-chips, human hair, horse-tail hair, or owls’ wings. They tear out their hair and beard, committed to this practice. They constantly stand, refusing seats. They squat, committed to the endeavor of squatting. They lie on a mat of thorns, making a mat of thorns their bed. They’re committed to the practice of immersion in water three times a day, including the evening. And so they live committed to practicing these various ways of mortifying and tormenting the body. This is called a person who mortifies themselves, being committed to the practice of mortifying themselves. 

And\marginnote{9.1} what person mortifies others, committed to the practice of mortifying others? It’s when a person is a slaughterer of sheep, pigs, or poultry, a hunter or trapper, a fisher, a bandit, an executioner, a butcher, a jailer, or someone with some other kind of cruel livelihood. This is called a person who mortifies others, being committed to the practice of mortifying others. 

And\marginnote{10.1} what person mortifies themselves and others, being committed to the practice of mortifying themselves and others? It’s when a person is an anointed aristocratic king or a well-to-do brahmin. He has a new temple built to the east of the city. He shaves off his hair and beard, dresses in a rough antelope hide, and smears his body with ghee and oil. Scratching his back with antlers, he enters the temple with his chief queen and the brahmin high priest. There he lies on the bare ground strewn with grass. The king feeds on the milk from one teat of a cow that has a calf of the same color. The chief queen feeds on the milk from the second teat. The brahmin high priest feeds on the milk from the third teat. The milk from the fourth teat is served to the sacred flame. The calf feeds on the remainder. He says: ‘Slaughter this many bulls, bullocks, heifers, goats, rams, and horses for the sacrifice! Fell this many trees and reap this much grass for the sacrificial equipment!’ His bondservants, employees, and workers do their jobs under threat of punishment and danger, weeping with tearful faces. This is called a person who mortifies themselves and others, being committed to the practice of mortifying themselves and others. 

And\marginnote{11.1} what person doesn’t mortify either themselves or others, but lives without wishes, extinguished, cooled, experiencing bliss, having become holy in themselves? 

It’s\marginnote{12.1} when a Realized One arises in the world, perfected, a fully awakened Buddha, accomplished in knowledge and conduct, holy, knower of the world, supreme guide for those who wish to train, teacher of gods and humans, awakened, blessed. He has realized with his own insight this world—with its gods, \textsanskrit{Māras} and \textsanskrit{Brahmās}, this population with its ascetics and brahmins, gods and humans—and he makes it known to others. He teaches Dhamma that’s good in the beginning, good in the middle, and good in the end, meaningful and well-phrased. And he reveals a spiritual practice that’s entirely full and pure. 

A\marginnote{13.1} householder hears that teaching, or a householder’s child, or someone reborn in some clan. They gain faith in the Realized One, and reflect: ‘Living in a house is cramped and dirty, but the life of one gone forth is wide open. It’s not easy for someone living at home to lead the spiritual life utterly full and pure, like a polished shell. Why don’t I shave off my hair and beard, dress in ocher robes, and go forth from the lay life to homelessness?’ After some time they give up a large or small fortune, and a large or small family circle. They shave off hair and beard, dress in ocher robes, and go forth from the lay life to homelessness. 

Once\marginnote{14.1} they’ve gone forth, they take up the training and livelihood of the mendicants. They give up killing living creatures, renouncing the rod and the sword. They’re scrupulous and kind, living full of compassion for all living beings. They give up stealing. They take only what’s given, and expect only what’s given. They keep themselves clean by not thieving. They give up unchastity. They are celibate, set apart, avoiding the common practice of sex. They give up lying. They speak the truth and stick to the truth. They’re honest and trustworthy, and don’t trick the world with their words. They give up divisive speech. They don’t repeat in one place what they heard in another so as to divide people against each other. Instead, they reconcile those who are divided, supporting unity, delighting in harmony, loving harmony, speaking words that promote harmony. They give up harsh speech. They speak in a way that’s mellow, pleasing to the ear, lovely, going to the heart, polite, likable and agreeable to the people. They give up talking nonsense. Their words are timely, true, and meaningful, in line with the teaching and training. They say things at the right time which are valuable, reasonable, succinct, and beneficial. They avoid injuring plants and seeds. They eat in one part of the day, abstaining from eating at night and food at the wrong time. They avoid dancing, singing, music, and seeing shows. They avoid beautifying and adorning themselves with garlands, perfumes, and makeup. They avoid high and luxurious beds. They avoid receiving gold and money, raw grains, raw meat, women and girls, male and female bondservants, goats and sheep, chickens and pigs, elephants, cows, horses, and mares, and fields and land. They avoid running errands and messages; buying and selling; falsifying weights, metals, or measures; bribery, fraud, cheating, and duplicity; mutilation, murder, abduction, banditry, plunder, and violence. 

They’re\marginnote{15.1} content with robes to look after the body and almsfood to look after the belly. Wherever they go, they set out taking only these things. They’re like a bird: wherever it flies, wings are its only burden. In the same way, a mendicant is content with robes to look after the body and almsfood to look after the belly. Wherever they go, they set out taking only these things. When they have this entire spectrum of noble ethics, they experience a blameless happiness inside themselves. 

When\marginnote{16.1} they see a sight with their eyes, they don’t get caught up in the features and details. If the faculty of sight were left unrestrained, bad unskillful qualities of desire and aversion would become overwhelming. For this reason, they practice restraint, protecting the faculty of sight, and achieving its restraint. When they hear a sound with their ears … When they smell an odor with their nose … When they taste a flavor with their tongue … When they feel a touch with their body … When they know a thought with their mind, they don’t get caught up in the features and details. If the faculty of mind were left unrestrained, bad unskillful qualities of desire and aversion would become overwhelming. For this reason, they practice restraint, protecting the faculty of mind, and achieving its restraint. When they have this noble sense restraint, they experience an unsullied bliss inside themselves. 

They\marginnote{17.1} act with situational awareness when going out and coming back; when looking ahead and aside; when bending and extending the limbs; when bearing the outer robe, bowl and robes; when eating, drinking, chewing, and tasting; when urinating and defecating; when walking, standing, sitting, sleeping, waking, speaking, and keeping silent. 

When\marginnote{18.1} they have this noble spectrum of ethics, this noble contentment, this noble sense restraint, and this noble mindfulness and situational awareness, they frequent a secluded lodging—a wilderness, the root of a tree, a hill, a ravine, a mountain cave, a charnel ground, a forest, the open air, a heap of straw. 

After\marginnote{19.1} the meal, they return from almsround, sit down cross-legged with their body straight, and establish mindfulness right there. Giving up desire for the world, they meditate with a heart rid of desire, cleansing the mind of desire. Giving up ill will, they meditate with a mind rid of ill will, full of compassion for all living beings, cleansing the mind of ill will and malevolence. Giving up dullness and drowsiness, they meditate with a mind rid of dullness and drowsiness, perceiving light, mindful and aware, cleansing the mind of dullness and drowsiness. Giving up restlessness and remorse, they meditate without restlessness, their mind peaceful inside, cleansing the mind of restlessness and remorse. Giving up doubt, they meditate having gone beyond doubt, not undecided about skillful qualities, cleansing the mind of doubt. 

They\marginnote{20.1} give up these five hindrances, corruptions of the heart that weaken wisdom. Then, quite secluded from sensual pleasures, secluded from unskillful qualities, they enter and remain in the first absorption, which has the rapture and bliss born of seclusion, while placing the mind and keeping it connected. 

As\marginnote{21.1} the placing of the mind and keeping it connected are stilled, they enter and remain in the second absorption, which has the rapture and bliss born of immersion, with internal clarity and confidence, and unified mind, without placing the mind and keeping it connected. 

And\marginnote{22.1} with the fading away of rapture, they enter and remain in the third absorption, where they meditate with equanimity, mindful and aware, personally experiencing the bliss of which the noble ones declare, ‘Equanimous and mindful, one meditates in bliss.’ 

Giving\marginnote{23.1} up pleasure and pain, and ending former happiness and sadness, they enter and remain in the fourth absorption, without pleasure or pain, with pure equanimity and mindfulness. 

When\marginnote{24.1} their mind has become immersed in \textsanskrit{samādhi} like this—purified, bright, flawless, rid of corruptions, pliable, workable, steady, and imperturbable—they extend it toward recollection of past lives. They recollect many kinds of past lives, that is, one, two, three, four, five, ten, twenty, thirty, forty, fifty, a hundred, a thousand, a hundred thousand rebirths; many eons of the world contracting, many eons of the world expanding, many eons of the world contracting and expanding. They remember: ‘There, I was named this, my clan was that, I looked like this, and that was my food. This was how I felt pleasure and pain, and that was how my life ended. When I passed away from that place I was reborn somewhere else. There, too, I was named this, my clan was that, I looked like this, and that was my food. This was how I felt pleasure and pain, and that was how my life ended. When I passed away from that place I was reborn here.’ And so they recollect their many kinds of past lives, with features and details. 

When\marginnote{25.1} their mind has become immersed in \textsanskrit{samādhi} like this—purified, bright, flawless, rid of corruptions, pliable, workable, steady, and imperturbable—they extend it toward knowledge of the death and rebirth of sentient beings. With clairvoyance that is purified and superhuman, they see sentient beings passing away and being reborn—inferior and superior, beautiful and ugly, in a good place or a bad place. They understand how sentient beings are reborn according to their deeds: ‘These dear beings did bad things by way of body, speech, and mind. They spoke ill of the noble ones; they had wrong view; and they chose to act out of that wrong view. When their body breaks up, after death, they’re reborn in a place of loss, a bad place, the underworld, hell. These dear beings, however, did good things by way of body, speech, and mind. They never spoke ill of the noble ones; they had right view; and they chose to act out of that right view. When their body breaks up, after death, they’re reborn in a good place, a heavenly realm.’ And so, with clairvoyance that is purified and superhuman, they see sentient beings passing away and being reborn—inferior and superior, beautiful and ugly, in a good place or a bad place. They understand how sentient beings are reborn according to their deeds. 

When\marginnote{26.1} their mind has become immersed in \textsanskrit{samādhi} like this—purified, bright, flawless, rid of corruptions, pliable, workable, steady, and imperturbable—they extend it toward knowledge of the ending of defilements. They truly understand: ‘This is suffering’ … ‘This is the origin of suffering’ … ‘This is the cessation of suffering’ … ‘This is the practice that leads to the cessation of suffering’. 

They\marginnote{27.1} truly understand: ‘These are defilements’ … ‘This is the origin of defilements’ … ‘This is the cessation of defilements’ … ‘This is the practice that leads to the cessation of defilements’. Knowing and seeing like this, their mind is freed from the defilements of sensuality, desire to be reborn, and ignorance. When they’re freed, they know they’re freed. 

They\marginnote{27.4} understand: ‘Rebirth is ended, the spiritual journey has been completed, what had to be done has been done, there is no return to any state of existence.’ 

This\marginnote{28.1} is called a person who neither mortifies themselves or others, being committed to the practice of not mortifying themselves or others. They live without wishes in the present life, extinguished, cooled, experiencing bliss, having become holy in themselves.” 

That\marginnote{28.3} is what the Buddha said. Satisfied, the mendicants were happy with what the Buddha said. 

%
\section*{{\suttatitleacronym MN 52}{\suttatitletranslation The Man From the City of Aṭṭhaka }{\suttatitleroot Aṭṭhakanāgarasutta}}
\addcontentsline{toc}{section}{\tocacronym{MN 52} \toctranslation{The Man From the City of Aṭṭhaka } \tocroot{Aṭṭhakanāgarasutta}}
\markboth{The Man From the City of Aṭṭhaka }{Aṭṭhakanāgarasutta}
\extramarks{MN 52}{MN 52}

\scevam{So\marginnote{1.1} I have heard. }At one time Venerable Ānanda was staying near \textsanskrit{Vesālī} in the little village of Beluva. 

Now\marginnote{1.3} at that time the householder Dasama from the city of \textsanskrit{Aṭṭhaka} had arrived at \textsanskrit{Pāṭaliputta} on some business. He went to the Chicken Monastery, approached a certain mendicant, bowed, sat down to one side, and said to him, “Sir, where is Venerable Ānanda now staying? For I want to see him.” 

“Householder,\marginnote{2.4} Venerable Ānanda is staying near \textsanskrit{Vesālī} in the little village of Beluva.” 

Then\marginnote{3.1} the householder Dasama, having concluded his business there, went to the little village of Beluva in \textsanskrit{Vesālī} to see Ānanda. He bowed, sat down to one side, and said to Ānanda: 

“Sir,\marginnote{3.2} Ānanda, is there one thing that has been rightly explained by the Blessed One—who knows and sees, the perfected one, the fully awakened Buddha—practicing which a diligent, keen, and resolute mendicant’s mind is freed, their defilements are ended, and they arrive at the supreme sanctuary?” 

“There\marginnote{3.3} is, householder.” 

“And\marginnote{3.4} what is that one thing?” 

“Householder,\marginnote{4.1} it’s when a mendicant, quite secluded from sensual pleasures, secluded from unskillful qualities, enters and remains in the first absorption, which has the rapture and bliss born of seclusion, while placing the mind and keeping it connected. Then they reflect: ‘Even this first absorption is produced by choices and intentions.’ They understand: ‘But whatever is produced by choices and intentions is impermanent and liable to cessation.’ Abiding in that they attain the ending of defilements. If they don’t attain the ending of defilements, with the ending of the five lower fetters they’re reborn spontaneously, because of their passion and love for that meditation. They are extinguished there, and are not liable to return from that world. This is one thing that has been rightly explained by the Blessed One—who knows and sees, the perfected one, the fully awakened Buddha—practicing which a diligent, keen, and resolute mendicant’s mind is freed, their defilements are ended, and they arrive at the supreme sanctuary. 

Furthermore,\marginnote{5.1} as the placing of the mind and keeping it connected are stilled, they enter and remain in the second absorption … third absorption … fourth absorption … 

Furthermore,\marginnote{8.1} a mendicant meditates spreading a heart full of love to one direction, and to the second, and to the third, and to the fourth. In the same way above, below, across, everywhere, all around, they spread a heart full of love to the whole world—abundant, expansive, limitless, free of enmity and ill will. Then they reflect: ‘Even this heart’s release by love is produced by choices and intentions.’ They understand: ‘But whatever is produced by choices and intentions is impermanent and liable to cessation.’ … 

Furthermore,\marginnote{9.1} a mendicant meditates spreading a heart full of compassion … rejoicing … equanimity … 

Furthermore,\marginnote{12.1} householder, a mendicant, going totally beyond perceptions of form, with the ending of perceptions of impingement, not focusing on perceptions of diversity, aware that ‘space is infinite’, enters and remains in the dimension of infinite space. Then they reflect: ‘Even this attainment of the dimension of infinite space is produced by choices and intentions.’ They understand: ‘But whatever is produced by choices and intentions is impermanent and liable to cessation.’ … 

Furthermore,\marginnote{13.1} a mendicant, going totally beyond the dimension of infinite space, aware that ‘consciousness is infinite’, enters and remains in the dimension of infinite consciousness. … 

Furthermore,\marginnote{14.1} a mendicant, going totally beyond the dimension of infinite consciousness, aware that ‘there is nothing at all’, enters and remains in the dimension of nothingness. Then they reflect: ‘Even this attainment of the dimension of nothingness is produced by choices and intentions.’ They understand: ‘But whatever is produced by choices and intentions is impermanent and liable to cessation.’ Abiding in that they attain the ending of defilements. If they don’t attain the ending of defilements, with the ending of the five lower fetters they’re reborn spontaneously because of their passion and love for that meditation. They are extinguished there, and are not liable to return from that world. This too is one thing that has been rightly explained by the Blessed One—who knows and sees, the perfected one, the fully awakened Buddha—practicing which a diligent, keen, and resolute mendicant’s mind is freed, their defilements are ended, and they arrive at the supreme sanctuary.” 

When\marginnote{15.1} he said this, the householder Dasama said to Venerable Ānanda, “Sir, suppose a person was looking for an entrance to a hidden treasure. And all at once they’d come across eleven entrances! In the same way, I was searching for the door to the deathless. And all at once I got to hear of eleven doors to the deathless. Suppose a person had a house with eleven doors. If the house caught fire they’d be able to flee to safety through any one of those doors. In the same way, I’m able to flee to safety through any one of these eleven doors to the deathless. Sir, those who follow other paths seek a fee for the teacher. Why shouldn’t I make an offering to Venerable Ānanda?” 

Then\marginnote{16.1} the householder Dasama, having assembled the \textsanskrit{Saṅgha} from \textsanskrit{Vesālī} and \textsanskrit{Pāṭaliputta}, served and satisfied them with his own hands with a variety of delicious foods. He clothed each and every mendicant in a pair of garments, with a set of three robes for Ānanda. And he had a dwelling worth five hundred built for Ānanda. 

%
\section*{{\suttatitleacronym MN 53}{\suttatitletranslation A Trainee }{\suttatitleroot Sekhasutta}}
\addcontentsline{toc}{section}{\tocacronym{MN 53} \toctranslation{A Trainee } \tocroot{Sekhasutta}}
\markboth{A Trainee }{Sekhasutta}
\extramarks{MN 53}{MN 53}

\scevam{So\marginnote{1.1} I have heard. }At one time the Buddha was staying in the land of the Sakyans, near Kapilavatthu in the Banyan Tree Monastery. 

Now\marginnote{2.1} at that time a new town hall had recently been constructed for the Sakyans of Kapilavatthu. It had not yet been occupied by an ascetic or brahmin or any person at all. Then the Sakyans of Kapilavatthu went up to the Buddha, bowed, sat down to one side, and said to him: 

“Sir,\marginnote{2.3} a new town hall has recently been constructed for the Sakyans of Kapilavatthu. It has not yet been occupied by an ascetic or brahmin or any person at all. May the Buddha be the first to use it, and only then will the Sakyans of Kapilavatthu use it. That would be for the lasting welfare and happiness of the Sakyans of Kapilavatthu.” The Buddha consented in silence. 

Then,\marginnote{3.2} knowing that the Buddha had consented, the Sakyans got up from their seat, bowed, and respectfully circled the Buddha, keeping him on their right. Then they went to the new town hall, where they spread carpets all over, prepared seats, set up a water jar, and placed a lamp. Then they went back to the Buddha, bowed, stood to one side, and told him of their preparations, saying, “Please, sir, come at your convenience.” 

Then\marginnote{4.1} the Buddha robed up and, taking his bowl and robe, went to the new town hall together with the \textsanskrit{Saṅgha} of mendicants. Having washed his feet he entered the town hall and sat against the central column facing east. The \textsanskrit{Saṅgha} of mendicants also washed their feet, entered the town hall, and sat against the west wall facing east, with the Buddha right in front of them. The Sakyans of Kapilavatthu also washed their feet, entered the town hall, and sat against the east wall facing west, with the Buddha right in front of them. 

The\marginnote{5.1} Buddha spent most of the night educating, encouraging, firing up, and inspiring the Sakyans with a Dhamma talk. Then he addressed Venerable Ānanda, “Ānanda, speak about the practicing trainee to the Sakyans of Kapilavatthu as you feel inspired. My back is sore, I’ll stretch it.” 

“Yes,\marginnote{5.5} sir,” Ānanda replied. And then the Buddha spread out his outer robe folded in four and laid down in the lion’s posture—on the right side, placing one foot on top of the other—mindful and aware, and focused on the time of getting up. 

Then\marginnote{6.1} Ānanda addressed \textsanskrit{Mahānāma} the Sakyan: 

“\textsanskrit{Mahānāma},\marginnote{6.2} a noble disciple is accomplished in ethics, guards the sense doors, eats in moderation, and is dedicated to wakefulness. They have seven good qualities, and they get the four absorptions—blissful meditations in the present life that belong to the higher mind—when they want, without trouble or difficulty. 

And\marginnote{7.1} how is a noble disciple accomplished in ethics? It’s when a noble disciple is ethical, restrained in the monastic code, conducting themselves well and seeking alms in suitable places. Seeing danger in the slightest fault, they keep the rules they’ve undertaken. That’s how a noble disciple is ethical. 

And\marginnote{8.1} how does a noble disciple guard the sense doors? When a noble disciple sees a sight with their eyes, they don’t get caught up in the features and details. If the faculty of sight were left unrestrained, bad unskillful qualities of desire and aversion would become overwhelming. For this reason, they practice restraint, protecting the faculty of sight, and achieving its restraint. When they hear a sound with their ears … When they smell an odor with their nose … When they taste a flavor with their tongue … When they feel a touch with their body … When they know a thought with their mind, they don’t get caught up in the features and details. If the faculty of mind were left unrestrained, bad unskillful qualities of desire and aversion would become overwhelming. For this reason, they practice restraint, protecting the faculty of mind, and achieving its restraint. That’s how a noble disciple guards the sense doors. 

And\marginnote{9.1} how does a noble disciple eat in moderation? It’s when a noble disciple reflects properly on the food that they eat: ‘Not for fun, indulgence, adornment, or decoration, but only to sustain this body, to avoid harm, and to support spiritual practice. In this way, I shall put an end to old discomfort and not give rise to new discomfort, and I will live blamelessly and at ease.’ That’s how a noble disciple eats in moderation. 

And\marginnote{10.1} how is a noble disciple dedicated to wakefulness? It’s when a noble disciple practices walking and sitting meditation by day, purifying their mind from obstacles. In the evening, they continue to practice walking and sitting meditation. In the middle of the night, they lie down in the lion’s posture—on the right side, placing one foot on top of the other—mindful and aware, and focused on the time of getting up. In the last part of the night, they get up and continue to practice walking and sitting meditation, purifying their mind from obstacles. That’s how a noble disciple is dedicated to wakefulness. 

And\marginnote{11.1} how does a noble disciple have seven good qualities? It’s when a noble disciple has faith in the Realized One’s awakening: ‘That Blessed One is perfected, a fully awakened Buddha, accomplished in knowledge and conduct, holy, knower of the world, supreme guide for those who wish to train, teacher of gods and humans, awakened, blessed.’ 

They\marginnote{12.1} have a conscience. They’re conscientious about bad conduct by way of body, speech, and mind, and conscientious about having any bad, unskillful qualities. 

They\marginnote{13.1} exercise prudence. They’re prudent when it comes to bad conduct by way of body, speech, and mind, and prudent when it comes to acquiring any bad, unskillful qualities. 

They’re\marginnote{14.1} very learned, remembering and keeping what they’ve learned. These teachings are good in the beginning, good in the middle, and good in the end, meaningful and well-phrased, describing a spiritual practice that’s entirely full and pure. They are very learned in such teachings, remembering them, reinforcing them by recitation, mentally scrutinizing them, and comprehending them theoretically. 

They\marginnote{15.1} live with energy roused up for giving up unskillful qualities and embracing skillful qualities. They’re strong, staunchly vigorous, not slacking off when it comes to developing skillful qualities. 

They’re\marginnote{16.1} mindful. They have utmost mindfulness and alertness, and can remember and recall what was said and done long ago. 

They’re\marginnote{17.1} wise. They have the wisdom of arising and passing away which is noble, penetrative, and leads to the complete ending of suffering. That’s how a noble disciple has seven good qualities. 

And\marginnote{18.1} how does a noble disciple get the four absorptions—blissful meditations in the present life that belong to the higher mind—when they want, without trouble or difficulty? It’s when a noble disciple, quite secluded from sensual pleasures, secluded from unskillful qualities, enters and remains in the first absorption … second absorption … third absorption … fourth absorption. That’s how a noble disciple gets the four absorptions—blissful meditations in the present life that belong to the higher mind—when they want, without trouble or difficulty. 

When\marginnote{19.1} a noble disciple is accomplished in ethics, guards the sense doors, eats in moderation, and is dedicated to wakefulness; and they have seven good qualities, and they get the four absorptions—blissful meditations in the present life that belong to the higher mind—when they want, without trouble or difficulty, they are called a noble disciple who is a practicing trainee. Their eggs are unspoiled, and they are capable of breaking out of their shell, becoming awakened, and achieving the supreme sanctuary. Suppose there was a chicken with eight or ten or twelve eggs. And she properly sat on them to keep them warm and incubated. Even if that chicken doesn’t wish, ‘If only my chicks could break out of the eggshell with their claws and beak and hatch safely!’ Still they can break out and hatch safely. 

In\marginnote{19.5} the same way, when a noble disciple is practicing all these things they are called a noble disciple who is a practicing trainee. Their eggs are unspoiled, and they are capable of breaking out of their shell, becoming awakened, and achieving the supreme sanctuary. 

Relying\marginnote{20.1} on this supreme purity of mindfulness and equanimity, that noble disciple recollects their many kinds of past lives. That is: one, two, three, four, five, ten, twenty, thirty, forty, fifty, a hundred, a thousand, a hundred thousand rebirths; many eons of the world contracting, many eons of the world expanding, many eons of the world contracting and expanding. … And so they recollect their many kinds of past lives, with features and details. This is their first breaking out, like a chick from an eggshell. 

Relying\marginnote{21.1} on this supreme purity of mindfulness and equanimity, that noble disciple, with clairvoyance that is purified and superhuman, sees sentient beings passing away and being reborn—inferior and superior, beautiful and ugly, in a good place or a bad place. … They understand how sentient beings are reborn according to their deeds. This is their second breaking out, like a chick from an eggshell. 

Relying\marginnote{22.1} on this supreme purity of mindfulness and equanimity, that noble disciple realizes the undefiled freedom of heart and freedom by wisdom in this very life. And they live having realized it with their own insight due to the ending of defilements. This is their third breaking out, like a chick from an eggshell. 

A\marginnote{23.1} noble disciple’s conduct includes the following: being accomplished in ethics, guarding the sense doors, moderation in eating, being dedicated to wakefulness, having seven good qualities, and getting the four absorptions when they want, without trouble or difficulty. 

A\marginnote{24.1} noble disciple’s knowledge includes the following: recollecting their past lives, clairvoyance that is purified and superhuman, and realizing the undefiled freedom of heart and freedom by wisdom in this very life due to the ending of defilements. 

This\marginnote{25.1} noble disciple is said to be ‘accomplished in knowledge’, and also ‘accomplished in conduct’, and also ‘accomplished in knowledge and conduct’. 

And\marginnote{25.2} \textsanskrit{Brahmā} \textsanskrit{Sanaṅkumāra} also spoke this verse: 

\begin{verse}%
‘The\marginnote{25.3} aristocrat is first among people \\
who take clan as the standard. \\
But one accomplished in knowledge and conduct \\
is first among gods and humans.’ 

%
\end{verse}

And\marginnote{25.7} that verse was well sung by \textsanskrit{Brahmā} \textsanskrit{Sanaṅkumāra}, not poorly sung; well spoken, not poorly spoken, beneficial, not harmful, and it was approved by the Buddha.” 

Then\marginnote{26.1} the Buddha got up and said to Venerable Ānanda, “Good, good, Ānanda! It’s good that you spoke to the Sakyans of Kapilavatthu about the practicing trainee.” 

This\marginnote{26.4} is what Venerable Ānanda said, and the teacher approved. Satisfied, the Sakyans of Kapilavatthu were happy with what Venerable Ānanda said. 

%
\section*{{\suttatitleacronym MN 54}{\suttatitletranslation With Potaliya the Householder }{\suttatitleroot Potaliyasutta}}
\addcontentsline{toc}{section}{\tocacronym{MN 54} \toctranslation{With Potaliya the Householder } \tocroot{Potaliyasutta}}
\markboth{With Potaliya the Householder }{Potaliyasutta}
\extramarks{MN 54}{MN 54}

\scevam{So\marginnote{1.1} I have heard. }At one time the Buddha was staying in the land of the Northern \textsanskrit{Āpaṇas}, near the town of theirs named \textsanskrit{Āpaṇa}. 

Then\marginnote{2.1} the Buddha robed up in the morning and, taking his bowl and robe, entered \textsanskrit{Āpaṇa} for alms. He wandered for alms in \textsanskrit{Āpaṇa}. After the meal, on his return from almsround, he went to a certain forest grove for the day’s meditation. Having plunged deep into it, he sat at the root of a certain tree for the day’s meditation. 

Potaliya\marginnote{3.1} the householder also approached that forest grove while going for a walk. He was well dressed in a cloak and sarong, with parasol and sandals. Having plunged deep into it, he went up to the Buddha, and exchanged greetings with him. When the greetings and polite conversation were over, he stood to one side, and the Buddha said to him, “There are seats, householder. Please sit if you wish.” 

When\marginnote{3.4} he said this, Potaliya was angry and upset. Thinking, “The ascetic Gotama addresses me as ‘householder’!” he stayed silent. 

For\marginnote{3.5} a second time … and a third time the Buddha said to him, “There are seats, householder. Please sit if you wish.” 

When\marginnote{3.8} he said this, Potaliya was angry and upset. Thinking, “The ascetic Gotama addresses me as ‘householder’!” he said to the Buddha, “Master Gotama, it is neither proper nor appropriate for you to address me as ‘householder’.” 

“Well,\marginnote{3.10} householder, you have the features, attributes, and signs of a householder.” 

“Master\marginnote{3.11} Gotama, it’s because I have refused all work and cut off all judgments.” 

“Householder,\marginnote{3.12} in what way have you refused all work and cut off all judgments?” 

“Master\marginnote{3.13} Gotama, all the money, grain, gold, and silver I used to have has been handed over to my children as their inheritance. And in this matter I do not advise or reprimand them, but live with nothing more than food and clothes. That’s how I have refused all work and cut off all judgments.” 

“The\marginnote{3.15} cutting off of judgments as you describe it is one thing, householder, but the cutting off of judgments in the noble one’s training is quite different.” 

“But\marginnote{3.16} what, sir, is cutting off of judgments in the noble one’s training? Sir, please teach me this.” 

“Well\marginnote{3.18} then, householder, listen and pay close attention, I will speak.” 

“Yes,\marginnote{3.19} sir,” said Potaliya. 

The\marginnote{4.1} Buddha said this: 

“Householder,\marginnote{4.2} these eight things lead to the cutting off of judgments in the noble one’s training. What eight? Killing living creatures should be given up, relying on not killing living creatures. Stealing should be given up, relying on not stealing. Lying should be given up, relying on speaking the truth. Divisive speech should be given up, relying on speech that isn’t divisive. Greed and lust should be given up, relying on not being greedy and lustful. Blaming and insulting should be given up, relying on not blaming and not insulting. Anger and distress should be given up, relying on not being angry and distressed. Arrogance should be given up, relying on not being arrogant. These are the eight things—stated in brief without being analyzed in detail—that lead to the cutting off of judgments in the noble one’s training.” 

“Sir,\marginnote{5.1} please teach me these eight things in detail out of compassion.” 

“Well\marginnote{5.2} then, householder, listen and pay close attention, I will speak.” 

“Yes,\marginnote{5.3} sir,” said Potaliya. The Buddha said this: 

“‘Killing\marginnote{6.1} living creatures should be given up, relying on not killing living creatures.’ That’s what I said, but why did I say it? It’s when a noble disciple reflects: ‘I am practicing to give up and cut off the fetters that might cause me to kill living creatures. But if I were to kill living creatures, because of that I would reprimand myself; sensible people, after examination, would criticize me; and when my body breaks up, after death, I could expect to be reborn in a bad place. And killing living creatures is itself a fetter and a hindrance. The distressing and feverish defilements that might arise because of killing living creatures do not occur in someone who does not kill living creatures.’ ‘Killing living creatures should be given up, relying on not killing living creatures.’ That’s what I said, and this is why I said it. 

‘Stealing\marginnote{7.1} … lying … divisive speech … greed and lust … blaming and insulting … anger and distress … 

Arrogance\marginnote{13.1} should be given up, relying on not being arrogant.’ That’s what I said, but why did I say it? It’s when a noble disciple reflects: ‘I am practicing to give up and cut off the fetters that might cause me to be arrogant. But if I were to be arrogant, because of that I would reprimand myself; sensible people, after examination, would criticize me; and when my body breaks up, after death, I could expect to be reborn in a bad place. And arrogance is itself a fetter and a hindrance. The distressing and feverish defilements that might arise because of arrogance do not occur in someone who is not arrogant.’ ‘Arrogance should be given up by not being arrogant.’ That’s what I said, and this is why I said it. 

These\marginnote{14.1} are the eight things—stated in brief and analyzed in detail—that lead to the cutting off of judgments in the noble one’s training. But just this much does not constitute the cutting off of judgments in each and every respect in the noble one’s training.” 

“But,\marginnote{14.3} sir, how is there the cutting off of judgments in each and every respect in the noble one’s training? Sir, please teach me this.” 

“Well\marginnote{14.5} then, householder, listen and pay close attention, I will speak.” 

“Yes,\marginnote{14.6} sir,” said Potaliya. The Buddha said this: 

\subsection*{1. The Dangers of Sensual Pleasures }

“Householder,\marginnote{15.1} suppose a dog weak with hunger was hanging around a butcher’s shop. Then a deft butcher or their apprentice would toss them a skeleton scraped clean of flesh and smeared in blood. What do you think, householder? Gnawing on such a fleshless skeleton, would that dog still get rid of its hunger?” 

“No,\marginnote{15.5} sir. Why not? Because that skeleton is scraped clean of flesh and smeared in blood. That dog will eventually get weary and frustrated.” 

“In\marginnote{15.9} the same way, a noble disciple reflects: ‘With the simile of a skeleton the Buddha said that sensual pleasures give little gratification and much suffering and distress, and they are all the more full of drawbacks.’ Having truly seen this with right understanding, they reject equanimity based on diversity and develop only the equanimity based on unity, where all kinds of grasping to the world’s material delights cease without anything left over. 

Suppose\marginnote{16.1} a vulture or a crow or a hawk was to grab a lump of meat and fly away. Other vultures, crows, and hawks would keep chasing it, pecking and clawing. What do you think, householder? If that vulture, crow, or hawk doesn’t quickly let go of that lump of meat, wouldn’t that result in death or deadly suffering for them?” 

“Yes,\marginnote{16.5} sir.” … 

“Suppose\marginnote{17.1} a person carrying a blazing grass torch was to walk against the wind. What do you think, householder? If that person doesn’t quickly let go of that blazing grass torch, wouldn’t they burn their hands or arm or other limb, resulting in death or deadly suffering for them?” 

“Yes,\marginnote{17.4} sir.” … 

“Suppose\marginnote{18.1} there was a pit of glowing coals deeper than a man’s height, full of glowing coals that neither flamed nor smoked. Then a person would come along who wants to live and doesn’t want to die, who wants to be happy and recoils from pain. Then two strong men would grab them by the arms and drag them towards the pit of glowing coals. What do you think, householder? Wouldn’t that person writhe and struggle to and fro?” 

“Yes,\marginnote{18.6} sir. Why is that? For that person knows: ‘If I fall in that pit of glowing coals, that’d result in my death or deadly pain.’” … 

“Suppose\marginnote{19.1} a person was to see delightful parks, woods, meadows, and lotus ponds in a dream. But when they woke they couldn’t see them at all. … 

Suppose\marginnote{20.1} a man had borrowed some goods—a gentleman’s carriage and fine jewelled earrings—and preceded and surrounded by these he proceeded through the middle of \textsanskrit{Āpaṇa}. When people saw him they’d say: ‘This must be a wealthy man! For that’s how the wealthy enjoy their wealth.’ But when the owners saw him, they’d take back what was theirs. What do you think? Would that be enough for that man to get upset?” 

“Yes,\marginnote{20.7} sir. Why is that? Because the owners took back what was theirs.” … 

“Suppose\marginnote{21.1} there was a dark forest grove not far from a town or village. And there was a tree laden with fruit, yet none of the fruit had fallen to the ground. And along came a person in need of fruit, wandering in search of fruit. Having plunged deep into that forest grove, they’d see that tree laden with fruit. They’d think: ‘That tree is laden with fruit, yet none of the fruit has fallen to the ground. But I know how to climb a tree. Why don’t I climb the tree, eat as much as I like, then fill my pouch?’ And that’s what they’d do. And along would come a second person in need of fruit, wandering in search of fruit, carrying a sharp axe. Having plunged deep into that forest grove, they’d see that tree laden with fruit. They’d think: ‘That tree is laden with fruit, yet none of the fruit has fallen to the ground. But I don’t know how to climb a tree. Why don’t I chop this tree down at the root, eat as much as I like, then fill my pouch?’ And so they’d chop the tree down at the root. What do you think, householder? If the first person, who climbed the tree, doesn’t quickly come down, when that tree fell wouldn’t they break their hand or arm or other limb, resulting in death or deadly suffering for them?” 

“Yes,\marginnote{21.19} sir.” 

“In\marginnote{21.20} the same way, a noble disciple reflects: ‘With the simile of the fruit tree the Buddha said that sensual pleasures give little gratification and much suffering and distress, and they are all the more full of drawbacks.’ Having truly seen this with right understanding, they reject equanimity based on diversity and develop only the equanimity based on unity, where all kinds of grasping to the world’s material delights cease without anything left over. 

Relying\marginnote{22.1} on this supreme purity of mindfulness and equanimity, that noble disciple recollects their many kinds of past lives. That is: one, two, three, four, five, ten, twenty, thirty, forty, fifty, a hundred, a thousand, a hundred thousand rebirths; many eons of the world contracting, many eons of the world expanding, many eons of the world contracting and expanding. … They recollect their many kinds of past lives, with features and details. 

Relying\marginnote{23.1} on this supreme purity of mindfulness and equanimity, that noble disciple, with clairvoyance that is purified and superhuman, sees sentient beings passing away and being reborn—inferior and superior, beautiful and ugly, in a good place or a bad place. … They understand how sentient beings are reborn according to their deeds. 

Relying\marginnote{24.1} on this supreme purity of mindfulness and equanimity, that noble disciple realizes the undefiled freedom of heart and freedom by wisdom in this very life. And they live having realized it with their own insight due to the ending of defilements. 

That’s\marginnote{25.1} how there is the cutting off of judgments in each and every respect in the noble one’s training. 

What\marginnote{25.2} do you think, householder? Do you regard yourself as having cut off judgments in a way comparable to the cutting off of judgments in each and every respect in the noble one’s training?” 

“Who\marginnote{25.4} am I compared to one who has cut off judgments in each and every respect in the noble one’s training? I am far from that. Sir, I used to think that the wanderers following other paths were thoroughbreds, and I fed them and treated them accordingly, but they were not actually thoroughbreds. I thought that the mendicants were not thoroughbreds, and I fed them and treated them accordingly, but they actually were thoroughbreds. But now I shall understand that the wanderers following other paths are not actually thoroughbreds, and I will feed them and treat them accordingly. And I shall understand that the mendicants actually are thoroughbreds, and I will feed them and treat them accordingly. The Buddha has inspired me to have love, confidence, and respect for ascetics! 

Excellent,\marginnote{26.1} sir! Excellent! As if he were righting the overturned, or revealing the hidden, or pointing out the path to the lost, or lighting a lamp in the dark so people with good eyes can see what’s there, the Buddha has made the teaching clear in many ways. I go for refuge to the Buddha, to the teaching, and to the mendicant \textsanskrit{Saṅgha}. From this day forth, may the Buddha remember me as a lay follower who has gone for refuge for life.” 

%
\section*{{\suttatitleacronym MN 55}{\suttatitletranslation With Jīvaka }{\suttatitleroot Jīvakasutta}}
\addcontentsline{toc}{section}{\tocacronym{MN 55} \toctranslation{With Jīvaka } \tocroot{Jīvakasutta}}
\markboth{With Jīvaka }{Jīvakasutta}
\extramarks{MN 55}{MN 55}

\scevam{So\marginnote{1.1} I have heard. }At one time the Buddha was staying near \textsanskrit{Rājagaha} in the Mango Grove of \textsanskrit{Jīvaka} \textsanskrit{Komārabhacca}. 

Then\marginnote{2.1} \textsanskrit{Jīvaka} went up to the Buddha, bowed, sat down to one side, and said to the Buddha: 

“Sir,\marginnote{3.1} I have heard this: ‘They slaughter living creatures specially for the ascetic Gotama. The ascetic Gotama knowingly eats meat prepared on purpose for him: this is a deed he caused.’ I trust that those who say this repeat what the Buddha has said, and do not misrepresent him with an untruth? Is their explanation in line with the teaching? Are there any legitimate grounds for rebuke and criticism?” 

“\textsanskrit{Jīvaka},\marginnote{4.1} those who say this do not repeat what I have said. They misrepresent me with what is false and untrue. 

In\marginnote{5.1} three cases I say that meat may not be eaten: it’s seen, heard, or suspected. These are three cases in which meat may not be eaten. 

In\marginnote{5.4} three cases I say that meat may be eaten: it’s not seen, heard, or suspected. These are three cases in which meat may be eaten. 

Take\marginnote{6.1} the case of a mendicant living supported by a town or village. They meditate spreading a heart full of love to one direction, and to the second, and to the third, and to the fourth. In the same way above, below, across, everywhere, all around, they spread a heart full of love to the whole world—abundant, expansive, limitless, free of enmity and ill will. A householder or their child approaches and invites them for the next day’s meal. The mendicant accepts if they want. 

When\marginnote{6.5} the night has passed, they robe up in the morning, take their bowl and robe, and approach that householder’s home, where they sit on the seat spread out. That householder or their child serves them with delicious almsfood. It never occurs to them, ‘It’s so good that this householder serves me with delicious almsfood! I hope they serve me with such delicious almsfood in the future!’ They don’t think that. They eat that almsfood untied, uninfatuated, unattached, seeing the drawback, and understanding the escape. 

What\marginnote{6.12} do you think, \textsanskrit{Jīvaka}? At that time is that mendicant intending to hurt themselves, hurt others, or hurt both?” 

“No,\marginnote{6.14} sir.” 

“Aren’t\marginnote{6.15} they eating blameless food at that time?” 

“Yes,\marginnote{7.1} sir. Sir, I have heard that \textsanskrit{Brahmā} abides in love. Now, I’ve seen the Buddha with my own eyes, and it is the Buddha who truly abides in love.” 

“Any\marginnote{7.6} greed, hate, or delusion that might give rise to ill will has been given up by the Realized One, cut off at the root, made like a palm stump, obliterated, and is unable to arise in the future. If that’s what you were referring to, I acknowledge it.” 

“That’s\marginnote{7.8} exactly what I was referring to.” 

“Take\marginnote{8{-}10.1} the case, \textsanskrit{Jīvaka}, of a mendicant living supported by a town or village. They meditate spreading a heart full of compassion … 

They\marginnote{8{-}10.3} meditate spreading a heart full of rejoicing … 

They\marginnote{8{-}10.4} meditate spreading a heart full of equanimity to one direction, and to the second, and to the third, and to the fourth. In the same way above, below, across, everywhere, all around, they spread a heart full of equanimity to the whole world—abundant, expansive, limitless, free of enmity and ill will. A householder or their child approaches and invites them for the next day’s meal. The mendicant accepts if they want. 

When\marginnote{8{-}10.8} the night has passed, they robe up in the morning, take their bowl and robe, and approach that householder’s home, where they sit on the seat spread out. That householder or their child serves them with delicious almsfood. It never occurs to them, ‘It’s so good that this householder serves me with delicious almsfood! I hope they serve me with such delicious almsfood in the future!’ They don’t think that. They eat that almsfood untied, uninfatuated, unattached, seeing the drawback, and understanding the escape. 

What\marginnote{8{-}10.15} do you think, \textsanskrit{Jīvaka}? At that time is that mendicant intending to hurt themselves, hurt others, or hurt both?” 

“No,\marginnote{8{-}10.17} sir.” 

“Aren’t\marginnote{8{-}10.18} they eating blameless food at that time?” 

“Yes,\marginnote{11.1} sir. Sir, I have heard that \textsanskrit{Brahmā} abides in equanimity. Now, I’ve seen the Buddha with my own eyes, and it is the Buddha who truly abides in equanimity.” 

“Any\marginnote{11.6} greed, hate, or delusion that might give rise to cruelty, discontent, or repulsion has been given up by the Realized One, cut off at the root, made like a palm stump, obliterated, and is unable to arise in the future. If that’s what you were referring to, I acknowledge it.” 

“That’s\marginnote{11.8} exactly what I was referring to.” 

“\textsanskrit{Jīvaka},\marginnote{12.1} anyone who slaughters a living creature specially for the Realized One or the Realized One’s disciple makes much bad karma for five reasons. 

When\marginnote{12.2} they say: ‘Go, fetch that living creature,’ this is the first reason. 

When\marginnote{12.4} that living creature experiences pain and sadness as it’s led along by a collar, this is the second reason. 

When\marginnote{12.5} they say: ‘Go, slaughter that living creature,’ this is the third reason. 

When\marginnote{12.7} that living creature experiences pain and sadness as it’s being slaughtered, this is the fourth reason. 

When\marginnote{12.8} they provide the Realized One or the Realized One’s disciple with unallowable food, this is the fifth reason. 

Anyone\marginnote{12.9} who slaughters a living creature specially for the Realized One or the Realized One’s disciple makes much bad karma for five reasons.” 

When\marginnote{13.1} he had spoken, \textsanskrit{Jīvaka} said to the Buddha: “It’s incredible, sir, it’s amazing! The mendicants indeed eat allowable food. The mendicants indeed eat blameless food. Excellent, sir! Excellent! … From this day forth, may the Buddha remember me as a lay follower who has gone for refuge for life.” 

%
\section*{{\suttatitleacronym MN 56}{\suttatitletranslation With Upāli }{\suttatitleroot Upālisutta}}
\addcontentsline{toc}{section}{\tocacronym{MN 56} \toctranslation{With Upāli } \tocroot{Upālisutta}}
\markboth{With Upāli }{Upālisutta}
\extramarks{MN 56}{MN 56}

\scevam{So\marginnote{1.1} I have heard. }At one time the Buddha was staying near \textsanskrit{Nālandā} in \textsanskrit{Pāvārika}’s mango grove. 

At\marginnote{2.1} that time \textsanskrit{Nigaṇṭha} \textsanskrit{Nāṭaputta} was residing at \textsanskrit{Nāḷandā} together with a large assembly of Jain ascetics. Then the Jain ascetic \textsanskrit{Dīgha} \textsanskrit{Tapassī} wandered for alms in \textsanskrit{Nāḷandā}. After the meal, on his return from almsround, he went to \textsanskrit{Pāvārika}’s mango grove. There he approached the Buddha, and exchanged greetings with him. 

When\marginnote{2.3} the greetings and polite conversation were over, he stood to one side. The Buddha said to him, “There are seats, \textsanskrit{Tapassī}. Please sit if you wish.” 

When\marginnote{3.1} he said this, \textsanskrit{Dīgha} \textsanskrit{Tapassī} took a low seat and sat to one side. The Buddha said to him, “\textsanskrit{Tapassī}, how many kinds of deed does \textsanskrit{Nigaṇṭha} \textsanskrit{Nātaputta} describe for performing bad deeds?” 

“Reverend\marginnote{3.4} Gotama, \textsanskrit{Nigaṇṭha} \textsanskrit{Nātaputta} doesn’t usually speak in terms of ‘deeds’. He usually speaks in terms of ‘rods’.” 

“Then\marginnote{3.6} how many kinds of rod does \textsanskrit{Nigaṇṭha} \textsanskrit{Nātaputta} describe for performing bad deeds?” 

“\textsanskrit{Nigaṇṭha}\marginnote{3.7} \textsanskrit{Nātaputta} describes three kinds of rod for performing bad deeds: the physical rod, the verbal rod, and the mental rod.” 

“But\marginnote{3.9} are these kinds of rod all distinct from each other?” 

“Yes,\marginnote{3.10} each is quite distinct.” 

“Of\marginnote{3.11} the three rods thus analyzed and differentiated, which rod does \textsanskrit{Nigaṇṭha} \textsanskrit{Nātaputta} describe as being the most blameworthy for performing bad deeds: the physical rod, the verbal rod, or the mental rod?” 

“\textsanskrit{Nigaṇṭha}\marginnote{3.12} \textsanskrit{Nātaputta} describes the physical rod as being the most blameworthy for performing bad deeds, not so much the verbal rod or the mental rod.” 

“Do\marginnote{3.13} you say the physical rod, \textsanskrit{Tapassī}?” 

“I\marginnote{3.14} say the physical rod, Reverend Gotama.” 

“Do\marginnote{3.15} you say the physical rod, \textsanskrit{Tapassī}?” 

“I\marginnote{3.16} say the physical rod, Reverend Gotama.” 

“Do\marginnote{3.17} you say the physical rod, \textsanskrit{Tapassī}?” 

“I\marginnote{3.18} say the physical rod, Reverend Gotama.” 

Thus\marginnote{3.19} the Buddha made \textsanskrit{Dīgha} \textsanskrit{Tapassī} stand by this point up to the third time. 

When\marginnote{4.1} this was said, \textsanskrit{Dīgha} \textsanskrit{Tapassī} said to the Buddha, “But Reverend Gotama, how many kinds of rod do you describe for performing bad deeds?” 

“\textsanskrit{Tapassī},\marginnote{4.3} the Realized One doesn’t usually speak in terms of ‘rods’. He usually speaks in terms of ‘deeds’.” 

“Then\marginnote{4.5} how many kinds of deed do you describe for performing bad deeds?” 

“I\marginnote{4.6} describe three kinds of deed for performing bad deeds: physical deeds, verbal deeds, and mental deeds.” 

“But\marginnote{4.8} are these kinds of deed all distinct from each other?” 

“Yes,\marginnote{4.9} each is quite distinct.” 

“Of\marginnote{4.10} the three deeds thus analyzed and differentiated, which deed do you describe as being the most blameworthy for performing bad deeds: physical deeds, verbal deeds, or mental deeds?” 

“I\marginnote{4.11} describe mental deeds as being the most blameworthy for performing bad deeds, not so much physical deeds or verbal deeds.” 

“Do\marginnote{4.12} you say mental deeds, Reverend Gotama?” 

“I\marginnote{4.13} say mental deeds, \textsanskrit{Tapassī}.” 

“Do\marginnote{4.14} you say mental deeds, Reverend Gotama?” 

“I\marginnote{4.15} say mental deeds, \textsanskrit{Tapassī}.” 

“Do\marginnote{4.16} you say mental deeds, Reverend Gotama?” 

“I\marginnote{4.17} say mental deeds, \textsanskrit{Tapassī}.” 

Thus\marginnote{4.18} the Jain ascetic \textsanskrit{Dīgha} \textsanskrit{Tapassī} made the Buddha stand by this point up to the third time, after which he got up from his seat and went to see \textsanskrit{Nigaṇṭha} \textsanskrit{Nātaputta}. 

Now\marginnote{5.1} at that time \textsanskrit{Nigaṇṭha} \textsanskrit{Nātaputta} was sitting together with a large assembly of laypeople of \textsanskrit{Bālaka} headed by \textsanskrit{Upāli}. \textsanskrit{Nigaṇṭha} \textsanskrit{Nātaputta} saw \textsanskrit{Dīgha} \textsanskrit{Tapassī} coming off in the distance and said to him, “So, \textsanskrit{Tapassī}, where are you coming from in the middle of the day?” 

“Just\marginnote{5.5} now, sir, I’ve come from the presence of the ascetic Gotama.” 

“But\marginnote{5.6} did you have some discussion with him?” 

“I\marginnote{5.7} did.” 

“And\marginnote{5.8} what kind of discussion did you have with him?” Then \textsanskrit{Dīgha} \textsanskrit{Tapassī} informed \textsanskrit{Nigaṇṭha} \textsanskrit{Nātaputta} of all they had discussed. When he had spoken, \textsanskrit{Nigaṇṭha} said to him, “Good, good, \textsanskrit{Tapassī}! \textsanskrit{Dīgha} \textsanskrit{Tapassī} has answered the ascetic Gotama like an educated disciple who rightly understands their teacher’s instructions. For how impressive is the measly mental rod when compared with the substantial physical rod? Rather, the physical rod is the most blameworthy for performing bad deeds, not so much the verbal rod or the mental rod.” 

When\marginnote{7.1} he said this, the householder \textsanskrit{Upāli} said to him, “Good, sir! Well done, \textsanskrit{Dīgha} \textsanskrit{Tapassī}! The honorable \textsanskrit{Tapassī} has answered the ascetic Gotama like an educated disciple who rightly understands their teacher’s instructions. For how impressive is the measly mental rod when compared with the substantial physical rod? Rather, the physical rod is the most blameworthy for performing bad deeds, not so much the verbal rod or the mental rod. 

I’d\marginnote{7.6} better go and refute the ascetic Gotama’s doctrine regarding this point. If he stands by the position that he stated to \textsanskrit{Dīgha} \textsanskrit{Tapassī}, I’ll take him on in debate and drag him to and fro and round about, like a strong man would drag a fleecy sheep to and fro and round about! Taking him on in debate, I’ll drag him to and fro and round about, like a strong brewer’s worker would toss a large brewer’s sieve into a deep lake, grab it by the corners, and drag it to and fro and round about! Taking him on in debate, I’ll shake him down and about and give him a beating, like a strong brewer’s mixer would grab a strainer by the corners and shake it down and about, and give it a beating! I’ll play a game of ear-washing with the ascetic Gotama, like a sixty-year-old elephant would plunge into a deep lotus pond and play a game of ear-washing! Sir, I’d better go and refute the ascetic Gotama’s doctrine on this point.” 

“Go,\marginnote{7.12} householder, refute the ascetic Gotama’s doctrine on this point. For either I should do so, or \textsanskrit{Dīgha} \textsanskrit{Tapassī}, or you.” 

When\marginnote{8.1} he said this, \textsanskrit{Dīgha} \textsanskrit{Tapassī} said to \textsanskrit{Nigaṇṭha} \textsanskrit{Nātaputta}, “Sir, I don’t believe it’s a good idea for the householder \textsanskrit{Upāli} to rebut the ascetic Gotama’s doctrine. For the ascetic Gotama is a magician. He knows a conversion magic, and uses it to convert the disciples of those who follow other paths.” 

“It\marginnote{8.4} is impossible, \textsanskrit{Tapassī}, it cannot happen that \textsanskrit{Upāli} could become Gotama’s disciple. But it is possible that Gotama could become \textsanskrit{Upāli}’s disciple. Go, householder, refute the ascetic Gotama’s doctrine on this point. For either I should do so, or \textsanskrit{Dīgha} \textsanskrit{Tapassī}, or you.” 

For\marginnote{9.1} a second time … and a third time, \textsanskrit{Dīgha} \textsanskrit{Tapassī} said to \textsanskrit{Nigaṇṭha} \textsanskrit{Nātaputta}, “Sir, I don’t believe it’s a good idea for the householder \textsanskrit{Upāli} to rebut the ascetic Gotama’s doctrine. For the ascetic Gotama is a magician. He knows a conversion magic, and uses it to convert the disciples of those who follow other paths.” 

“It\marginnote{9.5} is impossible, \textsanskrit{Tapassī}, it cannot happen that \textsanskrit{Upāli} could become Gotama’s disciple. But it is possible that Gotama could become \textsanskrit{Upāli}’s disciple. Go, householder, refute the ascetic Gotama’s doctrine on this point. For either I should do so, or \textsanskrit{Dīgha} \textsanskrit{Tapassī}, or you.” 

“Yes,\marginnote{9.9} sir,” replied the householder \textsanskrit{Upāli} to \textsanskrit{Nigaṇṭha} \textsanskrit{Nāṭaputta}. He got up from his seat, bowed, and respectfully circled him, keeping him on his right. Then he went to the Buddha, bowed, sat down to one side, and said to him, “Sir, did the Jain ascetic \textsanskrit{Dīgha} \textsanskrit{Tapassī} come here?” 

“He\marginnote{9.11} did, householder.” 

“But\marginnote{9.12} did you have some discussion with him?” 

“I\marginnote{9.13} did.” 

“And\marginnote{9.14} what kind of discussion did you have with him?” 

Then\marginnote{9.15} the Buddha informed \textsanskrit{Upāli} of all they had discussed. 

When\marginnote{10.1} he said this, the householder \textsanskrit{Upāli} said to him, “Good, sir, well done by \textsanskrit{Tapassī}! The honorable \textsanskrit{Tapassī} has answered the ascetic Gotama like an educated disciple who rightly understands their teacher’s instructions. For how impressive is the measly mental rod when compared with the substantial physical rod? Rather, the physical rod is the most blameworthy for performing bad deeds, not so much the verbal rod or the mental rod.” 

“Householder,\marginnote{10.6} so long as you debate on the basis of truth, we can have some discussion about this.” 

“I\marginnote{10.7} will debate on the basis of truth, sir. Let us have some discussion about this.” 

“What\marginnote{11.1} do you think, householder? Take a Jain ascetic who is sick, suffering, gravely ill. They reject cold water and use only hot water. Not getting cold water, they might die. Now, where does \textsanskrit{Nigaṇṭha} \textsanskrit{Nātaputta} say they would be reborn?” 

“Sir,\marginnote{11.5} there are gods called ‘mind-bound’. They would be reborn there. Why is that? Because they died with mental attachment.” 

“Think\marginnote{11.8} about it, householder! You should think before answering. What you said before and what you said after don’t match up. But you said that you would debate on the basis of truth.” 

“Even\marginnote{11.13} though the Buddha says this, still the physical rod is the most blameworthy for performing bad deeds, not so much the verbal rod or the mental rod.” 

“What\marginnote{12.1} do you think, householder? Take a Jain ascetic who is restrained in the fourfold restraint: obstructed by all water, devoted to all water, shaking off all water, pervaded by all water. When going out and coming back they accidentally injure many little creatures. Now, what result does \textsanskrit{Nigaṇṭha} \textsanskrit{Nātaputta} say they would incur?” 

“Sir,\marginnote{12.5} \textsanskrit{Nigaṇṭha} \textsanskrit{Nātaputta} says that unintentional acts are not very blameworthy.” 

“But\marginnote{12.6} if they are intentional?” 

“Then\marginnote{12.7} they are very blameworthy.” 

“But\marginnote{12.8} where does \textsanskrit{Nigaṇṭha} \textsanskrit{Nātaputta} say that intention is classified?” 

“In\marginnote{12.9} the mental rod, sir.” 

“Think\marginnote{12.10} about it, householder! You should think before answering. What you said before and what you said after don’t match up. But you said that you would debate on the basis of truth.” 

“Even\marginnote{12.15} though the Buddha says this, still the physical rod is the most blameworthy for performing bad deeds, not so much the verbal rod or the mental rod.” 

“What\marginnote{13.1} do you think, householder? Is this \textsanskrit{Nāḷandā} successful and prosperous and full of people?” 

“Indeed\marginnote{13.3} it is, sir.” 

“What\marginnote{13.4} do you think, householder? Suppose a man were to come along with a drawn sword and say: ‘In one moment I will reduce all the living creatures within the bounds of \textsanskrit{Nāḷandā} to one heap and mass of flesh!’ What do you think, householder? Could he do that?” 

“Sir,\marginnote{13.10} even ten, twenty, thirty, forty, or fifty men couldn’t do that. How impressive is one measly man?” 

“What\marginnote{13.12} do you think, householder? Suppose an ascetic or brahmin with psychic power, who has achieved mastery of the mind, were to come along and say: ‘I will reduce \textsanskrit{Nāḷandā} to ashes with a single malevolent act of will!’ What do you think, householder? Could he do that?” 

“Sir,\marginnote{13.18} an ascetic or brahmin with psychic power, who has achieved mastery of the mind, could reduce ten, twenty, thirty, forty, or fifty \textsanskrit{Nāḷandās} to ashes with a single malevolent act of will. How impressive is one measly \textsanskrit{Nāḷandā}?” 

“Think\marginnote{13.20} about it, householder! You should think before answering. What you said before and what you said after don’t match up. But you said that you would debate on the basis of truth.” 

“Even\marginnote{13.25} though the Buddha says this, still the physical rod is the most blameworthy for performing bad deeds, not so much the verbal rod or the mental rod.” 

“What\marginnote{13.26} do you think, householder? Have you heard how the wildernesses of \textsanskrit{Daṇḍaka}, \textsanskrit{Kāliṅga}, Mejjha, and \textsanskrit{Mātaṅga} came to be that way?” 

“I\marginnote{13.28} have, sir.” 

“What\marginnote{14.1} have you heard?” 

“I\marginnote{14.2} heard that it was because of a malevolent act of will by hermits that the wildernesses of \textsanskrit{Daṇḍaka}, \textsanskrit{Kāliṅga}, Mejjha, and \textsanskrit{Mātaṅga} came to be that way.” 

“Think\marginnote{14.3} about it, householder! You should think before answering. What you said before and what you said after don’t match up. But you said that you would debate on the basis of truth.” 

“Sir,\marginnote{15.1} I was already delighted and satisfied by the Buddha’s very first simile. Nevertheless, I wanted to hear the Buddha’s various solutions to the problem, so I thought I’d oppose you in this way. 

Excellent,\marginnote{15.3} sir! Excellent! As if he were righting the overturned, or revealing the hidden, or pointing out the path to the lost, or lighting a lamp in the dark so people with good eyes can see what’s there, the Buddha has made the teaching clear in many ways. I go for refuge to the Buddha, to the teaching, and to the mendicant \textsanskrit{Saṅgha}. From this day forth, may the Buddha remember me as a lay follower who has gone for refuge for life.” 

“Householder,\marginnote{16.1} you should act after careful consideration. It’s good for well-known people such as yourself to act after careful consideration.” 

“Now\marginnote{16.2} I’m even more delighted and satisfied with the Buddha, since he tells me to act after careful consideration. For if the followers of other paths were to gain me as a disciple, they’d carry a banner all over \textsanskrit{Nāḷandā}, saying: ‘The householder \textsanskrit{Upāli} has become our disciple!’ And yet the Buddha says: ‘Householder, you should act after careful consideration. It’s good for well-known people such as yourself to act after careful consideration.’ 

For\marginnote{16.7} a second time, I go for refuge to the Buddha, to the teaching, and to the mendicant \textsanskrit{Saṅgha}. From this day forth, may the Buddha remember me as a lay follower who has gone for refuge for life.” 

“For\marginnote{17.1} a long time now, householder, your family has been a well-spring of support for the Jain ascetics. You should consider giving to them when they come.” 

“Now\marginnote{17.2} I’m even more delighted and satisfied with the Buddha, since he tells me to consider giving to the Jain ascetics when they come. I have heard, sir, that the ascetic Gotama says this: ‘Gifts should only be given to me, not to others. Gifts should only be given to my disciples, not to the disciples of others. Only what is given to me is very fruitful, not what is given to others. Only what is given to my disciples is very fruitful, not what is given to the disciples of others.’ Yet the Buddha encourages me to give to the Jain ascetics. Well, sir, we’ll know the proper time for that. 

For\marginnote{17.10} a third time, I go for refuge to the Buddha, to the teaching, and to the mendicant \textsanskrit{Saṅgha}. From this day forth, may the Buddha remember me as a lay follower who has gone for refuge for life.” 

Then\marginnote{18.1} the Buddha taught the householder \textsanskrit{Upāli} step by step, with a talk on giving, ethical conduct, and heaven. He explained the drawbacks of sensual pleasures, so sordid and corrupt, and the benefit of renunciation. And when he knew that \textsanskrit{Upāli}’s mind was ready, pliable, rid of hindrances, elated, and confident he explained the special teaching of the Buddhas: suffering, its origin, its cessation, and the path. Just as a clean cloth rid of stains would properly absorb dye, in that very seat the stainless, immaculate vision of the Dhamma arose in \textsanskrit{Upāli}: “Everything that has a beginning has an end.” Then \textsanskrit{Upāli} saw, attained, understood, and fathomed the Dhamma. He went beyond doubt, got rid of indecision, and became self-assured and independent of others regarding the Teacher’s instructions. 

He\marginnote{18.9} said to the Buddha, “Well, now, sir, I must go. I have many duties, and much to do.” 

“Please,\marginnote{18.10} householder, go at your convenience.” 

And\marginnote{19.1} then the householder \textsanskrit{Upāli} approved and agreed with what the Buddha said. He got up from his seat, bowed, and respectfully circled the Buddha, keeping him on his right. Then he went back to his own home, where he addressed the gatekeeper, “My good gatekeeper, from this day forth close the gate to Jain monks and nuns, and open it for the Buddha’s monks, nuns, laymen, and laywomen. If any Jain ascetics come, say this to them: ‘Wait, sir, do not enter. From now on the householder \textsanskrit{Upāli} has become a disciple of the ascetic Gotama. His gate is closed to Jain monks and nuns, and opened for the Buddha’s monks, nuns, laymen, and laywomen. If you require almsfood, wait here, they will bring it to you.’” 

“Yes,\marginnote{19.8} sir,” replied the gatekeeper. 

\textsanskrit{Dīgha}\marginnote{20.1} \textsanskrit{Tapassī} heard that \textsanskrit{Upāli} had become a disciple of the ascetic Gotama. He went to \textsanskrit{Nigaṇṭha} \textsanskrit{Nātaputta} and said to him, “Sir, they say that the householder \textsanskrit{Upāli} has become a disciple of the ascetic Gotama.” 

“It\marginnote{20.5} is impossible, \textsanskrit{Tapassī}, it cannot happen that \textsanskrit{Upāli} could become Gotama’s disciple. But it is possible that Gotama could become \textsanskrit{Upāli}’s disciple.” 

For\marginnote{20.7} a second time … and a third time, \textsanskrit{Dīgha} \textsanskrit{Tapassī} said to \textsanskrit{Nigaṇṭha} \textsanskrit{Nātaputta}, “Sir, they say that the householder \textsanskrit{Upāli} has become a disciple of the ascetic Gotama.” 

“It\marginnote{20.10} is impossible, \textsanskrit{Tapassī}, it cannot happen that \textsanskrit{Upāli} could become Gotama’s disciple. But it is possible that Gotama could become \textsanskrit{Upāli}’s disciple.” 

“Well,\marginnote{20.12} sir, I’d better go and find out whether or not \textsanskrit{Upāli} has become Gotama’s disciple.” 

“Go,\marginnote{20.13} \textsanskrit{Tapassī}, and find out whether or not \textsanskrit{Upāli} has become Gotama’s disciple.” 

Then\marginnote{21.1} \textsanskrit{Dīgha} \textsanskrit{Tapassī} went to \textsanskrit{Upāli}’s home. The gatekeeper saw him coming off in the distance and said to him, “Wait, sir, do not enter. From now on the householder \textsanskrit{Upāli} has become a disciple of the ascetic Gotama. His gate is closed to Jain monks and nuns, and opened for the Buddha’s monks, nuns, laymen, and laywomen. If you require almsfood, wait here, they will bring it to you.” 

Saying,\marginnote{21.8} “No, mister, I do not require almsfood,” he turned back and went to \textsanskrit{Nigaṇṭha} \textsanskrit{Nātaputta} and said to him, “Sir, it’s really true that \textsanskrit{Upāli} has become Gotama’s disciple. Sir, I couldn’t get you to accept that it wasn’t a good idea for the householder \textsanskrit{Upāli} to rebut the ascetic Gotama’s doctrine. For the ascetic Gotama is a magician. He knows a conversion magic, and uses it to convert the disciples of those who follow other paths. The householder \textsanskrit{Upāli} has been converted by the ascetic Gotama’s conversion magic!” 

“It\marginnote{21.13} is impossible, \textsanskrit{Tapassī}, it cannot happen that \textsanskrit{Upāli} could become Gotama’s disciple. But it is possible that Gotama could become \textsanskrit{Upāli}’s disciple.” 

For\marginnote{21.15} a second time … and a third time, \textsanskrit{Dīgha} \textsanskrit{Tapassī} told \textsanskrit{Nigaṇṭha} \textsanskrit{Nātaputta} that it was really true. 

“It\marginnote{21.20} is impossible … 

Well,\marginnote{21.22} \textsanskrit{Tapassī}, I’d better go and find out for myself whether or not \textsanskrit{Upāli} has become Gotama’s disciple.” 

Then\marginnote{22.1} \textsanskrit{Nigaṇṭha} \textsanskrit{Nātaputta} went to \textsanskrit{Upāli}’s home together with a large following of Jain ascetics. The gatekeeper saw him coming off in the distance and said to him: ‘Wait, sir, do not enter. From now on the householder \textsanskrit{Upāli} has become a disciple of the ascetic Gotama. His gate is closed to Jain monks and nuns, and opened for the Buddha’s monks, nuns, laymen, and laywomen. If you require almsfood, wait here, they will bring it to you.” 

“Well\marginnote{22.8} then, my good gatekeeper, go to \textsanskrit{Upāli} and say: ‘Sir, \textsanskrit{Nigaṇṭha} \textsanskrit{Nātaputta} is waiting outside the gates together with a large following of Jain ascetics. He wishes to see you.’” 

“Yes,\marginnote{22.11} sir,” replied the gatekeeper. He went to \textsanskrit{Upāli} and relayed what was said. \textsanskrit{Upāli} said to him, “Well, then, my good gatekeeper, prepare seats in the hall of the middle gate.” 

“Yes,\marginnote{22.15} sir,” replied the gatekeeper. He did as he was asked, then returned to \textsanskrit{Upāli} and said, “Sir, seats have been prepared in the hall of the middle gate. Please go at your convenience.” 

Then\marginnote{23.1} \textsanskrit{Upāli} went to the hall of the middle gate, where he sat on the highest and finest seat. He addressed the gatekeeper, “Well then, my good gatekeeper, go to \textsanskrit{Nigaṇṭha} \textsanskrit{Nātaputta} and say to him: ‘Sir, \textsanskrit{Upāli} says you may enter if you wish.’” 

“Yes,\marginnote{23.5} sir,” replied the gatekeeper. He went to \textsanskrit{Nigaṇṭha} \textsanskrit{Nātaputta} and relayed what was said. 

Then\marginnote{23.8} \textsanskrit{Nigaṇṭha} \textsanskrit{Nātaputta} went to the hall of the middle gate together with a large following of Jain ascetics. Previously, when \textsanskrit{Upāli} saw \textsanskrit{Nigaṇṭha} \textsanskrit{Nātaputta} coming, he would go out to greet him and, having wiped off the highest and finest seat with his upper robe, he would put his arms around him and sit him down. But today, having seated himself on the highest and finest seat, he said to \textsanskrit{Nigaṇṭha} \textsanskrit{Nātaputta}, “There are seats, sir. Please sit if you wish.” 

When\marginnote{25.1} he said this, \textsanskrit{Nigaṇṭha} \textsanskrit{Nātaputta} said to him: “You’re mad, householder! You’re a moron! You said: ‘I’ll go and refute the ascetic Gotama’s doctrine.’ But you come back caught in the vast net of his doctrine. Suppose a man went to deliver a pair of balls, but came back castrated. Or they went to deliver eyes, but came back blinded. In the same way, you said: ‘I’ll go and refute the ascetic Gotama’s doctrine.’ But you come back caught in the vast net of his doctrine. You’ve been converted by the ascetic Gotama’s conversion magic!” 

“Sir,\marginnote{26.1} this conversion magic is excellent. This conversion magic is lovely! If my loved ones—relatives and kin—were to be converted by this, it would be for their lasting welfare and happiness. If all the aristocrats, brahmins, merchants, and workers were to be converted by this, it would be for their lasting welfare and happiness. If the whole world—with its gods, \textsanskrit{Māras} and \textsanskrit{Brahmās}, this population with its ascetics and brahmins, gods and humans—were to be converted by this, it would be for their lasting welfare and happiness. Well then, sir, I shall give you a simile. For by means of a simile some sensible people understand the meaning of what is said. 

Once\marginnote{27.1} upon a time there was an old brahmin, elderly and senior. His wife was a young brahmin lady who was pregnant and about to give birth. Then she said to the brahmin, ‘Go, brahmin, buy a baby monkey from the market and bring it back so it can be a playmate for my child.’ 

When\marginnote{27.4} she said this, the brahmin said to her, ‘Wait, my dear, until you give birth. If your child is a boy, I’ll buy you a male monkey, but if it’s a girl, I’ll buy a female monkey.’ 

For\marginnote{27.8} a second time, and a third time she said to the brahmin, ‘Go, brahmin, buy a baby monkey from the market and bring it back so it can be a playmate for my child.’ 

Then\marginnote{27.11} that brahmin, because of his love for the brahmin lady, bought a male baby monkey at the market, brought it to her, and said, ‘I’ve bought this male baby monkey for you so it can be a playmate for your child.’ 

When\marginnote{27.13} he said this, she said to him, ‘Go, brahmin, take this monkey to \textsanskrit{Rattapāṇi} the dyer and say, “Mister \textsanskrit{Rattapāṇi}, I wish to have this monkey dyed the color of yellow greasepaint, pounded and re-pounded, and pressed on both sides.”’ 

Then\marginnote{27.16} that brahmin, because of his love for the brahmin lady, took the monkey to \textsanskrit{Rattapāṇi} the dyer and said, ‘Mister \textsanskrit{Rattapāṇi}, I wish to have this monkey dyed the color of yellow greasepaint, pounded and re-pounded, and pressed on both sides.’ 

When\marginnote{27.18} he said this, \textsanskrit{Rattapāṇi} said to him, ‘Sir, this monkey can withstand a dying, but not a pounding or a pressing.’ 

In\marginnote{27.20} the same way, the doctrine of the foolish Jains looks fine initially—for fools, not for the astute—but can’t withstand being scrutinized or pressed. 

Then\marginnote{27.21} some time later that brahmin took a new pair of garments to \textsanskrit{Rattapāṇi} the dyer and said, ‘Mister \textsanskrit{Rattapāṇi}, I wish to have this new pair of garments dyed the color of yellow greasepaint, pounded and re-pounded, and pressed on both sides.’ 

When\marginnote{27.23} he said this, \textsanskrit{Rattapāṇi} said to him, ‘Sir, this pair of garments can withstand a dying, a pounding, and a pressing.’ 

In\marginnote{27.25} the same way, the doctrine of the Buddha looks fine initially—for the astute, not for fools—and it can withstand being scrutinized and pressed.” 

“Householder,\marginnote{28.1} the king and his retinue know you as a disciple of \textsanskrit{Nigaṇṭha} \textsanskrit{Nātaputta}. Whose disciple should we remember you as?” 

When\marginnote{29.1} he had spoken, the householder \textsanskrit{Upāli} got up from his seat, arranged his robe over one shoulder, raised his joined palms in the direction of the Buddha, and said to \textsanskrit{Nigaṇṭha} \textsanskrit{Nātaputta}, “Well then, sir, hear whose disciple I am: 

\begin{verse}%
The\marginnote{29.3} wise one, free of delusion, \\
rid of barrenness, victor in battle; \\
he’s untroubled and so even-minded, \\
with the virtue of an elder and the wisdom of a saint, \\
immaculate in the midst of it all: \\
he is the Buddha, and I am his disciple. 

He\marginnote{29.9} has no indecision, he’s content, \\
joyful, he has spat out the world’s bait; \\
he has completed the ascetic’s task as a human, \\
a man who bears his final body; \\
he’s beyond compare, he’s stainless: \\
he is the Buddha, and I am his disciple. 

He’s\marginnote{29.15} free of doubt, he’s skillful, \\
he’s a trainer, an excellent charioteer; \\
supreme, with brilliant qualities, \\
confident, his light shines forth; \\
he has cut off conceit, he’s a hero: \\
he is the Buddha, and I am his disciple. 

The\marginnote{29.21} chief bull, immeasurable, \\
profound, sagacious; \\
he is the builder of sanctuary, knowledgeable, \\
firm in principle and restrained; \\
he has got over clinging and is liberated: \\
he is the Buddha, and I am his disciple. 

He’s\marginnote{29.27} a giant, living remotely, \\
he’s ended the fetters and is liberated; \\
he’s skilled in dialogue and cleansed, \\
with banner put down, desireless; \\
he’s tamed, and doesn’t proliferate: \\
he is the Buddha, and I am his disciple. 

He\marginnote{29.33} is the seventh sage, free of deceit, \\
with three knowledges, he has attained to holiness, \\
he has bathed, he knows philology, \\
he’s tranquil, he understands what is known; \\
he crushes resistance, he is the lord: \\
he is the Buddha, and I am his disciple. 

The\marginnote{29.39} noble one, evolved, \\
he has attained the goal and explains it; \\
he is mindful, discerning, \\
neither leaning forward nor pulling back, \\
he’s unstirred, attained to mastery: \\
he is the Buddha, and I am his disciple. 

He\marginnote{29.45} has risen up, he practices absorption, \\
not following inner thoughts, he is pure, \\
independent, and fearless; \\
secluded, he has reached the peak, \\
crossed over, he helps others across: \\
he is the Buddha, and I am his disciple. 

He’s\marginnote{29.51} peaceful, his wisdom is vast, \\
with great wisdom, he’s free of greed; \\
he is the Realized One, the Holy One, \\
unrivaled, unequaled, \\
assured, and subtle: \\
he is the Buddha, and I am his disciple. 

He\marginnote{29.57} has cut off craving and is awakened, \\
free of fuming, unsullied; \\
a mighty spirit worthy of offerings, \\
best of men, inestimable, \\
grand, he has reached the peak of glory: \\
he is the Buddha, and I am his disciple.” 

%
\end{verse}

“But\marginnote{30.1} when did you compose these praises of the ascetic Gotama’s beautiful qualities, householder?” 

“Sir,\marginnote{30.2} suppose there was a large heap of many different flowers. A deft garland-maker or their apprentice could tie them into a colorful garland. In the same way, the Buddha has many beautiful qualities to praise, many hundreds of such qualities. Who, sir, would not praise the praiseworthy?” 

Unable\marginnote{31.1} to bear this honor paid to the Buddha, \textsanskrit{Nigaṇṭha} \textsanskrit{Nātaputta} spewed hot blood from his mouth there and then. 

%
\section*{{\suttatitleacronym MN 57}{\suttatitletranslation The Ascetic Who Behaved Like a Dog }{\suttatitleroot Kukkuravatikasutta}}
\addcontentsline{toc}{section}{\tocacronym{MN 57} \toctranslation{The Ascetic Who Behaved Like a Dog } \tocroot{Kukkuravatikasutta}}
\markboth{The Ascetic Who Behaved Like a Dog }{Kukkuravatikasutta}
\extramarks{MN 57}{MN 57}

\scevam{So\marginnote{1.1} I have heard. }At one time the Buddha was staying in the land of the Koliyans, where they have a town named Haliddavasana. 

Then\marginnote{2.1} \textsanskrit{Puṇṇa} Koliyaputta, who had taken a vow to behave like a cow, and Seniya, a naked ascetic who had taken a vow to behave like a dog, went to see the Buddha. \textsanskrit{Puṇṇa} bowed to the Buddha and sat down to one side, while Seniya exchanged greetings and polite conversation with him before sitting down to one side curled up like a dog. 

\textsanskrit{Puṇṇa}\marginnote{2.2} said to the Buddha, “Sir, this naked dog ascetic Seniya does a hard thing: he eats food placed on the ground. For a long time he has undertaken that observance to behave like a dog. Where will he be reborn in his next life?” 

“Enough,\marginnote{2.6} \textsanskrit{Puṇṇa}, let it be. Don’t ask me that.” 

For\marginnote{2.7} a second time … and a third time, \textsanskrit{Puṇṇa} said to the Buddha, “Sir, this naked dog ascetic Seniya does a hard thing: he eats food placed on the ground. For a long time he has undertaken that observance to behave like a dog. Where will he be reborn in his next life?” 

“Clearly,\marginnote{2.12} \textsanskrit{Puṇṇa}, I’m not getting through to you when I say: ‘Enough, \textsanskrit{Puṇṇa}, let it be. Don’t ask me that.’ Nevertheless, I will answer you. 

Take\marginnote{3.1} someone who develops the dog observance fully and uninterruptedly. They develop a dog’s ethics, a dog’s mentality, and a dog’s behavior fully and uninterruptedly. When their body breaks up, after death, they’re reborn in the company of dogs. But if they have such a view: ‘By this precept or observance or mortification or spiritual life, may I become one of the gods!’ This is their wrong view. An individual with wrong view is reborn in one of two places, I say: hell or the animal realm. So if the dog observance succeeds it leads to rebirth in the company of dogs, but if it fails it leads to hell.” 

When\marginnote{4.1} he said this, Seniya cried and burst out in tears. 

The\marginnote{4.2} Buddha said to \textsanskrit{Puṇṇa}, “This is what I didn’t get through to you when I said: ‘Enough, \textsanskrit{Puṇṇa}, let it be. Don’t ask me that.’” 

“Sir,\marginnote{4.5} I’m not crying because of what the Buddha said. But, sir, for a long time I have undertaken this observance to behave like a dog. Sir, this \textsanskrit{Puṇṇa} has taken a vow to behave like a cow. For a long time he has undertaken that observance to behave like a cow. Where will he be reborn in his next life?” 

“Enough,\marginnote{4.10} Seniya, let it be. Don’t ask me that.” 

For\marginnote{4.11} a second time … and a third time Seniya said to the Buddha, “Sir, this \textsanskrit{Puṇṇa} has taken a vow to behave like a cow. For a long time he has undertaken that observance to behave like a cow. Where will he be reborn in his next life?” 

“Clearly,\marginnote{4.16} Seniya, I’m not getting through to you when I say: ‘Enough, Seniya, let it be. Don’t ask me that.’ Nevertheless, I will answer you. 

Take\marginnote{5.1} someone who develops the cow observance fully and uninterruptedly. They develop a cow’s ethics, a cow’s mentality, and a cow’s behavior fully and uninterruptedly. When their body breaks up, after death, they’re reborn in the company of cows. But if they have such a view: ‘By this precept or observance or mortification or spiritual life, may I become one of the gods!’ This is their wrong view. An individual with wrong view is reborn in one of two places, I say: hell or the animal realm. So if the cow observance succeeds it leads to rebirth in the company of cows, but if it fails it leads to hell.” 

When\marginnote{6.1} he said this, \textsanskrit{Puṇṇa} cried and burst out in tears. 

The\marginnote{6.2} Buddha said to Seniya, “This is what I didn’t get through to you when I said: ‘Enough, Seniya, let it be. Don’t ask me that.’” 

“Sir,\marginnote{6.5} I’m not crying because of what the Buddha said. But, sir, for a long time I have undertaken this observance to behave like a cow. I am quite confident that the Buddha is capable of teaching me so that I can give up this cow observance, and the naked ascetic Seniya can give up that dog observance.” 

“Well\marginnote{6.9} then, \textsanskrit{Puṇṇa}, listen and pay close attention, I will speak.” 

“Yes,\marginnote{6.10} sir,” he replied. The Buddha said this: 

“\textsanskrit{Puṇṇa},\marginnote{7.1} I declare these four kinds of deeds, having realized them with my own insight. What four? 

\begin{enumerate}%
\item There are dark deeds with dark results; %
\item bright deeds with bright results; %
\item dark and bright deeds with dark and bright results; and %
\item neither dark nor bright deeds with neither dark nor bright results, which lead to the ending of deeds. %
\end{enumerate}

And\marginnote{8.1} what are dark deeds with dark results? It’s when someone makes hurtful choices by way of body, speech, and mind. Having made these choices, they’re reborn in a hurtful world, where hurtful contacts strike them. Touched by hurtful contacts, they experience hurtful feelings that are exclusively painful—like the beings in hell. This is how a being is born from a being. For your deeds determine your rebirth, and when you’re reborn contacts strike you. This is why I say that sentient beings are heirs to their deeds. These are called dark deeds with dark results. 

And\marginnote{9.1} what are bright deeds with bright results? It’s when someone makes pleasing choices by way of body, speech, and mind. Having made these choices, they are reborn in a pleasing world, where pleasing contacts strike them. Touched by pleasing contacts, they experience pleasing feelings that are exclusively happy—like the gods replete with glory. This is how a being is born from a being. For your deeds determine your rebirth, and when you’re reborn contacts strike you. This is why I say that sentient beings are heirs to their deeds. These are called bright deeds with bright results. 

And\marginnote{10.1} what are dark and bright deeds with dark and bright results? It’s when someone makes both hurtful and pleasing choices by way of body, speech, and mind. Having made these choices, they are reborn in a world that is both hurtful and pleasing, where hurtful and pleasing contacts strike them. Touched by both hurtful and pleasing contacts, they experience both hurtful and pleasing feelings that are a mixture of pleasure and pain—like humans, some gods, and some beings in the underworld. This is how a being is born from a being. For what you do brings about your rebirth, and when you’re reborn contacts strike you. This is why I say that sentient beings are heirs to their deeds. These are called dark and bright deeds with dark and bright results. 

And\marginnote{11.1} what are neither dark nor bright deeds with neither dark nor bright results, which lead to the ending of deeds? It’s the intention to give up dark deeds with dark results, bright deeds with bright results, and both dark and bright deeds with both dark and bright results. These are called neither dark nor bright deeds with neither dark nor bright results, which lead to the ending of deeds. 

These\marginnote{11.4} are the four kinds of deeds that I declare, having realized them with my own insight.” 

When\marginnote{12.1} he had spoken, \textsanskrit{Puṇṇa} Koliyaputta the observer of cow behavior said to the Buddha, “Excellent, sir! Excellent! … From this day forth, may the Buddha remember me as a lay follower who has gone for refuge for life.” 

And\marginnote{13.1} Seniya the naked dog ascetic said to the Buddha, “Excellent, sir! Excellent! … I go for refuge to the Buddha, to the teaching, and to the mendicant \textsanskrit{Saṅgha}. Sir, may I receive the going forth, the ordination in the Buddha’s presence?” 

“Seniya,\marginnote{14.1} if someone formerly ordained in another sect wishes to take the going forth, the ordination in this teaching and training, they must spend four months on probation. When four months have passed, if the mendicants are satisfied, they’ll give the going forth, the ordination into monkhood. However, I have recognized individual differences in this matter.” 

“Sir,\marginnote{14.3} if four months probation are required in such a case, I’ll spend four years on probation. When four years have passed, if the mendicants are satisfied, let them give me the going forth, the ordination into monkhood.” 

And\marginnote{15.1} the naked dog ascetic Seniya received the going forth, the ordination in the Buddha’s presence. Not long after his ordination, Venerable Seniya, living alone, withdrawn, diligent, keen, and resolute, soon realized the supreme end of the spiritual path in this very life. He lived having achieved with his own insight the goal for which gentlemen rightly go forth from the lay life to homelessness. 

He\marginnote{15.3} understood: “Rebirth is ended; the spiritual journey has been completed; what had to be done has been done; there is no return to any state of existence.” And Venerable Seniya became one of the perfected. 

%
\section*{{\suttatitleacronym MN 58}{\suttatitletranslation With Prince Abhaya }{\suttatitleroot Abhayarājakumārasutta}}
\addcontentsline{toc}{section}{\tocacronym{MN 58} \toctranslation{With Prince Abhaya } \tocroot{Abhayarājakumārasutta}}
\markboth{With Prince Abhaya }{Abhayarājakumārasutta}
\extramarks{MN 58}{MN 58}

\scevam{So\marginnote{1.1} I have heard. }At one time the Buddha was staying near \textsanskrit{Rājagaha}, in the Bamboo Grove, the squirrels’ feeding ground. 

Then\marginnote{2.1} Prince Abhaya went up to \textsanskrit{Nigaṇṭha} \textsanskrit{Nātaputta}, bowed, and sat down to one side. \textsanskrit{Nigaṇṭha} \textsanskrit{Nātaputta} said to him, “Come, prince, refute the ascetic Gotama’s doctrine. Then you will get a good reputation: ‘Prince Abhaya refuted the doctrine of the ascetic Gotama, so mighty and powerful!’” 

“But\marginnote{3.4} sir, how am I to do this?” 

“Here,\marginnote{3.5} prince, go to the ascetic Gotama and say to him: ‘Sir, might the Realized One utter speech that is disliked by others?’ When he’s asked this, if he answers: ‘He might, prince,’ say this to him, ‘Then, sir, what exactly is the difference between you and an ordinary person? For even an ordinary person might utter speech that is disliked by others.’ But if he answers, ‘He would not, prince,’ say this to him: ‘Then, sir, why exactly did you declare of Devadatta: “Devadatta is going to a place of loss, to hell, there to remain for an eon, irredeemable”? Devadatta was angry and upset with what you said.’ 

When\marginnote{3.16} you put this dilemma to him, the Buddha won’t be able to either spit it out or swallow it down. He’ll be like a man with an iron cross stuck in his throat, unable to either spit it out or swallow it down.” 

“Yes,\marginnote{4.1} sir,” replied Abhaya. He got up from his seat, bowed, and respectfully circled \textsanskrit{Nigaṇṭha} \textsanskrit{Nāṭaputta}, keeping him on his right. Then he went to the Buddha, bowed, and sat down to one side. 

Then\marginnote{4.2} he looked up at the sun and thought, “It’s too late to refute the Buddha’s doctrine today. I shall refute his doctrine in my own home tomorrow.” He said to the Buddha, “Sir, may the Buddha please accept tomorrow’s meal from me, together with three other monks.” The Buddha consented in silence. 

Then,\marginnote{5.1} knowing that the Buddha had consented, Abhaya got up from his seat, bowed, and respectfully circled the Buddha, keeping him on his right, before leaving. 

Then\marginnote{5.2} when the night had passed, the Buddha robed up in the morning and, taking his bowl and robe, went to Abhaya’s home, and sat down on the seat spread out. Then Abhaya served and satisfied the Buddha with his own hands with a variety of delicious foods. 

When\marginnote{5.4} the Buddha had eaten and washed his hand and bowl, Abhaya took a low seat, sat to one side, and said to him, “Sir, might the Realized One utter speech that is disliked by others?” 

“This\marginnote{6.3} is no simple matter, prince.” 

“Then\marginnote{6.4} the Jains have lost in this, sir.” 

“But\marginnote{6.5} prince, why do you say that the Jains have lost in this?” 

Then\marginnote{6.7} Abhaya told the Buddha all that had happened. 

Now\marginnote{7.1} at that time a little baby boy was sitting in Prince Abhaya’s lap. Then the Buddha said to Abhaya, “What do you think, prince? If—because of your negligence or his nurse’s negligence—your boy was to put a stick or stone in his mouth, what would you do to him?” 

“I’d\marginnote{7.5} try to take it out, sir. If that didn’t work, I’d hold his head with my left hand, and take it out using a hooked finger of my right hand, even if it drew blood. Why is that? Because I have compassion for the boy, sir.” 

“In\marginnote{8.1} the same way, prince, the Realized One does not utter speech that he knows to be untrue, false, and harmful, and which is disliked by others. The Realized One does not utter speech that he knows to be true and substantive, but which is harmful and disliked by others. The Realized One knows the right time to speak so as to explain what he knows to be true, substantive, and beneficial, but which is disliked by others. The Realized One does not utter speech that he knows to be untrue, false, and harmful, but which is liked by others. The Realized One does not utter speech that he knows to be true and substantive, but which is harmful, even if it is liked by others. The Realized One knows the right time to speak so as to explain what he knows to be true, substantive, and beneficial, and which is liked by others. Why is that? Because the Realized One has compassion for sentient beings.” 

“Sir,\marginnote{9.1} there are clever aristocrats, brahmins, householders, or ascetics who come to see you with a question already planned. Do you think beforehand that if they ask you like this, you’ll answer like that, or does the answer just appear to you on the spot?” 

“Well\marginnote{10.1} then, prince, I’ll ask you about this in return, and you can answer as you like. What do you think, prince? Are you skilled in the various parts of a chariot?” 

“I\marginnote{10.4} am, sir.” 

“What\marginnote{10.5} do you think, prince? When they come to you and ask: ‘What’s the name of this chariot part?’ Do you think beforehand that if they ask you like this, you’ll answer like that, or does the answer appear to you on the spot?” 

“Sir,\marginnote{10.9} I’m well-known as a charioteer skilled in a chariot’s parts. All the parts are well-known to me. The answer just appears to me on the spot.” 

“In\marginnote{11.1} the same way, when clever aristocrats, brahmins, householders, or ascetics come to see me with a question already planned, the answer just appears to me on the spot. Why is that? Because the Realized One has clearly comprehended the principle of the teachings, so that the answer just appears to him on the spot.” 

When\marginnote{12.1} he had spoken, Prince Abhaya said to the Buddha, “Excellent, sir! Excellent! … From this day forth, may Master Gotama remember me as a lay follower who has gone for refuge for life.” 

%
\section*{{\suttatitleacronym MN 59}{\suttatitletranslation The Many Kinds of Feeling }{\suttatitleroot Bahuvedanīyasutta}}
\addcontentsline{toc}{section}{\tocacronym{MN 59} \toctranslation{The Many Kinds of Feeling } \tocroot{Bahuvedanīyasutta}}
\markboth{The Many Kinds of Feeling }{Bahuvedanīyasutta}
\extramarks{MN 59}{MN 59}

\scevam{So\marginnote{1.1} I have heard. }At one time the Buddha was staying near \textsanskrit{Sāvatthī} in Jeta’s Grove, \textsanskrit{Anāthapiṇḍika}’s monastery. 

Then\marginnote{1.3} the master builder \textsanskrit{Pañcakaṅga} went up to Venerable \textsanskrit{Udāyī}, bowed, sat down to one side, and said to him, “Sir, how many feelings has the Buddha spoken of?” 

“Master\marginnote{1.5} builder, the Buddha has spoken of three feelings: pleasant, painful, and neutral. The Buddha has spoken of these three feelings.” 

When\marginnote{1.8} he said this, \textsanskrit{Pañcakaṅga} said to \textsanskrit{Udāyī}, “Sir, \textsanskrit{Udāyī}, the Buddha hasn’t spoken of three feelings. He’s spoken of two feelings: pleasant and painful. The Buddha said that neutral feeling is included as a peaceful and subtle kind of pleasure.” 

For\marginnote{2.1} a second time, \textsanskrit{Udāyī} said to \textsanskrit{Pañcakaṅga}, “The Buddha hasn’t spoken of two feelings, he’s spoken of three.” For a second time, \textsanskrit{Pañcakaṅga} said to \textsanskrit{Udāyī}, “The Buddha hasn’t spoken of three feelings, he’s spoken of two.” 

And\marginnote{3.1} for a third time, \textsanskrit{Udāyī} said to \textsanskrit{Pañcakaṅga}, “The Buddha hasn’t spoken of two feelings, he’s spoken of three.” 

And\marginnote{3.6} for a third time, \textsanskrit{Pañcakaṅga} said to \textsanskrit{Udāyī}, “The Buddha hasn’t spoken of three feelings, he’s spoken of two.” 

But\marginnote{3.11} neither was able to persuade the other. 

Venerable\marginnote{4.1} Ānanda heard this discussion between \textsanskrit{Udāyī} and \textsanskrit{Pañcakaṅga}. Then he went up to the Buddha, bowed, sat down to one side, and informed the Buddha of all they had discussed. When he had spoken, the Buddha said to him, 

“Ānanda,\marginnote{5.1} the explanation by the mendicant \textsanskrit{Udāyī}, which the master builder \textsanskrit{Pañcakaṅga} didn’t agree with, was quite correct. But the explanation by \textsanskrit{Pañcakaṅga}, which \textsanskrit{Udāyī} didn’t agree with, was also quite correct. In one explanation I’ve spoken of two feelings. In another explanation I’ve spoken of three feelings, or five, six, eighteen, thirty-six, or a hundred and eight feelings. I’ve explained the teaching in all these different ways. This being so, you can expect that those who don’t concede, approve, or agree with what has been well spoken will argue, quarrel, and fight, continually wounding each other with barbed words. I’ve explained the teaching in all these different ways. This being so, you can expect that those who do concede, approve, or agree with what has been well spoken will live in harmony, appreciating each other, without quarreling, blending like milk and water, and regarding each other with kindly eyes. 

There\marginnote{6.1} are these five kinds of sensual stimulation. What five? Sights known by the eye that are likable, desirable, agreeable, pleasant, sensual, and arousing. Sounds known by the ear … Smells known by the nose … Tastes known by the tongue … Touches known by the body that are likable, desirable, agreeable, pleasant, sensual, and arousing. These are the five kinds of sensual stimulation. The pleasure and happiness that arise from these five kinds of sensual stimulation is called sensual pleasure. 

There\marginnote{7.1} are those who would say that this is the highest pleasure and happiness that sentient beings experience. But I don’t grant them that. Why is that? Because there is another pleasure that is finer than that. And what is that pleasure? It’s when a mendicant, quite secluded from sensual pleasures, secluded from unskillful qualities, enters and remains in the first absorption, which has the rapture and bliss born of seclusion, while placing the mind and keeping it connected. This is a pleasure that is finer than that. 

There\marginnote{8.1} are those who would say that this is the highest pleasure and happiness that sentient beings experience. But I don’t grant them that. Why is that? Because there is another pleasure that is finer than that. And what is that pleasure? It’s when, as the placing of the mind and keeping it connected are stilled, a mendicant enters and remains in the second absorption, which has the rapture and bliss born of immersion, with internal clarity and confidence, and unified mind, without placing the mind and keeping it connected. … 

There\marginnote{9.1} is another pleasure that is finer than that. And what is that pleasure? It’s when, with the fading away of rapture, a mendicant enters and remains in the third absorption, where they meditate with equanimity, mindful and aware, personally experiencing the bliss of which the noble ones declare, ‘Equanimous and mindful, one meditates in bliss.’ … 

There\marginnote{10.1} is another pleasure that is finer than that. And what is that pleasure? It’s when, giving up pleasure and pain, and ending former happiness and sadness, a mendicant enters and remains in the fourth absorption, without pleasure or pain, with pure equanimity and mindfulness. … 

There\marginnote{11.1} is another pleasure that is finer than that. And what is that pleasure? It’s when a mendicant, going totally beyond perceptions of form, with the ending of perceptions of impingement, not focusing on perceptions of diversity, aware that ‘space is infinite’, enters and remains in the dimension of infinite space. … 

There\marginnote{12.1} is another pleasure that is finer than that. And what is that pleasure? It’s when a mendicant, going totally beyond the dimension of infinite space, aware that ‘consciousness is infinite’, enters and remains in the dimension of infinite consciousness. … 

There\marginnote{13.1} is another pleasure that is finer than that. And what is that pleasure? It’s when a mendicant, going totally beyond the dimension of infinite consciousness, aware that ‘there is nothing at all’, enters and remains in the dimension of nothingness. … 

There\marginnote{14.1} is another pleasure that is finer than that. And what is that pleasure? It’s when a mendicant, going totally beyond the dimension of nothingness, enters and remains in the dimension of neither perception nor non-perception. This is a pleasure that is finer than that. 

There\marginnote{15.1} are those who would say that this is the highest pleasure and happiness that sentient beings experience. But I don’t grant them that. Why is that? Because there is another pleasure that is finer than that. And what is that pleasure? It’s when a mendicant, going totally beyond the dimension of neither perception nor non-perception, enters and remains in the cessation of perception and feeling. This is a pleasure that is finer than that. 

It’s\marginnote{16.1} possible that wanderers who follow other paths might say, ‘The ascetic Gotama spoke of the cessation of perception and feeling, and he includes it in happiness. What’s up with that?’ 

When\marginnote{16.4} wanderers who follow other paths say this, you should say to them, ‘Reverends, when the Buddha describes what’s included in happiness, he’s not just referring to pleasant feeling. The Realized One describes pleasure as included in happiness wherever it’s found, and in whatever context.’” 

That\marginnote{16.7} is what the Buddha said. Satisfied, Venerable Ānanda was happy with what the Buddha said. 

%
\section*{{\suttatitleacronym MN 60}{\suttatitletranslation Guaranteed }{\suttatitleroot Apaṇṇakasutta}}
\addcontentsline{toc}{section}{\tocacronym{MN 60} \toctranslation{Guaranteed } \tocroot{Apaṇṇakasutta}}
\markboth{Guaranteed }{Apaṇṇakasutta}
\extramarks{MN 60}{MN 60}

\scevam{So\marginnote{1.1} I have heard. }At one time the Buddha was wandering in the land of the Kosalans together with a large \textsanskrit{Saṅgha} of mendicants when he arrived at a village of the Kosalan brahmins named \textsanskrit{Sālā}. 

The\marginnote{2.1} brahmins and householders of \textsanskrit{Sālā} heard: 

“It\marginnote{2.2} seems the ascetic Gotama—a Sakyan, gone forth from a Sakyan family—wandering in the land of the Kosalans has arrived at \textsanskrit{Sālā}, together with a large \textsanskrit{Saṅgha} of mendicants. He has this good reputation: ‘That Blessed One is perfected, a fully awakened Buddha, accomplished in knowledge and conduct, holy, knower of the world, supreme guide for those who wish to train, teacher of gods and humans, awakened, blessed.’ He has realized with his own insight this world—with its gods, \textsanskrit{Māras} and \textsanskrit{Brahmās}, this population with its ascetics and brahmins, gods and humans—and he makes it known to others. He teaches Dhamma that’s good in the beginning, good in the middle, and good in the end, meaningful and well-phrased. And he reveals a spiritual practice that’s entirely full and pure. It’s good to see such perfected ones.” 

Then\marginnote{3.1} the brahmins and householders of \textsanskrit{Sālā} went up to the Buddha. Before sitting down to one side, some bowed, some exchanged greetings and polite conversation, some held up their joined palms toward the Buddha, some announced their name and clan, while some kept silent. The Buddha said to them: 

“So,\marginnote{4.1} householders, is there some other teacher you’re happy with, in whom you have acquired grounded faith?” 

“No,\marginnote{4.2} sir.” 

“Since\marginnote{4.3} you haven’t found a teacher you’re happy with, you should undertake and implement this guaranteed teaching. For when the guaranteed teaching is undertaken, it will be for your lasting welfare and happiness. And what is the guaranteed teaching? 

There\marginnote{5.1} are some ascetics and brahmins who have this doctrine and view: ‘There’s no meaning in giving, sacrifice, or offerings. There’s no fruit or result of good and bad deeds. There’s no afterlife. There’s no obligation to mother and father. No beings are reborn spontaneously. And there’s no ascetic or brahmin who is well attained and practiced, and who describes the afterlife after realizing it with their own insight.’ 

And\marginnote{6.1} there are some ascetics and brahmins whose doctrine directly contradicts this. They say: ‘There is meaning in giving, sacrifice, and offerings. There are fruits and results of good and bad deeds. There is an afterlife. There is obligation to mother and father. There are beings reborn spontaneously. And there are ascetics and brahmins who are well attained and practiced, and who describe the afterlife after realizing it with their own insight.’ 

What\marginnote{6.4} do you think, householders? Don’t these doctrines directly contradict each other?” 

“Yes,\marginnote{6.6} sir.” 

“Since\marginnote{7.1} this is so, consider those ascetics and brahmins whose view is that there’s no meaning in giving, etc. You can expect that they will reject good conduct by way of body, speech, and mind, and undertake and implement bad conduct by way of body, speech, and mind. Why is that? Because those ascetics and brahmins don’t see that unskillful qualities are full of drawbacks, sordidness, and corruption, or that skillful qualities have the benefit and cleansing power of renunciation. 

Moreover,\marginnote{8.1} since there actually is another world, their view that there is no other world is wrong view. Since there actually is another world, their thought that there is no other world is wrong thought. Since there actually is another world, their speech that there is no other world is wrong speech. Since there actually is another world, in saying that there is no other world they contradict those perfected ones who know the other world. Since there actually is another world, in convincing another that there is no other world they are convincing them to accept an untrue teaching. And on account of that they glorify themselves and put others down. So they give up their former ethical conduct and are established in unethical conduct. And that is how these many bad, unskillful qualities come to be with wrong view as condition—wrong view, wrong thought, wrong speech, contradicting the noble ones, convincing others to accept untrue teachings, and glorifying oneself and putting others down. 

A\marginnote{9.1} sensible person reflects on this matter in this way: ‘If there is no other world, when this individual’s body breaks up they will keep themselves safe. And if there is another world, when their body breaks up, after death, they will be reborn in a place of loss, a bad place, the underworld, hell. But let’s assume that those who say that there is no other world are correct. Regardless, that individual is still criticized by sensible people in the present life as being an immoral individual of wrong view, a nihilist.’ But if there really is another world, they lose on both counts. For they are criticized by sensible people in the present life, and when their body breaks up, after death, they will be reborn in a place of loss, a bad place, the underworld, hell. They have wrongly undertaken this guaranteed teaching in such a way that it encompasses the positive outcomes of one side only, leaving out the skillful premise. 

Since\marginnote{10.1} this is so, consider those ascetics and brahmins whose view is that there is meaning in giving, etc. You can expect that they will reject bad conduct by way of body, speech, and mind, and undertake and implement good conduct by way of body, speech, and mind. Why is that? Because those ascetics and brahmins see that unskillful qualities are full of drawbacks, sordidness, and corruption, and that skillful qualities have the benefit and cleansing power of renunciation. 

Moreover,\marginnote{11.1} since there actually is another world, their view that there is another world is right view. Since there actually is another world, their thought that there is another world is right thought. Since there actually is another world, their speech that there is another world is right speech. Since there actually is another world, in saying that there is another world they don’t contradict those perfected ones who know the other world. Since there actually is another world, in convincing another that there is another world they are convincing them to accept a true teaching. And on account of that they don’t glorify themselves or put others down. So they give up their former unethical conduct and are established in ethical conduct. And that is how these many skillful qualities come to be with right view as condition—right view, right thought, right speech, not contradicting the noble ones, convincing others to accept true teachings, and not glorifying oneself or putting others down. 

A\marginnote{12.1} sensible person reflects on this matter in this way: ‘If there is another world, when this individual’s body breaks up, after death, they will be reborn in a good place, a heavenly realm. But let’s assume that those who say that there is no other world are correct. Regardless, that individual is still praised by sensible people in the present life as being a moral individual of right view, who affirms a positive teaching.’ So if there really is another world, they win on both counts. For they are praised by sensible people in the present life, and when their body breaks up, after death, they will be reborn in a good place, a heavenly realm. They have rightly undertaken this guaranteed teaching in such a way that it encompasses the positive outcomes of both sides, leaving out the unskillful premise. 

There\marginnote{13.1} are some ascetics and brahmins who have this doctrine and view: ‘The one who acts does nothing wrong when they punish, mutilate, torture, aggrieve, oppress, intimidate, or when they encourage others to do the same. They do nothing wrong when they kill, steal, break into houses, plunder wealth, steal from isolated buildings, commit highway robbery, commit adultery, and lie. If you were to reduce all the living creatures of this earth to one heap and mass of flesh with a razor-edged chakram, no evil comes of that, and no outcome of evil. If you were to go along the south bank of the Ganges killing, mutilating, and torturing, and encouraging others to do the same, no evil comes of that, and no outcome of evil. If you were to go along the north bank of the Ganges giving and sacrificing and encouraging others to do the same, no merit comes of that, and no outcome of merit. In giving, self-control, restraint, and truthfulness there is no merit or outcome of merit.’ 

And\marginnote{14.1} there are some ascetics and brahmins whose doctrine directly contradicts this. They say: ‘The one who acts does a bad deed when they punish, mutilate, torture, aggrieve, oppress, intimidate, or when they encourage others to do the same. They do a bad deed when they kill, steal, break into houses, plunder wealth, steal from isolated buildings, commit highway robbery, commit adultery, and lie. If you were to reduce all the living creatures of this earth to one heap and mass of flesh with a razor-edged chakram, evil comes of that, and an outcome of evil. If you were to go along the south bank of the Ganges killing, mutilating, and torturing, and encouraging others to do the same, evil comes of that, and an outcome of evil. If you were to go along the north bank of the Ganges giving and sacrificing and encouraging others to do the same, merit comes of that, and an outcome of merit. In giving, self-control, restraint, and truthfulness there is merit and outcome of merit.’ 

What\marginnote{14.7} do you think, householders? Don’t these doctrines directly contradict each other?” 

“Yes,\marginnote{14.9} sir.” 

“Since\marginnote{15.1} this is so, consider those ascetics and brahmins whose view is that the one who acts does nothing wrong when they punish, etc. You can expect that they will reject good conduct by way of body, speech, and mind, and undertake and implement bad conduct by way of body, speech, and mind. Why is that? Because those ascetics and brahmins don’t see that unskillful qualities are full of drawbacks, sordidness, and corruption, or that skillful qualities have the benefit and cleansing power of renunciation. 

Moreover,\marginnote{16.1} since action actually does have an effect, their view that action is ineffective is wrong view. Since action actually does have an effect, their thought that action is ineffective is wrong thought. Since action actually does have an effect, their speech that action is ineffective is wrong speech. Since action actually does have an effect, in saying that action is ineffective they contradict those perfected ones who teach that action is effective. Since action actually does have an effect, in convincing another that action is ineffective they are convincing them to accept an untrue teaching. And on account of that they glorify themselves and put others down. So they give up their former ethical conduct and are established in unethical conduct. And that is how these many bad, unskillful qualities come to be with wrong view as condition—wrong view, wrong thought, wrong speech, contradicting the noble ones, convincing others to accept untrue teachings, and glorifying oneself and putting others down. 

A\marginnote{17.1} sensible person reflects on this matter in this way: ‘If there is no effective action, when this individual’s body breaks up they will keep themselves safe. And if there is effective action, when their body breaks up, after death, they will be reborn in a place of loss, a bad place, the underworld, hell. But let’s assume that those who say that there is no effective action are correct. Regardless, that individual is still criticized by sensible people in the present life as being an immoral individual of wrong view, one who denies the efficacy of action.’ But if there really is effective action, they lose on both counts. For they are criticized by sensible people in the present life, and when their body breaks up, after death, they will be reborn in a place of loss, a bad place, the underworld, hell. They have wrongly undertaken this guaranteed teaching in such a way that it encompasses the positive outcomes of one side only, leaving out the skillful premise. 

Since\marginnote{18.1} this is so, consider those ascetics and brahmins whose view is that the one who acts does a bad deed when they punish, etc. You can expect that they will reject bad conduct by way of body, speech, and mind, and undertake and implement good conduct by way of body, speech, and mind. Why is that? Because those ascetics and brahmins see that unskillful qualities are full of drawbacks, sordidness, and corruption, and that skillful qualities have the benefit and cleansing power of renunciation. 

Moreover,\marginnote{19.1} since action actually does have an effect, their view that action is effective is right view. Since action actually does have an effect, their thought that action is effective is right thought. Since action actually does have an effect, their speech that action is effective is right speech. Since action actually does have an effect, in saying that action is effective they don’t contradict those perfected ones who teach that action is effective. Since action actually does have an effect, in convincing another that action is effective they are convincing them to accept a true teaching. And on account of that they don’t glorify themselves or put others down. So they give up their former unethical conduct and are established in ethical conduct. And that is how these many skillful qualities come to be with right view as condition—right view, right thought, right speech, not contradicting the noble ones, convincing others to accept true teachings, and not glorifying oneself or putting others down. 

A\marginnote{20.1} sensible person reflects on this matter in this way: ‘If there is effective action, when this individual’s body breaks up, after death, they will be reborn in a good place, a heavenly realm. But let’s assume that those who say that there is no effective action are correct. Regardless, that individual is still praised by sensible people in the present life as being a moral individual of right view, who affirms the efficacy of action.’ So if there really is effective action, they win on both counts. For they are praised by sensible people in the present life, and when their body breaks up, after death, they will be reborn in a good place, a heavenly realm. They have rightly undertaken this guaranteed teaching in such a way that it encompasses the positive outcomes of both sides, leaving out the unskillful premise. 

There\marginnote{21.1} are some ascetics and brahmins who have this doctrine and view: ‘There is no cause or condition for the corruption of sentient beings. Sentient beings are corrupted without cause or reason. There’s no cause or condition for the purification of sentient beings. Sentient beings are purified without cause or reason. There is no power, no energy, no human strength or vigor. All sentient beings, all living creatures, all beings, all souls lack control, power, and energy. Molded by destiny, circumstance, and nature, they experience pleasure and pain in the six classes of rebirth.’ 

And\marginnote{22.1} there are some ascetics and brahmins whose doctrine directly contradicts this. They say: ‘There is a cause and condition for the corruption of sentient beings. Sentient beings are corrupted with cause and reason. There is a cause and condition for the purification of sentient beings. Sentient beings are purified with cause and reason. There is power, energy, human strength and vigor. It is not the case that all sentient beings, all living creatures, all beings, all souls lack control, power, and energy, or that, molded by destiny, circumstance, and nature, they experience pleasure and pain in the six classes of rebirth.’ 

What\marginnote{22.9} do you think, householders? Don’t these doctrines directly contradict each other?” 

“Yes,\marginnote{22.11} sir.” 

“Since\marginnote{23.1} this is so, consider those ascetics and brahmins whose view is that there’s no cause or condition for the corruption of sentient beings, etc. You can expect that they will reject good conduct by way of body, speech, and mind, and undertake and implement bad conduct by way of body, speech, and mind. Why is that? Because those ascetics and brahmins don’t see that unskillful qualities are full of drawbacks, sordidness, and corruption, or that skillful qualities have the benefit and cleansing power of renunciation. 

Moreover,\marginnote{24.1} since there actually is causality, their view that there is no causality is wrong view. Since there actually is causality, their thought that there is no causality is wrong thought. Since there actually is causality, their speech that there is no causality is wrong speech. Since there actually is causality, in saying that there is no causality they contradict those perfected ones who teach that there is causality. Since there actually is causality, in convincing another that there is no causality they are convincing them to accept an untrue teaching. And on account of that they glorify themselves and put others down. So they give up their former ethical conduct and are established in unethical conduct. And that is how these many bad, unskillful qualities come to be with wrong view as condition—wrong view, wrong thought, wrong speech, contradicting the noble ones, convincing others to accept untrue teachings, and glorifying oneself and putting others down. 

A\marginnote{25.1} sensible person reflects on this matter in this way: ‘If there is no causality, when this individual’s body breaks up they will keep themselves safe. And if there is causality, when their body breaks up, after death, they will be reborn in a place of loss, a bad place, the underworld, hell. But let’s assume that those who say that there is no causality are correct. Regardless, that individual is still criticized by sensible people in the present life as being an immoral individual of wrong view, one who denies causality.’ But if there really is causality, they lose on both counts. For they are criticized by sensible people in the present life, and when their body breaks up, after death, they will be reborn in a place of loss, a bad place, the underworld, hell. They have wrongly undertaken this guaranteed teaching in such a way that it encompasses the positive outcomes of one side only, leaving out the skillful premise. 

Since\marginnote{26.1} this is so, consider those ascetics and brahmins whose view is that there is a cause and condition for the corruption of sentient beings, etc. You can expect that they will reject bad conduct by way of body, speech, and mind, and undertake and implement good conduct by way of body, speech, and mind. Why is that? Because those ascetics and brahmins see that unskillful qualities are full of drawbacks, sordidness, and corruption, and that skillful qualities have the benefit and cleansing power of renunciation. 

Moreover,\marginnote{27.1} since there actually is causality, their view that there is causality is right view. Since there actually is causality, their thought that there is causality is right thought. Since there actually is causality, their speech that there is causality is right speech. Since there actually is causality, in saying that there is causality they don’t contradict those perfected ones who teach that there is causality. Since there actually is causality, in convincing another that there is causality they are convincing them to accept a true teaching. And on account of that they don’t glorify themselves or put others down. So they give up their former unethical conduct and are established in ethical conduct. And that is how these many skillful qualities come to be with right view as condition—right view, right thought, right speech, not contradicting the noble ones, convincing others to accept true teachings, and not glorifying oneself or putting others down. 

A\marginnote{28.1} sensible person reflects on this matter in this way: ‘If there is causality, when this individual’s body breaks up, after death, they will be reborn in a good place, a heavenly realm. But let’s assume that those who say that there is no causality are correct. Regardless, that individual is still praised by sensible people in the present life as being a moral individual of right view, who affirms causality.’ So if there really is causality, they win on both counts. For they are praised by sensible people in the present life, and when their body breaks up, after death, they will be reborn in a good place, a heavenly realm. They have rightly undertaken this guaranteed teaching in such a way that it encompasses the positive outcomes of both sides, leaving out the unskillful premise. 

There\marginnote{29.1} are some ascetics and brahmins who have this doctrine and view: ‘There are no totally formless states of meditation.’ 

And\marginnote{30.1} there are some ascetics and brahmins whose doctrine directly contradicts this. They say: ‘There are totally formless states of meditation.’ 

What\marginnote{30.4} do you think, householders? Don’t these doctrines directly contradict each other?” 

“Yes,\marginnote{30.6} sir.” 

“A\marginnote{31.1} sensible person reflects on this matter in this way: ‘Some ascetics and brahmins say that there are no totally formless meditations, but I have not seen that. Some ascetics and brahmins say that there are totally formless meditations, but I have not known that. Without knowing or seeing, it would not be appropriate for me to take one side and declare, ‘This is the only truth, other ideas are silly.’ If those ascetics and brahmins who say that there are no totally formless meditations are correct, it is possible that I will be guaranteed rebirth among the gods who possess form and made of mind. If those ascetics and brahmins who say that there are totally formless meditations are correct, it is possible that I will be guaranteed rebirth among the gods who are formless and made of perception. Now, owing to form, bad things are seen: taking up the rod and the sword, quarrels, arguments, and disputes, accusations, divisive speech, and lies. But those things don’t exist where it is totally formless.’ Reflecting like this, they simply practice for disillusionment, dispassion, and cessation regarding forms. 

There\marginnote{32.1} are some ascetics and brahmins who have this doctrine and view: ‘There is no such thing as the total cessation of future lives.’ 

And\marginnote{33.1} there are some ascetics and brahmins whose doctrine directly contradicts this. They say: ‘There is such a thing as the total cessation of future lives.’ 

What\marginnote{33.4} do you think, householders? Don’t these doctrines directly contradict each other?” 

“Yes,\marginnote{33.6} sir.” 

“A\marginnote{34.1} sensible person reflects on this matter in this way: ‘Some ascetics and brahmins say that there is no such thing as the total cessation of future lives, but I have not seen that. Some ascetics and brahmins say that there is such a thing as the total cessation of future lives, but I have not known that. Without knowing or seeing, it would not be appropriate for me to take one side and declare, ‘This is the only truth, other ideas are silly.’ If those ascetics and brahmins who say that there is no such thing as the total cessation of future lives are correct, it is possible that I will be guaranteed rebirth among the gods who are formless and made of perception. If those ascetics and brahmins who say that there is such a thing as the total cessation of future lives are correct, it is possible that I will be fully extinguished in the present life. The view of those ascetics and brahmins who say that there is no such thing as the total cessation of future lives is close to greed, approving, attachment, and grasping. The view of those ascetics and brahmins who say that there is such a thing as the total cessation of future lives is close to non-greed, non-approving, non-attachment, and non-grasping.’ Reflecting like this, they simply practice for disillusionment, dispassion, and cessation regarding future lives. 

Householders,\marginnote{35.1} these four people are found in the world. What four? 

\begin{enumerate}%
\item One person mortifies themselves, committed to the practice of mortifying themselves. %
\item One person mortifies others, committed to the practice of mortifying others. %
\item One person mortifies themselves and others, committed to the practice of mortifying themselves and others. %
\item One person doesn’t mortify either themselves or others, committed to the practice of not mortifying themselves or others. They live without wishes in the present life, extinguished, cooled, experiencing bliss, having become holy in themselves. %
\end{enumerate}

And\marginnote{36.1} what person mortifies themselves, committed to the practice of mortifying themselves? It’s when someone goes naked, ignoring conventions. … And so they live committed to practicing these various ways of mortifying and tormenting the body. This is called a person who mortifies themselves, being committed to the practice of mortifying themselves. 

And\marginnote{37.1} what person mortifies others, committed to the practice of mortifying others? It’s when a person is a butcher of sheep, pigs, poultry, or deer, a hunter or fisher, a bandit, an executioner, a butcher of cattle, a jailer, or has some other cruel livelihood. This is called a person who mortifies others, being committed to the practice of mortifying others. 

And\marginnote{38.1} what person mortifies themselves and others, being committed to the practice of mortifying themselves and others? It’s when a person is an anointed aristocratic king or a well-to-do brahmin. … His bondservants, servants, and workers do their jobs under threat of punishment and danger, weeping, with tearful faces. This is called a person who mortifies themselves and others, being committed to the practice of mortifying themselves and others. 

And\marginnote{39.1} what person doesn’t mortify either themselves or others, committed to the practice of not mortifying themselves or others, living without wishes in the present life, extinguished, cooled, experiencing bliss, having become holy in themselves? 

It’s\marginnote{40{-}54.1} when a Realized One arises in the world, perfected, a fully awakened Buddha … A householder hears that teaching, or a householder’s child, or someone reborn in some good family. … They give up these five hindrances, corruptions of the heart that weaken wisdom. Then, quite secluded from sensual pleasures, secluded from unskillful qualities, they enter and remain in the first absorption … second absorption … third absorption … fourth absorption. 

When\marginnote{55.1} their mind has become immersed in \textsanskrit{samādhi} like this—purified, bright, flawless, rid of corruptions, pliable, workable, steady, and imperturbable—they extend it toward recollection of past lives. … They recollect their many kinds of past lives, with features and details. 

When\marginnote{55.3} their mind has become immersed in \textsanskrit{samādhi} like this—purified, bright, flawless, rid of corruptions, pliable, workable, steady, and imperturbable—they extend it toward knowledge of the death and rebirth of sentient beings. With clairvoyance that is purified and superhuman, they see sentient beings passing away and being reborn—inferior and superior, beautiful and ugly, in a good place or a bad place. … They understand how sentient beings are reborn according to their deeds. 

When\marginnote{55.5} their mind has become immersed in \textsanskrit{samādhi} like this—purified, bright, flawless, rid of corruptions, pliable, workable, steady, and imperturbable—they extend it toward knowledge of the ending of defilements. They truly understand: ‘This is suffering’ … ‘This is the origin of suffering’ … ‘This is the cessation of suffering’ … ‘This is the practice that leads to the cessation of suffering’. They truly understand: ‘These are defilements’ … ‘This is the origin of defilements’ … ‘This is the cessation of defilements’ … ‘This is the practice that leads to the cessation of defilements’. Knowing and seeing like this, their mind is freed from the defilements of sensuality, desire to be reborn, and ignorance. When they’re freed, they know they’re freed. 

They\marginnote{55.10} understand: ‘Rebirth is ended, the spiritual journey has been completed, what had to be done has been done, there is no return to any state of existence.’ 

This\marginnote{56.1} is called a person who neither mortifies themselves or others, being committed to the practice of not mortifying themselves or others. They live without wishes in the present life, extinguished, cooled, experiencing bliss, having become holy in themselves.” 

When\marginnote{57.1} he had spoken, the brahmins and householders of \textsanskrit{Sālā} said to the Buddha, “Excellent, Master Gotama! Excellent! As if he were righting the overturned, or revealing the hidden, or pointing out the path to the lost, or lighting a lamp in the dark so people with good eyes can see what’s there, Master Gotama has made the teaching clear in many ways. We go for refuge to Master Gotama, to the teaching, and to the mendicant \textsanskrit{Saṅgha}. From this day forth, may Master Gotama remember us as lay followers who have gone for refuge for life.” 

%
\addtocontents{toc}{\let\protect\contentsline\protect\nopagecontentsline}
\chapter*{The Chapter on Mendicants}
\addcontentsline{toc}{chapter}{\tocchapterline{The Chapter on Mendicants}}
\addtocontents{toc}{\let\protect\contentsline\protect\oldcontentsline}

%
\section*{{\suttatitleacronym MN 61}{\suttatitletranslation Advice to Rāhula at Ambalaṭṭhika }{\suttatitleroot Ambalaṭṭhikarāhulovādasutta}}
\addcontentsline{toc}{section}{\tocacronym{MN 61} \toctranslation{Advice to Rāhula at Ambalaṭṭhika } \tocroot{Ambalaṭṭhikarāhulovādasutta}}
\markboth{Advice to Rāhula at Ambalaṭṭhika }{Ambalaṭṭhikarāhulovādasutta}
\extramarks{MN 61}{MN 61}

\scevam{So\marginnote{1.1} I have heard. }At one time the Buddha was staying near \textsanskrit{Rājagaha}, in the Bamboo Grove, the squirrels’ feeding ground. 

Now\marginnote{2.1} at that time Venerable \textsanskrit{Rāhula} was staying at \textsanskrit{Ambalaṭṭhikā}. Then in the late afternoon, the Buddha came out of retreat and went to \textsanskrit{Ambalaṭṭhika} to see Venerable \textsanskrit{Rāhula}. \textsanskrit{Rāhula} saw the Buddha coming off in the distance. He spread out a seat and placed water for washing the feet. The Buddha sat on the seat spread out, and washed his feet. \textsanskrit{Rāhula} bowed to the Buddha and sat down to one side. 

Then\marginnote{3.1} the Buddha, leaving a little water in the pot, addressed \textsanskrit{Rāhula}, “\textsanskrit{Rāhula}, do you see this little bit of water left in the pot?” 

“Yes,\marginnote{3.3} sir.” 

“That’s\marginnote{3.4} how little of the ascetic’s nature is left in those who are not ashamed to tell a deliberate lie.” 

Then\marginnote{4.1} the Buddha, tossing away what little water was left in the pot, said to \textsanskrit{Rāhula}, “Do you see this little bit of water that was tossed away?” 

“Yes,\marginnote{4.3} sir.” 

“That’s\marginnote{4.4} how the ascetic’s nature is tossed away in those who are not ashamed to tell a deliberate lie.” 

Then\marginnote{5.1} the Buddha, turning the pot upside down, said to \textsanskrit{Rāhula}, “Do you see how this pot is turned upside down?” 

“Yes,\marginnote{5.3} sir.” 

“That’s\marginnote{5.4} how the ascetic’s nature is turned upside down in those who are not ashamed to tell a deliberate lie.” 

Then\marginnote{6.1} the Buddha, turning the pot right side up, said to \textsanskrit{Rāhula}, “Do you see how this pot is vacant and hollow?” 

“Yes,\marginnote{6.3} sir.” 

“That’s\marginnote{6.4} how vacant and hollow the ascetic’s nature is in those who are not ashamed to tell a deliberate lie. 

Suppose\marginnote{7.1} there was a royal bull elephant with tusks like chariot-poles, able to draw a heavy load, pedigree and battle-hardened. In battle it uses its fore-feet and hind-feet, its fore-quarters and hind-quarters, its head, ears, tusks, and tail, but it still protects its trunk. So its rider thinks: ‘This royal bull elephant still protects its trunk. It has not yet given its life.’ But when that royal bull elephant … in battle uses its fore-feet and hind-feet, its fore-quarters and hind-quarters, its head, ears, tusks, and tail, and its trunk, its rider thinks: ‘This royal bull elephant … in battle uses its fore-feet and hind-feet, its fore-quarters and hind-quarters, its head, ears, tusks, and tail, and its trunk. It has given its life. Now there is nothing that royal bull elephant would not do.’ 

In\marginnote{7.9} the same way, when someone is not ashamed to tell a deliberate lie, there is no bad deed they would not do, I say. So you should train like this: ‘I will not tell a lie, even for a joke.’ 

What\marginnote{8.1} do you think, \textsanskrit{Rāhula}? What is the purpose of a mirror?” 

“It’s\marginnote{8.3} for checking your reflection, sir.” 

“In\marginnote{8.4} the same way, deeds of body, speech, and mind should be done only after repeated checking. 

When\marginnote{9.1} you want to act with the body, you should check on that same deed: ‘Does this act with the body that I want to do lead to hurting myself, hurting others, or hurting both? Is it unskillful, with suffering as its outcome and result?’ If, while checking in this way, you know: ‘This act with the body that I want to do leads to hurting myself, hurting others, or hurting both. It’s unskillful, with suffering as its outcome and result.’ To the best of your ability, \textsanskrit{Rāhula}, you should not do such a deed. But if, while checking in this way, you know: ‘This act with the body that I want to do doesn’t lead to hurting myself, hurting others, or hurting both. It’s skillful, with happiness as its outcome and result.’ Then, \textsanskrit{Rāhula}, you should do such a deed. 

While\marginnote{10.1} you are acting with the body, you should check on that same act: ‘Does this act with the body that I am doing lead to hurting myself, hurting others, or hurting both? Is it unskillful, with suffering as its outcome and result?’ If, while checking in this way, you know: ‘This act with the body that I am doing leads to hurting myself, hurting others, or hurting both. It’s unskillful, with suffering as its outcome and result.’ Then, \textsanskrit{Rāhula}, you should desist from such a deed. But if, while checking in this way, you know: ‘This act with the body that I am doing doesn’t lead to hurting myself, hurting others, or hurting both. It’s skillful, with happiness as its outcome and result.’ Then, \textsanskrit{Rāhula}, you should continue doing such a deed. 

After\marginnote{11.1} you have acted with the body, you should check on that same act: ‘Does this act with the body that I have done lead to hurting myself, hurting others, or hurting both? Is it unskillful, with suffering as its outcome and result?’ If, while checking in this way, you know: ‘This act with the body that I have done leads to hurting myself, hurting others, or hurting both. It’s unskillful, with suffering as its outcome and result.’ Then, \textsanskrit{Rāhula}, you should confess, reveal, and clarify such a deed to the Teacher or a sensible spiritual companion. And having revealed it you should restrain yourself in future. But if, while checking in this way, you know: ‘This act with the body that I have done doesn’t lead to hurting myself, hurting others, or hurting both. It’s skillful, with happiness as its outcome and result.’ Then, \textsanskrit{Rāhula}, you should live in rapture and joy because of this, training day and night in skillful qualities. 

When\marginnote{12.1} you want to act with speech, you should check on that same deed: ‘Does this act of speech that I want to do lead to hurting myself, hurting others, or hurting both? …’ … 

If,\marginnote{14.1} while checking in this way, you know: ‘This act of speech that I have done leads to hurting myself, hurting others, or hurting both. It’s unskillful, with suffering as its outcome and result.’ Then, \textsanskrit{Rāhula}, you should confess, reveal, and clarify such a deed to the Teacher or a sensible spiritual companion. And having revealed it you should restrain yourself in future. But if, while checking in this way, you know: ‘This act of speech that I have done doesn’t lead to hurting myself, hurting others, or hurting both. It’s skillful, with happiness as its outcome and result.’ Then, \textsanskrit{Rāhula}, you should live in rapture and joy because of this, training day and night in skillful qualities. 

When\marginnote{15.1} you want to act with the mind, you should check on that same deed: ‘Does this act of mind that I want to do lead to hurting myself, hurting others, or hurting both? …’ … 

If,\marginnote{17.1} while checking in this way, you know: ‘This act of mind that I have done leads to hurting myself, hurting others, or hurting both. It’s unskillful, with suffering as its outcome and result.’ Then, \textsanskrit{Rāhula}, you should be horrified, repelled, and disgusted by that deed. And being repelled, you should restrain yourself in future. But if, while checking in this way, you know: ‘This act with the mind that I have done doesn’t lead to hurting myself, hurting others, or hurting both. It’s skillful, with happiness as its outcome and result.’ Then, \textsanskrit{Rāhula}, you should live in rapture and joy because of this, training day and night in skillful qualities. 

All\marginnote{18.1} the ascetics and brahmins of the past, future, and present who purify their physical, verbal, and mental actions do so after repeatedly checking. So \textsanskrit{Rāhula}, you should train yourself like this: ‘I will purify my physical, verbal, and mental actions after repeatedly checking.’” 

That\marginnote{18.6} is what the Buddha said. Satisfied, Venerable \textsanskrit{Rāhula} was happy with what the Buddha said. 

%
\section*{{\suttatitleacronym MN 62}{\suttatitletranslation The Longer Advice to Rāhula }{\suttatitleroot Mahārāhulovādasutta}}
\addcontentsline{toc}{section}{\tocacronym{MN 62} \toctranslation{The Longer Advice to Rāhula } \tocroot{Mahārāhulovādasutta}}
\markboth{The Longer Advice to Rāhula }{Mahārāhulovādasutta}
\extramarks{MN 62}{MN 62}

\scevam{So\marginnote{1.1} I have heard. }At one time the Buddha was staying near \textsanskrit{Sāvatthī} in Jeta’s Grove, \textsanskrit{Anāthapiṇḍika}’s monastery. 

Then\marginnote{2.1} the Buddha robed up in the morning and, taking his bowl and robe, entered \textsanskrit{Sāvatthī} for alms. And Venerable \textsanskrit{Rāhula} also robed up and followed behind the Buddha. 

Then\marginnote{3.1} the Buddha looked back at \textsanskrit{Rāhula} and said, “\textsanskrit{Rāhula}, you should truly see any kind of form at all—past, future, or present; internal or external; coarse or fine; inferior or superior; far or near: \emph{all} form—with right understanding: ‘This is not mine, I am not this, this is not my self.’” 

“Only\marginnote{3.3} form, Blessed One? Only form, Holy One?” 

“Form,\marginnote{3.4} \textsanskrit{Rāhula}, as well as feeling and perception and choices and consciousness.” 

Then\marginnote{4.1} \textsanskrit{Rāhula} thought, “Who would go to the village for alms today after being advised directly by the Buddha?” Turning back, he sat down at the root of a certain tree cross-legged, with his body straight, and established mindfulness right there. 

Venerable\marginnote{5.1} \textsanskrit{Sāriputta} saw him sitting there, and addressed him, “\textsanskrit{Rāhula}, develop mindfulness of breathing. When mindfulness of breathing is developed and cultivated it’s very fruitful and beneficial.” 

Then\marginnote{6.1} in the late afternoon, \textsanskrit{Rāhula} came out of retreat, went to the Buddha, bowed, sat down to one side, and said to him: 

“Sir,\marginnote{7.1} how is mindfulness of breathing developed and cultivated to be very fruitful and beneficial?” 

“\textsanskrit{Rāhula},\marginnote{8.1} the interior earth element is said to be anything hard, solid, and appropriated that’s internal, pertaining to an individual. This includes: head hair, body hair, nails, teeth, skin, flesh, sinews, bones, bone marrow, kidneys, heart, liver, diaphragm, spleen, lungs, intestines, mesentery, undigested food, feces, or anything else hard, solid, and appropriated that’s internal, pertaining to an individual. This is called the interior earth element. The interior earth element and the exterior earth element are just the earth element. This should be truly seen with right understanding like this: ‘This is not mine, I am not this, this is not my self.’ When you truly see with right understanding, you reject the earth element, detaching the mind from the earth element. 

And\marginnote{9.1} what is the water element? The water element may be interior or exterior. And what is the interior water element? Anything that’s water, watery, and appropriated that’s internal, pertaining to an individual. This includes: bile, phlegm, pus, blood, sweat, fat, tears, grease, saliva, snot, synovial fluid, urine, or anything else that’s water, watery, and appropriated that’s internal, pertaining to an individual. This is called the interior water element. The interior water element and the exterior water element are just the water element. This should be truly seen with right understanding like this: ‘This is not mine, I am not this, this is not my self.’ When you truly see with right understanding, you reject the water element, detaching the mind from the water element. 

And\marginnote{10.1} what is the fire element? The fire element may be interior or exterior. And what is the interior fire element? Anything that’s fire, fiery, and appropriated that’s internal, pertaining to an individual. This includes: that which warms, that which ages, that which heats you up when feverish, that which properly digests food and drink, or anything else that’s fire, fiery, and appropriated that’s internal, pertaining to an individual. This is called the interior fire element. The interior fire element and the exterior fire element are just the fire element. This should be truly seen with right understanding like this: ‘This is not mine, I am not this, this is not my self.’ When you truly see with right understanding, you reject the fire element, detaching the mind from the fire element. 

And\marginnote{11.1} what is the air element? The air element may be interior or exterior. And what is the interior air element? Anything that’s wind, windy, and appropriated that’s internal, pertaining to an individual. This includes: winds that go up or down, winds in the belly or the bowels, winds that flow through the limbs, in-breaths and out-breaths, or anything else that’s air, airy, and appropriated that’s internal, pertaining to an individual. This is called the interior air element. The interior air element and the exterior air element are just the air element. This should be truly seen with right understanding like this: ‘This is not mine, I am not this, this is not my self.’ When you truly see with right understanding, you reject the air element, detaching the mind from the air element. 

And\marginnote{12.1} what is the space element? The space element may be interior or exterior. And what is the interior space element? Anything that’s space, spacious, and appropriated that’s internal, pertaining to an individual. This includes: the ear canals, nostrils, and mouth; and the space for swallowing what is eaten and drunk, the space where it stays, and the space for excreting it from the nether regions. This is called the interior space element. The interior space element and the exterior space element are just the space element. This should be truly seen with right understanding like this: ‘This is not mine, I am not this, this is not my self.’ When you truly see with right understanding, you reject the space element, detaching the mind from the space element. 

\textsanskrit{Rāhula},\marginnote{13.1} meditate like the earth. For when you meditate like the earth, pleasant and unpleasant contacts will not occupy your mind. Suppose they were to toss both clean and unclean things on the earth, like feces, urine, spit, pus, and blood. The earth isn’t horrified, repelled, and disgusted because of this. In the same way, meditate like the earth. For when you meditate like the earth, pleasant and unpleasant contacts will not occupy your mind. 

Meditate\marginnote{14.1} like water. For when you meditate like water, pleasant and unpleasant contacts will not occupy your mind. Suppose they were to wash both clean and unclean things in the water, like feces, urine, spit, pus, and blood. The water isn’t horrified, repelled, and disgusted because of this. In the same way, meditate like water. For when you meditate like water, pleasant and unpleasant contacts will not occupy your mind. 

Meditate\marginnote{15.1} like fire. For when you meditate like fire, pleasant and unpleasant contacts will not occupy your mind. Suppose a fire were to burn both clean and unclean things, like feces, urine, spit, pus, and blood. The fire isn’t horrified, repelled, and disgusted because of this. In the same way, meditate like fire. For when you meditate like fire, pleasant and unpleasant contacts will not occupy your mind. 

Meditate\marginnote{16.1} like wind. For when you meditate like wind, pleasant and unpleasant contacts will not occupy your mind. Suppose the wind were to blow on both clean and unclean things, like feces, urine, spit, pus, and blood. The wind isn’t horrified, repelled, and disgusted because of this. In the same way, meditate like the wind. For when you meditate like wind, pleasant and unpleasant contacts will not occupy your mind. 

Meditate\marginnote{17.1} like space. For when you meditate like space, pleasant and unpleasant contacts will not occupy your mind. Just as space is not established anywhere, in the same way, meditate like space. For when you meditate like space, pleasant and unpleasant contacts will not occupy your mind. 

Meditate\marginnote{18.1} on love. For when you meditate on love any ill will will be given up. 

Meditate\marginnote{19.1} on compassion. For when you meditate on compassion any cruelty will be given up. 

Meditate\marginnote{20.1} on rejoicing. For when you meditate on rejoicing any discontent will be given up. Meditate on equanimity. 

For\marginnote{21.1} when you meditate on equanimity any repulsion will be given up. 

Meditate\marginnote{22.1} on ugliness. For when you meditate on ugliness any lust will be given up. 

Meditate\marginnote{23.1} on impermanence. For when you meditate on impermanence any conceit ‘I am’ will be given up. 

Develop\marginnote{24.1} mindfulness of breathing. When mindfulness of breathing is developed and cultivated it’s very fruitful and beneficial. And how is mindfulness of breathing developed and cultivated to be very fruitful and beneficial? 

It’s\marginnote{25.1} when a mendicant—gone to a wilderness, or to the root of a tree, or to an empty hut—sits down cross-legged, with their body straight, and establishes mindfulness right there. Just mindful, they breath in. Mindful, they breath out. 

When\marginnote{26.1} breathing in heavily they know: ‘I’m breathing in heavily.’ When breathing out heavily they know: ‘I’m breathing out heavily.’ When breathing in lightly they know: ‘I’m breathing in lightly.’ When breathing out lightly they know: ‘I’m breathing out lightly.’ They practice breathing in experiencing the whole body. They practice breathing out experiencing the whole body. They practice breathing in stilling the body’s motion. They practice breathing out stilling the body’s motion. 

They\marginnote{27.1} practice breathing in experiencing rapture. They practice breathing out experiencing rapture. They practice breathing in experiencing bliss. They practice breathing out experiencing bliss. They practice breathing in experiencing these emotions. They practice breathing out experiencing these emotions. They practice breathing in stilling these emotions. They practice breathing out stilling these emotions. 

They\marginnote{28.1} practice breathing in experiencing the mind. They practice breathing out experiencing the mind. They practice breathing in gladdening the mind. They practice breathing out gladdening the mind. They practice breathing in immersing the mind. They practice breathing out immersing the mind. They practice breathing in freeing the mind. They practice breathing out freeing the mind. 

They\marginnote{29.1} practice breathing in observing impermanence. They practice breathing out observing impermanence. They practice breathing in observing fading away. They practice breathing out observing fading away. They practice breathing in observing cessation. They practice breathing out observing cessation. They practice breathing in observing letting go. They practice breathing out observing letting go. 

Mindfulness\marginnote{30.1} of breathing, when developed and cultivated in this way, is very fruitful and beneficial. When mindfulness of breathing is developed and cultivated in this way, even when the final breaths in and out cease, they are known, not unknown.” 

That\marginnote{30.3} is what the Buddha said. Satisfied, Venerable \textsanskrit{Rāhula} was happy with what the Buddha said. 

%
\section*{{\suttatitleacronym MN 63}{\suttatitletranslation The Shorter Discourse With Māluṅkya }{\suttatitleroot Cūḷamālukyasutta}}
\addcontentsline{toc}{section}{\tocacronym{MN 63} \toctranslation{The Shorter Discourse With Māluṅkya } \tocroot{Cūḷamālukyasutta}}
\markboth{The Shorter Discourse With Māluṅkya }{Cūḷamālukyasutta}
\extramarks{MN 63}{MN 63}

\scevam{So\marginnote{1.1} I have heard. }At one time the Buddha was staying near \textsanskrit{Sāvatthī} in Jeta’s Grove, \textsanskrit{Anāthapiṇḍika}’s monastery. 

Then\marginnote{2.1} as Venerable \textsanskrit{Māluṅkya} was in private retreat this thought came to his mind: 

“There\marginnote{2.2} are several convictions that the Buddha has left undeclared; he has set them aside and refused to comment on them. For example: the cosmos is eternal, or not eternal, or finite, or infinite; the soul and the body are the same thing, or they are different things; after death, a Realized One exists, or doesn’t exist, or both exists and doesn’t exist, or neither exists nor doesn’t exist. The Buddha does not give me a straight answer on these points. I don’t like that, and do not accept it. I’ll go to him and ask him about this. If he gives me a straight answer on any of these points, I will lead the spiritual life under him. If he does not give me a straight answer on any of these points, I shall resign the training and return to a lesser life.” 

Then\marginnote{3.1} in the late afternoon, \textsanskrit{Māluṅkya} came out of retreat and went to the Buddha. He bowed, sat down to one side, and told the Buddha of his thoughts. He then continued: 

“If\marginnote{3.2} the Buddha knows that the cosmos is eternal, please tell me. If you know that the cosmos is not eternal, tell me. If you don’t know whether the cosmos is eternal or not, then it is straightforward to simply say: ‘I neither know nor see.’ If you know that the world is finite, or infinite; that the soul and the body are the same thing, or they are different things; that after death, a Realized One exists, or doesn’t exist, or both exists and doesn’t exist, or neither exists nor doesn’t exist, please tell me. If you don’t know any of these things, then it is straightforward to simply say: ‘I neither know nor see.’” 

“What,\marginnote{4.1} \textsanskrit{Māluṅkyaputta}, did I ever say to you: ‘Come, \textsanskrit{Māluṅkyaputta}, lead the spiritual life under me, and I will declare these things to you’?” 

“No,\marginnote{4.4} sir.” 

“Or\marginnote{4.5} did you ever say to me: ‘Sir, I will lead the spiritual life under the Buddha, and the Buddha will declare these things to me’?” 

“No,\marginnote{4.8} sir.” 

“So\marginnote{4.9} it seems that I did not say to you: ‘Come, \textsanskrit{Māluṅkyaputta}, lead the spiritual life under me, and I will declare these things to you.’ And you never said to me: ‘Sir, I will lead the spiritual life under the Buddha, and the Buddha will declare these things to me.’ In that case, you silly man, are you really in a position to be abandoning anything? 

Suppose\marginnote{5.1} someone were to say this: ‘I will not lead the spiritual life under the Buddha until the Buddha declares to me that the cosmos is eternal, or that the cosmos is not eternal … or that after death a Realized One neither exists nor doesn’t exist.’ That would still remain undeclared by the Realized One, and meanwhile that person would die. 

Suppose\marginnote{5.6} a man was struck by an arrow thickly smeared with poison. His friends and colleagues, relatives and kin would get a field surgeon to treat him. But the man would say: ‘I won’t pull out this arrow as long as I don’t know whether the man who wounded me was an aristocrat, a brahmin, a merchant, or a worker.’ He’d say: ‘I won’t pull out this arrow as long as I don’t know the following things about the man who wounded me: his name and clan; whether he’s tall, short, or medium; whether his skin is black, brown, or tawny; and what village, town, or city he comes from. I won’t pull out this arrow as long as I don’t know whether the bow that wounded me is made of wood or cane; whether the bow-string is made of swallow-wort fibre, sunn hemp fibre, sinew, sanseveria fibre, or spurge fibre; whether the shaft is made from a bush or a plantation tree; whether the shaft was fitted with feathers from a vulture, a heron, a hawk, a peacock, or a stork; whether the shaft was bound with sinews of a cow, a buffalo, a swamp deer, or a gibbon; and whether the arrowhead was spiked, razor-tipped, barbed, made of iron or a calf’s tooth, or lancet-shaped.’ That man would still not have learned these things, and meanwhile they’d die. 

In\marginnote{5.31} the same way, suppose someone was to say: ‘I will not lead the spiritual life under the Buddha until the Buddha declares to me that the cosmos is eternal, or that the cosmos is not eternal … or that after death a Realized One neither exists nor doesn’t exist.’ That would still remain undeclared by the Realized One, and meanwhile that person would die. 

It’s\marginnote{6.1} not true that if there were the view ‘the cosmos is eternal’ there would be the living of the spiritual life. It’s not true that if there were the view ‘the cosmos is not eternal’ there would be the living of the spiritual life. When there is the view that the cosmos is eternal or that the cosmos is not eternal, there is rebirth, there is old age, there is death, and there is sorrow, lamentation, pain, sadness, and distress. And it is the defeat of these things in this very life that I advocate. It’s not true that if there were the view ‘the world is finite’ … ‘the world is infinite’ … ‘the soul and the body are the same thing’ … ‘the soul and the body are different things’ … ‘a Realized One exists after death’ … ‘a Realized One doesn’t exist after death’ … ‘a Realized One both exists and doesn’t exist after death’ … ‘a Realized One neither exists nor doesn’t exist after death’ there would be the living of the spiritual life. When there are any of these views there is rebirth, there is old age, there is death, and there is sorrow, lamentation, pain, sadness, and distress. And it is the defeat of these things in this very life that I advocate. 

So,\marginnote{7.1} \textsanskrit{Māluṅkyaputta}, you should remember what I have not declared as undeclared, and what I have declared as declared. And what have I not declared? I have not declared the following: ‘the cosmos is eternal,’ ‘the cosmos is not eternal,’ ‘the world is finite,’ ‘the world is infinite,’ ‘the soul and the body are the same thing,’ ‘the soul and the body are different things,’ ‘a Realized One exists after death,’ ‘a Realized One doesn’t exist after death,’ ‘a Realized One both exists and doesn’t exist after death,’ ‘a Realized One neither exists nor doesn’t exist after death.’ 

And\marginnote{8.1} why haven’t I declared these things? Because they aren’t beneficial or relevant to the fundamentals of the spiritual life. They don’t lead to disillusionment, dispassion, cessation, peace, insight, awakening, and extinguishment. That’s why I haven’t declared them. 

And\marginnote{9.1} what have I declared? I have declared the following: ‘this is suffering,’ ‘this is the origin of suffering,’ ‘this is the cessation of suffering,’ ‘this is the practice that leads to the cessation of suffering.’ 

And\marginnote{10.1} why have I declared these things? Because they are beneficial and relevant to the fundamentals of the spiritual life. They lead to disillusionment, dispassion, cessation, peace, insight, awakening, and extinguishment. That’s why I have declared them. So, \textsanskrit{Māluṅkyaputta}, you should remember what I have not declared as undeclared, and what I have declared as declared.” 

That\marginnote{10.6} is what the Buddha said. Satisfied, Venerable \textsanskrit{Māluṅkyaputta} was happy with what the Buddha said. 

%
\section*{{\suttatitleacronym MN 64}{\suttatitletranslation The Longer Discourse With Māluṅkya }{\suttatitleroot Mahāmālukyasutta}}
\addcontentsline{toc}{section}{\tocacronym{MN 64} \toctranslation{The Longer Discourse With Māluṅkya } \tocroot{Mahāmālukyasutta}}
\markboth{The Longer Discourse With Māluṅkya }{Mahāmālukyasutta}
\extramarks{MN 64}{MN 64}

\scevam{So\marginnote{1.1} I have heard. }At one time the Buddha was staying near \textsanskrit{Sāvatthī} in Jeta’s Grove, \textsanskrit{Anāthapiṇḍika}’s monastery. There the Buddha addressed the mendicants, “Mendicants!” 

“Venerable\marginnote{1.5} sir,” they replied. The Buddha said this: 

“Mendicants,\marginnote{2.1} do you remember the five lower fetters that I taught?” 

When\marginnote{2.2} he said this, Venerable \textsanskrit{Māluṅkyaputta} said to him, “Sir, I remember them.” 

“But\marginnote{2.4} how do you remember them?” 

“I\marginnote{2.5} remember the lower fetters taught by the Buddha as follows: identity view, doubt, misapprehension of precepts and observances, sensual desire, and ill will. That’s how I remember the five lower fetters taught by the Buddha.” 

“Who\marginnote{3.1} on earth do you remember being taught the five lower fetters in that way? Wouldn’t the wanderers who follow other paths fault you using the simile of the infant? For a little baby doesn’t even have a concept of ‘identity’, so how could identity view possibly arise in them? Yet the underlying tendency to identity view still lies within them. A little baby doesn’t even have a concept of ‘teachings’, so how could doubt about the teachings possibly arise in them? Yet the underlying tendency to doubt still lies within them. A little baby doesn’t even have a concept of ‘precepts’, so how could misapprehension of precepts and observances possibly arise in them? Yet the underlying tendency to misapprehension of precepts and observances still lies within them. A little baby doesn’t even have a concept of ‘sensual pleasures’, so how could desire for sensual pleasures possibly arise in them? Yet the underlying tendency to sensual desire still lies within them. A little baby doesn’t even have a concept of ‘sentient beings’, so how could ill will for sentient beings possibly arise in them? Yet the underlying tendency to ill will still lies within them. Wouldn’t the wanderers who follow other paths fault you using the simile of the infant?” 

When\marginnote{4.1} he said this, Venerable Ānanda said to the Buddha, “Now is the time, Blessed One! Now is the time, Holy One! May the Buddha teach the five lower fetters. The mendicants will listen and remember it.” 

“Well\marginnote{4.4} then, Ānanda, listen and pay close attention, I will speak.” 

“Yes,\marginnote{4.5} sir,” Ānanda replied. The Buddha said this: 

“Ānanda,\marginnote{5.1} take an uneducated ordinary person who has not seen the noble ones, and is neither skilled nor trained in the teaching of the noble ones. They’ve not seen good persons, and are neither skilled nor trained in the teaching of the good persons. Their heart is overcome and mired in identity view, and they don’t truly understand the escape from identity view that has arisen. That identity view is reinforced in them, not eliminated: it is a lower fetter. 

Their\marginnote{5.5} heart is overcome and mired in doubt, and they don’t truly understand the escape from doubt that has arisen. That doubt is reinforced in them, not eliminated: it is a lower fetter. 

Their\marginnote{5.8} heart is overcome and mired in misapprehension of precepts and observances, and they don’t truly understand the escape from misapprehension of precepts and observances that has arisen. That misapprehension of precepts and observances is reinforced in them, not eliminated: it is a lower fetter. 

Their\marginnote{5.11} heart is overcome and mired in sensual desire, and they don’t truly understand the escape from sensual desire that has arisen. That sensual desire is reinforced in them, not eliminated: it is a lower fetter. 

Their\marginnote{5.14} heart is overcome and mired in ill will, and they don’t truly understand the escape from ill will that has arisen. That ill will is reinforced in them, not eliminated: it is a lower fetter. 

But\marginnote{6.1} an educated noble disciple has seen the noble ones, and is skilled and trained in the teaching of the noble ones. They’ve seen good persons, and are skilled and trained in the teaching of the good persons. Their heart is not overcome and mired in identity view, and they truly understand the escape from identity view that has arisen. That identity view, along with any underlying tendency to it, is given up in them. 

Their\marginnote{6.4} heart is not overcome and mired in doubt, and they truly understand the escape from doubt that has arisen. That doubt, along with any underlying tendency to it, is given up in them. 

Their\marginnote{6.7} heart is not overcome and mired in misapprehension of precepts and observances, and they truly understand the escape from misapprehension of precepts and observances that has arisen. That misapprehension of precepts and observances, along with any underlying tendency to it, is given up in them. 

Their\marginnote{6.10} heart is not overcome and mired in sensual desire, and they truly understand the escape from sensual desire that has arisen. That sensual desire, along with any underlying tendency to it, is given up in them. 

Their\marginnote{6.13} heart is not overcome and mired in ill will, and they truly understand the escape from ill will that has arisen. That ill will, along with any underlying tendency to it, is given up in them. 

There\marginnote{6.16} is a path and a practice for giving up the five lower fetters. It’s not possible to know or see or give up the five lower fetters without relying on that path and that practice. Suppose there was a large tree standing with heartwood. It’s not possible to cut out the heartwood without having cut through the bark and the softwood. In the same way, there is a path and a practice for giving up the five lower fetters. It’s not possible to know or see or give up the five lower fetters without relying on that path and that practice. 

There\marginnote{7.1} is a path and a practice for giving up the five lower fetters. It is possible to know and see and give up the five lower fetters by relying on that path and that practice. 

Suppose\marginnote{8.1} there was a large tree standing with heartwood. It is possible to cut out the heartwood after having cut through the bark and the softwood. In the same way, there is a path and a practice for giving up the five lower fetters. It is possible to know and see and give up the five lower fetters by relying on that path and that practice. Suppose the river Ganges was full to the brim so a crow could drink from it. Then along comes a feeble person, who thinks: ‘By swimming with my arms I’ll safely cross over to the far shore of the Ganges.’ But they’re not able to do so. In the same way, when the Dhamma is being taught for the cessation of identity view, someone whose mind isn’t eager, confident, settled, and decided should be regarded as being like that feeble person. Suppose the river Ganges was full to the brim so a crow could drink from it. Then along comes a strong person, who thinks: ‘By swimming with my arms I’ll safely cross over to the far shore of the Ganges.’ And they are able to do so. 

In\marginnote{8.13} the same way, when the Dhamma is being taught for the cessation of identity view, someone whose mind is eager, confident, settled, and decided should be regarded as being like that strong person. 

And\marginnote{9.1} what, Ānanda, is the path and the practice for giving up the five lower fetters? It’s when a mendicant—due to the seclusion from attachments, the giving up of unskillful qualities, and the complete settling of physical discomfort—quite secluded from sensual pleasures, secluded from unskillful qualities, enters and remains in the first absorption, which has the rapture and bliss born of seclusion, while placing the mind and keeping it connected. They contemplate the phenomena there—included in form, feeling, perception, choices, and consciousness—as impermanent, as suffering, as diseased, as a boil, as a dart, as misery, as an affliction, as alien, as falling apart, as empty, as not-self. They turn their mind away from those things, and apply it to the deathless element: ‘This is peaceful; this is sublime—that is, the stilling of all activities, the letting go of all attachments, the ending of craving, cessation, extinguishment.’ Abiding in that they attain the ending of defilements. If they don’t attain the ending of defilements, with the ending of the five lower fetters they’re reborn spontaneously, because of their passion and love for that meditation. They are extinguished there, and are not liable to return from that world. This is the path and the practice for giving up the five lower fetters. 

Furthermore,\marginnote{10{-}12.1} as the placing of the mind and keeping it connected are stilled, a mendicant enters and remains in the second absorption … third absorption … fourth absorption. They contemplate the phenomena there as impermanent … They turn their mind away from those things … If they don’t attain the ending of defilements, they’re reborn spontaneously … and are not liable to return from that world. This too is the path and the practice for giving up the five lower fetters. 

Furthermore,\marginnote{13.1} a mendicant, going totally beyond perceptions of form, with the ending of perceptions of impingement, not focusing on perceptions of diversity, aware that ‘space is infinite’, enters and remains in the dimension of infinite space. They contemplate the phenomena there as impermanent … They turn their mind away from those things … If they don’t attain the ending of defilements, they’re reborn spontaneously … and are not liable to return from that world. This too is the path and the practice for giving up the five lower fetters. 

Furthermore,\marginnote{14.1} a mendicant, going totally beyond the dimension of infinite space, aware that ‘consciousness is infinite’, enters and remains in the dimension of infinite consciousness. They contemplate the phenomena there as impermanent … They turn their mind away from those things … If they don’t attain the ending of defilements, they’re reborn spontaneously … and are not liable to return from that world. This too is the path and the practice for giving up the five lower fetters. 

Furthermore,\marginnote{15.1} a mendicant, going totally beyond the dimension of infinite consciousness, aware that ‘there is nothing at all’, enters and remains in the dimension of nothingness. They contemplate the phenomena there as impermanent … They turn their mind away from those things … If they don’t attain the ending of defilements, they’re reborn spontaneously … and are not liable to return from that world. This too is the path and the practice for giving up the five lower fetters.” 

“Sir,\marginnote{16.1} if this is the path and the practice for giving up the five lower fetters, how come some mendicants here are released in heart while others are released by wisdom?” 

“In\marginnote{16.2} that case, I say it is the diversity of their faculties.” 

That\marginnote{16.3} is what the Buddha said. Satisfied, Venerable Ānanda was happy with what the Buddha said. 

%
\section*{{\suttatitleacronym MN 65}{\suttatitletranslation With Bhaddāli }{\suttatitleroot Bhaddālisutta}}
\addcontentsline{toc}{section}{\tocacronym{MN 65} \toctranslation{With Bhaddāli } \tocroot{Bhaddālisutta}}
\markboth{With Bhaddāli }{Bhaddālisutta}
\extramarks{MN 65}{MN 65}

\scevam{So\marginnote{1.1} I have heard. }At one time the Buddha was staying near \textsanskrit{Sāvatthī} in Jeta’s Grove, \textsanskrit{Anāthapiṇḍika}’s monastery. There the Buddha addressed the mendicants, “Mendicants!” 

“Venerable\marginnote{1.5} sir,” they replied. The Buddha said this: 

“Mendicants,\marginnote{2.1} I eat my food in one sitting per day. Doing so, I find that I’m healthy and well, nimble, strong, and living comfortably. You too should eat your food in one sitting per day. Doing so, you’ll find that you’re healthy and well, nimble, strong, and living comfortably.” 

When\marginnote{3.1} he said this, Venerable \textsanskrit{Bhaddāli} said to the Buddha, “Sir, I’m not going to try to eat my food in one sitting per day. For when eating once a day I might feel remorse and regret.” 

“Well\marginnote{4.1} then, \textsanskrit{Bhaddāli}, eat one part of the meal in the place where you’re invited, and bring the rest back to eat. Eating this way, too, you will sustain yourself.” 

“Sir,\marginnote{4.3} I’m not going to try to eat that way, either. For when eating that way I might also feel remorse and regret.” Then, as this rule was being laid down by the Buddha and the \textsanskrit{Saṅgha} was undertaking it, \textsanskrit{Bhaddāli} announced he would not try to keep it. Then for the whole of that three months \textsanskrit{Bhaddāli} did not present himself in the presence of the Buddha, as happens when someone doesn’t fulfill the training according to the Teacher’s instructions. 

At\marginnote{5.1} that time several mendicants were making a robe for the Buddha, thinking that when his robe was finished and the three months of the rains residence had passed the Buddha would set out wandering. 

Then\marginnote{6.1} \textsanskrit{Bhaddāli} went up to those mendicants, and exchanged greetings with them. When the greetings and polite conversation were over, he sat down to one side. The mendicants said to \textsanskrit{Bhaddāli}, “Reverend \textsanskrit{Bhaddāli}, this robe is being made for the Buddha. When it’s finished and the three months of the rains residence have passed the Buddha will set out wandering. Come on, \textsanskrit{Bhaddāli}, learn your lesson. Don’t make it hard for yourself later on.” 

“Yes,\marginnote{7.1} reverends,” \textsanskrit{Bhaddāli} replied. He went to the Buddha, bowed, sat down to one side, and said to him, “I have made a mistake, sir. It was foolish, stupid, and unskillful of me that, as this rule was being laid down by the Buddha and the \textsanskrit{Saṅgha} was undertaking it, I announced I would not try to keep it. Please, sir, accept my mistake for what it is, so I will restrain myself in future.” 

“Indeed,\marginnote{8.1} \textsanskrit{Bhaddāli}, you made a mistake. It was foolish, stupid, and unskillful of you that, as this rule was being laid down by the Buddha and the \textsanskrit{Saṅgha} was undertaking it, you announced you would not try to keep it. 

And\marginnote{9.1} you didn’t realize this situation: ‘The Buddha is staying in \textsanskrit{Sāvatthī}, and he’ll know me as the mendicant named \textsanskrit{Bhaddāli} who doesn’t fulfill the training according to the Teacher’s instructions.’ 

And\marginnote{9.5} you didn’t realize this situation: ‘Several monks have commenced the rains retreat in \textsanskrit{Sāvatthī} … several nuns have commenced the rains retreat in \textsanskrit{Sāvatthī} … several laymen reside in \textsanskrit{Sāvatthī} … several laywomen reside in \textsanskrit{Sāvatthī}, and they’ll know me as the mendicant named \textsanskrit{Bhaddāli} who doesn’t fulfill the training according to the Teacher’s instructions. … 

Several\marginnote{9.21} ascetics and brahmins who follow various other paths have commenced the rains retreat in \textsanskrit{Sāvatthī}, and they’ll know me as the mendicant named \textsanskrit{Bhaddāli}, one of the senior disciples of Gotama, who doesn’t fulfill the training according to the Teacher’s instructions.’ You also didn’t realize this situation.” 

“I\marginnote{10.1} made a mistake, sir. It was foolish, stupid, and unskillful of me that, as this rule was being laid down by the Buddha and the \textsanskrit{Saṅgha} was undertaking it, I announced I would not try to keep it. Please, sir, accept my mistake for what it is, so I will restrain myself in future.” 

“Indeed,\marginnote{10.3} \textsanskrit{Bhaddāli}, you made a mistake. It was foolish, stupid, and unskillful of you that, as this rule was being laid down by the Buddha and the \textsanskrit{Saṅgha} was undertaking it, you announced you would not try to keep it. 

What\marginnote{11.1} do you think, \textsanskrit{Bhaddāli}? Suppose I was to say this to a mendicant who is freed both ways: ‘Please, mendicant, be a bridge for me to cross over the mud.’ Would they cross over themselves, or struggle to get out of it, or just say no?” 

“No,\marginnote{11.4} sir.” 

“What\marginnote{11.5} do you think, \textsanskrit{Bhaddāli}? Suppose I was to say the same thing to a mendicant who is freed by wisdom, or a personal witness, or attained to view, or freed by faith, or a follower of the teachings, or a follower by faith: ‘Please, mendicant, be a bridge for me to cross over the mud.’ Would they cross over themselves, or struggle to get out of it, or just say no?” 

“No,\marginnote{11.13} sir.” 

“What\marginnote{12.1} do you think, \textsanskrit{Bhaddāli}? At that time were you freed both ways, freed by wisdom, a personal witness, attained to view, freed by faith, a follower of the teachings, or a follower by faith?” 

“No,\marginnote{12.3} sir.” 

“Weren’t\marginnote{12.4} you void, hollow, and mistaken?” 

“Yes,\marginnote{13.1} sir.” 

“I\marginnote{13.2} made a mistake, sir. … Please, sir, accept my mistake for what it is, so I will restrain myself in future.” 

“Indeed,\marginnote{13.4} \textsanskrit{Bhaddāli}, you made a mistake. … But since you have recognized your mistake for what it is, and have dealt with it properly, I accept it. For it is growth in the training of the Noble One to recognize a mistake for what it is, deal with it properly, and commit to restraint in the future. 

\textsanskrit{Bhaddāli},\marginnote{14.1} take a mendicant who doesn’t fulfill the training according to the Teacher’s instructions. They think, ‘Why don’t I frequent a secluded lodging—a wilderness, the root of a tree, a hill, a ravine, a mountain cave, a charnel ground, a forest, the open air, a heap of straw. Hopefully I’ll realize a superhuman distinction in knowledge and vision worthy of the noble ones.’ So they frequent a secluded lodging. While they’re living withdrawn, they’re reprimanded by the Teacher, by sensible spiritual companions after examination, by deities, and by themselves. Being reprimanded in this way, they don’t realize any superhuman distinction in knowledge and vision worthy of the noble ones. Why is that? Because that’s how it is when someone doesn’t fulfill the training according to the Teacher’s instructions. 

But\marginnote{15.1} take a mendicant who does fulfill the training according to the Teacher’s instructions. They think, ‘Why don’t I frequent a secluded lodging—a wilderness, the root of a tree, a hill, a ravine, a mountain cave, a charnel ground, a forest, the open air, a heap of straw. Hopefully I’ll realize a superhuman distinction in knowledge and vision worthy of the noble ones.’ They frequent a secluded lodging—a wilderness, the root of a tree, a hill, a ravine, a mountain cave, a charnel ground, a forest, the open air, a heap of straw. While they’re living withdrawn, they’re not reprimanded by the Teacher, by sensible spiritual companions after examination, by deities, or by themselves. Not being reprimanded in this way, they realize a superhuman distinction in knowledge and vision worthy of the noble ones. 

Quite\marginnote{16.1} secluded from sensual pleasures, secluded from unskillful qualities, they enter and remain in the first absorption, which has the rapture and bliss born of seclusion, while placing the mind and keeping it connected. Why is that? Because that’s what happens when someone fulfills the training according to the Teacher’s instructions. 

Furthermore,\marginnote{17.1} as the placing of the mind and keeping it connected are stilled, a mendicant enters and remains in the second absorption, which has the rapture and bliss born of immersion, with internal clarity and confidence, and unified mind, without placing the mind and keeping it connected. Why is that? Because that’s what happens when someone fulfills the training according to the Teacher’s instructions. 

Furthermore,\marginnote{17.4} with the fading away of rapture, a mendicant enters and remains in the third absorption, where they meditate with equanimity, mindful and aware, personally experiencing the bliss of which the noble ones declare, ‘Equanimous and mindful, one meditates in bliss.’ Why is that? Because that’s what happens when someone fulfills the training according to the Teacher’s instructions. 

Furthermore,\marginnote{17.7} giving up pleasure and pain, and ending former happiness and sadness, a mendicant enters and remains in the fourth absorption, without pleasure or pain, with pure equanimity and mindfulness. Why is that? Because that’s what happens when someone fulfills the training according to the Teacher’s instructions. 

When\marginnote{18.1} their mind has become immersed in \textsanskrit{samādhi} like this—purified, bright, flawless, rid of corruptions, pliable, workable, steady, and imperturbable—they extend it toward recollection of past lives. They recollect many kinds of past lives, that is, one, two, three, four, five, ten, twenty, thirty, forty, fifty, a hundred, a thousand, a hundred thousand rebirths; many eons of the world contracting, many eons of the world expanding, many eons of the world contracting and expanding. … They recollect their many kinds of past lives, with features and details. Why is that? Because that’s what happens when someone fulfills the training according to the Teacher’s instructions. 

When\marginnote{19.1} their mind has become immersed in \textsanskrit{samādhi} like this—purified, bright, flawless, rid of corruptions, pliable, workable, steady, and imperturbable—they extend it toward knowledge of the death and rebirth of sentient beings. With clairvoyance that is purified and superhuman, they see sentient beings passing away and being reborn—inferior and superior, beautiful and ugly, in a good place or a bad place. They understand how sentient beings are reborn according to their deeds: ‘These dear beings did bad things by way of body, speech, and mind. … They’re reborn in the underworld, hell. These dear beings, however, did good things by way of body, speech, and mind. … they’re reborn in a good place, a heavenly realm.’ And so, with clairvoyance that is purified and superhuman … they understand how sentient beings are reborn according to their deeds. Why is that? Because that’s what happens when someone fulfills the training according to the Teacher’s instructions. 

When\marginnote{20.1} their mind has become immersed in \textsanskrit{samādhi} like this—purified, bright, flawless, rid of corruptions, pliable, workable, steady, and imperturbable—they extend it toward knowledge of the ending of defilements. They truly understand: ‘This is suffering’ … ‘This is the origin of suffering’ … ‘This is the cessation of suffering’ … ‘This is the practice that leads to the cessation of suffering’. They truly understand: ‘These are defilements’ … ‘This is the origin of defilements’ … ‘This is the cessation of defilements’ … ‘This is the practice that leads to the cessation of defilements’. 

Knowing\marginnote{21.1} and seeing like this, their mind is freed from the defilements of sensuality, desire to be reborn, and ignorance. When they’re freed, they know they’re freed. 

They\marginnote{21.3} understand: ‘Rebirth is ended, the spiritual journey has been completed, what had to be done has been done, there is no return to any state of existence.’ Why is that? Because that’s what happens when someone fulfills the training according to the Teacher’s instructions.” 

When\marginnote{22.1} he said this, Venerable \textsanskrit{Bhaddāli} said to the Buddha, “What is the cause, sir, what is the reason why they punish some monk, repeatedly pressuring him? And what is the cause, what is the reason why they don’t similarly punish another monk, repeatedly pressuring him?” 

“Take\marginnote{23.1} a monk who is a frequent offender with many offenses. When admonished by the monks, he dodges the issue, distracting the discussion with irrelevant points. He displays annoyance, hate, and bitterness. He doesn’t proceed properly, he doesn’t fall in line, he doesn’t proceed to get past it, and he doesn’t say: ‘I’ll do what pleases the \textsanskrit{Saṅgha}.’ In such a case, the monks say: ‘Reverends, this monk is a frequent offender, with many offenses. When admonished by the monks, he dodges the issue, distracting the discussion with irrelevant points. He displays annoyance, hate, and bitterness. He doesn’t proceed properly, he doesn’t fall in line, he doesn’t proceed to get past it, and he doesn’t say: “I’ll do what pleases the \textsanskrit{Saṅgha}.” It’d be good for the venerables to examine this monk in such a way that this disciplinary issue is not quickly settled.’ And that’s what they do. 

Take\marginnote{24.1} some other monk who is a frequent offender with many offenses. When admonished by the monks, he doesn’t dodge the issue, distracting the discussion with irrelevant points. He doesn’t display annoyance, hate, and bitterness. He proceeds properly, he falls in line, he proceeds to get past it, and he says: ‘I’ll do what pleases the \textsanskrit{Saṅgha}.’ In such a case, the monks say: ‘Reverends, this monk is a frequent offender, with many offenses. When admonished by the monks, he doesn’t dodge the issue, distracting the discussion with irrelevant points. He doesn’t display annoyance, hate, and bitterness. He proceeds properly, he falls in line, he proceeds to get past it, and he says: ‘I’ll do what pleases the \textsanskrit{Saṅgha}.’ It’d be good for the venerables to examine this monk in such a way that this disciplinary issue is quickly settled.’ And that’s what they do. 

Take\marginnote{25.1} some other monk who is an occasional offender without many offenses. When admonished by the monks, he dodges the issue … In such a case, the monks say: ‘Reverends, this monk is an occasional offender without many offenses. When admonished by the monks, he dodges the issue … It’d be good for the venerables to examine this monk in such a way that this disciplinary issue is not quickly settled.’ And that’s what they do. 

Take\marginnote{26.1} some other monk who is an occasional offender without many offenses. When admonished by the monks, he doesn’t dodge the issue … In such a case, the monks say: ‘Reverends, this monk is an occasional offender without many offenses. When admonished by the monks, he doesn’t dodge the issue … It’d be good for the venerables to examine this monk in such a way that this disciplinary issue is quickly settled.’ And that’s what they do. 

Take\marginnote{27.1} some other monk who gets by with mere faith and love. In such a case, the monks say: ‘Reverends, this monk gets by with mere faith and love. If we punish him, repeatedly pressuring him—no, let him not lose what little faith and love he has!’ 

Suppose\marginnote{28.1} there was a person with one eye. Their friends and colleagues, relatives and kin would protect that one eye: ‘Let them not lose the one eye that they have!’ In the same way, some monk gets by with mere faith and love. In such a case, the monks say: ‘Reverends, this monk gets by with mere faith and love. If we punish him, repeatedly pressuring him—no, let him not lose what little faith and love he has!’ This is the cause, this is the reason why they punish some monk, repeatedly pressuring him. And this is the cause, this is the reason why they don’t similarly punish another monk, repeatedly pressuring him.” 

“What\marginnote{29.1} is the cause, sir, what is the reason why there used to be fewer training rules but more enlightened mendicants? And what is the cause, what is the reason why these days there are more training rules and fewer enlightened mendicants?” 

“That’s\marginnote{30.1} how it is, \textsanskrit{Bhaddāli}. When sentient beings are in decline and the true teaching is disappearing there are more training rules and fewer enlightened mendicants. The Teacher doesnʼt lay down training rules for disciples as long as certain defiling influences have not appeared in the \textsanskrit{Saṅgha}. But when such defiling influences appear in the \textsanskrit{Saṅgha}, the Teacher lays down training rules for disciples to protect against them. 

And\marginnote{31.1} they donʼt appear until the \textsanskrit{Saṅgha} has attained a great size, an abundance of material support and fame, learning, and seniority. But when the \textsanskrit{Saṅgha} has attained these things, then such defiling influences appear in the \textsanskrit{Saṅgha}, and the Teacher lays down training rules for disciples to protect against them. 

There\marginnote{32.1} were only a few of you there at the time when I taught the exposition of the teaching on the simile of the thoroughbred colt. Do you remember that, \textsanskrit{Bhaddāli}?” 

“No,\marginnote{32.3} sir.” 

“What\marginnote{32.4} do you believe the reason for that is?” 

“Sir,\marginnote{32.5} it’s surely because for a long time now I haven’t fulfilled the training according to the Teacher’s instructions.” 

“That’s\marginnote{32.6} not the only reason, \textsanskrit{Bhaddāli}. Rather, for a long time I have comprehended your mind and known: ‘While I’m teaching, this silly man doesn’t pay heed, pay attention, engage wholeheartedly, or lend an ear.’ Still, \textsanskrit{Bhaddāli}, I shall teach the exposition of the teaching on the simile of the thoroughbred colt. Listen and pay close attention, I will speak.” 

“Yes,\marginnote{32.11} sir,” \textsanskrit{Bhaddāli} replied. The Buddha said this: 

“Suppose\marginnote{33.1} a deft horse trainer were to obtain a fine thoroughbred. First of all he’d make it get used to wearing the bit. Because it has not done this before, it still resorts to some tricks, dodges, and evasions. But with regular and gradual practice it quells that bad habit. 

When\marginnote{33.4} it has done this, the horse trainer next makes it get used to wearing the harness. Because it has not done this before, it still resorts to some tricks, dodges, and evasions. But with regular and gradual practice it quells that bad habit. 

When\marginnote{33.7} it has done this, the horse trainer next makes it get used to walking in procession, circling, prancing, galloping, charging, the protocols and traditions of court, and in the very best speed, fleetness, and friendliness. Because it has not done this before, it still resorts to some tricks, dodges, and evasions. But with regular and gradual practice it quells that bad habit. 

When\marginnote{33.10} it has done this, the horse trainer next rewards it with a grooming and a rub down. A fine royal thoroughbred with these ten factors is worthy of a king, fit to serve a king, and reckoned as a factor of kingship. 

In\marginnote{34.1} the same way, a mendicant with ten qualities is worthy of offerings dedicated to the gods, worthy of hospitality, worthy of a religious donation, worthy of veneration with joined palms, and is the supreme field of merit for the world. What ten? It’s when a mendicant has an adept’s right view, right thought, right speech, right action, right livelihood, right effort, right mindfulness, right immersion, right knowledge, and right freedom. A mendicant with these ten qualities is worthy of offerings dedicated to the gods, worthy of hospitality, worthy of a religious donation, worthy of veneration with joined palms, and is the supreme field of merit for the world.” 

That\marginnote{34.5} is what the Buddha said. Satisfied, Venerable \textsanskrit{Bhaddāli} was happy with what the Buddha said. 

%
\section*{{\suttatitleacronym MN 66}{\suttatitletranslation The Simile of the Quail }{\suttatitleroot Laṭukikopamasutta}}
\addcontentsline{toc}{section}{\tocacronym{MN 66} \toctranslation{The Simile of the Quail } \tocroot{Laṭukikopamasutta}}
\markboth{The Simile of the Quail }{Laṭukikopamasutta}
\extramarks{MN 66}{MN 66}

\scevam{So\marginnote{1.1} I have heard. }At one time the Buddha was staying in the land of the Northern \textsanskrit{Āpaṇas}, near the town of theirs named \textsanskrit{Āpaṇa}. 

Then\marginnote{2.1} the Buddha robed up in the morning and, taking his bowl and robe, entered \textsanskrit{Āpaṇa} for alms. He wandered for alms in \textsanskrit{Āpaṇa}. After the meal, on his return from almsround, he went to a certain forest grove for the day’s meditation. Having plunged deep into it, he sat at the root of a certain tree for the day’s meditation. 

Venerable\marginnote{3.1} \textsanskrit{Udāyī} also robed up in the morning and, taking his bowl and robe, entered \textsanskrit{Āpaṇa} for alms. He wandered for alms in \textsanskrit{Āpaṇa}. After the meal, on his return from almsround, he went to a certain forest grove for the day’s meditation. Having plunged deep into it, he sat at the root of a certain tree for the day’s meditation. Then as Venerable \textsanskrit{Udāyī} was in private retreat this thought came to his mind: 

“The\marginnote{4.1} Buddha has rid us of so many things that bring suffering and gifted us so many things that bring happiness! He has rid us of so many unskillful things and gifted us so many skillful things!” 

Then\marginnote{5.1} in the late afternoon, \textsanskrit{Udāyī} came out of retreat and went to the Buddha. He bowed, sat down to one side, and said to him: 

“Just\marginnote{6.1} now, sir, as I was in private retreat this thought came to mind: ‘The Buddha has rid us of so many things that bring suffering and gifted us so many things that bring happiness! He has rid us of so many unskillful things and gifted us so many skillful things!’ 

For\marginnote{6.4} we used to eat in the evening, the morning, and at the wrong time of day. But then there came a time when the Buddha addressed the mendicants, saying, ‘Please, mendicants, give up that meal at the wrong time of day.’ At that, sir, we became sad and upset, ‘But these faithful householders give us a variety of delicious foods at the wrong time of day. And the Blessed One tells us to give it up! The Holy One tells us to let it go!’ But when we considered our love and respect for the Buddha, and our sense of conscience and prudence, we gave up that meal at the wrong time of day. Then we ate in the evening and the morning. 

But\marginnote{6.11} then there came a time when the Buddha addressed the mendicants, saying, ‘Please, mendicants, give up that meal at the wrong time of night.’ At that, sir, we became sad and upset, ‘But that’s considered the more delicious of the two meals. And the Blessed One tells us to give it up! The Holy One tells us to let it go!’ Once it so happened that a certain person got some soup during the day. He said, ‘Come, let’s set this aside; we’ll enjoy it together this evening.’ Nearly all meals are prepared at night, only a few in the day. But when we considered our love and respect for the Buddha, and our sense of conscience and prudence, we gave up that meal at the wrong time of night. 

In\marginnote{6.19} the past, mendicants went wandering for alms in the dark of the night. They walked into a swamp, or fell into a sewer, or collided with a thorn bush, or collided with a sleeping cow, or encountered youths escaping a crime or on their way to commit one, or were invited by a female to commit a lewd act. 

Once\marginnote{6.20} it so happened that I wandered for alms in the dark of the night. A woman washing a pot saw me by a flash of lightning. Startled, she cried out, ‘Bloody hell! A goblin’s upon me!’ 

When\marginnote{6.24} she said this, I said to her, ‘Sister, I am no goblin. I’m a mendicant waiting for alms.’ 

‘Then\marginnote{6.27} it’s a mendicant whose ma died and pa died! You’d be better off having your belly sliced open with a meat cleaver than to wander for alms in the dark of night for the sake of your belly.’ 

Recollecting\marginnote{6.29} that, I thought, ‘The Buddha has rid us of so many things that bring suffering and gifted us so many things that bring happiness! He has rid us of so many unskillful things and gifted us so many skillful things!’” 

“This\marginnote{7.1} is exactly what happens when some foolish people are told by me to give something up. They say, ‘What, such a trivial, insignificant thing as this? This ascetic is much too strict!’ They don’t give it up, and they nurse bitterness towards me; and for the mendicants who want to train, that becomes a strong, firm, stout bond, a tie that has not rotted, and a heavy yoke. 

Suppose\marginnote{8.1} a quail was tied with a vine, and was waiting there to be injured, caged, or killed. Would it be right to say that, for that quail, that vine is weak, feeble, rotten, and insubstantial?” 

“No,\marginnote{8.5} sir. For that quail, that vine is a strong, firm, stout bond, a tie that has not rotted, and a heavy yoke.” 

“In\marginnote{8.7} the same way, when some foolish people are told by me to give something up, they say, ‘What, such a trivial, insignificant thing as this? This ascetic is much too strict!’ They don’t give it up, and they nurse bitterness towards me; and for the mendicants who want to train, that becomes a strong, firm, stout bond, a tie that has not rotted, and a heavy yoke. 

But\marginnote{9.1} when some gentlemen are told by me to give something up, they say, ‘What, we just have to give up such a trivial, insignificant thing as this, when the Blessed One tells us to give it up, the Holy One tells us to let it go?’ They give it up, and they don’t nurse bitterness towards me; and when the mendicants who want to train have given that up, they live relaxed, unruffled, surviving on charity, their hearts free as a wild deer. For them, that bond is weak, feeble, rotten, and insubstantial. 

Suppose\marginnote{10.1} there was a royal bull elephant with tusks like chariot-poles, able to draw a heavy load, pedigree and battle-hardened. And it was bound with a strong harness. But just by twisting its body a little, it would break apart its bonds and go wherever it wants. Would it be right to say that, for that bull elephant, that strong harness is a strong, firm, stout bond, a tie that has not rotted, and a heavy yoke?” 

“No,\marginnote{10.5} sir. For that bull elephant, that strong harness is weak, feeble, rotten, and insubstantial.” 

“In\marginnote{10.7} the same way, when some gentlemen are told by me to give something up, they say, ‘What, we just have to give up such a trivial, insignificant thing as this, when the Blessed One tells us to give it up, the Holy One tells us to let it go?’ They give it up, and they don’t nurse bitterness towards me; and when the mendicants who want to train have given that up, they live relaxed, unruffled, surviving on charity, their hearts free as a wild deer. For them, that bond is weak, feeble, rotten, and insubstantial. 

Suppose\marginnote{11.1} there was a poor man, with few possessions and little wealth. He had a single broken-down hovel open to the crows, not the best sort; a single broken-down couch, not the best sort; a single pot for storing grain, not the best sort; and a single wifey, not the best sort. He’d see a mendicant sitting in meditation in the cool shade, their hands and feet well washed after eating a delectable meal. He’d think, ‘The ascetic life is so very pleasant! The ascetic life is so very skillful! If only I could shave off my hair and beard, dress in ocher robes, and go forth from the lay life to homelessness.’ But he’s not able to give up his broken-down hovel, his broken-down couch, his pot for storing grain, or his wifey—none of which are the best sort—in order to go forth. Would it be right to say that, for that man, those bonds are weak, feeble, rotten, and insubstantial?” 

“No,\marginnote{11.12} sir. For that man, they are a strong, firm, stout bond, a tie that has not rotted, and a heavy yoke.” 

“In\marginnote{11.15} the same way, when some foolish people are told by me to give something up, they say, ‘What, such a trivial, insignificant thing as this? This ascetic is much too strict!’ They don’t give it up, and they nurse bitterness towards me; and for the mendicants who want to train, that becomes a strong, firm, stout bond, a tie that has not rotted, and a heavy yoke. 

Suppose\marginnote{12.1} there was a rich man, affluent, and wealthy. He had a vast amount of gold coin, grain, fields, lands, wives, and male and female bondservants. He’d see a mendicant sitting in meditation in the cool shade, their hands and feet well washed after eating a delectable meal. He’d think, ‘The ascetic life is so very pleasant! The ascetic life is so very skillful! If only I could shave off my hair and beard, dress in ocher robes, and go forth from the lay life to homelessness.’ And he is able to give up his vast amount of gold coin, grain, fields, lands, wives, and male and female bondservants in order to go forth. Would it be right to say that, for that man, they are a strong, firm, stout bond, a tie that has not rotted, and a heavy yoke?” 

“No,\marginnote{12.10} sir. For that man, those bonds are weak, feeble, rotten, and insubstantial.” 

“In\marginnote{12.13} the same way, when some gentlemen are told by me to give something up, they say, ‘What, we just have to give up such a trivial, insignificant thing as this, when the Blessed One tells us to give it up, the Holy One tells us to let it go?’ They give it up, and they don’t nurse bitterness towards me; and when the mendicants who want to train have given that up, they live relaxed, unruffled, surviving on charity, their hearts free as a wild deer. For them, that bond is weak, feeble, rotten, and insubstantial. 

\textsanskrit{Udāyī},\marginnote{13.1} these four people are found in the world. What four? 

Take\marginnote{14.1} a certain person practicing to give up and let go of attachments. As they do so, memories and thoughts connected with attachments beset them. They tolerate them and don’t give them up, get rid of them, eliminate them, and obliterate them. I call this person ‘fettered’, not ‘detached’. Why is that? Because I understand the diversity of faculties as it applies to this person. 

Take\marginnote{15.1} another person practicing to give up and let go of attachments. As they do so, memories and thoughts connected with attachments beset them. They don’t tolerate them, but give them up, get rid of them, eliminate them, and obliterate them. I call this person ‘fettered’, not ‘detached’. Why is that? Because I understand the diversity of faculties as it applies to this person. 

Take\marginnote{16.1} another person practicing to give up and let go of attachments. As they do so, every so often they lose mindfulness, and memories and thoughts connected with attachments beset them. Their mindfulness is slow to come up, but they quickly give up, get rid of, eliminate, and obliterate those thoughts. Suppose there was an iron cauldron that had been heated all day, and a person let two or three drops of water fall onto it. The drops would be slow to fall, but they’d quickly dry up and evaporate. 

In\marginnote{16.7} the same way, take a person practicing to give up and let go of attachments. As they do so, every so often they lose mindfulness, and memories and thoughts connected with attachments beset them. Their mindfulness is slow to come up, but they quickly give them up, get rid of, eliminate, and obliterate those thoughts. I also call this person ‘fettered’, not ‘detached’. Why is that? Because I understand the diversity of faculties as it applies to this person. 

Take\marginnote{17.1} another person who, understanding that attachment is the root of suffering, is freed with the ending of attachments. I call this person ‘detached’, not ‘fettered’. Why is that? Because I understand the diversity of faculties as it applies to this person. These are the four people found in the world. 

\textsanskrit{Udāyī},\marginnote{18.1} these are the five kinds of sensual stimulation. What five? Sights known by the eye that are likable, desirable, agreeable, pleasant, sensual, and arousing. Sounds known by the ear … Smells known by the nose … Tastes known by the tongue … Touches known by the body that are likable, desirable, agreeable, pleasant, sensual, and arousing. These are the five kinds of sensual stimulation. 

The\marginnote{19.1} pleasure and happiness that arise from these five kinds of sensual stimulation is called sensual pleasure—a filthy, ordinary, ignoble pleasure. Such pleasure should not be cultivated or developed, but should be feared, I say. 

Take\marginnote{20.1} a mendicant who, quite secluded from sensual pleasures, secluded from unskillful qualities, enters and remains in the first absorption … second absorption … third absorption … fourth absorption. 

This\marginnote{21.1} is called the pleasure of renunciation, the pleasure of seclusion, the pleasure of peace, the pleasure of awakening. Such pleasure should be cultivated and developed, and should not be feared, I say. 

Take\marginnote{22.1} a mendicant who, quite secluded from sensual pleasures, secluded from unskillful qualities, enters and remains in the first absorption. This belongs to the perturbable, I say. And what there belongs to the perturbable? Whatever placing of the mind and keeping it connected has not ceased there is what belongs to the perturbable. 

Take\marginnote{23.1} a mendicant who, as the placing of the mind and keeping it connected are stilled, enters and remains in the second absorption. This belongs to the perturbable, I say. And what there belongs to the perturbable? Whatever rapture and bliss has not ceased there is what belongs to the perturbable. 

Take\marginnote{24.1} a mendicant who, with the fading away of rapture, enters and remains in the third absorption. This belongs to the perturbable. And what there belongs to the perturbable? Whatever equanimous bliss has not ceased there is what belongs to the perturbable. 

Take\marginnote{25.1} a mendicant who, giving up pleasure and pain, enters and remains in the fourth absorption. This belongs to the imperturbable. 

Take\marginnote{26.1} a mendicant who, quite secluded from sensual pleasures, secluded from unskillful qualities, enters and remains in the first absorption. But this is not enough, I say: give it up, go beyond it. And what goes beyond it? 

Take\marginnote{27.1} a mendicant who, as the placing of the mind and keeping it connected are stilled, enters and remains in the second absorption. That goes beyond it. But this too is not enough, I say: give it up, go beyond it. And what goes beyond it? 

Take\marginnote{28.1} a mendicant who, with the fading away of rapture, enters and remains in the third absorption. That goes beyond it. But this too is not enough, I say: give it up, go beyond it. And what goes beyond it? 

Take\marginnote{29.1} a mendicant who, giving up pleasure and pain, enters and remains in the fourth absorption. That goes beyond it. But this too is not enough, I say: give it up, go beyond it. And what goes beyond it? 

Take\marginnote{30.1} a mendicant who, going totally beyond perceptions of form, with the ending of perceptions of impingement, not focusing on perceptions of diversity, aware that ‘space is infinite’, enters and remains in the dimension of infinite space. That goes beyond it. But this too is not enough, I say: give it up, go beyond it. And what goes beyond it? 

Take\marginnote{31.1} a mendicant who, going totally beyond the dimension of infinite space, aware that ‘consciousness is infinite’, enters and remains in the dimension of infinite consciousness. That goes beyond it. But this too is not enough, I say: give it up, go beyond it. And what goes beyond it? 

Take\marginnote{32.1} a mendicant who, going totally beyond the dimension of infinite consciousness, aware that ‘there is nothing at all’, enters and remains in the dimension of nothingness. That goes beyond it. But this too is not enough, I say: give it up, go beyond it. And what goes beyond it? 

Take\marginnote{33.1} a mendicant who, going totally beyond the dimension of nothingness, enters and remains in the dimension of neither perception nor non-perception. That goes beyond it. But this too is not enough, I say: give it up, go beyond it. And what goes beyond it? 

Take\marginnote{34.1} a mendicant who, going totally beyond the dimension of neither perception nor non-perception, enters and remains in the cessation of perception and feeling. That goes beyond it. 

So,\marginnote{34.2} \textsanskrit{Udāyī}, I even recommend giving up the dimension of neither perception nor non-perception. Do you see any fetter, large or small, that I don’t recommend giving up?” 

“No,\marginnote{34.4} sir.” 

That\marginnote{34.5} is what the Buddha said. Satisfied, Venerable \textsanskrit{Udāyī} was happy with what the Buddha said. 

%
\section*{{\suttatitleacronym MN 67}{\suttatitletranslation At Cātumā }{\suttatitleroot Cātumasutta}}
\addcontentsline{toc}{section}{\tocacronym{MN 67} \toctranslation{At Cātumā } \tocroot{Cātumasutta}}
\markboth{At Cātumā }{Cātumasutta}
\extramarks{MN 67}{MN 67}

\scevam{So\marginnote{1.1} I have heard. }At one time the Buddha was staying near \textsanskrit{Cātumā} in a myrobalan grove. 

Now\marginnote{2.1} at that time around five hundred mendicants headed by \textsanskrit{Sāriputta} and \textsanskrit{Moggallāna} arrived at \textsanskrit{Cātumā} to see the Buddha. And the visiting mendicants, while exchanging pleasantries with the resident mendicants, preparing their lodgings, and putting away their bowls and robes, made a dreadful racket. 

Then\marginnote{3.1} the Buddha said to Venerable Ānanda, “Ānanda, who’s making that dreadful racket? You’d think it was fishermen hauling in a catch!” 

And\marginnote{3.3} Ānanda told him what had happened. 

“Well\marginnote{4.1} then, Ānanda, in my name tell those mendicants that the teacher summons them.” 

“Yes,\marginnote{4.3} sir,” Ānanda replied. He went to those mendicants and said, “Venerables, the teacher summons you.” 

“Yes,\marginnote{4.5} reverend,” replied those mendicants. Then they rose from their seats and went to the Buddha, bowed, and sat down to one side. The Buddha said to them: 

“Mendicants,\marginnote{4.6} what’s with that dreadful racket? You’d think it was fishermen hauling in a catch!” 

And\marginnote{4.7} they told him what had happened. 

“Go\marginnote{5.1} away, mendicants, I dismiss you. You are not to stay in my presence.” 

“Yes,\marginnote{5.2} sir,” replied those mendicants. They got up from their seats, bowed, and respectfully circled the Buddha, keeping him on their right. They set their lodgings in order and left, taking their bowls and robes. 

Now\marginnote{6.1} at that time the Sakyans of \textsanskrit{Cātumā} were sitting together at the meeting hall on some business. Seeing those mendicants coming off in the distance, they went up to them and said, “Hello venerables, where are you going?” 

“Sirs,\marginnote{6.5} the mendicant \textsanskrit{Saṅgha} has been dismissed by the Buddha.” 

“Well\marginnote{6.6} then, venerables, sit here for a minute. Hopefully we’ll be able to restore the Buddha’s confidence.” 

“Yes,\marginnote{6.7} sirs,” replied the mendicants. 

Then\marginnote{7.1} the Sakyans of \textsanskrit{Cātumā} went up to the Buddha, bowed, sat down to one side, and said to him: 

“May\marginnote{7.2} the Buddha approve of the mendicant \textsanskrit{Saṅgha}! May the Buddha welcome the mendicant \textsanskrit{Saṅgha}! May the Buddha support the mendicant \textsanskrit{Saṅgha} now as he did in the past! There are mendicants here who are junior, recently gone forth, newly come to this teaching and training. If they don’t get to see the Buddha they may change and fall apart. If young seedlings don’t get water they may change and fall apart. In the same way, there are mendicants here who are junior, recently gone forth, newly come to this teaching and training. If they don’t get to see the Buddha they may change and fall apart. If a young calf doesn’t see its mother it may change and fall apart. In the same way, there are mendicants here who are junior, recently gone forth, newly come to this teaching and training. If they don’t get to see the Buddha they may change and fall apart. May the Buddha approve of the mendicant \textsanskrit{Saṅgha}! May the Buddha welcome the mendicant \textsanskrit{Saṅgha}! May the Buddha support the mendicant \textsanskrit{Saṅgha} now as he did in the past!” 

Then\marginnote{8.1} \textsanskrit{Brahmā} Sahampati knew what the Buddha was thinking. As easily as a strong person would extend or contract their arm, he vanished from the \textsanskrit{Brahmā} realm and reappeared in front of the Buddha. He arranged his robe over one shoulder, raised his joined palms toward the Buddha, and said: 

“May\marginnote{9.1} the Buddha approve of the mendicant \textsanskrit{Saṅgha}! May the Buddha welcome the mendicant \textsanskrit{Saṅgha}! May the Buddha support the mendicant \textsanskrit{Saṅgha} now as he did in the past! There are mendicants here who are junior, recently gone forth, newly come to this teaching and training. If they don’t get to see the Buddha they may change and fall apart. If young seedlings don’t get water they may change and fall apart. … If a young calf doesn’t see its mother it may change and fall apart. In the same way, there are mendicants here who are junior, recently gone forth, newly come to this teaching and training. If they don’t get to see the Buddha they may change and fall apart. May the Buddha approve of the mendicant \textsanskrit{Saṅgha}! May the Buddha welcome the mendicant \textsanskrit{Saṅgha}! May the Buddha support the mendicant \textsanskrit{Saṅgha} now as he did in the past!” 

The\marginnote{10.1} Sakyans of \textsanskrit{Cātumā} and \textsanskrit{Brahmā} Sahampati were able to restore the Buddha’s confidence with the similes of the seedlings and the calf. 

Then\marginnote{11.1} Venerable \textsanskrit{Mahāmoggallāna} addressed the mendicants, “Get up, reverends, and pick up your bowls and robes. The Buddha’s confidence has been restored by the Sakyans of \textsanskrit{Cātumā} and \textsanskrit{Brahmā} Sahampati with the similes of the seedlings and the calf.” 

“Yes,\marginnote{12.1} reverend,” replied those mendicants. Then they rose from their seats and, taking their bowls and robes, went to the Buddha, bowed, and sat down to one side. The Buddha said to Venerable \textsanskrit{Sāriputta}, “\textsanskrit{Sāriputta}, what did you think when the mendicant \textsanskrit{Saṅgha} was dismissed by me?” 

“Sir,\marginnote{12.3} I thought this: ‘The Buddha has dismissed the mendicant \textsanskrit{Saṅgha}. Now he will remain passive, dwelling in blissful meditation in the present life, and so will we.’” 

“Hold\marginnote{12.6} on, \textsanskrit{Sāriputta}, hold on! Don’t you ever think such a thing again!” 

Then\marginnote{13.1} the Buddha addressed Venerable \textsanskrit{Mahāmoggallāna}, “\textsanskrit{Moggallāna}, what did you think when the mendicant \textsanskrit{Saṅgha} was dismissed by me?” 

“Sir,\marginnote{13.3} I thought this: ‘The Buddha has dismissed the mendicant \textsanskrit{Saṅgha}. Now he will remain passive, dwelling in blissful meditation in the present life. Meanwhile, Venerable \textsanskrit{Sāriputta} and I will lead the mendicant \textsanskrit{Saṅgha}.’” 

“Good,\marginnote{13.6} good, \textsanskrit{Moggallāna}! For either I should lead the mendicant \textsanskrit{Saṅgha}, or else \textsanskrit{Sāriputta} and \textsanskrit{Moggallāna}.” 

Then\marginnote{14.1} the Buddha said to the mendicants: 

“Mendicants,\marginnote{14.2} when you go into the water you should anticipate four dangers. What four? The dangers of waves, marsh crocodiles, whirlpools, and gharials. These are the four dangers that anyone who enters the water should anticipate. 

In\marginnote{15.1} the same way, a gentleman who goes forth from the lay life to homelessness in this teaching and training should anticipate four dangers. What four? The dangers of waves, marsh crocodiles, whirlpools, and gharials. 

And\marginnote{16.1} what, mendicants, is the danger of waves? It’s when a gentleman has gone forth from the lay life to homelessness, thinking: ‘I’m swamped by rebirth, old age, and death; by sorrow, lamentation, pain, sadness, and distress. I’m swamped by suffering, mired in suffering. Hopefully I can find an end to this entire mass of suffering.’ When they’ve gone forth, their spiritual companions advise and instruct them: ‘You should go out like this, and come back like that. You should look to the front like this, and to the side like that. You should contract your limbs like this, and extend them like that. This is how you should bear your outer robe, bowl, and robes.’ They think: ‘Formerly, as laypeople, we advised and instructed others. And now these mendicants—who you’d think were our children or grandchildren—imagine they can advise and instruct us!’ They resign the training and return to a lesser life. This is called one who resigns the training and returns to a lesser life because they’re afraid of the danger of waves. ‘Danger of waves’ is a term for anger and distress. 

And\marginnote{17.1} what, mendicants, is the danger of marsh crocodiles? It’s when a gentleman has gone forth from the lay life to homelessness, thinking: ‘I’m swamped by rebirth, old age, and death; by sorrow, lamentation, pain, sadness, and distress. I’m swamped by suffering, mired in suffering. Hopefully I can find an end to this entire mass of suffering.’ When they’ve gone forth, their spiritual companions advise and instruct them: ‘You may eat, consume, taste, and drink these things, but not those. You may eat what’s allowable, but not what’s unallowable. You may eat at the right time, but not at the wrong time.’ They think: ‘Formerly, as laypeople, we used to eat, consume, taste, and drink what we wanted, not what we didn’t want. We ate and drank both allowable and unallowable things, at the right time and the wrong time. And these faithful householders give us a variety of delicious foods at the wrong time of day. But these mendicants imagine they can gag our mouths!’ They resign the training and return to a lesser life. This is called one who resigns the training and returns to a lesser life because they’re afraid of the danger of marsh crocodiles. ‘Danger of marsh crocodiles’ is a term for gluttony. 

And\marginnote{18.1} what, mendicants, is the danger of whirlpools? It’s when a gentleman has gone forth from the lay life to homelessness, thinking: ‘I’m swamped by rebirth, old age, and death; by sorrow, lamentation, pain, sadness, and distress. I’m swamped by suffering, mired in suffering. Hopefully I can find an end to this entire mass of suffering.’ When they’ve gone forth, they robe up in the morning and, taking their bowl and robe, enter a village or town for alms without guarding body, speech, and mind, without establishing mindfulness, and without restraining the sense faculties. There they see a householder or their child amusing themselves, supplied and provided with the five kinds of sensual stimulation. They think: ‘Formerly, as laypeople, we amused ourselves, supplied and provided with the five kinds of sensual stimulation. And it’s true that my family is wealthy. I can both enjoy my wealth and make merit.’ They resign the training and return to a lesser life. This is called one who resigns the training and returns to a lesser life because they’re afraid of the danger of whirlpools. ‘Danger of whirlpools’ is a term for the five kinds of sensual stimulation. 

And\marginnote{19.1} what, mendicants, is the danger of gharials? It’s when a gentleman has gone forth from the lay life to homelessness, thinking: ‘I’m swamped by rebirth, old age, and death; by sorrow, lamentation, pain, sadness, and distress. I’m swamped by suffering, mired in suffering. Hopefully I can find an end to this entire mass of suffering.’ When they’ve gone forth, they robe up in the morning and, taking their bowl and robe, enter a village or town for alms without guarding body, speech, and mind, without establishing mindfulness, and without restraining the sense faculties. There they see a female scantily clad, with revealing clothes. Lust infects their mind, so they resign the training and return to a lesser life. This is called one who resigns the training and returns to a lesser life because they’re afraid of the danger of gharials. ‘Danger of gharials’ is a term for females. 

These\marginnote{20.1} are the four dangers that a gentleman who goes forth from the lay life to homelessness in this teaching and training should anticipate.” 

That\marginnote{20.2} is what the Buddha said. Satisfied, the mendicants were happy with what the Buddha said. 

%
\section*{{\suttatitleacronym MN 68}{\suttatitletranslation At Naḷakapāna }{\suttatitleroot Naḷakapānasutta}}
\addcontentsline{toc}{section}{\tocacronym{MN 68} \toctranslation{At Naḷakapāna } \tocroot{Naḷakapānasutta}}
\markboth{At Naḷakapāna }{Naḷakapānasutta}
\extramarks{MN 68}{MN 68}

\scevam{So\marginnote{1.1} I have heard. }At one time the Buddha was staying in the land of the Kosalans near \textsanskrit{Naḷakapāna} in the Parrot Tree grove. 

Now\marginnote{2.1} at that time several very well-known gentlemen had gone forth from the lay life to homelessness out of faith in the Buddha—The venerables Anuruddha, Bhaddiya, Kimbila, Bhagu, \textsanskrit{Koṇḍañña}, Revata, Ānanda, and other very well-known gentlemen. 

Now\marginnote{3.1} at that time the Buddha was sitting in the open, surrounded by the mendicant \textsanskrit{Saṅgha}. Then the Buddha spoke to the mendicants about those gentlemen: “Mendicants, those gentlemen who have gone forth from the lay life to homelessness out of faith in me—I trust they’re satisfied with the spiritual life?” When this was said, the mendicants kept silent. 

For\marginnote{3.5} a second and a third time the Buddha asked the same question. For a third time, the mendicants kept silent. 

Then\marginnote{4.1} it occurred to the Buddha, “Why don’t I question just those gentlemen?” Then the Buddha said to Venerable Anuruddha, “Anuruddha and friends, I hope you’re satisfied with the spiritual life?” 

“Indeed,\marginnote{4.5} sir, we are satisfied with the spiritual life.” 

“Good,\marginnote{5.1} good, Anuruddha and friends! It’s appropriate for gentlemen like yourselves, who have gone forth in faith from the lay life to homelessness, to be satisfied with the spiritual life. Since you’re blessed with youth, in the prime of life, black-haired, you could have enjoyed sensual pleasures; yet you have gone forth from the lay life to homelessness. But you didn’t go forth because you were forced to by kings or bandits, or because you’re in debt or threatened, or to earn a living. Rather, didn’t you go forth thinking: ‘I’m swamped by rebirth, old age, and death; by sorrow, lamentation, pain, sadness, and distress. I’m swamped by suffering, mired in suffering. Hopefully I can find an end to this entire mass of suffering’?” 

“Yes,\marginnote{5.8} sir.” 

“But,\marginnote{6.1} Anuruddha and friends, when a gentleman has gone forth like this, what should he do? Take someone who doesn’t achieve the rapture and bliss that are secluded from sensual pleasures and unskillful qualities, or something even more peaceful than that. Their mind is still occupied by desire, ill will, dullness and drowsiness, restlessness and remorse, doubt, discontent, and sloth. That’s someone who doesn’t achieve the rapture and bliss that are secluded from sensual pleasures and unskillful qualities, or something even more peaceful than that. 

Take\marginnote{6.4} someone who does achieve the rapture and bliss that are secluded from sensual pleasures and unskillful qualities, or something even more peaceful than that. Their mind is not occupied by desire, ill will, dullness and drowsiness, restlessness and remorse, doubt, discontent, and sloth. That’s someone who does achieve the rapture and bliss that are secluded from sensual pleasures and unskillful qualities, or something even more peaceful than that. 

Is\marginnote{7.1} this what you think of me? ‘The Realized One has not given up the defilements that are corrupting, leading to future lives, hurtful, resulting in suffering and future rebirth, old age, and death. That’s why, after appraisal, he uses some things, endures some things, avoids some things, and gets rid of some things.’” 

“No\marginnote{7.4} sir, we don’t think of you that way. We think of you this way: ‘The Realized One has given up the defilements that are corrupting, leading to future lives, hurtful, resulting in suffering and future rebirth, old age, and death. That’s why, after appraisal, he uses some things, endures some things, avoids some things, and gets rid of some things.’” 

“Good,\marginnote{7.10} good, Anuruddha and friends! The Realized One has given up the defilements that are corrupting, leading to future lives, hurtful, resulting in suffering and future rebirth, old age, and death. He has cut them off at the root, made them like a palm stump, obliterated them so they are unable to arise in the future. Just as a palm tree with its crown cut off is incapable of further growth, in the same way, the Realized One has given up the defilements so they are unable to arise in the future. That’s why, after appraisal, he uses some things, endures some things, avoids some things, and gets rid of some things. 

What\marginnote{8.1} do you think, Anuruddha and friends? What advantage does the Realized One see in declaring the rebirth of his disciples who have passed away: ‘This one is reborn here, while that one is reborn there’?” 

“Our\marginnote{8.4} teachings are rooted in the Buddha. He is our guide and our refuge. Sir, may the Buddha himself please clarify the meaning of this. The mendicants will listen and remember it.” 

“The\marginnote{9.1} Realized One does not declare such things for the sake of deceiving people or flattering them, nor for the benefit of possessions, honor, or popularity, nor thinking, ‘So let people know about me!’ Rather, there are gentlemen of faith who are full of sublime joy and gladness. When they hear that, they apply their minds to that end. That is for their lasting welfare and happiness. 

Take\marginnote{10.1} a monk who hears this: ‘The monk named so-and-so has passed away. The Buddha has declared that, he was enlightened.’ And he’s either seen for himself, or heard from someone else, that that venerable had such ethics, such qualities, such wisdom, such meditation, or such freedom. Recollecting that monk’s faith, ethics, learning, generosity, and wisdom, he applies his mind to that end. That’s how a monk lives at ease. 

Take\marginnote{11.1} a monk who hears this: ‘The monk named so-and-so has passed away. The Buddha has declared that, with the ending of the five lower fetters, he’s been reborn spontaneously and will become extinguished there, not liable to return from that world.’ And he’s either seen for himself, or heard from someone else, that that venerable had such ethics, such qualities, such wisdom, such meditation, or such freedom. Recollecting that monk’s faith, ethics, learning, generosity, and wisdom, he applies his mind to that end. That too is how a monk lives at ease. 

Take\marginnote{12.1} a monk who hears this: ‘The monk named so-and-so has passed away. The Buddha has declared that, with the ending of three fetters, and the weakening of greed, hate, and delusion, he’s a once-returner. He’ll come back to this world once only, then make an end of suffering.’ And he’s either seen for himself, or heard from someone else, that that venerable had such ethics, such qualities, such wisdom, such meditation, or such freedom. Recollecting that monk’s faith, ethics, learning, generosity, and wisdom, he applies his mind to that end. That too is how a monk lives at ease. 

Take\marginnote{13.1} a monk who hears this: ‘The monk named so-and-so has passed away. The Buddha has declared that, with the ending of three fetters he’s a stream-enterer, not liable to be reborn in the underworld, bound for awakening.’ And he’s either seen for himself, or heard from someone else, that that venerable had such ethics, such qualities, such wisdom, such meditation, or such freedom. Recollecting that monk’s faith, ethics, learning, generosity, and wisdom, he applies his mind to that end. That too is how a monk lives at ease. 

Take\marginnote{14.1} a nun who hears this: ‘The nun named so-and-so has passed away. The Buddha has declared that, she was enlightened.’ And she’s either seen for herself, or heard from someone else, that that sister had such ethics, such qualities, such wisdom, such meditation, or such freedom. Recollecting that nun’s faith, ethics, learning, generosity, and wisdom, she applies her mind to that end. That’s how a nun lives at ease. 

Take\marginnote{15.1} a nun who hears this: ‘The nun named so-and-so has passed away. The Buddha has declared that, with the ending of the five lower fetters, she’s been reborn spontaneously and will become extinguished there, not liable to return from that world.’ And she’s either seen for herself, or heard from someone else, that that sister had such ethics, such qualities, such wisdom, such meditation, or such freedom. Recollecting that nun’s faith, ethics, learning, generosity, and wisdom, she applies her mind to that end. That too is how a nun lives at ease. 

Take\marginnote{16.1} a nun who hears this: ‘The nun named so-and-so has passed away. The Buddha has declared that, with the ending of three fetters, and the weakening of greed, hate, and delusion, she’s a once-returner. She’ll come back to this world once only, then make an end of suffering.’ And she’s either seen for herself, or heard from someone else, that that sister had such ethics, such qualities, such wisdom, such meditation, or such freedom. Recollecting that nun’s faith, ethics, learning, generosity, and wisdom, she applies her mind to that end. That too is how a nun lives at ease. 

Take\marginnote{17.1} a nun who hears this: ‘The nun named so-and-so has passed away. The Buddha has declared that, with the ending of three fetters she’s a stream-enterer, not liable to be reborn in the underworld, bound for awakening.’ And she’s either seen for herself, or heard from someone else, that that sister had such ethics, such qualities, such wisdom, such meditation, or such freedom. Recollecting that nun’s faith, ethics, learning, generosity, and wisdom, she applies her mind to that end. That too is how a nun lives at ease. 

Take\marginnote{18.1} a layman who hears this: ‘The layman named so-and-so has passed away. The Buddha has declared that, with the ending of the five lower fetters, he’s been reborn spontaneously and will become extinguished there, not liable to return from that world.’ And he’s either seen for himself, or heard from someone else, that that venerable had such ethics, such qualities, such wisdom, such meditation, or such freedom. Recollecting that layman’s faith, ethics, learning, generosity, and wisdom, he applies his mind to that end. That’s how a layman lives at ease. 

Take\marginnote{19.1} a layman who hears this: ‘The layman named so-and-so has passed away. The Buddha has declared that, with the ending of three fetters, and the weakening of greed, hate, and delusion, he’s a once-returner. He’ll come back to this world once only, then make an end of suffering.’ And he’s either seen for himself, or heard from someone else, that that venerable had such ethics, such qualities, such wisdom, such meditation, or such freedom. Recollecting that layman’s faith, ethics, learning, generosity, and wisdom, he applies his mind to that end. That too is how a layman lives at ease. 

Take\marginnote{20.1} a layman who hears this: ‘The layman named so-and-so has passed away. The Buddha has declared that, with the ending of three fetters he’s a stream-enterer, not liable to be reborn in the underworld, bound for awakening.’ And he’s either seen for himself, or heard from someone else, that that venerable had such ethics, such qualities, such wisdom, such meditation, or such freedom. Recollecting that layman’s faith, ethics, learning, generosity, and wisdom, he applies his mind to that end. That too is how a layman lives at ease. 

Take\marginnote{21.1} a laywoman who hears this: ‘The laywoman named so-and-so has passed away. The Buddha has declared that, with the ending of the five lower fetters, she’s been reborn spontaneously and will become extinguished there, not liable to return from that world.’ And she’s either seen for herself, or heard from someone else, that that sister had such ethics, such qualities, such wisdom, such meditation, or such freedom. Recollecting that laywoman’s faith, ethics, learning, generosity, and wisdom, she applies her mind to that end. That’s how a laywoman lives at ease. 

Take\marginnote{22.1} a laywoman who hears this: ‘The laywoman named so-and-so has passed away. The Buddha has declared that, with the ending of three fetters, and the weakening of greed, hate, and delusion, she’s a once-returner. She’ll come back to this world once only, then make an end of suffering.’ And she’s either seen for herself, or heard from someone else, that that sister had such ethics, such qualities, such wisdom, such meditation, or such freedom. Recollecting that laywoman’s faith, ethics, learning, generosity, and wisdom, she applies her mind to that end. That too is how a laywoman lives at ease. 

Take\marginnote{23.1} a laywoman who hears this: ‘The laywoman named so-and-so has passed away. The Buddha has declared that, with the ending of three fetters she’s a stream-enterer, not liable to be reborn in the underworld, bound for awakening.’ And she’s either seen for herself, or heard from someone else, that that sister had such ethics, such qualities, such wisdom, such meditation, or such freedom. Recollecting that laywoman’s faith, ethics, learning, generosity, and wisdom, she applies her mind to that end. That too is how a laywoman lives at ease. 

So\marginnote{24.1} it’s not for the sake of deceiving people or flattering them, nor for the benefit of possessions, honor, or popularity, nor thinking, ‘So let people know about me!’ that the Realized One declares the rebirth of his disciples who have passed away: ‘This one is reborn here, while that one is reborn there.’ Rather, there are gentlemen of faith who are full of joy and gladness. When they hear that, they apply their minds to that end. That is for their lasting welfare and happiness.” 

That\marginnote{24.6} is what the Buddha said. Satisfied, Venerable Anuruddha and friends were happy with what the Buddha said. 

%
\section*{{\suttatitleacronym MN 69}{\suttatitletranslation With Gulissāni }{\suttatitleroot Goliyānisutta}}
\addcontentsline{toc}{section}{\tocacronym{MN 69} \toctranslation{With Gulissāni } \tocroot{Goliyānisutta}}
\markboth{With Gulissāni }{Goliyānisutta}
\extramarks{MN 69}{MN 69}

\scevam{So\marginnote{1.1} I have heard. }At one time the Buddha was staying near \textsanskrit{Rājagaha}, in the Bamboo Grove, the squirrels’ feeding ground. 

Now\marginnote{2.1} at that time a wilderness mendicant of lax behavior named \textsanskrit{Gulissāni} had come down to the midst of the \textsanskrit{Saṅgha} on some business. There Venerable \textsanskrit{Sāriputta} spoke to the mendicants about \textsanskrit{Gulissāni}: 

“Reverends,\marginnote{3.1} a wilderness monk who has come to stay in the \textsanskrit{Saṅgha} should have respect and reverence for his spiritual companions. If he doesn’t, there’ll be some who say: ‘What’s the point of this wilderness venerable’s staying alone and autonomous in the wilderness, since he has no respect and reverence for his spiritual companions?’ That’s why a wilderness monk who has come to stay in the \textsanskrit{Saṅgha} should have respect and reverence for his spiritual companions. 

A\marginnote{4.1} wilderness monk who has come to stay in the \textsanskrit{Saṅgha} should be careful where he sits, thinking: ‘I shall sit so that I don’t intrude on the senior monks and I don’t block the junior monks from a seat.’ If he doesn’t, there’ll be some who say: ‘What’s the point of this wilderness venerable’s staying alone and autonomous in the wilderness, since he’s not careful where he sits?’ That’s why a wilderness monk who has come to stay in the \textsanskrit{Saṅgha} should be careful where he sits. 

A\marginnote{4.7} wilderness monk who has come to stay in the \textsanskrit{Saṅgha} should know even the supplementary regulations. If he doesn’t, there’ll be some who say: ‘What’s the point of this wilderness venerable’s staying alone and autonomous in the wilderness, since he doesn’t even know the supplementary regulations?’ That’s why a wilderness monk who has come to stay in the \textsanskrit{Saṅgha} should know even the supplementary regulations. 

A\marginnote{5.1} wilderness monk who has come to stay in the \textsanskrit{Saṅgha} shouldn’t enter the village too early or return too late in the day. If he does so, there’ll be some who say: ‘What’s the point of this wilderness venerable’s staying alone and autonomous in the wilderness, since he enters the village too early or returns too late in the day?’ That’s why a wilderness monk who has come to stay in the \textsanskrit{Saṅgha} shouldn’t enter the village too early or return too late in the day. 

A\marginnote{6.1} wilderness monk who has come to stay in the \textsanskrit{Saṅgha} shouldn’t socialize with families before or after the meal. If he does so, there’ll be some who say: ‘This wilderness venerable, staying alone and autonomous in the wilderness, must be used to wandering about at the wrong time, since he behaves like this when he’s come to the \textsanskrit{Saṅgha}.’ That’s why a wilderness monk who has come to stay in the \textsanskrit{Saṅgha} shouldn’t socialize with families before or after the meal. 

A\marginnote{7.1} wilderness monk who has come to stay in the \textsanskrit{Saṅgha} shouldn’t be restless and fickle. If he is, there’ll be some who say: ‘This wilderness venerable, staying alone and autonomous in the wilderness, must be used to being restless and fickle, since he behaves like this when he’s come to the \textsanskrit{Saṅgha}.’ That’s why a wilderness monk who has come to stay in the \textsanskrit{Saṅgha} shouldn’t be restless and fickle. 

A\marginnote{8.1} wilderness monk who has come to stay in the \textsanskrit{Saṅgha} shouldn’t be scurrilous and loose-tongued. If he is, there’ll be some who say: ‘What’s the point of this wilderness venerable’s staying alone and autonomous in the wilderness, since he’s scurrilous and loose-tongued?’ That’s why a wilderness monk who has come to stay in the \textsanskrit{Saṅgha} shouldn’t be scurrilous and loose-tongued. 

A\marginnote{9.1} wilderness monk who has come to stay in the \textsanskrit{Saṅgha} should be easy to admonish, with good friends. If he’s hard to admonish, with bad friends, there’ll be some who say: ‘What’s the point of this wilderness venerable’s staying alone and autonomous in the wilderness, since he’s hard to admonish, with bad friends?’ That’s why a wilderness monk who has come to stay in the \textsanskrit{Saṅgha} should be easy to admonish, with good friends. 

A\marginnote{10.1} wilderness monk should guard the sense doors. If he doesn’t, there’ll be some who say: ‘What’s the point of this wilderness venerable’s staying alone and autonomous in the wilderness, since he doesn’t guard the sense doors?’ That’s why a wilderness monk should guard the sense doors. 

A\marginnote{11.1} wilderness monk should eat in moderation. If he doesn’t, there’ll be some who say: ‘What’s the point of this wilderness venerable’s staying alone and autonomous in the wilderness, since he eats too much?’ That’s why a wilderness monk should eat in moderation. 

A\marginnote{12.1} wilderness monk should be committed to wakefulness. If he isn’t, there’ll be some who say: ‘What’s the point of this wilderness venerable’s staying alone and autonomous in the wilderness, since he’s not committed to wakefulness?’ That’s why a wilderness monk should be committed to wakefulness. 

A\marginnote{13.1} wilderness monk should be energetic. If he isn’t, there’ll be some who say: ‘What’s the point of this wilderness venerable’s staying alone and autonomous in the wilderness, since he’s not energetic?’ That’s why a wilderness monk should be energetic. 

A\marginnote{14.1} wilderness monk should be mindful. If he isn’t, there’ll be some who say: ‘What’s the point of this wilderness venerable’s staying alone and autonomous in the wilderness, since he’s not mindful?’ That’s why a wilderness monk should be mindful. 

A\marginnote{15.1} wilderness monk should have immersion. If he doesn’t, there’ll be some who say: ‘What’s the point of this wilderness venerable’s staying alone and autonomous in the wilderness, since he doesn’t have immersion?’ That’s why a wilderness monk should have immersion. 

A\marginnote{16.1} wilderness monk should be wise. If he isn’t, there’ll be some who say: ‘What’s the point of this wilderness venerable’s staying alone and autonomous in the wilderness, since he’s not wise?’ That’s why a wilderness monk should be wise. 

A\marginnote{17.1} wilderness monk should make an effort to learn the teaching and training. There are those who will question a wilderness monk about the teaching and training. If he is stumped, there’ll be some who say: ‘What’s the point of this wilderness venerable’s staying alone and autonomous in the wilderness, since he is stumped by a question about the teaching and training?’ That’s why a wilderness monk should make an effort to learn the teaching and training. 

A\marginnote{18.1} wilderness monk should practice meditation to realize the peaceful liberations that are formless, transcending form. There are those who will question a wilderness monk regarding the formless liberations. If he is stumped, there’ll be some who say: ‘What’s the point of this wilderness venerable’s staying alone and autonomous in the wilderness, since he is stumped by a question about the formless liberations?’ That’s why a wilderness monk should practice meditation to realize the peaceful liberations that are formless, transcending form. 

A\marginnote{19.1} wilderness monk should practice meditation to realize the superhuman state. There are those who will question a wilderness monk about the superhuman state. If he is stumped, there’ll be some who say: ‘What’s the point of this wilderness venerable’s staying alone and autonomous in the wilderness, since he doesn’t know the goal for which he went forth?’ That’s why a wilderness monk should practice meditation to realize the superhuman state.” 

When\marginnote{19.7} Venerable \textsanskrit{Sāriputta} said this, Venerable \textsanskrit{Mahāmoggallāna} said to him, “Reverend \textsanskrit{Sāriputta}, should these things be undertaken and followed only by wilderness monks, or by those who live within a village as well?” 

“Reverend\marginnote{19.9} \textsanskrit{Moggallāna}, these things should be undertaken and followed by wilderness monks, and still more by those who live within a village.” 

%
\section*{{\suttatitleacronym MN 70}{\suttatitletranslation At Kīṭāgiri }{\suttatitleroot Kīṭāgirisutta}}
\addcontentsline{toc}{section}{\tocacronym{MN 70} \toctranslation{At Kīṭāgiri } \tocroot{Kīṭāgirisutta}}
\markboth{At Kīṭāgiri }{Kīṭāgirisutta}
\extramarks{MN 70}{MN 70}

\scevam{So\marginnote{1.1} I have heard. }At one time the Buddha was wandering in the land of the \textsanskrit{Kāsīs} together with a large \textsanskrit{Saṅgha} of mendicants. There the Buddha addressed the mendicants: 

“Mendicants,\marginnote{2.1} I abstain from eating at night. Doing so, I find that I’m healthy and well, nimble, strong, and living comfortably. You too should abstain from eating at night. Doing so, you’ll find that you’re healthy and well, nimble, strong, and living comfortably.” 

“Yes,\marginnote{2.5} sir,” they replied. 

Then\marginnote{3.1} the Buddha, traveling stage by stage in the land of the \textsanskrit{Kāsīs}, arrived at a town of the \textsanskrit{Kāsīs} named \textsanskrit{Kīṭāgiri}, and stayed there. 

Now\marginnote{4.1} at that time the mendicants who followed Assaji and Punabbasuka were residing at \textsanskrit{Kīṭāgiri}. Then several mendicants went up to them and said, “Reverends, the Buddha abstains from eating at night, and so does the mendicant \textsanskrit{Saṅgha}. Doing so, they find that they’re healthy and well, nimble, strong, and living comfortably. You too should abstain from eating at night. Doing so, you’ll find that you’re healthy and well, nimble, strong, and living comfortably.” 

When\marginnote{4.7} they said this, the mendicants who followed Assaji and Punabbasuka said to them, “Reverends, we eat in the evening, the morning, and at the wrong time of day. Doing so, we find that we’re healthy and well, nimble, strong, and living comfortably. Why should we give up what is visible in the present to chase after what takes effect over time? We shall eat in the evening, the morning, and at the wrong time of day.” 

Since\marginnote{5.1} those mendicants were unable to persuade the mendicants who were followers of Assaji and Punabbasuka, they approached the Buddha, bowed, sat down to one side, and told him what had happened. 

So\marginnote{6.1} the Buddha addressed a certain monk, “Please, monk, in my name tell the mendicants who follow Assaji and Punabbasuka that the teacher summons them.” 

“Yes,\marginnote{6.4} sir,” that monk replied. He went to those mendicants and said, “Venerables, the teacher summons you.” 

“Yes,\marginnote{6.6} reverend,” those mendicants replied. They went to the Buddha, bowed, and sat down to one side. 

The\marginnote{6.7} Buddha said to them, “Is it really true, mendicants, that several mendicants went to you and said: ‘Reverends, the Buddha abstains from eating at night, and so does the mendicant \textsanskrit{Saṅgha}. Doing so, they find that they’re healthy and well, nimble, strong, and living comfortably. You too should abstain from eating at night. Doing so, you’ll find that you’re healthy and well, nimble, strong, and living comfortably.’ When they said this, did you really say to them: ‘Reverends, we eat in the evening, the morning, and at the wrong time of day. Doing so, we find that we’re healthy and well, nimble, strong, and living comfortably. Why should we give up what is visible in the present to chase after what takes effect over time? We shall eat in the evening, the morning, and at the wrong time of day.’” 

“Yes,\marginnote{6.17} sir.” 

“Mendicants,\marginnote{6.18} have you ever known me to teach the Dhamma like this: no matter what this individual experiences—pleasurable, painful, or neutral—their unskillful qualities decline and their skillful qualities grow?” 

“No,\marginnote{6.19} sir.” 

“Haven’t\marginnote{7.1} you known me to teach the Dhamma like this: ‘When someone feels this kind of pleasant feeling, unskillful qualities grow and skillful qualities decline. But when someone feels that kind of pleasant feeling, unskillful qualities decline and skillful qualities grow. When someone feels this kind of painful feeling, unskillful qualities grow and skillful qualities decline. But when someone feels that kind of painful feeling, unskillful qualities decline and skillful qualities grow. When someone feels this kind of neutral feeling, unskillful qualities grow and skillful qualities decline. But when someone feels that kind of neutral feeling, unskillful qualities decline and skillful qualities grow’?” 

“Yes,\marginnote{7.2} sir.” 

“Good,\marginnote{8.1} mendicants! Now, suppose I hadn’t known, seen, understood, realized, and experienced this with wisdom: ‘When someone feels this kind of pleasant feeling, unskillful qualities grow and skillful qualities decline.’ Not knowing this, would it be appropriate for me to say: ‘You should give up this kind of pleasant feeling’?” 

“No,\marginnote{8.5} sir.” 

“But\marginnote{8.6} I have known, seen, understood, realized, and experienced this with wisdom: ‘When someone feels this kind of pleasant feeling, unskillful qualities grow and skillful qualities decline.’ Since this is so, that’s why I say: ‘You should give up this kind of pleasant feeling.’ Now, suppose I hadn’t known, seen, understood, realized, and experienced this with wisdom: ‘When someone feels that kind of pleasant feeling, unskillful qualities decline and skillful qualities grow.’ Not knowing this, would it be appropriate for me to say: ‘You should enter and remain in that kind of pleasant feeling’?” 

“No,\marginnote{8.11} sir.” 

“But\marginnote{8.12} I have known, seen, understood, realized, and experienced this with wisdom: ‘When someone feels that kind of pleasant feeling, unskillful qualities decline and skillful qualities grow.’ Since this is so, that’s why I say: ‘You should enter and remain in that kind of pleasant feeling.’ 

Now,\marginnote{8.14} suppose I hadn’t known, seen, understood, realized, and experienced this with wisdom: ‘When someone feels this kind of painful feeling, unskillful qualities grow and skillful qualities decline.’ Not knowing this, would it be appropriate for me to say: ‘You should give up this kind of painful feeling’?” 

“No,\marginnote{8.17} sir.” 

“But\marginnote{9.1} I have known, seen, understood, realized, and experienced this with wisdom: ‘When someone feels this kind of painful feeling, unskillful qualities grow and skillful qualities decline.’ Since this is so, that’s why I say: ‘You should give up this kind of painful feeling.’ Now, suppose I hadn’t known, seen, understood, realized, and experienced this with wisdom: ‘When someone feels that kind of painful feeling, unskillful qualities decline and skillful qualities grow.’ Not knowing this, would it be appropriate for me to say: ‘You should enter and remain in that kind of painful feeling’?” 

“No,\marginnote{9.6} sir.” 

“But\marginnote{9.7} I have known, seen, understood, realized, and experienced this with wisdom: ‘When someone feels that kind of painful feeling, unskillful qualities decline and skillful qualities grow.’ Since this is so, that’s why I say: ‘You should enter and remain in that kind of painful feeling.’ 

Now,\marginnote{10.1} suppose I hadn’t known, seen, understood, realized, and experienced this with wisdom: ‘When someone feels this kind of neutral feeling, unskillful qualities grow and skillful qualities decline.’ Not knowing this, would it be appropriate for me to say: ‘You should give up this kind of neutral feeling’?” 

“No,\marginnote{10.4} sir.” 

“But\marginnote{10.5} I have known, seen, understood, realized, and experienced this with wisdom: ‘When someone feels this kind of neutral feeling, unskillful qualities grow and skillful qualities decline.’ Since this is so, that’s why I say: ‘You should give up this kind of neutral feeling.’ Now, suppose I hadn’t known, seen, understood, realized, and experienced this with wisdom: ‘When someone feels that kind of neutral feeling, unskillful qualities decline and skillful qualities grow.’ Not knowing this, would it be appropriate for me to say: ‘You should enter and remain in that kind of neutral feeling’?” 

“No,\marginnote{10.10} sir.” 

“But\marginnote{11.1} I have known, seen, understood, realized, and experienced this with wisdom: ‘When someone feels that kind of neutral feeling, unskillful qualities decline and skillful qualities grow.’ Since this is so, that’s why I say: ‘You should enter and remain in that kind of neutral feeling.’ 

Mendicants,\marginnote{12.1} I don’t say that all these mendicants still have work to do with diligence. Nor do I say that all these mendicants have no work to do with diligence. I say that mendicants don’t have work to do with diligence if they are perfected, with defilements ended, having completed the spiritual journey, done what had to be done, laid down the burden, achieved their own goal, utterly ended the fetters of rebirth, and become rightly freed through enlightenment. Why is that? They’ve done their work with diligence. They’re incapable of being negligent. 

I\marginnote{13.1} say that mendicants still have work to do with diligence if they are trainees, who haven’t achieved their heart’s desire, but live aspiring to the supreme sanctuary. Why is that? Thinking: ‘Hopefully this venerable will frequent appropriate lodgings, associate with good friends, and control their faculties. Then they might realize the supreme culmination of the spiritual path in this very life, and live having achieved with their own insight the goal for which gentlemen rightly go forth from the lay life to homelessness.’ Seeing this fruit of diligence for those mendicants, I say that they still have work to do with diligence. 

Mendicants,\marginnote{14.1} these seven people are found in the world. What seven? One freed both ways, one freed by wisdom, a personal witness, one attained to view, one freed by faith, a follower of the teachings, and a follower by faith. 

And\marginnote{15.1} what person is freed both ways? It’s a person who has direct meditative experience of the peaceful liberations that are formless, transcending form. And, having seen with wisdom, their defilements have come to an end. This person is called freed both ways. And I say that this mendicant has no work to do with diligence. Why is that? They’ve done their work with diligence. They’re incapable of being negligent. 

And\marginnote{16.1} what person is freed by wisdom? It’s a person who does not have direct meditative experience of the peaceful liberations that are formless, transcending form. Nevertheless, having seen with wisdom, their defilements have come to an end. This person is called freed by wisdom. I say that this mendicant has no work to do with diligence. Why is that? They’ve done their work with diligence. They’re incapable of being negligent. 

And\marginnote{17.1} what person is a personal witness? It’s a person who has direct meditative experience of the peaceful liberations that are formless, transcending form. And, having seen with wisdom, some of their defilements have come to an end. This person is called a personal witness. I say that this mendicant still has work to do with diligence. Why is that? Thinking: ‘Hopefully this venerable will frequent appropriate lodgings, associate with good friends, and control their faculties. Then they might realize the supreme culmination of the spiritual path in this very life, and live having achieved with their own insight the goal for which gentlemen rightly go forth from the lay life to homelessness.’ Seeing this fruit of diligence for this mendicant, I say that they still have work to do with diligence. 

And\marginnote{18.1} what person is attained to view? It’s a person who doesn’t have direct meditative experience of the peaceful liberations that are formless, transcending form. Nevertheless, having seen with wisdom, some of their defilements have come to an end. And they have clearly seen and clearly contemplated with wisdom the teaching and training proclaimed by the Realized One. This person is called attained to view. I say that this mendicant also still has work to do with diligence. Why is that? Thinking: ‘Hopefully this venerable will frequent appropriate lodgings, associate with good friends, and control their faculties. Then they might realize the supreme culmination of the spiritual path in this very life, and live having achieved with their own insight the goal for which gentlemen rightly go forth from the lay life to homelessness.’ Seeing this fruit of diligence for this mendicant, I say that they still have work to do with diligence. 

And\marginnote{19.1} what person is freed by faith? It’s a person who doesn’t have direct meditative experience of the peaceful liberations that are formless, transcending form. Nevertheless, having seen with wisdom, some of their defilements have come to an end. And their faith is settled, rooted, and planted in the Realized One. This person is called freed by faith. I say that this mendicant also still has work to do with diligence. Why is that? Thinking: ‘Hopefully this venerable will frequent appropriate lodgings, associate with good friends, and control their faculties. Then they might realize the supreme culmination of the spiritual path in this very life, and live having achieved with their own insight the goal for which gentlemen rightly go forth from the lay life to homelessness.’ Seeing this fruit of diligence for this mendicant, I say that they still have work to do with diligence. 

And\marginnote{20.1} what person is a follower of the teachings? It’s a person who doesn’t have direct meditative experience of the peaceful liberations that are formless, transcending form. Nevertheless, having seen with wisdom, some of their defilements have come to an end. And they accept the teachings proclaimed by the Realized One after considering them with a degree of wisdom. And they have the following qualities: the faculties of faith, energy, mindfulness, immersion, and wisdom. This person is called a follower of the teachings. I say that this mendicant also still has work to do with diligence. Why is that? Thinking: ‘Hopefully this venerable will frequent appropriate lodgings, associate with good friends, and control their faculties. Then they might realize the supreme culmination of the spiritual path in this very life, and live having achieved with their own insight the goal for which gentlemen rightly go forth from the lay life to homelessness.’ Seeing this fruit of diligence for this mendicant, I say that they still have work to do with diligence. 

And\marginnote{21.1} what person is a follower by faith? It’s a person who doesn’t have direct meditative experience of the peaceful liberations that are formless, transcending form. Nevertheless, having seen with wisdom, some of their defilements have come to an end. And they have a degree of faith and love for the Realized One. And they have the following qualities: the faculties of faith, energy, mindfulness, immersion, and wisdom. This person is called a follower by faith. I say that this mendicant also still has work to do with diligence. Why is that? Thinking: ‘Hopefully this venerable will frequent appropriate lodgings, associate with good friends, and control their faculties. Then they might realize the supreme culmination of the spiritual path in this very life, and live having achieved with their own insight the goal for which gentlemen rightly go forth from the lay life to homelessness.’ Seeing this fruit of diligence for this mendicant, I say that they still have work to do with diligence. 

Mendicants,\marginnote{22.1} I don’t say that enlightenment is achieved right away. Rather, enlightenment is achieved by gradual training, progress, and practice. 

And\marginnote{23.1} how is enlightenment achieved by gradual training, progress, and practice? It’s when someone in whom faith has arisen approaches a teacher. They pay homage, lend an ear, hear the teachings, remember the teachings, reflect on their meaning, and accept them after consideration. Then enthusiasm springs up; they make an effort, weigh up, and persevere. Persevering, they directly realize the ultimate truth, and see it with penetrating wisdom. 

Mendicants,\marginnote{24.1} there has not been that faith, that approaching, that paying homage, that listening, that hearing the teachings, that remembering the teachings, that reflecting on their meaning, that acceptance after consideration, that enthusiasm, that making an effort, that weighing up, or that striving. You’ve lost the way, mendicants! You’re practicing the wrong way! Just how far have these foolish people strayed from this teaching and training! 

There\marginnote{25.1} is an exposition in four parts, which a sensible person would quickly understand when it is recited. I shall recite it for you, mendicants. Try to understand it.” 

“Sir,\marginnote{25.3} who are we to be counted alongside those who understand the teaching?” 

“Even\marginnote{26.1} with a teacher who values material things, is an heir in material things, who lives caught up in material things, you wouldn’t get into such haggling: ‘If we get this, we’ll do that. If we don’t get this, we won’t do it.’ What then of the Realized One, who lives utterly detached from material things? 

For\marginnote{27.1} a faithful disciple who is practicing to fathom the Teacher’s instructions, this is in line with the teaching: ‘The Buddha is my Teacher, I am his disciple. The Buddha knows, I do not know.’ For a faithful disciple who is practicing to fathom the Teacher’s instructions, the Teacher’s instructions are nourishing and nutritious. For a faithful disciple who is practicing to fathom the Teacher’s instructions, this is in line with the teaching: ‘Gladly, let only skin, sinews, and bones remain! Let the flesh and blood waste away in my body! I will not relax my energy until I have achieved what is possible by human strength, energy, and vigor.’ A faithful disciple who is practicing to fathom the Teacher’s instructions can expect one of two results: enlightenment in the present life, or if there’s something left over, non-return.” 

That\marginnote{27.9} is what the Buddha said. Satisfied, the mendicants were happy with what the Buddha said. 

%
\addtocontents{toc}{\let\protect\contentsline\protect\nopagecontentsline}
\chapter*{The Chapter on Wanderers}
\addcontentsline{toc}{chapter}{\tocchapterline{The Chapter on Wanderers}}
\addtocontents{toc}{\let\protect\contentsline\protect\oldcontentsline}

%
\section*{{\suttatitleacronym MN 71}{\suttatitletranslation To Vacchagotta on the Three Knowledges }{\suttatitleroot Tevijjavacchasutta}}
\addcontentsline{toc}{section}{\tocacronym{MN 71} \toctranslation{To Vacchagotta on the Three Knowledges } \tocroot{Tevijjavacchasutta}}
\markboth{To Vacchagotta on the Three Knowledges }{Tevijjavacchasutta}
\extramarks{MN 71}{MN 71}

\scevam{So\marginnote{1.1} I have heard. }At one time the Buddha was staying near \textsanskrit{Vesālī}, at the Great Wood, in the hall with the peaked roof. 

Now\marginnote{2.1} at that time the wanderer Vacchagotta was residing in the Single Lotus Monastery of the wanderers. 

Then\marginnote{3.1} the Buddha robed up in the morning and, taking his bowl and robe, entered \textsanskrit{Vesālī} for alms. Then it occurred to him, “It’s too early to wander for alms in \textsanskrit{Vesālī}. Why don’t I visit the wanderer Vacchagotta at the Single Lotus Monastery?” So that’s what he did. 

Vacchagotta\marginnote{4.1} saw the Buddha coming off in the distance, and said to him, “Come, Blessed One! Welcome, Blessed One! It’s been a long time since you took the opportunity to come here. Please, sir, sit down, this seat is ready.” 

The\marginnote{4.7} Buddha sat on the seat spread out, while Vacchagotta took a low seat and sat to one side. Then Vacchagotta said to the Buddha: 

“Sir,\marginnote{5.1} I have heard this: ‘The ascetic Gotama claims to be all-knowing and all-seeing, to know and see everything without exception, thus: “Knowledge and vision are constantly and continually present to me, while walking, standing, sleeping, and waking.”’ I trust that those who say this repeat what the Buddha has said, and do not misrepresent him with an untruth? Is their explanation in line with the teaching? Are there any legitimate grounds for rebuke and criticism?” 

“Vaccha,\marginnote{5.5} those who say this do not repeat what I have said. They misrepresent me with what is false and untrue.” 

“So\marginnote{6.1} how should we answer so as to repeat what the Buddha has said, and not misrepresent him with an untruth? How should we explain in line with his teaching, with no legitimate grounds for rebuke and criticism?” 

“‘The\marginnote{6.2} ascetic Gotama has the three knowledges.’ Answering like this you would repeat what I have said, and not misrepresent me with an untruth. You would explain in line with my teaching, and there would be no legitimate grounds for rebuke and criticism. 

For,\marginnote{7.1} Vaccha, whenever I want, I recollect my many kinds of past lives. That is: one, two, three, four, five, ten, twenty, thirty, forty, fifty, a hundred, a thousand, a hundred thousand rebirths; many eons of the world contracting, many eons of the world expanding, many eons of the world contracting and expanding. I remember: ‘There, I was named this, my clan was that, I looked like this, and that was my food. This was how I felt pleasure and pain, and that was how my life ended. When I passed away from that place I was reborn somewhere else. There, too, I was named this, my clan was that, I looked like this, and that was my food. This was how I felt pleasure and pain, and that was how my life ended. When I passed away from that place I was reborn here.’ And so I recollect my many kinds of past lives, with features and details. 

And\marginnote{8.1} whenever I want, with clairvoyance that is purified and superhuman, I see sentient beings passing away and being reborn—inferior and superior, beautiful and ugly, in a good place or a bad place. I understand how sentient beings are reborn according to their deeds. 

And\marginnote{9.1} I have realized the undefiled freedom of heart and freedom by wisdom in this very life. I live having realized it with my own insight due to the ending of defilements. 

‘The\marginnote{10.1} ascetic Gotama has the three knowledges.’ Answering like this you would repeat what I have said, and not misrepresent me with an untruth. You would explain in line with my teaching, and there would be no legitimate grounds for rebuke and criticism.” 

When\marginnote{11.1} he said this, the wanderer Vacchagotta said to the Buddha, “Master Gotama, are there any laypeople who, without giving up the fetter of lay life, make an end of suffering when the body breaks up?” 

“No,\marginnote{11.3} Vaccha.” 

“But\marginnote{12.1} are there any laypeople who, without giving up the fetter of lay life, go to heaven when the body breaks up?” 

“There’s\marginnote{12.2} not just one hundred laypeople, Vaccha, or two or three or four or five hundred, but many more than that who, without giving up the fetter of lay life, go to heaven when the body breaks up.” 

“Master\marginnote{13.1} Gotama, are there any \textsanskrit{Ājīvaka} ascetics who make an end of suffering when the body breaks up?” 

“No,\marginnote{13.2} Vaccha.” 

“But\marginnote{14.1} are there any \textsanskrit{Ājīvaka} ascetics who go to heaven when the body breaks up?” 

“Vaccha,\marginnote{14.2} when I recollect the past ninety-one eons, I can’t find any \textsanskrit{Ājīvaka} ascetics who have gone to heaven, except one; and he taught the efficacy of deeds and action.” 

“In\marginnote{15.1} that case, Master Gotama, the sectarian tenets are empty even of the chance to go to heaven.” 

“Yes,\marginnote{15.2} Vaccha, the sectarian tenets are empty even of the chance to go to heaven.” 

That\marginnote{15.3} is what the Buddha said. Satisfied, the wanderer Vacchagotta was happy with what the Buddha said. 

%
\section*{{\suttatitleacronym MN 72}{\suttatitletranslation With Vacchagotta on Fire }{\suttatitleroot Aggivacchasutta}}
\addcontentsline{toc}{section}{\tocacronym{MN 72} \toctranslation{With Vacchagotta on Fire } \tocroot{Aggivacchasutta}}
\markboth{With Vacchagotta on Fire }{Aggivacchasutta}
\extramarks{MN 72}{MN 72}

\scevam{So\marginnote{1.1} I have heard. }At one time the Buddha was staying near \textsanskrit{Sāvatthī} in Jeta’s Grove, \textsanskrit{Anāthapiṇḍika}’s monastery. 

Then\marginnote{2.1} the wanderer Vacchagotta went up to the Buddha and exchanged greetings with him. When the greetings and polite conversation were over, he sat down to one side and said to the Buddha: 

“Master\marginnote{3.1} Gotama, is this your view: ‘The cosmos is eternal. This is the only truth, other ideas are silly’?” 

“That’s\marginnote{3.3} not my view, Vaccha.” 

“Then\marginnote{4.1} is this your view: ‘The cosmos is not eternal. This is the only truth, other ideas are silly’?” 

“That’s\marginnote{4.3} not my view, Vaccha.” 

“Then\marginnote{5.1} is this your view: ‘The world is finite. This is the only truth, other ideas are silly’?” 

“That’s\marginnote{5.3} not my view, Vaccha.” 

“Then\marginnote{6.1} is this your view: ‘The world is infinite. This is the only truth, other ideas are silly’?” 

“That’s\marginnote{6.3} not my view, Vaccha.” 

“Then\marginnote{7.1} is this your view: ‘The soul and the body are the same thing. This is the only truth, other ideas are silly’?” 

“That’s\marginnote{7.3} not my view, Vaccha.” 

“Then\marginnote{8.1} is this your view: ‘The soul and the body are different things. This is the only truth, other ideas are silly’?” 

“That’s\marginnote{8.3} not my view, Vaccha.” 

“Then\marginnote{9.1} is this your view: ‘A Realized One exists after death. This is the only truth, other ideas are silly’?” 

“That’s\marginnote{9.3} not my view, Vaccha.” 

“Then\marginnote{10.1} is this your view: ‘A Realized One doesn’t exist after death. This is the only truth, other ideas are silly’?” 

“That’s\marginnote{10.3} not my view, Vaccha.” 

“Then\marginnote{11.1} is this your view: ‘A Realized One both exists and doesn’t exist after death. This is the only truth, other ideas are silly’?” 

“That’s\marginnote{11.3} not my view, Vaccha.” 

“Then\marginnote{12.1} is this your view: ‘A Realized One neither exists nor doesn’t exist after death. This is the only truth, other ideas are silly’?” 

“That’s\marginnote{12.3} not my view, Vaccha.” 

“Master\marginnote{13.1} Gotama, when asked these ten questions, you say: ‘That’s not my view.’ Seeing what drawback do you avoid all these convictions?” 

“Each\marginnote{14.1} of these ten convictions is the thicket of views, the desert of views, the trick of views, the evasiveness of views, the fetter of views. They’re beset with anguish, distress, and fever. They don’t lead to disillusionment, dispassion, cessation, peace, insight, awakening, and extinguishment. Seeing this drawback I avoid all these convictions.” 

“But\marginnote{15.1} does Master Gotama have any convictions at all?” 

“The\marginnote{15.2} Realized One has done away with convictions. For the Realized One has seen: ‘Such is form, such is the origin of form, such is the ending of form. Such is feeling, such is the origin of feeling, such is the ending of feeling. Such is perception, such is the origin of perception, such is the ending of perception. Such are choices, such is the origin of choices, such is the ending of choices. Such is consciousness, such is the origin of consciousness, such is the ending of consciousness.’ That’s why the Realized One is freed with the ending, fading away, cessation, giving up, and letting go of all identifying, all worries, and all ego, possessiveness, or underlying tendency to conceit, I say.” 

“But\marginnote{16.1} Master Gotama, when a mendicant’s mind is freed like this, where are they reborn?” 

“‘They’re\marginnote{16.2} reborn’ doesn’t apply, Vaccha.” 

“Well\marginnote{16.3} then, are they not reborn?” 

“‘They’re\marginnote{16.4} not reborn’ doesn’t apply, Vaccha.” 

“Well\marginnote{16.5} then, are they both reborn and not reborn?” 

“‘They’re\marginnote{16.6} both reborn and not reborn’ doesn’t apply, Vaccha.” 

“Well\marginnote{16.7} then, are they neither reborn nor not reborn?” 

“‘They’re\marginnote{16.8} neither reborn nor not reborn’ doesn’t apply, Vaccha.” 

“Master\marginnote{17.1} Gotama, when asked all these questions, you say: ‘It doesn’t apply.’ I fail to understand this point, Master Gotama; I’ve fallen into confusion. And I’ve now lost even the degree of clarity I had from previous discussions with Master Gotama.” 

“No\marginnote{18.1} wonder you don’t understand, Vaccha, no wonder you’re confused. For this principle is deep, hard to see, hard to understand, peaceful, sublime, beyond the scope of logic, subtle, comprehensible to the astute. It’s hard for you to understand, since you have a different view, creed, preference, practice, and tradition. 

Well\marginnote{18.4} then, Vaccha, I’ll ask you about this in return, and you can answer as you like. 

What\marginnote{19.1} do you think, Vaccha? Suppose a fire was burning in front of you. Would you know: ‘This fire is burning in front of me’?” 

“Yes,\marginnote{19.4} I would, Master Gotama.” 

“But\marginnote{19.6} Vaccha, suppose they were to ask you: ‘This fire burning in front of you: what does it depend on to burn?’ How would you answer?” 

“I\marginnote{19.9} would answer like this: ‘This fire burning in front of me burns in dependence on grass and logs as fuel.’” 

“Suppose\marginnote{19.11} that fire burning in front of you was extinguished. Would you know: ‘This fire in front of me is extinguished’?” 

“Yes,\marginnote{19.13} I would, Master Gotama.” 

“But\marginnote{19.15} Vaccha, suppose they were to ask you: ‘This fire in front of you that is extinguished: in what direction did it go—east, south, west, or north?’ How would you answer?” 

“It\marginnote{19.18} doesn’t apply, Master Gotama. The fire depended on grass and logs as fuel. When that runs out, and no more fuel is added, the fire is reckoned to have become extinguished due to lack of fuel.” 

“In\marginnote{20.1} the same way, Vaccha, any form by which a Realized One might be described has been cut off at the root, made like a palm stump, obliterated, and unable to arise in the future. A Realized One is freed from reckoning in terms of form. They’re deep, immeasurable, and hard to fathom, like the ocean. ‘They’re reborn’, ‘they’re not reborn’, ‘they’re both reborn and not reborn’, ‘they’re neither reborn nor not reborn’—none of these apply. 

Any\marginnote{20.5} feeling … perception … choices … consciousness by which a Realized One might be described has been cut off at the root, made like a palm stump, obliterated, and unable to arise in the future. A Realized One is freed from reckoning in terms of consciousness. They’re deep, immeasurable, and hard to fathom, like the ocean. ‘They’re reborn’, ‘they’re not reborn’, ‘they’re both reborn and not reborn’, ‘they’re neither reborn nor not reborn’—none of these apply.” 

When\marginnote{21.1} he said this, the wanderer Vacchagotta said to the Buddha: 

“Master\marginnote{21.2} Gotama, suppose there was a large sal tree not far from a town or village. And because it’s impermanent, its branches and foliage, bark and shoots, and softwood would fall off. After some time it would be rid of branches and foliage, bark and shoots, and softwood, consisting purely of heartwood. In the same way, Master Gotama’s dispensation is rid of branches and foliage, bark and shoots, and softwood, consisting purely of heartwood. 

Excellent,\marginnote{22.1} Master Gotama! … From this day forth, may Master Gotama remember me as a lay follower who has gone for refuge for life.” 

%
\section*{{\suttatitleacronym MN 73}{\suttatitletranslation The Longer Discourse With Vacchagotta }{\suttatitleroot Mahāvacchasutta}}
\addcontentsline{toc}{section}{\tocacronym{MN 73} \toctranslation{The Longer Discourse With Vacchagotta } \tocroot{Mahāvacchasutta}}
\markboth{The Longer Discourse With Vacchagotta }{Mahāvacchasutta}
\extramarks{MN 73}{MN 73}

\scevam{So\marginnote{1.1} I have heard. }At one time the Buddha was staying near \textsanskrit{Rājagaha}, in the Bamboo Grove, the squirrels’ feeding ground. 

Then\marginnote{2.1} the wanderer Vacchagotta went up to the Buddha and exchanged greetings with him. When the greetings and polite conversation were over, he sat down to one side and said to the Buddha, “For a long time I have had discussions with Master Gotama. Please teach me in brief what is skillful and what is unskillful.” 

“Vaccha,\marginnote{3.3} I can teach you what is skillful and what is unskillful in brief or in detail. Still, let me do so in brief. Listen and pay close attention, I will speak.” 

“Yes,\marginnote{3.6} sir,” Vaccha replied. The Buddha said this: 

“Greed\marginnote{4.1} is unskillful, contentment is skillful. Hate is unskillful, love is skillful. Delusion is unskillful, understanding is skillful. So there are these three unskillful things and three that are skillful. 

Killing\marginnote{5.1} living creatures, stealing, and sexual misconduct; speech that’s false, divisive, harsh, or nonsensical; covetousness, ill will, and wrong view: these things are unskillful. Refraining from killing living creatures, stealing, and sexual misconduct; refraining from speech that’s false, divisive, harsh, or nonsensical; contentment, kind-heartedness, and right view: these things are skillful. So there are these ten unskillful things and ten that are skillful. 

When\marginnote{6.1} a mendicant has given up craving so it is cut off at the root, made like a palm stump, obliterated, and unable to arise in the future, that mendicant is perfected. They’ve ended the defilements, completed the spiritual journey, done what had to be done, laid down the burden, achieved their own true goal, utterly ended the fetters of rebirth, and are rightly freed through enlightenment.” 

“Leaving\marginnote{7.1} aside Master Gotama, is there even a single monk disciple of Master Gotama who has realized the undefiled freedom of heart and freedom by wisdom in this very life, and lives having realized it with their own insight due to the ending of defilements?” 

“There\marginnote{7.3} are not just one hundred such monks who are my disciples, Vaccha, or two or three or four or five hundred, but many more than that.” 

“Leaving\marginnote{8.1} aside Master Gotama and the monks, is there even a single nun disciple of Master Gotama who has realized the undefiled freedom of heart and freedom by wisdom in this very life, and lives having realized it with their own insight due to the ending of defilements?” 

“There\marginnote{8.3} are not just one hundred such nuns who are my disciples, Vaccha, or two or three or four or five hundred, but many more than that.” 

“Leaving\marginnote{9.1} aside Master Gotama, the monks, and the nuns, is there even a single layman disciple of Master Gotama—white-clothed and celibate—who, with the ending of the five lower fetters, is reborn spontaneously, to be extinguished there, not liable to return from that world?” 

“There\marginnote{9.3} are not just one hundred such celibate laymen who are my disciples, Vaccha, or two or three or four or five hundred, but many more than that.” 

“Leaving\marginnote{10.1} aside Master Gotama, the monks, the nuns, and the celibate laymen, is there even a single layman disciple of Master Gotama—white-clothed, enjoying sensual pleasures, following instructions, and responding to advice—who has gone beyond doubt, got rid of indecision, and lives self-assured and independent of others regarding the Teacher’s instruction?” 

“There\marginnote{10.3} are not just one hundred such laymen enjoying sensual pleasures who are my disciples, Vaccha, or two or three or four or five hundred, but many more than that.” 

“Leaving\marginnote{11.1} aside Master Gotama, the monks, the nuns, the celibate laymen, and the laymen enjoying sensual pleasures, is there even a single laywoman disciple of Master Gotama—white-clothed and celibate—who, with the ending of the five lower fetters, is reborn spontaneously, to be extinguished there, not liable to return from that world?” 

“There\marginnote{11.3} are not just one hundred such celibate laywomen who are my disciples, Vaccha, or two or three or four or five hundred, but many more than that.” 

“Leaving\marginnote{12.1} aside Master Gotama, the monks, the nuns, the celibate laymen, the laymen enjoying sensual pleasures, and the celibate laywomen, is there even a single laywoman disciple of Master Gotama—white-clothed, enjoying sensual pleasures, following instructions, and responding to advice—who has gone beyond doubt, got rid of indecision, and lives self-assured and independent of others regarding the Teacher’s instruction?” 

“There\marginnote{12.3} are not just one hundred such laywomen enjoying sensual pleasures who are my disciples, Vaccha, or two or three or four or five hundred, but many more than that.” 

“If\marginnote{13.1} Master Gotama was the only one to succeed in this teaching, not any monks, then this spiritual path would be incomplete in that respect. But because both Master Gotama and monks have succeeded in this teaching, this spiritual path is complete in that respect. 

If\marginnote{13.5} Master Gotama and the monks were the only ones to succeed in this teaching, not any nuns … celibate laymen … laymen enjoying sensual pleasures … celibate laywomen … 

laywomen\marginnote{13.21} enjoying sensual pleasures, then this spiritual path would be incomplete in that respect. But because Master Gotama, monks, nuns, celibate laymen, laymen enjoying sensual pleasures, celibate laywomen, and laywomen enjoying sensual pleasures have all succeeded in this teaching, this spiritual path is complete in that respect. 

Just\marginnote{14.1} as the Ganges river slants, slopes, and inclines towards the ocean, and keeps pushing into the ocean, in the same way Master Gotama’s assembly—with both laypeople and renunciates—slants, slopes, and inclines towards extinguishment, and keeps pushing into extinguishment. 

Excellent,\marginnote{15.1} Master Gotama! … I go for refuge to Master Gotama, to the teaching, and to the mendicant \textsanskrit{Saṅgha}. Sir, may I receive the going forth, the ordination in the Buddha’s presence?” 

“Vaccha,\marginnote{16.1} if someone formerly ordained in another sect wishes to take the going forth, the ordination in this teaching and training, they must spend four months on probation. When four months have passed, if the mendicants are satisfied, they’ll give the going forth, the ordination into monkhood. However, I have recognized individual differences in this matter.” 

“Sir,\marginnote{16.3} if four months probation are required in such a case, I’ll spend four years on probation. When four years have passed, if the mendicants are satisfied, let them give me the going forth, the ordination into monkhood.” And the wanderer Vaccha received the going forth, the ordination in the Buddha’s presence. 

Not\marginnote{17.1} long after his ordination, a fortnight later, Venerable Vacchagotta went to the Buddha, bowed, sat down to one side, and said to him, “Sir, I’ve reached as far as possible with the knowledge and understanding of a trainee. Please teach me further.” 

“Well\marginnote{18.1} then, Vaccha, further develop two things: serenity and discernment. When you have further developed these two things, they’ll lead to the penetration of many elements. 

Whenever\marginnote{19.1} you want, you’ll be capable of realizing the following, in each and every case: ‘May I wield the many kinds of psychic power: multiplying myself and becoming one again; appearing and disappearing; going unimpeded through a wall, a rampart, or a mountain as if through space; diving in and out of the earth as if it were water; walking on water as if it were earth; flying cross-legged through the sky like a bird; touching and stroking with my hand the sun and moon, so mighty and powerful; controlling my body as far as the \textsanskrit{Brahmā} realm.’ 

Whenever\marginnote{20.1} you want, you’ll be capable of realizing the following, in each and every case: ‘With clairaudience that is purified and superhuman, may I hear both kinds of sounds, human and divine, whether near or far.’ 

Whenever\marginnote{21.1} you want, you’ll be capable of realizing the following, in each and every case: ‘May I understand the minds of other beings and individuals, having comprehended them with my mind. May I understand mind with greed as “mind with greed”, and mind without greed as “mind without greed”; mind with hate as “mind with hate”, and mind without hate as “mind without hate”; mind with delusion as “mind with delusion”, and mind without delusion as “mind without delusion”; constricted mind as “constricted mind”, and scattered mind as “scattered mind”; expansive mind as “expansive mind”, and unexpansive mind as “unexpansive mind”; mind that is not supreme as “mind that is not supreme”, and mind that is supreme as “mind that is supreme”; mind immersed in \textsanskrit{samādhi} as “mind immersed in \textsanskrit{samādhi}”, and mind not immersed in \textsanskrit{samādhi} as “mind not immersed in \textsanskrit{samādhi}”; freed mind as “freed mind”, and unfreed mind as “unfreed mind”.’ 

Whenever\marginnote{22.1} you want, you’ll be capable of realizing the following, in each and every case: ‘May I recollect many kinds of past lives. That is: one, two, three, four, five, ten, twenty, thirty, forty, fifty, a hundred, a thousand, a hundred thousand rebirths; many eons of the world contracting, many eons of the world expanding, many eons of the world contracting and expanding. May I remember: “There, I was named this, my clan was that, I looked like this, and that was my food. This was how I felt pleasure and pain, and that was how my life ended. When I passed away from that place I was reborn somewhere else. There, too, I was named this, my clan was that, I looked like this, and that was my food. This was how I felt pleasure and pain, and that was how my life ended. When I passed away from that place I was reborn here.” May I recollect my many past lives, with features and details.’ 

Whenever\marginnote{23.1} you want, you’ll be capable of realizing the following, in each and every case: ‘With clairvoyance that is purified and superhuman, may I see sentient beings passing away and being reborn—inferior and superior, beautiful and ugly, in a good place or a bad place—and understand how sentient beings are reborn according to their deeds: “These dear beings did bad things by way of body, speech, and mind. They spoke ill of the noble ones; they had wrong view; and they chose to act out of that wrong view. When their body breaks up, after death, they’re reborn in a place of loss, a bad place, the underworld, hell. These dear beings, however, did good things by way of body, speech, and mind. They never spoke ill of the noble ones; they had right view; and they chose to act out of that right view. When their body breaks up, after death, they’re reborn in a good place, a heavenly realm.” And so, with clairvoyance that is purified and superhuman, may I see sentient beings passing away and being reborn—inferior and superior, beautiful and ugly, in a good place or a bad place. And may I understand how sentient beings are reborn according to their deeds.’ 

Whenever\marginnote{24.1} you want, you’ll be capable of realizing the following, in each and every case: ‘May I realize the undefiled freedom of heart and freedom by wisdom in this very life, and live having realized it with my own insight due to the ending of defilements.’ 

And\marginnote{25.1} then Venerable Vacchagotta approved and agreed with what the Buddha said. He got up from his seat, bowed, and respectfully circled the Buddha, keeping him on his right, before leaving. 

Then\marginnote{26.1} Vacchagotta, living alone, withdrawn, diligent, keen, and resolute, soon realized the supreme end of the spiritual path in this very life. He lived having achieved with his own insight the goal for which gentlemen rightly go forth from the lay life to homelessness. 

He\marginnote{26.2} understood: “Rebirth is ended; the spiritual journey has been completed; what had to be done has been done; there is no return to any state of existence.” And Venerable Vacchagotta became one of the perfected. 

Now\marginnote{27.1} at that time several mendicants were going to see the Buddha. Vacchagotta saw them coming off in the distance, went up to them, and said, “Hello venerables, where are you going?” 

“Reverend,\marginnote{27.5} we are going to see the Buddha.” 

“Well\marginnote{27.6} then, reverends, in my name please bow with your head at the Buddha’s feet and say: ‘Sir, the mendicant Vacchagotta bows with his head to your feet and says, “I have served the Blessed One! I have served the Holy One!”’” 

“Yes,\marginnote{28.1} reverend,” they replied. Then those mendicants went up to the Buddha, bowed, sat down to one side, and said to him, “Sir, the mendicant Vacchagotta bows with his head to your feet and says: ‘I have served the Blessed One! I have served the Holy One!’” 

“I’ve\marginnote{28.5} already comprehended Vacchagotta’s mind and understood that he has the three knowledges, and is very mighty and powerful. And deities also told me about this.” 

That\marginnote{28.9} is what the Buddha said. Satisfied, the mendicants were happy with what the Buddha said. 

%
\section*{{\suttatitleacronym MN 74}{\suttatitletranslation With Dīghanakha }{\suttatitleroot Dīghanakhasutta}}
\addcontentsline{toc}{section}{\tocacronym{MN 74} \toctranslation{With Dīghanakha } \tocroot{Dīghanakhasutta}}
\markboth{With Dīghanakha }{Dīghanakhasutta}
\extramarks{MN 74}{MN 74}

\scevam{So\marginnote{1.1} I have heard. }At one time the Buddha was staying near \textsanskrit{Rājagaha}, on the Vulture’s Peak Mountain in the Boar’s Cave. 

Then\marginnote{2.1} the wanderer \textsanskrit{Dīghanakha} went up to the Buddha, and exchanged greetings with him. When the greetings and polite conversation were over, he stood to one side, and said to the Buddha, “Master Gotama, this is my doctrine and view: ‘I believe in nothing.’” 

“This\marginnote{2.5} view of yours, Aggivessana—do you believe in that?” 

“If\marginnote{2.7} I believed in this view, Master Gotama, it wouldn’t make any difference, it wouldn’t make any difference!” 

“Well,\marginnote{3.1} Aggivessana, there are many more in the world who say, ‘It wouldn’t make any difference! It wouldn’t make any difference!’ But they don’t give up that view, and they grasp another view. And there are a scant few in the world who say, ‘It wouldn’t make any difference! It wouldn’t make any difference!’ And they give up that view by not grasping another view. 

There\marginnote{4.1} are some ascetics and brahmins who have this doctrine and view: ‘I believe in everything.’ There are some ascetics and brahmins who have this doctrine and view: ‘I believe in nothing.’ There are some ascetics and brahmins who have this doctrine and view: ‘I believe in some things, and not in others.’ Regarding this, the view of the ascetics and brahmins who believe in everything is close to greed, bondage, approving, attachment, and grasping. The view of the ascetics and brahmins who believe in nothing is far from greed, bondage, approving, attachment, and grasping.” 

When\marginnote{5.1} he said this, the wanderer \textsanskrit{Dīghanakha} said to the Buddha, “Master Gotama commends my conviction! He recommends my conviction!” 

“Now,\marginnote{5.3} regarding the ascetics and brahmins who believe in some things and not in others. Their view of what they believe in is close to greed, bondage, approving, attachment, and grasping. Their view of what they don’t believe in is far from greed, bondage, approving, attachment, and grasping. 

When\marginnote{6.1} it comes to the view of the ascetics and brahmins who believe in everything, a sensible person reflects like this: ‘I have the view that I believe in everything. Suppose I obstinately stick to this view and insist that, “This is the only truth, other ideas are silly.” Then I’d argue with two people—an ascetic or brahmin who believes in nothing, and an ascetic or brahmin who believes in some things and not in others. And when there’s arguing, there’s quarreling; when there’s quarreling there’s anguish; and when there’s anguish there’s harm.’ So, considering in themselves the potential for arguing, quarreling, anguish, and harm, they give up that view by not grasping another view. That’s how those views are given up and let go. 

When\marginnote{7.1} it comes to the view of the ascetics and brahmins who believe in nothing, a sensible person reflects like this: ‘I have the view that I believe in nothing. Suppose I obstinately stick to this view and insist that, “This is the only truth, other ideas are silly.” Then I’d argue with two people—an ascetic or brahmin who believes in everything, and an ascetic or brahmin who believes in some things and not in others. And when there’s arguing, there’s quarreling; when there’s quarreling there’s anguish; and when there’s anguish there’s harm.’ So, considering in themselves the potential for arguing, quarreling, anguish, and harm, they give up that view by not grasping another view. That’s how those views are given up and let go. 

When\marginnote{8.1} it comes to the view of the ascetics and brahmins who believe in some things and not in others, a sensible person reflects like this: ‘I have the view that I believe in some things and not in others. Suppose I obstinately stick to this view and insist that, “This is the only truth, other ideas are silly.” Then I’d argue with two people—an ascetic or brahmin who believes in everything, and an ascetic or brahmin who believes in nothing. And when there’s arguing, there’s quarreling; when there’s quarreling there’s anguish; and when there’s anguish there’s harm.’ So, considering in themselves the potential for arguing, quarreling, anguish, and harm, they give up that view by not grasping another view. That’s how those views are given up and let go. 

Aggivessana,\marginnote{9.1} this body is physical. It’s made up of the four primary elements, produced by mother and father, built up from rice and porridge, liable to impermanence, to wearing away and erosion, to breaking up and destruction. You should see it as impermanent, as suffering, as diseased, as a boil, as a dart, as misery, as an affliction, as alien, as falling apart, as empty, as not-self. Doing so, you’ll give up desire, affection, and subservience to the body. 

There\marginnote{10.1} are these three feelings: pleasant, painful, and neutral. At a time when you feel a pleasant feeling, you don’t feel a painful or neutral feeling; you only feel a pleasant feeling. At a time when you feel a painful feeling, you don’t feel a pleasant or neutral feeling; you only feel a painful feeling. At a time when you feel a neutral feeling, you don’t feel a pleasant or painful feeling; you only feel a neutral feeling. 

Pleasant,\marginnote{11.1} painful, and neutral feelings are impermanent, conditioned, dependently originated, liable to end, vanish, fade away, and cease. 

Seeing\marginnote{12.1} this, a learned noble disciple grows disillusioned with pleasant, painful, and neutral feelings. Being disillusioned, desire fades away. When desire fades away they’re freed. When they’re freed, they know they’re freed. 

They\marginnote{12.3} understand: ‘Rebirth is ended, the spiritual journey has been completed, what had to be done has been done, there is no return to any state of existence.’ 

A\marginnote{13.1} mendicant whose mind is freed like this doesn’t side with anyone or dispute with anyone. They speak the language of the world without misapprehending it.” 

Now\marginnote{14.1} at that time Venerable \textsanskrit{Sāriputta} was standing behind the Buddha fanning him. Then he thought, “It seems the Buddha speaks of giving up and letting go all these things through direct knowledge.” Reflecting like this, Venerable \textsanskrit{Sāriputta}’s mind was freed from the defilements by not grasping. 

And\marginnote{15.1} the stainless, immaculate vision of the Dhamma arose in the wanderer \textsanskrit{Dīghanakha}: “Everything that has a beginning has an end.” Then \textsanskrit{Dīghanakha} saw, attained, understood, and fathomed the Dhamma. He went beyond doubt, got rid of indecision, and became self-assured and independent of others regarding the Teacher’s instructions. He said to the Buddha: 

“Excellent,\marginnote{16.1} Master Gotama! Excellent! As if he were righting the overturned, or revealing the hidden, or pointing out the path to the lost, or lighting a lamp in the dark so people with good eyes can see what’s there, Master Gotama has made the teaching clear in many ways. I go for refuge to Master Gotama, to the teaching, and to the mendicant \textsanskrit{Saṅgha}. From this day forth, may Master Gotama remember me as a lay follower who has gone for refuge for life.” 

%
\section*{{\suttatitleacronym MN 75}{\suttatitletranslation With Māgaṇḍiya }{\suttatitleroot Māgaṇḍiyasutta}}
\addcontentsline{toc}{section}{\tocacronym{MN 75} \toctranslation{With Māgaṇḍiya } \tocroot{Māgaṇḍiyasutta}}
\markboth{With Māgaṇḍiya }{Māgaṇḍiyasutta}
\extramarks{MN 75}{MN 75}

\scevam{So\marginnote{1.1} I have heard. }At one time the Buddha was staying in the land of the Kurus, near the Kuru town named \textsanskrit{Kammāsadamma}, on a grass mat in the fire chamber of a brahmin of the \textsanskrit{Bhāradvāja} clan. 

Then\marginnote{2.1} the Buddha robed up in the morning and, taking his bowl and robe, entered \textsanskrit{Kammāsadamma} for alms. He wandered for alms in \textsanskrit{Kammāsadamma}. After the meal, on his return from almsround, he went to a certain forest grove for the day’s meditation. Having plunged deep into it, he sat at the root of a certain tree for the day’s meditation. 

Then\marginnote{3.1} as the wanderer \textsanskrit{Māgaṇḍiya} was going for a walk he approached that fire chamber. He saw the grass mat spread out there and asked the brahmin of the \textsanskrit{Bhāradvāja} clan, “Mister \textsanskrit{Bhāradvāja}, who has this grass mat been spread out for? It looks like an ascetic’s bed.” 

“There\marginnote{4.1} is the ascetic Gotama, a Sakyan, gone forth from a Sakyan family. He has this good reputation: ‘That Blessed One is perfected, a fully awakened Buddha, accomplished in knowledge and conduct, holy, knower of the world, supreme guide for those who wish to train, teacher of gods and humans, awakened, blessed.’ This bed has been spread for that Master Gotama.” 

“Well,\marginnote{5.1} it’s a sad sight, Mister \textsanskrit{Bhāradvāja}, a very sad sight indeed, to see a bed for Master Gotama, that life-destroyer!” 

“Be\marginnote{5.4} careful what you say, \textsanskrit{Māgaṇḍiya}, be careful what you say. Many astute aristocrats, brahmins, householders, and ascetics are devoted to Master Gotama. They’ve been guided by him in the noble method, the skillful teaching.” 

“Even\marginnote{5.7} if I was to see Master Gotama face to face, Mister \textsanskrit{Bhāradvāja}, I would say to his face: ‘The ascetic Gotama is a life-destroyer.’ Why is that? Because that’s what it implies in a discourse of ours.” 

“If\marginnote{5.11} you don’t mind, I’ll tell the ascetic Gotama about this.” 

“Don’t\marginnote{5.12} worry, Mister \textsanskrit{Bhāradvāja}. You may tell him exactly what I’ve said.” 

With\marginnote{6.1} clairaudience that is purified and superhuman, the Buddha heard this discussion between the brahmin of the \textsanskrit{Bhāradvāja} clan and the wanderer \textsanskrit{Māgaṇḍiya}. Coming out of retreat, he went to the brahmin’s fire chamber and sat on the grass mat. Then the brahmin of the \textsanskrit{Bhāradvāja} clan went to the Buddha and exchanged greetings with him. When the greetings and polite conversation were over, he sat down to one side. The Buddha said to him, “\textsanskrit{Bhāradvāja}, did you have a discussion with the wanderer \textsanskrit{Māgaṇḍiya} about this grass mat?” 

When\marginnote{6.6} he said this, the brahmin said to the Buddha, “I wanted to mention this very thing to Master Gotama, but you brought it up before I had a chance.” 

But\marginnote{7.1} this conversation between the Buddha and the brahmin was left unfinished. Then as the wanderer \textsanskrit{Māgaṇḍiya} was going for a walk he approached that fire chamber. He went up to the Buddha, and exchanged greetings with him. When the greetings and polite conversation were over, he sat down to one side, and the Buddha said to him: 

“\textsanskrit{Māgaṇḍiya},\marginnote{8.1} the eye likes sights, it loves them and enjoys them. That’s been tamed, guarded, protected and restrained by the Realized One, and he teaches Dhamma for its restraint. Is that what you were referring to when you called me a life-destroyer?” 

“That’s\marginnote{8.5} exactly what I was referring to. Why is that? Because that’s what it implies in a discourse of ours.” 

“The\marginnote{8.9} ear likes sounds … The nose likes smells … The tongue likes tastes … The body likes touches … The mind likes thoughts, it loves them and enjoys them. That’s been tamed, guarded, protected and restrained by the Realized One, and he teaches Dhamma for its restraint. Is that what you were referring to when you called me a life-destroyer?” 

“That’s\marginnote{8.24} exactly what I was referring to. Why is that? Because that’s what it implies in a discourse of ours.” 

“What\marginnote{9.1} do you think, \textsanskrit{Māgaṇḍiya}? Take someone who used to amuse themselves with sights known by the eye that are likable, desirable, agreeable, pleasant, sensual, and arousing. Some time later—having truly understood the origin, ending, gratification, drawback, and escape of sights, and having given up craving and dispelled passion for sights—they would live rid of thirst, their mind peaceful inside. What would you have to say to them, \textsanskrit{Māgaṇḍiya}?” 

“Nothing,\marginnote{9.4} Master Gotama.” 

“What\marginnote{9.5} do you think, \textsanskrit{Māgaṇḍiya}? Take someone who used to amuse themselves with sounds known by the ear … smells known by the nose … tastes known by the tongue … touches known by the body that are likable, desirable, agreeable, pleasant, sensual, and arousing. Some time later—having truly understood the origin, ending, gratification, drawback, and escape of touches, and having given up craving and dispelled passion for touches—they would live rid of thirst, their mind peaceful inside. What would you have to say to them, \textsanskrit{Māgaṇḍiya}?” 

“Nothing,\marginnote{9.12} Master Gotama.” 

“Well,\marginnote{10.1} when I was still a layperson I used to amuse myself, supplied and provided with sights known by the eye … sounds known by the ear … smells known by the nose … tastes known by the tongue … touches known by the body that are likable, desirable, agreeable, pleasant, sensual, and arousing. I had three stilt longhouses—one for the rainy season, one for the winter, and one for the summer. I stayed in a stilt longhouse without coming downstairs for the four months of the rainy season, where I was entertained by musicians—none of them men. Some time later—having truly understood the origin, ending, gratification, drawback, and escape of sensual pleasures, and having given up craving and dispelled passion for sensual pleasures—I live rid of thirst, my mind peaceful inside. I see other sentient beings who are not free from sensual pleasures being consumed by craving for sensual pleasures, burning with passion for sensual pleasures, indulging in sensual pleasures. I don’t envy them, nor do I hope to enjoy that. Why is that? Because there is a satisfaction that is apart from sensual pleasures and unskillful qualities, which even achieves the level of heavenly pleasure. Enjoying that satisfaction, I don’t envy what is inferior, nor do I hope to enjoy it. 

Suppose\marginnote{11.1} there was a householder or a householder’s child who was rich, affluent, and wealthy. And they would amuse themselves, supplied and provided with the five kinds of sensual stimulation. That is, sights known by the eye … sounds … smells … tastes … touches known by the body that are likable, desirable, agreeable, pleasant, sensual, and arousing. Having practiced good conduct by way of body, speech, and mind, when their body breaks up, after death, they’d be reborn in a good place, a heavenly realm, in the company of the gods of the Thirty-Three. There they’d amuse themselves in the Garden of Delight, escorted by a band of nymphs, supplied and provided with the five kinds of heavenly sensual stimulation. Then they’d see a householder or a householder’s child amusing themselves, supplied and provided with the five kinds of sensual stimulation. 

What\marginnote{11.7} do you think, \textsanskrit{Māgaṇḍiya}? Would that god—amusing themselves in the Garden of Delight, escorted by a band of nymphs, supplied and provided with the five kinds of heavenly sensual stimulation—envy that householder or householder’s child their five kinds of human sensual stimulation, or return to human sensual pleasures?” 

“No,\marginnote{11.8} Master Gotama. Why is that? Because heavenly sensual pleasures are better than human sensual pleasures.” 

“In\marginnote{12.1} the same way, \textsanskrit{Māgaṇḍiya}, when I was still a layperson I used to entertain myself with sights … sounds … smells … tastes … touches known by the body that are likable, desirable, agreeable, pleasant, sensual, and arousing. Some time later—having truly understood the origin, ending, gratification, drawback, and escape of sensual pleasures, and having given up craving and dispelled passion for sensual pleasures—I live rid of thirst, my mind peaceful inside. I see other sentient beings who are not free from sensual pleasures being consumed by craving for sensual pleasures, burning with passion for sensual pleasures, indulging in sensual pleasures. I don’t envy them, nor do I hope to enjoy that. Why is that? Because there is a satisfaction that is apart from sensual pleasures and unskillful qualities, which even achieves the level of heavenly pleasure. Enjoying that satisfaction, I don’t envy what is inferior, nor do I hope to enjoy it. 

Suppose\marginnote{13.1} there was a person affected by leprosy, with sores and blisters on their limbs. Being devoured by worms, scratching with their nails at the opening of their wounds, they’d cauterize their body over a pit of glowing coals. Their friends and colleagues, relatives and kin would get a field surgeon to treat them. The field surgeon would make medicine for them, and by using that they’d be cured of leprosy. They’d be healthy, happy, autonomous, master of themselves, able to go where they wanted. Then they’d see another person affected by leprosy, with sores and blisters on their limbs, being devoured by worms, scratching with their nails at the opening of their wounds, cauterizing their body over a pit of glowing coals. 

What\marginnote{13.6} do you think, \textsanskrit{Māgaṇḍiya}? Would that person envy that other person affected by leprosy for their pit of glowing coals or for taking medicine?” 

“No,\marginnote{13.8} Master Gotama. Why is that? Because you need to take medicine only when there’s a disease. When there’s no disease, there’s no need for medicine.” 

“In\marginnote{14.1} the same way, \textsanskrit{Māgaṇḍiya}, when I was still a layperson I used to entertain myself with sights … sounds … smells … tastes … touches known by the body that are likable, desirable, agreeable, pleasant, sensual, and arousing. Some time later—having truly understood the origin, ending, gratification, drawback, and escape of sensual pleasures, and having given up craving and dispelled passion for sensual pleasures—I live rid of thirst, my mind peaceful inside. I see other sentient beings who are not free from sensual pleasures being consumed by craving for sensual pleasures, burning with passion for sensual pleasures, indulging in sensual pleasures. I don’t envy them, nor do I hope to enjoy that. Why is that? Because there is a satisfaction that is apart from sensual pleasures and unskillful qualities, which even achieves the level of heavenly pleasure. Enjoying that satisfaction, I don’t envy what is inferior, nor do I hope to enjoy it. 

Suppose\marginnote{15.1} there was a person affected by leprosy, with sores and blisters on their limbs. Being devoured by worms, scratching with their nails at the opening of their wounds, they’d cauterize their body over a pit of glowing coals. Their friends and colleagues, relatives and kin would get a field surgeon to treat them. The field surgeon would make medicine for them, and by using that they’d be cured of leprosy. They’d be healthy, happy, autonomous, master of themselves, able to go where they wanted. Then two strong men would grab them by the arms and drag them towards the pit of glowing coals. 

What\marginnote{15.6} do you think, \textsanskrit{Māgaṇḍiya}? Wouldn’t that person writhe and struggle to and fro?” 

“Yes,\marginnote{15.8} Master Gotama. Why is that? Because that fire is really painful to touch, fiercely burning and scorching.” 

“What\marginnote{15.11} do you think, \textsanskrit{Māgaṇḍiya}? Is it only now that the fire is really painful to touch, fiercely burning and scorching, or was it painful previously as well?” 

“That\marginnote{15.13} fire is painful now and it was also painful previously. That person was affected by leprosy, with sores and blisters on their limbs. Being devoured by worms, scratching with their nails at the opening of their wounds, their sense faculties were impaired. So even though the fire was actually painful to touch, they had a distorted perception that it was pleasant.” 

“In\marginnote{16.1} the same way, sensual pleasures of the past, future, and present are painful to touch, fiercely burning and scorching. These sentient beings who are not free from sensual pleasures—being consumed by craving for sensual pleasures, burning with passion for sensual pleasures—have impaired sense faculties. So even though sensual pleasures are actually painful to touch, they have a distorted perception that they are pleasant. 

Suppose\marginnote{17.1} there was a person affected by leprosy, with sores and blisters on their limbs. Being devoured by worms, scratching with their nails at the opening of their wounds, they’re cauterizing their body over a pit of glowing coals. The more they scratch their wounds and cauterize their body, the more their wounds become foul, stinking, and infected. But still, they derive a degree of pleasure and gratification from the itchiness of their wounds. In the same way, I see other sentient beings who are not free from sensual pleasures being consumed by craving for sensual pleasures, burning with passion for sensual pleasures, indulging in sensual pleasures. The more they indulge in sensual pleasures, the more their craving for sensual pleasures grows, and the more they burn with passion for sensual pleasures. But still, they derive a degree of pleasure and gratification from the five kinds of sensual stimulation. 

What\marginnote{18.1} do you think, \textsanskrit{Māgaṇḍiya}? Have you seen or heard of a king or a royal minister of the past, future, or present, amusing themselves supplied and provided with the five kinds of sensual stimulation, who—without giving up craving for sensual pleasures and dispelling passion for sensual pleasures—lives rid of thirst, their mind peaceful inside?” 

“No,\marginnote{18.3} Master Gotama.” 

“Good,\marginnote{18.4} \textsanskrit{Māgaṇḍiya}. Neither have I. On the contrary, all the ascetics or brahmins of the past, future, or present who live rid of thirst, their minds peaceful inside, do so after truly understanding the origin, ending, gratification, drawback, and escape of sensual pleasures, and after giving up craving and dispelling passion for sensual pleasures.” 

Then\marginnote{19.1} on that occasion the Buddha expressed this heartfelt sentiment: 

\begin{verse}%
“Health\marginnote{19.2} is the ultimate blessing; \\
extinguishment, the ultimate happiness. \\
Of paths, the ultimate is eightfold—\\
it’s safe, and leads to the deathless.” 

%
\end{verse}

When\marginnote{19.6} he said this, \textsanskrit{Māgaṇḍiya} said to him, “It’s incredible, Master Gotama, it’s amazing! How well said this was by Master Gotama! ‘Health is the ultimate blessing; extinguishment, the ultimate happiness.’ I’ve also heard that wanderers of the past, the teachers of teachers, said: ‘Health is the ultimate blessing; extinguishment, the ultimate happiness.’ And it agrees, Master Gotama.” 

“But\marginnote{19.13} \textsanskrit{Māgaṇḍiya}, when you heard that wanderers of the past said this, what is that health? And what is that extinguishment?” When he said this, \textsanskrit{Māgaṇḍiya} stroked his own limbs with his hands, saying: 

“This\marginnote{19.16} is that health, Master Gotama, this is that extinguishment! For I am now healthy and happy, and have no afflictions.” 

“\textsanskrit{Māgaṇḍiya},\marginnote{20.1} suppose a person was born blind. They couldn’t see sights that are dark or bright, or blue, yellow, red, or magenta. They couldn’t see even and uneven ground, or the stars, or the moon and sun. They might hear a sighted person saying: ‘White cloth is really nice, it’s attractive, stainless, and clean.’ They’d go in search of white cloth. But someone would cheat them with a dirty, soiled garment, saying: ‘Sir, here is a white cloth for you, it’s attractive, stainless, and clean.’ They’d take it and put it on, expressing their gladness: ‘White cloth is really nice, it’s attractive, stainless, and clean.’ 

What\marginnote{20.10} do you think, \textsanskrit{Māgaṇḍiya}? Did that person blind from birth do this knowing and seeing, or out of faith in the sighted person?” 

“They\marginnote{20.13} did so not knowing or seeing, but out of faith in the sighted person.” 

“In\marginnote{21.1} the same way, the wanderers who follow other paths are blind and sightless. Not knowing health and not seeing extinguishment, they still recite this verse: ‘Health is the ultimate blessing; extinguishment, the ultimate happiness.’ For this verse was recited by the perfected ones, fully awakened Buddhas of the past: 

\begin{verse}%
‘Health\marginnote{21.4} is the ultimate blessing; \\
extinguishment, the ultimate happiness. \\
Of paths, the ultimate is eightfold—\\
it’s safe, and leads to the deathless.' 

%
\end{verse}

These\marginnote{21.8} days it has gradually become a verse used by ordinary people. But \textsanskrit{Māgaṇḍiya}, this body is a disease, a boil, a dart, a misery, an affliction. Yet you say of this body: ‘This is that health, this is that extinguishment!’ \textsanskrit{Māgaṇḍiya}, you don’t have the noble vision by which you might know health and see extinguishment.” 

“I\marginnote{22.1} am quite confident that Master Gotama is capable of teaching me so that I can know health and see extinguishment.” 

“\textsanskrit{Māgaṇḍiya},\marginnote{22.3} suppose a person was born blind. They couldn’t see sights that are dark or bright, or blue, yellow, red, or magenta. They couldn’t see even and uneven ground, or the stars, or the moon and sun. Their friends and colleagues, relatives and kin would get a field surgeon to treat them. The field surgeon would make medicine for them, but when they used it their eyes were not cured and they still could not see clearly. What do you think, \textsanskrit{Māgaṇḍiya}? Wouldn’t that doctor just get weary and frustrated?” 

“Yes,\marginnote{22.10} Master Gotama.” 

“In\marginnote{22.11} the same way, suppose I were to teach you the Dhamma, saying: ‘This is that health, this is that extinguishment.’ But you might not know health or see extinguishment, which would be wearying and troublesome for me.” 

“I\marginnote{23.1} am quite confident that Master Gotama is capable of teaching me so that I can know health and see extinguishment.” 

“\textsanskrit{Māgaṇḍiya},\marginnote{23.3} suppose a person was born blind. They couldn’t see sights that are dark or bright, or blue, yellow, red, or magenta. They couldn’t see even and uneven ground, or the stars, or the moon and sun. They might hear a sighted person saying: ‘White cloth is really nice, it’s attractive, stainless, and clean.’ 

They’d\marginnote{23.7} go in search of white cloth. But someone would cheat them with a dirty, soiled garment, saying: ‘Sir, here is a white cloth for you, it’s attractive, stainless, and clean.’ They’d take it and put it on. Their friends and colleagues, relatives and kin would get a field surgeon to treat them. The field surgeon would make medicine for them: emetics, purgatives, ointment, counter-ointment, or nasal treatment. And when they used it their eyes would be cured so that they could see clearly. As soon as their eyes were cured they’d lose all desire for that dirty, soiled garment. Then they would consider that person to be no friend, but an enemy, and might even think of murdering them: ‘For such a long time I’ve been cheated, tricked, and deceived by that person with this dirty, soiled garment when he said, “Sir, here is a white cloth for you, it’s attractive, stainless, and clean.”’ 

In\marginnote{24.1} the same way, \textsanskrit{Māgaṇḍiya}, suppose I were to teach you the Dhamma, saying: ‘This is that health, this is that extinguishment.’ You might know health and see extinguishment. And as soon as your vision arises you might give up desire for the five grasping aggregates. And you might even think: ‘For such a long time I’ve been cheated, tricked, and deceived by this mind. For what I have been grasping is only form, feeling, perception, choices, and consciousness. My grasping is a condition for continued existence. Continued existence is a condition for rebirth. Rebirth is a condition for old age and death, sorrow, lamentation, pain, sadness, and distress to come to be. That is how this entire mass of suffering originates.’” 

“I\marginnote{25.1} am quite confident that Master Gotama is capable of teaching me so that I can rise from this seat cured of blindness.” 

“Well\marginnote{25.3} then, \textsanskrit{Māgaṇḍiya}, you should associate with good people. When you associate with good people, you will hear the true teaching. When you hear the true teaching, you’ll practice in line with the teaching. When you practice in line with the teaching, you’ll know and see for yourself: ‘These are diseases, boils, and darts. And here is where diseases, boils, and darts cease without anything left over.’ When my grasping ceases, continued existence ceases. When continued existence ceases, rebirth ceases. When rebirth ceases, old age and death, sorrow, lamentation, pain, sadness, and distress cease. That is how this entire mass of suffering ceases.” 

When\marginnote{26.1} he said this, \textsanskrit{Māgaṇḍiya} said to him, “Excellent, Master Gotama! Excellent! As if he were righting the overturned, or revealing the hidden, or pointing out the path to the lost, or lighting a lamp in the dark so people with good eyes can see what’s there, Master Gotama has made the teaching clear in many ways. I go for refuge to Master Gotama, to the teaching, and to the mendicant \textsanskrit{Saṅgha}. Sir, may I receive the going forth, the ordination in the Buddha’s presence?” 

“\textsanskrit{Māgaṇḍiya},\marginnote{27.1} if someone formerly ordained in another sect wishes to take the going forth, the ordination in this teaching and training, they must spend four months on probation. When four months have passed, if the mendicants are satisfied, they’ll give the going forth, the ordination into monkhood. However, I have recognized individual differences in this matter.” 

“Sir,\marginnote{27.3} if four months probation are required in such a case, I’ll spend four years on probation. When four years have passed, if the mendicants are satisfied, let them give me the going forth, the ordination into monkhood.” 

And\marginnote{28.1} the wanderer \textsanskrit{Māgaṇḍiya} received the going forth, the ordination in the Buddha’s presence. Not long after his ordination, Venerable \textsanskrit{Māgaṇḍiya}, living alone, withdrawn, diligent, keen, and resolute, realized the supreme culmination of the spiritual path in this very life. He lived having achieved with his own insight the goal for which gentlemen rightly go forth from the lay life to homelessness. 

He\marginnote{28.3} understood: “Rebirth is ended; the spiritual journey has been completed; what had to be done has been done; there is no return to any state of existence.” And Venerable \textsanskrit{Māgaṇḍiya} became one of the perfected. 

%
\section*{{\suttatitleacronym MN 76}{\suttatitletranslation With Sandaka }{\suttatitleroot Sandakasutta}}
\addcontentsline{toc}{section}{\tocacronym{MN 76} \toctranslation{With Sandaka } \tocroot{Sandakasutta}}
\markboth{With Sandaka }{Sandakasutta}
\extramarks{MN 76}{MN 76}

\scevam{So\marginnote{1.1} I have heard. }At one time the Buddha was staying near Kosambi, in Ghosita’s Monastery. 

Now\marginnote{2.1} at that time the wanderer Sandaka was residing at the cave of the wavy leaf fig tree together with a large assembly of around five hundred wanderers. 

Then\marginnote{3.1} in the late afternoon, Venerable Ānanda came out of retreat and addressed the mendicants: “Come, reverends, let’s go to the Devakata Pool to see the cave.” 

“Yes,\marginnote{3.3} reverend,” they replied. Then Ānanda together with several mendicants went to the Devakata Pool. 

Now\marginnote{4.1} at that time, Sandaka and the large assembly of wanderers were sitting together making an uproar, a dreadful racket. They engaged in all kinds of unworthy talk, such as talk about kings, bandits, and ministers; talk about armies, threats, and wars; talk about food, drink, clothes, and beds; talk about garlands and fragrances; talk about family, vehicles, villages, towns, cities, and countries; talk about women and heroes; street talk and well talk; talk about the departed; motley talk; tales of land and sea; and talk about being reborn in this or that state of existence. 

Sandaka\marginnote{4.3} saw Ānanda coming off in the distance, and hushed his own assembly, “Be quiet, good sirs, don’t make a sound. The ascetic Ānanda, a disciple of the ascetic Gotama, is coming. He is included among the disciples of the ascetic Gotama, who is residing near \textsanskrit{Kosambī}. Such venerables like the quiet, are educated to be quiet, and praise the quiet. Hopefully if he sees that our assembly is quiet he’ll see fit to approach.” Then those wanderers fell silent. 

Then\marginnote{5.1} Venerable Ānanda went up to the wanderer Sandaka, who said to him, “Come, Master Ānanda! Welcome, Master Ānanda! It’s been a long time since you took the opportunity to come here. Please, sir, sit down, this seat is ready.” Ānanda sat down on the seat spread out, while Sandaka took a low seat and sat to one side. Ānanda said to Sandaka, “Sandaka, what were you sitting talking about just now? What conversation was left unfinished?” 

“Master\marginnote{5.10} Ānanda, leave aside what we were sitting talking about just now. It won’t be hard for you to hear about that later. It’d be great if Master Ānanda himself would give a Dhamma talk explaining his own tradition.” 

“Well\marginnote{5.13} then, Sandaka, listen and pay close attention, I will speak.” 

“Yes,\marginnote{5.14} sir,” replied Sandaka. Venerable Ānanda said this: 

“Sandaka,\marginnote{6.1} these things have been explained by the Blessed One, who knows and sees, the perfected one, the fully awakened Buddha: four ways that negate the spiritual life, and four kinds of unreliable spiritual life. A sensible person would, to the best of their ability, not practice such spiritual paths, and if they did practice them, they wouldn’t succeed in the procedure of the skillful teaching.” 

“But\marginnote{6.2} Master Ānanda, what are the four ways that negate the spiritual life, and the four kinds of unreliable spiritual life?” 

“Sandaka,\marginnote{7.1} take a certain teacher who has this doctrine and view: ‘There’s no meaning in giving, sacrifice, or offerings. There’s no fruit or result of good and bad deeds. There’s no afterlife. There’s no obligation to mother and father. No beings are reborn spontaneously. And there’s no ascetic or brahmin who is well attained and practiced, and who describes the afterlife after realizing it with their own insight. This person is made up of the four primary elements. When they die, the earth in their body merges and coalesces with the main mass of earth. The water in their body merges and coalesces with the main mass of water. The fire in their body merges and coalesces with the main mass of fire. The air in their body merges and coalesces with the main mass of air. The faculties are transferred to space. Four men with a bier carry away the corpse. Their footprints show the way to the cemetery. The bones become bleached. Offerings dedicated to the gods end in ashes. Giving is a doctrine for morons. When anyone affirms a positive teaching it’s just hollow, false nonsense. Both the foolish and the astute are annihilated and destroyed when their body breaks up, and they don’t exist after death.’ 

A\marginnote{8.1} sensible person reflects on this matter in this way: ‘This teacher has such a doctrine and view. If what that teacher says is true, both I who have not accomplished this and one who has accomplished it have attained exactly the same level. Yet I’m not one who says that both of us are annihilated and destroyed when our body breaks up, and we don’t exist after death. But it’s superfluous for this teacher to go naked, shaven, persisting in squatting, tearing out their hair and beard. For I’m living at home with my children, using sandalwood imported from \textsanskrit{Kāsi}, wearing garlands, perfumes, and makeup, and accepting gold and money. Yet I’ll have exactly the same destiny in the next life as this teacher. What do I know or see that I should lead the spiritual life under this teacher? This negates the spiritual life.’ Realizing this, they leave disappointed. 

This\marginnote{9.1} is the first way that negates the spiritual life. 

Furthermore,\marginnote{10.1} take a certain teacher who has this doctrine and view: ‘Nothing bad is done by the doer when they punish, mutilate, torture, aggrieve, oppress, intimidate, or when they encourage others to do the same. Nothing bad is done when they kill, steal, break into houses, plunder wealth, steal from isolated buildings, commit highway robbery, commit adultery, and lie. If you were to reduce all the living creatures of this earth to one heap and mass of flesh with a razor-edged chakram, no evil comes of that, and no outcome of evil. If you were to go along the south bank of the Ganges killing, mutilating, and torturing, and encouraging others to do the same, no evil comes of that, and no outcome of evil. If you were to go along the north bank of the Ganges giving and sacrificing and encouraging others to do the same, no merit comes of that, and no outcome of merit. In giving, self-control, restraint, and truthfulness there is no merit or outcome of merit.’ 

A\marginnote{11.1} sensible person reflects on this matter in this way: ‘This teacher has such a doctrine and view. If what that teacher says is true, both I who have not accomplished this and one who has accomplished it have attained exactly the same level. Yet I’m not one who says that when both of us act, nothing wrong is done. But it’s superfluous for this teacher to go naked, shaven, persisting in squatting, tearing out their hair and beard. For I’m living at home with my children, using sandalwood imported from \textsanskrit{Kāsi}, wearing garlands, perfumes, and makeup, and accepting gold and money. Yet I’ll have exactly the same destiny in the next life as this teacher. What do I know or see that I should lead the spiritual life under this teacher? This negates the spiritual life.’ Realizing this, they leave disappointed. 

This\marginnote{12.1} is the second way that negates the spiritual life. 

Furthermore,\marginnote{13.1} take a certain teacher who has this doctrine and view: ‘There is no cause or condition for the corruption of sentient beings. Sentient beings are corrupted without cause or reason. There’s no cause or condition for the purification of sentient beings. Sentient beings are purified without cause or reason. There is no power, no energy, no human strength or vigor. All sentient beings, all living creatures, all beings, all souls lack control, power, and energy. Molded by destiny, circumstance, and nature, they experience pleasure and pain in the six classes of rebirth.’ 

A\marginnote{14.1} sensible person reflects on this matter in this way: ‘This teacher has such a doctrine and view. If what that teacher says is true, both I who have not accomplished this and one who has accomplished it have attained exactly the same level. Yet I’m not one who says that both of us are purified without cause or reason. But it’s superfluous for this teacher to go naked, shaven, persisting in squatting, tearing out their hair and beard. For I’m living at home with my children, using sandalwood imported from \textsanskrit{Kāsi}, wearing garlands, perfumes, and makeup, and accepting gold and money. Yet I’ll have exactly the same destiny in the next life as this teacher. What do I know or see that I should lead the spiritual life under this teacher? This negates the spiritual life.’ Realizing this, they leave disappointed. 

This\marginnote{15.1} is the third way that negates the spiritual life. 

Furthermore,\marginnote{16.1} take a certain teacher who has this doctrine and view: ‘There are these seven substances that are not made, not derived, not created, without a creator, barren, steady as a mountain peak, standing firm like a pillar. They don’t move or deteriorate or obstruct each other. They’re unable to cause pleasure, pain, or neutral feeling to each other. What seven? The substances of earth, water, fire, air; pleasure, pain, and the soul is the seventh. These seven substances are not made, not derived, not created, without a creator, barren, steady as a mountain peak, standing firm like a pillar. They don’t move or deteriorate or obstruct each other. They’re unable to cause pleasure, pain, or neutral feeling to each other. And here there is no-one who kills or who makes others kill; no-one who learns or who educates others; no-one who understands or who helps others understand. If you chop off someone’s head with a sharp sword, you don’t take anyone’s life. The sword simply passes through the gap between the seven substances. There are 1.4 million main wombs, and 6,000, and 600. There are 500 deeds, and five, and three. There are deeds and half-deeds. There are 62 paths, 62 sub-eons, six classes of rebirth, and eight stages in a person’s life. There are 4,900 \textsanskrit{Ājīvaka} ascetics, 4,900 wanderers, and 4,900 naked ascetics. There are 2,000 faculties, 3,000 hells, and 36 realms of dust. There are seven percipient embryos, seven non-percipient embryos, and seven embryos without attachments. There are seven gods, seven humans, and seven goblins. There are seven lakes, seven winds, seven cliffs, and 700 cliffs. There are seven dreams and 700 dreams. There are 8.4 million great eons through which the foolish and the astute transmigrate before making an end of suffering. And here there is no such thing as this: “By this precept or observance or mortification or spiritual life I shall force unripened deeds to bear their fruit, or eliminate old deeds by experiencing their results little by little”—for that cannot be. Pleasure and pain are allotted. Transmigration lasts only for a limited period, so there’s no increase or decrease, no getting better or worse. It’s like how, when you toss a ball of string, it rolls away unraveling. In the same way, after transmigrating the foolish and the astute will make an end of suffering.’ 

A\marginnote{17.1} sensible person reflects on this matter in this way: ‘This teacher has such a doctrine and view. If what that teacher says is true, both I who have not accomplished this and one who has accomplished it have attained exactly the same level. Yet I’m not one who says that after transmigrating both of us will make an end of suffering. But it’s superfluous for this teacher to go naked, shaven, persisting in squatting, tearing out their hair and beard. For I’m living at home with my children, using sandalwood imported from \textsanskrit{Kāsi}, wearing garlands, perfumes, and makeup, and accepting gold and money. Yet I’ll have exactly the same destiny in the next life as this teacher. What do I know or see that I should lead the spiritual life under this teacher? This negates the spiritual life.’ Realizing this, they leave disappointed. 

This\marginnote{18.1} is the fourth way that negates the spiritual life. 

These\marginnote{19.1} are the four ways that negate the spiritual life that have been explained by the Blessed One, who knows and sees, the perfected one, the fully awakened Buddha. A sensible person would, to the best of their ability, not practice such spiritual paths, and if they did practice them, they wouldn’t succeed in the procedure of the skillful teaching.” 

“It’s\marginnote{20.1} incredible, Master Ānanda, it’s amazing, how these four ways that negate the spiritual life have been explained by the Buddha. But Master Ānanda, what are the four kinds of unreliable spiritual life?” 

“Sandaka,\marginnote{21.1} take a certain teacher who claims to be all-knowing and all-seeing, to know and see everything without exception, thus: ‘Knowledge and vision are constantly and continually present to me, while walking, standing, sleeping, and waking.’ He enters an empty house; he gets no almsfood; a dog bites him; he encounters a wild elephant, a wild horse, and a wild cow; he asks the name and clan of a woman or man; he asks the name and path to a village or town. When asked, ‘Why is this?’ he answers: ‘I had to enter an empty house, that’s why I entered it. I had to get no almsfood, that’s why I got none. I had to get bitten by a dog, that’s why I was bitten. I had to encounter a wild elephant, a wild horse, and a wild cow, that’s why I encountered them. I had to ask the name and clan of a woman or man, that’s why I asked. I had to ask the name and path to a village or town, that’s why I asked.’ 

A\marginnote{22.1} sensible person reflects on this matter in this way: ‘This teacher makes such a claim, but he answers in such a way. This spiritual life is unreliable.’ Realizing this, they leave disappointed. 

This\marginnote{23.1} is the first kind of unreliable spiritual life. 

Furthermore,\marginnote{24.1} take another teacher who is an oral transmitter, who takes oral transmission to be the truth. He teaches by oral transmission, by the lineage of testament, by canonical authority. But when a teacher takes oral transmission to be the truth, some of that is well learned, some poorly learned, some true, and some otherwise. 

A\marginnote{25.1} sensible person reflects on this matter in this way: ‘This teacher takes oral transmission to be the truth. He teaches by oral transmission, by the lineage of testament, by canonical authority. But when a teacher takes oral transmission to be the truth, some of that is well learned, some poorly learned, some true, and some otherwise. This spiritual life is unreliable.’ Realizing this, they leave disappointed. 

This\marginnote{26.1} is the second kind of unreliable spiritual life. 

Furthermore,\marginnote{27.1} take another teacher who relies on logic and inquiry. He teaches what he has worked out by logic, following a line of inquiry, expressing his own perspective. But when a teacher relies on logic and inquiry, some of that is well reasoned, some poorly reasoned, some true, and some otherwise. 

A\marginnote{28.1} sensible person reflects on this matter in this way: ‘This teacher relies on logic and inquiry. He teaches what he has worked out by logic, following a line of inquiry, expressing his own perspective. But when a teacher relies on logic and inquiry, some of that is well reasoned, some poorly reasoned, some true, and some otherwise. This spiritual life is unreliable.’ Realizing this, they leave disappointed. 

This\marginnote{29.1} is the third kind of unreliable spiritual life. 

Furthermore,\marginnote{30.1} take another teacher who is dull and stupid. Because of that, whenever he’s asked a question, he resorts to evasiveness and equivocation: ‘I don’t say it’s like this. I don’t say it’s like that. I don’t say it’s otherwise. I don’t say it’s not so. And I don’t deny it’s not so.’ 

A\marginnote{31.1} sensible person reflects on this matter in this way: ‘This teacher is dull and stupid. Because of that, whenever he’s asked a question, he resorts to evasiveness and equivocation: “I don’t say it’s like this. I don’t say it’s like that. I don’t say it’s otherwise. I don’t say it’s not so. And I don’t deny it’s not so.” This spiritual life is unreliable.’ Realizing this, they leave disappointed. 

This\marginnote{32.1} is the fourth kind of unreliable spiritual life. 

These\marginnote{33.1} are the four kinds of unreliable spiritual life that have been explained by the Blessed One, who knows and sees, the perfected one, the fully awakened Buddha. A sensible person would, to the best of their ability, not practice such spiritual paths, and if they did practice them, they wouldn’t complete the procedure of the skillful teaching.” 

“It’s\marginnote{34.1} incredible, Master Ānanda, it’s amazing, how these four kinds of unreliable spiritual life have been explained by the Buddha. But, Master Ānanda, what would a teacher say and explain so that a sensible person would, to the best of their ability, practice such a spiritual path, and once practicing it, they would complete the procedure of the skillful teaching?” 

“Sandaka,\marginnote{35{-}42.1} it’s when a Realized One arises in the world, perfected, a fully awakened Buddha, accomplished in knowledge and conduct, holy, knower of the world, supreme guide for those who wish to train, teacher of gods and humans, awakened, blessed. … He gives up these five hindrances, corruptions of the heart that weaken wisdom. Then, quite secluded from sensual pleasures, secluded from unskillful qualities, he enters and remains in the first absorption, which has the rapture and bliss born of seclusion, while placing the mind and keeping it connected. A sensible person would, to the best of their ability lead the spiritual life under a teacher who achieves such a high distinction, and, once practicing it, they would complete the procedure of the skillful teaching. 

Furthermore,\marginnote{43.1} as the placing of the mind and keeping it connected are stilled, a mendicant … enters and remains in the second absorption … third absorption … fourth absorption. A sensible person would, to the best of their ability lead the spiritual life under a teacher who achieves such a high distinction, and, once practicing it, they would complete the procedure of the skillful teaching. 

When\marginnote{47.1} their mind has become immersed in \textsanskrit{samādhi} like this—purified, bright, flawless, rid of corruptions, pliable, workable, steady, and imperturbable—they extend it toward recollection of past lives. They recollect many kinds of past lives. That is: one, two, three, four, five, ten, twenty, thirty, forty, fifty, a hundred, a thousand, a hundred thousand rebirths; many eons of the world contracting, many eons of the world expanding, many eons of the world contracting and expanding. … They recollect their many kinds of past lives, with features and details. A sensible person would, to the best of their ability lead the spiritual life under a teacher who achieves such a high distinction, and, once practicing it, they would complete the procedure of the skillful teaching. 

When\marginnote{48.1} their mind has become immersed in \textsanskrit{samādhi} like this—purified, bright, flawless, rid of corruptions, pliable, workable, steady, and imperturbable—they extend it toward knowledge of the death and rebirth of sentient beings. With clairvoyance that is purified and superhuman, they see sentient beings passing away and being reborn—inferior and superior, beautiful and ugly, in a good place or a bad place. … They understand how sentient beings are reborn according to their deeds. A sensible person would, to the best of their ability lead the spiritual life under a teacher who achieves such a high distinction, and, once practicing it, they would complete the procedure of the skillful teaching. 

When\marginnote{49.1} their mind has become immersed in \textsanskrit{samādhi} like this—purified, bright, flawless, rid of corruptions, pliable, workable, steady, and imperturbable—they extend it toward knowledge of the ending of defilements. They truly understand: ‘This is suffering’ … ‘This is the origin of suffering’ … ‘This is the cessation of suffering’ … ‘This is the practice that leads to the cessation of suffering’. They truly understand: ‘These are defilements’ … ‘This is the origin of defilements’ … ‘This is the cessation of defilements’ … ‘This is the practice that leads to the cessation of defilements’. 

Knowing\marginnote{50.1} and seeing like this, their mind is freed from the defilements of sensuality, desire to be reborn, and ignorance. When they’re freed, they know they’re freed. 

They\marginnote{50.3} understand: ‘Rebirth is ended, the spiritual journey has been completed, what had to be done has been done, there is no return to any state of existence.’ A sensible person would, to the best of their ability lead the spiritual life under a teacher who achieves such a high distinction, and, once practicing it, they would complete the procedure of the skillful teaching.” 

“But\marginnote{51.1} Master Ānanda, when a mendicant is perfected—with defilements ended, who has completed the spiritual journey, done what had to be done, laid down the burden, achieved their own true goal, utterly ended the fetters of rebirth, and is rightly freed through enlightenment—could they still enjoy sensual pleasures?” 

“Sandaka,\marginnote{51.2} a mendicant who is perfected—with defilements ended, who has completed the spiritual journey, done what had to be done, laid down the burden, achieved their own true goal, utterly ended the fetters of rebirth, and is rightly freed through enlightenment—can’t transgress in five respects. A mendicant with defilements ended can’t deliberately take the life of a living creature, take something with the intention to steal, have sex, tell a deliberate lie, or store up goods for their own enjoyment like they did as a lay person. A mendicant who is perfected can’t transgress in these five respects.” 

“But\marginnote{52.1} Master Ānanda, when a mendicant is perfected, would the knowledge and vision that their defilements are ended be constantly and continually present to them, while walking, standing, sleeping, and waking?” 

“Well\marginnote{52.3} then, Sandaka, I shall give you a simile. For by means of a simile some sensible people understand the meaning of what is said. Suppose there was a person whose hands and feet had been amputated. Would they be aware that their hands and feet had been amputated constantly and continually, while walking, standing, sleeping, and waking? Or would they be aware of it only when they checked it?” 

“They\marginnote{52.9} wouldn’t be aware of it constantly, only when they checked it.” 

“In\marginnote{52.13} the same way, when a mendicant is perfected, the knowledge and vision that their defilements are ended is not constantly and continually present to them, while walking, standing, sleeping, and waking. Rather, they are aware of it only when they checked it.” 

“But\marginnote{53.1} Reverend Ānanda, how many emancipators are there in this teaching and training?” 

“There\marginnote{53.2} are not just one hundred emancipators, Sandaka, or two or three or four or five hundred, but many more than that in this teaching and training.” 

“It’s\marginnote{53.3} incredible, Master Ānanda, it’s amazing! Namely, that there’s no glorifying one’s own teaching and putting down the teaching of others. The Dhamma is taught in its own field, and so many emancipators are recognized. But these \textsanskrit{Ājīvaka} ascetics, those sons of dead sons, glorify themselves and put others down. And they only recognize three emancipators: Nanda Vaccha, Kisa \textsanskrit{Saṅkicca}, and Makkhali \textsanskrit{Gosāla}.” 

Then\marginnote{54.1} the wanderer Sandaka addressed his own assembly, “Go, good sirs. The spiritual life is lived under the ascetic Gotama. It’s not easy for me to give up possessions, honor, or popularity now.” And that’s how the wanderer Sandaka sent his own assembly to lead the spiritual life under the Buddha. 

%
\section*{{\suttatitleacronym MN 77}{\suttatitletranslation The Longer Discourse with Sakuludāyī }{\suttatitleroot Mahāsakuludāyisutta}}
\addcontentsline{toc}{section}{\tocacronym{MN 77} \toctranslation{The Longer Discourse with Sakuludāyī } \tocroot{Mahāsakuludāyisutta}}
\markboth{The Longer Discourse with Sakuludāyī }{Mahāsakuludāyisutta}
\extramarks{MN 77}{MN 77}

\scevam{So\marginnote{1.1} I have heard. }At one time the Buddha was staying near \textsanskrit{Rājagaha}, in the Bamboo Grove, the squirrels’ feeding ground. 

Now\marginnote{2.1} at that time several very well-known wanderers were residing in the monastery of the wanderers in the peacocks’ feeding ground. They included \textsanskrit{Annabhāra}, Varadhara, \textsanskrit{Sakuludāyī}, and other very well-known wanderers. 

Then\marginnote{3.1} the Buddha robed up in the morning and, taking his bowl and robe, entered \textsanskrit{Rājagaha} for alms. Then it occurred to him, “It’s too early to wander for alms in \textsanskrit{Rājagaha}. Why don’t I visit the wanderer \textsanskrit{Sakuludāyī} at the monastery of the wanderers in the peacocks’ feeding ground?” 

So\marginnote{4.1} the Buddha went to the monastery of the wanderers. 

Now\marginnote{4.2} at that time, \textsanskrit{Sakuludāyī} was sitting together with a large assembly of wanderers making an uproar, a dreadful racket. They engaged in all kinds of unworthy talk, such as talk about kings, bandits, and ministers; talk about armies, threats, and wars; talk about food, drink, clothes, and beds; talk about garlands and fragrances; talk about family, vehicles, villages, towns, cities, and countries; talk about women and heroes; street talk and well talk; talk about the departed; motley talk; tales of land and sea; and talk about being reborn in this or that state of existence. 

\textsanskrit{Sakuludāyī}\marginnote{4.4} saw the Buddha coming off in the distance, and hushed his own assembly, “Be quiet, good sirs, don’t make a sound. Here comes the ascetic Gotama. The venerable likes quiet and praises quiet. Hopefully if he sees that our assembly is quiet he’ll see fit to approach.” Then those wanderers fell silent. 

Then\marginnote{5.1} the Buddha approached \textsanskrit{Sakuludāyī}, who said to him, “Come, Blessed One! Welcome, Blessed One! It’s been a long time since you took the opportunity to come here. Please, sir, sit down, this seat is ready.” The Buddha sat on the seat spread out, while \textsanskrit{Sakuludāyī} took a low seat and sat to one side. 

The\marginnote{5.10} Buddha said to him, “\textsanskrit{Udāyī}, what were you sitting talking about just now? What conversation was left unfinished?” 

“Sir,\marginnote{6.1} leave aside what we were sitting talking about just now. It won’t be hard for you to hear about that later. 

Sir,\marginnote{6.3} a few days ago several ascetics and brahmins who follow various other paths were sitting together at the debating hall, and this discussion came up among them: ‘The people of \textsanskrit{Aṅga} and Magadha are so fortunate, so very fortunate! For there are these ascetics and brahmins who lead an order and a community, and teach a community. They’re well-known and famous religious founders, regarded as holy by many people. And they have come down for the rainy season residence at \textsanskrit{Rājagaha}. They include \textsanskrit{Pūraṇa} Kassapa, Makkhali \textsanskrit{Gosāla}, Ajita Kesakambala, Pakudha \textsanskrit{Kaccāyana}, \textsanskrit{Sañjaya} \textsanskrit{Belaṭṭhiputta}, and \textsanskrit{Nigaṇṭha} \textsanskrit{Nāṭaputta}. This ascetic Gotama also leads an order and a community, and teaches a community. He’s a well-known and famous religious founder, regarded as holy by many people. And he too has come down for the rains residence at \textsanskrit{Rājagaha}. Which of these ascetics and brahmins is honored, respected, revered, and venerated by their disciples? And how do their disciples, after honoring and respecting them, remain loyal?’ 

Some\marginnote{6.17} of them said: ‘This \textsanskrit{Pūraṇa} Kassapa leads an order and a community, and teaches a community. He’s a well-known and famous religious founder, regarded as holy by many people. But he’s not honored, respected, revered, venerated, and esteemed by his disciples. And his disciples, not honoring and respecting him, don’t remain loyal to him. Once it so happened that he was teaching an assembly of many hundreds. Then one of his disciples made a noise, “My good sirs, don’t ask \textsanskrit{Pūraṇa} Kassapa about that. He doesn’t know that. I know it. Ask me about it, and I’ll answer you.” It happened that \textsanskrit{Pūraṇa} Kassapa didn’t get his way, though he called out with raised arms, “Be quiet, good sirs, don’t make a sound. They’re not asking you, they’re asking me! I’ll answer you!” Indeed, many of his disciples have left him after refuting his doctrine: “You don’t understand this teaching and training. I understand this teaching and training. What, you understand this teaching and training? You’re practicing wrong. I’m practicing right. I stay on topic, you don’t. You said last what you should have said first. You said first what you should have said last. What you’ve thought so much about has been disproved. Your doctrine is refuted. Go on, save your doctrine! You’re trapped; get yourself out of this—if you can!” That’s how \textsanskrit{Pūraṇa} Kassapa is not honored, respected, revered, venerated, and esteemed by his disciples. On the contrary, his disciples, not honoring and respecting him, don’t remain loyal to him. Rather, he’s reviled, and rightly so.’ 

Others\marginnote{6.34} said: 'This Makkhali \textsanskrit{Gosāla} … Ajita Kesakambala … Pakudha \textsanskrit{Kaccāyana} … \textsanskrit{Sañjaya} \textsanskrit{Belaṭṭhiputta} … \textsanskrit{Nigaṇṭha} \textsanskrit{Nāṭaputta} leads an order and a community, and teaches a community. He’s a well-known and famous religious founder, regarded as holy by many people. But he’s not honored, respected, revered, and venerated by his disciples. And his disciples, not honoring and respecting him, don’t remain loyal to him. Once it so happened that he was teaching an assembly of many hundreds. Then one of his disciples made a noise, “My good sirs, don’t ask \textsanskrit{Nigaṇṭha} \textsanskrit{Nātaputta} about that. He doesn’t know that. I know it. Ask me about it, and I’ll answer you.” It happened that \textsanskrit{Nigaṇṭha} \textsanskrit{Nātaputta} didn’t get his way, though he called out with raised arms, “Be quiet, good sirs, don’t make a sound. They’re not asking you, they’re asking me! I’ll answer you!” Indeed, many of his disciples have left him after refuting his doctrine: “You don’t understand this teaching and training. I understand this teaching and training. What, you understand this teaching and training? You’re practicing wrong. I’m practicing right. I stay on topic, you don’t. You said last what you should have said first. You said first what you should have said last. What you’ve thought so much about has been disproved. Your doctrine is refuted. Go on, save your doctrine! You’re trapped; get yourself out of this—if you can!” That’s how \textsanskrit{Nigaṇṭha} \textsanskrit{Nātaputta} is not honored, respected, revered, and venerated by his disciples. On the contrary, his disciples, not honoring and respecting him, don’t remain loyal to him. Rather, he’s reviled, and rightly so.’ 

Others\marginnote{6.55} said: ‘This ascetic Gotama leads an order and a community, and teaches a community. He’s a well-known and famous religious founder, regarded as holy by many people. He’s honored, respected, revered, and venerated by his disciples. And his disciples, honoring and respecting him, remain loyal to him. Once it so happened that he was teaching an assembly of many hundreds. Then one of his disciples cleared their throat. And one of their spiritual companions nudged them with their knee, to indicate, “Hush, venerable, don’t make sound! Our teacher, the Blessed One, is teaching!” While the ascetic Gotama is teaching an assembly of many hundreds, there is no sound of his disciples coughing or clearing their throats. That large crowd is poised on the edge of their seats, thinking, “Whatever the Buddha teaches, we shall listen to it.” It’s like when there’s a person at the crossroads pressing out pure manuka honey, and a large crowd is poised on the edge of their seats. In the same way, while the ascetic Gotama is teaching an assembly of many hundreds, there is no sound of his disciples coughing or clearing their throats. That large crowd is poised on the edge of their seats, thinking, “Whatever the Buddha teaches, we shall listen to it.” Even when a disciple of the ascetic Gotama resigns the training and returns to a lesser life, having been overly attached to their spiritual companions, they speak only praise of the teacher, the teaching, and the \textsanskrit{Saṅgha}. They blame only themselves, not others: “We were unlucky, we had little merit. For even after going forth in such a well explained teaching and training we weren’t able to practice for life the perfectly full and pure spiritual life.” They become monastery workers or lay followers, and they proceed having undertaken the five precepts. That’s how the ascetic Gotama is honored, respected, revered, and venerated by his disciples. And that’s how his disciples, honoring and respecting him, remain loyal to him.’” 

“But\marginnote{7.1} \textsanskrit{Udāyī}, how many qualities do you see in me, because of which my disciples honor, respect, revere, and venerate me; and after honoring and respecting me, they remain loyal to me?” 

“Sir,\marginnote{8.1} I see five such qualities in the Buddha. What five? 

The\marginnote{8.3} Buddha eats little and praises eating little. This is the first such quality I see in the Buddha. 

Furthermore,\marginnote{8.5} the Buddha is content with any kind of robe, and praises such contentment. This is the second such quality I see in the Buddha. 

Furthermore,\marginnote{8.7} the Buddha is content with any kind of almsfood, and praises such contentment. This is the third such quality I see in the Buddha. 

Furthermore,\marginnote{8.9} the Buddha is content with any kind of lodging, and praises such contentment. This is the fourth such quality I see in the Buddha. 

Furthermore,\marginnote{8.11} the Buddha is secluded, and praises seclusion. This is the fifth such quality I see in the Buddha. 

These\marginnote{8.13} are the five qualities I see in the Buddha, because of which his disciples honor, respect, revere, and venerate him; and after honoring and respecting him, they remain loyal to him.” 

“Suppose,\marginnote{9.1} \textsanskrit{Udāyī}, my disciples were loyal to me because I eat little. Well, there are disciples of mine who eat a cupful of food, or half a cupful; they eat a wood apple, or half a wood apple. But sometimes I even eat this bowl full to the brim, or even more. So if it were the case that my disciples are loyal to me because I eat little, then those disciples who eat even less would not be loyal to me. 

Suppose\marginnote{9.4} my disciples were loyal to me because I’m content with any kind of robe. Well, there are disciples of mine who have rag robes, wearing shabby robes. They gather scraps from charnel grounds, rubbish dumps, and shops, make them into a patchwork robe and wear it. But sometimes I wear robes offered by householders that are strong, yet next to which bottle-gourd down is coarse. So if it were the case that my disciples are loyal to me because I’m content with any kind of robe, then those disciples who wear rag robes would not be loyal to me. 

Suppose\marginnote{9.7} my disciples were loyal to me because I’m content with any kind of almsfood. Well, there are disciples of mine who eat only almsfood, wander indiscriminately for almsfood, happy to eat whatever they glean. When they’ve entered an inhabited area, they don’t consent when invited to sit down. But sometimes I even eat by invitation boiled fine rice with the dark grains picked out, served with many soups and sauces. So if it were the case that my disciples are loyal to me because I’m content with any kind of almsfood, then those disciples who eat only almsfood would not be loyal to me. 

Suppose\marginnote{9.10} my disciples were loyal to me because I’m content with any kind of lodging. Well, there are disciples of mine who stay at the root of a tree, in the open air. For eight months they don’t go under a roof. But sometimes I even stay in bungalows, plastered inside and out, draft-free, with latches fastened and windows shuttered. So if it were the case that my disciples are loyal to me because I’m content with any kind of lodging, then those disciples who stay at the root of a tree would not be loyal to me. 

Suppose\marginnote{9.13} my disciples were loyal to me because I’m secluded and I praise seclusion. Well, there are disciples of mine who live in the wilderness, in remote lodgings. Having ventured deep into remote lodgings in the wilderness and the forest, they live there, coming down to the midst of the \textsanskrit{Saṅgha} each fortnight for the recitation of the monastic code. But sometimes I live crowded by monks, nuns, laymen, and laywomen; by rulers and their ministers, and teachers of other paths and their disciples. So if it were the case that my disciples are loyal to me because I’m secluded and praise seclusion, then those disciples who live in the wilderness would not be loyal to me. 

So,\marginnote{9.16} \textsanskrit{Udāyī}, it’s not because of these five qualities that my disciples honor, respect, revere, and venerate me; and after honoring and respecting me, they remain loyal to me. 

There\marginnote{10.1} are five other qualities because of which my disciples honor, respect, revere, and venerate me; and after honoring and respecting me, they remain loyal to me. What five? 

Firstly,\marginnote{10.3} my disciples esteem me for the higher ethics: ‘The ascetic Gotama is ethical. He possesses the entire spectrum of ethical conduct to the highest degree.’ Since this is so, this is the first quality because of which my disciples are loyal to me. 

Furthermore,\marginnote{12.1} my disciples esteem me for my excellent knowledge and vision: ‘The ascetic Gotama only claims to know when he does in fact know. He only claims to see when he really does see. He teaches based on direct knowledge, not without direct knowledge. He teaches based on reason, not without reason. He teaches with a demonstrable basis, not without it.’ Since this is so, this is the second quality because of which my disciples are loyal to me. 

Furthermore,\marginnote{13.1} my disciples esteem me for my higher wisdom: ‘The ascetic Gotama is wise. He possesses the entire spectrum of wisdom to the highest degree. It’s not possible that he would fail to foresee grounds for future criticism, or to legitimately and completely refute the doctrines of others that come up.’ What do you think, \textsanskrit{Udāyī}? Would my disciples, knowing and seeing this, break in and interrupt me?” 

“No,\marginnote{13.6} sir.” 

“That’s\marginnote{13.7} because I don’t expect to be instructed by my disciples. Invariably, my disciples expect instruction from me. 

Since\marginnote{13.9} this is so, this is the third quality because of which my disciples are loyal to me. 

Furthermore,\marginnote{14.1} my disciples come to me and ask how the noble truth of suffering applies to the suffering in which they are swamped and mired. And I provide them with a satisfying answer to their question. They ask how the noble truths of the origin of suffering, the cessation of suffering, and the practice that leads to the cessation of suffering apply to the suffering that has overwhelmed them and brought them low. And I provide them with satisfying answers to their questions. Since this is so, this is the fourth quality because of which my disciples are loyal to me. 

Furthermore,\marginnote{15.1} I have explained to my disciples a practice that they use to develop the four kinds of mindfulness meditation. It’s when a mendicant meditates by observing an aspect of the body—keen, aware, and mindful, rid of desire and aversion for the world. They meditate observing an aspect of feelings … mind … principles—keen, aware, and mindful, rid of desire and aversion for the world. And many of my disciples meditate on that having attained perfection and consummation of insight. 

Furthermore,\marginnote{16.1} I have explained to my disciples a practice that they use to develop the four right efforts. It’s when a mendicant generates enthusiasm, tries, makes an effort, exerts the mind, and strives so that bad, unskillful qualities don’t arise. They generate enthusiasm, try, make an effort, exert the mind, and strive so that bad, unskillful qualities that have arisen are given up. They generate enthusiasm, try, make an effort, exert the mind, and strive so that skillful qualities arise. They generate enthusiasm, try, make an effort, exert the mind, and strive so that skillful qualities that have arisen remain, are not lost, but increase, mature, and are fulfilled by development. And many of my disciples meditate on that having attained perfection and consummation of insight. 

Furthermore,\marginnote{17.1} I have explained to my disciples a practice that they use to develop the four bases of psychic power. It’s when a mendicant develops the basis of psychic power that has immersion due to enthusiasm, and active effort. They develop the basis of psychic power that has immersion due to energy, and active effort. They develop the basis of psychic power that has immersion due to mental development, and active effort. They develop the basis of psychic power that has immersion due to inquiry, and active effort. And many of my disciples meditate on that having attained perfection and consummation of insight. 

Furthermore,\marginnote{18.1} I have explained to my disciples a practice that they use to develop the five faculties. It’s when a mendicant develops the faculties of faith, energy, mindfulness, immersion, and wisdom, which lead to peace and awakening. And many of my disciples meditate on that having attained perfection and consummation of insight. 

Furthermore,\marginnote{19.1} I have explained to my disciples a practice that they use to develop the five powers. It’s when a mendicant develops the powers of faith, energy, mindfulness, immersion, and wisdom, which lead to peace and awakening. And many of my disciples meditate on that having attained perfection and consummation of insight. 

Furthermore,\marginnote{20.1} I have explained to my disciples a practice that they use to develop the seven awakening factors. It’s when a mendicant develops the awakening factors of mindfulness, investigation of principles, energy, rapture, tranquility, immersion, and equanimity, which rely on seclusion, fading away, and cessation, and ripen as letting go. And many of my disciples meditate on that having attained perfection and consummation of insight. 

Furthermore,\marginnote{21.1} I have explained to my disciples a practice that they use to develop the noble eightfold path. It’s when a mendicant develops right view, right thought, right speech, right action, right livelihood, right effort, right mindfulness, and right immersion. And many of my disciples meditate on that having attained perfection and consummation of insight. 

Furthermore,\marginnote{22.1} I have explained to my disciples a practice that they use to develop the eight liberations. 

Having\marginnote{22.2} physical form, they see visions. This is the first liberation. 

Not\marginnote{22.4} perceiving form internally, they see visions externally. This is the second liberation. 

They’re\marginnote{22.6} focused only on beauty. This is the third liberation. 

Going\marginnote{22.8} totally beyond perceptions of form, with the ending of perceptions of impingement, not focusing on perceptions of diversity, aware that ‘space is infinite’, they enter and remain in the dimension of infinite space. This is the fourth liberation. 

Going\marginnote{22.10} totally beyond the dimension of infinite space, aware that ‘consciousness is infinite’, they enter and remain in the dimension of infinite consciousness. This is the fifth liberation. 

Going\marginnote{22.12} totally beyond the dimension of infinite consciousness, aware that ‘there is nothing at all’, they enter and remain in the dimension of nothingness. This is the sixth liberation. 

Going\marginnote{22.14} totally beyond the dimension of nothingness, they enter and remain in the dimension of neither perception nor non-perception. This is the seventh liberation. 

Going\marginnote{22.16} totally beyond the dimension of neither perception nor non-perception, they enter and remain in the cessation of perception and feeling. This is the eighth liberation. 

And\marginnote{22.18} many of my disciples meditate on that having attained perfection and consummation of insight. 

Furthermore,\marginnote{23.1} I have explained to my disciples a practice that they use to develop the eight dimensions of mastery. 

Perceiving\marginnote{23.2} form internally, someone sees visions externally, limited, both pretty and ugly. Mastering them, they perceive: ‘I know and see.’ This is the first dimension of mastery. 

Perceiving\marginnote{23.5} form internally, someone sees visions externally, limitless, both pretty and ugly. Mastering them, they perceive: ‘I know and see.’ This is the second dimension of mastery. 

Not\marginnote{23.8} perceiving form internally, someone sees visions externally, limited, both pretty and ugly. Mastering them, they perceive: ‘I know and see.’ This is the third dimension of mastery. 

Not\marginnote{23.11} perceiving form internally, someone sees visions externally, limitless, both pretty and ugly. Mastering them, they perceive: ‘I know and see.’ This is the fourth dimension of mastery. 

Not\marginnote{23.14} perceiving form internally, someone sees visions externally, blue, with blue color, blue hue, and blue tint. They’re like a flax flower that’s blue, with blue color, blue hue, and blue tint. Or a cloth from \textsanskrit{Bāraṇasī} that’s smoothed on both sides, blue, with blue color, blue hue, and blue tint. In the same way, not perceiving form internally, someone sees visions externally, blue, with blue color, blue hue, and blue tint. Mastering them, they perceive: ‘I know and see.’ This is the fifth dimension of mastery. 

Not\marginnote{23.19} perceiving form internally, someone sees visions externally that are yellow, with yellow color, yellow hue, and yellow tint. They’re like a champak flower that’s yellow, with yellow color, yellow hue, and yellow tint. Or a cloth from \textsanskrit{Bāraṇasī} that’s smoothed on both sides, yellow, with yellow color, yellow hue, and yellow tint. In the same way, not perceiving form internally, someone sees visions externally that are yellow, with yellow color, yellow hue, and yellow tint. Mastering them, they perceive: ‘I know and see.’ This is the sixth dimension of mastery. 

Not\marginnote{23.24} perceiving form internally, someone sees visions externally that are red, with red color, red hue, and red tint. They’re like a scarlet mallow flower that’s red, with red color, red hue, and red tint. Or a cloth from \textsanskrit{Bāraṇasī} that’s smoothed on both sides, red, with red color, red hue, and red tint. In the same way, not perceiving form internally, someone sees visions externally that are red, with red color, red hue, and red tint. Mastering them, they perceive: ‘I know and see.’ This is the seventh dimension of mastery. 

Not\marginnote{23.29} perceiving form internally, someone sees visions externally that are white, with white color, white hue, and white tint. They’re like the morning star that’s white, with white color, white hue, and white tint. Or a cloth from \textsanskrit{Bāraṇasī} that’s smoothed on both sides, white, with white color, white hue, and white tint. In the same way, not perceiving form internally, someone sees visions externally that are white, with white color, white hue, and white tint. Mastering them, they perceive: ‘I know and see.’ This is the eighth dimension of mastery. 

And\marginnote{23.34} many of my disciples meditate on that having attained perfection and consummation of insight. 

Furthermore,\marginnote{24.1} I have explained to my disciples a practice that they use to develop the ten universal dimensions of meditation. 

Someone\marginnote{24.2} perceives the meditation on universal earth above, below, across, non-dual and limitless. 

They\marginnote{24.3} perceive the meditation on universal water … the meditation on universal fire … the meditation on universal air … the meditation on universal blue … the meditation on universal yellow … the meditation on universal red … the meditation on universal white … the meditation on universal space … the meditation on universal consciousness above, below, across, non-dual and limitless. 

And\marginnote{24.12} many of my disciples meditate on that having attained perfection and consummation of insight. 

Furthermore,\marginnote{25.1} I have explained to my disciples a practice that they use to develop the four absorptions. 

It’s\marginnote{25.2} when a mendicant, quite secluded from sensual pleasures, secluded from unskillful qualities, enters and remains in the first absorption, which has the rapture and bliss born of seclusion, while placing the mind and keeping it connected. They drench, steep, fill, and spread their body with rapture and bliss born of seclusion. There’s no part of the body that’s not spread with rapture and bliss born of seclusion. It’s like when a deft bathroom attendant or their apprentice pours bath powder into a bronze dish, sprinkling it little by little with water. They knead it until the ball of bath powder is soaked and saturated with moisture, spread through inside and out; yet no moisture oozes out. In the same way, a mendicant drenches, steeps, fills, and spreads their body with rapture and bliss born of seclusion. There’s no part of the body that’s not spread with rapture and bliss born of seclusion. 

Furthermore,\marginnote{26.1} as the placing of the mind and keeping it connected are stilled, a mendicant enters and remains in the second absorption. It has the rapture and bliss born of immersion, with internal clarity and confidence, and unified mind, without placing the mind and keeping it connected. They drench, steep, fill, and spread their body with rapture and bliss born of immersion. There’s no part of the body that’s not spread with rapture and bliss born of immersion. It’s like a deep lake fed by spring water. There’s no inlet to the east, west, north, or south, and no rainfall to replenish it from time to time. But the stream of cool water welling up in the lake drenches, steeps, fills, and spreads throughout the lake. There’s no part of the lake that’s not spread through with cool water. In the same way, a mendicant drenches, steeps, fills, and spreads their body with rapture and bliss born of immersion. There’s no part of the body that’s not spread with rapture and bliss born of immersion. 

Furthermore,\marginnote{27.1} with the fading away of rapture, a mendicant enters and remains in the third absorption. They meditate with equanimity, mindful and aware, personally experiencing the bliss of which the noble ones declare, ‘Equanimous and mindful, one meditates in bliss.’ They drench, steep, fill, and spread their body with bliss free of rapture. There’s no part of the body that’s not spread with bliss free of rapture. It’s like a pool with blue water lilies, or pink or white lotuses. Some of them sprout and grow in the water without rising above it, thriving underwater. From the tip to the root they’re drenched, steeped, filled, and soaked with cool water. There’s no part of them that’s not soaked with cool water. In the same way, a mendicant drenches, steeps, fills, and spreads their body with bliss free of rapture. There’s no part of the body that’s not spread with bliss free of rapture. 

Furthermore,\marginnote{28.1} giving up pleasure and pain, and ending former happiness and sadness, a mendicant enters and remains in the fourth absorption. It is without pleasure or pain, with pure equanimity and mindfulness. They sit spreading their body through with pure bright mind. There’s no part of the body that’s not spread with pure bright mind. It’s like someone sitting wrapped from head to foot with white cloth. There’s no part of the body that’s not spread over with white cloth. In the same way, they sit spreading their body through with pure bright mind. There’s no part of the body that’s not spread with pure bright mind. And many of my disciples meditate on that having attained perfection and consummation of insight. 

Furthermore,\marginnote{29.1} I have explained to my disciples a practice that they use to understand this: ‘This body of mine is physical. It’s made up of the four primary elements, produced by mother and father, built up from rice and porridge, liable to impermanence, to wearing away and erosion, to breaking up and destruction. And this consciousness of mine is attached to it, tied to it.’ Suppose there was a beryl gem that was naturally beautiful, eight-faceted, well-worked, transparent and clear, endowed with all good qualities. And it was strung with a thread of blue, yellow, red, white, or golden brown. And someone with good eyesight were to take it in their hand and check it: ‘This beryl gem is naturally beautiful, eight-faceted, well-worked, transparent and clear, endowed with all good qualities. And it’s strung with a thread of blue, yellow, red, white, or golden brown.’ 

In\marginnote{29.9} the same way, I have explained to my disciples a practice that they use to understand this: ‘This body of mine is physical. It’s made up of the four primary elements, produced by mother and father, built up from rice and porridge, liable to impermanence, to wearing away and erosion, to breaking up and destruction. And this consciousness of mine is attached to it, tied to it.’ 

And\marginnote{29.12} many of my disciples meditate on that having attained perfection and consummation of insight. 

Furthermore,\marginnote{30.1} I have explained to my disciples a practice that they use to create from this body another body, consisting of form, mind-made, complete in all its various parts, not deficient in any faculty. Suppose a person was to draw a reed out from its sheath. They’d think: ‘This is the reed, this is the sheath. The reed and the sheath are different things. The reed has been drawn out from the sheath.’ Or suppose a person was to draw a sword out from its scabbard. They’d think: ‘This is the sword, this is the scabbard. The sword and the scabbard are different things. The sword has been drawn out from the scabbard.’ Or suppose a person was to draw a snake out from its slough. They’d think: ‘This is the snake, this is the slough. The snake and the slough are different things. The snake has been drawn out from the slough.’ In the same way, I have explained to my disciples a practice that they use to create from this body another body, consisting of form, mind-made, complete in all its various parts, not deficient in any faculty. 

And\marginnote{30.12} many of my disciples meditate on that having attained perfection and consummation of insight. 

Furthermore,\marginnote{31.1} I have explained to my disciples a practice that they use to wield the many kinds of psychic power: multiplying themselves and becoming one again; appearing and disappearing; going unimpeded through a wall, a rampart, or a mountain as if through space; diving in and out of the earth as if it were water; walking on water as if it were earth; flying cross-legged through the sky like a bird; touching and stroking with the hand the sun and moon, so mighty and powerful. They control the body as far as the \textsanskrit{Brahmā} realm. Suppose a deft potter or their apprentice had some well-prepared clay. They could produce any kind of pot that they like. Or suppose a deft ivory-carver or their apprentice had some well-prepared ivory. They could produce any kind of ivory item that they like. Or suppose a deft goldsmith or their apprentice had some well-prepared gold. They could produce any kind of gold item that they like. In the same way, I have explained to my disciples a practice that they use to wield the many kinds of psychic power … 

And\marginnote{31.6} many of my disciples meditate on that having attained perfection and consummation of insight. 

Furthermore,\marginnote{32.1} I have explained to my disciples a practice that they use so that, with clairaudience that is purified and superhuman, they hear both kinds of sounds, human and divine, whether near or far. Suppose there was a powerful horn blower. They’d easily make themselves heard in the four quarters. In the same way, I have explained to my disciples a practice that they use so that, with clairaudience that is purified and superhuman, they hear both kinds of sounds, human and divine, whether near or far. 

And\marginnote{32.4} many of my disciples meditate on that having attained perfection and consummation of insight. 

Furthermore,\marginnote{33.1} I have explained to my disciples a practice that they use to understand the minds of other beings and individuals, having comprehended them with their own mind. They understand mind with greed as ‘mind with greed’, and mind without greed as ‘mind without greed’; mind with hate as ‘mind with hate’, and mind without hate as ‘mind without hate’; mind with delusion as ‘mind with delusion’, and mind without delusion as ‘mind without delusion’; constricted mind as ‘constricted mind’, and scattered mind as ‘scattered mind’; expansive mind as ‘expansive mind’, and unexpansive mind as ‘unexpansive mind’; mind that is not supreme as ‘mind that is not supreme’, and mind that is supreme as ‘mind that is supreme’; mind immersed in \textsanskrit{samādhi} as ‘mind immersed in \textsanskrit{samādhi}’, and mind not immersed in \textsanskrit{samādhi} as ‘mind not immersed in \textsanskrit{samādhi}’; freed mind as ‘freed mind’, and unfreed mind as ‘unfreed mind’. Suppose there was a woman or man who was young, youthful, and fond of adornments, and they check their own reflection in a clean bright mirror or a clear bowl of water. If they had a spot they’d know ‘I have a spot’, and if they had no spots they’d know ‘I have no spots’. In the same way, I have explained to my disciples a practice that they use to understand the minds of other beings and individuals, having comprehended them with their own mind … 

And\marginnote{33.36} many of my disciples meditate on that having attained perfection and consummation of insight. 

Furthermore,\marginnote{34.1} I have explained to my disciples a practice that they use to recollect the many kinds of past lives. That is: one, two, three, four, five, ten, twenty, thirty, forty, fifty, a hundred, a thousand, a hundred thousand rebirths; many eons of the world contracting, many eons of the world expanding, many eons of the world contracting and expanding. ‘There, I was named this, my clan was that, I looked like this, and that was my food. This was how I felt pleasure and pain, and that was how my life ended. When I passed away from that place I was reborn somewhere else. There, too, I was named this, my clan was that, I looked like this, and that was my food. This was how I felt pleasure and pain, and that was how my life ended. When I passed away from that place I was reborn here.’ And so they recollect their many kinds of past lives, with features and details. Suppose a person was to leave their home village and go to another village. From that village they’d go to yet another village. And from that village they’d return to their home village. They’d think: ‘I went from my home village to another village. There I stood like this, sat like that, spoke like this, or kept silent like that. From that village I went to yet another village. There too I stood like this, sat like that, spoke like this, or kept silent like that. And from that village I returned to my home village.’ In the same way, I have explained to my disciples a practice that they use to recollect the many kinds of past lives. 

And\marginnote{34.4} many of my disciples meditate on that having attained perfection and consummation of insight. 

Furthermore,\marginnote{35.1} I have explained to my disciples a practice that they use so that, with clairvoyance that is purified and superhuman, they see sentient beings passing away and being reborn—inferior and superior, beautiful and ugly, in a good place or a bad place. They understand how sentient beings are reborn according to their deeds: ‘These dear beings did bad things by way of body, speech, and mind. They spoke ill of the noble ones; they had wrong view; and they chose to act out of that wrong view. When their body breaks up, after death, they’re reborn in a place of loss, a bad place, the underworld, hell. These dear beings, however, did good things by way of body, speech, and mind. They never spoke ill of the noble ones; they had right view; and they chose to act out of that right view. When their body breaks up, after death, they’re reborn in a good place, a heavenly realm.’ And so, with clairvoyance that is purified and superhuman, they see sentient beings passing away and being reborn—inferior and superior, beautiful and ugly, in a good place or a bad place. They understand how sentient beings are reborn according to their deeds. Suppose there were two houses with doors. A person with good eyesight standing in between them would see people entering and leaving a house and wandering to and fro. In the same way, I have explained to my disciples a practice that they use so that, with clairvoyance that is purified and superhuman, they see sentient beings passing away and being reborn … 

And\marginnote{35.4} many of my disciples meditate on that having attained perfection and consummation of insight. 

Furthermore,\marginnote{36.1} I have explained to my disciples a practice that they use to realize the undefiled freedom of heart and freedom by wisdom in this very life. And they live having realized it with their own insight due to the ending of defilements. Suppose there was a lake that was transparent, clear, and unclouded. A person with good eyesight standing on the bank would see the clams and mussels, and pebbles and gravel, and schools of fish swimming about or staying still. They’d think: ‘This lake is transparent, clear, and unclouded. And here are the clams and mussels, and pebbles and gravel, and schools of fish swimming about or staying still.’ 

In\marginnote{36.3} the same way, I have explained to my disciples a practice that they use to realize the undefiled freedom of heart and freedom by wisdom in this very life. And they live having realized it with their own insight due to the ending of defilements. 

And\marginnote{36.4} many of my disciples meditate on that having attained perfection and consummation of insight. 

This\marginnote{37.1} is the fifth quality because of which my disciples are loyal to me. 

These\marginnote{38.1} are the five qualities because of which my disciples honor, respect, revere, and venerate me; and after honoring and respecting me, they remain loyal to me.” 

That\marginnote{38.2} is what the Buddha said. Satisfied, the wanderer \textsanskrit{Sakuludāyī} was happy with what the Buddha said. 

%
\section*{{\suttatitleacronym MN 78}{\suttatitletranslation With Uggāhamāna Samaṇamuṇḍika }{\suttatitleroot Samaṇamuṇḍikasutta}}
\addcontentsline{toc}{section}{\tocacronym{MN 78} \toctranslation{With Uggāhamāna Samaṇamuṇḍika } \tocroot{Samaṇamuṇḍikasutta}}
\markboth{With Uggāhamāna Samaṇamuṇḍika }{Samaṇamuṇḍikasutta}
\extramarks{MN 78}{MN 78}

\scevam{So\marginnote{1.1} I have heard. }At one time the Buddha was staying near \textsanskrit{Sāvatthī} in Jeta’s Grove, \textsanskrit{Anāthapiṇḍika}’s monastery. 

Now\marginnote{1.3} at that time the wanderer \textsanskrit{Uggāhamāna} \textsanskrit{Samaṇamuṇḍikāputta} was residing together with around three hundred wanderers in \textsanskrit{Mallikā}’s single-halled monastery for group debates, set among the flaking pale-moon ebony trees. 

Then\marginnote{2.1} the master builder \textsanskrit{Pañcakaṅga} left \textsanskrit{Sāvatthī} in the middle of the day to see the Buddha. It occurred to him, “It’s the wrong time to see the Buddha, as he’s in retreat. And it’s the wrong time to see the esteemed mendicants, as they’re in retreat. Why don’t I go to \textsanskrit{Mallikā}’s monastery to visit the wanderer \textsanskrit{Uggāhamāna}?” So that’s what he did. 

Now\marginnote{3.1} at that time, \textsanskrit{Uggāhamāna} was sitting together with a large assembly of wanderers making an uproar, a dreadful racket. They engaged in all kinds of unworthy talk, such as talk about kings, bandits, and ministers; talk about armies, threats, and wars; talk about food, drink, clothes, and beds; talk about garlands and fragrances; talk about family, vehicles, villages, towns, cities, and countries; talk about women and heroes; street talk and well talk; talk about the departed; motley talk; tales of land and sea; and talk about being reborn in this or that state of existence. 

\textsanskrit{Uggāhamāna}\marginnote{3.3} saw \textsanskrit{Pañcakaṅga} coming off in the distance, and hushed his own assembly, “Be quiet, good sirs, don’t make a sound. Here comes \textsanskrit{Pañcakaṅga}, a disciple of the ascetic Gotama. He is included among the white-clothed lay disciples of the ascetic Gotama, who is residing in \textsanskrit{Sāvatthī}. Such venerables like the quiet, are educated to be quiet, and praise the quiet. Hopefully if he sees that our assembly is quiet he’ll see fit to approach.” Then those wanderers fell silent. 

Then\marginnote{4.1} \textsanskrit{Pañcakaṅga} approached \textsanskrit{Uggāhamāna}, and exchanged greetings with him. When the greetings and polite conversation were over, he sat down to one side. \textsanskrit{Uggāhamāna} said to him: 

“Householder,\marginnote{5.1} when an individual has four qualities I describe them as an invincible ascetic—accomplished in the skillful, excelling in the skillful, attained to the highest attainment. What four? It’s when they do no bad deeds with their body; speak no bad words; think no bad thoughts; and don’t earn a living by bad livelihood. When an individual has these four qualities I describe them as an invincible ascetic.” 

Then\marginnote{6.1} \textsanskrit{Pañcakaṅga} neither approved nor dismissed that mendicant’s statement. He got up from his seat, thinking, “I will learn the meaning of this statement from the Buddha himself.” 

Then\marginnote{7.1} he went to the Buddha, bowed, sat down to one side, and informed the Buddha of all that had been discussed. 

When\marginnote{8.1} he had spoken, the Buddha said to him, “Master builder, if what \textsanskrit{Uggāhamāna} says is true, a little baby boy is an invincible ascetic—accomplished in the skillful, excelling in the skillful, attained to the highest attainment. For a little baby doesn’t even have a concept of ‘a body’, so how could they possibly do a bad deed with their body, apart from just wriggling? And a little baby doesn’t even have a concept of ‘speech’, so how could they possibly speak bad words, apart from just crying? And a little baby doesn’t even have a concept of ‘thought’, so how could they possibly think bad thoughts, apart from just whimpering? And a little baby doesn’t even have a concept of ‘livelihood’, so how could they possibly earn a living by bad livelihood, apart from their mother’s breast? If what \textsanskrit{Uggāhamāna} says is true, a little baby boy is an invincible ascetic—accomplished in the skillful, excelling in the skillful, attained to the highest attainment. 

When\marginnote{8.8} an individual has four qualities I describe them, not as an invincible ascetic—accomplished in the skillful, excelling in the skillful, attained to the highest attainment—but as having achieved the same level as a little baby. What four? It’s when they do no bad deeds with their body; speak no bad words; think no bad thoughts; and don’t earn a living by bad livelihood. When an individual has these four qualities I describe them, not as an invincible ascetic, but as having achieved the same level as a little baby. 

When\marginnote{9.1} an individual has ten qualities, master builder, I describe them as an invincible ascetic—accomplished in the skillful, excelling in the skillful, attained to the highest attainment. 

But\marginnote{9.2} certain things must first be understood, I say. ‘These are unskillful behaviors.’ ‘Unskillful behaviors stem from this.’ ‘Here unskillful behaviors cease without anything left over.’ ‘Someone practicing like this is practicing for the cessation of unskillful behaviors.’ 

‘These\marginnote{9.10} are skillful behaviors.’ ‘Skillful behaviors stem from this.’ ‘Here skillful behaviors cease without anything left over.’ ‘Someone practicing like this is practicing for the cessation of skillful behaviors.’ 

‘These\marginnote{9.18} are unskillful thoughts.’ ‘Unskillful thoughts stem from this.’ ‘Here unskillful thoughts cease without anything left over.’ ‘Someone practicing like this is practicing for the cessation of unskillful thoughts.’ 

‘These\marginnote{9.26} are skillful thoughts.’ ‘Skillful thoughts stem from this.’ ‘Here skillful thoughts cease without anything left over.’ ‘Someone practicing like this is practicing for the cessation of skillful thoughts.’ 

And\marginnote{10.1} what, master builder, are unskillful behaviors? Unskillful deeds by way of body and speech, and bad livelihood. These are called unskillful behaviors. 

And\marginnote{10.4} where do these unskillful behaviors stem from? Where they stem from has been stated. You should say that they stem from the mind. What mind? The mind takes many and diverse forms. But unskillful behaviors stem from a mind that has greed, hate, and delusion. 

And\marginnote{10.10} where do these unskillful behaviors cease without anything left over? Their cessation has also been stated. It’s when a mendicant gives up bad conduct by way of body, speech, and mind, and develops good conduct by way of body, speech, and mind; they give up wrong livelihood and earn a living by right livelihood. This is where these unskillful behaviors cease without anything left over. 

And\marginnote{10.14} how is someone practicing for the cessation of unskillful behaviors? It’s when a mendicant generates enthusiasm, tries, makes an effort, exerts the mind, and strives so that bad, unskillful qualities don’t arise. They generate enthusiasm, try, make an effort, exert the mind, and strive so that bad, unskillful qualities that have arisen are given up. They generate enthusiasm, try, make an effort, exert the mind, and strive so that skillful qualities arise. They generate enthusiasm, try, make an effort, exert the mind, and strive so that skillful qualities that have arisen remain, are not lost, but increase, mature, and are completed by development. Someone practicing like this is practicing for the cessation of unskillful behaviors. 

And\marginnote{11.1} what are skillful behaviors? Skillful deeds by way of body and speech, and purified livelihood are included in behavior, I say. These are called skillful behaviors. 

And\marginnote{11.4} where do these skillful behaviors stem from? Where they stem from has been stated. You should say that they stem from the mind. What mind? The mind takes many and diverse forms. But skillful behaviors stem from a mind that is free from greed, hate, and delusion. 

And\marginnote{11.10} where do these skillful behaviors cease without anything left over? Their cessation has also been stated. It’s when a mendicant behaves ethically, but they don’t identify with their ethical behavior. And they truly understand the freedom of heart and freedom by wisdom where these skillful behaviors cease without anything left over. 

And\marginnote{11.14} how is someone practicing for the cessation of skillful behaviors? It’s when a mendicant generates enthusiasm, tries, makes an effort, exerts the mind, and strives so that bad, unskillful qualities don’t arise … so that unskillful qualities are given up … so that skillful qualities arise … so that skillful qualities that have arisen remain, are not lost, but increase, mature, and are fulfilled by development. Someone practicing like this is practicing for the cessation of skillful behaviors. 

And\marginnote{12.1} what are unskillful thoughts? Thoughts of sensuality, of malice, and of cruelty. These are called unskillful thoughts. 

And\marginnote{12.4} where do these unskillful thoughts stem from? Where they stem from has been stated. You should say that they stem from perception. What perception? Perception takes many and diverse forms. Perceptions of sensuality, malice, and cruelty—unskillful thoughts stem from this. 

And\marginnote{12.11} where do these unskillful thoughts cease without anything left over? Their cessation has also been stated. It’s when a mendicant, quite secluded from sensual pleasures, secluded from unskillful qualities, enters and remains in the first absorption, which has the rapture and bliss born of seclusion, while placing the mind and keeping it connected. This is where these unskillful thoughts cease without anything left over. 

And\marginnote{12.15} how is someone practicing for the cessation of unskillful thoughts? It’s when a mendicant generates enthusiasm, tries, makes an effort, exerts the mind, and strives so that bad, unskillful qualities don’t arise … so that unskillful qualities are given up … so that skillful qualities arise … so that skillful qualities that have arisen remain, are not lost, but increase, mature, and are fulfilled by development. Someone practicing like this is practicing for the cessation of unskillful thoughts. 

And\marginnote{13.1} what are skillful thoughts? Thoughts of renunciation, good will, and harmlessness. These are called skillful thoughts. 

And\marginnote{13.4} where do these skillful thoughts stem from? Where they stem from has been stated. You should say that they stem from perception. What perception? Perception takes many and diverse forms. Perceptions of renunciation, good will, and harmlessness—skillful thoughts stem from this. 

And\marginnote{13.11} where do these skillful thoughts cease without anything left over? Their cessation has also been stated. It’s when, as the placing of the mind and keeping it connected are stilled, a mendicant enters and remains in the second absorption, which has the rapture and bliss born of immersion, with internal clarity and confidence, and unified mind, without placing the mind and keeping it connected. This is where these skillful thoughts cease without anything left over. 

And\marginnote{13.15} how is someone practicing for the cessation of skillful thoughts? It’s when a mendicant generates enthusiasm, tries, makes an effort, exerts the mind, and strives so that bad, unskillful qualities don’t arise … so that unskillful qualities are given up … so that skillful qualities arise … so that skillful qualities that have arisen remain, are not lost, but increase, mature, and are fulfilled by development. Someone practicing like this is practicing for the cessation of skillful thoughts. 

Master\marginnote{14.1} builder, when an individual has what ten qualities do I describe them as an invincible ascetic—accomplished in the skillful, excelling in the skillful, attained to the highest attainment? It’s when a mendicant has an adept’s right view, right thought, right speech, right action, right livelihood, right effort, right mindfulness, right immersion, right knowledge, and right freedom. When an individual has these ten qualities, I describe them as an invincible ascetic—accomplished in the skillful, excelling in the skillful, attained to the highest attainment.” 

That\marginnote{14.4} is what the Buddha said. Satisfied, \textsanskrit{Pañcakaṅga} the master builder was happy with what the Buddha said. 

%
\section*{{\suttatitleacronym MN 79}{\suttatitletranslation The Shorter Discourse With Sakuludāyī }{\suttatitleroot Cūḷasakuludāyisutta}}
\addcontentsline{toc}{section}{\tocacronym{MN 79} \toctranslation{The Shorter Discourse With Sakuludāyī } \tocroot{Cūḷasakuludāyisutta}}
\markboth{The Shorter Discourse With Sakuludāyī }{Cūḷasakuludāyisutta}
\extramarks{MN 79}{MN 79}

\scevam{So\marginnote{1.1} I have heard. }At one time the Buddha was staying near \textsanskrit{Rājagaha}, in the Bamboo Grove, the squirrels’ feeding ground. Now at that time the wanderer \textsanskrit{Sakuludāyī} was residing together with a large assembly of wanderers in the monastery of the wanderers in the peacocks’ feeding ground. 

Then\marginnote{2.1} the Buddha robed up in the morning and, taking his bowl and robe, entered \textsanskrit{Rājagaha} for alms. Then it occurred to him, “It’s too early to wander for alms in \textsanskrit{Rājagaha}. Why don’t I visit the wanderer \textsanskrit{Sakuludāyī} at the monastery of the wanderers in the peacocks’ feeding ground?” Then the Buddha went to the monastery of the wanderers. 

Now\marginnote{3.1} at that time, \textsanskrit{Sakuludāyī} was sitting together with a large assembly of wanderers making an uproar, a dreadful racket. They engaged in all kinds of unworthy talk, such as talk about kings, bandits, and ministers; talk about armies, threats, and wars; talk about food, drink, clothes, and beds; talk about garlands and fragrances; talk about family, vehicles, villages, towns, cities, and countries; talk about women and heroes; street talk and well talk; talk about the departed; motley talk; tales of land and sea; and talk about being reborn in this or that state of existence. 

\textsanskrit{Sakuludāyī}\marginnote{3.3} saw the Buddha coming off in the distance, and hushed his own assembly, “Be quiet, good sirs, don’t make a sound. Here comes the ascetic Gotama. The venerable likes quiet and praises quiet. Hopefully if he sees that our assembly is quiet he’ll see fit to approach.” Then those wanderers fell silent. 

Then\marginnote{4.1} the Buddha approached \textsanskrit{Sakuludāyī}, who said to him, “Come, Blessed One! Welcome, Blessed One! It’s been a long time since you took the opportunity to come here. Please, sir, sit down, this seat is ready.” The Buddha sat on the seat spread out, while \textsanskrit{Sakuludāyī} took a low seat and sat to one side. The Buddha said to him, “\textsanskrit{Udāyī}, what were you sitting talking about just now? What conversation was left unfinished?” 

“Sir,\marginnote{5.1} leave aside what we were sitting talking about just now. It won’t be hard for you to hear about that later. When I don’t come to the assembly, they sit and engage in all kinds of unworthy talk. But when I have come to the assembly, they sit gazing up at my face alone, thinking, ‘Whatever the ascetic \textsanskrit{Udāyī} teaches, we shall listen to it.’ But when the Buddha has come to the assembly, both myself and the assembly sit gazing up at your face, thinking, ‘Whatever the Buddha teaches, we shall listen to it.’” 

“Well\marginnote{6.1} then, \textsanskrit{Udāyī}, suggest something for me to talk about.” 

“Master\marginnote{6.2} Gotama, a few days ago someone was claiming to be all-knowing and all-seeing, to know and see everything without exception, thus: ‘Knowledge and vision are constantly and continually present to me, while walking, standing, sleeping, and waking.’ When I asked them a question about the past, they dodged the issue, distracted the discussion with irrelevant points, and displayed annoyance, hate, and bitterness. That reminded me of the Buddha: ‘Surely it must be the Blessed One, the Holy One who is so skilled in such matters.’” 

“But\marginnote{6.6} \textsanskrit{Udāyī}, who was it that made such a claim and behaved in such a way?” 

“It\marginnote{6.7} was \textsanskrit{Nigaṇṭha} \textsanskrit{Nātaputta}, sir.” 

“\textsanskrit{Udāyī},\marginnote{7.1} someone who can recollect their many kinds of past lives, with features and details, might ask me a question about the past, or I might ask them a question about the past. And they might satisfy me with their answer, or I might satisfy them with my answer. 

Someone\marginnote{7.3} who, with clairvoyance that is purified and superhuman, understands how sentient beings are reborn according to their deeds might ask me a question about the future, or I might ask them a question about the future. And they might satisfy me with their answer, or I might satisfy them with my answer. 

Nevertheless,\marginnote{7.5} \textsanskrit{Udāyī}, leave aside the past and the future. I shall teach you the Dhamma: ‘When this exists, that is; due to the arising of this, that arises. When this doesn’t exist, that is not; due to the cessation of this, that ceases.’” 

“Well\marginnote{8.1} sir, I can’t even recall with features and details what I’ve undergone in this incarnation. How should I possibly recollect my many kinds of past lives with features and details, like the Buddha? For I can’t even see a mud-goblin right now. How should I possibly, with clairvoyance that is purified and superhuman, see sentient beings passing away and being reborn, like the Buddha? But then the Buddha told me, ‘Nevertheless, \textsanskrit{Udāyī}, leave aside the past and the future. I shall teach you the Dhamma: 

“When\marginnote{8.8} this exists, that is; due to the arising of this, that arises. When this doesn’t exist, that is not; due to the cessation of this, that ceases.”’ But that is even more unclear to me. Perhaps I might satisfy the Buddha by answering a question about my own tradition.” 

“But\marginnote{9.1} \textsanskrit{Udāyī}, what is your own tradition?” 

“Sir,\marginnote{9.2} it’s this: ‘This is the ultimate splendor, this is the ultimate splendor.’” 

“But\marginnote{9.4} what is that ultimate splendor?” 

“Sir,\marginnote{9.6} the ultimate splendor is the splendor compared to which no other splendor is finer.” 

“But\marginnote{9.7} what is that ultimate splendor compared to which no other splendor is finer?” 

“Sir,\marginnote{9.8} the ultimate splendor is the splendor compared to which no other splendor is finer.” 

“\textsanskrit{Udāyī},\marginnote{10.1} you could draw this out for a long time. You say, ‘The ultimate splendor is the splendor compared to which no other splendor is finer.’ But you don’t describe that splendor. Suppose a man was to say, ‘Whoever the finest lady in the land is, it is her that I want, her I desire!’ They’d say to him, ‘Mister, that finest lady in the land who you desire—do you know whether she’s an aristocrat, a brahmin, a merchant, or a worker?’ Asked this, he’d say, ‘No.’ They’d say to him, ‘Mister, that finest lady in the land who you desire—do you know her name or clan? Whether she’s tall or short or medium? Whether her skin is black, brown, or tawny? What village, town, or city she comes from?’ Asked this, he’d say, ‘No.’ They’d say to him, ‘Mister, do you desire someone who you’ve never even known or seen?’ Asked this, he’d say, ‘Yes.’ 

What\marginnote{10.14} do you think, \textsanskrit{Udāyī}? This being so, doesn’t that man’s statement turn out to have no demonstrable basis?” 

“Clearly\marginnote{10.16} that’s the case, sir.” 

“In\marginnote{10.17} the same way, you say, ‘The ultimate splendor is the splendor compared to which no other splendor is finer.’ But you don’t describe that splendor.” 

“Sir,\marginnote{11.1} suppose there was a beryl gem that was naturally beautiful, eight-faceted, well-worked. When placed on a cream rug it would shine and glow and radiate. Such is the splendor of the self that is well after death.” 

“What\marginnote{12.1} do you think, \textsanskrit{Udāyī}? Which of these two has a finer splendor: such a beryl gem, or a firefly in the dark of night?” 

“A\marginnote{12.3} firefly in the dark of night, sir.” 

“What\marginnote{13.1} do you think, \textsanskrit{Udāyī}? Which of these two has a finer splendor: a firefly in the dark of night, or an oil lamp in the dark of night?” 

“An\marginnote{13.3} oil lamp in the dark of night, sir.” 

“What\marginnote{14.1} do you think, \textsanskrit{Udāyī}? Which of these two has a finer splendor: an oil lamp in the dark of night, or a bonfire in the dark of night?” 

“A\marginnote{14.3} bonfire in the dark of night, sir.” 

“What\marginnote{15.1} do you think, \textsanskrit{Udāyī}? Which of these two has a finer splendor: a bonfire in the dark of night, or the Morning Star in a clear and cloudless sky at the crack of dawn?” 

“The\marginnote{15.3} Morning Star in a clear and cloudless sky at the crack of dawn, sir.” 

“What\marginnote{16.1} do you think, \textsanskrit{Udāyī}? Which of these two has a finer splendor: the Morning Star in a clear and cloudless sky at the crack of dawn, or the full moon at midnight in a clear and cloudless sky on the fifteenth day sabbath?” 

“The\marginnote{16.3} full moon at midnight in a clear and cloudless sky on the fifteenth day sabbath, sir.” 

“What\marginnote{17.1} do you think, \textsanskrit{Udāyī}? Which of these two has a finer splendor: the full moon at midnight in a clear and cloudless sky on the fifteenth day sabbath, or the sun at midday in a clear and cloudless sky in the last month of the rainy season?” 

“The\marginnote{17.3} sun at midday in a clear and cloudless sky in the last month of the rainy season, sir.” 

“Beyond\marginnote{18.1} this, \textsanskrit{Udāyī}, I know very many gods on whom the light of the sun and moon make no impression. Nevertheless, I do not say: ‘The splendor compared to which no other splendor is finer.’ But of the splendor inferior to a firefly you say, ‘This is the ultimate splendor.’ And you don’t describe that splendor.” 

“The\marginnote{19.1} Blessed One has cut short the discussion! The Holy One has cut short the discussion!” 

“But\marginnote{19.2} \textsanskrit{Udāyī}, why do you say this?” 

“Sir,\marginnote{19.4} it says this in our own tradition: ‘This is the ultimate splendor, this is the ultimate splendor.’ But when pursued, pressed, and grilled on our own tradition, we turned out to be void, hollow, and mistaken.” 

“But\marginnote{20.1} \textsanskrit{Udāyī}, is there a world of perfect happiness? And is there a grounded path for realizing a world of perfect happiness?” 

“Sir,\marginnote{20.2} it says this in our own tradition: ‘There is a world of perfect happiness. And there is a grounded path for realizing a world of perfect happiness.’” 

“Well,\marginnote{21.1} what is that grounded path for realizing a world of perfect happiness?” 

“Sir,\marginnote{21.2} it’s when someone gives up killing living creatures, stealing, sexual misconduct, and lying. And they proceed having undertaken some kind of mortification. This is the grounded path for realizing a world of perfect happiness.” 

“What\marginnote{22.1} do you think, \textsanskrit{Udāyī}? On an occasion when someone refrains from killing living creatures, is their self perfectly happy at that time, or does it have both pleasure and pain?” 

“It\marginnote{22.3} has both pleasure and pain.” 

“What\marginnote{22.4} do you think, \textsanskrit{Udāyī}? On an occasion when someone refrains from stealing … sexual misconduct … lying, is their self perfectly happy at that time, or does it have both pleasure and pain?” 

“It\marginnote{22.10} has both pleasure and pain.” 

“What\marginnote{22.11} do you think, \textsanskrit{Udāyī}? On an occasion when someone undertakes and follows some kind of mortification, is their self perfectly happy at that time, or does it have both pleasure and pain?” 

“It\marginnote{22.13} has both pleasure and pain.” 

“What\marginnote{22.14} do you think, \textsanskrit{Udāyī}? Is a perfectly happy world realized by relying on a practice of mixed pleasure and pain?” 

“The\marginnote{23.1} Blessed One has cut short the discussion! The Holy One has cut short the discussion!” 

“But\marginnote{23.2} \textsanskrit{Udāyī}, why do you say this?” 

“Sir,\marginnote{23.4} it says this in our own tradition: ‘There is a world of perfect happiness. And there is a grounded path for realizing a world of perfect happiness.’ But when pursued, pressed, and grilled on our own tradition, we turned out to be void, hollow, and mistaken. 

But\marginnote{23.7} sir, is there a world of perfect happiness? And is there a grounded path for realizing a world of perfect happiness?” 

“There\marginnote{24.1} is a world of perfect happiness, \textsanskrit{Udāyī}. And there is a grounded path for realizing a world of perfect happiness.” 

“Well\marginnote{24.2} sir, what is that grounded path for realizing a world of perfect happiness?” 

“It’s\marginnote{25.1} when a mendicant, quite secluded from sensual pleasures, secluded from unskillful qualities, enters and remains in the first absorption. As the placing of the mind and keeping it connected are stilled, they enter and remain in the second absorption. With the fading away of rapture, they enter and remain in the third absorption. This is the grounded path for realizing a world of perfect happiness.” 

“Sir,\marginnote{25.5} that’s not the grounded path for realizing a world of perfect happiness. At that point a perfectly happy world has already been realized.” 

“No,\marginnote{25.6} \textsanskrit{Udāyī}, at that point a perfectly happy world has not been realized. This is the grounded path for realizing a world of perfect happiness.” 

When\marginnote{26.1} he said this, \textsanskrit{Sakuludāyī}’s assembly made an uproar, a dreadful racket, “In that case, we’re lost, and so are our traditional teachings! We’re lost, and so are our traditional teachings! We know nothing higher than this!” 

Then\marginnote{26.4} \textsanskrit{Sakuludāyī}, having quieted those wanderers, said to the Buddha, “Well sir, at what point is a perfectly happy world realized?” 

“It’s\marginnote{27.2} when, giving up pleasure and pain, and ending former happiness and sadness, a mendicant enters and remains in the fourth absorption. There are deities who have been reborn in a perfectly happy world. That mendicant associates with them, converses, and engages in discussion. It’s at this point that a perfectly happy world has been realized.” 

“Surely\marginnote{28.1} the mendicants must lead the spiritual life under the Buddha for the sake of realizing this perfectly happy world?” 

“No,\marginnote{28.2} \textsanskrit{Udāyī}, the mendicants don’t lead the spiritual life under me for the sake of realizing this perfectly happy world. There are other things that are finer, for the sake of which the mendicants lead the spiritual life under me.” 

“But\marginnote{28.4} what are those finer things?” 

“It’s\marginnote{29{-}36.1} when a Realized One arises in the world, perfected, a fully awakened Buddha, accomplished in knowledge and conduct, holy, knower of the world, supreme guide for those who wish to train, teacher of gods and humans, awakened, blessed. … 

They\marginnote{37.1} give up these five hindrances, corruptions of the heart that weaken wisdom. Then, quite secluded from sensual pleasures, secluded from unskillful qualities, they enter and remain in the first absorption. This is one of the finer things for the sake of which the mendicants lead the spiritual life under me. 

Furthermore,\marginnote{38{-}40.1} as the placing of the mind and keeping it connected are stilled, a mendicant enters and remains in the second absorption … third absorption … fourth absorption. This too is one of the finer things. 

When\marginnote{41.1} their mind has become immersed in \textsanskrit{samādhi} like this—purified, bright, flawless, rid of corruptions, pliable, workable, steady, and imperturbable—they extend it toward recollection of past lives. They recollect many kinds of past lives. That is: one, two, three, four, five, ten, twenty, thirty, forty, fifty, a hundred, a thousand, a hundred thousand rebirths; many eons of the world contracting, many eons of the world expanding, many eons of the world contracting and expanding. They recollect their many kinds of past lives, with features and details. This too is one of the finer things. 

When\marginnote{42.1} their mind has become immersed in \textsanskrit{samādhi} like this—purified, bright, flawless, rid of corruptions, pliable, workable, steady, and imperturbable—they extend it toward knowledge of the death and rebirth of sentient beings. With clairvoyance that is purified and superhuman, they see sentient beings passing away and being reborn—inferior and superior, beautiful and ugly, in a good place or a bad place. They understand how sentient beings are reborn according to their deeds. This too is one of the finer things. 

When\marginnote{43.1} their mind has become immersed in \textsanskrit{samādhi} like this—purified, bright, flawless, rid of corruptions, pliable, workable, steady, and imperturbable—they extend it toward knowledge of the ending of defilements. They truly understand: ‘This is suffering’ … ‘This is the origin of suffering’ … ‘This is the cessation of suffering’ … ‘This is the practice that leads to the cessation of suffering’. They truly understand: ‘These are defilements’ … ‘This is the origin of defilements’ … ‘This is the cessation of defilements’ … ‘This is the practice that leads to the cessation of defilements’. 

Knowing\marginnote{44.1} and seeing like this, their mind is freed from the defilements of sensuality, desire to be reborn, and ignorance. When they’re freed, they know they’re freed. 

They\marginnote{44.3} understand: ‘Rebirth is ended, the spiritual journey has been completed, what had to be done has been done, there is no return to any state of existence.’ This too is one of the finer things. These are the finer things for the sake of which the mendicants lead the spiritual life under me.” 

When\marginnote{45.1} he had spoken, \textsanskrit{Sakuludāyī} said to the Buddha, “Excellent, sir! Excellent! As if he were righting the overturned, or revealing the hidden, or pointing out the path to the lost, or lighting a lamp in the dark so people with good eyes can see what’s there, the Buddha has made the teaching clear in many ways. I go for refuge to the Buddha, to the teaching, and to the mendicant \textsanskrit{Saṅgha}. Sir, may I receive the going forth, the ordination in the Buddha’s presence?” 

When\marginnote{46.1} he said this, \textsanskrit{Sakuludāyī}’s assembly said to him, “Master \textsanskrit{Udāyī}, don’t lead the spiritual life under the ascetic Gotama. You have been a teacher; don’t live as a student. The consequence for you will be as if a water jar were to become a water jug. Master \textsanskrit{Udāyī}, don’t lead the spiritual life under the ascetic Gotama. You have been a teacher; don’t live as a student.” And that’s how the wanderer \textsanskrit{Sakuludāyī}’s own assembly prevented him from leading the spiritual life under the Buddha. 

%
\section*{{\suttatitleacronym MN 80}{\suttatitletranslation With Vekhanasa }{\suttatitleroot Vekhanasasutta}}
\addcontentsline{toc}{section}{\tocacronym{MN 80} \toctranslation{With Vekhanasa } \tocroot{Vekhanasasutta}}
\markboth{With Vekhanasa }{Vekhanasasutta}
\extramarks{MN 80}{MN 80}

\scevam{So\marginnote{1.1} I have heard. }At one time the Buddha was staying near \textsanskrit{Sāvatthī} in Jeta’s Grove, \textsanskrit{Anāthapiṇḍika}’s monastery. 

Then\marginnote{2.1} the wanderer Vekhanasa went up to the Buddha, and exchanged greetings with him. When the greetings and polite conversation were over, he stood to one side, and expressed this heartfelt sentiment: “This is the ultimate splendor, this is the ultimate splendor.” 

“But\marginnote{2.5} \textsanskrit{Kaccāna}, why do you say: ‘This is the ultimate splendor, this is the ultimate splendor.’ What is that ultimate splendor?” 

“Master\marginnote{2.8} Gotama, the ultimate splendor is the splendor compared to which no other splendor is finer.” 

“But\marginnote{2.9} what is that ultimate splendor compared to which no other splendor is finer?” 

“Master\marginnote{2.10} Gotama, the ultimate splendor is the splendor compared to which no other splendor is finer.” 

“\textsanskrit{Kaccāna},\marginnote{3.1} you could draw this out for a long time. You say, ‘The ultimate splendor is the splendor compared to which no other splendor is finer.’ But you don’t describe that splendor. 

Suppose\marginnote{3.3} a man was to say, ‘Whoever the finest lady in the land is, it is her that I want, her I desire!’ 

They’d\marginnote{3.5} say to him, ‘Mister, that finest lady in the land who you desire—do you know whether she’s an aristocrat, a brahmin, a merchant, or a worker?’ 

Asked\marginnote{3.7} this, he’d say, ‘No.’ 

They’d\marginnote{3.8} say to him, ‘Mister, that finest lady in the land who you desire—do you know her name or clan? Whether she’s tall or short or medium? Whether her skin is black, brown, or tawny? What village, town, or city she comes from?’ 

Asked\marginnote{3.10} this, he’d say, ‘No.’ 

They’d\marginnote{3.11} say to him, ‘Mister, do you desire someone who you’ve never even known or seen?’ 

Asked\marginnote{3.13} this, he’d say, ‘Yes.’ 

What\marginnote{3.14} do you think, \textsanskrit{Kaccāna}? This being so, doesn’t that man’s statement turn out to have no demonstrable basis?” 

“Clearly\marginnote{3.16} that’s the case, sir.” 

“In\marginnote{3.17} the same way, you say, ‘The ultimate splendor is the splendor compared to which no other splendor is finer.’ But you don’t describe that splendor.” 

“Master\marginnote{4.1} Gotama, suppose there was a beryl gem that was naturally beautiful, eight-faceted, well-worked. When placed on a cream rug it would shine and glow and radiate. Such is the splendor of the self that is well after death.” 

“What\marginnote{5.1} do you think, \textsanskrit{Kaccāna}? Which of these two has a finer splendor: such a beryl gem, or a firefly in the dark of night?” 

“A\marginnote{5.3} firefly in the dark of night.” 

“What\marginnote{6.1} do you think, \textsanskrit{Kaccāna}? Which of these two has a finer splendor: a firefly in the dark of night, or an oil lamp in the dark of night?” 

“An\marginnote{6.3} oil lamp in the dark of night.” 

“What\marginnote{7.1} do you think, \textsanskrit{Kaccāna}? Which of these two has a finer splendor: an oil lamp in the dark of night, or a bonfire in the dark of night?” 

“A\marginnote{7.3} bonfire in the dark of night.” 

“What\marginnote{8.1} do you think, \textsanskrit{Kaccāna}? Which of these two has a finer splendor: a bonfire in the dark of night, or the Morning Star in a clear and cloudless sky at the crack of dawn?” 

“The\marginnote{8.3} Morning Star in a clear and cloudless sky at the crack of dawn.” 

“What\marginnote{9.1} do you think, \textsanskrit{Kaccāna}? Which of these two has a finer splendor: the Morning Star in a clear and cloudless sky at the crack of dawn, or the full moon at midnight in a clear and cloudless sky on the fifteenth day sabbath?” 

“The\marginnote{9.3} full moon at midnight in a clear and cloudless sky on the fifteenth day sabbath.” 

“What\marginnote{10.1} do you think, \textsanskrit{Kaccāna}? Which of these two has a finer splendor: the full moon at midnight in a clear and cloudless sky on the fifteenth day sabbath, or the sun at midday in a clear and cloudless sky in the last month of the rainy season?” 

“The\marginnote{10.3} sun at midday in a clear and cloudless sky in the last month of the rainy season.” 

“Beyond\marginnote{11.1} this, \textsanskrit{Kaccāna}, I know very many gods on whom the light of the sun and moon make no impression. Nevertheless, I do not say: ‘The splendor compared to which no other splendor is finer.’ But of the splendor inferior to a firefly you say, ‘This is the ultimate splendor.’ And you don’t describe that splendor. 

\textsanskrit{Kaccāna},\marginnote{12.1} there are these five kinds of sensual stimulation. What five? Sights known by the eye that are likable, desirable, agreeable, pleasant, sensual, and arousing. Sounds known by the ear … Smells known by the nose … Tastes known by the tongue … Touches known by the body that are likable, desirable, agreeable, pleasant, sensual, and arousing. These are the five kinds of sensual stimulation. 

The\marginnote{13.1} pleasure and happiness that arises from these five kinds of sensual stimulation is called sensual pleasure. So there is the saying: ‘From the senses comes sensual pleasure. From sensual pleasure comes the best kind of sensual pleasure, which is said to be the best thing there.’” 

When\marginnote{14.1} he said this, Vekhanasa said to the Buddha, “It’s incredible, Master Gotama, it’s amazing! How well said this was by Master Gotama! ‘From the senses comes sensual pleasure. From sensual pleasure comes the best kind of sensual pleasure, which is said to be the best thing there.’ Master Gotama, from the senses comes sensual pleasure. From sensual pleasure comes the best kind of sensual pleasure, which is said to be the best thing there.” 

“\textsanskrit{Kaccāna},\marginnote{14.6} it’s hard for you, who has a different view, creed, preference, practice, and tradition, to understand the senses, sensual pleasure, and the best kind of sensual pleasure. There are mendicants who are perfected, who have ended the defilements, completed the spiritual journey, done what had to be done, laid down the burden, achieved their own goal, utterly ended the fetters of rebirth, and are rightly freed through enlightenment. They can understand the senses, sensual pleasure, and the best kind of sensual pleasure.” 

When\marginnote{15.1} he said this, Vekhanasa became angry and upset. He even attacked and badmouthed the Buddha himself, saying, “The ascetic Gotama will be worsted!” He said to the Buddha, “This is exactly what happens with some ascetics and brahmins. Not knowing the past or seeing the future, they nevertheless claim: ‘We understand: “Rebirth is ended, the spiritual journey has been completed, what had to be done has been done, there is no return to any state of existence.’” Their statement turns out to be a joke—mere words, void and hollow.” 

“\textsanskrit{Kaccāna},\marginnote{16.1} there are some ascetics and brahmins who, not knowing the past or seeing the future, nevertheless claim: ‘We understand: “Rebirth is ended, the spiritual journey has been completed, what had to be done has been done, there is no return to any state of existence.’” There is a legitimate refutation of them. Nevertheless, \textsanskrit{Kaccāna}, leave aside the past and the future. Let a sensible person come—neither devious nor deceitful, a person of integrity. I teach and instruct them. Practicing as instructed they will soon know and see for themselves, ‘So this is how to be rightly released from the bond, that is, the bond of ignorance.’ Suppose there was a little baby bound with swaddling up to the neck. As they grow up and their senses mature, they’re accordingly released from those bonds. They’d know ‘I’m released,’ and there would be no more bonds. 

In\marginnote{16.11} the same way, let a sensible person come—neither devious nor deceitful, a person of integrity. I teach and instruct them. Practicing as instructed they will soon know and see for themselves, ‘So this is how to be rightly released from the bond, that is, the bond of ignorance.’” 

When\marginnote{17.1} he said this, Vekhanasa said to the Buddha, “Excellent, Master Gotama! … From this day forth, may Master Gotama remember me as a lay follower who has gone for refuge for life.” 

%
\addtocontents{toc}{\let\protect\contentsline\protect\nopagecontentsline}
\chapter*{The Chapter on Kings}
\addcontentsline{toc}{chapter}{\tocchapterline{The Chapter on Kings}}
\addtocontents{toc}{\let\protect\contentsline\protect\oldcontentsline}

%
\section*{{\suttatitleacronym MN 81}{\suttatitletranslation With Ghaṭīkāra }{\suttatitleroot Ghaṭikārasutta}}
\addcontentsline{toc}{section}{\tocacronym{MN 81} \toctranslation{With Ghaṭīkāra } \tocroot{Ghaṭikārasutta}}
\markboth{With Ghaṭīkāra }{Ghaṭikārasutta}
\extramarks{MN 81}{MN 81}

\scevam{So\marginnote{1.1} I have heard. }At one time the Buddha was wandering in the land of the Kosalans together with a large \textsanskrit{Saṅgha} of mendicants. Then the Buddha left the road, and at a certain spot he smiled. 

Then\marginnote{2.2} Venerable Ānanda thought, “What is the cause, what is the reason why the Buddha smiled? Realized Ones do not smile for no reason.” 

So\marginnote{2.5} Ānanda arranged his robe over one shoulder, raised his joined palms toward the Buddha, and said, “What is the cause, what is the reason why the Buddha smiled? Realized Ones do not smile for no reason.” 

“Once\marginnote{3.1} upon a time, Ānanda, there was a market town in this spot named \textsanskrit{Vebhaliṅga}. It was successful and prosperous and full of people. And Kassapa, a blessed one, a perfected one, a fully awakened Buddha, lived supported by \textsanskrit{Vebhaliṅga}. It was here, in fact, that he had his monastery, where he sat and advised the mendicant \textsanskrit{Saṅgha}.” 

Then\marginnote{4.1} Ānanda spread out his outer robe folded in four and said to the Buddha, “Well then, sir, may the Blessed One sit here! Then this piece of land will have been occupied by two perfected ones, fully awakened Buddhas.” The Buddha sat on the seat spread out. When he was seated he said to Venerable Ānanda: 

“Once\marginnote{5.1} upon a time, Ānanda, there was a market town in this spot named \textsanskrit{Vebhaliṅga}. It was successful and prosperous and full of people. And Kassapa, a blessed one, a perfected one, a fully awakened Buddha, lived supported by \textsanskrit{Vebhaliṅga}. It was here, in fact, that he had his monastery, where he sat and advised the mendicant \textsanskrit{Saṅgha}. 

The\marginnote{6.1} Buddha Kassapa had as chief supporter in \textsanskrit{Vebhaliṅga} a potter named \textsanskrit{Ghaṭīkāra}. \textsanskrit{Ghaṭīkāra} had a dear friend named \textsanskrit{Jotipāla}, a brahmin student. Then \textsanskrit{Ghaṭīkāra} addressed \textsanskrit{Jotipāla}, ‘Come, dear \textsanskrit{Jotipāla}, let’s go to see the Blessed One Kassapa, the perfected one, the fully awakened Buddha. For I regard it as holy to see that Blessed One.’ 

When\marginnote{6.6} he said this, \textsanskrit{Jotipāla} said to him, ‘Enough, dear \textsanskrit{Ghaṭīkāra}. What’s the use of seeing that baldy, that fake ascetic?’ 

For\marginnote{6.9} a second time … and a third time, \textsanskrit{Ghaṭīkāra} addressed \textsanskrit{Jotipāla}, ‘Come, dear \textsanskrit{Jotipāla}, let’s go to see the Blessed One Kassapa, the perfected one, the fully awakened Buddha. For I regard it as holy to see that Blessed One.’ 

For\marginnote{6.13} a third time, \textsanskrit{Jotipāla} said to him, ‘Enough, dear \textsanskrit{Ghaṭīkāra}. What’s the use of seeing that baldy, that fake ascetic?’ 

‘Well\marginnote{6.16} then, dear \textsanskrit{Jotipāla}, let’s take some bathing paste of powdered shell and go to the river to bathe.’ 

‘Yes,\marginnote{6.17} dear,’ replied \textsanskrit{Jotipāla}. So that’s what they did. 

Then\marginnote{7.1} \textsanskrit{Ghaṭīkāra} addressed \textsanskrit{Jotipāla}, ‘Dear \textsanskrit{Jotipāla}, the Buddha Kassapa’s monastery is not far away. Let’s go to see the Blessed One Kassapa, the perfected one, the fully awakened Buddha. For I regard it as holy to see that Blessed One.’ 

When\marginnote{7.5} he said this, \textsanskrit{Jotipāla} said to him, ‘Enough, dear \textsanskrit{Ghaṭīkāra}. What’s the use of seeing that baldy, that fake ascetic?’ 

For\marginnote{7.8} a second time … and a third time, \textsanskrit{Ghaṭīkāra} addressed \textsanskrit{Jotipāla}, ‘Dear \textsanskrit{Jotipāla}, the Buddha Kassapa’s monastery is not far away. Let’s go to see the Blessed One Kassapa, the perfected one, the fully awakened Buddha. For I regard it as holy to see that Blessed One.’ 

For\marginnote{7.13} a third time, \textsanskrit{Jotipāla} said to him, ‘Enough, dear \textsanskrit{Ghaṭīkāra}. What’s the use of seeing that baldy, that fake ascetic?’ 

Then\marginnote{8.1} \textsanskrit{Ghaṭīkāra} grabbed \textsanskrit{Jotipāla} by the belt and said, ‘Dear \textsanskrit{Jotipāla}, the Buddha Kassapa’s monastery is not far away. Let’s go to see the Blessed One Kassapa, the perfected one, the fully awakened Buddha. For I regard it as holy to see that Blessed One.’ 

So\marginnote{8.5} \textsanskrit{Jotipāla} undid his belt and said to \textsanskrit{Ghaṭīkāra}, ‘Enough, dear \textsanskrit{Ghaṭīkāra}. What’s the use of seeing that baldy, that fake ascetic?’ 

Then\marginnote{9.1} \textsanskrit{Ghaṭīkāra} grabbed \textsanskrit{Jotipāla} by the hair of his freshly-washed head and said, ‘Dear \textsanskrit{Jotipāla}, the Buddha Kassapa’s monastery is not far away. Let’s go to see the Blessed One Kassapa, the perfected one, the fully awakened Buddha. For I regard it as holy to see that Blessed One.’ 

Then\marginnote{9.5} \textsanskrit{Jotipāla} thought, ‘It’s incredible, it’s amazing, how this potter \textsanskrit{Ghaṭīkāra}, though born in a lower caste, should presume to grab me by the hair of my freshly-washed head! This must be no ordinary matter.’ He said to \textsanskrit{Ghaṭīkāra}, ‘You’d even milk it to this extent, dear \textsanskrit{Ghaṭīkāra}?’ 

‘I\marginnote{9.11} even milk it to this extent, dear \textsanskrit{Jotipāla}. For that is how holy I regard it to see that Blessed One.’ 

‘Well\marginnote{9.13} then, dear \textsanskrit{Ghaṭīkāra}, release me, we shall go.’ 

Then\marginnote{10.1} \textsanskrit{Ghaṭīkāra} the potter and \textsanskrit{Jotipāla} the brahmin student went to the Buddha Kassapa. \textsanskrit{Ghaṭīkāra} bowed and sat down to one side, but \textsanskrit{Jotipāla} exchanged greetings with the Buddha and sat down to one side. 

\textsanskrit{Ghaṭīkāra}\marginnote{10.2} said to the Buddha Kassapa, ‘Sir, this is my dear friend \textsanskrit{Jotipāla}, a brahmin student. Please teach him the Dhamma.’ Then the Buddha Kassapa educated, encouraged, fired up, and inspired \textsanskrit{Ghaṭīkāra} and \textsanskrit{Jotipāla} with a Dhamma talk. Then they got up from their seat, bowed, and respectfully circled the Buddha Kassapa, keeping him on their right, before leaving. 

Then\marginnote{11.1} \textsanskrit{Jotipāla} said to \textsanskrit{Ghatīkāra}, ‘Dear \textsanskrit{Ghaṭīkāra}, you have heard this teaching, so why don’t you go forth from the lay life to homelessness?’ 

‘Don’t\marginnote{11.3} you know, dear \textsanskrit{Jotipāla}, that I look after my blind old parents?’ 

‘Well\marginnote{11.4} then, dear \textsanskrit{Ghaṭīkāra}, I shall go forth from the lay life to homelessness.’ 

Then\marginnote{12.1} \textsanskrit{Ghaṭīkāra} and \textsanskrit{Jotipāla} went to the Buddha Kassapa, bowed and sat down to one side. \textsanskrit{Ghaṭīkāra} said to the Buddha Kassapa, ‘Sir, this is my dear friend \textsanskrit{Jotipāla}, a brahmin student. Please give him the going forth.’ And \textsanskrit{Jotipāla} the brahmin student received the going forth, the ordination in the Buddha’s presence. 

Not\marginnote{13.1} long after \textsanskrit{Jotipāla}’s ordination, a fortnight later, the Buddha Kassapa—having stayed in \textsanskrit{Vebhaliṅga} as long as he wished—set out for Benares. Traveling stage by stage, he arrived at Benares, where he stayed near Benares, in the deer park at Isipatana. King \textsanskrit{Kikī} of \textsanskrit{Kāsi} heard that he had arrived. He had the finest carriages harnessed. He then mounted a fine carriage and, along with other fine carriages, set out in full royal pomp from Benares to see the Buddha Kassapa. He went by carriage as far as the terrain allowed, then descended and approached the Buddha Kassapa on foot. He bowed and sat down to one side. The Buddha educated, encouraged, fired up, and inspired him with a Dhamma talk. 

Then\marginnote{14.5} King \textsanskrit{Kikī} said to the Buddha, ‘Sir, would the Buddha together with the mendicant \textsanskrit{Saṅgha} please accept tomorrow’s meal from me?’ The Buddha Kassapa consented in silence. 

Then,\marginnote{15.1} knowing that the Buddha had consented, King \textsanskrit{Kikī} got up from his seat, bowed, and respectfully circled the Buddha, keeping him on his right, before leaving. And when the night had passed, King \textsanskrit{Kikī} had a variety of delicious foods prepared in his own home—soft saffron rice with the dark grains picked out, served with many soups and sauces. Then he had the Buddha informed of the time, saying, ‘Sir, it’s time. The meal is ready.’ 

Then\marginnote{17.1} Kassapa Buddha robed up in the morning and, taking his bowl and robe, went to the home of King \textsanskrit{Kikī}, where he sat on the seat spread out, together with the \textsanskrit{Saṅgha} of mendicants. Then King \textsanskrit{Kikī} served and satisfied the mendicant \textsanskrit{Saṅgha} headed by the Buddha with his own hands with a variety of delicious foods. 

When\marginnote{17.3} the Buddha Kassapa had eaten and washed his hand and bowl, King \textsanskrit{Kikī} took a low seat and sat to one side. There he said to the Buddha Kassapa, ‘Sir, may the Buddha please accept my invitation to reside in Benares for the rainy season. The \textsanskrit{Saṅgha} will be looked after in the same style.’ 

‘Enough,\marginnote{17.7} great king. I have already accepted an invitation for the rains residence.’ 

For\marginnote{17.9} a second time … and a third time King \textsanskrit{Kikī} said to the Buddha Kassapa, ‘Sir, may the Buddha please accept my invitation to reside in Benares for the rainy season. The \textsanskrit{Saṅgha} will be looked after in the same style.’ 

‘Enough,\marginnote{17.13} Great King. I have already accepted an invitation for the rains residence.’ 

Then\marginnote{17.15} King \textsanskrit{Kikī}, thinking, ‘The Buddha does not accept my invitation to reside for the rains in Benares,’ became sad and upset. Then King \textsanskrit{Kikī} said to the Buddha Kassapa, ‘Sir, do you have another supporter better than me?’ 

‘Great\marginnote{18.1} king, there is a market town named \textsanskrit{Vebhaliṅga}, where there’s a potter named \textsanskrit{Ghaṭīkāra}. He is my chief supporter. Now, great king, you thought, “The Buddha does not accept my invitation to reside for the rains in Benares,” and you became sad and upset. But \textsanskrit{Ghaṭīkāra} doesn’t get upset, nor will he. 

\textsanskrit{Ghaṭīkāra}\marginnote{18.6} has gone for refuge to the Buddha, the teaching, and the \textsanskrit{Saṅgha}. He doesn’t kill living creatures, steal, commit sexual misconduct, lie, or take alcoholic drinks that cause negligence. He has experiential confidence in the Buddha, the teaching, and the \textsanskrit{Saṅgha}, and has the ethics loved by the noble ones. He is free of doubt regarding suffering, its origin, its cessation, and the practice that leads to its cessation. He eats in one part of the day; he’s celibate, ethical, and of good character. He has set aside gems and gold, and rejected gold and money. He’s put down the shovel and doesn’t dig the earth with his own hands. He takes what has crumbled off by a riverbank or been dug up by mice, and brings it back in a carrier. When he has made a pot, he says, “Anyone may leave bagged sesame, mung beans, or chickpeas here and take what they wish.” He looks after his blind old parents. And since he has ended the five lower fetters, \textsanskrit{Ghaṭīkāra} will be reborn spontaneously and will become extinguished there, not liable to return from that world. 

This\marginnote{19.1} one time, great king, I was staying near the market town of \textsanskrit{Vebhaliṅga}. Then I robed up in the morning and, taking my bowl and robe, went to the home of \textsanskrit{Ghaṭīkāra}’s parents, where I said to them, “Excuse me, where has Bhaggava gone?” 

“Your\marginnote{19.4} supporter has gone out, sir. But take rice from the pot and sauce from the pan and eat.” So that’s what I did. And after eating I got up from my seat and left. 

Then\marginnote{19.6} \textsanskrit{Ghaṭīkāra} went up to his parents and said, “Who took rice from the pot and sauce from the pan, ate it, and left?” 

“It\marginnote{19.8} was the Buddha Kassapa, my dear.” 

Then\marginnote{19.9} \textsanskrit{Ghaṭīkāra} thought, “I’m so fortunate, so very fortunate, in that the Buddha Kassapa trusts me so much!” Then joy and happiness did not leave him for a fortnight, or his parents for a week. 

Another\marginnote{20.1} time, great king, I was staying near that same market town of \textsanskrit{Vebhaliṅga}. Then I robed up in the morning and, taking my bowl and robe, went to the home of \textsanskrit{Ghaṭīkāra}’s parents, where I said to them, “Excuse me, where has Bhaggava gone?” 

“Your\marginnote{20.4} supporter has gone out, sir. But take porridge from the pot and sauce from the pan and eat.” So that’s what I did. And after eating I got up from my seat and left. 

Then\marginnote{20.6} \textsanskrit{Ghaṭīkāra} went up to his parents and said, “Who took porridge from the pot and sauce from the pan, ate it, and left?” 

“It\marginnote{20.8} was the Buddha Kassapa, my dear.” 

Then\marginnote{20.9} \textsanskrit{Ghaṭīkāra} thought, “I’m so fortunate, so very fortunate, to be trusted so much by the Buddha Kassapa!” Then joy and happiness did not leave him for a fortnight, or his parents for a week. 

Another\marginnote{21.1} time, great king, I was staying near that same market town of \textsanskrit{Vebhaliṅga}. Now at that time my hut leaked. So I addressed the mendicants, 

“Mendicants,\marginnote{21.4} go to \textsanskrit{Ghaṭīkāra}’s home and find some grass.” 

When\marginnote{21.5} I said this, those mendicants said to me, “Sir, there’s no grass there, but his workshop has a grass roof.” 

“Then\marginnote{21.7} go to the workshop and strip the grass.” So that’s what they did. 

Then\marginnote{21.9} \textsanskrit{Ghaṭīkāra}’s parents said to those mendicants, “Who’s stripping the grass from the workshop?” 

“It’s\marginnote{21.11} the mendicants, sister. The Buddha’s hut is leaking.” 

“Take\marginnote{21.12} it, sirs! Take it, my dears!” 

Then\marginnote{21.13} \textsanskrit{Ghaṭīkāra} went up to his parents and said, “Who stripped the grass from the workshop?” 

“It\marginnote{21.15} was the mendicants, dear. It seems the Buddha’s hut is leaking.” 

Then\marginnote{21.16} \textsanskrit{Ghaṭīkāra} thought, “I’m so fortunate, so very fortunate, to be trusted so much by the Buddha Kassapa!” Then joy and happiness did not leave him for a fortnight, or his parents for a week. 

Then\marginnote{21.20} the workshop remained with the sky for a roof for the whole three months, but no rain fell on it. And that, great king, is what \textsanskrit{Ghaṭīkāra} the potter is like.’ 

‘\textsanskrit{Ghaṭīkāra}\marginnote{21.22} the potter is fortunate, very fortunate, to be so trusted by the Buddha Kassapa.’ 

Then\marginnote{22.1} King \textsanskrit{Kikī} sent around five hundred cartloads of rice, soft saffron rice, and suitable sauce to \textsanskrit{Ghaṭīkāra}. Then one of the king’s men approached \textsanskrit{Ghaṭīkāra} and said, ‘Sir, these five hundred cartloads of rice, soft saffron rice, and suitable sauce have been sent to you by King \textsanskrit{Kikī} of \textsanskrit{Kāsī}. Please accept them.’ 

‘The\marginnote{22.5} king has many duties, and much to do. I have enough. Let this be for the king himself.’ 

Ānanda,\marginnote{23.1} you might think: ‘Surely the brahmin student \textsanskrit{Jotipāla} must have been someone else at that time?’ But you should not see it like this. I myself was the student \textsanskrit{Jotipāla} at that time.” 

That\marginnote{23.5} is what the Buddha said. Satisfied, Venerable Ānanda was happy with what the Buddha said. 

%
\section*{{\suttatitleacronym MN 82}{\suttatitletranslation With Raṭṭhapāla }{\suttatitleroot Raṭṭhapālasutta}}
\addcontentsline{toc}{section}{\tocacronym{MN 82} \toctranslation{With Raṭṭhapāla } \tocroot{Raṭṭhapālasutta}}
\markboth{With Raṭṭhapāla }{Raṭṭhapālasutta}
\extramarks{MN 82}{MN 82}

\scevam{So\marginnote{1.1} I have heard. }At one time the Buddha was wandering in the land of the Kurus together with a large \textsanskrit{Saṅgha} of mendicants when he arrived at a town of the Kurus named \textsanskrit{Thullakoṭṭhita}. 

The\marginnote{2.1} brahmins and householders of \textsanskrit{Thullakoṭṭhita} heard: 

“It\marginnote{2.2} seems the ascetic Gotama—a Sakyan, gone forth from a Sakyan family—has arrived at \textsanskrit{Thullakoṭṭhita}, together with a large \textsanskrit{Saṅgha} of mendicants. He has this good reputation: ‘That Blessed One is perfected, a fully awakened Buddha, accomplished in knowledge and conduct, holy, knower of the world, supreme guide for those who wish to train, teacher of gods and humans, awakened, blessed.’ He has realized with his own insight this world—with its gods, \textsanskrit{Māras} and \textsanskrit{Brahmās}, this population with its ascetics and brahmins, gods and humans—and he makes it known to others. He teaches Dhamma that’s good in the beginning, good in the middle, and good in the end, meaningful and well-phrased. And he reveals a spiritual practice that’s entirely full and pure. It’s good to see such perfected ones.” 

Then\marginnote{3.1} the brahmins and householders of \textsanskrit{Thullakoṭṭhita} went up to the Buddha. Before sitting down to one side, some bowed, some exchanged greetings and polite conversation, some held up their joined palms toward the Buddha, some announced their name and clan, while some kept silent. When they were seated, the Buddha educated, encouraged, fired up, and inspired them with a Dhamma talk. 

Now\marginnote{4.1} at that time a gentleman named \textsanskrit{Raṭṭhapāla}, the son of the leading clan in \textsanskrit{Thullakoṭṭhita}, was sitting in the assembly. He thought, “As I understand the Buddha’s teachings, it’s not easy for someone living at home to lead the spiritual life utterly full and pure, like a polished shell. Why don’t I shave off my hair and beard, dress in ocher robes, and go forth from lay life to homelessness?” 

Then,\marginnote{5.1} having approved and agreed with what the Buddha said, the brahmins and householders of \textsanskrit{Thullakoṭṭhita} got up from their seat, bowed, and respectfully circled the Buddha, keeping him on their right, before leaving. 

Soon\marginnote{6.1} after they left, \textsanskrit{Raṭṭhapāla} went up to the Buddha, bowed, sat down to one side, and said to him, “Sir, as I understand the Buddha’s teachings, it’s not easy for someone living at home to lead the spiritual life utterly full and pure, like a polished shell. I wish to shave off my hair and beard, dress in ocher robes, and go forth from the lay life to homelessness. Sir, may I receive the going forth, the ordination in the Buddha’s presence? May the Buddha please give me the going forth!” 

“But,\marginnote{6.6} \textsanskrit{Raṭṭhapāla}, do you have your parents’ permission?” 

“No,\marginnote{6.7} sir.” 

“\textsanskrit{Raṭṭhapāla},\marginnote{6.8} Buddhas don’t give the going forth to the child of parents who haven’t given their permission.” 

“I’ll\marginnote{6.9} make sure, sir, to get my parents’ permission.” 

Then\marginnote{7.1} \textsanskrit{Raṭṭhapāla} got up from his seat, bowed, and respectfully circled the Buddha. Then he went to his parents and said, “Mum and dad, as I understand the Buddha’s teachings, it’s not easy for someone living at home to lead the spiritual life utterly full and pure, like a polished shell. I wish to shave off my hair and beard, dress in ocher robes, and go forth from the lay life to homelessness. Please give me permission to go forth.” 

When\marginnote{7.5} he said this, \textsanskrit{Raṭṭhapāla}’s parents said to him, “But, dear \textsanskrit{Raṭṭhapāla}, you’re our only child. Youʼre dear to us and we love you. You’re dainty and raised in comfort. You know nothing of suffering. When you die we will lose you against our wishes. So how can we allow you to go forth while you’re still alive?” 

For\marginnote{7.10} a second time, and a third time, \textsanskrit{Raṭṭhapāla} asked his parents for permission, but got the same reply. 

Then\marginnote{7.20} \textsanskrit{Raṭṭhapāla} thought, “My parents don’t allow me to go forth.” He laid down there on the bare ground, saying, “I’ll either die right here or go forth.” And he refused to eat, up to the seventh meal. 

Then\marginnote{8.1} \textsanskrit{Raṭṭhapāla}’s parents said to him, “Dear \textsanskrit{Raṭṭhapāla}, youʼre our only child. You’re dear to us and we love you. You’re dainty and raised in comfort. You know nothing of suffering. When you die we will lose you against our wishes. So how can we allow you to go forth from lay life to homelessness while youʼre still living? Get up, \textsanskrit{Raṭṭhapāla}! Eat, drink, and amuse yourself. While enjoying sensual pleasures, delight in making merit. We don’t allow you to go forth. When you die we will lose you against our wishes. So how can we allow you to go forth while you’re still alive?” 

When\marginnote{8.11} they said this, \textsanskrit{Raṭṭhapāla} kept silent. 

For\marginnote{8.12} a second time, and a third time, \textsanskrit{Raṭṭhapāla}’s parents made the same request. 

And\marginnote{9.1} for a third time, \textsanskrit{Raṭṭhapāla} kept silent. \textsanskrit{Raṭṭhapāla}’s parents then went to see his friends. They told them of the situation and asked for their help. 

Then\marginnote{10.1} \textsanskrit{Raṭṭhapāla}’s friends went to him and said, “Our friend \textsanskrit{Raṭṭhapāla}, you are your parents’ only child. Youʼre dear to them and they love you. You’re dainty and raised in comfort. You know nothing of suffering. When you die your parents will lose you against their wishes. So how can they allow you to go forth while you’re still alive? Get up, \textsanskrit{Raṭṭhapāla}! Eat, drink, and amuse yourself. While enjoying sensual pleasures, delight in making merit. Your parents will not allow you to go forth. When you die your parents will lose you against their wishes. So how can they allow you to go forth while you’re still alive?” 

When\marginnote{10.11} they said this, \textsanskrit{Raṭṭhapāla} kept silent. 

For\marginnote{10.12} a second time, and a third time, \textsanskrit{Raṭṭhapāla}’s friends made the same request. And for a third time, \textsanskrit{Raṭṭhapāla} kept silent. 

Then\marginnote{11.1} \textsanskrit{Raṭṭhapāla}’s friends went to his parents and said, “Mum and dad, \textsanskrit{Raṭṭhapāla} is lying there on the bare ground saying: ‘I’ll either die right here or go forth.’ If you donʼt allow him to go forth, he’ll die there. But if you do allow him to go forth, you’ll see him again afterwards. And if he doesnʼt enjoy the renunciate life, where else will he have to go? He’ll come right back here. Please give \textsanskrit{Raṭṭhapāla} permission to go forth.” 

“Then,\marginnote{11.8} dears, we give \textsanskrit{Raṭṭhapāla} permission to go forth. But once gone forth he must visit his parents.” 

Then\marginnote{11.10} \textsanskrit{Raṭṭhapāla}’s friends went to him and said, “Get up, \textsanskrit{Raṭṭhapāla}! Your parents have given you permission to go forth from lay life to homelessness. But once gone forth you must visit your parents.” 

\textsanskrit{Raṭṭhapāla}\marginnote{12.1} got up and regained his strength. He went to the Buddha, bowed, sat down to one side, and said to him, “Sir, I have my parentsʼ permission to go forth from the lay life to homelessness. May the Buddha please give me the going forth.” 

And\marginnote{13.1} \textsanskrit{Raṭṭhapāla} received the going forth, the ordination in the Buddha’s presence. Not long after Venerable \textsanskrit{Raṭṭhapāla}’s ordination, a fortnight later, the Buddha—having stayed in \textsanskrit{Thullakoṭṭhita} as long as he wished—set out for \textsanskrit{Sāvatthī}. Traveling stage by stage, he arrived at \textsanskrit{Sāvatthī}, where he stayed in Jeta’s Grove, \textsanskrit{Anāthapiṇḍika}’s monastery. 

Then\marginnote{14.1} Venerable \textsanskrit{Raṭṭhapāla}, living alone, withdrawn, diligent, keen, and resolute, soon realized the supreme end of the spiritual path in this very life. He lived having achieved with his own insight the goal for which gentlemen rightly go forth from the lay life to homelessness. 

He\marginnote{14.2} understood: “Rebirth is ended; the spiritual journey has been completed; what had to be done has been done; there is no return to any state of existence.” And Venerable \textsanskrit{Raṭṭhapāla} became one of the perfected. 

Then\marginnote{15.1} he went up to the Buddha, bowed, sat down to one side, and said to him, “Sir, I’d like to visit my parents, if the Buddha allows it.” 

Then\marginnote{15.3} the Buddha focused on comprehending \textsanskrit{Raṭṭhapāla}’s mind. When he knew that it was impossible for \textsanskrit{Raṭṭhapāla} to resign the training and return to a lesser life, he said, “Please, \textsanskrit{Raṭṭhapāla}, go at your convenience.” 

And\marginnote{16.1} then \textsanskrit{Raṭṭhapāla} got up from his seat, bowed, and respectfully circled the Buddha, keeping him on his right. Then he set his lodgings in order and, taking his bowl and robe, set out for \textsanskrit{Thullakoṭṭhita}. Traveling stage by stage, he arrived at \textsanskrit{Thullakoṭṭhika}, where he stayed in King Koravya’s deer range. Then \textsanskrit{Raṭṭhapāla} robed up in the morning and, taking his bowl and robe, entered \textsanskrit{Thullakoṭṭhita} for alms. Wandering indiscriminately for almsfood, he approached his own father’s house. 

Now\marginnote{17.1} at that time \textsanskrit{Raṭṭhapāla}’s father was having his hair dressed in the hall of the middle gate. He saw \textsanskrit{Raṭṭhapāla} coming off in the distance and said, 

“Our\marginnote{17.4} dear and beloved only son was made to go forth by these shavelings, these fake ascetics!” And at his own father’s home \textsanskrit{Raṭṭhapāla} received neither alms nor a polite refusal, but only abuse. 

Now\marginnote{18.1} at that time a family bondservant wanted to throw away the previous night’s porridge. So \textsanskrit{Raṭṭhapāla} said to her, “If that’s to be thrown away, sister, pour it here in my bowl.” As she was pouring the porridge into his bowl, she recognized the features of his hands, feet, and voice. 

She\marginnote{18.5} then went to his mother and said, “Please, madam, you should know this. Master \textsanskrit{Raṭṭhapāla} has arrived.” 

“Wow!\marginnote{18.8} If you speak the truth, I’ll make you a free woman!” 

Then\marginnote{18.9} \textsanskrit{Raṭṭhapāla}’s mother went to his father and said, “Please householder, you should know this. It seems our son \textsanskrit{Raṭṭhapāla} has arrived.” 

Now\marginnote{19.1} at that time \textsanskrit{Raṭṭhapāla} was eating last night’s porridge by a wall. Then \textsanskrit{Raṭṭhapāla}’s father went up to him and said, “Dear \textsanskrit{Raṭṭhapāla}! There’s … and youʼll be eating last night’s porridge! Why not go to your own home?” 

“Householder,\marginnote{19.5} how could those of us who have gone forth from the lay life to homelessness have a house? We’re homeless, householder. I came to your house, but there I received neither alms nor a polite refusal, but only abuse.” 

“Come,\marginnote{19.9} dear \textsanskrit{Raṭṭhapāla}, let’s go to the house.” 

“Enough,\marginnote{19.10} householder. My meal is finished for today.” 

“Well\marginnote{19.11} then, dear \textsanskrit{Raṭṭhapāla}, please accept tomorrow’s meal from me.” \textsanskrit{Raṭṭhapāla} consented in silence. 

Then,\marginnote{20.1} knowing that \textsanskrit{Raṭṭhapāla} had consented, his father went back to his own house. He made a heap of gold coins and bullion and hid it under mats. Then he addressed \textsanskrit{Raṭṭhapāla}’s former wives, “Please, daughters-in-law, adorn yourselves in the way that our son \textsanskrit{Raṭṭhapāla} found you most adorable.” 

And\marginnote{21.1} when the night had passed \textsanskrit{Raṭṭhapāla}’s father had a variety of delicious foods prepared in his own home, and announced the time to the Venerable \textsanskrit{Raṭṭhapāla}, saying, “Sir, it’s time. The meal is ready.” 

Then\marginnote{22.1} \textsanskrit{Raṭṭhapāla} robed up in the morning and, taking his bowl and robe, went to his father’s home, and sat down on the seat spread out. \textsanskrit{Raṭṭhapāla}’s father, revealing the heap of gold coins and bullion, said to him, “Dear \textsanskrit{Raṭṭhapāla}, this is your maternal fortune. There’s another paternal fortune, and an ancestral one. You can both enjoy your wealth and make merit. Come, return to a lesser life, enjoy wealth, and make merit!” 

“If\marginnote{22.5} you’d follow my advice, householder, you’d have this heap of gold loaded on a cart and carried away to be dumped in the middle of the Ganges river. Why is that? Because this will bring you nothing but sorrow, lamentation, pain, sadness, and distress.” 

Then\marginnote{23.1} \textsanskrit{Raṭṭhapāla}’s former wives each clasped his feet and said, “What are they like, master, the nymphs for whom you lead the spiritual life?” 

“Sisters,\marginnote{23.3} I don’t lead the spiritual life for the sake of nymphs.” 

Saying,\marginnote{23.4} “Our master \textsanskrit{Raṭṭhapāla} addresses us as sisters!” they fainted right away. 

Then\marginnote{24.1} \textsanskrit{Raṭṭhapāla} said to his father, “If there is food to be given, householder, please give it. But don’t harass me.” 

“Eat,\marginnote{24.4} dear \textsanskrit{Raṭṭhapāla}. The meal is ready.” Then \textsanskrit{Raṭṭhapāla}’s father served and satisfied Venerable \textsanskrit{Raṭṭhapāla} with his own hands with a variety of delicious foods. 

When\marginnote{24.6} he had eaten and washed his hand and bowl, he recited this verse while standing right there: 

\begin{verse}%
“See\marginnote{25.1} this fancy puppet, \\
a body built of sores, \\
diseased, obsessed over, \\
in which nothing lasts at all. 

See\marginnote{25.5} this fancy figure, \\
with its gems and earrings; \\
it is bones wrapped in skin, \\
made pretty by its clothes. 

Rouged\marginnote{25.9} feet \\
and powdered face \\
may be enough to beguile a fool, \\
but not a seeker of the far shore. 

Hair\marginnote{25.13} in eight braids \\
and eyeliner \\
may be enough to beguile a fool, \\
but not a seeker of the far shore. 

A\marginnote{25.17} rotting body all adorned \\
like a freshly painted makeup box \\
may be enough to beguile a fool, \\
but not a seeker of the far shore. 

The\marginnote{25.21} hunter laid his snare, \\
but the deer didn’t spring the trap. \\
I’ve eaten the bait and now I go, \\
leaving the trapper to lament.” 

%
\end{verse}

Then\marginnote{26.1} \textsanskrit{Raṭṭhapāla}, having recited this verse while standing, went to King Koravya’s deer range and sat at the root of a tree for the day’s meditation. 

Then\marginnote{27.1} King Koravya addressed his gamekeeper, “My good gamekeeper, tidy up the park of the deer range. We will go to see the scenery.” 

“Yes,\marginnote{27.4} Your Majesty,” replied the gamekeeper. While tidying the deer range he saw \textsanskrit{Raṭṭhapāla} sitting in meditation. Seeing this, he went to the king, and said, “The deer range is tidy, sire. And the gentleman named \textsanskrit{Raṭṭhapāla}, the son of the leading clan in \textsanskrit{Thullakoṭṭhita}, of whom you have often spoken highly, is meditating there at the root of a tree.” 

“Well\marginnote{27.8} then, my good gamekeeper, that’s enough of the park for today. Now I shall pay homage to the Master \textsanskrit{Raṭṭhapāla}.” 

And\marginnote{28.1} then King Koravya said, “Give away all the different foods that have been prepared there.” He had the finest carriages harnessed. Then he mounted a fine carriage and, along with other fine carriages, set out in full royal pomp from \textsanskrit{Thullakoṭṭhita} to see \textsanskrit{Raṭṭhapāla}. He went by carriage as far as the terrain allowed, then descended and approached \textsanskrit{Raṭṭhapāla} on foot, together with a group of eminent officials. They exchanged greetings, and, when the greetings and polite conversation were over, he stood to one side, and said to \textsanskrit{Raṭṭhapāla}: 

“Here,\marginnote{28.3} Master \textsanskrit{Raṭṭhapāla}, sit on this elephant rug.” 

“Enough,\marginnote{28.4} great king, you sit on it. I’m sitting on my own seat.” 

So\marginnote{28.6} the king sat down on the seat spread out, and said: 

“Master\marginnote{29.1} \textsanskrit{Raṭṭhapāla}, there are these four kinds of decay. Because of these, some people shave off their hair and beard, dress in ocher robes, and go forth from the lay life to homelessness. What four? Decay due to old age, decay due to sickness, decay of wealth, and decay of relatives. 

And\marginnote{30.1} what is decay due to old age? It’s when someone is old, elderly, and senior, advanced in years, and has reached the final stage of life. They reflect: ‘I’m now old, elderly, and senior. I’m advanced in years and have reached the final stage of life. It’s not easy for me to acquire more wealth or to increase the wealth I’ve already acquired. Why don’t I shave off my hair and beard, dress in ocher robes, and go forth from the lay life to homelessness?’ So because of that decay due to old age they go forth. This is called decay due to old age. But Master \textsanskrit{Raṭṭhapāla} is now a youth, young, black-haired, blessed with youth, in the prime of life. You have no decay due to old age. So what did you know or see or hear that made you go forth? 

And\marginnote{31.1} what is decay due to sickness? It’s when someone is sick, suffering, gravely ill. They reflect: ‘I’m now sick, suffering, gravely ill. It’s not easy for me to acquire more wealth or to increase the wealth I’ve already acquired. Why don’t I go forth from the lay life to homelessness?’ So because of that decay due to sickness they go forth. This is called decay due to sickness. But Master \textsanskrit{Raṭṭhapāla} is now rarely ill or unwell. Your stomach digests well, being neither too hot nor too cold. You have no decay due to sickness. So what did you know or see or hear that made you go forth? 

And\marginnote{32.1} what is decay of wealth? It’s when someone is rich, affluent, and wealthy. But gradually their wealth dwindles away. They reflect: ‘I used to be rich, affluent, and wealthy. But gradually my wealth has dwindled away. It’s not easy for me to acquire more wealth or to increase the wealth I’ve already acquired. Why don’t I go forth from the lay life to homelessness?’ So because of that decay of wealth they go forth. This is called decay of wealth. But Master \textsanskrit{Raṭṭhapāla} is the son of the leading clan here in \textsanskrit{Thullakoṭṭhita}. You have no decay of wealth. So what did you know or see or hear that made you go forth? 

And\marginnote{33.1} what is decay of relatives? It’s when someone has many friends and colleagues, relatives and kin. But gradually their relatives dwindle away. They reflect: ‘I used to have many friends and colleagues, relatives and kin. But gradually they’ve dwindled away. It’s not easy for me to acquire more wealth or to increase the wealth I’ve already acquired. Why don’t I shave off my hair and beard, dress in ocher robes, and go forth from the lay life to homelessness?’ So because of that decay of relatives they go forth. This is called decay of relatives. But Master \textsanskrit{Raṭṭhapāla} has many friends and colleagues, relatives and kin right here in \textsanskrit{Thullakoṭṭhita}. You have no decay of relatives. So what did you know or see or hear that made you go forth? 

There\marginnote{34.1} are these four kinds of decay. Because of these, some people shave off their hair and beard, dress in ocher robes, and go forth from the lay life to homelessness. Master \textsanskrit{Raṭṭhapāla} has none of these. So what did you know or see or hear that made you go forth?” 

“Great\marginnote{35.1} king, the Blessed One who knows and sees, the perfected one, the fully awakened Buddha has taught these four summaries of the teaching for recitation. It was after knowing and seeing and hearing these that I went forth from the lay life to homelessness. 

What\marginnote{35.2} four? 

‘The\marginnote{36.1} world is unstable and swept away.’ This is the first summary. 

‘The\marginnote{36.3} world has no shelter and no savior.’ This is the second summary. 

‘The\marginnote{36.5} world has no owner—you must leave it all behind and pass on.’ This is the third summary. 

‘The\marginnote{36.7} world is wanting, insatiable, the slave of craving.’ This is the fourth summary. 

The\marginnote{37.1} Blessed One who knows and sees, the perfected one, the fully awakened Buddha taught these four summaries of the teaching. It was after knowing and seeing and hearing these that I went forth from the lay life to homelessness.” 

“‘The\marginnote{38.1} world is unstable and swept away.’ So Master \textsanskrit{Raṭṭhapāla} said. How should I see the meaning of this statement?” 

“What\marginnote{38.4} do you think, great king? When you were twenty or twenty-five years of age, were you proficient at riding elephants, horses, and chariots, and at archery and swordsmanship? Were you strong in thigh and arm, capable, and battle-hardened?” 

“I\marginnote{38.6} was, Master \textsanskrit{Raṭṭhapāla}. Sometimes it seems as if I had superpowers then. I don’t see anyone who could have equalled me in strength.” 

“What\marginnote{38.8} do you think, great king? These days are you just as strong in thigh and arm, capable, and battle-hardened?” 

“No,\marginnote{38.10} Master \textsanskrit{Raṭṭhapāla}. For now I am old, elderly, and senior, I’m advanced in years and have reached the final stage of life. I am eighty years old. Sometimes I intend to step in one place, but my foot goes somewhere else.” 

“This\marginnote{38.13} is what the Buddha was referring to when he said: ‘The world is unstable and swept away.’” 

“It’s\marginnote{38.16} incredible, Master \textsanskrit{Raṭṭhapāla}, it’s amazing, how well said this was by the Buddha. For the world is indeed unstable and swept away. 

In\marginnote{39.1} this royal court you can find divisions of elephants, cavalry, chariots, and infantry. They will serve to defend us from any threats. Yet you said: ‘The world has no shelter and no savior.’ How should I see the meaning of this statement?” 

“What\marginnote{39.5} do you think, great king? Do you have any chronic ailments?” 

“Yes,\marginnote{39.7} I do. Sometimes my friends and colleagues, relatives and family members surround me, thinking: ‘Now the king will die! Now the king will die!’” 

“What\marginnote{39.10} do you think, great king? Can you get your friends and colleagues, relatives and family members to help: ‘Please, my dear friends and colleagues, relatives and family members, all of you here share my pain so that I may feel less pain.’ Or must you alone feel that pain?” 

“I\marginnote{39.14} can’t get my friends to share my pain. Rather, I alone must feel it.” 

“This\marginnote{39.17} is what the Buddha was referring to when he said: ‘The world has no shelter and no savior.’” 

“It’s\marginnote{39.20} incredible, Master \textsanskrit{Raṭṭhapāla}, it’s amazing, how well said this was by the Buddha. For the world indeed has no shelter and no savior. 

In\marginnote{40.1} this royal court you can find abundant gold coin and bullion stored in dungeons and towers. Yet you said: ‘The world has no owner—you must leave it all behind and pass on.’ How should I see the meaning of this statement?” 

“What\marginnote{40.5} do you think, great king? These days you amuse yourself, supplied and provided with the five kinds of sensual stimulation. But is there any way to ensure that in the next life you will continue to amuse yourself in the same way, supplied and provided with the same five kinds of sensual stimulation? Or will others make use of this property, while you pass on according to your deeds?” 

“There’s\marginnote{40.8} no way to ensure that I will continue to amuse myself in the same way. Rather, others will take over this property, while I pass on according to my deeds.” 

“This\marginnote{40.11} is what the Buddha was referring to when he said: ‘The world has no owner—you must leave it all behind and pass on.’” 

“It’s\marginnote{40.14} incredible, Master \textsanskrit{Raṭṭhapāla}, it’s amazing, how well said this was by the Buddha. For the world indeed has no owner—you must leave it all behind and pass on. 

You\marginnote{41.1} also said this: ‘The world is wanting, insatiable, the slave of craving.’ How should I see the meaning of this statement?” 

“What\marginnote{41.4} do you think, great king? Do you dwell in the prosperous land of Kuru?” 

“Indeed\marginnote{41.6} I do.” 

“What\marginnote{41.7} do you think, great king? Suppose a trustworthy and reliable man were to come from the east. He’d approach you and say: ‘Please sir, you should know this. I come from the east. There I saw a large country that is successful and prosperous and full of people. They have many divisions of elephants, cavalry, chariots, and infantry. And there’s plenty of money and grain, plenty of gold coins and bullion, both worked and unworked, and plenty of women for the taking. With your current forces you can conquer it. Conquer it, great king!’ What would you do?” 

“I\marginnote{41.18} would conquer it and dwell there.” 

“What\marginnote{41.19} do you think, great king? Suppose a trustworthy and reliable man were to come from the west, north, south, or from over the ocean. He’d approach you and say the same thing. What would you do?” 

“I\marginnote{41.33} would conquer it and dwell there.” 

“This\marginnote{41.34} is what the Buddha was referring to when he said: ‘The world is wanting, insatiable, the slave of craving.’ And it was after knowing and seeing and hearing this that I went forth from the lay life to homelessness.” 

“It’s\marginnote{41.37} incredible, Master \textsanskrit{Raṭṭhapāla}, it’s amazing, how well said this was by the Buddha. For the world is indeed wanting, insatiable, the slave of craving.” 

This\marginnote{42.1} is what Venerable \textsanskrit{Raṭṭhapāla} said. Then he went on to say: 

\begin{verse}%
“I\marginnote{42.3} see rich people in the world who, \\
because of delusion, give not the wealth they’ve earned. \\
Greedily, they hoard their riches, \\
yearning for ever more sensual pleasures. 

A\marginnote{42.7} king who conquered the earth by force, \\
ruling the land from sea to sea, \\
unsatisfied with the near shore of the ocean, \\
would still yearn for the further shore. 

Not\marginnote{42.11} just the king, but others too, \\
reach death not rid of craving. \\
They leave the body still wanting, \\
for in this world sensual pleasures never satisfy. 

Relatives\marginnote{42.15} lament, their hair disheveled, \\
saying ‘Ah! Alas! They’re not immortal!’ \\
They take out the body wrapped in a shroud, \\
heap up a pyre, and burn it there. 

It’s\marginnote{42.19} poked with stakes while being burnt, \\
in just a single cloth, all wealth gone. \\
Relatives, friends, and companions \\
can’t help you when you’re dying. 

Heirs\marginnote{42.23} take your riches, \\
while beings fare on according to their deeds. \\
Riches don’t follow you when you die; \\
nor do children, wife, wealth, nor kingdom. 

Longevity\marginnote{42.27} isn’t gained by riches, \\
nor does wealth banish old age; \\
for the wise say this life is short, \\
it’s perishable and not eternal. 

The\marginnote{42.31} rich and the poor feel its touch; \\
the fool and the wise feel it too. \\
But the fool lies stricken by their own folly, \\
while the wise don’t tremble at the touch. 

Therefore\marginnote{42.35} wisdom’s much better than wealth, \\
since by wisdom you reach consummation in this life. \\
But if because of delusion you don’t reach consummation, \\
you’ll do evil deeds in life after life. 

One\marginnote{42.39} who enters a womb and the world beyond, \\
will transmigrate from one life to the next. \\
While someone of little wisdom, placing faith in them, \\
also enters a womb and the world beyond. 

As\marginnote{42.43} a bandit caught in the door \\
is punished for his own bad deeds; \\
so after departing, in the world beyond, \\
people are punished for their own bad deeds. 

Sensual\marginnote{42.47} pleasures are diverse, sweet, delightful; \\
appearing in disguise they disturb the mind. \\
Seeing danger in the many kinds of sensual stimulation, \\
I went forth, O King. 

As\marginnote{42.51} fruit falls from a tree, so people fall, \\
young and old, when the body breaks up. \\
Seeing this, too, I went forth, O King; \\
the ascetic life is guaranteed to be better.” 

%
\end{verse}

%
\section*{{\suttatitleacronym MN 83}{\suttatitletranslation About King Makhādeva }{\suttatitleroot Maghadevasutta}}
\addcontentsline{toc}{section}{\tocacronym{MN 83} \toctranslation{About King Makhādeva } \tocroot{Maghadevasutta}}
\markboth{About King Makhādeva }{Maghadevasutta}
\extramarks{MN 83}{MN 83}

\scevam{So\marginnote{1.1} I have heard. }At one time the Buddha was staying near \textsanskrit{Mithilā} in the \textsanskrit{Makhādeva} Mango Grove. Then the Buddha smiled at a certain spot. 

Then\marginnote{2.2} Venerable Ānanda thought, “What is the cause, what is the reason why the Buddha smiled? Realized Ones do not smile for no reason.” 

So\marginnote{2.5} Ānanda arranged his robe over one shoulder, raised his joined palms toward the Buddha, and said, “What is the cause, what is the reason why the Buddha smiled? Realized Ones do not smile for no reason.” 

“Once\marginnote{3.1} upon a time, Ānanda, right here in \textsanskrit{Mithilā} there was a just and principled king named \textsanskrit{Makhādeva}, a great king who stood by his duty. He justly treated brahmins and householders, and people of town and country. And he observed the sabbath on the fourteenth, fifteenth, and eighth of the fortnight. 

Then,\marginnote{4.1} after many years, many hundred years, many thousand years had passed, King \textsanskrit{Makhādeva} addressed his barber, ‘My dear barber, when you see grey hairs growing on my head, please tell me.’ 

‘Yes,\marginnote{4.3} Your Majesty,’ replied the barber. 

When\marginnote{4.4} many thousands of years had passed, the barber saw grey hairs growing on the king’s head. He said to the king, ‘The messengers of the gods have shown themselves to you. Grey hairs can be seen growing on your head.’ 

‘Well\marginnote{4.7} then, my dear barber, carefully pull them out with tweezers and place them in my cupped hands.’ 

‘Yes,\marginnote{4.8} Your Majesty,’ replied the barber, and he did as the king said. 

The\marginnote{4.9} king gave the barber a prize village, then summoned the crown prince and said, ‘Dear prince, the messengers of the gods have shown themselves to me. Grey hairs can be seen growing on my head. I have enjoyed human pleasures. Now it is time to seek heavenly pleasures. Come, dear prince, rule the realm. I shall shave off my hair and beard, dress in ocher robes, and go forth from the lay life to homelessness. 

For\marginnote{4.16} dear prince, you too will one day see grey hairs growing on your head. When this happens, after giving a prize village to the barber and carefully instructing the crown prince in kingship, you should shave off your hair and beard, dress in ocher robes, and go forth from the lay life to homelessness. Keep up this good practice that I have founded. Do not be my final man. Whatever generation is current when such good practice is broken, he is their final man. Therefore I say to you, “Keep up this good practice that I have founded. Do not be my final man.”’ 

And\marginnote{5.1} so, after giving a prize village to the barber and carefully instructing the crown prince in kingship, King \textsanskrit{Makhādeva} shaved off his hair and beard, dressed in ocher robes, and went forth from the lay life to homelessness here in this mango grove. He meditated spreading a heart full of love to one direction, and to the second, and to the third, and to the fourth. In the same way above, below, across, everywhere, all around, he spread a heart full of love to the whole world—abundant, expansive, limitless, free of enmity and ill will. He meditated spreading a heart full of compassion … rejoicing … equanimity to one direction, and to the second, and to the third, and to the fourth. In the same way above, below, across, everywhere, all around, he spread a heart full of equanimity to the whole world—abundant, expansive, limitless, free of enmity and ill will. 

For\marginnote{6.1} 84,000 years King \textsanskrit{Makhādeva} played games as a child, for 84,000 years he acted as viceroy, for 84,000 years he ruled the realm, and for 84,000 years he led the spiritual life after going forth here in this mango grove. Having developed these four \textsanskrit{Brahmā} meditations, when his body broke up, after death, he was reborn in a good place, a \textsanskrit{Brahmā} realm. 

Then,\marginnote{7.1} after many years, many hundred years, many thousand years had passed, King \textsanskrit{Makhādeva}’s son addressed his barber, ‘My dear barber, when you see grey hairs growing on my head, please tell me.’ And all unfolded as in the case of his father. And having developed the four \textsanskrit{Brahmā} meditations, when his body broke up, after death, \textsanskrit{Makhādeva}’s son was reborn in a good place, a \textsanskrit{Brahmā} realm. 

And\marginnote{10.1} a lineage of 84,000 kings, sons of sons of King \textsanskrit{Makhādeva}, shaved off their hair and beard, dressed in ocher robes, and went forth from the lay life to homelessness here in this mango grove. They meditated spreading a heart full of love … compassion … rejoicing … equanimity to one direction, and to the second, and to the third, and to the fourth. In the same way above, below, across, everywhere, all around, they spread a heart full of equanimity to the whole world—abundant, expansive, limitless, free of enmity and ill will. For 84,000 years they played games as a child, for 84,000 years they acted as viceroy, for 84,000 years they ruled the realm, and for 84,000 years they led the spiritual life after going forth here in this mango grove. And having developed the four \textsanskrit{Brahmā} meditations, when their bodies broke up, after death, they were reborn in a good place, a \textsanskrit{Brahmā} realm. 

Nimi\marginnote{12.1} was the last of those kings, a just and principled king, a great king who stood by his duty. He justly treated brahmins and householders, and people of town and country. And he observed the sabbath on the fourteenth, fifteenth, and eighth of the fortnight. 

Once\marginnote{13.1} upon a time, Ānanda, while the gods of the Thirty-Three were sitting together in the Hall of Justice, this discussion came up among them: ‘The people of Videha are so fortunate, so very fortunate to have Nimi as their king. He is a just and principled king, a great king who stands by his duty. He justly treats brahmins and householders, and people of town and country. And he observes the sabbath on the fourteenth, fifteenth, and eighth of the fortnight.’ 

Then\marginnote{13.6} Sakka, lord of gods, addressed the gods of the Thirty-Three, ‘Good sirs, would you like to see King Nimi?’ 

‘We\marginnote{13.8} would.’ 

Now\marginnote{13.9} at that time it was the fifteenth day sabbath, and King Nimi had bathed his head and was sitting upstairs in the royal longhouse to observe the sabbath. Then, as easily as a strong person would extend or contract their arm, Sakka vanished from the Thirty-Three gods and reappeared in front of King Nimi. He said to the king, ‘You’re fortunate, great king, so very fortunate. The gods of the Thirty-Three were sitting together in the Hall of Justice, where they spoke very highly of you. They would like to see you. I shall send a chariot harnessed with a thousand thoroughbreds for you, great king. Mount the heavenly chariot, great king! Do not waver.’ King Nimi consented in silence. 

Then,\marginnote{13.22} knowing that the king had consented, as easily as a strong person would extend or contract their arm, Sakka vanished from King Nimi and reappeared among the Thirty-Three gods. 

Then\marginnote{14.1} Sakka, lord of gods, addressed his charioteer \textsanskrit{Mātali}, ‘Come, dear \textsanskrit{Mātali}, harness the chariot with a thousand thoroughbreds. Then go to King Nimi and say, “Great king, this chariot has been sent for you by Sakka, lord of gods. Mount the heavenly chariot, great king! Do not waver.”’ 

‘Yes,\marginnote{14.5} lord,’ replied \textsanskrit{Mātali}. He did as Sakka asked, and said to the king, ‘Great king, this chariot has been sent for you by Sakka, lord of gods. Mount the heavenly chariot, great king! Do not waver. But which way should we go—the way of those who experience the result of bad deeds, or the way of those who experience the result of good deeds?’ 

‘Take\marginnote{14.9} me both ways, \textsanskrit{Mātali}.’ 

\textsanskrit{Mātali}\marginnote{15.1} brought King Nimi to the Hall of Justice. Sakka saw King Nimi coming off in the distance, and said to him: ‘Come, great king! Welcome, great king! The gods of the Thirty-Three who wanted to see you were sitting together in the Hall of Justice, where they spoke very highly of you. The gods of the Thirty-Three would like to see you. Enjoy divine glory among the gods!’ 

‘Enough,\marginnote{15.13} good sir. Send me back to Mithila right away. That way I shall justly treat brahmins and householders, and people of town and country. And I shall observe the sabbath on the fourteenth, fifteenth, and eighth of the fortnight.’ 

Then\marginnote{16.1} Sakka, lord of gods, addressed his charioteer \textsanskrit{Mātali}, ‘Come, dear \textsanskrit{Mātali}, harness the chariot with a thousand thoroughbreds and send King Nimi back to Mithila right away.’ 

‘Yes,\marginnote{16.3} lord,’ replied \textsanskrit{Mātali}, and did as Sakka asked. And there King Nimi justly treated his people, and observed the sabbath. 

Then,\marginnote{16.5} after many years, many hundred years, many thousand years had passed, King Nimi addressed his barber, ‘My dear barber, when you see grey hairs growing on my head, please tell me.’ And all unfolded as before. 

And\marginnote{17{-}19.1} having developed the four \textsanskrit{Brahmā} meditations, when his body broke up, after death, King Nimi was reborn in a good place, a \textsanskrit{Brahmā} realm. 

But\marginnote{20.1} King Nimi had a son named \textsanskrit{Kaḷārajanaka}. He didn’t go forth from the lay life to homelessness. He broke that good practice. He was their final man. 

Ānanda,\marginnote{21.1} you might think, ‘Surely King \textsanskrit{Makhādeva}, by whom that good practice was founded, must have been someone else at that time?’ But you should not see it like this. I myself was King \textsanskrit{Makhādeva} at that time. I was the one who founded that good practice, which was kept up by those who came after. 

But\marginnote{21.7} that good practice doesn’t lead to disillusionment, dispassion, cessation, peace, insight, awakening, and extinguishment. It only leads as far as rebirth in the \textsanskrit{Brahmā} realm. But now I have founded a good practice that does lead to disillusionment, dispassion, cessation, peace, insight, awakening, and extinguishment. 

And\marginnote{21.9} what is that good practice? It is simply this noble eightfold path, that is: right view, right thought, right speech, right action, right livelihood, right effort, right mindfulness, and right immersion. This is the good practice I have now founded that leads to disillusionment, dispassion, cessation, peace, insight, awakening, and extinguishment. 

Ānanda,\marginnote{21.13} I say to you: ‘You all should keep up this good practice that I have founded. Do not be my final men.’ Whatever generation is current when such good practice is broken, he is their final man. Ānanda, I say to you: ‘You all should keep up this good practice that I have founded. Do not be my final men.’” 

That\marginnote{21.18} is what the Buddha said. Satisfied, Venerable Ānanda was happy with what the Buddha said. 

%
\section*{{\suttatitleacronym MN 84}{\suttatitletranslation At Madhurā }{\suttatitleroot Madhurasutta}}
\addcontentsline{toc}{section}{\tocacronym{MN 84} \toctranslation{At Madhurā } \tocroot{Madhurasutta}}
\markboth{At Madhurā }{Madhurasutta}
\extramarks{MN 84}{MN 84}

\scevam{So\marginnote{1.1} I have heard. }At one time Venerable \textsanskrit{Mahākaccāna} was staying near \textsanskrit{Madhurā}, in Gunda’s Grove. 

King\marginnote{2.1} Avantiputta of \textsanskrit{Madhurā} heard, “It seems the ascetic \textsanskrit{Kaccāna} is staying near \textsanskrit{Madhurā}, in Gunda’s Grove. He has this good reputation: ‘He is astute, competent, clever, learned, a brilliant speaker, eloquent, mature, a perfected one.’ It’s good to see such perfected ones.” 

And\marginnote{3.1} then King Avantiputta had the finest carriages harnessed. He mounted a fine carriage and, along with other fine carriages, set out in full royal pomp from \textsanskrit{Madhurā} to see \textsanskrit{Mahākaccāna}. He went by carriage as far as the terrain allowed, then descended and approached \textsanskrit{Mahākaccāna} on foot. They exchanged greetings, and when the greetings and polite conversation were over, the king sat down to one side and said to \textsanskrit{Mahākaccāna}: 

“Master\marginnote{4.1} \textsanskrit{Kaccāna}, the brahmins say: ‘Only brahmins are the highest caste; other castes are inferior. Only brahmins are the light caste; other castes are dark. Only brahmins are purified, not others. Only brahmins are \textsanskrit{Brahmā}’s rightful sons, born of his mouth, born of \textsanskrit{Brahmā}, created by \textsanskrit{Brahmā}, heirs of \textsanskrit{Brahmā}.’ What does Master \textsanskrit{Kaccāna} have to say about this?” 

“Great\marginnote{5.1} king, that’s just propaganda. And here’s a way to understand that it’s just propaganda. 

What\marginnote{5.9} do you think, great king? Suppose an aristocrat prospers in money, grain, silver, or gold. Wouldn’t there be aristocrats, brahmins, merchants, and workers who would get up before him and go to bed after him, and be obliging, behaving nicely and speaking politely?” 

“There\marginnote{5.14} would, Master \textsanskrit{Kaccāna}.” 

“What\marginnote{5.18} do you think, great king? Suppose a brahmin … a merchant … a worker prospers in money, grain, silver, or gold. Wouldn’t there be workers, aristocrats, brahmins, and merchants who would get up before him and go to bed after him, and be obliging, behaving nicely and speaking politely?” 

“There\marginnote{5.39} would, Master \textsanskrit{Kaccāna}.” 

“What\marginnote{5.43} do you think, great king? If this is so, are the four castes equal or not? Or how do you see this?” 

“Certainly,\marginnote{5.46} Master \textsanskrit{Kaccāna}, in this case these four castes are equal. I can’t see any difference between them.” 

“And\marginnote{5.48} here’s another way to understand that the claims of the brahmins are just propaganda. 

What\marginnote{6.1} do you think, great king? Take an aristocrat who kills living creatures, steals, and commits sexual misconduct; uses speech that’s false, divisive, harsh, or nonsensical; and is covetous, malicious, and has wrong view. When their body breaks up, after death, would they be reborn in a place of loss, a bad place, the underworld, hell, or not? Or how do you see this?” 

“Such\marginnote{6.4} an aristocrat would be reborn in a bad place. That’s what I think, but I’ve also heard it from the perfected ones.” 

“Good,\marginnote{6.6} good, great king! It’s good that you think so, and it’s good that you’ve heard it from the perfected ones. What do you think, great king? Take a brahmin … a merchant … a worker who kills living creatures, steals, and commits sexual misconduct; uses speech that’s false, divisive, harsh, or nonsensical; and is covetous, malicious, and has wrong view. When their body breaks up, after death, would they be reborn in a place of loss, a bad place, the underworld, hell, or not? Or how do you see this?” 

“Such\marginnote{6.13} a brahmin, merchant, or worker would be reborn in a bad place. That’s what I think, but I’ve also heard it from the perfected ones.” 

“Good,\marginnote{6.15} good, great king! It’s good that you think so, and it’s good that you’ve heard it from the perfected ones. What do you think, great king? If this is so, are the four castes equal or not? Or how do you see this?” 

“Certainly,\marginnote{6.20} Master \textsanskrit{Kaccāna}, in this case these four castes are equal. I can’t see any difference between them.” 

“And\marginnote{6.22} here’s another way to understand that the claims of the brahmins are just propaganda. 

What\marginnote{7.1} do you think, great king? Take an aristocrat who doesn’t kill living creatures, steal, or commit sexual misconduct. They don’t use speech that’s false, divisive, harsh, or nonsensical. And they’re contented, kind-hearted, with right view. When their body breaks up, after death, would they be reborn in a good place, a heavenly realm, or not? Or how do you see this?” 

“Such\marginnote{7.4} an aristocrat would be reborn in a good place. That’s what I think, but I’ve also heard it from the perfected ones.” 

“Good,\marginnote{7.6} good, great king! It’s good that you think so, and it’s good that you’ve heard it from the perfected ones. What do you think, great king? Take a brahmin, merchant, or worker who doesn’t kill living creatures, steal, or commit sexual misconduct. They don’t use speech that’s false, divisive, harsh, or nonsensical. And they’re contented, kind-hearted, with right view. When their body breaks up, after death, would they be reborn in a good place, a heavenly realm, or not? Or how do you see this?” 

“Such\marginnote{7.11} a brahmin, merchant, or worker would be reborn in a good place. That’s what I think, but I’ve also heard it from the perfected ones.” 

“Good,\marginnote{7.13} good, great king! It’s good that you think so, and it’s good that you’ve heard it from the perfected ones. What do you think, great king? If this is so, are the four castes equal or not? Or how do you see this?” 

“Certainly,\marginnote{7.18} Master \textsanskrit{Kaccāna}, in this case these four castes are equal. I can’t see any difference between them.” 

“And\marginnote{7.20} here’s another way to understand that the claims of the brahmins are just propaganda. 

What\marginnote{8.1} do you think, great king? Take an aristocrat who breaks into houses, plunders wealth, steals from isolated buildings, commits highway robbery, and commits adultery. Suppose your men arrest him and present him to you, saying: ‘Your Majesty, this man is a bandit, a criminal. Punish him as you will.’ What would you do to him?” 

“I\marginnote{8.6} would have him executed, fined, or banished, or dealt with as befits the crime. Why is that? Because he’s lost his former status as an aristocrat, and is just reckoned as a bandit.” 

“What\marginnote{8.9} do you think, great king? Take a brahmin, merchant, or worker who breaks into houses, plunders wealth, steals from isolated buildings, commits highway robbery, and commits adultery. Suppose your men arrest him and present him to you, saying: ‘Your Majesty, this man is a bandit, a criminal. Punish him as you will.’ What would you do to him?” 

“I\marginnote{8.14} would have him executed, fined, or banished, or dealt with as befits the crime. Why is that? Because he’s lost his former status as a brahmin, merchant, or worker, and is just reckoned as a bandit.” 

“What\marginnote{8.17} do you think, great king? If this is so, are the four castes equal or not? Or how do you see this?” 

“Certainly,\marginnote{8.20} Master \textsanskrit{Kaccāna}, in this case these four castes are equal. I can’t see any difference between them.” 

“And\marginnote{8.22} here’s another way to understand that the claims of the brahmins are just propaganda. 

What\marginnote{9.1} do you think, great king? Take an aristocrat who shaves off their hair and beard, dresses in ocher robes, and goes forth from the lay life to homelessness. They refrain from killing living creatures, stealing, and lying. They abstain from eating at night, eat in one part of the day, and are celibate, ethical, and of good character. How would you treat them?” 

“I\marginnote{9.4} would bow to them, rise in their presence, or offer them a seat. I’d invite them to accept robes, almsfood, lodgings, and medicines and supplies for the sick. And I’d organize their lawful guarding and protection. Why is that? Because they’ve lost their former status as an aristocrat, and are just reckoned as an ascetic.” 

“What\marginnote{9.7} do you think, great king? Take a brahmin, merchant, or worker who shaves off their hair and beard, dresses in ocher robes, and goes forth from the lay life to homelessness. They refrain from killing living creatures, stealing, and lying. They abstain from eating at night, eat in one part of the day, and are celibate, ethical, and of good character. How would you treat them?” 

“I\marginnote{9.10} would bow to them, rise in their presence, or offer them a seat. I’d invite them to accept robes, almsfood, lodgings, and medicines and supplies for the sick. And I’d organize their lawful guarding and protection. Why is that? Because they’ve lost their former status as a brahmin, merchant, or worker, and are just reckoned as an ascetic.” 

“What\marginnote{9.13} do you think, great king? If this is so, are the four castes equal or not? Or how do you see this?” 

“Certainly,\marginnote{9.16} Master \textsanskrit{Kaccāna}, in this case these four castes are equal. I can’t see any difference between them.” 

“This\marginnote{9.18} is another way to understand that this is just propaganda: ‘Only brahmins are the highest caste; other castes are inferior. Only brahmins are the light caste; other castes are dark. Only brahmins are purified, not others. Only brahmins are \textsanskrit{Brahmā}’s rightful sons, born of his mouth, born of \textsanskrit{Brahmā}, created by \textsanskrit{Brahmā}, heirs of \textsanskrit{Brahmā}.’” 

When\marginnote{10.1} he had spoken, King Avantiputta of \textsanskrit{Madhurā} said to \textsanskrit{Mahākaccāna}, “Excellent, Master \textsanskrit{Kaccāna}! Excellent! As if he were righting the overturned, or revealing the hidden, or pointing out the path to the lost, or lighting a lamp in the dark so people with good eyes can see what’s there, Master \textsanskrit{Kaccāna} has made the teaching clear in many ways. I go for refuge to Master \textsanskrit{Kaccāna}, to the teaching, and to the mendicant \textsanskrit{Saṅgha}. From this day forth, may Master \textsanskrit{Kaccāna} remember me as a lay follower who has gone for refuge for life.” 

“Great\marginnote{10.6} king, don’t go for refuge to me. You should go for refuge to that same Blessed One to whom I have gone for refuge.” 

“But\marginnote{10.8} where is that Blessed One at present, the perfected one, the fully awakened Buddha?” 

“Great\marginnote{10.9} king, the Buddha has already become fully extinguished.” 

“Master\marginnote{11.1} \textsanskrit{Kaccāna}, if I heard that the Buddha was within ten leagues, or twenty, or even up to a hundred leagues away, I’d go a hundred leagues to see him. But since the Buddha has become fully extinguished, I go for refuge to that fully extinguished Buddha, to the teaching, and to the mendicant \textsanskrit{Saṅgha}. From this day forth, may Master \textsanskrit{Kaccāna} remember me as a lay follower who has gone for refuge for life.” 

%
\section*{{\suttatitleacronym MN 85}{\suttatitletranslation With Prince Bodhi }{\suttatitleroot Bodhirājakumārasutta}}
\addcontentsline{toc}{section}{\tocacronym{MN 85} \toctranslation{With Prince Bodhi } \tocroot{Bodhirājakumārasutta}}
\markboth{With Prince Bodhi }{Bodhirājakumārasutta}
\extramarks{MN 85}{MN 85}

\scevam{So\marginnote{1.1} I have heard. }At one time the Buddha was staying in the land of the Bhaggas on Crocodile Hill, in the deer park at \textsanskrit{Bhesakaḷā}’s Wood. 

Now\marginnote{2.1} at that time a new stilt longhouse named Pink Lotus had recently been constructed for Prince Bodhi. It had not yet been occupied by an ascetic or brahmin or any person at all. 

Then\marginnote{3.1} Prince Bodhi addressed the brahmin student \textsanskrit{Sañjikāputta}, “Please, dear \textsanskrit{Sañjikāputta}, go to the Buddha, and in my name bow with your head to his feet. Ask him if he is healthy and well, nimble, strong, and living comfortably. And then ask him whether he might accept tomorrow’s meal from me together with the mendicant \textsanskrit{Saṅgha}.” 

“Yes,\marginnote{4.1} sir,” \textsanskrit{Sañjikāputta} replied. He did as Prince Bodhi asked, and the Buddha consented in silence. 

Then,\marginnote{4.7} knowing that the Buddha had consented, \textsanskrit{Sañjikāputta} got up from his seat, went to Prince Bodhi, and said, “I gave the ascetic Gotama your message, and he accepted.” 

And\marginnote{5.1} when the night had passed Prince Bodhi had a variety of delicious foods prepared in his own home. He also had the Pink Lotus longhouse spread with white cloth down to the last step of the staircase. Then he said to \textsanskrit{Sañjikāputta}, “Please, dear \textsanskrit{Sañjikāputta}, go to the Buddha, and announce the time, saying, ‘Sir, it’s time. The meal is ready.’” 

“Yes,\marginnote{5.4} sir,” \textsanskrit{Sañjikāputta} replied, and he did as he was asked. 

Then\marginnote{6.1} the Buddha robed up in the morning and, taking his bowl and robe, went to Prince Bodhi’s home. 

Now\marginnote{7.1} at that time Prince Bodhi was standing outside the gates waiting for the Buddha. Seeing the Buddha coming off in the distance, he went out to greet him. After bowing and inviting the Buddha to go first, he approached the Pink Lotus longhouse. But the Buddha stopped by the last step of the staircase. 

Then\marginnote{7.5} Prince Bodhi said to him, “Sir, let the Blessed One ascend on the cloth! Let the Holy One ascend on the cloth! It will be for my lasting welfare and happiness.” But when he said this, the Buddha kept silent. 

For\marginnote{7.9} a second time … and a third time, Prince Bodhi said to him, “Sir, let the Blessed One ascend on the cloth! Let the Holy One ascend on the cloth! It will be for my lasting welfare and happiness.” 

Then\marginnote{7.13} the Buddha glanced at Venerable Ānanda. So Ānanda said to Prince Bodhi, “Fold up the cloth, Prince. The Buddha will not step upon white cloth. The Realized One has compassion for future generations.” 

So\marginnote{8.1} Prince Bodhi had the cloth folded up and the seats spread out upstairs in the longhouse. Then the Buddha ascended the longhouse and sat on the seats spread out together with the \textsanskrit{Saṅgha} of mendicants. 

Then\marginnote{9.1} Prince Bodhi served and satisfied the mendicant \textsanskrit{Saṅgha} headed by the Buddha with his own hands with a variety of delicious foods. When the Buddha had eaten and washed his hand and bowl, Prince Bodhi took a low seat, sat to one side, and said to him, “Sir, this is what I think: ‘Pleasure is not gained through pleasure; pleasure is gained through pain.’” 

“Prince,\marginnote{10.1} before my awakening—when I was still unawakened but intent on awakening—I too thought: ‘Pleasure is not gained through pleasure; pleasure is gained through pain.’ 

Some\marginnote{11.1} time later, while still black-haired, blessed with youth, in the prime of life—though my mother and father wished otherwise, weeping with tearful faces—I shaved off my hair and beard, dressed in ocher robes, and went forth from the lay life to homelessness. Once I had gone forth I set out to discover what is skillful, seeking the supreme state of sublime peace. I approached \textsanskrit{Āḷāra} \textsanskrit{Kālāma} and said to him, ‘Reverend \textsanskrit{Kālāma}, I wish to lead the spiritual life in this teaching and training.’ 

\textsanskrit{Āḷāra}\marginnote{11.4} \textsanskrit{Kālāma} replied, ‘Stay, venerable. This teaching is such that a sensible person can soon realize their own tradition with their own insight and live having achieved it.’ 

I\marginnote{11.7} quickly memorized that teaching. So far as lip-recital and oral recitation were concerned, I spoke with knowledge and the authority of the elders. I claimed to know and see, and so did others. Then it occurred to me, ‘It is not solely by mere faith that \textsanskrit{Āḷāra} \textsanskrit{Kālāma} declares: “I realize this teaching with my own insight, and live having achieved it.” Surely he meditates knowing and seeing this teaching.’ 

So\marginnote{12.1} I approached \textsanskrit{Āḷāra} \textsanskrit{Kālāma} and said to him, ‘Reverend \textsanskrit{Kālāma}, to what extent do you say you’ve realized this teaching with your own insight?’ When I said this, he declared the dimension of nothingness. 

Then\marginnote{12.4} it occurred to me, ‘It’s not just \textsanskrit{Āḷāra} \textsanskrit{Kālāma} who has faith, energy, mindfulness, immersion, and wisdom; I too have these things. Why don’t I make an effort to realize the same teaching that \textsanskrit{Āḷāra} \textsanskrit{Kālāma} says he has realized with his own insight?’ I quickly realized that teaching with my own insight, and lived having achieved it. 

So\marginnote{12.12} I approached \textsanskrit{Āḷāra} \textsanskrit{Kālāma} and said to him, ‘Reverend \textsanskrit{Kālāma}, have you realized this teaching with your own insight up to this point, and declare having achieved it?’ 

‘I\marginnote{12.14} have, reverend.’ 

‘I\marginnote{12.15} too have realized this teaching with my own insight up to this point, and live having achieved it.’ 

‘We\marginnote{12.16} are fortunate, reverend, so very fortunate to see a venerable such as yourself as one of our spiritual companions! So the teaching that I’ve realized with my own insight, and declare having achieved it, you’ve realized with your own insight, and live having achieved it. The teaching that you’ve realized with your own insight, and live having achieved it, I’ve realized with my own insight, and declare having achieved it. So the teaching that I know, you know, and the teaching you know, I know. I am like you and you are like me. Come now, reverend! We should both lead this community together.’ And that is how my teacher \textsanskrit{Āḷāra} \textsanskrit{Kālāma} placed me, his student, on the same position as him, and honored me with lofty praise. 

Then\marginnote{12.24} it occurred to me, ‘This teaching doesn’t lead to disillusionment, dispassion, cessation, peace, insight, awakening, and extinguishment. It only leads as far as rebirth in the dimension of nothingness.’ Realizing that this teaching was inadequate, I left disappointed. 

I\marginnote{12.27} set out to discover what is skillful, seeking the supreme state of sublime peace. I approached Uddaka, son of \textsanskrit{Rāma}, and said to him, ‘Reverend, I wish to lead the spiritual life in this teaching and training.’ 

Uddaka\marginnote{12.29} replied, ‘Stay, venerable. This teaching is such that a sensible person can soon realize their own tradition with their own insight and live having achieved it.’ 

I\marginnote{12.32} quickly memorized that teaching. So far as lip-recital and oral recitation were concerned, I spoke with knowledge and the authority of the elders. I claimed to know and see, and so did others. 

Then\marginnote{12.34} it occurred to me, ‘It is not solely by mere faith that \textsanskrit{Rāma} declared: “I realize this teaching with my own insight, and live having achieved it.” Surely he meditated knowing and seeing this teaching.’ 

So\marginnote{13.1} I approached Uddaka, son of \textsanskrit{Rāma}, and said to him, ‘Reverend, to what extent did \textsanskrit{Rāma} say he’d realized this teaching with his own insight?’ When I said this, Uddaka, son of \textsanskrit{Rāma}, declared the dimension of neither perception nor non-perception. 

Then\marginnote{13.4} it occurred to me, ‘It’s not just \textsanskrit{Rāma} who had faith, energy, mindfulness, immersion, and wisdom; I too have these things. Why don’t I make an effort to realize the same teaching that \textsanskrit{Rāma} said he had realized with his own insight?’ I quickly realized that teaching with my own insight, and lived having achieved it. 

So\marginnote{13.12} I approached Uddaka, son of \textsanskrit{Rāma}, and said to him, ‘Reverend, had \textsanskrit{Rāma} realized this teaching with his own insight up to this point, and declared having achieved it?’ 

‘He\marginnote{13.14} had, reverend.’ 

‘I\marginnote{13.15} too have realized this teaching with my own insight up to this point, and live having achieved it.’ 

‘We\marginnote{13.16} are fortunate, reverend, so very fortunate to see a venerable such as yourself as one of our spiritual companions! So the teaching that \textsanskrit{Rāma} had realized with his own insight, and declared having achieved it, you've realized with your own insight, and live having achieved it. The teaching that you’ve realized with your own insight, and live having achieved it, \textsanskrit{Rāma} had realized with his own insight, and declared having achieved it. So the teaching that \textsanskrit{Rāma} directly knew, you know, and the teaching you know, \textsanskrit{Rāma} directly knew. \textsanskrit{Rāma} was like you and you are like \textsanskrit{Rāma}. Come now, reverend! You should lead this community.’ And that is how my spiritual companion Uddaka, son of \textsanskrit{Rāma}, placed me in the position of a teacher, and honored me with lofty praise. 

Then\marginnote{13.24} it occurred to me, ‘This teaching doesn’t lead to disillusionment, dispassion, cessation, peace, insight, awakening, and extinguishment. It only leads as far as rebirth in the dimension of neither perception nor non-perception.’ Realizing that this teaching was inadequate, I left disappointed. 

I\marginnote{14.1} set out to discover what is skillful, seeking the supreme state of sublime peace. Traveling stage by stage in the Magadhan lands, I arrived at Senanigama near \textsanskrit{Uruvelā}. There I saw a delightful park, a lovely grove with a flowing river that was clean and charming, with smooth banks. And nearby was a village for alms. 

Then\marginnote{14.3} it occurred to me, ‘This park is truly delightful, a lovely grove with a flowing river that’s clean and charming, with smooth banks. And nearby there’s a village for alms. This is good enough for a gentleman who wishes to put forth effort in meditation.’ So I sat down right there, thinking, ‘This is good enough for meditation.’ 

And\marginnote{15.1} then these three examples, which were neither supernaturally inspired, nor learned before in the past, occurred to me. 

Suppose\marginnote{16.1} there was a green, sappy log, and it was lying in water. Then a person comes along with a drill-stick, thinking to light a fire and produce heat. What do you think, Prince? By drilling the stick against that green, sappy log lying in water, could they light a fire and produce heat?” 

“No,\marginnote{16.6} sir. Why is that? Because it’s a green, sappy log, and it’s lying in the water. That person will eventually get weary and frustrated.” 

“In\marginnote{17.1} the same way, there are ascetics and brahmins who don’t live withdrawn in body and mind from sensual pleasures. They haven’t internally given up or stilled desire, affection, infatuation, thirst, and passion for sensual pleasures. Regardless of whether or not they suffer painful, sharp, severe, acute feelings because of their efforts, they are incapable of knowledge and vision, of supreme awakening. This was the first example that occurred to me. 

Then\marginnote{18.1} a second example occurred to me. 

Suppose\marginnote{18.2} there was a green, sappy log, and it was lying on dry land far from the water. Then a person comes along with a drill-stick, thinking to light a fire and produce heat. What do you think, Prince? By drilling the stick against that green, sappy log on dry land far from water, could they light a fire and produce heat?” 

“No,\marginnote{18.7} sir. Why is that? Because it’s still a green, sappy log, despite the fact that it’s lying on dry land far from water. That person will eventually get weary and frustrated.” 

“In\marginnote{18.11} the same way, there are ascetics and brahmins who live withdrawn in body and mind from sensual pleasures. But they haven’t internally given up or stilled desire, affection, infatuation, thirst, and passion for sensual pleasures. Regardless of whether or not they suffer painful, sharp, severe, acute feelings because of their efforts, they are incapable of knowledge and vision, of supreme awakening. This was the second example that occurred to me. 

Then\marginnote{19.1} a third example occurred to me. 

Suppose\marginnote{19.2} there was a dried up, withered log, and it was lying on dry land far from the water. Then a person comes along with a drill-stick, thinking to light a fire and produce heat. What do you think, Prince? By drilling the stick against that dried up, withered log on dry land far from water, could they light a fire and produce heat?” 

“Yes,\marginnote{19.7} sir. Why is that? Because it’s a dried up, withered log, and it’s lying on dry land far from water.” 

“In\marginnote{19.10} the same way, there are ascetics and brahmins who live withdrawn in body and mind from sensual pleasures. And they have internally given up and stilled desire, affection, infatuation, thirst, and passion for sensual pleasures. Regardless of whether or not they suffer painful, sharp, severe, acute feelings because of their efforts, they are capable of knowledge and vision, of supreme awakening. This was the third example that occurred to me. These are the three examples, which were neither supernaturally inspired, nor learned before in the past, that occurred to me. 

Then\marginnote{20.1} it occurred to me, ‘Why don’t I, with teeth clenched and tongue pressed against the roof of my mouth, squeeze, squash, and torture mind with mind.’ So that’s what I did, until sweat ran from my armpits. It was like when a strong man grabs a weaker man by the head or throat or shoulder and squeezes, squashes, and tortures them. In the same way, with teeth clenched and tongue pressed against the roof of my mouth, I squeezed, squashed, and tortured mind with mind until sweat ran from my armpits. My energy was roused up and unflagging, and my mindfulness was established and lucid, but my body was disturbed, not tranquil, because I’d pushed too hard with that painful striving. 

Then\marginnote{21.1} it occurred to me, ‘Why don’t I practice the breathless absorption?’ So I cut off my breathing through my mouth and nose. But then winds came out my ears making a loud noise, like the puffing of a blacksmith’s bellows. My energy was roused up and unflagging, and my mindfulness was established and lucid, but my body was disturbed, not tranquil, because I’d pushed too hard with that painful striving. 

Then\marginnote{22.1} it occurred to me, ‘Why don’t I keep practicing the breathless absorption?’ So I cut off my breathing through my mouth and nose and ears. But then strong winds ground my head, like a strong man was drilling into my head with a sharp point. My energy was roused up and unflagging, and my mindfulness was established and lucid, but my body was disturbed, not tranquil, because I’d pushed too hard with that painful striving. 

Then\marginnote{23.1} it occurred to me, ‘Why don’t I keep practicing the breathless absorption?’ So I cut off my breathing through my mouth and nose and ears. But then I got a severe headache, like a strong man was tightening a tough leather strap around my head. My energy was roused up and unflagging, and my mindfulness was established and lucid, but my body was disturbed, not tranquil, because I’d pushed too hard with that painful striving. 

Then\marginnote{24.1} it occurred to me, ‘Why don’t I keep practicing the breathless absorption?’ So I cut off my breathing through my mouth and nose and ears. But then strong winds carved up my belly, like a deft butcher or their apprentice was slicing my belly open with a meat cleaver. My energy was roused up and unflagging, and my mindfulness was established and lucid, but my body was disturbed, not tranquil, because I’d pushed too hard with that painful striving. 

Then\marginnote{25.1} it occurred to me, ‘Why don’t I keep practicing the breathless absorption?’ So I cut off my breathing through my mouth and nose and ears. But then there was an intense burning in my body, like two strong men grabbing a weaker man by the arms to burn and scorch him on a pit of glowing coals. My energy was roused up and unflagging, and my mindfulness was established and lucid, but my body was disturbed, not tranquil, because I’d pushed too hard with that painful striving. 

Then\marginnote{26.1} some deities saw me and said, ‘The ascetic Gotama is dead.’ Others said, ‘He’s not dead, but he’s dying.’ Others said, ‘He’s not dead or dying. The ascetic Gotama is a perfected one, for that is how the perfected ones live.’ 

Then\marginnote{27.1} it occurred to me, ‘Why don’t I practice completely cutting off food?’ But deities came to me and said, ‘Good sir, don’t practice totally cutting off food. If you do, we’ll infuse divine nectar into your pores and you will live on that.’ Then it occurred to me, ‘If I claim to be completely fasting while these deities are infusing divine nectar in my pores, that would be a lie on my part.’ So I dismissed those deities, saying, ‘There’s no need.’ 

Then\marginnote{28.1} it occurred to me, ‘Why don’t I just take a little bit of food each time, a cup of broth made from mung beans, lentils, chickpeas, or green gram.’ So that’s what I did, until my body became extremely emaciated. Due to eating so little, my limbs became like the joints of an eighty-year-old or a corpse, my bottom became like a camel’s hoof, my vertebrae stuck out like beads on a string, and my ribs were as gaunt as the broken-down rafters on an old barn. Due to eating so little, the gleam of my eyes sank deep in their sockets, like the gleam of water sunk deep down a well. Due to eating so little, my scalp shriveled and withered like a green bitter-gourd in the wind and sun. Due to eating so little, the skin of my belly stuck to my backbone, so that when I tried to rub the skin of my belly I grabbed my backbone, and when I tried to rub my backbone I rubbed the skin of my belly. Due to eating so little, when I tried to urinate or defecate I fell face down right there. Due to eating so little, when I tried to relieve my body by rubbing my limbs with my hands, the hair, rotted at its roots, fell out. 

Then\marginnote{29.1} some people saw me and said, ‘The ascetic Gotama is black.’ Some said, ‘He’s not black, he’s brown.’ Some said, ‘He’s neither black nor brown. The ascetic Gotama has tawny skin.’ That’s how far the pure, bright complexion of my skin had been ruined by taking so little food. 

Then\marginnote{30.1} it occurred to me, ‘Whatever ascetics and brahmins have experienced painful, sharp, severe, acute feelings due to overexertion—whether in the past, future, or present—this is as far as it goes, no-one has done more than this. But I have not achieved any superhuman distinction in knowledge and vision worthy of the noble ones by this severe, gruelling work. Could there be another path to awakening?’ 

Then\marginnote{31.1} it occurred to me, ‘I recall sitting in the cool shade of the rose-apple tree while my father the Sakyan was off working. Quite secluded from sensual pleasures, secluded from unskillful qualities, I entered and remained in the first absorption, which has the rapture and bliss born of seclusion, while placing the mind and keeping it connected. Could that be the path to awakening?’ Stemming from that memory came the realization: ‘\emph{That} is the path to awakening!’ 

Then\marginnote{32.1} it occurred to me, ‘Why am I afraid of that pleasure, for it has nothing to do with sensual pleasures or unskillful qualities?’ Then it occurred to me, ‘I’m not afraid of that pleasure, for it has nothing to do with sensual pleasures or unskillful qualities.’ 

Then\marginnote{33.1} it occurred to me, ‘I can’t achieve that pleasure with a body so excessively emaciated. Why don’t I eat some solid food, some rice and porridge?’ So I ate some solid food. Now at that time the five mendicants were attending on me, thinking, ‘The ascetic Gotama will tell us of any truth that he realizes.’ But when I ate some solid food, they left disappointed in me, saying, ‘The ascetic Gotama has become indulgent; he has strayed from the struggle and returned to indulgence.’ 

After\marginnote{34{-}37.1} eating solid food and gathering my strength, quite secluded from sensual pleasures, secluded from unskillful qualities, I entered and remained in the first absorption … second absorption … third absorption … fourth absorption. When my mind had immersed in \textsanskrit{samādhi} like this—purified, bright, flawless, rid of corruptions, pliable, workable, steady, and imperturbable—I extended it toward recollection of past lives. I recollected many past lives. That is: one, two, three, four, five, ten, twenty, thirty, forty, fifty, a hundred, a thousand, a hundred thousand rebirths; many eons of the world contracting, many eons of the world expanding, many eons of the world contracting and expanding. And so I recollected my many kinds of past lives, with features and details. This was the first knowledge, which I achieved in the first watch of the night. Ignorance was destroyed and knowledge arose; darkness was destroyed and light arose, as happens for a meditator who is diligent, keen, and resolute. 

When\marginnote{38.1} my mind had immersed in \textsanskrit{samādhi} like this—purified, bright, flawless, rid of corruptions, pliable, workable, steady, and imperturbable—I extended it toward knowledge of the death and rebirth of sentient beings. With clairvoyance that is purified and superhuman, I saw sentient beings passing away and being reborn—inferior and superior, beautiful and ugly, in a good place or a bad place. I understood how sentient beings are reborn according to their deeds. 

This\marginnote{39.1} was the second knowledge, which I achieved in the middle watch of the night. Ignorance was destroyed and knowledge arose; darkness was destroyed and light arose, as happens for a meditator who is diligent, keen, and resolute. 

When\marginnote{40.1} my mind had immersed in \textsanskrit{samādhi} like this—purified, bright, flawless, rid of corruptions, pliable, workable, steady, and imperturbable—I extended it toward knowledge of the ending of defilements. I truly understood: ‘This is suffering’ … ‘This is the origin of suffering’ … ‘This is the cessation of suffering’ … ‘This is the practice that leads to the cessation of suffering’. I truly understood: ‘These are defilements’ … ‘This is the origin of defilements’ … ‘This is the cessation of defilements’ … ‘This is the practice that leads to the cessation of defilements’. 

Knowing\marginnote{41.1} and seeing like this, my mind was freed from the defilements of sensuality, desire to be reborn, and ignorance. When it was freed, I knew it was freed. 

I\marginnote{41.3} understood: ‘Rebirth is ended; the spiritual journey has been completed; what had to be done has been done; there is no return to any state of existence.’ 

This\marginnote{42.1} was the third knowledge, which I achieved in the last watch of the night. Ignorance was destroyed and knowledge arose; darkness was destroyed and light arose, as happens for a meditator who is diligent, keen, and resolute. 

Then\marginnote{43.1} it occurred to me, ‘This principle I have discovered is deep, hard to see, hard to understand, peaceful, sublime, beyond the scope of logic, subtle, comprehensible to the astute. But people like attachment, they love it and enjoy it. It’s hard for them to see this thing; that is, specific conditionality, dependent origination. It’s also hard for them to see this thing; that is, the stilling of all activities, the letting go of all attachments, the ending of craving, fading away, cessation, extinguishment. And if I were to teach the Dhamma, others might not understand me, which would be wearying and troublesome for me.’ And then these verses, which were neither supernaturally inspired, nor learned before in the past, occurred to me: 

\begin{verse}%
‘I’ve\marginnote{43.8} struggled hard to realize this, \\
enough with trying to explain it! \\
This teaching is not easily understood \\
by those mired in greed and hate. 

Those\marginnote{43.12} besotted by greed can’t see \\
what’s subtle, going against the stream, \\
deep, hard to see, and very fine, \\
for they’re shrouded in a mass of darkness.’ 

%
\end{verse}

And\marginnote{43.16} as I reflected like this, my mind inclined to remaining passive, not to teaching the Dhamma. 

Then\marginnote{44.1} \textsanskrit{Brahmā} Sahampati, knowing what I was thinking, thought, ‘Oh my goodness! The world will be lost, the world will perish! For the mind of the Realized One, the perfected one, the fully awakened Buddha, inclines to remaining passive, not to teaching the Dhamma.’ 

Then\marginnote{44.3} \textsanskrit{Brahmā} Sahampati, as easily as a strong person would extend or contract their arm, vanished from the \textsanskrit{Brahmā} realm and reappeared in front of me. He arranged his robe over one shoulder, raised his joined palms toward me, and said, ‘Sir, let the Blessed One teach the Dhamma! Let the Holy One teach the Dhamma! There are beings with little dust in their eyes. They’re in decline because they haven’t heard the teaching. There will be those who understand the teaching!’ 

That’s\marginnote{44.8} what \textsanskrit{Brahmā} Sahampati said. Then he went on to say: 

\begin{verse}%
‘Among\marginnote{44.10} the Magadhans there appeared in the past \\
an impure teaching thought up by those still stained. \\
Fling open the door to the deathless! \\
Let them hear the teaching the immaculate one discovered. 

Standing\marginnote{44.14} high on a rocky mountain, \\
you can see the people all around. \\
In just the same way, all-seer, wise one, \\
having ascended the Temple of Truth, 

rid\marginnote{44.18} of sorrow, look upon the people \\
swamped with sorrow, oppressed by rebirth and old age. \\
Rise, hero! Victor in battle, leader of the caravan, \\
wander the world without obligation. \\
Let the Blessed One teach the Dhamma! \\
There will be those who understand!’ 

%
\end{verse}

Then,\marginnote{45.1} understanding \textsanskrit{Brahmā}’s invitation, I surveyed the world with the eye of a Buddha, because of my compassion for sentient beings. And I saw sentient beings with little dust in their eyes, and some with much dust in their eyes; with keen faculties and with weak faculties, with good qualities and with bad qualities, easy to teach and hard to teach. And some of them lived seeing the danger in the fault to do with the next world, while others did not. It’s like a pool with blue water lilies, or pink or white lotuses. Some of them sprout and grow in the water without rising above it, thriving underwater. Some of them sprout and grow in the water reaching the water’s surface. And some of them sprout and grow in the water but rise up above the water and stand with no water clinging to them. Then I replied in verse to \textsanskrit{Brahmā} Sahampati: 

\begin{verse}%
‘Flung\marginnote{45.6} open are the doors to the deathless! \\
Let those with ears to hear commit to faith. \\
Thinking it would be troublesome, \textsanskrit{Brahmā}, I did not teach \\
the sophisticated, sublime Dhamma among humans.’ 

%
\end{verse}

Then\marginnote{45.10} \textsanskrit{Brahmā} Sahampati, knowing that his request for me to teach the Dhamma had been granted, bowed and respectfully circled me, keeping me on his right, before vanishing right there. 

Then\marginnote{46.1} it occurred to me, ‘Who should I teach first of all? Who will quickly understand the teaching?’ Then it occurred to me, ‘That \textsanskrit{Āḷāra} \textsanskrit{Kālāma} is astute, competent, clever, and has long had little dust in his eyes. Why don’t I teach him first of all? He’ll quickly understand the teaching.’ But a deity came to me and said, ‘Sir, \textsanskrit{Āḷāra} \textsanskrit{Kālāma} passed away seven days ago.’ 

And\marginnote{46.10} knowledge and vision arose in me, ‘\textsanskrit{Āḷāra} \textsanskrit{Kālāma} passed away seven days ago.’ Then it occurred to me, ‘This is a great loss for \textsanskrit{Āḷāra} \textsanskrit{Kālāma}. If he had heard the teaching, he would have understood it quickly.’ 

Then\marginnote{47.1} it occurred to me, ‘Who should I teach first of all? Who will quickly understand the teaching?’ Then it occurred to me, ‘That Uddaka, son of \textsanskrit{Rāma}, is astute, competent, clever, and has long had little dust in his eyes. Why don’t I teach him first of all? He’ll quickly understand the teaching.’ But a deity came to me and said, ‘Sir, Uddaka, son of \textsanskrit{Rāma}, passed away just last night.’ 

And\marginnote{47.10} knowledge and vision arose in me, ‘Uddaka, son of \textsanskrit{Rāma}, passed away just last night.’ Then it occurred to me, ‘This is a great loss for Uddaka. If he had heard the teaching, he would have understood it quickly.’ 

Then\marginnote{48.1} it occurred to me, ‘Who should I teach first of all? Who will quickly understand the teaching?’ Then it occurred to me, ‘The group of five mendicants were very helpful to me. They looked after me during my time of resolute striving. Why don’t I teach them first of all?’ Then it occurred to me, ‘Where are the group of five mendicants staying these days?’ With clairvoyance that is purified and superhuman I saw that the group of five mendicants were staying near Benares, in the deer park at Isipatana. 

So,\marginnote{49.1} when I had stayed in \textsanskrit{Uruvelā} as long as I wished, I set out for Benares. 

While\marginnote{49.2} I was traveling along the road between \textsanskrit{Gayā} and Bodhgaya, the \textsanskrit{Ājīvaka} ascetic Upaka saw me and said, ‘Reverend, your faculties are so very clear, and your complexion is pure and bright. In whose name have you gone forth, reverend? Who is your Teacher? Whose teaching do you believe in?’ 

I\marginnote{49.6} replied to Upaka in verse: 

\begin{verse}%
‘I\marginnote{49.7} am the champion, the knower of all, \\
unsullied in the midst of all things. \\
I’ve given up all, freed in the ending of craving. \\
Since I know for myself, whose follower should I be? 

I\marginnote{49.11} have no teacher. \\
There is no-one like me. \\
In the world with its gods, \\
I have no counterpart. 

For\marginnote{49.15} in this world, I am the perfected one; \\
I am the supreme Teacher. \\
I alone am fully awakened, \\
cooled, extinguished. 

I\marginnote{49.19} am going to the city of \textsanskrit{Kāsi} \\
to roll forth the Wheel of Dhamma. \\
In this world that is so blind, \\
I’ll beat the deathless drum!’ 

%
\end{verse}

‘According\marginnote{49.23} to what you claim, reverend, you ought to be the Infinite Victor.’ 

\begin{verse}%
‘The\marginnote{49.24} victors are those who, like me, \\
have reached the ending of defilements. \\
I have conquered bad qualities, Upaka—\\
that’s why I’m a victor.’ 

%
\end{verse}

When\marginnote{49.28} I had spoken, Upaka said: ‘If you say so, reverend.’ Shaking his head, he took a wrong turn and left. 

Traveling\marginnote{50.1} stage by stage, I arrived at Benares, and went to see the group of five mendicants in the deer park at Isipatana. The group of five mendicants saw me coming off in the distance and stopped each other, saying, ‘Here comes the ascetic Gotama. He’s so indulgent; he strayed from the struggle and returned to indulgence. We shouldn’t bow to him or rise for him or receive his bowl and robe. But we can set out a seat; he can sit if he likes.’ 

Yet\marginnote{50.7} as I drew closer, the group of five mendicants were unable to stop themselves as they had agreed. Some came out to greet me and receive my bowl and robe, some spread out a seat, while others set out water for washing my feet. But they still addressed me by name and as ‘reverend’. 

So\marginnote{51.1} I said to them, ‘Mendicants, don’t address me by name and as “reverend”. The Realized One is Perfected, a fully awakened Buddha. Listen up, mendicants: I have achieved the Deathless! I shall instruct you, I will teach you the Dhamma. By practicing as instructed you will soon realize the supreme end of the spiritual path in this very life. You will live having achieved with your own insight the goal for which gentlemen rightly go forth from the lay life to homelessness.’ 

But\marginnote{51.6} they said to me, ‘Reverend Gotama, even by that conduct, that practice, that grueling work you did not achieve any superhuman distinction in knowledge and vision worthy of the noble ones. How could you have achieved such a state now that you’ve become indulgent, strayed from the struggle and fallen into indulgence?’ 

So\marginnote{51.8} I said to them, ‘The Realized One has not become indulgent, strayed from the struggle and fallen into indulgence. The Realized One is Perfected, a fully awakened Buddha. Listen up, mendicants: I have achieved the Deathless! I shall instruct you, I will teach you the Dhamma. By practicing as instructed you will soon realize the supreme end of the spiritual path in this very life. You will live having achieved with your own insight the goal for which gentlemen rightly go forth from the lay life to homelessness.’ 

But\marginnote{51.13} for a second time they said to me, ‘Reverend Gotama … you’ve fallen into indulgence.’ 

So\marginnote{51.15} for a second time I said to them, ‘The Realized One has not become indulgent …’ 

But\marginnote{51.20} for a third time they said to me, ‘Reverend Gotama … you’ve fallen into indulgence.’ 

So\marginnote{52.1} I said to them, ‘Mendicants, have you ever known me to speak like this before?’ 

‘No,\marginnote{52.3} sir.’ 

‘The\marginnote{52.4} Realized One is Perfected, a fully awakened Buddha. Listen up, mendicants: I have achieved the Deathless! I shall instruct you, I will teach you the Dhamma. By practicing as instructed you will soon realize the supreme end of the spiritual path in this very life. You will live having achieved with your own insight the goal for which gentlemen rightly go forth from the lay life to homelessness.’ 

I\marginnote{53.1} was able to persuade the group of five mendicants. Then sometimes I advised two mendicants, while the other three went for alms. Then those three would feed all six of us with what they brought back. Sometimes I advised three mendicants, while the other two went for alms. Then those two would feed all six of us with what they brought back. 

As\marginnote{54.1} the group of five mendicants were being advised and instructed by me like this, they soon realized the supreme end of the spiritual path in this very life. They lived having achieved with their own insight the goal for which gentlemen rightly go forth from the lay life to homelessness.” 

When\marginnote{55.1} he had spoken, Prince Bodhi said to the Buddha, “Sir, when a mendicant has the Realized One as trainer, how long would it take for them to realize the supreme end of the spiritual path in this very life?” 

“Well\marginnote{55.3} then, prince, I’ll ask you about this in return, and you can answer as you like. What do you think, prince? Are you skilled in the art of wielding a hooked goad while riding an elephant?” 

“Yes,\marginnote{55.6} sir.” 

“What\marginnote{56.1} do you think, prince? Suppose a man were to come along thinking, ‘Prince Bodhi knows the art of wielding a hooked goad while riding an elephant. I’ll train in that art under him.’ If he’s faithless, he wouldn’t achieve what he could with faith. If he’s unhealthy, he wouldn’t achieve what he could with good health. If he’s devious or deceitful, he wouldn’t achieve what he could with honesty and integrity. If he’s lazy, he wouldn’t achieve what he could with energy. If he’s stupid, he wouldn’t achieve what he could with wisdom. What do you think, prince? Could that man still train under you in the art of wielding a hooked goad while riding an elephant?” 

“Sir,\marginnote{56.17} if he had even a single one of these factors he couldn’t train under me, let alone all five.” 

“What\marginnote{57.1} do you think, prince? Suppose a man were to come along thinking, ‘Prince Bodhi knows the art of wielding a hooked goad while riding an elephant. I’ll train in that art under him.’ If he’s faithful, he’d achieve what he could with faith. If he’s healthy, he’d achieve what he could with good health. If he’s honest and has integrity, he’d achieve what he could with honesty and integrity. If he’s energetic, he’d achieve what he could with energy. If he’s wise, he’d achieve what he could with wisdom. What do you think, prince? Could that man still train under you in the art of wielding a hooked goad while riding an elephant?” 

“Sir,\marginnote{57.17} if he had even a single one of these factors he could train under me, let alone all five.” 

“In\marginnote{58.1} the same way, prince, there are these five factors that support meditation. What five? It’s when a noble disciple has faith in the Realized One’s awakening: ‘That Blessed One is perfected, a fully awakened Buddha, accomplished in knowledge and conduct, holy, knower of the world, supreme guide for those who wish to train, teacher of gods and humans, awakened, blessed.’ They are rarely ill or unwell. Their stomach digests well, being neither too hot nor too cold, but just right, and fit for meditation. They’re not devious or deceitful. They reveal themselves honestly to the Teacher or sensible spiritual companions. They live with energy roused up for giving up unskillful qualities and embracing skillful qualities. They’re strong, staunchly vigorous, not slacking off when it comes to developing skillful qualities. They’re wise. They have the wisdom of arising and passing away which is noble, penetrative, and leads to the complete ending of suffering. These are the five factors that support meditation. 

When\marginnote{59.1} a mendicant with these five factors that support meditation has the Realized One as trainer, they could realize the supreme end of the spiritual path in seven years. Let alone seven years, they could realize the supreme end of the spiritual path in six years, or as little as one year. Let alone one year, when a mendicant with these five factors that support meditation has the Realized One as trainer, they could realize the supreme end of the spiritual path in seven months, or as little as one day. Let alone one day, when a mendicant with these five factors that support meditation has the Realized One as trainer, they could be instructed in the evening and achieve distinction in the morning, or be instructed in the morning and achieve distinction in the evening.” 

When\marginnote{60.1} he had spoken, Prince Bodhi said to the Buddha, “Oh, the Buddha! Oh, the teaching! Oh, how well explained is the teaching! For someone could be instructed in the evening and achieve distinction in the morning, or be instructed in the morning and achieve distinction in the evening.” 

When\marginnote{61.1} he said this, \textsanskrit{Sañjikāputta} said to Prince Bodhi, “Though Master Bodhi speaks like this, you don’t go for refuge to Master Gotama, to the teaching, and to the mendicant \textsanskrit{Saṅgha}.” 

“Don’t\marginnote{61.5} say that, dear \textsanskrit{Sañjikāputta}, don’t say that! I have heard and learned this in the presence of the lady, my mother. This one time the Buddha was staying near Kosambi, in Ghosita’s Monastery. Then my pregnant mother went up to the Buddha, bowed, sat down to one side, and said to him, ‘Sir, the prince or princess in my womb goes for refuge to the Buddha, the teaching, and the mendicant \textsanskrit{Saṅgha}. From this day forth, may the Buddha remember them as a lay follower who has gone for refuge for life.’ 

Another\marginnote{61.11} time the Buddha was staying here in the land of the Bhaggas on Crocodile Hill, in the deer park at \textsanskrit{Bhesakaḷā}’s Wood. Then my nurse, carrying me on her hip, went to the Buddha, bowed, stood to one side, and said to him, ‘Sir, this Prince Bodhi goes for refuge to the Buddha, to the teaching, and to the mendicant \textsanskrit{Saṅgha}. From this day forth, may the Buddha remember him as a lay follower who has gone for refuge for life.’ 

Now\marginnote{61.15} for a third time I go for refuge to the Buddha, to the teaching, and to the mendicant \textsanskrit{Saṅgha}. From this day forth, may the Buddha remember me as a lay follower who has gone for refuge for life.” 

%
\section*{{\suttatitleacronym MN 86}{\suttatitletranslation With Aṅgulimāla }{\suttatitleroot Aṅgulimālasutta}}
\addcontentsline{toc}{section}{\tocacronym{MN 86} \toctranslation{With Aṅgulimāla } \tocroot{Aṅgulimālasutta}}
\markboth{With Aṅgulimāla }{Aṅgulimālasutta}
\extramarks{MN 86}{MN 86}

\scevam{So\marginnote{1.1} I have heard. }At one time the Buddha was staying near \textsanskrit{Sāvatthī} in Jeta’s Grove, \textsanskrit{Anāthapiṇḍika}’s monastery. 

Now\marginnote{2.1} at that time in the realm of King Pasenadi of Kosala there was a bandit named \textsanskrit{Aṅgulimāla}. He was violent, bloody-handed, a hardened killer, merciless to living beings. He laid waste to villages, towns, and countries. He was constantly murdering people, and he wore their fingers as a necklace. 

Then\marginnote{3.1} the Buddha robed up in the morning and, taking his bowl and robe, entered \textsanskrit{Sāvatthī} for alms. Then, after the meal, on his return from almsround, he set his lodgings in order and, taking his bowl and robe, he walked down the road that led to \textsanskrit{Aṅgulimāla}. 

The\marginnote{3.3} cowherds, shepherds, farmers, and travelers saw him on the road, and said to him, “Don’t take this road, ascetic. On this road there is a bandit named \textsanskrit{Aṅgulimāla}. He is violent, bloody-handed, a hardened killer, merciless to living beings. He has laid waste to villages, towns, and countries. He is constantly murdering people, and he wears their fingers as a necklace. People travel along this road only after banding closely together in groups of ten, twenty, thirty, forty, or fifty. Still they meet their end by \textsanskrit{Aṅgulimāla}’s hand.” But when they said this, the Buddha went on in silence. 

For\marginnote{3.12} a second time … and a third time, they urged the Buddha to turn back. 

But\marginnote{4.1} when they said this, the Buddha went on in silence. 

The\marginnote{4.2} bandit \textsanskrit{Aṅgulimāla} saw the Buddha coming off in the distance, and thought, “It’s incredible, it’s amazing! People travel along this road only after banding closely together in groups of ten, twenty, thirty, forty, or fifty. Still they meet their end by my hand. But still this ascetic comes along alone and unaccompanied, like he had beaten me already. Why don’t I take his life?” 

Then\marginnote{5.1} \textsanskrit{Aṅgulimāla} donned his sword and shield, fastened his bow and arrows, and followed behind the Buddha. But the Buddha used his psychic power to will that \textsanskrit{Aṅgulimāla} could not catch up with him no matter how hard he tried, even though the Buddha kept walking at a normal speed. 

Then\marginnote{5.3} \textsanskrit{Aṅgulimāla} thought, “It’s incredible, it’s amazing! Previously, even when I’ve chased a speeding elephant, horse, chariot or deer, I’ve always caught up with them. But I can’t catch up with this ascetic no matter how hard I try, even though he’s walking at a normal speed.” 

He\marginnote{5.7} stood still and said, “Stop, stop, ascetic!” 

“I’ve\marginnote{5.9} stopped, \textsanskrit{Aṅgulimāla}—now you stop.” 

Then\marginnote{5.10} \textsanskrit{Aṅgulimāla} thought, “These Sakyan ascetics speak the truth. Yet while walking the ascetic Gotama says: ‘I’ve stopped, \textsanskrit{Aṅgulimāla}—now you stop.’ Why don’t I ask him about this?” 

Then\marginnote{6.1} he addressed the Buddha in verse: 

\begin{verse}%
“While\marginnote{6.2} walking, ascetic, you say ‘I’ve stopped.’ \\
And I have stopped, but you tell me I’ve not. \\
I’m asking you this, ascetic: \\
how is it you’ve stopped and I have not?” 

“\textsanskrit{Aṅgulimāla},\marginnote{6.6} I have forever stopped—\\
I’ve laid aside violence towards all creatures. \\
But you can’t stop yourself from harming living creatures; \\
that’s why I’ve stopped, but you have not.” 

“Oh,\marginnote{6.10} at long last a hermit, \\
a great sage who I honor, has entered this great forest. \\
Now that I’ve heard your verse on Dhamma, \\
I shall live without evil.” 

With\marginnote{6.14} these words, the bandit hurled his sword and weapons \\
down a cliff into an abyss. \\
He venerated the Holy One’s feet, \\
and asked him for the going forth right away. 

Then\marginnote{6.18} the Buddha, the compassionate great hermit, \\
the teacher of the world with its gods, \\
said to him, “Come, monk!” \\
And with that he became a monk. 

%
\end{verse}

Then\marginnote{7.1} the Buddha set out for \textsanskrit{Sāvatthī} with Venerable \textsanskrit{Aṅgulimāla} as his second monk. Traveling stage by stage, he arrived at \textsanskrit{Sāvatthī}, where he stayed in Jeta’s Grove, \textsanskrit{Anāthapiṇḍika}’s monastery. 

Now\marginnote{8.1} at that time a crowd had gathered by the gate of King Pasenadi’s royal compound making a dreadful racket, “In your realm, Your Majesty, there is a bandit named \textsanskrit{Aṅgulimāla}. He is violent, bloody-handed, a hardened killer, merciless to living beings. He has laid waste to villages, towns, and countries. He is constantly murdering people, and he wears their fingers as a necklace. Your Majesty must put a stop to him!” 

Then\marginnote{9.1} King Pasenadi drove out from \textsanskrit{Sāvatthī} in the middle of the day with around five hundred horses, heading for the monastery. He went by carriage as far as the terrain allowed, then descended and approached the Buddha on foot. He bowed and sat down to one side. The Buddha said to him, 

“What\marginnote{9.4} is it, great king? Is King Seniya \textsanskrit{Bimbisāra} of Magadha angry with you, or the Licchavis of \textsanskrit{Vesālī}, or some other opposing ruler?” 

“No,\marginnote{10.1} sir. In my realm there is a bandit named \textsanskrit{Aṅgulimāla}. He is violent, bloody-handed, a hardened killer, merciless to living beings. … I shall put a stop to him.” 

“But\marginnote{11.1} great king, suppose you were to see that \textsanskrit{Aṅgulimāla} had shaved off his hair and beard, dressed in ocher robes, and gone forth from the lay life to homelessness. And that he was refraining from killing living creatures, stealing, and lying; that he was eating in one part of the day, and was celibate, ethical, and of good character. What would you do to him?” 

“I\marginnote{11.2} would bow to him, rise in his presence, or offer him a seat. I’d invite him to accept robes, almsfood, lodgings, and medicines and supplies for the sick. And I’d organize his lawful guarding and protection. But sir, how could such an immoral, evil man ever have such virtue and restraint?” 

Now\marginnote{12.1} at that time Venerable \textsanskrit{Aṅgulimāla} was sitting not far from the Buddha. Then the Buddha pointed with his right arm and said to the king, “Great king, this is \textsanskrit{Aṅgulimāla}.” 

Then\marginnote{12.4} the king became frightened, scared, his hair standing on end. Knowing this, the Buddha said to him, “Do not fear, great king. You have nothing to fear from him.” Then the king’s fear died down. 

Then\marginnote{12.8} the king went over to \textsanskrit{Aṅgulimāla} and said, “Sir, is the venerable really \textsanskrit{Aṅgulimāla}?” 

“Yes,\marginnote{12.10} great king.” 

“What\marginnote{12.11} clans were your father and mother from?” 

“My\marginnote{12.12} father was a Gagga, and my mother a \textsanskrit{Mantāṇī}.” 

“May\marginnote{12.13} the venerable Gagga son of \textsanskrit{Mantāṇī} be happy. I’ll make sure that you’re provided with robes, almsfood, lodgings, and medicines and supplies for the sick.” 

But\marginnote{13.1} at that time Venerable \textsanskrit{Aṅgulimāla} lived in the wilderness, ate only almsfood, and owned just three robes. So he said to the king, “Enough, great king. My robes are complete.” 

Then\marginnote{13.4} the king went back to the Buddha, bowed, sat down to one side, and said to him, “It’s incredible, sir, it’s amazing! How the Buddha tames those who are wild, pacifies those who are violent, and extinguishes those who are unextinguished! For I was not able to tame him with the rod and the sword, but the Buddha tamed him without rod or sword. Well, now, sir, I must go. I have many duties, and much to do.” 

“Please,\marginnote{13.10} great king, go at your convenience.” Then King Pasenadi got up from his seat, bowed, and respectfully circled the Buddha, keeping him on his right, before leaving. 

Then\marginnote{14.1} Venerable \textsanskrit{Aṅgulimāla} robed up in the morning and, taking his bowl and robe, entered \textsanskrit{Sāvatthī} for alms. Then as he was wandering indiscriminately for almsfood he saw a woman undergoing a painful obstructed labor. Seeing this, it occurred to him, “Oh, beings undergo such travail! Oh, beings undergo such travail!” 

Then\marginnote{14.6} after wandering for alms in \textsanskrit{Sāvatthī}, after the meal, on his return from almsround, he went to the Buddha, bowed, sat down to one side, and told him what had happened. The Buddha said to him, “Well then, \textsanskrit{Aṅgulimāla}, go to that woman and say this: 

‘Ever\marginnote{15.2} since I was born, sister, I don’t recall having intentionally taken the life of a living creature. By this truth, may both you and your baby be safe.’” 

“But\marginnote{15.3} sir, wouldn’t that be telling a deliberate lie? For I have intentionally killed many living creatures.” 

“In\marginnote{15.5} that case, \textsanskrit{Aṅgulimāla}, go to that woman and say this: 

‘Ever\marginnote{15.6} since I was born in the noble birth, sister, I don’t recall having intentionally taken the life of a living creature. By this truth, may both you and your baby be safe.’” 

“Yes,\marginnote{15.7} sir,” replied \textsanskrit{Aṅgulimāla}. He went to that woman and said: 

“Ever\marginnote{15.8} since I was born in the noble birth, sister, I don’t recall having intentionally taken the life of a living creature. By this truth, may both you and your baby be safe.” 

Then\marginnote{15.9} that woman was safe, and so was her baby. 

Then\marginnote{16.1} \textsanskrit{Aṅgulimāla}, living alone, withdrawn, diligent, keen, and resolute, soon realized the supreme end of the spiritual path in this very life. He lived having achieved with his own insight the goal for which gentlemen rightly go forth from the lay life to homelessness. 

He\marginnote{16.2} understood: “Rebirth is ended; the spiritual journey has been completed; what had to be done has been done; there is no return to any state of existence.” And Venerable \textsanskrit{Aṅgulimāla} became one of the perfected. 

Then\marginnote{17.1} Venerable \textsanskrit{Aṅgulimāla} robed up in the morning and, taking his bowl and robe, entered \textsanskrit{Sāvatthī} for alms. Now at that time someone threw a stone that hit \textsanskrit{Aṅgulimāla}, someone else threw a stick, and someone else threw gravel. Then \textsanskrit{Aṅgulimāla}—with cracked head, bleeding, his bowl broken, and his outer robe torn—went to the Buddha. 

The\marginnote{17.4} Buddha saw him coming off in the distance, and said to him, “Endure it, brahmin! Endure it, brahmin! You’re experiencing in this life the result of deeds that might have caused you to be tormented in hell for many years, many hundreds or thousands of years.” 

Later,\marginnote{18.1} Venerable \textsanskrit{Aṅgulimāla} was experiencing the bliss of release while in private retreat. On that occasion he expressed this heartfelt sentiment: 

\begin{verse}%
“He\marginnote{18.3} who once was heedless, \\
but turned to heedfulness, \\
lights up the world, \\
like the moon freed from a cloud. 

Someone\marginnote{18.7} whose bad deed \\
is supplanted by the good, \\
lights up the world, \\
like the moon freed from a cloud. 

A\marginnote{18.11} young mendicant \\
devoted to the Buddha’s teaching, \\
lights up the world, \\
like the moon freed from a cloud. 

May\marginnote{18.15} even my enemies hear a Dhamma talk! \\
May even my enemies devote themselves to the Buddha’s teaching! \\
May even my enemies associate with those good people \\
who establish others in the Dhamma! 

May\marginnote{18.19} even my enemies hear Dhamma at the right time, \\
from those who speak on acceptance, \\
praising acquiescence; \\
and may they follow that path! 

For\marginnote{18.23} then they’d surely wish no harm \\
upon myself or others. \\
Having arrived at ultimate peace, \\
they’d look after creatures firm and frail. 

For\marginnote{18.27} irrigators guide the water, \\
and fletchers straighten arrows; \\
carpenters carve timber—\\
but the astute tame themselves. 

Some\marginnote{18.31} tame by using the rod, \\
some with goads, and some with whips. \\
But the poised one tamed me \\
without rod or sword. 

My\marginnote{18.35} name is ‘Harmless’, \\
though I used to be harmful. \\
The name I bear today is true, \\
for I do no harm to anyone. 

I\marginnote{18.39} used to be a bandit, \\
the notorious \textsanskrit{Aṅgulimāla}. \\
Swept away in a great flood, \\
I went to the Buddha as a refuge. 

I\marginnote{18.43} used to have blood on my hands, \\
the notorious \textsanskrit{Aṅgulimāla}. \\
See the refuge I’ve found—\\
the conduit to rebirth is eradicated. 

I’ve\marginnote{18.47} done many of the sort of deeds \\
that lead to a bad destination. \\
The result of my deeds has already struck me, \\
so I enjoy my food free of debt. 

Fools\marginnote{18.51} and half-wits \\
devote themselves to negligence. \\
But the wise protect diligence \\
as their best treasure. 

Don’t\marginnote{18.55} devote yourself to negligence, \\
or delight in sexual intimacy. \\
For if you’re diligent and practice absorption, \\
you’ll attain abundant happiness. 

It\marginnote{18.59} was welcome, not unwelcome, \\
the advice I got was good. \\
Of the well-explained teachings, \\
I arrived at the the best. 

It\marginnote{18.63} was welcome, not unwelcome, \\
the advice I got was good. \\
I’ve attained the three knowledges \\
and fulfilled the Buddha’s instructions.” 

%
\end{verse}

%
\section*{{\suttatitleacronym MN 87}{\suttatitletranslation Born From the Beloved }{\suttatitleroot Piyajātikasutta}}
\addcontentsline{toc}{section}{\tocacronym{MN 87} \toctranslation{Born From the Beloved } \tocroot{Piyajātikasutta}}
\markboth{Born From the Beloved }{Piyajātikasutta}
\extramarks{MN 87}{MN 87}

\scevam{So\marginnote{1.1} I have heard. }At one time the Buddha was staying near \textsanskrit{Sāvatthī} in Jeta’s Grove, \textsanskrit{Anāthapiṇḍika}’s monastery. 

Now\marginnote{2.1} at that time a certain householder’s dear and beloved only child passed away. After their death he didn’t feel like working or eating. He would go to the cemetery and wail, “Where are you, my only child? Where are you, my only child?” 

Then\marginnote{3.1} he went to the Buddha, bowed, and sat down to one side. The Buddha said to him, “Householder, you look like someone who’s not in their right mind; your faculties have deteriorated.” 

“And\marginnote{3.3} how, sir, could my faculties not have deteriorated? For my dear and beloved only child has passed away. Since their death I haven’t felt like working or eating. I go to the cemetery and wail: ‘Where are you, my only child? Where are you, my only child?’” 

“That’s\marginnote{3.8} so true, householder! That’s so true, householder! For our loved ones are a source of sorrow, lamentation, pain, sadness, and distress.” 

“Sir,\marginnote{3.10} who on earth could ever think such a thing! For our loved ones are a source of joy and happiness.” Disagreeing with the Buddha’s statement, rejecting it, he got up from his seat and left. 

Now\marginnote{4.1} at that time several gamblers were playing dice not far from the Buddha. That householder approached them and told them what had happened. 

“That’s\marginnote{4.17} so true, householder! That’s so true, householder! For our loved ones are a source of joy and happiness.” 

Thinking,\marginnote{4.19} “The gamblers and I are in agreement,” the householder left. 

Eventually\marginnote{5.1} that topic of discussion reached the royal compound. Then King Pasenadi addressed Queen \textsanskrit{Mallikā}, “Mallika, your ascetic Gotama said this: ‘Our loved ones are a source of sorrow, lamentation, pain, sadness, and distress.’” 

“If\marginnote{5.5} that’s what the Buddha said, great king, then that’s how it is.” 

“No\marginnote{5.6} matter what the ascetic Gotama says, \textsanskrit{Mallikā} agrees with him: ‘If that’s what the Buddha said, great king, then that’s how it is.’ You’re just like a student who agrees with everything their teacher says. Go away, \textsanskrit{Mallikā}, get out of here!” 

Then\marginnote{6.1} Queen \textsanskrit{Mallikā} addressed the brahmin \textsanskrit{Nāḷijaṅgha}, “Please, brahmin, go to the Buddha, and in my name bow with your head to his feet. Ask him if he is healthy and well, nimble, strong, and living comfortably. And then say: ‘Sir, did the Buddha make this statement: “Our loved ones are a source of sorrow, lamentation, pain, sadness, and distress”?’ Remember well how the Buddha answers and tell it to me. For Realized Ones say nothing that is not so.” 

“Yes,\marginnote{6.9} ma’am,” he replied. He went to the Buddha and exchanged greetings with him. When the greetings and polite conversation were over, he sat down to one side and said to the Buddha, “Master Gotama, Queen \textsanskrit{Mallikā} bows with her head to your feet. She asks if you are healthy and well, nimble, strong, and living comfortably. And she asks whether the Buddha made this statement: ‘Our loved ones are a source of sorrow, lamentation, pain, sadness, and distress.’” 

“That’s\marginnote{7.1} right, brahmin, that’s right! For our loved ones are a source of sorrow, lamentation, pain, sadness, and distress. 

And\marginnote{8.1} here’s a way to understand how our loved ones are a source of sorrow, lamentation, pain, sadness, and distress. Once upon a time right here in \textsanskrit{Sāvatthī} a certain woman’s mother passed away. And because of that she went mad and lost her mind. She went from street to street and from square to square saying, ‘Has anyone seen my mother? Has anyone seen my mother?’ 

And\marginnote{9{-}14.1} here’s another way to understand how our loved ones are a source of sorrow, lamentation, pain, sadness, and distress. 

Once\marginnote{9{-}14.2} upon a time right here in \textsanskrit{Sāvatthī} a certain woman’s father … brother … sister … son … daughter … husband passed away. And because of that she went mad and lost her mind. She went from street to street and from square to square saying, ‘Has anyone seen my husband? Has anyone seen my husband?’ 

And\marginnote{15{-}21.1} here’s another way to understand how our loved ones are a source of sorrow, lamentation, pain, sadness, and distress. 

Once\marginnote{15{-}21.2} upon a time right here in \textsanskrit{Sāvatthī} a certain man’s mother … father … brother … sister … son … daughter … wife passed away. And because of that he went mad and lost his mind. He went from street to street and from square to square saying, ‘Has anyone seen my wife? Has anyone seen my wife?’ 

And\marginnote{15{-}21.14} here’s another way to understand how our loved ones are a source of sorrow, lamentation, pain, sadness, and distress. 

Once\marginnote{22.1} upon a time right here in \textsanskrit{Sāvatthī} a certain woman went to live with her relative’s family. But her relatives wanted to divorce her from her husband and give her to another, who she didn’t want. So she told her husband about this. But he cut her in two and disemboweled himself, thinking, ‘We shall be together after death.’ That’s another way to understand how our loved ones are a source of sorrow, lamentation, pain, sadness, and distress.” 

Then\marginnote{23.1} \textsanskrit{Nāḷijaṅgha} the brahmin, having approved and agreed with what the Buddha said, got up from his seat, went to Queen \textsanskrit{Mallikā}, and told her of all they had discussed. Then Queen \textsanskrit{Mallikā} approached King Pasenadi and said to him, “What do you think, great king? Do you love Princess \textsanskrit{Vajirī}?” 

“Indeed\marginnote{24.3} I do, \textsanskrit{Mallikā}.” 

“What\marginnote{24.4} do you think, great king? If she were to decay and perish, would sorrow, lamentation, pain, sadness, and distress arise in you?” 

“If\marginnote{24.6} she were to decay and perish, my life would fall apart. How could sorrow, lamentation, pain, sadness, and distress not arise in me?” 

“This\marginnote{24.7} is what the Buddha was referring to when he said: ‘Our loved ones are a source of sorrow, lamentation, pain, sadness, and distress.’ 

What\marginnote{25.1} do you think, great king? Do you love Lady \textsanskrit{Vāsabhā}? … 

Do\marginnote{26.1} you love your son, General \textsanskrit{Viḍūḍabha}? … 

Do\marginnote{27.1} you love me?” 

“Indeed\marginnote{27.2} I do love you, \textsanskrit{Mallikā}.” 

“What\marginnote{27.3} do you think, great king? If I were to decay and perish, would sorrow, lamentation, pain, sadness, and distress arise in you?” 

“If\marginnote{27.5} you were to decay and perish, my life would fall apart. How could sorrow, lamentation, pain, sadness, and distress not arise in me?” 

“This\marginnote{27.6} is what the Buddha was referring to when he said: ‘Our loved ones are a source of sorrow, lamentation, pain, sadness, and distress.’ 

What\marginnote{28.1} do you think, great king? Do you love the realms of \textsanskrit{Kāsi} and Kosala?” 

“Indeed\marginnote{28.3} I do, \textsanskrit{Mallikā}. It’s due to the bounty of \textsanskrit{Kāsi} and Kosala that we use sandalwood imported from \textsanskrit{Kāsi} and wear garlands, perfumes, and makeup.” 

“What\marginnote{28.5} do you think, great king? If these realms were to decay and perish, would sorrow, lamentation, pain, sadness, and distress arise in you?” 

“If\marginnote{28.7} they were to decay and perish, my life would fall apart. How could sorrow, lamentation, pain, sadness, and distress not arise in me?” 

“This\marginnote{28.8} is what the Buddha was referring to when he said: ‘Our loved ones are a source of sorrow, lamentation, pain, sadness, and distress.’” 

“It’s\marginnote{29.1} incredible, \textsanskrit{Mallikā}, it’s amazing, how far the Buddha sees with penetrating wisdom, it seems to me. Come, \textsanskrit{Mallikā}, rinse my hands.” 

Then\marginnote{29.4} King Pasenadi got up from his seat, arranged his robe over one shoulder, raised his joined palms toward the Buddha, and expressed this heartfelt sentiment three times: 

“Homage\marginnote{29.5} to that Blessed One, the perfected one, the fully awakened Buddha! 

Homage\marginnote{29.6} to that Blessed One, the perfected one, the fully awakened Buddha! 

Homage\marginnote{29.7} to that Blessed One, the perfected one, the fully awakened Buddha!” 

%
\section*{{\suttatitleacronym MN 88}{\suttatitletranslation The Imported Cloth }{\suttatitleroot Bāhitikasutta}}
\addcontentsline{toc}{section}{\tocacronym{MN 88} \toctranslation{The Imported Cloth } \tocroot{Bāhitikasutta}}
\markboth{The Imported Cloth }{Bāhitikasutta}
\extramarks{MN 88}{MN 88}

\scevam{So\marginnote{1.1} I have heard. }At one time the Buddha was staying near \textsanskrit{Sāvatthī} in Jeta’s Grove, \textsanskrit{Anāthapiṇḍika}’s monastery. 

Then\marginnote{2.1} Venerable Ānanda robed up in the morning and, taking his bowl and robe, entered \textsanskrit{Sāvatthī} for alms. He wandered for alms in \textsanskrit{Sāvatthī}. After the meal, on his return from almsround, he went to the Eastern Monastery, the stilt longhouse of \textsanskrit{Migāra}’s mother, for the day’s meditation. 

Now\marginnote{3.1} at that time King Pasenadi of Kosala mounted the Single Lotus Elephant and drove out from \textsanskrit{Sāvatthī} in the middle of the day. He saw Ānanda coming off in the distance and said to the minister \textsanskrit{Sirivaḍḍha}, “My dear \textsanskrit{Sirivaḍḍha}, isn’t that Venerable Ānanda?” 

“Indeed\marginnote{3.5} it is, great king.” 

Then\marginnote{4.1} King Pasenadi addressed a man, “Please, mister, go to Venerable Ānanda, and in my name bow with your head to his feet. Say to him: ‘Sir, King Pasenadi of Kosala bows with his head at your feet.’ And then say: ‘Sir, if you have no urgent business, please wait a moment out of compassion.’” 

“Yes,\marginnote{5.1} Your Majesty,” that man replied. He did as the king asked. 

Ānanda\marginnote{6.1} consented in silence. 

Then\marginnote{6.2} King Pasenadi rode on the elephant as far as the terrain allowed, then descended and approached Ānanda on foot. He bowed, stood to one side, and said to Ānanda, “Sir, if you have no urgent business, it would be nice of you to go to the bank of the \textsanskrit{Aciravatī} river out of compassion.” 

Ānanda\marginnote{7.1} consented in silence. 

He\marginnote{7.2} went to the river bank and sat at the root of a certain tree on a seat spread out. Then King Pasenadi rode on the elephant as far as the terrain allowed, then descended and approached Ānanda on foot. He bowed, stood to one side, and said to Ānanda, “Here, Venerable Ānanda, sit on this elephant rug.” 

“Enough,\marginnote{7.5} great king, you sit on it. I’m sitting on my own seat.” 

So\marginnote{8.1} the king sat down on the seat spread out, and said, “Sir, might the Buddha engage in the sort of behavior—by way of body, speech, or mind—that is faulted by ascetics and brahmins?” 

“No,\marginnote{8.4} great king, the Buddha would not engage in the sort of behavior that is faulted by sensible ascetics and brahmins.” 

“It’s\marginnote{9.1} incredible, sir, it’s amazing! For I couldn’t fully express the question, but Ānanda’s answer completed it for me. I don’t believe that praise or criticism of others spoken by incompetent fools, without examining or scrutinizing, is the most important thing. Rather, I believe that praise or criticism of others spoken by competent and intelligent people after examining and scrutinizing is the most important thing. 

But\marginnote{10.1} sir, what kind of bodily behavior is faulted by sensible ascetics and brahmins?” 

“Unskillful\marginnote{10.2} behavior.” 

“But\marginnote{10.3} what kind of bodily behavior is unskillful?” 

“Blameworthy\marginnote{10.4} behavior.” 

“But\marginnote{10.5} what kind of bodily behavior is blameworthy?” 

“Hurtful\marginnote{10.6} behavior.” 

“But\marginnote{10.7} what kind of bodily behavior is hurtful?” 

“Behavior\marginnote{10.8} that results in suffering.” 

“But\marginnote{10.9} what kind of bodily behavior results in suffering?” 

“Bodily\marginnote{10.10} behavior that leads to hurting yourself, hurting others, and hurting both, and which makes unskillful qualities grow while skillful qualities decline. That kind of bodily behavior is faulted by sensible ascetics and brahmins.” 

“But\marginnote{11.1} what kind of verbal behavior … mental behavior is faulted by sensible ascetics and brahmins?” … 

“Mental\marginnote{12.4} behavior that leads to hurting yourself, hurting others, and hurting both, and which makes unskillful qualities grow while skillful qualities decline. That kind of mental behavior is faulted by sensible ascetics and brahmins.” 

“Sir,\marginnote{13.1} does the Buddha praise giving up all these unskillful things?” 

“Great\marginnote{13.2} king, the Realized One has given up all unskillful things and possesses skillful things.” 

“But\marginnote{14.1} sir, what kind of bodily behavior is not faulted by sensible ascetics and brahmins?” 

“Skillful\marginnote{14.2} behavior.” 

“But\marginnote{14.3} what kind of bodily behavior is skillful?” 

“Blameless\marginnote{14.4} behavior.” 

“But\marginnote{14.5} what kind of bodily behavior is blameless?” 

“Pleasing\marginnote{14.6} behavior.” 

“But\marginnote{14.7} what kind of bodily behavior is pleasing?” 

“Behavior\marginnote{14.8} that results in happiness.” 

“But\marginnote{14.9} what kind of bodily behavior results in happiness?” 

“Bodily\marginnote{14.10} behavior that leads to pleasing yourself, pleasing others, and pleasing both, and which makes unskillful qualities decline while skillful qualities grow. That kind of bodily behavior is not faulted by sensible ascetics and brahmins.” 

“But\marginnote{15.1} what kind of verbal behavior … mental behavior is not faulted by sensible ascetics and brahmins?” … 

“Mental\marginnote{16.6} behavior that leads to pleasing yourself, pleasing others, and pleasing both, and which makes unskillful qualities decline while skillful qualities grow. That kind of mental behavior is not faulted by sensible ascetics and brahmins.” 

“Sir,\marginnote{17.1} does the Buddha praise embracing all these skillful things?” 

“Great\marginnote{17.2} king, the Realized One has given up all unskillful things and possesses skillful things.” 

“It’s\marginnote{18.1} incredible, sir, it’s amazing! How well this was said by Venerable Ānanda! I’m delighted and satisfied with what you’ve expressed so well. So much so that if an elephant-treasure was suitable for you, I would give you one. If a horse-treasure was suitable for you, I would give you one. If a prize village was suitable for you, I would give you one. But, sir, I know that these things are not suitable for you. This imported cloth was sent to me by King \textsanskrit{Ajātasattu} Vedehiputta of Magadha packed in a parasol case. It’s exactly sixteen measures long and eight wide. May Venerable Ānanda please accept it out of compassion.” 

“Enough,\marginnote{18.12} great king. My three robes are complete.” 

“Sir,\marginnote{19.1} we have both seen this river \textsanskrit{Aciravatī} when it has rained heavily in the mountains, and the river overflows both its banks. In the same way, Venerable Ānanda can make a set of three robes for himself from this imported cloak. And you can share your old robes with your fellow monks. In this way my religious donation will come to overflow, it seems to me. Please accept the imported cloth.” 

So\marginnote{20.1} Ānanda accepted it. 

Then\marginnote{20.2} King Pasenadi said to him, “Well, now, sir, I must go. I have many duties, and much to do.” 

“Please,\marginnote{20.5} great king, go at your convenience.” Then King Pasenadi approved and agreed with what Ānanda said. He got up from his seat, bowed, and respectfully circled Ānanda, keeping him on his right, before leaving. 

Soon\marginnote{21.1} after he left, Ānanda went to the Buddha, bowed, sat down to one side, and told him what had happened. He presented the cloth to the Buddha. 

Then\marginnote{22.1} the Buddha said to the mendicants, 

“Mendicants,\marginnote{22.2} King Pasenadi is lucky, so very lucky, to get to see Ānanda and pay homage to him.” 

That\marginnote{22.4} is what the Buddha said. Satisfied, the mendicants were happy with what the Buddha said. 

%
\section*{{\suttatitleacronym MN 89}{\suttatitletranslation Shrines to the Teaching }{\suttatitleroot Dhammacetiyasutta}}
\addcontentsline{toc}{section}{\tocacronym{MN 89} \toctranslation{Shrines to the Teaching } \tocroot{Dhammacetiyasutta}}
\markboth{Shrines to the Teaching }{Dhammacetiyasutta}
\extramarks{MN 89}{MN 89}

\scevam{So\marginnote{1.1} I have heard. }At one time the Buddha was staying in the land of the Sakyans, near the Sakyan town named \textsanskrit{Medaḷumpa}. 

Now\marginnote{2.1} at that time King Pasenadi of Kosala had arrived at Townsville on some business. 

Then\marginnote{2.2} he addressed \textsanskrit{Dīgha} \textsanskrit{Kārāyana}, “My good \textsanskrit{Kārāyana}, harness the finest chariots. We will go to a park and see the scenery.” 

“Yes,\marginnote{2.4} Your Majesty,” replied \textsanskrit{Dīgha} \textsanskrit{Kārāyana}. He harnessed the chariots and informed the king, “Sire, the finest chariots are harnessed. Please go at your convenience.” 

Then\marginnote{3.1} King Pasenadi mounted a fine carriage and, along with other fine carriages, set out in full royal pomp from Townsville, heading for the park grounds. He went by carriage as far as the terrain allowed, then descended and entered the park on foot. 

As\marginnote{4.1} he was going for a walk in the park he saw roots of trees that were impressive and inspiring, quiet and still, far from the madding crowd, remote from human settlements, and fit for retreat. The sight reminded him right away of the Buddha: “These roots of trees, so impressive and inspiring, are like those where we used to pay homage to the Blessed One, the perfected one, the fully awakened Buddha.” 

He\marginnote{4.4} addressed \textsanskrit{Dīgha} \textsanskrit{Kārāyana}, “These roots of trees, so impressive and inspiring, are like those where we used to pay homage to the Blessed One, the perfected one, the fully awakened Buddha. My good \textsanskrit{Kārāyana}, where is that Buddha at present?” 

“Great\marginnote{5.1} king, there is a Sakyan town named \textsanskrit{Medaḷumpa}. That’s where the Buddha is now staying.” 

“But\marginnote{5.3} how far away is that town?” 

“Not\marginnote{5.4} far, great king, it’s three leagues. We can get there while it’s still light.” 

“Well\marginnote{5.7} then, harness the chariots, and we shall go to see the Buddha.” 

“Yes,\marginnote{5.8} Your Majesty,” replied \textsanskrit{Dīgha} \textsanskrit{Kārāyana}. He harnessed the chariots and informed the king, “Sire, the finest chariots are harnessed. Please go at your convenience.” 

Then\marginnote{6.1} King Pasenadi mounted a fine carriage and, along with other fine carriages, set out from Townsville to \textsanskrit{Medaḷumpa}. He reached the town while it was still light and headed for the park grounds. He went by carriage as far as the terrain allowed, then descended and entered the monastery on foot. 

At\marginnote{7.1} that time several mendicants were walking mindfully in the open air. King Pasenadi of Kosala went up to them and said, “Sirs, where is the Blessed One at present, the perfected one, the fully awakened Buddha? For I want to see him.” 

“Great\marginnote{8.1} king, that’s his dwelling, with the door closed. Approach it quietly, without hurrying; go onto the porch, clear your throat, and knock with the latch. The Buddha will open the door.” The king right away presented his sword and turban to \textsanskrit{Dīgha} \textsanskrit{Kārāyana}, who thought, “Now the king seeks privacy. I should wait here.” 

Then\marginnote{8.5} the king approached the Buddha’s dwelling and knocked, and the Buddha opened the door. 

King\marginnote{9.1} Pasenadi entered the dwelling, and bowed with his head at the Buddha’s feet, caressing them and covering them with kisses, and pronounced his name: “Sir, I am Pasenadi, king of Kosala! I am Pasenadi, king of Kosala!” 

“But\marginnote{9.4} great king, for what reason do you demonstrate such utmost devotion for this body, conveying your manifest love?” 

“Sir,\marginnote{10.1} I infer about the Buddha from the teaching: ‘The Blessed One is a fully awakened Buddha. The teaching is well explained. The \textsanskrit{Saṅgha} is practicing well.’ It happens, sir, that I see some ascetics and brahmins leading the spiritual life only for a limited period: ten, twenty, thirty, or forty years. Some time later—nicely bathed and anointed, with hair and beard dressed—they amuse themselves, supplied and provided with the five kinds of sensual stimulation. But here I see the mendicants leading the spiritual life entirely full and pure as long as they live, to their last breath. I don’t see any other spiritual life elsewhere so full and pure. That’s why I infer this about the Buddha from the teaching: ‘The Blessed One is a fully awakened Buddha. The teaching is well explained. The \textsanskrit{Saṅgha} is practicing well.’ 

Furthermore,\marginnote{11.1} kings fight with kings, aristocrats fight with aristocrats, brahmins fight with brahmins, householders fight with householders. A mother fights with her child, child with mother, father with child, and child with father. Brother fights with brother, brother with sister, sister with brother, and friend fights with friend. But here I see the mendicants living in harmony, appreciating each other, without quarreling, blending like milk and water, and regarding each other with kindly eyes. I don’t see any other assembly elsewhere so harmonious. So I infer this about the Buddha from the teaching: ‘The Blessed One is a fully awakened Buddha. The teaching is well explained. The \textsanskrit{Saṅgha} is practicing well.’ 

Furthermore,\marginnote{12.1} I have walked and wandered from monastery to monastery and from park to park. There I’ve seen some ascetics and brahmins who are thin, haggard, pale, and veiny—hardly a captivating sight, you’d think. It occurred to me: ‘Clearly these venerables lead the spiritual life dissatisfied, or they’re hiding some bad deed they’ve done. That’s why they’re thin, haggard, pale, and veiny—hardly a captivating sight, you’d think.’ I went up to them and said: ‘Venerables, why are you so thin, haggard, pale, and veiny—hardly a captivating sight, you’d think?’ They say: ‘We have jaundice, great king.’ But here I see mendicants always smiling and joyful, obviously happy, with cheerful faces, living relaxed, unruffled, surviving on charity, their hearts free as a wild deer. It occurred to me: ‘Clearly these venerables have realized a higher distinction in the Buddha’s instructions than they had before. That’s why these venerables are always smiling and joyful, obviously happy, with cheerful faces, living relaxed, unruffled, surviving on charity, their hearts free as a wild deer.’ So I infer this about the Buddha from the teaching: ‘The Blessed One is a fully awakened Buddha. The teaching is well explained. The \textsanskrit{Saṅgha} is practicing well.’ 

Furthermore,\marginnote{13.1} as an anointed aristocratic king I am able to execute, fine, or banish those who are guilty. Yet when I’m sitting in judgment they interrupt me. And I can’t get them to stop interrupting me and wait until I’ve finished speaking. But here I’ve seen the mendicants while the Buddha is teaching an assembly of many hundreds, and there is no sound of his disciples coughing or clearing their throats. Once it so happened that the Buddha was teaching an assembly of many hundreds. Then one of his disciples cleared their throat. And one of their spiritual companions nudged them with their knee, to indicate: ‘Hush, venerable, don’t make a sound! Our teacher, the Blessed One, is teaching!’ It occurred to me: ‘It’s incredible, it’s amazing, how an assembly can be so well trained without rod or sword!’ I don’t see any other assembly elsewhere so well trained. So I infer this about the Buddha from the teaching: ‘The Blessed One is a fully awakened Buddha. The teaching is well explained. The \textsanskrit{Saṅgha} is practicing well.’ 

Furthermore,\marginnote{14.1} I’ve seen some clever aristocrats who are subtle, accomplished in the doctrines of others, hair-splitters. You’d think they live to demolish convictions with their intellect. They hear: ‘So, gentlemen, that ascetic Gotama will come down to such and such village or town.’ They formulate a question, thinking: ‘We’ll approach the ascetic Gotama and ask him this question. If he answers like this, we’ll refute him like that; and if he answers like that, we’ll refute him like this.’ When they hear that he has come down they approach him. The Buddha educates, encourages, fires up, and inspires them with a Dhamma talk. They don’t even get around to asking their question to the Buddha, so how could they refute his answer? Invariably, they become his disciples. So I infer this about the Buddha from the teaching: ‘The Blessed One is a fully awakened Buddha. The teaching is well explained. The \textsanskrit{Saṅgha} is practicing well.’ 

Furthermore,\marginnote{15.1} I see some clever brahmins … some clever householders … some clever ascetics who are subtle, accomplished in the doctrines of others, hair-splitters. … They don’t even get around to asking their question to the Buddha, so how could they refute his answer? Invariably, they ask the ascetic Gotama for the chance to go forth. And he gives them the going-forth. Soon after going forth, living withdrawn, diligent, keen, and resolute, they realize the supreme end of the spiritual path in this very life. They live having achieved with their own insight the goal for which gentlemen rightly go forth from the lay life to homelessness. They say: ‘We were almost lost! We almost perished! For we used to claim that we were ascetics, brahmins, and perfected ones, but we were none of these things. But now we really are ascetics, brahmins, and perfected ones!’ So I infer this about the Buddha from the teaching: ‘The Blessed One is a fully awakened Buddha. The teaching is well explained. The \textsanskrit{Saṅgha} is practicing well.’ 

Furthermore,\marginnote{18.1} these chamberlains Isidatta and \textsanskrit{Purāṇa} share my meals and my carriages. I give them a livelihood and bring them renown. And yet they don’t show me the same level of devotion that they show to the Buddha. Once it so happened that while I was leading a military campaign and testing Isidatta and \textsanskrit{Purāṇa} I took up residence in a cramped house. They spent most of the night discussing the teaching, then they lay down with their heads towards where the Buddha was and their feet towards me. It occurred to me: ‘It’s incredible, it’s amazing! These chamberlains Isidatta and \textsanskrit{Purāṇa} share my meals and my carriages. I give them a livelihood and bring them renown. And yet they don’t show me the same level of devotion that they show to the Buddha. Clearly these venerables have realized a higher distinction in the Buddha’s instructions than they had before.’ So I infer this about the Buddha from the teaching: ‘The Blessed One is a fully awakened Buddha. The teaching is well explained. The \textsanskrit{Saṅgha} is practicing well.’ 

Furthermore,\marginnote{19.1} the Buddha is an aristocrat, and so am I. The Buddha is Kosalan, and so am I. The Buddha is eighty years old, and so am I. Since this is so, it’s proper for me to show the Buddha such utmost devotion and demonstrate such friendship. 

Well,\marginnote{20.1} now, sir, I must go. I have many duties, and much to do.” 

“Please,\marginnote{20.3} great king, go at your convenience.” Then King Pasenadi got up from his seat, bowed, and respectfully circled the Buddha, keeping him on his right, before leaving. 

Soon\marginnote{21.1} after the king had left, the Buddha addressed the mendicants: “Mendicants, before he got up and left, King Pasenadi spoke shrines to the teaching. Learn these shrines to the teaching! Memorize these shrines to the teaching! Remember these shrines to the teaching! These shrines to the teaching are beneficial and relate to the fundamentals of the spiritual life.” 

That\marginnote{21.7} is what the Buddha said. Satisfied, the mendicants were happy with what the Buddha said. 

%
\section*{{\suttatitleacronym MN 90}{\suttatitletranslation At Kaṇṇakatthala }{\suttatitleroot Kaṇṇakatthalasutta}}
\addcontentsline{toc}{section}{\tocacronym{MN 90} \toctranslation{At Kaṇṇakatthala } \tocroot{Kaṇṇakatthalasutta}}
\markboth{At Kaṇṇakatthala }{Kaṇṇakatthalasutta}
\extramarks{MN 90}{MN 90}

\scevam{So\marginnote{1.1} I have heard. }At one time the Buddha was staying near \textsanskrit{Ujuñña}, in the deer park at \textsanskrit{Kaṇṇakatthala}. 

Now\marginnote{2.1} at that time King Pasenadi of Kosala had arrived at \textsanskrit{Ujuñña} on some business. Then he addressed a man, “Please, mister, go to the Buddha, and in my name bow with your head to his feet. Ask him if he is healthy and well, nimble, strong, and living comfortably. And then say: ‘Sir, King Pasenadi of Kosala will come to see you today when he has finished breakfast.’” 

“Yes,\marginnote{2.7} Your Majesty,” that man replied. He did as the king asked. 

The\marginnote{3.1} sisters \textsanskrit{Somā} and \textsanskrit{Sakulā} heard this. While the meal was being served, they approached the king and said, “Great king, since you are going to the Buddha, please bow in our name with your head to his feet. Ask him if he is healthy and well, nimble, strong, and living comfortably.” 

When\marginnote{4.1} he had finished breakfast, King Pasenadi went to the Buddha, bowed, sat down to one side, and said to him, “Sir, the sisters \textsanskrit{Somā} and \textsanskrit{Sakulā} bow with their heads to your feet. They ask if you are healthy and well, nimble, strong, and living comfortably.” 

“But,\marginnote{4.3} great king, couldn’t they get any other messenger?” 

So\marginnote{4.4} Pasenadi explained the circumstances of the message. The Buddha said, “May the sisters \textsanskrit{Somā} and \textsanskrit{Sakulā} be happy, great king.” 

Then\marginnote{5.1} the king said to the Buddha, “I have heard, sir, that the ascetic Gotama says this: ‘There is no ascetic or brahmin who will claim to be all-knowing and all-seeing, to know and see everything without exception: that is not possible.’ Do those who say this repeat what the Buddha has said, and not misrepresent him with an untruth? Is their explanation in line with the teaching? Are there any legitimate grounds for rebuke and criticism?” 

“Great\marginnote{5.5} king, those who say this do not repeat what I have said. They misrepresent me with what is false and untrue.” 

Then\marginnote{6.1} King Pasenadi addressed General \textsanskrit{Viḍūḍabha}, “General, who introduced this topic of discussion to the royal compound?” 

“It\marginnote{6.3} was \textsanskrit{Sañjaya}, great king, the brahmin of the \textsanskrit{Ākāsa} clan.” 

Then\marginnote{7.1} the king addressed a man, “Please, mister, in my name tell \textsanskrit{Sañjaya} that King Pasenadi summons him.” 

“Yes,\marginnote{7.4} Your Majesty,” that man replied. He did as the king asked. 

Then\marginnote{8.1} the king said to the Buddha, “Sir, might the Buddha have spoken in reference to one thing, but that person believed it was something else? How then do you recall making this statement?” 

“Great\marginnote{8.4} king, I recall making this statement: ‘There is no ascetic or brahmin who knows all and sees all simultaneously: that is not possible.’” 

“What\marginnote{8.6} the Buddha says appears reasonable. 

Sir,\marginnote{9.1} there are these four classes: aristocrats, brahmins, merchants, and workers. Is there any difference between them?” 

“Of\marginnote{9.4} the four classes, two are said to be preeminent—the aristocrats and the brahmins. That is, when it comes to bowing down, rising up, greeting with joined palms, and observing proper etiquette.” 

“Sir,\marginnote{10.1} I am not asking you about the present life, but about the life to come.” 

“Great\marginnote{10.6} king, there are these five factors that support meditation. What five? It’s when a mendicant has faith in the Realized One’s awakening: ‘That Blessed One is perfected, a fully awakened Buddha, accomplished in knowledge and conduct, holy, knower of the world, supreme guide for those who wish to train, teacher of gods and humans, awakened, blessed.’ They are rarely ill or unwell. Their stomach digests well, being neither too hot nor too cold, but just right, and fit for meditation. They’re not devious or deceitful. They reveal themselves honestly to the Teacher or sensible spiritual companions. They live with energy roused up for giving up unskillful qualities and embracing skillful qualities. They’re strong, staunchly vigorous, not slacking off when it comes to developing skillful qualities. They’re wise. They have the wisdom of arising and passing away which is noble, penetrative, and leads to the complete ending of suffering. These are the five factors that support meditation. There are these four classes: aristocrats, brahmins, merchants, and workers. If they had these five factors that support meditation, that would be for their lasting welfare and happiness.” 

“Sir,\marginnote{11.1} there are these four classes: aristocrats, brahmins, merchants, and workers. If they had these five factors that support meditation, would there be any difference between them?” 

“In\marginnote{11.5} that case, I say it is the diversity of their efforts in meditation. Suppose there was a pair of elephants or horses or oxen in training who were well tamed and well trained. And there was a pair who were not tamed or trained. What do you think, great king? Wouldn’t the pair that was well tamed and well trained perform the tasks of the tamed, and reach the level of the tamed?” 

“Yes,\marginnote{11.9} sir.” 

“But\marginnote{11.10} would the pair that was not tamed and trained perform the tasks of the tamed and reach the level of the tamed, just like the tamed pair?” 

“No,\marginnote{11.11} sir.” 

“In\marginnote{11.12} the same way, there are things that must be attained by someone with faith, health, integrity, energy, and wisdom. It’s not possible for a faithless, unhealthy, deceitful, lazy, witless person to attain them.” 

“What\marginnote{12.1} the Buddha says appears reasonable. Sir, there are these four classes: aristocrats, brahmins, merchants, and workers. If they had these five factors that support meditation, and if they practiced rightly, would there be any difference between them?” 

“In\marginnote{12.6} that case, I say that there is no difference between the freedom of one and the freedom of the other. Suppose a person took dry teak wood and lit a fire and produced heat. Then another person did the same using \textsanskrit{sāl} wood, another used mango wood, while another used wood of the cluster fig. What do you think, great king? Would there be any difference between the fires produced by these different kinds of wood, that is, in the flame, color, or light?” 

“No,\marginnote{12.13} sir.” 

“In\marginnote{12.14} the same way, when fire has been kindled by energy and produced by effort, I say that there is no difference between the freedom of one and the freedom of the other.” 

“What\marginnote{13.1} the Buddha says appears reasonable. But sir, do gods survive?” 

“But\marginnote{13.3} what exactly are you asking?” 

“Whether\marginnote{13.5} those gods come back to this state of existence or not.” 

“Those\marginnote{13.6} gods who are subject to affliction come back to this state of existence, but those free of affliction do not come back.” 

When\marginnote{14.1} he said this, General \textsanskrit{Viḍūḍabha} said to the Buddha, “Sir, will the gods subject to affliction topple or expel from their place the gods who are free of affliction?” 

Then\marginnote{14.3} Venerable Ānanda thought, “This General \textsanskrit{Viḍūḍabha} is King Pasenadi’s son, and I am the Buddha’s son. Now is the time for one son to confer with another.” So Ānanda addressed General \textsanskrit{Viḍūḍabha}, “Well then, general, I’ll ask you about this in return, and you can answer as you like. What do you think, general? As far as the dominion of King Pasenadi of Kosala extends, where he rules as sovereign lord, can he topple or expel from that place any ascetic or brahmin, regardless of whether they are good or bad, or whether or not they are genuine spiritual practitioners?” 

“He\marginnote{14.11} can, mister.” 

“What\marginnote{14.12} do you think, general? As far as the dominion of King Pasenadi does not extend, where he does not rule as sovereign lord, can he topple or expel from that place any ascetic or brahmin, regardless of whether they are good or bad, or whether or not they are genuine spiritual practitioners?” 

“He\marginnote{14.14} cannot, mister.” 

“What\marginnote{14.15} do you think, general? Have you heard of the gods of the Thirty-Three?” 

“Yes,\marginnote{14.17} mister, I’ve heard of them, and so has the good King Pasenadi.” 

“What\marginnote{14.20} do you think, general? Can King Pasenadi topple or expel from their place the gods of the Thirty-Three?” 

“King\marginnote{14.22} Pasenadi can’t even see the gods of the Thirty-Three, so how could he possibly topple or expel them from their place?” 

“In\marginnote{14.23} the same way, general, the gods subject to affliction can’t even see the gods who are free of affliction, so how could they possibly topple or expel them from their place?” 

Then\marginnote{15.1} the king said to the Buddha, “Sir, what is this mendicant’s name?” 

“Ānanda,\marginnote{15.3} great king.” 

“A\marginnote{15.4} joy he is, and a joy he seems! What Venerable Ānanda says seems reasonable. But sir, does \textsanskrit{Brahmā} survive?” 

“But\marginnote{15.7} what exactly are you asking?” 

“Whether\marginnote{15.9} that \textsanskrit{Brahmā} comes back to this state of existence or not.” 

“Any\marginnote{15.10} \textsanskrit{Brahmā} who is subject to affliction comes back to this state of existence, but those free of affliction do not come back.” 

Then\marginnote{16.1} a certain man said to the king, “Great king, \textsanskrit{Sañjaya}, the brahmin of the \textsanskrit{Ākāsa} clan, has come.” 

Then\marginnote{16.3} King Pasenadi asked \textsanskrit{Sañjaya}, “Brahmin, who introduced this topic of discussion to the royal compound?” 

“It\marginnote{16.5} was General \textsanskrit{Viḍūḍabha}, great king.” 

But\marginnote{16.6} \textsanskrit{Viḍūḍabha} said, “It was \textsanskrit{Sañjaya}, great king, the brahmin of the \textsanskrit{Ākāsa} clan.” 

Then\marginnote{17.1} a certain man said to the king, “It’s time to depart, great king.” 

So\marginnote{17.3} the king said to the Buddha, “Sir, I asked you about omniscience, and you answered. I like and accept this, and am satisfied with it. I asked you about the four classes, about the gods, and about \textsanskrit{Brahmā}, and you answered in each case. Whatever I asked the Buddha about, he answered. I like and accept this, and am satisfied with it. Well, now, sir, I must go. I have many duties, and much to do.” 

“Please,\marginnote{17.16} great king, go at your convenience.” 

Then\marginnote{18.1} King Pasenadi approved and agreed with what the Buddha said. Then he got up from his seat, bowed, and respectfully circled the Buddha, keeping him on his right, before leaving. 

%
\addtocontents{toc}{\let\protect\contentsline\protect\nopagecontentsline}
\chapter*{The Chapter on Brahmins}
\addcontentsline{toc}{chapter}{\tocchapterline{The Chapter on Brahmins}}
\addtocontents{toc}{\let\protect\contentsline\protect\oldcontentsline}

%
\section*{{\suttatitleacronym MN 91}{\suttatitletranslation With Brahmāyu }{\suttatitleroot Brahmāyusutta}}
\addcontentsline{toc}{section}{\tocacronym{MN 91} \toctranslation{With Brahmāyu } \tocroot{Brahmāyusutta}}
\markboth{With Brahmāyu }{Brahmāyusutta}
\extramarks{MN 91}{MN 91}

So\marginnote{1.1} I have heard. At one time the Buddha was wandering in the land of the Videhans together with a large \textsanskrit{Saṅgha} of five hundred mendicants. 

Now\marginnote{2.1} at that time the brahmin \textsanskrit{Brahmāyu} was residing in \textsanskrit{Mithilā}. He was old, elderly, and senior, advanced in years, having reached the final stage of life; he was a hundred and twenty years old. He had mastered the three Vedas, together with their vocabularies, ritual, phonology and etymology, and the testament as fifth. He knew philology and grammar, and was well versed in cosmology and the marks of a great man. 

He\marginnote{3.1} heard: “It seems the ascetic Gotama—a Sakyan, gone forth from a Sakyan family—is wandering in the land of the Videhans, together with a large \textsanskrit{Saṅgha} of around five hundred mendicants. He has this good reputation: ‘That Blessed One is perfected, a fully awakened Buddha, accomplished in knowledge and conduct, holy, knower of the world, supreme guide for those who wish to train, teacher of gods and humans, awakened, blessed.’ He has realized with his own insight this world—with its gods, \textsanskrit{Māras} and \textsanskrit{Brahmās}, this population with its ascetics and brahmins, gods and humans—and he makes it known to others. He explains a teaching that’s good in the beginning, good in the middle, and good in the end, meaningful and well-phrased. And he reveals a spiritual practice that’s entirely full and pure. It’s good to see such perfected ones.” 

Now\marginnote{4.1} at that time the brahmin \textsanskrit{Brahmāyu} had a student named Uttara. He too had mastered the Vedic curriculum. \textsanskrit{Brahmāyu} told Uttara of the Buddha’s presence in the land of the Videhans, and added: “Please, dear Uttara, go to the ascetic Gotama and find out whether or not he lives up to his reputation. Through you I shall learn about Master Gotama.” 

“But\marginnote{5.1} sir, how shall I find out whether or not the ascetic Gotama lives up to his reputation?” 

“Dear\marginnote{5.3} Uttara, the thirty-two marks of a great man have been handed down in our hymns. A great man who possesses these has only two possible destinies, no other. If he stays at home he becomes a king, a wheel-turning monarch, a just and principled king. His dominion extends to all four sides, he achieves stability in the country, and he possesses the seven treasures. He has the following seven treasures: the wheel, the elephant, the horse, the jewel, the woman, the treasurer, and the counselor as the seventh treasure. He has over a thousand sons who are valiant and heroic, crushing the armies of his enemies. After conquering this land girt by sea, he reigns by principle, without rod or sword. But if he goes forth from the lay life to homelessness, he becomes a perfected one, a fully awakened Buddha, who draws back the veil from the world. But, dear Uttara, I am the one who gives the hymns, and you are the one who receives them.” 

“Yes,\marginnote{6.1} sir,” replied Uttara. He got up from his seat, bowed, and respectfully circled \textsanskrit{Brahmāyu} before setting out for the land of the Videhans where the Buddha was wandering. Traveling stage by stage, he came to the Buddha and exchanged greetings with him. When the greetings and polite conversation were over, he sat down to one side, and scrutinized his body for the thirty-two marks of a great man. He saw all of them except for two, which he had doubts about: whether the private parts are covered in a foreskin, and the largeness of the tongue. 

Then\marginnote{6.8} it occurred to the Buddha, “This brahmin student Uttara sees all the marks except for two, which he has doubts about: whether the private parts are covered in a foreskin, and the largeness of the tongue.” 

So\marginnote{7.1} the Buddha used his psychic power to will that Uttara would see his private parts covered in a foreskin. And he stuck out his tongue and stroked back and forth on his ear holes and nostrils, and covered his entire forehead with his tongue. 

Then\marginnote{8.1} Uttara thought, “The ascetic Gotama possesses the thirty-two marks. Why don’t I follow him and observe his deportment?” So Uttara followed the Buddha like a shadow for seven months. 

When\marginnote{8.5} seven months had passed he set out wandering towards \textsanskrit{Mithilā}. There he approached the brahmin \textsanskrit{Brahmāyu}, bowed, and sat down to one side. \textsanskrit{Brahmāyu} said to him, “Well, dear Uttara, does Master Gotama live up to his reputation or not?” 

“He\marginnote{8.9} does, sir. Master Gotama possesses the thirty-two marks. 

He\marginnote{9.1} has well-planted feet. 

On\marginnote{9.3} the soles of his feet there are thousand-spoked wheels, with rims and hubs, complete in every detail. 

He\marginnote{9.4} has projecting heels. 

He\marginnote{9.5} has long fingers. 

His\marginnote{9.6} hands and feet are tender. 

His\marginnote{9.7} hands and feet cling gracefully. 

His\marginnote{9.8} feet are arched. 

His\marginnote{9.9} calves are like those of an antelope. 

When\marginnote{9.10} standing upright and not bending over, the palms of both hands touch the knees. 

His\marginnote{9.11} private parts are covered in a foreskin. 

He\marginnote{9.12} is gold colored; his skin has a golden sheen. 

He\marginnote{9.13} has delicate skin, so delicate that dust and dirt don’t stick to his body. 

His\marginnote{9.14} hairs grow one per pore. 

His\marginnote{9.15} hairs stand up; they’re blue-black and curl clockwise. 

His\marginnote{9.16} body is as straight as \textsanskrit{Brahmā}’s. 

He\marginnote{9.17} has bulging muscles in seven places. 

His\marginnote{9.18} chest is like that of a lion. 

The\marginnote{9.19} gap between the shoulder-blades is filled in. 

He\marginnote{9.20} has the proportional circumference of a banyan tree: the span of his arms equals the height of his body. 

His\marginnote{9.21} torso is cylindrical. 

He\marginnote{9.22} has an excellent sense of taste. 

His\marginnote{9.23} jaw is like that of a lion. 

He\marginnote{9.24} has forty teeth. 

His\marginnote{9.25} teeth are even. 

His\marginnote{9.26} teeth have no gaps. 

His\marginnote{9.27} teeth are perfectly white. 

He\marginnote{9.28} has a large tongue. 

He\marginnote{9.29} has the voice of \textsanskrit{Brahmā}, like a cuckoo’s call. 

His\marginnote{9.30} eyes are deep blue. 

He\marginnote{9.31} has eyelashes like a cow’s. 

Between\marginnote{9.32} his eyebrows there grows a tuft, soft and white like cotton-wool. 

His\marginnote{9.33} head is shaped like a turban. 

These\marginnote{9.34} are the thirty-two marks of a great man possessed by Master Gotama. 

When\marginnote{10.1} he’s walking he takes the first step with the right foot. He doesn’t lift his foot too far or place it too near. He doesn’t walk too slow or too fast. He walks without knocking his knees or ankles together. When he’s walking he keeps his thighs neither too straight nor too bent, neither too tight nor too loose. When he walks, only the lower half of his body moves, and he walks effortlessly. When he turns to look he does so with the whole body. He doesn’t look directly up or down. He doesn’t look all around while walking, but focuses a plough’s length in front. Beyond that he has unhindered knowledge and vision. 

When\marginnote{11.1} entering an inhabited area he keeps his body neither too straight nor too bent, neither too tight nor too loose. 

He\marginnote{12.1} turns around neither too far nor too close to the seat. He doesn’t lean on his hand when sitting down. And he doesn’t just plonk his body down on the seat. When sitting in inhabited areas he doesn’t fidget with his hands or feet. He doesn’t sit with his knees or ankles crossed. He doesn’t sit with his hand holding his chin. When sitting in inhabited areas he doesn’t shake, tremble, quake, or get nervous, and so he is not nervous at all. When sitting in inhabited areas he still practices seclusion. 

When\marginnote{13.1} receiving water for rinsing the bowl, he holds the bowl neither too straight nor too bent, neither too tight nor too loose. 

He\marginnote{14.1} receives neither too little nor too much water. He rinses the bowl without making a sloshing noise, or spinning it around. He doesn’t put the bowl on the ground to rinse his hands; his hands and bowl are rinsed at the same time. He doesn’t throw the bowl rinsing water away too far or too near, or splash it about. When receiving rice, he holds the bowl neither too straight nor too bent, neither too close nor too loose. He receives neither too little nor too much rice. He eats sauce in a moderate proportion, and doesn’t spend too much time saucing his portions. He chews over each portion two or three times before swallowing. But no grain of rice enters his body unchewed, and none remain in his mouth. Only then does he raise another portion to his lips. He eats experiencing the taste, but without experiencing greed for the taste. 

He\marginnote{14.11} eats food thinking of eight reasons: ‘Not for fun, indulgence, adornment, or decoration, but only to sustain this body, to avoid harm, and to support spiritual practice. In this way, I shall put an end to old discomfort and not give rise to new discomfort, and I will live blamelessly and at ease.’ 

After\marginnote{15.1} eating, when receiving water for washing the bowl, he holds the bowl neither too straight nor too bent, neither too tight nor too loose. He receives neither too little nor too much water. He washes the bowl without making a sloshing noise, or spinning it around. He doesn’t put the bowl on the ground to wash his hands; his hands and bowl are washed at the same time. He doesn’t throw the bowl washing water away too far or too near, or splash it about. 

After\marginnote{16.1} eating he doesn’t put the bowl on the ground too far away or too close. He’s not careless with his bowl, nor does he spend too much time on it. 

After\marginnote{17.1} eating he sits for a while in silence, but doesn’t wait too long to give the verses of appreciation. After eating he expresses appreciation without criticizing the meal or expecting another one. Invariably, he educates, encourages, fires up, and inspires that assembly with a Dhamma talk. Then he gets up from his seat and leaves. 

He\marginnote{18.1} walks neither too fast nor too slow, without wanting to get out of there. 

He\marginnote{19.1} wears his robe on his body neither too high nor too low, neither too tight nor too loose. The wind doesn’t blow his robe off his body. And dust and dirt don’t stick to his body. 

When\marginnote{20.1} he has gone to the monastery he sits on a seat spread out and washes his feet. But he doesn’t waste time with pedicures. When he has washed his feet, he sits down cross-legged, with his body straight, and establishes mindfulness right there. He has no intention to hurt himself, hurt others, or hurt both. He only wishes for the welfare of himself, of others, of both, and of the whole world. In the monastery when he teaches Dhamma to an assembly, he neither flatters them nor rebukes them. Invariably, he educates, encourages, fires up, and inspires that assembly with a Dhamma talk. 

His\marginnote{21.3} voice has eight qualities: it is clear, comprehensible, charming, audible, lucid, undistorted, deep, and resonant. He makes sure his voice is intelligible as far as the assembly goes, but it doesn’t extend outside the assembly. And when they’ve been inspired with a Dhamma talk by Master Gotama they get up from their seats and leave looking back at him alone, and not forgetting their lesson. 

I\marginnote{22.1} have seen Master Gotama walking and standing; entering inhabited areas, and sitting and eating there; sitting silently after eating, and expressing appreciation; going to the monastery, sitting silently there, and teaching Dhamma to an assembly there. Such is Master Gotama; such he is and more than that.” 

When\marginnote{23.1} he had spoken, the brahmin \textsanskrit{Brahmāyu} got up from his seat, arranged his robe over one shoulder, raised his joined palms toward the Buddha, and uttered this aphorism three times: 

“Homage\marginnote{23.2} to that Blessed One, the perfected one, the fully awakened Buddha! 

Homage\marginnote{23.3} to that Blessed One, the perfected one, the fully awakened Buddha! 

Homage\marginnote{23.4} to that Blessed One, the perfected one, the fully awakened Buddha! 

Hopefully,\marginnote{23.5} some time or other I’ll get to meet him, and we can have a discussion.” 

And\marginnote{24.1} then the Buddha, traveling stage by stage in the Videhan lands, arrived at \textsanskrit{Mithilā}, where he stayed in the \textsanskrit{Makhādeva} Mango Grove. 

The\marginnote{24.3} brahmins and householders of \textsanskrit{Mithilā} heard: “It seems the ascetic Gotama—a Sakyan, gone forth from a Sakyan family—has arrived at \textsanskrit{Mithilā}, where he is staying in the \textsanskrit{Makhādeva} Mango Grove. He has this good reputation: ‘That Blessed One is perfected, a fully awakened Buddha, accomplished in knowledge and conduct, holy, knower of the world, supreme guide for those who wish to train, teacher of gods and humans, awakened, blessed.’ He has realized with his own insight this world—with its gods, \textsanskrit{Māras} and \textsanskrit{Brahmās}, this population with its ascetics and brahmins, gods and humans—and he makes it known to others. He teaches Dhamma that’s good in the beginning, good in the middle, and good in the end, meaningful and well-phrased. And he reveals a spiritual practice that’s entirely full and pure. It’s good to see such perfected ones.” 

Then\marginnote{25.1} the brahmins and householders of \textsanskrit{Mithilā} went up to the Buddha. Before sitting down to one side, some bowed, some exchanged greetings and polite conversation, some held up their joined palms toward the Buddha, some announced their name and clan, while some kept silent. 

The\marginnote{26.1} brahmin \textsanskrit{Brahmāyu} also heard that the Buddha had arrived. So he went to the \textsanskrit{Makhādeva} Mango Grove together with several disciples. 

Not\marginnote{26.3} far from the grove he thought, “It wouldn’t be appropriate for me to go to see the ascetic Gotama without first letting him know.” 

So\marginnote{26.5} he addressed one of his students: “Here, student, go to the ascetic Gotama and in my name bow with your head to his feet. Ask him if he is healthy and well, nimble, strong, and living comfortably. And then say: ‘Master Gotama, the brahmin \textsanskrit{Brahmāyu} is old, elderly, and senior, advanced in years, having reached the final stage of life; he is a hundred and twenty years old. He has mastered the three Vedas, together with their vocabularies, ritual, phonology and etymology, and the testament as fifth. He knows philology and grammar, and is well versed in cosmology and the marks of a great man. Of all the brahmins and householders residing in \textsanskrit{Mithilā}, \textsanskrit{Brahmāyu} is said to be the foremost in wealth, hymns, lifespan, and fame. He wants to see Master Gotama.’” 

“Yes,\marginnote{26.17} sir,” that student replied. He did as he was asked, and the Buddha said, “Please, student, let \textsanskrit{Brahmāyu} come when he’s ready.” 

The\marginnote{27.1} student went back to \textsanskrit{Brahmāyu} and said to him, “Your request for an audience with the ascetic Gotama has been granted. Please go at your convenience.” 

Then\marginnote{28.1} the brahmin \textsanskrit{Brahmāyu} went up to the Buddha. The assembly saw him coming off in the distance, and made way for him, as he was well-known and famous. 

\textsanskrit{Brahmāyu}\marginnote{28.4} said to that retinue, “Enough, gentlemen. Please sit on your own seats. I shall sit here by the ascetic Gotama.” 

Then\marginnote{29.1} the brahmin \textsanskrit{Brahmāyu} went up to the Buddha, and exchanged greetings with him. When the greetings and polite conversation were over, he sat down to one side, and scrutinized the Buddha’s body for the thirty-two marks of a great man. He saw all of them except for two, which he had doubts about: whether the private parts are covered in a foreskin, and the largeness of the tongue. Then \textsanskrit{Brahmāyu} addressed the Buddha in verse: 

\begin{verse}%
“I\marginnote{29.8} have learned of the thirty-two \\
marks of a great man. \\
There are two that I don’t see \\
on the body of the ascetic Gotama. 

Are\marginnote{29.12} the private parts covered in a foreskin, \\
O supreme person? \\
Though called by a word of the feminine gender, \\
perhaps your tongue is a manly one? 

Perhaps\marginnote{29.16} your tongue is large, \\
as we have been informed. \\
Please stick it out in its full extent, \\
and so, O hermit, dispel my doubt. 

For\marginnote{29.20} my welfare and benefit in this life, \\
and happiness in the next. \\
And I ask you to grant the opportunity \\
to ask whatever I desire.” 

%
\end{verse}

Then\marginnote{30.1} the Buddha thought, “\textsanskrit{Brahmāyu} sees all the marks except for two, which he has doubts about: whether the private parts are covered in a foreskin, and the largeness of the tongue.” 

So\marginnote{30.5} the Buddha used his psychic power to will that \textsanskrit{Brahmāyu} would see his private parts covered in a foreskin. And he stuck out his tongue and stroked back and forth on his ear holes and nostrils, and covered his entire forehead with his tongue. 

Then\marginnote{30.7} the Buddha replied to \textsanskrit{Brahmāyu} in verse: 

\begin{verse}%
“The\marginnote{31.1} thirty-two marks of a great man \\
that you have learned \\
are all found on my body: \\
so do not doubt, brahmin. 

I\marginnote{31.5} have known what should be known, \\
and developed what should be developed, \\
and given up what should be given up: \\
and so, brahmin, I am a Buddha. 

For\marginnote{31.9} your welfare and benefit in this life, \\
and happiness in the next: \\
I grant you the opportunity \\
to ask whatever you desire.” 

%
\end{verse}

Then\marginnote{32.1} \textsanskrit{Brahmāyu} thought: 

“My\marginnote{32.2} request has been granted. Should I ask him about what is beneficial in this life or the next?” Then he thought, “I’m well versed in the benefits that apply to this life, and others ask me about this. Why don’t I ask the ascetic Gotama about the benefit that specifically applies to lives to come?” 

So\marginnote{32.9} \textsanskrit{Brahmāyu} addressed the Buddha in verse: 

\begin{verse}%
“How\marginnote{32.10} do you become a brahmin? \\
And how do you become a knowledge master? \\
How a master of the three knowledges? \\
And how is one called a scholar? 

How\marginnote{32.14} do you become a perfected one? \\
And how a consummate one? \\
How do you become a sage? \\
And how is one declared to be awakened?” 

%
\end{verse}

Then\marginnote{33.1} the Buddha replied to \textsanskrit{Brahmāyu} in verse: 

\begin{verse}%
“One\marginnote{33.2} who knows their past lives, \\
and sees heaven and places of loss, \\
and has attained the end of rebirth: \\
that sage has perfect insight. 

They\marginnote{33.6} know their mind is pure, \\
completely freed from greed; \\
they’ve given up birth and death, \\
and have completed the spiritual journey. \\
Gone beyond all things, \\
such a one is declared to be awakened.” 

%
\end{verse}

When\marginnote{34.1} he said this, \textsanskrit{Brahmāyu} got up from his seat and arranged his robe on one shoulder. He bowed with his head at the Buddha’s feet, caressing them and covering them with kisses, and pronounced his name: “I am the brahmin \textsanskrit{Brahmāyu}, Master Gotama! I am the brahmin \textsanskrit{Brahmāyu}!” 

Then\marginnote{35.1} that assembly, their minds full of wonder and amazement, thought, “It’s incredible, it’s amazing, that \textsanskrit{Brahmāyu}, who is so well-known and famous, should show the Buddha such utmost devotion.” Then the Buddha said to \textsanskrit{Brahmāyu}, “Enough, brahmin. Get up, and sit in your own seat, since your mind has such confidence in me.” So \textsanskrit{Brahmāyu} got up and sat in his own seat. 

Then\marginnote{36.1} the Buddha taught him step by step, with a talk on giving, ethical conduct, and heaven. He explained the drawbacks of sensual pleasures, so sordid and corrupt, and the benefit of renunciation. And when the Buddha knew that \textsanskrit{Brahmāyu}’s mind was ready, pliable, rid of hindrances, elated, and confident he explained the special teaching of the Buddhas: suffering, its origin, its cessation, and the path. Just as a clean cloth rid of stains would properly absorb dye, in that very seat the stainless, immaculate vision of the Dhamma arose in the brahmin \textsanskrit{Brahmāyu}: “Everything that has a beginning has an end.” 

Then\marginnote{36.9} \textsanskrit{Brahmāyu} saw, attained, understood, and fathomed the Dhamma. He went beyond doubt, got rid of indecision, and became self-assured and independent of others regarding the Teacher’s instructions. He said to the Buddha: 

“Excellent,\marginnote{37.1} Master Gotama! Excellent! As if he were righting the overturned, or revealing the hidden, or pointing out the path to the lost, or lighting a lamp in the dark so people with good eyes can see what’s there, Master Gotama has made the teaching clear in many ways. I go for refuge to Master Gotama, to the teaching, and to the mendicant \textsanskrit{Saṅgha}. From this day forth, may Master Gotama remember me as a lay follower who has gone for refuge for life. Would you and the mendicant \textsanskrit{Saṅgha} please accept a meal from me tomorrow?” The Buddha consented in silence. Then, knowing that the Buddha had consented, \textsanskrit{Brahmāyu} got up from his seat, bowed, and respectfully circled the Buddha, keeping him on his right, before leaving. 

And\marginnote{38.1} when the night had passed \textsanskrit{Brahmāyu} had a variety of delicious foods prepared in his own home. Then he had the Buddha informed of the time, saying, “Itʼs time, Master Gotama, the meal is ready.” 

Then\marginnote{38.3} the Buddha robed up in the morning and, taking his bowl and robe, went to the home of the brahmin \textsanskrit{Brahmāyu}, where he sat on the seat spread out, together with the \textsanskrit{Saṅgha} of mendicants. For seven days, \textsanskrit{Brahmāyu} served and satisfied the mendicant \textsanskrit{Saṅgha} headed by the Buddha with his own hands with a variety of delicious foods. 

When\marginnote{39.1} the seven days had passed, the Buddha departed to wander in the Videhan lands. Not long after the Buddha left, \textsanskrit{Brahmāyu} passed away. 

Then\marginnote{39.3} several mendicants went up to the Buddha, bowed, sat down to one side, and said to him, “Sir, \textsanskrit{Brahmāyu} has passed away. Where has he been reborn in his next life?” 

“Mendicants,\marginnote{39.6} the brahmin \textsanskrit{Brahmāyu} was astute. He practiced in line with the teachings, and did not trouble me about the teachings. With the ending of the five lower fetters, he’s been reborn spontaneously and will become extinguished there, not liable to return from that world.” 

That\marginnote{39.8} is what the Buddha said. Satisfied, the mendicants were happy with what the Buddha said. 

%
\section*{{\suttatitleacronym MN 92}{\suttatitletranslation With Sela }{\suttatitleroot Selasutta}}
\addcontentsline{toc}{section}{\tocacronym{MN 92} \toctranslation{With Sela } \tocroot{Selasutta}}
\markboth{With Sela }{Selasutta}
\extramarks{MN 92}{MN 92}

\scevam{So\marginnote{1.1} I have heard. }At one time the Buddha was wandering in the land of the Northern \textsanskrit{Āpaṇas} together with a large \textsanskrit{Saṅgha} of 1,250 mendicants when he arrived at a town of the Northern \textsanskrit{Āpaṇas} named \textsanskrit{Āpaṇa}. 

The\marginnote{2.1} matted-hair ascetic \textsanskrit{Keṇiya} heard: “It seems the ascetic Gotama—a Sakyan, gone forth from a Sakyan family—has arrived at \textsanskrit{Āpaṇa}, together with a large \textsanskrit{Saṅgha} of 1,250 mendicants. He has this good reputation: ‘That Blessed One is perfected, a fully awakened Buddha, accomplished in knowledge and conduct, holy, knower of the world, supreme guide for those who wish to train, teacher of gods and humans, awakened, blessed.’ He has realized with his own insight this world—with its gods, \textsanskrit{Māras} and \textsanskrit{Brahmās}, this population with its ascetics and brahmins, gods and humans—and he makes it known to others. He teaches Dhamma that’s good in the beginning, good in the middle, and good in the end, meaningful and well-phrased. And he reveals a spiritual practice that’s entirely full and pure. It’s good to see such perfected ones.” 

So\marginnote{3.1} \textsanskrit{Keṇiya} approached the Buddha and exchanged greetings with him. When the greetings and polite conversation were over, he sat down to one side. The Buddha educated, encouraged, fired up, and inspired him with a Dhamma talk. 

Then\marginnote{3.4} he said to the Buddha, “Would Master Gotama together with the mendicant \textsanskrit{Saṅgha} please accept tomorrow’s meal from me?” 

When\marginnote{3.6} he said this, the Buddha said to him, “The \textsanskrit{Saṅgha} is large, \textsanskrit{Keṇiya}; there are 1,250 mendicants. And you are devoted to the brahmins.” 

For\marginnote{3.8} a second time, \textsanskrit{Keṇiya} asked the Buddha to accept a meal offering. “Never mind that the \textsanskrit{Saṅgha} is large, with 1,250 mendicants, and that I am devoted to the brahmins. Would Master Gotama together with the mendicant \textsanskrit{Saṅgha} please accept tomorrow’s meal from me?” And for a second time, the Buddha gave the same reply. For a third time, \textsanskrit{Keṇiya} asked the Buddha to accept a meal offering. “Never mind that the \textsanskrit{Saṅgha} is large, with 1,250 mendicants, and that I am devoted to the brahmins. Would Master Gotama together with the mendicant \textsanskrit{Saṅgha} please accept tomorrow’s meal from me?” The Buddha consented in silence. 

Then,\marginnote{4.1} knowing that the Buddha had consented, \textsanskrit{Keṇiya} got up from his seat and went to his own hermitage. There he addressed his friends and colleagues, relatives and family members, “My friends and colleagues, relatives and family members: please listen! The ascetic Gotama together with the mendicant \textsanskrit{Saṅgha} has been invited by me for tomorrow’s meal. Please help me with the preparations.” 

“Yes,\marginnote{4.5} sir,” they replied. Some dug ovens, some chopped wood, some washed dishes, some set out a water jar, and some spread out seats. Meanwhile, \textsanskrit{Keṇiya} set up the pavilion himself. 

Now\marginnote{5.1} at that time the brahmin Sela was residing in \textsanskrit{Āpaṇa}. He had mastered the three Vedas, together with their vocabularies, ritual, phonology and etymology, and the testament as fifth. He knew philology and grammar, and was well versed in cosmology and the marks of a great man. And he was teaching three hundred students to recite the hymns. 

And\marginnote{6.1} at that time \textsanskrit{Keṇiya} was devoted to Sela. Then Sela, while going for a walk escorted by the three hundred students, approached \textsanskrit{Keṇiya}’s hermitage. He saw the preparations going on, and said to \textsanskrit{Keṇiya}, “\textsanskrit{Keṇiya}, is your son or daughter being married? Or are you setting up a big sacrifice? Or has King Seniya \textsanskrit{Bimbisāra} of Magadha been invited for tomorrow’s meal?” 

“There\marginnote{8.1} is no marriage, Sela, and the king is not coming. Rather, I am setting up a big sacrifice. The ascetic Gotama has arrived at \textsanskrit{Āpaṇa}, together with a large \textsanskrit{Saṅgha} of 1,250 mendicants. He has this good reputation: ‘That Blessed One is perfected, a fully awakened Buddha, accomplished in knowledge and conduct, holy, knower of the world, supreme guide for those who wish to train, teacher of gods and humans, awakened, blessed.’ He has been invited by me for tomorrow’s meal together with the mendicant \textsanskrit{Saṅgha}.” 

“Mister\marginnote{9.1} \textsanskrit{Keṇiya}, did you say ‘the awakened one’?” 

“I\marginnote{9.2} said ‘the awakened one’.” 

“Did\marginnote{9.3} you say ‘the awakened one’?” 

“I\marginnote{9.4} said ‘the awakened one’.” 

Then\marginnote{10.1} Sela thought, “It’s hard to even find the word ‘awakened one’ in the world. The thirty-two marks of a great man have been handed down in our hymns. A great man who possesses these has only two possible destinies, no other. If he stays at home he becomes a king, a wheel-turning monarch, a just and principled king. His dominion extends to all four sides, he achieves stability in the country, and he possesses the seven treasures. He has the following seven treasures: the wheel, the elephant, the horse, the jewel, the woman, the treasurer, and the counselor as the seventh treasure. He has over a thousand sons who are valiant and heroic, crushing the armies of his enemies. After conquering this land girt by sea, he reigns by principle, without rod or sword. But if he goes forth from the lay life to homelessness, he becomes a perfected one, a fully awakened Buddha, who draws back the veil from the world.” 

“But\marginnote{11.1} \textsanskrit{Keṇiya}, where is the Blessed One at present, the perfected one, the fully awakened Buddha?” 

When\marginnote{11.2} he said this, \textsanskrit{Keṇiya} pointed with his right arm and said, “There, Mister Sela, at that line of blue forest.” 

Then\marginnote{12.1} Sela, together with his students, approached the Buddha. He said to his students, “Come quietly, gentlemen, tread gently. For the Buddhas are intimidating, like a lion living alone. When I’m consulting with the ascetic Gotama, don’t interrupt. Wait until I’ve finished speaking.” 

Then\marginnote{13.1} Sela went up to the Buddha, and exchanged greetings with him. When the greetings and polite conversation were over, he sat down to one side, and scrutinized the Buddha’s body for the thirty-two marks of a great man. 

He\marginnote{13.4} saw all of them except for two, which he had doubts about: whether the private parts are covered in a foreskin, and the largeness of the tongue. 

Then\marginnote{13.7} it occurred to the Buddha, “Sela sees all the marks except for two, which he has doubts about: whether the private parts are covered in a foreskin, and the largeness of the tongue.” 

The\marginnote{14.1} Buddha used his psychic power to will that Sela would see his private parts covered in a foreskin. And he stuck out his tongue and stroked back and forth on his ear holes and nostrils, and covered his entire forehead with his tongue. 

Then\marginnote{15.1} Sela thought, “The ascetic Gotama possesses the thirty-two marks completely, lacking none. But I don’t know whether or not he is an awakened one. I have heard that brahmins of the past who were elderly and senior, the teachers of teachers, said, ‘Those who are perfected ones, fully awakened Buddhas reveal themselves when praised.’ Why don’t I extoll him in his presence with fitting verses?” 

Then\marginnote{15.7} Sela extolled the Buddha in his presence with fitting verses: 

\begin{verse}%
“O\marginnote{16.1} Blessed One, your body’s perfect, \\
you’re radiant, handsome, lovely to behold; \\
golden colored, \\
with teeth so white; you’re strong. 

The\marginnote{16.5} characteristics \\
of a handsome man, \\
the marks of a great man, \\
are all found on your body. 

Your\marginnote{16.9} eyes are clear, your face is fair, \\
you’re formidable, upright, majestic. \\
In the midst of the \textsanskrit{Saṅgha} of ascetics, \\
you shine like the sun. 

You’re\marginnote{16.13} a mendicant fine to see, \\
with skin of golden sheen. \\
But with such excellent appearance, \\
what do you want with the ascetic life? 

You’re\marginnote{16.17} fit to be a king, \\
a wheel-turning monarch, chief of charioteers, \\
victorious in the four quarters, \\
lord of all India. 

Aristocrats,\marginnote{16.21} nobles, and kings \\
ought follow your rule. \\
Gotama, you should reign \\
as king of kings, lord of men!” 

“I\marginnote{17.1} am a king, Sela, \\
the supreme king of the teaching. \\
By the teaching I roll forth the wheel \\
which cannot be rolled back.” 

“You\marginnote{18.1} claim to be awakened, \\
the supreme king of the teaching. \\
‘I roll forth the teaching’: \\
so you say, Gotama. 

Then\marginnote{18.5} who is your general, \\
the disciple who follows the Teacher’s way? \\
Who keeps rolling the wheel \\
of teaching you rolled forth?” 

“By\marginnote{19.1} me the wheel was rolled forth,” \\
\scspeaker{said the Buddha, }\\
“the supreme wheel of teaching. \\
\textsanskrit{Sāriputta}, taking after the Realized One, \\
keeps it rolling on. 

I\marginnote{19.6} have known what should be known, \\
and developed what should be developed, \\
and given up what should be given up: \\
and so, brahmin, I am a Buddha. 

Dispel\marginnote{19.10} your doubt in me—\\
make up your mind, brahmin! \\
The sight of a Buddha \\
is hard to find again. 

I\marginnote{19.14} am a Buddha, brahmin, \\
the supreme surgeon, \\
one of those whose appearance in the world \\
is hard to find again. 

Holy,\marginnote{19.18} unequaled, \\
crusher of \textsanskrit{Māra}’s army; \\
having subdued all my opponents, \\
I rejoice, fearing nothing from any quarter.” 

“Pay\marginnote{20.1} heed, sirs, to what \\
is spoken by the seer. \\
The surgeon, the great hero, \\
roars like a lion in the jungle. 

Holy,\marginnote{20.5} unequaled, \\
crusher of \textsanskrit{Māra}’s army; \\
who would not be inspired by him, \\
even one whose nature is dark? 

Those\marginnote{20.9} who wish may follow me; \\
those who don’t may go. \\
Right here, I’ll go forth in his presence, \\
the one of such splendid wisdom.” 

“Sir,\marginnote{21.1} if you like \\
the teaching of the Buddha, \\
we’ll also go forth in his presence, \\
the one of such splendid wisdom.” 

“These\marginnote{22.1} three hundred brahmins \\
with joined palms held up, ask: \\
‘May we lead the spiritual life \\
in your presence, Blessed One?’” 

“The\marginnote{23.1} spiritual life is well explained,” \\
\scspeaker{said the Buddha, }\\
“visible in this very life, immediately effective. \\
Here the going forth isn’t in vain \\
for one who trains with diligence.” 

%
\end{verse}

And\marginnote{24.1} the brahmin Sela together with his assembly received the going forth, the ordination in the Buddha’s presence. 

And\marginnote{25.1} when the night had passed \textsanskrit{Keṇiya} had a variety of delicious foods prepared in his own home. Then he had the Buddha informed of the time, saying, “Itʼs time, Master Gotama, the meal is ready.” 

Then\marginnote{25.3} the Buddha robed up in the morning and, taking his bowl and robe, went to \textsanskrit{Keṇiya}’s hermitage, where he sat on the seat spread out, together with the \textsanskrit{Saṅgha} of mendicants. Then \textsanskrit{Keṇiya} served and satisfied the mendicant \textsanskrit{Saṅgha} headed by the Buddha with his own hands with a variety of delicious foods. When the Buddha had eaten and washed his hand and bowl, \textsanskrit{Keṇiya} took a low seat and sat to one side. The Buddha expressed his appreciation with these verses: 

\begin{verse}%
“The\marginnote{26.1} foremost of sacrifices is offering to the sacred flame; \\
the \textsanskrit{Gāyatrī} Mantra is the foremost of poetic meters; \\
of humans, the king is the foremost; \\
the ocean’s the foremost of rivers; 

the\marginnote{26.5} foremost of stars is the moon; \\
the sun is the foremost of lights; \\
for those who sacrifice seeking merit, \\
the \textsanskrit{Saṅgha} is the foremost.” 

%
\end{verse}

When\marginnote{26.9} the Buddha had expressed his appreciation to \textsanskrit{Keṇiya} the matted-hair ascetic with these verses, he got up from his seat and left. 

Then\marginnote{27.1} Venerable Sela and his assembly, living alone, withdrawn, diligent, keen, and resolute, soon realized the supreme end of the spiritual path in this very life. They lived having achieved with their own insight the goal for which gentlemen rightly go forth from the lay life to homelessness. 

They\marginnote{27.2} understood: “Rebirth is ended; the spiritual journey has been completed; what had to be done has been done; there is no return to any state of existence.” And Venerable Sela together with his assembly became perfected. 

Then\marginnote{28.1} Sela with his assembly went to see the Buddha. He arranged his robe over one shoulder, raised his joined palms toward the Buddha, and said: 

\begin{verse}%
“This\marginnote{28.2} is the eighth day since \\
we went for refuge, O seer. \\
In these seven days, Blessed One, \\
we’ve become tamed in your teaching. 

You\marginnote{28.6} are the Buddha, you are the Teacher, \\
you are the sage who has overcome \textsanskrit{Māra}; \\
you have cut off the underlying tendencies, \\
you’ve crossed over, and you bring humanity across. 

You\marginnote{28.10} have transcended attachments, \\
your defilements are shattered; \\
by not grasping, like a lion, \\
you’ve given up fear and dread. 

These\marginnote{28.14} three hundred mendicants \\
stand with joined palms raised. \\
Stretch out your feet, great hero: \\
let these giants bow to the Teacher.” 

%
\end{verse}

%
\section*{{\suttatitleacronym MN 93}{\suttatitletranslation With Assalāyana }{\suttatitleroot Assalāyanasutta}}
\addcontentsline{toc}{section}{\tocacronym{MN 93} \toctranslation{With Assalāyana } \tocroot{Assalāyanasutta}}
\markboth{With Assalāyana }{Assalāyanasutta}
\extramarks{MN 93}{MN 93}

\scevam{So\marginnote{1.1} I have heard. }At one time the Buddha was staying near \textsanskrit{Sāvatthī} in Jeta’s Grove, \textsanskrit{Anāthapiṇḍika}’s monastery. 

Now\marginnote{2.1} at that time around five hundred brahmins from abroad were residing in \textsanskrit{Sāvatthī} on some business. Then those brahmins thought, “This ascetic Gotama advocates purification for all four classes. Who is capable of having a dialogue with him about this?” 

Now\marginnote{3.1} at that time the brahmin student \textsanskrit{Assalāyana} was residing in \textsanskrit{Sāvatthī}. He was young, newly tonsured; he was sixteen years old. He had mastered the three Vedas, together with their vocabularies, ritual, phonology and etymology, and the testament as fifth. He knew philology and grammar, and was well versed in cosmology and the marks of a great man. 

Then\marginnote{3.2} those brahmins thought, “This \textsanskrit{Assalāyana} is capable of having a dialogue with the ascetic Gotama about this.” 

So\marginnote{4.1} they approached \textsanskrit{Assalāyana} and said to him, “This ascetic Gotama advocates purification for all four classes. Please, Mister \textsanskrit{Assalāyana}, have a dialogue with the ascetic Gotama about this.” 

When\marginnote{4.4} they said this, \textsanskrit{Assalāyana} said to them, “They say that the ascetic Gotama is a speaker of principle. But speakers of principle are hard to have a dialogue with. I’m not capable of having a dialogue with the ascetic Gotama about this.” 

For\marginnote{4.8} a second time, those brahmins said to him “This ascetic Gotama advocates purification for all four classes. Please, Mister \textsanskrit{Assalāyana}, have a dialogue with the ascetic Gotama about this. For you have lived as a wanderer.” And for a second time, \textsanskrit{Assalāyana} refused. 

For\marginnote{4.16} a third time, those brahmins said to him, “This ascetic Gotama advocates purification for all four classes. Please, Mister \textsanskrit{Assalāyana}, have a dialogue with the ascetic Gotama about this. For you have lived as a wanderer. Don’t admit defeat before going into battle!” 

When\marginnote{4.21} they said this, \textsanskrit{Assalāyana} said to them, “Clearly, gentlemen, I’m not getting through to you when I say: ‘They say that the ascetic Gotama is a speaker of principle. But speakers of principle are hard to have a dialogue with. I’m not capable of having a dialogue with the ascetic Gotama about this.’ Nevertheless, I shall go at your bidding.” 

Then\marginnote{5.1} \textsanskrit{Assalāyana} together with a large group of brahmins went to the Buddha and exchanged greetings with him. When the greetings and polite conversation were over, he sat down to one side and said to the Buddha: 

“Master\marginnote{5.3} Gotama, the brahmins say: ‘Only brahmins are the highest caste; other castes are inferior. Only brahmins are the light caste; other castes are dark. Only brahmins are purified, not others. Only brahmins are \textsanskrit{Brahmā}’s rightful sons, born of his mouth, born of \textsanskrit{Brahmā}, created by \textsanskrit{Brahmā}, heirs of \textsanskrit{Brahmā}.’ What do you say about this?” 

“But\marginnote{5.9} \textsanskrit{Assalāyana}, brahmin women are seen menstruating, being pregnant, giving birth, and breastfeeding. Yet even though they’re born from a brahmin womb they say: ‘Only brahmins are the highest caste; other castes are inferior. Only brahmins are the light caste; other castes are dark. Only brahmins are purified, not others. Only brahmins are \textsanskrit{Brahmā}’s rightful sons, born of his mouth, born of \textsanskrit{Brahmā}, created by \textsanskrit{Brahmā}, heirs of \textsanskrit{Brahmā}.’” 

“Even\marginnote{6.1} though you say this, still the brahmins maintain their belief.” 

“What\marginnote{6.4} do you think, \textsanskrit{Assalāyana}? Have you heard that in Greece and Persia and other foreign lands there are only two classes, masters and bonded servants; and that masters may become servants, and servants masters?” 

“Yes,\marginnote{6.6} I have heard that.” 

“Then\marginnote{6.7} what is the source of the brahmins’ self-confidence and forcefulness in this matter that they make this claim?” 

“Even\marginnote{7.1} though you say this, still the brahmins maintain their belief.” 

“What\marginnote{7.4} do you think, \textsanskrit{Assalāyana}? Suppose an aristocrat were to kill living creatures, steal, and commit sexual misconduct; to use speech that’s false, divisive, harsh, or nonsensical; and to be covetous, malicious, with wrong view. When their body breaks up, after death, they’d be reborn in a place of loss, a bad place, the underworld, hell. Would this happen only to an aristocrat, and not to a brahmin? Or suppose a merchant, or a worker were to act in the same way. Would that result befall only a merchant or a worker, and not to a brahmin?” 

“No,\marginnote{7.8} Master Gotama. If they acted the same way, the same result would befall an aristocrat, a brahmin, a merchant, or a worker. For if any of the four classes were to kill living creatures, steal, and commit sexual misconduct; to use speech that’s false, divisive, harsh, or nonsensical; and to be covetous, malicious, with wrong view, then, when their body breaks up, after death, they’d be reborn in a place of loss, a bad place, the underworld, hell.” 

“Then\marginnote{7.14} what is the source of the brahmins’ self-confidence and forcefulness in this matter that they make this claim?” 

“Even\marginnote{8.1} though you say this, still the brahmins maintain their belief.” 

“What\marginnote{8.4} do you think, \textsanskrit{Assalāyana}? Suppose a brahmin were to refrain from killing living creatures, stealing, and committing sexual misconduct; from using speech that’s false, divisive, harsh, or nonsensical; and from covetousness, malice, and wrong view. When their body breaks up, after death, they’d be reborn in a good place, a heavenly realm. Would this happen only to an brahmin, and not to an aristocrat, a merchant, or a worker?” 

“No,\marginnote{8.6} Master Gotama. If they acted the same way, the same result would befall an aristocrat, a brahmin, a merchant, or a worker. For if any of the four classes were to refrain from killing living creatures, stealing, and committing sexual misconduct; from using speech that’s false, divisive, harsh, or nonsensical; and from covetousness, malice, and wrong view, then, when their body breaks up, after death, they’d be reborn in a good place, a heavenly realm.” 

“Then\marginnote{8.12} what is the source of the brahmins’ self-confidence and forcefulness in this matter that they make this claim?” 

“Even\marginnote{9.1} though you say this, still the brahmins maintain their belief.” 

“What\marginnote{9.4} do you think, \textsanskrit{Assalāyana}? Is only a brahmin capable of developing a heart of love, free of enmity and ill will for this region, and not an aristocrat, merchant, or worker?” 

“No,\marginnote{9.6} Master Gotama. Aristocrats, brahmins, merchants, and workers can all do so. For all four classes are capable of developing a heart of love, free of enmity and ill will for this region.” 

“Then\marginnote{9.12} what is the source of the brahmins’ self-confidence and forcefulness in this matter that they make this claim?” 

“Even\marginnote{10.1} though you say this, still the brahmins maintain their belief.” 

“What\marginnote{10.4} do you think, \textsanskrit{Assalāyana}? Is only a brahmin capable of taking some bathing paste of powdered shell, going to the river, and washing off dust and dirt, and not an aristocrat, merchant, or worker?” 

“No,\marginnote{10.6} Master Gotama. All four classes are capable of doing this.” 

“Then\marginnote{10.11} what is the source of the brahmins’ self-confidence and forcefulness in this matter that they make this claim?” 

“Even\marginnote{11.1} though you say this, still the brahmins maintain their belief.” 

“What\marginnote{11.4} do you think, \textsanskrit{Assalāyana}? Suppose an anointed aristocratic king were to gather a hundred people born in different castes and say to them: ‘Please gentlemen, let anyone here who was born in a family of aristocrats, brahmins, or chieftains take a drill-stick made of teak, sal, frankincense wood, sandalwood, or cherry wood, light a fire and produce heat. And let anyone here who was born in a family of outcastes, hunters, bamboo-workers, chariot-makers, or waste-collectors take a drill-stick made from a dog’s drinking trough, a pig’s trough, a dustbin, or castor-oil wood, light a fire and produce heat.’ 

What\marginnote{11.8} do you think, \textsanskrit{Assalāyana}? Would only the fire produced by the high class people with good quality wood have flames, color, and radiance, and be usable as fire, and not the fire produced by the low class people with poor quality wood?” 

“No,\marginnote{11.11} Master Gotama. The fire produced by the high class people with good quality wood would have flames, color, and radiance, and be usable as fire, and so would the fire produced by the low class people with poor quality wood. For all fire has flames, color, and radiance, and is usable as fire.” 

“Then\marginnote{11.15} what is the source of the brahmins’ self-confidence and forcefulness in this matter that they make this claim?” 

“Even\marginnote{12.1} though you say this, still the brahmins maintain their belief.” 

“What\marginnote{12.4} do you think, \textsanskrit{Assalāyana}? Suppose an aristocrat boy was to sleep with a brahmin girl, and they had a child. Would that child be called an aristocrat after the father or a brahmin after the mother?” 

“They\marginnote{12.7} could be called either.” 

“What\marginnote{13.1} do you think, \textsanskrit{Assalāyana}? Suppose a brahmin boy was to sleep with an aristocrat girl, and they had a child. Would that child be called an aristocrat after the mother or a brahmin after the father?” 

“They\marginnote{13.4} could be called either.” 

“What\marginnote{14.1} do you think, \textsanskrit{Assalāyana}? Suppose a mare were to mate with a donkey, and she gave birth to a mule. Would that mule be called a horse after the mother or a donkey after the father?” 

“It’s\marginnote{14.4} a mule, as it is a crossbreed. I see the difference in this case, but not in the previous cases.” 

“What\marginnote{15.1} do you think, \textsanskrit{Assalāyana}? Suppose there were two brahmin students who were brothers who had shared a womb. One was an educated reciter, while the other was not an educated reciter. Who would the brahmins feed first at an offering of food for ancestors, an offering of a dish of milk-rice, a sacrifice, or a feast for guests?” 

“They’d\marginnote{15.4} first feed the student who was an educated reciter. For how could an offering to someone who not an educated reciter be very fruitful?” 

“What\marginnote{16.1} do you think, \textsanskrit{Assalāyana}? Suppose there were two brahmin students who were brothers who had shared a womb. One was an educated reciter, but was unethical, of bad character, while the other was not an educated reciter, but was ethical and of good character. Who would the brahmins feed first?” 

“They’d\marginnote{16.4} first feed the student who was not an educated reciter, but was ethical and of good character. For how could an offering to someone who is unethical and of bad character be very fruitful?” 

“Firstly\marginnote{17.1} you relied on birth, \textsanskrit{Assalāyana}, then you switched to education, then you switched to abstemious behavior. Now you’ve come around to believing in purification for the four classes, just as I advocate.” When he said this, \textsanskrit{Assalāyana} sat silent, embarrassed, shoulders drooping, downcast, depressed, with nothing to say. 

Knowing\marginnote{18.1} this, the Buddha said to him: 

“Once\marginnote{18.2} upon a time, \textsanskrit{Assalāyana}, seven brahmin hermits settled in leaf huts in a wilderness region. They had the following harmful misconception: ‘Only brahmins are the highest caste; other castes are inferior. Only brahmins are the light caste; other castes are dark. Only brahmins are purified, not others. Only brahmins are \textsanskrit{Brahmā}’s rightful sons, born of his mouth, born of \textsanskrit{Brahmā}, created by \textsanskrit{Brahmā}, heirs of \textsanskrit{Brahmā}.’ 

The\marginnote{18.6} hermit Devala the Dark heard about this. So he did up his hair and beard, dressed in magenta robes, put on his boots, grasped a golden staff, and appeared in the courtyard of the seven brahmin hermits. Then he wandered about the yard saying, ‘Where, oh where have those brahmin hermits gone? Where, oh where have those brahmin hermits gone?’ 

Then\marginnote{18.14} those brahmin hermits said, ‘Who’s this wandering about our courtyard like a cowpoke? Let’s curse him!’ 

So\marginnote{18.19} they cursed Devala the Dark, ‘Be ashes, lowlife! Be ashes, lowlife!’ But the more the hermits cursed him, the more attractive, good-looking, and lovely Devala the Dark became. 

Then\marginnote{18.23} those brahmin hermits said, ‘Our austerities are in vain! Our spiritual path is fruitless! For when we used to curse someone to become ashes, ashes they became. But the more we curse this one, the more attractive, good-looking, and lovely he becomes.’ 

‘Gentlemen,\marginnote{18.29} your austerities are not in vain; your spiritual path is not fruitless. Please let go of your malevolence towards me.’ 

‘We\marginnote{18.31} let go of our malevolence towards you. But who are you, sir?’ 

‘Have\marginnote{18.33} you heard of the hermit Devala the Dark?’ 

‘Yes,\marginnote{18.35} sir.’ 

‘I\marginnote{18.36} am he, sirs.’ Then they approached Devala and bowed to him. 

Devala\marginnote{18.38} said to them, ‘I heard that when the seven brahmin hermits had settled in leaf huts in a wilderness region, they had the following harmful misconception: “Only brahmins are the highest caste; other castes are inferior. Only brahmins are the light caste; other castes are dark. Only brahmins are purified, not others. Only brahmins are \textsanskrit{Brahmā}’s rightful sons, born of his mouth, born of \textsanskrit{Brahmā}, created by \textsanskrit{Brahmā}, heirs of \textsanskrit{Brahmā}.”’ 

‘That’s\marginnote{18.44} right, sir.’ 

‘But\marginnote{18.45} do you know whether your birth mother only had relations with a brahmin and not with a non-brahmin?’ 

‘We\marginnote{18.47} don’t know that.’ 

‘But\marginnote{18.48} do you know whether your birth mother’s mothers back to the seventh generation only had relations with brahmins and not with non-brahmins?’ 

‘We\marginnote{18.50} don’t know that.’ 

‘But\marginnote{18.51} do you know whether your birth father only had relations with a brahmin woman and not with a non-brahmin?’ 

‘We\marginnote{18.53} don’t know that.’ 

‘But\marginnote{18.54} do you know whether your birth father’s fathers back to the seventh generation only had relations with brahmins and not with non-brahmins?’ 

‘We\marginnote{18.56} don’t know that.’ 

‘But\marginnote{18.57} do you know how an embryo is conceived?’ 

‘We\marginnote{18.59} do know that, sir. An embryo is conceived when these three things come together—the mother and father come together, the mother is in the fertile part of her menstrual cycle, and the spirit being reborn is present.’ 

‘But\marginnote{18.62} do you know for sure whether that spirit is an aristocrat, a brahmin, a merchant, or a worker?’ 

‘We\marginnote{18.64} don’t know that.’ 

‘In\marginnote{18.66} that case, sirs, don’t you know what you are?’ 

‘In\marginnote{18.68} that case, sir, we don’t know what we are.’ 

So\marginnote{18.70} even those seven brahmin hermits were stumped when pursued, pressed, and grilled by the seer Devala on their own doctrine of ancestry. So how could you succeed, being grilled by me now on your own doctrine of ancestry—you who have not even mastered your own tradition?” 

When\marginnote{19.1} he had spoken, \textsanskrit{Assalāyana} said to him, “Excellent, Master Gotama! … From this day forth, may Master Gotama remember me as a lay follower who has gone for refuge for life.” 

%
\section*{{\suttatitleacronym MN 94}{\suttatitletranslation With Ghoṭamukha }{\suttatitleroot Ghoṭamukhasutta}}
\addcontentsline{toc}{section}{\tocacronym{MN 94} \toctranslation{With Ghoṭamukha } \tocroot{Ghoṭamukhasutta}}
\markboth{With Ghoṭamukha }{Ghoṭamukhasutta}
\extramarks{MN 94}{MN 94}

\scevam{So\marginnote{1.1} I have heard. }At one time Venerable Udena was staying near Benares in the Khemiya Mango Grove. 

Now\marginnote{2.1} at that time the brahmin \textsanskrit{Ghoṭamukha} had arrived at Benares on some business. Then as he was going for a walk he went to the Khemiya Mango Grove. At that time Venerable Udena was walking mindfully in the open air. \textsanskrit{Ghoṭamukha} approached and exchanged greetings with him. 

Walking\marginnote{2.5} alongside Udena, he said, “Mister ascetic, there is no such thing as a principled renunciate life; that’s what I think. And that’s without seeing gentlemen such as yourself, or a relevant teaching.” 

When\marginnote{3.1} he said this, Udena stepped down from the walking path, entered his dwelling, and sat down on the seat spread out. \textsanskrit{Ghoṭamukha} also stepped down from the walking path and entered the dwelling, where he stood to one side. Udena said to him, “There are seats, brahmin. Please sit if you wish.” 

“I\marginnote{3.6} was just waiting for you to sit down. For how could one such as I presume to sit first without being invited?” 

Then\marginnote{4.1} he took a low seat and sat to one side, where he said, “Mister ascetic, there is no such thing as a principled renunciate life; that’s what I think. And that’s without seeing gentlemen such as yourself, or a relevant teaching.” 

“Brahmin,\marginnote{4.6} we can discuss this. But only if you allow what should be allowed, and reject what should be rejected. And if you ask me the meaning of anything you don’t understand, saying: ‘Sir, why is this? What does that mean?’” 

“Let\marginnote{4.8} us discuss this. I will do as you say.” 

“Brahmin,\marginnote{5.1} these four people are found in the world. What four? 

\begin{enumerate}%
\item One person mortifies themselves, committed to the practice of mortifying themselves. %
\item One person mortifies others, committed to the practice of mortifying others. %
\item One person mortifies themselves and others, committed to the practice of mortifying themselves and others. %
\item One person doesn’t mortify either themselves or others, committed to the practice of not mortifying themselves or others. They live without wishes in the present life, extinguished, cooled, experiencing bliss, having become holy in themselves. %
\end{enumerate}

Which\marginnote{5.8} one of these four people do you like the sound of?” 

“Sir,\marginnote{5.9} I don’t like the sound of the first three people. I only like the sound of the last person, who doesn’t mortify either themselves or others.” 

“But\marginnote{6.1} why don’t you like the sound of those three people?” 

“Sir,\marginnote{6.2} the person who mortifies themselves does so even though they want to be happy and recoil from pain. That’s why I don’t like the sound of that person. The person who mortifies others does so even though others want to be happy and recoil from pain. That’s why I don’t like the sound of that person. The person who mortifies themselves and others does so even though both themselves and others want to be happy and recoil from pain. That’s why I don’t like the sound of that person. The person who doesn’t mortify either themselves or others—living without wishes, extinguished, cooled, experiencing bliss, having become holy in themselves—does not torment themselves or others, both of whom want to be happy and recoil from pain. That’s why I like the sound of that person.” 

“There\marginnote{7.1} are, brahmin, these two groups of people. What two? There’s one group of people who, being infatuated with jeweled earrings, seeks partners and children, male and female bondservants, fields and lands, and gold and money. 

And\marginnote{7.4} there’s another group of people who, not being infatuated with jeweled earrings, has given up partner and children, male and female bondservants, fields and lands, and gold and money, and goes forth from the lay life to homelessness. 

Now,\marginnote{7.5} brahmin, that person who doesn’t mortify either themselves or others—in which of these two groups of people do you usually find such a person?” 

“I\marginnote{7.9} usually find such a person in the group that has gone forth from the lay life to homelessness.” 

“Just\marginnote{8.1} now I understood you to say: ‘Mister ascetic, there is no such thing as a principled renunciate life; that’s what I think. And that’s without seeing gentlemen such as yourself, nor a relevant teaching.’” 

“Well,\marginnote{8.5} I obviously had my reasons for saying that, master Udena. But there is such a thing as a principled renunciate life; that’s what I think. Please remember me as saying this. Now, these four kinds of people that you’ve spoken of in a brief summary: please explain them to me in detail, out of compassion.” 

“Well\marginnote{9.1} then, brahmin, listen and pay close attention, I will speak.” 

“Yes,\marginnote{9.2} sir,” replied \textsanskrit{Ghoṭamukha}. Udena said this: 

“What\marginnote{10.1} person mortifies themselves, committed to the practice of mortifying themselves? It’s when someone goes naked, ignoring conventions. They lick their hands, and don’t come or wait when called. They don’t consent to food brought to them, or food prepared on purpose for them, or an invitation for a meal. They don’t receive anything from a pot or bowl; or from someone who keeps sheep, or who has a weapon or a shovel in their home; or where a couple is eating; or where there is a woman who is pregnant, breastfeeding, or who has a man in her home; or where there’s a dog waiting or flies buzzing. They accept no fish or meat or liquor or wine, and drink no beer. They go to just one house for alms, taking just one mouthful, or two houses and two mouthfuls, up to seven houses and seven mouthfuls. They feed on one saucer a day, two saucers a day, up to seven saucers a day. They eat once a day, once every second day, up to once a week, and so on, even up to once a fortnight. They live committed to the practice of eating food at set intervals. They eat herbs, millet, wild rice, poor rice, water lettuce, rice bran, scum from boiling rice, sesame flour, grass, or cow dung. They survive on forest roots and fruits, or eating fallen fruit. They wear robes of sunn hemp, mixed hemp, corpse-wrapping cloth, rags, lodh tree bark, antelope hide (whole or in strips), kusa grass, bark, wood-chips, human hair, horse-tail hair, or owls’ wings. They tear out their hair and beard, committed to this practice. They constantly stand, refusing seats. They squat, committed to persisting in the squatting position. They lie on a mat of thorns, making a mat of thorns their bed. They’re committed to the practice of immersion in water three times a day, including the evening. And so they live committed to practicing these various ways of mortifying and tormenting the body. This is called a person who mortifies themselves, being committed to the practice of mortifying themselves. 

And\marginnote{11.1} what person mortifies others, committed to the practice of mortifying others? It’s when a person is a slaughterer of sheep, pigs, poultry, or deer, a hunter or fisher, a bandit, an executioner, a butcher of cattle, a jailer, or has some other cruel livelihood. This is called a person who mortifies others, being committed to the practice of mortifying others. 

And\marginnote{12.1} what person mortifies themselves and others, being committed to the practice of mortifying themselves and others? It’s when a person is an anointed aristocratic king or a well-to-do brahmin. He has a new temple built to the east of the city. He shaves off his hair and beard, dresses in a rough antelope hide, and smears his body with ghee and oil. Scratching his back with antlers, he enters the temple with his chief queen and the brahmin high priest. There he lies on the bare ground strewn with grass. The king feeds on the milk from one teat of a cow that has a calf of the same color. The chief queen feeds on the milk from the second teat. The brahmin high priest feeds on the milk from the third teat. The milk from the fourth teat is served to the sacred flame. The calf feeds on the remainder. He says: ‘Slaughter this many bulls, bullocks, heifers, goats, rams, and horses for the sacrifice! Fell this many trees and reap this much grass for the sacrificial equipment!’ His bondservants, employees, and workers do their jobs under threat of punishment and danger, weeping with tearful faces. This is called a person who mortifies themselves and others, being committed to the practice of mortifying themselves and others. 

And\marginnote{13.1} what person doesn’t mortify either themselves or others, committed to the practice of not mortifying themselves or others, living without wishes in the present life, extinguished, cooled, experiencing bliss, having become holy in themselves? 

It’s\marginnote{14.1} when a Realized One arises in the world, perfected, a fully awakened Buddha, accomplished in knowledge and conduct, holy, knower of the world, supreme guide for those who wish to train, teacher of gods and humans, awakened, blessed. He has realized with his own insight this world—with its gods, \textsanskrit{Māras} and \textsanskrit{Brahmās}, this population with its ascetics and brahmins, gods and humans—and he makes it known to others. He teaches Dhamma that’s good in the beginning, good in the middle, and good in the end, meaningful and well-phrased. And he reveals a spiritual practice that’s entirely full and pure. 

A\marginnote{15.1} householder hears that teaching, or a householder’s child, or someone reborn in some clan. They gain faith in the Realized One, and reflect: ‘Living in a house is cramped and dirty, but the life of one gone forth is wide open. It’s not easy for someone living at home to lead the spiritual life utterly full and pure, like a polished shell. Why don’t I shave off my hair and beard, dress in ocher robes, and go forth from the lay life to homelessness?’ After some time they give up a large or small fortune, and a large or small family circle. They shave off hair and beard, dress in ocher robes, and go forth from the lay life to homelessness. Once they’ve gone forth, they take up the training and livelihood of the mendicants. They give up killing living creatures, renouncing the rod and the sword. They’re scrupulous and kind, living full of compassion for all living beings. 

They\marginnote{16.1} give up stealing. They take only what’s given, and expect only what’s given. They keep themselves clean by not thieving. 

They\marginnote{16.2} give up unchastity. They are celibate, set apart, avoiding the common practice of sex. 

They\marginnote{16.3} give up lying. They speak the truth and stick to the truth. They’re honest and trustworthy, and don’t trick the world with their words. 

They\marginnote{16.4} give up divisive speech. They don’t repeat in one place what they heard in another so as to divide people against each other. Instead, they reconcile those who are divided, supporting unity, delighting in harmony, loving harmony, speaking words that promote harmony. 

They\marginnote{16.5} give up harsh speech. They speak in a way that’s mellow, pleasing to the ear, lovely, going to the heart, polite, likable and agreeable to the people. 

They\marginnote{16.6} give up talking nonsense. Their words are timely, true, and meaningful, in line with the teaching and training. They say things at the right time which are valuable, reasonable, succinct, and beneficial. 

They\marginnote{16.7} avoid injuring plants and seeds. They eat in one part of the day, abstaining from eating at night and food at the wrong time. They avoid dancing, singing, music, and seeing shows. They avoid beautifying and adorning themselves with garlands, perfumes, and makeup. They avoid high and luxurious beds. They avoid receiving gold and money, raw grains, raw meat, women and girls, male and female bondservants, goats and sheep, chickens and pigs, elephants, cows, horses, and mares, and fields and land. They avoid running errands and messages; buying and selling; falsifying weights, metals, or measures; bribery, fraud, cheating, and duplicity; mutilation, murder, abduction, banditry, plunder, and violence. 

They’re\marginnote{17.1} content with robes to look after the body and almsfood to look after the belly. Wherever they go, they set out taking only these things. They’re like a bird: wherever it flies, wings are its only burden. In the same way, a mendicant is content with robes to look after the body and almsfood to look after the belly. Wherever they go, they set out taking only these things. When they have this entire spectrum of noble ethics, they experience a blameless happiness inside themselves. 

When\marginnote{18.1} they see a sight with their eyes, they don’t get caught up in the features and details. If the faculty of sight were left unrestrained, bad unskillful qualities of desire and aversion would become overwhelming. For this reason, they practice restraint, protecting the faculty of sight, and achieving its restraint. When they hear a sound with their ears … When they smell an odor with their nose … When they taste a flavor with their tongue … When they feel a touch with their body … When they know a thought with their mind, they don’t get caught up in the features and details. If the faculty of mind were left unrestrained, bad unskillful qualities of desire and aversion would become overwhelming. For this reason, they practice restraint, protecting the faculty of mind, and achieving its restraint. When they have this noble sense restraint, they experience an unsullied bliss inside themselves. 

They\marginnote{19.1} act with situational awareness when going out and coming back; when looking ahead and aside; when bending and extending the limbs; when bearing the outer robe, bowl and robes; when eating, drinking, chewing, and tasting; when urinating and defecating; when walking, standing, sitting, sleeping, waking, speaking, and keeping silent. 

When\marginnote{20.1} they have this noble spectrum of ethics, this noble sense restraint, and this noble mindfulness and situational awareness, they frequent a secluded lodging—a wilderness, the root of a tree, a hill, a ravine, a mountain cave, a charnel ground, a forest, the open air, a heap of straw. 

After\marginnote{21.1} the meal, they return from almsround, sit down cross-legged with their body straight, and establish mindfulness right there. Giving up desire for the world, they meditate with a heart rid of desire, cleansing the mind of desire. Giving up ill will and malevolence, they meditate with a mind rid of ill will, full of compassion for all living beings, cleansing the mind of ill will. Giving up dullness and drowsiness, they meditate with a mind rid of dullness and drowsiness, perceiving light, mindful and aware, cleansing the mind of dullness and drowsiness. Giving up restlessness and remorse, they meditate without restlessness, their mind peaceful inside, cleansing the mind of restlessness and remorse. Giving up doubt, they meditate having gone beyond doubt, not undecided about skillful qualities, cleansing the mind of doubt. 

They\marginnote{22.1} give up these five hindrances, corruptions of the heart that weaken wisdom. Then, quite secluded from sensual pleasures, secluded from unskillful qualities, they enter and remain in the first absorption, which has the rapture and bliss born of seclusion, while placing the mind and keeping it connected. 

As\marginnote{23.1} the placing of the mind and keeping it connected are stilled, they enter and remain in the second absorption, which has the rapture and bliss born of immersion, with internal clarity and confidence, and unified mind, without placing the mind and keeping it connected. 

And\marginnote{24.1} with the fading away of rapture, they enter and remain in the third absorption, where they meditate with equanimity, mindful and aware, personally experiencing the bliss of which the noble ones declare, ‘Equanimous and mindful, one meditates in bliss.’ 

Giving\marginnote{25.1} up pleasure and pain, and ending former happiness and sadness, they enter and remain in the fourth absorption, without pleasure or pain, with pure equanimity and mindfulness. 

When\marginnote{26.1} their mind has become immersed in \textsanskrit{samādhi} like this—purified, bright, flawless, rid of corruptions, pliable, workable, steady, and imperturbable—they extend it toward recollection of past lives. They recollect many kinds of past lives. That is: one, two, three, four, five, ten, twenty, thirty, forty, fifty, a hundred, a thousand, a hundred thousand rebirths; many eons of the world contracting, many eons of the world expanding, many eons of the world contracting and expanding. They remember: ‘There, I was named this, my clan was that, I looked like this, and that was my food. This was how I felt pleasure and pain, and that was how my life ended. When I passed away from that place I was reborn somewhere else. There, too, I was named this, my clan was that, I looked like this, and that was my food. This was how I felt pleasure and pain, and that was how my life ended. When I passed away from that place I was reborn here.’ And so they recollect their many kinds of past lives, with features and details. 

When\marginnote{27.1} their mind has become immersed in \textsanskrit{samādhi} like this—purified, bright, flawless, rid of corruptions, pliable, workable, steady, and imperturbable—they extend it toward knowledge of the death and rebirth of sentient beings. With clairvoyance that is purified and superhuman, they see sentient beings passing away and being reborn—inferior and superior, beautiful and ugly, in a good place or a bad place. They understand how sentient beings are reborn according to their deeds: ‘These dear beings did bad things by way of body, speech, and mind. They spoke ill of the noble ones; they had wrong view; and they chose to act out of that wrong view. When their body breaks up, after death, they’re reborn in a place of loss, a bad place, the underworld, hell. These dear beings, however, did good things by way of body, speech, and mind. They never spoke ill of the noble ones; they had right view; and they chose to act out of that right view. When their body breaks up, after death, they’re reborn in a good place, a heavenly realm.’ And so, with clairvoyance that is purified and superhuman, they see sentient beings passing away and being reborn—inferior and superior, beautiful and ugly, in a good place or a bad place. They understand how sentient beings are reborn according to their deeds. 

When\marginnote{28.1} their mind has become immersed in \textsanskrit{samādhi} like this—purified, bright, flawless, rid of corruptions, pliable, workable, steady, and imperturbable—they extend it toward knowledge of the ending of defilements. They truly understand: ‘This is suffering’ … ‘This is the origin of suffering’ … ‘This is the cessation of suffering’ … ‘This is the practice that leads to the cessation of suffering’. They truly understand: ‘These are defilements’ … ‘This is the origin of defilements’ … ‘This is the cessation of defilements’ … ‘This is the practice that leads to the cessation of defilements’. 

Knowing\marginnote{29.1} and seeing like this, their mind is freed from the defilements of sensuality, desire to be reborn, and ignorance. When they’re freed, they know they’re freed. 

They\marginnote{29.3} understand: ‘Rebirth is ended, the spiritual journey has been completed, what had to be done has been done, there is no return to any state of existence.’ 

This\marginnote{30.1} is called a person who neither mortifies themselves or others, being committed to the practice of not mortifying themselves or others. They live without wishes in the present life, extinguished, cooled, experiencing bliss, having become holy in themselves.” 

When\marginnote{31.1} he had spoken, \textsanskrit{Ghoṭamukha} said to him, “Excellent, Master Udena! Excellent! As if he were righting the overturned, or revealing the hidden, or pointing out the path to the lost, or lighting a lamp in the dark so people with good eyes can see what’s there, Master Udena has made the teaching clear in many ways. I go for refuge to Master Udena, to the teaching, and to the mendicant \textsanskrit{Saṅgha}. From this day forth, may Master Udena remember me as a lay follower who has gone for refuge for life.” 

“Brahmin,\marginnote{32.1} don’t go for refuge to me. You should go for refuge to that same Blessed One to whom I have gone for refuge.” 

“But\marginnote{32.3} Master Udena, where is the Blessed One at present, the perfected one, the fully awakened Buddha?” 

“Brahmin,\marginnote{32.4} the Buddha has already become fully extinguished.” 

“Master\marginnote{32.5} Udena, if I heard that the Buddha was within ten leagues, or twenty, or even up to a hundred leagues away, I’d go a hundred leagues to see him. 

But\marginnote{32.11} since the Buddha has become fully extinguished, I go for refuge to that fully extinguished Buddha, to the teaching, and to the \textsanskrit{Saṅgha}. From this day forth, may Master Udena remember me as a lay follower who has gone for refuge for life. Master Udena, the king of \textsanskrit{Aṅga} gives me a regular daily allowance. I will give you one portion of that.” 

“But\marginnote{33.2} brahmin, what does the king of \textsanskrit{Aṅga} give you as a regular daily allowance?” 

“Five\marginnote{33.3} hundred dollars.” 

“It’s\marginnote{33.4} not proper for us to receive gold and money.” 

“If\marginnote{33.5} that’s not proper, I will have a dwelling built for Master Udena.” 

“If\marginnote{33.6} you want to build me a dwelling, then build an assembly hall for the \textsanskrit{Saṅgha} at \textsanskrit{Pāṭaliputta}.” 

“Now\marginnote{33.7} I’m even more delighted and satisfied with Master Udena, since he encourages me to give to the \textsanskrit{Saṅgha}. So with this allowance and another one I will have an assembly hall built for the \textsanskrit{Saṅgha} at \textsanskrit{Pāṭaliputta}.” 

And\marginnote{33.9} so he had that hall built. And these days it’s called the “\textsanskrit{Ghoṭamukhī}”. 

%
\section*{{\suttatitleacronym MN 95}{\suttatitletranslation With Caṅkī }{\suttatitleroot Caṅkīsutta}}
\addcontentsline{toc}{section}{\tocacronym{MN 95} \toctranslation{With Caṅkī } \tocroot{Caṅkīsutta}}
\markboth{With Caṅkī }{Caṅkīsutta}
\extramarks{MN 95}{MN 95}

\scevam{So\marginnote{1.1} I have heard. }At one time the Buddha was wandering in the land of the Kosalans together with a large \textsanskrit{Saṅgha} of mendicants when he arrived at a village of the Kosalan brahmins named \textsanskrit{Opāsāda}. He stayed in a sal grove to the north of \textsanskrit{Opāsāda} called the “Gods’ Grove”. 

Now\marginnote{2.1} at that time the brahmin \textsanskrit{Caṅkī} was living in \textsanskrit{Opāsāda}. It was a crown property given by King Pasenadi of Kosala, teeming with living creatures, full of hay, wood, water, and grain, a royal endowment of the highest quality. 

The\marginnote{3.1} brahmins and householders of \textsanskrit{Opāsāda} heard: “It seems the ascetic Gotama—a Sakyan, gone forth from a Sakyan family—has arrived at \textsanskrit{Opāsāda} together with a large \textsanskrit{Saṅgha} of mendicants. He is staying in the God’s Grove to the north. He has this good reputation: ‘That Blessed One is perfected, a fully awakened Buddha, accomplished in knowledge and conduct, holy, knower of the world, supreme guide for those who wish to train, teacher of gods and humans, awakened, blessed.’ He has realized with his own insight this world—with its gods, \textsanskrit{Māras} and \textsanskrit{Brahmās}, this population with its ascetics and brahmins, gods and humans—and he makes it known to others. He teaches Dhamma that’s good in the beginning, good in the middle, and good in the end, meaningful and well-phrased. And he reveals a spiritual practice that’s entirely full and pure. It’s good to see such perfected ones.” 

Then,\marginnote{4.1} having departed \textsanskrit{Opāsāda}, they formed into companies and headed north to the God’s Grove. 

Now\marginnote{5.1} at that time the brahmin \textsanskrit{Caṅkī} had retired to the upper floor of his stilt longhouse for his midday nap. He saw the brahmins and householders heading for the God’s Grove, and addressed his steward, “My steward, why are the brahmins and householders heading north for the God’s Grove?” 

“The\marginnote{6.1} ascetic Gotama has arrived at \textsanskrit{Opāsāda} together with a large \textsanskrit{Saṅgha} of mendicants. He is staying in the God’s Grove to the north. He has this good reputation: ‘That Blessed One is perfected, a fully awakened Buddha, accomplished in knowledge and conduct, holy, knower of the world, supreme guide for those who wish to train, teacher of gods and humans, awakened, blessed.’ They’re going to see that Master Gotama.” 

“Well\marginnote{6.5} then, go to the brahmins and householders and say to them: “Sirs, the brahmin \textsanskrit{Caṅkī} asks you to wait, as he will also go to see the ascetic Gotama.” 

“Yes,\marginnote{6.8} sir,” replied the steward, and did as he was asked. 

Now\marginnote{7.1} at that time around five hundred brahmins from abroad were residing in \textsanskrit{Opāsāda} on some business. They heard that the brahmin \textsanskrit{Caṅkī} was going to see the ascetic Gotama. They approached \textsanskrit{Caṅkī} and said to him, “Is it really true that you are going to see the ascetic Gotama?” 

“Yes,\marginnote{7.6} gentlemen, it is true.” 

“Please\marginnote{8.1} don’t! It’s not appropriate for you to go to see the ascetic Gotama; it’s appropriate that he comes to see you. 

You\marginnote{8.4} are well born on both your mother’s and father’s side, of pure descent, irrefutable and impeccable in questions of ancestry back to the seventh paternal generation. For this reason it’s not appropriate for you to go to see the ascetic Gotama; it’s appropriate that he comes to see you. 

You’re\marginnote{8.7} rich, affluent, and wealthy. … 

You\marginnote{8.8} recite and remember the hymns, and have mastered the three Vedas, together with their vocabularies, ritual, phonology and etymology, and the testament as fifth. You know philology and grammar, and are well versed in cosmology and the marks of a great man. … 

You\marginnote{8.9} are attractive, good-looking, lovely, of surpassing beauty. You are magnificent, splendid, remarkable to behold. … 

You\marginnote{8.10} are ethical, mature in ethical conduct. … 

You’re\marginnote{8.11} a good speaker, with a polished, clear, and articulate voice that expresses the meaning. … 

You\marginnote{8.12} teach the teachers of many, and teach three hundred students to recite the hymns. … 

You’re\marginnote{8.13} honored, respected, revered, venerated, and esteemed by King Pasenadi of Kosala and the brahmin \textsanskrit{Pokkharasāti}. … 

You\marginnote{8.15} live in \textsanskrit{Opāsāda}, a crown property given by King Pasenadi of Kosala, teeming with living creatures, full of hay, wood, water, and grain, a royal endowment of the highest quality. 

For\marginnote{8.16} all these reasons it’s not appropriate for you to go to see the ascetic Gotama; it’s appropriate that he comes to see you.” 

When\marginnote{9.1} they had spoken, \textsanskrit{Caṅkī} said to those brahmins: 

“Well\marginnote{9.2} then, gentlemen, listen to why it’s appropriate for me to go to see the ascetic Gotama, and it’s not appropriate for him to come to see me. 

He\marginnote{9.4} is well born on both his mother’s and father’s side, of pure descent, irrefutable and impeccable in questions of ancestry back to the seventh paternal generation. For this reason it’s not appropriate for the ascetic Gotama to come to see me; rather, it’s appropriate for me to go to see him. 

When\marginnote{9.7} he went forth he abandoned abundant gold coin and bullion stored in dungeons and towers. … 

He\marginnote{9.8} went forth from the lay life to homelessness while still a youth, young, black-haired, blessed with youth, in the prime of life. … 

Though\marginnote{9.9} his mother and father wished otherwise, weeping with tearful faces, he shaved off his hair and beard, dressed in ocher robes, and went forth from the lay life to homelessness. … 

He\marginnote{9.10} is attractive, good-looking, lovely, of surpassing beauty. He is magnificent, splendid, remarkable to behold. … 

He\marginnote{9.11} is ethical, possessing ethical conduct that is noble and skillful. … 

He’s\marginnote{9.12} a good speaker, with a polished, clear, and articulate voice that expresses the meaning. … 

He’s\marginnote{9.13} a teacher of teachers. … 

He\marginnote{9.14} has ended sensual desire, and is rid of caprice. … 

He\marginnote{9.15} teaches the efficacy of deeds and action. He doesn’t wish any harm upon the community of brahmins. … 

He\marginnote{9.16} went forth from an eminent family of unbroken aristocratic lineage. … 

He\marginnote{9.17} went forth from a rich, affluent, and wealthy family. … 

People\marginnote{9.18} come from distant lands and distant countries to question him. … 

Many\marginnote{9.19} thousands of deities have gone for refuge for life to him. … 

He\marginnote{9.20} has this good reputation: ‘That Blessed One is perfected, a fully awakened Buddha, accomplished in knowledge and conduct, holy, knower of the world, supreme guide for those who wish to train, teacher of gods and humans, awakened, blessed.’ … 

He\marginnote{9.22} has the thirty-two marks of a great man. … 

King\marginnote{9.23} Seniya \textsanskrit{Bimbisāra} of Magadha and his wives and children have gone for refuge for life to the ascetic Gotama. … 

King\marginnote{9.24} Pasenadi of Kosala and his wives and children have gone for refuge for life to the ascetic Gotama. … 

The\marginnote{9.25} brahmin \textsanskrit{Pokkharasāti} and his wives and children have gone for refuge for life to the ascetic Gotama. … 

The\marginnote{9.26} ascetic Gotama has arrived to stay in the God’s Grove to the north of \textsanskrit{Opāsāda}. Any ascetic or brahmin who comes to stay in our village district is our guest, and should be honored and respected as such. For this reason, too, it’s not appropriate for Master Gotama to come to see me, rather, it’s appropriate for me to go to see him. 

This\marginnote{9.33} is the extent of Master Gotama’s praise that I have learned. But his praises are not confined to this, for the praise of Master Gotama is limitless. The possession of even a single one of these factors makes it inappropriate for Master Gotama to come to see me, rather, it’s appropriate for me to go to see him. Well then, gentlemen, let’s all go to see the ascetic Gotama.” 

Then\marginnote{10.1} \textsanskrit{Caṅkī} together with a large group of brahmins went to the Buddha and exchanged greetings with him. When the greetings and polite conversation were over, he sat down to one side. 

Now\marginnote{11.1} at that time the Buddha was sitting engaged in some polite conversation together with some very senior brahmins. And the brahmin student \textsanskrit{Kāpaṭika} was sitting in that assembly. He was young, newly tonsured; he was sixteen years old. He had mastered the three Vedas, together with their vocabularies, ritual, phonology and etymology, and the testament as fifth. He knew philology and grammar, and was well versed in cosmology and the marks of a great man. While the senior brahmins were conversing together with the Buddha, he interrupted. 

Then\marginnote{11.4} the Buddha rebuked \textsanskrit{Kāpaṭika}, “Venerable \textsanskrit{Bhāradvāja}, don’t interrupt the senior brahmins. Wait until they’ve finished speaking.” 

When\marginnote{11.7} he had spoken, \textsanskrit{Caṅkī} said to the Buddha, “Master Gotama, don’t rebuke the student \textsanskrit{Kāpaṭika}. He’s a gentleman, learned, astute, a good speaker. He’s capable of having a dialogue with Master Gotama about this.” 

Then\marginnote{12.1} it occurred to the Buddha, “Clearly the student \textsanskrit{Kāpaṭika} will talk about the scriptural heritage of the three Vedas. That’s why they put him at the front.” 

Then\marginnote{12.4} \textsanskrit{Kāpaṭika} thought, “When the ascetic Gotama looks at me, I’ll ask him a question.” Then the Buddha, knowing what \textsanskrit{Kāpaṭika} was thinking, looked at him. 

Then\marginnote{12.7} \textsanskrit{Kāpaṭika} thought, “The ascetic Gotama is engaging with me. Why don’t I ask him a question?” Then he said, “Master Gotama, regarding that which by the lineage of testament and by canonical authority is the ancient hymnal of the brahmins, the brahmins come to the definite conclusion: ‘This is the only truth, other ideas are silly.’ What do you say about this?” 

“Well,\marginnote{13.1} \textsanskrit{Bhāradvāja}, is there even a single one of the brahmins who says this: ‘I know this, I see this: this is the only truth, other ideas are silly’?” 

“No,\marginnote{13.4} Master Gotama.” 

“Well,\marginnote{13.5} is there even a single teacher of the brahmins, or a teacher’s teacher, or anyone back to the seventh generation of teachers, who says this: ‘I know this, I see this: this is the only truth, other ideas are silly’?” 

“No,\marginnote{13.8} Master Gotama.” 

“Well,\marginnote{13.9} what of the ancient hermits of the brahmins, namely \textsanskrit{Aṭṭhaka}, \textsanskrit{Vāmaka}, \textsanskrit{Vāmadeva}, \textsanskrit{Vessāmitta}, Yamadaggi, \textsanskrit{Aṅgīrasa}, \textsanskrit{Bhāradvāja}, \textsanskrit{Vāseṭṭha}, Kassapa, and Bhagu? They were the authors and propagators of the hymns. Their hymnal was sung and propagated and compiled in ancient times; and these days, brahmins continue to sing and chant it, chanting what was chanted and teaching what was taught. Did even they say: ‘We know this, we see this: this is the only truth, other ideas are silly’?” 

“No,\marginnote{13.13} Master Gotama.” 

“So,\marginnote{13.14} \textsanskrit{Bhāradvāja}, it seems that there is not a single one of the brahmins, not even anyone back to the seventh generation of teachers, nor even the ancient hermits of the brahmins who say: ‘We know this, we see this: this is the only truth, other ideas are silly.’ 

Suppose\marginnote{13.23} there was a queue of blind men, each holding the one in front: the first one does not see, the middle one does not see, and the last one does not see. In the same way, it seems to me that the brahmins’ statement turns out to be like a queue of blind men: the first one does not see, the middle one does not see, and the last one does not see. What do you think, \textsanskrit{Bhāradvāja}? This being so, doesn’t the brahmins’ faith turn out to be baseless?” 

“The\marginnote{14.1} brahmins don’t just honor this because of faith, but also because of oral transmission.” 

“First\marginnote{14.2} you relied on faith, now you speak of oral tradition. These five things can be seen to turn out in two different ways. What five? Faith, preference, oral tradition, reasoned contemplation, and acceptance of a view after consideration. Even though you have full faith in something, it may be void, hollow, and false. And even if you don’t have full faith in something, it may be true and real, not otherwise. Even though you have a strong preference for something … something may be accurately transmitted … something may be well contemplated … something may be well considered, it may be void, hollow, and false. And even if something is not well considered, it may be true and real, not otherwise. For a sensible person who is preserving truth this is not sufficient to come to the definite conclusion: ‘This is the only truth, other ideas are silly.’” 

“But\marginnote{15.1} Master Gotama, how do you define the preservation of truth?” 

“If\marginnote{15.3} a person has faith, they preserve truth by saying, ‘Such is my faith.’ But they don’t yet come to the definite conclusion: ‘This is the only truth, other ideas are silly.’ If a person has a preference … or has received an oral transmission … or has a reasoned reflection about something … or has accepted a view after contemplation, they preserve truth by saying, ‘Such is the view I have accepted after contemplation.’ But they don’t yet come to the definite conclusion: ‘This is the only truth, other ideas are silly.’ That’s how the preservation of truth is defined, \textsanskrit{Bhāradvāja}. I describe the preservation of truth as defined in this way. But this is not yet the awakening to the truth.” 

“That’s\marginnote{16.1} how the preservation of truth is defined, Master Gotama. We regard the preservation of truth as defined in this way. But Master Gotama, how do you define awakening to the truth?” 

“\textsanskrit{Bhāradvāja},\marginnote{17.1} take the case of a mendicant living supported by a town or village. A householder or their child approaches and scrutinizes them for three kinds of things: things that arouse greed, things that provoke hate, and things that promote delusion. ‘Does this venerable have any qualities that arouse greed? Such qualities that, were their mind to be overwhelmed by them, they might say that they know, even though they don’t know, or that they see, even though they don’t see; or that they might encourage others to do what is for their lasting harm and suffering?’ Scrutinizing them they find: ‘This venerable has no such qualities that arouse greed. Rather, that venerable has bodily and verbal behavior like that of someone without greed. And the principle that they teach is deep, hard to see, hard to understand, peaceful, sublime, beyond the scope of logic, subtle, comprehensible to the astute. It’s not easy for someone with greed to teach this.’ 

Scrutinizing\marginnote{18.1} them in this way they see that they are purified of qualities that arouse greed. Next, they search them for qualities that provoke hate. ‘Does this venerable have any qualities that provoke hate? Such qualities that, were their mind to be overwhelmed by them, they might say that they know, even though they don’t know, or that they see, even though they don’t see; or that they might encourage others to do what is for their lasting harm and suffering?’ Scrutinizing them they find: ‘This venerable has no such qualities that provoke hate. Rather, that venerable has bodily and verbal behavior like that of someone without hate. And the principle that they teach is deep, hard to see, hard to understand, peaceful, sublime, beyond the scope of logic, subtle, comprehensible to the astute. It’s not easy for someone with hate to teach this.’ 

Scrutinizing\marginnote{19.1} them in this way they see that they are purified of qualities that provoke hate. Next, they scrutinize them for qualities that promote delusion. ‘Does this venerable have any qualities that promote delusion? Such qualities that, were their mind to be overwhelmed by them, they might say that they know, even though they don’t know, or that they see, even though they don’t see; or that they might encourage others to do what is for their lasting harm and suffering?’ Scrutinizing them they find: ‘This venerable has no such qualities that promote delusion. Rather, that venerable has bodily and verbal behavior like that of someone without delusion. And the principle that they teach is deep, hard to see, hard to understand, peaceful, sublime, beyond the scope of logic, subtle, comprehensible to the astute. It’s not easy for someone with delusion to teach this.’ 

Scrutinizing\marginnote{20.1} them in this way they see that they are purified of qualities that promote delusion. Next, they place faith in them. When faith has arisen they approach the teacher. They pay homage, lend an ear, hear the teachings, remember the teachings, reflect on their meaning, and accept them after consideration. Then enthusiasm springs up; they make an effort, weigh up, and persevere. Persevering, they directly realize the ultimate truth, and see it with penetrating wisdom. That’s how the awakening to truth is defined, \textsanskrit{Bhāradvāja}. I describe the awakening to truth as defined in this way. But this is not yet the arrival at the truth.” 

“That’s\marginnote{21.1} how the awakening to truth is defined, Master Gotama. I regard the awakening to truth as defined in this way. But Master Gotama, how do you define the arrival at the truth?” 

“By\marginnote{21.4} the cultivation, development, and making much of these very same things there is the arrival at the truth. That’s how the arrival at the truth is defined, \textsanskrit{Bhāradvāja}. I describe the arrival at the truth as defined in this way.” 

“That’s\marginnote{22.1} how the arrival at the truth is defined, Master Gotama. I regard the arrival at the truth as defined in this way. But what quality is helpful for arriving at the truth?” 

“Striving\marginnote{22.4} is helpful for arriving at the truth. If you don’t strive, you won’t arrive at the truth. You arrive at the truth because you strive. That’s why striving is helpful for arriving at the truth.” 

“But\marginnote{23.1} what quality is helpful for striving?” 

“Weighing\marginnote{23.3} up the teachings is helpful for striving … 

Making\marginnote{24.1} an effort is helpful for weighing up the teachings … 

Enthusiasm\marginnote{25.1} is helpful for making an effort … 

Acceptance\marginnote{26.1} of the teachings after consideration is helpful for enthusiasm … 

Reflecting\marginnote{27.1} on the meaning of the teachings is helpful for accepting them after consideration … 

Remembering\marginnote{28.1} the teachings is helpful for reflecting on their meaning … 

Hearing\marginnote{29.1} the teachings is helpful for remembering the teachings … 

Listening\marginnote{30.1} is helpful for hearing the teachings … 

Paying\marginnote{31.1} homage is helpful for listening … 

Approaching\marginnote{32.1} is helpful for paying homage … 

Faith\marginnote{33.1} is helpful for approaching a teacher. If you don’t give rise to faith, you won’t approach a teacher. You approach a teacher because you have faith. That’s why faith is helpful for approaching a teacher.” 

“I’ve\marginnote{34.1} asked Master Gotama about the preservation of truth, and he has answered me. I like and accept this, and am satisfied with it. I’ve asked Master Gotama about awakening to the truth, and he has answered me. I like and accept this, and am satisfied with it. I’ve asked Master Gotama about the arrival at the truth, and he has answered me. I like and accept this, and am satisfied with it. I’ve asked Master Gotama about the things that are helpful for the arrival at the truth, and he has answered me. I like and accept this, and am satisfied with it. Whatever I have asked Master Gotama about he has answered me. I like and accept this, and am satisfied with it. 

Master\marginnote{34.11} Gotama, I used to think this: ‘Who are these shavelings, fake ascetics, riffraff, black spawn from the feet of our Kinsman to be counted alongside those who understand the teaching?’ The Buddha has inspired me to have love, confidence, and respect for ascetics! 

Excellent,\marginnote{35.1} Master Gotama! … From this day forth, may Master Gotama remember me as a lay follower who has gone for refuge for life.” 

%
\section*{{\suttatitleacronym MN 96}{\suttatitletranslation With Esukārī }{\suttatitleroot Esukārīsutta}}
\addcontentsline{toc}{section}{\tocacronym{MN 96} \toctranslation{With Esukārī } \tocroot{Esukārīsutta}}
\markboth{With Esukārī }{Esukārīsutta}
\extramarks{MN 96}{MN 96}

\scevam{So\marginnote{1.1} I have heard. }At one time the Buddha was staying near \textsanskrit{Sāvatthī} in Jeta’s Grove, \textsanskrit{Anāthapiṇḍika}’s monastery. 

Then\marginnote{2.1} \textsanskrit{Esukārī} the brahmin went up to the Buddha, and exchanged greetings with him. When the greetings and polite conversation were over, he sat down to one side and said to the Buddha: 

“Master\marginnote{3.1} Gotama, the brahmins prescribe four kinds of service: for a brahmin, an aristocrat, a merchant, and a worker. This is the service they prescribe for a brahmin: ‘A brahmin, an aristocrat, a merchant, and a worker may all serve a brahmin.’ This is the service they prescribe for an aristocrat: ‘An aristocrat, a merchant, and a worker may all serve an aristocrat.’ This is the service they prescribe for a merchant: ‘A merchant or a worker may serve a merchant.’ This is the service they prescribe for a worker: ‘Only a worker may serve a worker. For who else will serve a worker?’ These are the four kinds of service that the brahmins prescribe. What do you say about this?” 

“But\marginnote{4.1} brahmin, did the whole world authorize the brahmins to prescribe these four kinds of service?” 

“No,\marginnote{4.2} Master Gotama.” 

“It’s\marginnote{4.3} as if they were to force a steak on a poor, penniless person, telling them they must eat it and then pay for it. In the same way, the brahmins have prescribed these four kinds of service without the consent of these ascetics and brahmins. 

Brahmin,\marginnote{5.1} I don’t say that you should serve everyone, nor do I say that you shouldn’t serve anyone. I say that you shouldn’t serve someone if serving them makes you worse, not better. And I say that you should serve someone if serving them makes you better, not worse. 

If\marginnote{6.1} they were to ask an aristocrat this, ‘Who should you serve? Someone in whose service you get worse, or someone in whose service you get better?’ Answering rightly, an aristocrat would say, ‘Someone in whose service I get better.’ 

If\marginnote{7.1} they were to ask a brahmin … a merchant … or a worker this, ‘Who should you serve? Someone in whose service you get worse, or someone in whose service you get better?’ Answering rightly, a worker would say, ‘Someone in whose service I get better.’ 

Brahmin,\marginnote{7.7} I don’t say that coming from an eminent family makes you a better or worse person. I don’t say that being very beautiful makes you a better or worse person. I don’t say that being very wealthy makes you a better or worse person. 

For\marginnote{8.1} some people from eminent families kill living creatures, steal, and commit sexual misconduct. They use speech that’s false, divisive, harsh, or nonsensical. And they’re covetous, malicious, with wrong view. That’s why I don’t say that coming from an eminent family makes you a better person. 

But\marginnote{8.3} some people from eminent families also refrain from killing living creatures, stealing, and committing sexual misconduct. They refrain from using speech that’s false, divisive, harsh, or nonsensical. And they’re not covetous or malicious, and they have right view. That’s why I don’t say that coming from an eminent family makes you a worse person. 

People\marginnote{8.5} who are very beautiful, or not very beautiful, who are very wealthy, or not very wealthy, may also behave in the same ways. That’s why I don’t say that any of these things makes you a better or worse person. 

Brahmin,\marginnote{9.1} I don’t say that you should serve everyone, nor do I say that you shouldn’t serve anyone. And I say that you should serve someone if serving them makes you grow in faith, ethics, learning, generosity, and wisdom. I say that you shouldn’t serve someone if serving them doesn’t make you grow in faith, ethics, learning, generosity, and wisdom.” 

When\marginnote{10.1} he had spoken, \textsanskrit{Esukārī} said to him: 

“Master\marginnote{10.2} Gotama, the brahmins prescribe four kinds of wealth: for a brahmin, an aristocrat, a merchant, and a worker. The wealth they prescribe for a brahmin is living on alms. A brahmin who scorns his own wealth, living on alms, fails in his duty like a guard who steals. The wealth they prescribe for an aristocrat is the bow and quiver. An aristocrat who scorns his own wealth, the bow and quiver, fails in his duty like a guard who steals. The wealth they prescribe for a merchant is farming and animal husbandry. A merchant who scorns his own wealth, farming and animal husbandry, fails in his duty like a guard who steals. The wealth they prescribe for a worker is the scythe and flail. A worker who scorns his own wealth, the scythe and flail, fails in his duty like a guard who steals. These are the four kinds of wealth that the brahmins prescribe. What do you say about this?” 

“But\marginnote{11.1} brahmin, did the whole world authorize the brahmins to prescribe these four kinds of wealth?” 

“No,\marginnote{11.2} Master Gotama.” 

“It’s\marginnote{11.3} as if they were to force a steak on a poor, penniless person, telling them they must eat it and then pay for it. 

In\marginnote{11.4} the same way, the brahmins have prescribed these four kinds of wealth without the consent of these ascetics and brahmins. 

I\marginnote{12.1} declare that a person’s own wealth is the noble, transcendent teaching. But they are reckoned by recollecting the traditional family lineage of their mother and father wherever they are incarnated. If they incarnate in a family of aristocrats they are reckoned as an aristocrat. If they incarnate in a family of brahmins they are reckoned as a brahmin. If they incarnate in a family of merchants they are reckoned as a merchant. If they incarnate in a family of workers they are reckoned as a worker. 

It’s\marginnote{12.7} like fire, which is reckoned according to the specific conditions dependent upon which it burns. A fire that burns dependent on logs is reckoned as a log fire. A fire that burns dependent on twigs is reckoned as a twig fire. A fire that burns dependent on grass is reckoned as a grass fire. A fire that burns dependent on cow-dung is reckoned as a cow-dung fire. 

In\marginnote{12.12} the same way, I declare that a person’s own wealth is the noble, transcendent teaching. But they are reckoned by recollecting the traditional family lineage of their mother and father wherever they are incarnated. 

Suppose\marginnote{13.1} someone from a family of aristocrats goes forth from the lay life to homelessness. Relying on the teaching and training proclaimed by the Realized One they refrain from killing living creatures, stealing, and sex. They refrain from using speech that’s false, divisive, harsh, or nonsensical. And they’re not covetous or malicious, and they have right view. They succeed in the procedure of the skillful teaching. 

Suppose\marginnote{13.2} someone from a family of brahmins … merchants … workers goes forth from the lay life to homelessness. Relying on the teaching and training proclaimed by the Realized One … they succeed in the procedure of the skillful teaching. 

What\marginnote{14.1} do you think, brahmin? Is only a brahmin capable of developing a heart of love free of enmity and ill will for this region, and not an aristocrat, merchant, or worker?” 

“No,\marginnote{14.3} Master Gotama. Aristocrats, brahmins, merchants, and workers can all do so. For all four classes are capable of developing a heart of love free of enmity and ill will for this region.” 

“In\marginnote{14.9} the same way, suppose someone from a family of aristocrats, brahmins, merchants, or workers goes forth from the lay life to homelessness. Relying on the teaching and training proclaimed by the Realized One … they succeed in the procedure of the skillful teaching. 

What\marginnote{15.1} do you think, brahmin? Is only a brahmin capable of taking some bathing paste of powdered shell, going to the river, and washing off dust and dirt, and not an aristocrat, merchant, or worker?” 

“No,\marginnote{15.3} Master Gotama. All four classes are capable of doing this.” 

“In\marginnote{15.9} the same way, suppose someone from a family of aristocrats, brahmins, merchants, or workers goes forth from the lay life to homelessness. Relying on the teaching and training proclaimed by the Realized One … they succeed in the procedure of the skillful teaching. 

What\marginnote{16.1} do you think, brahmin? Suppose an anointed aristocratic king were to gather a hundred people born in different castes and say to them: ‘Please gentlemen, let anyone here who was born in a family of aristocrats, brahmins, or chieftains take a drill-stick made of teak, sal, frankincense wood, sandalwood, or cherry wood, light a fire and produce heat. And let anyone here who was born in a family of outcastes, hunters, bamboo-workers, chariot-makers, or waste-collectors take a drill-stick made from a dog’s drinking trough, a pig’s trough, a dustbin, or castor-oil wood, light a fire and produce heat.’ 

What\marginnote{16.5} do you think, brahmin? Would only the fire produced by the high class people with good quality wood have flames, color, and radiance, and be usable as fire, and not the fire produced by the low class people with poor quality wood?” 

“No,\marginnote{16.8} Master Gotama. The fire produced by the high class people with good quality wood would have flames, color, and radiance, and be usable as fire, and so would the fire produced by the low class people with poor quality wood. For all fire has flames, color, and radiance, and is usable as fire.” 

“In\marginnote{16.12} the same way, suppose someone from a family of aristocrats, brahmins, merchants, or workers goes forth from the lay life to homelessness. Relying on the teaching and training proclaimed by the Realized One they refrain from killing living creatures, stealing, and sex. They refrain from using speech that’s false, divisive, harsh, or nonsensical. And they’re not covetous or malicious, and they have right view. They succeed in the procedure of the skillful teaching.” 

When\marginnote{17.1} he had spoken, \textsanskrit{Esukārī} said to him, “Excellent, Master Gotama! Excellent! … From this day forth, may Master Gotama remember me as a lay follower who has gone for refuge for life.” 

%
\section*{{\suttatitleacronym MN 97}{\suttatitletranslation With Dhanañjāni }{\suttatitleroot Dhanañjānisutta}}
\addcontentsline{toc}{section}{\tocacronym{MN 97} \toctranslation{With Dhanañjāni } \tocroot{Dhanañjānisutta}}
\markboth{With Dhanañjāni }{Dhanañjānisutta}
\extramarks{MN 97}{MN 97}

\scevam{So\marginnote{1.1} I have heard. }At one time the Buddha was staying near \textsanskrit{Rājagaha}, in the Bamboo Grove, the squirrels’ feeding ground. 

Now\marginnote{2.1} at that time Venerable \textsanskrit{Sāriputta} was wandering in the Southern Hills together with a large \textsanskrit{Saṅgha} of mendicants. Then a certain mendicant who had completed the rainy season residence in \textsanskrit{Rājagaha} went to the Southern Hills, where he approached Venerable \textsanskrit{Sāriputta}, and exchanged greetings with him. When the greetings and polite conversation were over, he sat down to one side. \textsanskrit{Sāriputta} said to him, “Reverend, I hope the Buddha is healthy and well?” 

“He\marginnote{2.5} is, reverend.” 

“And\marginnote{2.6} I hope that the mendicant \textsanskrit{Saṅgha} is healthy and well.” 

“It\marginnote{2.7} is.” 

“Reverend,\marginnote{2.8} at the rice checkpoint there is a brahmin named \textsanskrit{Dhanañjāni}. I hope that he is healthy and well?” 

“He\marginnote{2.10} too is well.” 

“But\marginnote{2.11} is he diligent?” 

“How\marginnote{2.12} could he possibly be diligent? \textsanskrit{Dhanañjāni} robs the brahmins and householders in the name of the king, and he robs the king in the name of the brahmins and householders. His wife, a lady of faith who he married from a family of faith, has passed away. And he has taken a new wife who has no faith.” 

“Oh,\marginnote{2.16} it’s bad news to hear that \textsanskrit{Dhanañjāni} is negligent. Hopefully, some time or other I’ll get to meet him, and we can have a discussion.” 

When\marginnote{3.1} \textsanskrit{Sāriputta} had stayed in the Southern Hills as long as he wished, he set out for \textsanskrit{Rājagaha}. Traveling stage by stage, he arrived at \textsanskrit{Rājagaha}, where he stayed in the Bamboo Grove, the squirrels’ feeding ground. 

Then\marginnote{4.1} he robed up in the morning and, taking his bowl and robe, entered \textsanskrit{Rājagaha} for alms. Now at that time \textsanskrit{Dhanañjāni} was having his cows milked in a cow-shed outside the city. Then \textsanskrit{Sāriputta} wandered for alms in \textsanskrit{Rājagaha}. After the meal, on his return from almsround, he approached \textsanskrit{Dhanañjāni}. 

Seeing\marginnote{4.4} \textsanskrit{Sāriputta} coming off in the distance, \textsanskrit{Dhanañjāni} went to him and said, “Here, Master \textsanskrit{Sāriputta}, drink some fresh milk before the meal time.” 

“Enough,\marginnote{4.7} brahmin, I’ve finished eating for today. I shall be at the root of that tree for the day’s meditation. Come see me there.” 

“Yes,\marginnote{4.11} sir,” replied \textsanskrit{Dhanañjāni}. 

When\marginnote{5.1} \textsanskrit{Dhanañjāni} had finished breakfast he went to \textsanskrit{Sāriputta} and exchanged greetings with him. When the greetings and polite conversation were over, he sat down to one side. \textsanskrit{Sāriputta} said to him, “I hope you’re diligent, \textsanskrit{Dhanañjāni}?” 

“How\marginnote{5.4} can I possibly be diligent, Master \textsanskrit{Sāriputta}? I have to provide for my mother and father, my wives and children, and my bondservants and workers. And I have to make the proper offerings to friends and colleagues, relatives and kin, guests, ancestors, deities, and king. And then this body must also be fattened and built up.” 

“What\marginnote{6.1} do you think, \textsanskrit{Dhanañjāni}? Suppose someone was to behave in an unprincipled and unjust way for the sake of their parents. Because of this the wardens of hell would drag them to hell. Could they get out of being dragged to hell by pleading that they had acted for the sake of their parents? Or could their parents save them by pleading that the acts had been done for their sake?” 

“No,\marginnote{6.4} Master \textsanskrit{Sāriputta}. Rather, even as they were wailing the wardens of hell would cast them down into hell.” 

“What\marginnote{7.1} do you think, \textsanskrit{Dhanañjāni}? Suppose someone was to behave in an unprincipled and unjust way for the sake of their wives and children … bondservants and workers … friends and colleagues … relatives and kin … guests … ancestors … deities … king … fattening and building up their body. Because of this the wardens of hell would drag them to hell. Could they get out of being dragged to hell by pleading that they had acted for the sake of fattening and building up their body? Or could anyone else save them by pleading that the acts had been done for that reason?” 

“No,\marginnote{15.3} Master \textsanskrit{Sāriputta}. Rather, even as they were wailing the wardens of hell would cast them down into hell.” 

“Who\marginnote{16.1} do you think is better, \textsanskrit{Dhanañjāni}? Someone who, for the sake of their parents, behaves in an unprincipled and unjust manner, or someone who behaves in a principled and just manner?” 

“Someone\marginnote{16.3} who behaves in a principled and just manner for the sake of their parents. For principled and moral conduct is better than unprincipled and immoral conduct.” 

“\textsanskrit{Dhanañjāni},\marginnote{16.6} there are other livelihoods that are both profitable and legitimate. By means of these it’s possible to provide for your parents, avoid bad deeds, and practice the path of goodness. 

Who\marginnote{17.1} do you think is better, \textsanskrit{Dhanañjāni}? Someone who, for the sake of their wives and children … bondservants and workers … friends and colleagues … relatives and kin … guests … ancestors … deities … king … fattening and building up their body, behaves in an unprincipled and unjust manner, or someone who behaves in a principled and just manner?” 

“Someone\marginnote{25.3} who behaves in a principled and just manner. For principled and moral conduct is better than unprincipled and immoral conduct.” 

“\textsanskrit{Dhanañjāni},\marginnote{25.6} there are other livelihoods that are both profitable and legitimate. By means of these it’s possible to fatten and build up your body, avoid bad deeds, and practice the path of goodness.” 

Then\marginnote{26.1} \textsanskrit{Dhanañjāni} the brahmin, having approved and agreed with what Venerable \textsanskrit{Sāriputta} said, got up from his seat and left. 

Some\marginnote{27.1} time later \textsanskrit{Dhanañjāni} became sick, suffering, gravely ill. Then he addressed a man, “Please, mister, go to the Buddha, and in my name bow with your head to his feet. Say to him: ‘Sir, the brahmin \textsanskrit{Dhanañjāni} is sick, suffering, gravely ill. He bows with his head to your feet.’ Then go to Venerable \textsanskrit{Sāriputta}, and in my name bow with your head to his feet. Say to him: ‘Sir, the brahmin \textsanskrit{Dhanañjāni} is sick, suffering, gravely ill. He bows with his head to your feet.’ And then say: ‘Sir, please visit \textsanskrit{Dhanañjāni} at his home out of compassion.’” 

“Yes,\marginnote{27.11} sir,” that man replied. He did as \textsanskrit{Dhanañjāni} asked. \textsanskrit{Sāriputta} consented in silence. 

He\marginnote{28.1} robed up, and, taking his bowl and robe, went to \textsanskrit{Dhanañjāni}’s home, where he sat on the seat spread out and said to \textsanskrit{Dhanañjāni}, “I hope you’re keeping well, \textsanskrit{Dhanañjāni}; I hope you’re alright. And I hope the pain is fading, not growing, that its fading is evident, not its growing.” 

“I’m\marginnote{29.1} not keeping well, Master \textsanskrit{Sāriputta}, I’m not alright. The pain is terrible and growing, not fading; its growing is evident, not its fading. The winds piercing my head are so severe, it feels like a strong man drilling into my head with a sharp point. I’m not keeping well. The pain in my head is so severe, it feels like a strong man tightening a tough leather strap around my head. I’m not keeping well. The winds slicing my belly are so severe, like a deft butcher or their apprentice were slicing open a cows’s belly open with a meat cleaver. I’m not keeping well. The burning in my body is so severe, it feels like two strong men grabbing a weaker man by the arms to burn and scorch him on a pit of glowing coals. I’m not keeping well, Master \textsanskrit{Sāriputta}, I’m not alright. The pain is terrible and growing, not fading; its growing is evident, not its fading.” 

“\textsanskrit{Dhanañjāni},\marginnote{30.1} which do you think is better: hell or the animal realm?” 

“The\marginnote{30.3} animal realm is better.” 

“Which\marginnote{30.4} do you think is better: the animal realm or the ghost realm?” 

“The\marginnote{30.6} ghost realm is better.” 

“Which\marginnote{30.7} do you think is better: the ghost realm or human life?” 

“Human\marginnote{30.9} life is better.” 

“Which\marginnote{30.10} do you think is better: human life or as one of the Gods of the Four Great Kings?” 

“The\marginnote{30.12} Gods of the Four Great Kings.” 

“Which\marginnote{30.13} do you think is better: the Gods of the Four Great Kings or the Gods of the Thirty-Three?” 

“The\marginnote{30.15} Gods of the Thirty-Three.” 

“Which\marginnote{30.16} do you think is better: the Gods of the Thirty-Three or the Gods of Yama?” 

“The\marginnote{30.18} Gods of Yama.” 

“Which\marginnote{30.19} do you think is better: the Gods of Yama or the Joyful Gods?” 

“The\marginnote{30.21} Joyful Gods.” 

“Which\marginnote{30.22} do you think is better: the Joyful Gods or the Gods Who Love to Create?” 

“The\marginnote{30.24} Gods Who Love to Create.” 

“Which\marginnote{30.25} do you think is better: the Gods Who Love to Create or the Gods Who Control the Creations of Others?” 

“The\marginnote{30.27} Gods Who Control the Creations of Others.” 

“Which\marginnote{31.1} do you think is better: the Gods Who Control the Creations of Others or the \textsanskrit{Brahmā} realm?” 

“Master\marginnote{31.3} \textsanskrit{Sāriputta} speaks of the \textsanskrit{Brahmā} realm! Master \textsanskrit{Sāriputta} speaks of the \textsanskrit{Brahmā} realm!” 

Then\marginnote{31.5} \textsanskrit{Sāriputta} thought: 

“These\marginnote{31.6} brahmins are devoted to the \textsanskrit{Brahmā} realm. Why don’t I teach him a path to the company of \textsanskrit{Brahmā}?” 

“\textsanskrit{Dhanañjāni},\marginnote{31.8} I shall teach you a path to the company of \textsanskrit{Brahmā}. Listen and pay close attention, I will speak.” 

“Yes,\marginnote{31.10} sir,” replied \textsanskrit{Dhanañjāni}. Venerable \textsanskrit{Sāriputta} said this: 

“And\marginnote{32.1} what is a path to companionship with \textsanskrit{Brahmā}? Firstly, a mendicant meditates spreading a heart full of love to one direction, and to the second, and to the third, and to the fourth. In the same way above, below, across, everywhere, all around, they spread a heart full of love to the whole world—abundant, expansive, limitless, free of enmity and ill will. This is a path to companionship with \textsanskrit{Brahmā}. 

Furthermore,\marginnote{33{-}35.1} a mendicant meditates spreading a heart full of compassion … 

They\marginnote{33{-}35.2} meditate spreading a heart full of rejoicing … 

They\marginnote{33{-}35.3} meditate spreading a heart full of equanimity to one direction, and to the second, and to the third, and to the fourth. In the same way above, below, across, everywhere, all around, they spread a heart full of equanimity to the whole world—abundant, expansive, limitless, free of enmity and ill will. This is a path to companionship with \textsanskrit{Brahmā}.” 

“Well\marginnote{36.1} then, Master \textsanskrit{Sāriputta}, in my name bow with your head at the Buddha’s feet. Say to him: ‘Sir, the brahmin \textsanskrit{Dhanañjāni} is sick, suffering, gravely ill. He bows with his head to your feet.’” Then \textsanskrit{Sāriputta}, after establishing \textsanskrit{Dhanañjāni} in the inferior \textsanskrit{Brahmā} realm, got up from his seat and left while there was still more left to do. Not long after \textsanskrit{Sāriputta} had departed, \textsanskrit{Dhanañjāni} passed away and was reborn in the \textsanskrit{Brahmā} realm. 

Then\marginnote{37.1} the Buddha said to the mendicants, “Mendicants, \textsanskrit{Sāriputta}, after establishing \textsanskrit{Dhanañjāni} in the inferior \textsanskrit{Brahmā} realm, got up from his seat and left while there was still more left to do.” 

Then\marginnote{38.1} \textsanskrit{Sāriputta} went to the Buddha, bowed, sat down to one side, and said, “Sir, the brahmin \textsanskrit{Dhanañjāni} is sick, suffering, gravely ill. He bows with his head to your feet.” 

“But\marginnote{38.4} \textsanskrit{Sāriputta}, after establishing \textsanskrit{Dhanañjāni} in the inferior \textsanskrit{Brahmā} realm, why did you get up from your seat and leave while there was still more left to do?” 

“Sir,\marginnote{38.5} I thought: ‘These brahmins are devoted to the \textsanskrit{Brahmā} realm. Why don’t I teach him a path to the company of \textsanskrit{Brahmā}?’” 

“And\marginnote{38.7} \textsanskrit{Sāriputta}, the brahmin \textsanskrit{Dhanañjāni} has passed away and been reborn in the \textsanskrit{Brahmā} realm.” 

%
\section*{{\suttatitleacronym MN 98}{\suttatitletranslation With Vāseṭṭha }{\suttatitleroot Vāseṭṭhasutta}}
\addcontentsline{toc}{section}{\tocacronym{MN 98} \toctranslation{With Vāseṭṭha } \tocroot{Vāseṭṭhasutta}}
\markboth{With Vāseṭṭha }{Vāseṭṭhasutta}
\extramarks{MN 98}{MN 98}

\scevam{So\marginnote{1.1} I have heard. }At one time the Buddha was staying in a forest near \textsanskrit{Icchānaṅgala}. 

Now\marginnote{2.1} at that time several very well-known well-to-do brahmins were residing in \textsanskrit{Icchānaṅgala}. They included the brahmins \textsanskrit{Caṅkī}, \textsanskrit{Tārukkha}, \textsanskrit{Pokkharasāti}, \textsanskrit{Jāṇussoṇi}, Todeyya, and others. 

Then\marginnote{3.1} as the brahmin students \textsanskrit{Vāseṭṭha} and \textsanskrit{Bhāradvāja} were going for a walk they began to discuss the question: “How do you become a brahmin?” 

\textsanskrit{Bhāradvāja}\marginnote{3.3} said this: “When you’re well born on both your mother’s and father’s side, of pure descent, irrefutable and impeccable in questions of ancestry back to the seventh paternal generation—then you’re a brahmin.” 

\textsanskrit{Vāseṭṭha}\marginnote{3.6} said this: “When you’re ethical and accomplished in doing your duties—then you’re a brahmin.” 

But\marginnote{4.1} neither was able to persuade the other. 

So\marginnote{5.1} \textsanskrit{Vāseṭṭha} said to \textsanskrit{Bhāradvāja}, “Master \textsanskrit{Bhāradvāja}, the ascetic Gotama—a Sakyan, gone forth from a Sakyan family—is staying in a forest near \textsanskrit{Icchānaṅgala}. He has this good reputation: ‘That Blessed One is perfected, a fully awakened Buddha, accomplished in knowledge and conduct, holy, knower of the world, supreme guide for those who wish to train, teacher of gods and humans, awakened, blessed.’ Come, let’s go to see him and ask him about this matter. As he answers, so we’ll remember it.” 

“Yes,\marginnote{5.7} sir,” replied \textsanskrit{Bhāradvāja}. 

So\marginnote{6.1} they went to the Buddha and exchanged greetings with him. When the greetings and polite conversation were over, they sat down to one side, and \textsanskrit{Vāseṭṭha} addressed the Buddha in verse: 

\begin{verse}%
“We’re\marginnote{7.1} both authorized masters \\
of the three Vedas. \\
I’m a student of \textsanskrit{Pokkharasāti}, \\
and he of \textsanskrit{Tārukkha}. 

We’re\marginnote{7.5} fully qualified \\
in all the Vedic experts teach. \\
As philologists and grammarians, \\
we match our teachers in recitation. \\
We have a dispute \\
regarding the question of ancestry. 

For\marginnote{7.11} \textsanskrit{Bhāradvāja} says that \\
one is a brahmin due to birth, \\
but I declare it’s because of one’s actions. \\
Oh seer, know this as our debate. 

Since\marginnote{7.15} neither of us was able \\
to convince the other, \\
we’ve come to ask you, sir, \\
renowned as the awakened one. 

As\marginnote{7.19} people honor with joined palms \\
the moon on the cusp of waxing, \\
bowing, they revere \\
Gotama in the world. 

We\marginnote{7.23} ask this of Gotama, \\
the eye arisen in the world: \\
is one a brahmin due to birth, \\
or else because of actions? \\
We don’t know, please tell us, \\
so we can recognize a brahmin.” 

“I\marginnote{8.1} shall explain to you,” \\
\scspeaker{said the Buddha, }\\
“accurately and in sequence, \\
the taxonomy of living creatures, \\
for species are indeed diverse. 

Know\marginnote{8.6} the grass and trees, \\
though they lack self-awareness. \\
They’re defined by birth, \\
for species are indeed diverse. 

Next\marginnote{8.10} there are bugs and moths, \\
and so on, to ants and termites. \\
They’re defined by birth, \\
for species are indeed diverse. 

Know\marginnote{8.14} the quadrupeds, too, \\
both small and large. \\
They’re defined by birth, \\
for species are indeed diverse. 

Know,\marginnote{8.18} too, the long-backed snakes, \\
crawling on their bellies. \\
They’re defined by birth, \\
for species are indeed diverse. 

Next\marginnote{8.22} know the fish, \\
whose habitat is the water. \\
They’re defined by birth, \\
for species are indeed diverse. 

Next\marginnote{8.26} know the birds, \\
flying with wings as chariots. \\
They’re defined by birth, \\
for species are indeed diverse. 

While\marginnote{9.1} the differences between these species \\
are defined by birth, \\
the differences between humans \\
are not defined by birth. 

Not\marginnote{9.5} by hair nor by head, \\
not by ear nor by eye, \\
not by mouth nor by nose, \\
not by lips nor by eyebrow, 

not\marginnote{9.9} by shoulder nor by neck, \\
not by belly nor by back, \\
not by buttocks nor by breast, \\
not by groin nor by genitals, 

not\marginnote{9.13} by hands nor by feet, \\
not by fingers nor by nails, \\
not by knees nor by thighs, \\
not by color nor by voice: \\
none of these are defined by birth \\
as it is for other species. 

In\marginnote{9.19} individual human bodies \\
you can’t find such distinctions. \\
The distinctions among humans \\
are spoken of by convention. 

Anyone\marginnote{10.1} among humans \\
who lives off keeping cattle: \\
know them, \textsanskrit{Vāseṭṭha}, \\
as a farmer, not a brahmin. 

Anyone\marginnote{10.5} among humans \\
who lives off various professions: \\
know them, \textsanskrit{Vāseṭṭha}, \\
as a professional, not a brahmin. 

Anyone\marginnote{10.9} among humans \\
who lives off trade: \\
know them, \textsanskrit{Vāseṭṭha}, \\
as a trader, not a brahmin. 

Anyone\marginnote{10.13} among humans \\
who lives off serving others: \\
know them, \textsanskrit{Vāseṭṭha}, \\
as an employee, not a brahmin. 

Anyone\marginnote{10.17} among humans \\
who lives off stealing: \\
know them, \textsanskrit{Vāseṭṭha}, \\
as a bandit, not a brahmin. 

Anyone\marginnote{10.21} among humans \\
who lives off archery: \\
know them, \textsanskrit{Vāseṭṭha}, \\
as a soldier, not a brahmin. 

Anyone\marginnote{10.25} among humans \\
who lives off priesthood: \\
know them, \textsanskrit{Vāseṭṭha}, \\
as a sacrificer, not a brahmin. 

Anyone\marginnote{10.29} among humans \\
who taxes village and nation, \\
know them, \textsanskrit{Vāseṭṭha}, \\
as a ruler, not a brahmin. 

I\marginnote{11.1} don’t call someone a brahmin \\
after the mother or womb they came from. \\
If they still have attachments, \\
they’re just someone who says ‘sir’. \\
Having nothing, taking nothing: \\
that’s who I call a brahmin. 

Having\marginnote{11.7} cut off all fetters \\
they have no anxiety; \\
they’ve got over clinging, and are detached: \\
that’s who I call a brahmin. 

They’ve\marginnote{11.11} cut the strap and harness, \\
the reins and bridle too; \\
with cross-bar lifted, they’re awakened: \\
that’s who I call a brahmin. 

Abuse,\marginnote{11.15} killing, caging: \\
they endure these without anger. \\
Patience is their powerful army: \\
that’s who I call a brahmin. 

Not\marginnote{11.19} irritable or stuck up, \\
dutiful in precepts and observances, \\
tamed, bearing their final body: \\
that’s who I call a brahmin. 

Like\marginnote{11.23} rain off a lotus leaf, \\
like a mustard seed off the point of a pin, \\
sensual pleasures slip off them: \\
that’s who I call a brahmin. 

They\marginnote{11.27} understand for themselves \\
the end of suffering in this life; \\
with burden put down, detached: \\
that’s who I call a brahmin. 

Deep\marginnote{11.31} in wisdom, intelligent, \\
expert in the variety of paths; \\
arrived at the highest goal: \\
that’s who I call a brahmin. 

Socializing\marginnote{11.35} with neither \\
householders nor the homeless; \\
a migrant with no shelter, few in wishes: \\
that’s who I call a brahmin. 

They’ve\marginnote{11.39} laid aside violence \\
against creatures firm and frail; \\
not killing or making others kill: \\
that’s who I call a brahmin. 

Not\marginnote{11.43} fighting among those who fight, \\
extinguished among those who are armed, \\
not taking among those who take: \\
that’s who I call a brahmin. 

They’ve\marginnote{11.47} discarded greed and hate, \\
along with conceit and contempt, \\
like a mustard seed off the point of a pin: \\
that’s who I call a brahmin. 

The\marginnote{11.51} words they utter \\
are sweet, informative, and true, \\
and don’t offend anyone: \\
that’s who I call a brahmin. 

They\marginnote{11.55} don’t steal anything in the world, \\
long or short, \\
fine or coarse, beautiful or ugly: \\
that’s who I call a brahmin. 

They\marginnote{11.59} have no hope \\
for this world or the next; \\
with no need for hope, detached: \\
that’s who I call a brahmin. 

They\marginnote{11.63} have no clinging, \\
knowledge has freed them of indecision, \\
they’ve arrived at the culmination of the deathless: \\
that’s who I call a brahmin. 

They’ve\marginnote{11.67} escaped the snare \\
of both good and bad deeds; \\
sorrowless, stainless, pure: \\
that’s who I call a brahmin. 

Pure\marginnote{11.71} as the spotless moon, \\
clear and undisturbed, \\
they’ve ended desire to be reborn: \\
that’s who I call a brahmin. 

They’ve\marginnote{11.75} got past this grueling swamp \\
of delusion, transmigration. \\
Meditating in stillness, free of indecision, \\
they have crossed over to the far shore. \\
They’re extinguished by not grasping: \\
that’s who I call a brahmin. 

They’ve\marginnote{11.81} given up sensual stimulations, \\
and have gone forth from lay life; \\
they’ve ended rebirth in the sensual realm: \\
that’s who I call a brahmin. 

They’ve\marginnote{11.85} given up craving, \\
and have gone forth from lay life; \\
they’ve ended craving to be reborn: \\
that’s who I call a brahmin. 

They’ve\marginnote{11.89} given up human bonds, \\
and gone beyond heavenly bonds; \\
detached from all attachments: \\
that’s who I call a brahmin. 

Giving\marginnote{11.93} up discontent and desire, \\
they’re cooled and free of attachments; \\
a hero, master of the whole world: \\
that’s who I call a brahmin. 

They\marginnote{11.97} know the passing away \\
and rebirth of all beings; \\
unattached, holy, awakened: \\
that’s who I call a brahmin. 

Gods,\marginnote{11.101} fairies, and humans \\
don’t know their destiny; \\
the perfected ones with defilements ended: \\
that’s who I call a brahmin. 

They\marginnote{11.105} have nothing before or after, \\
or even in between. \\
Having nothing, taking nothing: \\
that’s who I call a brahmin. 

Leader\marginnote{11.109} of the herd, excellent hero, \\
great hermit and victor; \\
unstirred, washed, awakened: \\
that’s who I call a brahmin. 

They\marginnote{11.113} know their past lives, \\
and see heaven and places of loss, \\
and have attained the end of rebirth: \\
that’s who I call a brahmin. 

For\marginnote{12.1} name and clan are formulated \\
as mere convention in the world. \\
Produced by mutual agreement, \\
they’re formulated for each individual. 

For\marginnote{12.5} a long time this misconception \\
has prejudiced those who don’t understand. \\
Ignorant, they declare \\
that one is a brahmin by birth. 

You’re\marginnote{12.9} not a brahmin by birth, \\
nor by birth a non-brahmin. \\
You’re a brahmin by your deeds, \\
and by deeds a non-brahmin. 

You’re\marginnote{12.13} a farmer by your deeds, \\
by deeds you’re a professional; \\
you’re a trader by your deeds, \\
by deeds are you an employee; 

you’re\marginnote{12.17} a bandit by your deeds, \\
by deeds you’re a soldier; \\
you’re a sacrificer by your deeds, \\
by deeds you’re a ruler. 

In\marginnote{13.1} this way the astute regard deeds \\
in accord with truth. \\
Seeing dependent origination, \\
they’re expert in deeds and their results. 

Deeds\marginnote{13.5} make the world go on, \\
deeds make people go on; \\
sentient beings are bound by deeds, \\
like a moving chariot’s linchpin. 

By\marginnote{13.9} austerity and spiritual practice, \\
by restraint and by self-control: \\
that’s how to become a brahmin, \\
this is the supreme brahmin. 

Accomplished\marginnote{13.13} in the three knowledges, \\
peaceful, with rebirth ended, \\
know them, \textsanskrit{Vāseṭṭha}, \\
as \textsanskrit{Brahmā} and Sakka to the wise.” 

%
\end{verse}

When\marginnote{14.1} he had spoken, \textsanskrit{Vāseṭṭha} and \textsanskrit{Bhāradvāja} said to him, “Excellent, Master Gotama! Excellent! As if he were righting the overturned, or revealing the hidden, or pointing out the path to the lost, or lighting a lamp in the dark so people with good eyes can see what’s there, Master Gotama has made the teaching clear in many ways. We go for refuge to Master Gotama, to the teaching, and to the mendicant \textsanskrit{Saṅgha}. From this day forth, may Master Gotama remember us as lay followers who have gone for refuge for life.” 

%
\section*{{\suttatitleacronym MN 99}{\suttatitletranslation With Subha }{\suttatitleroot Subhasutta}}
\addcontentsline{toc}{section}{\tocacronym{MN 99} \toctranslation{With Subha } \tocroot{Subhasutta}}
\markboth{With Subha }{Subhasutta}
\extramarks{MN 99}{MN 99}

\scevam{So\marginnote{1.1} I have heard. }At one time the Buddha was staying near \textsanskrit{Sāvatthī} in Jeta’s Grove, \textsanskrit{Anāthapiṇḍika}’s monastery. 

Now\marginnote{2.1} at that time the brahmin student Subha, Todeyya’s son, was residing in \textsanskrit{Sāvatthī} at a certain householder’s home on some business. Then Subha said to that householder, “Householder, I have heard that \textsanskrit{Sāvatthī} does not lack for perfected ones. What ascetic or brahmin might we pay homage to today?” 

“Sir,\marginnote{2.6} the Buddha is staying near \textsanskrit{Sāvatthī} in Jeta’s Grove, \textsanskrit{Anāthapiṇḍika}’s monastery. You can pay homage to him.” 

Acknowledging\marginnote{3.1} that householder, Subha went to the Buddha and exchanged greetings with him. When the greetings and polite conversation were over, he sat down to one side and said to the Buddha: 

“Master\marginnote{4.1} Gotama, the brahmins say: ‘Laypeople succeed in the procedure of the skillful teaching, not renunciates.’ What do you say about this?” 

“On\marginnote{4.4} this point, student, I speak after analyzing the question, not definitively. I don’t praise wrong practice for either laypeople or renunciates. Because of wrong practice, neither laypeople nor renunciates succeed in the procedure of the skillful teaching. I praise right practice for both laypeople and renunciates. Because of right practice, both laypeople and renunciates succeed in the procedure of the skillful teaching.” 

“Master\marginnote{5.1} Gotama, the brahmins say: ‘Since the work of the lay life has many requirements, duties, issues, and undertakings it is very fruitful. But since the work of the renunciate has few requirements, duties, issues, and undertakings it is not very fruitful.’ What do you say about this?” 

“On\marginnote{5.5} this point, too, I speak after analyzing the question, not definitively. Some work has many requirements, duties, issues, and undertakings, and when it fails it’s not very fruitful. Some work has many requirements, duties, issues, and undertakings, and when it succeeds it is very fruitful. Some work has few requirements, duties, issues, and undertakings, and when it fails it’s not very fruitful. Some work has few requirements, duties, issues, and undertakings, and when it succeeds it is very fruitful. 

And\marginnote{6.1} what work has many requirements, duties, issues, and undertakings, and when it fails it’s not very fruitful? Farming. And what work has many requirements, duties, issues, and undertakings, and when it succeeds it is very fruitful? Again, it is farming. And what work has few requirements, duties, issues, and undertakings, and when it fails it’s not very fruitful? Trade. And what work has few requirements, duties, issues, and undertakings, and when it succeeds it is very fruitful? Again, it’s trade. 

The\marginnote{7.1} lay life is like farming in that it’s work with many requirements and when it fails it’s not very fruitful; but when it succeeds it is very fruitful. The renunciate life is like trade in that it’s work with few requirements and when it fails it’s not very fruitful; but when it succeeds it is very fruitful.” 

“Master\marginnote{8.1} Gotama, the brahmins prescribe five things for making merit and succeeding in the skillful.” 

“If\marginnote{8.3} you don’t mind, please explain these in this assembly.” 

“It’s\marginnote{8.5} no trouble when gentlemen such as yourself are sitting here.” 

“Well,\marginnote{8.6} speak then, student.” 

“Master\marginnote{9.1} Gotama, truth is the first thing. Austerity is the second thing. Celibacy is the third thing. Recitation is the fourth thing. Generosity is the fifth thing. These are the five things that the brahmins prescribe for making merit and succeeding in the skillful. What do you say about this?” 

“Well,\marginnote{9.8} student, is there even a single one of the brahmins who says this: ‘I declare the result of these five things after realizing it with my own insight’?” 

“No,\marginnote{9.10} Master Gotama.” 

“Well,\marginnote{9.11} is there even a single teacher of the brahmins, or a teacher’s teacher, or anyone back to the seventh generation of teachers, who says this: ‘I declare the result of these five things after realizing it with my own insight’?” 

“No,\marginnote{9.13} Master Gotama.” 

“Well,\marginnote{9.14} what of the ancient hermits of the brahmins, namely \textsanskrit{Aṭṭhaka}, \textsanskrit{Vāmaka}, \textsanskrit{Vāmadeva}, \textsanskrit{Vessāmitta}, Yamadaggi, \textsanskrit{Aṅgīrasa}, \textsanskrit{Bhāradvāja}, \textsanskrit{Vāseṭṭha}, Kassapa, and Bhagu? They were the authors and propagators of the hymns. Their hymnal was sung and propagated and compiled in ancient times; and these days, brahmins continue to sing and chant it, chanting what was chanted and teaching what was taught. Did even they say: ‘We declare the result of these five things after realizing it with our own insight’?” 

“No,\marginnote{9.17} Master Gotama.” 

“So,\marginnote{9.18} student, it seems that there is not a single one of the brahmins, not even anyone back to the seventh generation of teachers, nor even the ancient hermits of the brahmins who says: ‘We declare the result of these five things after realizing it with our own insight.’ 

Suppose\marginnote{9.25} there was a queue of blind men, each holding the one in front: the first one does not see, the middle one does not see, and the last one does not see. In the same way, it seems to me that the brahmins’ statement turns out to be comparable to a queue of blind men: the first one does not see, the middle one does not see, and the last one does not see.” 

When\marginnote{10.1} he said this, Subha became angry and upset with the Buddha because of the simile of the queue of blind men. He even attacked and badmouthed the Buddha himself, saying, “The ascetic Gotama will be worsted!” He said to the Buddha: 

“Master\marginnote{10.3} Gotama, the brahmin \textsanskrit{Pokkharasāti} \textsanskrit{Upamañña} of the Subhaga Forest says: ‘This is exactly what happens with some ascetics and brahmins. They claim to have a superhuman distinction in knowledge and vision worthy of the noble ones. But their statement turns out to be a joke—mere words, void and hollow. For how on earth can a human being know or see or realize a superhuman distinction in knowledge and vision worthy of the noble ones? That is not possible.’” 

“But\marginnote{11.1} student, does \textsanskrit{Pokkharasāti} understand the minds of all these ascetics and brahmins, having comprehended them with his mind?” 

“Master\marginnote{11.2} Gotama, \textsanskrit{Pokkharasāti} doesn’t even know the mind of his own bonded maid \textsanskrit{Puṇṇikā}, so how could he know all those ascetics and brahmins?” 

“Suppose\marginnote{12.1} there was a person blind from birth. They couldn’t see sights that are dark or bright, or blue, yellow, red, or magenta. They couldn’t see even and uneven ground, or the stars, or the moon and sun. They’d say: ‘There’s no such thing as dark and bright sights, and no-one who sees them. There’s no such thing as blue, yellow, red, magenta, even and uneven ground, stars, moon and sun, and no-one who sees these things. I don’t know it or see it, therefore it doesn’t exist.’ Would they be speaking rightly?” 

“No,\marginnote{12.14} Master Gotama. There are such things as dark and bright sights, and one who sees them. There is blue, yellow, red, magenta, even and uneven ground, stars, moon and sun, and one who sees these things. So it’s not right to say this: ‘I don’t know it or see it, therefore it doesn’t exist.’” 

“In\marginnote{13.1} the same way, \textsanskrit{Pokkharasāti} is blind and sightless. It’s not possible for him to know or see or realize a superhuman distinction in knowledge and vision worthy of the noble ones. 

What\marginnote{13.3} do you think, student? There are well-to-do brahmins of Kosala such as the brahmins \textsanskrit{Caṅkī}, \textsanskrit{Tārukkha}, \textsanskrit{Pokkharasāti}, \textsanskrit{Jāṇussoṇi}, and your father Todeyya. What’s better for them: that their speech agrees or disagrees with accepted usage?” 

“That\marginnote{13.6} it agrees, Master Gotama.” 

“What’s\marginnote{13.7} better for them: that their speech is thoughtful or thoughtless?” 

“That\marginnote{13.8} it is thoughtful.” 

“What’s\marginnote{13.9} better for them: that their speech follows reflection or is unreflective?” 

“That\marginnote{13.10} it follows reflection.” 

“What’s\marginnote{13.11} better for them: that their speech is beneficial or worthless?” 

“That\marginnote{13.12} it’s beneficial.” 

“What\marginnote{14.1} do you think, student? If this is so, does \textsanskrit{Pokkharasāti}’s speech agree or disagree with accepted usage?” 

“It\marginnote{14.3} disagrees, Master Gotama.” 

“Is\marginnote{14.4} it thoughtful or thoughtless?” 

“Thoughtless.”\marginnote{14.5} 

“Is\marginnote{14.6} it reflective or unreflective?” 

“Unreflective.”\marginnote{14.7} 

“Is\marginnote{14.8} it beneficial or worthless?” 

“Worthless.”\marginnote{14.9} 

“Student,\marginnote{15.1} there are these five hindrances. What five? The hindrances of sensual desire, ill will, dullness and drowsiness, restlessness and remorse, and doubt. These are the five hindrances. \textsanskrit{Pokkharasāti} is obstructed, shrouded, covered, and engulfed by these five hindrances. It’s not possible for him to know or see or realize a superhuman distinction in knowledge and vision worthy of the noble ones. 

There\marginnote{16.1} are these five kinds of sensual stimulation. What five? There are sights known by the eye that are likable, desirable, agreeable, pleasant, sensual, and arousing. There are sounds known by the ear … smells known by the nose … tastes known by the tongue … touches known by the body that are likable, desirable, agreeable, pleasant, sensual, and arousing. These are the five kinds of sensual stimulation. 

\textsanskrit{Pokkharasāti}\marginnote{16.9} enjoys himself with these five kinds of sensual stimulation, tied, infatuated, attached, blind to the drawbacks, and not understanding the escape. It’s not possible for him to know or see or realize a superhuman distinction in knowledge and vision worthy of the noble ones. 

What\marginnote{17.1} do you think, student? Which would have better flames, color, and radiance: a fire that depends on grass and logs as fuel, or one that does not?” 

“If\marginnote{17.3} it were possible for a fire to burn without depending on grass and logs as fuel, that would have better flames, color, and radiance.” 

“But\marginnote{17.4} it isn’t possible, except by psychic power. Rapture that depends on the five kinds of sensual stimulation is like a fire that depends on grass and logs as fuel. Rapture that’s apart from sensual pleasures and unskillful qualities is like a fire that doesn’t depend on grass and logs as fuel. 

And\marginnote{17.7} what is rapture that’s apart from sensual pleasures and unskillful qualities? It’s when a mendicant, quite secluded from sensual pleasures, secluded from unskillful qualities, enters and remains in the first absorption, which has the rapture and bliss born of seclusion, while placing the mind and keeping it connected. This is rapture that’s apart from sensual pleasures and unskillful qualities. 

Furthermore,\marginnote{17.10} as the placing of the mind and keeping it connected are stilled, they enter and remain in the second absorption, which has the rapture and bliss born of immersion, with internal clarity and confidence, and unified mind, without placing the mind and keeping it connected. This too is rapture that’s apart from sensual pleasures and unskillful qualities. 

Of\marginnote{18.1} the five things that the brahmins prescribe for making merit and succeeding in the skillful, which do they say is the most fruitful?” 

“Generosity.”\marginnote{18.2} 

“What\marginnote{19.1} do you think, student? Suppose a brahmin was setting up a big sacrifice. Then two brahmins came along, thinking to participate. Then one of those brahmins thought: ‘Oh, I hope that I alone get the best seat, the best drink, and the best almsfood in the refectory, not some other brahmin.’ But it’s possible that some other brahmin gets the best seat, the best drink, and the best almsfood in the refectory. Thinking, ‘Some other brahmin has got the best seat, the best drink, the best almsfood,’ they get angry and bitter. What do the brahmins say is the result of this?” 

“Master\marginnote{19.11} Gotama, brahmins don’t give gifts so that others will get angry and upset. Rather, they give only out of compassion.” 

“In\marginnote{19.14} that case, isn’t compassion a sixth ground for making merit?” 

“In\marginnote{19.16} that case, compassion is a sixth ground for making merit.” 

“Of\marginnote{20.1} the five things that the brahmins prescribe for making merit and succeeding in the skillful, where do you usually find them: among laypeople or renunciates?” 

“Mostly\marginnote{20.3} among renunciates, and less so among lay people. For a lay person has many requirements, duties, issues, and undertakings, and they can’t always tell the truth, practice austerities, be celibate, do lots of recitation, or be very generous. But a renunciate has few requirements, duties, issues, and undertakings, and they can always tell the truth, practice austerities, be celibate, do lots of recitation, and be very generous. Of the five things that the brahmins prescribe for making merit and succeeding in the skillful, I usually find them among renunciates, and less so among laypeople.” 

“I\marginnote{21.1} say that the five things prescribed by the brahmins for making merit are prerequisites of the mind for developing a mind free of enmity and ill will. 

Take\marginnote{21.3} a mendicant who speaks the truth. Thinking, ‘I’m truthful,’ they find inspiration in the meaning and the teaching, and find joy connected with the teaching. And I say that joy connected with the skillful is a prerequisite of the mind for developing a mind free of enmity and ill will. 

Take\marginnote{21.7} a mendicant who practices austerities … is celibate … does lots of recitation … and is very generous. Thinking, ‘I’m very generous,’ they find inspiration in the meaning and the teaching, and find joy connected with the teaching. And I say that joy connected with the skillful is a prerequisite of the mind for developing a mind free of enmity and ill will. I say that these five things prescribed by the brahmins for making merit are prerequisites of the mind for developing a mind free of enmity and ill will.” 

When\marginnote{22.1} he had spoken, Subha said to him, “Master Gotama, I have heard that the ascetic Gotama knows a path to companionship with \textsanskrit{Brahmā}.” 

“What\marginnote{22.4} do you think, student? Is the village of \textsanskrit{Naḷakāra} nearby?” 

“Yes\marginnote{22.6} it is, sir.” 

“What\marginnote{22.7} do you think, student? Suppose a person was born and raised in \textsanskrit{Naḷakāra}. And as soon as they left the town some people asked them for the road to \textsanskrit{Naḷakāra}. Would they be slow or hesitant to answer?” 

“No,\marginnote{22.10} Master Gotama. Why is that? Because they were born and raised in \textsanskrit{Naḷakāra}. They’re well acquainted with all the roads to the village.” 

“Still,\marginnote{22.13} it’s possible they might be slow or hesitant to answer. But the Realized One is never slow or hesitant when questioned about the \textsanskrit{Brahmā} realm or the practice that leads to the \textsanskrit{Brahmā} realm. I understand \textsanskrit{Brahmā}, the \textsanskrit{Brahmā} realm, and the practice that leads to the \textsanskrit{Brahmā} realm, practicing in accordance with which one is reborn in the \textsanskrit{Brahmā} realm.” 

“Master\marginnote{23.1} Gotama, I have heard that the ascetic Gotama teaches a path to companionship with \textsanskrit{Brahmā}. Please teach me that path.” 

“Well\marginnote{23.4} then, student, listen and pay close attention, I will speak.” 

“Yes,\marginnote{23.5} sir,” replied Subha. The Buddha said this: 

“And\marginnote{24.1} what is a path to companionship with \textsanskrit{Brahmā}? Firstly, a mendicant meditates spreading a heart full of love to one direction, and to the second, and to the third, and to the fourth. In the same way above, below, across, everywhere, all around, they spread a heart full of love to the whole world—abundant, expansive, limitless, free of enmity and ill will. When the heart’s release by love has been developed like this, any limited deeds they’ve done don’t remain or persist there. Suppose there was a powerful horn blower. They’d easily make themselves heard in the four quarters. In the same way, when the heart’s release by love has been developed like this, any limited deeds they’ve done don’t remain or persist there. This is a path to companionship with \textsanskrit{Brahmā}. 

Furthermore,\marginnote{25{-}27.1} a mendicant meditates spreading a heart full of compassion … 

They\marginnote{25{-}27.2} meditate spreading a heart full of rejoicing … 

They\marginnote{25{-}27.3} meditate spreading a heart full of equanimity to one direction, and to the second, and to the third, and to the fourth. In the same way above, below, across, everywhere, all around, they spread a heart full of equanimity to the whole world—abundant, expansive, limitless, free of enmity and ill will. When the heart’s release by equanimity has been developed and cultivated like this, any limited deeds they’ve done don’t remain or persist there. Suppose there was a powerful horn blower. They’d easily make themselves heard in the four quarters. In the same way, when the heart’s release by equanimity has been developed and cultivated like this, any limited deeds they’ve done don’t remain or persist there. This too is a path to companionship with \textsanskrit{Brahmā}.” 

When\marginnote{28.1} he had spoken, Subha said to him, “Excellent, Master Gotama! Excellent! As if he were righting the overturned, or revealing the hidden, or pointing out the path to the lost, or lighting a lamp in the dark so people with good eyes can see what’s there, Master Gotama has made the teaching clear in many ways. I go for refuge to Master Gotama, to the teaching, and to the mendicant \textsanskrit{Saṅgha}. From this day forth, may Master Gotama remember me as a lay follower who has gone for refuge for life. Well, now, Master Gotama, I must go. I have many duties, and much to do.” 

“Please,\marginnote{29.1} student, go at your convenience.” And then Subha approved and agreed with what the Buddha said. He got up from his seat, bowed, and respectfully circled the Buddha, keeping him on his right, before leaving. 

Now\marginnote{30.1} at that time the brahmin \textsanskrit{Jāṇussoṇi} drove out from \textsanskrit{Sāvatthī} in the middle of the day in an all-white chariot drawn by mares. He saw the student Subha coming off in the distance, and said to him, “So, Master \textsanskrit{Bhāradvāja}, where are you coming from in the middle of the day?” 

“Just\marginnote{30.5} now, good sir, I’ve come from the presence of the ascetic Gotama.” 

“What\marginnote{30.6} do you think of the ascetic Gotama’s lucidity of wisdom? Do you think he’s astute?” 

“My\marginnote{30.7} good man, who am I to judge the ascetic Gotama’s lucidity of wisdom? You’d really have to be on the same level to judge his lucidity of wisdom.” 

“Master\marginnote{30.9} \textsanskrit{Bhāradvāja} praises the ascetic Gotama with lofty praise indeed.” 

“Who\marginnote{30.10} am I to praise the ascetic Gotama? He is praised by the praised as the first among gods and humans. The five things that the brahmins prescribe for making merit and succeeding in the skillful he says are prerequisites of the mind for developing a mind free of enmity and ill will.” 

When\marginnote{31.1} he had spoken, \textsanskrit{Jāṇussoṇi} got down from his chariot, arranged his robe over one shoulder, raised his joined palms toward the Buddha, and expressed this heartfelt sentiment three times, “King Pasenadi of Kosala is lucky, so very lucky that the Realized One, the perfected one, the fully awakened Buddha is living in his realm!” 

%
\section*{{\suttatitleacronym MN 100}{\suttatitletranslation With Saṅgārava }{\suttatitleroot Saṅgāravasutta}}
\addcontentsline{toc}{section}{\tocacronym{MN 100} \toctranslation{With Saṅgārava } \tocroot{Saṅgāravasutta}}
\markboth{With Saṅgārava }{Saṅgāravasutta}
\extramarks{MN 100}{MN 100}

\scevam{So\marginnote{1.1} I have heard. }At one time the Buddha was wandering in the land of the Kosalans together with a large \textsanskrit{Saṅgha} of mendicants. Now at that time a brahmin lady named \textsanskrit{Dhanañjānī} was residing at \textsanskrit{Caṇḍalakappa}. She was devoted to the Buddha, the teaching, and the \textsanskrit{Saṅgha}. Once, she tripped and expressed this heartfelt sentiment three times: 

“Homage\marginnote{2.3} to that Blessed One, the perfected one, the fully awakened Buddha! 

Homage\marginnote{2.4} to that Blessed One, the perfected one, the fully awakened Buddha! 

Homage\marginnote{2.5} to that Blessed One, the perfected one, the fully awakened Buddha!” 

Now\marginnote{3.1} at that time the brahmin student \textsanskrit{Saṅgārava} was residing in \textsanskrit{Caṇḍalakappa}. He was young, newly tonsured; he was sixteen years old. He had mastered the three Vedas, together with their vocabularies, ritual, phonology and etymology, and the testament as fifth. He knew philology and grammar, and was well versed in cosmology and the marks of a great man. 

Hearing\marginnote{3.2} \textsanskrit{Dhanañjānī}’s exclamation, he said to her, “The brahmin lady named \textsanskrit{Dhanañjānī} is a disgrace! Though brahmins who are proficient in the three Vedas are found, she praises that shaveling, that fake ascetic.” 

“But\marginnote{3.5} my little dear, you don’t understand the Buddha’s ethics and wisdom. If you did, you’d never think of abusing or insulting him.” 

“Well\marginnote{3.7} then, ma’am, let me know when the Buddha arrives in \textsanskrit{Caṇḍalakappa}.” 

“I\marginnote{3.8} will, dear,” she replied. 

And\marginnote{4.1} then the Buddha, traveling stage by stage in the Kosalan lands, arrived at \textsanskrit{Caṇḍalakappa}, where he stayed in the mango grove of the Todeyya brahmins. 

\textsanskrit{Dhanañjānī}\marginnote{5.1} heard that he had arrived. So she went to \textsanskrit{Saṅgārava} and told him, adding, “Please, my little dear, go at your convenience.” 

“Yes,\marginnote{5.5} ma’am,” replied \textsanskrit{Saṅgārava}. He went to the Buddha and exchanged greetings with him. When the greetings and polite conversation were over, he sat down to one side and said to the Buddha: 

“Master\marginnote{6.1} Gotama, there are some ascetics and brahmins who claim to have mastered the fundamentals of the spiritual life having attained perfection and consummation of insight in the present life. Where do you stand regarding these?” 

“I\marginnote{7.1} say there is a diversity among those who claim to have mastered the fundamentals of the spiritual life having attained perfection and consummation of insight in the present life. There are some ascetics and brahmins who are oral transmitters. Through oral transmission they claim to have mastered the fundamentals of the spiritual life. For example, the brahmins who are proficient in the three Vedas. There are some ascetics and brahmins who solely by mere faith claim to have mastered the fundamentals of the spiritual life. For example, those who rely on logic and inquiry. There are some ascetics and brahmins who, having directly known for themselves the principle regarding teachings not learned before from another, claim to have mastered the fundamentals of the spiritual life. I am one of those. And here’s a way to understand that I am one of them. 

Before\marginnote{9.1} my awakening—when I was still unawakened but intent on awakening—I thought: ‘Living in a house is cramped and dirty, but the life of one gone forth is wide open. It’s not easy for someone living at home to lead the spiritual life utterly full and pure, like a polished shell. Why don’t I shave off my hair and beard, dress in ocher robes, and go forth from the lay life to homelessness?’ Some time later, while still black-haired, blessed with youth, in the prime of life—though my mother and father wished otherwise, weeping with tearful faces—I shaved off my hair and beard, dressed in ocher robes, and went forth from the lay life to homelessness. 

Once\marginnote{9.6} I had gone forth I set out to discover what is skillful, seeking the supreme state of sublime peace. I approached \textsanskrit{Āḷāra} \textsanskrit{Kālāma} and said to him, ‘Reverend \textsanskrit{Kālāma}, I wish to lead the spiritual life in this teaching and training.’ 

\textsanskrit{Āḷāra}\marginnote{9.8} \textsanskrit{Kālāma} replied, ‘Stay, venerable. This teaching is such that a sensible person can soon realize their own tradition with their own insight and live having achieved it.’ 

I\marginnote{9.11} quickly memorized that teaching. So far as lip-recital and oral recitation were concerned, I spoke with knowledge and the authority of the elders. I claimed to know and see, and so did others. 

Then\marginnote{9.13} it occurred to me, ‘It is not solely by mere faith that \textsanskrit{Āḷāra} \textsanskrit{Kālāma} declares: “I realize this teaching with my own insight, and live having achieved it.” Surely he meditates knowing and seeing this teaching.’ 

So\marginnote{10.1} I approached \textsanskrit{Āḷāra} \textsanskrit{Kālāma} and said to him: ‘Reverend \textsanskrit{Kālāma}, to what extent do you say you’ve realized this teaching with your own insight?’ When I said this, he declared the dimension of nothingness. 

Then\marginnote{10.4} it occurred to me, ‘It’s not just \textsanskrit{Āḷāra} \textsanskrit{Kālāma} who has faith, energy, mindfulness, immersion, and wisdom; I too have these things. Why don’t I make an effort to realize the same teaching that \textsanskrit{Āḷāra} \textsanskrit{Kālāma} says he has realized with his own insight?’ I quickly realized that teaching with my own insight, and lived having achieved it. 

So\marginnote{10.12} I approached \textsanskrit{Āḷāra} \textsanskrit{Kālāma} and said to him, ‘Reverend \textsanskrit{Kālāma}, have you realized this teaching with your own insight up to this point, and declare having achieved it?’ 

‘I\marginnote{10.14} have, reverend.’ 

‘I\marginnote{10.15} too have realized this teaching with my own insight up to this point, and live having achieved it.’ 

‘We\marginnote{10.16} are fortunate, reverend, so very fortunate to see a venerable such as yourself as one of our spiritual companions! So the teaching that I’ve realized with my own insight, and declare having achieved it, you’ve realized with your own insight, and live having achieved it. The teaching that you’ve realized with your own insight, and live having achieved it, I’ve realized with my own insight, and declare having achieved it. So the teaching that I know, you know, and the teaching you know, I know. I am like you and you are like me. Come now, reverend! We should both lead this community together.’ 

And\marginnote{10.23} that is how my teacher \textsanskrit{Āḷāra} \textsanskrit{Kālāma} placed me, his student, on the same position as him, and honored me with lofty praise. 

Then\marginnote{10.24} it occurred to me, ‘This teaching doesn’t lead to disillusionment, dispassion, cessation, peace, insight, awakening, and extinguishment. It only leads as far as rebirth in the dimension of nothingness.’ Realizing that this teaching was inadequate, I left disappointed. 

I\marginnote{11.1} set out to discover what is skillful, seeking the supreme state of sublime peace. I approached Uddaka, son of \textsanskrit{Rāma}, and said to him, ‘Reverend, I wish to lead the spiritual life in this teaching and training.’ 

Uddaka\marginnote{11.3} replied, ‘Stay, venerable. This teaching is such that a sensible person can soon realize their own tradition with their own insight and live having achieved it.’ 

I\marginnote{11.6} quickly memorized that teaching. So far as lip-recital and oral recitation were concerned, I spoke with knowledge and the authority of the elders. I claimed to know and see, and so did others. 

Then\marginnote{11.8} it occurred to me, ‘It is not solely by mere faith that \textsanskrit{Rāma} declared: “I realize this teaching with my own insight, and live having achieved it.” Surely he meditated knowing and seeing this teaching.’ 

So\marginnote{11.11} I approached Uddaka, son of \textsanskrit{Rāma}, and said to him, ‘Reverend, to what extent did \textsanskrit{Rāma} say he’d realized this teaching with his own insight?’ When I said this, Uddaka, son of \textsanskrit{Rāma}, declared the dimension of neither perception nor non-perception. 

Then\marginnote{11.14} it occurred to me, ‘It’s not just \textsanskrit{Rāma} who had faith, energy, mindfulness, immersion, and wisdom; I too have these things. Why don’t I make an effort to realize the same teaching that \textsanskrit{Rāma} said he had realized with his own insight?’ I quickly realized that teaching with my own insight, and lived having achieved it. 

So\marginnote{12.1} I approached Uddaka, son of \textsanskrit{Rāma}, and said to him, ‘Reverend, had \textsanskrit{Rāma} realized this teaching with his own insight up to this point, and declared having achieved it?’ 

‘He\marginnote{12.3} had, reverend.’ 

‘I\marginnote{12.4} too have realized this teaching with my own insight up to this point, and live having achieved it.’ 

‘We\marginnote{12.5} are fortunate, reverend, so very fortunate to see a venerable such as yourself as one of our spiritual companions! So the teaching that \textsanskrit{Rāma} had realized with his own insight, and declared having achieved it, you’ve realized with your own insight, and live having achieved it. The teaching that you’ve realized with your own insight, and live having achieved it, \textsanskrit{Rāma} had realized with his own insight, and declared having achieved it. So the teaching that \textsanskrit{Rāma} directly knew, you know, and the teaching you know, \textsanskrit{Rāma} directly knew. \textsanskrit{Rāma} was like you and you are like \textsanskrit{Rāma}. Come now, reverend! You should lead this community.’ And that is how my spiritual companion Uddaka, son of \textsanskrit{Rāma}, placed me in the position of a teacher, and honored me with lofty praise. 

Then\marginnote{12.13} it occurred to me, ‘This teaching doesn’t lead to disillusionment, dispassion, cessation, peace, insight, awakening, and extinguishment. It only leads as far as rebirth in dimension of neither perception nor non-perception.’ Realizing that this teaching was inadequate, I left disappointed. 

I\marginnote{13.1} set out to discover what is skillful, seeking the supreme state of sublime peace. Traveling stage by stage in the Magadhan lands, I arrived at Senanigama near \textsanskrit{Uruvelā}. There I saw a delightful park, a lovely grove with a flowing river that was clean and charming, with smooth banks. And nearby was a village for alms. Then it occurred to me, ‘This park is truly delightful, a lovely grove with a flowing river that’s clean and charming, with smooth banks. And nearby there’s a village to go for alms. This is good enough for a gentleman who wishes to put forth effort in meditation.’ So I sat down right there, thinking: ‘This is good enough for meditation.’ 

And\marginnote{14.1} then these three examples, which were neither supernaturally inspired, nor learned before in the past, occurred to me. Suppose there was a green, sappy log, and it was lying in water. Then a person comes along with a drill-stick, thinking to light a fire and produce heat. What do you think, \textsanskrit{Bhāradvāja}? By drilling the stick against that green, sappy log lying in water, could they light a fire and produce heat?” 

“No,\marginnote{14.7} Master Gotama. Why is that? Because it’s a green, sappy log, and it’s lying in the water. That person will eventually get weary and frustrated.” 

“In\marginnote{14.11} the same way, there are ascetics and brahmins who don’t live withdrawn in body and mind from sensual pleasures. They haven’t internally given up or stilled desire, affection, infatuation, thirst, and passion for sensual pleasures. Regardless of whether or not they suffer painful, sharp, severe, acute feelings due to overexertion, they are incapable of knowledge and vision, of supreme awakening. This was the first example that occurred to me. 

Then\marginnote{15.1} a second example occurred to me. Suppose there was a green, sappy log, and it was lying on dry land far from the water. Then a person comes along with a drill-stick, thinking to light a fire and produce heat. What do you think, \textsanskrit{Bhāradvāja}? By drilling the stick against that green, sappy log on dry land far from water, could they light a fire and produce heat?” 

“No,\marginnote{15.7} Master Gotama. Why is that? Because it’s still a green, sappy log, despite the fact that it’s lying on dry land far from water. That person will eventually get weary and frustrated.” “In the same way, there are ascetics and brahmins who live withdrawn in body and mind from sensual pleasures. But they haven’t internally given up or stilled desire, affection, infatuation, thirst, and passion for sensual pleasures. Regardless of whether or not they suffer painful, sharp, severe, acute feelings due to overexertion, they are incapable of knowledge and vision, of supreme awakening. This was the second example that occurred to me. 

Then\marginnote{16.1} a third example occurred to me. Suppose there was a dried up, withered log, and it was lying on dry land far from the water. Then a person comes along with a drill-stick, thinking to light a fire and produce heat. What do you think, \textsanskrit{Bhāradvāja}? By drilling the stick against that dried up, withered log on dry land far from water, could they light a fire and produce heat?” 

“Yes,\marginnote{16.7} Master Gotama. Why is that? Because it’s a dried up, withered log, and it’s lying on dry land far from water.” 

“In\marginnote{16.10} the same way, there are ascetics and brahmins who live withdrawn in body and mind from sensual pleasures. And they have internally given up and stilled desire, affection, infatuation, thirst, and passion for sensual pleasures. Regardless of whether or not they suffer painful, sharp, severe, acute feelings due to overexertion, they are capable of knowledge and vision, of supreme awakening. This was the third example that occurred to me. These are the three examples, which were neither supernaturally inspired, nor learned before in the past, that occurred to me. 

Then\marginnote{17.1} it occurred to me, ‘Why don’t I, with teeth clenched and tongue pressed against the roof of my mouth, squeeze, squash, and torture mind with mind.’ So that’s what I did, until sweat ran from my armpits. It was like when a strong man grabs a weaker man by the head or throat or shoulder and squeezes, squashes, and tortures them. In the same way, with teeth clenched and tongue pressed against the roof of my mouth, I squeezed, squashed, and tortured mind with mind until sweat ran from my armpits. My energy was roused up and unflagging, and my mindfulness was established and lucid, but my body was disturbed, not tranquil, because I’d pushed too hard with that painful striving. 

Then\marginnote{18.1} it occurred to me, ‘Why don’t I practice the breathless absorption?’ So I cut off my breathing through my mouth and nose. But then winds came out my ears making a loud noise, like the puffing of a blacksmith’s bellows. My energy was roused up and unflagging, and my mindfulness was established and lucid, but my body was disturbed, not tranquil, because I’d pushed too hard with that painful striving. 

Then\marginnote{19.1} it occurred to me, ‘Why don’t I keep practicing the breathless absorption?’ So I cut off my breathing through my mouth and nose. But then strong winds ground my head, like a strong man was drilling into my head with a sharp point. My energy was roused up and unflagging, and my mindfulness was established and lucid, but my body was disturbed, not tranquil, because I’d pushed too hard with that painful striving. 

Then\marginnote{20.1} it occurred to me, ‘Why don’t I keep practicing the breathless absorption?’ So I cut off my breathing through my mouth and nose. But then I got a severe headache, like a strong man was tightening a tough leather strap around my head. My energy was roused up and unflagging, and my mindfulness was established and lucid, but my body was disturbed, not tranquil, because I’d pushed too hard with that painful striving. 

Then\marginnote{21.1} it occurred to me, ‘Why don’t I keep practicing the breathless absorption?’ So I cut off my breathing through my mouth and nose. But then strong winds carved up my belly, like a deft butcher or their apprentice was slicing my belly open with a meat cleaver. My energy was roused up and unflagging, and my mindfulness was established and lucid, but my body was disturbed, not tranquil, because I’d pushed too hard with that painful striving. 

Then\marginnote{22.1} it occurred to me, ‘Why don’t I keep practicing the breathless absorption?’ So I cut off my breathing through my mouth and nose. But then there was an intense burning in my body, like two strong men grabbing a weaker man by the arms to burn and scorch him on a pit of glowing coals. My energy was roused up and unflagging, and my mindfulness was established and lucid, but my body was disturbed, not tranquil, because I’d pushed too hard with that painful striving. 

Then\marginnote{23.1} some deities saw me and said, ‘The ascetic Gotama is dead.’ Others said, ‘He’s not dead, but he’s dying.’ Others said, ‘He’s not dead or dying. The ascetic Gotama is a perfected one, for that is how the perfected ones live.’ 

Then\marginnote{24.1} it occurred to me, ‘Why don’t I practice completely cutting off food?’ 

But\marginnote{24.3} deities came to me and said, ‘Good sir, don’t practice totally cutting off food. If you do, we’ll infuse divine nectar into your pores and you will live on that.’ 

Then\marginnote{24.7} it occurred to me, ‘If I claim to be completely fasting while these deities are infusing divine nectar in my pores, that would be a lie on my part.’ So I dismissed those deities, saying, ‘There’s no need.’ 

Then\marginnote{25.1} it occurred to me, ‘Why don’t I just take a little bit of food each time, a cup of broth made from mung beans, lentils, chickpeas, or green gram.’ So that’s what I did, until my body became extremely emaciated. Due to eating so little, my limbs became like the joints of an eighty-year-old or a corpse, my bottom became like a camel’s hoof, my vertebrae stuck out like beads on a string, and my ribs were as gaunt as the broken-down rafters on an old barn. Due to eating so little, the gleam of my eyes sank deep in their sockets, like the gleam of water sunk deep down a well. Due to eating so little, my scalp shriveled and withered like a green bitter-gourd in the wind and sun. Due to eating so little, the skin of my belly stuck to my backbone, so that when I tried to rub the skin of my belly I grabbed my backbone, and when I tried to rub my backbone I rubbed the skin of my belly. Due to eating so little, when I tried to urinate or defecate I fell face down right there. Due to eating so little, when I tried to relieve my body by rubbing my limbs with my hands, the hair, rotted at its roots, fell out. 

Then\marginnote{26.1} some people saw me and said: ‘The ascetic Gotama is black.’ Some said: ‘He’s not black, he’s brown.’ Some said: ‘He’s neither black nor brown. The ascetic Gotama has tawny skin.’ That’s how far the pure, bright complexion of my skin had been ruined by taking so little food. 

Then\marginnote{27.1} it occurred to me, ‘Whatever ascetics and brahmins have experienced painful, sharp, severe, acute feelings due to overexertion—whether in the past, future, or present—this is as far as it goes, no-one has done more than this. But I have not achieved any superhuman distinction in knowledge and vision worthy of the noble ones by this severe, grueling work. Could there be another path to awakening?’ 

Then\marginnote{28.1} it occurred to me, ‘I recall sitting in the cool shade of the rose-apple tree while my father the Sakyan was off working. Quite secluded from sensual pleasures, secluded from unskillful qualities, I entered and remained in the first absorption, which has the rapture and bliss born of seclusion, while placing the mind and keeping it connected. Could that be the path to awakening?’ Stemming from that memory came the realization: ‘\emph{That} is the path to awakening!’ 

Then\marginnote{29.1} it occurred to me, ‘Why am I afraid of that pleasure, for it has nothing to do with sensual pleasures or unskillful qualities?’ I thought, ‘I’m not afraid of that pleasure, for it has nothing to do with sensual pleasures or unskillful qualities.’ 

Then\marginnote{30.1} it occurred to me, ‘I can’t achieve that pleasure with a body so excessively emaciated. Why don’t I eat some solid food, some rice and porridge?’ So I ate some solid food. 

Now\marginnote{30.5} at that time the five mendicants were attending on me, thinking, ‘The ascetic Gotama will tell us of any truth that he realizes.’ But when I ate some solid food, they left disappointed in me, saying, ‘The ascetic Gotama has become indulgent; he has strayed from the struggle and returned to indulgence.’ 

After\marginnote{31{-}33.1} eating solid food and gathering my strength, quite secluded from sensual pleasures, secluded from unskillful qualities, I entered and remained in the first absorption … As the placing of the mind and keeping it connected were stilled, I entered and remained in the second absorption … third absorption … fourth absorption. 

When\marginnote{34.1} my mind had immersed in \textsanskrit{samādhi} like this—purified, bright, flawless, rid of corruptions, pliable, workable, steady, and imperturbable—I extended it toward recollection of past lives. I recollected many past lives. That is: one, two, three, four, five, ten, twenty, thirty, forty, fifty, a hundred, a thousand, a hundred thousand rebirths; many eons of the world contracting, many eons of the world expanding, many eons of the world contracting and expanding. And so I recollected my many kinds of past lives, with features and details. 

This\marginnote{35.1} was the first knowledge, which I achieved in the first watch of the night. Ignorance was destroyed and knowledge arose; darkness was destroyed and light arose, as happens for a meditator who is diligent, keen, and resolute. 

When\marginnote{36{-}38.1} my mind had immersed in \textsanskrit{samādhi} like this—purified, bright, flawless, rid of corruptions, pliable, workable, steady, and imperturbable—I extended it toward knowledge of the death and rebirth of sentient beings. With clairvoyance that is purified and superhuman, I saw sentient beings passing away and being reborn—inferior and superior, beautiful and ugly, in a good place or a bad place. I understood how sentient beings are reborn according to their deeds … 

This\marginnote{36{-}38.3} was the second knowledge, which I achieved in the middle watch of the night. Ignorance was destroyed and knowledge arose; darkness was destroyed and light arose, as happens for a meditator who is diligent, keen, and resolute. 

When\marginnote{39.1} my mind had immersed in \textsanskrit{samādhi} like this—purified, bright, flawless, rid of corruptions, pliable, workable, steady, and imperturbable—I extended it toward knowledge of the ending of defilements. I truly understood: ‘This is suffering’ … ‘This is the origin of suffering’ … ‘This is the cessation of suffering’ … ‘This is the practice that leads to the cessation of suffering’. I truly understood: ‘These are defilements’ … ‘This is the origin of defilements’ … ‘This is the cessation of defilements’ … ‘This is the practice that leads to the cessation of defilements’. 

Knowing\marginnote{40.1} and seeing like this, my mind was freed from the defilements of sensuality, desire to be reborn, and ignorance. When it was freed, I knew it was freed. 

I\marginnote{40.3} understood: ‘Rebirth is ended; the spiritual journey has been completed; what had to be done has been done; there is no return to any state of existence.’ 

This\marginnote{41.1} was the third knowledge, which I achieved in the last watch of the night. Ignorance was destroyed and knowledge arose; darkness was destroyed and light arose, as happens for a meditator who is diligent, keen, and resolute.” 

When\marginnote{42.1} he had spoken, \textsanskrit{Saṅgārava} said to the Buddha, “Master Gotama’s effort was steadfast and appropriate for a good person, since he is a perfected one, a fully awakened Buddha. But Master Gotama, do gods survive?” 

“I’ve\marginnote{42.5} understood about gods in terms of causes.” 

“But\marginnote{42.7} Master Gotama, when asked ‘Do gods survive?’ why did you say that you have understood about gods in terms of causes? If that’s the case, isn’t it a hollow lie?” 

“When\marginnote{42.9} asked ‘Do gods survive’, whether you reply ‘Gods survive’ or ‘I’ve understood in terms of causes’ a sensible person would come to the definite conclusion that gods survive.” 

“But\marginnote{42.12} why didn’t you say that in the first place?” 

“It’s\marginnote{42.13} widely agreed in the world that gods survive.” 

When\marginnote{43.1} he had spoken, \textsanskrit{Saṅgārava} said to the Buddha, “Excellent, Master Gotama! Excellent! As if he were righting the overturned, or revealing the hidden, or pointing out the path to the lost, or lighting a lamp in the dark so people with good eyes can see what’s there, Master Gotama has made the Teaching clear in many ways. I go for refuge to Master Gotama, to the teaching, and to the mendicant \textsanskrit{Saṅgha}. From this day forth, may Master Gotama remember me as a lay follower who has gone for refuge for life.” 

%
\backmatter%
\chapter*{Colophon}
\addcontentsline{toc}{chapter}{Colophon}
\markboth{Colophon}{Colophon}

\section*{The Translator}

Bhikkhu Sujato was born as Anthony Aidan Best on 4/11/1966 in Perth, Western Australia. He grew up in the pleasant suburbs of Mt Lawley and Attadale alongside his sister Nicola, who was the good child. His mother, Margaret Lorraine Huntsman née Pinder, said “he’ll either be a priest or a poet”, while his father, Anthony Thomas Best, advised him to “never do anything for money”. He attended Aquinas College, a Catholic school, where he decided to become an atheist. At the University of WA he studied philosophy, aiming to learn what he wanted to do with his life. Finding that what he wanted to do was play guitar, he dropped out. His main band was named Martha’s Vineyard, which achieved modest success in the indie circuit. Then it broke up, because everyone thought they personally were reason for the success, which, oddly enough, turns out not to have been the case. 

A seemingly random encounter with a roadside joey took him to Thailand, where he entered his first meditation retreat at Wat Ram Poeng, Chieng Mai in 1992. He decided to devote himself to the Buddha’s path, and took full ordination in Wat Pa Nanachat in 1994, where his teachers were Ajahn Pasanno and Ajahn Jayasaro. In 1997 he returned to Perth to study with Ajahn Brahm at Bodhinyana Monastery. 

He spent several years practicing in seclusion in Malaysia and Thailand before establishing Santi Forest Monastery in Bundanoon, NSW, in 2003. There he was instrumental in supporting the establishment of the Theravada bhikkhuni order in Australia and advocating for women’s rights. He continues to teach in Australia and globally, with a special concern for the moral implications of climate change and other forms of environmental destruction. He has published a series of books of original and groundbreaking research on early Buddhism. 

In 2005 he founded SuttaCentral together with Rod Bucknell and John Kelly. In 2015, seeing the need for a complete, accurate, plain English translation of the Pali texts, he undertook the task, spending nearly three years in isolation on the isle of Qi Mei off the coast of the nation of Taiwan. He completed the four main \textsanskrit{Nikāyas} in 2018, and the early books of the Khuddaka \textsanskrit{Nikāya} were complete by 2021. All this work is dedicated to the public domain and is entirely free of copyright encumbrance. 

In 2019 he returned to Sydney where, together with Bhikkhu Akaliko, he established Lokanta Vihara (The Monastery at the End of the World). 

\section*{Creation Process}

Primary source was the digital \textsanskrit{Mahāsaṅgīti} edition of the Pali \textsanskrit{Tipiṭaka}. Translated from the Pali, with reference to several English translations, especially those of Bhikkhu Bodhi.

\section*{The Translation}

This translation was part of a project to translate the four Pali \textsanskrit{Nikāyas} with the following aims: plain, approachable English; consistent terminology; accurate rendition of the Pali; free of copyright. It was made during 2016–2018 while Bhikkhu Sujato was staying in Qimei, Taiwan.

\section*{About SuttaCentral}

SuttaCentral publishes early Buddhist texts. Since 2005 we have provided root texts in Pali, Chinese, Sanskrit, Tibetan, and other languages, parallels between these texts, and translations in many modern languages. We build on the work of generations of scholars, and offer our contribution freely.

SuttaCentral is driven by volunteer contributions, and in addition we employ professional developers. We offer a sponsorship program for high quality translations from the original languages. Financial support for SuttaCentral is handled by the SuttaCentral Development Trust, a charitable trust registered in Australia.

\section*{About Bilara}

“Bilara” means “cat” in Pali, and it is the name of our Computer Assisted Translation (CAT) software. Bilara is a web app that enables translators to translate early Buddhist texts into their own language. These translations are published on SuttaCentral with the root text and translation side by side.

\section*{About SuttaCentral Editions}

The SuttaCentral Editions project makes high quality books from selected Bilara translations. These are published in formats including HTML, EPUB, PDF, and print.

If you want to print any of our Editions, please let us know and we will help prepare a file to your specifications.

%
\end{document}